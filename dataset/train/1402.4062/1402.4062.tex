\documentclass[oribibl,envcountsame,envcountsect,runningheads]{llncs}



\newif\ifignore 

\ignorefalse

\clearpage{}
\def \tnil {\langle\rangle}
\spnewtheorem{assumption}[theorem]{Assumption}{\bfseries}{\upshape}


\newcommand{\auxproof}[1]{
\ifignore\mbox{}\newline
\textbf{PROOF:} \dotfill\newline
{\it #1}\mbox{}\newline
\textbf{ENDPROOF}\dotfill
\fi}

\usepackage[english]{babel}
\selectlanguage{english}
\usepackage{amsmath}
\usepackage{amssymb}
\usepackage{url}
\usepackage{enumerate}
\usepackage[shortlabels]{enumitem}
\usepackage{times}
\usepackage{amstext}

\usepackage{wrapfig}
\setlength{\intextsep}{.1\intextsep}

\message{<Paul Taylor's Proof Trees, 2 August 1996>}


\def\introrule{{\cal I}}\def\elimrule{{\cal E}}\def\andintro{\using{\land}\introrule\justifies}\def\impelim{\using{\Rightarrow}\elimrule\justifies}\def\allintro{\using{\forall}\introrule\justifies}\def\allelim{\using{\forall}\elimrule\justifies}\def\falseelim{\using{\bot}\elimrule\justifies}\def\existsintro{\using{\exists}\introrule\justifies}

\def\andelim#1{\using{\land}#1\elimrule\justifies}\def\orintro#1{\using{\lor}#1\introrule\justifies}

\def\impintro#1{\using{\Rightarrow}\introrule_{#1}\justifies}\def\orelim#1{\using{\lor}\elimrule_{#1}\justifies}\def\existselim#1{\using{\exists}\elimrule_{#1}\justifies}



\newdimen\proofrulebreadth \proofrulebreadth=.05em
\newdimen\proofdotseparation \proofdotseparation=1.25ex
\newdimen\proofrulebaseline \proofrulebaseline=2ex
\newcount\proofdotnumber \proofdotnumber=3
\let\then\relax
\def\hfi{\hskip0pt plus.0001fil}
\mathchardef\squigto="3A3B
\newif\ifinsideprooftree\insideprooftreefalse
\newif\ifonleftofproofrule\onleftofproofrulefalse
\newif\ifproofdots\proofdotsfalse
\newif\ifdoubleproof\doubleprooffalse
\let\wereinproofbit\relax
\newdimen\shortenproofleft
\newdimen\shortenproofright
\newdimen\proofbelowshift
\newbox\proofabove
\newbox\proofbelow
\newbox\proofrulename
\def\shiftproofbelow{\let\next\relax\afterassignment\setshiftproofbelow\dimen0 }
\def\shiftproofbelowneg{\def\next{\multiply\dimen0 by-1 }\afterassignment\setshiftproofbelow\dimen0 }
\def\setshiftproofbelow{\next\proofbelowshift=\dimen0 }
\def\setproofrulebreadth{\proofrulebreadth}

\def\prooftree{\ifnum  \lastpenalty=1
\then   \unpenalty
\else   \onleftofproofrulefalse
\fi
\ifonleftofproofrule
\else   \ifinsideprooftree
        \then   \hskip.5em plus1fil
        \fi
\fi
\bgroup \setbox\proofbelow=\hbox{}\setbox\proofrulename=\hbox{}\let\justifies\proofover\let\leadsto\proofoverdots\let\Justifies\proofoverdbl
\let\using\proofusing\let\endprooftree\fi
\proofdotsfalse\doubleprooffalse
\let\thickness\setproofrulebreadth
\let\shiftright\shiftproofbelow \let\shift\shiftproofbelow
\let\shiftleft\shiftproofbelowneg
\let\ifwasinsideprooftree\ifinsideprooftree
\insideprooftreetrue
\setbox\proofabove=\hbox\bgroup\egroup  \shortenproofleft=\dimen0
\shortenproofright=\dimen1
\proofrulebreadth=\dimen2
\proofbelowshift=\dimen3
\proofdotseparation=\dimen4
\proofdotnumber=\count255
}

\def\proofover{\eproofbit \setbox\proofbelow=\hbox\bgroup \let\wereinproofbit\proofover
\displaystyle
}\def\proofoverdots{\eproofbit \proofdotstrue
\setbox\proofbelow=\hbox\bgroup \let\wereinproofbit\proofoverdots

}

\def\endprooftree{\eproofbit \dimen5 =0pt\dimen0=\wd\proofabove \advance\dimen0-\shortenproofleft
\advance\dimen0-\shortenproofright
\dimen1=.5\dimen0 \advance\dimen1-.5\wd\proofbelow
\dimen4=\dimen1
\advance\dimen1\proofbelowshift \advance\dimen4-\proofbelowshift
\ifdim  \dimen1<0pt
\then   \advance\shortenproofleft\dimen1
        \advance\dimen0-\dimen1
        \dimen1=0pt
\ifdim  \shortenproofleft<0pt
        \then   \setbox\proofabove=\hbox{\kern-\shortenproofleft\unhbox\proofabove}\shortenproofleft=0pt
        \fi
\fi
\ifdim  \dimen4<0pt
\then   \advance\shortenproofright\dimen4
        \advance\dimen0-\dimen4
        \dimen4=0pt
\fi
\ifdim  \shortenproofright<\wd\proofrulename
\then   \shortenproofright=\wd\proofrulename
\fi
\dimen2=\shortenproofleft \advance\dimen2 by\dimen1
\dimen3=\shortenproofright\advance\dimen3 by\dimen4
\ifproofdots
\then
        \dimen6=\shortenproofleft \advance\dimen6 .5\dimen0
        \setbox1=\vbox to\proofdotseparation{\vss\hbox{}\vss}\setbox0=\hbox{\advance\dimen6-.5\wd1
                \kern\dimen6
                \unhbox\proofrulename}\else   \dimen6=\fontdimen22\the\textfont2 \dimen7=\dimen6
        \advance\dimen6by.5\proofrulebreadth
        \advance\dimen7by-.5\proofrulebreadth
        \setbox0=\hbox{\kern\shortenproofleft
                \ifdoubleproof
                \then   \hbox to\dimen0{\mkern-2mu=\mkern-2mu}\else   \vrule height\dimen6 depth-\dimen7 width\dimen0
                \fi
                \unhbox\proofrulename}\ht0=\dimen6 \dp0=-\dimen7
\fi
\let\doll\relax
\ifwasinsideprooftree
\then   \let\VBOX\vbox
\else   \ifmmode\else\fi
        \let\VBOX\vcenter
\fi
\VBOX   {\baselineskip\proofrulebaseline \lineskip.2ex
        \expandafter\lineskiplimit\ifproofdots0ex\else-0.6ex\fi
        \hbox   spread\dimen5   {\hfi\unhbox\proofabove\hfi}\hbox{\box0}\hbox   {\kern\dimen2 \box\proofbelow}}\doll \global\dimen2=\dimen2
\global\dimen3=\dimen3
\egroup \ifonleftofproofrule
\then   \shortenproofleft=\dimen2
\fi
\shortenproofright=\dimen3
\onleftofproofrulefalse
\ifinsideprooftree
\then   \hskip.5em plus 1fil \penalty2
\fi
}

 
\renewcommand{\arraystretch}{1.3}
\setlength{\arraycolsep}{3pt}






\newcommand{\QEDbox}{\square}
\newcommand{\QED}{\hspace*{\fill}}

\newcommand{\Lft}{\mathcal{L}}
\newcommand{\Dst}{\mathcal{D}}
\newcommand{\Dstfin}{\Dst_{\textrm{fin}}}
\newcommand{\Pow}{\mathcal{P}}
\newcommand{\Powfin}{\Pow_{\textrm{fin}}}
\newcommand{\E}{\mathcal{E}}
\newcommand{\Graph}[1]{\textrm{Graph}(#1)}
\newcommand{\support}{\textrm{supp}}
\newcommand{\after}{\mathrel{\circ}}
\newcommand{\idmap}{\textrm{id}}
\newcommand{\Hom}{\mathit{Hom}}
\newcommand{\cat}[1]{\ensuremath{\mathbf{#1}}}
\newcommand{\Cat}[1]{\ensuremath{\mathbf{#1}}}
\newcommand{\Sets}{\Cat{Sets}}
\newcommand{\Rel}{\Cat{Rel}}
\newcommand{\CL}{\Cat{CL}}
\newcommand{\op}[1]{#1^{\textrm{op}}}
\newcommand{\Kl}{\mathcal{K}{\kern-.2ex}\ell}
\newcommand{\EM}{\mathcal{E}{\kern-.2ex}\mathcal{M}}
\newcommand{\Alg}{\mathbf{Alg}}
\newcommand{\CoAlg}{\mathbf{CoAlg}}
\newcommand{\lift}[1]{\smash{\widehat{#1}}}
\newcommand{\free}[1]{{#1^*}}
\newcommand{\tuple}[1]{\langle#1\rangle}
\newcommand{\lam}[2]{\lambda#1.\,#2}
\newcommand{\lamin}[3]{\lambda#1\in#2.\,#3}
\newcommand{\all}[2]{\forall#1.\,#2}
\newcommand{\allin}[3]{\forall#1\in#2.\,#3}
\newcommand{\ex}[2]{\exists#1.\,#2}
\newcommand{\exin}[3]{\exists#1\in#2.\,#3}
\newcommand{\set}[2]{\{#1\;|\;#2\}}
\newcommand{\setin}[3]{\{#1\in#2\;|\;#3\}}
\newcommand{\st}{\ensuremath{\mathrm{st}}}
\newcommand{\dst}{\ensuremath{\mathrm{dst}}}
\newcommand{\evmap}{\ensuremath{\mathrm{ev}}}
\newcommand{\bc}{\ensuremath{\mathrm{bc}}\xspace}
\newcommand{\tr}{\ensuremath{\mathrm{tr}}}
\newcommand{\extnt}{\ensuremath{\frak{e}}}
\newcommand{\Tr}{\ensuremath{\mathrm{Tr}}}
\newcommand{\Ev}{\ensuremath{\mathrm{Ev}}}
\newcommand{\klplus}{\mathrel{\dot{+}}}
\newcommand{\klkappa}{\dot{\kappa}}
\newcommand{\klafter}{\circ}
\newcommand{\bdbefore}{\mathrel{\raisebox{.1em}{}}}
\newcommand{\bdafter}{\mathrel{\bullet}}
\newcommand{\toFinal}[1]{{!_{#1}}}\DeclareMathSymbol{\fromInit}{\mathord}{operators}{"3C}
\newcommand{\up}{\ensuremath{\mathrm{up}}}
\newcommand{\IfThenElse}[3]{\textsf{if }#1\textsf{ then }#2\textsf{ else }#3}
\newcommand{\N}{\mathbb{N}} \newcommand{\Z}{\mathbb{Z}} \newcommand{\C}{\cat{C}} 

\newcommand{\cppo}{\cat{Cppo}}
\newcommand{\J}{\mathcal{J}}



\newcommand\M{\mathcal M}
\newcommand\D{\mathcal D}
\newcommand\V{\mathcal V}

\newcommand{\state}[1]{*++[F-:<10pt>]{#1}}
\newcommand{\fstate}[1]{*++[F=:<10pt>]{#1}}


\mathchardef\ls="213C    \mathchardef\gr="213E    \newcommand{\<}{\langle}
\renewcommand{\>}{\rangle}
\usepackage{eucal}
\usepackage{multirow}
\usepackage[cmtip,all]{xy}
\DeclareMathOperator{\supp}{supp}
\DeclareMathOperator{\dist}{dLTS}
\newcommand{\bb}[1]{[\![ #1 ]\!]}

\newcommand{\congrightarrow}{\mathrel{\stackrel{
           \raisebox{.5ex}{}}{
           \raisebox{0ex}[0ex][0ex]{}}}}
\newcommand{\conglongrightarrow}{\mathrel{\stackrel{
           \raisebox{.5ex}{}}{
           \raisebox{0ex}[0ex][0ex]{}}}}












\newcommand{\takeout}[1]{\empty}

\def\id{\mathrm{id}}
\def\Id{\mathrm{Id}}
\def\sol#1{{#1}^\bullet}
\def\eps{\epsilon}
\def\inj{\mathsf{in}}
\def\inl{\mathsf{inl}}
\def\inr{\mathsf{inr}}
\def\refeq#1{(\ref{#1})}
\def\colim{\mathop{\textrm{colim}}}



\newtheorem{construction}{Construction}
\def\lsol#1{{#1}^\dag} \def\gensol#1{{#1}^\dag} \def\ssol#1{{#1}^{\star}} \def\cansol#1{{#1}^\dag}
\def\altsol#1{{#1}^\ddag}
\def\quotsol#1{{#1}^{\sim}}
\def\epselim#1{{#1}\!\setminus\! \epsilon}
\def\carrier{I}
\newcommand{\circone}[1]{\mbox{\textcircled{\scriptsize #1}}}\newcommand{\can}[1]{\mathit{can}_{#1}}

\newcommand{\bbq}[1]{\bb{#1}_{\sim}} 

\def\To{\Rightarrow}
\def\:{\colon}
\def\GF{F} \def\GFG{G} \def\MM{R} \def\quot{\xi} \def\quotG{\gamma} 

\clearpage{}
\usepackage{ifthen}










\newtheorem{definition_th}{Definition}[section]  \newtheorem{theorem_th}[definition_th]{Theorem}
\newtheorem{proposition_th}[definition_th]{Proposition}
\newtheorem{corollary_th}[definition_th]{Corollary}
\newtheorem{lemma_th}[definition_th]{Lemma}
\newtheorem{problem_th}[definition_th]{Problem}
\newtheorem{algo_th}[definition_th]{Algorithm}
\newtheorem{example_th}[definition_th]{Example}





\newenvironment{theorem_for}[2][\empty]{\bigskip\noindent{\bf
    Theorem~\ref{#2}}\ifthenelse{\equal{#1}{\empty}}{{\bf.}}{ {\bf (#1).}}\it}{\vspace{0.5cm}}
\newenvironment{corollary_for}[2][\empty]{\bigskip\noindent{\bf
    Corollary~\ref{#2}}\ifthenelse{\equal{#1}{\empty}}{{\bf.}}{ {\bf (#1).}}\it}{\vspace{0.5cm}}
\newenvironment{proposition_for}[2][\empty]{\bigskip\noindent{\bf
    Proposition~\ref{#2}}\ifthenelse{\equal{#1}{\empty}}{{\bf.}}{ {\bf
      (#1).}}\it}{\vspace{0.5cm}}
\newenvironment{lemma_for}[2][\empty]{\bigskip\noindent{\bf
    Lemma~\ref{#2}}\ifthenelse{\equal{#1}{\empty}}{{\bf.}}{ {\bf (#1).}}\it}{\vspace{0.5cm}}
\newenvironment{algo}[1]{\begin{algo_th}#1
    \rm}{\hfill\vspace{0.3cm}
  \end{algo_th}}
\newenvironment{algo_ind}[1]{\begin{algo_th}#1 \rm \begin{tabbing}
    xx \= xx \= xx \= xx \= xx \= xx \= xx \= xx \kill \\}
  {\end{tabbing}\hfill\vspace{0.5cm} \end{algo_th}}
\newenvironment{descr}[1]{\noindent{\bf #1}}{\quad}
 



\pagestyle{plain}


\begin{document}




\title{How to Kill Epsilons with a Dagger}
\subtitle{A Coalgebraic Take on Systems with Algebraic Label Structure}

\author{Filippo Bonchi\inst{1} \and Stefan Milius\inst{2} \and Alexandra Silva\inst{3} \and Fabio Zanasi\inst{1} }
\institute{ENS Lyon, U. de Lyon, CNRS, INRIA, UCBL, France \and Lehrstuhl f\"ur Theoretische Informatik, Friedrich-Alexander Universit\"at Erlangen-N\"urnberg \and Institute for Computing and Information Sciences,
  Radboud University Nijmegen\thanks{\scriptsize Also affiliated to Centrum Wiskunde \& Informatica (Amsterdam, The Netherlands) and HASLab / INESC TEC, Universidade do Minho (Braga, Portugal).} }


\maketitle

\begin{abstract}
We propose an abstract framework for modeling state-based systems with internal behavior as e.g. given by silent or -transitions. Our approach employs monads with a parametrized fixpoint operator  to give a semantics to those systems and implement a sound procedure of abstraction of the internal transitions, whose labels are seen as the unit of a free monoid. More broadly, our approach extends the standard coalgebraic framework for state-based systems by taking into account the algebraic structure of the labels of their transitions. This allows to consider a wide range of other examples, including Mazurkiewicz traces for concurrent systems.
\end{abstract}

\section{Introduction}\label{Sec:Intro}
The theory of coalgebras provides an elegant mathematical framework to express the semantics of computing devices:
the operational semantics, which is usually given as a state machine, is modeled as a coalgebra for a functor; the denotational semantics as the unique map into the final coalgebra of that functor. While  the denotational semantics is often \emph{compositional} (as, for instance, ensured by the bialgebraic approach of \cite{plotkin-semop}), it is sometimes not \emph{fully-abstract}, i.e, it discriminates systems that are equal from the point of view of an external observer. This is due to the presence of internal transitions (also called -transitions) that are not observable but that are not abstracted away by the usual coalgebraic semantics using the unique homomorphism into the final coalgebra.



In this paper, we focus on the problem of giving trace semantics to systems with internal transitions.
Our approach stems from an elementary observation (pointed out in previous work, e.g. \cite{Sobocinski2012}): the labels of transitions form a monoid and the internal transitions are those labeled by the unit of the monoid. Thus, there is an \emph{algebraic structure} on the labels that needs to be taken into account when modeling the denotational semantics of those systems.
To illustrate this point, consider the following two non-deterministic automata (NDA).

The one on the left (that we call ) is an NDA with -transitions: its transitions are labeled either by the symbols of the alphabet  or by the empty word . The one on the right (that we call ) has transitions labeled by languages in , here represented as regular expressions.
The monoid structure on the labels is explicit on , while it is less evident in  since the set of labels  does not form a monoid. However, this set can be trivially embedded into  by looking at each symbols as the corresponding singleton language. For this reason each automaton with -transitions, like , can be regarded as an automaton with transitions labeled by languages, like . Furthermore, we can define the semantics of NDA with -transitions by defining the semantics of NDA with transitions labeled by languages: a word  is accepted by a state  if there is a path  where  is a final state, and there exist a decomposition  such that .
Observe that, with this definition,  and  accept the same language: all words over  that end with  or . In fact,  was obtained from  in a well-known process to compute the regular expression denoting the language accepted by a given automaton~\cite{Hopcroft}.

We propose to define the semantics of systems with internal transitions following the same idea as in the above example. Given some transition type (i.e.~an endofunctor) , one first defines an embedding of -systems with internal transitions into -system, where  has been derived from  by making explicit the algebraic structure on the labels. Next one models the semantics of an -system as the one of the corresponding -system . Naively, one could think of defining the semantics of  as the unique map  into the final coalgebra for . However, this approach turns out to be too fine grained, essentially because it ignores the underlying algebraic structure on the labels of . The same problem can be observed in the example above:  and the representation of  as an automaton with languages as labels have different final semantics---they accept the same language only modulo the equations of monoids.

Thus we need to extend the standard coalgebraic framework by taking into account the algebraic structure on labels. To this end, we develop our theory for systems whose transition type  has a \emph{canonical fixpoint}, i.e. its initial algebra and final coalgebra coincide. This is the case for many relevant examples, as observed in \cite{HasuoJS:07}. Our \emph{canonical fixpoint semantics} will be given as the composite , where  is a coalgebra morphism given by finality and  is an algebra morphism given by initiality. The target of  will be an algebra for  encoding the equational theory associated with the labels of -systems. Intuitively,  being an \emph{algebra} morphism, will take the quotient of the semantics given by  modulo those equations. Therefore the extension provided by  is the technical feature allowing us to take into account the algebraic structure on labels.

To study the properties of our canonical fixpoint semantics, it will be convenient to formulate it as an operator  assigning to systems (seen as sets of equations) a certain \emph{solution}. Within the same perspective we will implement a different kind of solution  turning any system  with internal transitions into one  where those have been abstracted away. By comparing the operators  and , we will then be able to show that such a procedure (also called \emph{-elimination}) is sound with respect to the canonical fixpoint semantics.

To conclude, we will explore further the flexibility of our framework. In particular, we will model the case in which the algebraic structure of the labels is quotiented under some equations, resulting in a coarser equivalence than the one given by the canonical fixpoint semantics. As a relevant example of this phenomenon, we give the first coalgebraic account of Mazurkiewicz traces.

\paragraph{Synopsis} After recalling the necessary background in Section \ref{Sec:Trace}, we discuss our motivating examples---automata with -transitions and automata on words---in Section \ref{SSec:Mot}. Section \ref{sec:Theory} is devoted to present the canonical fixpoint semantics and the sound procedure of -elimination. This framework is then instantiated to the examples of Section \ref{SSec:Mot}. Finally, in Section \ref{ssec:quot} we show how a quotient of the algebra on labels induces a coarser canonical fixpoint semantics. We propose Mazurkiewicz traces as a motivating example for such a construction. A full version of this paper with all proofs and extra material can be found in \url{http://arxiv.org/abs/1402.4062}.





\section{Preliminaries}\label{Sec:Trace}
In this section we introduce the basic notions we need for our abstract framework. We assume some familiarity with
category theory. We will use boldface capitals  to denote categories,  for objects and  for morphisms.
We use Greek letters and double arrows, e.g. , for natural transformations, monad morphisms and any kind of 2-cells. If  has coproducts we will denote them by  and use  for the coproduct injections.


\subsection{Monads}

We recall the basics of the theory of monads, as needed here. For more information, see \textit{e.g.}~\cite{MacLane71}. A monad is a functor  together with two natural transformations, a \emph{unit}  and a \emph{multiplication} , which are required to satisfy the following equations, for every :  and .

A \emph{morphism of monads} from  to  is a natural transformation  that preserves unit and multiplication:  and . A \emph{quotient of monads} is a morphism of monads with epimorphic components.



\begin{example}\label{ex:mnds}
We briefly describe the examples of monads that
we use in this paper.
\begin{enumerate}\item \label{pt:powersetmonad} Let . The powerset monad  maps a set  to the set  of
  subsets of , and a function  to  given by direct image. The unit is given by the
  singleton set map  and multiplication by union
  .

\item \label{pt:exceptionmonad}  Let  be a category with coproducts and  an object of . The exception monad  is defined on objects as 
and on arrows  as . Its unit and multiplication are given on  respectively as  and , where  is the codiagonal. When ,  can be thought as a set of {\em exceptions} and this monad is often used to encode computations that might fail throwing an exception chosen from the set .

\item \label{pt:freemonad} Let  be an endofunctor on a category  such that for every
  object  there exists a free -algebra  on 
  (equivalently, an initial
  -algebra) with the structure  and
  universal morphism . Then as proved by Barr~\cite{barr:70} (see also
  Kelly~\cite{kelly:80})  is the functor part of a \emph{free
  monad} on  with the unit given by the above  and the
  multiplication given by the freeness of :  is the
  unique -algebra homomorphism from  to
   such that .
  Also notice that for a complete category every free monad
  arises in this way. Finally, for later use we fix the notation  for the universal natural transformation of the free monad.
\end{enumerate}
\end{example}
Given a monad , its \emph{Kleisli category}  has the same objects as , but morphisms  in  are morphisms  in . The identity map  in  is 's unit ; and composition  in  uses 's multiplication: . There is a forgetful functor , sending  to  and  to . This functor has a left adjoint , given by  and . The Kleisli category  inherits coproducts from the underlying category . More precisely, for every objects  and  their coproduct  in  is also a coproduct in  with the injections  and .
\subsection{Distributive laws and liftings}\label{SSec:Distributive}
The most interesting examples of the theory that we will present in Section~\ref{sec:Theory} concern coalgebras
for functors  that are obtained as liftings of endofunctors  on .
Formally, given a monad , a \emph{lifting} of  to  is an endofunctor
  such that . The lifting of a monad  is a monad  such that
 is a lifting of  and ,  are given on  (i.e. ) respectively as  and .

A natural way of lifting functors and monads is by mean of distributive laws.
A {\em distributive law} of a monad  over a monad  is a
natural transformation , that commutes
appropriately with the unit and multiplication of both monads; more precisely, the diagrams below commute:

A distributive law of a {\em functor }  over a {\em monad}  is a
natural transformation  such that only the two topmost squares above commute.

The following ``folklore'' result gives an alternative description of
distributive laws in terms of liftings to Kleisli categories, see also~\cite{Johnstone75}, \cite{Mulry93} or~\cite{BalanK11}.
\begin{proposition}[\cite{Mulry93}]
\label{LiftProp}
Let  be a monad on a category . Then the following holds:
\begin{enumerate}
\item For every endofunctor  on , there is a bijective correspondence between liftings of  to  and distributive laws of  over .
\item For every monad  on , there is a bijective correspondence between liftings of  to  and distributive laws of  over .
 \end{enumerate}
\end{proposition}
In what follows we shall simply write  for the
lifting of an endofunctor .
\begin{proposition}[\cite{HasuoJS:07}]\label{prop:liftinginitialalgebra}
 Let  be a monad and  be a functor with a lifting
 .
 If  has an initial algebra  (in ),
 then  is an initial algebra for  (in ).
\end{proposition}
In our examples, we will often consider the free monad (Example \ref{ex:mnds}.\ref{pt:freemonad})  generated by a lifted functor . The following result will be pivotal.
\newcommand{\propliftingfreemonad}{ Let  be a functor and  be a monad such that there is a lifting
 . Then the free monad  lifts to a monad
 . Moreover, .}
\begin{proposition}\label{prop:liftingfreemonad}
\propliftingfreemonad
\end{proposition}


Recall from~\cite{HasuoJS:07} that for every polynomial endofunctor
 on  there exists a canonical distributive law of  over
any \emph{commutative} monad  (equivalently, a canonical lifting of
 to ); this result was later extended to so-called analytic
endofunctors of  (see~\cite{mps:09}). This can be used in our
applications since the power-set functor  is commutative, and so
is the exception monad  iff .




\subsection{-enriched categories}
\label{sec:cppo}

For our general theory we are going to assume that we work in a
category where the hom-sets carry a cpo structure. Recall that a
\emph{cpo} is a partially ordered set in which all -chains
have a join. A cpo with bottom is a cpo with a least element . A
function between cpos is called \emph{continuous} if it preserves
joins of -chains. Cpos with bottom and continuous maps form a category
that we denote by .

A \emph{\cppo-enriched category}  is a category where (a)
each hom-set  is a cpo with a bottom element  and
(b) composition is continuous, that is:

The composition is called \emph{left strict} if  for all arrows 
.

In our applications,  will mostly be a Kleisli category for a
monad on . Throughout this subsection we assume
that  is a -enriched category.

An endofunctor  is said to be \emph{locally
  continuous} if for any -chain ,  in  we have:


We are going to make use of the fact that a locally continuous endofunctor
 on  has a \emph{canonical fixpoint}, i.e. whenever its
initial algebra exists it is also its final coalgebra:

\begin{theorem}[\cite{Freyd}]\label{thm:Freyd}
 Let  be a locally continuous endofunctor on the \cppo-enriched category  whose composition is left-strict.
 If an initial -algebra  exists,
 then  is a final -coalgebra.
\end{theorem}


In the sequel, we will be interested in free algebras for a
functor  on  and the free monad 
(cf.~Example~\ref{ex:mnds}.\ref{pt:freemonad}). For this observe that coproducts in  are always -enriched, i.e.~all copairing maps  are continuous; in fact, it is easy to show that this map is continuous in both of its arguments using that composition with the coproduct injections is continuous.

\newcommand{\lemHstar}{  Let  be -enriched with composition left-strict.
  Furthermore, let  be locally continuous and assume that all
  free -algebras exist. Then the free monad  is locally continuous.
}

\begin{proposition}
  \label{lem:Hstar}
\lemHstar
  \end{proposition}

\subsection{Final Coalgebras in Kleisli categories}\label{SSec:Coalg}

In our applications the -enriched
category will be the Kleisli category  of a
monad on  and the endofunctors of interest are liftings
 of endofunctors  on . It is known that in this
setting a final coalgebra  for the lifting  can be obtained
as a lifting of an initial -algebra (see Hasuo et
al.~\cite{HasuoJS:07}).
 The following result is a variation of Theorem~3.3 in~\cite{HasuoJS:07}:

\begin{theorem}\label{thm:HJS}
Let  be a monad and  be a functor such that
\begin{enumerate}[(a)]
 \item  is \cppo-enriched with composition left strict;
 \item  is accessible (i.e.,  preserves -filtered colimits for some cardinal ) and has a lifting  which is
   locally continuous.
\end{enumerate}
If  is the initial algebra for the functor ,
then
\begin{enumerate}
 \item  is the initial algebra for the functor ;
 \item  is the final coalgebra for the functor .
\end{enumerate}
\end{theorem}
The first item follows from Proposition~\ref{prop:liftinginitialalgebra} and the second one follows
from Theorem~\ref{thm:Freyd}. There are two differences with Theorem~3.3 in~\cite{HasuoJS:07}:
\begin{enumerate}[(1)]
\item The functor  is supposed to preserve -colimits rather that being accessible. We use the assumption of accessibility because it guarantees the existence of all free algebras for  and for , which implies also that for all  an initial -algebra exists. This property of  will be needed for applying our framework of Section \ref{sec:Theory}.

\item We assume that the lifting  is locally continuous rather than locally monotone. We will need continuity to ensure the double dagger law in Remark~\ref{rmk:doubledagger}. This assumption is not really restrictive since, as explained in Section 3.3.1 of~\cite{HasuoJS:07}, in all the meaningful examples where  is locally monotone, it is also locally continuous.
\end{enumerate}

\begin{example}[NDA]\label{ex:NDA} Consider the powerset monad  (Example \ref{ex:mnds}.\ref{pt:powersetmonad}) and the functor  on  (with ). The functor  lifts to  on  as follows: for any  in  (that is  in ),  is given by  and .

  Non-deterministic automata (NDA) over the input alphabet  can be
regarded as coalgebras for the functor
. Consider, on the left, a 3-state NDA, where the only final state is marked by a double circle.

  It can be represented as a coalgebra , that is a function , given above on the right, which assigns to each state  a set which: contains
 if  is final; and  for
all transitions .
\end{example}

It is easy to see that  and  above satisfy the conditions of Theorem \ref{thm:HJS} and therefore both the final -coalgebra and the initial -algebra are the lifting of the initial algebra for the functor , given by  with structure  which maps  to  and  to .

For an NDA , the final coalgebra homomorphism  is the function  that maps every state in  to the language that it accepts. In :



\subsection{Monads with Fixpoint Operators}\label{Sec:Elgot}

In order to develop our theory of systems with internal behavior, we will adopt an equational perspective on coalgebras. In the sequel we recall some preliminaries on this viewpoint.

Let  be a monad on any category . Any morphism  (i.e.~a coalgebra for the functor
) may be understood as a system of mutually recursive equations. In our
applications we are interested in the case where  and
  is a (lifted) free monad. As in the example of NDA (Example \ref{ex:NDA}) take  and . Now, set  and consider the following system of mutually
recursive equations

where  are \emph{recursion variables},  is a \emph{parameter} and . A \emph{solution} assigns to each
of the two variables  an element of  such that
the formal equations  become actual identities in :

Observe that the above system of equations corresponds to an \emph{equation morphism}  and the solution to a morphism , both in . The property that  is a solution for  is expressed by the following equation in :

So  is a \emph{parametrized fixpoint operator}, i.e.~a family of fixpoint operators indexed by parameter sets .
\begin{remark} \label{rmk:doubledagger} In our
applications we shall need a certain equational property of the operator : for all  and equation morphism ,
the following equation, called \emph{double dagger law}, holds:

This and other laws of parametrized fixpoint operators have been
studied by Bloom and \'Esik in the context of \emph{iteration
  theories}~\cite{be93}. A closely related notion is that of \emph{Elgot monads}~\cite{amv10,amv11}.
\end{remark}
\begin{example}[Least fixpoint solutions]\label{ex:LfpSolCPO}
  Let  be a locally continuous monad on the \cppo-enriched category . Then  is equipped wi th a parametrized fixpoint operator obtained by taking least fixpoints: given a morphism  consider the function  on  given by . Then  is continuous and we take  to be the least fixpoint of . Since , equation~\refeq{diag:fixp} holds, and it follows from the argument in Theorem~8.2.15 and Exercise~8.2.17 in~\cite{be93} that the operator  satisfies the axioms of iteration theories (or Elgot monads, respectively). In particular the double dagger law holds for the least fixpoint operator .
\end{example}

\section{Motivating examples}\label{SSec:Mot}

The work of~\cite{HasuoJS:07} bridged a gap in the theory of coalgebras: for certain functors, taking the final coalgebra directly in  does not give the right notion of equivalence. For instance, for NDA, one would obtain bisimilarity instead of language equivalence. The change to Kleisli categories allowed the recovery of the usual language semantics for NDA and, more generally, led to the development of \emph{coalgebraic trace semantics}.


In the Introduction we argued that there are relevant examples for which this approach still yields the unwanted notion of equivalence, the problem being that it does not consider the extra algebraic structure on the label set. In the sequel, we motivate the reader for the generic theory we will develop by detailing two case studies in which this phenomenon can be observed: NDA with -transitions and NDA with word transitions. Later on, in Example \ref{Sec:MazurTraces}, we will also consider Mazurkiewicz traces~\cite{Mazurkiewicz77}.

\paragraph{NDA with -transitions.} In the world of automata, -transitions are considered in order to enable easy composition of automata and compact representations of languages. These transitions are to be interpreted as the empty word when computing the language accepted by a state. Consider, on the left, the following simple example of an NDA with -transitions, where states  and  just make  transitions. The intended semantics in this example is that  all states accept words in .


Note that, more explicitly, these are just NDA where the alphabet has a distinguished symbol . So, they are coalgebras for the functor  (where  is the functor of Example~\ref{ex:NDA}), i.e.~functions , as made explicit for the above automaton in the middle.

The final coalgebra for  is simply  and the final map  assigns to each state the language in  that it accepts. However, the equivalence induced by  is too fine grained: for the automata above,  maps ,  and  to three different languages (on the right), where the number of  plays an explicit role, but the intended semantics should disregard 's.



\paragraph{NDA with word transitions.} This is a variation on the motivating example of the introduction: instead of languages, transitions are labeled by words\footnote{More generally, one could consider labels from an arbitrary monoid.}. Formally, consider again the functor  from Example~\ref{ex:NDA}. Then NDA with word transitions are coalgebras for the functor , that is, functions . We observe that they are like NDA  but (1) transitions are labeled by words in , rather than just symbols of the alphabet , and (2) states have associated output languages, rather than just . We will draw them as ordinary automata plus an arrow  to denote the output language of a state (no  stands for the empty language). For an example, consider the following word automaton and associated transition function .

The semantics of NDA with word transitions is given by languages over , obtained by concatenating the words in the transitions and ending with a word from the output language. For instance,  above accepts word  but not .

However, if we consider the final coalgebra semantics we again have a mismatch.
The initial -algebra has carrier   that can be represented as the set of non-empty lists of words over , where  indicates possibly empty lists of words. Its structure  maps  into  and  into . Here, we use  to denote the empty list and  is the append operation. By Theorem \ref{thm:HJS},
the final -coalgebra has the same carrier and structure . The final map, as a function , is then defined by commutativity of the following square (in ):


Once more, the semantics given by  is too fine grained: in the above example,  and  whereas the intended semantics would equate both  and , since they both accept the language .

\medskip

Note that any NDA can be regarded as word automaton. Recall the natural transformation  defined in Example \ref{ex:mnds}.\ref{pt:freemonad}: for the functor  of NDA,  maps any pair  into  and  into . Composing an NDA  with , one obtains the word automaton .

In the same way, every NDA with -transitions can also be seen as a word automaton by postcomposing with the natural transformation . Here,  is the unit of the free monad
 defined on a given set  below (the multiplication  is shown on the right).

In the next section, we propose to define the semantics of -coalgebras via a canonical fixpoint operator rather than with the final map which as we saw above might yield unwanted semantics. Then, using the observation above, the semantics of -coalgebras and -coalgebras will be defined by embedding them into -coalgebras via the natural transformations  and  described above.
\section{Canonical Fixpoint Solutions}\label{sec:Theory}
In this section we lay the foundations of our approach. A construction is introduced assigning canonical solutions to coalgebras seen as equation morphisms (\emph{cf.} Section \ref{Sec:Elgot}) in a \cppo-enriched setting. We will be working under the following assumptions.
\begin{assumption}\label{ass:CPPOenrichTcont}
  Let  be a \cppo-enriched category with coproducts and  composition left-strict.
  Let  be a locally continuous monad on  such
  that, for all object , an initial algebra for  exists.
\end{assumption}

As seen in Example \ref{ex:LfpSolCPO}, in this setting an equation morphism  may be given the least solution. Here, we take  a different approach, exploiting the initial algebra-final coalgebra coincidence of Theorem~\ref{thm:Freyd}.



For every parameter object , the endofunctor  is
a locally continuous monad because it is the composition of  with the (locally continuous) exception monad . Thus, by Theorem \ref{thm:Freyd} applied to , the initial -algebra  yields a final -coalgebra . This allows us to associate with any equation morphism  a canonical morphism of type  as in the following diagram.

In \eqref{diag:canonicalSol}, the map  is the unique morphism of  -coalgebras given by finality of  , whereas  is the unique morphism of -algebras given by initiality of .

We call the composite  the \emph{canonical fixpoint solution} of . In the following we check that the canonical fixpoint solution is indeed a solution of , in fact, it coincides with the least solution. 


\newcommand{\proplfpcfpsolution}{Given a morphism , then the least solution of  as in Example \ref{ex:LfpSolCPO} is the canonical fixpoint solution:  as in \eqref{diag:canonicalSol}.}
\begin{proposition} \label{prop:lfp=cfpsolution}
\proplfpcfpsolution
\end{proposition}


As recalled in Example \ref{ex:LfpSolCPO}, the least fixpoint operator  satisfies the double dagger law. Thus Proposition \ref{prop:lfp=cfpsolution} yields the following result\footnote{The equality of least and canonical fixpoint solutions can be used to state a stronger result, namely that canonical fixpoint solutions satisfy the axioms of iteration theories (\emph{cf.} Example \ref{ex:LfpSolCPO}). However, the double dagger law is the only property that we need here, explaining the statement of Corollary \ref{cor:cansolElgot}.}

\begin{corollary}\label{cor:cansolElgot} Let  and  be as in Assumption \ref{ass:CPPOenrichTcont}. Then the canonical fixpoint operator  associated with  satisfies the double dagger law.
\end{corollary}

We now introduce a factorisation result on the operator , which is useful for comparing solutions provided by different monads connected via a monad morphism.
\newcommand{\propfact}{Suppose that  and  are monads on  satisfying Assumption \ref{ass:CPPOenrichTcont} and  is a monad morphism. For any morphism :

where  is provided by the canonical fixpoint solution for  and  by the one for .}
\begin{proposition}[Factorisation Lemma] \label{prop:factorizationLemma}
\propfact
\end{proposition}
\subsection{A Theory of Systems with Internal Behavior}

We now use canonical fixpoint solutions to provide an abstract theory of systems with internal behavior, that we will later instantiate to the motivating examples of Section \ref{SSec:Mot}. Throughout this section, we will develop our framework for the following ingredients.

\begin{assumption} \label{ass:TheoryHSystems} Let  be a \cppo-enriched category with coproducts and composition left-strict and let  be a locally continuous functor for which all free -algebras exist. Consider the following two monads derived from :
\begin{itemize}
  \item the free monad  (\emph{cf.} Example \ref{ex:mnds}.\ref{pt:freemonad}), for which we suppose that an initial -algebra exists for all ;
  \item for a fixed , the exception monad  (\emph{cf.} Example \ref{ex:mnds}.\ref{pt:exceptionmonad}), for which we suppose that an initial -algebra exists for all .
\end{itemize}
\end{assumption}
In the next proposition we verify that the construction introduced in the previous section applies to the two monads of Assumption \ref{ass:TheoryHSystems}.
\newcommand{\propass}{Let , ,   and   be as in Assumption \ref{ass:TheoryHSystems}. Then  and the monads  and  satisfy Assumption \ref{ass:CPPOenrichTcont}. Thus both  and   are monads with canonical fixpoint solution (which satisfy the double dagger law by Corollary \ref{cor:cansolElgot}).}
 \begin{proposition}\label{prop:ass2->ass1}
 \propass
 \end{proposition}
To avoid ambiguity, we denote with  the canonical fixpoint operator associated with  and with  the one associated with .

We will employ the additional structure of those two monads for the analysis of \emph{-systems with internal transitions}. An -system is simply an -coalgebra , where we take the operational point of view of seeing  as a space of states and  as the transition type of . An -system with internal transitions is an -coalgebra , where the component  of the codomain is targeted by those transitions representing the internal (non-interacting) behavior of system .

A key observation for our analysis is that -systems---with or without internal transitions---enjoy a standard representation as -systems, that is, coalgebras of the form .
\begin{definition}[-systems as -systems] \label{def:reprHtofreeH} Let  be as in Example~\ref{ex:mnds}.\ref{pt:freemonad}. We introduce the following encoding  of -systems and -systems with internal transitions as -systems.
\begin{itemize}
  \item Given an -system , define  as

\item Given an -system with internal transitions , define  as .
  \end{itemize}
\end{definition}
Thus -systems (with or without internal transitions) may be seen as equation morphisms  for the monad  (with the initial object  as parameter), with solutions by canonical fixpoint ({\em cf.}~Section~\ref{Sec:Elgot}). This will allow us to achieve the following.
\begin{enumerate}[label={(\arabic*)}]
  \item[\bf \S 1] We supply a uniform trace semantics for -systems, possibly with internal transitions, and -systems, based on the canonical fixpoint solution operator of . \item[\bf \S 2] We use the canonical fixpoint operator of  to transform any -system  with internal transitions into an -system  without internal transitions. \item[\bf \S 3] We prove that the transformation of {\bf \S 2} is sound with respect to the semantics of {\bf \S 1}.
\end{enumerate}
\paragraph{\bf \S 1: Uniform trace semantics.}  The canonical fixpoint semantics of -systems, with or without internal transitions, and -systems is defined as follows.
 \begin{definition}[Canonical Fixpoint Semantics] \label{def:canfixSem}
\begin{itemize}
  \item For an -system , its semantics  is defined as 
(note that  can be seen as  an equation morphism for  on parameter ).
\item For an -system , its semantics  is defined as .
\item For an -system with internal transitions , its semantics  is defined as .
\end{itemize}
\end{definition}
The underlying intuition of Definition \ref{def:canfixSem} is that canonical fixpoint solutions may be given an operational understanding. Given an -system , its solution  is formally defined as the composite  (\emph{cf.} \eqref{diag:canonicalSol}): we can see the coalgebra morphism  as a map that gives the \emph{behavior} of system  without taking into account the structure of labels and the algebra morphism 
as evaluating this structure, e.g. flattening words of words, using the initial algebra  for the monad . In particular, the action of  is what makes our semantics suitable for modeling ``algebraic'' operations on internal transitions such as -elimination, as we will see in concrete instances of our framework.

\begin{remark} The canonical fixpoint semantics of Definition \ref{def:canfixSem} encompasses the framework for traces in \cite{HasuoJS:07}, where the semantics of an -system ---without internal transitions---is defined as the unique morphism  from  into the final -coalgebra . Indeed, using finality of , it can be shown that . Theorem~\ref{th:comparewithHJS} below guarantees compatibility with Assumption~\ref{ass:TheoryHSystems}.
\end{remark}
The following result is instrumental in our examples and in comparing our theory with the one developed in \cite{HasuoJS:07} for trace semantics in Kleisli categories.
\newcommand{\thcomparewithHJS}{Let  be a monad and  be a functor satisfying the assumptions of Theorem \ref{thm:HJS}, that is:
\begin{enumerate}[(a)]
 \item  is \cppo-enriched and composition is left strict;
 \item  is accessible and has a locally continuous lifting .
\end{enumerate}
Then , ,  and  (for a given set ) satisfy Assumption \ref{ass:TheoryHSystems}.}
\begin{theorem}\label{th:comparewithHJS}
\thcomparewithHJS
\end{theorem}
\begin{example}[Semantics of NDA with word transitions]
In Section \ref{SSec:Mot}, we have modeled NDA with word transitions as -coalgebras on , where  and  are defined as for NDA (see Example \ref{ex:NDA}). By Proposition~\ref{prop:liftingfreemonad},  and thus, by virtue of Theorem \ref{th:comparewithHJS},  satisfies Assumption \ref{ass:TheoryHSystems}. Therefore we can define the semantics of NDA with word transitions  via canonical fixpoint solutions as :

Observe that the above diagram is just \eqref{diag:canonicalSol} instantiated with  and . Moreover, this diagram is in  and hence the explicit definition of  as a function  is given by .

Both  and  can be defined uniquely by the commutativity of the above diagram.
We have already defined  in diagram~\eqref{eq:finalmap_wordaut} and the definition of  is given in the right-hand square of the above diagram. The isomorphism in the middle and  were defined in Section~\ref{SSec:Mot}.

Using the above formula  we now have the semantics of :

This definition is precisely the language semantics: a word  is accepted by a state  if there exists a decomposition  such that .
Take again the automaton of the motivating example. We can calculate the semantics and observe that we now get exactly what was expected: .

The key role played by the monad structure on  can be appreciated by comparing the graphs of  and  as in the example above. The algebra morphism  maps values from the initial algebra  for the \emph{endofunctor}  into the initial algebra  for the \emph{monad} : its action is precisely to take into account the additional equations encoded by the algebraic theory of the monad . For instance, we can see the mapping of  into the word  as the result of concatenating the words , ,  and then quotienting out of the equation  in the monoid .
\end{example}
\begin{remark}[Multiple Solutions]\label{rm:multiple}
The canonical solution  is not the unique solution. Indeed, the uniqueness of  in the left-hand square and of  in the right-hand square of the diagram above does not imply the uniqueness of . To see this, take for instance the automaton

Both  and  are solutions. The canonical one is the least one, i.e., .
\end{remark}
\begin{example}[{Semantics of NDA with -transitions}]\label{ex:epselim}
NDA with -transitions are modeled as -coalgebras on , where  and  are defined as for NDA (see Example~\ref{ex:NDA}). We can define the semantics of NDA with -transitions via canonical fixpoint solutions as , where  is the automaton with word transitions corresponding to  (see Definition~\ref{def:reprHtofreeH}). The first example in Section~\ref{SSec:Mot}
would be represented as follows,

where  and  are defined as at the end of Section~\ref{SSec:Mot}.
By using \eqref{eq:language}, it can be easily checked that the semantics  maps  ,  and  into .
\end{example}
\paragraph{\bf \S 2: Elimination of internal transitions.}
We view an -system  with internal transitions as an equation morphism for the monad , with parameter . Thus we can use the canonical fixpoint solution of  to obtain an -system , which we denote by . The construction is depicted below.

\begin{example}[-elimination]
Using the automaton of Example~\ref{ex:epselim}, we can perform -elimination, as defined in~\eqref{diag:cansolHX+Id}, using the canonical solution for the monad :

We obtain the following NDA .

The semantics  is defined as , where  is the representation of the NDA  as an automaton with word transitions (Definition \ref{def:reprHtofreeH}). It is immediate to see, in this case, that . This fact is an instance of Theorem \ref{th:epselimsound} below.
\end{example}
\begin{remark} Note that -elimination was recently defined using a trace operator on a Kleisli category~\cite{Hasuo06,SW13,Asada}. These works are based on the trace semantics of Hasuo et al.~\cite{HasuoJS:07} and tailored for -elimination. They do not take into account any algebraic structure of the labels and are hence not applicable to the other examples we consider in this paper. \end{remark}
\paragraph{\bf \S 3: Soundness of -elimination.}
We now formally prove that the canonical fixpoint semantics of  and  coincide. To this end, first we show how the construction  can be expressed in terms of the canonical fixpoint solution of . This turns out to be an application of the factorisation lemma (Proposition \ref{prop:factorizationLemma}), for which we introduce the natural transformation  defined at  by

Since  is a monad with canonical fixpoint solutions, it can be verified that so is . Moreover,  is a monad morphism between  and . These observations allow us to prove the following.

\newcommand{\propfactEpsilonElim}{ For any -system  with internal transitions, consider the equation morphism . Then:
}
\begin{proposition}[Factorisation property of ] \label{prop:factEpsilonElim}
\propfactEpsilonElim
\end{proposition}
\begin{proof}
  This follows simply by an application of Proposition~\ref{prop:factorizationLemma} to  and  with . \qed
\end{proof}


\noindent We are now in position to show point {\bf \S 3}: soundness of -elimination.
\newcommand{\propEpsilonElimSound}{For any -system  with internal transitions,

}
\begin{theorem}[Eliminating internal transitions is sound] \label{th:epselimsound}
\propEpsilonElimSound
\end{theorem}
\begin{proof} The statement is shown by the following derivation.

\qed
\end{proof}
\section{Quotient Semantics}\label{ssec:quot}
When considering behavior of systems it is common to encounter spectrums of successively coarser equivalences. For instance, in basic process algebra trace equivalence can be obtained by quotienting bisimilarity with an axiom stating the distributivity of action prefixing by non-determinism~\cite{Rabinovich93}. There are many more examples of this phenomenon, including Mazurkiewicz traces, which we will describe below.

In this section we develop a variant of the canonical fixpoint semantics, where we can encompass in a uniform manner behaviors which are quotients of the canonical behaviors of the previous section (that is, the object ).
\begin{assumption}\label{ass:quotient} Let , ,  and  be as in Assumption \ref{ass:TheoryHSystems} and  a monad quotient for some monad . Moreover, suppose that for all  an initial -algebra exists.
\end{assumption}
Observe that, as Assumption \ref{ass:quotient} subsumes Assumption \ref{ass:TheoryHSystems}, we are within the framework of previous section, with the canonical fixpoint solution of  providing semantics for - and -systems. For our extension, one is interested in  as a semantic domain coarser than  and we aim at defining an interpretation for -systems in . To this aim, we first check that  has canonical fixpoint solutions.
\newcommand{\propQElgot}{Let , ,  and  be as in Assumption \ref{ass:quotient}. Then Assumption \ref{ass:CPPOenrichTcont} holds for  and , meaning that  is a monad with canonical fixpoint solutions (which satisfy the double dagger law by Corollary \ref{cor:cansolElgot}).}
\begin{proposition}\label{prop:QElgot}
\propQElgot
\end{proposition}
We use the notation  for the canonical fixpoint operator of . This allows us to define the semantics of -systems, analogously to what we did for -systems in Definition \ref{def:canfixSem}. Moreover, the connecting monad morphism  yields an extension of this semantics to include also systems of transition type  and .
\begin{definition}[Quotient Semantics] \label{def:quotSem}
The quotient semantics of -systems, with or without internal transitions, -systems and -systems is defined as follows.
\begin{itemize}
  \item For a -system , its semantics  is defined as  (note that  can be regarded as an equation morphism for  with ).
\item For an -system , its semantics  is defined as .
\item For an -system ---with or without internal transitions---its semantics  is defined as , where  is as in Definition \ref{def:reprHtofreeH}.
\end{itemize}
\end{definition}
The Factorisation Lemma (Proposition \ref{prop:factorizationLemma}) allows us to establish a link between the canonical fixpoint semantics  and the quotient semantics .
\newcommand{\propfactQuotient}{ Let  be either an -system or an -system (with or without internal transitions). Then:
}
\begin{proposition}[Factorisation for the quotient semantics] \label{prop:factQuotient}
\propfactQuotient
\end{proposition}
As a corollary we obtain that eliminating internal transitions is sound also for quotient semantics.
\newcommand{\corSoundQuot}{
For any -system  with internal transitions,

}
\begin{corollary} \label{for:sound-quot}
\corSoundQuot
\end{corollary}
The quotient semantics can be formulated in a Kleisli category  by further assuming  below. This is needed to lift a quotient of monads from  to .

\newcommand{\thquotcomparewithHJS}{Let  be a monad and  be an accessible functor satisfying the assumptions of Theorem \ref{thm:HJS}. By Proposition \ref{prop:liftingfreemonad} the free monad  on  lifts to a monad  via a distributive law  with . Let  be a monad and  a monad quotient such~that
\begin{itemize}
\item[(c)] for each set , there is a map
  making the following commute.
 
 \end{itemize}
Then the following hold:
\begin{enumerate}
\item there is a monad  lifting  and a monad morphism  defined as ; \label{pt:QuotKleisli2}
 \item , , ,  (for a given set ),  and  satisfy Assumption \ref{ass:quotient}. \label{pt:QuotKleisli3}
     \end{enumerate}}
\begin{theorem}\label{th:quotcomparewithHJS}
\thquotcomparewithHJS
\end{theorem}

Notice that condition~(c) and the first part of statement \ref{pt:QuotKleisli2}~are
related to~\cite[Theorem~1]{BHKR13}; however, that paper treats distributive
laws of monads over endofunctors.

\begin{example}[Mazurkiewicz traces] \label{Sec:MazurTraces}
This example, using a known equivalence in concurrency theory, illustrates the use of the quotient semantics developed in Section~\ref{ssec:quot}.

The trace semantics proposed by Mazurkiewicz~\cite{Mazurkiewicz77} accounts for concurrent actions. Intuitively, let  be the action alphabet and . We will call  and  concurrent, and write , if the order in which these actions occur is not relevant. This means that we equate words that only differ in the order of these two actions, e.g.~ and  denote the same Mazurkiewicz trace.

To obtain the intended semantics of Mazurkiewicz traces we use the quotient semantics defined above\footnote{Mazurkiewicz traces were defined over labelled transition systems which are similar to NDA but where every state is final. For simplicity, we consider LTS here immediately as NDA.}. In particular, for Mazurkiewisz traces one considers a symmetric and irreflexive ``independence'' relation  on the label set . Let  be the least congruence relation on the free monoid  such that


We now have two monads on , namely  and . There is the canonical quotient of monads  given by identifying words of the same -equivalence class. It can be checked that those data satisfy the assumptions of Theorem \ref{th:quotcomparewithHJS} and thus we are allowed to apply the quotient semantics .
This will be given on an NDA  by first embedding it into  as  and then into  as .
To this morphism we apply the canonical fixpoint operator of  to obtain , that is, the semantics .
It is easy to see that this definition captures the intended semantics: for all states 

Indeed, by Proposition \ref{prop:factQuotient},  and  is just  where  maps every word  into its equivalence class .
\end{example}

\section{Discussion}\label{Sec:Discussion}
The framework introduced in this paper provides a uniform way to express the semantics of systems with internal behaviour via canonical fixpoint solutions. Moreover, these solutions are exploited to eliminate internal transitions in a sound way, i.e., preserving the semantics. We have shown our approach at work on NDA with -transitions but, by virtue of Theorem \ref{th:comparewithHJS}, it also covers all the examples in~\cite{HasuoJS:07} (like probabilistic systems) and more (like the weighted automata on positive reals of~\cite{SW13}).

It is worth noticing that, in principle, our framework is applicable also to examples that do not arise from Kleisli categories. Indeed the theory of Section~\ref{sec:Theory} is formulated for a general category : Assumption \ref{ass:TheoryHSystems} only requires  to be -enriched and the monad  to be locally continuous. The role of these assumptions is two-fold: (a) ensuring the initial algebra-final coalgebra coincidence and (b) guaranteeing that the canonical fixpoint operator  satisfies the \emph{double dagger law}. If (a) implies (b), we could have formulated our theory just assuming the coincidence of initial algebra and final coalgebra and without any -enrichment. Condition~(a) holds for some interesting examples not based on Kleisli categories, e.g. for examples in the category of join semi-lattices. Therefore it is of relevance to investigate the following question: given a monad  with initial algebra-final coalgebra coincidence, under which conditions does the canonical fixpoint solution provided by  satisfy the double dagger law?

As a concluding remark, let us recall that our original question concerned the problem of modeling the semantics of systems where labels carry an algebraic structure.
In this paper we have mostly been focusing on automata theory, but there are many other examples in which the information carried by the labels has relevance for the semantics of the systems under consideration:
in logic programming labels are substitutions of terms; in (concurrent) constraint programming they are elements of a lattice; in process calculi they are actions representing syntactical contexts and in tile systems~\cite{DBLP:conf/birthday/GadducciM00} they are morphisms in a category. We believe that our approach provides various insights towards a coalgebraic semantics for these computational models.

\paragraph{Acknowledgments.}
We are grateful to the anonymous referees for valuable comments.
The work of Alexandra Silva is partially funded by the ERDF through the
Programme COMPETE and by the Portuguese Foundation for Science and
Technology, project ref.~\texttt{FCOMP-01-0124-FEDER-020537} and
\texttt{SFRH/BPD/}\texttt{71956/2010}.
The first and the fourth author acknowledge support by project \texttt{ANR 12IS0} \texttt{2001 PACE}.

\bibliographystyle{splncs03}
\begin{thebibliography}{10}
\providecommand{\url}[1]{\texttt{#1}}
\providecommand{\urlprefix}{URL }

\bibitem{adamek:74}
Ad\'{a}mek, J.: Free algebras and automata realizations in the language of
  categories. Comment.~Math.~Univ.~Carolin.  15,  589--602 (1974)

\bibitem{amv10}
Ad\'amek, J., Milius, S., Velebil, J.: Equational properties of iterative
  monads. Inform.~and Comput.  208,  1306--1348 (2010),
  doi:10.1016/j.ic.2009.10.006

\bibitem{amv11}
Ad\'amek, J., Milius, S., Velebil, J.: Elgot theories: a new perspective of the
  equational properties of iteration. Math.~Structures Comput.~Sci.  21(2),
  417--480 (2011)

\bibitem{ap:04}
Ad\'amek, J., Porst, H.E.: On tree coalgebras and coalgebra presentations.
  Theoret.~Comput.~Sci.  311,  257--283 (2004)

\bibitem{Asada}
Asada, K., Hidaka, S., Kato, H., Hu, Z., Nakano, K.: A parameterized graph
  transformation calculus for finite graphs with monadic branches. In:
  Pe{\~n}a, R., Schrijvers, T. (eds.) PPDP. pp. 73--84. ACM (2013)

\bibitem{BalanK11}
Balan, A., Kurz, A.: On coalgebras over algebras. Theoret.~Comput.~Sci. 412(38),
   4989--5005 (2011)

\bibitem{barr:70}
Barr, M.: Coequalizers and free triples. Math.~Z.  116,  307--322 (1970)

\bibitem{be93}
Bloom, S.L., \'Esik, Z.: Iteration Theories: the equational logic of iterative
  processes. EATCS Monographs on Theoretical Computer Science, Springer (1993)

\bibitem{BHKR13}
Bonsangue, M.M., Hansen, H.H., Kurz, A., Rot, J.: Presenting distributive laws.
  In: Heckel and Milius  \cite{DBLP:conf/calco/2013}, pp. 95--109

\bibitem{DBLP:conf/mfps/1993}
Brookes, S.D., Main, M.G., Melton, A., Mislove, M.W., Schmidt, D.A. (eds.): Proceedings MFPS 1993. Lecture
  Notes in Computer Science, vol. 802. Springer (1994)

\bibitem{Freyd}
Freyd, P.J.: Remarks on algebraically compact categories. London Mathematical Society Lecture Notes
  Series, vol. 177. Cambridge University Press (1992)

\bibitem{DBLP:conf/birthday/GadducciM00}
Gadducci, F., Montanari, U.: The tile model. In: Plotkin, G.D., Stirling, C.,
  Tofte, M. (eds.) Proof, Language, and Interaction. pp. 133--166. The MIT
  Press (2000)

\bibitem{Hasuo06}
Hasuo, I., Jacobs, B., Sokolova, A.: Generic forward and backward simulations.
  In: (Partly in Japanese) Proceedings of {\em JSSST Annual Meeting} (2006)

\bibitem{HasuoJS:07}
Hasuo, I., Jacobs, B., Sokolova, A.: Generic trace semantics via coinduction.
  Log. Methods Comput. Sci.  3(4:11),  1--36 (2007)

\bibitem{DBLP:conf/calco/2013}
Heckel, R., Milius, S. (eds.): Proceedings CALCO 2013. Lecture Notes in Computer Science, vol. 8089. Springer (2013)

\bibitem{Hopcroft}
Hopcroft, J., Motwani, R., Ullman, J.: Introduction to Automata Theory,
  Languages, and Computation (3rd Edition). Wesley (2006)

\bibitem{Johnstone75}
Johnstone, P.: Adjoint lifting theorems for categories of algebras.
  Bull.~London Math.~Soc.  7,  294--297 (1975)

\bibitem{kelly:80}
Kelly, G.M.: A unified treatment of transfinite constructions for free
  algebras, free monoids, colimits, associated sheaves, and so on.
  Bull.~Austral.~Math.~Soc.  22,  1--83 (1980)

\bibitem{Mazurkiewicz77}
Mazurkiewicz, A.: Concurrent Program Schemes and Their Interpretation. Aarhus
  University, Comp. Science Depart., DAIMI PB-78 (July 1977)

\bibitem{MacLane71}
\mbox{Mac Lane}, S.: Categories for the Working Mathematician. Springer, Berlin
  (1971)

\bibitem{mps:09}
Milius, S., Palm, T., Schwencke, D.: Complete iterativity for algebras with
  effects. In: Kurz et al. (eds.) Proc. CALCO'09. LNCS,
  vol. 5728, pp. 34--48. Springer (2009)

\bibitem{Mulry93}
Mulry, P.S.: Lifting theorems for kleisli categories. In: Brookes et~al.
  \cite{DBLP:conf/mfps/1993}, pp. 304--319

\bibitem{Rabinovich93}
Rabinovich, A.M.: A complete axiomatisation for trace congruence of finite
  state behaviors. In: Brookes et~al.  \cite{DBLP:conf/mfps/1993}, pp. 530--543

\bibitem{SW13}
Silva, A., Westerbaan, B.: A coalgebraic view of -transitions. In: Heckel and
  Milius  \cite{DBLP:conf/calco/2013}, pp. 267--281

\bibitem{Sobocinski2012}
Soboci\'{n}ski, P.: Relational presheaves as labelled transition systems. In:
Proc.~Coalgebraic Methods in Computer Science ({CMCS'12}). LNCS, vol. 7399, pp.
  40--50. Springer (2012)

\bibitem{plotkin-semop}
Turi, D., Plotkin, G.: {Towards a Mathematical Operational Semantics}. In:
  {Proc. Logic in Computer Science (LICS'97)}. IEEE Computer Society ({1997})

\end{thebibliography}

\newpage
\appendix
\appendix



\section{Proofs of Section~\ref{Sec:Trace}}
In this appendix, we show the proofs of Proposition~\ref{prop:liftingfreemonad} and \ref{lem:Hstar}.
The proofs of the other results shown in Section~\ref{Sec:Trace} can be found in the referred literature.

\begin{proposition_for}{prop:liftingfreemonad}
\propliftingfreemonad
\end{proposition_for}
\begin{proof}

Let  be the distributive law of the functor  over the monad  corresponding to the lifting  (see Proposition~\ref{LiftProp}).
For an object , we define  by the universal property of the initial -algebra .

By diagram chasing, one can show that  is a distributive law of the monad 
over the monad  and, by Proposition~\ref{LiftProp}, we have a lifting .

For proving ,
take  to be the initial -algebra and observe that  is the initial
-algebra (Proposition~\ref{prop:liftinginitialalgebra}). The fact that the units and the multiplications of  and  coincide is immediately proved by functoriality of .
\qed
\end{proof}




\begin{proposition_for}{lem:Hstar}
 \lemHstar
\end{proposition_for}
\begin{proof}
  First recall that  is a free -algebra with the
  structure  and the universal morphism 
  (cf.~Example~\ref{ex:mnds}(5)). Equivalently,  is an initial algebra for
  . Given a morphism ,  is defined by
  initiality; more precisely,  is the unique morphism such
  that the following diagram commutes:
  
  Now recall that  is an isomorphism and consider the following function
  
  with
  
  Since  is locally continuous, we see that  is continuous (in
  both arguments). Clearly,  is the unique fixpoint of . To see that  is
  locally continuous let  be an -chain in . It is easy to see that 
  is a fixpoint of ; indeed we
  have (using continuity of ):
  
  Thus, by the uniqueness of the fixpoint  we have
  
  as desired. \qed
\end{proof}

Finally, we record a simple lemma for future use:

\begin{proposition}
  \label{lem:quot}
 Let  be a quotient functor of the locally continuous functor  on the \cppo-enriched category . Then  is locally continuous, too.
\end{proposition}
\begin{proof}
 Suppose that  is an epi natural transformation.
 Consider an -chain  in . To prove that  we show that

and we use that  is epi.
\qed
\end{proof}










\section{Proofs of Section~\ref{sec:Theory}}\label{App:proofsTheory}

In this appendix, we report the proofs of the results stated in Section~\ref{sec:Theory}, apart from Theorem~\ref{th:comparewithHJS} that we prove separately in the next appendix.

\begin{proposition_for}{prop:lfp=cfpsolution}
\proplfpcfpsolution
\end{proposition_for}
\begin{proof} It suffices to show that  is the least fixpoint of the continuous function  on , defined as in Example~\ref{ex:LfpSolCPO}. To this aim, first observe that the least fixpoint of  can be obtained as the -join

where  and .

An analogous observation can be made for the coalgebra morphism . By finality of ,  is the unique map making the left-hand square in \eqref{diag:canonicalSol} commute. In particular, it is the \emph{least} function---in the cpo ---to do so: thus it is the least fixpoint of a continuous function, expressed by the -join

 where  and .
 Analogously, by initiality of ,  is the unique---and thus the least---fixpoint of a continuous function on , calculated as follows:

 where  and .

\noindent We now show by induction on  that :
\begin{itemize}
  \item for , by left-strictness of composition we have
      
  \item For the inductive step, consider the following derivation:
      
\end{itemize}

\noindent Thus we are allowed to conclude:

where the third equality is given by continuity of composition in .\qed
\end{proof}

\begin{remark}\label{rm:uniqueness} In the proof of Proposition~\ref{prop:lfp=cfpsolution} one observes that both  and  are \emph{unique} fixpoints for the continuous functions on  and , respectively, corresponding to commutativity of the two inner squares in \eqref{diag:canonicalSol}. Nonetheless, the same is not true for their composite , which we just prove to be the \emph{least} solution for : there are possibly other maps making the outer rectangle in \eqref{diag:canonicalSol} commute (cf. Remark~\ref{rm:multiple}).
\end{remark}

 \begin{proposition_for}{prop:factorizationLemma}
 \propfact
 \end{proposition_for}
\begin{proof} First we construct the canonical fixpoint solution for  and . The former will factor through the initial -algebra  and the latter through the initial -algebra  as in the diagram:

The statement of the Proposition amounts to show that the top face of the diagram commutes, that is,

We are going to prove \eqref{eq:fact} by exploiting the initiality of  and finality of . For this purpose, it is convenient to make the following observation:
\begin{itemize}
  \item[] any -algebra  canonically induces a -algebra   and the same---by naturality of ---for algebra homomorphisms. Dually, any -coalgebra  canonically induces a -algebra  and the same for coalgebra homomorphisms.
      
\end{itemize}
By observation ,  has a -algebra structure and thus by initiality there is a unique -algebra morphism . Then our claim \eqref{eq:fact} reduces to the commutativity of the following diagram.
      
We address commutativity of  and of  separately.
\begin{itemize}
  \item[--] By observation ,  has also a -algebra structure. Then, by initiality of , for commutativity of  it suffices to show that  and  are -algebra morphisms. For this purpose, first observe that by construction  and  are -algebra morphism and the same for  in virtue of observation . Hence it suffices to prove that also  is a -algebra morphism. That is given by commutativity of the following diagram
        
      where  and  commute because  is a monad morphism and  by naturality of .
  \item[--] To show that also  in \eqref{diag:fact12} commutes, we first check that  is also a -coalgebra morphism:
        
      In the diagram above, the pentagon with angle  commutes because  is a -algebra morphism, whereas the pentagon with angle  commutes by naturality of  applied to the maps  and .

      To conclude, observe that  (by construction) and  (by observation ) are also -coalgebra morphisms. Thus  by finality of , meaning that  in \eqref{diag:fact12} commutes.
\end{itemize}\qed
\end{proof}


 \begin{proposition_for}{prop:ass2->ass1}
 \propass
 \end{proposition_for}
 \begin{proof} We check that the two monads satisfy Assumption~\ref{ass:CPPOenrichTcont}. For all , the condition on the existence of initial algebras for the endofunctors  and  is already guaranteed by Assumption~\ref{ass:TheoryHSystems}. It remains to show local continuity. As  is locally continuous and all free -algebras exist, the monad  is also locally continuous by Proposition~\ref{lem:Hstar}. Local continuity of  is immediate by the fact that all copairing maps  in the -enriched category  are continuous (\emph{cf.} Section \ref{sec:cppo}).
 \qed
 \end{proof}


 \begin{proposition_for}[Factorisation property of ]{prop:factEpsilonElim}
\propfactEpsilonElim
\end{proposition_for}
\begin{proof}
Let us use the notation  for the canonical fixpoint solution operator of . We now apply Proposition~\ref{prop:factorizationLemma} to show that solutions of  factorize through the ones of . The connecting monad morphism is , defined above. Proposition~\ref{prop:factorizationLemma} yields the following factorisation property:
\begin{itemize}
\item[] for any  and equation morphism , consider . The solution  provided by  factorises as , where  is the solution provided by  to .
\end{itemize}
If we fix  and , then  says: for any -system  with internal computation, consider the equation morphism  for  with parameter . Then the following diagram commutes:

To conclude our argument, we observe that the the system  can be also seen as an equation for  with parameter . This means that also  provides a solution to such equation, which can be checked to coincide with the one given by , that is, . Then the main statement is proven by be the following derivation:
 \qed
\end{proof}


\section{Proof of Theorem~\ref{th:comparewithHJS}}\label{app:proofthComparewithHJS}

This section is devoted to prove Theorem~\ref{th:comparewithHJS}. To this aim, we first give more details on accessible endofunctors and how they yield a canonical free algebra construction.

\begin{remark}
  \label{rem:chain}
  \begin{enumerate}[(1)]
  \item \label{pt:bounded} Ad\'amek and Porst~\cite{ap:04} showed that an endofunctor  on  is
    accessible iff is it bounded in the following sense: there exists
    a cardinal  such that for every set , every element of
     lies in the image of  for some 
    of less than  elements.

  \item \label{pt:HaccessibleFreeAlg} Recall from~\cite{adamek:74} that for an accessible endofunctor
     on a cocomplete category  (not only the initial but) all
    \emph{free} -algebras exist and are obtained from an inductive
    construction. More precisely, for every object  of  define
    the following ordinal indexed \emph{free-algebra-chain}:
    
    Its connecting morphisms  are
    uniquely determined by
    
    Indeed, this defines an ordinal indexed chain uniquely (up to
    isomorphism). The ``missing'' connecting maps are determined by
    the universal property of colimits, e.g.~ is
    unique such that  for all .

    Now suppose that  preserves -filtered
    colimits. Then  is an isomorphism and one
    can show that  is a free -algebra on  with the
    structure and universal morphism given by
    .
  \item \label{pt:freeHaccessible} As we saw previously, the assignment of a free -algebra on 
    to any object  yields a free monad on ; thus, in item~(2)
    above we have . Now notice that the construction
    in the previous point can be written object free; we obtain 
    after  steps of the following chain in the category of
    endofunctors on :
    
    The connecting natural transformations  have the
    components described as connecting morphisms in item~(2).

    As a consequence we see that if  is accessible then so is
    ; indeed, all   preserve -filtered colimits if
     does.
  \end{enumerate}
\end{remark}

The next Proposition is instrumental in relating accessiblity of an endofunctor with the existence of initial algebras for its lifting.

\newcommand{\propliftingCoproduct}{Let  a cocomplete category,  be a monad and  be an accessible endofunctor with a lifting . Then for all  both the initial -algebra and the initial -algebra exist.}
\begin{proposition} \label{prop:liftingCoproduct}
\propliftingCoproduct
\end{proposition}
\begin{proof} As the left adjoint  is defined as the identity on objects, without loss of generality we can prove our statement for an object , where .

First we observe that the endofunctor  (\emph{cf.} Example~\ref{ex:mnds}.\ref{pt:exceptionmonad}) always has a lifting to . Indeed, because the left adjoint  preserves coproducts, we have
 
 implying that  is a lifting of .

 Now we can compose the -endofunctors  and  in two different ways, obtaining  and . It is straightforward to check that the composite of two liftings is a lifting of the composite functor. This means that we have liftings  and  respectively of  and .

 The next step is to use accessibility to get initial algebras in  that will be then lifted to . To this aim, we observe that both functors  and  are accessible, because the functor  is clearly accessible and  is assumed to have this property.

Thus as observed in Remark~\ref{rem:chain}.\ref{pt:HaccessibleFreeAlg} both an initial -algebra and an initial -algebra exist. Then Proposition~\ref{prop:liftinginitialalgebra} yields the existence both of an initial -algebra and an initial -algebra. \qed
\end{proof}

We are now ready to supply a proof of Theorem \ref{th:comparewithHJS}.

\begin{theorem_for}{th:comparewithHJS}
\thcomparewithHJS
\end{theorem_for}
\begin{proof}
Since  inherits coproducts from , we only need to check the following properties:
\begin{enumerate}
  \item all free -algebras exist;
  \item for all , the initial -algebra exists;
  \item for all , the initial -algebra exists.
\end{enumerate}
In virtue of Proposition~\ref{prop:liftingCoproduct}, the three properties are implied respectively by the following statements:
\begin{enumerate}
  \item the functor  is accessible;
  \item the functor  is the lifting of  and  is accessible;
  \item the functor  is the lifting of  and  is accessible.
\end{enumerate}
The first point is given by assumption. For the second point,  is accessible by Remark~\ref{rem:chain}.\ref{pt:freeHaccessible} and  is its lifting by Proposition~\ref{prop:liftingfreemonad}. For the third point, since the identity  and the constant functor  are clearly accessible and coproducts preserve this property, then  is also accessible. As the left adjoint  preserves coproducts, it is immediate to check that  is the lifting of . Indeed:

This concludes the proof of the three properties above. \qed
\end{proof}





\section{Proofs of Section \ref{ssec:quot}}

In this appendix, we report the proofs of the results stated in Section~\ref{ssec:quot}, apart from Theorem~\ref{th:quotcomparewithHJS} that we prove separately in the next appendix.


\begin{proposition_for}{prop:QElgot}
\propQElgot
\end{proposition_for}
\begin{proof} We need to check the following:
 \begin{enumerate}
   \item for all  an initial -algebra exist and
   \item  is locally continuous.
 \end{enumerate}
 The first point is given by Assumption~\ref{ass:quotient}. For the second point, we already checked with Proposition~\ref{prop:ass2->ass1} that our assumptions on  and  imply that  is locally continuous. Then, by Proposition~\ref{lem:quot},  has the same property. \qed
\end{proof}



\begin{proposition_for}[Factorisation for the quotient semantics]{prop:factQuotient}
\propfactQuotient
\end{proposition_for}
\begin{proof} We instantiate the statement of Proposition~\ref{prop:factorizationLemma} to the monads ,  and the monad morphism . It amounts to commutativity of the following diagram for a given -system  and the parameter .

Thus for -systems the equality \eqref{eq:factQ} is immediate, because  by Definition~\ref{def:quotSem} and  by commutativity of \eqref{diag:factfreeHQ}.

Starting instead from an -system  based on state space , with or without internal computations, consider the following chain of equalities:

The first and third equalities are given by unfolding the definition of  and , whereas the second one is due to commutativity of \eqref{diag:factfreeHQ} applied to the -system  in place of . \qed
\end{proof}

\begin{corollary_for} {for:sound-quot}
\corSoundQuot
\end{corollary_for}
\begin{proof} The statement is immediately given by the following derivation

where the first and third equalities hold by Proposition \ref{prop:factQuotient} and the second equality by Theorem \ref{th:epselimsound}. \qed
\end{proof}





\section{Proof of Theorem~\ref{th:quotcomparewithHJS}}
Finally, we can prove Theorem~\ref{th:quotcomparewithHJS}. The following lemma provides sufficient conditions for lifting the quotient of an endofunctor to .
\begin{proposition}\label{lemma:quotient}
Let  be monads such that there exists a distributive law  and
 let  be the corresponding lifting.
 Let  be a monad quotient  such that
\begin{itemize}
\item[(c)] for each , there is a map
  making the following commute.
 
 \end{itemize}
 Then  lifts to a monad
  and  defined as  is a monad quotient.
\end{proposition}
\begin{proof}
We first prove that  given by  is a natural transformation.
Let  be a morphism in . As each -component is epi, it suffices to check that . For this purpose we construct the following cube.
 
The bottom face commutes by naturality of ; the leftmost and the righmost faces commute by naturality of ; the backward and the front face commute because of . It is therefore easy to see that .

Now, we prove that  is a distributive law of monads. The argument for the four diagrams is analogous, so we just show the one for , depicted in the triangle , below.

Observe that  commutes by naturality of ,  commutes since  is a distributive law of monads and  commute by .
Therefore the first equality of the following equation holds  and the second equality holds by naturality of . The commutativity of  follows since  is epi.

By Proposition~\ref{LiftProp}, and the fact that , then  has a monad lifting .

We now prove that  is a monad morphism. First, we need to check that it is a natural transformation, that is for all morphisms  in , the following diagram commutes.

By spelling out the definitions of  and , the above diagram corresponds to the following in .

Observe that  and  commute by naturality of ,  commutes by naturality of  and  commutes by .

Verifying that  is a also morphism of monads is immediate:

and .

All its components are epi since  is a left adjoint and thus preserves epis.
\qed
\end{proof}




\begin{theorem_for}{th:quotcomparewithHJS}
\thquotcomparewithHJS
\end{theorem_for}
\begin{proof} The conditions of Point~\ref{pt:QuotKleisli2} are guaranteed by Proposition~\ref{lemma:quotient}. In particular, the morphism  is of the right type because  by Proposition~\ref{prop:liftingfreemonad}. For point~\ref{pt:QuotKleisli3} we observe that, for , ,  and , proving Assumption~\ref{ass:quotient} amounts to show Assumption~\ref{ass:TheoryHSystems}, which we already did in Theorem~\ref{th:comparewithHJS}.

Thus it only remains to prove that for all  an initial -algebra exists. In virtue of Proposition~\ref{prop:liftingCoproduct}, it suffices to show that  is accessible. The accessibility of the quotient  of  is guaranteed from the fact that  is accessible (Remark~\ref{rem:chain}\ref{pt:freeHaccessible}) and thus bounded (Remark~\ref{rem:chain}\ref{pt:bounded}) and that the quotients of bounded functors are also bounded.
\qed
\end{proof}

\section{Modeling Mazurkiewicz Trace Semantics}\label{app:mazurAss}

 The following statement allows to apply the framework of quotient semantics (Section \ref{ssec:quot}) to the modeling of Mazurkiewicz trace semantics (Section \ref{Sec:MazurTraces}). The functor , the monads  and , the quotient of monads  and the congruence relation  are as in Example \ref{Sec:MazurTraces}.

\begin{proposition}The monads  and , the functor  and the quotient of monads   satisfy the assumptions of Theorem \ref{th:quotcomparewithHJS}.
\end{proposition}
\begin{proof}
Clearly the functor  is accessible. The remaining properties of  and of the monad  are as in Theorem \ref{thm:HJS} and have been already verified in \cite{HasuoJS:07}. Thus it remains to show that the quotient  satisfies condition  of Theorem \ref{th:quotcomparewithHJS}. For this purpose, fix . The desired morphism  will be given by universal property of a standard coequalizer diagram induced by the congruence relation . First we define the set  as

Intuitively,  is the set of equations on  induced by . There are evident projection maps . It is immediate to verify that the following is a coequalizer diagram.

Also one can check that the morphism  (where  is a distributive law as in the statement of Theorem \ref{th:quotcomparewithHJS}) gives the same values if precomposed with  or with . Thus the universal property of coequalizer yields a unique morphism  making the following commute.

Commutativity of the above diagram yields condition  of Theorem \ref{th:quotcomparewithHJS}. \qed
\end{proof} 

 
\end{document}
