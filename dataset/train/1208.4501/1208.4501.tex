\documentclass[letterpaper, 12 pt]{article}  \usepackage{amssymb}
\usepackage{amsmath,mathrsfs}
\usepackage{amsthm}
\usepackage{psfrag}
\usepackage{epsfig,fullpage}
\usepackage{color}

\title{
On Multisequences and their Extensions  
}

\author{Srinivasan Krishnaswamy*, H. K. Pillai}
\newtheorem{theorem}{Theorem}[section]
\newtheorem{corollary}[theorem]{Corollary}
\newtheorem{example}[theorem]{Example}
\newtheorem{remark}[theorem]{Remark}
\newtheorem{proposition}[theorem]{Proposition}
\newtheorem{lemma}[theorem]{Lemma}
\newtheorem{definition}[theorem]{Definition}
\newtheorem{algorithm}[theorem]{Algorithm}
\newtheorem{problem}{Problem}
\newtheorem{subproblem}{Problem}[problem]
\newtheorem{subroutine}[theorem]{Subroutine}
\newcommand{\F}{\mathbb{F}}
\linespread{1.3}
\begin{document}
\maketitle

\begin{abstract}
In this paper we deal with the dimension of multisequences and related
properties. For a given multisequence  and , we define
the extension of . Further we count the number of multisequences 
whose extensions have maximum dimension and give an algorithm to derive
such multisequences. We then go on to use this theory to count the number of
Linear Feedback Shift Register(LFSR) configurations with multi input multi
output delay blocks for any given primitive characteristic polynomial and
also to design such LFSRs. Further, we use the result on multisequences to count the number of Hankel matrices of
any given dimension. 
\end{abstract}

\section{Introduction}
Linear recurring sequences, over a finite field , with maximum period have been shown to 
exhibit several important randomness properties such as -level
autocorrelation property and span- property (all nonzero subsequences of
length  occur once in every period)\cite{Golomb}. As a result they  
find applications in a wide array of areas including  cryptography 
\cite{Schneier}, error correcting
codes  \cite{peterson} and spread spectrum communication \cite{pickholtz}. 

An obvious extension of a sequence of scalars is a sequence of vectors over the
given finite field. Such a sequence of vectors is known as a multisequence.
Periodic multisequences and linear relations among the elements of such
sequences,  have been a subject of study for a considerable period of time
\cite{Daykin}, \cite{Mullen}, \cite{Yucas}. Generating multisequences with a
given minimal polynomial has been an important problem motivating papers like
\cite{Ecuyer}, \cite{Neider1} and \cite{Neider2}.  

In this paper, we start by deriving some basic theorems regarding
multisequences. We then introduce the concept of an extension of a
multisequence, for . We then derive an
algorithm to generate multisequences whose extensions have maximum dimension. Further
we derive a formula for the number of multisequences having this property.
As an application, we show that the problem of generating some special class of LFSR configurations for any given primitive characteristic polynomial is a special case of the above problem. We then go on to count the number of such LFSR configurations using the
formula derived for multisequences. Finally we demonstrate another application of the theory developed : a novel way to count the number of full rank Hankel matrices with entries from a given finite field. 

In the remainder of this paper,  denotes a field of cardinality , where
 is a prime power.  denotes the ring of polynomials in  with
coefficients from . The group of all full rank  matrices with
entries from  is denoted by . The cardinality of any set
 is given by . The set of positive integers is denoted by
.
For some integer , we denote the vector in
, with  in the  position and  in the remaining positions,
by . For any matrix , we denote the submatrix of , where
 the row indices run from  to  and the column indices run from  to
 , by . We denote the -th row of a matrix by . A matrix with row vectors  is represented as . The column span of a
matrix  is
denoted by . 
\section{Multisequences }
\label{Multisequences}






We define a sequence  in  as map from  to . 
A sequence , in a finite field 
is called periodic if there exists an integer  such that  for
all . 
The smallest such nonnegative  is called the period of the sequence. There
are linear relations amongst the elements of a periodic sequence. One obvious
example of such a relation being . A general form of such a
relation is
{\small

}
These are called  Linear Recurring Relations (LRRs). The integer  in equation
\eqref{eq1} is called the
order of the LRR. Given an LRR we can uniquely associate a monic polynomial
with it. For example the polynomial associated with the LRR in equation
\eqref{eq1} is .
Since we are dealing with periodic sequences, without loss of generality, we can
assume that   \cite[Theorem 6.11]{lidl}.

It is easy to check that all polynomials associated with LRRs of a given
sequence , form an ideal  in the polynomial ring .
Since  is a principal ideal domain, every ideal has a unique monic
generating polynomial. The generating polynomial of  is called
the minimal polynomial of the sequence .
The degree of the minimal polynomial is called the linear complexity of the
sequence. 

Given an LRR of degree , there are many sequences that satisfy this relation.
In fact, the collection of all sequences that satisfy this relation form a
vector space over . The maximum possible period of sequences in this
vector space is equal to the order of the polynomial associated with the LRR. In
particular, if the polynomial associated with 
the LRR is a primitive polynomial of degree , then every nonzero sequence in
the corresponding vector space has a period equal to  ( \cite[Theorem
6.33]{lidl}).

Consider a sequence of linear complexity . Given  consecutive elements of
the sequence, every subsequent element can be generated using the LRR
corresponding to the minimal polynomial. The vector consisting of 
consecutive elements of the sequence is called the state vector of the sequence.
We denote the -th state vector of the sequence by  i.e., . Observe that if the minimal polynomial of the
sequence is primitive then the sequence has  different state
vectors i.e., every nonzero vector in  is a state vector of the
sequence. 

Let  denote the sequence got by shifting the sequence  once
to the left i.e., . The -th state vector of
 is denoted by . Therefore .
Observe that  , where

This matrix is the companion matrix of the polynomial .  Observe that the 
companion matrix associated to the polynomial is unique. The sequence obtained by
shifting  
 times to the left, is denoted by  i.e .

Similar to sequences, we define a multisequence in  as a map from
 to . 
 A multisequence    is called
periodic if exists a finite
integer  such that , for all . As in the case
of scalar sequences, there exist linear  recurring relations between the
elements of the
multisequence. These relations are of the form
{\small

}
 Analogous to scalar sequences, one can define
a polynomial ,
which can be associated with every LRR.
Again, the polynomials associated to all LRRs of a 
given periodic multisequence, form an ideal in the principal ideal domain
 and the monic
generator of this ideal is called the minimal polynomial of the multisequence.
The degree of the minimal polynomial is defined as the linear complexity of the
multisequence.    
      
  The  component of each vector in  gives a sequence of
scalars in . We call this sequence the
 component sequence of , denoted by . Clearly, the
minimal polynomial of the multisequence is the least common multiple of the
minimal polynomials of the component sequences. Therefore, the minimal
polynomials of each of the component sequences divide the minimal polynomial of
the multisequence. Hence, if the minimal polynomial of the multisequence is an
irreducible polynomial , each of the nonzero component sequences also have
 as
their minimal polynomial.



Note that a multisequence, with linear complexity  is completely determined
by the first  terms (vectors).  The state of a multisequence can 
therefore be thought of as  consecutive elements (vectors) of the 
multisequence. Each state is thus an  matrix.
 Thus, we have a sequence of  matrix states associated with every
multisequence. We denote the -th matrix state of the multisequence  by
, i.e. . The
 row, denoted by , of  is the th state
vector of the component sequence . 
In the following theorem
we prove that for a periodic multisequence , all matrix states have
the same column span.

\begin{lemma}
\label{span}
For a periodic multisequence, the column span of the matrix states is an
invariant.
\end{lemma}

\begin{proof}
Consider a periodic multisequence . It is enough to show that
, for any given integer .
 Let the minimal polynomial of the multisequence be . Since , therefore .
Thus,

Since , , i.e., . Hence 

Therefore . Hence proved.  


\end{proof}

We define the dimension of a multisequence as follows
\begin{definition}
The dimension of a multisequence  is defined as the rank of its matrix
states.
\end{definition}

As in the case of scalar sequences, any nonzero multisequence with a primitive
minimal polynomial  of degree , has a period of .
In this paper, we henceforth assume that the multisequences considered
have primitive minimal polynomials.

The first problem we address is the following:
\begin{problem}
 Given a positive integer  and a primitive polynomial  of degree
, how many
multisequences of dimension  exist in  with  as its minimal
polynomial. (Clearly )
\end{problem}
 
 {\bf Two multisequences are considered the same if they
are
shifted versions of one another}, i.e., the multisequence  is the same as 
its shifted version   for any . 

 We denote the collection of  dimensional
subspaces of
 by . The cardinality of   is given by

 
\begin{lemma}
\label{NoMult}
Given a primitive polynomial  of degree , the number of
multisequences in , with minimal polynomial , having dimension
 is .
\end{lemma}

\begin{proof}
 Given a multisequence  of dimension , by Lemma \ref{span},  
the column space of the matrix states  is a unique 
dimensional
subspace of .
Observe that there are  subspaces of  that have
dimension . Consider any one such  dimensional space . Fix  a
basis
for , say , where . Let  be
the matrix . Any  whose column span is  can then be written as ,
where . The number of such matrices  is equal to 
 (choosing  independent vectors in
). As the polynomial  is primitive, 
each multisequence has  distinct matrix states. Thus,
the number of multisequences with column span  is
equal to . Therefore, given a primitive
polynomial  of degree , the number of
multisequences in  with minimal polynomial  having dimension
 is .
\end{proof}

If a multisequence in  has dimension , its component sequences are
linearly independent. We can therefore give the following corollary to Lemma
\ref{NoMult}.

\begin{corollary}
\label{maincor}
Given a primitive minimal polynomial  of degree , the number of
multisequences in , with minimal polynomial , having linearly
independent component sequences is
.
\end{corollary}

\section{Extensions of Multisequences}

We now look to extend an -dimensional 
multisequence  in   to an -dimensional multisequence in  
where . Further, we impose a condition that the minimal polynomial of the new multisequence is 
the same as the minimal polynomial of . An obvious way of keeping the 
minimal polynomial unchanged is by appending to  its component sequences 
or their linear combinations. Thus  for , 
where . The extended multisequence however continues to have dimension .
On other hand, appending  with shifted versions of the component sequences
may perhaps increase the dimension of the multisequence.

Let , with . We define
the -extension of the multisequence  in  as the multisequence
 in , whose component sequences are obtained from the component 
sequences of  in the following order :  . Clearly, the
minimal polynomial of the 
multisequences  and  are the same. We can
therefore ask the following question.

\begin{problem}
\label{mainProblem}
 Given , 
with , how many multisequences  of rank  in  give 
-extended multisequences in  having dimension ?
\end{problem}

The solution to this problem is given by the following theorem.

 \begin{theorem}
\label{maintheorem}
Let  such that 
 and let  be a primitive polynomial of degree
. The number  of multisequences in
 with minimal polynomial  whose extensions have dimension 
is equal to .
\end{theorem}

In the remainder of this section we give a constructive proof to this theorem.
Starting with a multisequence in  with dimension , we recursively
generate a series
of  multisequences in  culminating in a desired multisequence
whose extension has dimension . We first prove a few preparatory
results which when put together gives us the constructive proof.   
 
 For any , let . Let  define the following map from  to
. 

Note that the repeated action of  on any element of
 eventually gives . Thus, given
, 
defines a unique path from  to . We call this path the `road'.
\begin{example}
 The road for  
is          .
\end{example}
Clearly given any point  on an road, for any
other point
 lying on the path from  to ,  . Besides, the map  ensures
the following: \begin{itemize}{\item[-] If ,  if and only if }\end{itemize} 


We now look to retrace the -road from  to . As a first step, we
prove the following lemma.

\begin{lemma}
For every point  on the road, there exists
a coordinate  which satisfies
at least one of the following conditions:
\begin{enumerate}
 \item  and .
 \item  and .
\end{enumerate}
\end{lemma}

\begin{proof}
 For every point  on the road, there exists
a unique point  on the road such that .
Now, , where  . Also, since  is on the path from  to
, . Therefore, . If 
 then . If instead, there exists an  such
that , then . Hence proved.
\end{proof}

We therefore have the following definition.
\begin{definition}
Consider an . For every point , 
on the road the active coordinate is defined as follows: 
\begin{enumerate}
\item
If there exists a coordinate  such that  and ,
then the active coordinate is the coordinate corresponding to the largest such
. 
\item
In the event of there being no  that satisfies point 1, the active
coordinate is the
coordinate corresponding to the largest  such that  and
.
\end{enumerate}
\end{definition}


 


It can be easily seen that one can traverse the road backwards from  to 
by repeatedly
incrementing the active coordinate at every point. This is demonstrated in the
following example:
\begin{example}
 Let . Starting from  the road is traversed
backwards as follows: (At every point the active coordinate is underlined) 
  
  
  
 .
\end{example}
 
Detecting the active
coordinate of any point  involves the following steps:
\begin{itemize}
 \item Find .
 \item Find the largest  such that the th coordinate has value
 and is less than .
 \item In the event of there being no  satisfying the preceding condition, find
the largest  such that
the th coordinate has value  and is less than .
\end{itemize}

Notice that each of the above steps can be implemented in  operations. 





For generating multisequences with -extensions having maximum dimension, 
we travel backwards along the -road from  to . During this
backward 
traversal, at every point  on the -road, we recursively generate a 
multisequence whose -extension has maximum dimension.



We now make the following observation: Given a matrix  in the companion form and a vector
, for ,  has the
following form

 where the s are elements in , whose values depend on the matrix .
Therefore, the matrix  has the following
structure.
    

For any , let  denote the number of multisequences
in  with a given primitive minimal polynomial of degree , whose extensions have maximum dimension.
\begin{theorem}
\label{starLemma}
Let , and let  and 
be consecutive points on the road. Then,  

where  is any integer greater than .
\end{theorem}

\begin{proof}
Let  be the smallest integer such that . Therefore . Let  be a multisequence in  whose minimal 
polynomial  is a primitive polynomial of degree . Further 
assume that the extension of  has dimension .
 Each matrix state of  is therefore a matrix in  with
full row rank. As  is a primitive polynomial 
of degree , there is a matrix state  of , whose th row is 
. For , let
 be
the th
row of this . Therefore, . Now expand  to a matrix  as follows:
\begin{enumerate}
 \item For every , append the th row of  with any element 
of . Therefore, the th row of  is , for some .
\item Let the th row of  be  i.e., .
 
\end{enumerate}
Let  be a primitive polynomial of degree . Using  
as a matrix state, one can generate a multisequence  whose minimal 
polynomial is . We claim that  has a extension with dimension 
.  

As  is a matrix state of , the following matrix  is a
 matrix state of the  extension of :

 The th block of rows of
 has the following structure:
 

For , let .
The corresponding th block of rows of  has the following
structure:

The s shown in the blocks above represent entries from  which depend on
the matrix . Since  , the s appear only in
the last  columns of  (shown as the trailing submatrix after the vertical line). As extension of  has
rank , therefore  has rank . 

Similarly, corresponding to the matrix state  of , we have the
following matrix state of the extension of .



 For , the th block of  is
 (recall that ), where  is the companion matrix of the polynomial
. This block has the following structure


 The th block of  is
.
This block has the following structure.
 
 

Let  be the submatrix of  got by removing its last column and the
first row of its th block. Observe that . By the
structure of the th block of
 one can clearly see that this submatrix  can be modified to 
using elementary row operations. Hence this submatrix  has rank . This
implies that  has rank . Therefore,  does have a extension with dimension .

Note that each of the s can be chosen in
 ways. Each such choice yields a different matrix  and hence a
different multisequence . As a
result {\bf for every multisequence  with minimal polynomial ,
the above process gives us 
multisequences  with minimal polynomial }. 
Therefore, 


Conversely, consider a multisequence  in  with primitive minimal
polynomial  whose extension has rank . Consider its matrix state
 whose th row is . Now  can be
reduced to a matrix   as follows:
\begin{enumerate}
 \item For  remove the last entry of the th row.
 \item Let the th row of  be 
\end{enumerate}
 
Let  generate a multisequence  having primitive minimal polynomial . Using similar arguments as those used earlier in the proof, one can prove that
the extension of  has dimension . 
 Note that the matrix  is independent  of  the last entries of the rows of
. Hence, there are  matrices (including ),
 with th row , which have the same first  columns as
 .  By the above process each one of these matrices 
  gives the same matrix  (and hence the same multisequence ).
 Besides if we start with a matrix with th row  which differs
from  in any entry corresponding to the first  columns, it results
in a different  (and hence a different multisequence ). Therefore,
 
 Thus, from equations \eqref{eqn1} and \eqref{eqn2} we can conclude that
 

\end{proof}


Using this result, Theorem \ref{maintheorem} can by proved in the following
manner:
\begin{proof}[Proof of Theorem \ref{maintheorem}]
 For each , such that , let  be a given
 primitive polynomial of degree . For every point 
on the road, let . As we have seen in the proof of
Theorem \ref{starLemma}, starting from a multisequence in  with
dimension
 (i.e., its extension has maximum dimension), having minimal
polynomial , we can recursively generate multisequences in
, with minimal polynomial , whose extensions have
maximum dimension, for every  on the road.

By Theorem \ref{starLemma}, for any two consecutive points,  and  in the path from  to ,
 where .
The path from  to  has  such steps. Therefore,

 However,  is the number of multisequences in
 of dimension , with a given primitive minimal polynomial  of degree
.
Therefore, by Corollary \ref{maincor}, . Hence,

Hence proved.
\end{proof}

\begin{remark}
  does not depend on the integers  but just
their sum.
\end{remark}
One can therefore ask the following question.

\begin{problem}
 Given any , how many multisequences in  having dimension 
are extensions of multisequences in  for some  where . 
\end{problem}

This problem is answered in the following lemma.


\begin{lemma}
The number of multisequences in  which are extensions of
multisequences in  is given by,

\end{lemma}

\begin{proof}
 For any , define the following subset  of
. 


Therefore, 

Corresponding to each element of , say ,
we can define a monomial, . Therefore,
calculating   is equivalent to finding the number of monomials
of degree  where each variable 
is raised to a nonzero index. However, every such monomial can be written as
, where
 is a monomial of degree . Consequently, the
cardinality of 
 is equal to the number of monomials of degree . This
number is equal to . As a result, 

\end{proof}

Given , let .
Let  be a series of primitive polynomials where the
index  denotes the degree of the respective polynomial. Let s be
their corresponding companion matrices. Let  and  be consecutive
points on the road. Further, let  be the position of the active
coordinate of . Consider a multisequence  in  with a minimal
polynomial , whose extension has maximum dimension.
Note that  can be uniquely determined by any of its matrix states. Let 
be its matrix state whose th row is . The proof of
Theorem \ref{starLemma} gives us a procedure to go from , to the matrix
state  of a multisequence  with a primitive minimal polynomial
 of degree , whose extension has maximum dimension.
Thus, using this procedure one can generate a sequence of matrices
 starting with a matrix  having full row rank, and culminating in a matrix  . Each matrix  in the above sequence uniquely corresponds to a
point  on the road and can be seen as a matrix state of a multisequence
with minimal polynomial  whose corresponding extension has maximum
dimension. The following is an algorithm to generate this sequence: 
  





\begin{algorithm}
\label{algo}
(The variable  is used to store the respective matrix state at every step of
the algorithm. The current point in the path from  to  is stored in
the variable . The variable  stores the position of
the active coordinate of . The variable  stores the summation of the
values of the coordinates of . ) \\
{\bf Initialization:}
\begin{itemize}
 \item Initialize  to .  
 \item Initialize the value of  to .
 \item Initialize  to any matrix in  that has full
row rank.  
\end{itemize}
{\bf Main Loop:}
 \begin{itemize}
    \item While  do the following
    \begin{itemize}
    \item Find the position of the active coordinate of  and store it
in . 
    \item Find a polynomial  such
that .
                       \item . (This gives us the matrix
state whose th row is ).
                       \item  For all , append the th row of 
with any   to get the row vector . Change the th
row of  to .
                       \item Increment  and  by . 
       \end{itemize}
   \end{itemize}
\end{algorithm}
The most complex part of the above algorithm is to find the polynomial .
This can be done using the following subroutine.
\begin{subroutine}
\begin{itemize}
 \item Construct the matrix .
 \item Solve the set of linear equations 

 \item If  is the solution to the above set of
equations, . Therefore .
\end{itemize}
\end{subroutine}

Let  and  be the active coordinates of 
and  respectively. Algorithm \ref{algo} can be thought of as a map from the space of
matrices in  which have full row rank and whose
th rows are , to the space of matrices in  which have full row rank and whose th rows are .  There are precisely 
matrices in  whose th row
is . During each iteration of the while loop one can chose s 
in  ways. Therefore, corresponding to each choice of matrix  there are 
possible candidates for . No two distinct choices for the matrix  can give the
same . Therefore, we have  possible matrices which
can occur as an output to Algorithm \ref{algo}. The
number of full row rank matrices in , whose th row is
, is however .
Out of these matrices, precisely those matrices that occur as matrix
states of multisequences whose extensions have full rank are the ones that can be obtained from the above algorithm.

We now determine the computational complexity of Algorithm
\ref{algo}. We begin by evaluating the computational complexity of calculating
: 

\begin{itemize}
 \item For any , .
Therefore, knowing ,  can
be calculated in  steps. Consequently, the matrix  can be
generated in  operations.
\item Using decomposition, solving the set of linear equations
\eqref{polycalc} takes  operations.
\item For any , such that , the th
row of  is given by . As we have already seen each element in
the above summation can be generated in  steps. Each row of
 can thus be calculated in  steps. Therefore, 
 can be calculated in  operations.
\end{itemize}
As we have already seen, the active coordinate of  can be found in 
steps. Also, appending   rows of  takes  operations. Further,
incrementing  and  has complexity . Therefore each iteration of the
while loop takes  time.  Since the number of steps from  to
 is , the while loop runs for a maximum of  iterations.
 As a result, {\bf the computational complexity of algorithm \ref{algo} is
}.
 Therefore for a fixed  and  this computational complexity is . This is therefore a polynomial time algorithm.    

 We now proceed to see an application of the above developed theory.

    
\section{Word Based Linear Feedback Shift Registers}
The theory developed in the preceding sections finds an application in word
based Linear Feedback Shift Register (LFSR) design. We begin our discussion by
giving a brief introduction to Linear Feedback Shift
Registers (LFSR)s. 

LFSRs are electronic circuits that implement LRRs. These are widely used
in the field of pseudo-random number generation and coding theory. LFSRs
consist
of delay elements, feedback elements and adders. 
 {\small
 \begin{figure*}[h]
  \label{spiderman}
\psfrag{a0}{\tiny }
 \psfrag{a1}{\tiny }
 \psfrag{a2}{\tiny }
 \psfrag{ak-3}{\tiny }
 \psfrag{ak-2}{\tiny }
 \psfrag{ak-1}{\tiny }
 \psfrag{D0}{\tiny }
 \psfrag{D1}{\tiny }
 \psfrag{D2}{\tiny }
 \psfrag{Dk-3}{\tiny }
 \psfrag{Dk-2}{\tiny }
 \psfrag{Dk-1}{\tiny }
 \psfrag{op}{\tiny Output}
       \centering
       \includegraphics[scale=0.6]{LFSR.eps}
 \begin{center}
  \caption{Linear Feedback Shift Register}
 \end{center}
  \end{figure*}
 }
For example, the LFSR corresponding to the LRR  is as shown in Figure 1. LFSRs with
primitive characteristic polynomials are of particular interest since they
generate
sequences with desirable randomness properties like  -level
autocorrelation property and span- property (all nonzero subsequences of
length  occur once in every period)\cite{Golomb}.
An LFSR can be seen as a state machine where the states are the outputs of the
delay blocks. Its state transition matrix is the companion matrix of the
characteristic polynomial of the LRR, (i.e., matrix  in equation \eqref{A}).


Conventional LFSRs use bitwise operations and hence are incapable of 
efficiently utilizing the parallelism provided by word based processors. In the
1994 conference on fast software encryption, a challenge was set
forth to design LFSR's which exploit the parallelism offered by the word
oriented operations of modern processors \cite{Preneel}.  A
special case of this scheme was implemented by Tsaban and Vishne
in their  paper \cite{Tsabman}. Here, they
introduced a family of efficient word oriented LFSRs with multiple input
multiple output delay blocks. 
The design of Tsaban and Vishne was further generalized in \cite{zeng} wherein
the structure shown in figure \ref{f4} was proposed to implement the
mathematical scheme proposed in \cite{Neider2}. 
\begin{figure*}[h]
\psfrag{a0}{\tiny }
\psfrag{a1}{\tiny }
\psfrag{ak-2}{\tiny }
\psfrag{ak-1}{\tiny }
\psfrag{D0}{\tiny }
\psfrag{D1}{\tiny }
\psfrag{Dk-2}{\tiny }
\psfrag{Dk-1}{\tiny  }
\psfrag{mbits}{\tiny bits }
\psfrag{op}{\tiny Output}
     
\begin{center}
\includegraphics[scale=0.6]{LFSRm.eps}
 \caption{Linear Feedback Shift Register with m-Input m-Output Delay Blocks}
\label{f4}
\end{center}      
\end{figure*}

Consider the LFSR shown in Figure \ref{f4} 
Let  be the output
of the LFSR at the
-th time instant. Due to the structure of the LFSR, the following algebraic
relation is satisfied by the vectors generated by it. 

where . Therefore, for all ,

where,

We henceforth call the structure of the matrix  as the companion
structure. The matrix  is called the transition matrix of the LFSR and
it uniquely characterizes the LFSR.  The characteristic
polynomial
of an LFSR is the characteristic polynomial of the respective transition matrix.
As in the scalar case,  LFSRs with primitive characteristic polynomials are
of special interest. Interestingly, in this design, different combinations of
feedback matrices can be used to get the same characteristic polynomial. This
gives rise to the following question:
 \begin{problem}
\label{mainproblem}
 Given integers  and a primitive polynomial , of degree , how
many different LFSR realizations, using m-input m-output delay elements, have
 as their characteristic polynomial?
\end{problem}
From the discussion above, it is clear that this
number is equal to the number of m-companion matrices that have the given
primitive polynomial as their characteristic polynomial. 
Therefore, problem \ref{mainproblem} can be restated as follows.
\begin{problem}
  Given integers  and a primitive polynomial , of degree ,
how many companion matrices in  have  as their
characteristic polynomial?
\end{problem}

This question was addressed by us in \cite{HP} where the solution was found for
the cases ,  and . We will now demonstrate how the theory
developed earlier in the paper can be used to solve this problem for the
general case and in addition give an algorithm for generating such
configurations.

Consider an LFSR with input output delay blocks with
primitive characteristic polynomial  of degree . Let  be
the transition matrix of the LFSR.
The output of such an LFSR can be seen as a
multisequence  in . 
Let  be the component sequences of . Each component
sequence of  satisfies the LRR corresponding to . Therefore, the
minimal polynomial of  divides . Since  is a primitive
polynomial, the minimal polynomial of  is . The matrix state of 
at time instant  is . Let  be the state vector of the
component sequence  at time instant . Therefore  can also be
written as . Let  and let  be the
extension of . Therefore, the matrix state of  at time instant 
is . By permuting the rows of  we get the
matrix  
 which can also be written as follows:

Since  has a primitive characteristic polynomial, for any non zero
vector , the vectors  are
linearly independent. Therefore,  has rank . As a consequence,
 has rank  i.e.,  has dimension . We therefore have the
following lemma:

\begin{lemma}
Let . Given an LFSR with
input output delay blocks having a primitive characteristic polynomial
 of degree , the extension of any non zero multisequence generated
by it has an dimension  (i.e., maximum dimension).
\end{lemma}

Further, since  has primitive characteristic polynomial, repeated
action of  on any nonzero vector  will generate all
nonzero vectors in . Therefore, by Equation \eqref{recurrence}, for any
initial nonzero state of the LFSR all possible nonzero states of the LFSR will
be covered. As a result, all multisequences generated by the LFSR are just
shifted versions of each other. In other words, each LFSR with a primitive
characteristic polynomial has a unique multisequence associated with it.

Conversely, consider a multisequence  with
primitive minimal polynomial  of degree , whose extension has
dimension . Let  be the companion matrix of  and  be the 
matrix state of  at instant . One can construct the following full rank
matrices  by permuting the rows of the matrix states of the
extension of .

{\small
 }

Clearly, for any  , . Therefore,
. Thus
  is independent of  and is a constant matrix for a given
multisequence . Let this matrix be denoted by . Therefore, for
all , 

Further, given any matrix state  of ,  can be
constructed as follows: 

where  is got from Equation \eqref{consmat}.

It can be easily verified that the matrix  satisfying equation
\eqref{invrec} has an canonical structure. Let  be as follows: 

Therefore, for any , . Thus we have an LRR (and hence an LFSR) generating the  
the multisequence .

Thus, we have demonstrated a one to one correspondence between LFSRs with
input output delay blocks having a given primitive minimal polynomial
 of degree  and multisequences in  with minimal polynomial
 whose extensions have dimension . Therefore, by Theorem
\ref{maintheorem} we have the following:

\begin{theorem}
Number of LFSRs with input output delay blocks whose transition
matrices have a given primitive polynomial  of degree , as their
characteristic polynomial is

\end{theorem}

Thus, for , every
multisequence  with a primitive minimal polynomial , such that 
has dimension , is generated by a unique LFSR whose transition matrix
has characteristic polynomial . Besides, given a matrix state of
  one can uniquely determine the transition
matrix  of the LFSR by Equation \eqref{invrec}.

Therefore, the problem of finding LFSRs generating multisequences with a given
primitive polynomial reduces to a special case of problem \ref{mainProblem}
where  and . Hence algorithm
\ref{algo} can be used to obtain desired LFSR configurations as demonstrated in
the following example.

\subsection{Example}
We demonstrate Algorithm \ref{algo} by generating a companion matrix over  with primitive characteristic polynomial . We therefore generate a multisequence in  whose
extension has maximum dimension i.e., . Note that here . Consider
 which are primitive polynomials over  of corresponding degrees. Let s be the companion matrices of the respective s, for . 
We start with a multisequence in  with minimal polynomial  and the following matrix state

We initialize 
\begin{enumerate}
 \item[Iteration 1:] . Therefore, the active coordinate of 
is the rd coordinate. (The rd row of  is already  and hence
 is the desired matrix state).  
Let us append the first and second rows of  with  and  respectively and
change the third row to . We therefore get the matrix.

  This is the state matrix of a multisequence  with characteristic
polynomial . Increment the active coordinate of  to get . It can be verified that  is a multisequence whose extension has dimension .

 \item[Iteration 2:] . Therefore, the active coordinate of  is
the nd coordinate. The matrix state of  with second row being 
is 

Suppose we append the first and third rows of  with  and  respectively and
change the second row to . This gives the following matrix:
 
 This is the state matrix of a multisequence  with characteristic
polynomial . Increment
the active coordinate of  to get . Note that  is a multisequence whose extension has dimension .

 \item[Iteration 3:] . Therefore, the active coordinate of  is
the st coordinate. The matrix state of  with first row being 
is 

Suppose we append the second and third rows of  with  and  respectively and
change the second row to . This gives the following matrix:
 
 This is the state matrix of a multisequence  with characteristic
polynomial . Increment the active coordinate of  to get . Now  is a multisequence whose extension has maximum
dimension, i.e., .
\end{enumerate}


Using the matrix  we can construct the following matrix 
: 

The companion matrix  can now be obtained as follows:

This corresponds to an LFSR whose output multisequence will satisfy the
following linear recurring recurring relation: 




 \section{Counting the number of non-singular block Hankel matrices}
Hankel matrices are specially structured matrices which frequently
appear in the fields of signal processing \cite{Hasan}, image
processing and control theory. In this section
we derive a formula for the number of non-singular Hankel matrices of a given
size, over a given finite field , by using the theory developed in the
preceding sections. 

A Hankel matrix is a matrix which is constant along the anti-diagonals. For
example:

 It can be easily seen that the space of  Hankel matrices is a
 dimensional space and each Hankel matrix can be uniquely determined by
the corresponding vector . 
Since the number of Hankel matrices over a finite field  is finite, one
may pose the following question.

\begin{problem}
Given , find the number of Hankel matrices in  that
have full rank.
\end{problem}

We solve this problem by proving a bijection between the set of full rank
Hankel matrices in  and the set of multisequences in 
with a given primitive minimal polynomial  whose extensions have
maximum dimension, for .

\begin{theorem}
Let  be a primitive polynomial of degree . Let .
Consider a Hankel matrix  corresponding to the vector . The matrix  has full rank
if and only if the matrix  is a matrix state of a
multisequence  with minimal polynomial , whose extension has
maximum dimension. 
\end{theorem}
\begin{proof}
Consider the multisequence  with primitive polynomial  and matrix state
 . Therefore the corresponding matrix state  of the extension of 
is as follows:


Clearly, the submatrix  (the top right
submatrix) has full rank. Therefore,  has full rank if and only if the
submatrix  (the bottom left submatrix) has full rank.
However,  is the Hankel matrix . Hence, the matrix 
has full rank if and only if the matrix  has full rank. In other words, the
matrix  has full rank if and only if the extension  of the multisequence
 has maximum dimension. Hence proved.
\end{proof}

Every multisequence  in , with primitive minimal polynomial 
is uniquely characterized by any of its matrix states. Therefore, the number of
full rank Henkel matrices in 
is equal to the number of multisequences in  whose extensions have
maximum dimension. Hence, by Theorem \ref{maintheorem} we have the following
theorem

\begin{theorem}
The number of Hankel matrices in  having full rank is
. 
\end{theorem}
 




\section{Conclusions}
In this paper we have introduced the concept of matrix states. Using matrix
states, we have defined the dimension of a multisequence and calculated the
number of multisequences with a given dimension. The concept of extensions
has then been introduced. We have calculated the number of multisequences
whose
extensions have maximum dimension. Further we give an algorithm to
generate such multisequences.  We have then demonstrated an application of the
theory developed for extensions. For any given , we have derived a
formula for the number of LFSR configurations, with  input  output
delay blocks, that generate multisequences with a given primitive minimal
polynomial. Further, we have demonstrated the use of the algorithm developed for
extensions, for the generation of such LFSR configurations. Finally, using
the theory developed, we have derived a formula for the number of Hankel
matrices in  that have full rank.
       
 \nopagebreak
\bibliographystyle{ieeetr}
\bibliography{LSFR}


\end{document}