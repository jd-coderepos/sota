



\documentclass[conference,compsoc,a4paper]{IEEEtran}[2015/08/26]

\usepackage{pbalance}

\usepackage{upquote}

\usepackage[ngerman,main=english]{babel}
\addto\extrasenglish{\languageshorthands{ngerman}\useshorthands{"}}

\usepackage[hyphens]{url}



\makeatletter
\g@addto@macro{\UrlBreaks}{\UrlOrds}
\makeatother




\usepackage[zerostyle=b,scaled=.75]{newtxtt}

\usepackage[T1]{fontenc}



\usepackage[
  babel=true, expansion=alltext,
  protrusion=alltext-nott, final ]{microtype}

\DisableLigatures{encoding = T1, family = tt* }



\usepackage{graphicx}

\usepackage{diagbox}

\usepackage{xcolor}

\usepackage{listings}

\definecolor{eclipseStrings}{RGB}{42,0.0,255}
\definecolor{eclipseKeywords}{RGB}{127,0,85}
\colorlet{numb}{magenta!60!black}
\newcommand{\skipping}[1]{}



\lstdefinelanguage{json}{
    basicstyle=\normalfont\ttfamily,
    commentstyle=\color{eclipseStrings}, stringstyle=\color{eclipseKeywords}, numbers=left,
    numberstyle=\scriptsize,
    stepnumber=1,
    numbersep=8pt,
    showstringspaces=false,
    breaklines=true,
    frame=lines,
string=[s]{"}{"},
    comment=[l]{:\ "},
    morecomment=[l]{:"},
    literate=
        *{0}{{{\color{numb}0}}}{1}
         {1}{{{\color{numb}1}}}{1}
         {2}{{{\color{numb}2}}}{1}
         {3}{{{\color{numb}3}}}{1}
         {4}{{{\color{numb}4}}}{1}
         {5}{{{\color{numb}5}}}{1}
         {6}{{{\color{numb}6}}}{1}
         {7}{{{\color{numb}7}}}{1}
         {8}{{{\color{numb}8}}}{1}
         {9}{{{\color{numb}9}}}{1}
}

\lstset{
escapeinside={(*}{*)},
language=json,
showstringspaces=false,
extendedchars=true,
basicstyle=\footnotesize\ttfamily,
commentstyle=\slshape,
stringstyle=\ttfamily,
breaklines=true,
breakatwhitespace=true,
columns=flexible,
numbers=left,
numberstyle=\tiny,
basewidth=.5em,
xleftmargin=.5cm,
captionpos=b
}

\lstset{literate=
  {á}{{\'a}}1 {é}{{\'e}}1 {í}{{\'i}}1 {ó}{{\'o}}1 {ú}{{\'u}}1
  {Á}{{\'A}}1 {É}{{\'E}}1 {Í}{{\'I}}1 {Ó}{{\'O}}1 {Ú}{{\'U}}1
  {à}{{\`a}}1 {è}{{\`e}}1 {ì}{{\`i}}1 {ò}{{\`o}}1 {ù}{{\`u}}1
  {À}{{\`A}}1 {È}{{\'E}}1 {Ì}{{\`I}}1 {Ò}{{\`O}}1 {Ù}{{\`U}}1
  {ä}{{\"a}}1 {ë}{{\"e}}1 {ï}{{\"i}}1 {ö}{{\"o}}1 {ü}{{\"u}}1
  {Ä}{{\"A}}1 {Ë}{{\"E}}1 {Ï}{{\"I}}1 {Ö}{{\"O}}1 {Ü}{{\"U}}1
  {â}{{\^a}}1 {ê}{{\^e}}1 {î}{{\^i}}1 {ô}{{\^o}}1 {û}{{\^u}}1
  {Â}{{\^A}}1 {Ê}{{\^E}}1 {Î}{{\^I}}1 {Ô}{{\^O}}1 {Û}{{\^U}}1
  {Ã}{{\~A}}1 {ã}{{\~a}}1 {Õ}{{\~O}}1 {õ}{{\~o}}1
  {œ}{{\oe}}1 {Œ}{{\OE}}1 {æ}{{\ae}}1 {Æ}{{\AE}}1 {ß}{{\ss}}1
  {ű}{{\H{u}}}1 {Ű}{{\H{U}}}1 {ő}{{\H{o}}}1 {Ő}{{\H{O}}}1
  {ç}{{\c c}}1 {Ç}{{\c C}}1 {ø}{{\o}}1 {å}{{\r a}}1 {Å}{{\r A}}1
}

\usepackage[autostyle=true]{csquotes}

\defineshorthand{"`}{\openautoquote}
\defineshorthand{"'}{\closeautoquote}

\usepackage{booktabs}

\usepackage{paralist}



\usepackage[square,        comma,         numbers,       sort&compress ]{natbib}

\renewcommand{\bibfont}{\normalfont\footnotesize}

\usepackage{etoolbox}
\makeatletter
\patchcmd{\NAT@test}{\else \NAT@nm}{\else \NAT@hyper@{\NAT@nm}}{}{}
\makeatother

\usepackage{pdfcomment}

\newcommand{\todo}[1]{\commentatside{#1}}

\newcommand{\TODO}[1]{\commentatside{#1}}

\usepackage{stfloats}
\fnbelowfloat

\usepackage[group-minimum-digits=4,per-mode=fraction]{siunitx}
\addto\extrasgerman{\sisetup{locale = DE}}

\usepackage{footmisc}

\usepackage{hyperxmp}

\usepackage{hyperref}

\hypersetup{
  keeppdfinfo,
  hidelinks,
  colorlinks=true,
  allcolors=black,
  pdfstartview=Fit,
  breaklinks=true,
  pdftitle={High-Resolution Multi-Scale RAFT},
  pdfsubject={Technical Report for Robust Vision Challenge 2022},
  pdfauthor={Azin Jahedi, Maximilian Luz, Lukas Mehl, Marc Rivinius, Andrés Bruhn},
  pdfkeywords={
    Robust Vision Challenge 2022,
    Optical Flow,
    Multi-scale,
    Coarse-to-fine},
}

\usepackage[all]{hypcap}

\usepackage[caption=false,font=footnotesize]{subfig}



\usepackage[capitalise,nameinlink,noabbrev]{cleveref}

\crefname{listing}{Listing}{Listings}
\Crefname{listing}{Listing}{Listings}
\crefname{lstlisting}{Listing}{Listings}
\Crefname{lstlisting}{Listing}{Listings}

\usepackage{lipsum}

\usepackage[math]{blindtext}
\usepackage{mwe}
\usepackage[realmainfile]{currfile}
\usepackage{tcolorbox}
\tcbuselibrary{listings}

\DeclareFontFamily{U}{MnSymbolC}{}
\DeclareSymbolFont{MnSyC}{U}{MnSymbolC}{m}{n}
\DeclareFontShape{U}{MnSymbolC}{m}{n}{
  <-6>    MnSymbolC5
  <6-7>   MnSymbolC6
  <7-8>   MnSymbolC7
  <8-9>   MnSymbolC8
  <9-10>  MnSymbolC9
  <10-12> MnSymbolC10
  <12->   MnSymbolC12}{}
\DeclareMathSymbol{\powerset}{\mathord}{MnSyC}{180}

\usepackage{xspace}
\newcommand{\eg}{e.\,g.,\ }
\newcommand{\ie}{i.\,e.,\ }

\makeatletter
\newcommand{\hydash}{\penalty\@M-\hskip\z@skip}
\makeatother



\hyphenation{op-tical net-works semi-conduc-tor}

\input glyphtounicode
\pdfgentounicode=1

\IEEEoverridecommandlockouts

\begin{document}


\title{High-Resolution Multi-Scale RAFT \\ (Robust Vision Challenge 2022)}

\author{\IEEEauthorblockN{Azin Jahedi\IEEEauthorrefmark{1}, Maximilian Luz\IEEEauthorrefmark{1}, Lukas Mehl\IEEEauthorrefmark{1}, Marc Rivinius\IEEEauthorrefmark{2}, Andrés Bruhn\IEEEauthorrefmark{1}}
  \IEEEauthorblockA{\IEEEauthorrefmark{1}Institute for Visualization and Interactive Systems, University of Stuttgart, Germany}
  \IEEEauthorblockA{\IEEEauthorrefmark{2}Institute of Information Security, University of Stuttgart, Germany}

   
   \thanks{The authors thank the Deutsche Forschungsgemeinschaft (DFG, German Research Foundation) -- Project-ID 251654672 -- TRR 161 (B04) for funding and the International Max Planck Research School for Intelligent Systems (IMPRS-IS) for supporting Azin Jahedi.}
}




\maketitle

\thispagestyle{plain}
\pagestyle{plain}

\iffalse
  \IEEEoverridecommandlockouts
  \IEEEpubid{\begin{minipage}{\textwidth}\ \0.5mm]
			All Fl-all    &   4.88  &   {\bf 4.15}  &   {\bf +15.0\%} \\
			Non-occluded Fl-all  &  2.80  &    {\bf 2.18}& {\bf +22.1} \\
			\midrule
			{\bf Sintel Clean (Test)}   &  &    &   \0.5mm]
			
			EPE all    & {\bf 2.667}  & 2.682   &  \bf{-0.1}\% \\
			EPE matched  & \bf{1.190}  &  1.278   &   \bf{-6.8\%} \\
s0-10  &  0.468   &   {\bf 0.420}   &  {\bf +10.2\%}
	 		\\
	 		\midrule
			{\bf Middlebury (Train)}   &     &     &  \-0.2mm]
			 & {\bf \scriptsize {(Test)}}  &  {\bf \scriptsize (Test) }  &  {\bf \scriptsize (Test)  } & {\bf \scriptsize (Test) } \\
			 & {\scriptsize EPE all}  &  {\scriptsize EPE all }  &  {\scriptsize EPE all} & {\scriptsize Avg-Rank } \\
			\midrule
		 {\bf {Cold}}   & 1.63 & 2.94 &  {\bf0.1725} & {\bf8.8}\0.5mm]
			{\bf Cold warm}    &  {\bf 1.23}  &   {\bf 2.68}  & {\bf 0.1725} & 9.2 \\
		

			\bottomrule
		\end{tabular}
		

\end{table}

 
\subsection{Robust Vision Challenge}
The results for the Robust Vision Challenge 2022 (RVC) are shown in \cref{fig:leaderboard}. Further details for the separate benchmarks, which we refer to in the following, can be found at the corresponding benchmark websites.

In the Middlebury benchmark, our MS-RAFT+ method achieves Rank 4 overall\footnote{\label{fn:rank}The overall rank also considers anonymous methods.} and Rank 2 among all RVC approaches. Since we did not use Middlebury training data for our mixed fine-tuning, this also shows the good generalization performance of our method.
In the RVC ranking for this benchmark, which also considers other metrics, 
it achieves a final rank of 2. 

In the KITTI 2015 benchmark, our method achieves Rank 8 (Fl-all) and Rank 2 (Fl-all non-occluded) overall\footref{fn:rank}, as well as Rank 2 (Fl-all) and Rank 1 (Fl-all non-occluded) among all RVC approaches. It is even on par with the leading scene flow method (CamLiFlow++) in non-occluded regions. 
In the RVC ranking for this benchmark, which also considers other metrics, our method achieves the 2nd place.

In the Sintel benchmark, MS-RAFT+ achieves overall\footref{fn:rank} Rank 19 (final, EPE all) and Rank 6 (clean, EPE-all), as well as Rank 3 (both final and clean, EPE-all) considering only RVC submissions.
Our method performs again particularly well in non-occluded regions (clean). Moreover, it shows very good results for small displacements (both final and clean), where it achieves 1st ranks in the corresponding metrics.
In the RVC ranking, 
that also considers many other metrics, our approach finally ranks 2nd.

Eventually, in the VIPER benchmark, our method achieves Rank 1 overall\footref{fn:rank} and Rank 1 among all RVC approaches. By clearly outperforming all other methods in all categories (day, sunset, rain, snow, night), it sets a new state-of-the-art for this benchmark. Consequently, in the final RVC ranking thereof, it ranks 1st.


With RVC-ranks of 2 for Middlebury, KITTI 2015, and Sintel, as well as 1 for VIPER, MS-RAFT+ shows a good generalization performance across all benchmarks. In the final leaderboard of the RVC, this results in Rank 1 of all participating methods (see \cref{fig:leaderboard}) and thereby our method won the Robust Vision Challenge 2022 for the optical flow estimation task.

\begin{figure}
    \centering
       \includegraphics[width=1\columnwidth]{rvc_small_cropped.pdf}
    \caption{Final leaderboard of Robust Vision Challenge 2022 (Top 5).} 
    \label{fig:leaderboard}
\end{figure}








\bibliographystyle{IEEEtranN} \bibliography{report}





\end{document}
