\documentclass[conference]{IEEEtran}

\IEEEoverridecommandlockouts

\usepackage{amsmath,amssymb,epsf,latexsym,graphicx,color}
\usepackage{mathrsfs}

\usepackage{cite}




\newtheorem{assumption}{\hspace{0pt}\bf AS\hspace{-0.15cm}}

\newcounter{lemma} \setcounter{lemma}{1}
\newenvironment{lemma}{ \vspace{-0.1cm} \noindent {\bf Lemma
\thelemma :} \addtocounter{lemma}{-1}\refstepcounter{lemma} \em} {\addtocounter{lemma}{1}
\vspace{-0.1cm} \normalfont }

\newcounter{corollary} \setcounter{corollary}{1}
\newenvironment{corollary}{ \vspace{-0.1cm} \noindent {\bf Corollary
\thecorollary :} \addtocounter{corollary}{-1}\refstepcounter{corollary} \em} {\addtocounter{corollary}{1}
\vspace{-0.1cm} \normalfont }

\newcounter{proposition} \setcounter{proposition}{1}
\newenvironment{proposition}{ \vspace{-0.1cm} \noindent {\bf Proposition
\theproposition :} \addtocounter{proposition}{-1}\refstepcounter{proposition}\em } { \addtocounter{proposition}{1}
\vspace{-0.1cm} \normalfont }

\newtheorem{observation}{\hspace{0pt}\bf Observation}
\newtheorem{theorem}{\hspace{0pt}\bf Theorem}
\newtheorem{fact}{\hspace{0pt}\bf Fact}

\newcounter{remarkjs} \setcounter{remarkjs}{1}
\newenvironment{remarkjs}{ \vspace{-0.1cm} \noindent {\bf Remark
\theremarkjs :}
\addtocounter{remarkjs}{-1}\refstepcounter{remarkjs}\em } {
\addtocounter{remarkjs}{1} \vspace{-0.1cm} \normalfont }



\newtheorem{test}{\hspace{0pt}\it Test Case}
\newtheorem{definition}{\hspace{0pt}\bf Definition}
\newtheorem{example}{\hspace{0pt}\bf Example}

\def\proof      {\noindent\hspace{0pt}{\bf{Proof: }}}
\def\myQED      {\hfill\vspace{0.3cm}}



\def\bss{{\ensuremath{\boldsymbol{s}}}}
\def\bsz{{\ensuremath{\boldsymbol{z}}}}
\def\bsv{{\ensuremath{\boldsymbol{v}}}}

\def\bs{{\ensuremath{\boldsymbol{}}}}

\def\bbarA{{\ensuremath{\bar A}}}
\def\bbarB{{\ensuremath{\bar B}}}
\def\bbarC{{\ensuremath{\bar C}}}
\def\bbarD{{\ensuremath{\bar D}}}
\def\bbarE{{\ensuremath{\bar E}}}
\def\bbarF{{\ensuremath{\bar F}}}
\def\bbarG{{\ensuremath{\bar G}}}
\def\bbarH{{\ensuremath{\bar H}}}
\def\bbarI{{\ensuremath{\bar I}}}
\def\bbarJ{{\ensuremath{\bar J}}}
\def\bbarK{{\ensuremath{\bar K}}}
\def\bbarL{{\ensuremath{\bar L}}}
\def\bbarM{{\ensuremath{\bar M}}}
\def\bbarN{{\ensuremath{\bar N}}}
\def\bbarO{{\ensuremath{\bar O}}}
\def\bbarP{{\ensuremath{\bar P}}}
\def\bbarQ{{\ensuremath{\bar Q}}}
\def\bbarR{{\ensuremath{\bar R}}}
\def\bbarW{{\ensuremath{\bar W}}}
\def\bbarU{{\ensuremath{\bar U}}}
\def\bbarV{{\ensuremath{\bar V}}}
\def\bbarS{{\ensuremath{\bar S}}}
\def\bbarT{{\ensuremath{\bar T}}}
\def\bbarX{{\ensuremath{\bar X}}}
\def\bbarY{{\ensuremath{\bar Y}}}
\def\bbarZ{{\ensuremath{\bar Z}}}
\def\bbara{{\ensuremath{\bar a}}}
\def\bbarb{{\ensuremath{\bar b}}}
\def\bbarc{{\ensuremath{\bar c}}}
\def\bbard{{\ensuremath{\bar d}}}
\def\bbare{{\ensuremath{\bar e}}}
\def\bbarf{{\ensuremath{\bar f}}}
\def\bbarg{{\ensuremath{\bar g}}}
\def\bbarh{{\ensuremath{\bar h}}}
\def\bbari{{\ensuremath{\bar i}}}
\def\bbarj{{\ensuremath{\bar j}}}
\def\bbark{{\ensuremath{\bar k}}}
\def\bbarl{{\ensuremath{\bar l}}}
\def\bbarm{{\ensuremath{\bar m}}}
\def\bbarn{{\ensuremath{\bar n}}}
\def\bbaro{{\ensuremath{\bar o}}}
\def\bbarp{{\ensuremath{\bar p}}}
\def\bbarq{{\ensuremath{\bar q}}}
\def\bbarr{{\ensuremath{\bar r}}}
\def\bbarw{{\ensuremath{\bar w}}}
\def\bbaru{{\ensuremath{\bar u}}}
\def\bbarv{{\ensuremath{\bar v}}}
\def\bbars{{\ensuremath{\bar s}}}
\def\bbart{{\ensuremath{\bar t}}}
\def\bbarx{{\ensuremath{\bar x}}}
\def\bbary{{\ensuremath{\bar y}}}
\def\bbarz{{\ensuremath{\bar z}}}
\def\ccalA{{\ensuremath{\mathcal A}}}
\def\ccalB{{\ensuremath{\mathcal B}}}
\def\ccalC{{\ensuremath{\mathcal C}}}
\def\ccalD{{\ensuremath{\mathcal D}}}
\def\ccalE{{\ensuremath{\mathcal E}}}
\def\ccalF{{\ensuremath{\mathcal F}}}
\def\ccalG{{\ensuremath{\mathcal G}}}
\def\ccalH{{\ensuremath{\mathcal H}}}
\def\ccalI{{\ensuremath{\mathcal I}}}
\def\ccalJ{{\ensuremath{\mathcal J}}}
\def\ccalK{{\ensuremath{\mathcal K}}}
\def\ccalL{{\ensuremath{\mathcal L}}}
\def\ccalM{{\ensuremath{\mathcal M}}}
\def\ccalN{{\ensuremath{\mathcal N}}}
\def\ccalO{{\ensuremath{\mathcal O}}}
\def\ccalP{{\ensuremath{\mathcal P}}}
\def\ccalQ{{\ensuremath{\mathcal Q}}}
\def\ccalR{{\ensuremath{\mathcal R}}}
\def\ccalW{{\ensuremath{\mathcal W}}}
\def\ccalU{{\ensuremath{\mathcal U}}}
\def\ccalV{{\ensuremath{\mathcal V}}}
\def\ccalS{{\ensuremath{\mathcal S}}}
\def\ccalT{{\ensuremath{\mathcal T}}}
\def\ccalX{{\ensuremath{\mathcal X}}}
\def\ccalY{{\ensuremath{\mathcal Y}}}
\def\ccalZ{{\ensuremath{\mathcal Z}}}
\def\ccala{{\ensuremath{\mathcal a}}}
\def\ccalb{{\ensuremath{\mathcal b}}}
\def\ccalc{{\ensuremath{\mathcal c}}}
\def\ccald{{\ensuremath{\mathcal d}}}
\def\ccale{{\ensuremath{\mathcal e}}}
\def\ccalf{{\ensuremath{\mathcal f}}}
\def\ccalg{{\ensuremath{\mathcal g}}}
\def\ccalh{{\ensuremath{\mathcal h}}}
\def\ccali{{\ensuremath{\mathcal i}}}
\def\ccalj{{\ensuremath{\mathcal j}}}
\def\ccalk{{\ensuremath{\mathcal k}}}
\def\ccall{{\ensuremath{\mathcal l}}}
\def\ccalm{{\ensuremath{\mathcal m}}}
\def\ccaln{{\ensuremath{\mathcal n}}}
\def\ccalo{{\ensuremath{\mathcal o}}}
\def\ccalp{{\ensuremath{\mathcal p}}}
\def\ccalq{{\ensuremath{\mathcal q}}}
\def\ccalr{{\ensuremath{\mathcal r}}}
\def\ccalw{{\ensuremath{\mathcal w}}}
\def\ccalu{{\ensuremath{\mathcal u}}}
\def\ccalv{{\ensuremath{\mathcal v}}}
\def\ccals{{\ensuremath{\mathcal s}}}
\def\ccalt{{\ensuremath{\mathcal t}}}
\def\ccalx{{\ensuremath{\mathcal x}}}
\def\ccaly{{\ensuremath{\mathcal y}}}
\def\ccalz{{\ensuremath{\mathcal z}}}
\def\ccal0{{\ensuremath{\mathcal 0}}}
\def\hhatA{{\ensuremath{\hat A}}}
\def\hhatB{{\ensuremath{\hat B}}}
\def\hhatC{{\ensuremath{\hat C}}}
\def\hhatD{{\ensuremath{\hat D}}}
\def\hhatE{{\ensuremath{\hat E}}}
\def\hhatF{{\ensuremath{\hat F}}}
\def\hhatG{{\ensuremath{\hat G}}}
\def\hhatH{{\ensuremath{\hat H}}}
\def\hhatI{{\ensuremath{\hat I}}}
\def\hhatJ{{\ensuremath{\hat J}}}
\def\hhatK{{\ensuremath{\hat K}}}
\def\hhatL{{\ensuremath{\hat L}}}
\def\hhatM{{\ensuremath{\hat M}}}
\def\hhatN{{\ensuremath{\hat N}}}
\def\hhatO{{\ensuremath{\hat O}}}
\def\hhatP{{\ensuremath{\hat P}}}
\def\hhatQ{{\ensuremath{\hat Q}}}
\def\hhatR{{\ensuremath{\hat R}}}
\def\hhatW{{\ensuremath{\hat W}}}
\def\hhatU{{\ensuremath{\hat U}}}
\def\hhatV{{\ensuremath{\hat V}}}
\def\hhatS{{\ensuremath{\hat S}}}
\def\hhatT{{\ensuremath{\hat T}}}
\def\hhatX{{\ensuremath{\hat X}}}
\def\hhatY{{\ensuremath{\hat Y}}}
\def\hhatZ{{\ensuremath{\hat Z}}}
\def\hhata{{\ensuremath{\hat a}}}
\def\hhatb{{\ensuremath{\hat b}}}
\def\hhatc{{\ensuremath{\hat c}}}
\def\hhatd{{\ensuremath{\hat d}}}
\def\hhate{{\ensuremath{\hat e}}}
\def\hhatf{{\ensuremath{\hat f}}}
\def\hhatg{{\ensuremath{\hat g}}}
\def\hhath{{\ensuremath{\hat h}}}
\def\hhati{{\ensuremath{\hat i}}}
\def\hhatj{{\ensuremath{\hat j}}}
\def\hhatk{{\ensuremath{\hat k}}}
\def\hhatl{{\ensuremath{\hat l}}}
\def\hhatm{{\ensuremath{\hat m}}}
\def\hhatn{{\ensuremath{\hat n}}}
\def\hhato{{\ensuremath{\hat o}}}
\def\hhatp{{\ensuremath{\hat p}}}
\def\hhatq{{\ensuremath{\hat q}}}
\def\hhatr{{\ensuremath{\hat r}}}
\def\hhatw{{\ensuremath{\hat w}}}
\def\hhatu{{\ensuremath{\hat u}}}
\def\hhatv{{\ensuremath{\hat v}}}
\def\hhats{{\ensuremath{\hat s}}}
\def\hhatt{{\ensuremath{\hat t}}}
\def\hhatx{{\ensuremath{\hat x}}}
\def\hhaty{{\ensuremath{\hat y}}}
\def\hhatz{{\ensuremath{\hat z}}}
\def\tdA{{\ensuremath{\tilde A}}}
\def\tdB{{\ensuremath{\tilde B}}}
\def\tdC{{\ensuremath{\tilde C}}}
\def\tdD{{\ensuremath{\tilde D}}}
\def\tdE{{\ensuremath{\tilde E}}}
\def\tdF{{\ensuremath{\tilde F}}}
\def\tdG{{\ensuremath{\tilde G}}}
\def\tdH{{\ensuremath{\tilde H}}}
\def\tdI{{\ensuremath{\tilde I}}}
\def\tdJ{{\ensuremath{\tilde J}}}
\def\tdK{{\ensuremath{\tilde K}}}
\def\tdL{{\ensuremath{\tilde L}}}
\def\tdM{{\ensuremath{\tilde M}}}
\def\tdN{{\ensuremath{\tilde N}}}
\def\tdO{{\ensuremath{\tilde O}}}
\def\tdP{{\ensuremath{\tilde P}}}
\def\tdQ{{\ensuremath{\tilde Q}}}
\def\tdR{{\ensuremath{\tilde R}}}
\def\tdW{{\ensuremath{\tilde W}}}
\def\tdU{{\ensuremath{\tilde U}}}
\def\tdV{{\ensuremath{\tilde V}}}
\def\tdS{{\ensuremath{\tilde S}}}
\def\tdT{{\ensuremath{\tilde T}}}
\def\tdX{{\ensuremath{\tilde X}}}
\def\tdY{{\ensuremath{\tilde Y}}}
\def\tdZ{{\ensuremath{\tilde Z}}}
\def\tda{{\ensuremath{\tilde a}}}
\def\tdb{{\ensuremath{\tilde b}}}
\def\tdc{{\ensuremath{\tilde c}}}
\def\tdd{{\ensuremath{\tilde d}}}
\def\tde{{\ensuremath{\tilde e}}}
\def\tdf{{\ensuremath{\tilde f}}}
\def\tdg{{\ensuremath{\tilde g}}}
\def\tdh{{\ensuremath{\tilde h}}}
\def\tdi{{\ensuremath{\tilde i}}}
\def\tdj{{\ensuremath{\tilde j}}}
\def\tdk{{\ensuremath{\tilde k}}}
\def\tdl{{\ensuremath{\tilde l}}}
\def\tdm{{\ensuremath{\tilde m}}}
\def\tdn{{\ensuremath{\tilde n}}}
\def\tdo{{\ensuremath{\tilde o}}}
\def\tdp{{\ensuremath{\tilde p}}}
\def\tdq{{\ensuremath{\tilde q}}}
\def\tdr{{\ensuremath{\tilde r}}}
\def\tdw{{\ensuremath{\tilde w}}}
\def\tdu{{\ensuremath{\tilde u}}}
\def\tdv{{\ensuremath{\tilde r}}}
\def\tds{{\ensuremath{\tilde s}}}
\def\tdt{{\ensuremath{\tilde t}}}
\def\tdx{{\ensuremath{\tilde x}}}
\def\tdy{{\ensuremath{\tilde y}}}
\def\tdz{{\ensuremath{\tilde z}}}
\def\chka{{\ensuremath{\check a}}}
\def\chkb{{\ensuremath{\check b}}}
\def\chkc{{\ensuremath{\check c}}}
\def\chkd{{\ensuremath{\check d}}}
\def\chke{{\ensuremath{\check e}}}
\def\chkf{{\ensuremath{\check f}}}
\def\chkg{{\ensuremath{\check g}}}
\def\chkh{{\ensuremath{\check h}}}
\def\chki{{\ensuremath{\check i}}}
\def\chkj{{\ensuremath{\check j}}}
\def\chkk{{\ensuremath{\check k}}}
\def\chkl{{\ensuremath{\check l}}}
\def\chkm{{\ensuremath{\check m}}}
\def\chkn{{\ensuremath{\check n}}}
\def\chko{{\ensuremath{\check o}}}
\def\chkp{{\ensuremath{\check p}}}
\def\chkq{{\ensuremath{\check q}}}
\def\chkr{{\ensuremath{\check r}}}
\def\chkw{{\ensuremath{\check w}}}
\def\chku{{\ensuremath{\check u}}}
\def\chkv{{\ensuremath{\check v}}}
\def\chks{{\ensuremath{\check s}}}
\def\chkt{{\ensuremath{\check t}}}
\def\chkx{{\ensuremath{\check x}}}
\def\chky{{\ensuremath{\check y}}}
\def\chkz{{\ensuremath{\check z}}}
\def\bbA{{\ensuremath{\mathbf A}}}
\def\bbB{{\ensuremath{\mathbf B}}}
\def\bbC{{\ensuremath{\mathbf C}}}
\def\bbD{{\ensuremath{\mathbf D}}}
\def\bbE{{\ensuremath{\mathbf E}}}
\def\bbF{{\ensuremath{\mathbf F}}}
\def\bbG{{\ensuremath{\mathbf G}}}
\def\bbH{{\ensuremath{\mathbf H}}}
\def\bbI{{\ensuremath{\mathbf I}}}
\def\bbJ{{\ensuremath{\mathbf J}}}
\def\bbK{{\ensuremath{\mathbf K}}}
\def\bbL{{\ensuremath{\mathbf L}}}
\def\bbM{{\ensuremath{\mathbf M}}}
\def\bbN{{\ensuremath{\mathbf N}}}
\def\bbO{{\ensuremath{\mathbf O}}}
\def\bbP{{\ensuremath{\mathbf P}}}
\def\bbQ{{\ensuremath{\mathbf Q}}}
\def\bbR{{\ensuremath{\mathbf R}}}
\def\bbW{{\ensuremath{\mathbf W}}}
\def\bbU{{\ensuremath{\mathbf U}}}
\def\bbV{{\ensuremath{\mathbf V}}}
\def\bbS{{\ensuremath{\mathbf S}}}
\def\bbT{{\ensuremath{\mathbf T}}}
\def\bbX{{\ensuremath{\mathbf X}}}
\def\bbY{{\ensuremath{\mathbf Y}}}
\def\bbZ{{\ensuremath{\mathbf Z}}}
\def\bba{{\ensuremath{\mathbf a}}}
\def\bbb{{\ensuremath{\mathbf b}}}
\def\bbc{{\ensuremath{\mathbf c}}}
\def\bbd{{\ensuremath{\mathbf d}}}
\def\bbe{{\ensuremath{\mathbf e}}}
\def\bbf{{\ensuremath{\mathbf f}}}
\def\bbg{{\ensuremath{\mathbf g}}}
\def\bbh{{\ensuremath{\mathbf h}}}
\def\bbi{{\ensuremath{\mathbf i}}}
\def\bbj{{\ensuremath{\mathbf j}}}
\def\bbk{{\ensuremath{\mathbf k}}}
\def\bbl{{\ensuremath{\mathbf l}}}
\def\bbm{{\ensuremath{\mathbf m}}}
\def\bbn{{\ensuremath{\mathbf n}}}
\def\bbo{{\ensuremath{\mathbf o}}}
\def\bbp{{\ensuremath{\mathbf p}}}
\def\bbq{{\ensuremath{\mathbf q}}}
\def\bbr{{\ensuremath{\mathbf r}}}
\def\bbw{{\ensuremath{\mathbf w}}}
\def\bbu{{\ensuremath{\mathbf u}}}
\def\bbv{{\ensuremath{\mathbf v}}}
\def\bbs{{\ensuremath{\mathbf s}}}
\def\bbt{{\ensuremath{\mathbf t}}}
\def\bbx{{\ensuremath{\mathbf x}}}
\def\bby{{\ensuremath{\mathbf y}}}
\def\bbz{{\ensuremath{\mathbf z}}}
\def\bb0{{\ensuremath{\mathbf 0}}}


\def\bbarbbA{{\bar{\ensuremath{\mathbf A}} }}
\def\bbarbbB{{\bar{\ensuremath{\mathbf B}} }}
\def\bbarbbC{{\bar{\ensuremath{\mathbf C}} }}
\def\bbarbbD{{\bar{\ensuremath{\mathbf D}} }}
\def\bbarbbE{{\bar{\ensuremath{\mathbf E}} }}
\def\bbarbbF{{\bar{\ensuremath{\mathbf F}} }}
\def\bbarbbG{{\bar{\ensuremath{\mathbf G}} }}
\def\bbarbbH{{\bar{\ensuremath{\mathbf H}} }}
\def\bbarbbI{{\bar{\ensuremath{\mathbf I}} }}
\def\bbarbbJ{{\bar{\ensuremath{\mathbf J}} }}
\def\bbarbbK{{\bar{\ensuremath{\mathbf K}} }}
\def\bbarbbL{{\bar{\ensuremath{\mathbf L}} }}
\def\bbarbbM{{\bar{\ensuremath{\mathbf M}} }}
\def\bbarbbN{{\bar{\ensuremath{\mathbf N}} }}
\def\bbarbbO{{\bar{\ensuremath{\mathbf O}} }}
\def\bbarbbP{{\bar{\ensuremath{\mathbf P}} }}
\def\bbarbbQ{{\bar{\ensuremath{\mathbf Q}} }}
\def\bbarbbR{{\bar{\ensuremath{\mathbf R}} }}
\def\bbarbbS{{\bar{\ensuremath{\mathbf S}} }}
\def\bbarbbT{{\bar{\ensuremath{\mathbf T}} }}
\def\bbarbbU{{\bar{\ensuremath{\mathbf U}} }}
\def\bbarbbV{{\bar{\ensuremath{\mathbf V}} }}
\def\bbarbbW{{\bar{\ensuremath{\mathbf W}} }}
\def\bbarbbX{{\bar{\ensuremath{\mathbf X}} }}
\def\bbarbbY{{\bar{\ensuremath{\mathbf Y}} }}
\def\bbarbbZ{{\bar{\ensuremath{\mathbf Z}} }}
\def\bbarbba{{\bar{\ensuremath{\mathbf a}} }}
\def\bbarbbb{{\bar{\ensuremath{\mathbf b}} }}
\def\bbarbbc{{\bar{\ensuremath{\mathbf c}} }}
\def\bbarbbd{{\bar{\ensuremath{\mathbf d}} }}
\def\bbarbbe{{\bar{\ensuremath{\mathbf e}} }}
\def\bbarbbf{{\bar{\ensuremath{\mathbf f}} }}
\def\bbarbbg{{\bar{\ensuremath{\mathbf g}} }}
\def\bbarbbh{{\bar{\ensuremath{\mathbf h}} }}
\def\bbarbbi{{\bar{\ensuremath{\mathbf i}} }}
\def\bbarbbj{{\bar{\ensuremath{\mathbf j}} }}
\def\bbarbbk{{\bar{\ensuremath{\mathbf k}} }}
\def\bbarbbl{{\bar{\ensuremath{\mathbf l}} }}
\def\bbarbbm{{\bar{\ensuremath{\mathbf m}} }}
\def\bbarbbn{{\bar{\ensuremath{\mathbf n}} }}
\def\bbarbbo{{\bar{\ensuremath{\mathbf o}} }}
\def\bbarbbp{{\bar{\ensuremath{\mathbf p}} }}
\def\bbarbbq{{\bar{\ensuremath{\mathbf q}} }}
\def\bbarbbr{{\bar{\ensuremath{\mathbf r}} }}
\def\bbarbbs{{\bar{\ensuremath{\mathbf s}} }}
\def\bbarbbt{{\bar{\ensuremath{\mathbf t}} }}
\def\bbarbbu{{\bar{\ensuremath{\mathbf u}} }}
\def\bbarbbv{{\bar{\ensuremath{\mathbf v}} }}
\def\bbarbbw{{\bar{\ensuremath{\mathbf w}} }}
\def\bbarbbx{{\bar{\ensuremath{\mathbf x}} }}
\def\bbarbby{{\bar{\ensuremath{\mathbf y}} }}
\def\bbarbbz{{\bar{\ensuremath{\mathbf z}} }}
\def\hhatbbA{{\hat{\ensuremath{\mathbf A}} }}
\def\hhatbbB{{\hat{\ensuremath{\mathbf B}} }}
\def\hhatbbC{{\hat{\ensuremath{\mathbf C}} }}
\def\hhatbbD{{\hat{\ensuremath{\mathbf D}} }}
\def\hhatbbE{{\hat{\ensuremath{\mathbf E}} }}
\def\hhatbbF{{\hat{\ensuremath{\mathbf F}} }}
\def\hhatbbG{{\hat{\ensuremath{\mathbf G}} }}
\def\hhatbbH{{\hat{\ensuremath{\mathbf H}} }}
\def\hhatbbI{{\hat{\ensuremath{\mathbf I}} }}
\def\hhatbbJ{{\hat{\ensuremath{\mathbf J}} }}
\def\hhatbbK{{\hat{\ensuremath{\mathbf K}} }}
\def\hhatbbL{{\hat{\ensuremath{\mathbf L}} }}
\def\hhatbbM{{\hat{\ensuremath{\mathbf M}} }}
\def\hhatbbN{{\hat{\ensuremath{\mathbf N}} }}
\def\hhatbbO{{\hat{\ensuremath{\mathbf O}} }}
\def\hhatbbP{{\hat{\ensuremath{\mathbf P}} }}
\def\hhatbbQ{{\hat{\ensuremath{\mathbf Q}} }}
\def\hhatbbR{{\hat{\ensuremath{\mathbf R}} }}
\def\hhatbbS{{\hat{\ensuremath{\mathbf S}} }}
\def\hhatbbT{{\hat{\ensuremath{\mathbf T}} }}
\def\hhatbbU{{\hat{\ensuremath{\mathbf U}} }}
\def\hhatbbV{{\hat{\ensuremath{\mathbf V}} }}
\def\hhatbbW{{\hat{\ensuremath{\mathbf W}} }}
\def\hhatbbX{{\hat{\ensuremath{\mathbf X}} }}
\def\hhatbbY{{\hat{\ensuremath{\mathbf Y}} }}
\def\hhatbbZ{{\hat{\ensuremath{\mathbf Z}} }}
\def\hhatbba{{\hat{\ensuremath{\mathbf a}} }}
\def\hhatbbb{{\hat{\ensuremath{\mathbf b}} }}
\def\hhatbbc{{\hat{\ensuremath{\mathbf c}} }}
\def\hhatbbd{{\hat{\ensuremath{\mathbf d}} }}
\def\hhatbbe{{\hat{\ensuremath{\mathbf e}} }}
\def\hhatbbf{{\hat{\ensuremath{\mathbf f}} }}
\def\hhatbbg{{\hat{\ensuremath{\mathbf g}} }}
\def\hhatbbh{{\hat{\ensuremath{\mathbf h}} }}
\def\hhatbbi{{\hat{\ensuremath{\mathbf i}} }}
\def\hhatbbj{{\hat{\ensuremath{\mathbf j}} }}
\def\hhatbbk{{\hat{\ensuremath{\mathbf k}} }}
\def\hhatbbl{{\hat{\ensuremath{\mathbf l}} }}
\def\hhatbbm{{\hat{\ensuremath{\mathbf m}} }}
\def\hhatbbn{{\hat{\ensuremath{\mathbf n}} }}
\def\hhatbbo{{\hat{\ensuremath{\mathbf o}} }}
\def\hhatbbp{{\hat{\ensuremath{\mathbf p}} }}
\def\hhatbbq{{\hat{\ensuremath{\mathbf q}} }}
\def\hhatbbr{{\hat{\ensuremath{\mathbf r}} }}
\def\hhatbbs{{\hat{\ensuremath{\mathbf s}} }}
\def\hhatbbt{{\hat{\ensuremath{\mathbf t}} }}
\def\hhatbbu{{\hat{\ensuremath{\mathbf u}} }}
\def\hhatbbv{{\hat{\ensuremath{\mathbf v}} }}
\def\hhatbbw{{\hat{\ensuremath{\mathbf w}} }}
\def\hhatbbx{{\hat{\ensuremath{\mathbf x}} }}
\def\hhatbby{{\hat{\ensuremath{\mathbf y}} }}
\def\hhatbbz{{\hat{\ensuremath{\mathbf z}} }}
\def\tdbbA{{\tilde{\ensuremath{\mathbf A}} }}
\def\tdbbB{{\tilde{\ensuremath{\mathbf B}} }}
\def\tdbbC{{\tilde{\ensuremath{\mathbf C}} }}
\def\tdbbD{{\tilde{\ensuremath{\mathbf D}} }}
\def\tdbbE{{\tilde{\ensuremath{\mathbf E}} }}
\def\tdbbF{{\tilde{\ensuremath{\mathbf F}} }}
\def\tdbbG{{\tilde{\ensuremath{\mathbf G}} }}
\def\tdbbH{{\tilde{\ensuremath{\mathbf H}} }}
\def\tdbbI{{\tilde{\ensuremath{\mathbf I}} }}
\def\tdbbJ{{\tilde{\ensuremath{\mathbf J}} }}
\def\tdbbK{{\tilde{\ensuremath{\mathbf K}} }}
\def\tdbbL{{\tilde{\ensuremath{\mathbf L}} }}
\def\tdbbM{{\tilde{\ensuremath{\mathbf M}} }}
\def\tdbbN{{\tilde{\ensuremath{\mathbf N}} }}
\def\tdbbO{{\tilde{\ensuremath{\mathbf O}} }}
\def\tdbbP{{\tilde{\ensuremath{\mathbf P}} }}
\def\tdbbQ{{\tilde{\ensuremath{\mathbf Q}} }}
\def\tdbbR{{\tilde{\ensuremath{\mathbf R}} }}
\def\tdbbS{{\tilde{\ensuremath{\mathbf S}} }}
\def\tdbbT{{\tilde{\ensuremath{\mathbf T}} }}
\def\tdbbU{{\tilde{\ensuremath{\mathbf U}} }}
\def\tdbbV{{\tilde{\ensuremath{\mathbf V}} }}
\def\tdbbW{{\tilde{\ensuremath{\mathbf W}} }}
\def\tdbbX{{\tilde{\ensuremath{\mathbf X}} }}
\def\tdbbY{{\tilde{\ensuremath{\mathbf Y}} }}
\def\tdbbZ{{\tilde{\ensuremath{\mathbf Z}} }}
\def\tdbba{{\tilde{\ensuremath{\mathbf a}} }}
\def\tdbbb{{\tilde{\ensuremath{\mathbf b}} }}
\def\tdbbc{{\tilde{\ensuremath{\mathbf c}} }}
\def\tdbbd{{\tilde{\ensuremath{\mathbf d}} }}
\def\tdbbe{{\tilde{\ensuremath{\mathbf e}} }}
\def\tdbbf{{\tilde{\ensuremath{\mathbf f}} }}
\def\tdbbg{{\tilde{\ensuremath{\mathbf g}} }}
\def\tdbbh{{\tilde{\ensuremath{\mathbf h}} }}
\def\tdbbi{{\tilde{\ensuremath{\mathbf i}} }}
\def\tdbbj{{\tilde{\ensuremath{\mathbf j}} }}
\def\tdbbk{{\tilde{\ensuremath{\mathbf k}} }}
\def\tdbbl{{\tilde{\ensuremath{\mathbf l}} }}
\def\tdbbm{{\tilde{\ensuremath{\mathbf m}} }}
\def\tdbbn{{\tilde{\ensuremath{\mathbf n}} }}
\def\tdbbo{{\tilde{\ensuremath{\mathbf o}} }}
\def\tdbbp{{\tilde{\ensuremath{\mathbf p}} }}
\def\tdbbq{{\tilde{\ensuremath{\mathbf q}} }}
\def\tdbbr{{\tilde{\ensuremath{\mathbf r}} }}
\def\tdbbs{{\tilde{\ensuremath{\mathbf s}} }}
\def\tdbbt{{\tilde{\ensuremath{\mathbf t}} }}
\def\tdbbu{{\tilde{\ensuremath{\mathbf u}} }}
\def\tdbbv{{\tilde{\ensuremath{\mathbf v}} }}
\def\tdbbw{{\tilde{\ensuremath{\mathbf w}} }}
\def\tdbbx{{\tilde{\ensuremath{\mathbf x}} }}
\def\tdbby{{\tilde{\ensuremath{\mathbf y}} }}
\def\tdbbz{{\tilde{\ensuremath{\mathbf z}} }}
\def\bbcalA{\mbox{\boldmath }}
\def\bbcalB{\mbox{\boldmath }}
\def\bbcalC{\mbox{\boldmath }}
\def\bbcalD{\mbox{\boldmath }}
\def\bbcalE{\mbox{\boldmath }}
\def\bbcalF{\mbox{\boldmath }}
\def\bbcalG{\mbox{\boldmath }}
\def\bbcalH{\mbox{\boldmath }}
\def\bbcalI{\mbox{\boldmath }}
\def\bbcalJ{\mbox{\boldmath }}
\def\bbcalK{\mbox{\boldmath }}
\def\bbcalL{\mbox{\boldmath }}
\def\bbcalM{\mbox{\boldmath }}
\def\bbcalN{\mbox{\boldmath }}
\def\bbcalO{\mbox{\boldmath }}
\def\bbcalP{\mbox{\boldmath }}
\def\bbcalQ{\mbox{\boldmath }}
\def\bbcalR{\mbox{\boldmath }}
\def\bbcalW{\mbox{\boldmath }}
\def\bbcalU{\mbox{\boldmath }}
\def\bbcalV{\mbox{\boldmath }}
\def\bbcalS{\mbox{\boldmath }}
\def\bbcalT{\mbox{\boldmath }}
\def\bbcalX{\mbox{\boldmath }}
\def\bbcalY{\mbox{\boldmath }}
\def\bbcalZ{\mbox{\boldmath }}
\def\bbcale{\mbox{\boldmath }}
\def\tdupsilon{\tilde\upsilon}
\def\tdalpha{\tilde\alpha}
\def\tdbeta{\tilde\beta}
\def\tdgamma{\tilde\gamma}
\def\tddelta{\tilde\delta}
\def\tdepsilon{\tilde\epsilon}
\def\tdvarepsilon{\tilde\varepsilon}
\def\tdzeta{\tilde\zeta}
\def\tdeta{\tilde\eta}
\def\tdtheta{\tilde\theta}
\def\tdvartheta{\tilde\vartheta}

\def\tdiota{\tilde\iota}
\def\tdkappa{\tilde\kappa}
\def\tdlambda{\tilde\lambda}
\def\tdmu{\tilde\mu}
\def\tdnu{\tilde\nu}
\def\tdxi{\tilde\xi}
\def\tdpi{\tilde\pi}
\def\tdrho{\tilde\rho}
\def\tdvarrho{\tilde\varrho}
\def\tdsigma{\tilde\sigma}
\def\tdvarsigma{\tilde\varsigma}
\def\tdtau{\tilde\tau}
\def\tdupsilon{\tilde\upsilon}
\def\tdphi{\tilde\phi}
\def\tdvarphi{\tilde\varphi}
\def\tdchi{\tilde\chi}
\def\tdpsi{\tilde\psi}
\def\tdomega{\tilde\omega}

\def\tdGamma{\tilde\Gamma}
\def\tdDelta{\tilde\Delta}
\def\tdTheta{\tilde\Theta}
\def\tdLambda{\tilde\Lambda}
\def\tdXi{\tilde\Xi}
\def\tdPi{\tilde\Pi}
\def\tdSigma{\tilde\Sigma}
\def\tdUpsilon{\tilde\Upsilon}
\def\tdPhi{\tilde\Phi}
\def\tdPsi{\tilde\Psi}
\def\bbarupsilon{\bar\upsilon}
\def\bbaralpha{\bar\alpha}
\def\bbarbeta{\bar\beta}
\def\bbargamma{\bar\gamma}
\def\bbarsigma{\bar\sigma}
\def\bbardelta{\bar\delta}
\def\bbarepsilon{\bar\epsilon}
\def\bbarvarepsilon{\bar\varepsilon}
\def\bbarzeta{\bar\zeta}
\def\bbareta{\bar\eta}
\def\bbartheta{\bar\theta}
\def\bbarvartheta{\bar\vartheta}

\def\bbariota{\bar\iota}
\def\bbarkappa{\bar\kappa}
\def\bbarlambda{\bar\lambda}
\def\bbarmu{\bar\mu}
\def\bbarnu{\bar\nu}
\def\bbarxi{\bar\xi}
\def\bbarpi{\bar\pi}
\def\bbarrho{\bar\rho}
\def\bbarvarrho{\bar\varrho}
\def\bbarvarsigma{\bar\varsigma}
\def\bbarphi{\bar\phi}
\def\bbarvarphi{\bar\varphi}
\def\bbarchi{\bar\chi}
\def\bbarpsi{\bar\psi}
\def\bbaromega{\bar\omega}

\def\bbarGamma{\bar\Gamma}
\def\bbarDelta{\bar\Delta}
\def\bbarTheta{\bar\Theta}
\def\bbarLambda{\bar\Lambda}
\def\bbarXi{\bar\Xi}
\def\bbarPi{\bar\Pi}
\def\bbarSigma{\bar\Sigma}
\def\bbarUpsilon{\bar\Upsilon}
\def\bbarPhi{\bar\Phi}
\def\bbarPsi{\bar\Psi}
\def\chkupsilon{\check\upsilon}
\def\chkalpha{\check\alpha}
\def\chkbeta{\check\beta}
\def\chkgamma{\check\gamma}
\def\chkdelta{\check\delta}
\def\chkepsilon{\check\epsilon}
\def\chkvarepsilon{\check\varepsilon}
\def\chkzeta{\check\zeta}
\def\chketa{\check\eta}
\def\chktheta{\check\theta}
\def\chkvartheta{\check\vartheta}

\def\chkiota{\check\iota}
\def\chkkappa{\check\kappa}
\def\chklambda{\check\lambda}
\def\chkmu{\check\mu}
\def\chknu{\check\nu}
\def\chkxi{\check\xi}
\def\chkpi{\check\pi}
\def\chkrho{\check\rho}
\def\chkvarrho{\check\varrho}
\def\chksigma{\check\sigma}
\def\chkvarsigma{\check\varsigma}
\def\chktau{\check\tau}
\def\chkupsilon{\check\upsilon}
\def\chkphi{\check\phi}
\def\chkvarphi{\check\varphi}
\def\chkchi{\check\chi}
\def\chkpsi{\check\psi}
\def\chkomega{\check\omega}

\def\chkGamma{\check\Gamma}
\def\chkDelta{\check\Delta}
\def\chkTheta{\check\Theta}
\def\chkLambda{\check\Lambda}
\def\chkXi{\check\Xi}
\def\chkPi{\check\Pi}
\def\chkSigma{\check\Sigma}
\def\chkUpsilon{\check\Upsilon}
\def\chkPhi{\check\Phi}
\def\chkPsi{\check\Psi}
\def\bbupsilon{{\mbox{\boldmath }}}
\def\bbalpha{{\mbox{\boldmath }}}
\def\bbbeta{{\mbox{\boldmath }}}
\def\bbgamma{{\mbox{\boldmath }}}
\def\bbdelta{{\mbox{\boldmath }}}
\def\bbepsilon{{\mbox{\boldmath }}}
\def\bbvarepsilon{{\mbox{\boldmath }}}
\def\bbzeta{{\mbox{\boldmath }}}
\def\bbeta{{\mbox{\boldmath }}}
\def\bbtheta{{\mbox{\boldmath }}}
\def\bbvartheta{{\mbox{\boldmath }}}

\def\bbiota{{\mbox{\boldmath }}}
\def\bbkappa{{\mbox{\boldmath }}}
\def\bblambda{{\mbox{\boldmath }}}
\def\bbmu{{\mbox{\boldmath }}}
\def\bbnu{{\mbox{\boldmath }}}
\def\bbxi{{\mbox{\boldmath }}}
\def\bbpi{{\mbox{\boldmath }}}
\def\bbrho{{\mbox{\boldmath }}}
\def\bbvarrho{{\mbox{\boldmath }}}
\def\bbvarsigma{{\mbox{\boldmath }}}
\def\bbphi{{\mbox{\boldmath }}}
\def\bbvarphi{{\mbox{\boldmath }}}
\def\bbchi{{\mbox{\boldmath }}}
\def\bbpsi{{\mbox{\boldmath }}}
\def\bbomega{{\mbox{\boldmath }}}

\def\bbGamma{{\mbox{\boldmath }}}
\def\bbDelta{{\mbox{\boldmath }}}
\def\bbTheta{{\mbox{\boldmath }}}
\def\bbLambda{{\mbox{\boldmath }}}
\def\bbXi{{\mbox{\boldmath }}}
\def\bbPi{{\mbox{\boldmath }}}
\def\bbSigma{{\mbox{\boldmath }}}
\def\bbUpsilon{{\mbox{\boldmath }}}
\def\bbPhi{{\mbox{\boldmath }}}
\def\bbPsi{{\mbox{\boldmath }}}

\def\bbarbbupsilon{\bar\bbupsilon}
\def\bbarbbalpha{\bar\bbalpha}
\def\bbarbbbeta{\bar\bbbeta}
\def\bbarbbgamma{\bar\bbgamma}
\def\bbarbbdelta{\bar\bbdelta}
\def\bbarbbepsilon{\bar\bbepsilon}
\def\bbarbbvarepsilon{\bar\bbvarepsilon}
\def\bbarbbzeta{\bar\bbzeta}
\def\bbarbbeta{\bar\bbeta}
\def\bbarbbtheta{\bar\bbtheta}
\def\bbarbbvartheta{\bar\bbvartheta}

\def\bbarbbiota{\bar\bbiota}
\def\bbarbbkappa{\bar\bbkappa}
\def\bbarbblambda{\bar\bblambda}
\def\bbarbbmu{\bar\bbmu}
\def\bbarbbnu{\bar\bbnu}
\def\bbarbbxi{\bar\bbxi}
\def\bbarbbpi{\bar\bbpi}
\def\bbarbbrho{\bar\bbrho}
\def\bbarbbvarrho{\bar\bbvarrho}
\def\bbarbbvarsigma{\bar\bbvarsigma}
\def\bbarbbphi{\bar\bbphi}
\def\bbarbbvarphi{\bar\bbvarphi}
\def\bbarbbchi{\bar\bbchi}
\def\bbarbbpsi{\bar\bbpsi}
\def\bbarbbomega{\bar\bbomega}

\def\bbarbbGamma{\bar\bbGamma}
\def\bbarbbDelta{\bar\bbDelta}
\def\bbarbbTheta{\bar\bbTheta}
\def\bbarbbLambda{\bar\bbLambda}
\def\bbarbbXi{\bar\bbXi}
\def\bbarbbPi{\bar\bbPi}
\def\bbarbbSigma{\bar\bbSigma}
\def\bbarbbUpsilon{\bar\bbUpsilon}
\def\bbarbbPhi{\bar\bbPhi}
\def\bbarbbPsi{\bar\bbPsi}
\def\hhatbbupsilon{\hat\bbupsilon}
\def\hhatbbalpha{\hat\bbalpha}
\def\hhatbbbeta{\hat\bbbeta}
\def\hhatbbgamma{\hat\bbgamma}
\def\hhatbbdelta{\hat\bbdelta}
\def\hhatbbepsilon{\hat\bbepsilon}
\def\hhatbbvarepsilon{\hat\bbvarepsilon}
\def\hhatbbzeta{\hat\bbzeta}
\def\hhatbbeta{\hat\bbeta}
\def\hhatbbtheta{\hat\bbtheta}
\def\hhatbbvartheta{\hat\bbvartheta}

\def\hhatbbiota{\hat\bbiota}
\def\hhatbbkappa{\hat\bbkappa}
\def\hhatbblambda{\hat\bblambda}
\def\hhatbbmu{\hat\bbmu}
\def\hhatbbnu{\hat\bbnu}
\def\hhatbbxi{\hat\bbxi}
\def\hhatbbpi{\hat\bbpi}
\def\hhatbbrho{\hat\bbrho}
\def\hhatbbvarrho{\hat\bbvarrho}
\def\hhatbbvarsigma{\hat\bbvarsigma}
\def\hhatbbphi{\hat\bbphi}
\def\hhatbbvarphi{\hat\bbvarphi}
\def\hhatbbchi{\hat\bbchi}
\def\hhatbbpsi{\hat\bbpsi}
\def\hhatbbomega{\hat\bbomega}

\def\hhatbbGamma{\hat\bbGamma}
\def\hhatbbDelta{\hat\bbDelta}
\def\hhatbbTheta{\hat\bbTheta}
\def\hhatbbLambda{\hat\bbLambda}
\def\hhatbbXi{\hat\bbXi}
\def\hhatbbPi{\hat\bbPi}
\def\hhatbbSigma{\hat\bbSigma}
\def\hhatbbUpsilon{\hat\bbUpsilon}
\def\hhatbbPhi{\hat\bbPhi}
\def\hhatbbPsi{\hat\bbPsi}
\def\tdbbupsilon{\tilde\bbupsilon}
\def\tdbbalpha{\tilde\bbalpha}
\def\tdbbbeta{\tilde\bbbeta}
\def\tdbbgamma{\tilde\bbgamma}
\def\tdbbdelta{\tilde\bbdelta}
\def\tdbbepsilon{\tilde\bbepsilon}
\def\tdbbvarepsilon{\tilde\bbvarepsilon}
\def\tdbbzeta{\tilde\bbzeta}
\def\tdbbeta{\tilde\bbeta}
\def\tdbbtheta{\tilde\bbtheta}
\def\tdbbvartheta{\tilde\bbvartheta}

\def\tdbbiota{\tilde\bbiota}
\def\tdbbkappa{\tilde\bbkappa}
\def\tdbblambda{\tilde\bblambda}
\def\tdbbmu{\tilde\bbmu}
\def\tdbbnu{\tilde\bbnu}
\def\tdbbxi{\tilde\bbxi}
\def\tdbbpi{\tilde\bbpi}
\def\tdbbrho{\tilde\bbrho}
\def\tdbbvarrho{\tilde\bbvarrho}
\def\tdbbvarsigma{\tilde\bbvarsigma}
\def\tdbbphi{\tilde\bbphi}
\def\tdbbvarphi{\tilde\bbvarphi}
\def\tdbbchi{\tilde\bbchi}
\def\tdbbpsi{\tilde\bbpsi}
\def\tdbbomega{\tilde\bbomega}

\def\tdbbGamma{\tilde\bbGamma}
\def\tdbbDelta{\tilde\bbDelta}
\def\tdbbTheta{\tilde\bbTheta}
\def\tdbbLambda{\tilde\bbLambda}
\def\tdbbXi{\tilde\bbXi}
\def\tdbbPi{\tilde\bbPi}
\def\tdbbSigma{\tilde\bbSigma}
\def\tdbbUpsilon{\tilde\bbUpsilon}
\def\tdbbPhi{\tilde\bbPhi}
\def\tdbbPsi{\tilde\bbPsi}
\def\deltat{\triangle t}
\def\hhattheta{\hat\theta}
\def\var{{\rm var}}
\def\bbtau{{\mbox{\boldmath }}}

\makeatletter
\def\wideubar{\underaccent{{\cc@style\underline{\mskip8mu}}}}
\def\Wideubar{\underaccent{{\cc@style\underline{\mskip6mu}}}}
\makeatother

\makeatletter
\def\widebar{\accentset{{\cc@style\underline{\mskip8mu}}}}
\def\Widebar{\accentset{{\cc@style\underline{\mskip6mu}}}}
\makeatother



 
\usepackage{subfigure}

\usepackage{algorithm}
\def \trace {{\rm tr}}



\begin{document}

\title{Moving-Horizon Dynamic Power System State Estimation Using Semidefinite Relaxation}


\author{\IEEEauthorblockN{Gang Wang, Seung-Jun Kim, and Georgios B. Giannakis}
\IEEEauthorblockA{ School of Automation, Beijing Institute of Technology\\
Beijing 100081, China\\
 Dept. of ECE and Digital Tech. Center, Univ. of Minnesota\\
Minneapolis, MN 55455, USA\\
E-mail: \{wang4937,seungjun,georgios\}@umn.edu}
\thanks{This work was supported by NSF grant 1202135, and by the Institute of Renewable Energy and the Environment (IREE) at the University of Minnesota, under grant No.~RL-0010-13. G. Wang was supported in part by the China Scholarship Council.}}


\maketitle

\begin{abstract}
Accurate power system state estimation (PSSE) is an essential prerequisite for reliable operation of power systems. Different from static PSSE, dynamic PSSE can exploit past measurements based on a dynamical state evolution model, offering improved accuracy and state predictability. A key challenge is the nonlinear measurement model, which is often tackled using linearization, despite divergence and local optimality issues. In this work, a moving-horizon estimation (MHE) strategy is advocated, where model nonlinearity can be accurately captured with strong performance guarantees. To mitigate local optimality, a semidefinite relaxation approach is adopted, which often provides solutions close to the global optimum. Numerical tests show that the proposed method can markedly improve upon an extended Kalman filter (EKF)-based alternative.

\end{abstract}

\begin{IEEEkeywords}
Dynamic power system state estimation, moving-horizon state estimation, semidefinite relaxation.
\end{IEEEkeywords}




\section{Introduction}
\label{sec1}
The electric power system is a large-scale cyber-physical system, composed of thousands of physical and computational modules, spanning over a wide geographical area. The energy management system (EMS)/supervisory control and data acquisition (SCADA) systems are responsible for monitoring, control and optimization of the power grid, performing a slew of tasks including bad data detection and analysis, economic dispatch, and optimal power flow~\cite{book:abur2004, ieee:monticelli2000, spm:giannakis2013}. Accurate power system state estimation (PSSE) is an essential prerequisite for these functions, providing the operator with basic visibility to real-time states of power systems. PSSE is also critical for security assessment necessary to detect instabilities and contingencies, and to determine necessary corrective actions~\cite{spm:huang2012}.




The state of a power system refers to the complex voltages consisting of voltage magnitudes and phase angles, at all buses in the grid. Given the network topology and impedance parameters, all nodal and line electrical quantities of interest are completely characterized by the system states. The goal of PSSE is to estimate the system states from the measurements of related quantities, such as power injections and flows, and voltage magnitudes and angles, at a subset of buses. Depending on whether system dynamics are taken into account, PSSE can be divided into two paradigms: static PSSE and dynamic PSSE (also called forecasting-aided PSSE)~\cite{spm:huang2012}.

When SCADA measurements are involved, the PSSE problem becomes nonlinear and nonconvex. Traditionally, PSSE has been solved via weighted nonlinear least-squares, invoking Gauss-Newton iterations. Thus, the method is potentially susceptible to locally optimal solutions, sensitive to initialization, and troubled with convergence issues. This may become increasingly problematic in the challenging scenarios of future power systems, where system states may change significantly between measurements due to, e.g., massive integration renewables, the presence of bad data, or, cyber-attacks.



A recent progress made for mitigating these issues in the context of static PSSE is based on a semidefinite relaxation (SDR) approach, which was demonstrated empirically to yield solutions close to the globally optimal ones at polynomial-time complexity~\cite{naps:zhu2011}. SDR is well motivated in various applications in signal processing and communication~\cite{spm:luo2010}, as well as in optimal power flow problems~\cite{bai2008, tps:low2012, tsm:dall2013}. In a nutshell, the measurement model, which is nonlinear in system states , is re-expressed as {\em linear} in the rank- outer product  ( denotes Hermitian transpose), which leads to a semidefinite programming problem except for the rank- constraint. Dropping the nonconvex rank constraints yields a convex problem, from whose solutions the desired rank- solutions can be recovered using various heuristics.




While static PSSE utilizes only the measurements of current time, dynamic PSSE can leverage past measurements as well, based on the dynamical model governing the system states. The dynamics of power systems could be due to the changing frequency, or the changing line parameters. To circumvent the nonlinearity in the measurement model, approximate state estimation techniques such as the extended Kalman filter (EKF) and the unscented Kalman filter (UKF) have been advocated~\cite{HuS02, Val11}. However, such approximations can suffer from divergence due to their inability to accurately incorporate the underlying nonlinear dynamics.





Recently, moving-horizon estimation (MHE) for nonlinear dynamical systems has attracted much attention~\cite{auto:alessandri2008}, because it can provide state estimates with bounded error under appropriate assumptions~\cite{auto:alessandri2008}. Moreover, constrained MHE has been shown to offer an asymptotically stable estimator for nonlinear dynamical systems with deterministic noise terms~\cite{tac:rao2003}. Rigorous comparison of MHE and EKF for nonlinear chemical processes corroborated the robustness and improved estimation performance of the MHE method~\cite{haseltine2005critical}. When applied to the dynamic PSSE problem, however, the MHE formulation is still nonconvex, and thus difficult to yield globally optimal solutions. The key contribution of the present paper is to leverage SDR to convexify the problem and thus attain efficient near-optimal solutions.




The remainder of this paper is organized as follows. The power grid model and the (static) PSSE formulation are introduced in Section~\ref{sec2}. The SDR approach for PSSE is reviewed in Section~\ref{sec3}. In Section~\ref{sec4}, the MHE strategy for dynamic PSSE is presented and the SDR reformulation is described. The results of numerical tests are presented in Section~\ref{sec5}, and the conclusions are drawn in Section~\ref{sec6}.

\emph{Notations:} All matrices (vectors) are denoted by boldface letters.  and  represent transpose and complex-conjugate transpose, respectively;  is the matrix Frobenius norm,  the vector Euclidean norm,  the matrix trace, and  the matrix rank; finally,  signifies the magnitude of a complex number.



\section{Modeling and Problem Formulation}
\label{sec2}
Consider a power transmission network with  buses. The set of all buses is denoted by  and the set of all lines by  In order to estimate the complex voltages  at all buses, collected in the state vector   measurements of the following types are taken: Active (reactive) power injection at bus  denoted by  ; active (reactive) power flow out of bus  to bus  denoted by  ; and voltage magnitude at bus  denoted by .
Then, collect the measurements in an  vector   with     and  denoting the sets of buses or lines where the corresponding measurements are taken.

It turns out that the measured quantities  ,  are quadratic functions of  To specify this, collect injected currents at all buses in vector  and let  denote the so-called bus admittance matrix, whose entries are defined as

where  is line admittance between buses  and   the shunt admittance of bus  to the ground; and  the set of buses with transmission lines connected to bus  Upon denoting the shunt admittance at bus  corresponding to line  by , the current flowing from bus  to bus  is given by  Then the complex power injection at bus  is  and the complex power flowing out from bus  to bus  is  Likewise, the squared bus voltage magnitude can also be expressed as  Then, the measurement model is given by

where  is quadratic in , and  is the measurement noise.




The goal of PSSE is to obtain an estimate of  from  The static PSSE is formulated as a weighted nonlinear least-squares (LS) problem given by

where  represents the weight for the -th measurement, inversely proportional to the variance of  Problem (\ref{wnls}) is nonlinear and nonconvex. Thus, iterative algorithms based on Gauss-Newton updates are often employed to find locally optimal solutions. Next, an SDR approach that targets globally optimal solutions is reviewed.

\section{SDR Approach for PSSE}
\label{sec3}
The idea is to re-express the quadratic function  of  as a linear function of the  rank- matrix  Let  denote the -th canonical basis of  and define a number of admittance-related matrices

together with

With these definitions, the following relations hold for every  and every 

Then,  can be expressed as

where  is one of  and  corresponding to the type of the -th measurement. Then problem (\ref{wnls}) is equivalent to

\big\{\hat\bbV\big\}:=&{\rm{arg}}\mathop{\rm{min}}
\limits_{\bbV\in\mathbb{C}^{n\times n}}\sum_{\ell=1}^{L}w^\ell\Big(z^\ell-{\rm{Tr}}\big(\bbH^\ell\bbV\big)\Big)^2\label{ssevkn}\\
{\text{s. to}}~ \bbV&\succeq \bb0,~{\text{and}}~{\rm{rank}}(\bbV)=1\label{sserank1}.

SDR amounts to dropping the nonconvex rank constraint in (\ref{sserank1}), yielding a convex optimization problem, which can be efficiently solved.




\section{SDR-Based MHE for Dynamic PSSE}
\label{sec4}
\subsection{MHE for Dynamic PSSE}
For dynamic PSSE, the state-space model adopted is:

\bbv_{k+1}=&\,\bbF_k{\bbv}_k+{\bbxi}_k\label{system eq}\\
\bbz_k=&\,\bbh(\bbv_k)+\bbeta_k\label{observation eq}

with the following notations








Different from standard Kalman filtering set-ups, the initial state  the process noise  and the measurement noise  in MHE are assumed to be unknown deterministic vectors, which take values from  and  respectively. The constraints  and  can be interpreted as a strategy for modeling the bounded disturbances or random variables with truncated densities \cite{tac:rao2003}. We resort to the MHE strategy to perform dynamic PSSE because of well-appreciated advantages of MHE in nonlinear state estimation, such as accurate yet tractable incorporation of nonlinearities with consequent asymptotic stability. To be specific, given appropriate assumptions including that the nonlinear system is uniformly observable, and that the states belong to a compact set, MHE turns out to be an asymptotically stable observer  \cite[Prop. 3.4]{tac:rao2003}. \iffalse; see, e.g., Prop. 3.4 in \cite{tac:rao2003} for more details.\fi


The key idea behind MHE is to capitalize on a sliding window of past observations to perform state estimation. Thus, the information vector containing
 past measurements as well as the current one at time  is given by


Let  denote the smoothed state estimate at time  given  The
MHE strategy focuses on obtaining  at any time  based on the most recent
estimate  and  This prior estimate  is simply obtained as

where  is an \emph{a priori} prediction of initial state 
Denote by  the estimates of states  respectively, to be calculated at time  A notable simplification of the estimation scheme to obtain estimates  can be based upon  through the ``noise-free`` dynamic update, that is,

Therefore, per time instant , it is only necessary to determine  since the other  estimates can be iteratively computed via (\ref{estimates}).





Considering that the statistics of  are unknown, the LS estimation criterion is given by

where the nonnegative weights  and  are design parameters, tuned depending on the relative confidence in the state prediction  and the measurements, respectively. In a nutshell, the MHE strategy can be stated as follows.

{\bf{\emph{MHE Strategy:}}}\label{problem 1} At any time  given  find estimates  via

and (\ref{estimates}), where  is propagated as

with an initialization 

\subsection{SDR for MHE}

The limitation of MHE is that it requires online solutions of dynamic (nonconvex) optimization problems [cf. (\ref{problem})], which are typically solved by Gauss-Newton iterations. Instead, the fresh idea here is to leverage the SDR technique to find the near globally optimum solutions for MHE-based dynamic PSSE.

In light of (\ref{estimates}),  can be directly calculated once we obtain the estimate  Similarly, by defining  we can propagate the noise-free dynamics to obtain [cf. (\ref{estimates})]

Since  holds, by substituting (\ref{matrix}) into this relation, and defining  one obtains

\bbH_{s}^{P,n}:=\,&\frac{1}{2}\bbT_s^\ccalH\left(\bbY^n+(\bbY^ns)^{\ccalH}\right)\bbT_s\label{hspn}\\
\bbH_{s}^{P,mn}:=\,&\frac{1}{2}\bbT_s^\ccalH\left(\bbY^{mn}+(\bbY^{mn})^{\ccalH}\right)\bbT_s\\
\bbH_{s}^{Q,n}:=\,&\frac{j}{2}\bbT_s^\ccalH\left(\bbY^n-(\bbY^n)^{\ccalH}\right)\bbT_s\label{hsqn}\\
\bbH_{s}^{Q,mn}:=\,&\frac{j}{2}\bbT_s^\ccalH\left(\bbY^{mn}-(\bbY^{mn})^{\ccalH}\right)\bbT_s\\
{\text{and~~}}\bbH_{s}^{V,n}:=\,&\bbT_s^\ccalH(\bbe^n(\bbe^n)^\ccalT)\bbT_s\label{hsvn}.

Then, it can be clearly seen that

Moreover, let  be the outer product formed from the prior estimate  Approaches to obtain  from  will be discussed later.

Then, the SDR-based MHE problem can be formulated as

&\left\{\hat\bbV_{k-M|k}\right\}:={\rm{arg}}\mathop{\rm{min}}
\limits_{\bbV_{k\!-\!M\!|\!k}\in\mathbb{C}^{N\!\times\! N}}\mu \big\|\bbV_{k-M|k}-\bar\bbV_{k-M}\big\|_F^2+\nonumber\\
&\qquad\qquad\qquad\lambda\sum_{s=0}^{M}\!\sum_{\ell=1}^{L}\Big(z_{k\!-\!M+s}^\ell\!-\!{\rm{Tr}}
\big(\bbH_{s}^\ell\bbV_{k\!-\!M|k}\big)\Big)^2\label{vkn}\\
&\quad{\text{s. to}}\quad \bbV_{k-M|k}\succeq \bb0 \label{psd1}\\
&\quad\quad\quad\quad{\rm{rank}}(\bbV_{k-M|k})=1\label{rank1}

where

with . The positive semidefinite constraint \eqref{psd1} together with the rank constraint \eqref{rank1} ensure that  can be recovered from . However, this formulation is nonconvex due to the rank constraint. Thus, SDR approach amounts to removing (\ref{rank1}) to obtain a convex optimization problem.

Due to the relaxation, optimal solution  to the SDR problem (\ref{vkn})-(\ref{psd1}) may be of rank greater than 1. Still, one can recover   from  using a number of heuristics. One way is to perform eigen-decomposition of  as
 
where  is the rank of ,  are the ordered eigenvalues of , and  is the eigenvector corresponding to  eigenvalue . Then, the best rank-one approximation of  in the LS sense is . Thus,  can be approximated to .
Another approach is to resort to randomization, where one generates random vectors according to , and picks the one that minimizes the cost.
Such a randomization procedure has been empirically found to yield reasonable performance~\cite{spm:luo2010}. Once  is obtained via any of such heuristics, estimates   can be found again using (\ref{estimates}).

The computational complexity of solving \eqref{vkn}--\eqref{psd1} is rather high in the present form, although it is still polynomial in . There are two promising directions under investigation. One is to exploit rich sparsity structure in  to simplify the formulation. Another direction is to consider special network structures such as the radial topology common in transmission networks. This allows second-order cone programming (SOCP) formulations, which can be solved faster~\cite{Far13}.

\section{Numerical Tests}
\label{sec5}


 \begin{figure}
\centering
\includegraphics[scale=0.62]{rmse3.eps}
\caption{RMSE performance comparison.}
\label{fig1}
\end{figure}

 \begin{figure}
\centering
\includegraphics[width=9.1cm,height=8.3cm]{v22.eps}
\caption{Evolution of the real and the imaginary parts of .}
\label{fig2}
\end{figure}

 \begin{figure}
\centering
\includegraphics[width=9.1cm,height=8.3cm]{v44.eps}
\caption{Evolution of the real and the imaginary parts of .}
\label{fig3}
\end{figure}

 \begin{figure}
\centering
\includegraphics[width=9.1cm,height=8.3cm]{v66.eps}
\caption{Evolution of the real and the imaginary parts of .}
\label{fig4}
\end{figure}



The proposed SDR-based MHE approach was tested using the IEEE 6-bus system with 11 transmission lines, and compared to an existing approach that is based on the EKF \cite{HuS02}. For this, a Matlab toolbox called MATPOWER \cite{tps:thomas2011}
 was used to generate the pertinent power flows and meter measurements. To solve (\ref{vkn})-(\ref{psd1}), the CVX and SeDuMi packages were used \cite{cvx}, \cite{sedumi}.

 To simulate the slow evolution of the power system states, transition matrix  was employed and each entry of the process noise  was generated by having both the magnitude and angle sampled according to a uniform distribution over the interval 
 The voltage magnitudes of initial state  was formed to have Gaussian distributed entries with mean  and standard deviation  and angles uniformly distributed over  Bus  was chosen as the reference with angle  in order to fix the phase angle ambiguity \cite{naps:zhu2011}. The active and reactive power flows across lines 1-7 together with voltage magnitudes at all 6 buses were measured. Every measurement was corrupted by noise randomly generated over interval   The simulation horizon, the length of the sliding window are, respectively,  and the design parameters are set to  and 

Fig. \ref{fig1} compares the root-mean-square-errors (RMSEs) of the proposed approach against those of EKF, where the results were based upon 100 independent realizations. Fig. \ref{fig2}-\ref{fig4} depict the dynamic evolution of both estimates calculated by the two approaches together with the true states  of bus    of bus   of bus  respectively, where the real part is shown at the top panel and the imaginary part at the bottom. It can be clearly seen that the proposed method exhibits improved RMSE performance relative to EKF, which may even diverge from the true state depending on the initialization.



\section{Conclusion}
\label{sec6}
A dynamic PSSE algorithm has been proposed for power systems, which capitalizes on a set of recent measurements in a sliding window fashion. Since the measurement model for power grids is inherently nonlinear, traditional dynamic PSSE methods have relied on EKF/UKF approaches. Unfortunately, depending on initialization and the severity of dynamics, existing algorithms may be divergent. In contrast, the proposed approach leverages the MHE strategy and the SDR technique to accurately incorporate nonlinear dynamics, thus providing improved estimation accuracy and robustness. Numerical tests using the IEEE 6-bus system corroborated those performance claims. Further enhancements to account for false data injection as well as to reduce computational complexity are left for future work. 















\begin{thebibliography}{10}
\providecommand{\url}[1]{#1}
\csname url@samestyle\endcsname
\providecommand{\newblock}{\relax}
\providecommand{\bibinfo}[2]{#2}
\providecommand{\BIBentrySTDinterwordspacing}{\spaceskip=0pt\relax}
\providecommand{\BIBentryALTinterwordstretchfactor}{4}
\providecommand{\BIBentryALTinterwordspacing}{\spaceskip=\fontdimen2\font plus
\BIBentryALTinterwordstretchfactor\fontdimen3\font minus
  \fontdimen4\font\relax}
\providecommand{\BIBforeignlanguage}[2]{{\expandafter\ifx\csname l@#1\endcsname\relax
\typeout{** WARNING: IEEEtran.bst: No hyphenation pattern has been}\typeout{** loaded for the language `#1'. Using the pattern for}\typeout{** the default language instead.}\else
\language=\csname l@#1\endcsname
\fi
#2}}
\providecommand{\BIBdecl}{\relax}
\BIBdecl

\bibitem{book:abur2004}
A.~Abur and A.~G. Exposito, \emph{Power System State Estimation: Theory and
  Implementation}.\hskip 1em plus 0.5em minus 0.4em\relax CRC Press, 2004.



\bibitem{ieee:monticelli2000}
A.~Monticelli, ``Electric power system state estimation,'' \emph{Proc. IEEE},
  vol.~88, no.~2, pp. 262--282, 2000.

\bibitem{spm:giannakis2013}
G.~B. Giannakis, V.~Kekatos, N.~Gatsis, S.-J. Kim, H.~Zhu, and B.~F.
  Wollenberg, ``Monitoring and optimization for power grids,'' \emph{IEEE Sig.
  Proc. Mag.}, vol.~30, no.~5, pp. 107--128, Sep. 2013.

\bibitem{spm:huang2012}
Y.-F. Huang, S.~Werner, J.~Huang, N.~Kashyap, and V.~Gupta, ``State estimation
  in electric power grids: Meeting new challenges presented by the requirements
  of the future grid,'' \emph{IEEE Sig. Proc. Mag.}, vol.~29, no.~5, pp.
  33--43, 2012.

\bibitem{naps:zhu2011}
H.~Zhu and G.~B. Giannakis, ``Estimating the state of {AC} power systems using
  semidefinite programming,'' in \emph{IEEE PES General Meeting}, Boston, MA,
  Aug. 2011, pp. 1--7.

\bibitem{spm:luo2010}
Z.-Q. Luo, W.-K. Ma, A.-C. So, Y.~Ye, and S.~Zhang, ``Semidefinite relaxation
  of quadratic optimization problems,'' \emph{IEEE Sig. Proc. Mag.}, vol.~27,
  no.~3, pp. 20--34, 2010.

\bibitem{bai2008}
X.~Bai, H.~Wei, K.~Fujisawa, and Y.~Wang, ``Semidefinite programming for
  optimal power flow problems,'' \emph{Intl. J. Electr. Power \& Energy Syst.},
  vol.~30, no. 6--7, pp. 383--392, Jul. 2008.

\bibitem{tps:low2012}
J.~Lavaei and S.~H. Low, ``Zero duality gap in optimal power flow problem,''
  \emph{IEEE Trans. Power Syst.}, vol.~27, no.~1, pp. 92--107, Feb. 2012.

\bibitem{tsm:dall2013}
E.~Dall'Anese, H.~Zhu, and G.~Giannakis, ``Distributed optimal power flow for
  smart microgrids,'' \emph{IEEE Trans. Smart Grid}, vol.~4, no.~3,
  pp. 1464--1475, 2013.

\bibitem{HuS02}
S.-J. Huang and K.-R. Shih, ``Dynamic-state-estimation scheme including
  nonlinear measurement-function considerations,'' \emph{IEE Proc. Gener.
  Transm. Distrib.}, vol. 149, no.~6, pp. 673--678, Nov. 2002.

\bibitem{Val11}
G.~Valverde and V.~Terzija, ``Unscented Kalman filter for power system dynamic
  state estimation,'' \emph{IET Gener. Transm. Distrib.}, vol.~5, no.~1, pp.
  29--37, 2011.

\bibitem{auto:alessandri2008}
A.~Alessandri, M.~Baglietto, and G.~Battistelli, ``Moving-horizon state
  estimation for nonlinear discrete-time systems: New stability results and
  approximation schemes,'' \emph{Automatica}, vol.~44, no.~7, pp. 1753--1765,
  2008.

\bibitem{tac:rao2003}
C.~V. Rao, J.~B. Rawlings, and D.~Q. Mayne, ``Constrained state estimation for
  nonlinear discrete-time systems: Stability and moving horizon
  approximations,'' \emph{IEEE Trans. Automat. Contr.}, vol.~48, no.~2, pp.
  246--258, 2003.

\bibitem{haseltine2005critical}
E.~L. Haseltine and J.~B. Rawlings, ``Critical evaluation of extended Kalman
  filtering and moving-horizon estimation,'' \emph{Industrial and Engineering
  Chemistry Res.}, vol.~44, no.~8, pp. 2451--2460, 2005.

\bibitem{Far13}
M. Farivar and S. H. Low, ``Branch flow model: Relaxations and convexification---Part I," \emph{IEEE Trans. Power Syst.}, vol.~28, no.~3, pp.~2554--2564, Aug.~2013.

\bibitem{tps:thomas2011}
R.~D. Zimmerman, C.~E. Murillo-S{\'a}nchez, and R.~J. Thomas, ``MATPOWER:
  Steady-state operations, planning, and analysis tools for power systems
  research and education,'' \emph{IEEE Trans. Power Syst.}, vol.~26, no.~1, pp.
  12--19, 2011.

\bibitem{cvx}
G.~M and S.~Boyd, ``CVX: Matlab software for disciplined convex programming,''
  Version 2.0 (beta), November 2013. [Online]. Available: http://cvxr.com/cvx/.

\bibitem{sedumi}
J.~F. Sturm, ``Using SeDuMi 1.02, a matlab toolbox for optimization over
  symmetric cones,'' \emph{Optimization Methods and Software}, vol.~11, no.
  1-4, pp. 625--653, 1999.

\end{thebibliography}

\end{document}
