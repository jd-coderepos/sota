\documentclass[envcountsect,envcountsame,runningheads,a4paper]{llncs}

\renewcommand{\topfraction}{0.9}
\renewcommand{\bottomfraction}{0.8}
\renewcommand{\textfraction}{0.05}

\usepackage{graphicx}
\usepackage{cite}
\usepackage{amssymb,amsmath,amsfonts} 
\usepackage{color}
\usepackage{subfigure}

\renewcommand{\paragraph}[1]{\smallskip\noindent\textit{#1}}

\newtheorem{conj}[theorem]{Conjecture}
\newtheorem{lem}[theorem]{Lemma}
\newtheorem{fact}[theorem]{Fact}
\newtheorem{prop}[theorem]{Proposition}
\newtheorem{obs}[theorem]{Observation}
\newtheorem{corol}[theorem]{Corollary}
\newtheorem{defn}[theorem]{Definition}



\DeclareMathOperator{\conv}{CH}
\DeclareMathOperator{\NN}{NN}
\DeclareMathOperator{\UH}{UH}
\DeclareMathOperator{\lev}{lev}

\newcommand{\eqdef}{:=}
\newcommand{\eps}{\varepsilon}
\DeclareMathOperator{\EX}{\mathbf{Exp}}
\def\SS{{\mathbb S}}

\newcommand\polylog{{\rm polylog}}
\newcommand\A{{\mathcal A}}
\newcommand\B{{\mathcal B}}
\newcommand\C{{\mathcal C}}
\newcommand\D{{\mathcal D}}
\newcommand\E{{\mathcal E}}
\newcommand\F{{\mathcal F}}
\newcommand\G{{\mathcal G}}
\def\L{{\mathcal L}}
\def\K{{\mathcal K}}
\def\M{{\mathcal M}}
\def\N{{\mathbb N}}
\def\P{{\mathcal P}}
\def\Q{{\mathcal Q}}
\def\cR{{\mathcal R}}
\def\S{{\mathcal S}}
\def\U{{\mathcal U}}
\def\T{{\mathcal T}}
\def\W{{\mathcal W}}
\def\V{{\mathcal V}}
\def\Z{{\mathbb Z}}
\def\UC{{\mathcal U\! C}}
\def\etal{\textit{et~al.}}
\def\DD{{\small \Delta}}
\def\bd{{\partial}}


\newcommand\R{{\mathbb R}}
\def\aa{{\bf a}}
\def\bb{{\bf b}}
\def\hh{{\bf h}}
\def\kk{{\bf k}}
\def\pp{{\bf p}}
\def\qq{{\bf q}}
\def\xx{{\bf x}}
\def\yy{{\bf y}}
\def\uu{{\bf u}}
\def\vv{{\bf v}}
\def\ww{{\bf w}}


\title{Interference Minimization in Asymmetric Sensor Networks\thanks{KB supported in part by the Netherlands Organisation for Scientific 
Research (NWO) under project no.\ 612.001.207.
WM supported in part by DFG Grants MU 3501/1
and MU 3502/2.
}
}

\author{
Yves Brise\inst{1}
\and
Kevin Buchin\inst{2}
\and Dustin Eversmann\inst{3}
\and Michael Hoffmann\inst{1}
\and Wolfgang~Mulzer\inst{3}
}

\authorrunning{Y.~Brise, K.~Buchin, D.~Eversmann, M.~Hoffmann and W.~Mulzer}

\institute{
ETH Z\"urich, Switzerland,
{\tt hoffmann@inf.ethz.ch}
\and
TU Eindhoven, The Netherlands,
{\tt k.a.buchin@tue.de}
\and
FU Berlin, Germany,
{\tt mulzer@inf.fu-berlin.de}
}
\begin{document}
\maketitle

\begin{abstract}
A fundamental problem in wireless sensor networks is to connect
a given set of sensors while minimizing the \emph{receiver
interference}.
This is modeled as follows: each sensor node corresponds
to a point in  and each \emph{transmission range} corresponds
to a ball. The receiver interference of a sensor node is
defined as the number of transmission ranges it lies in.
Our goal is to choose transmission radii that
minimize the maximum interference while maintaining a
strongly connected asymmetric communication graph.

For the two-dimensional case, we show that it is NP-complete
to decide whether one can achieve a receiver interference
of at most . In the one-dimensional case, we prove that there are
optimal solutions with nontrivial structural properties. These
properties can be exploited to obtain an
exact algorithm that runs in quasi-polynomial time.
This generalizes a result by Tan~et al.~to the asymmetric case.
\end{abstract}
\section{Introduction}

Wireless sensor networks constitute a popular paradigm in mobile networks:
several small independent devices are distributed in a certain region,
and each device has limited computational resources. The devices
can communicate through a wireless network. Since battery life is limited,
it is imperative that the overhead for the communication be kept as small
as possible.
One major concern when trying to achieve this goal is to control the
\emph{interference} caused by competing senders. This enables us
to reduce the range of the senders, thus increasing
battery life. At the same time, we need to
ensure that the resulting communication graph remains connected.

There are many different ways to formalize the problem of
interference minimization.  Usually, the devices are modeled as
points in -dimensional space, and
the transmission ranges are modeled as -dimensional balls.
Each point can choose the radius of its transmission range, and
different choices of transmission ranges lead to different
reachability structures.
There are two ways to interpret the resulting
communication graph. In the \emph{symmetric} case,
the communication graph is undirected, and it contains an edge between
two points  and  if and only if both  and  lie in the transmission
range of the other point. For a valid
assignment of transmission ranges, we require that
the communication graph is connected.
In the \emph{asymmetric} case, the communication graph is directed,
and there is an edge from  to  if and only if  lies in the
transmission range of . We require that the communication graph
is strongly connected, or, in a slightly different model,
that there is one point that is reachable from
every other point through a directed path.

In both the symmetric and the asymmetric case, the
(\emph{receiver-centric}) \emph{interference}
of a point is defined as the number of transmission
ranges that it lies in~\cite{RickenbachWaZo09}. The goal is to find a
valid assignment of transmission ranges that makes the maximum interference
as small as possible. We refer to the resulting interference as \emph{minimum interference}.
The minimum interference under the two models for the 
asymmetric case differs by at most one: if there is a point reachable 
from every other, we can increase its transmission range to 
include all other points. As a result, the communication 
graph becomes strongly connected, while the minimum 
interference increases by at most one.

Let  be the number of points.
In the symmetric case,
one can always achieve interference
, and this is sometimes necessary~\cite{HalldorssonTo08,
RickenbachWaZo09}. In the one-dimensional case, there is an
efficient approximation algorithm with
approximation factor ~\cite{RickenbachWaZo09}.
Furthermore,
Tan~et al.~\cite{TanLoWaHuLa11}
prove the existence of optimal solutions
with interesting structural properties in one dimension.
This can be used to obtain a nontrivial exact algorithm for
this case.
In the asymmetric case, the interference is significantly smaller:
one can always achieve interference , which is sometimes
optimal (e.g.,~\cite{Korman12}).

\paragraph{Our results.}
We consider interference minimization in asymmetric wireless
sensor networks in one and two dimensions. We show that for
two dimensions, it is NP-complete to find a valid
assignment that minimizes the maximum interference.
In one dimension we consider our second model requiring
one point that is reachable from
every other point through a directed path.
 Generalizing the
result by Tan~et al.~\cite{TanLoWaHuLa11}, we show that
there is an optimal solution that
exhibits a certain binary tree structure. By means of
dynamic programming, this structure can
be leveraged for a nontrivial exact algorithm. Unlike the
symmetric case, this algorithm always runs in quasi-polynomial
time , making it unlikely that the
one-dimensional problem is NP-hard.
Nonetheless, a polynomial time algorithm remains elusive.


\section{Preliminaries and Notation}

We now formalize our interference model for the
planar case.
Let  be a planar -point set.
A \emph{receiver assignment} 
is a function that assigns to each point in  the furthest
point that receives data from .
The resulting (asymmetric) \emph{communication graph}
 is the directed graph
with vertex set  and edge set
,
i.e., from each point  there are edges
to all points that are at least as close as the
assigned receiver .
The receiver assignment  is \emph{valid}
if  is strongly connected.

For  and , let  denote
the closed disk with center  and radius .
We define  as the
set that contains for each 
a disk with center  and  on the boundary.
The disks in  are called the \emph{transmission
ranges} for .
The \emph{interference}
of , , is the maximum number of transmission ranges
that cover a point in , i.e.,
.
In the \emph{interference minimization problem}, we are looking for
a valid receiver assignment with minimum interference.

\section{NP-completeness in Two Dimensions}

We show that the following problem is
NP-complete: given a planar point set
, does there exist a valid receiver assignment
 for  with ?
It follows that the minimum interference for
planar point sets is
NP-hard to approximate within a
factor of .

The problem is clearly in NP.
To show that interference minimization is NP-hard,
we reduce from the problem of deciding
whether a grid graph of maximum degree  contains
a Hamiltonian path:
a \emph{grid graph}  is a graph whose vertex
set  is a finite
subset of the integer grid.
Two vertices  are adjacent in 
if and only if , i.e., if  and 
are neighbors in the integer grid.
A \emph{Hamiltonian path} in  is a path
that visits every vertex in  exactly once.
Papadimitriou and Vazirani showed that
it is NP-complete to decide whether a grid graph 
of maximum degree  contains a Hamiltonian
cycle~\cite{PapadimitriouVa84}.
Note that we may assume that 
is connected;
otherwise there can be no Hamiltonian path.

Our reduction proceeds by replacing each vertex 
of the given grid graph  by a \emph{vertex gadget} ;
see Fig.~\ref{fig:gadget_bare}.
\begin{figure}[b]
  \centering
  \includegraphics[scale=0.8]{figs/asymmetric_gadget_rotated}
  \caption{The vertex gadget. }
  \label{fig:gadget_bare}
\end{figure}
The vertex gadget consists of 13 points, and it has five parts:
(a) the \emph{main point}  with
the same coordinates as ;
(b) three \emph{satellite stations}
with two points each: , ,
, , , .
The coordinates of the  are
chosen from 
so that there is a satellite station for each
edge in  that is incident to . If  has degree two, the third 
satellite station can be placed in any of the two remaining directions.
The  lie
at the corresponding clockwise positions from
,
for a sufficiently small ;
(c) the \emph{connector} , a point
that lies roughly at the remaining position from
 that is
not occupied by a satellite station, but
an -unit further away from . For example,
if  has no satellite station, then  lies
at ; and
(d) the \emph{inhibitor}, consisting of five
points .
The point  is the center of the inhibitor
and  is the point closest to .
The position of  is
, that is,
the distance between  and
 is an -unit larger than the distance  between 
and :
.
The points  are placed at
the positions ,
with  closest to .

\begin{figure*}
  \centering
  \includegraphics[scale=0.6]{figs/asymmetric_example_grid_graph}
  \caption{An example reduction.}
  \label{fig:gadget_example}
\end{figure*}

Given a grid graph , the reduction can be
carried out in polynomial time: just replace each
vertex  of  by the corresponding gadget ; see
Fig.~\ref{fig:gadget_example} for an example.
Let  be the resulting point set.
Two satellite stations in  that correspond to the same
edge of  are called \emph{partners}.
First, we investigate the interference in any valid
receiver assignment for .

\begin{lemma}
\label{lem:gadget_NN}
Let  be a valid receiver assignment for .
Then in each vertex gadget, the points  and 
have interference as least ,
and the points , and  have interference at
least .
\end{lemma}

\begin{proof}
For each point , the transmission
range  must contain
at least the nearest
neighbor of . Furthermore, in each
satellite station and in each inhibitor,
at least one point must have an assigned receiver outside
of the satellite station or inhibitor; otherwise,
the communication graph  would not be strongly connected.
This forces interference of  at  and at : each satellite
station and  must have
an edge to , and  all must have an
edge to . Similarly, for , the main
point  and the satellite  must have an edge to ;
see Fig.~\ref{fig:gadget_NN+}.
\qed{}
\end{proof}


\begin{figure}[htbp]
  \centering
  \includegraphics[scale=0.8]{figs/asymmetric_gadget_rotated_NN+}
  \caption{The nearest
    neighbors in a vertex gadget. }
  \label{fig:gadget_NN+}
\end{figure}

Let  be a valid receiver assignment,  and
let  be a vertex gadget in . An \emph{outgoing}
edge for  is an edge in  that originates
in  and ends in a different vertex gadget.
An \emph{incoming} edge for  is an edge that
originates in a different gadget and ends in .
A \emph{connecting} edge for  is either an outgoing
or an incoming edge for . If  holds, then
Lemma~\ref{lem:gadget_NN} implies that a connecting
edge can be incident only to satellite
stations. The proof of the following lemma is given in Appendix~\ref{sec:omitted}.

\begin{lemma}\label{lem:outedge}
Let  be a valid receiver assignment for 
with .
Let  be a vertex gadget of  and
 an outgoing edge from 
to another vertex gadget .
Then  goes from a satellite station of  to
its partner satellite station in .
Furthermore, in each satellite station of ,
at most one point is incident to outgoing edges.
\end{lemma}

\noindent
Next, we show that the edges between the vertex gadgets are
quite restricted.

\begin{lemma}
\label{lem:inout}
Let  be a valid receiver assignment for
 with .
For every vertex gadget  in ,
at most two satellite stations in 
are incident to connecting edges in .
\end{lemma}

\begin{proof}
By Lemma~\ref{lem:outedge} connecting edges are between satellite stations and
by Lemma~\ref{lem:gadget_NN},
the satellite points  in  have
interference at least .

First, assume that all three satellite
stations in  have outgoing edges.
This would increase the interference
at all three  to . Then,  could
not have any incoming edge from another vertex gadget,
because this would increase the interference for
at least one  (note that due to the placement of
the , every incoming edge causes interference at
an ). If  had no incoming edge,  would
not be strongly connected.
It follows that   has at most two satellite stations
with outgoing edges.

Next, assume that two satellite stations in
 have outgoing edges. Then, the third satellite station
of  cannot have an incoming edge,
as the two outgoing edges already increase the interference at the
third satellite station to .

Hence, we know that every vertex gadget  either
(i) has connecting edges with all three partner
gadgets, exactly one of which is outgoing,
or  (ii) is connected to
at most two other vertex gadgets. Take a vertex gadget  of type (i)
with partners .
Suppose that  has incoming edges from  and

and that the outgoing edge goes to . Follow the outgoing edge to
. If  is of type (i), follow the outgoing
edge from ; if  is of type (ii) and has an
outgoing edge to a vertex gadget we have not seen yet, follow this edge.
Continue this process until  is reached again or until
the next vertex gadget has been visited already. This gives
all vertex gadgets that are reachable from  on a
directed path.
However, in each step there is only one choice for the next
vertex gadget. Thus,
the process cannot discover  and , since both of them
would lead to  in the next step, causing the process to stop.
It follows that at least one of  or
 is not reachable from , although
 should be strongly connected. Therefore, all vertex gadgets in
 must be of type (ii), as claimed in the lemma.
\qed{}
\end{proof}


\noindent
We can now prove the main theorem of this section.

\begin{theorem}\label{thm:asym_NPC}
Given a point set , it is
\textup{NP}-complete to decide whether
there exists a valid receiver assignment  for
 with .
\end{theorem}

\begin{proof}
Using the receiver assignment  as certificate,
the problem is easily seen to be in NP.
To show NP-hardness, we use the polynomial time reduction
from the Hamiltonian path problem in grid graphs:
given a grid graph  of maximum degree
, we construct a planar point set  as above.
It remains to verify that  has a Hamiltonian path
if and only if  has a valid receiver assignment 
with .

Given a Hamilton path  in
, we construct
a valid receiver assignment  for
 as follows:
in each vertex gadget, we set ,
, and . For
 we set  and
.
Finally, we set .
This essentially creates the edges from
Fig.~\ref{fig:gadget_NN+}, plus the edge
from  to .
Next, we encode  into : for each  on
an edge of , we set   to the corresponding
 in the partner station. For the remaining
, we set .
Since  is Hamiltonian,
 is
strongly connected (note that
each vertex gadget induces a strongly connected
subgraph).
It can now be verified that  and  have
interference ;
, ,  have interference ; and
 has interference .
The point  has interference between  and
, depending on whether  and  are
on edges of .
The satellites  and 
have interference at most  and ,
respectively.

Now consider a valid receiver assignment  for
 with
.
Let  be the set of edges in  that correspond
to  pairs of vertex gadgets with a connecting edge
in .
Let  be the subgraph that  induces in .
By Lemma~\ref{lem:inout},  has maximum degree .
Furthermore, since  is strongly connected,
the graph  is connected and meets all vertices of .
Thus,  is a Hamiltonian path (or cycle) for ,
as desired.
\qed{}
\end{proof}

\textbf{Remark}.  A similar result to Theorem~\ref{thm:asym_NPC}
also holds for symmetric communication graphs networks~\cite{Buchin08}.

\section{The One-Dimensional Case}

For the one-dimensional case we
minimize receiver interference under the second model discussed in the introduction:
given  and
a receiver assignment
, the graph
 now has a directed edge from each point 
to its assigned receiver , and no other edges.
 is \emph{valid} if  is acyclic and if
there is a sink  that is reachable from every
point in . The sink has no outgoing edge. The interference
of , ,  is defined as before.

\subsection{Properties of Optimal Solutions}

We now explore the structure of optimal
receiver assignments. Let  and  be a valid receiver
assignment for  with sink . We can interpret
 as a directed tree, so we
call  the \emph{root} of .
For a directed edge  in , we say that  is a \emph{child} of 
and  is the \emph{parent} of .
We write  if there is a
directed path from  to  in .
If ,
then  is an \emph{ancestor} of  and
 a
\emph{descendant} of . Note that  is both an
ancestor and a descendant of . Two points  are \emph{unrelated}
if  is neither an ancestor nor a descendant of .
For two points , , we define 
as the open interval bounded by  and , and
 as the closure of
.
An edge  of  is a \emph{cross edge}
if the interval  contains at least one point that is
not a descendant of .

Our main structural result is that there is always an optimal
receiver assignment for  without cross edges.
A similar property was observed by Tan~et al.~for the
symmetric case~\cite{TanLoWaHuLa11}.

\begin{lemma}\label{lem:nocross}
Let  be a valid receiver assignment for  with
minimum interference.
There is a valid assignment  for  with

such that  has no cross edges.
\end{lemma}

\begin{proof}
Pick a valid assignment
 with minimum interference that minimizes the total
length of the cross edges

where  are the cross-edges of .
If , we are done.
Thus, suppose . Pick a cross edge  such that
the hop-distance (i.e., the number of edges) from  to the root
is maximum among all cross edges.
Let  be the leftmost and  the rightmost descendant of . We refer 
to Appendix~\ref{sec:omitted} for a proof of the following lemma.

\begin{prop}\label{prop:descendants}
The interval  contains only descendants of .
\end{prop}

Let  be the points in  that
are not descendants of . Each point in 
is either unrelated to , or it is an ancestor of .
Let  be the point in  that is closest
to  (i.e.,   either lies directly to the left of  or directly
to the right of .
We now describe how to construct a new valid assignment ,
from which we will eventually derive a contradiction to the choice of
.
The construction is as follows:
replace the edge  by . Furthermore, if
(i) ;
(ii) the last edge   on the path
from  to  crosses the interval ; and (iii)
 is not a cross-edge, we also
change the edge  to the edge that connects  to the
closer of  or . We give the proof of the following proposition in Appendix~\ref{sec:omitted}.

\begin{prop}\label{clm:validass}
 is a valid assignment.
\end{prop}


\begin{prop}\label{clm:optass}
We have .
\end{prop}

\begin{proof}
Since the new edges are shorter than the edges they replace, each
transmission range for  is contained in the corresponding
transmission range for . The interference cannot decrease since  is
optimal.
\qed{}
\end{proof}

\begin{prop}\label{clm:lesscross}
We have .
\end{prop}

\begin{proof}
First, we claim that  contains
no new cross edges, except possibly :
suppose  is a cross edge
of , but not of .
This means that  contains a point 
with ,
but .
Then 
must be a descendant of  in  and in
, because
as we saw in the proof of Claim~\ref{clm:validass},
for any , we have that if ,
then .

Hence,  and  intersect.
Since  is a cross edge, the choice of  now implies that
.
Thus,  lies in , because  is a direct neighbor of  or .
We claim that . Indeed, otherwise we would have 
(since  is not a cross edge in ), and thus also
. However, we already observed
, so we would have
 (recall that we introduce the edge 
in ).
This contradicts our choice of .

Now it follows that  is the last
edge on the path from  to , because if  were not an ancestor
of , then  would already be a cross-edge in .
Hence, (i)  is an ancestor of ; (ii)  crosses the interval
; and (iii)  is not a cross edge in .
These are the conditions for the edge  that
we remove from .
The new edge  from  to  or  cannot be a cross edge,
because  is not a cross edge in  and  does
not cover any descendants of .

Hence,   contain no new cross-edges,
except possibly  which replaces
the cross edge .
By construction, , so
.
\qed{}
\end{proof}

\noindent
Propositions~\ref{clm:validass}--\ref{clm:lesscross}
yield a contradiction to the choice of .
It follows that we must have , as desired.
\qed{}
\end{proof}

Let .
We say that a valid assignment  for  has the \emph{BST-property} if the
the following holds for any vertex  of :
(i)  has at most one child  with  and at most
one child  with ; and (ii) let  be the
leftmost and  the rightmost descendant of .
Then  contains only descendants of .
In other words:  constitutes a binary
search tree for the (coordinates of the) points in .
A valid assignment without cross edges has
the BST-property. The following is therefore an immediate consequence of Lemma~\ref{lem:nocross}.

\begin{theorem}\label{thm:BST}
Every  has an optimal valid assignment
with the \emph{BST-property}.\qed
\end{theorem}

\subsection{A Quasi-Polynomial Algorithm}\label{sec:qalgo}

We now show how to use Theorem~\ref{thm:BST} for
a quasi-polynomial time algorithm to minimize the
interference. The algorithm uses dynamic
programming. A subproblem  for the dynamic program consists of four
parts: (i) an interval  of \emph{consecutive} points
in ;
(ii) a root ; (iii) a set
 of \emph{incoming interference}; and (iv) a set 
of \emph{outgoing interference}.

The objective of  is to find an optimal valid assignment  for
 subject to
(i) the root of  is ;
(ii) the set  contains all transmission ranges of
 that cover points in  plus potentially a
transmission range with center ;
(iii) the set  contains
transmission ranges that cover points in  and have their center in .
The interference of  is defined as the maximum number of transmission ranges in
 that cover any given point of .
The transmission ranges in  are given as pairs , where
 is the center and  a point on the boundary of the range.

Each range in  covers a boundary point of
.
Since it is known that there is always an assignment with interference
 (see \cite{RickenbachWaZo09} and Observation~\ref{obs:nna_upper}),
no point of  lies in more than 
ranges of . Thus, we can assume that
, and the total number
of subproblems is .

A subproblem  can be solved recursively as follows.
Let  be the points in  to the left of ,
and  the points in  to the right of .
We enumerate all pairs  of subproblems with
 and ,
and we connect the roots  and  to .
Then we check whether , , , ,
, and  are \emph{consistent}.
This means that   contains all ranges from  with center in
 plus the range for the edge 
(if it does not lie in  yet).
Furthermore,
 may contain additional ranges with center in  that cover
points in  but not in .
The set  must contain all ranges in  and 
that cover points in , as
well as the range from  with center , if it exists and if it
covers a point in .
The conditions for  are analogous.

Let  be the valid assignment for  obtained
by taking optimal valid assignments  and 
for  and  and
by adding edges from  and  to .
The interference of  is then defined with respect to
the ranges in  plus the range with
center  in  (the other ranges of  must lie in
. We take the
pair  of subproblems which minimizes this interference.
This step takes  time, because the number
of subproblem pairs is  and the overhead per pair is
polynomial in .

The recursion ends if  contains a single point . If 
contains only one range, namely the edge from  to its parent, the
interference of  is given by .
If  is empty or contains more than one range, then
the interference for  is .

To find the overall optimum, we start the recursion with
,  and every possible root, taking the
minimum of all results. By implementing the recursion with dynamic programming,
we obtain the following result.

\begin{theorem}\label{thm:qp_algo}
Let  with . The optimum interference of  can be
found in time .\qed
\end{theorem}

Theorem~\ref{thm:qp_algo} can be improved slightly. The number of subproblems
depends on the maximum number of transmission ranges that cover the boundary points
of  in an optimum assignment. This number is bounded by the optimum
interference of .  Using exponential search, we get the following
theorem.
\begin{theorem}
Let  with . The optimum
interference  for  can be found in time
.\qed
\end{theorem}

\section{Further Structural Properties in One Dimension}\label{sec:further}

In this section, we explore further structural properties of
optimal valid receiver assignments for one-dimensional point sets.
It is well known that for any -point set , there always exists a valid assignment  with
. Furthermore, there exist point sets such that
any valid assignment  for them must have ~\cite{RickenbachWaZo09}.
For completeness, we include proofs for these facts in Section~\ref{sec:nna}. 
Below we show that there may be an arbitrary number
of left-right turns in an optimal solution. To the best of our knowledge, this result is new, and
it shows that in a certain sense, Theorem~\ref{thm:BST} cannot be improved.


In Theorem~\ref{thm:BST} we proved that there always exists an
optimal solution with the BST-property. Now, we will
show that the structure of an optimal solution cannot
be much simpler than that. Let  be finite
and let  be a valid receiver assignment for . A \emph{bend} in 
is an edge between two non-adjacent points. We will show that
for any  there is a point set  such that any optimal
assignment for  has at least  bends.

For this, we inductively define sets , ,  as follows.
For each , let  denote the diameter of .
 is just the origin (and ). Given , we let 
consist of two copies of , where the second copy is translated
by  to the right, see Fig.~\ref{fig:lowerb}. By induction,
it follows that
 and .

\begin{figure}
\centering
\includegraphics[scale=0.8]{figs/lowerb}
\caption{Inductive construction of .}
\label{fig:lowerb}
\end{figure}

\begin{prop}\label{prop:lowerb}
Every valid assignment for  has interference at least
.
\end{prop}

We give the proof of Proposition~\ref{prop:lowerb} in Appendix~\ref{sec:omitted}.

\begin{lemma}\label{lem:opt_ass}
For , there exists a valid assignment  for
 that achieves interference . Furthermore,  can
be chosen with the following properties: (i)  has the \textup{BST}-property;
(ii) the leftmost or the rightmost point of  is the root
of ; (iii) the interference at the root is
, the interference at the other extreme point of  is
.
\end{lemma}

\begin{proof}
We construct  inductively. The point set 
has two points at distance , so any valid
assignment has the claimed properties.

Given , we construct : recall that 
consists of two copies of  at distance
. Let  be the left and  the right copy.
To get an assignment  with the leftmost point
as root, we use the assignment  with the left point as root
for  and for ,
and we connect the root of  to the rightmost
point of . This yields a valid assignment.
Since the distance between  and 
is , the interference for all points in
 increases by .  The interferences for
 do not change, except for the rightmost point,
whose interference increases by . Since ,
the desired properties follow by induction.
The assignment with the rightmost point as root is constructed
symmetrically.
\qed{}
\end{proof}

The point set  is constructed recursively.
 consists of a single point  and a copy 
of  translated to the right by  units.
Let  be the diameter of .
To construct  from , we add a copy  of
, at distance  from . If  is odd,
we add  to the left, and if  is even,
we add  to the right; see Fig.~\ref{fig:bends}.
\begin{figure}
\centering
\includegraphics[scale=0.7]{figs/bends}
\caption{The structure of . The arrows indicate the
bends of an optimal assignment.}
\label{fig:bends}
\end{figure}

\begin{theorem}
We have the following properties: (i) the diameter
 is ;
(ii) the optimum interference of  is ; and (iii) every optimal
assignment for  has at least  bends.
\end{theorem}

\begin{proof}
By construction, we have  and  ,
for .  Solving the recursion yields the claimed bound.

In order to prove (ii), we first exhibit an assignment
 for  that achieves interference
. We construct  as follows: first, for
, we take for 
the assignment  from
Lemma~\ref{lem:opt_ass} whose root is
the closest point of  to . Then, we
connect  to the closest point in , and
for ,
we connect the root of  to the root of
. Using the properties
from Lemma~\ref{lem:opt_ass}, we can check
that this
assignment has interference .

Next, we show that all valid assignments for
 have interference at least .
Let  be an assignment for .
Let  be the leftmost point of
, and let  be the last point on the
path from  to the root of  that lies
in .
We change the assignment  such that all
edges leaving  now go to
. This yields a valid assignment 
for  with root . Thus, ,
by Proposition~\ref{prop:lowerb}. Hence,
by construction, , since
.

For (iii), let  be an optimal assignment for .
We prove by induction that the root of
 lies in , and that  has  bends,
all of which originate outside of .
As argued above, we have .
As before, let  be the leftmost point of 
and  the last point on the path from  to
the root of . Suppose that  is not
the root of . Then  has an outgoing edge that
increases the interference of all points in 
by . Furthermore, by constructing a valid assignment
 for  as in the previous paragraph,
we see that the interference in  of all edges that originate
from 
is at least . If follows that ,
although  is optimal.

Thus, the root  of  lies in . Let  be a
point outside  with .
The outgoing edge from  increases
the interference of all points in 
by . Furthermore, we can construct a valid assignment
 for  by redirecting all
edges leaving  to . By construction, ,
so by (ii),  is optimal for 
with interference . By induction,  has its root
in  and has at least  bends, all of which originate
outside . Thus,  must lie in .
Since  was arbitrary, it follows that all bends of 
are also bends of . The edge from  in  is
also a bend, so the claim follows.
\qed{}
\end{proof}

\section{Conclusion}

We have shown that interference minimization in two-dimensional planar sensor
networks is NP-complete. In one dimension, there exists an algorithm
that runs in quasi-polynomial time, based on the fact that there are always
optimal solutions with the BST-property. Since it is generally believed that
NP-complete problems do not have quasi-polynomial algorithms, our result indicates
that one-dimensional interference minimization is probably not NP-complete. However,
no polynomial-time algorithm for the problem is known so far. Furthermore,
our structural result in Section~\ref{sec:further} indicates that optimal solutions
can exhibit quite complicated behavior, so further ideas will be necessary for a
better algorithm.


\section*{Acknowledgments}
We would like to thank Maike Buchin, Tobias Christ, Martin Jaggi, Matias Korman, Marek Sulovsk\'y,
and Kevin Verbeek for fruitful discussions.

\bibliographystyle{abbrv}
\bibliography{interfere}

\newpage
\appendix
\section{Omitted Proofs}\label{sec:omitted}

\begin{proof}[of Lemma~\ref{lem:outedge}]
By Lemma~\ref{lem:gadget_NN}, both
 and  in  have interference
at least .
This implies that neither , nor , nor
any point in the inhibitor of  can be
incident to an outgoing edge of :
such an edge would increase the interference at
 or at .
In particular, note that the
distance between the inhibitors in
two distinct vertex gadgets is at least
,
the distance between 
and its corresponding inhibitor; see the
dotted line in Fig~\ref{fig:gadget_example}.

Thus, all outgoing edges for  must originate in
a satellite station.
If there were a satellite station in  where
both points are incident to outgoing edges,
the interference at  would increase.
Furthermore, if there were a satellite station in  with
an outgoing edge that does not go the partner station,
this would increase the interference at the main point
of the partner vertex gadget, or at the inhibitor center  of .
\qed{}
\end{proof}

\begin{proof}[of Proposition~\ref{prop:descendants}]
Since  and  each have a
path to , the  interval  is
covered by edges that begin in proper descendants of .
Thus, if  contains a point  that is not a descendant
of , then  would be covered by an edge  with
 a proper descendant of .
Thus,  would be a cross edge with larger hop-distance to the
root, despite the choice of .
\qed{}
\end{proof}


\begin{proof}[of Proposition~\ref{clm:validass}]
We must show that all points in  can reach the root.
At most two edges change:  and (potentially)
.
First, consider the change of  to .
This affects only the descendants of .
Since  is not a descendant of ,
the path from  to the root does not use the edge , and hence
all descendants of  can still reach the root.
Second, consider the change of  to an edge from  to
 or . Both  and  have  as ancestor (since we
introduced the edge ), so all descendants of 
can still reach the root.
\qed{}
\end{proof}

\begin{proof}[of Proposition~\ref{prop:lowerb}]
The proof is by induction on . For  and , the claim
is clear.


Now consider a valid assignment  for  with sink . Let  and
 be the two  subsets of , and suppose
without loss of generality that .
Let  be the edges that cross from  to .
Fix a point , and let  be the last vertex on the
path from  to  that lies in . We replace every
edge  with 
by the edge . By the definition of , this does not increase the interference.
We thus obtain a valid assignment
 with sink  such that
, since the ball for the edge
between  and  covers all of . By induction,
we have , so , as claimed.
\qed{}
\end{proof}

\section{Nearest Neighbor Algorithm and Lower Bound}\label{sec:nna}

First, we prove  that we can always obtain
interference , a fact used in Section~\ref{sec:qalgo}.
This is achieved by the \emph{Nearest-Neighbor-Algorithm}
(NNA)~\cite{RickenbachWaZo09,Korman12}.
It works as follows.

At each step, we maintain a partition
 of , such that
the convex hulls of the  are disjoint. Each set 
has a designated sink  and an assignment
 such that the graph
 is acyclic and has  as the only sink.
Initially,  consists of  singletons, one
for each point in . Each point in  is the sink of its
set, and the assignments are trivial.

Now we describe how to go from a partition
 to a new partition .
For each sink , we define the successor  as the closest
point to  in . We will ensure that this closest
point is unique in every round after the first.
In the first round, we break
ties arbitrarily
Consider the directed graph  that has vertex set  and
contains all edges from the component graphs  together
with edges , for . Let  be the
components of . Each such component  contains exactly one cycle,
and each such cycle contains exactly two sinks  and .
Pick  such that the distances between
 and the closest points in the neighboring components
 and  are distinct (if they exist). At least
one of  and  has this property, because  and
 are distinct. Suppose that  (the other case
is analogous). We make  the new sink of , and we
let  be the union of  and
the assignments  for all
components . Clearly,  is a valid assignment
for . We set .
This process continues until a single component remains.

\begin{obs}\label{obs:nna_upper}
The nearest neighbor algorithm ensures interference at most
.
\end{obs}

\begin{proof}
Since each component in  is combined with at least one
other component of , we have ,
so there are at most  rounds.

Now fix a point . We claim that in the interference
of  increases by at most  in each round, except for
possibly two rounds in which the interference increases by .
Indeed, in the first round, the interference increases by at most ,
since each point connects to its nearest neighbor (the increase by  can
happen if there is a point with two nearest neighbors).
In the following rounds, if  lies in the interior of a connected component ,
its interference increases by at most  (through the edge from
 to ). If  lies on the boundary of
, its interference may increase by  (through the edge between
 and  and the edge that connects a neighboring component
to ). In this case, however,  does not appear on the boundary of
any future components, so the increase by  can happen at most once.
\qed{}
\end{proof}


Next, we show that interference
 is sometimes necessary.
We make use of the points sets  constructed in Section~\ref{sec:further}.


\begin{corol}
For every , there exists a point set  with  points such that
every valid assignment for  has interference .
\end{corol}

\begin{proof}
Take the point set  from Section~\ref{sec:further} and add
 points sufficiently far away.
The bound on the interference follows from Proposition~\ref{prop:lowerb}.
\qed{}
\end{proof}


\end{document}
