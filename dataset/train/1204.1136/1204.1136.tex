\documentclass[11pt,a4paper]{article}

\usepackage{fullpage}
\usepackage{amsthm}
\usepackage{amsmath}
\usepackage{amssymb}
\usepackage{amsthm}
\usepackage{xspace}
\usepackage[boxed]{algorithm2e}
\usepackage{mathrsfs}
\usepackage[text={16cm,23.7cm}]{geometry}

\newtheorem{definition}{Definition}
\newtheorem{proposition}{Proposition}
\newtheorem{theorem}{Theorem}
\newtheorem{corollary}{Corollary}
\newtheorem{lemma}{Lemma}
\newtheorem{property}{Property}

\newcommand{\Broder}{Broder \etal}
\newcommand{\etal}{\textsl{et al.}\xspace}
\renewcommand{\O}{\ensuremath \widetilde O}
\newcommand{\Tt}{\ensuremath \widetilde \Theta}
\newcommand{\To}{\ensuremath \widetilde \Omega}
\newcommand{\s}{\ensuremath  \mathscr S}
\renewcommand{\t}{\ensuremath \mathscr  T}
\newcommand{\E}{\ensuremath \mathbb E}
\renewcommand{\Pr}{\ensuremath \mathrm {Pr}}
\renewcommand{\*}{\hspace*{5mm}}

\author{Adrian Kosowski\footnote{Inria Bordeaux Sud-Ouest, 33400 Talence, France. E-mail: adrian.kosowski@inria.fr}}

\title{Faster Walks in Graphs:\\ A  Time-Space Trade-off for Undirected - Connectivity}

\date{}

\begin{document}
\maketitle\thispagestyle{empty}
\setcounter{page}{0}
\vspace{8mm}

\begin{abstract}
\noindent
In this paper, we make use of the Metropolis-type walks due to Nonaka \etal\ (2010) to provide a faster solution to the --connectivity problem in undirected graphs (USTCON).

As our main result, we propose a family of randomized algorithms for USTCON which achieves a time-space product of  in graphs with  nodes and  edges (where the -notation disregards poly-logarithmic terms). This improves the previously best trade-off of , due to Feige (1995). Our algorithm consists in deploying several short Metropolis-type walks, starting from landmark nodes distributed using the scheme of Broder \etal (1994) on a modified input graph. In particular, we obtain an algorithm running in time  which is, in general, more space-efficient than both BFS and DFS. 

We close the paper by showing how to fine-tune the Metropolis-type walk so as to match the performance parameters (e.g., average hitting time) of the unbiased random walk for any graph, while preserving a worst-case bound of  on cover time.
\end{abstract}
\vspace{8mm}
\textbf{Keywords:} undirected - connectivity, time-space trade-off, graph exploration, Metropolis-Hastings walk, parallel random walks.
\newpage

\section{Introduction}

In the \emph{undirected - connectivity problem} (USTCON), the input to the algorithm is an undirected graph  with  vertices and  edges. Two of the vertices of the graph, , are distinguished. The goal is to determine whether  and  belong to the same connected component of . USTCON has a spectrum of applications in various areas of computer science, ranging from tasks of network discovery to computer-aided verification. The problem has also made its mark on complexity theory, most famously, playing a central part in the rise and eventual collapse of the complexity class SL.

The time complexity of algorithms for USTCON depends on the amount of space available to the algorithm. Given  space, USTCON can be solved deterministically in time  by fast algorithms such as BFS or DFS. Given  space, the problem can still be solved deterministically~\cite{Rei} in polynomial time. However, in this case the fastest known solutions are randomized. Aleliunas \etal~\cite{AKLLR} proposed a log-space algorithm with bounded error probability, which consists in running a random walk, starting from node  for  steps, and testing if node  has been reached. 

The study of the interplay between the space complexity  and the time complexity  of randomized algorithms for USTCON was initiated by \Broder~\cite{BKRU}. They observed that both BFS/DFS, and the random walk, admit the same time-space trade-off of , and investigated whether there exist algorithms with such a trade-off for an arbitrary choice of , , where  is some model-dependent constant. After a sequence of papers relying on the deployment of many short random walks, this question was eventually settled in the affirmative by Feige~\cite{F}, who proposed a family of algorithms which achieve such a time-space trade-off in the whole of the considered range of space bounds. 

The main result of this paper is an improved time-space trade-off for USTCON. Since the cover time of the random walk is precisely  for some graphs, any improvement with respect to Aleliunas \etal~\cite{AKLLR} or Feige~\cite{F} requires a refinement of the performed walk on graphs. Instead of the random walk, we make use of the Metropolis-Hastings walk on graphs, with weighting proposed by Nonaka~\etal~\cite{NOSY}. This walk covers any undirected graph in  steps, but its transition probabilities rely on knowledge of the degrees of neighboring nodes at every step. We start the technical sections of this paper with an explicit implementation of the walk from~\cite{NOSY} using the Metropolis sampling algorithm from~\cite{M51}. This yields a solution to USTCON in  time and logarithmic space. Our contribution lies in completing this quadratic time-space trade-off for larger bounds on the space complexity of the algorithm. The main technical difficulty concerns overcoming problems with short runs of the Metropolis-Hastings walk, which sometimes exhibits inferior behavior to the random walk in terms of the speed of discovering new nodes. 

For the entire range of space bounds (), we propose algorithms running in time . In other words, we obtain  for , and  for . (Note that  is a lower bound on execution time for any algorithm for USTCON, regardless of the space bound.) In particular, we prove that USTCON can be solved in time  using space , which is, in general, less than the space requirement of BFS/DFS. 

All of the considered algorithms for USTCON are randomized (in the Monte Carlo sense), with bounded probability of one-sided error. This means that the positive answer ``connected'' may only be reached by the algorithm when  and  belong the same connected component of , whereas the negative answer ``not connected'' signifies that,  with probability at least ,  and  belong to different components of .

\subsection{Related work}

Much of the work on undirected - connectivity has focused around its role as the fundamental complete problem for the symmetric log-space complexity class (SL). A survey of other important problems identified as belonging to the class SL, such as simulating symmetric Turing machines, and testing if a graph is bipartite, is provided in~{AGCR}. A major line of study concerned determining the minimum space required to solve USTCON deterministically. The bound on the required space was reduced, over several decades, from the  bound given by Savitch's theorem~\cite{Sav}, through  \cite{SZ}, and ~\cite{ATWZ}. Finally, in 2004, Reingold's~\cite{Rei} new construction of universal graph exploration sequences provided the first log-space algorithm for USTCON, showing that SL=L. Befor Reingold's paper, Nisan~\cite{Nis} had shown a deterministic algorithm for USTCON running in polynomial time and  space. Borodin \etal~\cite{BCDRT} proposed a log-space Las-Vegas type algorithm for USTCON (with no-error) which runs in expected polynomial time.

When considering randomized algorithms with bounded one-sided error, the unbiased random walk was shown to solve USTCON in  space and  time by Aleliunas \etal~\cite{AKLLR}. Several years later, \Broder~\cite{BKRU} proposed a family of algorithms based on short random walks starting from landmark nodes. Relying on landmarks chosen on the set of nodes according to the stationary distribution of the walk, they achieved a time-space trade-off of . Subsequent algorithms from the literature~\cite{BF,F} make use of different landmark distribution schemes. Barnes and Feige~\cite{BF} achieve a trade-off of  by using a mixed landmark distribution scheme, which places half of the landmarks according to the stationary distribution of the random walk, and half according to the uniform distribution on nodes. Feige~\cite{F} introduces the inverse distribution scheme, which likewise places half of the landmarks according to the stationary distribution of the random walk, and the other half according to the inverse of node degrees. He achieves a time-space trade-off of  in general, where  is the minimum degree of the graph. Thus, the trade-off of  is reached for the case of (nearly) regular graphs.

Undirected - connectivity is a special case of the more general reachability problem in directed graphs (STCON), which is a complete problem for the class NL. STCON can also be solved deterministically in  space using Savitch's theorem~\cite{Sav}. So far, it has resisted fast solutions in small space. This problem was extensively studied in different variants of a model of computation based on Jumping Automata on Graphs (JAG-s). The memory of a JAG is organized in the form of  pebbles placed in the graph and  states of the automaton, with space defined as . Cook and Rackoff~\cite{CR} show a way of solving STCON in the JAG model deterministically in  space, and also prove an almost matching lower bound on space of . This lower bound is also known to apply to randomized JAG-s running in slightly super-polynomial time~\cite{BS}. Gopalan \etal~\cite{GLM} propose a family of algorithms for STCON based on short random walks, whose runtime increases from  to  as space decreases from  to .

Finally, we remark on recent developments in the area of graph exploration with biased random walks. Ikeda \etal~\cite{IKY} and Nonaka \etal~\cite{NOSY} studied possible adjustments to the transition matrix of the walk based on the availability of local topological information (otherwise known as ``look-ahead''). In general, the idea of these approaches is to increase the probability of transition to a node of lower degree. The former paper introduces a new type of walk, called the -walk, whose transition matrices are biased so that transition from a node to its neighbor of degree  is proportional to . Such a walk was shown to visit all nodes of the graph in  steps in expectation for an optimal choice of parameter . Nonaka \etal~\cite{NOSY} later used the key lemmas from this work to prove an analogous result for a walk with a modified transition matrix, which fits into the class of Metropolis-Hastings walks. This walk is the starting point for considerations in our paper. A somewhat different approach was adopted by Berenbrink \etal~\cite{BCERS}, who show that a random walk with the additional capability of marking one unvisited node in its neighborhood as visited can be used to speed up exploration.

\subsection{Overview of the paper}

The organization of the technical parts of the paper is the following. In Section~\ref{sec2}, we recall the definition of the Metropolis-Hastings walk and provide its efficient implementation using the Metropolis algorithm. In this way, given a representation of graph , each step of the walk can be simulated by a procedure running in  time and using  bits of space.

We subsequently identify the key properties of the unit-potential Metropolis-Hastings walk, denoted , which allow it to be used as a replacement for the (unbiased) random walk on , denoted , in algorithms solving USTCON. The major difference between these types of walks is that a short random walk  has the desirable property of \emph{low edge-return rate}, i.e., each edge of the graph is visited  times in expectation during  steps of the walk (for sufficiently small ). However, no analogous property hold for the Metropolis-Hastings walk. In fact, on some graphs (e.g.,\ the \emph{glitter star} defined in~\cite{NOSY}), the Metropolis-Hastings walk , will in expectation discover only  edges during  steps of the walk, visiting each of these edges  times (for any choice of ). We overcome this problem in two stages:
\begin{itemize}

\item In Section~\ref{sec22} we prove that in a graph of maximum degree , the Metropolis-Hastings walk  begins to achieve a \emph{low node-return rate} starting from a threshold length of  steps: a Metropolis-Hastings walk of length , , visits each node of the graph  times in expectation. This property is formally stated as Lemma~\ref{lemiii}. 

\item In Section~\ref{sec3} we show how to obtain the trade-off   for an arbitrary choice of space bound . Our initial approach makes use of a modification of a technique introduced by \Broder~\cite{BKRU}. It consists in running  walks of length  each, which originate from an appropriately chosen subset of  nodes of the graph called \emph{landmarks}. In our formulation, the walks used are Metropolis-Hastings walks (rather than random walks on ), and the set of landmarks is sampled uniformly on . By observing the visits of each of these walks to other landmarks from the set, it is possible to obtain information about paths connecting different landmarks. When the performed Metropolis-Hastings walks have a low node-return rate (i.e., when ), the obtained information turns out to be w.h.p.\ sufficient to find an answer to - connectivity with a low probability of error. Otherwise, when , we modify the approach, performing a logical transformation of graph . We split each node of degree greater than , so that the maximum degree of the modified graph does not exceed . Then, all of the considerations are performed for this modified graph. In particular, the set of landmark nodes is chosen by uniform sampling on the set of nodes of this modified graph. The overhead associated with this transformation is just small enough for our algorithm to have the claimed time complexity of . An implementation of the complete algorithm is provided in Appendix~A.
\end{itemize}

Finally, in the closing Section~\ref{sec5} we discuss the tightness of the obtained results. We also propose a modified weighting of the Metropolis-Hastings walk which performs faster than uniform-weighted Metropolis-Hastings for many classes of graphs, while still covering all the nodes of the graph in  time. This walk satisfies the property that its commute time between any pair of nodes (and consequently also the average hitting time) is asymptotically upper-bounded by the values of the respective parameters for the unbiased random walk. In particular, it covers all the nodes of the previously mentioned glitter star, in expected  steps.

\subsection{Notation and model}

The input graph , with  and , is simple and not necessarily connected. In order to simplify notation for complexity bounds, we assume . The degree of a node  is denoted by , the neighborhood of node  by , and  the closed neighborhood of  by . The maximum degree of the graph is denoted by . The arc set  of undirected graph  is understood as the set of arcs of all edges and self-loops of : . An arc  is sometimes denoted as  for compactness of notation. Note that the symbols , , , ,  always refer to the input graph . When considering a different graph , we will sometimes denote its vertex, edge, and arc sets by , , and , respectively. 

Our algorithms are designed for the classical RAM model of computation. No special assumptions are made on the representation of graph , except that for any node , there should exist a local ordering on the set of its neighbors, given by the bijective function . Each of the following operations should be possible to implement in  time: computing , computing  for a node , and ``traversing an edge'' by computing , for port . An example of a permissible representation is a lexicographically sorted array of ordered pairs of identifiers of neighboring nodes , taken over .

For most of the paper, we consider weighted reversible Markovian processes corresponding to a random walk  on some weighted undirected graph  with positive weights on arcs. The walk is located on the nodes of graph , and the next state of the walk is reached by following an arc incident to the current node, chosen with probability proportional to the weight of this arc. By a slight abuse of notation, we denote the transition matrix of the walk in the same way as the weighted graph. Most other notation follows that of Aldous and Fill~\cite{AF}. In particular, we consider the following random variables:
\begin{itemize}
\item  denotes the number of steps in the time interval  during which the walk visits , where the symbol  may represent a node, edge, or arc of the graph.
\item  denotes the first moment of time  at which the walk first visits (or returns to) a node from , where the symbol  may represent a subset of nodes or a single node of the graph.
\end{itemize}
By writing  and , respectively, we mean the expectation of random variable , and the probability of event  occurring, taken over walks starting from probability distribution  (which may be concentrated on a single node or arc). A walk starting from an arc is understood as one which starts from the head of the arc at time , and then moves to the tail of the arc at time .

Given a weighted graph , we denote by  the commute time between nodes . Throughout the paper, we consider only walks representing reversible Markovian processes, corresponding to symmetric weightings of the graph: , for all . In some of the proofs, we rely on the resistor network representation of reversible walks: for each edge  having weight  on each of its arc, a resistor with resistance  is placed between nodes  and  of the resistor network. The symbol  denotes the resistance of replacement between nodes  and  of the network. We recall that .~\cite{CRRST}

\section{Preliminaries: The Metropolis-Hastings Walk on Graphs}\label{sec2}\label{sec22}


The \emph{Metropolis-Hastings walk} with potential function  is defined as a walk on the weighted graph  with the following assignment of weights :


We recall that for a walk in state , the next state is chosen as  with probability proportional to the weight . By a classical result due to Metropolis \etal~\cite{M51}, for a given representation of graph , a single step of the Metropolis-Hastings walk  can be simulated in  time and space by means of the procedure shown in Algorithm~\ref{algo1}. The algorithm takes advantage of the fact that , for all . For a walk located at node , it samples a node  with uniform probability , and accepts  as the new state with the appropriate probability. We remark that a step of  can  also be simulated by a log-space automaton which pushes a pebble along the arc . The pebble remains at  if state  is accepted, and otherwise reverts to  by traversing the arc . Thus, one step of  can be simulated by at most two moves of a pebble.

\IncMargin{0.5em}
\begin{algorithm}
\textbf{function} next\_state (: node) \{\\
\*  neighbor of  in  chosen uniformly at random; \quad// \emph{pick a new state}\\
\* \textbf{with probability}  \textbf{do} \textbf{return} ; \ // \emph{accept: move to new state}\\
\* \textbf{return} ; \quad// \emph{do not accept: keep current state}\\
\}
\caption{State transition function on  for the walk .
}
\label{algo1}
\end{algorithm}

\begin{definition}
We denote by  the weighted graph  for the unit potential function .
\end{definition}


From the next two sections, we focus on the Metropolis-Hastings walk . We note that the weights on the edges of  are now simply given by .

The bound on the time required by the Metropolis-Hastings walk to discover w.h.p.\ the entire connected component containing the starting node of the walk follows from the considerations of Nonaka \etal~\cite{NOSY}. (All omitted proofs are provided in the Appendix.)

\begin{lemma}[\cite{NOSY}]\label{lemtwo}
Let , let  be the connected component of  containing node , and let . Then:
\begin{itemize}
\item a walk  of length  starting from  covers an arbitrary node  with probability at least .
\item a walk  of length  starting from  covers all nodes from  with probability at least .
\end{itemize}
\end{lemma}

By the above Lemma, a solution to USTCON, with probability , is obtained by running the walk , starting from , for  steps. USTCON can therefore be solved in log-space by running Algorithm~\ref{algo1} in a loop for  iterations. (We are unaware of any previous reference in the literature for this observation.)

\begin{corollary}
There is a log-space algorithm for USTCON which runs in time , with probability of one-sided error bounded by .
\qed
\end{corollary}

For our purposes, we will need a more detailed analysis of the behavior of the Metropolis-Hastings walk. We start by recalling that the Metropolis-Hastings walk  is a reversible Markovian process, since  for all arcs. Its stationary distribution is the uniform distribution , with

for all . This allows us to show the following key lemma which captures the ``low node-return rate'' property of the Metropolis-Hastings walk, as highlighted in the Introduction. The first claim states that a Metropolis-Hastings walk starting within any subset of nodes  is likely to leave it within  steps, while its second claim shows that a Metropolis-Hastings walk of length  is likely to return to its starting node not more than  times. However, both of the above statements hold only when considering walks of duration .
\begin{lemma}\label{lemiii}
Suppose that  is connected. Let , and let . For a weighted random walk  starting from node :
\begin{enumerate}
\item[(i)] the expected time to reach a node from  is bounded by:

\item[(ii)] the expected number of visits to node  before any time , , is bounded by:

\end{enumerate}
\end{lemma}
The proof of the lemma follows by an analysis of resistances of replacement along shortest paths in the resistor network for the weighted graph .

\section{A time-space trade-off for USTCON}\label{sec3}

The time-space tradoffs for USTCON proposed by \Broder~\cite{BKRU} make use of a number of short random walks, originating from a subset of nodes of the graph called \emph{landmarks}. Herein, we design an algorithm which replaces these random walks by Metropolis-Hastings walks.

We start by a brief overview of the landmark-based approach. When considering an algorithm using space , the size of the set of landmarks is defined by a parameter . The algorithm first chooses a set of landmarks  consisting of  nodes: node , node , and  nodes picked (in the case of our work) uniformly at random from . Then, a walk of suitably chosen length  is released from each of the landmarks. Throughout this process, the algorithm maintains a disjoint-set data structure (also known as ``Union-Find''~\cite{HU}) on the set of landmarks, with each set corresponding to the landmarks identified as belonging to the same connected component of the graph.

Initially, each landmark belongs to a separate set. Whenever a walk starting from one landmark hits some other landmark, the algorithm updates the data structure, merging the classes corresponding to these two landmarks. At the end of the process, if landmarks  and  belong to the same class, then, with certainty, there exists an - path in , and the answer to USTCON is positive. Otherwise, the algorithm returns a negative result, and, in the rest of this Section, we focus on proving that this result is correct w.h.p.

The runtime of the algorithm of \Broder~is determined by the time of running  random walks of length  each, thus . To achieve the claimed trade-off of , we will therefore need to use walks of length roughly .

\subsection{An initial approach}\label{sec31}

We fix a value of the parameter , describing the number of landmark nodes. The landmark-based algorithms are built around the premise that landmarks belonging to the same connected component of  quickly discover each other with the help of the short walks they release. In particular, it is desirable that the set of landmarks in each connected component of  has the property that for any partition of the set of landmarks into two subsets, a short walk originating from a landmark in one of these subsets is likely to reach some landmark from the other subset. \Broder~\cite{BKRU} observe (cf.\ also~\cite{F} for a high-level exposition of the argument) that this property is satisfied if the considered set of landmarks is \emph{good}, i.e., it fulfills the following two assumptions. Firstly, the set of short walks originating from all of the landmarks should be likely to jointly cover all the arcs of the graph. Secondly, a short walk originating from an arbitrary starting node of the graph should be likely to reach at least one landmark from the set.

Most of the analysis and key lemmas in this subsection follow along the lines proposed by \Broder We confine ourselves to a summary of the approach, highlighting the subtle differences resulting from the use of Metropolis-Hastings walks. We start by re-setting the good landmark property in the context of Metropolis-Hastings walks  of a specifically chosen length .

\begin{property}\label{begood}
Let  be the set of  landmark nodes, let  be a connected component of , and let , where  is a suitably chosen absolute constant (whose value follows from the proof of Lemma~\ref{lemGood}). We say that the set of landmarks  is \emph{good with respect to } if the following properties hold:
\begin{itemize}
\item With probability at least , a set of  walks  of length  each, with one walk originating from each landmark from , covers an arbitrarily chosen arc of .
\item With probability at least , a walk  of length , originating from an arbitrarily chosen node of , hits some landmark from .
\end{itemize}
\end{property}

In the above property, the choice of the length  of the walk takes into account that walks of length  lead us to the sought time complexity of  for the algorithm. However, in order to ensure that a uniformly sampled landmark set is likely to be good, we will make use of the low node-return rate of the Metropolis-Hastings walk from Lemma~\ref{lemiii}, and thus we need to have .

We will now show that that Property~\ref{begood} holds w.h.p.\ for a set of landmarks, each of which is chosen according to the uniform distribution  on the set of nodes . To achieve this, we capture the ``contribution'' of a single Metropolis-Hastings walk to the probability of success of the events described in the Property. It turns out that a Metropolis-Hastings walk  of the chosen length , when starting from a landmark, has probability  of reaching an arbitrary arc of the graph. When starting from an arbitrary node from , such a walk has probability  of reaching any specific landmark. These claims are formulated in a slightly more general way as the two lemmas below. Their proofs take into account the low node-return rate property from Lemma~\ref{lemiii}, and the properties of a walk starting from its stationary distribution .

\begin{lemma}\label{lemA}
Suppose that  is connected. For a weighted walk  starting from a node chosen according to the uniform distribution , the probability of traversing (a fixed) non-loop arc  before time , where , is:

\end{lemma}

\begin{lemma}\label{lemB}
Suppose that  is connected. Let  be picked according to the uniform distribution . For a weighted walk  starting from some node , the probability of reaching  before time , where , is:

\end{lemma}

After combining the above lemmas and applying some elementary arguments about unions of independent events, we finally obtain that Property~\ref{begood} is satisfied w.h.p.\ by landmarks uniformly chosen from , provided that the considered connected component is sufficiently large.

\begin{lemma}\label{lemGood}
If a connected component  has  nodes, then a (multi)set of  nodes, picked with uniform probability from , is a good set of landmarks with respect to  with probability at least .
\end{lemma}


The results of \Broder imply directly that if a set of landmarks is good with respect to a connected component , then all landmarks in  can be identified as belonging to the same connected component by releasing a small number of walks from each landmark, and applying Union-Find type operations on a disjoint-set datastructure on the landmarks. Since the proof does not rely on any other assumptions beyond the properties of good landmarks, the result is directly applicable to our considerations of the Metropolis-Hastings walk.

\begin{lemma}[\cite{BKRU}]\label{bro}
Let  be a set of good landmarks with respect to connected component . Then, a set of walks of length  each, with  walks originating from each of the landmarks, with probability at least  discovers that all landmarks located within  belong to the same connected component.
\end{lemma}
\noindent
In the above, the absolute constant  can be chosen as .

Our algorithm for USTCON is now obtained as follows. We pick a set of landmarks , consisting of , , and  nodes picked uniformly at random from , and then follow  Metropolis-Hastings walks from each landmark, updating the disjoint-set data structure. Finally, the algorithm decides whether  and  are connected based on whether these two landmarks have been identified as belonging to the same connected component.

The algorithm never provides a false-positive answer. The probability of identifying a pair of nodes  and  from the same component  as not being connected, can be bounded using the following argument adapted from \Broder Let  be the connected component of  containing node . If , then by Lemma~\ref{lemGood}, the set  is a set of good landmarks with respect to  with probability at least  (note that adding nodes  and  to a good set of landmarks cannot make this set of landmarks a bad one). Conditioned on this, by Lemma~\ref{bro}, we obtain a correct answer to USTCON with probability . Thus, the algorithm works correctly with probability at least . In the case when , we consider only the walks originating from landmark . There are  such (independent) walks, each of length . It follows from Lemma~3, putting  and , that in this case, node  will be reached with probability at least . This completes the proof of correctness.

\begin{proposition}
For all , there is an algorithm solving USTCON using space  and time , where , with probability of one-sided error bounded by .\qed
\end{proposition}

For the case when , we have obtained the trade-off . We now show how to obtain the claimed trade-off in the general case.

\subsection{Removing the dependence on maximum degree}\label{sec32}

We now remove the dependence of length of the used walks on the value of . We design a graph  by subdividing the nodes of , so that each node from  turns into a path of nodes in  with maximum degree bounded by , where  is an integer parameter, whose value is specified later. Formally, graph  is defined as follows:
\begin{itemize}
\item For each node ,  contains  copies of , labeled .
\item Nodes  and , , are connected by an edge in  if and only if , , and .
\item Nodes  and , for all , are connected by an edge of , with special port labels  and  at its endpoints.
\end{itemize}
Let  and  be the number of nodes and the maximum degree of , respectively. We have , and the following bound on  holds:

Solving USTCON on  between nodes  and  can be reduced to solving USTCON on  between nodes  and , since the transformation of  into  does not affect connectivity. In order to apply the algorithm for USTCON to , rather than to , we introduce the following modifications:
\begin{itemize}
\item Landmarks need to be distributed following the uniform distribution on . This can be achieved by picking  integers uniformly at random from the range , then enumerating all the nodes of  in order, and associating the landmarks with the corresponding nodes from . This operation can be performed in  time, which is always .
\item The performed walks need to follow , rather than . A simulation of one step of the walk  can be performed in  time.
\item The duration of each of the performed walks is given as , where:

\end{itemize}
It follows that the time complexity of the entire algorithm is bounded by the  complexity of landmark distribution and the  complexity of simulating the Metropolis-Hastings walks on . Substituting the expression from~\eqref{eqn0} for , we have:

Now, putting  gives , and we obtain the required time bound . Since the proposed solution can be implemented with a space bound of , we have proven the main theorem of the paper.

\begin{theorem}\label{thm:main}
For all , where  is some model-dependent constant, there is an algorithm solving USTCON using space  and time , with probability of one-sided error bounded by .\qed
\end{theorem}


\section{Remarks}\label{sec5}

\paragraph{Tightness of the trade-off.}\label{sec51}

For a space bound , we cannot hope for an algorithm with smaller run-time than , achieved in Theorem~\ref{thm:main}.
In fact, the lower bound of  holds for the RAM model under most reasonable representations of  in the memory (cf. Proposition~\ref{lbm} in Appendix B for a standard proof).

For smaller values of , the optimality of the achieved trade-off  is open. For the extremal case of , the results of~\cite{BBRRT} imply that  for any deterministic algorithm using a jumping automaton (JAG) with at most one movable pebble. There is also little hope of improving the time complexity using randomized algorithms similar to the Metropolis-Hastings walk, since Nonaka \etal~\cite{NOSY} showed that any walk, having a stationary distribution which is (almost) uniform on the nodes of the graph, has  cover time for some graphs.

Even more strongly, one can ask whether there exists an algorithm for USTCON which runs in  space and  time. This appears unlikely in view of the negative result of Edmonds~\cite{E}, who showed that a randomized JAG using  space and  pebbles requires in expectation  time to explore certain -regular graphs.

\paragraph{Fine-tuning the Metropolis-Hastings walk.}\label{sec52}

In view of Lemma~\ref{lemtwo}, the Metropolis-Hastings walk visits all the nodes of a graph within  steps. This is an improvement with respect to the bound of  on the cover time of an unbiased random walk. Nevertheless, the Metropolis-Hastings walk may perform worse than the random walk for specific graph classes. A generic example of such a graph, called the \emph{glitter star}, was defined by~\cite{NOSY} as a tree on  nodes, with one central node of degree  connected to  nodes of degree , which are in turn connected to  leaves. On the glitter star, the cover time of the random walk is , and the cover time of the Metropolis-Hastings walk is .

Below we propose a walk  with a different potential function which combines some of the advantages of the random walk and the Metropolis-Hastings walk.

\begin{proposition}\label{probla}
For a graph , let the node potential function  be given as , where  is the average degree of the graph.  Then, for any pair of nodes , the walk  achieves a commute time of:

where  and  denote the commute times for the random walk on  and the Metropolis-Hastings walk, respectively. A step of the walk  can be simulated using  space and time.
\end{proposition}

The above Proposition implies that for any graph, the walk  with , is asymptotically not slower than the unbiased random walk in terms of parameters such as maximum hitting time and (arbitrarily weighted) average hitting time. At the same time, this walk preserves the upper bound of  on the cover time in the graph, making it an interesting alternative to the unbiased random walk in practical applications, e.g., for different random graph models.

We remark that there exist different ways of combining the unbiased random walk and the Metropolis-Hastings walk. For example, one may consider an automaton which iteratively performs a phase of the walk , followed by a phase of the walk  of the same length, doubling the lengths of both walks in each subsequent iteration. Such a walk visits all the nodes of the graph in expected time asymptotically equal to the cover time of the faster of the two walks.

\newpage
\pagenumbering{roman}
\setcounter{page}{1}
\begin{thebibliography}{19}

\bibitem{AF}
D.~Aldous, J.~Fill.
\emph{Reversible Markov Chains and Random Walks on Graphs}.
Book draft available at \texttt{http://stat-www.berkeley.edu/users/aldous/RWG/book.html}, 2001.

\bibitem{AKLLR}
R.~Aleliunas, R.M.~Karp, R.J.~Lipton, L.~Lovasz, C.~Rackoff.
Random walks, universal traversal sequences, and the complexity of maze problems.
\emph{Proc. 20th Annual Symposium on Foundations of Computer Science (FOCS)}, 1979, pp.~218--223.

\bibitem{ACGR}
C.~Alvarez, R.~Greenlaw.
A compendium of problems complete for symmetric logarithmic space.
\emph{Computational Complexity }9:2 (2000), 123--145.

\bibitem{ATWZ}
R.~Armoni, A.~Ta-Shma, A.~Wigderson, S.~Zhou.
An  space algorithm for  connectivity in undirected graphs.
\emph{Journal of the ACM }47:2 (2000), 294–-311.

\bibitem{BF}
G.~Barnes, U. Feige.
Short random walks on graphs.
\emph{Proc. 25th Annual ACM Symposium of the Theory of Computing (STOC)}, 1993, pp.~728--737.

\bibitem{BBRRT}
P.~Beame, A.~Borodin, P.~Raghavan, W.L.~Ruzzo, M.~Tompa.
Time-Space Tradeoffs for Undirected Graph Traversal by Graph Automata.
\emph{Information and Computation }130:2 (1996), 101--129.

\bibitem{BCERS}
P.~Berenbrink, C.~Cooper, R.~Els\"{a}sser, T.~Radzik, T.~Sauerwald.
Speeding Up Random Walks with Neighborhood Exploration.
\emph{Proc. 21st Annual ACM-SIAM Symposium on Discrete Algorithms (SODA)}, 2010, pp.~1422--1435.

\bibitem{BS}
P.~Berman, J.~Simon.
Lower bounds on graph threading by probabilistic machines.
\emph{Proc. 24th Annual Symposium on Foundations of Computer Science (FOCS)}, 1983, pp.~304--311.

\bibitem{BCDRT}
A.~Borodin, S.A.~Cook, P.W.~Dymond, W.L.~Ruzzo, M.~Tompa.
Two applications of inductive counting for complementation problems.
\emph{SIAM Journal on Computing }18:3 (1989), 559–-578.

\bibitem{BKRU}
A.Z.~Broder, A.R.~Karlin, P.~Raghavan, E.~Upfal.
Trading space for time in undirected - connectivity.
\emph{SIAM Journal on Computing }23 (1994), 324--334.

\bibitem{CRRST}
A.~Chandra, P.~Raghavan, W.L.~Ruzzo, R.~Smolensky, P. Tiwari.
The electrical resistance of a graph captures its commute and cover times.
\emph{Proc. 21st Annual ACM Symposium of the Theory of Computing (STOC)}, 1989, pp.~574--586.

\bibitem{CR}
S.A.~Cook, C.W.~Rackoff
Space lower bounds for maze threadability on restricted machines.
\emph{SIAM Journal on Computing }9:3 (1980), 630--652.

\bibitem{E}
J.~Edmonds.
Time-space trade-offs for undirected st-connectivity on a JAG.
\emph{Proc. 25th Annual ACM Symposium of the Theory of Computing (STOC)}, 1993, pp.~718--727.

\bibitem{F}
U.~Feige.
A Randomized Time-Space Trade-off of  for USTCON.
\emph{Proc. 34th Annual Symposium on Foundations of Computer Science (FOCS)}, 1993, pp.~238--246.

\bibitem{GLM}
P.~Gopalan, R.J.~Lipton, A.~Mehta.
Randomized Time-Space Tradeoffs for Directed Graph Connectivity.
\emph{Proc. Foundations of Software Technology and Theoretical Computer Science (FSTTCS)}, LNCS 2914, 2003, pp.~208--216.

\bibitem{HU}
J.E.~Hopcroft, J.D.~Ullman.
Set Merging Algorithms.
\emph{SIAM Journal on Computing }2:4 (1973), 294-–303.

\bibitem{IKY}
S.~Ikeda, I.~Kubo, M.~Yamashita.
The hitting and cover times of random walks on finite graphs using local degree information.
\emph{Theoretical Computer Science }410:1 (2009), 94--100.

\bibitem{M51}
N.~Metropolis, A.W.~Rosenbluth, M.N.~Rosenbluth, A.H.~Teller, and E.~Teller,
Equation of State Calculations by Fast Computing Machines,
\emph{Journal of Chemical Physics} 21 (1953), 1087--1092.

\bibitem{Nis}
N.~Nisan.
.
\emph{Proc. 24th Annual ACM Symposium of the Theory of Computing (STOC)}, 1992, pp.~619--623.

\bibitem{NOSY}
Y.~Nonaka, H.~Ono, K.~Sadakane, M.~Yamashita.
The hitting and cover times of Metropolis walks.
\emph{Theoretical Computer Science} 411:16--18 (2010), 1889--1894.

\bibitem{Rei}
O.~Reingold.
Undirected connectivity in log-space.
\emph{Journal of the ACM} 55:4 (2008), 1--24.

\bibitem{Sav}
W.J.~Savitch.
Relationships between nondeterministic and deterministic tape complexities.
\emph{Journal of Computer and System Sciences} 4 (1970), 177--192.

\bibitem{SZ}
M.~Saks, S.~Zhou.
.
\emph{Journal of Computer and System Sciences} 58:2 (1999), 376--403.

\end{thebibliography}

\newpage
\section*{Appendix A: Implementation}

For the sake of completeness, below we provide the pseudocode of the algorithm for USTCON announced in Theorem~\ref{thm:main}. The implementation is self-contained, except for the following subroutines. The disjoint-set data structure is implemented by the procedures:  which adds a new set containing only element  to the data structure,  which returns (the identifier of) the set containing element , and  which replaces sets  and  by set  in the data structure. Each of these operations is performed in amortized  time.

The routine , for a node , returns a pair , such that  with , and . This routine can be performed in  time in the RAM model, as well as in most JAG-based models.

We recall the values of the absolute constants:  and .

\medskip
\noindent
// \emph{Solution to USTCON using  auxiliary landmarks}\\
\textbf{procedure} test\_connectivity (: nodes from ) \{\\
\* ;\\
\* ;\\
\* ;\2mm]
\* // \emph{From each landmark, run  Metropolis walks  of length  each}\\
\* \textbf{repeat}  times \{\\
\* \* \textbf{for}  \textbf{do} \{\\
\* \* \* ; \\
\* \* \* \textbf{repeat}  times \{\\
\* \* \* \*  next\_state* ();\\
\* \* \* \* ;\\
\* \* \* \}\\
\* \* \}\\
\* \}\\
\* \textbf{if}  \textbf{then} \textbf{return} ``\emph{connected}'';\\
\* \textbf{return} ``\emph{probably not connected}'';\\
\}\\
\\
// \emph{Simulate one step of the walk  from state }\\
\textbf{function} next\_state* (: node, : integer) \{\\
\* get\_degree*;\\
\*  get\_random\_port*;\\
\* \textbf{if}  \textbf{then} \{\\
\* \* ;\\
\* \* ;\\
\* \} \textbf{else if}  \textbf{then} \{\\
\* \* ;\\
\* \* ;\\
\* \} \textbf{else} \{ \emph{//  is an integer corresponding to a port at  in }\\
\* \* ;\\
\* \* ;\\
\* \}\\
\*  get\_degree*;\\
\* \textbf{with probability} 
 \textbf{do} \textbf{return} ;\\
\* \textbf{return} ;\\
\}\\
\\
// \emph{Return the degree of  in }\\
\textbf{function} get\_degree* (: node, : integer) \{\\
\* ;\\
\* ;\\
\* ;\\
\* \textbf{if}  \textbf{then} ;\\
\* \textbf{if}  \textbf{then} ;\\
\* \textbf{return} ;\\
\}\\
\\
// \emph{Return a port at node  in  chosen uniformly at random}\\
\textbf{function} get\_random\_port* (: node, : integer) \{\\
\* ;\\
\* ;\\
\*  get\_degree* ;\\
\* \textbf{with probability}  \textbf{do}\\
\* \* \textbf{return} integer from range  chosen uniformly at random;\\
\* ;\\
\* \textbf{if}  \textbf{then} ;\\
\* \textbf{if}  \textbf{then} ;\\
\* \textbf{return} element of  chosen uniformly at random;\\
\}\\

\newpage
\section*{Appendix B: Auxiliary claims}


\begin{lemma}\label{lemrev}
For all , 
\end{lemma}

\begin{proof}
Let , , denote the random variable equal to  if a walk of length  is located at  after  steps, and  otherwise. Since  is a reversible Markovian process, by the properties of the -th power of the transition matrix of the walk (cf.~\cite{AF}, Chapter 3.1), we have:

Since , it follows that . Taking into account that , , by linearity of expectation we obtain the claim.
\end{proof}


\begin{lemma}\label{lembla}
For any node :

and for any arc  of  corresponding to an edge :

\end{lemma}
\begin{proof}
Follows directly from the stationary distribution of the Metropolis-Hastings walk on nodes and edges.
\end{proof}

\begin{proposition}\label{lbm}
Any algorithm for USTCON requires time .
\end{proposition}
\begin{proof}
Consider a generic instance of USTCON defined as follows. Take two disjoint copies of some -edge-connected graph  on  nodes, with one distinguished node . The two copies of  are assigned the subscripts  and , respectively. Now, as the considered instance of USTCON we use, with probability , the disconnected graph  with  and . Otherwise, we pick an edge  of  uniformly at random, and use as the instance the connected graph , likewise with  and . Subject to a choice of node identifiers in the representations, the connected and disconnected instances differ on precisely  memory cells in the adjacency lists of the graph (for nodes , , , and ), and these cells, taken over the choices of edge , form a partition of the memory representation of the graph. Consequently, the expected number of memory reads for an algorithm deciding connectivity with probability  is lower-bounded by , and is thus , within the range .
\end{proof}


\section*{Appendix B: Proofs of technical lemmas}


\subsection*{Proof of Lemma~\ref{lemiii}}

The interested reader may see this proof as an analogue of the discussion for short random walks in regular graphs, cf.~Aldous and Fill, Chapter 6, Proposition 16.

\emph{Claim} : Consider a shortest path  in graph  from  to a nearest vertex . Let , where , and ,  for . Let  be the subgraph of  induced by nodes from set , their neighbors in , and node : . Since any random walk in  which starts from  and does not enter  is confined to nodes and edges of graph , we have the following relation between the walks  and :

where the latter equality follows from the electrical network representation of random walks. The resistance  is upper-bounded by the resistance of the series connection going through the nodes of path  in :


Since the path  is a shortest path in graph  between nodes  and , such that  and , it follows that (cf.~\cite{AF}):

and:

Since the total weight of edges and self-loops of  incident to a vertex in  is equal to , we have:

Claim  follows from inequalities~\eqref{eq1},~\eqref{eq2}, and~\eqref{eq3}.

\medskip
\noindent
\emph{Claim} : Suppose that , and let:

Since the considered walk hits nodes from  a total of (at most)  times, we have , and the considerations performed in the proof of Lemma~\ref{lemiii} can be applied for the above-defined set .

First, we bound the expected number of returns to node  for a walk starting at  before reaching  for the first time:

Taking into account~\cite{AF} (Chapter 3, eq. (28) and Corollary 11) and bound~\eqref{eq2}, we have:


It follows from Lemma~\ref{lemrev} that the definition of set  may be rewritten as:

Thus, , which means that if a walk starting from  reaches , it will return to  at most  times in expectation before time . So, using~\eqref{eq4}, we obtain the claim:


\qed


\subsection*{Proof of Lemma~\ref{lemA}}

Fix an arbitrary arc , with . We will bound the sought probability from the inequality:

The expected number of traversals of  for a walk of even length starting from the stationary distribution on  is given by equation~\eqref{eq6}.

To bound the expectation from the denominator of~\eqref{eq5}, we note that by Lemma~\ref{lemiii},  , and that arc  is chosen with probability  during each visit to :

Considering a walk starting from a traversal of arc , we observe that after its traversal of   the walk must return to node  before traversing  again:


By combining inequalities~\eqref{eq6},~\eqref{eq5},~\eqref{eq7}, and taking into account that , we obtain the claim:

\qed


\subsection*{Proof of Lemma~\ref{lemB}}

Pick a node  according to the uniform probability distribution . We will bound the sought probability from the inequality:

Taking into account Lemma~\ref{lemrev} and condition~\eqref{eqU}, and noting that  is chosen according to the uniform distribution  on , we have:

The expectation from the denominator of~\eqref{eq5b} is bounded by Lemma~\ref{lemiii}, . By combining the above relations, and taking into account that , we obtain:

\qed

\subsection*{Proof of Lemma~\ref{lemGood}}

Fixing a connected component  with , we introduce the following notation for a set of landmarks :
\begin{itemize}
\item let ,
\item let  denote the event that ,
\item let  be the random variable over  describing the maximum, over all non-loops arcs  belonging to , of the probability that a set of  random walks  of length  each, with one random walk originating from each landmark from , does not cover arc .
\item let  be the random variable over  describing the maximum, over all nodes , of the probability that a random walk  of length , originating from , does not hit any landmark of .
\end{itemize}
Suppose that  is a set of  nodes picked according to the uniform distribution  on . To prove the claim of the Lemma, we need to show the following bound:

We observe that each landmark from  belongs to  with probability . Let . A w.h.p.\ lower bound on the size of  follows from the Chernoff bound applied to  Bernoulli trials with success probability :

where we took into account that , and that . In the following, we only need to show that, conditioned on the event  holding,  is a good set of landmarks with probability . Note that all the landmarks from  are distributed uniformly at random on , also when conditioned on .

To bound , fix a non-loop arc  of  as the arc maximizing the failure probability in the definition of . By applying Lemma~\ref{lemA} to graph , the probability that a walk  of length , starting from the uniform distribution on , does not cover arc , is at most . Thus, considering that:

the probability  that no random walk starting from a landmark hits arc  is bounded by:


Likewise, to bound , fix a node  maximizing the probability that a walk  of length , originating from , does not hit any landmark of . By Lemma~\ref{lemB}, the probability that the considered walk of length  does not cover a node chosen according to the uniform distribution on , is at most . Thus, taking into account that , the probability that the walk does not hit any landmark can once again be bounded as less than :

It follows that:

and by the Markov bound:

Now, inequalities~\eqref{eq12} and~\eqref{eq13} imply that inequality~\eqref{eq11} holds, which completes the proof.
\qed

\subsection*{Proof of Proposition~\ref{probla}}

We begin by observing that the unbiased random walk on  can be described as a weighted Metropolis-Hastings walk , where, for all , the potential function on nodes is given as , where  is an arbitrarily chosen constant of proportionality ( for all edges). Now, looking at the electrical networks analogy, by identifying with each other the corresponding nodes of the electrical networks describing the walks  and , and leaving the edges of both these networks in parallel connection, we obtain a new network on  with edge weights  given by:

corresponding to the potential function on nodes:

It follows that the resistance of replacement of the network of  for any two nodes  can be bounded as:

Moreover, the following relations hold between resistances and commute times:



Fixing , i.e., , we obtain from all of the above relations:

\qed
\end{document}
