In this section we present an automata-theoretic algorithm for solving the reversal- and register-bounded verification problem. Before we give an overview of our algorithm and outline the automata constructions it comprises, we recall some basic definitions from automata theory.

\subsection{Preliminaries}
 A 2-way nondeterministic finite automaton (2NFA) is a tuple
, where 
 is a finite set of states,
 is the initial state,
 is a finite alphabet,
 is the transition relation and 
 is a set of accepting states.
 is deterministic iff  is a function from  to .
 is a 1-way NFA (NFA) iff  for each . 

For , ,  and , let  iff  and
(1) if , then , , either  or  and  and 
(2) if , then , , either  or  and .
 
   
If  is an NFA, we define  for  in the obvious way.


Let  and , where , be designated symbols and .

If  is a 2NFA, then  iff
 or 
 for some .
If  is an NFA, then  iff .

A push-down automaton (PDA) is a tuple , where
 is a finite set of states,
 is the initial state,
 is a finite input alphabet,
 is a finite stack alphabet,
 is the start symbol and
 is the transition relation. For , , , ,  we define 
 iff
.

For a PDA ,  iff .


\subsection{Overview of the algorithm}
We now outline the construction of a PDA , which we use in order to reduce the reversal- and register-bounded verification problem to checking emptiness of a PDA. We begin by describing the input of the  automata involved in the construction and then proceed to give an overview of the construction followed by a  
formal definition of the input alphabets of these automata.

\looseness=-1
The automaton  reads words that consist of traces in . 
In order to reflect sufficient information about the corresponding nodes and
registers in the underlying graph, these traces are annotated as follows. 
First, since graphs derived by  contain at most  registers,
we assume these registers to have unique identifiers from the set .
Thus, a triple  
consists of an edge label  and the identifiers of the registers at the source and target node 
of the edge, where  indicates no register at the respective node.
We add additional annotation to reflect which nodes are shared in the corresponding
paths, that is, positions where paths in the series-parallel graph branch off or join. 

The automaton  reads such annotated traces and checks the existence of a run
by emulating the behaviour of the machines on these traces by guessing an execution
for each of them. 
An \emph{execution} of  is a sequence  
 
such that 
for some .
In addition to verifying that each guess is indeed an execution,
 needs to also check that the values written to and read from
the shared registers by different machines are consistent. 

Formally, an annotated trace and executions of the machines  define a
\emph{read-write} sequence , 
where, intuitively,  is the location that is read and/or written.
Such a read-write sequence  is \emph{valid w.r.t.\ an initial register valuation 
} iff each read operation reads the value
written by the most recent write operation, or the initial value from  if it is not overwritten,
that is, for  with  it holds that 
if there is  such that , then 
for the largest such , and otherwise .

\looseness=-1
Thus, the automaton  accepts tuples of traces in some graph derived by , annotated with information about registers and about nodes shared by the corresponding paths in the graph. 
 also guesses an execution for each of the  machines. 
The PDA  is constructed as the intersection of a PDA  and an NFA  (Section~\ref{sec:pda-traces}). 
 checks that its input word encodes a tuple of traces in some graph derived by  and that these are correctly annotated with information about registers and the nodes that are shared among the paths corresponding to these traces. 
The NFA  guesses and verifies the executions of the machines. It is obtained as the intersection of  NFAs:
 NFAs , one for each , an NFA  and an NFA .
The NFA  verifies that the guess of an execution of machine  is correct.
We describe the construction of  as a 2NFA (Section~\ref{sec:2nfa-executions}) which is then converted to an NFA using standard techniques~\cite{HopcroftUllman}.
Automaton  checks the validity of the read-write sequence corresponding to the annotated traces and the guessed executions  (Section~\ref{sec:2nfa-read-write}). 
Automaton  (Section~\ref{sec:1nfa-properties}) checks that a configuration in  is reached.
The reversal- and register-bounded verification problem thus reduces to checking emptiness of the language of the constructed automaton .

\looseness=-1
According to Section~\ref{sec:runs-to-words}, it suffices to reason about  traces in graphs derived by .  To this end, we define the \emph{trace alphabet}

Words over  are tuples of  traces in some graph , annotated with additional information. Each letter in  contains one row for each of the  machines and each of the  paths corresponding to it. There are two types of letters. Each row in a letter of the first type consists of a letter in  (or the special symbol ) together with two \emph{register identifiers} in   and a \emph{path index} in . The letters of the second type are -tuples of path indices in , where equal indices indicate paths sharing a node. 

The \emph{execution alphabet}

is used to describe tuples of executions, one for each of the  machines. 
Each letter contains  rows for each machine, one for each of its paths. Each row in the letter consists of a \emph{block number} in , two \emph{register values} (one for the read and one for the write operations), a \emph{successor state} and an \emph{index of a row} in an associated trace word (word in ).
Let  be the product of the trace and execution alphabets.

In what follows, if , then 
 
denotes the elements of the -th letter of the word  for , and we use
 
to denote the -th row of  for .
Similarly, if , 
the -th letter is

for , 
and the -th row is
,
for .
For ,  and ,
the corresponding letter from  is denoted
.

In the remainder of this section we present the intuition behind the automata 
constructions and their properties.

