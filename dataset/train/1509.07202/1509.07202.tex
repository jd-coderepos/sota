In this section we present an automata-theoretic algorithm for solving the reversal- and register-bounded verification problem. Before we give an overview of our algorithm and outline the automata constructions it comprises, we recall some basic definitions from automata theory.

\subsection{Preliminaries}
 A 2-way nondeterministic finite automaton (2NFA) is a tuple
$\mc A = (Q,q^0,\Sigma,\delta,A)$, where 
$Q$ is a finite set of states,
$q^0 \in Q$ is the initial state,
$\Sigma$ is a finite alphabet,
$\delta \subseteq Q \times \Sigma \times  Q \times \{-1,1\}$ is the transition relation and 
$A \subseteq Q$ is a set of accepting states.
$\mc A$ is deterministic iff $\delta$ is a function from $Q \times \Sigma$ to $Q \times \{-1,1\}$.
$\mc A$ is a 1-way NFA (NFA) iff $d = 1$ for each $(q,\sigma,q',d) \in \delta$. 

For $q,q' \in Q$, $w',w'',w''',w'''' \in \Sigma^*$, $\sigma \in \Sigma$ and $\sigma' \in \Sigma \cup \{\epsilon\}$, let $\langle q,w',\sigma,w''\rangle \Rightarrow_{\mc A} \langle q',w''',\sigma',w''''\rangle$ iff $(q,\sigma,q',d) \in \delta$ and
(1) if $d = 1$, then $w''' = w'.\sigma$, $w'' =\sigma'.w''''$, either $\sigma' \in \Sigma$ or $\sigma' = \epsilon$ and $w'''' =\epsilon$ and 
(2) if $d = -1$, then $w'''' = \sigma . w''$, $w' =w'''.\sigma'$, either $\sigma' \in \Sigma$ or $\sigma' = \epsilon$ and $w''' =\epsilon$.
 
   
If $\mc A$ is an NFA, we define $\delta(q,w)$ for $w \in \Sigma^*$ in the obvious way.


Let $\vdash \in \Sigma$ and $\dashv \in \Sigma$, where $\vdash \neq \dashv$, be designated symbols and $w \in (\Sigma \setminus \{\vdash,\dashv\})^*$.

If $\mc A$ is a 2NFA, then $w \in \mc L(\mc A)$ iff
$\langle q^0,\epsilon,\vdash, w \dashv \rangle \Rightarrow_{\mc A}^* \langle q,\vdash w \dashv,\epsilon,\epsilon \rangle$ or 
$\langle q^0,\epsilon,\vdash, w \dashv \rangle \Rightarrow_{\mc A}^* \langle q,\epsilon,\epsilon,\vdash w \dashv \rangle$ for some $q \in A$.
If $\mc A$ is an NFA, then $w \in \mc L(\mc A)$ iff $\delta(q^0,\vdash w \dashv) \cap A \neq \emptyset$.

A push-down automaton (PDA) is a tuple $\mc P = (Q,q^0,\Sigma,\Delta,\bot,\delta)$, where
$Q$ is a finite set of states,
$q^0 \in Q$ is the initial state,
$\Sigma$ is a finite input alphabet,
$\Delta$ is a finite stack alphabet,
$\bot$ is the start symbol and
$\delta \subseteq Q \times (\Sigma \cup \{\epsilon\}) \times \Delta \times Q \times \Delta^*$ is the transition relation. For $q,q' \in Q$, $\sigma \in \Sigma \cup \{\epsilon\}$, $w \in \Sigma^*$, $a \in \Delta$, $\alpha,\beta \in \Delta^*$ we define 
$\langle q, \sigma.w,a.\alpha\rangle \Rightarrow_{\mc P} \langle q', w,\beta.\alpha\rangle$ iff
$(q,\sigma,a,q',\beta) \in \delta$.

For a PDA $\mc P$, $w \in \mc L(\mc P)$ iff $\langle q^0,\vdash w \dashv,\bot\rangle \Rightarrow_{\mc P}^* \langle q,\epsilon,\bot \rangle$.


\subsection{Overview of the algorithm}
We now outline the construction of a PDA $\mc A$, which we use in order to reduce the reversal- and register-bounded verification problem to checking emptiness of a PDA. We begin by describing the input of the  automata involved in the construction and then proceed to give an overview of the construction followed by a  
formal definition of the input alphabets of these automata.

\looseness=-1
The automaton $\mc A$ reads words that consist of traces in $\Sigma^*$. 
In order to reflect sufficient information about the corresponding nodes and
registers in the underlying graph, these traces are annotated as follows. 
First, since graphs derived by ${\mc G}^k$ contain at most $k$ registers,
we assume these registers to have unique identifiers from the set $\{1,\ldots,k\}$.
Thus, a triple $(\sigma,j_1,j_2) \in \Sigma\times\{0,\ldots,k\}\times\{0,\ldots,k\}$ 
consists of an edge label $\sigma$ and the identifiers of the registers at the source and target node 
of the edge, where $0$ indicates no register at the respective node.
We add additional annotation to reflect which nodes are shared in the corresponding
paths, that is, positions where paths in the series-parallel graph branch off or join. 

The automaton $\mc A$ reads such annotated traces and checks the existence of a run
by emulating the behaviour of the machines on these traces by guessing an execution
for each of them. 
An \emph{execution} of $\mc M = (Q,q^0,\Sigma,\delta)$ is a sequence  
$\xi = q_0,(\sigma_1,b_1,b_1'),q_1,\ldots,(\sigma_f,b_f,b_f'),q_f$ 
such that $(q_{l-1},\sigma_l,b_l,q_l,d_l,b_l') \in \delta$
for some $d_l \in \{1,-1\}$.
In addition to verifying that each guess is indeed an execution,
$\mc A$ needs to also check that the values written to and read from
the shared registers by different machines are consistent. 

Formally, an annotated trace and executions of the machines  define a
\emph{read-write} sequence $\eta = (j_1,b_1,b'_1),\ldots,(j_f,b_f,b'_f) \in 
(\{0,\ldots,k\} \times \mathbb{B} \times \mathbb{B})^*$, 
where, intuitively, $j_i$ is the location that is read and/or written.
Such a read-write sequence $\eta$ is \emph{valid w.r.t.\ an initial register valuation 
$\beta_0 : \{1,\ldots,k\} \to \mathbb{B}$} iff each read operation reads the value
written by the most recent write operation, or the initial value from $\beta_0$ if it is not overwritten,
that is, for $i \in \{1,\ldots,f\}$ with $j_i > 0$ it holds that 
if there is $i' < i$ such that $j_{i'} =  j_i$, then $b_i = b_{i'}'$
for the largest such $i'$, and otherwise $b_i = \beta_0(j_i)$.

\looseness=-1
Thus, the automaton $\mc A$ accepts tuples of traces in some graph derived by ${\mc G}^k$, annotated with information about registers and about nodes shared by the corresponding paths in the graph. 
$\mc A$ also guesses an execution for each of the $m$ machines. 
The PDA $\mc A$ is constructed as the intersection of a PDA $\mc P_{\mathsf t}$ and an NFA $\mc A_{\mathsf e}$ (Section~\ref{sec:pda-traces}). 
$\mc P_{\mathsf t}$ checks that its input word encodes a tuple of traces in some graph derived by ${\mc G}^k$ and that these are correctly annotated with information about registers and the nodes that are shared among the paths corresponding to these traces. 
The NFA $\mc A_{\mathsf{e}}$ guesses and verifies the executions of the machines. It is obtained as the intersection of $m+2$ NFAs:
$m$ NFAs $A_i$, one for each $i \in \{1,\ldots,m\}$, an NFA $\mc A_{\mathsf c}$ and an NFA $\mc A_{\mathsf s}$.
The NFA $\mc A_i$ verifies that the guess of an execution of machine $i \in \{1,\ldots,m\}$ is correct.
We describe the construction of $\mc A_i$ as a 2NFA (Section~\ref{sec:2nfa-executions}) which is then converted to an NFA using standard techniques~\cite{HopcroftUllman}.
Automaton $\mc A_{\mathsf c}$ checks the validity of the read-write sequence corresponding to the annotated traces and the guessed executions  (Section~\ref{sec:2nfa-read-write}). 
Automaton $\mc A_{\mathsf s}$ (Section~\ref{sec:1nfa-properties}) checks that a configuration in $F$ is reached.
The reversal- and register-bounded verification problem thus reduces to checking emptiness of the language of the constructed automaton $\mc A$.

\looseness=-1
According to Section~\ref{sec:runs-to-words}, it suffices to reason about $t = m \cdot (r+1)$ traces in graphs derived by ${\mc G}^k$.  To this end, we define the \emph{trace alphabet}
$\galph = 
\Big(\big(\Sigma\ \dot{\cup}\ \{\flat\}\big) \times \{0,\ldots,k\}^2 \times
\{1,\ldots,t\} \Big)^t\ \dot{\cup} \ \{1,\ldots,t\}^t.$
Words over $\galph$ are tuples of $t$ traces in some graph $G$, annotated with additional information. Each letter in $\galph$ contains one row for each of the $m$ machines and each of the $\widetilde r = r+1$ paths corresponding to it. There are two types of letters. Each row in a letter of the first type consists of a letter in $\Sigma$ (or the special symbol $\flat$) together with two \emph{register identifiers} in  $\{0,\ldots,k\}$ and a \emph{path index} in $\{1,\ldots,t\}$. The letters of the second type are $t$-tuples of path indices in $\{1,\ldots,t\}$, where equal indices indicate paths sharing a node. 

The \emph{execution alphabet}
$\calph = \big(
\{0,\ldots,p\}\times {\mathbb B} \times {\mathbb B}
 \times
Q \times \{1,\ldots,t\}
\big)^t$
is used to describe tuples of executions, one for each of the $m$ machines. 
Each letter contains $\widetilde r = r+1$ rows for each machine, one for each of its paths. Each row in the letter consists of a \emph{block number} in $\{0,\ldots,p\}$, two \emph{register values} (one for the read and one for the write operations), a \emph{successor state} and an \emph{index of a row} in an associated trace word (word in $\galph^*$).
Let $\widetilde \Sigma =  \galph \times \calph$ be the product of the trace and execution alphabets.

In what follows, if $\widetilde \tau = \widetilde \sigma_1\ldots\widetilde\sigma_f \in \galph^*$, then 
$\widetilde \sigma_j = (\widetilde \sigma_{1,j},\ldots,\widetilde \sigma_{t,j})$ 
denotes the elements of the $j$-th letter of the word $\widetilde \tau$ for $j \in \{1,\ldots,f\}$, and we use
$\widetilde \tau_i  = \widetilde \sigma_{i,1}\ldots\widetilde\sigma_{i,f}$ 
to denote the $i$-th row of $\widetilde\tau$ for $i \in \{1,\ldots,t\}$.
Similarly, if $\widetilde \tau = \widetilde \sigma_1\ldots\widetilde\sigma_f \in {\widetilde \Sigma}^*$, 
the $j$-th letter is
$\widetilde \sigma_j = (\widetilde \sigma_{1,j},\ldots,\widetilde\sigma_{t,j},
\widetilde\eta_{1,1,j},\ldots,\widetilde\eta_{1,\widetilde r,j},\ldots,
\widetilde\eta_{m,1,j},\ldots,\widetilde\eta_{m,\widetilde r,j}),$
for $j \in \{1,\ldots,f\}$, 
and the $i$-th row is
$\widetilde \tau_i  = \widetilde \sigma_{i,1}\ldots\widetilde\sigma_{i,f}$,
for $i \in \{1,\ldots,t\}$.
For $n \in \{1,\ldots, m\}$, $h \in \{1,\ldots,\widetilde r\}$ and $j\in \{1,\ldots,f\}$,
the corresponding letter from $\calph$ is denoted
$\widetilde\eta_{n,h,j} = (p_{n,h,j},b_{n,h,j},b'_{n,h,j},q'_{n,h,j},t_{n,h,j})$.

In the remainder of this section we present the intuition behind the automata 
constructions and their properties.

