\documentclass{bmvc2k}

\usepackage{mathtools}
\usepackage{amsfonts}
\usepackage{pifont}
\usepackage{booktabs}
\usepackage{flexisym}


\title{Construct Dynamic Graphs for Hand Gesture Recognition via Spatial-Temporal Attention}

\addauthor{Yuxiao Chen}{yc984@cs.rutgers.edu}{1}
\addauthor{Long Zhao}{lz311@cs.rutgers.edu}{1}
\addauthor{Xi Peng}{xipeng@udel.com}{2}
\addauthor{Jianbo Yuan}{jyuan10@cs.rochester.edu}{3}
\addauthor{Dimitris N. Metaxas}{dnm@cs.rutgers.edu}{1}

\addinstitution{
 Department of Computer Science\\
 Rutgers University\\
 New Jersey, USA
}
\addinstitution{
 Department of Computer and Information Sciences\\
 University of Delaware\\
 Delaware, USA
}
\addinstitution{
 Department of Computer Science\\
 University of Rochester\\
 New York, USA
}

\runninghead{Yuxiao Chen}{Dynamic Graphs for Hand Gesture Recognition}

\def\eg{\emph{e.g}\bmvaOneDot}
\def\ie{\emph{i.e}\bmvaOneDot}
\def\Eg{\emph{E.g}\bmvaOneDot}
\def\etal{\emph{et al}\bmvaOneDot}

\begin{document}
\maketitle

\begin{abstract}
We propose a Dynamic Graph-Based Spatial-Temporal Attention (DG-STA) method for hand gesture recognition. The key idea is to first construct a fully-connected graph from a hand skeleton, where the node features and edges are then automatically learned via a self-attention mechanism that performs in both spatial and temporal domains. 
We further propose to leverage the spatial-temporal cues of joint positions to guarantee robust recognition in challenging conditions. 
In addition, a novel spatial-temporal mask is applied to significantly cut down the computational cost by 99\%. We carry out extensive experiments on benchmarks (DHG-14/28 and SHREC'17) and prove the superior performance of our method compared with the state-of-the-art methods. The source code can be found
at \url{https://github.com/yuxiaochen1103/DG-STA}.
\end{abstract}

\section{Introduction}
\label{sec:intro}

Hand gesture recognition has been an active research area due to its wide range of applications such as human computer interaction, gaming and nonverbal communication analysis including sign language recognition~\cite{nikam2016sign,wachs2011vision,rautaray2015vision}. Previous work can be classified into two categories based on the modality of their inputs: image-based~\cite{freeman1995orientation,molchanov2015hand,wang2015superpixel} and skeleton-based~\cite{de2016skeleton,hou2018spatial,de2017shrec,chen2017motion,nunez2018convolutional} methods. Image-based methods take RGB or RGB-D images as inputs and rely on image-level features for recognition. On the other hand, skeleton-based methods make predictions by a sequence of hand joints with 2D or 3D coordinates. They are more robust to varying lighting conditions and occlusions given the accurate joint coordinates. Thanks to the low-cost depth cameras (\eg, Microsoft Kinect or Intel RealSense) and great progress made on hand pose estimation~\cite{oberweger2015hands,oberweger2017deepprior,oberweger2015training}, accurate coordinates of hand joints are easy to be obtained. Therefore, we follow the skeleton-based method in this work.


Conventional methods~\cite{ohn2013joint,de2016skeleton,de2017shrec} of skeleton-based hand gesture recognition aim to design powerful feature descriptors to model the action of hands. However, these hand-crafted features have limited generalization capability. Recent studies~\cite{hou2018spatial,chen2017motion,nunez2018convolutional} have achieved significant improvement using deep learning. They usually concatenate the joint coordinates into a tensor which is fed into a neural network, and then hand features are directly learned by the network during training. Nevertheless, the spatial structures and temporal dynamics of hand skeletons are not explicitly exploited in these deep learning based approaches.

More recent studies~\cite{yan2018spatial,si2018skeleton,zhao2019semantic} attempt to incorporate structures and dynamics of skeletons based on skeleton graphs. Specifically, given a sequence of skeletons, they define a spatial-temporal graph where structures and dynamics of skeletons are embedded. The feature representation of the graph is then extracted for action recognition. However, a pre-defined graph with fixed structure lacks the flexibility to capture the variance and dynamics across different actions, yielding sub-optimal performance in practice.  

To this end, we propose a \emph{Dynamic Graph-Based Spatial Temporal Attention (DG-STA)} model for hand gesture recognition. The key idea is to perform a self-attention mechanism in both spatial and temporal domains to modify a unified graph dynamically in order to model different actions.
Figure~\ref{fig:overview} gives an overview of our approach. There are three crucial designs that distinguish our approach from previous methods.
\textbf{First}, instead of a pre-defined graph with fixed structure, we propose to construct a unified graph where the edges and nodes are dynamically optimized according to different actions. This makes it is possible to achieve action-specific graphs with improved expressive power.
\textbf{Second}, we propose \emph{spatial-temporal position embedding} which improves the temporal position embedding~\cite{vaswani2017attention}. It encodes the identity and temporal order information of each node in the graph. Combining node features with their position embeddings can further improve the performance of our approach.
\textbf{Third}, to implement our DG-STA more efficiently, we present a novel \emph{spatial-temporal mask operation} which is directly applied to the matrix of scaled dot-products among all nodes. It significantly improves the computational efficiency of our model and allows easier input data arrangement.



\begin{figure}[t]
\includegraphics[width=\textwidth]{fig1.pdf}
\caption{Illustration of our method. The nodes in the graph correspond to hand joints and the dashed lines represent disconnected edges. The proposed DG-STA calculates edge weights and learn node features in both spatial and temporal domains of the hand skeleton graph.}
\label{fig:overview}
\end{figure}

To evaluate the effectiveness of our approach, we conduct comprehensive experiments on two standard benchmarks: DHG-14/28 Dataset~\cite{de2016skeleton} and SHREC'17 Track Dataset~\cite{de2017shrec}. The results demonstrate that our method outperforms the state-of-the-art methods. In summary, our main contributions are summarised as follows:
\begin{itemize}
\item We propose Dynamic Graph-Based Spatial-Temporal Attention (DG-STA) for skeleton-based hand gesture recognition. The structures and dynamics of hand skeletons can be learned automatically and more efficiently by our approach.
\item We propose spatial-temporal position embedding which encodes the identity and temporal order information of nodes to boost the performance of our model, and a spatial-temporal mask operation for efficient implementation of DG-STA.
\item We conduct comprehensive experiments to validate our approach on two standard benchmarks. The proposed DG-STA achieves the state-of-the-art performance.
\end{itemize} \section{Related Work}

In this section, we review recent work on self-attention  and recent developments in skeleton-based action recognition, which motivated our approach.

\textbf{Self-Attention.} 
The self-attention mechanism is widely used in computer vision and natural language processing tasks~\cite{chen2018factual,chen2018twitter,chen2018commerce,ZhangGMO19,lin2017structured,tan2018deep,verga2018simultaneously,tian2018cr}.
Vaswani~\etal~\cite{vaswani2017attention} proposed to apply the self-attention  module to model temporal and semantic relationships among words within a sentence for machine translation. Instead, we study applying the self-attention mechanism to learn spatial-temporal information contained in hand skeletons represented by graphs, which are largely different from sequences. Graph Attention Networks (GATs)~\cite{velivckovic2017graph} employed self-attention to learn node embeddings of graphs. By contrast, our approach is able to capture additional temporal information as well as node identities. 



\textbf{Skeleton-Based Hand Gesture Recognition.} Skeleton-based hand gesture is a well-studied but still challenging task. Traditional methods~\cite{ohn2013joint,de2016skeleton,de2017shrec,lu2003using} mainly focus on designing powerful hand feature descriptors. Smedt~\etal~\cite{de2016skeleton} proposed the Shape of Connected Joints descriptor to represent the hand shape of hand skeleton. Recent studies~\cite{hou2018spatial,chen2017motion,nunez2018convolutional} apply deep neural networks for this task and achieve significant performance improvement. Convolutional Neural Networks and Long Short-Term Memory are leveraged to learn the spatial and temporal features from the sequence of hand joints for hand gesture classification in~\cite{nunez2018convolutional}. One limitation of these learning-based methods is that they do not explicitly explore the structures and dynamics of human hands.

\textbf{Skeleton-Based Human Action Recognition.} Recent studies in skeleton-base human action recognition~\cite{zhao2018learning,tang2018quantized,peng2018jointly} started to incorporate structures and dynamics of human bodies by building skeleton graphs~\cite{yan2018spatial,si2018skeleton,zhao2019semantic}. This idea is first introduced by~\cite{dlid2017graph} which employs Graph CNNs. Recently, Yan~\etal~\cite{yan2018spatial} built a skeleton graph based on the natural structure of human body, and extract its representation by Graph Convolution Networks~\cite{kipf2016semi} for action recognition. Nevertheless, it is difficult to define an optimal skeleton graph which represents all action-specific structures and dynamics information. Instead, our method can automatically learn multiple action-specific graphs with the multi-head attention mechanism~\cite{vaswani2017attention}, which efficiently encode structures and dynamics of hand gestures.

%
 \section{Methodology}

The overview of our approach is shown in Figure~\ref{fig:overview}. First, a fully-connected skeleton graph is constructed from the input sequence of hand skeletons as described in Section~\ref{sec:init_g}. In Section~\ref{sec:sta_graph}, we devise DG-STA to learn the edge weights and node embeddings within the graph. The learned node features by DG-STA are then average-pooled into a vector which captures the structures and dynamics of the input skeleton graph. We use it for hand gesture classification.
Section~\ref{sec:pos_embedding} presents the spatial-temporal position embedding which is combined with node features to incorporate node identity and temporal order information contained in the hand skeletons. Moreover, a spatial-temporal mask operation is introduced in Section~\ref{sec:efficient_imp} which implements our proposed DG-STA more efficiently.     

\subsection{Skeleton Graph Initialization}
\label{sec:init_g}
Given a video of  frames,  hand joints are extracted from each frame to represent the hand skeleton. Then a fully-connected skeleton graph  is constructed from this sequence of hand skeletons. Let  denote the node set where  represents the -th hand joint at the time step . The node features are represented by , where  indicates the feature vector of the node . They are extracted from the 3D coordinates of nodes. Note that each node is connected with all other nodes including itself. For clarity, we define three types of edges on the edge set  as follows. 
\begin{itemize}
\item A spatial edge  connects two different nodes at the same time step.
\item A temporal edge  connects two nodes at different time steps.
\item A self-connected edge  connects the node with itself. 
\end{itemize}





\subsection{Dynamic Graph Construction via Spatial-Temporal Attention}
\label{sec:sta_graph}



The proposed  DG-STA consists of the spatial attention model  and temporal attention model  which are employed to extract spatial and temporal information from the hand skeleton graph respectively. Both  and  are based on multi-head attention~\cite{vaswani2017attention}.  first takes the initial node features  as the input and updates them to encode spatial information. The updated node features are then fed to  to further learn temporal information. Finally, the results are average-pooled to a vector which is used as the feature representation of the skeleton graph for classification.


Specifically, given the input feature  of the node  in the skeleton graph, the -th spatial attention head first applies three fully-connected layers to map  into the key, query and value vectors respectively, which are formulated as:

where ,  and  represent the key, query and value vectors of the node; ,  and  are the corresponding weight matrices of the three fully-connected layers of the -th spatial attention head.

The spatial attention head computes the weights of the spatial and self-connected edges in two steps. First, it calculates the ``scaled dot-product''~\cite{vaswani2017attention} between the query vectors and key vectors of the nodes within the same time step. Then it normalizes the results by a Softmax function. These two steps are formulated as: 

where  is the dimension of the key, query and value vectors;  is the scaled dot-product of the node  and ;  represents the inner product operation;  is the attention weight between the node  and , which measures the importance of information from node  to node . Meanwhile, the weights for all temporal edges are set to 0 in order to block the information passing in the temporal domain. As a result, each spatial attention head produces a weighted skeleton graph which represents a specific type of spatial structure of the hand. 

The attention head calculates the spatial attention feature of the node  as the weighted sum of the value vectors within the same time step, which is defined as: 

where  denotes the spatial attention feature of the node . Intuitively, the computation mechanism of the spatial attention features is essentially the process that each node in the graph sends some information to the others within the same time step and then aggregates the received information based on the learned edge weights.

The spatial attention model  finally concatenates the spatial attention features learned by all spatial attention heads into  which is employed as the spatial feature of the node :

where  is the number of spatial attention heads. 
The obtained node features encode multiple types of structural information represented by the weighted skeleton graphs which are learned by different spatial attention heads. 

The temporal attention model  takes the output node features from the spatial attention model as the input, and then applies the above multi-head attention mechanism in the temporal domain. The node feature which is the output from the temporal attention model encodes both spatial and temporal information carried by the input sequence of the hand skeletons. We average-pool these node features to a vector as the feature representation of the input sequence for hand gesture recognition.


































\subsection{Spatial-Temporal Position Embedding}
\label{sec:pos_embedding}



The original node features  extracted from the coordinates of the input hand skeletons do not contain spatial identity information describing which hand joint a node corresponds to, and temporal information indicating which time step a node is at. To incorporate these messages, we propose the spatial-temporal position embedding.

Specifically, our spatial-temporal position embedding is made up of  the spatial position embedding and the temporal position embedding. The spatial one consists of  vectors and each  represents a hand joint. Meanwhile, the temporal one is composed of  distinct vectors and each of them corresponds to a node in the hand skeleton graph. The feature vector of a specific node is added with the corresponding spatial and temporal position embedding vectors before fed into the DG-STA. Therefore, we have: 

where  denotes the final output feature of node ,  is the spatial position embedding of the -th hand joint, and  denotes the temporal position embedding of the -th hand joint at -th time step. These embeddings are with the same dimension as , and their values are set using the sine and cosine functions of different frequencies following~\cite{vaswani2017attention}.

\subsection{Efficient Implementation}
\label{sec:efficient_imp}

 It is not straightforward to implement the proposed DG-STA because the input data have to be arranged in a complex format. However, we find that the computation of attention weights and features without domain constraints is straightforward, which can be implemented efficiently using matrix multiplication operations. Therefore, we propose a novel scheme to facilitate the implementation of the DG-STA. The main idea is to first compute the matrix of the scaled dot-products among all nodes and then apply the proposed spatial-temporal mask operation to the matrix in order to let the model focus on the spatial or temporal domain. 
 
\begin{figure}[h]
\includegraphics[width=1\textwidth]{fig2.pdf}
\caption{Illustration of the proposed spatial and temporal mask operations.}
\label{fig:mask_op}
\end{figure}
 
An illustration of our mask operation is shown in Figure~\ref{fig:mask_op}. For a specific attention head, we compute a query matrix  where each row represents the query vector of each node, and a key matrix  where each row corresponds to the key vector of each node. The matrix of the scaled dot-products  (\ie, the edge weights before normalization) can be obtained by:

where  is the matrix multiplication, and  denotes the matrix transpose operation. Then the proposed spatial mask operation sets the value of each element in  which represents the temporal edge to  (\ie, a number close to negative infinity) and keeps the values of other elements unchanged. Therefore, the resulting matrix after the spatial mask operation   is calculated:

where  denotes the element-wise dot operation,  is the spatial mask where the elements are 1 if they represent the spatial or self-connect edges and 0 otherwise. We set  to  in our implementation. The Softmax activation  essentially normalizes weights across spatial edges, because the exponential value of the  is close to 0. As a result, all weights of the temporal edges in  are set to 0. Equations~(\ref{eq:scale_dot}) and (\ref{eq:mask}) efficiently implement the calculation of edge weights in the spatial domain formulated in Equation~(\ref{eq:s_att}). Moreover, the matrix  can be directly employed to implement the computation of node features represented by Equation~(\ref{eq:s_value}) by performing the matrix multiplication operation with the matrix of value vectors. 

We define the temporal mask operation following the same way as Equation~(\ref{eq:mask}). The difference is that we use the temporal mask  instead of  to compute the matrix after the temporal mask operation  . The elements of  are 1 if they represent the temporal or self-connect edges and 0 otherwise. With the help of the proposed spatial-temporal mask operation, we experimentally find that the computation time is reduced by 99\%.
%
 \section{Experiments}
In this section, we first describe our network structure in Section~\ref{sec:arch}. In Section~\ref{sec:settings}, we introduce the datasets and settings employed in the experiments. Then we conduct ablation studies in Section~\ref{sec:abl_study} to evaluate the effectiveness of each component proposed in our method. Finally, we report our results and comparisons with the state of the art in Section~\ref{sec:comparison}.

\subsection{Implementation Details}
\label{sec:arch}

Our network structure is shown in Figure \ref{fig:arch}. We set the head number of the spatial and temporal attention models to 8. The dimension  of the query, key and value vector is set to 32. Layer Normalization~\cite{ln} is utilized to normalize the intermediate outputs of our network. The input 3D coordinate of a hand joint is projected into an initial node feature of 128 dimension. It is then added with the corresponding spatial position embedding and fed into the spatial attention model, which produces a node feature of 256 dimension. This node feature is projected into a vector of 128 dimension which is added with the corresponding temporal position embedding. The temporal attention model takes it as the input and generates the final node feature. Finally, we average-pool the features of all nodes into a vector and feed it into a fully-connected layer for classification.

\begin{figure}[h]
\includegraphics[width=1\textwidth]{fig3.pdf}
\caption{The network architecture of the proposed DG-STA.}
\label{fig:arch}
\end{figure}

\subsection{Datasets and Settings}
\label{sec:settings}

We evaluate our method on the DHG-14/28 Dataset~\cite{de2016skeleton} and the SHREC'17 Track Dataset~\cite{de2017shrec}. Both datasets contain 2800 video sequences of 14 hand gestures which are performed in two configurations: using one single finger or the whole hand. The videos of the two datasets are captured by the Intel Realsense camera. The 3D coordinates of 22 hand joints in real-world space are provided per frame for network training and evaluation.

\textbf{Network Training.}
The proposed DG-STA is implemented based on the PyTorch platform. The Adam~\cite{kingma2014adam} optimizer with a learning rate of 0.001 is employed to train our model. The batch size is set to 32 and the dropout rate~\cite{srivastava2014dropout} is set to 0.2. We uniformly sample 8 frames from each video as the input. For fair comparison, we perform data augmentation by applying the same operations as proposed in~\cite{de2017shrec,nunez2018convolutional} including scaling, shifting, time interpolation and adding noise. We also subtract every skeleton sequence by the palm position of the first frame following Smedt~\etal~\cite{de2017shrec} for alignment.

\textbf{Evaluation Protocols.}
On the DHG-14/28 Dataset, models are evaluated by using the \emph{leave-one-subject-out cross-validation} strategy~\cite{de2016skeleton}. Specifically, we perform one experiment for each subject in this dataset. In each experiment, one subject is selected for testing and the remaining 19 subjects are used for training. The average accuracy of 14 gestures (without the single-finger configuration) or 28 gestures (with the single-finger configuration) over the 20 cross-validation folds are reported. For the SHREC'17 Track Dataset, we use the same data split as provided by~\cite{de2017shrec} and report the accuracy of both 14 and 28 gestures.

\subsection{Ablation Study}
\label{sec:abl_study}
Our proposed approach consists of three major components, including the fully-connected skeleton graph structure (FSG), the spatial-temporal attention model (STA) and the spatial-temporal position embedding (STE). We validate the effectiveness of these components in this section. The results are shown in Table~\ref{tbl:abl_std}.

\begin{table}[h]
\begin{center}
\begin{tabular}{|c|c|c|c|c|}
\hline
Setting & FSG+STA & FSG+GAT+STE & SSG+STA+STE & \textbf{DG-STA}\\
\hline\hline
14 Gestures~(D) & 84.3 & 90.8 & 89.8 & \textbf{91.9}\\
28 Gestures~(D) & 77.3 & 87.8 & 86.6 & \textbf{88.0}\\
\hline\hline
14 Gestures~(S) & 88.9 & 92.7 & 91.5 & \textbf{94.4}\\
28 Gestures~(S) & 80.1 & 86.2 & 87.7 & \textbf{90.7}\\
\hline
\end{tabular}
\end{center}
\caption{Ablation study of accuracy (\%) on the DHG-14/28 Dataset (D) and SHREC'17 Track Dataset (S). Our full model (DG-STA) achieves the best performance.}
\label{tbl:abl_std}
\end{table}

\textbf{Evaluation of Fully-Connected Graph Structure.} We compare the proposed FSG with the sparse skeleton graph structure (SSG) introduced by Yan~\etal~\cite{yan2018spatial}, where spatial edges are defined based on the natural connections of the hand joints and temporal edges connect the same joints between consecutive frames. We can see that our model significantly outperforms the one trained on SSG. This is because SSG may be sub-optimal for some hand gestures, while FSG has little constrains on the model so that it is able to learn action-specific graph structures.

\textbf{Evaluation of Spatial-Temporal Attention.} The proposed STA downgrades to Graph Attention (GAT)~\cite{velivckovic2017graph} if only one attention model is applied to the whole graph without distinguishing the spatial and temporal domains. We implement GAT by replacing the spatial and temporal attention models in our network with one attention model, and train it under the same setting of our model. We can observe that the STA-based model achieves better performance than the GAT-based model, which demonstrates the effectiveness of STA.

\textbf{Evaluation of Spatial-Temporal Position Embedding.} We validate the effectiveness of the proposed STE by training a variant of our method where STE is removed. We can see that our model outperforms the model without STE, which demonstrates the importance of the identity and temporal order information encoded by STE.




\subsection{Comparison with Previous Methods}
\label{sec:comparison}

We compare our method with the state-of-the-art methods on the DHG-14/28 Dataset~\cite{de2016skeleton} and the SHREC'17 Track Dataset~\cite{de2017shrec}, respectively. The compared state-of-the-art methods include traditional hand-crafted feature approaches~\cite{chen2017motion,oreifej2013hon4d,devanne20153,ohn2013joint,de2017dynamic,de2016skeleton,caputo2018comparing,boulahia2017dynamic}, deep learning based approaches~\cite{nunez2018convolutional,hou2018spatial,de2017shrec} and a graph-based method~\cite{yan2018spatial}. The results are shown in Tables~\ref{tbl:dhg} and \ref{tbl:SH17}. Note that for ST-GCN~\cite{yan2018spatial}, we implement it following the \emph{distance partitioning} setting and use a three-layer ST-GCN with 128 channels for fair comparison. We collect the results of other baseline methods from~\cite{hou2018spatial}.









\begin{table}[h]
\begin{center}
\begin{tabular}{|c|c|c|}
\hline
Method & 14 Gestures & 28 Gestures\\
\hline\hline
SoCJ+HoHD+HoWR~\cite{de2016skeleton} & 83.1 & 80.0\\
Chen~\etal~\cite{chen2017motion} & 84.7 & 80.3\\
CNN+LSTM~\cite{nunez2018convolutional} & 85.6 & 81.1\\
Res-TCN~\cite{hou2018spatial} & 86.9 & 83.6\\
STA-Res-TCN~\cite{hou2018spatial} & 89.2 & 85.0\\
ST-GCN~\cite{yan2018spatial} & 91.2 & 87.1\\
\hline\hline
\textbf{DG-STA (Ours)} & \textbf{91.9} & \textbf{88.0}\\
\hline
\end{tabular}
\end{center}
\caption{Comparisons of accuracy (\%) on DHG-14/28 Dataset.}
\label{tbl:dhg}
\end{table}

\textbf{Results on DHG-14/28 Dataset}. From Table~\ref{tbl:dhg}, we can see that our method achieves the state-of-the-arts performance under both 14-gesture and 28-gesture setting. Moreover, both our method and ST-GCN~\cite{yan2018spatial} outperform other methods which do not explicitly exploit structures and dynamics of hands, which demonstrates that these messages are important for skeleton-based hand gesture recognition.

\begin{table}[h]
\begin{center}
\begin{tabular}{|c|c|c|}
\hline
Method & 14 Gestures & 28 Gestures\\
\hline\hline
Oreifej~\etal~\cite{oreifej2013hon4d} & 78.5 & 74.0\\
Devanne~\etal~\cite{devanne20153} & 79.4 & 62.0\\
Classify Sequence by Key Frames~\cite{de2017shrec} & 82.9 & 71.9\\
Ohn-Bar~\etal~\cite{ohn2013joint} & 83.9 & 76.5\\
SoCJ+Direction+Rotation~\cite{de2017dynamic} & 86.9 & 84.2\\
SoCJ+HoHD+HoWR~\cite{de2016skeleton} & 88.2 & 81.9\\
Caputo~\etal~\cite{caputo2018comparing} & 89.5 & -\\
Boulahia~\etal~\cite{boulahia2017dynamic} & 90.5 & 80.5\\
Res-TCN~\cite{hou2018spatial} & 91.1 & 87.3\\
STA-Res-TCN~\cite{hou2018spatial} & 93.6 & \textbf{90.7}\\
ST-GCN~\cite{yan2018spatial} & 92.7 & 87.7\\
\hline\hline
\textbf{DG-STA (Ours)} & \textbf{94.4} & \textbf{90.7}\\
\hline
\end{tabular}
\end{center}
\caption{Comparisons of accuracy (\%) on SHREC'17 Track Dataset.}
\label{tbl:SH17}
\end{table}

\textbf{Results on SHREC'17 Track Dataset}. Different from the DHG-14/28 Dataset where videos are cropped by human-labeled beginnings and ends of the gestures~\cite{de2016skeleton}, the SHREC'17 Track Dataset provides raw captured video sequences with noisy frames, and hence is more challenging. We can see that our method achieves the state-of-the-arts performance under the 14-gesture setting, and obtains comparable performance with STA-Res-TCN~\cite{hou2018spatial} under the 28-gesture setting. In addition, we can observe that our method and ST-GCN~\cite{yan2018spatial} outperform all other methods which do not explicitly exploit structures and dynamics of hands.

%
 \section{Conclusions}

In this paper, we proposed a graph-based spatial-temporal attention method for skeleton-based hand-gesture recognition. It utilizes two attention models in the spatial and temporal domains of the fully-connected hand skeleton graph to learn edge weights and extract spatial and temporal information for hand gesture recognition. Extensive experiments demonstrate
the effectiveness of our framework. Our proposed method provides a general framework that can be further used for other tasks aiming to learn spatial and temporal information from graph-based data, \eg, skeleton-based human action recognition.

\section{Acknowledgments}

This work was funded partly by ARO-MURI-68985NSMUR and NSF 1763523, 1747778, 1733843, 1703883 to Dimitris N. Metaxas. 
\begin{thebibliography}{42}
\providecommand{\natexlab}[1]{#1}
\providecommand{\url}[1]{\texttt{#1}}
\expandafter\ifx\csname urlstyle\endcsname\relax
  \providecommand{\doi}[1]{doi: #1}\else
  \providecommand{\doi}{doi: \begingroup \urlstyle{rm}\Url}\fi

\bibitem[Boulahia et~al.(2017)Boulahia, Anquetil, Multon, and
  Kulpa]{boulahia2017dynamic}
Said~Yacine Boulahia, Eric Anquetil, Franck Multon, and Richard Kulpa.
\newblock Dynamic hand gesture recognition based on {3D} pattern assembled
  trajectories.
\newblock In \emph{International Conference on Image Processing Theory, Tools
  and Applications (IPTA)}, pages 1--6, 2017.

\bibitem[Caputo et~al.(2018)Caputo, Prebianca, Carcangiu, Spano, and
  Giachetti]{caputo2018comparing}
Fabio~M Caputo, Pietro Prebianca, Alessandro Carcangiu, Lucio~D Spano, and
  Andrea Giachetti.
\newblock Comparing {3D} trajectories for simple mid-air gesture recognition.
\newblock \emph{Computers \& Graphics}, 73:\penalty0 17--25, 2018.

\bibitem[Chen et~al.(2018{\natexlab{a}})Chen, Chen, Guo, and
  Luo]{chen2018commerce}
Tianlang Chen, Yuxiao Chen, Han Guo, and Jiebo Luo.
\newblock When e-commerce meets social media: Identifying business on wechat
  moment using bilateral-attention lstm.
\newblock In \emph{Proceedings of the World Wide Web Conference (WWW)}, pages
  343--350, 2018{\natexlab{a}}.

\bibitem[Chen et~al.(2018{\natexlab{b}})Chen, Zhang, You, Fang, Wang, Jin, and
  Luo]{chen2018factual}
Tianlang Chen, Zhongping Zhang, Quanzeng You, Chen Fang, Zhaowen Wang, Hailin
  Jin, and Jiebo Luo.
\newblock ``factual''or``emotional'': Stylized image captioning with adaptive
  learning and attention.
\newblock In \emph{Proceedings of the European Conference on Computer Vision
  (ECCV)}, pages 519--535, 2018{\natexlab{b}}.

\bibitem[Chen et~al.(2017)Chen, Guo, Wang, and Zhang]{chen2017motion}
Xinghao Chen, Hengkai Guo, Guijin Wang, and Li~Zhang.
\newblock Motion feature augmented recurrent neural network for skeleton-based
  dynamic hand gesture recognition.
\newblock In \emph{Proceedings of the IEEE International Conference on Image
  Processing (ICIP)}, pages 2881--2885, 2017.

\bibitem[Chen et~al.(2018{\natexlab{c}})Chen, Yuan, You, and
  Luo]{chen2018twitter}
Yuxiao Chen, Jianbo Yuan, Quanzeng You, and Jiebo Luo.
\newblock Twitter sentiment analysis via bi-sense emoji embedding and
  attention-based lstm.
\newblock In \emph{Proceedings of the ACM Multimedia Conference on Multimedia
  Conference (MM)}, pages 117--125, 2018{\natexlab{c}}.

\bibitem[De~Smedt(2017)]{de2017dynamic}
Quentin De~Smedt.
\newblock \emph{Dynamic hand gesture recognition-From traditional handcrafted
  to recent deep learning approaches}.
\newblock PhD thesis, Universit{\'e} de Lille 1, Sciences et Technologies;
  CRIStAL UMR 9189, 2017.

\bibitem[De~Smedt et~al.(2016)De~Smedt, Wannous, and
  Vandeborre]{de2016skeleton}
Quentin De~Smedt, Hazem Wannous, and Jean-Philippe Vandeborre.
\newblock Skeleton-based dynamic hand gesture recognition.
\newblock In \emph{Proceedings of the IEEE Conference on Computer Vision and
  Pattern Recognition Workshops (CVPRW)}, pages 1--9, 2016.

\bibitem[De~Smedt et~al.(2017)De~Smedt, Wannous, Vandeborre, Guerry, Le~Saux,
  and Filliat]{de2017shrec}
Quentin De~Smedt, Hazem Wannous, Jean-Philippe Vandeborre, Joris Guerry,
  Bertrand Le~Saux, and David Filliat.
\newblock {SHREC'17 Track}: {3D} hand gesture recognition using a depth and
  skeletal dataset.
\newblock In \emph{Eurographics Workshop on 3D Object Retrieval}, 2017.

\bibitem[Devanne et~al.(2015)Devanne, Wannous, Berretti, Pala, Daoudi, and
  Del~Bimbo]{devanne20153}
Maxime Devanne, Hazem Wannous, Stefano Berretti, Pietro Pala, Mohamed Daoudi,
  and Alberto Del~Bimbo.
\newblock {3-D} human action recognition by shape analysis of motion
  trajectories on riemannian manifold.
\newblock \emph{IEEE Transactions on Cybernetics}, 45\penalty0 (7):\penalty0
  1340--1352, 2015.

\bibitem[Edwards and Xie(2017)]{dlid2017graph}
M.~Edwards and X.~Xie.
\newblock Graph-based {CNN} for human action recognition from {3D} pose.
\newblock In \emph{British Machine Vision Conference Workshop: Deep Learning on
  Irregular Domains}, pages 1.1--1.10, 2017.

\bibitem[Freeman and Roth(1995)]{freeman1995orientation}
William~T Freeman and Michal Roth.
\newblock Orientation histograms for hand gesture recognition.
\newblock In \emph{International Workshop on Automatic Face and Gesture
  Recognition}, volume~12, pages 296--301, 1995.

\bibitem[Hou et~al.(2018)Hou, Wang, Chen, Xue, Zhu, and Yang]{hou2018spatial}
Jingxuan Hou, Guijin Wang, Xinghao Chen, Jing-Hao Xue, Rui Zhu, and Huazhong
  Yang.
\newblock Spatial-temporal attention {Res-TCN} for skeleton-based dynamic hand
  gesture recognition.
\newblock In \emph{Proceedings of the European Conference on Computer Vision
  (ECCV)}, pages 273--286, 2018.

\bibitem[Kingma and Ba(2014)]{kingma2014adam}
Diederik~P Kingma and Jimmy Ba.
\newblock Adam: A method for stochastic optimization.
\newblock \emph{arXiv preprint arXiv:1412.6980}, 2014.

\bibitem[Kipf and Welling(2016)]{kipf2016semi}
Thomas~N Kipf and Max Welling.
\newblock Semi-supervised classification with graph convolutional networks.
\newblock \emph{arXiv preprint arXiv:1609.02907}, 2016.

\bibitem[Lei~Ba et~al.(2016)Lei~Ba, Kiros, and Hinton]{ln}
Jimmy Lei~Ba, Jamie~Ryan Kiros, and Geoffrey~E Hinton.
\newblock Layer normalization.
\newblock \emph{arXiv preprint arXiv:1607.06450}, 2016.

\bibitem[Lin et~al.(2017)Lin, Feng, Santos, Yu, Xiang, Zhou, and
  Bengio]{lin2017structured}
Zhouhan Lin, Minwei Feng, Cicero Nogueira~dos Santos, Mo~Yu, Bing Xiang, Bowen
  Zhou, and Yoshua Bengio.
\newblock A structured self-attentive sentence embedding.
\newblock \emph{arXiv preprint arXiv:1703.03130}, 2017.

\bibitem[Lu et~al.(2003)Lu, Metaxas, Samaras, and Oliensis]{lu2003using}
Shan Lu, Dimitris Metaxas, Dimitris Samaras, and John Oliensis.
\newblock Using multiple cues for hand tracking and model refinement.
\newblock In \emph{Proceedings of the IEEE Conference on Computer Vision and
  Pattern Recognition (CVPR)}, volume~2, pages 443--450, 2003.

\bibitem[Molchanov et~al.(2015)Molchanov, Gupta, Kim, and
  Kautz]{molchanov2015hand}
Pavlo Molchanov, Shalini Gupta, Kihwan Kim, and Jan Kautz.
\newblock Hand gesture recognition with {3D} convolutional neural networks.
\newblock In \emph{Proceedings of the IEEE Conference on Computer Vision and
  Pattern Recognition Workshops (CVPRW)}, pages 1--7, 2015.

\bibitem[Nikam and Ambekar(2016)]{nikam2016sign}
Ashish~S Nikam and Aarti~G Ambekar.
\newblock Sign language recognition using image based hand gesture recognition
  techniques.
\newblock In \emph{Proceedings of the International Conference on Green
  Engineering and Technologies (IC-GET)}, pages 1--5, 2016.

\bibitem[N{\'u}{\~n}ez et~al.(2018)N{\'u}{\~n}ez, Cabido, Pantrigo, Montemayor,
  and V{\'e}lez]{nunez2018convolutional}
Juan~C N{\'u}{\~n}ez, Raul Cabido, Juan~J Pantrigo, Antonio~S Montemayor, and
  Jos{\'e}~F V{\'e}lez.
\newblock Convolutional neural networks and long short-term memory for
  skeleton-based human activity and hand gesture recognition.
\newblock \emph{Pattern Recognition}, 76:\penalty0 80--94, 2018.

\bibitem[Oberweger and Lepetit(2017)]{oberweger2017deepprior}
Markus Oberweger and Vincent Lepetit.
\newblock Deepprior++: Improving fast and accurate {3D} hand pose estimation.
\newblock In \emph{Proceedings of the IEEE Conference on Computer Vision and
  Pattern Recognition (CVPR)}, pages 585--594, 2017.

\bibitem[Oberweger et~al.(2015{\natexlab{a}})Oberweger, Wohlhart, and
  Lepetit]{oberweger2015hands}
Markus Oberweger, Paul Wohlhart, and Vincent Lepetit.
\newblock Hands deep in deep learning for hand pose estimation.
\newblock \emph{arXiv preprint arXiv:1502.06807}, 2015{\natexlab{a}}.

\bibitem[Oberweger et~al.(2015{\natexlab{b}})Oberweger, Wohlhart, and
  Lepetit]{oberweger2015training}
Markus Oberweger, Paul Wohlhart, and Vincent Lepetit.
\newblock Training a feedback loop for hand pose estimation.
\newblock In \emph{Proceedings of the IEEE Conference on Computer Vision and
  Pattern Recognition (CVPR)}, pages 3316--3324, 2015{\natexlab{b}}.

\bibitem[Ohn-Bar and Trivedi(2013)]{ohn2013joint}
Eshed Ohn-Bar and Mohan Trivedi.
\newblock Joint angles similarities and {HOG2} for action recognition.
\newblock In \emph{Proceedings of the IEEE Conference on Computer Vision and
  Pattern Recognition Workshops (CVPRW)}, pages 465--470, 2013.

\bibitem[Oreifej and Liu(2013)]{oreifej2013hon4d}
Omar Oreifej and Zicheng Liu.
\newblock {HON4D}: Histogram of oriented {4D} normals for activity recognition
  from depth sequences.
\newblock In \emph{Proceedings of the IEEE Conference on Computer Vision and
  Pattern Recognition (CVPR)}, pages 716--723, 2013.

\bibitem[Peng et~al.(2018)Peng, Tang, Yang, Feris, and
  Metaxas]{peng2018jointly}
Xi~Peng, Zhiqiang Tang, Fei Yang, Rogerio~S Feris, and Dimitris Metaxas.
\newblock Jointly optimize data augmentation and network training: Adversarial
  data augmentation in human pose estimation.
\newblock In \emph{Proceedings of the IEEE Conference on Computer Vision and
  Pattern Recognition (CVPR)}, pages 2226--2234, 2018.

\bibitem[Rautaray and Agrawal(2015)]{rautaray2015vision}
Siddharth~S Rautaray and Anupam Agrawal.
\newblock Vision based hand gesture recognition for human computer interaction:
  a survey.
\newblock \emph{Artificial Intelligence Review}, 43\penalty0 (1):\penalty0
  1--54, 2015.

\bibitem[Si et~al.(2018)Si, Jing, Wang, Wang, and Tan]{si2018skeleton}
Chenyang Si, Ya~Jing, Wei Wang, Liang Wang, and Tieniu Tan.
\newblock Skeleton-based action recognition with spatial reasoning and temporal
  stack learning.
\newblock In \emph{Proceedings of the European Conference on Computer Vision
  (ECCV)}, pages 103--118, 2018.

\bibitem[Srivastava et~al.(2014)Srivastava, Hinton, Krizhevsky, Sutskever, and
  Salakhutdinov]{srivastava2014dropout}
Nitish Srivastava, Geoffrey Hinton, Alex Krizhevsky, Ilya Sutskever, and Ruslan
  Salakhutdinov.
\newblock Dropout: a simple way to prevent neural networks from overfitting.
\newblock \emph{The Journal of Machine Learning Research}, 15\penalty0
  (1):\penalty0 1929--1958, 2014.

\bibitem[Tan et~al.(2018)Tan, Wang, Xie, Chen, and Shi]{tan2018deep}
Zhixing Tan, Mingxuan Wang, Jun Xie, Yidong Chen, and Xiaodong Shi.
\newblock Deep semantic role labeling with self-attention.
\newblock In \emph{Proceedings of the AAAI Conference on Artificial
  Intelligence}, 2018.

\bibitem[Tang et~al.(2018)Tang, Peng, Geng, Wu, Zhang, and
  Metaxas]{tang2018quantized}
Zhiqiang Tang, Xi~Peng, Shijie Geng, Lingfei Wu, Shaoting Zhang, and Dimitris
  Metaxas.
\newblock Quantized densely connected {U-Nets} for efficient landmark
  localization.
\newblock In \emph{Proceedings of the European Conference on Computer Vision
  (ECCV)}, pages 339--354, 2018.

\bibitem[Tian et~al.(2018)Tian, Peng, Zhao, Zhang, and Metaxas]{tian2018cr}
Yu~Tian, Xi~Peng, Long Zhao, Shaoting Zhang, and Dimitris~N Metaxas.
\newblock {CR-GAN}: Learning complete representations for multi-view
  generation.
\newblock In \emph{Proceedings of the International Joint Conference on
  Artificial Intelligence (IJCAI)}, pages 942--948, 2018.

\bibitem[Vaswani et~al.(2017)Vaswani, Shazeer, Parmar, Uszkoreit, Jones, Gomez,
  Kaiser, and Polosukhin]{vaswani2017attention}
Ashish Vaswani, Noam Shazeer, Niki Parmar, Jakob Uszkoreit, Llion Jones,
  Aidan~N Gomez, {\L}ukasz Kaiser, and Illia Polosukhin.
\newblock Attention is all you need.
\newblock In \emph{Advances in Neural Information Processing Systems (NIPS)},
  pages 5998--6008, 2017.

\bibitem[Veli{\v{c}}kovi{\'c} et~al.(2017)Veli{\v{c}}kovi{\'c}, Cucurull,
  Casanova, Romero, Lio, and Bengio]{velivckovic2017graph}
Petar Veli{\v{c}}kovi{\'c}, Guillem Cucurull, Arantxa Casanova, Adriana Romero,
  Pietro Lio, and Yoshua Bengio.
\newblock Graph attention networks.
\newblock \emph{arXiv preprint arXiv:1710.10903}, 2017.

\bibitem[Verga et~al.(2018)Verga, Strubell, and
  McCallum]{verga2018simultaneously}
Patrick Verga, Emma Strubell, and Andrew McCallum.
\newblock Simultaneously self-attending to all mentions for full-abstract
  biological relation extraction.
\newblock \emph{arXiv preprint arXiv:1802.10569}, 2018.

\bibitem[Wachs et~al.(2011)Wachs, K{\"o}lsch, Stern, and Edan]{wachs2011vision}
Juan~Pablo Wachs, Mathias K{\"o}lsch, Helman Stern, and Yael Edan.
\newblock Vision-based hand-gesture applications.
\newblock \emph{Communications of the ACM}, 54\penalty0 (2):\penalty0 60--71,
  2011.

\bibitem[Wang et~al.(2015)Wang, Liu, and Chan]{wang2015superpixel}
Chong Wang, Zhong Liu, and Shing-Chow Chan.
\newblock Superpixel-based hand gesture recognition with kinect depth camera.
\newblock \emph{IEEE Transactions on Multimedia}, 17\penalty0 (1):\penalty0
  29--39, 2015.

\bibitem[Yan et~al.(2018)Yan, Xiong, and Lin]{yan2018spatial}
Sijie Yan, Yuanjun Xiong, and Dahua Lin.
\newblock Spatial temporal graph convolutional networks for skeleton-based
  action recognition.
\newblock In \emph{Proceedings of the AAAI Conference on Artificial
  Intelligence}, 2018.

\bibitem[Zhang et~al.(2019)Zhang, Goodfellow, Metaxas, and Odena]{ZhangGMO19}
Han Zhang, Ian~J. Goodfellow, Dimitris~N. Metaxas, and Augustus Odena.
\newblock Self-attention generative adversarial networks.
\newblock In \emph{Proceedings of the International Conference on Machine
  Learning (ICML)}, pages 7354--7363, 2019.

\bibitem[Zhao et~al.(2018)Zhao, Peng, Tian, Kapadia, and
  Metaxas]{zhao2018learning}
Long Zhao, Xi~Peng, Yu~Tian, Mubbasir Kapadia, and Dimitris Metaxas.
\newblock Learning to forecast and refine residual motion for image-to-video
  generation.
\newblock In \emph{Proceedings of the European Conference on Computer Vision
  (ECCV)}, pages 387--403, 2018.

\bibitem[Zhao et~al.(2019)Zhao, Peng, Tian, Kapadia, and
  Metaxas]{zhao2019semantic}
Long Zhao, Xi~Peng, Yu~Tian, Mubbasir Kapadia, and Dimitris~N Metaxas.
\newblock Semantic graph convolutional networks for {3D} human pose regression.
\newblock In \emph{Proceedings of the IEEE Conference on Computer Vision and
  Pattern Recognition (CVPR)}, pages 3425--3435, 2019.

\end{thebibliography}
 \end{document}
