

\begin{theorem}\label{thm:pc}
  For mixed/packing covering, there are -approximation algorithms
  running
  
  \noindent 
  (i) in time ,

  \noindent 
  (ii) in parallel time ,
  doing work .
\end{theorem}











\begin{algorithm}[t]
  \caption{Generic approximation algorithm for mixed packing/covering \LPs\label{alg:pc}}
  \begin{algorithmic}[1]
    \Function{Packing-Covering}{matrices , ; initial solution , }
    \State\label{pc:defn}
    {\renewcommand{\arraystretch}{1.3}
      \begin{tabular}[t]{@{}lr@{\,~}l@{~~~~~}l@{~\,}r@{\,~}l}
        Define:&  & \\
        &  & 
        &  & & ~~if~, else \\ 
        &  &  
        &  & & \\
        &  & 
        &  & & .
        \i) returns a -approximate solution
  (i.e.,  such that  and ),
  \iii) increments  (lines~\ref{op:a}--\ref{op:a:end}) at most  times.
\end{lemma}
See the appendix for a proof,
which follows the proofs of Lemmas 1--5 of~\cite{Young01Sequential}. 
After initialization, Alg.~\ref{alg:pc} simply 
repeats one of two operations: {\bf (a)}
incrementing the current solution  by some vector ,
or {\bf (b)} scaling  by .
In each iteration, it can do either operation whose precondition is met,
and when incrementing  there are many valid ways to choose .

\subsection{Proof of part (i), sequential algorithm}
Alg.~\ref{alg:pc:imp}, which we use to prove part (i) of Thm.~\ref{thm:pc},
repeats these two operations in a particular way.
To reduce the run time to nearly linear,
instead of computing , ,  and  exactly as it proceeds,
Alg.~\ref{alg:pc:imp} maintains estimates:
, , , and .
To prove correctness,
we show that the estimates suffice to ensure
a correct implementation of Alg.~\ref{alg:pc},
which is correct by Lemma~\ref{lemma:pc} (i).

\begin{algorithm}[t]
  \caption{Sequential implementation 
    of Alg.~\ref{alg:pc}\label{alg:pc:imp}
    for mixed packing/covering \LPs}
  \begin{algorithmic}[1]
    \Function{Sequential-Packing-Covering}{}
    \State Initialize  for ,
    ,
    and .
    \State\label{pc:imp:estimate1} 
    Maintain vectors , , , and , to satisfy invariant
    \vspace*{-1.5ex}
    0pt]
          \est C_i &\in& (C_i \,x - 1, C_i\, x]
          &
          \est c_i &=& {(1-\eps)}^{\est C_i} \text{ if } \est C_i \le U, \text{ else } \est c_i=0.
        \end{array}
      \end{array}
      \vspace*{-0.39in}
    3pt]
    \end{tabular}
    \State For , 
    let  (and, implicitly,  for ),
    \State~~choosing  
    such that .
    \State For , let .
    \State For , update  and .
    For , update  and .
    \State\label{pc:par:update}
    For , update ,  , and .
    \State If  then {\bf return} . \Comment{finished}
    \EndBlock
    \EndWhile 
    \Block {\bf operation (b): scale }\label{op:bp}
    \Comment{\emph{assertion:}
      {}}
    \State Let .
    \EndBlock
    \EndRepeat
    \EndFunction 
  \end{algorithmic}
\end{algorithm}

\subsection{Proof of part (ii), parallel algorithm}
Next we prove part (ii) of Thm.~\ref{thm:pc}, using Alg.~\ref{alg:pc:par}.
By careful inspection, Alg.~\ref{alg:pc:par}
just repeats the two operations of Alg.~\ref{alg:pc}
(increment  or scale ),
so is correct by Lemma~\ref{lemma:pc} (i).
To finish, we detail how to implement the steps
so a careful accounting yields the desired time and work bounds.

Call each iteration of the repeat loop a {\em phase}.
Each phase scales , so by Lemma~\ref{lemma:pc} (ii), there are  phases.
By inspection, , so .  
Within any given phase, 
for the \emph{first} increment,
compute all quantities directly
in  time and  total work.
In each subsequent increment within the phase,
update all quantities incrementally,
in time  and doing total work
linear in the sizes of the sets

and 
of \emph{active edges}.

(For example:
update each  by noting that the increment
increases  by ;
update  by noting that the increment increases 
it by .
Update  by noting that  only increases within the phase,
so  only shrinks, so it suffices to delete a given  from 
in the first increment when  exceeds .)

\paragraph{Bounding the work and time.}
In each increment, if a given  is in , then the increment increases .
When that happens the parameter  is at least 
(using  and )
so  increases by at least a factor of .
The value of  is initially at least 
and finally .
It follows that  is in the set  during at most  increments.
Thus, for any given non-zero , the pair  is in 
in at most  increments.
Likewise, for any given non-zero , the pair  is in 
in at most  increments.
Hence, the total work for Alg.~\ref{alg:pc:par}
is , as desired.

To bound the total time, note that, within each of the  phases,
some  remains in  throughout the phase.
As noted above, no  is in  for more than  increments.
Hence, each phase has  increments.
To finish, recall that each increment takes  time.
This concludes the proof of Thm.~\ref{thm:pc}. \hfill \qed

