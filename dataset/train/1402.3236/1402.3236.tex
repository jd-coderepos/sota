\pdfoutput=1
\documentclass[a4paper,11pt]{article}
\usepackage{amsthm,amsfonts,amssymb,amsmath,wasysym,stmaryrd,textcomp}
\usepackage[section]{placeins}
\usepackage[pdftex]{graphicx}
  \DeclareGraphicsExtensions{.pdf}
\usepackage{color}
\usepackage{booktabs}
\newtheorem{theorem}{Theorem}[section]
\newtheorem{lemma}[theorem]{Lemma}
\newtheorem{proposition}[theorem]{Proposition}
\newtheorem{corollary}[theorem]{Corollary}
\newtheorem{definition}[theorem]{Definition}
\newtheorem{example}[theorem]{Example}
\newtheorem{remark}[theorem]{Remark}
\newtheorem{notation}{Notation}
\setcounter{topnumber}{2}
\setcounter{bottomnumber}{2}
\setcounter{totalnumber}{4}
\renewcommand{\topfraction}{0.85}
\renewcommand{\bottomfraction}{0.85}
\renewcommand{\textfraction}{0.15}
\renewcommand{\floatpagefraction}{0.7}
\begin{document}
\title{A graph-theoretical approach for the computation of connected iso-surfaces based on
volumetric data}
\author{Abdulaziz Ali and Dieter Bothe\\
\mbox{}\\
Mathematical Modeling and Analysis\\
Center of Smart Interfaces\\
TU Darmstadt\\
Alarich-Weiss-Str.\ 10\\
64287 Darmstadt\\
Germany}
\maketitle
\newpage
\tableofcontents
\newpage
\begin{abstract}
The existing combinatorial methods for iso-surface computation are efficient for pure visualization purposes,
but it is known that the resulting iso-surfaces can have holes, and topological problems like missing or wrong connectivity
can appear. To avoid such problems, we introduce a graph-theoretical method for the computation of iso-surfaces on cuboid
meshes in . The method for the generation of iso-surfaces employs labeled cuboid graphs
 such that  is the set of vertices of a cuboid ,  is the set of
edges of  and . The nodes of  are weighted by the values of 
which represents the volumetric information, e.g. from a Volume of Fluid method. Using a given iso-level
, we first obtain all iso-points, i.e. points where the value  is attained by the edge-interpolated
-field. The iso-surface is then built from iso-elements which are composed of triangles and are
such that their polygonal boundary has only iso-points as vertices. All vertices lie on the faces of a single mesh cell.

We give a proof that the generated iso-surface is connected up to the boundary of the domain and it can be
decomposed into different oriented components. Two different components may have discrete points or line segments
in common. The graph-theoretical method for the computation of iso-surfaces developed in this paper enables to
recover local information of the iso-surface that can be used e.g. to compute discrete mean curvature and to
solve surface PDEs. Concerning the computational effort, the resulting
algorithm is as efficient as existing combinatorial methods.
\end{abstract}
\noindent{\bf Keywords: } connected iso-surface, iso-surface topology, iso-path, surface pseudo-normal.
\newpage
\section{Introduction}
An iso-surface is a level set of a continuous function whose domain is, in the considered case, .
Iso-surfaces are for example used to visualize scalar volume data processed in medicine, computational
fluid dynamics (CFD), geophysics, and chemistry. In medical imaging, by applying X-ray computed tomography
(CT) one obtains volume data which can be used to detect bones, tumors and cancer. In two-fluid systems,
the Volume of Fluid (VOF) method provides VOF-data which implicitly describes the interface between the fluids.

The Marching Cubes method~\cite{Lorensen87marchingcubes:} is a well known method for volume
visualization. It is a combinatorial and not a graph-theoretical method, being based on the tabulation
of 256 different configurations. This set of possible configurations is not complete and, hence,
resulting iso-surfaces can have holes. More generally, it is known that commonly employed iso-surface
algorithms can lead to surfaces with wrong connectivities and holes; see~\cite{Etiene:2012:TVI:2197070.2197097}
and further references therein. While this may not be an issue in a pure visualization context, it is a severe
problem if surface transport equations are to be solved, like surfactant transport on a fluid
surface~\cite{Alke_and_Bothe}. Other works like~\cite{Nielson:1991:ADR:949607.949621}
and~\cite{Chernyaev95marchingcubes} resolve the
ambiguity in Marching Cubes. In~\cite{conf/dgci/Lachaud96}, a topological approach for the
computation of iso-surfaces is given, some geometrical properties of the iso-surface are derived and an
algorithm for the computation of iso-surfaces is given. But there are still configurations which are not
covered by~\cite{Nielson:1991:ADR:949607.949621} as well as by~\cite{conf/dgci/Lachaud96}. For
example configurations in which an iso-surface only touches one or more vertices of a cuboid are not investigated
in~\cite{Chernyaev95marchingcubes},~\cite{conf/dgci/Lachaud96} and~\cite{Nielson:1991:ADR:949607.949621}.

The present work, we introduce a novel graph-theoretical method for the generation of iso-surfaces. We will show
that the generated iso-surface is connected up to the boundary of the domain and it can be decomposed into
different oriented components. Two different components may have discrete points or line segments as an
intersection. If two different components have common points or line segments, then each single component
can be identified and computed as if they are disjoint. This decomposition
of iso-surfaces into oriented components can be used e.g. to compute discrete mean curvature and to solve
surface PDEs for instance by means of a Finite Area method.

Our graph-theoretical approach employs labeled cuboid graphs. A labeled cuboid graph is denoted by
 such that  is the set of vertices of a cuboid  , 
is the set of edges of  and . The nodes of  are weighted
by the values of  and the weights lie in . Using a given iso-level ,
we interpolate on each edge of the graph to get all points, where the value  is attained.
We call such a point on an edge of , where the interpolated values equals , an iso-point if one
of the edge end points has a label greater than  and the other one less than or equal to . From
this we get a piece of iso-surface whose boundary is a polygon such that its vertices
are iso-points and each of the edges lies on a face of . We call such an iso-surface piece an iso-element, its
boundary an iso-path and each of the edges of the iso-path an iso-line. To compute the iso-element we use
a center point  which is the arithmetic mean of the corresponding iso-points.
Then the iso-element is defined as the union of all triangles spanned by an iso-line edge and the vertex .
Each of the sketches  to  of Figure~\ref{image_1} shows an iso-element of a labeled
cuboid graph. Sketch  has a triangular iso-element. Sketches  to  of
Figure~\ref{image_1} have iso-elements such that the iso-paths may not lie on a common plane.

The iso-elements described above result from iso-paths which run on the faces of a single graph .
Besides this, iso-elements can also occur from two neighboring graphs in which case they lie on their common
face. This special case is of fundamental importance for connectivity of the iso-surface.

\begin{notation}
Let  be a labeled cuboid graph and  be an iso-level. Then we use the symbols
 in sketches which illustrate  as well as subgraphs
of it. The symbols are used to characterize the labels of the nodes of  and for interpolated values at points
that lie on edges of . The symbol  correspond to labels less than . The symbol 
correspond to labels greater than  and the symbol  to labels equal to . In addition, the symbol
 correspond to labels less than or equal to .
\label{int:not-1}
\end{notation}
\begin{figure}[!ht]
\includegraphics[width=0.8\linewidth]{image/image_1}
\caption{The sketches  to  show iso-elements of labeled cuboid graphs. Each iso-element
is determined by its polygonal boundary, an iso-path.}
\label{image_1}
\end{figure}
\FloatBarrier

The iso-surface construction algorithms known so far do not yield in a simple way the local
information of the iso-surface such as neighborhood relations at common points or
edges of two or more iso-elements. Such topological information is required
to compute discrete mean curvature and to solve PDEs on an iso-surface. Hence, the development of a
new iso-surface computation method which provides a decomposition of the iso-surface into connected
components is required. We achieve this by introducing a purely graph-theoretical method for the
computation and decomposition of iso-surfaces. The resulting algorithm is very efficient.

The sketches in Figure~\ref{image_1} show labeled cuboid graphs having one iso-path.
But a labeled cuboid graph can have more than one iso-path and hence more than
one iso-element. The sketches in Figure~\ref{image_2} demonstrate that the number of
iso-paths depends on the labels of the graph. These are not meaningless, pathological cases, but they
occur naturally in dynamic processes with topological changes such as the crown splash of an impacting
droplet considered as an application in Section~5. Therefore, one of the main tasks is to introduce a
classification of labeled
cuboid graphs according to their subgraphs such that the number of iso-paths in a labeled cuboid graph
can be identified. The identification of the number of iso-paths and further analysis of the subgraphs of
a labeled cuboid graph is the main part of the present work.
\begin{figure}[!ht]
\includegraphics[width=0.8\linewidth]{image/image_2}
\caption{Sketches  to  show labeled cuboid graphs with one to four distinct iso-paths.}
\label{image_2}
\end{figure}

To understand the topology of an iso-surface, we need to investigate pairs of labeled cuboid graphs
 and  for a common iso-level ,
where the underlying cuboids  and , have a common face. Such graphs will be called
face-neighbored. If both labeled cuboid graphs have iso-paths with a common iso-line lying on the
common face of the cuboids, then they have iso-elements which have a common edge
(see Figure~\ref{image_3}). In case both end points of the common edge
are vertices of the common face then there is a possibility that more than two iso-paths can meet at
the common edge. Additionally, if the common iso-line of two iso-paths is a diagonal of a cuboid face,
then in each cuboid we can have a maximum of two iso-paths passing through the common edge; hence, it
can be a common edge for four different iso-paths. An edge of a cuboid can be a common edge for four
distinct cuboids and in each cuboid we can have a maximum of two iso-paths passing
through the common edge. Hence, there can be up to eight distinct iso-paths passing through the common edge.
All these cases have to be treated for a rigorous iso-surface computation. The sketches in Figure~\ref{image_3}
demonstrate two iso-paths of a pair of labeled cuboid graphs with a common edge.
\begin{figure}[!ht]
\includegraphics[width=0.6\linewidth]{image/image_3}
\caption{Each pair of sketches  and  shows the iso-paths of
face-neighbored labeled cuboid graphs. The common face edges in both sketches is bold-framed.
Each pair of iso-paths in both sketches above has a common iso-line which is drawn in bold.}
\label{image_3}
\end{figure}
\FloatBarrier

An iso-point  of an iso-path can be a common point of  or  iso-paths.
Let  for  lie on the -th iso-path  such that  is
an iso-line of . Let  be the center of . Then we construct a
piece of iso-surface region  using all iso-points  and all iso-path centers
 and . Such a region which will be computed in Section 8 is required for computing
discrete mean curvature at the iso-point  (see e.g. \cite{Meyer02Vismath}).

The iso-surface computation algorithm needs a partition  of a closed polygonal domain
 into cuboids such that two cuboids have either a common face or
a common vertex or a common edge or they are disjoint. Here, polygonal domain means domain with
polygonal boundary. Then, to obtain labeled cuboid graphs, the vertices of each of the cuboids in
the partition  of  are labeled with real numbers in . The labeled
cuboid graphs will be divided into two classes, depending on whether they have a single iso-path
or at least two iso-paths (cf.,Figure~\ref{image_2}).
Labeled cuboid graphs of the first class, having only a single iso-path, are called irreducible;
members of the other class are called reducible. This classification of labeled cuboid graphs is
fundamental for our algorithm.

The algorithm for iso-surface construction uses operations on labeled cuboid graphs such
that a reducible labeled cuboid graph  will be transformed to an irreducible one
. Each step in the transformation of  from reducibility to irreducibility
gives an iso-element of . An irreducible labeled cuboid graph finally has a single iso-element.
In addition, we can have an iso-path on single faces of a reducible labeled cuboid graph, on
which at least two iso-lines lie.

We get the iso-surfaces by collecting all iso-elements of each of the labeled cuboid graphs in
. The algorithm provides rich local information on the iso-surface like the number of
iso-elements with a common edge, or iso-elements with at least one edge of it being an edge of the
underlying cuboid net, or at least one edge of the iso-element being a face diagonal
of a cuboid. These local information is used for a simple identification of the connected components
of the iso-surface.

Another main result of the present work is that the number  of iso-elements with a
common iso-line is an element of . Hence, the number of connected components of the
iso-surface with a common iso-line may be an element of . In particular, we will show
that triple lines at which three surfaces meet do not occur.

The same principle ideas works to discretization using tetrahedra \cite{second-article-ali-bothe}.
It is very likely that the same principle ideas can be adapted to other discretizations (more
general polyhedral cells) and also to space dimensions different from 3.

This algorithm for computation of iso-surfaces has computational complexity of order , where  is
the number of cuboids in . Iso-surface computation algorithms based on
combinatorial approaches have as well a complexity of , but the algorithms do not give the full
topological information of the iso-surfaces. More severe, the iso-surface will in general be not complete.\\

\vspace{-0.3cm}
\noindent The paper is organized as follows:\\
\vspace{-0.3cm}

\noindent In Section 2 we introduce the main definitions and notations. We define
the concept of a labeled cuboid graph and its subgraphs. In addition, graph operations on
labeled cuboid graphs and on subgraphs are defined. Furthermore, we define iso-paths which
are the boundary of iso-elements that will be computed using labeled cuboid graphs. In Section 3,
equivalence classes of labeled cuboid graphs and its subgraphs are introduced. Rules for computation
of iso-paths of labeled cuboid graphs are given in Section 4. The classification of the different
types of labeled cuboid graphs for the computation of iso-paths and iso-path computation rules are
given in Section 5. Additionally, the algorithm for a complete iso-path computation of all labeled
cuboid graphs  is given. Furthermore, we include figures illustrating computed iso-surfaces
for snap shots of a simulated collision of two liquid droplets and
for a crown splash of an impacting droplet. The connectedness of the constructed iso-surface is proven
in Section 6. Finally, in Section 7 we give definitions of neighborhoods of iso-paths on which
the decomposition of iso-surfaces in connected components is based. Having this decomposition, we
show in Section 8 how to efficiently compute iso-surface normals with common orientation on a single component
and how to compute a surface region around an iso-point within the component from which discrete mean curvature
can be computed.

In this paper we introduce appropriate symbols and definitions which do not always follow the traditional way
but efficiently describe the method. Proofs are often given with the help of appropriate figures.
\newpage
\section{Labeled Cuboid Graphs}
In this paper a {\it labeled graph}  denotes a triple  of
,  and  such
that  is a set of vertices (nodes),  is a set of edges with end points in  and
 assigns real numbers from  as weights (labels) to the nodes.

We call  a {\it cuboid} if .
Let  be a labeled graph. We call  a {\it labeled cuboid graph},
if  and  are the set of vertices and edges of ,
respectively. In this case we call  the {\it cuboid} of . We say that a graph
 is a {\it subgraph} of  if ,
, and  is the restriction of  to .

Let  be a graph and , . We call  {\it incident} to 
if  and  are the end points of an edge . In case  is
a subgraph of  we write " {\it is incident to}  {\it in} " to
express the fact that  and  are the end points of an edge .

In the present paper, we only consider labeled graphs  having the following
connectivity property: For any node  there exists another node  such
that  is incident to . \\

\noindent{\bf Note: }If suitable, we use the abbreviation  for a labeled cuboid graph
. Additionally, we sometimes abbreviate a subgraph 
of  by  and also use the notation , abbreviated as , for
labeled cuboid graphs , where . Analogous abbreviations
will be used for subgraphs of . Moreover, we use the shorthand notation  for the labeled
cuboid graph ; analogously for subgraphs of it.

\subsection{Notations}
Below we give some notations which will be used in this work. Recall from Notation~\ref{int:not-1}
the symbols  which will be used to indicate the weight of the
nodes of a labeled cuboid graph, its subgraphs and iso-points.\\

\noindent{\bf Partition of a domain into cuboids: }Let  be a polygonal domain.
Here, polygonal domain means a domain with polygonal boundary. We denote by  the partition
of  into cuboids such that two cuboids have either a common face or a common
vertex or a common edge or they are disjoint. We call such a partition  of 
{\it partition of}  {\it into cuboids}.\\

\noindent{\bf Cuboid grid: }Let  be a polygonal domain and let 
be the partition of  into cuboids. Then we call the vertices of all cuboids in
 a cuboid grid.\\

\noindent{\bf Parallel faces: } We say that two faces  and  of a cuboid  are {\it parallel} if
both  and  have no common nodes; this is abbreviated by . In case  and 
have common nodes we say that both faces are not parallel, symbolized as .\\

\noindent{\bf Face subgraph (a face, for short): }We say that a graph  is a
{\it face subgraph} of a labeled cuboid graph , if  contains a single face of  and 
is the set of nodes of the face in .\\

\noindent{\bf Edge subgraph (an edge, for short) or edge: }We say that a graph  is an
{\it edge subgraph} of a labeled cuboid graph , if  contains a single edge of  and 
is the set of nodes of the edge in .\\

\noindent{\bf Face and Edge neighbors: }We say that two labeled cuboid graphs  and
 are {\it face-neighbors} if, given  and  as the cuboids of  and
, respectively, the intersection  is a face of both cuboids. We say that two cuboid graphs
 and  are {\it edge-neighbors} if the intersection  is an edge of both cuboids.

\subsection{Interpolations and Iso-paths}
Suppose we have a labeled cuboid graph  and an iso-level . Then we
interpolate linearly between the end points of  and their weights to get a possible point on 
with the value . Then, by connecting two distinct interpolated points of  that lie on the same face
of , we get a line. If, by joining each pair of these points that lie on the same face of , we obtain
a closed path that does not cross itself, we call this path a simple closed path. Any such closed path
is the boundary of a continuous 2-dimensional manifold in , which will be defined later.

In this section we give notations and definitions which will be used throughout the text.
\begin{definition}(Iso-point). Let  be a labeled cuboid graph, 
the union of all edges of  and  be an iso-level. We define a function
 by piecewise definition:  on , let  be defined by

where  and  are the nodes of  and , .
For  we call the value  the f-value of . If  such that  or
 then we call  an {\it iso-point} of . For an iso-point  we have

\label{def:labeled-graph-5}
\end{definition}

\noindent{\bf Iso-nodes, disperse nodes and continuous nodes: }Let  be a labeled cuboid
graph and  an iso-level. Then we call all nodes of  with label greater than  {\it disperse nodes},
otherwise {\it continuous nodes}. All nodes of  which are iso-points are also called {\it iso-nodes}. We denote
by  the total number of disperse nodes of . \\

\noindent{\bf disperse/continuous graph, face and edge: }Let  be a labeled cuboid graph and
 an iso-level. We call  a {\it disperse graph} if all nodes of  are disperse nodes and in case
all nodes of  are continuous we call  a {\it continuous graph}. A disperse graph  is symbolized as
 and a continuous graph  is symbolized as . A face subgraph  of
 is a {\it disperse face} if all nodes of  are disperse; we call it a {\it continuous face} if all nodes of
 are continuous. We call an edge subgraph  of  a {\it disperse edge} if all nodes of
 are disperse; we call it a {\it continuous edge} if all nodes of  are continuous.\\

\noindent{\bf L-face and Trivial L-face: }Let  be a labeled cuboid graph and
 an iso-level. We call a face subgraph  of 
an {\it L-face} if there exist two disperse and two continuous nodes of  such that
the disperse nodes are not incident in . An L-face 
is called a {\it trivial L-face}, if both continuous nodes are iso-nodes. All other L-faces are called
{\it non-trivial L-faces}. We denote by  the set of all L-faces of . Figure~\ref{image_4} shows
all possible L-faces of .
\begin{figure}[!ht]
\includegraphics[width=0.7\linewidth]{image/image_4}
\caption{Sketches  and  represent non-trivial L-faces and sketch  represents a trivial L-face.}
\label{image_4}
\end{figure}
\FloatBarrier

\noindent{\bf Singular/regular face: }Let  be a labeled cuboid graph and
 an iso-level. A face subgraph  of
 is called a {\it singular face} if three nodes of  are disperse and the other node is
an iso-node. We say that a face subgraph  of  is a
{\it regular face} if   is not an L-face, not a disperse face, not a continuous face and not a
singular face. Figure~\ref{image_5} shows all possible regular faces and a singular face in
a labeled cuboid graph .
\begin{figure}[!ht]
\includegraphics[width=0.9\linewidth]{image/image_5}
\caption{Sketches ,  and  represent regular faces and sketch  represents a singular face.}
\label{image_5}
\end{figure}
\FloatBarrier

\subsubsection{Iso-path and Iso-Elements}
The definitions of an iso-element and an iso-path are intertwined to each other. An iso-element is
a 2-dimensional manifold in  composed of flat triangular patches and computed
by using the iso-points of a labeled cuboid graph for a given iso-level . An iso-path is
the boundary of an iso-element.\\

\noindent{\bf Cyclically ordered points: }Let  be distinct points in , where
. Suppose that  with  for  and
 is a simply closed path. Then we call the points
 {\it cyclically ordered} and denote the simply closed path  by .

We denote by  the center of  and define a surface with
boundary  by

where  denotes the filled triangle spanned by the points  of
. Note that the patches
which form such a surface do not lie in a common plane, in general, but for  we have

The surface  is an oriented and connected manifold in .\\

\noindent{\bf Iso-line: }Let  be a labeled cuboid graph and  be a
given iso-level. We say that a line segment  is an {\it iso-line} of  if its two end points
 and  are both iso-points, lying on the same face of the cuboid of . We then call 
incident to . This defines incidence between iso-points.

\begin{definition}(Iso-path and iso-element). Let  be a labeled
cuboid graph and let  be an iso-level. Let  have  iso-points
. Let there be a subset  with 
such that the set of line segments 
is a subset of the set of iso-lines of . Furthermore, let the iso-points 
define a simply closed path  with  being its center. Then we call 
an inner iso-path of  if  does not lie on a single non-trivial L-face of .

We call  an outer iso-path of  if it lies on a single non-trivial L-face
 of  and satisfies

where  are the inner iso-paths of  and its face-neighbor ,
where  and  is the common face of the cuboids of  and .

We call  an iso-path if it is an inner or an outer iso-path, and we then call
 an iso-element of .
\label{def:labeled-graph-6}
\end{definition}

\noindent{\bf Corresponding iso-element, iso-path and iso-line: }Let  be a labeled
cuboid graph and let  be an iso-level. Let us denote by  one of the iso-paths
of  and by  the iso-element which is bounded by all of the edges of 
as given by Definition~\ref{def:labeled-graph-6}. Then we say  {\it corresponds to}  or 
{\it corresponds to} . Furthermore, if  is an iso-line of  which lies in  then we say
that  {\it corresponds to} .\\

\noindent{\bf Iso-line neighbor: }Let  and 
be labeled cuboid graphs and  a given iso-level. Assume  or  and  are
face or edge-neighbors according to the given context. Let  and  be distinct
iso-paths of  and , respectively. Furthermore, let  and  be iso-lines
corresponding to  and , respectively such that  and  have the same end points.
Then we call the iso-lines  and  {\it neighbors}.\\

\noindent{\bf Connectedness of iso-path: }Let  be a labeled cuboid graph and
let  be an iso-level. Let  be an iso-path of . By definition of an iso-path,
 contains  iso-lines  of . We say that the iso-path 
of  is {\it connected} if every  lies on an iso-path  of , where 
are labeled cuboid graphs with the same iso-level  and  for all .
This means, here {\it connectedness} refers to {\it connectedness to all sides of the iso-path}.\\

\noindent{\bf Iso-surfaces and connectedness of iso-surfaces: }Let  be a
polygonal domain and let  be the partition of  into cuboids.
Furthermore, let the vertices of the cuboids in  be labeled by real weights from .
Let  be an iso-level. Then we call the surfaces obtained by joining all iso-elements
of the labeled cuboids in  {\it iso-surfaces}. We say that an iso-surface is {\it connected}
if each iso-path which does not have an edge that lies on the boundary of  is connected. That
means, connected iso-surfaces have {\it no holes} except, possibly, at the boundary .
For further theoretical investigations we here assume that the iso-surfaces do not touch the boundary
of .

\subsection{Mapping between labeled graphs}
Let  be a cuboid with set of vertices  and set of edges . Then we denote by 
the set of all labeled cuboid graphs  with .
Let  be a given function. Then we define a graph operation
 by

where . We denote by  the identity
mapping on .

\begin{definition}(Subgraph mapping). Let  and , where  and  are the set
of vertices and the set of edges of a cuboid , respectively. Then we denote by  the
set of all subgraphs with set of vertices  and set of edges  of labeled graphs
. Let  be a given function.
Then we define the graph operation  by

where .
\label{def:labeled-graph-2}
\end{definition}

\begin{definition}(Subgraph replacement). Let  be a subgraph of the labeled cuboid
graph . Given , we define a function 
on  by

We then set

and say that  is obtained from the graph  by replacing
the subgraph  of  by .
\label{def:labeled-graph-3}
\end{definition}

\begin{definition}(Subgraph replacement to a continuous graph). Let  be a subgraph
of the labeled cuboid graph . Additionally, let  be a continuous graph
with  and let  be the subgraph of 
corresponding to . The subgraphs  and  have the same nodes and edges. Then we define a function
 on  by

We set

and say that  is obtained from the graph  by replacing
the subgraph  of  by .
\label{def:labeled-graph-3-1}
\end{definition}

\begin{definition}(Subgraph replacement to a disperse graph). Let  be a subgraph
of the labeled cuboid graph . Additionally, let  be a disperse graph
with  and let  be the subgraph of 
corresponding to . The subgraphs  and  have the same nodes and edges. Then we define a function

on  by

We set

and say that  is obtained from the graph  by replacing
the subgraph  of  by .
\label{def:labeled-graph-3-2}
\end{definition}

Now we define a so-called general labeled graph  which we get by substituting
part of a labeled cuboid graph  for a given iso-level  by a graph, where the
nodes of the graph are labeled with values in . The graph that will be substituted consists of
one or more inner iso-paths of . Therefore, such a general labeled cuboid graph contains at least one
inner iso-path of .
\begin{definition}(General labeled graph). Let  be a labeled cuboid graph.
Let  be a set of  points such that each  lies on
an edge . Let  be a given set of edges with end points in . Assume that each
 is an end point of two edges in the set .
Let  be a labeling on .
Then we call  a labeled graph.
Assume that  on . Then we define a labeled graph
 by

where

Then  is called a general labeled graph, or labeled graph for short.
\label{def:labeled-graph-4a}
\end{definition}

\begin{definition}(General subgraph replacement). Let  be a subgraph of the labeled
cuboid graph . Let  be a labeled graph
such that , ,  and  
on . We define a function  on  by

Then we set

where  and .
We say that  is obtained from the graph  by replacing
the subgraph  of  by the graph
. Note that the labeled graph 
may no longer be a labeled cuboid graph.
\label{def:labeled-graph-4}
\end{definition}

\noindent{\bf Note: }In this paper, any graph operation applied on a labeled graph with a given iso-level 
does not change the iso-level , i.e. the transformed labeled graph has the same iso-level
.

\section{Equivalence Classes of Labeled Cuboid Graphs}
From here on, whenever we consider (one or several) labeled cuboid graphs, it is understood that also
an iso-level  has been chosen. We then speak of "a labeled cuboid graph  with
iso-level ".

For the computation of iso-paths for a labeled cuboid graph  with iso-level 
we compare the node-labels with . The exact values of the nodes are not important for the
investigation of iso-paths of the graph . Therefore, in the following two definitions we introduce
the important concept of equivalence classes of labeled cuboid graphs. The first definition considers
for each node of  whether the node is disperse or not. In the second definition of an equivalence
class of labeled cuboid graphs, besides the disperse nodes of , the differences between iso-nodes
and nodes with node value less than  are accounted for.

\begin{definition}
Suppose  and  are labeled cuboid graphs with
iso-level . We call the graphs  and  -equivalent if the following conditions
are satisfied:
\begin{enumerate}
\item[(i)] ,
\item[(ii)] both  and  have the same number of L-faces,
\item[(iii)] to each  with  and  such that  are
incident to , there exists  with  and  such that
 are incident to  and, to each , one of the following holds:
  \begin{enumerate}
   \item[(a)] if  then ,
   \item[(b)] if  then .
  \end{enumerate}
\end{enumerate}
The mapping from  to  is required to be a bijection. We denote the -equivalence between 
and  by . Additionally, we denote by 
the -equivalence class, defined by

\label{def:equivalence-1}
\end{definition}

\begin{definition}
Suppose  and  are labeled cuboid graphs
with iso-level . We call the graphs  and  -equivalent if the following
conditions are satisfied:
\begin{enumerate}
\item[(i)] ,
\item[(ii)] both  and  have the same number of -faces,
\item[(iii)] to each  with  and  such that  are
incident to , there exists  with  and  such that
 are incident to  and, to each , one of the following holds:
  \begin{enumerate}
   \item[(a)] if  then ,
   \item[(b)] if  then ,
   \item[(c)] if  then .
  \end{enumerate}
\end{enumerate}
The mapping from  to  is required to be a bijection. We denote the -equivalence between 
and  by . Additionally, we denote by 
the -equivalence class, defined by

\label{def:equivalence-2}
\end{definition}
\noindent{\bf Illustration of -equivalence class: }Let  be a labeled cuboid
graph with iso-level . Let  be the nodes of  and let   be the
cuboid of . Consider a sketch that shows the edges and vertices of , and such that the vertices
of  are marked as follows:
\begin{enumerate}
\item[(i)]  is marked by  if ,
\item[(ii)]  is marked by  if .
\end{enumerate}
Then we say that the sketch represents .\\

\noindent{\bf Illustration of -equivalence class: }Let  be a labeled cuboid
graph with iso-level . Let  with 
be the nodes of , where  for 
and  for . Let  be the cuboid of . Consider a sketch
that shows the edges and vertices of , and such that the vertices of  are marked as follows:
\begin{enumerate}
\item[(i)]  is marked by  for all ,
\item[(ii)]  is marked by  for all  with 
and  is not incident to any one of the nodes in ,
\item[(iii)]  is marked by  for all  with 
and  is incident to one of the nodes in ,
\item[(iv)]   is marked by  for all  with 
and  is incident to any one of the nodes in .
\end{enumerate}
Then we say that the sketch represents .\\

Sketches will be "numbered" by  or,  throughout this paper. Examples
for - and -equivalence classes are given in Figure~\ref{image_6}.
\begin{figure}[!ht]
\includegraphics[width=0.8\linewidth]{image/image_6}
\caption{Sketches  represent equivalence classes ,
,  and , respectively.}
\label{image_6}
\end{figure}
\FloatBarrier

\noindent{\bf Equivalence class of a subgraph: }Suppose  is a subgraph of
a labeled cuboid graph  with iso-level . Then we denote by
 the -equivalence class of  such that each element
in  satisfies Definition~\ref{def:equivalence-1} restricted
to . Analogously, we denote by  the
-equivalence class of  such that each element in 
satisfies Definition~\ref{def:equivalence-2} restricted to .\\

\noindent{\bf Illustration of - and -equivalence subclasses: }Analogously to
- and -equivalence classes, we can represent both subclasses by sketches.
Given a sketch , we denote by  and  the equivalence
classes of the graph or subgraph represented by .\\

\noindent Figure~\ref{image_7} illustrates some examples of - and -equivalence
subclasses.
\begin{figure}[!ht]
\includegraphics[width=0.9\linewidth]{image/image_7}
\caption{Sketches  are representations of equivalence subclasses
, ,  and , respectively.}
\label{image_7}
\end{figure}
\FloatBarrier

Finally, we define an additional class of labeled cuboid graphs which we call -class and
-class, where for certain nodes no restriction on the label is done.
\begin{definition}
Let  be a labeled cuboid graph with iso-level . Let
. Then we define the -class of  corresponding to  by

where

\label{def:equivalence-3}
\end{definition}

\begin{definition}
Let  be a labeled cuboid graph with iso-level . Let .
Then we define the -class of  corresponding to  by

where

\label{def:equivalence-4}
\end{definition}

\noindent{\bf Illustration of -class: }Suppose a sketch  represents
, i.e. . Consider a sketch
 which is a copy of  but those nodes in  corresponding to 
are marked by the symbol . We then say that  represents  and write  for this. \\

\noindent{\bf Illustration of -equivalence class: }Suppose a sketch  represents
, i.e. . Consider a sketch
 which is a copy of  but those nodes in  corresponding to 
are marked by the symbol . We then say that  represents
 and write  for this. \\

\noindent{\bf -subclass and -subclass: }Suppose 
is a subgraph of a labeled cuboid graph  with iso-level . Let 
and assume . With an analogous definition as \eqref{eq:equivalence-3}
for , we get the  -subclass of  corresponding
to  which is denoted by . With an analogous
definition as \eqref{eq:equivalence-5} for , we get the  -subclass of
 corresponding to  which is denoted by
.\\

\noindent{\bf Illustration of -subclass and -subclass: }Suppose
, i.e. the sketch  represents .
Consider a sketch  which is a copy of  but the nodes in  corresponding to
 are marked by the symbol . We then say that  represents
 and write  for this. If we start
with  instead, we obtain a sketch  that represents
the subclass ; we write  for this.\\

\noindent Figure~\ref{image_8_9} shows examples of - and
-classes and subclasses with iso-level .
\begin{figure}[!ht]
\includegraphics[width=0.8\linewidth]{image/image_8}\\ \\
\includegraphics[width=0.8\linewidth]{image/image_9}
\caption{Sketches  and  represent -classes and sketches  and  represent
-classes. Sketches  and  represent -subclasses and
sketches  and  represent -subclasses.}
\label{image_8_9}
\end{figure}
\FloatBarrier

\begin{notation}
Sometimes we use sketches which can have either all four symbols  or
all three symbols  to represent a class of labeled cuboid graphs or subgraphs of it.
These special sketches that we use in the next sections are given in Figure~\ref{image_10}. We
call the equivalence classes represented by the sketches ,  and  the -equivalence classes
and the equivalence class represented by the sketch  the -equivalence subclass.
\label{not:special}
\end{notation}
\begin{figure}[!ht]
\includegraphics[width=0.8\linewidth]{image/image_10}
\caption{Sketches ,  and  represent -equivalence classes and sketch
 represents -equivalence subclass.}
\label{image_10}
\end{figure}

\section{Rules of Iso-path Computations}
In this section we define mappings that will be applied on . They
will be used for the computation of iso-paths in labeled cuboid graphs with iso-level .
These mappings replace subgraphs of a graph by other graphs just as described by
Definitions~\ref{def:labeled-graph-3} and~\ref{def:labeled-graph-4}.

The distinction between different types of face subgraphs of a labeled cuboid graph 
with iso-level  introduced in Section 2 is important
for the computation of iso-paths of . We illustrate by Figure~\ref{image_11} the different types
of faces of . Sketch  represents the -subclass  of
regular faces and sketch  represents the -equivalence subclass 
of regular faces. Sketch  represents the -equivalence subclass of singular faces.
Sketches ,  and  represent the possible - and -equivalence
subclasses of L-faces. Here  is a -equivalence subclass of trivial L-faces
and  is a -equivalence subclass of non-trivial L-faces.
\begin{figure}[!ht]
\includegraphics[width=0.7\linewidth]{image/image_11}
\caption{Sketches  illustrate the different types of face subgraphs.}
\label{image_11}
\end{figure}
\FloatBarrier


\subsection{Removing Singular and Isolated Iso-paths}
Let  be a labeled cuboid graph with iso-level . In this section
we introduce graph operations on  which are called -rules and -rules.
The -rules are denoted by  and  and remove {\it singular iso-paths} in , where
a singular iso-path is a degenerate closed path with only one or two nodes. Singular
iso-paths correspond to iso-elements of surface measure zero. The -rule removes singular
iso-paths with only one node and the -rule removes singular iso-paths with two nodes.

The -rules are graph operations denoted by  and . The -rule removes
iso-paths in  such that the iso-element computed from the iso-paths separates only
disperse nodes of . The -rule removes an iso-path of  if  has the
labeled cuboid graph  with the same iso-level 
as a face neighbor and both  and  contain four iso-nodes and four disperse
nodes such that the iso-nodes lie on the common face. In this case, the
iso-element computed from the iso-path separates only the disperse nodes of both
face neighboring graphs. An iso-path of  which is removed using the - or -rule is
called an {\it isolated iso-path}.\\

\noindent{\bf Isolated iso-element: }An iso-element computed from an isolated iso-path is called
an {\it isolated iso-element}.\\

\noindent{\bf Regular labeled cuboid graph: }Let  be a labeled cuboid graph
with iso-level . Assume  is neither a disperse nor a continuous graph. Furthermore,
assume  has neither singular iso-paths nor an isolated iso-path. Then we call  a {\it regular
labeled cuboid graph}. Sometimes we use the abbreviation  is {\it regular} instead of  is a
regular labeled cuboid graph.

\subsubsection{T- and F-graphs}
In this section we introduce the T-subgraphs and F-graphs of a labeled cuboid graph 
with iso-level . The existence of such graphs in  is a possible indication of singular
iso-path or isolated iso-path existence in . The detection of isolated iso-paths is necessary
for the computation of iso-paths of , since isolated iso-paths may not be connected. Removing
isolated iso-paths guarantees the iso-path connectedness as will be shown in Section 6.

\begin{definition}(-subgraph). Let  be a labeled cuboid graph
with iso-level . Let  be a subgraph of  such that the following
conditions hold:
\begin{enumerate}
\item the number of points in  is four (),
\item there exist a point  such that  is incident to all points  in  and

\end{enumerate}
Then  is called a -subgraph.
\label{def:iso-path-2}
\end{definition}

\begin{definition}(-subgraph). Let  be a labeled cuboid graph
with iso-level . Let  be a subgraph of  such that the following
conditions hold:
\begin{enumerate}
\item the number of points in  is seven (),
\item there exist two points  such that  is incident to  in  and

\end{enumerate}
Then  is called a -subgraph.
\label{def:iso-path-3}
\end{definition}
\noindent - and -subgraphs are subsumed as -subgraphs.

\begin{definition}(-graph). Let  be a labeled cuboid graph
with iso-level . Let there be  with  and
\begin{enumerate}
\item each  is incident only to two points in ,
\item ,
\item .
\end{enumerate}
Then  is called an -graph.
\label{def:iso-path-4}
\end{definition}

\begin{definition}(-graph). Suppose  and 
are face-neighbored labeled cuboid graphs with iso-level . Let the following properties hold:
\begin{enumerate}
\item ,
\item ,
\item .
\end{enumerate}
Then  is called an -graph.
\label{def:iso-path-5}
\end{definition}
\noindent - and -graphs are subsumed as -graphs.

Figure~\ref{image_12} illustrates the -subgraphs and -graphs. The sketches  and 
shown in Figure~\ref{image_12} represent the equivalence subclasses  and
. The - and -subgraphs lie in  and ,
respectively. The sketches  in Figure~\ref{image_12} represent ,
 and , respectively. The -graph lies in 
and the -graph lies in  as well as in .
\begin{figure}[!ht]
\includegraphics[width=0.9\linewidth]{image/image_12}
\caption{Sketches  and  represent - and -subgraphs, respectively and -
and -graphs are represented by sketch  and the sketches , respectively.}
\label{image_12}
\end{figure}

\subsubsection{T- and F-rules}
A labeled cuboid graph  with iso-level  can have
one to four singular iso-paths or an isolated iso-path, but not both types of iso-paths. This section
is devoted to the indication and deletion of singular iso-paths or of an isolated iso-path.

Let  and  be an iso-level. Let
 be defined by

and

Let . We define mappings  by
 and , where

and


\begin{definition}(-rule). Let  be a labeled cuboid graph with iso-level
. Let  be a -subgraph of . We call the mapping
, defined by setting  in \eqref{eq:iso-path-4}, a -rule. If required for clarity,
we speak of the -rule with respect to .
\label{def:iso-path-6}
\end{definition}

\begin{definition}(-rule). Let  be a labeled cuboid graph with iso-level
. Let  be a -subgraph of . We call the mapping
, defined by setting  in \eqref{eq:iso-path-4}, a -rule. If required for clarity,
we speak of the -rule with respect to .
\label{def:iso-path-7}
\end{definition}
\noindent We subsume the - and -rules as -rules.

\begin{definition}(-rule). Let  be a labeled cuboid graph with iso-level
. Let  have  distinct -subgraphs, denoted as .  We consider four cases, where
in the case , the -rule changes the node values of  according to:
\begin{enumerate}
\item Case :  (-rule w.r. to )
\item Case :  (-rule w.r. to )
\item Case :  (-rule w.r. to )
\item Case :  (-rule w.r. to ).
\end{enumerate}
Then we write

We call this -rule. If we apply the -rule to  then 
will have no more -subgraphs.
\label{def:iso-path-star-map}
\end{definition}

\begin{definition}(-rules). Let  be a labeled cuboid graph with iso-level
. Let  be an - or an -graph. Let 
 be such that  for all . Then we call the mapping ,
defined by setting  in \eqref{eq:iso-path-4}, an -rule or an -rule if  is an
- or -graph, respectively. The - and -rules are denoted by  and ,
respectively.
\label{def:iso-path-8}
\end{definition}
\noindent We subsume the - and -rules as -rules.

The - and -rules are illustrated in Figure~\ref{image_13_14} and~\ref{image_15_16}, respectively.
The sequence  and  in Figure~\ref{image_13_14} represents - and -rules, respectively.
The sequence  and  in Figure~\ref{image_15_16} represents - and -rules, respectively.
\begin{figure}[!ht]
.
\begin{tabular}[c]{l}
\includegraphics[width=0.4\linewidth]{image/image_13}
\end{tabular}\\ \\

.
\begin{tabular}[c]{l}
\includegraphics[width=0.4\linewidth]{image/image_14}
\end{tabular}
\caption{The sequence  and  illustrate the - and -rules, respectively.}
\label{image_13_14}
\end{figure}
\begin{figure}[!ht]
.
\begin{tabular}[c]{l}
\includegraphics[width=0.4\linewidth]{image/image_15}
\end{tabular}\\ \\

.
\begin{tabular}[c]{l}
\includegraphics[width=0.6\linewidth]{image/image_16}
\end{tabular}
\caption{The sequence  and  illustrate the - and -rules, respectively.}
\label{image_15_16}
\end{figure}
\FloatBarrier

The following result about singular faces will be proved using -rules.
\begin{proposition}(Singular faces). Let  be a labeled cuboid graph with iso-level
. Let  be regular. Suppose that  has a singular face. Then the maximum number of
singular faces which  can have is three. Furthermore, if  has  or  singular faces then
there exist a total of  iso-points which are iso-nodes in  such that these iso-nodes lie on the
singular faces. Then each pair of these iso-nodes lies on a regular face of  or on a space diagonal
of the cuboid of . Moreover, an iso-node can never be on two distinct singular faces.
\label{prop:iso-path-1}
\end{proposition}
\begin{proof}
We give the proof of Proposition~\ref{prop:iso-path-1} by using Figure~\ref{image_17}.
If  then  has at least one singular face. The other possibilities
for  to have two singular faces occur only if  or .
The only possibility for  to have three singular faces is .
The singular faces are marked by bold lines as displayed in Figure~\ref{image_17}. There is no possibility
to get more than three singular faces of . The pair of iso-nodes corresponding to singular faces
of  in case  and  lies on regular face diagonals of . But in case  the pair of
iso-nodes corresponding to singular faces of  lies on a diagonal of the cuboid of . Furthermore,
the rule  forbids the possibility of an iso-node being on two distinct singular faces, which
proves the last claim.
\begin{figure}[!ht]
\includegraphics[width=0.8\linewidth]{image/image_17}
\caption{Sketches , ,  and  illustrate the possibilities of a labeled
cuboid graph to have singular faces.}
\label{image_17}
\end{figure}
\end{proof}
\FloatBarrier

\subsection{Iso-path Computation Rules}
In this section we give rules for iso-path computation for a labeled cuboid graph  with
iso-level . We consider two types of rules which are called -rules and -rules.
-rules compute iso-paths in , whereas -rules compute iso-lines in . By combining - and
-rules we get the iso-paths in . Furthermore, consecutive application of -rules to a labeled
cuboid graph , on which no -rules apply gives an additional iso-path in .

In this section we consider not only labeled cuboid graphs but as well iso-points, iso-lines and iso-paths.
We also give graphical sketches to illustrate for a given labeled cuboid graph the corresponding iso-points,
iso-lines and iso-paths. Figure~\ref{image_18} illustrates that for 
the iso-points on the edges are marked by the symbol  as shown in sketch , and the iso-lines
that connect two iso-points on a face are marked by \!\textminus\!\textminus\!
as shown in sketch . The simple closed path shown in sketch  is an iso-path of .
\begin{figure}[!ht]
\includegraphics[width=0.6\linewidth]{image/image_18}
\caption{Sketches ,  and  illustrate the steps of iso-path computation.}
\label{image_18}
\end{figure}
\FloatBarrier
\subsubsection{-subgraphs, -cuboid graphs and -rules}
Let  be a labeled cuboid graph with iso-level . The -rules are denoted
by  and will be applied on  for computing and deleting iso-paths of
. They are graph operations which will be applied on so-called -subgraphs of  as described below.

\begin{definition}(-subgraph). Let  be a labeled cuboid graph
with iso-level . Let  be a subgraph of  such that the following
conditions hold:
\begin{enumerate}
\item the number of points in  is four (),
\item there exist a point  such that  is incident to all points  in  and

\end{enumerate}
Then  is called an -subgraph.
\label{def:iso-path-10}
\end{definition}

\begin{definition}(-subgraph). Let  be a labeled cuboid graph
with iso-level . Let  be a subgraph of  such that the following
conditions hold:
\begin{enumerate}
\item the number of points in  is six (),
\item there exist two points  such that  is incident to  in ,
each  and  are incident in  to three points in  and

\end{enumerate}
Then  is called an -subgraph.
\label{def:iso-path-11}
\end{definition}

\begin{definition}(-subgraph). Let  be a labeled cuboid graph
with iso-level . Let  be a subgraph of  such that the following
conditions hold:
\begin{enumerate}
\item the number of points in  is four (),
\item there exist a point  such that  is incident to all points  in  and

\end{enumerate}
Then  is called an -subgraph.
\label{def:iso-path-12}
\end{definition}
\noindent -, - and -subgraphs are subsumed as -subgraphs.

In Figure~\ref{image_19_hier}, sketches ,  and  represent ,
 and , respectively. The -, - and -subgraphs
are elements of ,  and , respectively.
\begin{figure}[!ht]
\includegraphics[width=0.6\linewidth]{image/image_19}
\caption{The sketches ,  and  illustrate -, - and  -subgraphs,
respectively.}
\label{image_19_hier}
\end{figure}
\FloatBarrier
\begin{definition}(-cuboid graphs). Let  be a labeled cuboid graph
with iso-level . Let  be an -subgraph of . Then by
using Definition~\ref{def:labeled-graph-3-1} we get a labeled cuboid graph ,
where  and  are defined as given by the Definition~\ref{def:labeled-graph-3-1}. Then
we call  the -cuboid graph corresponding to . If  is an
-subgraph of  then we call  the -cuboid graph corresponding to .
Analogously, in case  is an -subgraph of  using Definition~\ref{def:labeled-graph-3-2}
we get a labeled cuboid graph , where  and  are defined as given
by the Definition~\ref{def:labeled-graph-3-2}. Then we call  the -cuboid graph
corresponding to .
\label{def:iso-path-10-11-12}
\end{definition}
\noindent -, - and -cuboid graphs are subsumed as -cuboid graphs.

In Figure~\ref{image_20}, sketches ,  and  represent ,
 and , respectively. The -, - and -cuboid graphs
are elements of ,  and , respectively.
\begin{figure}[!ht]
\includegraphics[width=0.6\linewidth]{image/image_20}
\caption{The sketches ,  and  illustrate -, - and  -cuboid graphs,
respectively.}
\label{image_20}
\end{figure}
\FloatBarrier
\begin{definition}(Subgraph with iso-path).
Let  be an - or -subgraph and let  be an iso-level.
Let  be the set of iso-points
corresponding to . Let .
Then we define a labeled graph , where
,  and

We call  an - or -subgraph with iso-path if  is an - or -subgraph,
respectively.
\label{def:iso-path-13}
\end{definition}

\begin{definition}(Subgraph with iso-path).
Let  be an -subgraph and let  be an iso-level.
Let  be the set of iso-points
corresponding to . Let  with  and
 are cyclically ordered\}. Then we define a labeled graph
, where ,  and

We call  an -subgraph with iso-path.
\label{def:iso-path-14}
\end{definition}
\noindent -, - and -subgraphs with iso-path are subsumed as -subgraphs with iso-path.
In Figure~\ref{image_21}, sketches ,  and  represent -, - and
-subgraphs with iso-paths, respectively.
\begin{figure}[!ht]
\includegraphics[width=0.6\linewidth]{image/image_21}
\caption{Sketches ,  and  illustrate -, - and -subgraphs with
iso-paths, respectively.}
\label{image_21}
\end{figure}
\FloatBarrier
In the following three definitions we use the functions  and  defined by \eqref{eq:iso-path-1} and
\eqref{eq:iso-path-2}.
\begin{definition}(Iso-path free subgraph).
Let  be an -subgraph  and let  be an iso-level.
Then we call the labeled graph  an iso-path free -subgraph, where
 in case  and  if
. Iso-path free -, - and -subgraphs are subsumed as iso-path free -subgraphs.
\label{def:iso-path-16-17-18}
\end{definition}

In Figure~\ref{image_22}, sketches ,  and  illustrate the iso-path
free \mbox{-,} - and -subgraphs which are contained in , 
and , respectively.
\begin{figure}[!ht]
\includegraphics[width=0.6\linewidth]{image/image_22}
\caption{Sketches ,  and  illustrate iso-path free -, -
and -subgraphs, respectively.}
\label{image_22}
\end{figure}
\FloatBarrier
Let  be an -subgraph of  and let  be
an iso-level. Then there exists an iso-path of , since we have a corresponding -subgraph with
iso-path. This iso-path is as well an iso-path of . In the overall iso-surface construction,
the iso-paths that we get from -graphs will be recorded in a list. Thereafter, the -subgraph
is substituted in  with an iso-path free -subgraph. Hence we get from  a new labeled cuboid
graph  with a new labeling and a reduced number of iso-paths. The complete
procedure of computing an iso-path of  corresponding to an -subgraph, recording the corresponding
iso-path and then substituting the -subgraph in  with an iso-path free subgraph is called
an -rule. These -rules are also called -, - and -rules if they correspond to
the -, - and -subgraphs, respectively.

The symbols used in the following two definitions have been introduced in
Definition~\ref{def:labeled-graph-4}.
\begin{definition}(-rule on a labeled cuboid graph). Let  be a labeled cuboid
graph with iso-level  and let  be an -subgraph of .
Let  be the -subgraph with iso-path
corresponding to . In addition, let  
be the iso-path free -subgraph corresponding to . Then we call the following sequence an
-rule of  corresponding to :

Likewise, we define for  - and -subgraphs the corresponding - and -rules on ,
using the labelings  and , respectively.
We subsume the -, - and -rules on  as -rules.
\label{def:iso-path-20}
\end{definition}

\begin{definition}(-rules on a labeled cuboid graph). Let  be a labeled
cuboid graph with iso-level  and let ,
 be -subgraphs of .
Let  and
 be -subgraphs
with iso-paths corresponding to  and , respectively. In addition, let
 and
 be the iso-path free
-subgraphs corresponding to  and , respectively. Then we call the following sequence an
-rule of , corresponding to  and :

Likewise, we define for - and -subgraphs the - and -rules of . In analogy, we define
-rules. All these -rules will be subsumed as -rules, where  if the type of -rule
is  or  if the type of -rule is  or .
\label{def:iso-path-21}
\end{definition}
The -rules will be written in a simplified {\it graph-theoretical rules} as displayed in
Figure~\ref{image_23_24_25}.
\begin{figure}[!ht]
.
\begin{tabular}[c]{l}
\includegraphics[width=0.7\linewidth]{image/image_23}
\end{tabular}\\ \\

.
\begin{tabular}[c]{l}
\includegraphics[width=0.7\linewidth]{image/image_24}
\end{tabular}\\ \\

.
\begin{tabular}[c]{l}
\includegraphics[width=0.7\linewidth]{image/image_25}
\end{tabular}
\caption{The sequence ,  and  represent the -, - and -rules, respectively.}
\label{image_23_24_25}
\end{figure}
\FloatBarrier

\subsubsection{-subgraphs and -rules}
In this section we describe the so-called -rules. -rules are graph operations which operate
on the regular faces of a labeled cuboid graph  with iso-level .
Applying a -rule to a regular face of  gives an iso-line that lies on this face.

\begin{definition}(-subgraphs). Let  be a labeled cuboid graph with
iso-level . Let  be a regular face subgraph of  such that
one of the following properties is satisfied:
\begin{enumerate}
\item  contains one disperse node,
\item  contains two disperse nodes,
\item  contains three disperse nodes.
\end{enumerate}
In case  satisfies 1 we call  a -subgraph. Likewise, we call  a - or a -subgraph
if  satisfies 2 or 3, respectively. We subsume the -, - and -subgraphs as -subgraphs.
-subgraphs are regular faces and vice versa.
\label{def:iso-path-22}
\end{definition}

\begin{definition}(C-subgraphs with iso-lines and -rules).
Let  be a -subgraph and let  be an iso-level.
Let  be the set of iso-points
corresponding to . Let .
Then we define a labeled graph , where
,  and

We call  a -subgraph with iso-line. Analogously, we define - and
-subgraphs with iso-lines in case  is a -subgraph or a -subgraph, respectively.
We subsume these subgraphs as C-subgraphs with iso-lines. In addition, we call the transformation
process which takes  to a -subgraph with iso-line a -rule . The -, -
and -rules are subsumed as -rules.
\label{def:iso-path-23}
\end{definition}
\noindent The -rules will be written in a simplified {\it graph-theoretical rules} as shown in
Figure~\ref{image_26_27_28}.
\begin{figure}[!ht]
.
\begin{tabular}[c]{l}
\includegraphics[width=0.7\linewidth]{image/image_26}
\end{tabular}\\ \\ \\

.
\begin{tabular}[c]{l}
\includegraphics[width=0.7\linewidth]{image/image_27}
\end{tabular}\\ \\ \\

.
\begin{tabular}[c]{l}
\includegraphics[width=0.55\linewidth]{image/image_28}
\end{tabular}
\caption{The sequence ,  and  illustrate the -, - and -rules, respectively.}
\label{image_26_27_28}
\end{figure}
\FloatBarrier
\noindent{\bf Note: }A singular face contains only one iso-point and hence it is not possible
to compute an iso-line on it. Therefore, -rules will not be applied to singular faces.
\begin{proposition}
The application of any of the  -rules as well as any of the -rules to a labeled cuboid
graph  with iso-level  gives the same iso-lines on the
regular faces of .
\label{prop:iso-path-2}
\end{proposition}
\begin{proof}
Both - and -rules compute iso-lines by connecting iso-points on the same
face of . Since a regular face of  has only two iso-points
we only get one iso-line on the face. Therefore, the - and -rules give the same
iso-line on a regular face .
\end{proof}

\begin{notation}(Corresponding node or nodes of iso-line and corresponding iso-path). Let 
be a labeled cuboid graph with iso-level  and let  be regular. Let 
be a regular face of , having one or two or three disperse nodes (cf. sketches ,  and  of
Figure~\ref{image_29}). Note that in case  has three disperse nodes then the fourth (continuous) node
of  is not allowed to be an iso-node, since otherwise  is a singular face.
Let  be the iso-line of  which is computed by applying either the -, or - or the
-rule to . The types of the -rules which will be applied to  are chosen according to
the graph-theoretical rules as given in Figure~\ref{image_26_27_28}. Then we say that the disperse
node or nodes of  correspond to the iso-line .

Now, let  be a non-trivial L-face of  (cf. sketches  and  of
Figure~\ref{image_29}). Let  be one of the continuous nodes of  and let  be not an iso-node.
Note that at least one continuous node of a non-trivial L-face is not an iso-node, since otherwise  is a
trivial L-face. By joining the iso-points in  that are incident to  in , we get an iso-line  and we
say that the continuous node  corresponds to the iso-line . Moreover, if there exists an inner
iso-path  of  that passes through  we say that  corresponds to . Additionally, let 
be one of the disperse nodes of . By joining the iso-points in  that are incident to  in , we get
an iso-line  and we say that the disperse node  corresponds to the iso-line . Furthermore,
if there exists an inner iso-path  of  that passes through  we say that  corresponds to .
\label{note:corresponding-node}
\end{notation}
The sketches ,  and  in Figure~\ref{image_29} show cases with an iso-line on a regular face. In
case  the iso-line on the face corresponds to the disperse node of the face and in the other two cases
each of the iso-lines on the faces corresponds to the disperse nodes of the face. The Sketches  and
 in Figure~\ref{image_29} show non-trivial L-faces with possible iso-lines. Iso-lines corresponding to
the continuous node (denoted by ) are drawn bold for  and , while iso-lines
corresponding to disperse nodes are drawn light.
\begin{figure}[!ht]
\includegraphics[width=0.99\linewidth]{image/image_29}
\caption{Sketches ,  and  show cases with an iso-line on a regular face.
Sketches  and  show cases with iso-lines on non-trivial L-faces.}
\label{image_29}
\end{figure}
\FloatBarrier

\section{Classification of Labeled Cuboid Graphs and\\ Computation of Iso-paths}
The primary objective of this section is to find a correspondence between iso-paths and subgraphs
of a labeled cuboid graph  with iso-level . The secondary objective
of this section is to find the set of subgraphs of  that correspond to the complete set of iso-paths
of , without knowing the iso-paths of . Finally, we give an algorithm to find the set of subgraphs
of  that correspond to the complete set of {\it inner iso-paths} of , where an {\it inner iso-path}
is  an iso-path of  which does not lie on a single non-trivial L-face of .

For these purposes we define three different types of labeled subgraphs of  as follows:
\begin{enumerate}
\item subgraphs of surface measure zero which do not lie on a face of the cuboid of ,
\item subgraphs of positive surface measure which do not lie on a face of the cuboid of ,
\item L-face subgraphs, lying on a face of the cuboid of .
\end{enumerate}

Subgraphs of  of surface measure zero which do not lie on a face of the cuboid of  correspond to an
iso-element without surface area. The possible surface measure zero subgraphs of 
are illustrated by the sketches  and  in Figure~\ref{image_30}. We denote by  and
 arbitrary subgraphs contained in the -equivalence classes  and
, respectively. The subgraphs  and  are denoted {\it basic
zero subgraphs} of a labeled cuboid graph, since any surface measure zero subgraph of  contains 
or  as a subgraph.
\begin{figure}[!ht]
\includegraphics[width=0.37\linewidth]{image/image_30}
\caption{Sketches  and  illustrate basic zero subgraphs of a labeled cuboid graph. The
labeled subgraphs corresponding to  and  are denoted by  and ,
respectively.}
\label{image_30}
\end{figure}
\FloatBarrier
Subgraphs of  of positive surface measure which do not lie on a face of the cuboid of  correspond
to iso-elements with positive surface area. If we change the disperse nodes of  and
 to continuous nodes and the iso-node of  and the iso-nodes of  to disperse
nodes then we get labeled graphs denoted by  and , respectively. The labeled graphs  and 
are contained in the -equivalence classes  and , respectively,
where sketches  and  are shown in Figure~\ref{image_19}. If we change the iso-node of graph
 to a label less than the iso-value  then we get a labeled graph denoted by  which
is contained in the -equivalence class , where sketch  is given in
Figure~\ref{image_19}.
\begin{figure}[!ht]
\includegraphics[width=0.6\linewidth]{image/image_19}
\caption{The sketches ,  and   illustrate basic positive subgraphs of a labeled cuboid graph.
The labeled subgraphs corresponding to ,  and   are denoted by ,  and ,
respectively.}
\label{image_19}
\end{figure}
\FloatBarrier
The labeled graphs ,  and  have positive surface measure and are called {\it basic positive subgraphs}
of a labeled cuboid graph. Here "{\it basic positive subgraphs}" means that the surface measure that we get
from the labeled graphs ,  and  is positive and they have the smallest number of edges compared to
other positive surface measure subgraphs which do not lie on a face of a labeled cuboid graph.

Now we give a definition which characterizes the positive surface measure subgraphs of a labeled cuboid graph
which contain a single iso-path of the graph which does not lie on a single non-trivial L-face of the graph.
\begin{definition}(Reduced positive surface measure subgraph). Let  be a regular
labeled cuboid graph with iso-level . Then we call the subgraph 
of  a reduced positive surface measure subgraph if either  is in , or if  has at
least two disperse nodes and satisfies the following conditions:
\begin{enumerate}
\item  contains all incidence relations present in  between the disperse nodes of
,
\item for any two different disperse nodes of  there exists a path which connects both disperse nodes such that
the path passes only through disperse edges of ,
\item all continuous nodes of  which are incident to the disperse nodes of  are in  and these
are the only continuous nodes in ,
\item any edge  has at least one disperse node as an end point,
\item  contains no L-faces,
\item  does not contain two different subgraphs of  which are in .
\end{enumerate}
\label{def:positive-surface-measure-graph}
\end{definition}
The labeled graphs contained in the -equivalence classes ,
where sketches  are given in Figure~\ref{image_31}, illustrate examples of reduced
positive surface measure subgraphs which are not basic positive subgraphs.
\begin{figure}[!ht]
\includegraphics[width=0.8\linewidth]{image/image_31}
\caption{Sketches  to  illustrate reduced positive surface measure subgraphs of a labeled
cuboid graph which are not basic and do not lie on a face.}
\label{image_31}
\end{figure}
\FloatBarrier
The existence of basic positive measure subgraphs in a labeled cuboid graph  with iso-level 
induces a classification of  as reducible or irreducible as will be defined in the next subsection.

\subsection{Classification of a Labeled Cuboid Graph}
Let  be a regular labeled cuboid graph with iso-level . Then  can
be reducible or irreducible. This classification of regular graphs is important for the computation of
iso-paths. Reducible labeled cuboid graphs will be transformed stepwise to irreducible labeled cuboid
graphs, using the -rules. Each step of transformation from reducibility to irreducibility of a
labeled cuboid graph  gives an iso-element of  and, furthermore, an irreducible labeled cuboid
graph has a single iso-element.

Any regular labeled cuboid graph  with iso-level  has precisely
one of the following properties
\begin{enumerate}
\item[(a)]  is L-face free,
\item[(b)]  has only non-trivial L-faces,
\item[(c)]  has one trivial L-face.
\end{enumerate}
If  is regular and L-face free then it has one of the following forms:
\begin{enumerate}
\item there exist two disperse nodes such that each of them is incident only to continuous nodes,
\item there exist two continuous nodes such that each of them is incident only to disperse nodes,
\item all nodes satisfy none of the conditions given by 1 and 2 from above.
\end{enumerate}

\noindent Recall that  is the set of all L-faces of a labeled cuboid graph  with iso-level
 and  denotes the number of disperse nodes of .
\begin{definition}(Reducible/irreducible labeled cuboid graph). Let  be a labeled
cuboid graph with iso-level  and let  be regular. Then we call  reducible if one of
the following conditions holds:
\begin{enumerate}
\item ,
\item  and there is  such that ,
\item  and there is  such that .
\end{enumerate}
We call  irreducible if  is not reducible.
\label{def:class-1}
\end{definition}
\noindent{\bf Note: }Reducible and irreducible labeled cuboid graphs are regular. A reducible
labeled cuboid graph contains at least two inner iso-paths, while an irreducible labeled cuboid graph
has one iso-path.\\

Reducible graphs will be decomposed with respect to inner iso-paths using the basic positive subgraphs.
Let  be a labeled cuboid graph with iso-level . Assume  is reducible.
Then there are   and  such that -rule is applicable on  and

where  is the set of all inner iso-paths that we get by applying the -rule to  and 
is the iso-path of a {\it reduced positive surface measure subgraph} 
of a labeled cuboid graph  (with the same iso-level ), where  and 
have the same set of nodes and  contains all disperse nodes of . We call  a {\it rest graph}
of . This means there exists a decomposition  of  with respect to inner iso-paths of  as

where  denote the  distinct -subgraphs of  and 
is the rest graph of  which we get after we apply the -rule to . The rest graph  of 
is irreducible. In addition, we define a decomposition  of  into labeled cuboid graphs
(with the same iso-level ) by

where  is the -th -cuboid graph of  for  and  is the rest graph of .
Note that the  are irreducible labeled cuboid graphs. By the definition of the -th
-cuboid graph of , the following holds:
\begin{itemize}
\item for all , the iso-path of  is the same as the iso-path that we get by
applying the -rule to .
\end{itemize}
Theoretical investigations and algorithmical computation of iso-surfaces and surface normals corresponding
inner iso-paths of  are easier if we use the labeled cuboid graphs  for  and the rest
graph  of  as given by the decomposition~\eqref{eq:class-3-1} instead of .

\subsection{Inner Iso-paths of Labeled Cuboid Graphs}
In this section we compute inner iso-paths of a given labeled cuboid graph  with iso-level
. For the computation of the inner iso-paths we repeatedly refer to the labeled graphs ,
 and  as explained above in this section.

\begin{theorem}\label{thm:class-1}
Let  be a labeled cuboid graph with iso-level . Assume  is regular and
has at least one L-face. Then  contains a subgraph  which is an element of
 or  or .
\end{theorem}
\begin{proof}
We consider each case from  to , separately. In the following,  is an element of a
- or -equivalence class represented by the sketch of a labeled cuboid graph as shown on
the left side of the figures below. We use as well special cases of -equivalence for  as given
in Notation~\ref{not:special}.
\begin{enumerate}
\item : see Figure~\ref{image_32}. Evidently, there is .
\begin{figure}[!ht]
\includegraphics[width=0.3\linewidth]{image/image_32}
\caption{The left part represents  such that  and .}
\label{image_32}
\end{figure}
\FloatBarrier
\item : see Figure~\ref{image_33_34}. In case of sketch  we have  and in case
of sketch  we have  and .
\begin{figure}[!ht]

\begin{tabular}[c]{l}
\includegraphics[width=0.3\linewidth]{image/image_33}
\end{tabular}
\hspace{1cm}
\begin{tabular}[c]{l}
\includegraphics[width=0.35\linewidth]{image/image_34}
\end{tabular}
\caption{The left side of the sketches  and  represent the two possibilities of  such that  and
.}
\label{image_33_34}
\end{figure}
\FloatBarrier
\item : see Figure~\ref{image_35_36}. In case of sketch  we have  and in case
of sketch  we have .
\begin{figure}[!ht]

\begin{tabular}[c]{l}
\includegraphics[width=0.37\linewidth]{image/image_35}
\end{tabular}
\hspace{0.8cm}
\begin{tabular}[c]{l}
\includegraphics[width=0.35\linewidth]{image/image_36}
\end{tabular}
\caption{The left side of the sketches  and  represent the three possibilities of  such that 
and .}
\label{image_35_36}
\end{figure}
\FloatBarrier
\item : see Figure~\ref{image_37_38}. In case of sketch  we have  and
 and in case of sketch  we have .
\begin{figure}[!ht]

\begin{tabular}[c]{l}
\includegraphics[width=0.35\linewidth]{image/image_37}
\end{tabular}
\hspace{0.8cm}
\begin{tabular}[c]{l}
\includegraphics[width=0.33\linewidth]{image/image_38}
\end{tabular}
\caption{The left side of the sketches  and  represent the two possibilities of  such that 
and .}
\label{image_37_38}
\end{figure}
\FloatBarrier
\item : see Figure~\ref{image_39}. It holds that .
\begin{figure}[!ht]
\includegraphics[width=0.3\linewidth]{image/image_39}
\caption{The sketch on the left represents  such that  and .}
\label{image_39}
\end{figure}
\end{enumerate}
\vspace{-0.9cm}
\end{proof}
\FloatBarrier
\noindent {\bf Note: }A labeled cuboid graph  with iso-level  and 
or  has no L-faces, since on any L-face there exists two disperse and two continuous nodes.\\

Propositions \ref{prop:class-1} and \ref{prop:class-2} draw consequences of Theorem~\ref{thm:class-1}.
\begin{proposition}Let  be a labeled cuboid graph with iso-level 
and let  be regular. Then, if  has L-faces at all, the numbers of possible L-faces in dependence
of  is given in Table~\ref{table_1} .
\label{prop:class-1}
\end{proposition}

\begin{proposition}(Removing L-faces). Let  be a labeled cuboid graph with
iso-level . Suppose that  is regular and has at least one L-face. If an -rule is
chosen according to row three of Table~\ref{table_1} and if we apply this -rule to , we get an
L-face free labeled cuboid graph . Here,  has to be chosen according
to row three of Table~\ref{table_1}. The computation of  is described in
Definition~\ref{def:iso-path-21}. Here we denote by  the total number of L-faces in 
and, as before,  denotes the number of disperse nodes of .
\begin{table}[h]
\begin{center}
\begin{tabular}{|l||c|c|c|p{3.5cm}|c|c|c|c|}
\hline
&{\bf 2}&\multicolumn{2}{|c|}{\bf 3}&\multicolumn{2}{|c|}{\bf 4}
&\multicolumn{2}{|c|}{\bf 5}&{\bf 6}\\ \hline\hline
&1&1&3&\hspace{1.6cm}2&6&1&3&1\\ \hline
-rule&&&&\begin{tabular}{c|c}L-faces 
& L-faces  \\
& \end{tabular}&&&&\\ \hline
-rule&&\multicolumn{2}{|c|}{}&\begin{tabular}{c|c}&
\end{tabular}&&&&\\ \hline
\end{tabular}
\end{center}
\caption{Rules for removing L-faces. The signs  and  correspond to
parallel and non-parallelity of the L-faces in case of two L-faces.}
\label{table_1}
\end{table}
\label{prop:class-2}
\end{proposition}
\FloatBarrier
We get the results of Proposition~\ref{prop:class-1} and~\ref{prop:class-2} given in Table~\ref{table_1}
by the same arguments as used to prove Theorem~\ref{thm:class-1} for the cases  to .
Therefore, a detailed proof is omitted.\\



\noindent {\bf Note: }From here on, when we say that an -rule applies on a labeled cuboid graph
 with iso-level , where  has at least one L-face, it is understood
that the corresponding -rule is chosen according to Table~\ref{table_1}. Moreover, if we say that
we apply the -rule  to , it is understood that the -rule is chosen
according to Table~\ref{table_1}.

\begin{proposition}
Let  be a labeled cuboid graph with iso-level . Assume  is regular with
at least one L-face and . Suppose the subgraph  of 
is an L-face. Let the -rule that will be applied on  be chosen according to Table~\ref{table_1}. If
the - or -rule is applied on  then one of the two disperse nodes of  is the disperse node
of an - or -subgraph of , respectively. If the -rule is applied on  then one of the
two continuous nodes of  which is not an iso-node is the continuous node of an -subgraph of .
\label{prop:class-3}
\end{proposition}
\begin{proof}
Choosing the -rule for  according to Table~\ref{table_1} and using the arguments used to prove
Theorem~\ref{thm:class-1} proofs the claim.
\end{proof}

\begin{theorem}\label{thm:class-2}
Let  be a labeled cuboid graph with iso-level . Assume  is regular
and . Suppose that  has no L-face and let one of the following hold:
\begin{enumerate}
\item[] for , the two disperse nodes are on the diagonal of the cuboid of ,
\item[] for , the two continuous nodes are on the diagonal of the cuboid of .
\end{enumerate}
Then  contains two subgraphs  and 
which are in  for the case  and in  for the case , respectively.
\end{theorem}
\begin{proof}In the following,  is an element of the -equivalence class represented by the
sketch of a labeled cuboid graph as shown on the left side of the figures below.
\begin{enumerate}
\item Case : see Figure~\ref{image_40}. Evidently, there is .
\FloatBarrier
\begin{figure}[!ht]
\includegraphics[width=0.35\linewidth]{image/image_40}
\caption{The sketch on the left represents  such that  and the two disperse nodes
are on the space diagonal of the cuboid of .}
\label{image_40}
\end{figure}
\FloatBarrier
\item Case : see Figure~\ref{image_41}. Evidently, there is .
\begin{figure}[!ht]
\includegraphics[width=0.35\linewidth]{image/image_41}
\caption{The sketch on the left represents  such that  and the two
continuous nodes are on the space diagonal of the cuboid of .}
\label{image_41}
\end{figure}
\FloatBarrier
\end{enumerate}
\vspace{-0.9cm}
\end{proof}
\FloatBarrier

\begin{proposition}
Let  be a labeled cuboid graph with iso-level  and let  be irreducible.
Then the following holds:
\begin{enumerate}
\item ,
\item  has no L-face,
\item if  or  then  satisfies neither the assumption  nor  of Theorem~\ref{thm:class-2}.
\end{enumerate}
\label{prop:class-4}
\end{proposition}
\begin{proof}
The results follow using Table~\ref{table_1} and Theorem~\ref{thm:class-2} and the fact that  is as well
irreducible in case  or .
\end{proof}

\begin{theorem}\label{thm:class-3}
Let  be a labeled cuboid graph with iso-level  and let  contain
only one reduced positive surface measure subgraph  and let  contain
all disperse nodes of . Then  is irreducible and hence contains only one iso-path. Vice versa,
if  is irreducible then there exists a subgraph of  which satisfies the above stated properties
of .
\end{theorem}

\begin{proof} We give the prove by computing the iso-paths of  by joining the iso-points with iso-lines.
The steps of the iso-path computation are given in each case by figures with a sequence of sketches from
left to right. The symbols on the rightmost side with only disperse or only continuous nodes are used
to characterize the resulting type of inner iso-path. The sketch on the left shows the respective
labeled cuboid graph, in the second sketch the iso-points are marked and in the third sketch the iso-lines
are inserted by applying the -rules.
\begin{enumerate}
\item  and .
\begin{itemize}
\item[] : see Figure~\ref{image_42}.
\begin{figure}[!ht]
\includegraphics[width=0.9\linewidth]{image/image_42}
\caption{Computation of the iso-path if .}
\label{image_42}
\end{figure}
\FloatBarrier
\item[] : see Figure~\ref{image_43}.
\begin{figure}[!ht]
\includegraphics[width=0.9\linewidth]{image/image_43}
\caption{Computation of the iso-path if  and .}
\label{image_43}
\end{figure}
\FloatBarrier
\item[] : see Figure~\ref{image_44_45_46}.
\begin{figure}[!ht]

\begin{tabular}[c]{l}
\includegraphics[width=0.8\linewidth]{image/image_44}
\end{tabular}

\vspace{0.2cm}

\begin{tabular}[c]{l}
\includegraphics[width=0.8\linewidth]{image/image_45}
\end{tabular}

\vspace{0.2cm}

\begin{tabular}[c]{l}
\includegraphics[width=0.8\linewidth]{image/image_46}
\end{tabular}
\caption{Computation of the iso-path if  and  in three different cases as shown by
the sketches (c1), (c2) and (c3).}
\label{image_44_45_46}
\end{figure}
\FloatBarrier
\item[] : see Figure~\ref{image_47}.
\begin{figure}[!ht]
\includegraphics[width=0.8\linewidth]{image/image_47}
\caption{Computation of the iso-path if 
and .}
\label{image_47}
\end{figure}
\FloatBarrier
\item[] : see Figure~\ref{image_48}.
\begin{figure}[!ht]
\includegraphics[width=0.8\linewidth]{image/image_48}
\caption{Computation of the iso-path if .}
\label{image_48}
\end{figure}
\FloatBarrier
\end{itemize}
\item  or .
\begin{itemize}
\item[] : see Figure~\ref{image_49}.
\begin{figure}[!ht]
\includegraphics[width=0.8\linewidth]{image/image_49}
\caption{Computation of the iso-path if  and both disperse nodes lie on the same edge of the
cuboid of .}
\label{image_49}
\end{figure}
\FloatBarrier
\item[] : see Figure~\ref{image_50}.
\begin{figure}[!ht]
\includegraphics[width=0.8\linewidth]{image/image_50}
\caption{Computation of the iso-path if  and the two continuous nodes lie on the same edge of
the cuboid of .}
\label{image_50}
\end{figure}
\FloatBarrier
\end{itemize}
\end{enumerate}
The simple paths constructed above are iso-paths, since they satisfy the criteria for an
inner iso-path given in Definition~\ref{def:labeled-graph-6}. Furthermore, in all cases,
 contains only one reduced positive surface measure subgraph containing all disperse nodes of
. Therefore,  is irreducible.

To show the last statement, observe that the graphs given in all cases above are the only irreducible
graphs of  which proves the last claim.
\end{proof}

Proposition \ref{prop:class-5} draws a consequence of Theorem~\ref{thm:class-1} and~\ref{thm:class-3}.
\begin{proposition}Let  be a labeled cuboid graph with iso-level .
Assume  is regular and has at least one L-face. Let us apply the corresponding -rules to  given
in Table~\ref{table_1} to obtain a labeled cuboid graph . Then  contains only one
iso-path.
\label{prop:class-5}
\end{proposition}
\begin{proof}
The application of -rule to , where the -rule is chosen according to Table~\ref{table_1},
gives an irreducible graph . Hence, the irreducible graph  has a single iso-path as proven in
Theorem~\ref{thm:class-3}. The iso-path of  is as well one of the inner iso-paths of .
\end{proof}

Proposition \ref{prop:class-6} draws a consequence of Theorem~\ref{thm:class-1}, \ref{thm:class-2}
and~\ref{thm:class-3}.
\begin{proposition}Let  be a labeled cuboid graph with iso-level  and let
 be reducible. Then the successive application of one of the -rules to  transforms 
to an irreducible graph. Here the -rule is chosen according to Table~\ref{table_1} if  has at
least one L-face and, otherwise, the -rule is chosen using the results given in Theorem~\ref{thm:class-2}.
\label{prop:class-6}
\end{proposition}
\begin{proof} First, if  has at least one L-face we apply
Proposition~\ref{prop:class-5}. Second, if  is L-face free, then if  then apply once the
-rule to  to get an irreducible graph  with only one disperse node. In case 
apply once the -rule to  to get an irreducible graph  with seven disperse nodes.

The second result follows by the arguments used to prove Theorem~\ref{thm:class-2}.
\end{proof}

\begin{proposition}
Let  be a labeled cuboid graph with iso-level  and let  be regular.
Suppose that the subgraph  of  is a regular face. Then there exists only one
iso-path of  that passes through the iso-line on the regular face  if one of the following
conditions is satisfied:
\begin{itemize}
\item[(a)]  is irreducible,
\item[(b)]  has no L-face,
\item[(c)]  has one or more L-faces, but there is no subgraph  of
 such that .
\end{itemize}
\label{prop:class-7}
\end{proposition}
\begin{proof} We consider two cases.\\

\noindent{\bf Case 1: }If  is an irreducible graph then there exists exactly one iso-path in  and the
iso-path passes through all iso-lines of regular faces of  as proven by Theorem~\ref{thm:class-3}. \\

\noindent{\bf Case 2: }If  is reducible then we consider two subcases.

First, if  has no L-face then application of the -rule to  in case  changes the disperse
node on the face to a continuous node such that the resulting face is a continuous face. In case 
application of the -rule to  changes the continuous node on the face to a disperse node such that
the resulting face is a disperse face. These results follow from the arguments used to prove
Theorem~\ref{thm:class-2} for the cases  and . Now, note that no iso-lines of  lie on
disperse or continuous faces of . Hence in both cases only one iso-path of  passes through the iso-line
on the regular face .

Second, if  has at least one L-face then application of one of the -rules chosen according to
Table~\ref{table_1} will decrease the number of L-faces of  and never create new L-faces. Hence, there
exists only one iso-path of  that passes through the iso-line on the regular face . This holds,
because after computing an iso-path of  that passes through an iso-line on a regular face, then the
number of disperse nodes of  that lie on the regular face increase or decrease. In case the number of
disperse nodes on the regular face increases, then the resulting face is a disperse face or a singular face.
But in case the number of disperse nodes of the regular face decreases, then the resulting face is a
continuous face. Note that no iso-lines of  lie on singular faces of . The arguments given in the
second subcase follow from the arguments used to prove Theorem~\ref{thm:class-1} for the cases 
to . Hence, through the iso-line of a regular face  passes only one iso-path of .
\end{proof}

\vspace{0.2cm}

\begin{proposition}
Let  be a labeled cuboid graph with iso-level . Assume  is regular and
has at least one non-trivial L-face . Let the -rule that will be applied on 
be chosen according to Table~\ref{table_1}. If the -rule that will be applied on  is either the
- or -rule then there exist two iso-lines on  and to each iso-line there exists an inner
iso-path of  that passes through it. If the -rule will be applied on  then one of the following
holds:
\begin{itemize}
\item[(a)] in case  has no iso-nodes then there exist two iso-lines on  and to each iso-line there
exists an iso-path of  that passes through it such that none of the iso-paths lies on the L-face.
\item[(b)] in case one continuous node of  is an iso-node then there exists one iso-line on  and to the
iso-line there exists an iso-path of  that passes through it such that the iso-path does not lie on the
L-face.
\end{itemize}
\label{prop:class-8}
\end{proposition}
\begin{proof} We consider two cases. \\

\noindent{\bf Case 1: }Either the - or -rule will be applied on . Then, applying
Proposition~\ref{prop:class-3} we find that through one of the iso-lines passes an iso-path of .
The - or -rule then changes the disperse node corresponding the iso-line to a continuous node.
Then there remain only one disperse node on the face. Hence the L-face will be transformed to a regular face
with only one disperse node. On a regular face lies only a single iso-line and through this iso-line
passes only one iso-path of  according to Proposition~\ref{prop:class-7}. But  is a non-trivial L-face
and, hence, both iso-paths passes through two different iso-lines on  and both iso-paths are different.\\

\noindent{\bf Case 2: }The -rule will be applied on . Then, applying Proposition~\ref{prop:class-3}
we find that through one of the iso-lines passes an iso-path of . The -rule then changes one of the
continuous nodes of  which is not an iso-node of  to a disperse node. Then there remains only one
continuous node on the face. Hence, the L-face will be transformed to a regular face if  has no
iso-node. On a regular face lies only a single iso-line and through this iso-line passes only one iso-path
of  according to Proposition~\ref{prop:class-7}. In this case,  is a non-trivial L-face
and, hence, both iso-paths passes through two different iso-lines on  and both iso-paths are different.
But in case,  has iso-node then the L-face will be transformed to a singular face. But on a singular
face lies no iso-line. Therefore, in this case, only one iso-path passes through an iso-line of the L-face .
\end{proof}

\subsection{Outer Iso-path of a labeled cuboid graph}
Until now we have not considered iso-paths on an L-face of a labeled cuboid graph  with
iso-level  which is regular and has at least one L-face. This section is devoted to compute
outer iso-paths of . As we know on each regular and on each trivial L-face of  lies only one iso-line
and on a singular face of  lies no iso-line. Consequently, there is no iso-path on each of these faces
of . But on a non-trivial L-face of  we can have an iso-path if there exists a face-neighbored
labeled cuboid graph  with iso-level  which is regular.

Let  and  be labeled cuboid graphs with iso-level
. Suppose that  and  are regular and face neighbors. Let  be a face of  and  be
a face of  such that . Note that, if  is an L-face of  this does not mean
that the face  is an L-face of . Hence, if we have an iso-path in , this does not mean that
we have an iso-path in . Hence, the node values of the common face of  and  can be different
in case both  and  are not regular faces. All these claims will be proved in this section.

If  is an L-face then we give a {\it graph-theoretical rule} to indicate if an iso-path on it exists.
This graph-theoretical rule is used for the computation of iso-paths on L-faces in Section 5.4. The
graph-theoretical rules depend on the type of the L-face of , on the type of the -rule
which will be applied on  and is chosen according to Table~\ref{table_1}, and on the type of the face
 of , and in case  has at least one L-face then, in addition, from the type of the -rule
which will be applied on  and is chosen according to Table~\ref{table_1}.\\

\noindent{\bf Note: }When needed we use in special cases -equivalence for a labeled cuboid graph
with iso-level  as given in Notation~\ref{not:special}.

\begin{theorem}\label{thm:class-4}
Let ,  be labeled cuboid graphs with iso-level
. We assume that both  and  are face neighbors such that the common face
 is a non-trivial L-face. Furthermore, suppose that both  and  are
regular. Let the -rules that will be applied on  and on  be chosen according to
Table~\ref{table_1}. Assume that on  the rules  or  apply and on  the rule 
applies. Then there exists an iso-path in . Furthermore, to each iso-line on  there exists only one
inner iso-path that passes through the iso-line.
\end{theorem}
\begin{proof} Let  and  with
the sketches  and  as shown in Figure~\ref{image_51}. Then the L-face  is
in  or in  as shown in Figure~\ref{image_51}. In the case
, application of the - or -rule to  gives a graph represented
by the sketch  of Figure~\ref{image_52_53}. For the same case, application of the -rule
to  gives a graph represented by sketch  of Figure~\ref{image_52_53}. These graph
operations transform  to a graph represented by the sketch  of Figure~\ref{image_52_53}.
This is shown by the graph-theoretical rule  and by the sequence  in Figure~\ref{image_52_53}.
Hence we get an iso-path on . For the case  we apply the same procedure to
transform  to a graph represented by the sketch  of Figure~\ref{image_54_55}. These procedures
are shown by the graph-theoretical rule  and by the sequence  in Figure~\ref{image_54_55}.
Hence we get an iso-path on .

The last claim holds true because to each disperse node in  there is a {\it corresponding} iso-path
in  (we get this using the - or -rule on ), and to the continuous node of  which
is not an iso-node, there exists a corresponding iso-path in  (we get this using the -rule
on ). Here, the word "{\it corresponding}" is to be understood in the sense of
Notation~\ref{note:corresponding-node}.
\end{proof}
\begin{figure}[!ht]
\includegraphics[width=0.7\linewidth]{image/image_51}
\caption{Sketches  and  represent  and .
The common face of  and  is a non-trivial
L-face. The sketches  and  represent  and  in which the two
different types of non-trivial L-faces are obtained.}
\label{image_51}
\end{figure}
\FloatBarrier
\begin{figure}[!ht]

\begin{tabular}[c]{l}
\includegraphics[width=0.6\linewidth]{image/image_52}
\end{tabular}\\ \\

\begin{tabular}[c]{l}
\includegraphics[width=0.89\linewidth]{image/image_53}
\end{tabular}
\caption{The graph-theoretical rule given by  and the sequence  illustrate the procedure
how to compute an iso-path that lies on a non-trivial L-face which is in .}
\label{image_52_53}
\end{figure}
\FloatBarrier

\begin{figure}[!ht]

\begin{tabular}[c]{l}
\includegraphics[width=0.6\linewidth]{image/image_54}
\end{tabular}
\\ \\

\begin{tabular}[c]{l}
\includegraphics[width=0.89\linewidth]{image/image_55}
\end{tabular}
\caption{The graph-theoretical rule given by  and the sequence  illustrate the procedure
to compute an iso-path that lies on a non-trivial L-face which is in .}
\label{image_54_55}
\end{figure}
\FloatBarrier
\begin{theorem}\label{thm:class-5}
Let  and  be labeled cuboid
graphs with iso-level . Assume that both  and  are face neighbors
with a common L-face . Furthermore, suppose that  and  are
isolated iso-path free. Let  and ,
where the sketches  and  are as shown in Figure \ref{image_56}. Let the -rules that will
be applied on  be chosen according to Table~\ref{table_1}. Let one of the rules  or  be
applicable on . Then there exists an iso-path in . But if rule  is applicable on  then
there exists no iso-path in . Furthermore, to each iso-line on  there exists only one inner iso-path
that passes through the iso-line.
\end{theorem}
\begin{proof} First, transform  and  to  and
, respectively, by applying the -rule to each of
them. Then we have  and , where sketch  is as
shown in Figure~\ref{image_56}. Then  will be transformed to a graph in
 and  will be transformed to a graph in ,
with sketches  and  from Figure~\ref{image_56}. We then apply the -rule
to  and one of the rules  or  to  in order to compute the iso-path on 
as shown in sketch  of the sequence  of Figure~\ref{image_57_58}. This procedure is
illustrated by the graph-theoretical rule given by  in Figure~\ref{image_57_58}.

The last claim holds true because to each disperse node in  there is a {\it corresponding}
iso-path in  (we get this using the - or -rule on ), and to the continuous
node of  which is not an iso-node, there exists a corresponding iso-path in  only if the
-rule is applied on . Here, the word "{\it corresponding}" is to be understood in the
sense of Notation~\ref{note:corresponding-node}.
\end{proof}
\begin{figure}[!ht]
\includegraphics[width=0.7\linewidth]{image/image_56}
\caption{The sketches  and  represent  and ,
respectively. Sketches  and  represent faces of  and
, respectively. Both faces  and  have the same nodes, but
different node weights.}
\label{image_56}
\end{figure}
\FloatBarrier
\begin{figure}[!ht]

\begin{tabular}[c]{l}
\includegraphics[width=0.89\linewidth]{image/image_57}
\end{tabular}\\ \\
\mbox{}\\ \\

\begin{tabular}[c]{l}
\includegraphics[width=0.89\linewidth]{image/image_58}
\end{tabular}
\caption{The graph-theoretical rule given by  and the sequence  illustrate the procedure
to compute an iso-path that lies on a non-trivial L-face which is in .}
\label{image_57_58}
\end{figure}
\FloatBarrier
\noindent{\bf Explanation of Figure~\ref{image_57_58} : }The graph-theoretical rule  together with
the sequence of the sketches  means that we first apply the -rule to , obtaining .
Then we apply to the common face of  and  the -rule to get . By applying the - or
-rule to , we get . But if we apply the -rule to  we get
. From  we get an iso-path on the common face of  and . The
sequence  in Figure~\ref{image_57_58} shows that the regular face obtained in  and the
L-face obtained in  have iso-lines as shown by  and  or , respectively.
In case , the common face of  and  has an iso-path as shown by . In case , there
is no iso-path on the common face of  and .
\FloatBarrier

\subsection{Algorithm for Complete Iso-path Computation}
In this section we will give an algorithm for the complete iso-paths computation of a labeled cuboid graph
 with iso-level  which is neither disperse nor continuous.

The complete iso-paths of  will be computed in three steps. In the first step we delete all singular
iso-paths or an isolated iso-path of . These deletions will be carried out by transforming the graph
 to a graph . If  is regular then the second and the third step will be
to compute the iso-paths of  which are as well iso-paths of . The only difference between the
iso-paths of  and  is that  contains no singular iso-paths and no isolated iso-path. Here,
if  then  is regular but if  then  contains only singular or isolated iso-paths.
In this case  is a disperse labeled cuboid graph and has no iso-path and hence  has as well no iso-path.
This means, in case  is a disperse graph we consider as well  as a disperse graph.\\

Let  be a polygonal domain allowing a partition into cuboids,
i.e.  for the partition .  Let
 be a set of labeled cuboid graphs
with common iso-level , and let  be the cuboid of  for . The following
algorithm computes the complete iso-paths of  for .

\subsection*{Algorithm for Iso-paths Computation}
\noindent{\bf Notations}: Let  be a labeled cuboid graph with iso-level .
Denote by  the total number of disperse nodes of  and by  and  the set
of L-faces and non-trivial L-faces of , respectively. Recall the -mapping as defined in
Definition~\ref{def:iso-path-star-map}.\\

\noindent {\bf For }  {\bf do:}\\
\mbox{}\hspace{0.3cm}    {\bf Step 1}: {\bf removing singular iso-paths or isolated iso-path}\\
\mbox{}\hspace{0.6cm}      i) removing singular iso-paths if \,:\\
\mbox{}\hspace{1.1cm}         a) \\
\mbox{}\hspace{1.1cm}         b) if  then no iso-path, , {\bf go to} Step 1\\
\mbox{}\hspace{0.6cm}      ii) removing isolated iso-path if \,:\\
\mbox{}\hspace{1.1cm}         a) \\
\mbox{}\hspace{1.1cm}         b) if  then no iso-path, , {\bf go to} Step 1\\

\noindent\mbox{}\hspace{0.3cm}    {\bf Step 2}: {\bf iso-path computation of} 
{\bf for the case} \\
\mbox{}\hspace{0.6cm}      (i) register all L-faces: \\
\mbox{}\hspace{1.2cm}          let  and \\
\mbox{}\hspace{0.6cm}      (ii) register all non-trivial L-faces of  if \,:\\
\mbox{}\hspace{1.2cm}           let  and \\
\mbox{}\hspace{0.6cm}      (iii) let  for  be all face neighbors of  such that \\
\mbox{}\hspace{1.4cm}            the common nodes of  and  are the nodes of \\
\mbox{}\hspace{0.6cm}      (iv) use Table~\ref{table_1} to get the type of the -rule corresponding to \\
\mbox{}\hspace{1.3cm}            a) use -rule to compute iso-paths of  \\
\mbox{} \hspace{1.7cm}              \\
\mbox{} \hspace{1.7cm}              {\bf Note: } is the rest graph of  and hence irreducible\\
\mbox{}\hspace{1.3cm}            b) use -rules to compute the iso-path of  (see Theorem~\ref{thm:class-1})\\
\mbox{}\hspace{0.6cm}     (v) {\bf if}  {\bf then} compute iso-path on non-trivial L-faces of \,:\\
\mbox{}\hspace{1.2cm}         {\bf for} \\
\mbox{}\hspace{1.7cm}             a) use Table~\ref{table_1} to get the type of the -rule corresponding to \\
\mbox{}\hspace{1.7cm}             b) use Theorems~\ref{thm:class-4} and~\ref{thm:class-5} to compute a possible iso-path on\\
\mbox{}\hspace{2.27cm}                \\

\noindent\mbox{}\hspace{0.3cm}   {\bf Step 3}: {\bf iso-path computation of}  {\bf for the case} \\
\mbox{}\hspace{0.6cm}        {\bf if}  and {\bf if} there exists  in  or in
, {\bf where}\\
\mbox{}\hspace{1.1cm}              , use Theorem~\ref{thm:class-2} to compute iso-path\\
\mbox{}\hspace{0.6cm}        {\bf else}\\
\mbox{}\hspace{1.1cm}           apply -rules to compute iso-path \\
\mbox{}\hspace{0.6cm}        {\bf endif}\\

Note that the complexity of this algorithm is .

\subsection{Application}
Tracking or capturing interfaces of two-phase systems is an important issue for instance in computational
fluid dynamics simulations~\cite{opac-b1133222}. The interfaces are used not only to track the phases but
there can be adsorbed quantities on them like surfactants as well, which affect the hydrodynamics of the
system~\cite{Alke_and_Bothe}. Two well-known volume tracking methods are the Volume of Fluid (VOF)
method~\cite{Nich81} and the Level Set method~\cite{Stan02}. The Level Set method uses a signed distance
function which implicitly determines the interface as the zero level set. While level set methods are
advantageous concerning discrete mean curvature computation, they suffer from volume loss of the disperse phase.
The latter requires reinitialization of the level set function which introduces non-physical changes.
The VOF-method conserves the phase volumes, but the standard interface reconstruction using a piecewise
planar approximation (PLIC,~\cite{Rider97reconstructingvolume}) leads to disconnected interface representations.
The present iso-surface algorithm can be employed to obtain a connected interface approximation instead.

Consider a polygonal domain  and a domain partition
 of  into cuboids. Given a function  which gives the volume fractions of one of the phases (say the disperse phase),
we define a labeling function , where 
is the set of vertices of all cuboids in , by

where  if and only if . Using the labeling function , we get from
 labeled cuboid graphs .
Next, we interpolate the node values onto the edges according to \mbox{Definition}~\ref{def:labeled-graph-5}. Then, for
a given , we determine for all graphs  a common iso-level  by solving the
inequality

where the domain  is the union of all bounded volumes enclosed by
the iso-surface for the given iso-level . Note that  contains all disperse nodes
of .

The function  given by

measures the deviation between the total volume of the given disperse phase and that of the computed
enclosed volume. The function  is decreasing, but not necessarily strictly decreasing.
Furthermore,  can have (small) jumps. The latter can appear at an iso-level  if
 attains the value  on several, complete edges. In such cases, the error bound 
in~\eqref{eq:application-2} cannot be chosen arbitrarily small.

Using the iso-surface algorithm from Section 5.4 we first computed iso-surfaces for snap shots of the
simulated collision of two liquid droplets. The snap shots are taken at different time steps and the
resulting iso-surfaces are shown in Figure~\ref{image_59_60_61_62}. We have also applied the
iso-surface algorithm to the outcome of a crown splash, where one typical snap shot is shown in
Figure~\ref{image_63}. For the iso-surface computation of the binary droplet collision and
the splash we computed the iso-level  using  in~\eqref{eq:application-2}.
\begin{figure}[!ht]
\includegraphics[width=0.3\linewidth]{image/image_59}\hspace{1cm}
\includegraphics[width=0.4\linewidth]{image/image_60}\\ \\
\includegraphics[width=0.4\linewidth]{image/image_61}\hspace{1cm}
\includegraphics[width=0.4\linewidth]{image/image_62}
\caption{The sequence of sketches show iso-surface meshes for snap shots of a boundary droplet collisions taken at different
time steps.}
\label{image_59_60_61_62}
\end{figure}
\begin{figure}[!ht]
\includegraphics[width=0.7\linewidth]{image/image_63}
\caption{The figure shows the iso-surface mesh for a snap shot of a crown splash.}
\label{image_63}
\end{figure}
\FloatBarrier

The highly dynamic impact of a droplet into a liquid layer which produces the crown-splash is a good
validation example, since during the splash several special cases occur in the iso-surface computation.
Indeed, there appear labeled cuboid graphs  containing
\begin{enumerate}
\item singular faces,
\item a trivial L-face,
\item non-trivial L-faces,
\item edges with iso-node end points.
\end{enumerate}

Sketches ,  and  of Figure~\ref{image_64_65_66} illustrate some of the special
cases which appear. Sketch  illustrates a labeled cuboid graph and its iso-path, where the graph
contains a singular face. Sketch  illustrates a labeled cuboid graph and its iso-path, where the
graph contains a trivial L-face. Sketch  illustrates a labeled cuboid graph and its iso-path, where
the graph contains two non-trivial L-faces and an edge with iso-nodes as end points of it.
\begin{figure}[!ht]
\includegraphics[width=0.16\linewidth]{image/image_64}\hspace{2cm}
\includegraphics[width=0.16\linewidth]{image/image_65}\hspace{2cm}
\includegraphics[width=0.16\linewidth]{image/image_66}
\caption{Sketches ,  and  illustrate labeled cuboid graphs and their iso-paths, where each of the
cuboids has at least one iso-node.}
\label{image_64_65_66}
\end{figure}
\FloatBarrier

\section{Connectedness of Iso-paths}
Let  be a labeled cuboid graph with iso-level . Then an iso-line in 
can be common for two distinct iso-paths that lie in . In all other cases an iso-line in 
can be common for at least two iso-paths, one corresponding to  and one or more corresponding to another
labeled cuboid graph , where  is a face or edge neighbor of . Iso-lines which
are common for at least two iso-paths are as well common for the corresponding iso-elements. Therefore,
a common iso-line of at least two iso-paths is common for at least two distinct iso-elements. This means
that iso-elements are connected via such iso-lines. We also say that the iso-paths are connected at
the common iso-line. If each edge of an iso-path is common for at least two distinct iso-paths then
we say that the iso-path is connected. Connected iso-paths give rise to connected iso-surfaces. Therefore,
connectedness of iso-paths is a very important property which will be investigated in this section.

Consider a polygonal domain  having a cuboid partition 
such that . Let  be a set of labeled cuboid
graphs with common iso-level  and  be the cuboid of  for . We compute the
iso-paths in each  by applying the algorithm given in Section 5.4. Then the following questions arise:
\begin{enumerate}
\item is it possible to show iso-surface connectivity?
\item is it possible to decompose iso-surfaces into components such that each edge of an iso-element
in a component is a common to only two distinct iso-elements in the same component?
\item how to compute discrete mean curvature at iso-points?
\item is it possible to get all local topological information of an iso-surface in a simple way?
\end{enumerate}
The first problem will be answered affirmative in this section and the remaining questions will
receive a positive answer in Sections 7 and 8.\\

\noindent{\bf Note: }When needed we use in special cases -equivalence for a labeled cuboid graph
with iso-level  as given in Notation~\ref{not:special}.

\begin{theorem}\label{thm:connect-1}
Let  be a labeled cuboid graph with iso-level . Suppose  is regular
and has a trivial L-face. Then  contains exactly two distinct inner iso-paths. Furthermore,  has
only one trivial L-face.
\end{theorem}
\begin{proof} It holds that  or  or ,
where the sketches  are as given in Figure~\ref{image_67}, since precisely in these cases
 is isolated iso-path free. We then apply the -rule to  if ,
or apply - and -rules to  if , or apply the -rule to  if
. We get in all cases two distinct inner iso-paths in . Furthermore, if
 then application of the -rule to  transforms  to a labeled cuboid
graph  where the sketches  and  are as given in
Figure~\ref{image_67}. But  is a disperse graph and has no L-faces. Hence,  has only one trivial
L-face.
\end{proof}
\begin{figure}[!ht]
\includegraphics[width=0.7\linewidth]{image/image_67}
\caption{Sketch  represents . Then, by substituting in  for the
unknowns of the two nodes marked by the symbol  the symbols
 or , we get four different types of labeled cuboid graphs that lie in
, ,  or , respectively.}
\label{image_67}
\end{figure}
\FloatBarrier
\begin{lemma}\label{lemma:connect-1}
Let  and  be labeled cuboid graphs with
iso-level . We assume that both  and  are face neighbors with a trivial
L-face . Furthermore, suppose that  and  are isolated iso-path
free. Then there exist exactly two distinct inner iso-paths in  that pass through the iso-line
in .
\end{lemma}
\begin{proof} We consider two different labelings of  and
 as given by the graphical sequences in  and  of
Figure~\ref{image_68_69}. Transform first  and  to  and
, respectively, by applying the -rule to each of
them. Then we have  and , where sketch  is as
shown in Figure~\ref{image_68_69}. Furthermore, the common trivial L-face of graphs  and
 will be transformed to  and
to , respectively. Sketches
 and  are as shown in Figure~\ref{image_68_69}. Both  and  are isolated iso-path
free and, hence, there exists an iso-path in . But since we have a trivial L-face in ,
we have at least two distinct inner iso-paths in  that pass through the iso-line in  according
to Theorem~\ref{thm:connect-1}. Application of - or -rule or - and -rules
to , where  is isolated iso-path free, we get exactly two distinct inner
iso-paths of .
\end{proof}
\begin{figure}[!ht]
\noindent 
\begin{tabular}[c]{l}
\includegraphics[width=0.7\linewidth]{image/image_68}
\end{tabular}
\\ \\

\noindent 
\begin{tabular}[c]{l}
\includegraphics[width=0.7\linewidth]{image/image_69}
\end{tabular}
\caption{Application of -rule to  and to 
transforms the common trivial L-face of both graphs to graphs in  and in ,
respectively. The sequences of sketches  and  show the complete possible results.}
\label{image_68_69}
\end{figure}
\FloatBarrier
\begin{theorem}\label{thm:connect-2}
Let  and  with iso-level  where sketches  and  are as
shown in Figure~\ref{image_70_71_72}. Both  and  are face neighbors with a regular
face  which is in  or  or
, where sketches  are as shown in Figure~\ref{image_73}.
Then there is only one iso-line in .
\end{theorem}
\begin{proof} Apply the -rules to . We consider in Figure \ref{image_70_71_72} different
labelings of  and  which give different types of regular faces as a face neighbor
of both graphs. In all these cases as shown by the graphical sequences ,  and  in
Figure~\ref{image_70_71_72} we get, by applying the -rules to each of the regular faces,
a single iso-line.
\end{proof}
\begin{figure}[!ht]
\noindent 
\begin{tabular}[c]{l}
\includegraphics[width=0.8\linewidth]{image/image_70}
\end{tabular}
\\ \\

\noindent 
\begin{tabular}[c]{l}
\includegraphics[width=0.8\linewidth]{image/image_71}
\end{tabular}
\\ \\

\noindent 
\begin{tabular}[c]{l}
\includegraphics[width=0.8\linewidth]{image/image_72}
\end{tabular}
\caption{The sequences , , and  show that for a common regular face of
 and  we get an iso-line
on the common face on  as well as on  as shown in the last sketch of each sequence.}
\label{image_70_71_72}
\end{figure}
\begin{figure}[!ht]
\includegraphics[width=0.7\linewidth]{image/image_73}
\caption{Regular faces of one, two and three disperse nodes. Each of the regular faces of
two and three disperse nodes has at least one non-iso-node.}
\label{image_73}
\end{figure}
\FloatBarrier

\begin{lemma}\label{lemma:connect-2}
Let  be a labeled cuboid graph with iso-level . Let ,
where the sketch  is as shown in Figure~\ref{image_74}. Then  contains exactly two distinct
inner iso-paths. Both iso-paths pass through the edge  of the graph .
\begin{figure}[!ht]
\includegraphics[width=0.15\linewidth]{image/image_74}
\caption{Sketch  illustrates . The edge  of  has
two iso-node end points.}
\label{image_74}
\end{figure}
\FloatBarrier
\end{lemma}
\begin{proof}
Apply the -rule to .
\end{proof}

\begin{proposition}
Let  be a labeled cuboid graph with iso-level . Suppose  is regular.
Let  have two iso-points which are iso-nodes that lie on the same edge of a cuboid of . Then the
maximum number of distinct iso-paths that pass through both iso-nodes is two.
\label{prop:connect-1}
\end{proposition}
\begin{proof}
According to Lemma~\ref{lemma:connect-2} any graph in , where sketch  is as shown
in Figure~\ref{image_74}, has two distinct iso-paths that pass through both iso-nodes. The maximum
number of inner iso-paths that passes through two iso-points which are iso-nodes and are end points of
an edge of  is attained only if .
\end{proof}
\FloatBarrier

\begin{definition}(System of cuboids and system of labeled cuboid graphs at an edge). Let
 be a cuboid with vertices . Let  be an edge of .
Then let  be distinct cuboids such that   for  have the edge 
in common and if  for  then  and  have a common face.
Then we say  are the system of cuboids with common edge . Furthermore, let
 for  be labeled cuboid graphs with a common iso-level
 and be singular iso-path free and isolated iso-path free. In addition, let  be
the cuboid of  for . Then we say  are the system of labeled
cuboid graphs with common edge .
\label{def:system-cuboid-1}
\end{definition}

The following lemma is a direct geometrical consequence of Definition~\ref{def:system-cuboid-1}.
\begin{lemma}
Let  and  for  be a system of cuboids and a system
of labeled cuboid graphs, respectively, corresponding to the common edge  such that  is a
cuboid of . Then to each cuboid  there exist two distinct cuboids
 and  in  such that  and  as well as  and  have a
common face. Analogously, to each labeled cuboid graph  there exist two
distinct graphs  and  in  such that  and  as well as 
and  are face neighbored.
\label{lem:system-cuboid-1}
\end{lemma}

\begin{theorem}
Let  and  for  be a system of cuboids and a system of
labeled cuboid graphs, respectively, corresponding to the common edge  such that  is a
cuboid of . Let  and assume that the nodes of  corresponding to
the end points of  are iso-nodes (), where 
is the common iso-level of  for . Let the total number of iso-paths
that pass through  be denoted by . Then , where
. If , denote the iso-paths by . Additionally,
denote by  the labeled cuboid graph corresponding to  and by 
the cuboid of  for . If  then to any  there exists ,
, such that one of the following holds:
\begin{itemize}
\item[(i)]  and  are face neighbored and have two common
disperse nodes,
\item[(ii)] if (i) does not hold then  and there exists a unique  which
depends on  and  such that  is face neighbored to  and
 is face neighbored to , where . Furthermore, if  then  is face neighbored to 
for . In addition, each  for  has at least seven disperse
nodes and, hence, they have no iso-path that passes through . Note that for  it holds
.
\end{itemize}
Then we say the pair  and  are disperse connected with respect to the common
iso-line . Furthermore, this property is unique and, hence, there is no other iso-paths which
are disperse connected either to  or  with respect to .\\

\noindent Note that all graphs and cuboids stated above are in the system of labeled cuboid
graphs and in the system of cuboids corresponding to the edge , respectively.
\label{thm:connect-3}
\end{theorem}
\begin{proof}
The proof will be given using the following two parts.\\

\noindent{\bf 1.} Uniqueness of disperse connectivity of  and  with respect to .\\

If this is not the case, then either  or  has an additional disperse node
and there exists a graph , where  such that
 is either face neighbored to  or to  and  is disperse or  contains
no iso-path that passes through . If  is not disperse but has no iso-path that passes through 
then  contains seven disperse nodes (this follows from the application of -rule to ). But
in this case the face neighbored graph to , which is  or , has
two additional disperse nodes in common with . Then either  or  has at
least five disperse nodes. But then application of -rule to the graph with at least five disperse
nodes leads to a labeled cuboid graph with seven disperse nodes. Then the graph will have no iso-path that
passes through . But this is a contradiction to the regularity of  and .
This proves the claim for the unique disperse connectivity of  and .\\

\noindent{\bf 2.} Proof validity of cases (i) and (ii).\\

Let us assume that an iso-path  exists in  such that the iso-path passes
through . Then  is regular and, hence,  has at least two disperse nodes.
From Lemma~\ref{lem:system-cuboid-1} we know that it is possible to rename the graphs 
such that  is face neighbored to  for  and  is face neighbored
to . Now the proof follows using the following two steps:\\

\noindent{\bf Step 1.} Since  is regular there exists a face neighbor
 to  such that  and  have two common disperse
nodes corresponding to . If  is regular and has an iso-path that passes through  then case (i)
is satisfied.\\

\noindent{\bf Step 2.} But if  given in Step 1 has no iso-path that passes through  then from the
application of -rule to  we get , where  has at least seven disperse nodes.
Then  has no iso-path that passes through . But since all graphs  are
singular iso-path free and isolated iso-path free we have . Again  is face neighbored
with  such that . Hence,  and  have at least
two disperse nodes in common. If  is regular and has an iso-path that passes through  then case
(ii) is satisfied by choosing . But if  has no iso-path that passes through  then
 has at least seven disperse nodes (again after application of -rule to ). Hence, in
case  has at least seven disperse nodes set , and  and then apply again
Step 2 until case (ii) is satisfied.

Note that there exists a
regular graph , disperse connected to  with respect to edge . If this is
not the case, then each face neighbored graphs  and  of , where
 and , have at least seven disperse nodes.
Then  and  have two common disperse nodes. But since  is face neighbored
to  and not face neighbored to ,  and  have another two common
disperse nodes. But then  has at least four disperse nodes. Furthermore, there exist
four disperse nodes of  denoted by  such that each 
and  are face vertices of the cuboid of , where  are the end points
of . Furthermore, there exists a graph  which is face neighbored with  and . In
addition,   has at least seven disperse nodes (this follows from the assumption that there exists
no  which is disperse connected with ). If  is regular and has
only four disperse nodes then application of Proposition~\ref{prop:connect-1} gives that 
has two iso-paths and both iso-paths passes through  and hence both iso-paths are disperse connected
with respect to ; therefore, case (ii) is satisfied. In this case,
we use , ,   in (ii) of Theorem~\ref{thm:connect-3}. But then
in  there exist two iso-paths  and  which are disperse connected
with respect to . This is a contradiction to the assumption that there exists no
 which is disperse connected to  with respect to . In case  has
five disperse nodes then application of -rule to  gives that  has at
least seven disperse nodes and, hence,  has no iso-path that passes through . This is
a contradiction to the assumption that  has an iso-path that passes through
, hence this case does not occur. Consequently, case (ii) holds.\\

To conclude, the set of all iso-paths that pass through  can be uniquely decomposed into pairs
such that each pair is disperse connected with respect to . Hence, the total number of iso-paths in
the system of graphs which pass through  is an even number. The maximum number is 8 as follows
from Proposition~\ref{prop:connect-1}).
\end{proof}

\begin{theorem}\label{thm:connect-4}(Connectedness of iso-paths).
Let the polygonal domain  have a domain partition 
into cuboids. Let  be a set of labeled
cuboid graphs with common iso-level , and  be the cuboid of  for .
Assume that all , , singular iso-path free and isolated iso-path free.
Compute the complete iso-paths in each  by applying the algorithm given in Section 5.4.
Then each iso-line of an arbitrary  is common for at least two distinct iso-paths, where
the iso-paths can be in  or in face- or in edge-neighbors of .
\end{theorem}
\begin{proof} We prove the claim by distinguishing different cases:\\

\noindent {\bf Case 1:} In case an iso-line  is an edge of  then
Theorem~\ref{thm:connect-3} says that we have at least two iso-paths in  which have 
as a common iso-line.\\

\noindent {\bf Case 2:} In case an iso-line  lies on a trivial L-face of ,
Theorem~\ref{thm:connect-1} says that we have two iso-paths in  which have  as a common iso-line.
We even get four iso-paths with  as a common iso-line if the trivial L-face is common for  and
, where  is a face neighbor of .\\

\noindent {\bf Case 3:} Let  be a regular face of . Assume that  is as well a face
of , where  is a face neighbor of . Then Proposition~\ref{prop:class-7}
implies that each of the graphs  and  contains exactly one inner iso-path running through .\\

\noindent {\bf Case 4:} Let an iso-line  lie on a non-trivial L-face  of
. Then we consider two subcases:
\begin{itemize}
\item[(a)] Suppose  is a common non-trivial L-face of  and . Then
one of the following holds:
\begin{itemize}
\item[(i)] if there exists an iso-path on each of the L-faces then each iso-line on the L-face is part of an
iso-path that does not lie on the L-face as shown in Proposition~\ref{prop:class-8}. Therefore, to each iso-line
there exist two iso-paths, where the first iso-path is an inner iso-path and the second iso-path is the
iso-path on the L-face.
\item[(ii)] if there exists no iso-path on each of the L-faces then Proposition~\ref{prop:class-8} says that
to each iso-line on the L-face there exists an iso-path in  and in the face neighboring graph. Hence to
each iso-line on the L-face there exist two iso-paths.
\end{itemize}
\item[(b)] Suppose  is a face neighbor of , where
 is a regular face of  and . Then one of the following
holds:
\begin{itemize}
\item[(i)] if there exists an iso-path on the L-face then we have the same conclusion as in (i) of (a).
\item[(ii)] if there exists no iso-path on the L-face then we have the same conclusion as in (ii) of (a).
\end{itemize}
\end{itemize}
\vspace{-0.7cm}
\end{proof}

\section{Components of iso-surfaces}
In this section we show how to compute separate components of connected iso-surfaces such that
on each component normals and discrete mean curvature can be calculated. We give definitions required to define
iso-path connectivity such that an iso-surface can be decomposed into its components. These components
are orientable and connected.

We have proved in Section 6 using Theorem~\ref{thm:connect-4} that the iso-surfaces computed by
applying the algorithm given in Section 5.4 are connected. This means to each iso-line  of
an iso-path there exist at least two iso-paths such that  is a common edge to them.
Theorem~\ref{thm:connect-3} and Theorem~\ref{thm:connect-1} show that an iso-line  can be
common for up to eight or four iso-paths, respectively. It is clear that discrete mean curvature
computation at the end points of , where  is common to more than two iso-paths, is not defined.
Hence, we will give a definition which allows to decompose the iso-surfaces into connected components
such that at each iso-point of the connected components discrete mean curvature computation is possible.

To define components of closed iso-surfaces we need the notion of a {\it disperse path} which is a simple
path but not a loop such that it consists solely of edges of cuboids which only have disperse nodes as
end points. Such a path runs within the {\it system of graphs}, which is defined next.

\begin{definition}(System of cuboids and system of graphs).  Let  be a cuboid with vertices
 and edges . Then let 
be the system of cuboids (see Definition \ref{def:system-cuboid-1}) with respect to edges
, respectively. For each  and  we denote by
 the cuboids such that

where  for all , and the following holds:
\begin{itemize}
\item for all  there exists a neighborhood  of 
      such that  and any two distinct cuboids
      in the set  have a common edge
      or a common face.
\end{itemize}
Then we say  and  for ,  are the system of cuboids
of . Furthermore, let  and  for ,  be labeled cuboid graphs with a common
iso-level  and singular iso-path free and isolated iso-path free. In addition, let
 and  be the cuboids of  and  for , ,
respectively. Then we say that  and  for ,  are the
system of labeled cuboid graphs of , where  for all .
Additionally, we call  the number of system of graphs of  which is given by

where  for ,  is the set of faces of
,  is the set of edges of face  of  for , and  denotes
the number of elements in a set. It holds  for  and . Hence,
in the present case of cuboids, .
\label{def:curve-1}
\end{definition}
The derivation of formula~\eqref{eq:system-graph-1} is easy and is therefore left to the reader.
Equation~\eqref{eq:system-graph-1} can be even applied to arbitrary partition
of a polygonal domain which has the form as given in Definition~\ref{def:curve-1}.


\begin{definition}(Disperse path). Let  be a simple, but not closed path in the system of graphs
of a given labeled cuboid graph . If there is  and to each 
there exists  in the system of graphs of  and a disperse edge 
of  such that  is of the form , we call  a disperse path in
the system of graphs of .
\label{def:disperse-path}
\end{definition}

\begin{remark}
In the following, when we say "corresponding nodes of an iso-line" or
"corresponding iso-line of nodes", the word "corresponding" is to be understood in the
sense of Notation~\ref{note:corresponding-node}. Furthermore, when we say "disperse
node or nodes corresponding to iso-line  with respect to an iso-path", then it means that
 is an edge of the iso-path and  corresponds to the disperse node or nodes.
\label{rem:corresponding-node-line}
\end{remark}

\begin{definition}(Disperse connectedness of two iso-paths at a common edge). Let  be
a labeled cuboid graph with iso-level . Let  be regular and  be an iso-line of .
Let  and  be two distinct iso-paths with , where 
is one of the iso-paths of  and  is an iso-path in the system of graphs of . We say that
 and  are disperse connected with respect to the common edge  if one of the
following holds:
\begin{enumerate}
\item  and  are the only iso-paths in the system of graphs of  such that
,
\item  is an inner iso-path of ,  is an inner iso-path for one labeled cuboid
graph in the system of graphs of , and one of the following holds:
\begin{enumerate}
\item at least one of the disperse nodes corresponding to  is the same for the iso-line  with
respect to  and ,
\item there exist two distinct disperse nodes  and  in the system of graphs of  such that
 is a node of  and  may not a node of . The disperse nodes  and  correspond
to  with respect to  and , respectively, and there exists a disperse path in the
system of graphs of  which connects  and .
\end{enumerate}
\end{enumerate}
We say that two distinct iso-elements  and  in the system of
graphs of  are {\it neighbored} with respect to the iso-line  of  if the corresponding
iso-paths  and  of  and , respectively, are disperse connected with respect
to .
\label{def:component-1}
\end{definition}

\noindent{\bf Convention of iso-elements disjointness: }Let  be  a labeled cuboid graph
with iso-level . Let  be regular and  be an iso-line of . Let  and  be two
distinct iso-elements in the system of graphs of  which have  as a common iso-line (),
but are {\it not} neighbored with respect to . Then, concerning their connectivity, we consider both
iso-elements  and  as disjoint. This means, given arbitrary points  and ,
there exists no path  that joins  and . This convention helps to
decompose iso-surfaces into connected components such that on each component computation of surface PDEs
and discrete mean curvature is possible.\\

\noindent The next theorem will be used to decompose iso-surfaces into components.
\begin{theorem}Let  be a labeled cuboid graph with iso-level  and let
 be regular. Let  be an iso-path of , given by  with
. We set  for  and  which
are iso-lines of  and as well edges of . Then there exist iso-paths  in
the system of labeled cuboid graphs of  such that  and
each pair  is disperse connected with respect to  for .
\label{thm:iso-path-connectedness-disperse-paths}
\end{theorem}
\begin{proof}
We prove the claim by considering three cases. In all these cases, we let  be a
graph in the system of graphs of  such that  and  are face-neighbored and
the nodes of the face  of  are common to both graphs  and .
Suppose  is the face of  such that .\\

\noindent{\bf Case 1: }An iso-line  lies on a regular face 
of . \\

We consider three subcases.
\begin{enumerate}
\item Assume that  contains only two disperse nodes and at least one of the continuous nodes is not
an iso-node. Then the following holds:
\begin{enumerate}
\item .
\item An iso-line on  is common only for two iso-paths. The first iso-path lies in  and the second in
. This follows from Proposition~\ref{prop:class-7}. The common iso-line of both iso-paths corresponds to
the disperse nodes of  (cf. Remark~\ref{rem:corresponding-node-line}).
\end{enumerate}
Hence, condition  of Definition~\ref{def:component-1} is satisfied.

\item  has two disperse nodes and two iso-nodes. Then the following holds:
\begin{enumerate}
\item[(a)] If  is common for  and  and if there exists an iso-path  in  that passes
through  such that the corresponding disperse nodes of  and  with respect to  are
the same, then both iso-paths are disperse connected.

\item[(b)] The other cases are shown in Theorem~\ref{thm:connect-3}.
\end{enumerate}

Hence, condition  or condition  of Definition~\ref{def:component-1} are satisfied.
\item The face  of  is a non-trivial L-face. Then one of the following holds:
\begin{enumerate}
\item Let  be the only iso-line on . Then the following holds:
\begin{enumerate}
\item  contains exactly one iso-node and  contains three disperse nodes and one continuous node
which is not an iso-node.
\item The iso-line  is common to a pair of iso-paths such that the iso-paths are disperse
connected with respect to the common iso-line.
\end{enumerate}
These results follow from Proposition~\ref{prop:class-7} and \ref{prop:class-8}. Hence, condition  of
Definition~\ref{def:component-1} is satisfied.
\item Let there be an iso-path on . Then the following holds:
\begin{enumerate}
\item There exist three iso-lines on .
\item There is one iso-line on .
\item To each iso-line in  there exists an iso-path that does not lie on .
\item Each iso-line on  is common to the iso-path on  and another iso-path that does not lie on
. Both iso-paths are disperse connected with respect to their common iso-line .
\end{enumerate}
These results follow from Proposition~\ref{prop:class-7} and \ref{prop:class-8}. Hence, condition  of
Definition~\ref{def:component-1} is satisfied.
\end{enumerate}
\end{enumerate}
\noindent{\bf Case 2: }An iso-line  lies on a trivial L-face
 of . \\

We consider two subcases.
\begin{enumerate}
\item Let  be a singular or a disperse face of .

Then, according to Theorem~\ref{thm:connect-1} there are two iso-paths in  having  as a common
iso-line. These iso-paths are disperse connected with respect to .

Hence, condition  of Definition~\ref{def:component-1} is satisfied.
\item Let  be a trivial L-face of . Then the following holds:
\begin{enumerate}
\item .
\item According to Theorem~\ref{thm:connect-1} we get two iso-paths in  and two iso-paths in  and
all four iso-paths have the common iso-line . Then there exists a unique pairing of iso-paths such that
the common iso-line  of each pair of iso-paths corresponds to a disperse node of . Both pairs of
iso-paths are then disperse connected with respect to . But it is not possible that three of them
are disperse connected. If this is the case, then there is a disperse path which connects the disperse
nodes on . But for this to happen we need at least five disperse nodes, say in . Then application of
the -rule to  will change one of the iso-nodes of  on  to a disperse node. But then
 has no iso-path that passes through . Therefore, there cannot exist three iso-paths that pass
through  and which are disperse connected.
\end{enumerate}
Hence, condition  of Definition~\ref{def:component-1} is satisfied.
\end{enumerate}
\noindent{\bf Case 3: }An iso-line  lies on a non-trivial L-face
 of .
\begin{enumerate}
\item Let there be no iso-path on the L-faces  and . Then the following holds:
\begin{enumerate}
\item .
\item There exist two iso-lines on .
\item Each iso-line from (b) is common to a pair of iso-paths which are disperse connected with respect to
the common iso-line on . This follows from Proposition~\ref{prop:class-8}.
\end{enumerate}
Hence, condition  of Definition~\ref{def:component-1} is satisfied.
\item Let there be an iso-path on the L-faces. Then the following holds:
\begin{enumerate}
\item .
\item There exist three iso-lines on  if  has an iso-node; otherwise, there exist four iso-lines on .
\item To each iso-line from (b) there exists an iso-path that does not lie on .
\item Each iso-line from (b) is common to the iso-path on  and another iso-path which does not lie on
. Both iso-paths are disperse connected with respect to the common iso-line .
\end{enumerate}
These results follow from Proposition~\ref{prop:class-7} and \ref{prop:class-8}. Hence, condition  of
Definition~\ref{def:component-1} is satisfied.
\end{enumerate}
\end{proof}

\noindent Based on Theorem~\ref{thm:iso-path-connectedness-disperse-paths} we give the following definition.
\begin{definition}(Disperse connected iso-elements and iso-surface component).
Let the polygonal domain  have a domain partition  into cuboids. Let  be a set of labeled
cuboid graphs with common iso-level , and  be the cuboid of  for .
Assume that all  are singular iso-path free and isolated iso-path free.
Compute the complete iso-paths in each  by applying the algorithm given in Section 5.4.
Let the union of all iso-surfaces in  be denoted by  and let
 and  be two iso-elements in . If there is  and
if there exist iso-elements  of  such that
,  and all pairs ,  are neighbored with respect
to their common iso-line, then we say  and  are disperse connected iso-elements.

We define a component of  with respect to an iso-element , denoted by , by
\begin{itemize}
\item  consists of  and of all iso-elements of  which are disperse connected to .
\end{itemize}
\label{def:component-2}
\end{definition}

\begin{theorem}(Connectedness and uniqueness of a component).
An iso-surface component defined as in Definition~\ref{def:component-2} is connected. Furthermore,
for any two distinct iso-elements  and  of  it holds that either
 or  in the sense of
{\it convention of iso-elements disjointness}. Consequently, the components of  are uniquely
determined.
\label{thm:iso-element-connected}
\end{theorem}
\begin{proof}We give the proof in two steps. \\

\noindent{\bf 1. Uniqueness of components: }Note that the relation between iso-elements to be disperse
connected is reflexive, symmetric and transitive by its definition. Hence, the set of all iso-elements
decomposes into disjoint (in the sense of iso-elements disjointness) equivalence classes, which are
precisely the components of . This proves the uniqueness.\\


\noindent{\bf 2. Connectedness of a component: }Let  be a component of  and let
 be iso-elements. Then  is connected to  and  is connected to ,
hence each two iso-elements in  are disperse connected.
\end{proof}


The intersection of two different components is  in the sense that the intersection does not
contain an iso-element, but it may contain discrete points or line segments (cf. {\it Convention of
iso-elements disjointness}). The following remark provides additional information on the intersection
of two distinct components of .

\begin{remark}(Relations of two distinct components).
We denote from here on the components of  by  if  has 
components, where  and  are as in Definition~\ref{def:component-2}. Note that the only
possibility for two distinct components to have an iso-element in common is if both have a common trivial L-face
such that on the L-face there is an iso-element. But such an L-face belongs to only one component, since if there
exists an iso-path on the L-face then each iso-line on the L-face is common to only two distinct iso-paths as
shown in Proposition~\ref{prop:class-8}. The first iso-path is the iso-path on the L-face and the second iso-path
is an inner iso-path. Therefore, according to Definition~\ref{def:component-2}, the iso-path on the L-face belongs
to only one component and, hence, all other inner iso-paths which have a common iso-line with the iso-path on the
L-face belong to the same component as well. This follows from condition  of Definition~\ref{def:component-1}.

If two different components have an iso-point or iso-points in common, where none of them are end points of
an iso-line for both components, then the common points are vertices of cuboids in the partition .
In addition, if two different components have an iso-line or iso-lines in common, then the common line segments
are edges (cf. Theorem~\ref{thm:connect-3}) or face diagonals (cf. Lemma~\ref{lemma:connect-1}) of cuboids in the
partition . This follows from Definition~\ref{def:component-2} and
Theorem~\ref{thm:iso-element-connected}.
\end{remark}

\section{Surface Normal Vectors and discrete Curvature}
For iso-surfaces  computed according to the algorithm given in Section 5.4 there exists
a decomposition of  into components according to Definition~\ref{def:component-2} and
Theorem~\ref{thm:iso-element-connected}. By construction, a component is oriented and connected.
This section is devoted to solve the following problems:
\begin{enumerate}
\item how to compute surface normal vectors of a component,
\item how many iso-points of a component are connected to an iso-point  of the component,
\item constructing an appropriate region in one component around an iso-point  of the component
on which discrete mean curvature computation for  is possible.
\end{enumerate}
In this paper we are not giving details on how to compute discrete mean curvature at discrete points of
a component since this can be found in the literature (see e.g. \cite{Meyer02Vismath}), but we
show how to identify the required surface region inside a component.

\subsection{Surface Normal Vectors}
The computation of surface normal vectors of an iso-surface is straight forward, except for
determining the same orientation for all iso-surface normal vectors. For all triangles of the triangulated
iso-elements, let  be an arbitrarily oriented surface normal. Then the oriented normal field
 is obtained by choosing  or , locally. For this purpose we first
need to know a direction in which the desired surface normal shall be approximately pointing. We call
a vector which approximates the surface normal vector, while giving the same orientation (i.e. angle to
 is below ), a {\it surface pseudo-normal}. Usually, for the computation of such a
surface pseudo-normal one needs to know the gradient of the nodal function  of a labeled
cuboid graph . We give here an alternative method which does not use the gradient
of the function  but instead uses all node positions of , distinguishing between continuous
and disperse nodes.

Let  be an irreducible labeled cuboid graph with iso-level . Let there be
 iso-points  of the iso-path, denoted by , in  and let  be the
center of the iso-element  corresponding . By definition of an iso-surface
component as given in Definition~\ref{def:component-2}, the orientation of the normal field on an
iso-element of  should be pointing towards the continuous nodes of . Therefore, a surface pseudo-normal
for the iso-element, denoted as , has to be determined using the position of
disperse and continuous nodes of  such that it points from the disperse to the continuous phase. Let
Let  be a normal of the triangle , say. Then the surface normal  on
 is


\begin{remark} For iso-surface normal computation, the consideration of irreducible labeled cuboid graphs,
is sufficient. First, computation of surface normals of iso-paths which lie on L-faces of a labeled cuboid graph
is straight forward. Second, a labeled cuboid graph  can be decomposed into irreducible graphs with respect
to inner iso-paths of  as given in~\eqref{eq:class-3-1}.
\label{rem:second-remark}
\end{remark}

\noindent{\bf Surface pseudo-normal: }Let  be an irreducible labeled cuboid graph
with iso-level . Let  and let  and 
correspond to the set of continuous and disperse node indices of , respectively.
Then . We compute a surface pseudo-normal for  by adding all vectors
 for  with  defined by

where  and  are continuous and disperse nodes of , respectively.
Then we get

\begin{proposition}The surface pseudo-normal vector given in \eqref{eq:curve-3} satisfies

where .
\label{prop:curve-2}
\end{proposition}
\begin{proof}
We get the following by using  and  :

By using the above relation we get

\end{proof}

\subsection{Discrete Curvature}
An iso-surface  which is computed according to the algorithm given in Section 5.4 is polygonal
(piecewise planar). Hence, on a component  of  only discrete mean curvature computation
has a meaning. The discrete mean curvature can be computed on points in  which are vertices of the
triangulation of . Any iso-element of  is triangulated using a center 
of the iso-element and its iso-points. Generally, the discrete mean curvature at  is nearly zero. Hence,
discrete mean curvature computation is mainly important at the iso-points of .

Discrete mean curvature computation methods (see e.g. \cite{Meyer02Vismath}) use for the computation of
discrete mean curvature at an iso-point  of  a piece of triangulated surface region around  in
which  is contained. This surface region is contained in . Hence, the aim of this section is to
compute such a surface region.

The following theorem will give us the minimum and maximum number of neighbors of an iso-point  of
 such that the neighbors are iso-points in  and each of them is incident to .

\begin{theorem}Let  be a labeled cuboid graph with iso-level  and let
 be regular. Let  be an iso-element of  and let the iso-path  corresponding to  be
given by  with . We set  for
 and , which are iso-lines and edges of . Let us fix
one of the , say . Then there exist  iso-paths 
in the system of labeled cuboid graphs of , where  such that
\begin{enumerate}
\item ,
\item ,
\end{enumerate}
with  and
\begin{itemize}
\item[(i)]  and  are disperse connected with respect to  for ,
\item[(ii)]  and  are disperse connected with respect to .
\end{itemize}
This means, if  is the iso-surface computed according to the algorithm given in Section 5.4 then all iso-paths
 belong to the same component  of  and all these iso-paths have in
common the iso-point  and these iso-paths are the only iso-paths in  with this property.
\label{thm:iso-path-region-curvature}
\end{theorem}
\begin{proof}
We prove the claim by considering four cases. \\

\noindent{\bf Case 1: }All faces in the system of graphs of , where the iso-point  lies, are regular.\\

\noindent In this case the point  is an end point of two iso-lines of . Both iso-lines
lie in  but on different regular faces. Note that  does not lie on an L-face and hence both
regular faces do not each have two disperse and two iso-nodes (according to Lemma~\ref{lemma:connect-2}).
Hence, on each of these regular faces lies an edge of a unique iso-path according to
Proposition~\ref{prop:class-7}. Therefore, to each neighbor of the faces there exists an iso-path
which passes through an iso-line on the face. Hence, we have two additional iso-paths in two different
labeled cuboids. From these two additional iso-paths we get two additional iso-lines with an end point
. Therefore, we get a total of four distinct iso-lines with an end point . But only three
iso-paths cannot give a connected iso-surface at the point  and hence, there exist at least one
additional iso-path that passes through . Hence, there exist  satisfying
the claim of the proposition which is .\\

\noindent{\bf Case 2: }The iso-point  is not an iso-node of  and all faces in the system of
graphs of , where the iso-point  lies, are L-faces.\\

\noindent The iso-point  is common to four faces in the system of graphs of . If each of the
L-faces is non-trivial, then on each face we have two iso-lines with end point . Then we have a
total of   iso-lines which are neighbors of . If some of the L-faces are trivial L-faces,
we get .\\

\noindent{\bf Case 3: }The iso-point  is an iso-node of  and all faces in the system of graphs
of , where the iso-point  lies, are L-faces.\\

\noindent The iso-point  is common to eight faces in the system of graphs of . The prove uses
the same argument as in Case 1 to get that  is at least four. In case each of the L-faces is non-trivial,
we get the same result as in Case 2. In this case, the same argument as in Case 2 can be applied.\\

\noindent{\bf Case 4: }The iso-point  of  lies on an L-face in the system of graphs of .\\

\noindent By combining the cases 1, 2 and 3 we then get .
\end{proof}

In the next definition, we give the so-called {\it surface region}  of an iso-point  within the
component  of . The surface region  contains  and all iso-points in 
which are incident to . Additionally,  contains each center of the iso-elements which have  in
common. The surface region  of  can be used to compute discrete mean curvature at 
(see e.g. \cite{Meyer02Vismath}).

\begin{definition}(Neighboring iso-lines, iso-points, points and surface region). Let  be an
iso-element of the iso-surface  computed according to the algorithm given in Section 5.4.
and let  an iso-point. Then we call the iso-lines  which we get from
Theorem~\ref{thm:iso-path-region-curvature} using , neighboring iso-lines to . Then
 and  lie in the component . Let the iso-points  be
the other end points of the , i.e.  for .
We call  neighboring iso-points to . Note that to each iso-line  there exists an
iso-path  in  such that . All the iso-paths
 are different according to Theorem~\ref{thm:iso-path-region-curvature}.
Let  be the center of  and let us denote by  the number of iso-lines
of . We define

for , and

We then call the piece of iso-surface  defined by

the surface region of . In addition, we define a set of points
 by

for , and

Finally, we set .
\label{def:neighboring-points-1}
\end{definition}

\noindent{\bf Computation of discrete mean curvature: }Let  and  be as defined in
Definition~\ref{def:neighboring-points-1}. The discrete mean curvature at the iso-point  is computed by
integrating  over , where  denotes the
Laplace-Beltrami operator. The surface  is piecewise linear. Therefore, we can give
nodal functions defined on  with values  on one of the points 
and zero on the other points, where  is computed according to
Definition~\ref{def:neighboring-points-1}. Applying these nodal functions as a basis for 
and using Gauss's theorem for surface integration of  over  we get
the discrete mean curvature at the iso-point . For more details on this computation of discrete mean curvature
see \cite{Meyer02Vismath}.




\section*{Acknowledgements}
The authors gratefully acknowledge financial support by the German Research Foundation within the
Priority Program "Transport Processes at Fluidic Interfaces" (SPP 1506).
\newpage
\bibliographystyle{amsplain}
\bibliography{references}
\end{document}
