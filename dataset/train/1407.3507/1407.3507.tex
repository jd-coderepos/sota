\documentclass[runningheads,a4paper]{llncs}
\usepackage{url}
\usepackage{epsf}
\usepackage{graphicx}
\usepackage{amsfonts}
\usepackage{amssymb}
\usepackage{amsmath}
\usepackage{latexsym}
\usepackage{color}
\usepackage{multirow}
\usepackage{setspace, float}
\usepackage{refcount}
\usepackage{hyperref}
\usepackage{makeidx}  


\usepackage{thmtools,thm-restate}



\newcommand{\figref}[1]{\autoref{fig:#1}}



\renewcommand{\sectionautorefname}{Section}
\newcommand{\lemmaautorefname}{Lemma}
\renewcommand{\figureautorefname}{Figure}

\newcommand{\ceil}[1]{\ensuremath{\protect\lceil#1\rceil}}
\newcommand{\CEIL}[1]{\ensuremath{\protect\left\lceil#1\right\rceil}}
\newcommand{\FLOOR}[1]{\ensuremath{\protect\left\lfloor#1\right\rfloor}}
\newcommand{\floor}[1]{\ensuremath{\protect\lfloor#1\rfloor}}
\newcommand{\arr}[1]{\ensuremath{\protect\overrightarrow{#1}}}










\newcommand{\ABox}{
\raisebox{3pt}{\framebox[6pt]{\rule{6pt}{0pt}}}
}


\newcommand{\pp}{\xi}
\newcommand{\red}[1]{\textcolor{red}{#1}}
\newcommand{\ang}{\measuredangle}



\urldef{\mailmd}\path|mirela.damian@villanova.edu|
\urldef{\mailvv}\path|dvoicu@villanova.edu|


\begin{document}

\mainmatter  

\title{Spanning Properties of Theta-Theta Graphs\thanks{This work was supported by NSF grant CCF-1218814.}}
\titlerunning{Spanning Properties of Theta-Theta Graphs}

\author{Mirela Damian and Dumitru V. Voicu}
\authorrunning{M. Damian and D. V. Voicu}

\institute{Department of Computer Science \\ Villanova University, Villanova, PA 19085 \\
\mailmd\\
\mailvv}

\date{}

\maketitle

\begin{abstract}
We study the spanning properties of Theta-Theta graphs. Similar in spirit with the Yao-Yao graphs, Theta-Theta graphs partition the space around each vertex into a set of  cones, for some fixed integer , and select at most one edge per cone. The difference is in the way edges are selected. Yao-Yao graphs select an edge of minimum length, whereas Theta-Theta graphs select an edge of minimum orthogonal projection onto the cone bisector. It has been established that the Yao-Yao graphs with parameter  have spanning ratio , for . In this paper we establish a first spanning ratio of  for Theta-Theta graphs, for the same values of . We also extend the class of Theta-Theta spanners with parameter , and establish a spanning ratio of  for . We surmise that these stronger results are mainly due to a tighter analysis in this paper, rather than Theta-Theta being superior to Yao-Yao as a spanner. We also show that the spanning ratio of Theta-Theta graphs decreases to  as  increases to .  These are the first results on the spanning properties of Theta-Theta graphs. 
\keywords{Yao graph, Theta graph, Yao-Yao, Theta-Theta, spanner}
\end{abstract}

\section{Introduction}
Let  be a set of  points in the plane, and let  be an undirected plane graph with vertex set . The \emph{length} of a path in  is the sum of the Euclidean lengths of its constituent edges. The distance in  between any two points  is the length of a shortest path between  and . We say that  is a \emph{spanner} if it preserves distances between each pair of points in , up to a given factor. Specifically, for a fixed integer , we say that  is a -\emph{spanner} if any two points  at distance  in the plane are at distance at most  in . The smallest integer  for which this property holds is called the \emph{spanning ratio} of . Clearly there is a tradeoff between the spanning ratio and the sparsity of : the smaller the spanning ratio, the denser the spanner and the better the approximation of the original distances. 


One way to control the tradeoff between the spanning ratio and the sparsity of the spanner is to partition the space around each point into equiangular cones of angle , for some integer , and connect each point to a ``nearest'' point in each cone. Intuitively, this construction promises a short detour between any two points , by following the edge from  aiming in the direction of  (the one lying in the cone with apex  containing ). 
The definition of a ``nearest'' point comes in two flavors, in the context of \emph{Yao} graphs~\cite{Yao82} and \emph{Theta}-graphs (or -graphs)~\cite{Clark87,Keil88}. For Yao graphs, the ``nearest'' point is simply the point that minimizes the -distance, whereas for Theta graphs, the ``nearest'' point in a cone  is the point whose orthogonal projection onto the bisector of  minimizes the -distance.  Both Yao and Theta graphs are parameterized by a positive integer , which controls the cone angle . In the following we will refer to the Yao graph as  and Theta graphs as , for a fixed .  Both  and  are known to be efficient spanners, for . The spanning ratios of these graphs are summarized in~\autoref{tab:spanningratios}.


\begin{table}
\begin{center}
\begin{tabular} {|c|c|c|c|c|c|}
\hline
\multirow{2}{*}{Parameter } & \multicolumn{5}{|c|}{Spanning Ratio}  \\
\cline{2-6}
&  & \multicolumn{2}{|c|}{} &  &  \\
\hline \hline
\small{} & \multicolumn{5}{|c|}{\small{~\cite{MollaThesis09}}} \\
\hline
\small{} & \small{~\cite{BDD+10}} & \multicolumn{2}{|c|}{\small{237~\cite{BarbaBD13}}} & \multicolumn{2}{|c|}{\small{~\cite{DMP09}}} \\
\hline
\small{} & \small{~\cite{BarbaBD14}} & \multicolumn{2}{|c|}{\small{~\cite{BoseMRS14}}} &  
\small{OPEN} & \small{~\cite{KX13}} \\
\hline
\small{} & \small{5.8~\cite{BarbaBD14}}  & \multicolumn{2}{|c|}{\small{~\cite{BGH+10}}} & \small{~\cite{MollaThesis09}} & \small{OPEN} \\
\hline
\small{} & \multirow{3}{*}{~\small{\cite{BDDx10}}} & \multicolumn{2}{|c|}{~\small{\cite{RS91}} }
& \raisebox{-0.3em}{\small{~for}} & \raisebox{-0.3em}{\red{\small{~for}}} \\
\cline{1-1} \cline{3-4}
\small{} & & \small{} & \multirow{3}{*}{\small{\cite{BoseRV13,BoseCM+14}}} &  \small{~and} & \red{\small{~and}} \\
\cline{1-1} \cline{3-3}
\small{} & &  & &  \raisebox{0.3em}{\small{}} &  \raisebox{0.3em}{\red{\small{}}}  \\
\cline{1-3}
\small{} & ~\small{\cite{BarbaBD14}} &  & &  \raisebox{0.4em}{\small{\cite{DB13}}} & \raisebox{0.4em}{\small{\red{[HERE]}}} \\
\hline
\end{tabular}
\end{center}
\caption{Spanning ratios of Yao and Theta graphs for various  values.}
\label{tab:spanningratios}
\end{table}

\vspace{-2em}
Interest in Yao and Theta graphs has increased with the advancement of wireless ad hoc networks and the need for efficient communication (see~\cite{RodittyMU08,Chris08,KanjPGe09,CowenLW00} and the references therein). Designing routing algorithms for wireless ad hoc networks is an extremely difficult 
task and research in this area is still in progress. The overlay communication graph formed by the wireless links should be a spanner to ensure fast delivery of information, and should also have low degree to ensure a low maintenance cost and reduced MAC-level contention and interference~\cite{HamR06}. We observe that both Yao and Theta graphs obey the first requirement (as detailed in   \autoref{tab:spanningratios}), but fail to satisfy the second requirement. One simple example consists of  points equally distributed around a circle centered at an n point . Then, for , both  and  will have an edge directed from each of the  points towards , because  is ``nearest'' in one of their cones. So each of  and  has out-degree , but in-degree . To reduce the in-degree, alternate spanner structures based on Yao and Theta graphs have been proposed, such as Yao-Yao~\cite{WL03}, Sink~\cite{LiWanWang01,AryaYY95}, Stable Roommates~\cite{BoseCCCKL13}, and Ordered-Yao~\cite{Song04}. 

The \emph{Yao-Yao} graph with integer parameter , denoted , is a subgraph of  obtained by applying a second Yao step to the set of incoming edges in each cone.  More precisely, for each point  and each cone with apex  containing two or more incoming edges,  retains only a shortest incoming edge and discards the rest. Ties are broken arbitrarily.  This construction guarantees a degree of at most  at each node in  (one incoming and one outgoing edge per cone), however the spanning property of  is still under investigation.  The only existing result shows that , for , is a spanner with spanning ratio . For ,  the spanning ratio of  drops to ~\cite{DB13}. 

\emph{Sink} spanners~\cite{LiWanWang01,AryaYY95} transform bounded outdegree spanners, such as  and , into bounded degree spanners, by replacing  each directed star consisting of all links directed into a point  and lying in a cone with apex , 
by a tree of bounded degree with ``sink'' . The result is a spanner with degree at most  and spanning ratio . 

The \emph{Stable Roommates} spanner introduced in~\cite{BoseCCCKL13} has degree 
at most  and spanning ratio matching the spanning ratio of , so this spanner combines both qualities -- low spanning ratio and low degree -- of the Yao and Yao-Yao graphs, respectively. 
The only drawback of this approach is that it processes pairs of points in non-decreasing order by their distances, making it unsuitable for a fast local implementation. (The authors present a distributed implementation that requires  rounds of communication.) 

The \emph{ordered} Theta approach~\cite{BoseGM04} reduces the potentially linear degree of the Theta graph to a logarithmic degree.  
Similar to the stable roommates approach, the ordered Theta approach imposes a particular ordering on the input points. 
The authors show that careful orderings can produce graphs with spanning ratio  and degree . 




Similar in spirit with the Yao-Yao graph, in this paper we introduce the \emph{Theta-Theta} graph , parameterized by integer , and study the spanning properties of this graph.  The graph  is obtained by applying a filtering step to the edges of  as follows. For each point  and each cone  with apex , we consider all edges in  directed into , and maintain only a ``shortest'' edge while discarding the rest. Recall that in the context of Theta graphs, a ``shortest'' edge minimizes the length of its projection on the cone bisector.  Ties are arbitrarily broken.

Our main result shows that  is a spanner, for any . This result relies on a result by Bonichon et al.~\cite{BGH+10}, who prove that  is a -spanner. Our main contribution is showing that   contains a short path between the endpoints of each edge in . More precisely, we show that for each edge , there is a path between  and  in  no longer than , for . This, combined with the fact that  is a -spanner, yields an upper bound of  on the spanning ratio of . 
A similar approach has been used in~\cite{DB13} to establish that  has spanning ratio , for .  
We observe that the spanning ratio of  decreases to ,  and  as  increases to , , and above , respectively. 
The spanning ratios established in this paper for  are stronger than the ones obtained in~\cite{DB13} for , for the same parameter values . We surmise that this is mainly due to the tighter analysis in this paper, rather than  being superior to  as a spanner.







\subsection{Definitions}
\label{sec:defs}
Throughout the paper,  will refer to a fixed set of  points in the plane. The directed Yao graph  with integer parameter  on  is constructed as follows. For each point , starting with the direction of the positive -axis, extend  equally spaced rays  originating at , in counterclockwise order (see \autoref{fig:defs}a for ). These rays divide the plane into  cones, denoted by , each of angle . To avoid overlapping boundaries, we assume that each cone is half-open and half-closed, meaning that  includes  but excludes  (here  wraps around).  
In each cone of , draw a directed edge from  to its ``closest'' point  in that cone (the one that minimizes the -distance ). Ties are broken arbitrarily. These directed edges collectively form the edge set for the directed Yao graph. The undirected Yao graph (or simply Yao graph) on  is obtained by simply ignoring the directions of these edges. The Theta graph  is defined in a similar way, with the only difference being in the definition of ``closest'': in each cone  with apex , draw a directed edge from  to the point  that minimizes the distance between  and the orthogonal projection of  on the bisector of the cone.  \begin{figure}[pht]
\centering
\begin{tabular}{c@{\hspace{0.1\linewidth}}c}
\includegraphics[width=0.35\linewidth]{ipefigs/conerays.pdf} & 
\raisebox{2em}{\includegraphics[width=0.42\linewidth]{ipefigs/yaovstheta.pdf}} \\
(a) & (b) 
\end{tabular}
\caption{Definitions (a) Rays defining the cones at point  (b) Theta edges , .}
\label{fig:defs}
\end{figure}
For example, looking at the cone  in~\autoref{fig:defs}b, notice that  minimizes the -distance to , whereas  minimizes the -distance between its projection onto the cone bisector and . Consequently,  will be added to , and  to .  Similarly,  will be added to , and  to .
\autoref{fig:yaothetaex}a shows the Yao graph  for the point set depicted in \autoref{fig:defs}b, and  
\autoref{fig:yaothetaex}c shows the Theta graph  for the same point set. 


The Yao-Yao graph  is obtained from  by applying a reverse Yao step to the set of incoming Yao edges in . That is, for each node   and each cone with apex  containing two or more incoming edges,  retains a shortest incoming edge and discards the rest. Ties are broken arbitrarily.  The Theta-Theta graph  is obtained from  in a similar way, with the only difference being in the requirement that a ``shortest'' incoming edge 
in a  cone minimizes the length of its projection onto the cone bisector.
\autoref{fig:yaothetaex}b shows the graph  derived from the graph  depicted in  \autoref{fig:yaothetaex}a, and  
\autoref{fig:yaothetaex}d shows the graph  derived from the graph  depicted in  \autoref{fig:yaothetaex}c. 

When the choice of a particular cone is either irrelevant or is clear from the context, we ignore the cone subscript and use  to denote any of the cones .  For any two points , let  denote the cone with apex  that contains .  Let  be the canonical triangle with two of its sides along the rays bounding , and the third side orthogonal to the bisector of 
 and passing through . For example, shaded in~\autoref{fig:defs}b are the canonical triangles  and . 

\begin{figure}[htbp]
\centering
\begin{tabular}{c@{\hspace{0.1\linewidth}}c}
\includegraphics[width=0.4\linewidth]{ipefigs/yaoexample.pdf} & 
\includegraphics[width=0.4\linewidth]{ipefigs/yaoyaoexample.pdf} \\
(a) & (b) \\
\includegraphics[width=0.4\linewidth]{ipefigs/thetaexample.pdf} & 
\includegraphics[width=0.4\linewidth]{ipefigs/thetathetaexample.pdf} \\
(c) & (d) 
\end{tabular}
\caption{Graph examples (a)  (b)  (c)  (d) .}
\label{fig:yaothetaex}
\end{figure}


For any pair of vertices  and  in an undirected graph , let  denote a \emph{shortest} path in  between  and . 
For example,  refers to a shortest path in  from  to . 


Our main goal is to establish a short path in  between the endpoints of each edge in . Our arguments will rely on the assumption that, for each point , each cone  is entirely contained in , hence . 
Throughout the rest of the paper, will will work with a quadruple of distinct points  in the following configuration:  is an arbitrary edge in ;  is the edge in  that lies in the cone ; and  is the edge in  that lies in the cone . We will refer to this configuration as a \emph{canonical -configuration}, to avoid repeating these definitions in different contexts.
For a snapshot of a canonical -configuration, see ahead to~\autoref{fig:abba}a. 
We will further assume, without loss of generality, that in a canonical -configuration  lies in , and the bisector of  lies below, or aligns with, the bisector of . Any other configuration is equivalent to this canonical -configuration under rotational and/or reflectional symmetry.  

\section{Preliminaries}
\label{sec:basic}
In this section we present a few isolated lemmas that will be used in our main proof from~\autoref{sec:main}. 
For the sake of clarity and continuity in the flow of our exposition, we defer the proofs of most of these lemmas to the appendix. 
We encourage the reader to skip ahead to~\autoref{sec:main}, and refer back to these lemmas from the context of~\autoref{thm:maintheta}, where their role will become evident.
We begin this section with the statement of an existing result. 

\begin{theorem}{\emph{\cite{BGH+10}}}
For any pair of points , there is a path in  whose total length is bounded above by .
\label{thm:theta6}
\end{theorem}
The key ingredient in the result of~\autoref{thm:theta6} is a specific subgraph of , called \emph{half-}. This graph preserves half of the edges in , those belonging to non-consecutive cones. Bonichon et al.~\cite{BGH+10} show that half- is a
\emph{triangular-distance}\footnote{The \emph{triangular distance} from a point  to a point 
is the side length of the smallest equilateral triangle centered at  that touches  and has one horizontal side.} 
Delaunay triangulation, computed as the dual of the Voronoi diagram based on
the triangular distance function. Combined with Chew's proof that any triangular-distance Delaunay triangulation is a -spanner~\cite{Chew89}, this result settles \autoref{thm:theta6}. 
The structure of , viewed as the union of two planar -spanners,  has been used in establishing spanning properties of other graphs as well~\cite{Bon2+10,jDR12,DB13}.

\begin{figure}[htpb]
\centering
\includegraphics[width=0.4\linewidth]{ipefigs/trapezoid.pdf} 
\caption{\autoref{lem:thetapath}: .}
\label{fig:trapezoid}
\end{figure}


\begin{figure}[htpb]
\centering
\begin{tabular}{c@{\hspace{0.05\linewidth}}c}
\includegraphics[width=0.4\linewidth]{ipefigs/canonical.pdf} & 
\includegraphics[width=0.43\linewidth]{ipefigs/abba.pdf} \\
(a) & (b) 
\end{tabular}
\caption{(a) Canonical -configuration: ,  and  (b) Bounding ,  and .}
\label{fig:abba}
\end{figure}


\noindent
Before stating our preliminary results, we define the term ) parameterized by angle  as 

This term will occur frequently in our analysis, and this definition will come in handy. The upper bound of  follows from the fact that ) decreases as  increases, therefore .  
The following lemma plays a central role in the proofs of Lemmas~\ref{lem:paa1} and~\ref{lem:paasecond}. 
\begin{lemma}{\emph{\cite{DB13}}}
Let  and let  and  be the other two vertices of . If  is empty of points in , then . Moreover, each edge of  is no longer than . \emph{[Refer to~\autoref{fig:trapezoid}.]}
\label{lem:thetapath}
\end{lemma}
Note that~\autoref{lem:thetapath} does not specify which of the two sides  and  lies clockwise from , so the lemma applies in both situations. 
The following lemma establishes fundamental relationships on the distances between points in a canonical -configuration. 

\begin{restatable}{lemma}{abbalemma}
\label{lem:abba}
Let  be points in a canonical -configuration. Then each of  and  is no longer than . In addition, if  and  are the angles formed by the horizontal through  with  and the lower ray of , respectively, and if , then 

\emph{[Refer to~\autoref{fig:abba}b.]} 
\end{restatable}

\noindent
Lemmas~\ref{lem:paa1} through~\ref{lem:paa5} isolate specific situations that will arise in the analysis of our main result. 
We state them independently in this section.



\begin{figure}[htbp]
\centering
\begin{tabular}{c@{\hspace{0.05\linewidth}}c}
\includegraphics[width=0.45\linewidth]{ipefigs/casebb.pdf} &
\includegraphics[width=0.45\linewidth]{ipefigs/caseaa.pdf} \\
(a) & (b)  
\end{tabular}
\caption{Bounding  (a)~\autoref{lem:paa1}:  above  (b)~\autoref{lem:paasecond}:  below .} 
\label{fig:casebb}
\end{figure}


\begin{restatable}{lemma}{aalemma}
\label{lem:paa1}
Let  be points in a canonical -configuration, with the additional constraint that . Let  and  be the angles formed by the horizontal through  with  and the lower ray of , respectively. 
Then 
Here the term  is as defined in~\emph{(\ref{eq:T})}. Furthermore, each edge of  and  is strictly smaller than , for . \emph{[Refer to~\autoref{fig:casebb}a.]}
\end{restatable}





\begin{restatable}{lemma}{aasecondlemma}
\label{lem:paasecond}
Let  be points in a canonical -configuration, with the additional constraints that , and the angle  formed by  with the horizontal through  is at most . 
Then 
Furthermore, each edge of  and  is strictly shorter than , for . \emph{[Refer to~\autoref{fig:casebb}b.]}
\end{restatable}

\begin{restatable}{lemma}{aathirdlemma}
\label{lem:paa5}
Let  be points in a canonical -configuration, with the additional constraint that either  
, or  and the angle formed by  with the horizontal through  is above . 
Then \begin{small}

\end{small}Furthermore, each edge of  and  is strictly shorter than , for . \end{restatable}


Our approach to finding a short path in  between the endpoints of each edge in  uses induction on the Euclidean lengths of the edges in . The following lemma will be useful in proving the inductive step in various situations. 

\begin{lemma}
\label{lem:pab}
Let  be points in a canonical -configuration, and let  be a fixed real value.  Assume that, for each edge  no longer than , the inequality  holds. 
Let . 
If  and , then 

Furthermore, if , then 
  for any real value  such that 

Here the symbol  is used to denote the path concatenation operator. 
\end{lemma}
\begin{proof}
Because , each edge on  must be shorter than . This along with the lemma statement implies that, for each edge  on the path , the inequality  holds. Summing up these inequalities for all edges along the path  yields .  Similar arguments show that . Thus the 
first inequality stated by this lemma holds. 
Using the upper bound on  from~\autoref{lem:abba}, and the assumption that 
, this inequality can be easily reorganized into  for any real value  that satisfies~(\ref{eq:t}).
{\hfill\ABox}\end{proof}

\section{ is a Spanner, for }
\label{sec:main}
This section presents our main result, which shows that  is a spanner, provided that  and  (and so . In particular, we show that for each edge , there is a path in  no longer than . This, combined with the result of~\autoref{thm:theta6}, yields our main result that  is a -spanner, for . The spanning ratio decreases to  for ,  which is superior to the spanning ratio of  established in~\cite{jDR12} for , with . We also show that the spanning ratio of  drops to  for . 

Our approach takes advantage of the fact that each edge  is embedded in an equilateral triangle  empty of points in . The restriction  is necessary in our analysis to guarantee that each cone used in constructing  and  is a subset of a cone used in constructing , therefore it inherits a large area empty of 
points in . This property is crucial in establishing a ``short'' path in  between the endpoints of each edge in . Although we search for \emph{undirected} paths in the undirected version of , we sometimes point out the direction of an edge if significant in the context. 

\begin{theorem}
Let  be a positive integer, with . For each edge , a shortest path in  between  and  satisfies , where  is a positive real with values , ,  and  corresponding to  values , , , and above , respectively.
\label{thm:maintheta}
\end{theorem}
\begin{proof}
Recall that , so in the context of this theorem . Throughout this proof will refer to the value  from the theorem statement as the \emph{stretch} factor, with the understanding that it measures the ``stretch'' in  of an edge , and to be distinguished from the spanning ratio of  (which by~\autoref{thm:theta6} is at most 2). 

The proof is by induction on the Euclidean length of the edges in . 
The base case corresponds to a shortest edge .
In this case we show that  and . Assume to the contrary that
 and let   be the edge that lies in . 
~\autoref{lem:paa1} does not impose any restrictions on the relative position of the  and , therefore the result that 
each edge on  is strictly shorter than  applies in this context. 
This contradicts our assumption that  is a shortest edge in . This shows that . 
Similar arguments, used in conjunction with Lemmas~\ref{lem:paa1},~\ref{lem:paasecond} and~\ref{lem:paa5} (which distinguish between different locations of  relative to ), show that .

Our inductive hypothesis states that the theorem holds for all edges in  of length strictly lower than some fixed value . To prove the inductive step, pick a shortest edge  of length  or higher, and find a ``short'' path  that satisfies the conditions of the theorem. Let  and  be the other two points in  which, along with  and , complete a canonical -configuration:  lies in , and 
 lies in . Refer to~\autoref{fig:abba}a. Also recall our general assumptions that in a canonical -configuration , and the bisector of  aligns with, or lies below, the bisector of . 
The locus of  is , which is an area completely inside . 
The locus of  is , which is an area that may overlap two or three of the cones ,  and .  Note that  may not lie in , due to our assumption that the bisector of  is no higher than the bisector of . 

Our intent is to use the result of~\autoref{lem:pab} to establish the existence of a path between  and  of length at most , for some fixed real constant . The two key ingredients needed by~\autoref{lem:pab} are ``short'' paths in  between  and , and between  and . 
We discuss three cases,  depending on whether  lies in ,  or . The case 
 is the simplest, so we will save it for last. 
Let ,  and  be the angles formed by the horizontal through  with , , and the lower ray of , respectively. 

\paragraph{Case .} This case is depicted in~\autoref{fig:casebb}a. By \autoref{lem:paa1}, we have 

where  is as defined in~(\ref{eq:T}). 
Notice the restrictions on the angles  and :

The upper bound on  is due to our assumption that the bisector of  is no higher than the bisector of . The bounds on  follow immediately from the definitions of  and . 
Next we determine a maximum for the quantity on the right hand side of~(\ref{eq:paa1bound}).  
We consider two situations, depending on ranges of , which affect the sign of . Observe that  is always positive, since  for any . 

Assume first that , so  is no higher than the bisector of . 
In this case  and , therefore  is positive. Substituting in~(\ref{eq:paa1bound}) the upper bound on  and the lower bound on  from~\autoref{lem:abba} yields

Let  denote the quantity on the right hand side of the inequality above. Note that  increases as  increases, therefore  for . ~\autoref{fig:paa1plot}a shows how   varies with  and , for fixed . 
\begin{figure}[hptb]
\centering
\begin{tabular}{c@{\hspace{0.1\linewidth}}c}
\includegraphics[width=0.4\linewidth]{ipefigs/paa1-bound1.pdf} &
\includegraphics[width=0.4\linewidth]{ipefigs/paa1-bound2.pdf} \\
(a) & (b) 
\end{tabular}
\caption{Case : upper bound on  for  and  (a)  (b) }
\label{fig:paa1plot}
\end{figure}
It can be verified that 
, for any . This along with~\autoref{lem:pab} yields a stretch factor  for the path in  between  and . 
The stretch factor  decreases with  as shown in the second column of Table~\ref{tab:paa1bound}. 

\begin{table}[hptb]
\begin{center}
\begin{tabular} {|c|c|c|}
\hline
\multirow{3}{*}{} & \multicolumn{2}{|c|}{Case : stretch factor  from~\autoref{lem:pab}} \\
\cline{2-3}
& ~~~~~~~~~~~~~~~~~ &  \\
\hline
 &  &  \\
 &  &  \\
 &  &  \\
 &  &  \\
\hline
\end{tabular}
\end{center}
\caption{Case : real constant  from~\autoref{lem:pab} for various  values.}
\label{tab:paa1bound}
\end{table}

Assume now that , so  lies above the bisector of . 
In this case  is negative, and by~(\ref{eq:gamma}) we have . Substituting in~(\ref{eq:paa1bound}) the upper bound on  and  from~\autoref{lem:abba} yields

Let  denote the quantity on the right hand side of the inequality above.  Because  is positive and  is negative,  is positive and therefore   increases as  increases. It follows that  for . ~\autoref{fig:paa1plot}b shows how  varies with  and , for . It can be verified that 
, for any .
This along with~\autoref{lem:pab} yields a stretch factor  for the path in  between  and . 
The stretch factor  decreases with  as shown in the third column of Table~\ref{tab:paa1bound}. 



\paragraph{Case .} This case is depicted in~\autoref{fig:casebb}b. We discuss two situations, depending on whether  lies above or below the bisector of . Assume first that  is no higher than the bisector of , so . Thus we are in the context of~\autoref{lem:paasecond}, which gives us an upper bound  
, where 

Note that  decreases as  increases, therefore  for any . It can be verified that , for any . 
By~\autoref{lem:pab}, we have 
 
Simple calculations show that the right hand side of the inequality above does not exceed  for any . This bound decreases with  as shown in the second column of Table~\ref{tab:paa2bound}. 

Assume now that  lies above the bisector of , so . Intuitively, this forces  and  to lie close to each other (for sufficiently small  values), and similarly for  and , so we can work with somewhat looser upper bounds without exceeding the spanning ratio established so far. Our context matches the context of~\autoref{lem:paa5}, which tells us that . The bound  increases with , therefore .  This together with~\autoref{lem:pab} yields 
 for any . 
This bound decreases with  as shown in the third column of Table~\ref{tab:paa2bound}. 




\begin{table}[hptb]
\begin{center}
\begin{tabular} {|c|c|c|}
\hline
\multirow{3}{*}{} & \multicolumn{2}{|c|}{Case : stretch factor  from~\autoref{lem:pab}} \\
\cline{2-3}
& ~~~~~~~~~~~~~~~~~~~ &  \\
\hline
 &  &  \\
 &  &  \\
 &  &  \\
 &  &  \\
\hline
\end{tabular}
\end{center}
\caption{Case : real constant  from~\autoref{lem:pab} for various  values.}
\label{tab:paa2bound}
\end{table}
\paragraph{Case .} The bound on  provided by~\autoref{lem:paa5} applies here as well, therefore the analysis for this case is identical to the one for the previous case (with  and  above the bisector of ), yielding the spanning ratios listed in the third column of Table~\ref{tab:paa2bound}. 

To derive the results listed in Tables~\ref{tab:paa1bound} and~\ref{tab:paa2bound}, we worked with a quadruplet of \emph{distinct} points  in a  -configuration. The cases where  and  coincide, or  and  coincide, are special instances of this general case and yield lower stretch factors. 
The results listed in Tables~\ref{tab:paa1bound} and~\ref{tab:paa2bound} 
indicate that the stretch factor is highest when  lies above  and  is below the bisector of . The largest stretch factor value is  for , and it drops to ,  and  for  values 
,  and , respectively. This completes the proof.
{\hfill\ABox}\end{proof}
Combined with the result of~\autoref{thm:theta6}, the result of~\autoref{thm:maintheta} yields the main result of this paper, stated by~\autoref{thm:main} below. 

\begin{theorem} 
\label{thm:main}
The graph , with  and , is a -spanner. The spanning ratio decreases to ,  and  as  increases to , , and above , respectively.
\end{theorem}

\section{Conclusions}
In this paper we present the first results on the spanning property of -graphs. We show that, for any integer , the graph  is a spanner with spanning ratio . The spanning ratio drops to  for , which is superior to the spanning ratio of  established in~\cite{jDR12} for , with . The framework of our analysis seems inadequate to handle all graphs , for all , because it relies on the fact that each cone used in constructing  is a subset of a cone used in constructing . 
It is unclear whether a fundamentally new technique is  required to handle all  graphs, for . Proving or disproving that these graphs are spanners remains the main open problem in this area. 

\bibliographystyle{plain}
\bibliography{spannerbib}



\newpage
\section*{Appendix: Deferred Proofs}
\label{sec:defer}



\subsection{Proof of~\autoref{lem:abba}}

\abbalemma*
\begin{proof}
Let  be the height of the isosceles triangle , and let  be the length of its two equal sides. They are related by . Because both  and  lie inside , their length may not exceed . Also  may not be lower than , since  is on the base of . This implies 
, so the upper bound on  holds.  
Observe now that  and  are similar, and the side length of   does not exceed  (because  lies inside ). Similar arguments can then be used to establish the same upper bound on . 

Let  be the intersection point between  and the right side of 
. Then . By the Law of Sines applied on , we have 
. Let  be the lower right corner of . Note that  (as angle interior to ) and  (as angle exterior to ). Also because ,  is acute, therefore . These together show that , so the lower bound on  holds. This completes the proof. 
{\hfill\ABox}\end{proof}


\subsection{Proof of~\autoref{lem:paa1}}
\aalemma*
\begin{proof}
First we determine an upper bound on . Let  and  be the right and left corners of , respectively. Because  is interior to , the perpendicular from  to the bisector of  intersects the line segment , so  the perpendicular from  to  falls left of . This implies that . 
Note that  meets the conditions of~\autoref{lem:thetapath}, with  empty of points in , therefore   
. 
Let the horizontal through  intersect the left rays of  and  in points  and , respectively. 
Then  and , so we have 

We determine  and  in terms of  by applying the Law of Sines on : 
. Note that ,  therefore both  and 
 are acute.  This along with the fact that  implies , and 
  implies . Combining these inequalities together yields

Next we determine  and  in terms of  by applying the Law of Sines on : 
. Plugging in the angle values 
 and  yields 

Combining inequalities~(\ref{eq:pbb1}),~(\ref{eq:pbb2}) and~(\ref{eq:pbb3}) together yields 

Next we determine an upper bound on .
Let  be the left corner of  (refer to~\autoref{fig:casebb}a.) By~\autoref{lem:thetapath}, .  
Let the horizontal through  intersect the left side of  and the line supporting  in points  and , respectively. Then  and . These together imply  

Note that  and , so the bounds from~(\ref{eq:pbb3}) apply here as well. 
Next we determine  and  in terms of  by applying the Law of Sines on : 
. Because the upper ray of 
 is parallel to the lower ray of ,  we have 
 and . 
Since both angles are acute, we get  
 and . These together imply 

Combining inequalities~(\ref{eq:paa1}),~(\ref{eq:pbb3}) and~(\ref{eq:paa2}) together yields 

This along with~(\ref{eq:pbbfinal}) settles the first part of the lemma. 
We now turn to the second claim of the lemma. By~\autoref{lem:thetapath}, each edge on 
 is no longer than  (cf.~(\ref{eq:pbb3})), and  
each edge on  is no longer than . 
To simplify discussion, let . 
It suffices to show that  in order to settle the second part of the lemma. 
Because the bisector of  is no higher than the bisector of , we have that , therefore . Substituting the upper bound on  from~\autoref{lem:abba} yields 

 It can be verified that the right hand side of this inequality is strictly smaller than , for any . This completes the proof.
{\hfill\ABox}\end{proof}

\subsection{Proof of~\autoref{lem:paasecond}}

\aasecondlemma*

\begin{proof}
We define the following points:  and  are the right and left corners of ;  is the left corner of ;  and  are the points where the right ray of  intersects  and the horizontal through , respectively;  is the point where the line supporting  intersects ; and  is the intersection point between  and . Refer to~\autoref{fig:casebb}b. Arguments similar to the ones used in the proof of~\autoref{lem:paa1} show that . This along with  and  implies 

By~\autoref{lem:thetapath} we have  
. This together with the inequality above and the fact that 
,  yields 

Using the similarity property of  and , we derive  and 
.
Using the Law of Sines on , we derive 
 and 
.
Observe that  (because the ray shooting from  towards , parallel to , lies inside  of angle , and  is equal to the angle formed by this ray with ), and  (as angle exterior to ). It follows that . These together imply

These inequalities along with~(\ref{eq:paa52}) yield
the upper bound on  stated by this lemma.



For the second part of the lemma, it can be verified that the term  is strictly positive for any  and . This along with the upper bound established by this lemma shows that , therefore each edge on each of the paths  and  is strictly shorter than . This completes the proof.
{\hfill\ABox}\end{proof}

\subsection{Proof of~\autoref{lem:paa5}}

\aathirdlemma*

\begin{proof}
The conditions stated by the lemma suggest that either  and  lie close to each other (if ), or  and  lie close to each other (if  is above the bisector of ). Intuitively, the upper bounds established for these two cases must be within a small factor of each other.  

Let  be the intersection point between  and . By the lemma statement  lies below the horizontal through , therefore the point  exists.  
Observe that a ray shooting from  towards , parallel to , lies inside  of angle , and  is equal to the angle formed by this ray with , therefore . By the Law of Sines applied on triangle , we have 
. This along with~\autoref{thm:theta6} and the fact that  implies 

Similarly arguments used on  show that  

Consider first the case where , and  is above the bisector of . In this case  and  (since  is below the horizontal through ). By the definition of a -configuration, the bisector of  lies below the bisector of , therefore the angle formed by  with the bisector of  is at most . It follows that 
 and similarly . These together show that 
 and , which along with~(\ref{eq:small1}) and~(\ref{eq:small2}) yield 

Substituting  the upper bound on  from~\autoref{lem:abba} results in . Thus the upper bound claimed by the lemma holds for this case. 

Assume now that . In this case , and similarly  (because  lies exterior to  and above ). Since neither of these angles can extend as far as , the inequalities 
 and  hold. These along with~(\ref{eq:small1}) and~(\ref{eq:small2}) yield 

This shows that the bound from~(\ref{eq:small3}) established for the previous case applies in this case as well. This settles the first part of the lemma. For the second part, simple calculations show that  for any . This implies that , therefore each edge of  and  is strictly shorter than . This completes the proof.
{\hfill\ABox}\end{proof}
\end{document}
