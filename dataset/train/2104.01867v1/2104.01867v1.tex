

\documentclass[final]{cvpr}
\usepackage{times}
\usepackage{epsfig}
\usepackage[numbers,sort,compress]{natbib}
\usepackage{graphicx}
\usepackage{amsmath}
\usepackage{amssymb}
\usepackage{here}
\usepackage[pagebackref=true,breaklinks=true,colorlinks,bookmarks=false]{hyperref}
\usepackage{pifont}\usepackage{cuted}
\usepackage{capt-of}
\usepackage{tabu}
\usepackage{subfig}
\usepackage{soul}
\usepackage{booktabs}
\newcommand{\cmark}{\ding{51}}

\def\cvprPaperID{5284} \def\confYear{CVPR 2021}
\setcounter{page}{4321} \newcommand{\anh}[1]{{\textcolor{cyan}{[Anh: #1]}}}
\newcommand{\thao}[1]{{\textcolor{blue}{[Thao: #1]}}}
\newcommand{\add}[1]{\textcolor{red}{#1}}
\newcommand{\Sref}[1]{Sec.~\ref{#1}}
\newcommand{\Eref}[1]{Eq.~(\ref{#1})}
\newcommand{\Fref}[1]{Fig.~\ref{#1}}
\newcommand{\Tref}[1]{Table~\ref{#1}}

\pagenumbering{gobble}
\begin{document}
\def\mA{\mathcal{A}}
\def\mB{\mathcal{B}}
\def\mC{\mathcal{C}}
\def\mD{\mathcal{D}}
\def\mE{\mathcal{E}}
\def\mF{\mathcal{F}}
\def\mG{\mathcal{G}}
\def\mH{\mathcal{H}}
\def\mI{\mathcal{I}}
\def\mJ{\mathcal{J}}
\def\mK{\mathcal{K}}
\def\mL{\mathcal{L}}
\def\mM{\mathcal{M}}
\def\mN{\mathcal{N}}
\def\mO{\mathcal{O}}
\def\mP{\mathcal{P}}
\def\mQ{\mathcal{Q}}
\def\mR{\mathcal{R}}
\def\mS{\mathcal{S}}
\def\mT{\mathcal{T}}
\def\mU{\mathcal{U}}
\def\mV{\mathcal{V}}
\def\mW{\mathcal{W}}
\def\mX{\mathcal{X}}
\def\mY{\mathcal{Y}}
\def\mZ{\mathcal{Z}} 

\def\bbN{\mathbb{N}} 
\def\bbR{\mathbb{R}} 
\def\bbP{\mathbb{P}} 
\def\bbQ{\mathbb{Q}} 
\def\bbE{\mathbb{E}} 

 

\def\1n{\mathbf{1}_n}
\def\0{\mathbf{0}}
\def\1{\mathbf{1}}


\def\A{{\bf A}}
\def\B{{\bf B}}
\def\C{{\bf C}}
\def\D{{\bf D}}
\def\E{{\bf E}}
\def\F{{\bf F}}
\def\G{{\bf G}}
\def\H{{\bf H}}
\def\I{{\bf I}}
\def\J{{\bf J}}
\def\K{{\bf K}}
\def\L{{\bf L}}
\def\M{{\bf M}}
\def\N{{\bf N}}
\def\O{{\bf O}}
\def\P{{\bf P}}
\def\Q{{\bf Q}}
\def\R{{\bf R}}
\def\S{{\bf S}}
\def\T{{\bf T}}
\def\U{{\bf U}}
\def\V{{\bf V}}
\def\W{{\bf W}}
\def\X{{\bf X}}
\def\Y{{\bf Y}}
\def\Z{{\bf Z}}

\def\a{{\bf a}}
\def\b{{\bf b}}
\def\c{{\bf c}}
\def\d{{\bf d}}
\def\e{{\bf e}}
\def\f{{\bf f}}
\def\g{{\bf g}}
\def\h{{\bf h}}
\def\i{{\bf i}}
\def\j{{\bf j}}
\def\k{{\bf k}}
\def\l{{\bf l}}
\def\m{{\bf m}}
\def\n{{\bf n}}
\def\o{{\bf o}}
\def\p{{\bf p}}
\def\q{{\bf q}}
\def\r{{\bf r}}
\def\s{{\bf s}}
\def\t{{\bf t}}
\def\u{{\bf u}}
\def\v{{\bf v}}
\def\w{{\bf w}}
\def\x{{\bf x}}
\def\y{{\bf y}}
\def\z{{\bf z}}

\def\balpha{\mbox{\boldmath{}}}
\def\bbeta{\mbox{\boldmath{}}}
\def\bdelta{\mbox{\boldmath{}}}
\def\bgamma{\mbox{\boldmath{}}}
\def\blambda{\mbox{\boldmath{}}}
\def\bsigma{\mbox{\boldmath{}}}
\def\btheta{\mbox{\boldmath{}}}
\def\bomega{\mbox{\boldmath{}}}
\def\bxi{\mbox{\boldmath{}}}
\def\bnu{\mbox{\boldmath{}}}                                  
\def\bphi{\mbox{\boldmath{}}}
\def\bmu{\mbox{\boldmath{}}}

\def\bDelta{\mbox{\boldmath{}}}
\def\bOmega{\mbox{\boldmath{}}}
\def\bPhi{\mbox{\boldmath{}}}
\def\bLambda{\mbox{\boldmath{}}}
\def\bSigma{\mbox{\boldmath{}}}
\def\bGamma{\mbox{\boldmath{}}}
                                  
\newcommand{\myprob}[1]{\mathop{\mathbb{P}}_{#1}}

\newcommand{\myexp}[1]{\mathop{\mathbb{E}}_{#1}}

\newcommand{\mydelta}[1]{1_{#1}}


\newcommand{\myminimum}[1]{\mathop{\textrm{minimum}}_{#1}}
\newcommand{\mymaximum}[1]{\mathop{\textrm{maximum}}_{#1}}    
\newcommand{\mymin}[1]{\mathop{\textrm{minimize}}_{#1}}
\newcommand{\mymax}[1]{\mathop{\textrm{maximize}}_{#1}}
\newcommand{\mymins}[1]{\mathop{\textrm{min.}}_{#1}}
\newcommand{\mymaxs}[1]{\mathop{\textrm{max.}}_{#1}}  
\newcommand{\myargmin}[1]{\mathop{\textrm{argmin}}_{#1}} 
\newcommand{\myargmax}[1]{\mathop{\textrm{argmax}}_{#1}} 
\newcommand{\myst}{\textrm{s.t. }}

\newcommand{\denselist}{\itemsep -1pt}
\newcommand{\sparselist}{\itemsep 1pt}



\definecolor{pink}{rgb}{0.9,0.5,0.5}
\definecolor{purple}{rgb}{0.5, 0.4, 0.8}   
\definecolor{gray}{rgb}{0.3, 0.3, 0.3}
\definecolor{mygreen}{rgb}{0.2, 0.6, 0.2}  


\newcommand{\cyan}[1]{\textcolor{cyan}{#1}}
\newcommand{\red}[1]{\textcolor{red}{#1}}  
\newcommand{\blue}[1]{\textcolor{blue}{#1}}
\newcommand{\magenta}[1]{\textcolor{magenta}{#1}}
\newcommand{\pink}[1]{\textcolor{pink}{#1}}
\newcommand{\green}[1]{\textcolor{green}{#1}} 
\newcommand{\gray}[1]{\textcolor{gray}{#1}}    
\newcommand{\mygreen}[1]{\textcolor{mygreen}{#1}}    
\newcommand{\purple}[1]{\textcolor{purple}{#1}}       

\definecolor{greena}{rgb}{0.4, 0.5, 0.1}
\newcommand{\greena}[1]{\textcolor{greena}{#1}}

\definecolor{bluea}{rgb}{0, 0.4, 0.6}
\newcommand{\bluea}[1]{\textcolor{bluea}{#1}}
\definecolor{reda}{rgb}{0.6, 0.2, 0.1}
\newcommand{\reda}[1]{\textcolor{reda}{#1}}

\def\changemargin#1#2{\list{}{\rightmargin#2\leftmargin#1}\item[]}
\let\endchangemargin=\endlist
                                               
\newcommand{\cm}[1]{}

\newcommand{\mhoai}[1]{{\color{magenta}\textbf{[Hoai: #1]}}}

\newcommand{\mtodo}[1]{{\color{red}\textbf{[TODO: #1]}}}
\newcommand{\myheading}[1]{\vspace{1ex}\noindent \textbf{#1}}
\newcommand{\htimesw}[2]{\mbox{}}




\newif\ifshowsolution
\showsolutiontrue

\ifshowsolution  
\newcommand{\Comment}[1]{\paragraph{\bf  COMMENT:} {\sf #1} \bigskip}
\newcommand{\Solution}[2]{\paragraph{\bf  SOLUTION:} {\sf #2} }
\newcommand{\Mistake}[2]{\paragraph{\bf  COMMON MISTAKE #1:} {\sf #2} \bigskip}
\else
\newcommand{\Solution}[2]{\vspace{#1}}
\fi

\newcommand{\truefalse}{
\begin{enumerate}
	\item True
	\item False
\end{enumerate}
}

\newcommand{\yesno}{
\begin{enumerate}
	\item Yes
	\item No
\end{enumerate}
}

 
\title{Lipstick ain't enough: Beyond Color Matching for In-the-Wild Makeup Transfer}




\author{
Thao Nguyen \quad Anh Tuan Tran \quad Minh Hoai \\
VinAI Research, Hanoi, Vietnam,
VinUniversity, Hanoi, Vietnam,\\
Stony Brook University, Stony Brook, NY 11790, USA\\
{\tt\small \{v.thaontp79,v.anhtt152,v.hoainm\}@vinai.io}
}

\makeatletter
\let\@oldmaketitle\@maketitle

\renewcommand{\@maketitle}{\@oldmaketitle
\vspace{-8mm}
\centering
\includegraphics[width=0.95\linewidth,page=5]{imgs/teaser_3.pdf}
\vskip -0.1in
\captionof{figure}{In-the-wild facial makeup consists of both color transfer and pattern addition. We propose a holistic method that can transfer the color and pattern from a reference makeup style to another image.} \label{fig:feature-graphic}

\vspace{3mm}
}
\makeatletter


\maketitle
\begin{abstract}
Makeup transfer is the task of applying on a source face the makeup style from a reference image. Real-life makeups are diverse and wild, which cover not only color-changing but also patterns, such as stickers, blushes, and jewelries. However, existing works overlooked the latter components and confined makeup transfer to color manipulation, focusing only on light makeup styles. In this work, we propose a holistic makeup transfer framework that can handle all the mentioned makeup components. It consists of an improved color transfer branch and a novel pattern transfer branch to learn all makeup properties, including color, shape, texture, and location. To train and evaluate such a system, we also introduce new makeup datasets for real and synthetic extreme makeup. Experimental results show that our framework achieves the state of the art performance on both light and extreme makeup styles. Code is available at \url{https://github.com/VinAIResearch/CPM}.
   
\end{abstract}


\section{Introduction}
\vspace{-2mm}



Across thousands of years of history, humankind has been fascinated with facial beauty. Humans, particularly females, want to be attractive, and facial appearance is a crucial part of this. The cosmetic industry, as reported in 2007, generates a turnover of about \I_s^nsnI_r^mrmI_s^m\mFI_{s}^{n}, I_{r}^{m}T_{s}^{n}, T_{r}^{m}T_{s}^{m}I_{s}^{m}I\mathcal{UV}ST\mathcal{UV}^{-1}\mathcal{U}\mathcal{V}I_{s}^{n}I_{r}^{m}(S_{s}, T_{s}^{n})(S_{r}, T_{r}^{m})S_{s}S_{r}T_{s}^{n}T_{r}^{m}T_{s}^{m}I_{s}^{m} = \mathcal{UV}^{-1}(S_{s}, T_{s}^{m})\mathcal{C}T_{s}^{m_C}, T_{r}^{n_C} :=\mathcal{C}(T_{s}^{n}, T_{r}^{m})\mathcal{C}\mathcal{L}_{adv}T_{s}^{m_C}T_{r}^{n_C}\mathcal{L}_{cyc}\mathcal{L}_{per}\mathcal{L}_{hist}\mathcal{L}_{hist}HM\lambda^{eyes}, \lambda^{lips}, \lambda^{skin}\mathcal{L}_{hist}^ii\odot\Gamma_{s}^i\Gamma_{r}^ii\Gamma_{s}^i = \Gamma_{r}^i = \Gamma^iT_{r}^{m}\mathcal{L_{DC}} = \frac{2|\Gamma^{gt}\cap \Gamma^{pr}|}{|\Gamma^{gt}|+|\Gamma^{pr}|}\Gamma^{gt}\Gamma^{pr}\Gamma^{m}T_{s}^{m_C}\Gamma^{m}T_{s}^{m_C}I_{s}^{m}I_{s}^{m} = \mathcal{UV}^{-1}(S_{s}, T_{s}^{m})150{\times}150nI_s^nI_r^nI_s^{gt}I_r^mI_s^n, I_r^m, I_s^{gt}\mathcal{UV}\mathcal{C}256{\times}256\lambda_{adv}{=}1\lambda_{cyc}{=}10\lambda_{per}{=}0.005\lambda_{hist}{=}1\lambda_{skin}{=}0.1\lambda_{eyes}{=}1\lambda_{lips}{=}1256{\times}25651\alphaI_s^nI_{r_1}^{m_1}I_{r_2}^{m_2}T_{s}^{m_1}T_{s}^{m_2}\alpha \in [0, 1]$, and render to get the interpolated output. \Fref{fig:interpolation} displays some interpolated results in case one or two reference styles are given. The results are smooth and natural, even in extreme regions such as heavy eye-shadow and cheek color.

\myheading{Partial makeup transfer.} Further exploiting the UV position map, we can use it together with facial segmentation to perform partial makeup transfer. Instead of transferring makeup on the entire face, we can do it on a face region defined by some input mask. This controllable mechanism was proposed in the previous works \cite{beautyglow,jiang2019psgan} and can be easily implemented in our system. \Fref{fig:partial} provides an example in which we transferred makeup partially for the lips, eye shadow, and skin region, then generated a makeup composition on the entire face. 



\section{Conclusion}
\vspace{-2mm}
In this paper, we extend the definition of the makeup transfer task and propose a novel holistic framework to deal with in-the-wild makeup styles. Makeup styles are now interpreted as a combination of color-matching and pattern-addition, respectively, solved by our Color Transfer Branch and Pattern Transfer Branch. UV representation is incorporated to improve the results of both branches. The experiments show our framework can achieve state-of-the-art qualitative and quantitative results. Moreover, we propose novel datasets to leverage  makeup-transfer studies and encourage future development.

{\small
\setlength{\bibsep}{0pt}
\bibliographystyle{plainnat}
\setlength{\bibsep}{0pt}
\bibliography{longstrings,egbib}
}

\end{document}
