Before we present the reachability algorithm, let us first introduce a reachability fragment of CTL that is used in the algorithm. A formula of the logic is given by the abstract syntax: 

where ,  and
.

The semantics of formulae is given in terms of a TLTS  and a labeling function  assigning sets of true atomic propositions to states. We define the set of atomic propositions  and the labeling function  as  and . The intuition is that a proposition  is true in a marking  if the number of tokens in the place  satisfies the proposition with respect to . Since atomic propositions only depend on the discrete part of a marking, we adopt the same definition of  for symbolic markings. For a state  and a formula , we define the satisfaction relation  inductively as follows:




As the  and  temporal operators are dual, it is enough to
design an algorithm for deciding the validity of .
Note that because the predicates do not allow us to query the ages
of tokens in the net, the presence of age invariants in the TAPN model
adds an expressive power (otherwise we could conjunct the age invariants
with the intervals on input arcs and add to the formulae the requirement that
no place contains any token exceeding the invariant bound).

We say that a place  in a boolean predicate 
defined according to \cref{eqn:psi} is \emph{monotonicity-breaking}
if  contains an atomic proposition of the form
, ,  or . In other words, the 
inequality imposes some upper bound or an exact comparison to a
concrete number in the place .

\begin{lemma}\label{lem:inc_empty_all_properties}
Let  and  be symbolic markings and let  be a boolean predicate 
defined by \cref{eqn:psi} and let
the set  of inclusion places do not contain any
monotonicity-breaking place. 
If  and  then .
\end{lemma}
\begin{proof}
By structural induction on . The induction step
is trivial; we discuss here only the base case for a proposition of the formi
.
Let  and  be symbolic markings 
such that .
Let . 

If  is a monotonicity-breaking place than 
and all tokens in the place  belong to the set .
By Condition \ref{def:inclusion-op-bijection-condition} of the inclusion
ordering
there exists a bijection  such that 
for all  
we have  and hence
in the marking 
the number of tokens in the place  is equal to the number
of tokens in the place  in the marking  and
we get .

If  is not a monotonicity-breaking place, the constraint on 
has the form  or .
If the tokens in the place  belong to 
we are done by the arguments as above.
If the tokens in the place  belong to 
then by Condition \ref{def:inclusion-op-inc-count}
of the inclusion ordering 
the number of tokens placed in  in the marking
 is at least the number of tokens in the marking 
 and because the proposition on  states only
a lower-bound, we can again conclude that .
\qed
\end{proof}



In order to present an efficient reachability algorithm, we need a finite representation for the potentially infinite sets of valuations discussed in \cref{sec:discrete-inclusion}. We will thus use DBMs. However, we have to implement the operations used on the sets of valuations, such as delay, clock reset and intersection, on DBMs. Similarly, we need to define a DBM which represents a guard or invariant of the form  where  is a clock and  is a well-formed interval---here 
 is either closed or open left parenthesis and
similarly for .

\begin{proposition}
Let  and  be canonical DBMs over the clocks . Then the following operations and DBMs can be computed efficiently
\begin{enumerate}
\item (Delay)  is a canonical DBM s.t. .
\item (Reset)  where  is a canonical DBM s.t. .
\item (Intersection)  is a canonical DBM s.t. .
\item (Interval DBM)  is a canonical DBM s.t. .
\item (Discrete Inclusion) Let  and  be placement functions. The expression  can be computed efficiently.
\end{enumerate}
\end{proposition}
All these operations can be efficiently implemented for 
DBMs (see e.g. \cite{pettersson_phd, DBLP:conf/ac/BengtssonY03})
for details on operations 1--4; the fifth operator can be implemented using 
DBMs, as showed in the full version of the paper.



We can now perform a standard search through the state-space of symbolic
markings using the passed/waiting list approach. We start by
adding the initial marking to the waiting list. As long as the waiting
list is nonempty, a symbolic marking  is removed from the waiting list,
added to the passed list, and all symbolic extrapolated successors of 
are explored. If a successor  of  is below (w.r.t. the
ordering ) some marking on the passed or waiting
list, we ignore it. Otherwise we add  to the waiting list and
remove from the waiting and passed lists all markings that are below .
We stop with a positive answer once we find a marking satisfying a property
we are searching for.
If the whole state-space is searched without finding such a marking, we
return a negative answer.
The search is performed only upto  tokens in the net where this
number is supplied by the user (it is undecidable whether
there is some  such that the net is -bounded~\cite{memics_paper}).
If the net is -bounded for the given  (this can be automatically
verified) then this gives
a conclusive answer, otherwise the search can give a conclusive
answer only if it finds a marking satisfying the given property.

\begin{algorithm}[p]
\textbf{Name: } \textbf{succ()}\;
\KwIn{A -bounded TAPN  and a symbolic marking .}
\KwOut{The set of successor markings for .}
\Begin{
	successors := \;
	\ForAll{}{
		Let \;
		\ForAll{sets  where  and }{
			Let \;
			 := move(\place{}, \IN, )\;
			\;
			\If{ is consistent}{
			successors := successors \;
			}
		}
	}
	\Return successors\;
}
\caption{Successor generation algorithm\label{alg:simple_successor_generation}}  
\end{algorithm}

\begin{algorithm}[p]
\textbf{Name:} \textbf{Reach()}\;
\KwIn{A marked -bounded TAPN , a formula 
 and a set  not containing any
monotonicity-breaking place in .}
\KwOut{YES if , NO otherwise.}
\Begin{ 
	\;
	Create DBM  such that \;
	\lIf{}{ \Return YES\; }
	\;

	\While{}{
		Remove some  from \;		
		\;
	
			\ForAll{}
			{
				\If{\nllabel{alg_line:discrete_inclusion_subset}}
	{				
\nllabel{alg_line:discrete_inclusion_superset_start}\;
						\; \nllabel{alg_line:discrete_inclusion_superset_end}
\lIf{}{ \Return YES\; }
					\;
				}
			}
		}		
	\Return NO\;
}
\caption{Reachability algorithm\label{alg:reachability}}  
\end{algorithm}
The successor generation algorithm is presented in 
Algorithm~\ref{alg:simple_successor_generation} and the reachability 
algorithm is given in Algorithm~\ref{alg:reachability}.
Observe, as remarked above, that the algorithm will discard any generated successor marking if a larger marking is already present in the PASSED or WAITING list (line \ref{alg_line:discrete_inclusion_subset}). Similarly, if a generated successor marking is larger than some marking in the PASSED or WAITING list, then it will remove all such smaller markings from the PASSED and WAITING list (lines \ref{alg_line:discrete_inclusion_superset_start} to \ref{alg_line:discrete_inclusion_superset_end}). 



\begin{lemma}\label{lem:algorithm_termination}
Algorithm \ref{alg:reachability} terminates.
\end{lemma}
\begin{proof}
Let  be a -bounded TAPN. We must argue that the state-space of the symbolic semantics is finite. Since  is a -bounded TAPN, it follows that there are only finitely many placement functions. Further, from \cref{lem:finite_number_of_extrapolated_dbms} we know that there are only finitely many extrapolated DBMs for a given placement function. Thus, we may conclude that there are only finitely many symbolic markings in the symbolic semantics using the extrapolation operator. Observe that Algorithm \ref{alg:reachability} will add each symbolic marking to the  list at most once. Thus, it follows that the algorithm terminates.
\qed
\end{proof}

\begin{lemma}\label{lem:algorithm_yes_implies_initial_marking_satisfies_property}
If Algorithm \ref{alg:reachability} returns "YES", then .
\end{lemma}
\begin{proof}
Assume that Algorithm \ref{alg:reachability} returns "YES". We must show that  such that . We define  for any placement function  and canonical DBM  (for any set of valuations  that cannot be represented by a DBM we assume ).

Since the algorithm returned "YES", it must have found some symbolic marking  such that . Observe that Algorithm \ref{alg:reachability} (and Algorithm \ref{alg:simple_successor_generation}) will alternate between using the identity abstraction for discrete transition firings and  for time delays. Thus, from the way the algorithm searches through the state-space, we may conclude that there must exist symbolic markings  such that

where . By \cref{lem:extrapolation_below_abstraction_equiv} and \cref{thm:soundness_completeness}, we have that  is sound. Thus, there exists a concrete marking  such that  and . Since atomic propositions depend only on the discrete part of a marking (placement function), it follows that .\qed
\end{proof}

\begin{lemma}\label{lem:init_marking_satisfies_property_implies_algorithm_yes}
If  then 
Algorithm \ref{alg:reachability} returns "YES".
\end{lemma}

\begin{proof}
Assume that . This means that 
 and . We must show that Algorithm \ref{alg:reachability} 
returns "YES".
We define  as before.
By \cref{lem:extrapolation_below_abstraction_equiv} and \cref{thm:soundness_completeness}, we get that  is complete. Thus, there exists a symbolic marking  such that  where  and .

We will now argue that Algorithm~\ref{alg:reachability} will find a symbolic marking   such that . It is easy to see that Algorithm~\ref{alg:reachability} together with Algorithm~\ref{alg:simple_successor_generation} implements a symbolic exploration of the form . 
However, notice that the algorithm discards some of the discovered symbolic markings (lines \ref{alg_line:discrete_inclusion_subset} to \ref{alg_line:discrete_inclusion_superset_end} in Algorithm~\ref{alg:reachability}). If the algorithm finds a symbolic marking  for which  for some  in the PASSED or WAITING list, it will discard  (line \ref{alg_line:discrete_inclusion_subset}). Similarly, if  for some  in the PASSED or WAITING list, it will remove all markings  from both the PASSED and WAITING list (lines \ref{alg_line:discrete_inclusion_superset_start} to \ref{alg_line:discrete_inclusion_superset_end}). However, by \cref{thm:monotonicity_theorem} it is safe to skip these symbolic markings since the future behaviour of the smaller symbolic markings is included in the larger symbolic marking. 
Thus, it follows that Algorithm~\ref{alg:reachability} will find a symbolic marking  such that . 
By \cref{lem:inc_empty_all_properties}, we have that if the smaller marking satisfies  then the larger marking  also satisfies .  
Thus, Algorithm~\ref{alg:reachability} returns "YES".
\qed
\end{proof}










The correctness of the reachability algorithm is hence established.
