\documentclass{ifacconf}

\usepackage{amsmath} \usepackage{amssymb}  \usepackage{MnSymbol}
\usepackage{subfig}
\usepackage{xcolor}


\usepackage{graphicx}      \usepackage{natbib}        

\usepackage{tikz}
\usepackage{pgfplots}
\usepackage{verbatim}
\allowdisplaybreaks
\pgfplotsset{every axis y label/.style={at={(ticklabel cs:0.5)},rotate=90,anchor=center}}
\newtheorem{corollary}[thm]{Corollary}

\begin{document}
\begin{frontmatter}

\title{Geometric tracking control for a nonholonomic system: a spherical robot} 




\author[First]{Sneha Gajbhiye} 
\author[First]{Ravi N. Banavar} 


\address[First]{Systems and Control Engineering, Indian Institute of Technology Bombay,
India, 400076. (e-mail: sneha@sc.iitb.ac.in, banavar@iitb.ac.in)}


\begin{abstract}  This paper presents tracking control laws for two different objectives of a nonholonomic system - a spherical robot - using a geometric approach. The first control law addresses orientation tracking using a modified trace potential function. The second law addresses contact position tracking using a  transport map for the angular velocity error. A special case of this is position and reduced orientation stabilization. Both control laws are coordinate free. The performance of the feedback control laws are demonstrated through simulations.

\end{abstract}

\begin{keyword}
Differential geometry, feedback stabilization, spherical robot.
\end{keyword}

\end{frontmatter}


\section{Introduction}
The problem of tracking of nonholonomic systems is a challenging one in control theory. Applications include robotics, rolling and locomotive mechanisms. A better understanding of  the system's intrinsic properties simplify the control synthesis. Geometric control theory plays an important role in accomplishing such design strategies, see \citep{book_isidori}, \citep{Zenkov}, \citep{ostrowski_thesis}. In this paper we study the tracking problem of one such nonholonomic system - a \textit{spherical robot}.

A spherical mobile robot is a spherical shell actuated by a driving mechanism mounted inside to make the shell roll. In this paper we consider the driving actuators as three rotors. So the robot has three input degrees of freedom (rotors) which are used to control two translation and three rotational degrees of freedom (shell). Several modeling approaches and motion planning algorithms have been proposed for the spherical robot to achieve desired orientation and position,
 see \citep{joshi_banavar}, \citep{reg_n_chaotic_2013}, \citep{mukherjee1999}, \citep{bicchi1997}, \citep{Zhan_2008}, that are solely based on coordinate dependent approach like quaternions and Euler parametrizations. Geometric control addressed the development of control laws for systems evolving on manifolds in a coordinate free setting. Recently, \cite{schneider} has derived the dynamic model of the  Chaplygin's sphere using geometric mechanics and presented  orientation stabilization of a Chaplygin's sphere with a rotor by the controlled Lagrangian matching condition. In \citep{karimpur_2012}, \citep{muralidharan_mahindrakar}, the authors address the control methods based on quaternions and stereographic projection respectively. In \cite{karimpur_2012}, the authors applied backstepping to achieve position stabilization and tracking by expressing attitude in quaternion representation. In \citep{shen2008}, the authors propose motion planning algorithms using symmetric products on manifold (Lie group) to achieve position convergence with arbitrary orientation and vice versa. As spherical robot is a nonholonomic system, it fails to satisfy a necessary condition for asymptotic stabilization on  by a continuous feedback law, see Brockett \citep{brockett}. Due to this negative result, point-to-point stabilization of position and orientation of spherical robot through continuous state-feedback is not possible. In \citep{muralidharan_mahindrakar}, the authors consider stereographic projection map and design smooth kinematic control law to achieve position tracking and position with reduced attitude stabilization. 

The contribution of this paper is to present two geometric control laws to achieve two different objectives: 1) tracking of a desired orientation trajectory; 2) contact position trajectory tracking asymptotically. The intermediate result of contact position tracking law is position and reduced attitude stabilization. The use of the transport map for velocity error on  gives a better and complete understanding of the nonholonomic constraint in case of position tracking. For orientation tracking, a potential function which is the trace of the relative orientation and desired orientation, is constructed. The stability result is derived using Lyapunov direct method \citep{nijmeijer} which is recently restated in \citep{bullo_stability} to achieve asymptotic stability. While the notion of transport map in velocity error has been considered (see, \citep{book_bullo} for tracking of fully actuated and \citep{taeyoung_lee} for underactuated systems), this is the first instance where such a treatment is considered in the presence of a nonholonomic constraint and with underactuation. 

The model is derived using Lagrangian reduction defined on a symmetry group. The well developed theory on geometric nonholonomic mechanics is presented in \citep{marsden}, \citep{CMR}, \citep{CHMR}, \citep{BKMM}, \citep{bloch2003}. By symmetry we can study the dynamics of a mechanical system on a reduced space and the reduced equations are in the Euler-Poincar\'{e} form. Due to nonholonomic constraints the system may or may not have full symmetry as in the case of the rigid body with gravitational field, for example, a heavy top; and the Euler-Poincar\'{e} equation will depend on an advection term \citep{schneider}. In this paper we follow this modelling tool and derive the reduced equations of motion. The paper is organized as follows: In Section 2 we present the description and modelling of the spherical robot using Lagrangian reduction theory. In section 3 we formulate the control problem for orientation tracking and then position tracking and axis stabilization. We identify this stabilization as position and axis stabilization. Section 4 follows with the concluding remarks on the above control strategies.

\section{Description of a spherical robot}
Consider a spherical mobile robot  with internal rotors which can roll without slipping on a flat surface under a uniform gravitational field. All the three rotors are placed along three mutually orthogonal axes of the sphere-body frame, as shown in Fig. (\ref{rotor_fig}). To balance the mass symmetrically, the rotor is placed on one side and a dead weight is placed on the diametrically opposite side. All the rotors and dead weights are placed such that the center of mass of the robot coincides with the geometric center of the sphere. 
\begin{figure}[h]
\centering
\includegraphics[scale=0.3]{Picture1.pdf}
\caption{Spherical robot on horizontal plane}
\label{rotor_fig}
\end{figure}
Let the sphere body coordinate frame be located with its origin at the center of the sphere. Let  be the position of the center of the sphere in an inertial frame, and let  be the rotation matrix which maps from the sphere body coordinate frame to the inertial coordinate frame. The relative motion of three rotors with respect to the sphere body frame is given by generalized shape coordinates ,  where . Hence, the configuration space is , where . The following notation is adopted here:
\begin{itemize}
\item  - Unit vectors in inertial frame,
\item ,  - Angular velocity of the sphere in inertial frame and sphere frame respectively;  - Angular velocity of the  rotor,
\item  - Mass of the sphere and  rotors;  - Inertia matrix of the sphere without rotors about its center of mass in sphere frame and  - Moment of inertia of the rotors about the three principal axes, i.e.  with .
\end{itemize}
The Lagrangian of the system consists only of kinetic energy and is given as

where ,  and . The Lagrangian is now expressed as 

where , ,  . The rolling without slipping assumption on the robot yields a nonholonomic constraint given as

\subsection{Dynamics of the spherical robot} 
The configuration space  is a smooth manifold,  is the velocity space called the tangent bundle and a smooth distribution  defines the constraints, the set of admissible velocities. 
With the Lagrangian  defined in (\ref{lagrange}) and distribution  satisfying the constraint (\ref{constr1}), let  be a group with its Lie algebra , where  is the Lie algebra of . The group action of  on  is given by ; . It is seen that the Lagrangian  and distribution  is invariant with respect to the subgroup  of  given as
. When the Lagrangian  and the distribution  are invariant under the action of the subgroup group , the system is reduced to the space  and the Lagrangian is termed as the reduced Lagrangian . Define 

where  is the velocity of the contact point in the sphere frame and  is called an advected variable \citep{lhmnlc2012}. The reduced Lagrangian  

and the rolling constraint is now expressed in the sphere body coordinate frame as , 
where ,   and  is the (left-invariant) sphere-body angular velocity. Substituting  in , the system is reduced to the quotient space  given by the reduced-constraint Lagrangian  as

Due to subgroup symmetry, there is an advection dynamic and differentiating (\ref{advection}) it is calculated as

The dynamics is calculated using the intermediate theorem given by \citep{schneider}. The equations of motion is given by the Euler-Poincar\'{e} equation for the group variable  and the Euler-Lagrange equation for the shape variable .
Let  be the angular momentum of the sphere(momentum conjugate to ) and  be the angular momentum of the  rotor,the dynamics is given as, 

Recasting the dynamic equation \eqref{equ3} as,

where  and using the solution  of the equation (\ref{reduced_dyanamic}), we can find the curve  by solving the reconstruction equation 

Hence, equations (\ref{constr1}), (\ref{reduced_dyanamic}) and (\ref{final_adv}), together with the reconstruction equation (\ref{rec}), give the complete dynamics of the spherical robot. If , it is easily seen that any configuration is an equilibrium and hence the equilibrium manifold is the whole configuration manifold . By expressing the control in terms of the gradient of a potential function (or error function), the equilibrium can be changed to any desired point. A similar procedure is followed in the next two sections to achieve tracking. 
\subsection*{Remarks on controllability}
The controllability for the three rotor case has been analysed by \cite{reg_n_chaotic_2013}, \cite{joshi_banavar} in the literature. One can use fiber configuration controllability definition to check the controllability. This controllability has been studied for the Chaplygin's sphere with rotors in \cite{shen2008}, \cite{karimpur_2012}. We will mention this result and then design stabilization/tracking control laws for our system. To check the controllability, equation (\ref{reduced_dyanamic}) is cast in an affine-control form as

where, , the drift vector field  and the control vector fields , are expressed as 

where . If   is an equivalent control input, where  is a transformed control input; then the control vector fields on  are written as 

where  is the  column of . Let , then from  (\ref{reduced_dyanamic}), which is the Euler-Poincar\'{e} form, we see 
that is, the inertial momentum  is conserved. Suppose that the system is initially at equilibrium, then  and therefore , where  is called as \textit{mechanical connection} \cite{schneider}. From (\ref{advection}), , then the control vector fields for the complete configuration  are expressed as 

The iterative Lie brackets  span the tangent space of  (termed as the \textit{fiber configuration}) at any configuration. Hence, the system is \textit{fiber} configuration accessible at any configuration and therefore fiber configuration controllable. 
\section{Orientation trajectory tracking}
The control objective here is to design a feedback control law which tracks a desired orientation trajectory .  The rotational system dynamics described by (\ref{reduced_dyanamic}) and (\ref{rec}) can be expressed in the standard control form with  as

where

where for notational simplification we write  and  are the columns of .
We now define a scalar valued potential function to achieve this objective and then prove the stability of the system. Subsequently, we add a damping term to get asymptotic convergence to the equilibrium. Let  be an error function about  constructed by a modified trace function as

where  with  and . The modified trace function was first employ by \citep{chillingworth_marsden_wan} for the purpose of feedback stabilization. Define  and set  and the error in angular velocity as . Taking time derivative of ,

and from the equality , where  is a \textit{hat} map and  is a \textit{breve} map (inverse of hat map), it follows that
 
where  can also be termed as differential of  with respect to .

We compute the feedforward (FF) control term which tracks the desired velocity and add the proportional-derivative (PD) term to stabilize/track the orientation asymptotically. The velocity error has geometric interpretation since  is Lie algebraic element. Since  and  are the two velocities taking values in different tangent spaces, to define the error velocity we need to compare tangent vectors in the same tangent space. This can be achieved by the transport map  as defined in [\citep{book_bullo}, \S 11]. If  and  are the two vectors at the points  and  respectively, then a right transport map  transforms  into a vector at  and the error is expressed as

Now, the feedforward control term is calculated by taking the covariant derivative of the transport map along . Associated with a Riemannian manifold is the notion of the affine connection  that defines the covariant derivative. For details on Riemannain manifolds and affine differential geometry one can refer to \citep{do_carmo}, \citep{book_bullo}. For a given affine connection , two vector fields  with  its Lie algebra, the covariant derivative is defined as . If  are left-invariant vector fields on , then the covariant derivative is , 
where  is a bilinear map defined as 

In our case ,  and  with  and . With this the  is calculated as 

From (\ref{covariant}) calculating the bilinear map and therefore 

\begin{thm}
Under the feedback torque  the closed loop system (\ref{control_equation2}) is Lyapunov stable about .
\end{thm}
\textit{Proof}: Define the function 

where  is the Riemanian metric on . Since,  is an error function and , it follows that the function  is locally positive definite around . It follows

Thus,  is a Lyapunov function about  and therefore  is stable in the sense of Lyapunov for system (\ref{control_equation2}). \\

The next step is to introduce damping or the dissipative term  to achieve asymptotic stability. Introducing damping to the control by defining  
where . Then the closed loop control system becomes 

where  and

\textit{Lemma 1:} The control system (\ref{control_equation3}) is locally controllable on .\\
\textit{Proof}: The proof is given in Appendix.  \\
We now prove asymptotic stability of our system about the desired equilibrium. To prove this we use the stability result stated in [\citep{bullo_stability},theorem 1].
\begin{thm}\label{theorem2}
Consider the system (\ref{control_equation3}) with input torque . Let  be described in (\ref{lyapunov_function}). If  and  is the dissipative input, then the closed loop system asymptotically stabilize .
\end{thm}
\textit{Proof:} Consider the Lyapunov function  as defined in (\ref{lyapunov_function}), Computing the rate of  we get

From (\ref{lyap:2}), we see that,  which implies

Defining  yield . We know that  then calculating  as

where . From this the dissipative control is calculated as

where  is a symmetric positive-definite matrix and  is a positive constant. Substituting the value of  in (\ref{asymptotic_H}) we get 
Since, system is locally controllable from Lemma 1 and  is negative semidefinite we conclude from the Theorem 1 of \citep{bullo_stability} that the point  is local asymptotically stable. \\

\begin{corollary}
 If  implies , then the system (\ref{control_equation2}) with control  is local asymptotically stable about .
\end{corollary}
Infact the system is local exponential stable about
. To check exponential stability, we compute the second variation of . If the second variation is positive definite about the equilibrium point we say that the equilibrium is exponentially stable. From (\ref{lyapunov_function})  where  and for  being a kinetic energy, yields . The second variation of the error functions  is calculated as follows; let , then

At  we have  and , where . Since, the second variation of  is positive definite at equilibrium one can conclude the system achieves the desired orientation exponentially.

\subsubsection*{Simulation:}
\label{section_simu}
We choose the model parameters as:  ; ;
and control parameters as  and . Choosing rotation matrix , the simulations are carried out for both attitude tracking and stabilization by three rotors with the following control law:
 
Keeping the desired orientation trajectory as , then Fig. (\ref{error_angular_velocity}(a)) and (\ref{error_angular_velocity}(b)) shows the error in angular velocity () of the sphere and the error norm of , indicating asymptotic convergence to the desired trajectory. The error norm is calculated as  
For stabilization, setting the desired orientation as  and initial angular velocity as . Then Fig. (\ref{control}(a)) shows the torque applied at the internal rotors. The angular velocity of the three rotors is shown in Fig (\ref{control}(b)) converges to the initial momentum.
\begin{figure}[h]
			\subfloat[]{
\includegraphics[scale=0.41]{rot_ang_vel.pdf}}			
			\subfloat[]{
\includegraphics[scale=0.43]{rot_error.pdf}}
		\caption{(a)Angular velocity of spherical robot. (b) Error norm of orientation}				
				\label{angular_velocity}			
\end{figure}
\begin{figure}
			\subfloat[]{
\includegraphics[scale=0.4]{rot_torque.pdf}}			
			\subfloat[]{
\includegraphics[scale=0.38]{rot_rotor_angle.pdf}}
		\caption{(a) Torque to the internal rotors. (b) Angular position of the rotors. }				
				\label{control}				
\end{figure}
\section{Contact point tracking and axis stabilization}
In this section we derive a control law based on a configuration error function which ensures contact position tracking by tracking the angular velocity. The control objective is to design a control law which aligns  to a desired angular velocity and stabilizes/tracks the contact position on the plane asymptotically. Suppose  is the desired contact point velocity, where  is a desired orientation. Note that given the nonholonomic constraint,  gets determined by  which eventually determine  . Let  be a potential function given by 
Taking the time derivative of  along the system's trajectory,

Set  and define the error in angular velocity as . The proportional-derivative (PD) control term is given as , 
where  and  are positive definite matrices. With this PD control and (\ref{feedfrwd}), the nonlinear controller is given as

\begin{thm}
The system (\ref{reduced_dyanamic}) with control input (\ref{control_track}), given by

is local asymptotically stable at .
\end{thm}
\textit{Proof}: Define a candidate error function

The time derivative of the Lyapunov function is

Let ,where  and since  all the trajectories are bounded and contained within . Define  to be the set of all points  of  satisfying . From (\ref{lyap:1}), we have . As  implies  which yields the dynamics,  and . Since the robot rolls on a horizontal plane, at any point . So the only possibility of  to happen is when . Hence, the largest invariant set will be the set  in . And from LaSalle's invariance principle, the trajectories in  converge to  as , i.e, to the equilibrium . 
\subsubsection*{Position and reduced attitude stabilization:} 
For position stabilization there are two cases: 1) when     ; and 2) when   , where  is any scalar. Case 1 is immediate. When  the control law  and the robot converges to the origin asymptotically. In case 2,  implies that the robot's final contact position is the origin and the angular velocity is about the axis. The control law in this case is expressed as 

The first term in the control law is responsible for the contact position stabilization and remaining terms will orient the sphere such that the angular velocity tracks . Such is a case of reduced attitude stabilization where stabilizing  upto a rotation about  is equivalent to stabilizing the angular velocity direction of the axis  \citep{bullo_murray_track}. Thus, we can restate the attitude as  and conclude that the control law (\ref{control_track}) gives the contact point and reduced attitude stabilization in terms of the points in . 
\subsubsection*{Simulation:}
We take the model parameters as in section (\ref{section_simu}) with initial orientation  and starting point on the horizontal plane as . Setting the desired orientation  and  which satisfy , Fig. \ref{position_track1}(a)) and Fig. (\ref{fig:gam}(a)) shows that as the angular velocity achieve  asymptotically, the sphere attains the desired Line-of-sight that is  converges to . The initial oscillations in  plot are due to the sphere rotating in the spiral type motion on plane and then asymptotic converges to . The position on  plane is illustrated in Fig. (\ref{position_track1}(b)).
\begin{figure}[h]
\centering
			\subfloat[]{
\includegraphics[scale=0.39]{ang_vel_pos.pdf}}			
			\subfloat[]{
\includegraphics[scale=0.45]{posi_xy.pdf}}
		\caption{a)Angular velocity of spherical robot. b) Position on the plane.}				
		\label{position_track1}				
\end{figure}
\begin{figure}[h]
\centering
			\subfloat[]{
\includegraphics[scale=0.45]{posi_gamms.pdf}}			
			\subfloat[]{
\includegraphics[scale=0.45]{posi_torque.pdf}}
			\caption{  plot and torques on rotors.}				
				\label{fig:gam}
\end{figure}
To illustrate contact point trajectory tracking, we choose  to track line and circle, as shown in Fig. (\ref{fig:5}), (\ref{fig:6}) and (\ref{fig:7}). Keeping , then Fig. (\ref{fig:5})(a) and (b) shows the angular velocity of the sphere and the phase plane of position. Setting  which yields . Fig. (\ref{fig:6}) shows the spherical robot follows the desired circular trajectory. Setting  and  then the sphere will rotate at constant speed shown in Fig. (\ref{fig:7}) and follows the line given by .
\begin{figure}[h]
\centering
\includegraphics[scale=0.35]{pos_phase_plane.pdf}
\caption{Phase plane of  position.}			
\label{fig:4}
\end{figure}
\begin{figure}[h]
\centering
			\subfloat[]{
\includegraphics[scale=0.47]{trans_ang_vel_stable.pdf}}			
			\subfloat[]{
\includegraphics[scale=0.45]{trans_torque_point.pdf}}
\caption{Angular velocity and torques on rotors.}				
\label{fig:5}
\end{figure}
\begin{figure}[h]
\centering
			\subfloat[]{
\includegraphics[scale=0.4]{ang_vel_circ.pdf}}			
			\subfloat[]{
\includegraphics[scale=0.5]{transport_circle.pdf}}
\caption{Angular velocity and phase plane of  position tracks the circular trajectory.}			
\label{fig:6}
\end{figure}
\begin{figure}[h]
\centering
			\subfloat[]{
\includegraphics[scale=0.48]{trans_ang_vel_line.pdf}}			
			\subfloat[]{
\includegraphics[scale=0.5]{trans_xy_line.pdf}}
\caption{Angular velocity and phase plane of  position tracks the line trajectory.}			
\label{fig:7}
\end{figure}
\section{Discussion}
In conclusion, we say that both the control strategies derived using the geometric approach, without parametrization,  illustrate a more general philosophy on the control design, preserving the mechanical notions of the system. To the best of our knowledge this is the first instance where such a strategy has been employed to a nonholonomic system. Both the control  strategies are derived using the notions of the affine connection, the error functions and a transport map on tangent spaces. The first feedback strategy results in a continuous feedback law which tracks the desired orientation trajectory. In position tracking strategy, an intermediate result while proving the stability is , which provides an interpretation about feedforward control . The closed-loop system with  has the property that  vanishes along the trajectory. That is, if  is zero at initial time, it will remain zero at final time. And if we keep  we get the  for all time, and the result is contact position and a reduced attitude stabilization. 
\section*{Appendix}
\subsubsection{Proof of Lemma 1:} 
\label{A3}
In this section we will compute the Lie brackets of  and , where . Given two vector field  the Lie derivative (bracket) of  along  is , 
where  is the flow of  and  is pull-back of a vector field . From system (\ref{control_equation3}), the Lie bracket of  and  will be

where  is the flow of . The control vector field   
then flow of  is given as  
and  is the identity map on the manifold .

Similarly,  and  are calculated and given as

where  denotes some functions we are not interested in. The vectors ,, are linearly independent since  are linearly independent. To see the linear independence, we write all the six vectors as

Now, for these vectors to be linearly independent, all the scalars  's and 's equal to zero. From the values of  and  , for (\ref{lin_indep2}) to hold  that, it follows that

for all . Since  is linearly independent, (\ref{lin_indep1}) will hold only when .  And    is linear independent, then  to satisfy (\ref{lin_indep2}). Hence, the set
 
are linearly independent on  of dimensional six and spans the tangent space of the configuration space at any configuration. Therefore, the system is locally controllable. 
\begin{thebibliography}{26}
\providecommand{\natexlab}[1]{#1}
\providecommand{\url}[1]{\texttt{#1}}
\providecommand{\urlprefix}{URL }
\expandafter\ifx\csname urlstyle\endcsname\relax
  \providecommand{\doi}[1]{doi:\discretionary{}{}{}#1}\else
  \providecommand{\doi}{doi:\discretionary{}{}{}\begingroup
  \urlstyle{rm}\Url}\fi

\bibitem[{Bicchi et~al.(1996)Bicchi, Balluchi, Prattichizzo, and
  Gorelli}]{bicchi1997}
Bicchi, A., Balluchi, A., Prattichizzo, D., and Gorelli, A. (1996).
\newblock Introducing the 'sphericle: An experimental testbed for research and
  teaching nonholonomy'.
\newblock \emph{Inter. Conf. on Robotics and Automation}, 36, 2620--2625.

\bibitem[{Bloch(2003)}]{bloch2003}
Bloch, A. (2003).
\newblock \emph{Nonholonomic Mechanics and Control}.
\newblock New York: Springer-Verlag.

\bibitem[{Bloch et~al.(1996)Bloch, Krishnaprasad, Marsden, and Murray}]{BKMM}
Bloch, A., Krishnaprasad, P., Marsden, J., and Murray, R. (1996).
\newblock Nonholonomic mechanical systems with symmetry.
\newblock \emph{Arch. for Rat. Mech. and Anal.}, 136, 21--99.

\bibitem[{Brockett(1983)}]{brockett}
Brockett, R.W. (1983).
\newblock asymptotic stability and feedback stabilization.
\newblock \emph{Differential {G}eometry {C}ontrol {T}heory, {B}irkhauser}, 1,
  181--191.

\bibitem[{Bullo(1999)}]{bullo_stability}
Bullo, F. (1999).
\newblock Stabilization of relative equilibria for underactuated systems on
  {R}iemannian manifolds.
\newblock \emph{Automatica}, 1--29.

\bibitem[{Bullo et~al.(1995)Bullo, Murray, and Sarti}]{bullo_murray_track}
Bullo, F., Murray, R.M., and Sarti, A. (1995).
\newblock Control on the sphere and reduced attitude stabilization.
\newblock \emph{Proc. of the {IFAC} Symposium on Nonlinear control systems
  design ({NOLCOS}).}, 495--501.

\bibitem[{Cendra et~al.(1998)Cendra, Holm, Marsden, and Ratiu}]{CHMR}
Cendra, H., Holm, D.D., Marsden, J.E., and Ratiu, T.S. (1998).
\newblock Lagrangian reduction, the {E}uler-{P}oincar\'{e} equations and
  semidirect products.
\newblock \emph{American Mathematical Society Translation(2)}, 186.

\bibitem[{Chillingworth et~al.(1982)Chillingworth, Marsden, and
  Wan}]{chillingworth_marsden_wan}
Chillingworth, D.J.R., Marsden, J., and Wan, Y.H. (1982).
\newblock Symmetry and bifurcation in three-dimensional elasticity: {P}art {I}.
\newblock \emph{{A}rch. {R}ation. {M}ech. {A}nal.}, 80, 295--331.

\bibitem[{Do~Carmo(2009)}]{do_carmo}
Do~Carmo, M.P. (2009).
\newblock \emph{Riemannian Geometry}.
\newblock ISBN press, Birkhauser, Berlin.

\bibitem[{Gajbhiye and Banavar(2012)}]{lhmnlc2012}
Gajbhiye, S. and Banavar, R. (2012).
\newblock The {E}uler-{P}oincar\'{e} equations for a spherical robot actuated
  by a pendulum.
\newblock \emph{Proc. of 4th IFAC Workshop on Lagrangian and Hamiltonian
  methods for Non-Linear Cont.}, 4, 72--77.

\bibitem[{Holm et~al.(1998)Holm, Marsden, and Ratiu}]{CMR}
Holm, D., Marsden, J., and Ratiu, T. (1998).
\newblock The {E}uler-{P}oincar\'{e} equations and semidirect products with
  applications to continuum theories.
\newblock \emph{Adv. in Mathematics}, 137.

\bibitem[{Isidori(1995)}]{book_isidori}
Isidori, A. (1995).
\newblock \emph{Nonlinear Control Systems}.
\newblock Springer-Verlag New York, Inc., NJ, USA, 3rd edition.

\bibitem[{Joshi and Banavar(2009)}]{joshi_banavar}
Joshi, V. and Banavar, R.N. (2009).
\newblock Motion analysis of a spherical mobile robot.
\newblock \emph{Robotica, Cambridge University Press}, 27, 343--353.

\bibitem[{Karimpour et~al.(2012)Karimpour, Kashmiri, and
  Mahzoon}]{karimpur_2012}
Karimpour, H., Kashmiri, M., and Mahzoon, M. (2012).
\newblock Stabilization of an autonomous rolling sphere navigating in a
  labyrinth arena: A geometric mechanics perspective.
\newblock \emph{Systems and Control Letters}, 61, 495--505.

\bibitem[{Lee et~al.(2010)Lee, Leok, and McClamroch}]{taeyoung_lee}
Lee, T., Leok, M., and McClamroch, N.H. (2010).
\newblock Geometric tracking control of a quadrotor uav on {SE}(3) for extreme
  maneuverability.
\newblock In \emph{Proc. IFAC World Congress}, volume~18, 6337--6342.

\bibitem[{Lewis and Bullo(2005)}]{book_bullo}
Lewis, A.D. and Bullo, F. (2005).
\newblock \emph{Geometric Control of Mechanical Systems}.
\newblock New York: Springer.

\bibitem[{Marsden and Ratiu(1994)}]{marsden}
Marsden, J.E. and Ratiu, T.S. (1994).
\newblock \emph{Introduction to Mechanics and Symmetry}.
\newblock Springer-Verlag, New York.

\bibitem[{Mukherjee et~al.(1999)Mukherjee, Minor, and
  Pukrushpan}]{mukherjee1999}
Mukherjee, R., Minor, M., and Pukrushpan, J. (1999).
\newblock Simple motion planning strategies for spherobot: A spherical mobile
  robot.
\newblock \emph{Proc. of 38th CDC}, 2132--2138.

\bibitem[{Muralidharan and Mahindrakar(2015)}]{muralidharan_mahindrakar}
Muralidharan, V. and Mahindrakar, A. (2015).
\newblock Geometric controllability and stabilization of spherical robot
  dynamics.
\newblock \emph{{IEEE} {T}rans. in {A}uto. {C}ontrol}, PP(99), 1--6.

\bibitem[{Nijmeijer and van~der Schaft(1990)}]{nijmeijer}
Nijmeijer, H. and van~der Schaft, A. (1990).
\newblock \emph{Nonlinear Dynamical Control Systems}.
\newblock New York: Springer-Verlag.

\bibitem[{Ostrowski(1996)}]{ostrowski_thesis}
Ostrowski, J.P. (1996).
\newblock \emph{The mechanics and control of undulatory Robotic Locomotion}.
\newblock Ph.D. thesis, California Institute of Technology, Pasadena,
  California.

\bibitem[{Schneider(2002)}]{schneider}
Schneider, D. (2002).
\newblock Non-holonomic {E}uler-{P}oincar\'{e} equations and stability in
  {C}haplygin's sphere.
\newblock \emph{Dynamical Systems}, 87--130.

\bibitem[{Shen et~al.(2008)Shen, Schneider, and Bloch}]{shen2008}
Shen, J., Schneider, D.A., and Bloch, A.M. (2008).
\newblock Controllability and motion planning of multibody {C}haplygin's sphere
  and {C}haplygin's top.
\newblock \emph{International Journal on Robust and Nonlinear Control}, 18,
  905--945.

\bibitem[{Svinin et~al.(2013)Svinin, Morinaga, and
  Yamamoto}]{reg_n_chaotic_2013}
Svinin, M., Morinaga, A., and Yamamoto, M. (2013).
\newblock On the dynamic model and motion planning for a spherical rolling
  robot actuated by orthogonal internal rotors.
\newblock \emph{Regular and Chaotic Dynamics}, 18, 126--143.

\bibitem[{Zenkov et~al.(1999)Zenkov, Bloch, and Marsden}]{Zenkov}
Zenkov, D., Bloch, A., and Marsden, J. (1999).
\newblock The lyapunov-malkin theorem and stabilization of the unicycle with
  rider.
\newblock \emph{Systems Control Letters}, 45, 293--302.

\bibitem[{Zhan et~al.(2008)Zhan, Zengbo, and Yao}]{Zhan_2008}
Zhan, Q., Zengbo, L., and Yao, C. (2008).
\newblock A back-stepping based trajectory tracking controller for a
  non-chained nonholonomic spherical robot.
\newblock \emph{Chinese Journal of Aeronautics}, 21(5), 472 -- 480.

\end{thebibliography}

















\end{document}
