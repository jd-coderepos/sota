\documentclass{elsart}
\usepackage{natbib,amsmath,amssymb,latexsym,url,graphicx}
\usepackage{graphics}
\usepackage{psfig}
\usepackage{color}
\def\para{\vspace{3mm}}




\def\deg{{\rm deg}}
\def\tdeg{{\rm tdeg}}
\def\card{{\rm Card}}

\def\qed{\hfill  \framebox(5,5){}}
\def\lcoeff{{\rm lcoeff}}
\def\coeff{{\rm coeff}}
\def\mcd{{\rm gcd}}
\def\StHa{{\rm StHa}}
\def\stha{{\rm stha}}
\def\Sres{{\rm Sres}}
\def\Res{{\rm Res}}
\def\ord{{\rm ord}}
\def\pp{{\mathcal  P}}
\def\cc{{\mathcal  C}}
\def\oo{{\mathcal  O}_d({\mathcal C})}
\def\sig{{\rm sign}}

\newtheorem{theorem}{{\bf Theorem}}
\newtheorem{remark}{{\bf Remark}}
\newtheorem{definition}[theorem]{{\bf Definition}}
\newtheorem{corollary}[theorem]{{\bf Corollary}}
\newtheorem{proposition}[theorem]{{\bf Proposition}}
\newtheorem{lemma}[theorem]{{\bf Lemma}}
\newtheorem{example}{{\bf Example}}




\begin{document}


\begin{frontmatter}






\title{Topology of 2D and 3D Rational Curves}



\author[a]{Juan Gerardo Alc\'azar\thanksref{proy}},
\ead{juange.alcazar@uah.es}
\author[b]{Gema Mar\'{\i}a D\'{\i}az-Toca\thanksref{proy}},
\ead{gemadiaz@um.es}




\address[a]{Departamento de Matem\'aticas, Universidad de Alcal\'a,
E-28871 Madrid, Spain}
\address[b]{Departamento de Matem\'atica Aplicada, Universidad de
Murcia,  30100 Murcia, Spain}




\thanks[proy]{Supported by the Spanish `` Ministerio de
Ciencia e Innovacion" under the Project MTM2008-04699-C03-01 }



\begin{abstract}
In this paper we present algorithms for computing the topology of planar and space rational
curves defined by a parametrization. The algorithms given here work directly with the
parametrization of the curve, and do not require to compute or use the implicit equation of the curve (in the case of planar curves) or
of any projection (in the case of space curves). Moreover, these algorithms
have been implemented in Maple; the examples considered and the timings obtained show good performance
skills.
\end{abstract}
\end{frontmatter}

\section{Introduction}\label{section-introduction}

The topology of planar algebraic curves, implicitly given, is a
well-studied problem (see \cite{Arnon}, \cite{gianni},
\cite{LaloCompl}, \cite{Lalo}, \cite{Hong}, and the more recent
works \cite{Eigen}, \cite{seidel}, among others); more recently,
the problem for space algebraic curves has also received certain
attention (see \cite{JG-Sendra}, \cite{Diat}, \cite{ElKa}). In all
these works it is assumed that the curve is given by means of
implicit equations, and the  considered algorithms deal with the
curves in this form. However, in this paper we address the
problem, apparently not discussed up to now, of computing the
topology of a rational curve (i.e. constructing a planar or space
graph describing the shape of the curve) starting directly from
its parametrization, without computing or making use of the
implicit equation of the curve.  This question may be of special
interest in the field of computer-aided geometric design (CAGD), where many of the curves used
are rational and even directly provided in parametric
form (e.g. Bezier curves, B-splines, NURBS).

Perhaps the reason for the absence of previous studies in this
direction is the common belief that if the parametric equations of
a curve are available, the curve is easy to visualize. This is
essentially true, but if the goal is to get a global idea of how
the curve is like, then there are still some difficulties. On the
one hand, one should previously compute a parameter interval such
that the plotting of the curve over the interval shows the main
features of the curve; this includes handling the case of possible
missing points/branches (which happens if some point of the curve
is generated when the parameter of the curve tends to infinity,
see for example \cite{Andradas}). On the other hand, the plotting,
as pointed out by Gonzalez-Vega and Necula in the introduction to
\cite{Lalo}, may not always provide a clear idea of the topology
of the curve, and hence auxiliary tools for describing the shape
of the curve may be of help.

In this paper we address both planar and space rational curves. As in other topology algorithms, we require the
input curves to satisfy certain conditions that can be achieved
with generality. In the planar case it is
required that the curve has neither vertical asymptotes nor
vertical components, and that the parametrization is {\it proper}
(see Section \ref{sec-background}). Initially,
the algorithm works in a similar way to existing
algorithms, i.e. first one computes the critical
points and the points of the curve lying on the lines 
containing some critical point, and then one
appropriately connects these points. However, the connection phase
is carried out not in the usual way, but taking
advantage of the fact that a  parametrization
 is available (see Theorem \ref{th-connect} in
Section \ref{plane-case}). More precisely, the algorithm computes how the {\it parameter} values are
connected; so, two points are joined whenever the
algorithm detects that the parameter values giving rise to them
need to be connected. In particular, and unlike many classical
algorithms, this strategy does not require the curves to be in generic position (as defined in \cite{Lalo}).



The method is specially profitable in the case of space rational
curves. Existing implicit algorithms compute the topology of the
curve by projecting it onto a plane (the -plane, in our case),
and then lifting to space the topology of this projection. In
\cite{JG-Sendra}, this lifting phase is carried out in general by
using a second, auxiliary projection; however, in \cite{Diat},
\cite{ElKa} no auxiliary projection is needed. In any case, the
lifting of the singularities of the projection is a delicate
operation. In our case, we use a similar strategy for 3D curves.
However, here the lifting operation (which is performed without
auxiliary projections) presents no difficulties since the space
points are identified by the parameter values giving rise to them
(previously computed when addressing the projection). In the case
of 3D curves, our requirements are: (i) the curve has no
asymptotes or components normal to the -plane; (ii) the
projection onto the -plane fulfills the requirements of the 2D
algorithm.

We have implemented the algorithms in Maple 13; outputs and
timings of several examples are given in Section \ref{Examp}. In
our implementation we give to the user the option of computing
isolated points of the curve or not. The reason for this is that
isolated points correspond to points generated by complex,
non-real, values of the parameter, and therefore they may not be
of interest for certain users; moreover, the number of isolated
points is certified by means of Hermite's method (see \cite{cox})
and therefore it may be time-consuming.

The structure of the paper is the following. In Section
\ref{sec-background} we provide the necessary background on
rational curves; hence, notions like properness, normality,
critical and singular points are reviewed here, jointly with
related results. In Section \ref{plane-case} we provide the
algorithm for the 2D case. In Section \ref{sec-space}, the
algorithm for the 3D case is given. Finally, in Section
\ref{Examp} we describe some details of the implementation, and we
provide the outputs and timings of different examples in 2D and
3D. The parametrizations used in the examples
are given in Appendix I and Appendix II.


\section{Background on Rational Curves}\label{sec-background}

In this section we briefly recall the background on affine rational curves
that we need in order to develop our results. So, in the sequel
we will consider an affine rational
curve  defined by a rational parametrization

where  and  for all . In our case  or ; so, we will usually write  instead of . Moreover, since the parametrization is assumed to be real, we have that  is a real curve (i.e. that it consists of infinitely many real points), although for theoretical reasons when necessary we will see the curve embedded in . Nevertheless, our goal will always be the description of the shape of its real part.





A point  is {\it reached} by the
parametrization  if there exists 
such that ; in this case, we will also say that
 generates . Notice that the value of the parameter
generating a real point may be either real or complex, and that
there may be points generated by several (real or complex) values
of the parameter. In this sense, we will say that the
parametrization  is {\sf proper} if almost all points
of  are reached by just one value of the parameter
, i.e. if  is injective for almost all the points
of . So, if  is proper then there are
just finitely many points of  generated by several
different values of the parameter, corresponding to the {\it
self-intersections} of the curve. In order to check whether
 is proper, we will use the following criterion. Let

Then, the following theorem, directly deducible from Proposition 7
in \cite{Rubio} (see also Theorem 4.30 in \cite{SWPD}, for the
planar case), holds.

\begin{theorem} \label{th-charact-proper}
The parametrization  is proper iff .
\end{theorem}

On the other hand, we will say that  is {\sf normal} if
every point in  is reached by at least one value of
the parameter, i.e. if
. If  is
not normal, then (see Proposition 4.2 in \cite{Andradas})
there is just one point of  non-reached by the parametrization,
namely the point Notice that  exists
if and only if
 for all . Furthermore, if  exists,
it may still be reached by some (real or complex) value of the parameter. If we denote
,  is reached iff 
Also, observe that if  exists then it is obtained as the limit of a
sequence of real points of , and therefore it cannot be isolated.
 Hence, if  is reached by some value  then it is a self-intersection
 of the curve, because it is a crossing of two branches of the curve, one corresponding to  and the
 other corresponding to . On the other hand,
 if  exists but it is not reached, one can
 reparametrize the curve so that it is reached (see Theorem 7.30
 in \cite{SWPD}). However, reparametrizations may complicate the
 equations of the curve, or bring other difficulties, like
 improperness or the introduction of algebraic numbers. Hence, in our case whenever we meet this
 phenomenon, we will understand that this reparametrization has
 not been performed.

If every point of  is reachable via  by real values of the parameter
 one says that  is -{\sf normal}. We refer to \cite{Andradas}, \cite{SWPD} for a
 thorough study of this phenomenon. If  is not -{\sf normal}, then there exist real
 points  reachable only by complex values of the parameter. Moreover, the following
 result (see  Proposition 4.2 in \cite{Andradas}) clarifies the nature of these points.

\begin{proposition}\label{R-normal}
Let  be a proper parametrization of .
Then ,  is non-reached by any real value of the
parameter if and only if it
is a real isolated point of .
\end{proposition}





\subsection{Critical Points of Planar Rational Curves}\label{add-back}

In the rest of the section we assume that , i.e.
that  is a real rational curve
 parametrized by . Let 
be its implicit equation; then we have the following classical definitions:


\begin{definition} \label{def-crit-points}
A point  is  called: (a) {\sf a critical point}
if ; (b) {\sf a singular
point}, if it is critical and ; (c) {\sf a ramification point} if it is critical, but
non-singular; (d) {\sf a  regular point} if it is not critical.
\end{definition}

One may easily see that ramification points
correspond to those points satisfying that  but , and that
singular points correspond to either points where , or to
self-intersections of the curve. Singularities of a rational parametrization
can be computed directly from the parametrization, without converting to
implicit form. More precisely, the following result holds (see Theorem 10 and Theorem 11 in \cite{Sonia}).
Here, we denote , , and we write
.

\begin{theorem} \label{th-sing-param}
Let  be a parametrization of , and let  be an affine singularity
of , reacheable by some value  of the parameter. Then, .
\end{theorem}

\begin{remark} \label{inf-sing}
If  is reached by some  (in that case it is a
self-intersection of
the curve, and therefore a
singularity,
as we observed before), then  must be a root of 
(see Theorem 10 in
\cite{Sonia}).
\end{remark}


Whenever  is proper, one may
deduce that  is not identically ; therefore, in that situation  has finitely many roots and from Theorem \ref{th-sing-param}, the -values generating reachable singularities are among these roots. Now let us denote the numerator of  by , and let us write the square-free part of  as ; also, let , and let . Then, the following corollary on the real critical points of 
can be deduced.

\begin{corollary} \label{gen-critical}
The real critical points of  are included in the (finite) set consisting of: (i)  (if it exists); (ii) the real points generated by (real or complex) roots of .
\end{corollary}


\section{Computation of the Graph Associated with a Planar Curve} \label{plane-case}

Let  be a planar algebraic curve,
parametrized by


In this section we address the problem of algorithmically
computing a graph  homeomorphic to the curve
. In order to do so, we will follow the usual
strategy widely used in the implicit case (see \cite{Eigen},
\cite{Lalo}, \cite{Hong}, \cite{seidel}):
\begin{itemize}
\item [(1)] Compute the critical points of 
(see Definition \ref{def-crit-points} in Subsection \ref{add-back}). Let  be the -coordinates of
the critical points of ; also, let , .
 \item [(2)] Compute the points of  lying on the
 vertical lines ,  passing through the critical points;
 we will refer to these lines as {\sf critical lines}.
 \item [(3)] For , compute the points of  lying on the vertical line ; similarly for , . We will refer to these lines as ``non-critical" lines.
 \item [(4)] Connect, by means of segments, the points of  lying on each non-critical line, with the points in the critical lines immediately
 on its right and on its left, respectively.
     \end{itemize}

In our case, we will take advantage of the fact that a parametrization of the curve is available; this will be specially useful in order to carry out step (4). Moreover, in order to apply the method presented in this section, we need that certain hypotheses are fulfilled by . These hypotheses are introduced in Subsection \ref{subsec-hyp}. Then, in Subsection \ref{vertices} and Subsection \ref{edges} we show how to compute the vertices and edges, respectively, of the planar graph. Finally, in the last subsection we provide the full algorithm. The reader may find
several examples of the output of this algorithm in Section \ref{Examp}.


\subsection{Hypotheses}\label{subsec-hyp}

In the rest of the section, we assume that the following hypotheses are fulfilled:

\begin{itemize}
\item [(i)]  is proper.
\item [(ii)]  has no vertical asymptotes; in particular, it is not a vertical line.
\end{itemize}

The first hypothesis guarantees that  is traced just
once when following the parametrization . In order to
check this hypothesis, Theorem
\ref{th-charact-proper} can be applied. Moreover, if this hypothesis does not
hold, one can always reparametrize the curve (see Chapter 6.1 in
\cite{SWPD}) so that it is fulfilled. In order to check the second
hypothesis, one can use the following result, which is easy to prove.

\begin{lemma} \label{asymp-plane}
 has a vertical asymptote iff one of the following
conditions occurs: (a)  has some real root which is not a
real root of ; (b)  but
.
\end{lemma}








If  has some vertical asymptote, one proceeds in the following way:

\begin{itemize}
\item If  
has no horizontal asymptotes (which can be checked by
appropriately adapting Lemma \ref{asymp-plane}), then by interchanging
the axes  and  the condition is fulfilled. Notice that this is an affine
transformation, which therefore does not change the topology of the curve.
\item If  has also horizontal asymptotes, then
almost all changes of
coordinates of the type , with , set the curve properly (see Proposition 3.2 in
\cite{LaloCompl}). Observe that if  is proper, the curve  obtained by applying such a transformation is properly
parametrized by .
\end{itemize}

\subsection{Vertices of the Graph.}\label{vertices}

The notable points of  are the real critical points.
Now from Corollary \ref{gen-critical}, we have that these are
among the following points:

\begin{itemize}
\item [(i)]  (if it exists).
\item [(ii)] The points of  generated (via ) by the real roots of the polynomial  in Corollary \ref{gen-critical}.
    \item [(iii)] The real points of  generated (via ) by complex roots of .
    \end{itemize}

    The computation of  is described in Section
    \ref{sec-background}. Moreover, once the real roots of
     are computed, the points in (ii) are obtained by
    evaluating  at these roots. Now we consider as vertices of the graph  not only these points, but
    also the points of  lying on the vertical lines
    containing the points in (i) and (ii). In order to compute these points, we recall
    the definition of the polynomials
, introduced in Section
\ref{sec-background}, and we consider the polynomials

Then, given a point , , the real roots of  provide the -values of the
points lying in the line ; then, the coordinates of those
points can be obtained by evaluating  at these
-values. Observe that we get not only the coordinates, but also
the -values generating the points, via . This is
important for the connection phase.

So, let us consider the points in (iii). If a point in (iii) is
also generated by a real value of the parameter, then it will have
already been computed as a point in (ii). So, if this is not the
case, by Proposition \ref{R-normal} it is an isolated point. Now
these points might
    be computed by seeking complex
    roots of  giving rise (when evaluating ) to real points of .
    However, in the sequel we will provide an alternative way for carrying out this computation,
    that allows to certify the existence or non-existence of this kind of points. For this purpose, we denote a
complex value of the parameter , where  and
, and we represent the complex modulus as . Also, we write
    
and

Then the following result, that can be easily verified, holds. Here, we denote
the result of substituting  in , as .

\begin{lemma} \label{comp-isol}
Let . Then,  is generated
by a complex value of the parameter  if and only if
there exists  satisfying that  is
a real solution of the system

\end{lemma}



In order to certify the number of real solutions of System
(\ref{sistema_pa_2d}) we apply Hermite's method (see for example
\cite{cox}). However, these solutions include the complex values
of the parameter generating real points that are also reached by
real values of the parameter. In order to identify the existence
of those solutions, we compute, also by Hermite's method, the
number of real solutions of the system obtained by adding the
following equations to System (\ref{sistema_pa_2d}):


So, real isolated points of  correspond to solutions of System
(\ref{sistema_pa_2d}) which are not solutions of System
(\ref{sistema_pb_2d}).








\subsection{Edges of the Graph.} \label{edges}

In this section, we address the problem of connecting the vertices
of  (to compute the edges of the graph). For this
purpose, the idea is to introduce between two consecutive critical
lines an intermediate ``non-critical" line, and to connect the
points of  on each ``non-critical" line with the
points of  on the critical line immediately on its
right or on its left. In order to do this, we take advantage of
the fact that a parametrization of the curve is available, and we
connect the points just by comparing the parameters generating the
points in the two vertical lines (one of them critical, and the
other one ``non-critical"). The idea is made precise in the
following theorem. Here, we will consider  as
``generated" by both  and , besides other real
values that may also generate it; as usual,  (resp.
) is considered less (resp. greater) than any other real
number compared with it, and . This result is
illustrated by Figure 1.

\begin{theorem} \label{th-connect}
Let  satisfying that: (i)  (resp.
); (ii) there is no critical line  such that
 (resp. ). Also, let  be a
real point of  lying on the line , generated
by , and let  (including ,
if  belongs to the line ) be the set of real
values generating the real points of .
The following statements are true:
\begin{itemize}
\item [(1)] If , then  must be connected with the point  of  generated by
the least (resp. greatest) element of  which is
greater (resp. less) than .
    \item [(2)] If , then  must be connected with the point  of  generated by the greatest (resp. least) element of  which is less (resp. greater) than .
\end{itemize}
\end{theorem}

{\bf Proof.} We prove (1) for the case when ; the proofs
of (1) for the case , and of (2) in both cases, are
analogous. Now let  be the least element of
 which is greater than . Since by hypothesis
 has no vertical asymptotes, then  must be
connected either with exactly one real point of  generated by a real value of the parameter, or with
. Now, we distinguish two different cases, depending
on whether  belongs to the line , or not. We
begin with the case when  does not belong to .
So,  is connected with a point of  generated by some . First of
all, observe that . Indeed, by hypothesis
 has no vertical asymptotes. Then,  is defined
for every  between  and , and since 
is a quotient of polynomials,  is also differentiable
there. Moreover,  cannot vanish between  and
 because by hypothesis there does not exist any
critical line between  and . Hence, the sign of
 is constant in the interval lying between  and
, and since , then  in that interval; therefore,
 is increasing there. So, since 
we deduce that .

Now our aim is to prove that . For this purpose,
 observe that  because  is
the least element of  greater than ; hence,
we just have to prove that  cannot occur. Assume by
contradiction that . Since
 and  is differentiable along
, by Rolle's Theorem  must vanish at some
point of . However, this is absurd because
 is strictly positive in , which contains
.

Finally, let
us consider the case when  belongs to the line
. If there exists , , with , then  must be connected with
, since otherwise by adapting the above
argument one reaches a
contradiction. On the other hand,
if  is greater than every real element of ,
then  cannot be connected with any other point of  but ; however, since we consider
 generated by , and , the rule also
holds in this case. \qed




\begin{figure}[ht]
\begin{center}
\centerline{}
\end{center}
\caption{Connecting Points}
\end{figure}














\subsection{Full Algorithm} \label{algorit-planar}

The following algorithm {\tt Planar-Top} can be derived from the preceding subsections.

\underline{\tt Planar-Top Algorithm:}

{\sf Input:}  a planar curve , parametrized by

fulfilling: (i)  for ,
 for ; (ii)  is proper; (iii)
 has no vertical asymptotes.

{\sf Output:} a planar graph  homeomorphic to the
curve.

\begin{itemize}
\item [(1)] (Critical Points) Compute the polynomial  in Corollary \ref{gen-critical}, and the real roots of . Then, compute:
 \begin{itemize}
 \item [(1.1)] The critical points of  (by evaluating  at the real roots of ). Store these points in a list For each of these points, store its coordinates, and the list of real -values generating them.
 \item [(1.2)] The point  (if it exists), and the list of -values generating it.
 \end{itemize}
\item [(2)] (Points of  on Critical Lines)
\begin{itemize}
\item [(2.1)] For  from 1 to , compute the real points of  lying on the line . Store these points in a list
    For each of these points, store its coordinates, and the list of real -values generating them.
\item [(2.2)] Check whether  belongs to some of the above lines . In the affirmative case, go to (3); otherwise,
compute the real points of  lying on . Store these points in a list
For each of these points, store its coordinates, and the list of real -values generating them.
    \end{itemize}
    \item [(3)] (Points of  on Non-Critical lines)
    \begin{itemize}
    \item [(3.1)] Let , , be the set consisting
    of the -coordinates of the critical points computed in (1.1) and (1.2). Also, let , ,
    and for  from 1 to
     let .
        \item [(3.2)] For  from 0 to , compute the real points of  lying on the line ; store these points in a list
            For each of these points, store its coordinates, and the list of real -values generating them.
        \end{itemize}
    \item [(4)] (Edges)
    \begin{itemize}
    \item [(4.1)] For  from 0 to , connect the points of  lying on  and  by applying Theorem \ref{th-connect}.
        \item [(4.2)] For  from 1 to , connect the points of  lying on  and  by applying Theorem \ref{th-connect}.
            \end{itemize}
            \item [(5)] (Isolated vertices) Compute the real isolated points of the curve, and add them to the graph.
\end{itemize}

We will provide several examples of the output of the algorithm in Section \ref{Examp}.


\section{Computation of the Graph Associated with a Space Curve}\label{sec-space}

In this section we let  be a real curve, parametrized by

where . In the sequel, we consider the problem of algorithmically computing a graph  homeomorphic to . In order to do that, we will follow the strategy used to address the implicit case in \cite{JG-Sendra}, \cite{Diat}, \cite{ElKa}; more precisely, we need to perform the following steps:
\begin{itemize}
\item [(1)] Project the curve onto a coordinate plane (the -plane, in our case)
 \item [(2)] Compute the graph  associated with the projection (by using the algorithm given in Section \ref{plane-case})
     \item [(3)] Lift the graph  of the projection, to get .
     \end{itemize}
     As in \cite{Diat} and \cite{ElKa}, here we will require just one projection
     in order to perform the lifting phase. Now in the following subsections we first describe the hypotheses that we request on the input curve (essentially, that it is properly parametrized, and that it is ``correctly placed" in space); then, we present the ideas and results needed for computing the vertices and edges of the graph, and finally we provide the full algorithm.



\subsection{Hypotheses}\label{hyp-3d}

Since  is rational, if it is not a line parallel to
the -axis, then its projection onto the  plane, denoted as
, is an algebraic rational curve and can
be parametrized by

Thus, in the sequel we assume that the following hypotheses hold:

\begin{itemize}
\item [(i)]  has no asymptotes parallel to the -axis (in particular, it is not normal to the -plane).
\item [(ii)]  is a proper parametrization of .
\item [(iii)]  has no asymptotes parallel to the -axis.
\end{itemize}

In particular, hypotheses (ii) and (iii) imply that the graph of
 can be computed by using the Planar-Top
Algorithm. Now if the  parametrization  of  is not proper, then  cannot be a proper
parametrization of  either; then, in
particular (ii) implies that  is properly
parametrized. However, the converse does not necessarily hold,
i.e. it can happen that  is proper, but  is
not. From Section \ref{plane-case}, we know how to check
hypotheses (ii) and (iii). In order to check hypothesis (i), the
next lemma, analogous to Lemma \ref{asymp-plane}, can be applied.

\begin{lemma} \label{asint-space}
 has an asymptote parallel to the -axis iff one of the following conditions happen: (a)  has some real root, which is not a real root of ; (b)  but , .
\end{lemma}

Moreover, hypothesis (i) implies the following relationship between the points , . Recall from Section \ref{sec-background} that they are the only points of  and , respectively, that {\it may} not be reached by any complex value of the parameter.

\begin{lemma} \label{inf-points}
Assume that hypothesis (i) holds. Then,  exists iff  exists, and .
\end{lemma}

{\bf Proof.} If  exists, then it is clear that
 exists and is the projection of .
Conversely, if 
then  exists because   has no
asymptotes. \qed

On the other hand, hypothesis (ii) leads to the following result. Here, the notion of {\it birationality} arises; essentially, the projection of  is said to be {\it birational} if there are not two different branches of  whose projections overlap (see Chapter 5 in \cite{cox-1} for further information on birationality).





\begin{theorem} \label{lem-third-hyp}
Assume that  is not a line parallel to the -axis. Then, if  is proper, the projection of  onto the -plane is birational. Conversely, if  is proper and the projection of  onto the -plane is birational, then  is proper.
\end{theorem}

{\bf Proof.} Let us see . For this purpose, let , , satisfying that there are at least two different points  projecting onto . Since , by Lemma \ref{inf-points} none of these points is , and hence both are reached by . Let  be the -values generating , respectively. Then, , and thus  is generated by two different values of the parameter. But since  is proper, this can only happen for finitely many points, and thus the projection is birational. Conversely, given any , , not generated by any root of  (notice that we are excluding finitely many points), the -values reaching  are exactly those ones generating the points of  that are projected onto . Since  is proper, almost all points of  are generated by just one value of the parameter. And since the projection is birational, we conclude that almost all points of  come from just one point of , and therefore almost all points of  are generated by just one value of the parameter. So,  holds. \qed

It is well-known that almost all affine transformations of the
type  transform 
so that its -projection is birational. So,
if  is proper, almost all of these transformations set
 proper. Moreover, if  is not proper
there exist reparametrization algorithms (see \cite{Tomas},
\cite{Seder}). Therefore, in the sequel we will assume that the
above hypotheses hold.

\subsection{Definition of the Space Graph.}\label{graph-3d}

By assuming the hypotheses of the preceding subsection hold, we
can compute the graph  associated with
 with the Planar-Top Algorithm described
in Section \ref{plane-case}. Hence, in the following we will
assume that this process has already been carried out.

Now we make precise the definition of the graph 
that we want to compute.

\begin{definition} \label{graph-assoc-3d}
Let  be a space curve in the above conditions. Then, the {\sf graph associated with }, , is the following graph:
\begin{itemize}
\item [(i)] Its vertices are the real points of  giving rise (by projection) to the vertices of .
    \item [(ii)] Its edges are the result of ``lifting" to space the edges of , i.e. of computing, for each edge  of , an space segment  corresponding to the branch of  giving rise (by projection) to .
        \end{itemize}
        \end{definition}

        Hence, we have to lift to space the vertices and edges of  in order to compute . Let us see that this lifting operation is well-defined.


\begin{theorem} \label{th-vert-lift}
Every vertex of , except perhaps the isolated vertices, lifts to at least one real space point of .
\end{theorem}

{\bf Proof.} Every point of  fulfills one of the following conditions: (1)
; (2) there exists  satisfying that
; (3)  does not fulfill (2), but there exists
 such that . In the first case,
is lifted to  by Lemma \ref{inf-points}.  If 
belongs to the second group, then it is lifted to 
because  has no asymptotes parallel to the -axis.
Finally, if  belongs to the third group then it is an isolated
point of ; in this case,  comes from a
real point of  iff . \qed

\begin{remark} \label{rem-isol}
Real isolated points of  may come from real isolated points of , or from points of  whose -coordinate is complex. In any case, thanks to hypothesis (i) they do not come from branches of  normal to the -plane.
\end{remark}


Now let us consider the lifting of the edges of the planar graph.
The next result guarantees that, under the considered hypotheses,
this lifting process can be always carried out. In particular, it
implies that there are no real branches of  coming from complex components of  (which is a
phenomenon that in general can happen when working with space
algebraic curves; see for example p. 734 in \cite{JG-Sendra}).

\begin{theorem} \label{th-edges-lift}
Under the considered hypothesis, for every edge  of  there exists one and just one branch of  giving rise to .
\end{theorem}

{\bf Proof.} Let  be an edge of . By
construction of the graph provided in Section \ref{plane-case}, ,
if  exists, it is always included as a vertex of
. So there exists a real open interval
 such that  generates the real branch
of , that we denote by ,
corresponding to . On the other hand, for every  we
have that  must be defined, because otherwise 
has an asymptote parallel to the -axis. Then,  is
defined for every , and gives rise to a real connected
branch of  projecting as .
Furthermore, since  is proper by hypothesis, the
projection onto the -plane is birational by Theorem
\ref{lem-third-hyp}. Hence, there are just finitely many points of
 giving rise, by projection, to the same point of
; but none of these points can give rise
to a point of , because such a point would create a
singularity of  which would split 
into two different edges, and  is already an edge of
. Then, we conclude that 
lifts to a unique connected real branch of . \qed

\subsection{Computation of the Vertices}\label{vertices-3d}
From Definition \ref{graph-assoc-3d}, this process is the lifting of the vertices of
. From the construction of the planar
graph, one may see that for each vertex  of
 the algorithm stores the real values
 of the parameter generating it. For a
fixed ,  is well-defined for ,
since otherwise  has an asymptote parallel to the
-axis. Hence,  is lifted to the space points
 Furthermore, if  exists, then it is lifted to  and to the space points reached by
the real values of the parameter generating , if any. Proceeding this way, the only remaining
space vertices are the real isolated ones (which, by Proposition \ref{R-normal}, are generated
by complex values of the parameter). So, in the rest of the subsection we consider this kind of points.

From Theorem \ref{th-vert-lift}, the real isolated vertices of
 do not necessarily come from real
isolated points of  (since they may be the
projection of complex space points). Conversely, a real isolated
point of  does not necessarily project as an
isolated point of , because its projection
may coincide with the projection of some other real point of
 which is not isolated. However, the next result
ensures that isolated points of  always project as
vertices of ; therefore, these points are
computed when lifting the planar vertices.

\begin{lemma} \label{lemma-isol}
Let  be a real isolated point. Then,  is a vertex of .
\end{lemma}

{\bf Proof.} If  is an isolated point of
, then the statement is true. Otherwise,
there exists a point  in a real branch of 
such that . Observe that  cannot be
 because it is isolated. Therefore, suppose that  it
is reached via  by . Now we
distinguish the cases  or ,
respectively. If , then 
with . Thus,  is generated via
 by two different values of the parameter, namely
, and since  is proper,  is a
self-intersection of . Hence, it is a
singularity of , and the statement
follows. Finally, if  then
 and therefore it is also a vertex of
. \qed

Then, we might recover isolated singularities of  by determining the complex values of the parameter that generate (by projection) vertices of , and by computing those real points of  which are generated by those values. Nevertheless, in the sequel we consider an alternative method, analogous to that in Subsection \ref{vertices}. For this purpose, the following lemma is needed. Here, we denote a complex value of the parameter  as , where  and . Also, we write

and

Then the following
result, analogous to Lemma \ref{comp-isol}, holds. Here,  denotes the result of substituting  in . As in Lemma \ref{comp-isol}, by applying the following result one computes a finite set of complex points which contains the complex points generating the isolated singularities of the space curve.

\begin{lemma} \label{isol-points-3d}
{Let . Then,  is generated by
a complex value of the parameter  if and only if
there exists  satisfying that  is
a real solution of the system }


\end{lemma}


As in the planar case, one can certify the number of real
solutions of the system by Hermite's method; also, one can
construct another system whose solutions correspond to complex
values of the parameter generating points that are also reached by
real values of the parameter, and proceed as in
the 2D case.

\subsection{Computation of the Edges}\label{edges-3d}

The method consists of the lifting of the edges of
. So, let  be an edge of
; by Theorem \ref{th-edges-lift},  is
lifted to an space edge . In order to
compute , the crucial observation is that the computation
of the edges of  is in fact done by
connecting not points, but values of the parameter . Hence,
each edge  can be identified with a pair 
where  belong to , and where the vertices of
 defining  are ,
 (see also Figure 1); here,
. Hence,  is lifted to the
space segment connecting the points ,
; also, . Notice that this idea works perfectly when 
is the projection of several real points of .














\subsection{Full Algorithm}

The following algorithm {\tt Space-Top} can be derived from the preceding subsections.

\underline{\tt Space-Top Algorithm:}

{\sf Input:} a space curve , parametrized by 
fulfilling: (i)  for ,  for ; (ii)  has no asymptotes parallel to the -axis; (iii)  is proper; (iv)  has no asymptotes parallel to the -axis.

{\sf Output:}  a space graph  homeomorphic to .

\begin{itemize}
\item [(1)] (Projection) Compute the graph  associated with the projection  of  onto the -plane, parametrized by , by applying {\tt Planar-Top}.
    \item [(2)] (Lifting phase)
    \begin{itemize}
    \item [(2.1)] (Vertices) For each vertex of : if  is generated by  where , then  lifts to the points ;  , if it exists, lifts to , and we write .
        \item [(2.2)] (Edges) For each edge of : if  is identified (according to Subsection \ref{edges-3d}) with , where , then it is lifted to the space edge obtained by connecting  by means of a segment.
            \end{itemize}
    \item [(3)] (Isolated vertices) Add to  the real isolated singularities of .
    \end{itemize}

 Several examples of the output of this algorithm are presented in the next section.



\section{Experimentation and Examples} \label{Examp}

The algorithm has been implemented in \texttt{Maple 13}, and the examples
run on an Intel Core 2 Duo processor with speeds revving up to
1.83 GHz. The implementation allows the option of computing
isolated points, or not. The reason for introducing this option is
that the number of isolated points is certified by means of
Hermite's method, and this method may be costly.

On the other hand, the user can decide the number of digits used
in the computation. Suppose we denote such a number by . Then,
when running the algorithm, the computing starts using  digits.
However, if the algorithm detects that the number of points in a
vertical line is not the right number, the precision is
automatically increased by 5 more digits and the whole process
starts again. In our experimentations, we usually set , the
default value of \texttt{Digits} variable in \texttt{Maple}, and
in the implementation, the number of digits is limited to a
maximum of 500, although we have never needed more than 70 digits.





Next, we first present examples of the 2D algorithm. In Table 1, we include, for each curve, the degree of the
parametrization (i.e. the maximum exponent of the parameter in the
numerators and denominators of the components of the
parametrization, ), the total degree of the implicit equation
(), the number of terms of the implicit equation (n.terms),
 the
timings in seconds corresponding to the graph without computing isolated
points () or computing them (), and the number of digits
used in the computations. The parametrizations corresponding to
these examples are given in Appendix I.

\begin{center}
\begin{tabular}{|c|l|l|c|c|c|c|c|} \hline
Example &  &  & n.terms  &  &  & Digits  \\
\hline \hline 1 & 3 & 6 & 16  & 0.359 & 1.672 & 10  \\
\hline 2 & 8 & 8 & 25  & 0.891 & 1.078 & 10  \\
\hline 3 & 8 & 8 & 9  & 10.250 & 71.172 & 40  \\
\hline 4 & 4 & 4 & 7  & 0.109 & 0.110 & 10 \\
\hline 5 & 6 & 6 & 28  & 0.203 & 11.859 & 10 \\
\hline 6 & 8 & 8 & 21  & 0.171 & 2.50 & 10 \\
\hline 7 & 23 & 23 & 335  & 49.797 &  h. & 10 \\
\hline 8 & 6 & 12 & 49  & 13.625 &  h. & 10 \\
\hline 9 & 17 & 17 & 171  & 1.656 &  h. & 10 \\
\hline
\end{tabular}

{\bf Table 1:} 2D Examples.
\end{center}


One may notice that as the degree increases (see Example 7 or
Example 9) the computation of the isolated points turns very
costly. An alternative for those cases could be to detect isolated
points directly by checking the existence of complex values of the
parameter corresponding to real singular points; users interested in certifying
rigourously the number of isolated singularities, can choose to apply
Hermite's method later.

The pictures corresponding to the examples in Table 1 can be found
in Figure 2; from left to right we have Examples 1, 2, 3 in the
first row, 4, 5, 6 in the second row and 7, 8, 9 in the third one.
Examples 2 and 6 are the offsets of the cardioid and of the
cubical cusp, respectively; furthermore, Example 4 corresponds to
the epitrochoid. Notice that the curves in Examples 2, 3 and 4 are
not
in generic position.


\begin{figure}[ht]
\begin{center}
\centerline{}
\end{center}
\caption{Examples of the 2D algorithm.}
\end{figure}


Finally, we present examples of the 3D algorithm. In Table 2, for each curve we include: the degree of the
parametrization (), the total degree of the implicit equation
of the projection onto the -plane (), the number of terms
of this projection (n.terms), the timing without computing
isolated points (), the timing including the
computation of isolated points (), and the number of digits
used. As in the 2D-case, in all cases the computations start with
10 digits, and the algorithm increases the number of digits when
it is necessary. The parametrizations corresponding to these
examples are given in Appendix II.

\begin{center}
\begin{tabular}{|c|l|l|c|c|c|c|l|} \hline
Example &  &  & n.terms  &  &  & Digits  \\
\hline \hline 1 & 8 & 8 & 38  & 5.578 & 6.188 & 30  \\
\hline 2 & 10 & 10 & 65  & 3.516 & 3.297 & 10 \\
\hline 3 & 21 & 21 & 234  & 4.453 & 4.515 & 10 \\
\hline 4 & 4 & 7 & 8  & 0.657 & 0.625 & 10 \\
\hline 5 & 6 & 6 & 28  & 0.437 & 0.266 & 10 \\
\hline 6 & 8 & 4 & 5  & 0.141 & 0.109 & 10 \\
\hline 7 & 4 & 4 & 15  & 0.125 & 0.500 & 10 \\
\hline 8 & 12 & 12 & 91  & 1.015 & 0.875 & 10 \\
\hline 9 & 16 & 16 & 142  & 74.00 & 74.094 & 65\\
 \hline
\end{tabular}

{\bf Table 2:} 3D Examples.
\end{center}

The pictures corresponding to these curves can be found in Figure
3. The diamond in each picture points out the origin of the system
of coordinates; moreover, in Example 7 we have not plotted the
axes for the isolated point to be better appreciated.


\begin{figure}[ht]
\begin{center}
\centerline{}
\end{center}
\caption{Examples of the 3D algorithm.}
\end{figure}



\section{Acknowledgements}
We want to thank Prof. J.R. Sendra for his excellent ideas and the
time and energy he devoted to us. We also want to thank Prof.
Gonz\'alez-Vega for suggesting the problem and discussing it with
us.

\newpage

\begin{thebibliography}{56}

\bibitem{Arnon} Arnon
D.,  MacCallum S. (1988). {\it A polynomial time algorithm for the
topology type of a real algebraic curve}, Journal of Symbolic
Computation, vol. 5 pp 213-236.

\bibitem{JG-Sendra} Alcazar J.G., Sendra R. (2005) {\it Computation of
the Topology of Real Algebraic Space Curves}, Journal of Symbolic
Computation 39, pp. 719-744.





\bibitem{Tomas} Andradas C., Recio T., Sendra J.R. (1997) {\it A relatively optimal
reparametrization algorithm through canonical divisors}, Proceedings ISAAC 97, ACM press, pp. 349-355.

\bibitem{Andradas} Andradas C., Recio T. (2007) {\it Missing points and branches of real parametric curves}, Applicable Algebra in Engineering and Computing 18 (1-2), pp. 107-126



\bibitem{cox-1} Cox D., Little J., O'Shea D. (1992). {\it Ideals, Varieties and Algorithms}. Springer.

\bibitem{cox} Cox D., Little J., O'Shea D. (2005). {\it Using Algebraic Geometry}. Second Edition. Springer.

\bibitem{Diat} Diatta D.N., Mourrain B. and Ruatta O. (2008) {\it On the Computation of the Topology of a Non-Reduced Implicit Space Curve}. In Proceedings ISSAC 2008, ed. David Jeffrey, pp. 47-55.

\bibitem{Eigen} Eigenwilling A., Kerber M., Wolpert N. (2007) {\it Fast and Exact Geometric Analysis of Real Algebraic Plane Curves}, in C.W. Brown, editor, Proc. Int. Symp. Symbolic and Algebraic Computation, pp. 151-158, Waterloo, Canada. ACM.

\bibitem{ElKa} El Kahoui M. (2008) {\it Topology of Real Algebraic Space Curves}. Journal of Symbolic
Computation vol. 43, pp. 235-258.

\bibitem{gianni} Gianni P.,
Traverso C. (1983). {\it Shape determination of real curves and
surfaces}, Ann. Univ. Ferrera Sez VII Sec. Math. XXIX pp 87-109.

\bibitem{LaloCompl} Gonzalez-Vega L.,   El
Kahoui M. (1996). {\it  An improved upper complexity bound for the
topology computation of a real algebraic plane curve}, J.
Complexity 12 pp 527-544.

\bibitem{Lalo} Gonzalez-Vega L., Necula I. (2002).
{\it Efficient topology determination of implicitly defined
algebraic plane curves}, Computer Aided Geometric Design, vol. 19
pp. 719-743.

\bibitem{Hong} Hong H. (1996). {\it An effective
method for analyzing the topology of plane real algebraic curves},
Math. Comput. Simulation 42 pp. 571-582

\bibitem{Sonia} P\'erez-D\'{\i}az S. (2007). {\it Computation of the singularities of parametric plane curves}, Journal of Symbolic Computation, vol. 42, pp. 835-857.



\bibitem{Rubio} Rubio R., Serradilla J.M., V\'elez M.P. (2008). {\it Detecting real singularities of a space curve from a real rational parametrization}, Journal of Symbolic Computation, etc.




\bibitem{Seder} Sederberg T.W. (1986). {\it Improperly parametrized rational curves}, Computer Aided Geometric Design 3, 67-75.

\bibitem{seidel} Seidel R., Wolpert N. (2005) {\it On the Exact Computation of the
Topology of Real Algebraic Curves}. Proc. of the 21st Ann. ACM Symp. on Comp. Geom. (SCG 2005). ACM, 2005 107--115.

\bibitem{S02} Sendra J.\,R. (2002).
{\it Normal Parametrizations of Algebraic Plane Curves}.
 Journal of Symbolic Computation vol. 33, pp. 863--885.







\bibitem{SWPD} Sendra J.R., Winkler F., Perez-Diaz P. (2008). {\it Rational Algebraic Curves}, Springer-Verlag.





\end{thebibliography}


\newpage

\section*{Appendix I: Parametrizations of the planar curves used in
the experimentation}\label{appen}

\underline{Example 1:}




\underline{Example 2:}

\\


\underline{Example 3:}



\underline{Example 4:}




\underline{Example 5:}




\underline{Example 6:}




\underline{Example 7:}



\underline{Example 8:}



\underline{Example 9:}




\section*{Appendix II: Parametrizations of the space curves used in
the experimentation}\label{appen-2}

For each example, will use the notation
.

\underline{Example 1:}







with


\underline{Example 2:}







\underline{Example 3:}





\underline{Example 4:}




\underline{Example 5:}







\underline{Example 6:}



\underline{Example 7:}







\underline{Example 8:}







\underline{Example 9:}









\end{document}
