\documentclass[copyright,creativecommons]{eptcs}
\usepackage{amsmath,amssymb,amsthm}
\usepackage{graphicx}
\usepackage{alltt}
\usepackage{subfig}
\usepackage{boxedminipage}
\usepackage{afterpage}


  \newcommand{\defeq}{\ensuremath{\stackrel{def}{=}}}
  \newcommand{\arrow}{\longrightarrow}
\renewcommand{\partial}{\rightharpoonup}
  \newcommand{\proves}{\vdash}
\renewcommand{\models}{\vDash}

\DeclareMathOperator{\id}{id}
\DeclareMathOperator{\rem}{rem}

\newcommand{\comment}[1]{}
\newcommand{\todo}[1]{\marginpar{\bf\large}{\sf[#1]}}
\newcommand{\angles}[1]{\langle #1 \rangle}
\newcommand{\eg}{\textit{e.g.,}}
\newcommand{\ie}{\textit{i.e.,}}
\newcommand{\Eg}{\textit{E.g.,}}
\newcommand{\Ie}{\textit{I.e.,}}
\newcommand{\etc}{\textit{etc.}}
 \newcommand{\E}{\mathcal{E}}
\newcommand{\N}{\mathbb{N}}
 \graphicspath{./figs}
\providecommand{\event}{RTRTS 2010} 

\DeclareMathOperator{\rd}{rd}
\DeclareMathOperator{\src}{src}

\title{Using the PALS Architecture to Verify a Distributed Topology Control
Protocol for Wireless Multi-Hop Networks in the Presence of Node Failures
\thanks{This research was supported in part by the Boeing Corporation, grant C8088-557395, and by the NSF, grant CNS 08-34709}
}
\author{
  Michael Katelman and Jos\'{e} Meseguer
  \institute{Department of Computer Science \\
             University of Illinois at Urbana-Champaign \\
             Urbana, IL 61801, USA}
  \email{\{katelman,meseguer\}@uiuc.edu}
}

\def\titlerunning{Using PALS}
\def\authorrunning{Michael Katelman and Jos\'{e} Meseguer}

\begin{document}
\maketitle

\begin{abstract}
The PALS architecture reduces distributed, real-time asynchronous system design
to the design of a synchronous system under reasonable requirements. Assuming
logical synchrony leads to fewer system behaviors and provides a conceptually
simpler paradigm for engineering purposes. One of the current limitations of
the framework is that from a set of independent ``synchronous machines'', one
must compose the entire synchronous system by hand, which is tedious and
error-prone. We use Maude's meta-level to automatically generate a synchronous
composition from user-provided component machines and a description of how the
machines communicate with each other. We then use the new capabilities to
verify the correctness of a distributed topology control protocol for wireless
networks in the presence of nodes that may fail.
\end{abstract}

\section{Introduction}

The design and verification of a distributed embedded system (DES) such as
those in avionics, cars, medical systems, and sensor networks, poses serious
challenges for formal verification, particularly for model checking
verification,  for at least two reasons: (i) their real-time nature has to be
taken into consideration in their modeling and verification, and this usually
makes verification harder or may require restrictions such as the use of
time-bounded properties; (ii) their distributed nature can easily cause an
explosion in the size of the state space, making it infeasible for a model
checker to verify a system.

The Physically Asynchronous but Logically Synchronous (PALS) architectural
pattern \cite{meseguer_2010_01,meseguer_2009_01,sha_2009_01,miller_2009_01} has been recently
introduced to greatly reduce the design, verification, and implementation
efforts for a large class of DES systems, including many in avionics
applications, which can be conceptually conceived to operate in a synchronous
way, but which are in fact implemented as asynchronous systems.  Up to now, the
design of such systems has been very labor-intensive and error-prone, and their
formal verification has been infeasible due to state space explosion even for
modest-sized systems.  The essence of the PALS idea is to allow the designer
and formal verifier to specify the system as a \emph{synchronous composition of
abstract machines}, and to then \emph{automatically derive} from this
synchronous design a corresponding asynchronous version which is correct by
construction.

Conceptually, PALS can be understood as a \emph{model transformation},  which
takes as arguments both the simpler synchronous design and a collection of
\emph{performance-related upper bounds}, such as the maximum clock skew in an
underlying clock synchronization algorithm, the maximum network delay for
message transmission between any two components, and the maximum computation
time for an abstract machine to perform a one-step transition.  The result of
the PALS transformation is the corresponding asynchronous system that will
correctly realize the synchronous design with a provable minimal time period of
operation.

For model checking verification purposes, the great advantage of using PALS is
that the two difficulties (i)--(ii) described above for DES model checking
verification are greatly reduced.  Difficulty (i), their real-time nature,
completely goes away, since the synchronous system can be viewed as a
\emph{single abstract machine} obtained by composing the different machines
connected together in the design.  Such a single abstract machine, together
with its environment specification, can then be treated as an ordinary Kripke
structure \emph{with no explicit real-time features}, and can therefore be reasoned
about with standard model checkers.  Difficulty (ii) is enormously reduced,
because what in the synchronous model corresponds to a \emph{single state
transition} is achieved in its asynchronous version by possibly many transitions
(at least one per distributed component), with possibly many interleavings.
This means that the synchronous model will typically have many fewer states
than the asynchronous one, so that it may be often possible to model check the
synchronous model while it is infeasible to do so for its asynchronous version.
Because of the \emph{semantic equivalence} between a synchronous design and its
PALS asynchronous transform, both systems satisfy the same temporal logic
properties, so that it is enough to verify the much simpler synchronous design.

For the moment, the potential of PALS has not yet been fully exploited at the
formal specification and verification levels in languages such as Maude and
Real-Time Maude.  For this to happen, two important \emph{theory
transformations} need to be supported and automated, namely: 
\begin{enumerate}
\item the transformation performing the \emph{synchronous composition} of a
collection of abstract machines as specified by a \emph{wiring diagram}
connecting those machines; and

\item the PALS transformation itself, which takes the collection of such
machines and their wiring together with the performance parameters and produces
the equivalent asynchronous model.  
\end{enumerate} 
Thanks to the reflective features of rewriting logic and their Maude support by
its \texttt{META-LEVEL} module, transformations (1) and (2) can be specified
within Maude as reflective module transformations.  Note, however, that for
model checking verification purposes, since only the synchronous model needs to
be verified, \emph{only transformation} (1) \emph{is needed}.

This paper addresses this need for supporting PALS in Maude and Real-Time Maude
and makes the following contributions:
\begin{itemize}
\item It provides a meta-level  implementation in Maude of transformation (1)
which is both parametric on the wiring diagram and generic on the actual
abstract machines that may then be composed according to the specified diagram.

\item It applies the PALS transformation for the first time to an application
in the area of sensor networks, illustrating how it can be used to greatly
simplify the formal verification of  a topology control protocol (LMST)
\emph{in the presence of failures}, so that some of the nodes may fail.

\item It illustrates by the LMST example a more general method by which
real-time distributed object-based systems can be modeled in a much simpler
synchronous way using the PALS architecture, provided that the objects in the
system in question are supposed to \emph{only communicate with each other at
pre-established times}, and change their state at those times only as a result
of the messages they then receive.  
\end{itemize}

The remainder of the paper is organized as follows. Section \ref{sec:pals}
reviews the PALS framework, in particular the synchronous model of design.
Section \ref{sec:sc} then describes our generic implementation in Maude for
combining many independent synchronous machines into one large machine,
accomplishing transformation (1) above. In Section \ref{sec:lmst} we then
describe the LMST topology control protocol, our modeling of it in PALS using
the methods of Section \ref{sec:sc}, and prove its correctness with respect to
node failure using Maude's LTL model checker. Finally, Section
\ref{sec:conclusions} contains some concluding remarks and discusses future
work.

\section{Background: The PALS Synchronous Model}
\label{sec:pals}

To design a distributed, real-time system with the PALS architecture
\cite{meseguer_2010_01,meseguer_2009_01,sha_2009_01,miller_2009_01}, one starts with a
\emph{synchronous} model of the system; then, a very general
\emph{transformation}, formally specified in \cite{meseguer_2010_01,meseguer_2009_01}, takes the
synchronous model into an asynchronous one suitable for deployment. A
certain kind of \emph{bisimulation} between the two systems (see
\cite{meseguer_2010_01,meseguer_2009_01}) allows one to reduce (a) verification of a property
against the asynchronous machine to (b) verification of the property against
the synchronous design; where the properties in question are given by temporal
logic formulae. The synchronous machine typically has a much smaller state
space and can therefore be model checked more efficiently.

In this paper we are concerned only with the \emph{synchronous model} (for
information on the asynchronous model and transformation, see
\cite{meseguer_2010_01,meseguer_2009_01}), the key notions of which are \emph{synchronous
machines}, \emph{environment}, and \emph{wiring diagram}. Consider the small
logic circuit given in Figure \ref{fig:circuit}. In terms of the synchronous
model, it is comprised of three synchronous machines,  -- , an
environment defined by the unconnected wires at the boundary box, and a wiring
diagram specifying that the two xor gates get their inputs from the environment
and send their outputs to the and gate, which finally outputs its result to the
environment.

\begin{figure}
\begin{center}
\includegraphics[scale=0.5]{graphics/circuit.pdf}
\end{center}
\caption{A simple logic circuit.}
\label{fig:circuit}
\end{figure}
 
Formally, a component machine  is defined as a four-tuple, , where , specifies the inputs to the machine, , specifies the outputs of the machine,
 is its internal state, and  is its transition function, specifying
how the state is updated and what outputs are produced from a given input and
current state. Each of the component input types  and
component output types  are assumed to be non-empty; for
technical reasons, it is also assumed that .

An environment is a pair  with , the set of inputs \emph{to} the environment,
and , the output
\emph{from} the environment.  Again, each of the
 and  are
assumed to be non-empty.

Let  be a () indexed set of machines and  an environment. A \emph{wiring diagram} for 
and  is a function , where 

that maps each input port to the output port, or \emph{source}, from which it
receives data. Ports exist as part of the component machines, , or as part of the environment, .

Returning to the example in Figure \ref{fig:circuit}, each of the machines
 --  have as input set , and as output set ;
they have no internal state, but because the  component is required to be
non-empty we represent the internal state with the unit type, we we denote by .
The output part of their transition functions perform exclusive-or in the case
of  and , and logical and in the case of . The input type of the
environment is , since there is one input to the environment, and
output type , since the environment furnishes four values to the
machine. The wiring diagram, , is given by

The composed machine operates just like the logic circuit, with values
propagating through one level of logic gate per round. This could, for example,
represent the time it takes for values to propagate through the logic
elements to the output of the circuit.

\section{Automatic Synchronous Composition}
\label{sec:sc}

We now describe the infrastructure we have built in Maude to compose a set of
synchronous machines into one larger machine, as illustrated above in Section
\ref{sec:pals} when we composed the three logic gates into a circuit. Given (a)
for each machine , values  specifying the number of inputs and
outputs, respectively, of , (b) values  for the number of inputs
and outputs of the environment, and (c) a wiring diagram, we automatically
generate a \emph{parameterized module} \cite[Ch. 10]{clavel_2007_01}
corresponding to the synchronous composition of a set of abstract (
unspecified) machines composed according to the given wiring diagram. This is
accomplished using Maude's \emph{meta-level} \cite[Ch. 14]{clavel_2007_01}. The
module has a fixed topology, but is \emph{generic} in the operation of the
individual synchronous machines. The discussion in this section assumes a firm
understanding of parameterized and meta-level programming in Maude (see
\cite{clavel_2007_01} for a review).

Parameterized programming in Maude uses the notions of ``\emph{theories}'' and
``\emph{views}'' (see \cite[\S 8.3.1]{clavel_2007_01} and \cite[\S 8.3.2]{clavel_2007_01}). Theories are
used to specify a parameter's interface, and views are used to instantiate an
interface. Theories are like regular functional and system modules in Maude,
except that they do not need to satisfy the same conditions for executability.
In general, they also may omit constructors for defined sorts or equations
defining the declared operators since these will be mapped to other sorts and
functions later, using a \emph{view}. However, they can state \emph{axioms}
that any instance of the symbols in the parameter theory must satisfy.

For (a) above, the user provides a term of sort {\tt Machines}:

\begin{center}
\begin{small}
\begin{boxedminipage}{0.85\textwidth}
\begin{verbatim}
including MAP{NzNat,IOSize} * (sort Map{NzNat,IOSize} to Machines) .
\end{verbatim}
\end{boxedminipage}
\end{small}
\end{center}

\noindent
presumed to give a mapping from a prefix of the non-zero natural numbers,
isomorphic to the index set , to pairs of numbers , the number
of inputs and outputs of , respectively. The sort for this pair of numbers
is {\tt IOSize}, defined with the following constructor

\begin{center}
\begin{small}
\begin{boxedminipage}{0.50\textwidth}
\begin{verbatim}
op _#_ : NzNat NzNat -> IOSize [ctor] .
\end{verbatim}
\end{boxedminipage}
\end{small}
\end{center}

With a term of sort {\tt Machines}, it is possible to generate the set of
parameters for the .  We simply iterate through the mappings,
creating a new \emph{parameter} for each  pair requiring a view for
a theory 

\begin{center}
\begin{small}
\begin{boxedminipage}{0.75\textwidth}
\begin{verbatim}
  op smParams : Machines NzNat -> ParameterDeclList .
  eq smParams(MS, J) = 
      if ==hasMapping} is pre-defined in the {\tt MAP} module; it
determines whether a given term has a mapping, and we use it to determine when
we are finished iterating through the  modules.

We assume a set of theories, n_jm_j, giving a
general interface for a synchronous machine with  inputs and 
outputs; the general form of {\tt SM--} is given in Figure
\ref{fig:sm}, it specifies the component input and output sorts, product types
for the input and output, projection functions, and a split transition
function. Figure \ref{fig:sminstance} shows how to instantiate a view of {\tt
SM-2-1} corresponding to a ``synchronous machine'' for the two xor gates given
above in Figure \ref{fig:circuit}. The {\tt TUPLE} module operation provides a
very general way to create product types (see \cite[\S
18.3.1]{clavel_2007_01}); it only works in Full Maude \cite[Part
II]{clavel_2007_01}, but by a slight abuse of notation we employ it in Figure
\ref{fig:sminstance}, and throughout this document, as if it can be used in
Core Maude \cite[Part I]{clavel_2007_01}; also, we call the projection
functions {\tt pi1}, \etc\, instead of {\tt p1\_} which is used in \cite[\S
18.3.1]{clavel_2007_01}. In addition, it is worth noting that while it would be
nice to use the {\tt TUPLE[\_]} module operation in the {\tt SM--}
theories, parameterized theories are not currently allowed in Maude.

\begin{figure}[ht]
\begin{center}
\begin{boxedminipage}{0.55\textwidth}
\begin{small}
\begin{alltt}
fth SM-- is
  sorts Di Di-1  Di- .
  op `(_,,_`) : Di-1  Di- -> Di [ctor] .
  op pi1 : Di -> Di-1 .
  eq pi1(( X1:Di-1,  )) = X1:Di-1 .
  
  op pi : Di -> Di- .
  eq pi(( , X:Di- )) = X:Di- .
  sort S .
  sorts Do Do-1  Do- .
  op `(_,,_`) : Do-1  Do- -> Do [ctor] .
  op pi1 : Do -> Do-1 .
  eq pi1(( X1:Do-1,  )) = X1:Do-1 .
  
  op pi : Do -> Do- .
  eq pi(( , X:Do- )) = X:Do- .
  op delta1 : Di S -> S .
  op delta2 : Di S -> Do .
endfth
\end{alltt}
\end{small}
\end{boxedminipage}
\end{center}
\caption{The ``functional theory'' for a synchronous machine with  inputs and  outputs.}
\label{fig:sm}
\end{figure}
 
Consider again the example of Figure \ref{fig:circuit}. The corresponding {\tt
Machines} is given by

\begin{center}
\begin{small}
\begin{boxedminipage}{0.50\textwidth}
\begin{verbatim}
  1 |-> 2 # 1, 2 |-> 2 # 1, 3 |-> 2 # 1
\end{verbatim}
\end{boxedminipage}
\end{small}
\end{center}

\noindent
and the value produced by {\tt smParams} is

\begin{center}
\begin{small}
\begin{boxedminipage}{0.60\textwidth}
\begin{verbatim}
  'M1 :: 'SM-2-1 , 'M2 :: 'SM-2-1 , 'M3 :: 'SM-2-1 
\end{verbatim}
\end{boxedminipage}
\end{small}
\end{center}

\begin{figure}[ht]
\subfloat[Synchronous machine for an xor gate.]{
\begin{boxedminipage}{0.48\textwidth}
\begin{small}
\begin{verbatim}
fmod XOR2 is
  including BOOL .
  including UNIT .
  including TUPLE[2]{Bool,Bool} * 
    (sort Tuple{Bool,Bool} to Bool^2) .
  including TUPLE[1]{Bool} * 
    (sort Tuple{Bool} to Bool^1) .

  var I : Bool^2 .
  var S : Unit .
  op delta1 : Bool^2 Unit -> Unit .
  eq delta1(I, S) = * .
  op delta2 : Bool^2 Unit -> Bool^1 .
  eq delta2(I, S) = 
      ( pi1(I) xor pi2(I) ) .
endfm
\end{verbatim}
\end{small}
\end{boxedminipage}
 }
\subfloat[View of the xor gate in the appropriate theory.]{
\begin{boxedminipage}{0.45\textwidth}
\begin{small}
\begin{verbatim}
view Xor2 from SM-2-1 to XOR2 is
  sort Di-1 to Bool . 
  sort Di-2 to Bool .
  sort S to Unit .
  sort Do-1 to Bool .
endv
\end{verbatim}
\end{small}
\end{boxedminipage}
 }
\caption{Instantiating a synchronous machine for the xor gate using a view.}
\label{fig:sminstance}
\end{figure}
 
The values , provided by the user and corresponding to (b) above,
determine the interface of the \emph{environment}, just as the 
determined the interface to the component synchronous machines.  Similar to the
{\tt SM--} theories, we assume theories {\tt E--} for the
environments. These are exactly the same as the {\tt SM--} theories,
except that they omit the sort {\tt S} and the transition functions {\tt
delta1} and {\tt delta2}. Therefore, to generate the header for the module
giving the synchronous composition of a set of machines we can simply use

\begin{center}
\begin{small}
\begin{boxedminipage}{0.80\textwidth}
\begin{verbatim}
op scHeader : Machines IOSize -> Header .
eq scHeader(MS, N # M) = 
  'SC { smParams(MS, 1), E :: -index(-index('E, N), M) } .
\end{verbatim}
\end{boxedminipage}
\end{small}
\end{center}

The third component, (c) above, that we need to give is the \emph{wiring
diagram}. A wiring diagram is treated as a mapping between \emph{ports}, where
a port is defined as follows:

\begin{center}
\begin{small}
\begin{boxedminipage}{0.50\textwidth}
\begin{verbatim}
sort MIdx . subsort NzNat < MIdx . 
op e : -> MIdx [ctor] . 
sort Port . 
op `(_,_`) : MIdx NzNat -> Port [ctor] .
\end{verbatim}
\end{boxedminipage}
\end{small}
\end{center}

\noindent
Then, a wiring diagram is just (with a view for ports instantiated in the obvious way)

\begin{center}
\begin{small}
\begin{boxedminipage}{0.75\textwidth}
\begin{verbatim}
  including MAP{Port,Port} * (sort Map{Port,Port} to Wiring) .
\end{verbatim}
\end{boxedminipage}
\end{small}
\end{center}

To generate a module for the synchronous composition of machines , and environment , and a wiring diagram , we have a function

\begin{center}
\begin{small}
\begin{boxedminipage}{0.60\textwidth}
\begin{verbatim}
  op gensc : Machines IOSize Wiring -> Module .
\end{verbatim}
\end{boxedminipage}
\end{small}
\end{center}

\noindent 
where the arguments correspond to the three pieces of information, (a) -- (c)
respectively, above. The composed machine will be denoted by , following
the notation of \cite{meseguer_2010_01,meseguer_2009_01}.

The meta-level sort {\tt Module} in {\tt META-MODULE} contains the following
constructor for functional modules

\begin{center}
\begin{small}
\begin{boxedminipage}{0.90\textwidth}
\begin{alltt}
op fmod_is_sorts_.____endfm : Header ImportList SortSet SubsortDeclSet
  OpDeclSet MembAxSet EquationSet -> Module [ctor ].
\end{alltt}
\end{boxedminipage}
\end{small}
\end{center}

\noindent
The function {\tt gensc} is defined at the top level by instantiating each of
the arguments of the above operator as explained below

\begin{center}
\begin{small}
\begin{boxedminipage}{0.40\textwidth}
\begin{verbatim}
eq gensc(MS, E, W) =
    fmod
      scHeader  (MS, E, W) is 
      nil 
      sorts 
      scSorts   (MS, E, W) . 
      scSubsorts(MS, E, W) 
      scOpDecls (MS, E, W) 
      none 
      scEqs     (MS, E, W) 
    endfm .
\end{verbatim}
\end{boxedminipage}
\end{small}
\end{center}

The implementation of {\tt scHeader} is given above (albeit with the third
argument omitted, since it is unused and {\tt Wiring} had not been introduced).
For {\tt scSorts} we simply need to give \emph{names} for the relevant sorts of
: (1) the input type for the composed machine, , (2) the state
type, , and (3) the output type, .

\begin{center}
\begin{small}
\begin{boxedminipage}{0.60\textwidth}
\begin{verbatim}
op scSorts : Machines IOSize Wiring -> SortSet .
eq scSorts(MS, E, W) = 'Di^E ; 'Do^E ; 'S^E .
\end{verbatim}
\end{boxedminipage}
\end{small}
\end{center}

Let us jump ahead briefly to define the internal state of , \ie\ the
constructor for sort {\tt S\^{}E}, since it requires the notion of
\emph{internal node}, which we will need when defining {\tt scSubsorts}.
Let 

 is called the set of \emph{internal nodes}. Then, the state of  is
defined as

where  denotes the sort of the  output of machine .
For example, the set of internal nodes for the circuit in Figure
\ref{fig:circuit} is , the outputs of the xor gates;
therefore we generate an {\tt OpDecl} for the state constructor as follows
(where the parameters are assumed to be the same as above)

\begin{center}
\begin{small}
\begin{boxedminipage}{0.95\textwidth}
\begin{verbatim}
(op `(_`,_`,_`,_`,_`) : 'M1S 'M3Do-1 'M2\EDo < 'Di^E .)
(subsort 'EM_1M_3Do-1 < 'M3\star\EDi < Do^E .
  subsort EDo-1 < M1Do-2 < M1Do-3 < M2Do-4 < M2Do-1 < M3Do-1 < M3Do-1 < EDo < ES M2S M1Do-1 -> S^E [ctor] .
  op pi1 : S^E -> M1S  .
  op pi3 : S^E -> M3Do-1  .
  op pi-2-1 : S^E -> M2S,X2:M2S,X4:M1Do-1)) 
        = X1:M1S,X2:M2S,X4:M1Do-1)) 
        = X2:M2S,X2:M2S,X4:M1Do-1)) 
        = X3:M3S,X2:M2S,X4:M1Do-1))
        = X4:M1S,X2:M2S,X4:M1Do-1)) 
        = X5:M2\rd[t, t + \rd]t\rd\rdN_1,\dotsc,N_5N_1,\dotsc,N_5nmN_1,\dotsc,N_5N_1,\dotsc,N_5M_1Do < 'Di^E .)
\end{verbatim}
\end{boxedminipage}
\end{small}
\end{center}

\noindent
which is just a -tuple with every component of sort {\tt Status}, that is,
exactly the information we expect from the environment. Then, we add a
\emph{rule} to non-deterministically generate any possible output from the
environment (equivalently, \emph{input} to the device) at each step

\begin{center}
\begin{small}
\begin{boxedminipage}{0.60\textwidth}
\begin{alltt}
crl [fromEnv] : S => delta1^E(I, S) .
 if I,IS := possibleInputsSet
\end{alltt}
\end{boxedminipage}
\end{small}
\end{center}

\noindent
where {\tt possibleInputsSet} is a set all possible values of sort {\tt Di\^{}E}.
Using the function {\tt natToDi\^{}E} above, it is straightforward to generate

\begin{center}
\begin{small}
\begin{boxedminipage}{0.75\textwidth}
\begin{verbatim}
op genDi^EUpTo : Nat -> Set{Di^E} .
eq genDi^EUpTo(0   ) = natToDi^E(0   ) .
eq genDi^EUpTo(s(X)) = natToDi^E(s(X)) , genDi^EUpTo(X) .

op possibleInputsSet : -> Set{Di^E} .
eq possibleInputsSet = genDi^EUpTo(31) .
\end{verbatim}
\end{boxedminipage}
\end{small}
\end{center}


We still need to define a notion of correctness for LMST. At a high level we
say that the protocol is correct if the network always stays connected whenever
there are no new node failures during a round. The top-level LTL formula is
given by

\begin{center}
\begin{small}
\begin{boxedminipage}{0.70\textwidth}
\begin{verbatim}
op correct? : -> Formula .
eq correct? = O ([] no-new-failures? -> ((O connected?))) .
\end{verbatim}
\end{boxedminipage}
\end{small}
\end{center}

\noindent
which says, more precisely, that after the first time step it is always the
case that whenever the set of failing nodes is \emph{stable}, then during the
next round the network is \emph{connected}. The formula characterizing when
there are no new node failures is defined as

\begin{center}
\begin{small}
\begin{boxedminipage}{0.50\textwidth}
\begin{alltt}
op no-new-failures? : -> Formula .
eq no-new-failures? = 
    (failed?(1) <-> O failed?(1)) 
    (failed?(2) <-> O failed?(2)) 
    (failed?(3) <-> O failed?(3)) 
    (failed?(4) <-> O failed?(4)) 
    (failed?(5) <-> O failed?(5)) .

eq S |= failed?(1) = pi1(S) == failed .

eq S |= failed?(5) = pi5(S) == failed .
\end{alltt}
\end{boxedminipage}
\end{small}
\end{center}

\noindent
that is, that the failed proposition for each node is consistent between the
current state and the next state for every node individually. The {\tt
connected?} formula is more complicated, but essentially it traverses the
routing tables of all non-failed nodes to determine if there is a multi-hop
route from each non-failed node to every other one (see \cite{katelman_2010_01}
for details).

Finally, we can model check a  node system against {\tt correct?} using
Maude's LTL model checker, showing that for the particular topology, LMST is
correct in the presence of node failure. Therefore, any \emph{asynchronous}
implementation of the system had through the PALS transformation would satisfy
the same notion of correctness.

\begin{center}
\begin{small}
\begin{boxedminipage}{0.85\textwidth}
\begin{verbatim}
Maude> red modelCheck(init, correct?) .
reduce in CHECK : modelCheck(init, correct?) .
rewrites: 317674 in 218ms cpu (226ms real) (1456678 rewrites/second)
result Bool: true
\end{verbatim}
\end{boxedminipage}
\end{small}
\end{center}

\section{Conclusions}
\label{sec:conclusions}

We have addressed the need to automatically support the synchronous composition
of abstract machines within Maude so that the PALS architecture can be
exploited for model checking purposes.  This is accomplished via
a meta-level module transformation in Maude that can automatically generate the
single abstract machine which is the composition of an ensemble of abstract
machines connected by a wiring diagram.  The transformation makes it easy to
verify a complex asynchronous DES system by model checking a much simpler
synchronous version where the user is responsible only for the individual
machine specifications and the wiring diagram.  

We have then illustrated how this transformation can be
applied to greatly simplify the formal verification of a key connectedness
property in the LMST topology control protocol for sensor networks.  As
explained in the introduction, a sensor network protocol such as LMST is an
example of a much broader class of distributed object-based DESs whose objects
only communicate with each other at pre-established times, and which change
their state at those times only as a result of the messages they then receive.
It would be quite useful to identify other examples of systems within this
category; also, the module transformation that we have presented could be
\emph{specialized to object-based systems} of this kind, so that it is not
necessary to specify such objects explicitly as abstract machines.

Besides the extension of the present work just outlined, much work remains
ahead.  For example, what is called transformation (2) in the Introduction
(passing from a synchronous model to its asynchronous PALS equivalent) should
also be automated at the meta-level, not for verification purposes, but for
purposes such as asynchronous design, code generation and also for system
emulation in physical time, when Real-Time Maude specifications are transformed
into actual real-time implementations.

\nocite{*}
\bibliographystyle{eptcs} 
\bibliography{rtrts_2010_camera}

\end{document}
