In this section we present the algorithm for solving LLFP clause
sequences, which extends the differential worklist algorithm by
Nielson et al. \cite{bib:ssforalfp,bib:sssuite}. The algorithm
computes the relations in increasing order on their rank and therefore
the negations present no obstacles. It completely abandons a
worklist-like data structures, which are typical for most classical
iterative fixpoint algorithms
\cite{DBLP:journals/scp/FechtS99}. Instead, we adapt the recursive
topdown approach of Le Charlier and van Hentenryck
\cite{bib:LeCharlier92} which is enhanced by continuation based
semi-naive iteration \cite{bib:Balbin87,bib:FechtSeidl98}.



In the following we assume that prior to solving the LLFP formula, all
the clauses are transformed into a form such that all \textit{applied}
occurrences of variables  in preconditions,
i.e.~, are not followed by their \textit{defining} occurrences,
i.e.~ and . This is necessary to
correctly perform late bindings of variables  in
the presence of  construct.

The algorithm operates with (intermediate) representations of the two
interpretations  and  of the semantics; we
shall call them {\tt env} and {\tt result}, respectively, in the
following. The data structure {\tt env} is supplied as a parameter to
the functions of the algorithms, and it represents partial
environment. The data structure {\tt result} is an imperative data
structure that is updated as we progress.

The partial environment {\tt env} is implemented as a map from
variables to their optional values. In the case the variable is
undefined it is mapped into . Otherwise, depending on its type
it is mapped to  or , which means that the
variable is bound to , or ,
respectively. The main operation on {\tt env} is the function
\textsc{unify}, defined as follows

It uses two auxiliary functions that perform unifications on each
component of the relation. For the first component, which ranges over
the universe , the function is given by

It performs a unification of an argument  with an element  in the environment {\tt env}. In the case when the unification
succeeds the modified environment is returned, otherwise the function
fails. The funcion is extended to -tuples in a straightforward
way. The definition of the unification function for the lattice
component is given by

The function is parametrized with , defined
in Section \ref{sec:alfp-lat}. It performs a unification of an lattice
term  with an element  in the environment {\tt env}. In
the case when the unification succeeds the set of unified environments
is returned, otherwise the function returns empty set.

The other important operation on the partial environment is given by
the function {\sc unifiable}. The function when applied to {\tt env}
and a tuple , returns a set of tuples for which {\sc
  unify} would succeed. The function is defined by means of two
auxiliary functions, formally we have

where

and

Both auxiliary funcions are extended to -tuples in a
straightforward way.

The global data structure \texttt{result}, which is updated
incrementally during computations, is represented as a mapping from
predicate names to the prefix trees that for each predicate  record
the tuples currently known to belong to . There are three main
operations on the data structure \texttt{result}: the operation
\texttt{result.}\textsc{has} checks whether a given tuple is
associated with a given predicate, the operation
\texttt{result.}\textsc{sub} returns a list of the tuples associated
with a given predicate and the operation \texttt{result.}\textsc{add}
adds a tuple to the interpretation of a given predicate.

Since  is updated as the algorithm progresses, it may
happen that a query  inside a precondition fails to be
satisfied at the given point in time, but may hold in the future when
a new tuple  is added to the interpretation of . If we
are not careful we may lose the consequences that adding 
to  will have on the contents of other predicates. This gives rise
to the data structure \texttt{infl} that records computations that
have to be resumed for the new tuples; these future computations are
called \emph{consumers}. The \texttt{infl} data structure is also
represented as a mapping from the predicate names to prefix trees that
for each predicate  record consumers that have to be resumed when
the interpretation of  is updated. There are two main operations on
the data structure {\tt infl}: the operation
\texttt{infl.}\textsc{register} that adds a new consumer for a given
predicate and \texttt{infl.}\textsc{consumers} that returns all the
consumers currently associated with a given predicate.

In the algorithm, we have one function for each of the three syntactic
categories. The function \textsc{solve} takes a \emph{clause sequence}
as input and calls the function \textsc{execute} on each of the
individual clauses

where we write [\ ] for the empty environment reflecting that we have
no free variables in the clause sequences.

Let us now turn to the description of the function
\textsc{execute}. The function takes a \emph{clause}  as a
parameter and a representation {\tt env} of the interpretation of the
variables. We have one case for each of the forms of ; the pseudo
code is given in Figure \ref{figure:function-execute}.
\begin{figure}
\centering
\begin{minipage}{.77\textwidth}
\begin{algorithmic}
\State \Call{execute}{\texttt{env} =
\State \indent \textbf{let} \Call{iterFun}{}  =
\State \indent \indent \textbf{match} \texttt{result.}\Call{has}{} \textbf{with}
\State \indent \indent \indent    ()
\State \indent \indent \indent   
\State \indent \indent \indent \indent \texttt{result.}\Call{add}{}
\State \indent \indent \indent \indent \Call{iter}{} (\textbf{fun}    ) (\texttt{infl.}\Call{consumers}{} 
\State \indent \textbf{in} \Call{iter}{} \Call{iterFun}{} (\Call{unifiable}{\texttt{env},})
\end{algorithmic}
\begin{algorithmic}
\State
\end{algorithmic}
\begin{algorithmic}
  \State
\end{algorithmic}
\begin{algorithmic}
  \State
\end{algorithmic}
\begin{algorithmic}
  \State
\end{algorithmic}
\end{minipage}
\caption[]{The \textsc{execute} function.}
\label{figure:function-execute}
\end{figure}
Let us explain the case of an assertion first. The algorithm uses the
auxiliary function \textsc{iter}, which applies the function
\textsc{iterFun} to each element of the list of tuples that can be
unified with the argument . Given a tuple ,
the function \textsc{iterFun} adds the tuple to the interpretation of
 stored in \texttt{result} if it is not already present. If the
\textsc{add} operation succeeds, we first create a list of all the
consumers currently registered for predicate  by calling the
function \texttt{infl.}\textsc{consumers}. Thereafter, we resume the
computations by iterating over the list of consumers and calling
corresponding continuations.
The cases of always true clause, {\bf 1}, is straightforward; the
function simply returns the unit, without performing any other
actions.
In the case of the conjunction of clauses the algorithm calls the
\textsc{execute} function for both conjuncts and the current
environment \texttt{env}.
In the case of implication we make use of the function \textsc{check}
that in addition to the precondition and the environment also takes
the continuation  as an argument.
In the case of universal quantification, we simply extend the
environment to record that the value of the new variable is unknown
and then we recurse. The case of universal quantification over a
variable  is exactly the same and hence omitted.

Now, let us present the function \textsc{check}. It takes a
\emph{precondition}, a continuation and an environment as
parameters. The pseudo code is given in Figure
\ref{fig:function-check}.
\begin{figure}
\centering
\begin{minipage}{.95\textwidth}
\begin{algorithmic}
\State \Call{check}{\texttt{env} =
\State \indent \textbf{let} \Call{consumer}{}  =
\State \indent \indent \textbf{match} \Call{unify}{\texttt{env}} \textbf{with}
\State \indent \indent \indent  fail  ()
\State \indent \indent \indent  \texttt{envs} 
\Call{iter}{}  \texttt{envs}
\State \indent \textbf{in} \texttt{infl.}\Call{register}{}(,\Call{consumer}{}); \Call{iter}{} \Call{consumer}{} (\texttt{result.}\Call{sub}{} )
\end{algorithmic}
\begin{algorithmic}
\State \Call{check}{\texttt{env} =
\State \indent \textbf{let} \Call{iterFun}{}  =
\State \indent \indent \textbf{match} \texttt{result.}\Call{has}{} \textbf{with}
\State \indent \indent \indent    ()
\State \indent \indent \indent    \Call{iter}{}
 (\Call{unify}{\texttt{env}})
\State \indent \textbf{in} \Call{iter}{} \Call{iterFun}{} (\Call{unifiable}{\texttt{env}})
\end{algorithmic}
\begin{algorithmic}
\State \Call{check}{\texttt{env} =
\State \indent \textbf{let} \texttt{env'} = \textbf{if}
\texttt{env} \textbf{then} \texttt{env} \textbf{else} \texttt{env}
\State \indent \textbf{in} \textbf{let} \textsc{f} a = {\tt env'}{\tt env'}
\State \indent \textbf{in} \textbf{match} \texttt{env'}(x) \textbf{with}
\State \indent \indent    \textsc{f} a
\State \indent \indent    \Call{iter}{} \textsc{f} 
\end{algorithmic}
\begin{algorithmic}
  \State
\end{algorithmic}
\begin{algorithmic}
\State \Call{check}{\texttt{env} = \Call{check}{}\texttt{env}; \Call{check}{}\texttt{env}
\end{algorithmic}
\begin{algorithmic}
\State \Call{check}{\texttt{env} =
\Call{check}{(\textsc{remove} )}(\texttt{env})
\end{algorithmic}
\end{minipage}
\caption[]{The \textsc{check} function.}
\label{fig:function-check}
\end{figure}
In the case of positive queries we first ensure that the consumer is
registered in {\tt infl}, by calling function \textsc{register}, so
that future tuples associated with  will be processed. Thereafter,
the function inspects the data structure \texttt{result} to obtain the
list of tuples associated with the predicate . Then, the auxiliary
function \textsc{consumer} unifies  with each tuple; and
if the operation succeeds, the continuation  is invoked on each
of the updated new environments in the returned set \texttt{envs}.
In the case of negated query, the algorithm first computes the tuples
unifiable with  in the environment \texttt{env}. Then,
for each tuple it checks whether the tuple is already in  and if
not, the tuple is unified with  to produce set of new
environments. Thereafter, the continuation  is evaluated in each
of the environments contained in the returned set. Notice that in the
case of negative queries we do not register a consumer for the
relation . This is because the stratification condition introduced
in Definition \ref{def:stratification} ensures that the relation is
fully evaluated before it is queried negatively. Thus, there is no
need to register future computations since the interpretation of 
will not change.
Now, let us consider function \textsc{check} in the case of ,
where . The function begins with creating an
environment {\tt env'} that is exactly as {\tt env} except that the
binding for the variable  is set to  in the case  is
undefined in {\tt env}. Then, we define an auxiliary function that
checks whether {\tt env'} over-approximates the abstraction of an
argument , denoted by , and if so the continuation is
called in the environment {\tt env'}. Finally, the
function checks the binding for the variable  in the environment
{\tt env'} and if it is bound to  the function \textsc{f}
applied to  is called. Otherwise, the function \textsc{f} is called
for each element of the universe, using the \textsc{iter}
function. The case of , where  is essentially the same
as the case explained above, except that we do not have to handle the
case when  is undefined in {\tt env}.
For conjunction of preconditions we exploit a continuation passing
programming style. More precisely, we call the \textsc{check} function
for the precondition , and as a continuation we pass a call to
the \textsc{check} function partially applied to the precondition
 and the continuation .
In the case of disjunction of preconditions the function simply checks
preconditions  and  respectively in the current
environment \texttt{env}. In order to be efficient we use memoization;
this means that if both checks yield the same bindings of variables,
the second check does not need to consider the continuation, as it has
already been done.
The algorithm for existential quantification checks the precondition
 in the environment extended with the quantified variable. The
continuation that is passed is a composition of functions  and
\textsc{remove} , where the function \textsc{remove} removes
variable passed as a first argument from the environment passed as a
second argument. In order to be efficient we again use a memoization
to avoid redundant computations. The case of existential
quantification over a variable  is exactly the same
and hence omitted.
