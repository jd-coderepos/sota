



\documentclass[preprint,12pt]{elsarticle}







\usepackage{amssymb}
\usepackage{algorithm}
\usepackage{algpascal}
\usepackage{algpseudocode}
\usepackage{etoolbox}







\biboptions{sort&compress}



\algtext*{EndWhile}\algtext*{EndIf}

\begin{document}

\pagestyle{empty}
\begin{frontmatter}





\title{A new algorithm for Many to Many Matching with Demands and Capacities}



\author[la1]{Fatemeh Rajabi-Alni}

\corref{cor1}\ead {fatemehrajabialni@yahoo.com}
\cortext[cor1]{Corresponding author.}


 \address[la1]{Department of Computer Engineering, Islamic Azad University,\\North Tehran Branch, Tehran, Iran.\fnref{label3}}




\begin{abstract}
Let  and  with , \textit {the many to many point matching with demands and capacities} matches each point  to at least  and at most  points in , and each point  to at least  and at most  points in  for all  and . In this paper, we present an  time and  space algorithm for this problem.
\end{abstract}

\begin{keyword}


many to many matching\sep Hungarian method\sep bipartite graph\sep points with demands and capacities 


\end{keyword}

\end{frontmatter}






\section{Introduction}


A \textit {matching} between two sets defines a relationship between their elements. The matching is used in various fields such as computational biology \cite{1}, pattern recognition \cite{2}, computer vision \cite{3}, music information retrieval \cite{4}, and computational music theory \cite{5}. 
A \textit {many-to-many matching} between  and  assigns each point in  to one or more points in , and vise versa.

Let  and  be two sets with , Eiter and Mannila \cite{6} proposed an  algorithm for the minimum many-to-many matching problem between  and  by reducing the problem to the minimum-weight perfect matching problem in a bipartite graph. 

\textit {The minimum many-to-many matching with demands and capacities}, here called \textit {MMDC} matching, is a matching in which each point  is matched to at least  and at most  points in , and each point  is matched to at least  and at most  points in , such that sum of the matching costs is minimized.
Schrijver \cite{7} solved the MMDC matching problem in strongly polynomial time. In this paper, we present a new algorithm that computes an MMDC matching between  and  in  time using  space. In section \ref{Preliminaries}, we review the basic Hungarian algorithm and some preliminary definitions. In section \ref{newalgorithms}, we present our new algorithm. 

\section {Preliminaries}
\label{Preliminaries}
Given an undirected bipartite graph , A \textit {maximum matching}  is a matching that for any other matching , we have . A path with the edges alternating between  and  is called an \textit {alternating path}. Each vertex  that is incident to one edge in  is called a \textit {matched vertex}; otherwise it is a \textit {free vertex}. An alternating path that its both endpoints are free is called an \textit {augmenting path}. Note that if the  edges of an augmenting path is replaced with the  ones, its size increases by . Let , a \textit {vertex labeling function}  assigns a label to each vertex . A vertex labeling that in which  for all  and  is called a \textit {feasible labeling}. The equality graph of a feasible labeling  is a graph  such that . The \textit {neighbors} of a vertex  is defined as . Consider a set of the vertices , the neighbors of  is .
\newtheorem{lemma}{Lemma}
\begin{lemma}
\label{lem1}
Consider a feasible labeling  of an undirected bipartite graph  and   with , let 
 If the labels of the vertices of  is updated such that:
 
then,  is also a feasible labeling.
\end{lemma}

\textbf {Proof.} Note that  is a feasible labeling, so we have  for each edge  of . After the update four cases arise:
\begin{itemize}
\item  and . In this case  
\item  and . We have 
\item  and . We see that 
\item  and . In this situation we have 

Two cases arises: 
\begin{itemize}

\item . So 
 Hence, .  

\item . Obviously 


\end{itemize}

\end{itemize}
\qed

\newtheorem{theorem}{Theorem}
\begin{theorem}
If  is feasible labeling and  is a Perfect matching in , then  is a max-weight matching \cite{8}.
\end{theorem}
\textbf {Proof.} Suppose that  is a perfect matching in , since each  vertex is incident to exactly one edge of  we have:
 So  is an upper bound for each perfect matching. Now assume that  is a perfect matching in :
 It is obvious that  is an optimal matching. 
\qed

In the following, we briefly describe the basic Hungarian algorithm which computes the maximum many to many matching between two sets. The input bipartite graph  is a complete bipartite graph that in which . 


\makeatletter
\expandafter\patchcmd\csname\string\algorithmic\endcsname{\itemsep\z@}{\itemsep=0.5ex plus0.5pt}{}{}
\makeatother

\algsetblock[Name]{Initial}{}{3}{1cm}
\alglanguage{pseudocode}
\begin{algorithm}
\caption{The Basic Hungarian algorithm(,)}
\begin{algorithmic}[1]
\Initial \Comment Find an initial feasible labeling  and a matching  in 
\State Let 
\State 
\State 
\While { is not perfect}

 \State Select a free vertex  and set , 
  \Repeat
     \While {}
        \State Update the labels according to Lemma \ref{lem1}
        \EndWhile
         \State Select  
          \If { is not free}\Comment ( is matched to the vertex , extend the alternating tree)
            \State .
           \EndIf
          \Until { is free}   
    \State Augment 
\EndWhile
\Return 
\end{algorithmic}
 \end{algorithm}


In line , we label all points of  with zero and each point  with  to get an initial feasible labeling. Note that  can be empty. It is obvious that for computing the minimum cost many to many matching using the Hungarian algorithm we must weight the edge  by .   


\begin{lemma}
Each augmenting path is a 4-vertex path.
\end{lemma}
\textbf {Proof.} Suppose that the lemma is false. Let  be an augmenting path with more than four vertices, that is . Note that  and  are free nodes. It is obvious that the first edge is in , so the second , third, and fourth edges of  are in , , and , respectively. Since the third edge  is in , the fourth edge  must be in . Note that  is a free node. A contradiction. 
\qed

\section{The algorithm}
\label{newalgorithms}
In this section, we describe our new algorithm which is based on the well known Hungarian algorithm. Consider two point sets  and  with . Let  and  denote the demand sets of  and , respectively. Let  and  be the capacity sets of  and , respectively. Without loss of generality, we assume that . 

\begin{theorem}
Let  and  be two sets with , an MMDC matching between  and  can be computed in  time.
\end{theorem}

\textbf {Proof.} 

We first construct a bipartite graph as follows. Consider the complete bipartite graph  where  and  (see Figure \ref{fig:1}). 
A \textit {complete connection} between two sets is a connection that in which each element of one set is connected to all elements of the other set. We show each set of the vertices by a rectangle and the complete connection between them by a line connecting the two corresponding rectangles. 

Given  and , there exists a complete connection between  and  such that the weight of  is equal to the cost of matching the point  to  for all  and . Let  and , each point of  is connected to the all points of  such that the weight of  is equal to the weight of . There exists also a complete connection between the sets  and  such that the weight of  is equal to the weight of . We have a set  that in which . In fact, we use  to get . Each vertex of  is connected to all vertices of  with zero weighted edges. 



Now we apply our new algorithm, Algorithm \ref{MMDC}, on above bipartite graph . Let  and  denote the capacity and the demand of the vertex ; so for all  we have , , , and . 

In our algorithm, a vertex  is free to another vertex  if  is not matched with  in  and has at least one empty capacity.  
So  and  are called free vertices to a vertex  that are not matched with it in , if \newline  and , respectively. \newline Also the vertices  and  are free to another vertex that is not incident in  to them, when \newline  and , respectively.


\begin{figure}
\vspace{-8cm}
\hspace{-11cm}
\resizebox{2.5\textwidth}{!}{\includegraphics{fig1.eps}
}
\vspace{-37.5cm}
\caption{Our constructed complete bipartite graph with .}
\label{fig:1}       \end{figure}


In fact, we save the current number of the vertices that are matched to the vertices of , , , and  in the arrays , , , and , respectively; for example  shows the number of the nodes that are matched to . The initial values of the arrays is ; when a new point is matched to their representing node their values are increased by . Assume that  returns the number of the vertices that are matched to  so far. So , , , and finally . Note that the procedures  and  return  if  and , respectively. So in the augmenting path ,  is free to ,  is matched to , and  is free to . Now we change the basic Hungarian algorithm as follows.


\algsetblock[Name]{Initialize}{}{4}{1cm}

\alglanguage{pseudocode}
\begin{algorithm}
\caption{The MMDC Hungarian algorithm(, , , )}
\label{MMDC}
\begin{algorithmic}[1]

\Initialize \Comment Find an initial feasible labeling  and a matching  in 
\State Let 
\State 
\State 
\State Let  
\item[]
   \While{}

 \State Select  with  
 \State Set 
\item[]
  \Repeat
     \While {}
        \Statex \Comment Update the labels according to Lemma \ref{lem1}

\State Let 
\State Let
 

        \EndWhile
\item[]
         \State Select 
          \If { }\Comment ()
\Statex\Comment ( is matched to some vertices )

            \State .
           \EndIf
          \Until {}   
\item[]
    \State 
    \EndWhile

\end{algorithmic}
\end{algorithm}


We first label the vertices of our bipartite graph  using an initial feasible labeling in lines . Algorithm \ref{MMDC} has a  loop where  times iterates and  edges are selected. In each iteration of our algorithm  increases by . Let  In line  ofAlgorithm \ref{MMDC}, the values of all slacks must be updated when a vertex is moved form  to . This is done in  time. During our algorithm  vertices are moved from  to , so it takes the total time of . 

In lines , we can compute the value of  by:

in  time. After computing the value of  and updating the labels of the vertices, we must also update the values of the slacks. This can be done using:
 
In each iteration the value of  may be computed at most  times, that takes  time each time, so running each iteration takes at most  time.
Our algorithm has  iteration with  time, so it runs in  time. 

\qed



\section{Conclusion}
\label{ConclusionSect}
In this paper, we presented an  time and  space algorithm for computing an MMDC matching between  and  with total cardinality . In fact, we modified the basic Hungarian algorithm to get a new algorithm, called the MMDC matching algorithm. Then, we construct a bipartite graph  and apply our new algorithm on .

\begin{thebibliography}{8}


\bibitem{1}
A. Ben-Dor, R.M. Karp, B. Schwikowski, R. Shamir, The restriction scaffold problem, J. Comput. Biol. 10 (2003) 385-398.

\bibitem{2}
S.R. Buss, P.N. Yianilos, A bipartite matching approach to approximate string comparison and search, Technical report, NEC Research Institute, Princeton, New Jersey (1995).

\bibitem{3}
M.F. Demirci, A. Shokoufandeh, Y. Keselman, L. Bretzner, S. Dickinson, Object recognition as many-to-many feature matching, Int. J. Comput. Vis. 69 (2006) 203-222.

\bibitem{4}
G.T. Toussaint, A comparison of rhythmic similarity measures, 5th International Conference on Music Information Retrieval, (2004) 242-245.


\bibitem{5}
G.T. Toussaint, The geometry of musical rhythm, Japan Conference on Discrete and Computational Geometry, Berlin-Heidelberg, (2005) 198-212.

\bibitem{6}
T. Eiter, H. Mannila, Distance measures for point sets and their computation, Acta Inform. 34 (1997) 109-133.

\bibitem{7}
A. Schrijver, Combinatorial optimization. polyhedra and efficiency, vol. A, Algorithms and Combinatorics, no. 24, Springer-Verlag, Berlin (2003).

\bibitem{8} L. R. Foulds, Combinatorial Optimization for Undergraduates, Springer-Verlag, Berlin (1984).

\end{thebibliography}

\end{document}
