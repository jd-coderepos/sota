\documentclass[envcountsect,envcountsame]{llncs}
\usepackage{amssymb,stmaryrd}
\usepackage{amsmath}
\usepackage{epsfig}
\usepackage{makeidx}
\usepackage{enumitem}
\usepackage{array}
\usepackage{algorithmic}
\usepackage[ruled]{algorithm}

\algsetup{indent=2em}
\renewcommand{\algorithmicendif}{\textbf{fi}}
\renewcommand{\algorithmicendfor}{\textbf{done}}
\renewcommand{\algorithmicendwhile}{\textbf{done}}
\renewcommand{\algorithmiccomment}[1]{\textit{#1}}
\renewcommand{\algorithmicrequire}{\textbf{Input:}}
\renewcommand{\algorithmicensure}{\textbf{Output:}}
\newcommand{\algorithmicdone}{\textbf{done}}
 \newcommand{\OLFORALL}[2]
{\STATE \algorithmicfor \mbox{ #1 }\algorithmicdo\mbox{ #2 }\
  \algorithmicdone}
\newcommand{\OLWHILE}[2]
{\STATE \algorithmicwhile \mbox{ #1 }\algorithmicdo\mbox{ #2 }\
  \algorithmicendwhile}
\newcommand{\OLIF}[2]
{\STATE \algorithmicif \mbox{ #1 }\algorithmicthen\mbox{ #2 }\algorithmicendif}
\newcommand{\OLELIF}[3]
{\STATE \algorithmicif \mbox{ #1 }\algorithmicthen\mbox{ #2 }\
  \algorithmicelse \mbox{ #3} \algorithmicendif}

\pagestyle{plain}
\pagenumbering{arabic}

\begin{document}

\title{\mbox{Generalized Maneuvers in Route Planning}}
\author{Petr Hlin\v{e}n\'y ~\and~ Ondrej Mori\v{s}}

\institute{Faculty of Informatics, Masaryk University \\
  Botanick\'a 68a, 602 00 Brno, Czech Republic  \\
  \email{hlineny@fi.muni.cz, xmoris@fi.muni.cz}}

\maketitle

\begin{abstract}
We study an important practical aspect of the route planning problem in 
real-world road networks -- \emph{maneuvers}. Informally, maneuvers 
represent various irregularities of the road network graph such as
turn-prohibitions, traffic light delays, round-abouts, forbidden passages 
and so on. We propose a generalized model which can handle arbitrarily complex
(and even negative) maneuvers, and outline how to enhance Dijkstra's 
algorithm in order to solve route planning queries in this model without 
prior adjustments of the underlying road network graph.
\end{abstract}

\section{Introduction}

Since mass introduction of GPS navigation devices, the \emph{route planning 
problem}, has received considerable attention. This problem is in fact
an instance of the well-known single pair shortest path (SPSP) problem
in graphs representing real-world road networks. However, it involves 
many challenging difficulties compared to ordinary SPSP. Firstly, classical 
algorithms such as Dijkstra's \cite{Dijkstra1959}, A* \cite{Hart1972} or their 
bidirectional variants \cite{Pohl1969} are not well suited for the route
planning despite their optimality in wide theoretical sense. It is mainly 
because graphs representing real-world road networks are so huge that even 
an algorithm with linear time and space complexity cannot be feasibly run 
on typical mobile devices. 

Secondly, these classical approaches disregard certain important aspects of
real-world road networks, namely route restrictions, traffic regulations, or
actual traffic info. Hence a route found by such algorithms might not be 
optimal or not even feasible. Additional attributes are needed in this regard.

The first difficulty has been intensively studied in the past decade, and 
complexity overheads of classical algorithms have been largely improved by 
using various preprocessing approaches. For a brief overview, we refer the 
readers to \cite{Cherkassky1994,Delling2009B,Schultes2008} or our 
\cite{HM2011A}. In this paper we focus on the second mentioned difficulty 
as it is still receiving significantly less attention.

\subsubsection{Related Work. } 
The common way to model required additional attributes of road networks is
with so called {\em maneuvers}\/; Definition~\ref{def:maneuver}. Maneuvers 
do not seem to be in the center of interest of route-planning research papers:
They are either assumed to be encoded into the underlying graph of a road 
network, or they are addressed only partially with rather simple types of 
restriction attributes such as turn-penalties and path prohibitions.

Basically, the research directions are represented either by modifications 
of the underlying graph during preprocessing 
\cite{Jiang2002,Pallottino1997,Ziliaskopoulos1996}, or by adjusting a query 
algorithm \cite{Kirby1969,Villeneuve2005} in order to resolve simple types 
of restrictions during queries. 

The first, and seemingly the simplest, solution is commonly used as it makes 
a~road network graph maneuver-free and so there is no need to adjust the 
queries in any way. Unfortunately, it can significantly increase the size 
of the graph \cite{Winter2002}; for instance, replacing a single 
turn-prohibition can add up to eight new vertices in place of one 
original~\cite{Gutierrez2008}. A solution like this one thus conflicts with 
the aforementioned (graph-size) objectives. Another approach \cite{Anez1996} 
uses so-called dual graph representation instead of the original one, where 
allowed turns are modeled by dual edges. 

To summarize, a sufficiently general approach for arbitrarily complex 
maneuvers seems to be missing in the literature despite the fact that such 
a solution could be really important. We would like to emphasize that all 
the cited works suffer from the fact that they consider only ``simple'' types 
of maneuvers.

\vspace{-1.5ex}
\subsubsection{Our Contribution. } Firstly, we introduce a formal model of a 
generic maneuver -- from a single vertex to a long self-intersecting walk 
-- with either positive or negative effects (penalties); being enforced, 
recommended, not recommended or even prohibited. Our model can capture 
virtually any route restriction, most traffic regulations and even some 
dynamic properties of real-world road networks. 

Secondly, we integrate this model into Dijkstra's algorithm, rising its 
worst-case time complexity only slightly (depending on a structure of 
maneuvers). The underlying graph is not modified at all and no preprocessing 
is needed. Even though our idea is fairly simple and relative easy to 
understand, it is novel in the respect that no comparable solution has been 
published to date. Furthermore, some important added benefits of our 
algorithm are as follows:

\begin{itemize}
\parskip 2pt
\item It can be directly used bidirectionally with any alternation strategy
  using an appropriate termination condition; it can be extended also to the 
  A* algorithm by applying a ``potential function to maneuver effects''.
  
\item Many route planning approaches use Dijkstra or A* in the core of their 
  query algorithms, and hence our solution can be incorporated into many of 
  them (for example, those based on a reach, landmarks or various types of 
  separators) quite naturally under additional assumptions.

\item Our algorithm tackles maneuvers ``on-line'' -- that is\ no maneuver 
  is processed before it is reached. And since the underlying graph of a 
  road network is not changed (no vertices or edges are removed or added),
  it is possible to add or remove maneuvers dynamically even during queries
  to some extent.
\end{itemize}

\section{Maneuvers: Basic Terms}
\label{sec:maneuvers}

A \emph{(directed) graph}  is a pair of a finite set  of vertices 
and a finite multiset  of edges (self-loops and 
parallel edges are allowed). The vertex set of  is referred to as , 
its edge multiset as . \emph{A~subgraph}  of a graph  is denoted 
by . 

\emph{A walk}  is an alternating sequence of vertices and 
edges   such that  for , the multiset of all edges of a walk 
is denoted by . \emph{A concatenation}  of 
walks  and  is the walk  . 
If  represents a single edge, we write~. If edges are 
clear from the graph, then we write a walk simply as .

A walk  is a {\em prefix} of another walk  if  is a subwalk of
 starting with the same index, and analogically with {\em suffix}.
The \emph{prefix set} of a walk  is  and analogically 
. 
A prefix (suffix) of a walk  thus is a member of 
(), and it is {\em nontrivial} if .

\emph{The weight} of a walk  with respect to a weighting  of  is defined as  and 
denoted by . \emph{A distance} from  to  in , , 
is the minimum weight of a walk  over all such 
walks and  is then called \emph{optimal} (with respect to weighting ). 
If there no such walk then . \emph{A~path} is a walk 
without repeating vertices and edges.

Virtually any route restriction or traffic regulation in a road network, 
such as turn-prohibitions, traffic lights delays, forbidden passages, 
turn-out lanes, suggested directions or car accidents by contrast, can be 
modeled by \emph{maneuvers} -- walks having extra (either positive or negative) 
``cost effects''. Formally:

\begin{definition}[Maneuver]
\label{def:maneuver}
\emph{A maneuver}  of  is a walk in  that is assigned a penalty 
. A set of all maneuvers of  is 
denoted by .
\end{definition}

\begin{remark}
A maneuver with a negative or positive penalty is called \emph{negative} 
or \emph{positive}, respectively. Furthermore, there are two special kinds 
of maneuvers the \emph{restricted} ones of penalty 0 and the \emph{prohibited} 
ones of penalty~. 
\end{remark}

The cost effect of a maneuver is formalized next:
\begin{definition}[Penalized Weight]
\label{def:penalized_weight}
Let  be a graph with a weighting  and a set of maneuvers . 
The \emph{penalized weight} of a walk  containing the maneuvers 
 as subwalks is defined as .
\end{definition}

Then, the intended meaning of maneuvers in route planning is as follows.

\begin{itemize}
\parskip2pt
\item If a driver enters a restricted maneuver, she must pass it completely
  (cf.~Definition~\ref{def:valid_walk}); 
  she must obey the given direction(s) regardless of the cost effect.
  Examples are headings to be followed or specific round-abouts.

\item By contrast, if a driver enters a prohibited maneuver, she must 
  not pass it completely. She must get off it before reaching its end, 
  otherwise it makes her route infinitely bad. Examples are forbidden 
  passages or temporal closures.

\item Finally, if a driver enters a positive or negative maneuver, she is
  not required to pass it completely;
  but if she does, then this will increase or decrease the 
  cost of her route accordingly. Negative maneuvers make her route 
  better (more desirable) and positive ones make it worse.
  Examples of positive maneuvers are, for instance, traffic lights delays, lane
  changes, or left-turns.
  Examples of negative ones are turn-out lanes, shortcuts, or implicit routes.
\end{itemize}

\begin{definition}[Valid Walk]
\label{def:valid_walk}
Let  be as in Definition~\ref{def:penalized_weight}. A walk 
 in  is \emph{valid} if and only if  and, for any
restricted maneuver , it holds that if a nontrivial prefix of
 is a subwalk of , then whole  is a~subwalk of  or a suffix of 
is contained in~ (that is  ends there).
\end{definition}

We finally get to the summarizing definition. A structure of a road network 
is naturally represented by a graph  such that the junctions are 
represented by  and the roads by . The chosen cost function 
(for example travel time, distance, expenses) is represented by a \emph{non-negative}
weighting  assigned to , and the additional 
attributes such as traffic regulations are represented by maneuvers as above.
We say that two walks  are {\em divergent} if, up to symmetry
between , a nontrivial prefix of  is contained in  but
the whole  is not a subwalk of~. Moreover, we say that  
{\em overhangs}  if a~nontrivial prefix of  is a suffix of~
(particularly, ).

\begin{definition}[Road Network]
\label{def:road_network}
Let  be a graph with a non-negative weighting  and a set of maneuvers 
. \emph{A road network} is the triple . 
Furthermore, it is called \emph{proper} if:
\begin{itemize}
\parskip 2pt
\item[i.] no two restricted maneuvers in  are divergent,
\item[ii.] no two negative maneuvers in  overhang one another, and
\item[iii.] for all ,  (that is, the penalized weight of every walk in  is 
  non-negative). \end{itemize}
\vspace{-4pt}
\end{definition}

\begin{figure}[t]
  \centering
  \centerline{\epsfig{file=./maneuver.eps, scale=0.6}}
  \vspace*{-.5ex}
  \caption{A road network containing maneuvers  
    with  (prohibited left turn) and  
    with  (right turn traffic lights delay). All edges
    have weight 1. The penalized weight of the walk  is , the penalized weight of the walk  is . Therefore the optimal walk (with respect to the 
    penalized weight) from  to  is  
    with the penalized weight .}
  \label{fig:maneuver}
\end{figure}

Within a road network, only valid walks (Definition~\ref{def:valid_walk})
are allowed further, and the distance from  to , , is the minimum penalized weight (Definition~\ref{def:penalized_weight})
of a valid walk ; such a walk  is then 
called \emph{optimal with respect to the penalized weight}. If there is no 
such walk, then . See~Fig.~\ref{fig:maneuver}.

Motivation for the required properties i.--iii. in 
Definition~\ref{def:road_network} is of both natural and practical character:
As for i., it simply says that no two restricted maneuvers are in a
conflict (that is no route planning deadlocks). Point ii. concerning only 
negative maneuvers is needed for a fast query algorithm, and it is indeed 
a~natural requirement (to certain extent, overhanging maneuvers can be modeled 
without overhangs). We remark that other studies usually allow no negative 
maneuvers at all. Finally, iii. states that no negative maneuvers can result 
in a negative overall cost of any walk -- another very natural property.
In informal words, a negative penalty of a maneuver somehow ``cannot 
influence'' suitability of a~route before entering and after exiting the 
maneuver.

\subsection{Strongly Connected Road Network}
\label{sec:connectivity}

The traditional graph theoretical notion of strong connectivity also needs to
be refined, it must suit our road networks to dismiss possible route 
planning traps now imposed by maneuvers. 

First, we need to define a notion of a \emph{``context''} of a vertex  
in  -- a~maximal walk in  ending at  such that it is a proper 
prefix of a maneuver in , or  otherwise. A set of all such
walks for  is denoted by . For example, on the road 
network depicted on Fig.~\ref{fig:maneuver}, 
. More formally: 

\begin{definition}
  \label{def:context}
  Let  be a set of maneuvers. We define 
  
  
  This  is the {\em maneuver-prefix set} at ,
  that is\ the set of all proper prefixes of walks from  that end
  right at~, including the mandatory empty walk.
  An element of  is called a {\em context} of the
  position  within the road network.
\end{definition}


\emph{The reverse graph}  of  is a graph on the same set of vertices 
with all of the edges reversed. Let  be a road network, 
a \emph{reverse road network} is defined as , where 
,  and , .

\begin{definition}
  \label{def:Mconnectivity}
  A road network  is \emph{strongly connected} if, for every 
  pair of edges  and for each possible 
  context  of  in  and each one 
  of  in , that is \ , there exists a valid walk starting with  and 
  ending with~.
\end{definition}

We remark that Definition~\ref{def:Mconnectivity} naturally corresponds to 
strong connectivity in an~amplified road network modeling the maneuvers 
within underlying graph.

\section{Route Planning Queries}
\label{sec:query}

At first, let us recall classical Dijkstra's algorithm \cite{Dijkstra1959}.
It solves SPSP\footnote{Given a graph and two vertices find a shortest path 
from one to another.} problem a graph  with a~non-negative weighting  
for a pair  of vertices.

\begin{itemize}
\parskip 3pt
\item The algorithm maintains, for all , a 
  {\em (temporary) distance estimate} of the shortest path from  to  
  found so far in , and a predecessor of  on that path in . 

\item The scanned vertices, that is those with , 
  are stored in the set ; and the reached but not yet scanned vertices, 
  that is those with , are stored in the set . 

\item The algorithm work as follows: it iteratively picks a vertex  
  with minimum value  and relaxes all the edges  leaving .
  Then  is removed from  and added to . {\em Relaxing} an edge 
  means to check if a shortest path estimate from  to  may be improved 
  via ; if so, then  and  are updated. Finally,  is added 
  into  if is not there already. 

\item The algorithm terminates when  is scanned or when  is empty.
\end{itemize}

Time complexity depends on the implementation of ; such as it is 
 with the Fibonacci heap.

\subsection{-Dijkstra's Algorithm}
\label{sec:Mdijkstra}

In this section we will briefly sketch the core ideas of our natural extension 
of Dijkstra's algorithm. We refer a reader to Algorithm \ref{alg:m-dijkstra}
for a full-scale pseudocode of this -Dijkstra's algorithm.

\begin{enumerate}
\parskip 3pt
\item Every vertex  scanned during the algorithm is considered 
  together with its context  
  (Definition~\ref{def:context}); that is\ as a pair .
  The intention is for  to record how  has  been reached in the 
  algorithm, and same  can obviously be reached and scanned more than 
  once, with different contexts.
  For instance,  can be reached with the empty or  contexts 
  on the road network depicted on Fig.~\ref{fig:maneuver}.

\item Temporary distance estimates are stored in the algorithm as 
  for such vertex-context pairs . At each step the 
  algorithm selects a next pair  such that it is minimal with respect 
  to the following partial order .

  \begin{remark}{Partial order :}
  \label{rmk:order}
  2pt] 
      (d[v_1,X_1] = d[v_2,X_2] &\land \> X_1 \in \mathit{Suf\!fix}(X_2)\,)\big).
    \end{array}\right.
   m= |V(G)|+\sum_{M\in\cal M}(|M| - 1)\leq c_{\cal{M}} \cdot |V(G)| \,, r= \sum_{u\in V(G)}|{\cal X_M}(u)|\cdot\mathit{out\mbox-deg}(u) + q
  \leq (c_{\cal{M}} + 1) \cdot|E(G)|3pt]
 &  \3pt]
 &  \2pt]
\hline\hline
1 &  & 0 &  \2pt]
3 &  & 2 &  \2pt]
5 &  & 2 &  
\2pt]
7 &  & 3 &  \2pt]
9 &  & 4 &  \2pt]
11 &  & 5 &  \2pt]
13 &  & 7 &  \2pt]
15 &  & 9 &  \\
\hline
\end{tabular}
\end{table}

\begin{figure}[H]
  \centering
  \epsfig{file=./alg-example-02.ps, scale=0.33}
  ~
  \epsfig{file=./alg-example-03.ps, scale=0.33}
  ~
  \epsfig{file=./alg-example-04.ps, scale=0.33}
  \vskip 20pt
  \epsfig{file=./alg-example-05.ps, scale=0.33}
  ~
  \epsfig{file=./alg-example-06.ps, scale=0.33}
  ~
  \epsfig{file=./alg-example-07.ps, scale=0.33}
  \vskip 20pt
  \epsfig{file=./alg-example-08.ps, scale=0.33}
  ~
  \epsfig{file=./alg-example-09.ps, scale=0.33}  
  ~
  \epsfig{file=./alg-example-10.ps, scale=0.33}
  \vskip 20pt
  \epsfig{file=./alg-example-11.ps, scale=0.33}
  ~
  \epsfig{file=./alg-example-12.ps, scale=0.33}  
  ~
  \epsfig{file=./alg-example-13.ps, scale=0.33}
  \caption{A computation of an optimal walk w.r.t. the penalized weight from 
     to  in . Numbers represent the distance from the start . 
    Black vertices are reached or scanned and black edges were relaxed.
    Dotted edges represent maneuver edges. 
    Steps 6 and 7 are depicted in the same figure (they are equal),
    analogously for steps 8 and 9.}
  \label{fig:example}
\end{figure}

\section{Conclusion}

We have introduced a novel generic model of maneuvers that is able to capture
almost arbitrarily complex route restrictions, traffic regulations and even
some dynamic aspects of the route planning problem. It can model anything 
from single vertices to long self-intersecting walks as restricted, negative, 
positive or prohibited maneuvers. We have shown how to incorporate this model 
into Dijkstra's algorithm so that no adjustment of the underlying road network 
graph is needed. The running time of the proposed Algorithm~\ref{alg:m-dijkstra} 
is only marginally larger than that of ordinary Dijkstra's algorithm 
(Theorem~\ref{thm:MDijkstra}) in practical networks. 

Our algorithm can be relatively straightforwardly extended to a bidirectional 
algorithm by running it simultaneously from the start vertex in the original 
network and from the target vertex in the reversed network. A termination
condition must reflect the fact that chained contexts of vertex-context pairs 
scanned in both directions might contain maneuvers as subwalks. 
Furthermore, since the A* algorithm is just an ordinary Dijkstra's algorithm 
with edge weights adjusted by a potential function, our extension remains
correct for A* if the road network is proper (Definition~\ref{def:road_network}, 
namely iii.) even with respect to this potential function.

Finally, we would like to highlight that, under reasonable assumptions, our 
model can be incorporated into many established route planning approaches.

\bibliographystyle{plain}
\bibliography{references}

\end{document}
