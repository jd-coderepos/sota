

\documentclass[11pt]{llncs}
 


\usepackage{amsmath,amsfonts,amssymb,enumerate}
\usepackage[T1]{fontenc}
\usepackage{hyperref}
\usepackage[noadjust,nospace]{cite}


\usepackage{boxedminipage}

\makeatletter\renewcommand\subparagraph{\@startsection{subparagraph}{5}{0ex} {3.25ex \@plus1ex \@minus .2ex} {-1em} {\normalfont\normalsize\bfseries}}\makeatother

\newcommand{\opts}{opt_S}
\newcommand{\vd}{{\sf vd}}
\newcommand{\ed}{{\sf ed}}
\newcommand{\ea}{{\sf ea}}
\newcommand{\cdpe}{{\sc CDPE}}
\newcommand{\dpe}{{\sc DPE}}
\newcommand{\cdbe}{{\sc CDBE}}
\newcommand{\dbe}{{\sc DBE}}

\newcommand\displaycase[1]{{\bf #1}}

\renewcommand{\P}{{\sf P}}
\newcommand{\NP}{{\sf NP}}
\newcommand{\FPT}{{\sf FPT}}
\newcommand{\W}{{\sf W[1]}}

\bibliographystyle{abbrv}

\pagestyle{plain}

\title{Editing to Eulerian Graphs\thanks{The research 
leading to these results has received funding from the European Research Council under the European Union's Seventh Framework Programme (FP/2007-2013)/ERC Grant Agreement n. 267959 and from EPSRC Grant EP/K025090/1.
An extended abstract of this paper will appear in the proceedings of FSTTCS 2014.
}} 
\author{Konrad K. Dabrowski\inst{1} \and Petr A. Golovach\inst{2} \and Pim van 't Hof\inst{2} \and\\ Dani{\"e}l Paulusma\inst{1}}

\institute{
School of Engineering and  Computing Sciences, Durham University,\\ 
Science Laboratories, South Road, Durham DH1 3LE, United Kingdom\\
\email{\{konrad.dabrowski,daniel.paulusma\}@durham.ac.uk}
\and
Department of Informatics, University of Bergen,\\
PB 7803, 5020 Bergen, Norway\\ 
\email{\{petr.golovach,pim.vanthof\}@ii.uib.no}
}



\begin{document}

\maketitle

\begin{abstract}
We investigate the problem of modifying a graph into a connected graph in which
the degree of each vertex satisfies a prescribed parity constraint. Let ,
 and  denote the operations 
edge addition, edge deletion and vertex
deletion respectively. For any , we define
\textsc{Connected Degree Parity Editing} (\cdpe()) to be the problem
that takes as input a graph~, an integer  and a function , and asks whether  can be modified into a connected
graph  with  for each , using
at most~ operations from . We prove that
\begin{itemize}
\item if  or , then \cdpe() can be solved in polynomial time;
\item if , then \cdpe() is \NP-complete and \W-hard when parameterized by~, even if .
\end{itemize}
Together with known results by Cai and Yang and by Cygan, Marx, Pilipczuk,
Pilipczuk and Schlotter, our results completely classify the classical and
parameterized complexity of the \cdpe() problem for all . We obtain the same classification for a natural variant of
the \cdpe() problem on directed graphs, where the target is a weakly
connected digraph in which the difference between the in- and out-degree of
every vertex equals a prescribed value.
As an important implication of our results, we obtain polynomial-time
algorithms for the {\sc Eulerian Editing} problem and its directed variant. 
\end{abstract}

\begin{keywords}
Eulerian graphs, graph editing, polynomial algorithm
\end{keywords}

\section{Introduction}\label{sec:intro}
Graph modification problems play a central role in algorithmic graph theory,
partly due to the fact that they naturally arise in numerous practical
applications. A graph modification problem takes as input a graph  and an
integer , and asks whether  can be modified into a graph
belonging to a prescribed graph class , using at most  operations
of a certain type. The most common operations that are considered in this
context are edge additions (-{\sc Completion}), edge deletions
(-{\sc Edge Deletion}), vertex deletions (-{\sc Vertex Deletion}),
and a combination of edge additions and edge deletions (-{\sc
Editing}). The intensive study of graph modification problems has produced a
plethora of classical and parameterized complexity results (see
e.g.~\cite{BoeschST77,BurzynBD06,Ca96,CaiY11,CechlarovaS10,CrowstonGJY12,CyganMPPS14,DornMNW13,FroeseNN14,GoyalMPPS14,Golovach13,Golovach13a,HohnJM12,LesniakO86,LY80,MathiesonS12,MoserT09,NatanzonSS01}).


An undirected graph is Eulerian if it is connected and every vertex has even
degree,
while 
a directed graph is Eulerian if it is strongly
connected\footnote{Replacing ``strongly connected'' by ``weakly connected''
yields an equivalent definition of Eulerian digraphs, as it is well-known that
a balanced digraph is strongly connected if and only it is weakly connected
(see e.g.~\cite{CyganMPPS14}).} and balanced, i.e. the in-degree of every
vertex equals its out-degree. Eulerian graphs form a well-known graph class
both within algorithmic and structural graph theory. 
Several groups of authors have investigated the problem of deciding 
whether
a given undirected graph can be made Eulerian using a small number of operations. Boesch et
al.~\cite{BoeschST77} presented a polynomial-time algorithm for {\sc Eulerian
Completion}, and 
Cai and Yang~\cite{CaiY11} showed that the problems {\sc
Eulerian Vertex Deletion} and {\sc Eulerian Edge Deletion} are
\NP-complete~\cite{CaiY11}. When parameterized by~, 
it is known that
{\sc Eulerian Vertex Deletion} is \W-hard~\cite{CaiY11}, while {\sc
Eulerian Edge Deletion} is fixed-parameter tractable~\cite{CyganMPPS14}. Cygan
et al.~\cite{CyganMPPS14} showed that the classical and parameterized
complexity results for {\sc Eulerian Vertex Deletion} and {\sc Eulerian Edge
Deletion} also hold for the directed variants of these problems.
Recently, Goyal et al.~\cite{GoyalMPPS14} improved the fixed-parameter tractability  results of Cygan et al.~\cite{CyganMPPS14} for the directed and undirected variants 
of {\sc Eulerian Edge
Deletion}. The same authors also proved that the {\sc Undirected Connected Odd Edge Deletion} problem, which asks whether it is possible to obtain a connected graph in which all vertices have odd degree by deleting at most  edges, is fixed-parameter tractable when parameterized by .

Another problem that can be seen as involving editing to an Eulerian multigraph
is the \textsc{Chinese Postman} problem, also known as the \textsc{Route
Inspection} problem~\cite{Kwan62}. In this problem a connected graph ,
together with an integer , is given and the question is whether there exists
a closed walk in  that contains all edges of , but that has length at
most~. In other words, can a total of at most  copies of existing edges
be added to  in order to modify  into an Eulerian multigraph? Edmonds and
Johnson~\cite{EdmondsJ73} showed that both the undirected and directed variant
of this problem can be solved in polynomial time.

\subparagraph{Our Contribution}
We generalize, extend and complement known results on graph modification
problems dealing with Eulerian graphs and digraphs. 
The main contribution of
this paper consists of two non-trivial polynomial-time algorithms: one for
solving the {\sc Eulerian Editing} problem, and one for solving the directed
variant of this problem.
Given the aforementioned \NP-completeness result
for {\sc Eulerian Edge Deletion} and the fact that -{\sc Editing} is
\NP-complete for almost all natural graph classes ~\cite{BurzynBD06,NatanzonSS01}, we find it particularly interesting that
{\sc Eulerian Editing} turns out to be polynomial-time solvable. To the best of
our knowledge, 
the only other natural 
non-trivial graph class  for which -{\sc Editing} is known to be polynomial-time solvable is the class of split
graphs~\cite{HammerS81}.

In fact, our polynomial-time algorithms are implications of two more general
results. In order to formally state these results, we need to introduce some
terminology. Let ,  and  denote the operations edge addition,
edge deletion and vertex deletion, respectively. For any set  and non-negative integer , we say that a graph  can be
{\em -modified} into a graph  if  can be obtained from  by
using at most  operations from~. We define the following problem for
every :
\begin{center}
\begin{boxedminipage}{.99\textwidth}
\begin{tabular}{rl}
\textsc{\cdpe():} & \textsc{Connected Degree Parity Editing}\\
\textit{~~~~Instance:} & A graph , an integer \\
                       & and a function .\\
\textit{Question:} & Can  be -modified into a connected graph \\ 
                   & with  for each ?
                   \end{tabular}
\end{boxedminipage}
\end{center}

Inspired by the work of Cygan et al.~\cite{CyganMPPS14} on directed Eulerian
graphs, we also study a natural directed variant of the \cdbe() problem.
Denoting the in- and out-degree of a vertex  in a digraph~ by 
and , respectively, we define the following problem for every
:
\begin{center}
\begin{boxedminipage}{.99\textwidth}
\begin{tabular}{rl}
\textsc{\cdbe():}   & \textsc{Connected Degree Balance Editing}\\
\textit{~~~~Instance:} & A digraph , an integer  and\\
                       & a function .\\
\textit{Question:}     & Can  be -modified into a weakly connected\\& digraph 
                        with  for each\\
& ?
\end{tabular}
\end{boxedminipage}
\end{center}

In Section~\ref{sec:undirected}, we prove that \cdpe() can be solved in polynomial time when   
and when .
The first of these two results extends the result by Boesch et
al.~\cite{BoeschST77} on {\sc Eulerian Completion}
 and the second yields the first
polynomial-time algorithm for {\sc Eulerian Editing}, as these problems are
equivalent to \cdpe() and \cdpe(, respectively, when we
set . 
The complexity of the problem drastically changes when vertex deletion is allowed: we prove that for every subset  with , the \cdpe() problem is
\NP-complete and \W-hard with parameter~, even when . 
This complements results by Cai and Yang~\cite{CaiY11} stating that \cdpe()
is \NP-complete and \W-hard with parameter~ when 
and  or . 
Our results, together with the aforementioned results due to Cygan et al.~\cite{CyganMPPS14}\footnote{The \FPT-results
by Cygan et al.~\cite{CyganMPPS14} only cover \cdpe() and
\cdbe() when , but it can easily be seen that their
results carry over to \cdpe() and \cdbe() for any function
.} and Cai and
Yang~\cite{CaiY11}, yield a complete classification of both the classical and the parameterized
complexity of \cdpe() for all ; 
see the middle column of Table~\ref{t-thetable}. 

In Section~\ref{sec:directed}, we use different and more involved arguments to classify the classical and parameterized complexity of the \cdbe() problem for all . 
Interestingly, the classification we obtain for \cdbe() turns out to be identical to the one we obtained for \cdpe().  
In particular, our proof of the fact that \cdbe() is polynomial-time solvable when  and  implies that the directed variants of {\sc Eulerian Completion} and {\sc Eulerian Editing} are not significantly harder than their undirected counterparts.
All results on \cdbe() are summarized in the right column of Table~\ref{t-thetable}. 

\begin{table}[htb]
\begin{center}
\begin{tabular}{l|l|l}
                    &  \cdpe()                             &  \cdbe()                           \\ \hline
           &  \P                                             &  \P                                            \\
                  &  \P                                             &   \P                                             \\
                  &  \FPT~\cite{CyganMPPS14} &  \FPT~\cite{CyganMPPS14} \\
                   &  \W-hard~\cite{CaiY11}               &  \W-hard~\cite{CyganMPPS14}\\
            &  \W-hard                               &  \W-hard\\
            &   \W-hard                              &  \W-hard \\
     &  \W-hard                             &  \W-hard \\
\end{tabular}
\end{center}
\caption{A summary of the results for \cdpe() and \cdbe(). All results are new except those for which a reference is given.
The number of allowed operations  is the parameter in the parameterized results, and if a parameterized result is stated, then the corresponding problem is
\NP-complete.} 
\label{t-thetable}
\end{table}

We would like to emphasize that there are no obvious hardness reductions between the different problem variants. The parameter  in the problem definitions represents the budget for all operations in total; adding a new operation to  may completely change the problem, as there is no way of forbidding its use. Hence, our polynomial-time algorithms for \cdpe() and \cdbe() do not generalize the polynomial-time algorithms for \cdpe() and \cdbe(), and as such require significantly different arguments. In particular, our main result, stating that {\sc Eulerian Editing} is polynomial-time solvable, is not a generalization of the fact that {\sc Eulerian Completion} is polynomial-time solvable and stands in no relation to the \FPT-result by Cygan et al.~\cite{CyganMPPS14} for {\sc Eulerian Edge Deletion}.

\medskip
\noindent
We end this section by mentioning two similar graph modification frameworks in the literature that formed a direct motivation for the framework defined in this paper.
Mathieson and Szeider~\cite{MathiesonS12} considered the {\sc Degree Constraint Editing()} problem, which is that of testing whether a graph  can be -modified into a graph  in which the degree of every vertex belongs to some list associated with that vertex;
recently some new results for this problem were obtained by Froese et al.~\cite{FroeseNN14} and Golovach~\cite{Golovach13a}.
Golovach~\cite{Golovach13} performed a similar study to that of Mathieson and Szeider~\cite{MathiesonS12}, but with the additional condition that the resulting graph must be connected.

\section{Preliminaries}\label{sec:prelim}

We consider finite graphs  that may be undirected or directed; in the
latter case we will always call them digraphs. All our undirected graphs will
be without loops or multiple edges; in particular, this is the case for both the input and the output graph in every undirected problem we consider. 
Similarly, for every directed problem that we consider, we do not allow the input or output digraph to contain multiple arcs.
In our proofs we will also make use of {\em directed multigraphs}, which are digraphs that are permitted to have multiple arcs.

We denote an edge between two vertices  and  in a graph by .  We denote an arc between
two vertices  and~ by , where  is the \emph{tail} of 
and  is the \emph{head}. The disjoint union of two graphs  and  is denoted .
The complete graph on  vertices is denoted  and the complete bipartite graph with classes of size  and 
is denoted~.

Let  be a graph or a digraph.  Throughout the paper we assume that
 and .  For , we let   be the graph (digraph)
with vertex set  and an edge (arc) between two vertices  and~ if and
only if this is the case in ; we say that  is \emph{induced by}~.
We write .  For , we let  be the
graph (digraph) with edge (arc) set~ whose vertex set consists of the
end-vertices of the edges in ; we say that  is \emph{edge-induced
by}~.  Let  be a set of (ordered) pairs of vertices of .  We let
 be the graph (digraph) obtained by deleting all edges (arcs) of 
from~, and we let  be the graph (digraph) obtained by adding all edges
(arcs) of  to .  We may write  or  if .

Let  be a graph.  A {\em component} of  is a maximal connected subgraph
of~.  The {\em complement}  of   is the graph 
with vertex set~ and an edge between two distinct vertices  and  if and only
if~.   For a vertex , we
let  denote its \emph{(open) neighbourhood}.  The
\emph{degree} of   is denoted . The graph   is {\em even}
if all its vertices have even degree, and it is {\em Eulerian} if it is even
and connected.  We say that a set  is an {\em edge cut} in 
if~ is connected but  is not.  An edge cut of size~ is called a {\em
bridge} in~.

A \emph{matching} of a graph  is a set of edges, in which no two edges have
a common end-vertex; it is called a {\em maximum} matching if its number of
edges is maximum over all matchings of .  We need the following lemma due to
Micali and Vazirani.

\begin{lemma}[\cite{MicaliV80}]\label{l-mic}
A maximum matching of an -vertex graph can be found in  
time.
\end{lemma}

Let  be a digraph.  If  is an arc, then  is the {\em
reverse} of this arc.  For a subset , we let  denote the set of arcs whose reverse is in~.  The \emph{underlying}
graph of~ is the undirected graph with vertex set  where two vertices
 are adjacent if and only if  or  is an arc in .  We
say that  is \emph{(weakly) connected} if its underlying graph is connected.
A {\em component} of~ is a connected component of its underlying graph.  An
arc  is a \emph{bridge} in  if it is a bridge in the underlying
graph of .  A vertex  is an \emph{in-neighbour} or \emph{out-neighbour}
of a vertex~ if  or , respectively. Let
 and , where we
call  and  the \emph{in-degree} and
\emph{out-degree} of , respectively. 
A vertex  is {\em balanced} if , 
or equivalently, its {\em degree balance} .
Recall that~ is {\em Eulerian} if it
is connected and {\em balanced}, that is, the out-degree of every vertex is
equal to its in-degree. 

Let  be a graph and let . A subset  is a
\emph{-join} if the set of odd-degree vertices in  is precisely .
If  is connected and  is even then  has at least one -join. In
Section~\ref{sec:undirected} we need to find a {\em minimum} -join, that is,
one of minimum size. We use the following result of Edmonds and Johnson~\cite{EdmondsJ73} to do
so.

\begin{lemma}[\cite{EdmondsJ73}]\label{lem:t-join}
Let  be  a graph, and let .  Then a minimum -join (if
one exists) can be found in  time. 
\end{lemma}
Lemma~\ref{lem:t-join} was used by Cygan et al.~\cite{CyganMPPS14} to solve
-{\sc Edge Deletion} in polynomial time when~ is the class
of even graphs. It would  immediately yield a polynomial-time algorithm for
\cdpe() if we dropped the connectivity condition.

We need a variant of Lemma~\ref{lem:t-join} for digraphs in
Section~\ref{sec:directed}.  Let  be a directed multigraph and let  be a function for some .  A multiset
 with  is a {\em directed -join} in 
if the following two conditions hold: 
 for
every  and  for every .  A directed -join is {\em minimum} if it has minimum
size.  The next lemma was used by Cygan et al.~\cite{CyganMPPS14} to solve
-{\sc Edge Deletion} in polynomial time when  is the class
of balanced digraphs; it would also yield a polynomial-time algorithm for
\cdbe() if we dropped the connectivity condition. 

\begin{lemma}[\cite{CyganMPPS14}]\label{lem:dir-t-join}
Let  be a directed multigraph and  be  a
function for some .  A minimum directed -join  (if one
exists) can be found in  time. Moreover,~ consists
of mutually arc-disjoint directed paths from vertices~ with  to
vertices~ with .
\end{lemma}




\section{Connected Degree Parity Editing}\label{sec:undirected}

Let .  
In Section~\ref{s-polyun} we will show that
\cdpe() is polynomial-time solvable if  or  and
in Section~\ref{s-wun} we will show 
that it is \NP-complete and \W-hard with parameter~ if .

\subsection{The Polynomial-Time Solvable Cases}\label{s-polyun}
First, let .  Let  be an
instance of \cdpe() with .  Let~ be a set of edges not in ,
and let  be a set of edges in~, with  if .  We
say that  is a {\em solution} for  if its {\em size}
, the congruence  holds for every vertex~ and
the graph  is connected; if  is
not connected then  is a {\em semi-solution} for . If
 we may denote the solution by  rather than  (since
).  We consider the optimization version for \cdpe.  The
input is a pair , and we aim to find the minimum  such that
 has a solution (if one exists).  We call such a solution {\em
optimal} and denote its size by .  We say that a
(semi)-solution for  is also a (semi)-solution for .
If  has no solution for any value of , then  is a
{\em no-instance} of  \cdpe and .

Let .  Define
 if  and  if .  Note that
if  then  contains no edges of , so in this case any
-join in  can only contain edges in . The following
key lemma is an easy observation.

\begin{sloppypar}
\begin{lemma}\label{lem:struct-undir}
Let . Let  be an instance of
\cdpe and~, .  Then
 is a semi-solution of \cdpe if and only if  is a
-join in~.
\end{lemma}
\end{sloppypar}

We extend the result of Boesch et al.~\cite{BoeschST77} for  to arbitrary . Our proof
is based around similar ideas but we also had to do some further analysis. The
main difference in the two proofs is the following. If  then
none of the added edges in a solution will be a bridge in the modified graph
(as the number of vertices of odd degree in a graph is always even). However
this is no longer true for arbitrary  and extra arguments are needed.

\begin{theorem}\label{thm:add-undi}
Let . Then \cdpe can be solved in  time.
\end{theorem} 

\begin{proof}
Let  and let  be an instance of \cdpe.  We first
use Lemma~\ref{lem:t-join} to check in  time whether  has a
-join.  If not then  has no semi-solution by
Lemma~\ref{lem:struct-undir}, and thus no solution either.  We may therefore
assume that~ is even and~ is a minimum -join in~. (Recall that
Lemma~\ref{lem:t-join} states that we can find  in  time if it
exists.) We also assume that either~ or  is not connected,
otherwise the trivial solution  is clearly optimal.  Let~ be
the number of components of~ that do not contain any vertex of~ and
let~ be the number of components of~ that contain at least one vertex
of~. We will prove the following series of statements.

\begin{itemize}
\item  is a no-instance if  and  for .
\item  if  and  for .
\item  if  and~ has a component that is not complete.
\item  if .
\item  if .
\end{itemize}

We split our proof into two parts depending on the value of~.

\medskip
\noindent
\displaycase{Case 1:} .\\
In this case , so by Lemma~\ref{lem:struct-undir} for any
semi-solution~, every vertex in  must have even degree in .
In other words, every vertex of~ must be incident to an even number of edges
in~. Since , we assumed above that~ was disconnected, so 
and any solution~ must be non-empty. This means that  must contain a
cycle, so . Recall that  is
a subgraph of .

Suppose . If  for  then , which
does not contain a cycle. Therefore  is a no-instance in this case.
If  for  then , which contains no
cycles of length 3. Therefore  in this case.  Indeed, if
 are vertices in the~ component of~ and  are vertices in the~
component, then  is a solution of size 4 and this
solution must therefore be optimal.  Finally, suppose~ contains exactly two
components, at least one of which is not a clique. Let~ be non-adjacent
vertices in this component and let~ be a vertex in the other component. Then
 is a solution of size~3, which must therefore be optimal.

Finally, suppose that . Since  must be connected for any
solution~, every component in~ must contain at least one vertex
incident to an edge of~. By Lemma~\ref{lem:struct-undir}, this vertex must
be incident to an even number of edges of~, meaning that it must be incident
to at least two such edges. Therefore .  Indeed, if we
choose vertices , one from each component of~ then
 is a solution of size~, which
is therefore optimal.

This concludes the  case.

\medskip
\noindent
\displaycase{Case 2:} .\\
In this case . We first show that . Since~ is a minimum -join in~,
Lemma~\ref{lem:struct-undir} implies that . Since 
has  components, any solution  must contain at least  edges to
ensure that  is connected, so . Finally, let
 be the components of~ that do not contain any vertices
of~. If  is a solution then every component~ must contain a vertex
incident to some edge in~.  By Lemma~\ref{lem:struct-undir}, this vertex
must be incident to an even number of edges of , meaning that it must be
incident to at least two such edges.  By Lemma~\ref{lem:struct-undir}, every
vertex of~ must be incident to some edge in~. Therefore~ must contain
at least  edges, so .

Next we show that we can always construct a solution of size
. To do this, we try to replace edges of~
in such way that~ remains a minimum -join in~, but the number of
components in  is reduced. After we have finished this process, if 
is connected then setting  gives a solution of size~, which is
therefore optimal. Otherwise, we will be able to use the structure of~ to
construct a solution of size either  or .

Consider the graph . Since~ is a minimum -join,  cannot
contain any cycles (otherwise the edges in the cycle could be removed from~
to give a smaller -join).  We claim that~ does not contain a path of
length .  Suppose, for contradiction, that there is such a path with
edge set~ and end-vertices~ and~.  Note that~ and~ are in the
same component of . Since  is not connected (otherwise  would be
an optimal solution of size ), there must be a vertex  which
is in a different component of  from the one containing~ and~. In this
case .  Let~. Then~ is
also a -join in , since the degree parity of any vertex in  is
the same as its degree parity in . However, , which contradicts
the fact that~ is a minimum -join.  Therefore  must be a forest
that contains no paths of length 3. In other words  is a forest of
stars.

Now suppose that , such that~ is not a bridge in  and
the vertices~ and~ are in different components of .
Let~. Then~ is also a minimum
-join in~. However,  has one component less than .  Indeed,
since~ is not a bridge in , the vertices  must all be in
the same component of . Therefore, if such edges  exist, we
replace~ by~. We do this exhaustively until no further such pairs of
edges exist. At this point either every edge in~ must be a bridge or every
edge in~ is in the same component of . We consider these possibilities
separately.

First suppose that every edge in~ is a bridge. Choose  and let
 be the components of , with . Note that
since every edge in~ is a bridge, . Now let  for
. Let  if  and  otherwise. Now every vertex in 
has the same degree parity as in , so~ is a -join in~.  The
graph  is connected, so~ is a solution.  However,
.  Therefore~ is an optimal solution.

We may now assume that every edge in~ is in the same component of . If
 is connected, then  is a solution of size  and is therefore
optimal, so we may assume that  is not connected. Suppose . Then
, otherwise we could replace~ in  by  to get a
smaller -join in . Suppose that  do not form a cut-set in .
In other words, we suppose that  and  are in the same component of
. Let~ be a vertex in a different component of 
from the one containing . Then .  Let
. Then  must also be a minimum
-join in~. However,  has one less component than .
Indeed,~ is in the same component of  as . In this case we may
replace~ by~. Again, we apply this replacement exhaustively until it can
no longer be applied. This process ends when either  becomes connected (in
which case  is an optimal solution of size ) or, for every pair of
edges of the form , we find that  is a cut-set in
. We may assume the latter is the case.

Now suppose . Consider the component  of
 containing . We claim that  contains no vertices
of . Suppose, for contradiction, that  ( is not
necessarily distinct from ). Then by Lemma~\ref{lem:struct-undir},~ must
be the end-vertex of some edge in , say  (again  is
not necessarily distinct from ). Note that  and~ are in the same
component of , which is different from the component
containing  and . Let , then~
is also a -join in , but , contradicting the minimality of
. Therefore  must be one of the  components of  that contain no
vertices of~.

Now  contains  paths and  edges, so we can
decompose  into  paths of length 1 and 
paths of length~2. We can do this in such a way that the ends of each path lie
in .  Also, by the arguments above, the middle vertex of every path of length~2
lies in a different one of those~ components of~ that do not contain any
vertices of .  Let  be the components of  such that
 is the only component containing vertices of . Note that
.  Let  for .
Choose  and let . Then every vertex
in  has the same degree parity as in  and the graph  is
connected, so  is a solution. Furthermore, , so  is an optimal solution.  This concludes the proof of
Case 2.

\medskip
\noindent
Recall that a minimum -join in~ can be found in  time by
Lemma~\ref{lem:t-join}, so the value of  can be computed in
 time. Note that the constructive proofs for Cases~1 and 2 can be
turned into
 time algorithms, so an optimal solution  can also be found in
 time.
\end{proof}


We are now ready to present the main result of this section. 
Proving this result requires significantly different arguments than the ones used in the proof of Theorem~\ref{thm:add-undi}.
Let  and let  be an instance of \cdpe(). If~
is a -join in , let  and . Then by
Lemma~\ref{lem:struct-undir},  is a semi-solution.  Note that if~ is
a minimum -join in~ then it is a matching in which every vertex of~
is incident to precisely one edge of~, so . We will show
how this allows us to calculate  directly from the structure
of~, without having to find a -join. We will also show that there are
only trivial no-instances for this problem,
namely when~ is odd or~ contains only two vertices.

\begin{theorem}\label{thm:edit-undir}
Let . Then \cdpe can be solved in  time and an
optimal solution (if one exists) can be found in  time.
\end{theorem}

\begin{proof}
Let  and let  be an instance of \cdpe.  By
Lemma~\ref{lem:struct-undir}, we may assume that~ is even, otherwise
 is a no-instance.  If  and , or  and
, then  is a no-instance.  
If  and  then, trivially, , and if
 and  then .
To avoid these trivial
instances, we therefore assume that~ contains at least three vertices.
Under these assumptions we will show that  is always finite and give exact formulas for the value of . 
Let~ be the number of components of~ that do not contain
any vertex of~ and let~ be the number of components of~ that contain
at least one vertex of~.
We  prove the following series of statements.

\begin{itemize}
\item  if ,
\item  if ,
\item  if ,  
, for some , and each edge of~ is a bridge of ,
\item  in all other cases.
\end{itemize}

\smallskip
\noindent
Note that if , then the first statement applies and the trivial
solution  is optimal.
We now consider the remaining three cases separately.

\medskip
\noindent
\displaycase{Case 1:} {\em and .}\\
Then , so by Lemma~\ref{lem:struct-undir} for any
semi-solution~, every vertex in  must have even degree in
.  In other words, every vertex of~ must be incident to an
even number of edges in~.  Since , the graph~ is
disconnected, so any solution  is non-empty.  This means that
 must contain a cycle, so  if a solution
exits.
Suppose . As~ has at least three vertices, it contains a component containing an edge~. Let  be a vertex in its
other component. We set  and  to obtain a solution for .
Since , this solution is optimal. Suppose . Since  must be connected for any
solution~, every component in~ must contain at least one vertex
incident to an edge of~. By Lemma~\ref{lem:struct-undir}, this vertex must
be incident to an even number of edges of~, meaning that it must be
incident to at least two such edges. Therefore .
Indeed, if we choose vertices , one from each component of~,
then setting  and 
gives a solution of size~, which is therefore optimal.
This concludes Case~1.


\medskip
\noindent
\displaycase{Case 2:} {\em ,  for some  and each edge
of~ is a bridge of .}\\
Then  is connected.  Let  be the central vertex of the star and
let  be the leaves.  By Lemma~\ref{lem:struct-undir}, in any
semi-solution , every vertex of  must be incident to an odd number of
edges in , so . Suppose 
is a semi-solution of size . Then  must be a
matching with each edge joining a pair of vertices of . However, then
 for some~. Since , we must have
. However, since  is a bridge of ,  and 
must then be in different components of , so  is not connected
and  is not a solution.  Therefore .

Next we show how to find a solution of size .  Since~ is
even,  must be odd. First suppose that . Since~ is connected and
 is a bridge,  has exactly two components.
Since~ contains at least three vertices, one of these components contains
another vertex . Without loss of generality assume , in which
case  .
Then setting  and  gives a
semi-solution.  Since  are all in the same component of , the
graph  must be connected, so  is a solution. Since
, this solution is optimal.  
Now suppose .
Let  and . Then  is a semi-solution and since
 are all in the same component of , we find that   is a
solution. Since , this solution is
optimal.
This concludes Case 2.

\medskip
\noindent
\displaycase{Case 3:} {\em  and Case~2 does not hold.}\\
Then .
Let  be the components of~ without vertices of~  and let
. Note that  if  and that~ is not
the empty graph, as .  Choose  for .

We first show that .
Since~ has  components, any solution  must contain at least
 edges in  to ensure that  is connected, so
. If  is a solution then every
component~ must contain a vertex incident to some edge in~.  By
Lemma~\ref{lem:struct-undir}, this vertex must be incident to an even number of
edges of , meaning that it must be incident to at least two such
edges. By Lemma~\ref{lem:struct-undir}, every vertex of~ must be incident to
some edge in~. Therefore~ must contain at least
 edges, so .

We now show how to find a solution of size . We
start by finding a maximum matching~ in~. Let~ be the
set of vertices in~ that are not incident to any edge in~. We divide the
argument into two cases, depending on the size of .

\medskip
\noindent
\displaycase{Case 3a:} .\\
In this case, by Lemma~\ref{lem:struct-undir}, setting  and 
gives a semi-solution. Now suppose that , such that~ is not a
bridge in  and the vertices~ and~ are in different components of
.  Let~. Then~ is also a
maximum matching in~. However,  has one component less
than .  Indeed, since~ is not a bridge in , the vertices
 must all be in the same component of . Therefore, if such
edges  exist, we replace~ by~. We do this exhaustively
until no further such pairs of edges exist. At this point either every edge
in~ is a bridge
in 
or every edge in~ is in the same component of .  We
consider these possibilities separately.

First suppose that every edge in~ is a bridge
in . 
Choose  and let
 be the components of , with .  Note that
since every edge in~ is a bridge, . Now let  for
. Let  and let  if  and  otherwise.  Now
every vertex in  has the same degree parity as in , so~ is a
semi-solution by Lemma~\ref{lem:struct-undir}. The graph  is connected,
so~ is a solution. As  , we find that  is an optimal solution.

Now suppose that every edge in~ is in the same component of . Note that
 are the remaining components of .  Choose . Let
 and let  if  and  otherwise.  Then every vertex in
 has the same parity as in  and  is connected, so by
Lemma~\ref{lem:struct-undir}  is a solution. Since
, this solution is optimal.
This concludes Case 3a.

\medskip
\noindent
\displaycase{Case 3b:} .\\
Note that~ must be even since~ is even. Every pair of vertices
in~ must be non-adjacent in~, as otherwise  would not be
maximum. Therefore  is a
clique. Let .

We claim that  is connected. Clearly every vertex of the clique~
must be in the same component of . Suppose for contradiction that 
is a component of~ that does not contain~. Then~ must contain
some edge . However, in this case  is a larger matching in  than ,
which contradicts the maximality of . Therefore~ is connected.

Let . If  then since  is a
clique,  is connected. If  set  and . If 
set  and . Then  is connected, so  is a solution by
Lemma~\ref{lem:struct-undir}. This solution has size ,
so it is optimal.

Now suppose that . Then . If , let  and . Then  is connected, so
 is a solution by Lemma~\ref{lem:struct-undir}. This solution has size
, so it is optimal.
Assume that , so  contains only one component. If~ is
not a bridge in , let  and . Then  is connected,
so  is a solution. This solution has size , so
it is optimal.

Now assume that  is a bridge in . Let  and  denote
the components of  with  and . Note
that  is also a bridge in~.  We claim that the edges of~ are
either all in  or all in~. Suppose for contradiction that  and . Then  would be a larger matching in~ than
, contradicting the maximality of~.  Without loss of generality, we may
therefore assume that all edges of~ are in .

Let , where .  We claim that
 must be adjacent in~ to all vertices of . Suppose
for contradiction that~ is non-adjacent in~ to some vertex of . Since , this vertex would have to be
incident to some edge in~. Without loss of generality, assume . Then  would be a larger
matching in  than , contradicting the maximality of .
Therefore~ is adjacent in~ to every vertex of . In
particular, since , it follows that  and  is connected.

Suppose that every edge between  and  is a bridge in
. Then no two vertices of  can be adjacent, and
. However, then Case~2 applies, which we assumed was not the
case. Without loss of generality, we may therefore assume that  is not
a bridge in . Let  and .
Then  is connected, so  is a solution. Since
, this solution is optimal.
This concludes Case~3b and therefore also concludes Case~3.

\medskip
\noindent
It is clear that  can be computed in  time.  We also
observe that the above proof is constructive, that is, we not only solve the
decision variant of \cdpe() but we can also find an optimal solution.
To do so, we must find a maximum matching in~. This takes
 time by Lemma~\ref{l-mic}.
However, the bottleneck is in Case 3a, where we are glueing components by replacing two matching edges by two other matching edges, which takes  time.
As the total number of times we may need to do this is , this procedure may take  time in total.
Hence, we can obtain an optimal solution in
 time.
\end{proof}

\subsection{The \W-Hard Cases}\label{s-wun}

We first describe the problem used in our \W-hardness construction.
A {\em red/blue graph} is a bipartite graph  whose
vertices are partitioned into independent sets  (the red vertices)
and  (the blue vertices). A non-empty set  is an
{\em odd set} if every vertex in  has an odd number of neighbours in
. The {\sc Odd Set} problem takes as input a red/blue graph  and an integer , and asks whether there is an odd set
 of size at most . This problem is known to be
\NP-complete as well as \W-hard when parameterized
by~~\cite{DowneyFVW99}. For our purposes, we need to show that the same
holds for the following restricted version of the problem.

\begin{center}
\begin{boxedminipage}{.99\textwidth}
\begin{tabular}{rl}
& \textsc{Odd-Sized Odd Set}\\
\textit{~~~~Instance:} & A red/blue graph  where  is odd, and\\
                       & an odd integer .\\
\textit{Question:} & Is there an odd set  such that  and \\ & is odd?\\
\end{tabular}
\end{boxedminipage}
\end{center}

\begin{lemma}
\label{l-oddsized}
{\sc Odd-Sized Odd Set} is \NP-complete as well as \W-hard when
parameterized by .
\end{lemma}

\begin{proof}
The {\sc Odd-Sized Odd Set} problem trivially belongs to \NP. To prove
that the problem is \NP-hard and \W-hard when parameterized
by~, we give a parameterized reduction from {\sc Odd Set}. 
Recall that this problem is \NP-complete as well as \W-hard when
parameterized by~~\cite{DowneyFVW99}.

Given an instance  of {\sc Odd Set}, where  is
a red/blue graph with  and  and  is 
a positive integer, we construct an instance  of {\sc Odd-Sized Odd
Set} as follows. We start with the disjoint union  of two copies
of , where . We then add an independent set
. For each , we make~
adjacent to the two copies of  in . We then add
a vertex  that is made adjacent to all vertices in , as well as
a vertex  that is made adjacent to  only. Let  denote the obtained red/blue graph, where  and . 
Notice that  and  is odd.  We set
. 
Clearly,   is odd.  We claim that  is a yes-instance of {\sc
Odd-Sized Odd Set} if and only if  is a yes-instance of {\sc Odd Set}.

First suppose  is a yes-instance of {\sc Odd Set}. Then there is an odd
set  such that . Consider the set  consisting of the two copies of  in , plus the vertex . For
each vertex , the number of vertices  has in
 equals the number of neighbours the corresponding vertex in  has
in . Since  is an odd set in , this number is odd for every vertex in
. Let . If , then 
has three neighbours in , namely the two copies of  in  and vertex . If , then  is the only
neighbour of  in . Finally,  has exactly one neighbour in ,
namely . This proves that  is an odd set. Since  and  is odd, we conclude that  is a yes-instance of
{\sc Odd-Sized Odd Set}. 

Now suppose that  is a yes-instance of {\sc Odd-Sized Odd Set}, and
let  be an odd set in~ such that  and
 is odd. Since  is the only neighbour of  in~, it holds
that . This implies that every vertex in  must have either
two or zero neighbours in . Let  be the set
consisting of those vertices in  that have exactly two neighbours in
. Since no two vertices in  have a common
neighbour other than~ and , we find that . Let , and let~
and  denote the corresponding vertices in~ and ,
respectively. For each , the two neighbours of  other
than  are exactly the two copies of  in . This implies that
. By the definition of  and the
construction of , every vertex in  has an odd number of
neighbours in . Consequently, every vertex in  has an odd
number of neighbours in . This implies that  is an odd set in  of size
at most .
\end{proof}
We are now ready to prove the hardness results of this section.

\begin{sloppypar}
\begin{theorem}\label{thm:vertex-undir}
Let . Then \cdpe is
\NP-complete and \W-hard when parameterized by , even if
.
\end{theorem}
\end{sloppypar}

\begin{proof}
The \cdpe() problem clearly belongs to \NP. To prove that the problem
is \NP-complete and \W-hard when parameterized by , even if
, we reduce from {\sc Odd-Sized Odd Set}. The latter problem is
\NP-complete as well as \W-hard when parameterized by  due to
Lemma~\ref{l-oddsized}, and this clearly remains true when we assume that
 and every vertex in  has at least one neighbour in
.

\begin{sloppypar}
Let  be an instance of {\sc Odd-Sized Odd Set}, where  is a red/blue graph with  and , and where  and every vertex in  has at least one neighbour in . We construct a graph  as
follows. We start with two copies  of , as
well as  copies  of . Let  and .
For any two vertices  and , we add the edge
 if and only if the corresponding vertices in  are adjacent. For every
vertex , we add an edge between  and  in~ if and
only if  has even degree in~, where  denote the copies of  in
 and , respectively. For every ,
we add an independent set  
of size , and make all the vertices in  adjacent to every
vertex in~. Let . Finally,
we add two vertices  and make each of them adjacent to every vertex in
. This completes the construction of . We define a parity
function  by setting  for every
.
\end{sloppypar}

We will show that  is a yes-instance of \cdpe() if and only
if  is a yes-instance of {\sc Odd-Sized Odd Set}. We first make some
observations about the vertex degrees in . Recall that both 
and  are odd by the definition of {\sc Odd-Sized Odd Set}. With this in
mind, it is easy to verify that every vertex in  has
odd degree, while every vertex in  has even degree.

Suppose  is a yes-instance of {\sc Odd-Sized Odd Set}. Then there exists
an odd set  in  such that  and  is odd.
Fix an arbitrary order on the vertices of . For each , delete from  the copy of the th vertex of
. If , then for each , we delete the copy of
 from  (regardless of whether or not ); since~
is odd and  is odd, we delete an even number of copies of 
in this second step. Let  denote the obtained graph. Observe that we
obtained  from  by deleting exactly  vertices. We claim that~
is Eulerian.

Since we deleted exactly one vertex from each set , the degree of
each vertex in  decreased by exactly~, making the degrees of all
these vertices even. Consider an arbitrary vertex . Recall
that  has odd degree in . The vertex in  corresponding to  has an
odd number of neighbours in  due to the fact that  is an odd set. Exactly
one copy of each of these neighbours was deleted from , plus an additional
even number of copies of  in case . This means that out of all the
neighbours of  in , an odd number are deleted, implying that  has
even degree in . Now consider the degrees of the vertices in . Observe that these vertices form an independent set in
, and every vertex that is deleted from  belongs to this set. Hence,
the parity of the degrees of the vertices in  does
not change, so all these vertices have even degree in . It remains to argue
that  is connected. Recall that we assume that  and every
vertex in  has at least one neighbour in . Since we deleted
exactly one vertex from each set , there is at least one edge in
 between a remaining vertex of  and a vertex in .
This, together with the fact that the vertices in 
are all present in , implies that  is connected. We conclude that 
is Eulerian.

\begin{sloppypar}
For the reverse direction, suppose  is a yes-instance of
\cdpe(). Then there is a sequence  of at most  operations from~
transforming  into a Eulerian graph . We claim that  consists of
exactly  vertex deletions, and that  deletes exactly one vertex from each
set . Recall that each vertex in~ has odd degree in
. 
Let . In order to change the (parity of the) degree of a
vertex , we need to perform (at least) one of the following operations:
\end{sloppypar}
\begin{enumerate}[(i)]
\renewcommand{\theenumi}{(\roman{enumi})}
\renewcommand\labelenumi{\theenumi}
\item delete , 
\item delete an edge incident with , 
\item add an edge incident with , or 
\item delete one of the neighbours of . 
\end{enumerate}
Operations~(i)--(iii) 
leave the parity of at least two vertices in  unaltered.
Hence, from the construction of  and the fact that , it follows
that  deletes exactly one vertex from each set .

Let  denote the set of vertices that are deleted from
 by performing the operations in . Note that , and hence
 has odd size. Let  be the set of those vertices in  of
which  contains an odd number of copies, i.e. . We claim that  is a
solution for the instance  of {\sc Odd-Sized Odd Set}. Since  is
odd,~ must be odd as well. It therefore remains to show that~ is an odd
set in~. For contradiction, suppose there is a vertex  that
has an even number of neighbours in . Consider the copy of  in ; let us denote this copy by . Recall that for every ,
vertex  is adjacent either to all copies of  in  or to none of
these copies. The fact that  has an even number of neighbours in  implies
that  is adjacent to an even number of vertices in . This means that
the degree of  in  has the same parity as the degree of  in~.
Since  has odd degree in  and  is Eulerian, we have thus obtained
the desired contradiction.
\end{proof}


\section{Connected Degree Balance Editing}\label{sec:directed}

Let .  In Section~\ref{s-polyund} 
we will show that
\cdbe() is polynomial-time solvable if  and
in Section~\ref{s-wund} we will show 
that it is \NP-complete
and \W-hard with parameter~ if .

\subsection{The Polynomial-Time Solvable Cases}\label{s-polyund} 

Let  .  
Let
 be an instance of \cdbe() with . Let~ be a set of
arcs not in , and let~ be a set of arcs in~, with  if
.  We say that  is a {\em solution} for  if its
{\em size} , the equation  holds for every vertex~ and
the graph  is connected; if~ is not connected
then  is a {\em semi-solution} for .  Just as in
Section~\ref{s-polyun} 
we consider the optimization version for \cdbe and we use the same terminology.

Let  be an instance of (the optimization version) of \cdbe
where .  Let  be the set of vertices  such that
.  Define a function  by  for every
. 

We construct a directed multigraph  with vertex set~ 
and arc set
determined as follows.
If , for each pair of distinct
vertices~ and~ in~, if , add the arc  to~
(these arcs are precisely those that can be added to~). If ,
for each pair of distinct vertices~ and~, if , add the
arc  to~ (these arcs are precisely those whose reverse can be
deleted from~). Note that adding a
(missing)
arc has the same effect on the degree
balance of the vertices in a digraph as deleting the reverse of the arc 
(if it exists).
Also 
observe that  becomes a directed multigraph rather than a digraph only if
 and there are distinct vertices~ and~ such that  and  applies.  Moreover,~ contains at most two
copies of any arc, and if there are two copies of  then  is not
in .

Let  be a minimum directed
-join in  (if one exists).  Note that~ may contains two copies of
the same arc if  is a directed multigraph.  
Also note that for any pair of vertices
, either  or , otherwise  would be a smaller -join in , contradicting the
minimality of . 

We define two sets  and~ which, as we will show, correspond to a
semi-solution  of .  Initially set .
Consider the arcs in .   
If  contains  exactly once then add  to  
 if  and add  to  if  (in this case  holds).
If  contains two copies of  then add  to  and  to ;
note that by definition of~ and~, in this case  and .
Observe that the sets  and  are not multisets.
We need the following lemma, which consists of seven 
easy observations.

\begin{lemma}\label{l-weakly}
\begin{sloppypar}
Let .  Let  be an
instance of \cdbe where .  Let  be a minimum directed
-join. The following statements hold.
\end{sloppypar}
\begin{enumerate}[(i)]
\renewcommand{\theenumi}{(\roman{enumi})}
\renewcommand\labelenumi{\theenumi}
\item \label{stat:uv-A-uv-not-E} If  then .
\item \label{stat:uv-D-uv-E} If  then .
\item \label{stat:A-D-disjoint}  and moreover,  if and only if  or . 
\item \label{stat:two-copies-F} There are two copies of  in  if and only if  and .
\item \label{stat:no-ed-D-empty} If , then .
\item \label{stat:never-disconnect} If vertices  and  are joined by an arc in  then they are
joined by an arc in .
\item \label{stat:never-disconnect-changed} If  then  and  are connected by an arc in .
\end{enumerate}
\end{lemma}

\begin{proof}
Statements~\ref{stat:uv-A-uv-not-E} and~\ref{stat:uv-D-uv-E} follow directly from the definitions of  and
, respectively.  The fact that  follows directly
from Statements~\ref{stat:uv-A-uv-not-E} and~\ref{stat:uv-D-uv-E}. The second part of Statement~\ref{stat:A-D-disjoint} follows
directly from the definitions of  and . Statement~\ref{stat:two-copies-F} follows
directly from the definition of  and~.

To prove Statement~\ref{stat:no-ed-D-empty}, suppose for contradiction that  and
. By Statement~\ref{stat:uv-D-uv-E}, .  
Since ,  can contain at most one copy of . By definition of  and , it follows that  and .
However, since  and ,  is not an arc in
 by definition of . Therefore  cannot be an -join in ,
which is a contradiction.

Next we consider Statement~\ref{stat:never-disconnect}. First suppose that . If
 and~ are not connected by an arc in , then . Then, by Statement~\ref{stat:A-D-disjoint}, . However, as stated
earlier, this cannot happen, since  is minimum. Now suppose  and
. If  and  are not connected by an arc in ,
then . By Statement~\ref{stat:A-D-disjoint}, . Then  must contain two copies of , since , so .
However in this case  and  are
connected by an arc in . This completes the proof of Statement~\ref{stat:never-disconnect}.

Finally, we consider Statement~\ref{stat:never-disconnect-changed}. Suppose . If 
then by Statement~\ref{stat:A-D-disjoint},  is an arc in . Otherwise, by
Statement~\ref{stat:A-D-disjoint}, , so  by Statement~\ref{stat:uv-D-uv-E}. However,
in this case Statement~\ref{stat:never-disconnect} implies that  and  are connected by an arc in
.
\end{proof}

If  and  are sets, then   is the multiset that consists of one
copy of each element that occurs in exactly one of  and  and two copies
of each element that occurs in both.

The next lemma provides the starting point for our algorithm.

\begin{lemma}\label{l-onetoone}
\begin{sloppypar}
Let  .  Let  be an
instance of \cdbe where . The following holds:
\end{sloppypar}
\begin{enumerate}[(i)]
\renewcommand{\theenumi}{(\roman{enumi})}
\renewcommand\labelenumi{\theenumi}
\item  \label{stat:f-join-is-semisolution} If  is a minimum directed -join in , then  is
a semi-solution for  of size~.
\item \label{stat:semisolution-is-f-join} If  is a semi-solution for , then 
is a directed -join in~ of size .
\end{enumerate}
\end{lemma}

\begin{proof}
First consider Statement~\ref{stat:f-join-is-semisolution}.  Suppose  is a minimum directed -join
in~.  By Lemma~\ref{l-weakly}~\ref{stat:A-D-disjoint} and~\ref{stat:two-copies-F},  has size
.

Let .  Let .  Let  and  be the sets of
arcs in  with~ as tail or head, respectively, that were put into~.
Let  and  be the set of arcs in  with~ as tail or head,
respectively, whose reverse was put into .

Suppose . Define  if  is not in  and  if .
Then by the
definition of a directed -join, we have
5pt]
&= d^+_{G_S(F)}(u)-d^-_{G_S(F)}(u)\
If  then  by Lemma~\ref{l-weakly}~\ref{stat:uv-A-uv-not-E}.  If
 then  by Lemma~\ref{l-weakly}~\ref{stat:uv-D-uv-E}. Moreover, in
that case, .  Consequently, we find that
5pt]
5pt]
&= d^+_G(u)-d^-_G(u)+\delta(u)-(d^+_G(u)-d^-_G(u))\

d^+_G(u)-d^-_G(u)+|A^+(u)|-|A^-(u)| - (|D^+(u)|-|D^-(u)|)\
\hspace*{5em}&= d^+_H(u)-d^-_H(u)\5pt]
&= d^+_G(u)-d^-_G(u)+f(u),

where we define  if .  This leads to
 Let .  Suppose
 appears once in .  Let . Then . By
definition,~ contains .  Let . Then , so . By
definition,~ contains . Suppose  appears twice in .
Then  and . Hence,  and ,
and moreover, .  Then  appears twice in .  We
conclude that  is a subset of the arcs in .  Let  and
 be the set of arcs in  with  as tail or head, respectively.
Then  and . We find that, for all
,
5pt]
&= |A^+(u)|-|A^-(u)|-(|D^+(u)|-|D^-(u)|)\
Hence,  is a directed -join.  It follows from the corresponding
definitions that the size of  is .
This completes the proof of Lemma~\ref{l-onetoone}.
\end{proof} 

 
Let  be an instance of \cdbe.  Let~ be the
number of components of~ that contain no vertex of~.
Let~ be the number of components of~ that contain at least
one vertex of~.
Let .

We now state the following lemma;
its proof is based on Lemmas~\ref{lem:dir-t-join},~\ref{l-weakly} and~\ref{l-onetoone}.

\begin{sloppypar}
\begin{lemma}\label{l-algo}
Let  .  Let  be an
instance of \cdbe with .  If~ is a (given) minimum directed -join in
, then  has a solution that has size at most
, which can be found in
 time.
\end{lemma}
\end{sloppypar}

\begin{proof}
Let  be a minimum directed -join in . If  is connected,
then the statement of the theorem holds by Lemma~\ref{l-onetoone}.  Suppose~
is not connected.  We will try to replace arcs in  to obtain a different
minimum directed -join~ such that  will have fewer
components. Either this will eventually cause the graph to be connected (in
which case the corresponding solution will still have size~), or else the
structure of this directed -join will enable us to find a solution for
\cdbe() of size either  or . Our changes to  will
be such that no additional arcs are ever added to the corresponding set .
Thus, if , then the property  will be preserved.

By Lemma~\ref{lem:dir-t-join},  must only consist of mutually
arc-disjoint directed paths from vertices~ with  to vertices 
with . We claim that all such paths must be of length at most~2.
Suppose, for contradiction, that there is a directed path of length at least~3
in  from some vertex~ to some vertex~. Note that  and  are
in the same component of . Since  is not  connected, there must be a
vertex  in some other component of~. By Lemma~\ref{l-weakly}~\ref{stat:never-disconnect}, this
means that  is not in the same component of  as~ or , so 
and  are arcs in .  Replacing the directed path from  to  in
 by the arcs  would yield a smaller directed -join in ,
which is a contradiction.  Therefore all directed paths in  must be of
length at most~2. 

Let  and  be arcs in .  Note that by
Lemma~\ref{l-weakly}~\ref{stat:never-disconnect-changed},  and  are in the same component of  and
 and  are in the same component of .  Suppose that  and
 are chosen such that  and  are in a different component of 
from the one containing   and~ and that one of the following situations
holds:
\begin{enumerate}[(i)]
\renewcommand{\theenumi}{(\roman{enumi})}
\renewcommand\labelenumi{\theenumi}
\item \label{case:uv-in-A-no-bridge} either  and  is not a bridge in , or
\item \label{case:vu-in-D} .
\end{enumerate}
By Lemma~\ref{l-weakly}~\ref{stat:never-disconnect}, vertex  is not in the same component of  as
 and vertex~ is not in the same component of  as . Hence, by the
definition of , the arcs  and  are in .  As such, we
may replace  and  in  by  and . This yields
another minimum directed -join in  which, as we explain below, reduces
the number of components in  by one.  Because~ and  are not in the
same components of  as  or , adding  and   to 
means that these two arcs will be put into .  Suppose~\ref{case:uv-in-A-no-bridge} holds. Then the
vertices in the original component of~ that contained  and  will still
be connected, whereas the vertices in the original component of~ that
contained~ and  will still be connected as well (if necessary via a path
that uses the new arcs  and ). Thus,  has one component
less.  Suppose~\ref{case:vu-in-D} holds.  Then removing  from~ means removing it
from . Hence, in , the arc  is restored and we can apply the
same arguments.

We apply the above replacement operation exhaustively. At termination, we have
modified~ into a minimum directed -join of , in which either every
arc in  will be a bridge in  and , or the end-vertices
of every arc in  will all be in the same component of~.  We discuss these
two cases separately.

\medskip
\noindent
\displaycase{Case 1:} {\em Every arc in  is a bridge in  and }.\\ 
Then . We claim that every directed path in  has length~1.  For
contradiction, suppose  and  are two arcs in .  Since both
 and  are bridges in , we must have that  is not an arc
in .  Then replacing  and  in  by  would yield a
smaller directed -join in , which would contradict the minimality of
.

As every directed path in  has length~1, every arc  must
be such that  and .  Hence,  contains exactly 
arcs.

Let  be the components of~. Because every arc in  is a
bridge in  and , we find that .  Suppose
. Then~ is connected, so .  Hence we
have a solution for \cdbe that uses  arcs.  Suppose .
Choose an arc   arbitrarily and assume without loss of
generality that  and  are in~. Next, choose a vertex  in 
for .  Replace the arc  in  by the arcs
.  This gives a solution for
\cdbe that uses 
arcs. 

\medskip
\noindent
\displaycase{Case 2:} {\em The end-vertices of each arc in  are all in the same component~of~}.\\  
Suppose  has at least one other component; let  be a vertex in such a
component.  Suppose that  and  are two distinct arcs in  such
that the following situation holds:~ and~ are in the same component of
the graph obtained from  after removing  and .  Because  is
a minimum directed -join,  and  are distinct vertices.  By
Lemma~\ref{l-weakly}~\ref{stat:never-disconnect}, vertices  and  are not in the component of 
that contains .  Hence, by the definition of~, the arcs  and
 are in .  As such, we may replace  and  in~ by
 and .  This yields another minimum directed -join in 
which, as we explain below, reduces the number of components in  by one.

Because  and  are not in the component of  that contains , we find
that  and  will be put into . Because  is a minimum
directed -join,  must be in  already, so  or . By Lemma~\ref{l-weakly}~\ref{stat:never-disconnect} and~\ref{stat:never-disconnect-changed},~ and  are still in the
same component after our replacement.  Consequently, all vertices~
will be in the same  component.  Hence, the number of components in  is
reduced by one.

We apply the above replacement operation exhaustively.  If  becomes
connected, then since  is (still) a minimum directed -join, we have found
a solution of size .  Assume  does not become connected. Then, at
termination of our procedure, we have obtained the following situation.  For
every two distinct arcs  and , we have that  and  are in
different components of the graph~ obtained from  after removing 
and . Moreover,  is in the same component of  as~ (by our
earlier arguments, we have that ).

Let  be the component of~ that contains~.
We claim that  and , and that  contains no vertices incident to arcs in . This can be seen as follows.  Because~ does not contain  or , we find that  and  are both
in  due to Lemma~\ref{l-weakly}~\ref{stat:never-disconnect-changed}. If~
contains a vertex incident to some arc in , then this component must also contain the other end-vertex of
this arc by Lemma~\ref{l-weakly}~\ref{stat:never-disconnect-changed}. Suppose  are in~ and . (Note
that we do not insist that  or .) Then we find a smaller
directed -join of  by replacing ,  and  in~
by the arcs  and  (which are not in  already due to Lemma~\ref{l-weakly}~\ref{stat:never-disconnect}). This contradicts the
minimality of .

We now do as follows.  Recall that every directed path in  has length at
most~2.  Hence, we can partition  into  arcs  with  and
 and  pairs of arcs  with  and
.  We deduced above that every directed path ,  reduces
the number of components in  by one.  Hence, the number of components in~
is .  

Let  be the components of~ that do not contain any
vertex~ with . Note that .
Because~ is not connected and every vertex  with
 belongs to the same component of~, we find that .
Choose an arbitrary arc  from~ and for ,  choose
an arbitrary vertex  in .  Remove  from  if 
or add  to   otherwise (by Lemma~\ref{l-weakly}~\ref{stat:A-D-disjoint} 
if ).  Add the arcs
 to~.  This gives a
solution for \cdbe that uses
 arcs.

\medskip
\noindent
It is readily seen that all steps in the algorithm described above cost 
time. 
This completes the proof of Lemma~\ref{l-algo}.
\end{proof}


\noindent
The next result is our first main result of this section. 
We prove it by showing that the upper bound in Lemma~\ref{l-algo} is also a lower bound for (almost) any instance of \cdbe with  that has a semi-solution.


\begin{theorem}\label{thm:edit-dir}
For ,  \cdbe can be
solved in time .
\end{theorem}


\begin{proof}
\begin{sloppypar}
Let  , and let  be an
instance of \cdbe.  We first use Lemma~\ref{lem:dir-t-join} to check
\end{sloppypar}
whether  has a directed -join.  Because  has at most  arcs,
this takes  time.  If  has no directed -join
then  has no semi-solution by Lemma~\ref{l-onetoone}, and thus no
solution either.  
Assume that  has a directed -join, and let  be a minimum directed
-join that can be found in time  by Lemma~\ref{lem:dir-t-join}. 
As before,~ denotes the
number of components of~ that do not contain any vertex of~, while
 is the number of components of~ that contain at least
one vertex of~, 
and . 

We will prove the following series of statements.

\begin{itemize}
\item  if , ,
\item  if , ,
\item  if .
\end{itemize}

If  and  then  is an optimal solution.  If
 and , to ensure connectivity and preserve degree balance, for
every component of~ there must be at least one arc whose head is in this
component and at least one arc whose tail is in this component, thus any
solution must contain at least~ arcs. Let  be the components
of  and arbitrarily choose vertices  for . Let 
and .  Then  is a solution which has size~ and is
therefore optimal.

Suppose .  By Lemma~\ref{l-algo} we find a solution  for
 of size at most  in 
time.  Hence, the total running time is , and it
remains to show that any solution has size at least
. 

Let  be an arbitrary solution. Then  is also semi-solution. Every
semi-solution has size at least~ by Lemma~\ref{l-onetoone}~\ref{stat:semisolution-is-f-join}. Therefore
 has size at least .

Since there are  components in , we must add at least  arcs to
ensure  is connected.  Therefore  has size at least . 

Finally, for every vertex  with 
(resp. ) we find that  must be such that at least  arcs
are either in  and have  as a tail (resp. head) or else are in~ and
have  as a head (resp. tail). For every component containing only vertices
 with , there must be at least one arc in  whose head is in this
component and at least one arc in  whose tail is in this component (to
ensure connectivity and to ensure that the degree balance is not changed for any
vertex in this component).  Therefore we have that  has size at least
. This completes the proof of Theorem~\ref{thm:edit-dir}.
\end{proof}


\subsection{The \W-Hard Cases}\label{s-wund}

Recall that Cygan et al.~\cite{CyganMPPS14} proved that \cdbe() is
\NP-complete and \W-hard when parameterized by , even when
. Our next results shows that this remains true if we allow not
only vertex deletions, but also edge deletions and/or edge additions.

\begin{sloppypar}
\begin{theorem}\label{thm:vertex-dir}
Let . Then \cdbe is
\NP-complete and \W-hard when parameterized by , even if
.
\end{theorem}
\end{sloppypar}

\begin{proof}
Let . The \cdbe() problem
trivially belongs to \NP. To prove hardness, we describe a parameterized
reduction from {\sc Directed Balanced Node Deletion}. This problem takes as
input a digraph  and an integer , and asks whether there exists a
set~ of at most  vertices whose deletion yields a balanced digraph. This
problem is known to be \NP-complete and \W-hard with
parameter~~\cite{CyganMPPS14}.  

Let  be an instance of {\sc Directed Balanced Node Deletion}, and let
. We construct a digraph  as follows. We start with a copy of
, where for every , we write~ to denote the copy of  in
. Let . We add  isolated vertices
. 
For each , we construct a gadget  consisting of
vertices  and arcs  and
 for every .  We make every vertex  adjacent to each of the gadgets by adding arcs
 and  for every . This completes the
construction of . We define a function 
by setting  for every .

We claim that  is a yes-instance of \cdbe() if and only if
 is a yes-instance of {\sc Directed Balanced Node Deletion}.

First suppose  is a yes-instance of {\sc Directed Balanced Node
Deletion}. Then there is a set  of size at most  such
that  is balanced. We define a set  of size~ as
follows. If , then we set . If , then we set . We claim that  is Eulerian. Since the gadgets
are connected and every vertex outside the gadgets is adjacent to each of the
gadgets, it is clear that  is connected. It remains to show that every
vertex in  is balanced. In~, the in- and out-degrees of each vertex
 
equal  and , respectively, while the in- and out-degrees of each vertex
 equal  and , respectively.  Since each of the  vertices in
 is an in-neighbour of  and an out-neighbour of , it holds that 
 for
each .  All other vertices in the gadgets, already
balanced in , remain balanced in . The same holds for the vertices
in ; the in- and out-degree of each of these
vertices, both in  and in , equals . For every vertex , it holds that  and
. Since  for
every  due to the assumption that  balanced, it
holds that every  is balanced in . We conclude
that  is Eulerian.

For the reverse direction, suppose there exists a sequence  of operations
from  that transforms~ into a Eulerian digraph. We first argue that 
deletes exactly  vertices from . 
As we mentioned before, the in- and out-degrees of each vertex  in 
equal  and  in~, respectively, while the in- and out-degrees of
each vertex~ in  equal  and , respectively.  Since  by
assumption, this means that the operations in  need to either delete or
balance each of the  vertices in the set
. Since  and each edge
deletion or edge addition changes the degree of at most two vertices in ,
there is a gadget  such that  neither deletes a vertex of  nor
adds or deletes an edge incident with any of the vertices of~. The fact
that the vertices of~, and  and  in particular, are balanced
after applying the operations in  implies that  deletes exactly 
in-neighbours of~ (all of which are out-neighbours of ). We conclude
that~ deletes exactly  vertices from .

Let  be the set of at most  vertices that are deleted
from~ by~, and let  be the corresponding
set of vertices in . Let . From the construction of
, it holds that 
 and
.  Since
, we have that .
This shows that  is balanced, and hence  is a yes-instance of {\sc
Directed Balanced Node Deletion}.
\end{proof}


\section{Conclusions}\label{sec:concl}

By extending previous work~\cite{BoeschST77,CaiY11,CyganMPPS14} we completely
classified both the classical and parameterized complexity of \cdpe() and
\cdbe(), as summarized in Table~\ref{t-thetable}.  Our work followed the
framework used~\cite{Golovach13,MathiesonS12} for  {\sc (Connected) Degree
Constraint Editing()}. 
Our study was motivated by Eulerian graphs. As such,
the variants \dpe() and \dbe() of \cdpe() and \cdbe(),
respectively, in which the graph  is no longer required to be connected,
were beyond the scope of this paper.  It follows from results of Cai and
Yang~\cite{CaiY11} and Cygan~\cite{CyganMPPS14}, respectively, that for
, \dpe() and  \dbe() are \NP-complete and, when parameterized
by , \W-hard, whereas they are polynomial-time solvable for 
as a result of Lemmas~\ref{lem:t-join} and~\ref{lem:dir-t-join}, respectively.
The problems \dpe and \dbe are also polynomial-time solvable if
; this is in fact proven by combining
Lemmas~\ref{lem:t-join} and~\ref{lem:struct-undir} for the undirected case, and
Lemmas~\ref{lem:dir-t-join} and~\ref{l-onetoone} for the directed case.  We
expect the remaining (hardness) results of Table~\ref{t-thetable} to carry over
as well.

Let  be an integer. Here is a natural generalization of \cdpe().

\begin{center}
\begin{boxedminipage}{.99\textwidth}
\begin{tabular}{rl}
\textsc{-CDME():} & \textsc{Connected Degree Modulo--Editing}\\
\textit{~~~~Instance:} & A graph , integer  and\\ & 
                        a function
                        .\\
\textit{Question:} & Can  be -modified into a connected graph \\& with 
          for each ?
                   \end{tabular}
\end{boxedminipage}
\end{center}
Note that -\textsc{CDME}() is \cdpe().  
The following theorem shows
that the complexity of -\textsc{CDME}() may differ from
-\textsc{CDME}().

\begin{theorem}\label{thm:mod-3}
-\textsc{CDME} is \NP-complete even if .
\end{theorem}

\begin{proof}
Reduce from the \textsc{Hamiltonicity} problem, which is \NP-complete for
connected cubic graphs~\cite{GareyJ79}.  Let  be a connected cubic graph.
Let  for every , and take .  Then 
has a Hamiltonian cycle if and only if  can be -modified into a
connected graph  with  for all .
\end{proof}

\begin{sloppypar}
It is natural to ask whether
-\textsc{CDME} is fixed-parameter tractable with parameter~.
\end{sloppypar}

Finally, another direction for future research is to investigate how the complexity of  \cdpe() and \cdbe() changes
if we permit other graph operations, such as edge contraction, to be in the set~.
For instance, Belmonte et al.~\cite{BGHP14} considered this operation and
obtained the first results extending the work of Mathieson and Szeider~\cite{MathiesonS12} in this direction.

\bibliography{editing-euler}
\end{document}
