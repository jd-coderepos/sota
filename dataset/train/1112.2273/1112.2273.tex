\documentclass[envcountsame]{llncs}

\usepackage{amsfonts,epsfig}
\usepackage{enumerate}
\usepackage{comment}

\begin{document}

\title{Steiner Forest Orientation Problems \thanks{A preliminary version of this article was presented at the 20th European Symposium on Algorithms (ESA 2012).}}

\author{Marek Cygan\inst{1}
 \thanks{Partially supported by National Science Centre grant no. N206 567140, Foundation for Polish Science, ERC Starting Grant NEWNET 279352, NSF CAREER award 1053605, DARPA/AFRL award FA8650-11-1-7162 and ONR YIP grant no. N000141110662.}
\and 
Guy Kortsarz\inst{2} \thanks{Partially supported by NSF support grant award number 0829959.}
\and 
Zeev Nutov\inst{3}
        }

\institute{
  IDSIA, University of Lugano, Switzerland \ \email{marek@idsia.ch}
\and
Rutgers University, Camden \                             \email{guyk@camden.rutgers.edu}
\and
The Open University of Israel \                          \email{nutov@openu.ac.il}
           }


\maketitle

\newcommand {\ignore} [1] {}


\newcommand{\cur}{{\rm cur}}

\begin{abstract}
We consider connectivity problems with orientation constra\-ints.
Given a directed graph  and a collection of ordered node pairs 
let .
In the {\sf Steiner Forest Orientation} problem we are given  
an undirected graph  with edge-costs and a set  of ordered node pairs. 
The goal is to find a minimum-cost subgraph  of  and an orientation  of  
such that . We give a -approximation algorithm for this problem.

In the {\sf Maximum Pairs Orientation} problem we are given a  
graph  and a multi-collection of ordered node pairs  on .
The goal is to find an orientation  of  such that  is maximum.
Generalizing the result of Arkin and Hassin~[DAM'02] for , we will show that 
for a mixed graph  (that may have both directed and undirected edges),
one can decide in  time whether  has an orientation  with 
(for undirected graphs this problem admits a polynomial time algorithm for any , but it is NP-complete on mixed graphs).
For undirected graphs, we will show that one can decide 
whether  admits an orientation  with  
in  time; hence this decision problem is 
fixed-parameter tractable, which answers an open question from Dorn et al.~[AMB'11].
We also show that {\sf Maximum Pairs Orientation} admits ratio ,
which is better than the ratio  of Gamzu et al.~[WABI'10] when .

Finally, we show that the following node-connectivity problem can be solved in polynomial time:
given a graph  with edge-costs, , and an integer , 
find a min-cost subgraph  of  with an orientation  
such that  contains  internally-disjoint -paths, 
and  internally-disjoint -paths.
\end{abstract}


\section{Introduction}

\subsection{Problems considered and our results}

We consider connectivity problems with orientation constraints.
Unless stated otherwise, graphs are assumed to be undirected (and may not be simple), 
but we also consider directed graphs,
and even {\em mixed graphs}, which may have both directed and undirected edges.
Given a mixed graph , an {\em orientation of } is a directed graph  
obtained from  by assigning to each undirected edge one of the two possible directions.
For a mixed graph  on node set  and a multi-collection of 
ordered node pairs (that is convenient to consider as a set of directed edges)  on  
let  denote the subset of the pairs (or edges) in 
for which  contains a -path. We say that  {\em satisfies}  if ,
and that  is {\em -orientable} if  admits an orientation  that satisfies .
We note that for undirected graphs 
it is easy to check in polynomial time whether  is -orientable, cf. \cite{HM}
and Section~\ref{s:mix} in this paper.
Let  denote the number of nodes in  and  
the total number of edges and arcs in  and ordered pairs in .

Our first problem is the classic {\sf Steiner Forest} problem with orientation constraints.

\vspace{0.1cm}

\begin{center} 
\fbox{
\begin{minipage}{0.96\textwidth}
{\sf Steiner Forest Orientation} \\
{\em Instance:} 
A graph  with edge-costs and a set  of 
\hphantom{\em Instance:} \  
ordered node pairs. \\
{\em Objective:}
Find a minimum-cost subgraph  of  with an orientation  that 
\hphantom{\em Objective:} satisfies .
\end{minipage}
}
\end{center}

\begin{theorem} \label{t:min}
{\sf Steiner Forest Orientation} admits a -approximation algorithm. 
\end{theorem}

Our next bunch of results deals with maximization problems of finding an orientation that 
satisfies the maximum number of pairs in . 

\vspace{0.1cm}

\begin{center} 
\fbox{
\begin{minipage}{0.96\textwidth}
{\sf Maximum Pairs Orientation} \\
{\em Instance:} \ 
A graph  and a multi-collection of ordered node pairs (i.e., a set 
\hphantom{\em Instance:} \ of directed edges)  on~. \\
{\em Objective:}
Find an orientation  of  such that the number  of pairs 
\hphantom{\em Objective:} satisfied by  is maximum.
\end{minipage}
}
\end{center}

\vspace{0.1cm}

Let {\sf  Pairs Orientation} be the decision problem of determining whether 
{\sf Ma\-xi\-mum Pairs Orientation} has a solution of value at least .
Let {\sf -Orientation} be the decision problem of determining whether 
 is -orientable (this is the {\sf  Pairs Orientation} with ).
As was mentioned, for undirected graphs {\sf -Orientation} can be easily decided 
in polynomial time \cite{HM}.
Arkin and Hassin~\cite{H1} proved that on mixed graphs, {\sf -Orientation}
is NP-complete, but it is polynomial-time solvable for .
Using new techniques, we widely generalize the result of \cite{H1} as follows.

\begin{theorem} \label{t:mix}
Given a mixed graph  and  one can decide in  time
whether  is -orientable;
namely, {\sf -Orientation} with a mixed graph  can be decided in  time.
In particular, the problem can be decided in polynomial time for any instance with constant .
\end{theorem}

In several papers, for example in \cite{ZV}, 
it is stated that any instance of {\sf Ma\-xi\-mum Pairs Orientation} 
admits a solution  such that . 
Furthermore Gamzu et al.~\cite{segev-approx} show that
{\sf Maximum Pairs Orientation} admits an -approximation algorithm.
In \cite{fpt} it is shown that {\sf  Pairs Orientation} is fixed-parameter 
tractable\footnote{``Fixed-parameter tractable'' means the following. 
In the parameterized complexity setting, an instance of a decision problem 
comes with an integer parameter .
A problem is said to be {\em fixed-parameter tractable} (w.r.t. ) if there exists an algorithm 
that decides any instance  in time  for some (usually exponential) 
computable function .}
when parameterized by the maximum number of pairs that can be connected via one node.
They posed an open question if the problem is fixed-parameter tractable when parameterized by 
(the number of pairs that should be connected),
namely, whether {\sf  Pairs Orientation} can be decided in  time, 
for some computable function .
Our next result answers this open question, and for  improves the approximation 
ratio  for {\sf Maximum Pairs Orientation} of \cite{ZV,segev-approx}. 

\begin{theorem} \label{t:kernel}
Any instance of {\sf Maximum Pairs Orientation} 
admits a solution , that can be computed in polynomial time, such that .
Furthermore
\begin{itemize}
\item[{\em (i)}]
{\sf  Pairs Orientation} can be decided in  time;
thus it is fixed-parameter tractable when parameterized by .
\item[{\em (ii)}]
{\sf Maximum Pairs Orientation} admits an -approximation algorithm.
\end{itemize}
\end{theorem} 

Note that  may be much smaller than , say .
While this size of  does not allow exhaustive search in time 
polynomial in , we do get an approximation ratio of 
, 
which is better than the ratio  of Gamzu et al.~\cite{segev-approx}.

One may also consider ``high-connectivity'' orientation problems,
to satisfy prescribed connectivity demands.
Several papers considered min-cost edge-connectivity orientation problems, cf. \cite{KNS}.
Almost nothing is known about min-cost node-connectivity orientation problems.
We consider the following simple but still nontrivial variant.

\vspace{0.1cm}

\begin{center} 
\fbox{
\begin{minipage}{0.96\textwidth}
{\sf  Disjoint Paths Orientation} \\
{\em Instance:} \  
A graph  with edge-costs, , and an integer . \\
{\em Objective:}
Find a minimum-cost subgraph  of  with an orientation~ 
\hphantom{\em Objective:} such that  contains  internally-disjoint -paths, 
and  internally-\hphantom{\em Objective:} disjoint  -paths.
\end{minipage}
}
\end{center}

\vspace{0.1cm}

Checking whether {\sf  Disjoint Paths Orientation} admits a feasible solution can 
be done in polynomial time using the characterization of feasible solutions of 
Egawa, Kaneko, and Matsumoto \cite{EKM} (see Theorem~\ref{t:EKM} in Section~\ref{s:nc});
we use this characterization to prove the following.

\begin{theorem} \label{t:nc}
{\sf  Disjoint Paths Orientation} can be solved in polynomial time.
\end{theorem}

Theorems \ref{t:min}, \ref{t:mix}, \ref{t:kernel} and \ref{t:nc}, 
are proved in
sections \ref{s:min}, \ref{s:mix}, \ref{s:kernel} and \ref{s:nc}, 
respectively.


\subsection{Previous and related work}

Let  denote the -edge-connectivity in a graph ,
namely, the ma\-xi\-mum number of pairwise edge-disjoint -paths in .
Similarly, let  denote the -node-connectivity in , 
namely, the maximum number of pairwise internally node-disjoint -paths in .
Given an edge-connectivity demand function , we say that 
 {\em satisfies}  if  for all ; similarly, 
for node connectivity demands, we say that  {\em satisfies}  
if  for all .

\vspace{0.1cm}

\begin{center} 
\fbox{
\begin{minipage}{0.96\textwidth}
{\sf Survivable Network Orientation} \\
{\em Instance:} \ 
A graph  with edge-costs and edge/node-connectivity 
\hphantom{\em Instance:} \ demand function . \\
{\em Objective:}
Find a minimum-cost subgraph  of  with orientation  that 
\hphantom{\em Objective:} satisfies .
\end{minipage}
}
\end{center}

\vspace{0.1cm}

So far we assumed that the orienting costs are symmetric; 
this means that orienting an undirected edge connecting  and  
in each one of the two directions is the same, namely, that .
This assumption is reasonable in practical problems,  
but in a theoretic more general setting, we might have non-symmetric costs .
Note that the version with non-symmetric costs includes the min-cost version 
of the corresponding directed connectivity problem, and also the case when the input graph 
 is a mixed graph, by assigning large/infinite costs to non-feasible orientations.
For example, {\sf Steiner Forest Orientation} with non-symmetric costs 
includes the {\sf Directed Steiner Forest} problem,
which is {\sf Label-Cover} hard to approximate \cite{DK}.
This is another reason to consider the symmetric costs version.

Khanna, Naor, and Shepherd \cite{KNS} considered several orientation problems with non-symmetric costs.
They showed that when  is required to be -edge-outconnected from a given roots 
(namely,  contains  edge-disjoint paths from  to every other node),
then the problem admits a polynomial time algorithm.
In fact they considered a more general problem of finding an orientation that covers 
an intersecting supermodular or crossing supermodular set-function.
See \cite{KNS} for precise definitions. Further generalization of this result due to 
Frank, T.~Kir\'{a}ly, and Z.~Kir\'{a}ly was presented in \cite{FKK}.
For the case when  should be strongly connected, \cite{KNS} obtained 
a -approximation algorithm; note that our {\sf Steiner Forest Orientation} problem 
has much more general demands, that are {\em not} captured by  
intersecting supermodular or crossing supermodular set-functions, but we 
consider symmetric edge-costs (otherwise the problem includes the {\sf Directed Steiner Forest} problem). 
For the case when  is required to be 
-edge-connected, , \cite{KNS} obtained a pseudo-approximation algorithm 
that computes a -edge-connected subgraph of cost at most  times the cost 
of an optimal -connected subgraph.

We refer the reader to \cite{FK} for a survey on characterization of graphs that admit orientations satisfying 
prescribed connectivity demands, and here mention only 
the following central theorem,
that can be used to obtain a pseudo-approximation for edge-connectivity orientation problems.

\begin{theorem} [Well-Balanced Orientation Theorem, Nash-Williams \cite{NW}] \label{t:NW} \ 
Any undirected graph  has an orientation  for which

for all . 
\end{theorem}

We note that given , an orientation as in Theorem~\ref{t:NW} can be computed in polynomial time.
It is easy to see that if  has 
an orientation  that satisfies  then  satisfies the demand function  defined by 
.
Theorem~\ref{t:NW} implies that edge-connectivity {\sf Survivable Network Orientation} admits 
a polynomial time algorithm that computes a subgraph  of  and an orientation  of 
such that  and  

This is achieved by applying Jain's \cite{Jain} algorithm to compute a 
-approximate solution  for the corresponding undirected 
edge-connectivity {\sf Survivable Network} instance with demands ,
and then computing an orientation  of  as in Theorem~\ref{t:NW}.
This implies that if the costs are symmetric, then by cost at most 
we can satisfy almost half of the demand of every pair, 
and if also the demands are symmetric then we can satisfy all the demands.
The above algorithm also applies for non-symmetric edge-costs, 
invoking an additional cost factor of .
Summarizing, we have the following observation, which we failed to find in the literature. 

\begin{corollary} \label{c:NW}
Edge-connectivity {\sf Survivable Network Orientation} (with non-sym\-me\-tric costs) admits 
a polynomial time algorithm that computes a subgraph  of  and an orientation  of 
such that  and  
 for all . 
In particular, the problem  admits a -approximation algorithm if both 
the costs and the demands are symmetric.
\end{corollary}



\section{Algorithm for  {\sf Steiner Forest Orientation} (Theorem~\ref{t:min})}

\label{s:min}

In this section we prove Theorem \ref{t:min}. 
For a mixed graph or an edge set  on a node set  and  let
 denote the set of all (directed and undirected) edges in  from  to 
and let  denote their number;
for brevity,  and ,
where .

Given an integral set-function  on subsets of  we say that 
 covers  if  for all .
Define a set-function  by  and for every 

Note that the set-function  is symmetric, namely, that  for all .

\begin{lemma} \label{l:feasible}
If an undirected edge set  has an orientation  that satisfies an edge-connectivity demand function  then  covers . 
\end{lemma}
\begin{proof}
Let . By Menger's Theorem, any orientation  of  that satisfies 
has at least  edges from  to ,
and at least  edges from  to .
The statement follows.
\qed
\end{proof}

Recall that in the {\sf Steiner Forest Orientation} problem 
we have  if  and  otherwise. 
We will show that if 
then the inverse to Lemma~\ref{l:feasible} is also true, namely,
if  covers  then  has an orientation that satisfies ;
for this case, we also give a -approximation algorithm for the problem of computing 
a minimum-cost subgraph that covers .
We do not know if these results can be extended for .

\begin{lemma} \label{l:1}
For , if an undirected edge set  covers  then  has an orientation that satisfies .
\end{lemma}
\begin{proof}
Observe that if  (namely, if ) 
then  belong to the same connected component of .
Hence it sufficient to consider the case when  is connected.
Let  be an orientation of  obtained as follows.
Orient every -edge-connected component of  to be strongly connected
(recall that a directed graph is strongly connected if there is a directed path from 
any of its nodes to any other); this is possible by Theorem~\ref{t:NW}.
Now we orient the bridges of . 
Consider a bridge  of . 
The removal of  partitions  into two connected
components .
Note that  or 
, since .
If , we orient  from  to ;
if , we orient  from  to ;
and if , we orient  arbitrarily.
It is easy to see that the obtained orientation  of  satisfies .
\qed
\end{proof}

We say that an edge-set or a graph  {\em covers} a set-family  if 
 for all .
A set-family  is said to be {\em uncrossable} if for any  
the following holds:  or 
.
The problem of finding a minimum-cost set of undirected edges that covers an 
uncrossable set-family  admits a primal-dual -approximation algorithm, 
provided the inclusion-minimal
members of  can be computed in polynomial time \cite{GGPS}.
It is known that the undirected {\sf Steiner Forest} problem is a particular case of the 
problem of finding a min-cost cover of an uncrossable family, and thus admits a 
-approximation algorithm.
 
\begin{lemma} \label{l:F-uncross}
Let  be a mixed graph, where edges in  are undirected and edges in  are directed, such that for every  both 
 belong to the same connected component of the graph . Then the set-family 
 is 
uncrossable, and its inclusion minimal members can be computed in polynomial time.
\end{lemma} 
\begin{proof}
Let  be the set of connected components of the graph .
Let .
Any bridge  of  partitions  into two parts 
such that  is the unique edge in  connecting them. 
Note that the condition  is equivalent to the following condition (C1), 
while if condition (C1) holds then the condition  
is equivalent to the following condition (C2)
(since no edge in  connects two distinct connected components of ).
\begin{enumerate}[(C1)]
\item
There exists  and a bridge  of ,
such that  is a union of one of the sets 
and sets in .
\item
.
\end{enumerate}
Hence we have the following characterization of the sets in :
{\em  if, and only if, conditions (C1), (C2) hold for .} 
This implies that every inclusion-minimal member of  is 
 or , for some bridge  of .
In particular, the inclusion-minimal members of  
can be computed in polynomial time.



\begin{figure} 
\centering 
\epsfbox{comps.eps}
   \caption{Illustration to the proof of Lemma~\ref{l:F-uncross};
            components distinct from  are shown by gray ellipses.} 
\label{f:comps}
\end{figure}


Now let  (so conditions (C1), (C2) hold for each one of ), 
let  be the corresponding connected components and 
 the corresponding bridges 
(possibly , in which case we also may have ), and let 
 and  be the corresponding partitions of  and , respectively.
Since  are bridges, at least one of the sets 

must be empty, say .
Note that the set-family  is symmetric, hence to prove that 
 or ,
it is sufficient to prove 
that  
for some pair  such that .
E.g., if  and , then 
 and , hence 
 together with the 
symmetry of  implies .
Similarly, if  and , then 
 and 
, hence 
  
implies .
Thus w.l.o.g. we may assume that 
 and , see Figure~\ref{f:comps},
and we show that .
Recall that  and
hence  is a (possibly empty) union of some sets in 
.
Thus  is a union of  and some sets in 
.
This implies that conditions (C1), (C2) hold for , hence 
; the proof that
 is similar.
This concludes the proof of the lemma.
\qed
\end{proof}

\begin{lemma} \label{l:3}
Given a {\sf Steiner Forest Orientation} instance, the problem of compu\-ting a 
minimum-cost subgraph  of  that covers  admits a -approximation algorithm.
\end{lemma}
\begin{proof}
The algorithm has two phases.
In the first phase we solve the corresponding undirected {\sf Steiner Forest} instance
with the same demand function . The {\sf Steiner Forest} 
problem admits a -approximation algorithm, hence .
Let  be a subgraph of  computed by such a -approximation algorithm.
Note that  for all .
Hence to obtain a cover of  it is sufficient to cover 
the family  of the deficient sets w.r.t. .
The key point is that the family  is uncrossable, and that the inclusion-minimal 
members of  can be computed in polynomial time.
In the second phase we compute a -approximate cover of this  using the 
algorithm of \cite{GGPS}.
Observe that the set , that is 
the set of edges of the optimum solution with edges of  removed,
covers the family  and therefore the cost of the second phase 
is at most .
Consequently, the problem of covering  is reduced to solving two problems
of covering an uncrossable set-family.  

To show that  is uncrossable we use Lemma~\ref{l:F-uncross}.
Note that for any  both  belong to the same connected component of , 
and that  if, and only if,  and ,
hence .
Consequently, by Lemma~\ref{l:F-uncross}, the family  is uncrossable and its 
inclusion-minimal members can be computed in polynomial time.
This concludes the proof of the lemma.
\qed
\end{proof}

The proof of Theorem~\ref{t:min} is complete.

\section{Algorithm for {\sf -Orientation} on mixed graphs (Theorem~\ref{t:mix})} \label{s:mix}

In this section we prove Theorem \ref{t:mix}.
The following (essentially known) statement is straightforward. 

\begin{lemma} \label{lem:contract-cycle}
Let  be a mixed graph, let  be a set of directed edges on , 
and let  be a subgraph of  that admits a strongly connected orientation. 
Let  be obtained from  by contracting  into a single node.
Then  is -orientable if, and only if,  is -orientable. 
In particular, this is so if  is a cycle. \hfill 
\end{lemma}

\begin{corollary} \label{cor:tree}
{\sf -orientation} (with an undirected graph ) can be decided in polynomial time.
\end{corollary}

\begin{proof}
By repeatedly contracting a cycle of , we obtain an equivalent instance,
by Lemma~\ref{lem:contract-cycle}.
Hence we may assume that  is a tree.
Then for every  there is a unique -path in , 
which imposes an orientation on all the edges of this path.
Hence if suffices to check that no two pairs in  
impose different orientations of the same edge of the tree.
\qed
\end{proof}

Our algorithm for mixed graphs is based on a similar idea.
We say that a mixed graph is an {\em ori-cycle} if it admits an orientation that is a 
directed simple cycle. We need the following statement.

\begin{lemma} \label{lem:find-cycle}
Let  be a mixed graph, where edges of  are undirected and  contains directed arcs, and let  be obtained from  by contracting every
connected component of the undirected graph  into a single node.
If there is a directed cycle (possibly  a self-loop)  in  then there is an ori-cycle  in ,
and such  can be found in polynomial time.
\end{lemma}
\begin{proof}
If  is also a directed cycle in , then we take .
Otherwise, we replace every node  of  that corresponds to a contracted connected component  of 
by a path, as follows.
Let  be the arc entering  in  and let  be the arc leaving  in .
Let  be the head of  and similarly let  be a the tail of .
Since  is a connected component in , there is a -path in , and we replace  by 
this path. The result is the required ori-cycle  (possibly a self-loop) in .
It is easy to see that such  can be obtained from  in polynomial time. 
\qed
\end{proof}

By Lemmas~\ref{lem:find-cycle} and \ref{lem:contract-cycle}
we may assume that the directed graph  obtained from  by contracting every
connected component of , is a directed acyclic multigraph (with no self-loops).
This preprocessing step is similar to the one used by Silverbush et al.~\cite{silverbush}.
Let . Let  be the function
which for each node  of  assigns a node  in 
that represents the connected component of  that contains 
(in other words the function  shows a correspondence between
nodes before and after contractions).

The first step of our algorithm is to guess the first and the last edge
on the path for each of the  pairs in , by trying all  possibilities.
If for the -th pair an undirected edge is selected as the first or the last one
on the corresponding path, then we orient it accordingly and move it from  to .
Thus by functions  we denote
the guessed first and last arc for each of the  paths.

Now we present a branching algorithm with exponential time complexity
which we later convert to  time by applying a method of memoization.
Let  be a topological ordering of .
By  we denote the most recently chosen
arc from  for each of the  paths (initially ).
In what follows we consider subsequent nodes  of  with respect to 
and branch on possible orientations of the connected component  of .
We use this orientation to update the function  for all the arguments 
such that  is an arc entering a node mapped to .

Let  be the first node w.r.t. to 
which was not yet considered by the branching algorithm.
Let  be the set of indices  such that  for ,
and .
If  then we skip  and proceed to the next node in .
Otherwise for each  we branch on choosing an arc  such that ,
that is we select an arc that the -path will use just after leaving the connected component of 
corresponding to the node  (note that there are at most  branches).
Before updating the arcs  for each  in a branch, we check whether the connected component 
of  consisting of nodes  is accordingly orientable by using Corollary~\ref{cor:tree} (see Figure~\ref{fig:zoom}).
Finally after considering all the nodes in  we check whether for each  
we have .
If this is the case our algorithm returns YES and otherwise it returns NO in this branch.

\begin{figure} 
\centering 
\epsfysize=3.1cm
\epsfbox{fig1.eps}
   \caption{Our algorithm considers what orientation the connected component  (of )
will have. Currently we have ,  and , hence .
If in a branch we set new values  and  then by Corollary~\ref{cor:tree} we can
verify that it is possible to orient , so that there is a path from the end-point 
of  to the start-point of  and from the end-point of  to the start-point of .
However the branch with new values  and  will be terminated,
since it is not possible to orient  accordingly.}
\label{fig:zoom}
\end{figure}

The correctness of our algorithm follows from the invariant that each node  is considered
at most once, since all the updated values  are changed to arcs that are
to the right with respect to~.

Observe that when considering a node  it is not important what orientations previous nodes in  have,
because all the relevant information is contained in the  function. 
Therefore to improve the currently exponential time complexity we apply the standard technique of 
memoization, that is, store results of all the previously computed recursive calls.
Consequently for any index of the currently considered node in  
and any values of the function , there is at most one branch
for which we compute the result, since for the subsequent recursive calls
we use the previously computed results.
This leads to  branches and  total time and space complexity.

\section{Algorithms for {\sf Maximum Pairs Orientation} (Theorem~\ref{t:kernel})} \label{s:kernel}

In this section we prove Theorem \ref{t:kernel}.

\begin{lemma}
\label{lem:kernel}
There exists a linear time algorithm that given an instance 
of {\sf Maximum Pairs Orientation} or of {\sf k Pairs Orientation},
transforms it into an equivalent instance such that the input graph is a tree
with at most  nodes.
\end{lemma}

\begin{proof}
As is observed in \cite{HM}, and also follows from Lemma~\ref{lem:contract-cycle},
we can assume that the input graph  is a tree; such a tree can be constructed in linear time
by contracting the -edge-connected components of .
For each edge  of  we compute an integer ,
that is the number of pairs  in , such that 
belongs to the shortest path between  and  in .
If for an edge  we have , we contract this edge,
since we can always orient it as desired by at most one -path.
Note that after this operation each leaf belongs to at least two pairs in .
If a node  has degree  in the tree and does not belong to any pair,
we contract one of the edges incident to ; this is since in any inclusion minimal solution,
one of the two edges enters  if, and only if, the other leaves~. 

The linear time implementation of the presented reductions is as follows.
First, using a linear time algorithm for computing 2-edge-connected components,
and by scanning every edge in , we can see which components it connects,
thus obtaining an equivalent instance where  is a tree.
To compute all the values , we root the tree in an arbitrary node
and create a multiset of pairs ,
where  is the lowest common ancestor of  and ,
which can be computed in linear time~\cite{HT}.
Note that in  the first coordinate of each pair is an descendant of the second coordinate node
and pairs in  represent upward paths.
Let  and observe that 
if  is a parent of  in the tree , then  equals
the sum of values  over all descendants  of the node  (including ).
Therefore we can compute all the values  in linear time
and contract all the edges with .
Note that after the contractions are done we need to relabel pairs in ,
since some nodes may have their labels changed,
but this can be also done in linear time by storing a new label for each node
in a table.
Finally using a graph search algorithm we find maximal paths such
that each internal node is of degree two and does not belong to any pair.
For each such path we contract all but one edge.

We claim that after these reductions are implemented, the tree  obtained has at most  nodes. 
Let  be the number of leaves and  the number of nodes of degree  in . 
As each node of degree less than  in  is an  or , .
Since each leaf belongs to at least two pairs in , .
The number of nodes of degree at least  is at most 
 and so .
This concludes the proof of the lemma.
\qed
\end{proof}

After applying Lemma~\ref{lem:kernel}, the number of nodes 
of the returned tree is at most .
Therefore, by~\cite{ZV}, one can find in polynomial time
a solution , such that .
Therefore, if for a given {\sf  Pairs Orientation} instance
we have , then clearly it is a YES instance.
However if , then . 
In order to solve the {\sf  Pairs Orientation} instance
we consider all possible  subsets  of exactly 
pairs from , and check if the graph is -orientable.
Observe that

where the second inequality follows from Stirling's formula.
Therefore the running time is , which proves (i).

Combining Lemma~\ref{lem:kernel} with the -approximation algorithm of 
Gamzu et al. \cite{segev-approx} proves (ii).
Thus the proof of Theorem~\ref{t:kernel} is complete.

\section{Conclusions and open problems}

In this paper we considered minimum-cost and maximum pairs orientation problems.
Our main results are a -approximation algorithm for {\sf Steiner Forest Orientation},
an  time algorithm for {\sf -Orientation} on mixed graphs,
and an  time algorithm for {\sf  Pairs Orientation} 
(which implies that {\sf  Pairs Orientation} is fixed-parameter tractable when parameterized by ,
solving an open question from \cite{fpt}).
We now mention some open problems, most of them related to the work of Khanna, Naor, and Shepherd \cite{KNS}.

We have shown that the {\sf -Orientation} problem on mixed graphs parameterized by  belongs to XP,
however to the best of our knowledge it is not known whether this problem is fixed parameter tractable
or -hard.

As was mentioned, \cite{KNS} 
showed that the problem of computing a minimum-cost 
-edge-outconnected orientation can be solved in polynomial time, even for non-symmetric edge-costs.
To the best of our knowledge, for {\em node-connectivity}, and even for the simpler notion 
of {\em element-connectivity}, no non-trivial approximation ratio is known even for 
symmetric costs and .
Moreover, even the decision version of determining whether 
an undirected graph admits a -outconnected orientation is not known to be in P nor NP-complete.  

For the case when the orientation  is required to be 
-edge-connected, , \cite{KNS} obtained a pseudo-approximation algorithm 
that computes a -edge-connected subgraph of cost at most  times the cost 
of an optimal -connected subgraph.
It is an open question if the problem admits a non-trivial true approximation 
algorithm even for .

\section{Algorithm for {\sf  Disjoint Paths Orientation} (Theorem~\ref{t:nc})} \label{s:nc}

In this section we prove Theorem~\ref{t:nc}.
We need the following characterization due to \cite{EKM} 
of feasible solutions to {\sf  Disjoint Paths Orientation}.

\begin{theorem} [\cite{EKM}] \label{t:EKM}
Let  be an undirected graph and let . 
Then  has an orientation  such that 
 if, and only if,

Furthermore, if  satisfies (\ref{e:EKM}), then an orientation  of 
that satisfies \\ 
 can be computed in polynomial time. 
\end{theorem}

Now, let us use the following version of Menger's Theorem for node and edge capacitated graphs;
this version can be deduced from the original Menger's Theorem by elementary constructions.

\begin{lemma} \label{l:M}
Let  be two nodes in a directed/undirected graph  with edge and node capacities 
. 
Then the maximum number of -paths such that every  
appears in at most  of them equals to
. \hfill 
\end{lemma}

From Lemma~\ref{l:M} we deduce the following.

\begin{corollary} \label{c:EKM}
An undirected graph  satisfies (\ref{e:EKM}) if, and only if, the following condition holds: 
 contains  edge-disjoint -paths such that every  
belongs to at most 2 of them.
\end{corollary}
\begin{proof}
Assign capacity  to every  and capacity  to every .
By Lemma~\ref{l:M}, the condition in the corollary is equivalent to the condition

Since  for all  and  for all , 
the latter condition is equivalent to the condition

The above condition is equivalent to (\ref{e:EKM}), since 
for every  we have 
,
by applying Menger's Theorem on the graph .
The statement follows.
\qed
\end{proof}


Now consider the following problem.

\vspace{0.1cm}

\begin{center} 
\fbox{
\begin{minipage}{0.96\textwidth}
{\sf Node-Capacitated Min-Cost -Flow} \\ 
{\em Instance:} \ A graph  with edge-costs,
, node-capacities  
\hphantom{\em Instance:} \ 
, and an integer . \\
{\em Objective:} Find a set  of  edge-disjoint paths such that every  
\hphantom{\em Objective:} belongs to at most  paths in .
\end{minipage}
}
\end{center}

\vspace{0.1cm}

From Corollary~\ref{c:EKM}, we see that {\sf  Disjoint Paths Orientation} is a particular case of 
{\sf Node-Capacitated Min-Cost -Flow} when  is undirected, , and all node capacities are .

{\sf Node-Capacitated Min-Cost -Flow} can be solved in polynomial time, for both directed and undirected graphs,
by reducing the problem to the standard {\sf Edge-Capacitated Min-Cost -Flow} problem.
For directed graphs this can be done by a standard reduction of converting node-capacities
to edge-capacities: replace every node  by the two
nodes , connected by the edge  having the same capacity as , and redirect the
heads of the edges entering  to  and the tails of the edges leaving  to .
The undirected case is easily reduced to the directed one, by solving the problem on the 
bidirection graph of , obtained from  by replacing every undirected edge  connecting 
by a pair of antiparallel directed edges  of the same cost as .

The proof of Theorem~\ref{t:nc} is complete.

\bibliographystyle{abbrv}
\bibliography{u}

\end{document}
