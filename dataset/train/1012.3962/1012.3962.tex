\documentclass{llncs}

\usepackage{multicol}

\usepackage{amsmath}
\usepackage{amsfonts}
\usepackage{amssymb}

\usepackage{graphicx}
\usepackage{epic}
\usepackage{eepic}
\usepackage{epsfig,float}
\usepackage{color}
\usepackage{tabularx}

\pagestyle{plain}
\DeclareGraphicsRule{.tif}{png}{.png}{`convert #1 `dirname #1`/`basename #1 .tif`.png}

\renewcommand{\le}{\leqslant}
\renewcommand{\ge}{\geqslant}

\newcommand{\ol}{\overline}
\newcommand{\eps}{\varepsilon}
\newcommand{\emp}{\emptyset}
\newcommand{\rhoR}{R}
\newcommand{\Sig}{\Sigma}
\newcommand{\sig}{\sigma}
\newcommand{\noin}{\noindent}
\newcommand{\pf}{prefix-focused}
\newcommand{\ur}{uniquely reachable}
\newcommand{\bi}{\begin{itemize}}
\newcommand{\ei}{\end{itemize}}
\newcommand{\be}{\begin{enumerate}}
\newcommand{\ee}{\end{enumerate}}
\newcommand{\bd}{\begin{description}}
\newcommand{\ed}{\end{description}}
\newcommand{\bq}{\begin{quote}}
\newcommand{\eq}{\end{quote}}
\newcommand{\txt}[1]{\mbox{ #1 }}

\newcommand{\etc}{\mbox{\it etc.}}
\newcommand{\ie}{\mbox{\it i.e.}}
\newcommand{\eg}{\mbox{\it e.g.}}


\newcommand{\inv}[1]{\mbox{}}

\newcommand{\stress}[1]{{\fontfamily{cmtt}\selectfont #1}}

\def\shu{\mathbin{\mathchoice
{\rule{.3pt}{1ex}\rule{.3em}{.3pt}\rule{.3pt}{1ex}
\rule{.3em}{.3pt}\rule{.3pt}{1ex}}
{\rule{.3pt}{1ex}\rule{.3em}{.3pt}\rule{.3pt}{1ex}
\rule{.3em}{.3pt}\rule{.3pt}{1ex}}
{\rule{.2pt}{.7ex}\rule{.2em}{.2pt}\rule{.2pt}{.7ex}
\rule{.2em}{.2pt}\rule{.2pt}{.7ex}}
{\rule{.3pt}{1ex}\rule{.3em}{.3pt}\rule{.3pt}{1ex}
\rule{.3em}{.3pt}\rule{.3pt}{1ex}}\mkern2mu}}

\newcommand{\cA}{{\mathcal A}}
\newcommand{\cB}{{\mathcal B}}
\newcommand{\cC}{{\mathcal C}}
\newcommand{\cD}{{\mathcal D}}
\newcommand{\cL}{{\mathcal L}}
\newcommand{\cN}{{\mathcal N}}
\newcommand{\cP}{{\mathcal P}}
\newcommand{\cS}{{\mathcal S}}
\newcommand{\cR}{{\mathcal R}}
\newcommand{\cT}{{\mathcal T}}

\newcommand{\one}{{\mathbf 1}}

\newcommand{\Lra}{{\hspace{.1cm}\Leftrightarrow\hspace{.1cm}}}
\newcommand{\lra}{{\hspace{.1cm}\leftrightarrow\hspace{.1cm}}}
\newcommand{\la}{{\hspace{.1cm}\leftarrow\hspace{.1cm}}}
\newcommand{\raL}{{\hspace{.1cm}{\rightarrow_L} \hspace{.1cm}}}
\newcommand{\lraL}{{\hspace{.1cm}{\leftrightarrow_L} \hspace{.1cm}}}

\newcommand{\sn}{{semiautomaton}}
\newcommand{\sa}{{semiautomata}}
\newcommand{\Sn}{{Semiautomaton}}
\newcommand{\Sa}{{Semiautomata}}
\newcommand{\se}{{settable}}
\newcommand{\Se}{{Settable}}
\newcommand{\pc}{{prefix-continuous}}
\newcommand{\tr}{{transformation}}
\newcommand{\noni}{{non-increasing}}
\newcommand{\qedb}{\hfill} 

\newtheorem{open}[theorem]{Open problem}
\title{Quotient Complexity of Star-Free Languages
\thanks{This work was supported by the Natural Sciences and Engineering Research Council of Canada under grant No.~OGP0000871
}
}

\author{Janusz~Brzozowski and Bo Liu
 }

\authorrunning{Brzozowski, Liu}   

\institute{David R. Cheriton School of Computer Science, University of Waterloo, \\
Waterloo, ON, Canada N2L 3G1\\
\{ {\tt brzozo, b23liu} \}{\tt @uwaterloo.ca} 
}


\begin{document}

\maketitle
\today
\begin{abstract}
The quotient complexity, also known as state complexity, of a regular language is the number of distinct left quotients of the language.
The quotient complexity of an operation is the maximal quotient complexity of the language resulting from the operation, as a function of the quotient complexities of the operands.
The class of star-free languages is the smallest class  containing the  finite languages and closed under boolean operations and concatenation.
We prove that the tight bounds on the quotient complexities of union, intersection, difference,  symmetric difference, concatenation, and star  for star-free languages are the same as those for regular languages, with some small exceptions, whereas the bound for reversal is .
\bigskip

\noin
{\bf Keywords:}
aperiodic, automaton,  complexity,  language,  operation, quotient, regular, star-free, state complexity
\end{abstract}


\section{Introduction}

The class of regular languages can be defined as the smallest class containing the finite languages and closed under union, concatenation and star. Since regular languages are also closed under complementation, one can redefine them as the smallest class containing the finite languages and closed under boolean operations, concatenation and star. In this new formulation, a natural question is that of the \emph{generalized star height} of a regular language, which is the minimum number of nested stars required to define the language when boolean operations are allowed. 
It is not clear who first considered the problem of generalized star height, but McNaughton and Papert reported in their 1971 monograph~\cite{McPa71} that this problem had been open ``for many years''.
There exist regular languages of star height 0 and 1, but it is not even known whether there exists a language of star height 2. See 
{\small\tt http://liafa.jussieu.fr/\~{}jep/Problemes/starheight.html.}



We consider regular languages of star height 0, which are also called \emph{star-free}. 
In 1965, Sch\"utzenberger proved~\cite{Sch65} that a language is star-free if and only if its syntactic monoid is \emph{group-free}, that is, has only trivial subgroups. An equivalent condition is that the minimal deterministic automaton of a star-free language is \emph{permutation-free}, that is, has only trivial permutations. Another point of view is that these automata are \emph{counter-free}, since they cannot count modulo any integer greater than 1. They can, however, \emph{count to a threshold}, that is  \emph{or more}.
Such automata are called \emph{aperiodic,} and this is the term that we use.

The \emph{state complexity} of a regular language~\cite{Yu01} is the number of states in the minimal deterministic finite automaton accepting that language. We prefer  the equivalent concept of \emph{quotient complexity}~\cite{Brz10}, which is the number of distinct left quotients of the language, because quotient complexity has some advantages.
The \emph{quotient complexity of an operation} in a subclass of regular languages is the maximal quotient complexity of the language resulting from the operation, as a function of the quotient complexities of the operands when they range over all the languages in the subclass. 
The complexities of basic operations in the class of regular languages were studied by Maslov~\cite{Mas70} and Yu, Zhuang and Salomaa~\cite{YZS94}.

The complexities of operations were also considered in several subclasses of regular languages: 
unary~\cite{PiSh02,YZS94}, finite~\cite{CCSY01,Yu01}, ideal~\cite{BJL10}, closed~\cite{BJZ10}, prefix-free~\cite{HSW09},  suffix-free~\cite{HaSa09}, bifix-, factor-, and subword-free~\cite{BJS10},   and convex~\cite{Brz10a}. 
The complexity of operations can be significantly lower 
in a subclass of regular languages than in the general case. 
We prove that this is \emph{not} the case for star-free languages, which meet the bounds for regular languages, with  small exceptions.

It was shown in~\cite{BHK09} that the tight bound for converting an -state aperiodic nondeterministic automaton to a deterministic one is .

In Section~\ref{sec:terminology} we define our terminology and notation.
Boolean operations, concatenation, star,
and reversal  are studied in Sections~\ref{sec:boolean}--\ref{sec:reversal}, respectively.
Unary languages are treated in Section~\ref{sec:unary}, and
Section~\ref{sec:conclusions} concludes the paper.

\section{Terminology and Notation}
\label{sec:terminology}
If  is a finite non-empty alphabet, then  is the set of all  words over this alphabet,
with  as the empty word. 
For , , let  be the length of , and  , the number of 's in . A~language is any subset of .


We use the following set operations on languages:  {\em complement\/} (),  {\em union\/}  (),  {\em intersection\/} (),  {\em difference\/} (), and {\em symmetric difference\/} (). 
We also  use \emph{product}, also called  \emph{(con)catenation}  ()  and   \emph{star} ().
The reverse  of a word  is defined by: , and .
The \emph{reverse} of a language  is 
.

\emph{Regular} languages are the smallest class of languages containing the finite languages and closed under boolean operations, product  and  star. 
\emph{Star-free} languages are the languages one can construct from finite languages using only boolean operations and concatenation.
Some examples of star-free languages are , , 
 over ,  and
 over . We do not write such expressions for star-free  languages, but denote them as usual.

The \emph{(left) quotient} of a language  by a word  is defined as . The number of distinct quotients of a language  is called its \emph{quotient complexity}  and is denoted by . 
A~quotient  is \emph{accepting} if ; 
otherwise it is \emph{rejecting}.


A \emph{deterministic finite automaton (DFA)} is a quintuple , where  is a finite set of \emph{states,}  is a finite \emph{alphabet,}  is the \emph{transition function,}  is the \emph{initial state,} and  is the set of \emph{final} or \emph{accepting states\/}. 
As usual, the transition function is extended to . 
A DFA  accepts  if , and the \emph{language} accepted by  is . 
The \emph{language of a state}  of  is the language  accepted by the automaton . 
If the language of a state is empty, that state is \emph{empty}.

Let  if , and , otherwise.
The \emph{quotient automaton} of a regular language  is 
, where , , 
,  , and .
Since this  is the minimal DFA accepting , the quotient complexity of  is equal to the state complexity of , and we call it simply  \emph{complexity}.

A \emph{transformation} of a set  into itself is a mapping 

where  for . 
Each word in  performs a transformation of the set  of states of a DFA .
A DFA is \emph{aperiodic} if no word performs a permutation, other than the identity permutation, of a subset of .
Since testing if a DFA is aperiodic is PSPACE-complete~\cite{ChHu91}, we  use a subclass of aperiodic automata. 
Without loss of generality, we assume that .
A transformation is \emph{non-decreasing} if  implies .
A~non-decreasing transformation cannot have a non-trivial permutation, and the composition of non-decreasing transformations is non-decreasing. 
Hence a DFA with non-decreasing input transformations is aperiodic.

A \emph{nondeterministic finite automaton  (NFA)} is defined as a quintuple

 where 
 , , and  are as in a DFA, 
  is the \emph{transition function} and
  is the \emph{set of initial states}. 
If  also allows , 
that is, , 
we call  an \emph{-NFA}.




\section{Boolean Operations}
\label{sec:boolean}
We now consider the quotient complexity of  union, intersection, symmetric difference, and difference in the class of star-free languages. 
The upper bound for these four operations in the class of regular languages is ~\cite{Brz10,Mas70,YZS94}.



\begin{theorem}
\label{thm:boolean}
For each of the operations  union, intersection, symmetric difference, and difference, there exist binary star-free languages  and  with quotient complexities  and , respectively, that meet the bound .
\end{theorem}

\begin{proof}


Let . We examine union first. 
For , let  and let  be any binary star-free language with 
.
Then .
Similarly, if , let  and let  be any binary star-free language with 
. Then .

For , let , and 
; then  and , and both  and  are star-free. 
The quotient automata of  and  are  in Fig.~\ref{fig:unionKL} for  and ,
and  their direct product  for , in Fig.~\ref{fig:union}.


 \begin{figure}[t]
\begin{center}
\setlength{\unitlength}{0.00039370in}
\begingroup\makeatletter\ifx\SetFigFont\undefined \gdef\SetFigFont#1#2#3#4#5{\reset@font\fontsize{#1}{#2pt}\fontfamily{#3}\fontseries{#4}\fontshape{#5}\selectfont}\fi\endgroup {\renewcommand{\dashlinestretch}{30}
\begin{picture}(10794,1367)(0,-10)
\put(3286,257){\makebox(0,0)[lb]{\smash{{\SetFigFont{9}{10.8}{\rmdefault}{\mddefault}{\updefault}}}}}
\put(3369.500,772.929){\arc{394.717}{2.4948}{6.9299}}
\blacken\thicklines
\path(3529.033,794.097)(3527.000,654.000)(3600.998,772.977)(3529.033,794.097)
\thinlines
\put(2206.500,750.929){\arc{394.717}{2.4948}{6.9299}}
\blacken\thicklines
\path(2366.033,772.097)(2364.000,632.000)(2437.998,750.977)(2366.033,772.097)
\thinlines
\put(1044.500,757.929){\arc{394.717}{2.4948}{6.9299}}
\blacken\thicklines
\path(1204.033,779.097)(1202.000,639.000)(1275.998,757.977)(1204.033,779.097)
\thinlines
\put(7225.500,739.929){\arc{394.717}{2.4948}{6.9299}}
\blacken\thicklines
\path(7385.033,761.097)(7383.000,621.000)(7456.998,739.977)(7385.033,761.097)
\thinlines
\put(8305.500,731.929){\arc{394.717}{2.4948}{6.9299}}
\blacken\thicklines
\path(8465.033,753.097)(8463.000,613.000)(8536.998,731.977)(8465.033,753.097)
\thinlines
\put(9385.500,731.929){\arc{394.717}{2.4948}{6.9299}}
\blacken\thicklines
\path(9545.033,753.097)(9543.000,613.000)(9616.998,731.977)(9545.033,753.097)
\thinlines
\put(10472.500,739.929){\arc{394.717}{2.4948}{6.9299}}
\blacken\thicklines
\path(10632.033,761.097)(10630.000,621.000)(10703.998,739.977)(10632.033,761.097)
\thinlines
\put(6145.500,724.929){\arc{394.717}{2.4948}{6.9299}}
\blacken\thicklines
\path(6305.033,746.097)(6303.000,606.000)(6376.998,724.977)(6305.033,746.097)
\thinlines
\put(1044,356){\ellipse{630}{630}}
\put(2215,343){\ellipse{630}{630}}
\put(3372,365){\ellipse{540}{540}}
\put(4537,354){\ellipse{630}{630}}
\put(3369,363){\ellipse{630}{630}}
\put(8312,329){\ellipse{630}{630}}
\put(7219,339){\ellipse{630}{630}}
\put(10471,325){\ellipse{630}{630}}
\put(9394,330){\ellipse{540}{540}}
\put(9391,329){\ellipse{630}{630}}
\put(6141,323){\ellipse{630}{630}}
\path(2537,347)(3047,347)
\blacken\thicklines
\path(2912.000,309.500)(3047.000,347.000)(2912.000,384.500)(2912.000,309.500)
\thinlines
\path(1374,354)(1884,354)
\blacken\thicklines
\path(1749.000,316.500)(1884.000,354.000)(1749.000,391.500)(1749.000,316.500)
\thinlines
\path(3692,354)(4202,354)
\blacken\thicklines
\path(4067.000,316.500)(4202.000,354.000)(4067.000,391.500)(4067.000,316.500)
\thinlines
\path(444,358)(714,358)
\blacken\thicklines
\path(579.000,320.500)(714.000,358.000)(579.000,395.500)(579.000,320.500)
\thinlines
\path(7542,336)(7992,336)
\blacken\thicklines
\path(7857.000,298.500)(7992.000,336.000)(7857.000,373.500)(7857.000,298.500)
\thinlines
\path(8629,343)(9079,343)
\blacken\thicklines
\path(8944.000,305.500)(9079.000,343.000)(8944.000,380.500)(8944.000,305.500)
\thinlines
\path(9702,336)(10152,336)
\blacken\thicklines
\path(10017.000,298.500)(10152.000,336.000)(10017.000,373.500)(10017.000,298.500)
\thinlines
\path(6447,343)(6897,343)
\blacken\thicklines
\path(6762.000,305.500)(6897.000,343.000)(6762.000,380.500)(6762.000,305.500)
\thinlines
\path(5530,336)(5800,336)
\blacken\thicklines
\path(5665.000,298.500)(5800.000,336.000)(5665.000,373.500)(5665.000,298.500)
\put(4277,1097){\makebox(0,0)[lb]{\smash{{\SetFigFont{9}{10.8}{\familydefault}{\mddefault}{\updefault}}}}}
\put(985,1089){\makebox(0,0)[lb]{\smash{{\SetFigFont{9}{10.8}{\familydefault}{\mddefault}{\updefault}}}}}
\put(2132,1103){\makebox(0,0)[lb]{\smash{{\SetFigFont{9}{10.8}{\familydefault}{\mddefault}{\updefault}}}}}
\put(3264,1090){\makebox(0,0)[lb]{\smash{{\SetFigFont{9}{10.8}{\familydefault}{\mddefault}{\updefault}}}}}
\put(3797,475){\makebox(0,0)[lb]{\smash{{\SetFigFont{9}{10.8}{\familydefault}{\mddefault}{\updefault}}}}}
\put(2605,467){\makebox(0,0)[lb]{\smash{{\SetFigFont{9}{10.8}{\familydefault}{\mddefault}{\updefault}}}}}
\put(1486,482){\makebox(0,0)[lb]{\smash{{\SetFigFont{9}{10.8}{\familydefault}{\mddefault}{\updefault}}}}}
\put(2117,264){\makebox(0,0)[lb]{\smash{{\SetFigFont{9}{10.8}{\rmdefault}{\mddefault}{\updefault}}}}}
\put(4449,249){\makebox(0,0)[lb]{\smash{{\SetFigFont{9}{10.8}{\rmdefault}{\mddefault}{\updefault}}}}}
\put(976,272){\makebox(0,0)[lb]{\smash{{\SetFigFont{9}{10.8}{\rmdefault}{\mddefault}{\updefault}}}}}
\put(15,262){\makebox(0,0)[lb]{\smash{{\SetFigFont{9}{10.8}{\rmdefault}{\mddefault}{\updefault}}}}}
\put(6611,474){\makebox(0,0)[lb]{\smash{{\SetFigFont{9}{10.8}{\familydefault}{\mddefault}{\updefault}}}}}
\put(5208,241){\makebox(0,0)[lb]{\smash{{\SetFigFont{9}{10.8}{\rmdefault}{\mddefault}{\updefault}}}}}
\put(7130,256){\makebox(0,0)[lb]{\smash{{\SetFigFont{9}{10.8}{\rmdefault}{\mddefault}{\updefault}}}}}
\put(8231,249){\makebox(0,0)[lb]{\smash{{\SetFigFont{9}{10.8}{\rmdefault}{\mddefault}{\updefault}}}}}
\put(9311,249){\makebox(0,0)[lb]{\smash{{\SetFigFont{9}{10.8}{\rmdefault}{\mddefault}{\updefault}}}}}
\put(10392,234){\makebox(0,0)[lb]{\smash{{\SetFigFont{9}{10.8}{\rmdefault}{\mddefault}{\updefault}}}}}
\put(6072,249){\makebox(0,0)[lb]{\smash{{\SetFigFont{9}{10.8}{\rmdefault}{\mddefault}{\updefault}}}}}
\put(7698,474){\makebox(0,0)[lb]{\smash{{\SetFigFont{9}{10.8}{\familydefault}{\mddefault}{\updefault}}}}}
\put(8786,474){\makebox(0,0)[lb]{\smash{{\SetFigFont{9}{10.8}{\familydefault}{\mddefault}{\updefault}}}}}
\put(9851,459){\makebox(0,0)[lb]{\smash{{\SetFigFont{9}{10.8}{\familydefault}{\mddefault}{\updefault}}}}}
\put(6050,1059){\makebox(0,0)[lb]{\smash{{\SetFigFont{9}{10.8}{\familydefault}{\mddefault}{\updefault}}}}}
\put(7122,1052){\makebox(0,0)[lb]{\smash{{\SetFigFont{9}{10.8}{\familydefault}{\mddefault}{\updefault}}}}}
\put(8179,1059){\makebox(0,0)[lb]{\smash{{\SetFigFont{9}{10.8}{\familydefault}{\mddefault}{\updefault}}}}}
\put(9260,1052){\makebox(0,0)[lb]{\smash{{\SetFigFont{9}{10.8}{\familydefault}{\mddefault}{\updefault}}}}}
\put(10225,1066){\makebox(0,0)[lb]{\smash{{\SetFigFont{9}{10.8}{\familydefault}{\mddefault}{\updefault}}}}}
\thinlines
\put(4539.500,757.929){\arc{394.717}{2.4948}{6.9299}}
\blacken\thicklines
\path(4699.033,779.097)(4697.000,639.000)(4770.998,757.977)(4699.033,779.097)
\end{picture}
}
 \end{center}
\caption{Witnesses  and   for union with  and .} 
\label{fig:unionKL}
\end{figure}

 \begin{figure}[t]
\begin{center}
\setlength{\unitlength}{0.00039370in}
\begingroup\makeatletter\ifx\SetFigFont\undefined \gdef\SetFigFont#1#2#3#4#5{\reset@font\fontsize{#1}{#2pt}\fontfamily{#3}\fontseries{#4}\fontshape{#5}\selectfont}\fi\endgroup {\renewcommand{\dashlinestretch}{30}
\begin{picture}(5517,4798)(0,-10)
\put(3659,100){\makebox(0,0)[lb]{\smash{{\SetFigFont{9}{10.8}{\familydefault}{\mddefault}{\updefault}}}}}
\put(5198.250,3114.000){\arc{337.500}{4.0689}{8.4975}}
\blacken\thicklines
\path(5236.935,2986.036)(5097.000,2979.000)(5220.516,2912.855)(5236.935,2986.036)
\thinlines
\put(507.000,582.750){\arc{337.500}{5.6397}{10.0683}}
\blacken\thicklines
\path(379.376,544.083)(372.000,684.000)(306.156,560.324)(379.376,544.083)
\thinlines
\put(2667.000,537.750){\arc{337.500}{5.6397}{10.0683}}
\blacken\thicklines
\path(2539.376,499.083)(2532.000,639.000)(2466.156,515.324)(2539.376,499.083)
\thinlines
\put(5078.413,611.609){\arc{416.445}{4.8018}{9.5567}}
\blacken\thicklines
\path(5235.368,796.968)(5097.000,819.000)(5204.181,728.759)(5235.368,796.968)
\thinlines
\put(3777.000,515.750){\arc{337.500}{5.6397}{10.0683}}
\blacken\thicklines
\path(3649.376,477.083)(3642.000,617.000)(3576.156,493.324)(3649.376,477.083)
\thinlines
\put(5273.250,2041.000){\arc{337.500}{4.0689}{8.4975}}
\blacken\thicklines
\path(5311.935,1913.036)(5172.000,1906.000)(5295.516,1839.855)(5311.935,1913.036)
\thinlines
\put(1572.000,552.750){\arc{337.500}{5.6397}{10.0683}}
\blacken\thicklines
\path(1444.376,514.083)(1437.000,654.000)(1371.156,530.324)(1444.376,514.083)
\thinlines
\put(1602,3120){\ellipse{630}{630}}
\put(2667,3114){\ellipse{630}{630}}
\put(3747,3114){\ellipse{630}{630}}
\put(4821,3114){\ellipse{630}{630}}
\put(3749,3115){\ellipse{720}{720}}
\put(1587,2027){\ellipse{630}{630}}
\put(1586,2026){\ellipse{720}{720}}
\put(2669,2027){\ellipse{630}{630}}
\put(2659,954){\ellipse{630}{630}}
\put(2685,4210){\ellipse{630}{630}}
\put(469,2027){\ellipse{630}{630}}
\put(505,954){\ellipse{630}{630}}
\put(4833,4194){\ellipse{630}{630}}
\put(3747,4209){\ellipse{630}{630}}
\put(3750,4207){\ellipse{720}{720}}
\put(508,3112){\ellipse{630}{630}}
\put(2668,2023){\ellipse{720}{720}}
\put(1582,4190){\ellipse{630}{630}}
\put(1579,947){\ellipse{630}{630}}
\put(4838,2035){\ellipse{630}{630}}
\put(3766,954){\ellipse{630}{630}}
\put(3769,954){\ellipse{720}{720}}
\put(497,4178){\ellipse{630}{630}}
\put(470,2026){\ellipse{720}{720}}
\put(4835,2033){\ellipse{720}{720}}
\put(3739,2034){\ellipse{630}{630}}
\put(3742,2034){\ellipse{720}{720}}
\put(4827,962){\ellipse{630}{630}}
\path(822,4194)(1272,4194)
\blacken\thicklines
\path(1137.000,4156.500)(1272.000,4194.000)(1137.000,4231.500)(1137.000,4156.500)
\thinlines
\path(822,3114)(1272,3114)
\blacken\thicklines
\path(1137.000,3076.500)(1272.000,3114.000)(1137.000,3151.500)(1137.000,3076.500)
\thinlines
\path(822,954)(1272,954)
\blacken\thicklines
\path(1137.000,916.500)(1272.000,954.000)(1137.000,991.500)(1137.000,916.500)
\thinlines
\path(1902,954)(2352,954)
\blacken\thicklines
\path(2217.000,916.500)(2352.000,954.000)(2217.000,991.500)(2217.000,916.500)
\thinlines
\path(1902,3114)(2352,3114)
\blacken\thicklines
\path(2217.000,3076.500)(2352.000,3114.000)(2217.000,3151.500)(2217.000,3076.500)
\thinlines
\path(2982,4194)(3387,4194)
\blacken\thicklines
\path(3252.000,4156.500)(3387.000,4194.000)(3252.000,4231.500)(3252.000,4156.500)
\thinlines
\path(2982,3114)(3387,3114)
\blacken\thicklines
\path(3252.000,3076.500)(3387.000,3114.000)(3252.000,3151.500)(3252.000,3076.500)
\thinlines
\path(4107,4194)(4512,4194)
\blacken\thicklines
\path(4377.000,4156.500)(4512.000,4194.000)(4377.000,4231.500)(4377.000,4156.500)
\thinlines
\path(4107,3114)(4512,3114)
\blacken\thicklines
\path(4377.000,3076.500)(4512.000,3114.000)(4377.000,3151.500)(4377.000,3076.500)
\thinlines
\path(1587,2799)(1587,2394)
\blacken\thicklines
\path(1549.500,2529.000)(1587.000,2394.000)(1624.500,2529.000)(1549.500,2529.000)
\thinlines
\path(507,2799)(507,2394)
\blacken\thicklines
\path(469.500,2529.000)(507.000,2394.000)(544.500,2529.000)(469.500,2529.000)
\thinlines
\path(2667,2799)(2667,2394)
\blacken\thicklines
\path(2629.500,2529.000)(2667.000,2394.000)(2704.500,2529.000)(2629.500,2529.000)
\thinlines
\path(507,1674)(507,1269)
\blacken\thicklines
\path(469.500,1404.000)(507.000,1269.000)(544.500,1404.000)(469.500,1404.000)
\thinlines
\path(1587,1674)(1587,1269)
\blacken\thicklines
\path(1549.500,1404.000)(1587.000,1269.000)(1624.500,1404.000)(1549.500,1404.000)
\thinlines
\path(2667,1674)(2667,1269)
\blacken\thicklines
\path(2629.500,1404.000)(2667.000,1269.000)(2704.500,1404.000)(2629.500,1404.000)
\thinlines
\path(507,3879)(507,3429)
\blacken\thicklines
\path(469.500,3564.000)(507.000,3429.000)(544.500,3564.000)(469.500,3564.000)
\thinlines
\path(1587,3879)(1587,3429)
\blacken\thicklines
\path(1549.500,3564.000)(1587.000,3429.000)(1624.500,3564.000)(1549.500,3564.000)
\thinlines
\path(2667,3879)(2667,3429)
\blacken\thicklines
\path(2629.500,3564.000)(2667.000,3429.000)(2704.500,3564.000)(2629.500,3564.000)
\thinlines
\path(3027,2034)(3387,2034)
\blacken\thicklines
\path(3252.000,1996.500)(3387.000,2034.000)(3252.000,2071.500)(3252.000,1996.500)
\thinlines
\path(4827,3879)(4827,3429)
\blacken\thicklines
\path(4789.500,3549.000)(4827.000,3429.000)(4864.500,3549.000)(4789.500,3549.000)
\thinlines
\path(3747,2754)(3747,2394)
\blacken\thicklines
\path(3709.500,2529.000)(3747.000,2394.000)(3784.500,2529.000)(3709.500,2529.000)
\thinlines
\path(1924,4194)(2374,4194)
\blacken\thicklines
\path(2239.000,4156.500)(2374.000,4194.000)(2239.000,4231.500)(2239.000,4156.500)
\thinlines
\path(4130,954)(4535,954)
\blacken\thicklines
\path(4400.000,916.500)(4535.000,954.000)(4400.000,991.500)(4400.000,916.500)
\thinlines
\path(2997,954)(3402,954)
\blacken\thicklines
\path(3267.000,916.500)(3402.000,954.000)(3267.000,991.500)(3267.000,916.500)
\thinlines
\path(3754,1666)(3754,1306)
\blacken\thicklines
\path(3716.500,1441.000)(3754.000,1306.000)(3791.500,1441.000)(3716.500,1441.000)
\thinlines
\path(1940,2034)(2300,2034)
\blacken\thicklines
\path(2165.000,1996.500)(2300.000,2034.000)(2165.000,2071.500)(2165.000,1996.500)
\thinlines
\path(859,2034)(1219,2034)
\blacken\thicklines
\path(1084.000,1996.500)(1219.000,2034.000)(1084.000,2071.500)(1084.000,1996.500)
\thinlines
\path(3754,3841)(3754,3481)
\blacken\thicklines
\path(3716.500,3616.000)(3754.000,3481.000)(3791.500,3616.000)(3716.500,3616.000)
\thinlines
\path(4834,2806)(4834,2401)
\blacken\thicklines
\path(4796.500,2536.000)(4834.000,2401.000)(4871.500,2536.000)(4796.500,2536.000)
\thinlines
\path(4834,1666)(4834,1261)
\blacken\thicklines
\path(4796.500,1396.000)(4834.000,1261.000)(4871.500,1396.000)(4796.500,1396.000)
\thinlines
\path(4114,2034)(4474,2034)
\blacken\thicklines
\path(4339.000,1996.500)(4474.000,2034.000)(4339.000,2071.500)(4339.000,1996.500)
\thinlines
\path(12,4771)(304,4411)
\blacken\thicklines
\path(189.834,4492.224)(304.000,4411.000)(248.082,4539.469)(189.834,4492.224)
\put(635,3617){\makebox(0,0)[lb]{\smash{{\SetFigFont{9}{10.8}{\familydefault}{\mddefault}{\updefault}}}}}
\put(980,4308){\makebox(0,0)[lb]{\smash{{\SetFigFont{9}{10.8}{\familydefault}{\mddefault}{\updefault}}}}}
\put(2060,4316){\makebox(0,0)[lb]{\smash{{\SetFigFont{9}{10.8}{\familydefault}{\mddefault}{\updefault}}}}}
\put(3080,4323){\makebox(0,0)[lb]{\smash{{\SetFigFont{9}{10.8}{\familydefault}{\mddefault}{\updefault}}}}}
\put(4227,4323){\makebox(0,0)[lb]{\smash{{\SetFigFont{9}{10.8}{\familydefault}{\mddefault}{\updefault}}}}}
\put(5036,151){\makebox(0,0)[lb]{\smash{{\SetFigFont{9}{10.8}{\familydefault}{\mddefault}{\updefault}}}}}
\put(267,4088){\makebox(0,0)[lb]{\smash{{\SetFigFont{9}{10.8}{\familydefault}{\mddefault}{\updefault}}}}}
\put(5502,4118){\makebox(0,0)[lb]{\smash{{\SetFigFont{9}{10.8}{\familydefault}{\mddefault}{\updefault}}}}}
\put(1353,4096){\makebox(0,0)[lb]{\smash{{\SetFigFont{9}{10.8}{\familydefault}{\mddefault}{\updefault}}}}}
\put(2442,4113){\makebox(0,0)[lb]{\smash{{\SetFigFont{9}{10.8}{\familydefault}{\mddefault}{\updefault}}}}}
\put(3505,4112){\makebox(0,0)[lb]{\smash{{\SetFigFont{9}{10.8}{\familydefault}{\mddefault}{\updefault}}}}}
\put(4600,4097){\makebox(0,0)[lb]{\smash{{\SetFigFont{9}{10.8}{\familydefault}{\mddefault}{\updefault}}}}}
\put(274,3023){\makebox(0,0)[lb]{\smash{{\SetFigFont{9}{10.8}{\familydefault}{\mddefault}{\updefault}}}}}
\put(1362,3024){\makebox(0,0)[lb]{\smash{{\SetFigFont{9}{10.8}{\familydefault}{\mddefault}{\updefault}}}}}
\put(2435,3024){\makebox(0,0)[lb]{\smash{{\SetFigFont{9}{10.8}{\familydefault}{\mddefault}{\updefault}}}}}
\put(3507,3024){\makebox(0,0)[lb]{\smash{{\SetFigFont{9}{10.8}{\familydefault}{\mddefault}{\updefault}}}}}
\put(4586,3024){\makebox(0,0)[lb]{\smash{{\SetFigFont{9}{10.8}{\familydefault}{\mddefault}{\updefault}}}}}
\put(252,1938){\makebox(0,0)[lb]{\smash{{\SetFigFont{9}{10.8}{\familydefault}{\mddefault}{\updefault}}}}}
\put(1370,1944){\makebox(0,0)[lb]{\smash{{\SetFigFont{9}{10.8}{\familydefault}{\mddefault}{\updefault}}}}}
\put(2434,1944){\makebox(0,0)[lb]{\smash{{\SetFigFont{9}{10.8}{\familydefault}{\mddefault}{\updefault}}}}}
\put(3521,1944){\makebox(0,0)[lb]{\smash{{\SetFigFont{9}{10.8}{\familydefault}{\mddefault}{\updefault}}}}}
\put(4602,1944){\makebox(0,0)[lb]{\smash{{\SetFigFont{9}{10.8}{\familydefault}{\mddefault}{\updefault}}}}}
\put(274,865){\makebox(0,0)[lb]{\smash{{\SetFigFont{9}{10.8}{\familydefault}{\mddefault}{\updefault}}}}}
\put(1347,864){\makebox(0,0)[lb]{\smash{{\SetFigFont{9}{10.8}{\familydefault}{\mddefault}{\updefault}}}}}
\put(2412,857){\makebox(0,0)[lb]{\smash{{\SetFigFont{9}{10.8}{\familydefault}{\mddefault}{\updefault}}}}}
\put(3534,872){\makebox(0,0)[lb]{\smash{{\SetFigFont{9}{10.8}{\familydefault}{\mddefault}{\updefault}}}}}
\put(4596,871){\makebox(0,0)[lb]{\smash{{\SetFigFont{9}{10.8}{\familydefault}{\mddefault}{\updefault}}}}}
\put(1700,3624){\makebox(0,0)[lb]{\smash{{\SetFigFont{9}{10.8}{\familydefault}{\mddefault}{\updefault}}}}}
\put(2765,3617){\makebox(0,0)[lb]{\smash{{\SetFigFont{9}{10.8}{\familydefault}{\mddefault}{\updefault}}}}}
\put(3853,3617){\makebox(0,0)[lb]{\smash{{\SetFigFont{9}{10.8}{\familydefault}{\mddefault}{\updefault}}}}}
\put(4940,3617){\makebox(0,0)[lb]{\smash{{\SetFigFont{9}{10.8}{\familydefault}{\mddefault}{\updefault}}}}}
\put(605,2544){\makebox(0,0)[lb]{\smash{{\SetFigFont{9}{10.8}{\familydefault}{\mddefault}{\updefault}}}}}
\put(1701,2551){\makebox(0,0)[lb]{\smash{{\SetFigFont{9}{10.8}{\familydefault}{\mddefault}{\updefault}}}}}
\put(2773,2544){\makebox(0,0)[lb]{\smash{{\SetFigFont{9}{10.8}{\familydefault}{\mddefault}{\updefault}}}}}
\put(3846,2536){\makebox(0,0)[lb]{\smash{{\SetFigFont{9}{10.8}{\familydefault}{\mddefault}{\updefault}}}}}
\put(4948,2528){\makebox(0,0)[lb]{\smash{{\SetFigFont{9}{10.8}{\familydefault}{\mddefault}{\updefault}}}}}
\put(620,1450){\makebox(0,0)[lb]{\smash{{\SetFigFont{9}{10.8}{\familydefault}{\mddefault}{\updefault}}}}}
\put(1708,1442){\makebox(0,0)[lb]{\smash{{\SetFigFont{9}{10.8}{\familydefault}{\mddefault}{\updefault}}}}}
\put(2781,1450){\makebox(0,0)[lb]{\smash{{\SetFigFont{9}{10.8}{\familydefault}{\mddefault}{\updefault}}}}}
\put(3854,1450){\makebox(0,0)[lb]{\smash{{\SetFigFont{9}{10.8}{\familydefault}{\mddefault}{\updefault}}}}}
\put(4949,1427){\makebox(0,0)[lb]{\smash{{\SetFigFont{9}{10.8}{\familydefault}{\mddefault}{\updefault}}}}}
\put(973,3236){\makebox(0,0)[lb]{\smash{{\SetFigFont{9}{10.8}{\familydefault}{\mddefault}{\updefault}}}}}
\put(2053,3258){\makebox(0,0)[lb]{\smash{{\SetFigFont{9}{10.8}{\familydefault}{\mddefault}{\updefault}}}}}
\put(966,2148){\makebox(0,0)[lb]{\smash{{\SetFigFont{9}{10.8}{\familydefault}{\mddefault}{\updefault}}}}}
\put(2039,2156){\makebox(0,0)[lb]{\smash{{\SetFigFont{9}{10.8}{\familydefault}{\mddefault}{\updefault}}}}}
\put(3126,2156){\makebox(0,0)[lb]{\smash{{\SetFigFont{9}{10.8}{\familydefault}{\mddefault}{\updefault}}}}}
\put(4199,2148){\makebox(0,0)[lb]{\smash{{\SetFigFont{9}{10.8}{\familydefault}{\mddefault}{\updefault}}}}}
\put(981,1076){\makebox(0,0)[lb]{\smash{{\SetFigFont{9}{10.8}{\familydefault}{\mddefault}{\updefault}}}}}
\put(2054,1083){\makebox(0,0)[lb]{\smash{{\SetFigFont{9}{10.8}{\familydefault}{\mddefault}{\updefault}}}}}
\put(3134,1091){\makebox(0,0)[lb]{\smash{{\SetFigFont{9}{10.8}{\familydefault}{\mddefault}{\updefault}}}}}
\put(5502,3015){\makebox(0,0)[lb]{\smash{{\SetFigFont{9}{10.8}{\familydefault}{\mddefault}{\updefault}}}}}
\put(5502,1951){\makebox(0,0)[lb]{\smash{{\SetFigFont{9}{10.8}{\familydefault}{\mddefault}{\updefault}}}}}
\put(4228,3251){\makebox(0,0)[lb]{\smash{{\SetFigFont{9}{10.8}{\familydefault}{\mddefault}{\updefault}}}}}
\put(3118,3258){\makebox(0,0)[lb]{\smash{{\SetFigFont{9}{10.8}{\familydefault}{\mddefault}{\updefault}}}}}
\put(4213,1091){\makebox(0,0)[lb]{\smash{{\SetFigFont{9}{10.8}{\familydefault}{\mddefault}{\updefault}}}}}
\put(418,137){\makebox(0,0)[lb]{\smash{{\SetFigFont{9}{10.8}{\familydefault}{\mddefault}{\updefault}}}}}
\put(1499,129){\makebox(0,0)[lb]{\smash{{\SetFigFont{9}{10.8}{\familydefault}{\mddefault}{\updefault}}}}}
\put(2579,122){\makebox(0,0)[lb]{\smash{{\SetFigFont{9}{10.8}{\familydefault}{\mddefault}{\updefault}}}}}
\thinlines
\put(5243.250,4194.000){\arc{337.500}{4.0689}{8.4975}}
\blacken\thicklines
\path(5281.935,4066.036)(5142.000,4059.000)(5265.516,3992.855)(5281.935,4066.036)
\end{picture}
}
 \end{center}
\caption{Quotient automaton of .} 
\label{fig:union}
\end{figure}

Let , and consider the quotients of  by the  words , , and ;
these quotients  correspond to states  in the direct-product automaton for .
We begin with the rejecting quotients of . 
First, , and all the other quotients are non-empty.
Next, if  and  (rows 1 to , columns 1 to ), then 
the  pair  of non-empty words  belongs to  and to no other rejecting quotient.
If , then  (rows 1 to , column ) contains , but has no words from .
If  , then  (row , columns 1 to )  contains , but has no words from .
So all  rejecting  quotients are distinct.


Now turn to the accepting quotients.
For , quotient  (rows 1 to , column ) contains , and this word is not contained in any other quotient  with , and  has no words from . 
Thus all the quotients in column  are distinct.
For ,   (row , columns 1 to ) contains , and this word is not contained in any other quotient  with , 
and  has no words from . 
Thus all the quotients in row  are distinct.
Excluding , each quotient in  column   contains  but not , each quotient in row   contains  but not , and   contains both  and . Hence all accepting quotients are distinct, and
 our claim holds for union.
\goodbreak

For difference, we can use  and , where  and  meet the bound  for union,
because
.

For intersection, it was shown in~\cite{BJL10} that the languages 
 and
 meet the bound .
Since both languages are star-free, our claim holds for intersection.
These languages  also meet the bound  for symmetric difference~\cite{BJL10}.
\qed
\end{proof}



\section{Product}
\label{sec:product}


The tight bound for  product of  regular languages~\cite{Mas70,YZS94} is  .
We  show that this bound can be met by star-free languages, with some  exceptions. 

In subset constructions, we use the notation  to mean that subset  under input word  moves to subset .
\begin{theorem}
\label{thm:product}
There exist
quaternary star-free languages  and  with quotient complexities  and , respectively, such that  .\end{theorem}
\begin{proof}
Let the quotient automaton for  be , where
, , , , and 




Next, let the quotient automaton for  be , where
, , , , and 

The automaton  for  is shown in Fig.~\ref{fig:product}, where the transition labeled
 should be ignored for now. 
The automaton  for  is also shown in  Fig.~\ref{fig:product}. 
If the transition labeled  is taken into account and  is made a rejecting state, then we have an -NFA for .
Here the initial state is , the set of accepting states is , and the transitions are as shown.

\begin{figure}[t]
\begin{center}
\setlength{\unitlength}{0.00039370in}
\begingroup\makeatletter\ifx\SetFigFont\undefined \gdef\SetFigFont#1#2#3#4#5{\reset@font\fontsize{#1}{#2pt}\fontfamily{#3}\fontseries{#4}\fontshape{#5}\selectfont}\fi\endgroup {\renewcommand{\dashlinestretch}{30}
\begin{picture}(11419,3038)(0,-10)
\put(9861,1737){\makebox(0,0)[lb]{\smash{{\SetFigFont{9}{10.8}{\familydefault}{\mddefault}{\updefault}}}}}
\put(6719.500,1413.929){\arc{394.717}{2.4948}{6.9299}}
\blacken\thicklines
\path(6879.033,1435.097)(6877.000,1295.000)(6950.998,1413.977)(6879.033,1435.097)
\thinlines
\put(8863.500,1420.929){\arc{394.717}{2.4948}{6.9299}}
\blacken\thicklines
\path(9023.033,1442.097)(9021.000,1302.000)(9094.998,1420.977)(9023.033,1442.097)
\thinlines
\put(9981.500,1435.929){\arc{394.717}{2.4948}{6.9299}}
\blacken\thicklines
\path(10141.033,1457.097)(10139.000,1317.000)(10212.998,1435.977)(10141.033,1457.097)
\thinlines
\put(11106.500,1420.929){\arc{394.717}{2.4948}{6.9299}}
\blacken\thicklines
\path(11266.033,1442.097)(11264.000,1302.000)(11337.998,1420.977)(11266.033,1442.097)
\thinlines
\put(727.071,1302.500){\arc{394.717}{0.9240}{5.3591}}
\blacken\thicklines
\path(705.908,1462.355)(846.000,1460.000)(727.193,1534.271)(705.908,1462.355)
\thinlines
\put(4947.928,1309.500){\arc{394.716}{4.0656}{8.5007}}
\blacken\thicklines
\path(4969.092,1149.646)(4829.000,1152.000)(4947.807,1077.729)(4969.092,1149.646)
\thinlines
\put(3388.500,1765.929){\arc{394.717}{2.4948}{6.9299}}
\blacken\thicklines
\path(3548.033,1787.097)(3546.000,1647.000)(3619.998,1765.977)(3548.033,1787.097)
\thinlines
\put(2241.500,1735.929){\arc{394.717}{2.4948}{6.9299}}
\blacken\thicklines
\path(2401.033,1757.097)(2399.000,1617.000)(2472.998,1735.977)(2401.033,1757.097)
\thinlines
\put(1164,1308){\ellipse{630}{630}}
\put(4511,1327){\ellipse{630}{630}}
\put(7768,1020){\ellipse{630}{630}}
\put(8861,1010){\ellipse{630}{630}}
\put(3382,1339){\ellipse{630}{630}}
\put(2247,1303){\ellipse{630}{630}}
\put(9982,1017){\ellipse{630}{630}}
\put(9982,1017){\ellipse{540}{540}}
\put(11096,1009){\ellipse{630}{630}}
\put(6711,1002){\ellipse{630}{630}}
\put(4514,1327){\ellipse{540}{540}}
\path(1364,1534)(1994,1534)
\blacken\thicklines
\path(1859.000,1496.500)(1994.000,1534.000)(1859.000,1571.500)(1859.000,1496.500)
\thinlines
\path(2024,1085)(1394,1085)
\blacken\thicklines
\path(1529.000,1122.500)(1394.000,1085.000)(1529.000,1047.500)(1529.000,1122.500)
\thinlines
\path(3149,1107)(2519,1107)
\blacken\thicklines
\path(2654.000,1144.500)(2519.000,1107.000)(2654.000,1069.500)(2654.000,1144.500)
\thinlines
\path(4259,1115)(3629,1115)
\blacken\thicklines
\path(3764.000,1152.500)(3629.000,1115.000)(3764.000,1077.500)(3764.000,1152.500)
\thinlines
\path(2489,1527)(3119,1527)
\blacken\thicklines
\path(2984.000,1489.500)(3119.000,1527.000)(2984.000,1564.500)(2984.000,1489.500)
\thinlines
\path(3645,1542)(4275,1542)
\blacken\thicklines
\path(4140.000,1504.500)(4275.000,1542.000)(4140.000,1579.500)(4140.000,1504.500)
\thinlines
\path(5924,994)(6374,994)
\blacken\thicklines
\path(6239.000,956.500)(6374.000,994.000)(6239.000,1031.500)(6239.000,956.500)
\thinlines
\path(7552,800)(6967,800)
\blacken\thicklines
\path(7102.000,837.500)(6967.000,800.000)(7102.000,762.500)(7102.000,837.500)
\thinlines
\path(8616,800)(8031,800)
\blacken\thicklines
\path(8166.000,837.500)(8031.000,800.000)(8166.000,762.500)(8166.000,837.500)
\thinlines
\path(9711,800)(9126,800)
\blacken\thicklines
\path(9261.000,837.500)(9126.000,800.000)(9261.000,762.500)(9261.000,837.500)
\thinlines
\path(10836,807)(10251,807)
\blacken\thicklines
\path(10386.000,844.500)(10251.000,807.000)(10386.000,769.500)(10386.000,844.500)
\thinlines
\path(8030,1235)(8615,1235)
\blacken\thicklines
\path(8480.000,1197.500)(8615.000,1235.000)(8480.000,1272.500)(8480.000,1197.500)
\thinlines
\path(9126,1227)(9711,1227)
\blacken\thicklines
\path(9576.000,1189.500)(9711.000,1227.000)(9576.000,1264.500)(9576.000,1189.500)
\thinlines
\path(10251,1227)(10836,1227)
\blacken\thicklines
\path(10701.000,1189.500)(10836.000,1227.000)(10701.000,1264.500)(10701.000,1189.500)
\thinlines
\path(6936,1242)(7521,1242)
\blacken\thicklines
\path(7386.000,1204.500)(7521.000,1242.000)(7386.000,1279.500)(7386.000,1204.500)
\thinlines
\path(614,2029)(966,1593)
\blacken\thicklines
\path(852.019,1674.484)(966.000,1593.000)(910.375,1721.597)(852.019,1674.484)
\thinlines
\path(4679,1062)(4680,1061)(4682,1059)
	(4685,1055)(4690,1050)(4698,1042)
	(4708,1031)(4720,1018)(4736,1003)
	(4753,985)(4773,965)(4795,943)
	(4819,920)(4845,895)(4873,870)
	(4902,844)(4932,818)(4963,791)
	(4996,765)(5030,740)(5065,715)
	(5101,690)(5139,666)(5179,643)
	(5220,621)(5264,600)(5310,580)
	(5358,561)(5409,544)(5462,529)
	(5516,517)(5572,507)(5632,500)
	(5690,497)(5746,497)(5800,499)
	(5850,505)(5899,512)(5944,522)
	(5988,533)(6029,546)(6069,561)
	(6107,576)(6144,593)(6179,610)
	(6213,628)(6246,647)(6277,665)
	(6306,683)(6334,701)(6360,718)
	(6383,734)(6403,749)(6421,762)
	(6436,773)(6448,782)(6457,788)(6471,799)
\blacken\thicklines
\path(6388.015,686.107)(6471.000,799.000)(6341.679,745.081)(6388.015,686.107)
\thinlines
\path(2467,1602)(2468,1603)(2469,1606)
	(2472,1610)(2477,1617)(2484,1627)
	(2493,1641)(2504,1657)(2518,1677)
	(2534,1700)(2553,1725)(2574,1754)
	(2596,1784)(2621,1816)(2647,1850)
	(2675,1884)(2704,1919)(2734,1954)
	(2765,1989)(2798,2024)(2831,2058)
	(2866,2090)(2902,2122)(2939,2153)
	(2978,2181)(3019,2209)(3062,2234)
	(3106,2258)(3153,2279)(3202,2298)
	(3254,2313)(3307,2325)(3363,2333)
	(3419,2337)(3475,2336)(3531,2330)
	(3585,2321)(3636,2308)(3686,2291)
	(3733,2273)(3778,2251)(3822,2228)
	(3863,2202)(3903,2175)(3941,2146)
	(3977,2116)(4013,2085)(4047,2053)
	(4080,2020)(4113,1987)(4144,1953)
	(4173,1920)(4202,1886)(4229,1854)
	(4254,1823)(4277,1794)(4298,1766)
	(4317,1741)(4334,1719)(4348,1700)
	(4360,1684)(4369,1671)(4377,1662)(4387,1647)
\blacken\thicklines
\path(4280.914,1738.526)(4387.000,1647.000)(4343.317,1780.128)(4280.914,1738.526)
\thinlines
\path(1169,1654)(1170,1655)(1171,1656)
	(1173,1660)(1177,1665)(1183,1672)
	(1191,1682)(1200,1695)(1212,1710)
	(1227,1728)(1244,1749)(1263,1773)
	(1285,1800)(1309,1829)(1335,1860)
	(1364,1894)(1394,1929)(1427,1966)
	(1461,2004)(1497,2043)(1534,2082)
	(1572,2122)(1612,2162)(1653,2202)
	(1695,2241)(1739,2280)(1784,2319)
	(1829,2356)(1877,2393)(1925,2428)
	(1976,2462)(2027,2495)(2081,2527)
	(2137,2557)(2195,2585)(2255,2611)
	(2317,2636)(2382,2658)(2450,2677)
	(2519,2694)(2592,2708)(2666,2718)
	(2742,2725)(2819,2727)(2896,2725)
	(2973,2719)(3048,2710)(3121,2697)
	(3193,2680)(3262,2661)(3329,2640)
	(3394,2616)(3456,2590)(3517,2563)
	(3575,2533)(3632,2502)(3686,2470)
	(3740,2436)(3792,2401)(3842,2365)
	(3892,2328)(3940,2290)(3987,2251)
	(4033,2212)(4077,2173)(4121,2133)
	(4163,2094)(4204,2055)(4243,2016)
	(4281,1979)(4316,1942)(4350,1907)
	(4382,1874)(4411,1843)(4438,1814)
	(4462,1787)(4484,1764)(4502,1743)
	(4518,1724)(4532,1709)(4543,1697)
	(4551,1687)(4558,1680)(4567,1669)
\blacken\thicklines
\path(4452.489,1749.738)(4567.000,1669.000)(4510.536,1797.231)(4452.489,1749.738)
\put(2714,1617){\makebox(0,0)[lb]{\smash{{\SetFigFont{9}{10.8}{\familydefault}{\mddefault}{\updefault}}}}}
\put(1649,1616){\makebox(0,0)[lb]{\smash{{\SetFigFont{9}{10.8}{\familydefault}{\mddefault}{\updefault}}}}}
\put(3906,837){\makebox(0,0)[lb]{\smash{{\SetFigFont{9}{10.8}{\familydefault}{\mddefault}{\updefault}}}}}
\put(4394,1280){\makebox(0,0)[lb]{\smash{{\SetFigFont{9}{10.8}{\rmdefault}{\mddefault}{\updefault}}}}}
\put(1063,1250){\makebox(0,0)[lb]{\smash{{\SetFigFont{9}{10.8}{\rmdefault}{\mddefault}{\updefault}}}}}
\put(2137,1264){\makebox(0,0)[lb]{\smash{{\SetFigFont{9}{10.8}{\rmdefault}{\mddefault}{\updefault}}}}}
\put(3269,1278){\makebox(0,0)[lb]{\smash{{\SetFigFont{9}{10.8}{\rmdefault}{\mddefault}{\updefault}}}}}
\put(1701,829){\makebox(0,0)[lb]{\smash{{\SetFigFont{9}{10.8}{\familydefault}{\mddefault}{\updefault}}}}}
\put(2834,837){\makebox(0,0)[lb]{\smash{{\SetFigFont{9}{10.8}{\familydefault}{\mddefault}{\updefault}}}}}
\put(2768,2270){\makebox(0,0)[lb]{\smash{{\SetFigFont{9}{10.8}{\familydefault}{\mddefault}{\updefault}}}}}
\put(2280,2780){\makebox(0,0)[lb]{\smash{{\SetFigFont{9}{10.8}{\familydefault}{\mddefault}{\updefault}}}}}
\put(3660,1624){\makebox(0,0)[lb]{\smash{{\SetFigFont{9}{10.8}{\familydefault}{\mddefault}{\updefault}}}}}
\put(6628,930){\makebox(0,0)[lb]{\smash{{\SetFigFont{9}{10.8}{\rmdefault}{\mddefault}{\updefault}}}}}
\put(7686,937){\makebox(0,0)[lb]{\smash{{\SetFigFont{9}{10.8}{\rmdefault}{\mddefault}{\updefault}}}}}
\put(8795,930){\makebox(0,0)[lb]{\smash{{\SetFigFont{9}{10.8}{\rmdefault}{\mddefault}{\updefault}}}}}
\put(9898,938){\makebox(0,0)[lb]{\smash{{\SetFigFont{9}{10.8}{\rmdefault}{\mddefault}{\updefault}}}}}
\put(11015,938){\makebox(0,0)[lb]{\smash{{\SetFigFont{9}{10.8}{\rmdefault}{\mddefault}{\updefault}}}}}
\put(10730,1740){\makebox(0,0)[lb]{\smash{{\SetFigFont{9}{10.8}{\familydefault}{\mddefault}{\updefault}}}}}
\put(6305,1732){\makebox(0,0)[lb]{\smash{{\SetFigFont{9}{10.8}{\familydefault}{\mddefault}{\updefault}}}}}
\put(2586,273){\makebox(0,0)[lb]{\smash{{\SetFigFont{9}{10.8}{\rmdefault}{\mddefault}{\updefault}}}}}
\put(8736,109){\makebox(0,0)[lb]{\smash{{\SetFigFont{9}{10.8}{\rmdefault}{\mddefault}{\updefault}}}}}
\put(5445,673){\makebox(0,0)[lb]{\smash{{\SetFigFont{9}{10.8}{\familydefault}{\mddefault}{\updefault}}}}}
\put(10508,522){\makebox(0,0)[lb]{\smash{{\SetFigFont{9}{10.8}{\familydefault}{\mddefault}{\updefault}}}}}
\put(9376,507){\makebox(0,0)[lb]{\smash{{\SetFigFont{9}{10.8}{\familydefault}{\mddefault}{\updefault}}}}}
\put(8273,508){\makebox(0,0)[lb]{\smash{{\SetFigFont{9}{10.8}{\familydefault}{\mddefault}{\updefault}}}}}
\put(7193,507){\makebox(0,0)[lb]{\smash{{\SetFigFont{9}{10.8}{\familydefault}{\mddefault}{\updefault}}}}}
\put(10266,1333){\makebox(0,0)[lb]{\smash{{\SetFigFont{9}{10.8}{\familydefault}{\mddefault}{\updefault}}}}}
\put(9193,1333){\makebox(0,0)[lb]{\smash{{\SetFigFont{9}{10.8}{\familydefault}{\mddefault}{\updefault}}}}}
\put(8091,1333){\makebox(0,0)[lb]{\smash{{\SetFigFont{9}{10.8}{\familydefault}{\mddefault}{\updefault}}}}}
\put(7146,1323){\makebox(0,0)[lb]{\smash{{\SetFigFont{9}{10.8}{\familydefault}{\mddefault}{\updefault}}}}}
\put(15,1212){\makebox(0,0)[lb]{\smash{{\SetFigFont{9}{10.8}{\familydefault}{\mddefault}{\updefault}}}}}
\put(5226,1234){\makebox(0,0)[lb]{\smash{{\SetFigFont{9}{10.8}{\familydefault}{\mddefault}{\updefault}}}}}
\put(3319,2027){\makebox(0,0)[lb]{\smash{{\SetFigFont{9}{10.8}{\familydefault}{\mddefault}{\updefault}}}}}
\put(2173,2028){\makebox(0,0)[lb]{\smash{{\SetFigFont{9}{10.8}{\familydefault}{\mddefault}{\updefault}}}}}
\put(7678,1737){\makebox(0,0)[lb]{\smash{{\SetFigFont{9}{10.8}{\familydefault}{\mddefault}{\updefault}}}}}
\put(8759,1737){\makebox(0,0)[lb]{\smash{{\SetFigFont{9}{10.8}{\familydefault}{\mddefault}{\updefault}}}}}
\thinlines
\put(7768.500,1428.929){\arc{394.717}{2.4948}{6.9299}}
\blacken\thicklines
\path(7928.033,1450.097)(7926.000,1310.000)(7999.998,1428.977)(7928.033,1450.097)
\end{picture}
}
 \end{center}
\caption{-NFA  of .} 
\label{fig:product}
\end{figure}

For  , , , , and ,  denote  by .
Similarly, for  ,  and ,  denote  by .

We first show by induction on the size of  that all  subsets of the form , where , , and , are reachable. 

When , the set  is reached by , for .
Now suppose we want to reach , where , , 
, and . 
Let ; by the induction assumption,  is reachable. Then
. 
Thus  is also reachable.




Next, we prove that the  subsets of the form , where  is any subset of , are reachable.
If , then  is the initial subset.
Let  and  be be as above.
Then .

If , there are two cases. If , then start with , which has already been shown to be reachable. We then have

If , then start with .
Now
.

Finally, we show that the  subsets of the form 
, 
where , and  are reachable. We have
.

In summary,  different subsets are reachable.
We now prove that all these subsets are pairwise distinguishable.

For , state  of  accepts the word , and state  accepts the word ; moreover,  each of these words  is accepted by only that one state  of , and none of these words is accepted by state , if .
Hence, if  is in  or in , then 
  and   are distinguished by .


First, let , and consider  and , where  ,  and  and  differ by state .
Then  and  are distinguished by .
Next, let  and  take  and , where  .
First apply ; then we reach  and , where
.
Then  accepts , whereas  rejects this word.


Second, suppose  and   and  differ by state ; then  and  are distinguished by .

Third, consider , where  and , where  and  . 
Then   is accepted by  but not by . 

Since all reachable sets are pairwise distinguishable,   the bound is met.
\qed
\end{proof}

\begin{corollary}
\label{cor:prod}
There exists a ternary star-free language  with quotient complexity  , such that  .
\end{corollary}
\begin{proof}
If ,  the DFA  has one  state, which is both initial and accepting.
Now   is not needed in the proofs of reachability and
distinguishability.
\qed
\end{proof}

A \emph{right (left) ideal}~\cite{BJL10} is a language  satisfying  (). 
If  (), then  is the right (left) ideal generated by .
Corollary~\ref{cor:prod} shows that  the bound  on the quotient complexity of the left ideal generated by a regular language can also be met by a star-free language.

If  in Theorem~\ref{thm:product}, then either  and , or  is the right ideal generated by . In the second case, it is known~\cite{YZS94} that  is a tight bound for , and that the language  is a witness~\cite{BJL10}. Since that witness is star-free, the general bound holds also for star-free languages.

The case  and  remains. 
For , the best bound for product of  regular languages is 6, whereas it is 4 for star-free languages. This was verified with the \emph{GAP} package \emph{Automata}~\cite{GAP} by enumerating all products of 2-state aperiodic automata.

There are only three types of inputs possible for a 2-state aperiodic DFA:
the input that takes both states to state 1, the input that takes both states to state 2, and the identity input.
If 1 is the accepting state, then subsets  and  are not distinguishable.
Therefore a rejecting quotient  of  can appear with only three subsets of quotients of  in the DFA of  instead of ,  and an accepting quotient, only with one subset instead of two.
The complexity is maximized when there is only one accepting quotient of . Hence . 
If 2 is the accepting state, then   and  are not distinguishable.
Hence  in this case.

\begin{theorem}
\label{thm:m2}
There exist ternary star-free languages  and  with quotient complexities  and , respectively, such that .
\end{theorem}
\begin{proof}
Let  be the DFA in the proof of Theorem~\ref{thm:product} restricted to input alphabet .
Let , where

For , subset   is reached by ,  
, by , 
and , by . 
Finally,  is reached by .
This gives  subsets.

For ,    accepts no words from ,  accepts , and  accepts  but not . 
Hence subsets  and  with ,
, and , are distinguishable. 
Next,  and  with  are distinguished by .
Also,  and  are distinguished from 
by , and  from  by . Therefore all  subsets are distinguishable.
\qed
\end{proof}
We do not know whether the bound  can be reached. However, we have verified with \emph{GAP} that it cannot be reached if .

\section{Star}
\label{sec:star}

The following DFA plays a key part in finding bounds on the quotient complexities of stars of star-free languages.
Let , and , where 
 and 

Since all the inputs perform non-decreasing transformations,  is 
aperiodic. 

In Fig.~\ref{fig:star}, if we ignore state 0 and its outgoing transitions, and also the  transition, then the figure
 shows the automaton . With state 0 and the  transition it
depicts the -NFA of .


\begin{figure}[h]
\begin{center}
\setlength{\unitlength}{0.00039370in}
\begingroup\makeatletter\ifx\SetFigFont\undefined \gdef\SetFigFont#1#2#3#4#5{\reset@font\fontsize{#1}{#2pt}\fontfamily{#3}\fontseries{#4}\fontshape{#5}\selectfont}\fi\endgroup {\renewcommand{\dashlinestretch}{30}
\begin{picture}(8975,2206)(0,-10)
\put(4332,1871){\makebox(0,0)[lb]{\smash{{\SetFigFont{9}{10.8}{\familydefault}{\mddefault}{\updefault}}}}}
\put(1762.780,1308.139){\arc{331.829}{2.2779}{7.3005}}
\blacken\thicklines
\path(1893.711,1300.119)(1850.000,1167.000)(1956.106,1258.503)(1893.711,1300.119)
\thinlines
\put(2848,872){\ellipse{630}{630}}
\put(7218,875){\ellipse{630}{630}}
\put(7220,874){\ellipse{702}{702}}
\put(3918,874){\ellipse{630}{630}}
\put(6078,879){\ellipse{630}{630}}
\put(8295,818){\ellipse{630}{630}}
\put(642,874){\ellipse{702}{702}}
\put(641,872){\ellipse{630}{630}}
\put(5010,884){\ellipse{630}{630}}
\put(1754,874){\ellipse{630}{630}}
\path(1985,1099)(2615,1099)
\blacken\thicklines
\path(2480.000,1061.500)(2615.000,1099.000)(2480.000,1136.500)(2480.000,1061.500)
\thinlines
\path(2615,649)(1985,649)
\blacken\thicklines
\path(2120.000,686.500)(1985.000,649.000)(2120.000,611.500)(2120.000,686.500)
\thinlines
\path(3065,1099)(3695,1099)
\blacken\thicklines
\path(3560.000,1061.500)(3695.000,1099.000)(3560.000,1136.500)(3560.000,1061.500)
\thinlines
\path(3695,649)(3065,649)
\blacken\thicklines
\path(3200.000,686.500)(3065.000,649.000)(3200.000,611.500)(3200.000,686.500)
\thinlines
\path(4145,1099)(4775,1099)
\blacken\thicklines
\path(4640.000,1061.500)(4775.000,1099.000)(4640.000,1136.500)(4640.000,1061.500)
\thinlines
\path(5225,1099)(5855,1099)
\blacken\thicklines
\path(5720.000,1061.500)(5855.000,1099.000)(5720.000,1136.500)(5720.000,1061.500)
\thinlines
\path(6305,1099)(6935,1099)
\blacken\thicklines
\path(6800.000,1061.500)(6935.000,1099.000)(6800.000,1136.500)(6800.000,1061.500)
\thinlines
\path(7475,1099)(8105,1099)
\blacken\thicklines
\path(7970.000,1061.500)(8105.000,1099.000)(7970.000,1136.500)(7970.000,1061.500)
\thinlines
\path(4775,649)(4145,649)
\blacken\thicklines
\path(4280.000,686.500)(4145.000,649.000)(4280.000,611.500)(4280.000,686.500)
\thinlines
\path(5855,649)(5225,649)
\blacken\thicklines
\path(5360.000,686.500)(5225.000,649.000)(5360.000,611.500)(5360.000,686.500)
\thinlines
\path(6935,649)(6305,649)
\blacken\thicklines
\path(6440.000,686.500)(6305.000,649.000)(6440.000,611.500)(6440.000,686.500)
\thinlines
\path(12,882)(282,882)
\blacken\path(162.000,852.000)(282.000,882.000)(162.000,912.000)(162.000,852.000)
\path(987,874)(1437,874)
\blacken\thicklines
\path(1302.000,836.500)(1437.000,874.000)(1302.000,911.500)(1302.000,836.500)
\thinlines
\path(8015,657)(7497,657)
\blacken\thicklines
\path(7632.000,694.500)(7497.000,657.000)(7632.000,619.500)(7632.000,694.500)
\thinlines
\path(1985,619)(1992,619)
\path(1992,619)(1985,619)
\path(755,536)(756,535)(758,533)
	(761,529)(767,523)(774,514)
	(784,503)(797,489)(813,472)
	(830,453)(851,432)(874,409)
	(899,384)(925,359)(954,332)
	(984,305)(1016,278)(1049,252)
	(1083,225)(1119,200)(1156,175)
	(1195,151)(1236,128)(1280,107)
	(1325,87)(1373,69)(1424,53)
	(1478,39)(1535,27)(1595,19)
	(1658,14)(1722,12)(1783,14)
	(1843,20)(1902,28)(1959,39)
	(2014,53)(2067,68)(2117,85)
	(2166,104)(2213,124)(2258,145)
	(2301,167)(2343,191)(2384,215)
	(2424,240)(2463,265)(2500,291)
	(2537,317)(2571,342)(2605,368)
	(2637,393)(2666,416)(2694,439)
	(2719,460)(2742,479)(2762,496)
	(2779,510)(2793,523)(2804,532)
	(2812,540)(2825,551)
\blacken\path(2752.772,450.585)(2825.000,551.000)(2714.015,496.389)(2752.772,450.585)
\path(7205,1242)(7204,1242)(7203,1243)
	(7200,1245)(7196,1248)(7190,1252)
	(7182,1258)(7171,1265)(7158,1274)
	(7142,1284)(7123,1297)(7102,1311)
	(7077,1328)(7049,1346)(7019,1365)
	(6985,1387)(6949,1410)(6910,1434)
	(6869,1460)(6825,1486)(6779,1514)
	(6731,1542)(6680,1572)(6629,1601)
	(6575,1631)(6520,1661)(6464,1691)
	(6406,1721)(6347,1751)(6286,1781)
	(6225,1810)(6162,1838)(6097,1866)
	(6031,1894)(5964,1920)(5895,1946)
	(5825,1971)(5753,1995)(5679,2018)
	(5602,2040)(5524,2061)(5443,2081)
	(5360,2099)(5274,2115)(5186,2131)
	(5095,2144)(5002,2156)(4906,2165)
	(4808,2172)(4708,2177)(4607,2179)
	(4505,2179)(4403,2176)(4303,2170)
	(4204,2161)(4107,2150)(4012,2137)
	(3920,2122)(3831,2105)(3744,2087)
	(3660,2067)(3579,2046)(3501,2024)
	(3425,2000)(3351,1975)(3279,1949)
	(3210,1923)(3142,1895)(3076,1867)
	(3012,1838)(2949,1808)(2888,1778)
	(2828,1747)(2770,1715)(2712,1684)
	(2657,1652)(2602,1620)(2549,1588)
	(2497,1556)(2447,1524)(2398,1492)
	(2352,1461)(2307,1431)(2264,1402)
	(2223,1374)(2184,1347)(2148,1321)
	(2114,1297)(2083,1274)(2054,1253)
	(2029,1234)(2006,1217)(1985,1202)
	(1968,1189)(1953,1177)(1941,1168)
	(1931,1161)(1924,1155)(1918,1150)(1910,1144)
\blacken\thicklines
\path(1995.500,1255.000)(1910.000,1144.000)(2040.500,1195.000)(1995.500,1255.000)
\put(2772,807){\makebox(0,0)[lb]{\smash{{\SetFigFont{9}{10.8}{\rmdefault}{\mddefault}{\updefault}}}}}
\put(3844,799){\makebox(0,0)[lb]{\smash{{\SetFigFont{9}{10.8}{\rmdefault}{\mddefault}{\updefault}}}}}
\put(4925,815){\makebox(0,0)[lb]{\smash{{\SetFigFont{9}{10.8}{\rmdefault}{\mddefault}{\updefault}}}}}
\put(5996,800){\makebox(0,0)[lb]{\smash{{\SetFigFont{9}{10.8}{\rmdefault}{\mddefault}{\updefault}}}}}
\put(7136,793){\makebox(0,0)[lb]{\smash{{\SetFigFont{9}{10.8}{\rmdefault}{\mddefault}{\updefault}}}}}
\put(2255,1181){\makebox(0,0)[lb]{\smash{{\SetFigFont{9}{10.8}{\familydefault}{\mddefault}{\updefault}}}}}
\put(3223,363){\makebox(0,0)[lb]{\smash{{\SetFigFont{9}{10.8}{\familydefault}{\mddefault}{\updefault}}}}}
\put(2067,373){\makebox(0,0)[lb]{\smash{{\SetFigFont{9}{10.8}{\familydefault}{\mddefault}{\updefault}}}}}
\put(987,1038){\makebox(0,0)[lb]{\smash{{\SetFigFont{9}{10.8}{\familydefault}{\mddefault}{\updefault}}}}}
\put(3313,1204){\makebox(0,0)[lb]{\smash{{\SetFigFont{9}{10.8}{\familydefault}{\mddefault}{\updefault}}}}}
\put(4385,1211){\makebox(0,0)[lb]{\smash{{\SetFigFont{9}{10.8}{\familydefault}{\mddefault}{\updefault}}}}}
\put(5450,1219){\makebox(0,0)[lb]{\smash{{\SetFigFont{9}{10.8}{\familydefault}{\mddefault}{\updefault}}}}}
\put(6545,1219){\makebox(0,0)[lb]{\smash{{\SetFigFont{9}{10.8}{\familydefault}{\mddefault}{\updefault}}}}}
\put(7565,1204){\makebox(0,0)[lb]{\smash{{\SetFigFont{9}{10.8}{\familydefault}{\mddefault}{\updefault}}}}}
\put(4295,363){\makebox(0,0)[lb]{\smash{{\SetFigFont{9}{10.8}{\familydefault}{\mddefault}{\updefault}}}}}
\put(5374,355){\makebox(0,0)[lb]{\smash{{\SetFigFont{9}{10.8}{\familydefault}{\mddefault}{\updefault}}}}}
\put(6461,350){\makebox(0,0)[lb]{\smash{{\SetFigFont{9}{10.8}{\familydefault}{\mddefault}{\updefault}}}}}
\put(7623,350){\makebox(0,0)[lb]{\smash{{\SetFigFont{9}{10.8}{\familydefault}{\mddefault}{\updefault}}}}}
\put(8960,777){\makebox(0,0)[lb]{\smash{{\SetFigFont{9}{10.8}{\familydefault}{\mddefault}{\updefault}}}}}
\put(1451,124){\makebox(0,0)[lb]{\smash{{\SetFigFont{9}{10.8}{\familydefault}{\mddefault}{\updefault}}}}}
\put(559,785){\makebox(0,0)[lb]{\smash{{\SetFigFont{9}{10.8}{\rmdefault}{\mddefault}{\updefault}}}}}
\put(8201,770){\makebox(0,0)[lb]{\smash{{\SetFigFont{9}{10.8}{\rmdefault}{\mddefault}{\updefault}}}}}
\put(1684,792){\makebox(0,0)[lb]{\smash{{\SetFigFont{9}{10.8}{\rmdefault}{\mddefault}{\updefault}}}}}
\put(1531,1594){\makebox(0,0)[lb]{\smash{{\SetFigFont{9}{10.8}{\familydefault}{\mddefault}{\updefault}}}}}
\thinlines
\put(8740.333,822.000){\arc{333.333}{3.7851}{8.7813}}
\blacken\thicklines
\path(8743.529,690.520)(8607.000,722.000)(8707.733,624.614)(8743.529,690.520)
\end{picture}
}
 \end{center}
\caption{-NFA  of ,  . Transitions under  (not shown) are all to state 7.} 
\label{fig:star}
\end{figure}


We first study  , the restriction of  to the alphabet .
\begin{lemma}
\label{lem:star}
If , and  is the  star-free language accepted by , then 
. 
\end{lemma}
\begin{proof}
Consider the subsets of  in the subset construction of the DFA for . Since 0 can only appear in , the remaining reachable subsets are subsets of . 
The empty subset cannot be reached because there is a transition from each state under every letter. Since state  cannot occur without state 1, we eliminate  subsets.
Because state  always appears with state , and state  can only be reached from state  by , the subset  first appears with state 2, and afterwards, always with a state from ; hence  cannot be reached.
Also,    and  cannot appear together without , because   cannot be reached by , and  1 cannot be reached by~ without including . This eliminates another  subsets.
So  subsets are unreachable, and .

Now turn to the reachable subsets, and note that subsets  and  are reached by  and , respectively.

First, let . 
 All singleton sets  are reached by  from . 
 Now let  ,
, where ,\;  , and 
; then

Thus any  can be reached from a smaller , and so all subsets in  are reachable.

Second, let ; then 
, as above,
and all  subsets in  are reachable.

Third, let . 
If , then  is reachable from  by . 
 Now suppose  is not empty.
 If , then . 
 So  .
Now, if , then 

If , then 

In either case, 

and all  subsets in  are reachable.

Fourth, let . We have shown that 
, if .
Since also 
, we have
.
Hence all  subsets  in  are reachable.

Altogether,  subsets are reachable.
It remains to be shown that all the reachable subsets are pairwise distinguishable. 
State 0 does not accept , while  accepts it. 
Each state 
 with  accepts  and each of these words is  accepted by only that one state, and  accepts . So any two subsets  and  are distinguishable.
\qed
\end{proof}

\begin{theorem}
\label{thm:stars}
For  there exists a quaternary star-free language  with  such that .
For ,  the tight upper bound is~2. 
\end{theorem}
\begin{proof}
For , there are only two languages,  and , and both are star-free. We have , and . For , there are two star-free unary languages,  and , and the bound cannot be met if .  If , then  meets the bound .
For , we analyzed all 3-state aperiodic automata using \emph{GAP}.
The bound 6 is met by  defined above, and  bounds 5 and 4 are met by   and , respectively.
These bounds cannot be improved.

We now turn to  the general case. 
We will show that the following sets of states are reachable in  the nondeterministic automaton  (see Fig.~\ref{fig:star}) from the initial state 0: the set , all subsets of  containing , and all  
 non-empty subsets of .
By Lemma~\ref{lem:star}, we can reach 
all these subsets by words in , except 
 and the subsets of  containing .

We have
; hence  is reachable.
Now consider , where .
Let ; then using  we move to 
 ,  and by  we reach
. Since  is reachable by Lemma~\ref{lem:star},  is also reachable. Thus we can reach all the subsets of  containing  by words in .
The only set missing now is , and it is reached by .

In Lemma~\ref{lem:star}, we have already shown that any two subsets  such that  are distinguishable by words in .
\qed
\end{proof}



Table~\ref{tab:StarSummary} summarizes our results for the quotient complexity of  in case  is star-free. For unary languages, see Section~\ref{sec:unary}. The figures in boldface type are known to be tight upper bounds. 
For , we analyzed all 4-state automata with non-decreasing input transformations.
Automata ,  , and  meet the bounds 12,
11, and 9, respectively.
The bounds 11 and 9 cannot be improved in the class of automata with non-decreasing input transformations.
For the rest, the bounds for  and  are met by 
, and , respectively.

\setlength{\extrarowheight}{2pt}
\begin{table}[ht]
\caption{Quotient  complexities for stars of star-free languages.}
\label{tab:StarSummary}
\begin{center}

\end{center}
\end{table}



\section{Reversal}
\label{sec:reversal}
For regular binary languages, the tight bound for reversal~\cite{Lei81}  is . For star-free languages the bound   can be met, but with  letters.

\begin{theorem}
\label{thm:reversal}
For each  there exists a star-free language  with quotient complexity  such that .
For , the bound is met if ,  for , if , and for , if .
\end{theorem}
\begin{proof}
For  and ,  is a witness. 
For  and ,  is a witness. 
We have verified using \emph{GAP} that all star-free languages  with  
satisfy ; hence this bound cannot be increased.

Now let , and let 

 where 
, ,  , and 

Since all the inputs perform non-decreasing transformations,  is 
aperiodic. 
Figure~\ref{fig:reversalodd} shows the NFA  which is the reverse of  DFA .


\begin{figure}[tbh]
\begin{center}
\setlength{\unitlength}{0.00039370in}
\begingroup\makeatletter\ifx\SetFigFont\undefined \gdef\SetFigFont#1#2#3#4#5{\reset@font\fontsize{#1}{#2pt}\fontfamily{#3}\fontseries{#4}\fontshape{#5}\selectfont}\fi\endgroup {\renewcommand{\dashlinestretch}{30}
\begin{picture}(11618,2051)(0,-10)
\put(10337,350){\makebox(0,0)[lb]{\smash{{\SetFigFont{9}{10.8}{\familydefault}{\mddefault}{\updefault}}}}}
\put(11296.500,1218.929){\arc{394.717}{2.4948}{6.9299}}
\blacken\thicklines
\path(11456.033,1240.097)(11454.000,1100.000)(11527.998,1218.977)(11456.033,1240.097)
\thinlines
\put(7869.500,1233.929){\arc{394.717}{2.4948}{6.9299}}
\blacken\thicklines
\path(8029.033,1255.097)(8027.000,1115.000)(8100.998,1233.977)(8029.033,1255.097)
\thinlines
\put(6181.500,1226.929){\arc{394.717}{2.4948}{6.9299}}
\blacken\thicklines
\path(6341.033,1248.097)(6339.000,1108.000)(6412.998,1226.977)(6341.033,1248.097)
\thinlines
\put(4449.500,1233.929){\arc{394.717}{2.4948}{6.9299}}
\blacken\thicklines
\path(4609.033,1255.097)(4607.000,1115.000)(4680.998,1233.977)(4609.033,1255.097)
\thinlines
\put(2731.500,1218.929){\arc{394.717}{2.4948}{6.9299}}
\blacken\thicklines
\path(2891.033,1240.097)(2889.000,1100.000)(2962.998,1218.977)(2891.033,1240.097)
\thinlines
\put(1029.500,1233.929){\arc{394.717}{2.4948}{6.9299}}
\blacken\thicklines
\path(1189.033,1255.097)(1187.000,1115.000)(1260.998,1233.977)(1189.033,1255.097)
\thinlines
\put(1030,791){\ellipse{630}{630}}
\put(2730,791){\ellipse{630}{630}}
\put(4455,800){\ellipse{630}{630}}
\put(6166,797){\ellipse{630}{630}}
\put(7867,807){\ellipse{630}{630}}
\put(1031,792){\ellipse{540}{540}}
\put(9584,801){\ellipse{630}{630}}
\put(11295,805){\ellipse{630}{630}}
\path(2739,12)(2739,462)
\blacken\thicklines
\path(2776.500,327.000)(2739.000,462.000)(2701.500,327.000)(2776.500,327.000)
\thinlines
\path(6167,12)(6167,462)
\blacken\thicklines
\path(6204.500,327.000)(6167.000,462.000)(6129.500,327.000)(6204.500,327.000)
\thinlines
\path(1314,665)(2431,665)
\blacken\thicklines
\path(2296.000,627.500)(2431.000,665.000)(2296.000,702.500)(2296.000,627.500)
\thinlines
\path(4188,935)(3071,935)
\blacken\thicklines
\path(3206.000,972.500)(3071.000,935.000)(3206.000,897.500)(3206.000,972.500)
\thinlines
\path(3032,665)(4149,665)
\blacken\thicklines
\path(4014.000,627.500)(4149.000,665.000)(4014.000,702.500)(4014.000,627.500)
\thinlines
\path(4749,665)(5866,665)
\blacken\thicklines
\path(5731.000,627.500)(5866.000,665.000)(5731.000,702.500)(5731.000,627.500)
\thinlines
\path(5889,935)(4772,935)
\blacken\thicklines
\path(4907.000,972.500)(4772.000,935.000)(4907.000,897.500)(4907.000,972.500)
\thinlines
\path(7585,935)(6468,935)
\blacken\thicklines
\path(6603.000,972.500)(6468.000,935.000)(6603.000,897.500)(6603.000,972.500)
\thinlines
\path(6451,657)(7568,657)
\blacken\thicklines
\path(7433.000,619.500)(7568.000,657.000)(7433.000,694.500)(7433.000,619.500)
\thinlines
\path(8154,672)(9271,672)
\blacken\thicklines
\path(9136.000,634.500)(9271.000,672.000)(9136.000,709.500)(9136.000,634.500)
\thinlines
\path(9288,935)(8171,935)
\blacken\thicklines
\path(8306.000,972.500)(8171.000,935.000)(8306.000,897.500)(8306.000,972.500)
\thinlines
\path(11020,935)(9903,935)
\blacken\thicklines
\path(10038.000,972.500)(9903.000,935.000)(10038.000,897.500)(10038.000,972.500)
\thinlines
\path(9886,672)(11003,672)
\blacken\thicklines
\path(10868.000,634.500)(11003.000,672.000)(10868.000,709.500)(10868.000,634.500)
\thinlines
\path(9624,19)(9624,469)
\blacken\thicklines
\path(9661.500,334.000)(9624.000,469.000)(9586.500,334.000)(9661.500,334.000)
\thinlines
\path(2440,935)(1323,935)
\blacken\thicklines
\path(1458.000,972.500)(1323.000,935.000)(1458.000,897.500)(1458.000,972.500)
\put(953,683){\makebox(0,0)[lb]{\smash{{\SetFigFont{9}{10.8}{\rmdefault}{\mddefault}{\updefault}}}}}
\put(2648,689){\makebox(0,0)[lb]{\smash{{\SetFigFont{9}{10.8}{\rmdefault}{\mddefault}{\updefault}}}}}
\put(4372,690){\makebox(0,0)[lb]{\smash{{\SetFigFont{9}{10.8}{\rmdefault}{\mddefault}{\updefault}}}}}
\put(6091,691){\makebox(0,0)[lb]{\smash{{\SetFigFont{9}{10.8}{\rmdefault}{\mddefault}{\updefault}}}}}
\put(7794,683){\makebox(0,0)[lb]{\smash{{\SetFigFont{9}{10.8}{\rmdefault}{\mddefault}{\updefault}}}}}
\put(9504,705){\makebox(0,0)[lb]{\smash{{\SetFigFont{9}{10.8}{\rmdefault}{\mddefault}{\updefault}}}}}
\put(10381,1620){\makebox(0,0)[lb]{\smash{{\SetFigFont{9}{10.8}{\familydefault}{\mddefault}{\updefault}}}}}
\put(11236,690){\makebox(0,0)[lb]{\smash{{\SetFigFont{9}{10.8}{\rmdefault}{\mddefault}{\updefault}}}}}
\put(1966,1769){\makebox(0,0)[lb]{\smash{{\SetFigFont{9}{10.8}{\familydefault}{\mddefault}{\updefault}}}}}
\put(3847,1566){\makebox(0,0)[lb]{\smash{{\SetFigFont{9}{10.8}{\familydefault}{\mddefault}{\updefault}}}}}
\put(5588,1761){\makebox(0,0)[lb]{\smash{{\SetFigFont{9}{10.8}{\familydefault}{\mddefault}{\updefault}}}}}
\put(7247,1551){\makebox(0,0)[lb]{\smash{{\SetFigFont{9}{10.8}{\familydefault}{\mddefault}{\updefault}}}}}
\put(8926,1754){\makebox(0,0)[lb]{\smash{{\SetFigFont{9}{10.8}{\familydefault}{\mddefault}{\updefault}}}}}
\put(15,1567){\makebox(0,0)[lb]{\smash{{\SetFigFont{9}{10.8}{\familydefault}{\mddefault}{\updefault}}}}}
\put(1786,350){\makebox(0,0)[lb]{\smash{{\SetFigFont{9}{10.8}{\familydefault}{\mddefault}{\updefault}}}}}
\put(3310,350){\makebox(0,0)[lb]{\smash{{\SetFigFont{9}{10.8}{\familydefault}{\mddefault}{\updefault}}}}}
\put(1817,1064){\makebox(0,0)[lb]{\smash{{\SetFigFont{9}{10.8}{\familydefault}{\mddefault}{\updefault}}}}}
\put(3512,1063){\makebox(0,0)[lb]{\smash{{\SetFigFont{9}{10.8}{\familydefault}{\mddefault}{\updefault}}}}}
\put(5221,1049){\makebox(0,0)[lb]{\smash{{\SetFigFont{9}{10.8}{\familydefault}{\mddefault}{\updefault}}}}}
\put(6909,1055){\makebox(0,0)[lb]{\smash{{\SetFigFont{9}{10.8}{\familydefault}{\mddefault}{\updefault}}}}}
\put(8633,1047){\makebox(0,0)[lb]{\smash{{\SetFigFont{9}{10.8}{\familydefault}{\mddefault}{\updefault}}}}}
\put(10336,1055){\makebox(0,0)[lb]{\smash{{\SetFigFont{9}{10.8}{\familydefault}{\mddefault}{\updefault}}}}}
\put(5012,343){\makebox(0,0)[lb]{\smash{{\SetFigFont{9}{10.8}{\familydefault}{\mddefault}{\updefault}}}}}
\put(6708,350){\makebox(0,0)[lb]{\smash{{\SetFigFont{9}{10.8}{\familydefault}{\mddefault}{\updefault}}}}}
\put(8403,350){\makebox(0,0)[lb]{\smash{{\SetFigFont{9}{10.8}{\familydefault}{\mddefault}{\updefault}}}}}
\thinlines
\put(9586.500,1226.929){\arc{394.717}{2.4948}{6.9299}}
\blacken\thicklines
\path(9746.033,1248.097)(9744.000,1108.000)(9817.998,1226.977)(9746.033,1248.097)
\end{picture}
}
 \end{center}
\caption{NFA  of ,  odd.} 
\label{fig:reversalodd}
\end{figure}

Assume initially that  is odd.
Let   be a subset of , and let .
Then  NFA  has the following properties:
\goodbreak


\noin
{\bf P1}
If ,  and , then input  deletes state  from  without changing any of the other states. 
\smallskip

\noin
{\bf P2}
If ,  ,  and , then input  adds state  to 
without changing any of the other states.


We now examine the sets of reachable states in .
The set  of all the odd states cannot be reached. For suppose that it is reached from some set . If it is reached by , then  must be a subset of  . 
However, the successor under  of such a set   also contains  if it contains .
If we use , then  must be a subset of .
But then the successor of   also contains 2 if it contains~1.
If we use  with  odd, then  must be a subset of , and  must also have . But then the successor of  also contains , which is even, if it contains .
If we use  with  even, then we also get .


If , there are no  inputs. Set   is initial,  can be reached by  and  by .
We can get  by  ,   and   by  and , respectively, and   by . 
Set  is unreachable. So assume .

First, consider subsets  of , the set of \emph{middle states}; these are subsets of  containing neither 1 nor . If  start with
. 
By using inputs , delete  or not, add  or not, etc., until we reach 2, which cannot be removed by any .
If , then    has 2, is a subset of , and so is reachable; then  is reached  by   from .


Second, consider subsets  of  containing 1 but not .
If , start with  and apply  to reach . 
Each state in , except  2, is without a predecessor in .
Hence, by using  inputs , we can construct any such  .
If ,
start with  and apply  to reach , where  of all the odd states.
By using inputs , we can construct any such set .



Third,  examine subsets  of  containing  but not 1.
If , start with  and apply  to reach , and then apply  to get .
Construct any such set  using inputs .
If , then   is a subset of  containing . 
Since the set    is a subset of , it is reachable; then  is reached  by   from .


Finally, consider subsets  containing both 1 and .
Apply  to  to reach . From this set we can reach any set containing .

Now assume that .
We now show that  is reachable for every even  in .
Apply  to  to reach .
 If , we are done; otherwise, delete  and  by  and  in that order. Then insert  and  by  and  in that order.
 If , we are done; otherwise,  continue in this fashion.
If we reach , then , and the process stops. 

If , then we can reach .
From  we can get , , and .
We are missing only  , which is unreachable.





If , from  we can reach by  inputs all the subsets containing    but not , except those subsets containing  without .
From now on,  we are interested only in the missing subsets, which are with , without , and have  without .
Then take . 
From here we can reach all subsets containing    without , except those containing  without . If , then , and we are missing only 
, which is unreachable.

Continuing in this fashion, we can reach all the subsets containing 
 but not 2, except . Together with the case where , we have all the states containing ,  except .



\begin{figure}[tbh]
\begin{center}
\setlength{\unitlength}{0.00037620in}
\begingroup\makeatletter\ifx\SetFigFont\undefined \gdef\SetFigFont#1#2#3#4#5{\reset@font\fontsize{#1}{#2pt}\fontfamily{#3}\fontseries{#4}\fontshape{#5}\selectfont}\fi\endgroup {\renewcommand{\dashlinestretch}{30}
\begin{picture}(9614,1856)(0,-10)
\put(8581,1573){\makebox(0,0)[lb]{\smash{{\SetFigFont{9}{10.8}{\familydefault}{\mddefault}{\updefault}}}}}
\put(7576.500,1233.929){\arc{394.717}{2.4948}{6.9299}}
\blacken\thicklines
\path(7736.033,1255.097)(7734.000,1115.000)(7807.998,1233.977)(7736.033,1255.097)
\thinlines
\put(5888.500,1226.929){\arc{394.717}{2.4948}{6.9299}}
\blacken\thicklines
\path(6048.033,1248.097)(6046.000,1108.000)(6119.998,1226.977)(6048.033,1248.097)
\thinlines
\put(4156.500,1233.929){\arc{394.717}{2.4948}{6.9299}}
\blacken\thicklines
\path(4316.033,1255.097)(4314.000,1115.000)(4387.998,1233.977)(4316.033,1255.097)
\thinlines
\put(2438.500,1218.929){\arc{394.717}{2.4948}{6.9299}}
\blacken\thicklines
\path(2598.033,1240.097)(2596.000,1100.000)(2669.998,1218.977)(2598.033,1240.097)
\thinlines
\put(736.500,1233.929){\arc{394.717}{2.4948}{6.9299}}
\blacken\thicklines
\path(896.033,1255.097)(894.000,1115.000)(967.998,1233.977)(896.033,1255.097)
\thinlines
\put(737,791){\ellipse{630}{630}}
\put(2437,791){\ellipse{630}{630}}
\put(4162,800){\ellipse{630}{630}}
\put(5873,797){\ellipse{630}{630}}
\put(7574,807){\ellipse{630}{630}}
\put(738,792){\ellipse{540}{540}}
\put(9291,801){\ellipse{630}{630}}
\path(2446,12)(2446,462)
\blacken\thicklines
\path(2483.500,327.000)(2446.000,462.000)(2408.500,327.000)(2483.500,327.000)
\thinlines
\path(5874,12)(5874,462)
\blacken\thicklines
\path(5911.500,327.000)(5874.000,462.000)(5836.500,327.000)(5911.500,327.000)
\thinlines
\path(1021,665)(2138,665)
\blacken\thicklines
\path(2003.000,627.500)(2138.000,665.000)(2003.000,702.500)(2003.000,627.500)
\thinlines
\path(3895,935)(2778,935)
\blacken\thicklines
\path(2913.000,972.500)(2778.000,935.000)(2913.000,897.500)(2913.000,972.500)
\thinlines
\path(2739,665)(3856,665)
\blacken\thicklines
\path(3721.000,627.500)(3856.000,665.000)(3721.000,702.500)(3721.000,627.500)
\thinlines
\path(4456,665)(5573,665)
\blacken\thicklines
\path(5438.000,627.500)(5573.000,665.000)(5438.000,702.500)(5438.000,627.500)
\thinlines
\path(5596,935)(4479,935)
\blacken\thicklines
\path(4614.000,972.500)(4479.000,935.000)(4614.000,897.500)(4614.000,972.500)
\thinlines
\path(7292,935)(6175,935)
\blacken\thicklines
\path(6310.000,972.500)(6175.000,935.000)(6310.000,897.500)(6310.000,972.500)
\thinlines
\path(6158,657)(7275,657)
\blacken\thicklines
\path(7140.000,619.500)(7275.000,657.000)(7140.000,694.500)(7140.000,619.500)
\thinlines
\path(7861,672)(8978,672)
\blacken\thicklines
\path(8843.000,634.500)(8978.000,672.000)(8843.000,709.500)(8843.000,634.500)
\thinlines
\path(8995,935)(7878,935)
\blacken\thicklines
\path(8013.000,972.500)(7878.000,935.000)(8013.000,897.500)(8013.000,972.500)
\thinlines
\path(9331,19)(9331,469)
\blacken\thicklines
\path(9368.500,334.000)(9331.000,469.000)(9293.500,334.000)(9368.500,334.000)
\thinlines
\path(2147,935)(1030,935)
\blacken\thicklines
\path(1165.000,972.500)(1030.000,935.000)(1165.000,897.500)(1165.000,972.500)
\put(660,683){\makebox(0,0)[lb]{\smash{{\SetFigFont{9}{10.8}{\rmdefault}{\mddefault}{\updefault}}}}}
\put(2355,689){\makebox(0,0)[lb]{\smash{{\SetFigFont{9}{10.8}{\rmdefault}{\mddefault}{\updefault}}}}}
\put(4079,690){\makebox(0,0)[lb]{\smash{{\SetFigFont{9}{10.8}{\rmdefault}{\mddefault}{\updefault}}}}}
\put(5798,691){\makebox(0,0)[lb]{\smash{{\SetFigFont{9}{10.8}{\rmdefault}{\mddefault}{\updefault}}}}}
\put(7501,683){\makebox(0,0)[lb]{\smash{{\SetFigFont{9}{10.8}{\rmdefault}{\mddefault}{\updefault}}}}}
\put(9211,705){\makebox(0,0)[lb]{\smash{{\SetFigFont{9}{10.8}{\rmdefault}{\mddefault}{\updefault}}}}}
\put(1493,350){\makebox(0,0)[lb]{\smash{{\SetFigFont{9}{10.8}{\familydefault}{\mddefault}{\updefault}}}}}
\put(3017,350){\makebox(0,0)[lb]{\smash{{\SetFigFont{9}{10.8}{\familydefault}{\mddefault}{\updefault}}}}}
\put(1524,1064){\makebox(0,0)[lb]{\smash{{\SetFigFont{9}{10.8}{\familydefault}{\mddefault}{\updefault}}}}}
\put(3219,1063){\makebox(0,0)[lb]{\smash{{\SetFigFont{9}{10.8}{\familydefault}{\mddefault}{\updefault}}}}}
\put(4928,1049){\makebox(0,0)[lb]{\smash{{\SetFigFont{9}{10.8}{\familydefault}{\mddefault}{\updefault}}}}}
\put(6616,1055){\makebox(0,0)[lb]{\smash{{\SetFigFont{9}{10.8}{\familydefault}{\mddefault}{\updefault}}}}}
\put(8340,1047){\makebox(0,0)[lb]{\smash{{\SetFigFont{9}{10.8}{\familydefault}{\mddefault}{\updefault}}}}}
\put(4719,343){\makebox(0,0)[lb]{\smash{{\SetFigFont{9}{10.8}{\familydefault}{\mddefault}{\updefault}}}}}
\put(6415,350){\makebox(0,0)[lb]{\smash{{\SetFigFont{9}{10.8}{\familydefault}{\mddefault}{\updefault}}}}}
\put(8373,335){\makebox(0,0)[lb]{\smash{{\SetFigFont{9}{10.8}{\familydefault}{\mddefault}{\updefault}}}}}
\put(1875,1574){\makebox(0,0)[lb]{\smash{{\SetFigFont{9}{10.8}{\familydefault}{\mddefault}{\updefault}}}}}
\put(5497,1573){\makebox(0,0)[lb]{\smash{{\SetFigFont{9}{10.8}{\familydefault}{\mddefault}{\updefault}}}}}
\put(7209,1567){\makebox(0,0)[lb]{\smash{{\SetFigFont{9}{10.8}{\familydefault}{\mddefault}{\updefault}}}}}
\put(3795,1574){\makebox(0,0)[lb]{\smash{{\SetFigFont{9}{10.8}{\familydefault}{\mddefault}{\updefault}}}}}
\put(15,1574){\makebox(0,0)[lb]{\smash{{\SetFigFont{9}{10.8}{\familydefault}{\mddefault}{\updefault}}}}}
\thinlines
\put(9293.500,1226.929){\arc{394.717}{2.4948}{6.9299}}
\blacken\thicklines
\path(9453.033,1248.097)(9451.000,1108.000)(9524.998,1226.977)(9453.033,1248.097)
\end{picture}
}
 \end{center}
\caption{NFA  of ,  even.} 
\label{fig:reversaleven}
\end{figure}

 
The case where  is even is similar. The NFA  is shown in Fig.~\ref{fig:reversaleven} for .
By an argument similar to that for  odd,   cannot be reached. 

Any subset of  can be reached as follows.
If , apply  to  to get , and then  to get to .
Now any subset of  containing  can be reached by inputs  .
If , then any subset of  can be reached from  by  inputs .

Second, consider subsets  of  containing 1 but not .
If , start with  and apply  to reach . 
Then apply  to get .
Now any subset of  containing  can be reached by inputs  .
If , start with  and apply  to reach .
By using inputs , we can construct any subset  of  containing 1 and not 2,  except the subsets that have  without .
In case , we can reach , , and , but not .
From now on,  we are interested only in the missing subsets.
As in the even case, we can get subsets containing  without  by deleting
 and ,  adding , and re-inserting . Now we are unable to reach
states having  without . 
We verify that  is reachable for every even  with , and continue as  in the odd case.
We can keep moving  this problem to the left, until we reach . Then state 4 cannot be removed because  is not reachable.


Third,  examine subsets  of  containing  but not 1.
If , all such subsets are reachable by inputs  from .
If , then   is a subset of  containing . 
Since     is a subset of , it is reachable; then  is reached  by   from .

Finally, consider subsets  containing both 1 and .
If , apply  to reach . From here we can reach any set containing  by inputs .
If , we reach  from  by .
From here we can reach any set containing  but not 2 by inputs .

We still need to verify that all the reachable subsets are pairwise distinguishable. 
State , and only state , accepts .  Hence, if  and  and  differ by state , then they are distinguishable by .
\qed
\end{proof}
\section{Unary Languages}
\label{sec:unary}
The case of unary languages is special. For regular unary languages, the tight bounds for each boolean operation , product , star , and reversal  are , , , and , respectively~\cite{YZS94}. With the exception of the bound for reversal, these bounds cannot be met by star-free unary languages.


\begin{theorem}
\label{thm:unary}
Let  and  be unary star-free languages with quotient complexities  and , respectively. \\
  \hglue10pt  1. 
For each boolean operation  ,  and the bound is tight.\\
 \hglue10pt  2. 
For product,  , 
and the bound is tight.\\ 
 \hglue10pt  3. For the star, the tight bound is
 \\
 \hglue10pt  4. For reversal, .
\end{theorem}

\begin{proof} If a unary star-free language  is finite and , its longest word has length ; if it is infinite,   the longest word not in  has length  .\\
 \hglue10pt  1. 
One verifies that . 
The witness languages are   and   for union and symmetric difference, 
 and  for intersection, and  and  for difference, since .
 \hglue10pt  2. 
One verifies that , and    and  are witnesses.\\
 \hglue10pt  3. 
 If  is infinite, then , and ; hence
.
For , the bounds actually met in the infinite case are 1, 1, 3, 4, 5, respectively.
If  is finite,  it must contain , and if it has , then .
The tight bounds for finite unary star-free languages are
2, 2, 1, 2, 3, respectively. 
Hence the tight bounds for all unary star-free languages for the first five values of  are 2, 2, 3, 4, 5, and the witnesses are , , , , and , respectively.

It was shown in~\cite{CCSY01} that for a finite unary language ,  for . 
For , this bound applies here, and
a witness is .\\
 \hglue10pt  4. 
For unary languages, we have ; hence .
\qed
\end{proof}



\section{Conclusions}
\label{sec:conclusions}

We have shown that all the commonly used regular operations in the class of star-free languages meet the quotient complexity bounds of arbitrary regular languages. The only exceptions are in the product for , reversal, and operations on unary languages.



\providecommand{\noopsort}[1]{}
\begin{thebibliography}{10}

\bibitem{BHK09}
Bordin, H., Holzer, M., Kutrib, M.:
\newblock Determination of finite automata accepting subregular languages.
\newblock Theoret. Comput. Sci. \textbf{410} (2009)  3209--3249

\bibitem{Brz10}
Brzozowski, J.:
\newblock Quotient complexity of regular languages.
\newblock In Dassow, J., Pighizzini, G., Truthe, B., eds.: Proceedings of the
  11th International Workshop on Descriptional Complexity of Formal Systems,
  Magdeburg, Germany, Otto-von-Guericke-Universit{\"a}t (2009)  25--42.

\bibitem{Brz10a}
Brzozowski, J.:
\newblock Complexity in convex languages.
\newblock In Dediu, A.H., Fernau, H., Martin-Vide, C., eds.: Proceedings of the
  4th International Conference on Language and Automata Theory LATA\/.
  Volume 6031 of LNCS, Springer (2010)  1--15

\bibitem{BJL10}
Brzozowski, J., Jir{\'a}skov{\'a}, G., Li, B.:
\newblock Quotient complexity of ideal languages.
\newblock In L\'opez-Ortiz, A., ed.: Proceedings of the 9th Latin American
  Theoretical Informatics Symposium, LATIN\/. Volume 6034 of LNCS,
  Springer (2010)  208--211

\bibitem{BJS10}
Brzozowski, J., Jir\'askov\'a, G., Smith, J.:
\newblock Quotient complexity of bifix-, factor-, and subword-free languages.
\newblock {\tt http://arxiv.org/abs/1006.4843)} (2010)

\bibitem{BJZ10}
Brzozowski, J., Jir{\'a}skov{\'a}, G., Zou, C.:
\newblock Quotient complexity of closed languages.
\newblock In Ablayev, F., Mayr, E.W., eds.: Proceedings of the 5th
  International Computer Science Symposium in Russia, CSR\/. Volume 6072
  of LNCS, Springer (84--95)  208--211

\bibitem{CCSY01}
C\^ampeanu, C., Culik~II, K., Salomaa, K., Yu, S.:
\newblock State complexity of basic operations on finite languages.
\newblock In Boldt, O., J{\"u}rgensen, H., eds.: Revised Papers from the 4th
  International Workshop on Automata Implementation, WIA\/. Volume 2214
  of LNCS, Springer (2001)  60--70

\bibitem{ChHu91}
Cho, S., Huynh, D.T.:
\newblock \mbox{Finite-automaton aperiodicity is PSPACE-complete}.
\newblock Theoret. Comput. Sci. \textbf{88}(1) (1991)  99--116

\bibitem{GAP}
GAP-Group:
\newblock GAP - Groups, Algorithms, Programming - a System for Computational
  Discrete Algebra, {\tt http://www.gap-system.org} (2010)

\bibitem{HaSa09}
Han, Y.S., Salomaa, K.:
\newblock State complexity of basic operations on suffix-free regular
  languages.
\newblock Theoret. Comput. Sci. \textbf{410}(27-29) (2009)  2537--2548

\bibitem{HSW09}
Han, Y.S., Salomaa, K., Wood, D.:
\newblock Operational state complexity of prefix-free regular languages.
\newblock In {\'E}sik, Z., F{\"u}l{\"o}p, Z., eds.: Automata, Formal Languages,
  and Related Topics, University of Szeged, Hungary (2009)  99--115

\bibitem{Lei81}
Leiss, E.:
\newblock Succinct representation of regular languages by boolean automata.
\newblock Theoret. Comput. Sci. \textbf{13} (2009)  323--330

\bibitem{Mas70}
Maslov, A.N.:
\newblock Estimates of the number of states of finite automata.
\newblock Dokl. Akad. Nauk SSSR \textbf{194} (1970)  1266--1268 (Russian).
  English translation: Soviet Math. Dokl. {\bf 11} (1970), 1373--1375.

\bibitem{McPa71}
McNaughton, R., Papert, S.:
\newblock Counter-free automata.
\newblock The MIT Press, Cambridge, MA (1971)

\bibitem{PiSh02}
Pighizzini, G., Shallit, J.:
\newblock Unary language operations, state complexity and \mbox{Jacobsthal's}
  function.
\newblock Internat.\ J.\ Found.\ Comput.\ Sci. \textbf{13} (2002)  145--159

\bibitem{Sch65}
Sch\"utzenberger, M.:
\newblock On finite monoids having only trivial subgroups.
\newblock Inform. and Control \textbf{8} (1965)  190--194

\bibitem{Yu01}
Yu, S.:
\newblock State complexity of regular languages.
\newblock J. Autom. Lang. Comb. \textbf{6} (2001)  221--234

\bibitem{YZS94}
Yu, S., Zhuang, Q., Salomaa, K.:
\newblock The state complexities of some basic operations on regular languages.
\newblock Theoret. Comput. Sci. \textbf{125}(2) (1994)  315--328

\end{thebibliography}

\end{document}
