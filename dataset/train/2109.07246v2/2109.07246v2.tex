\documentclass[10pt,twocolumn,letterpaper]{article}

\usepackage{iccv}
\usepackage{times}
\usepackage{epsfig}
\usepackage{graphicx}
\usepackage{amsmath}
\usepackage{amssymb}
\usepackage{csquotes}
\usepackage{soul}
\usepackage{enumerate}
\usepackage[normalem]{ulem}
\usepackage{algorithm}
\usepackage{algorithmicx}
\usepackage{algpseudocode}
\usepackage{booktabs}
\usepackage{array}
\usepackage{rotating}
\usepackage{arydshln}
\usepackage{ragged2e}
\usepackage{multirow}



\usepackage[pagebackref=true,breaklinks=true,colorlinks,bookmarks=false]{hyperref}

\newcommand{\fdp}[1]{{\textcolor{blue}{#1}}}


\iccvfinalcopy 

\def\iccvPaperID{3167} \def\httilde{\mbox{\tt\raisebox{-.5ex}{\symbol{126}}}}

\ificcvfinal\pagestyle{empty}\fi

\def\NB#1{{\color{green}{\bf [NB:} {\it{#1}}{\bf ]}}}
\def\YC#1{{\color{blue}{\bf [Yuchao:} {\it{#1}}{\bf ]}}}

\def\YR#1{{\color{cyan}{\bf [Yiran:} {\it{#1}}{\bf ]}}}
\def\XY#1{{\color{blue}{\bf [Xin:} {\it{#1}}{\bf ]}}}
\def\Jing#1{{\color{magenta}{\bf [Jing:} {\it{#1}}{\bf ]}}}
\def\ourmodel{CMINet}
\def\ourdataset{COME15K}

\begin{document}

\title{RGB-D Saliency Detection via Cascaded Mutual Information Minimization}

\author{
Jing Zhang~
Deng-Ping Fan~
Yuchao Dai~
Xin Yu~ 
Yiran Zhong~ 
Nick Barnes~ 
Ling Shao\\
 Australian National University\quad
 IIAI\quad
 Northwestern Polytechnical University\\
 University of Technology Sydney\quad
 SenseTime\\
}

\maketitle
\ificcvfinal\thispagestyle{empty}\fi

\begin{abstract}
Existing RGB-D saliency detection models do not explicitly encourage RGB and depth to achieve effective multi-modal learning. In this paper, we introduce a novel multi-stage cascaded learning framework via mutual information minimization to \textbf{explicitly} model the multi-modal information between RGB image and depth data. Specifically, we first map the feature of each mode to a lower dimensional feature vector, and adopt mutual information minimization as a regularizer to reduce the redundancy between appearance features from RGB and geometric features from depth. We then perform multi-stage cascaded learning to impose the mutual information minimization constraint at every stage of the network. Extensive experiments on benchmark RGB-D saliency datasets illustrate the effectiveness of our framework. Further, to prosper the development of this field, 
we contribute the largest (7 larger than NJU2K) \ourdataset~dataset, which contains 15,625 image pairs with
high quality polygon-/scribble-/object-/instance-/rank-level annotations. Based on these rich labels, we additionally construct four new benchmarks
with strong baselines and observe some interesting phenomena, which can motivate future model design. Source code and dataset are available at \url{https://github.com/JingZhang617/cascaded_rgbd_sod}.
\end{abstract}


































































\section{Introduction}
\label{sec:intro}
Saliency detection models are trained to discover the regions of an image that attract human attention. Conventionally, saliency detection is performed mostly
on RGB images only
\cite{BASNet_Sal,SCRN_iccv,F3Net_aaai2020,Iter_Coop_CVPR,A2dele_cvpr2020}. With the availability of
depth data as shown in Table~\ref{tab:existing_rgbd_dataset},
RGB-D saliency detection \cite{jing2020uc,dmra_iccv19, zhang2020bilateral,zhao2019Contrast} attracts
great attention. The extra depth data provides real-world geometric information, which is useful for scenarios when the foreground shares similar appearance to
the background. Further, the robustness of depth sensors (\eg Microsoft Kinect) against lighting changes can also benefit the saliency detection task.








\begin{figure}[t!]
   \begin{center}
   \begin{tabular}{ c@{ } c@{ } c@{ } c@{ } c@{ }}
   {\includegraphics[width=0.185\linewidth]{figure1/000698_left_img.jpg}} &
   {\includegraphics[width=0.185\linewidth]{figure1/000698_left_depth.png}} & {\includegraphics[width=0.185\linewidth]{figure1/000698_left_gt.png}} & {\includegraphics[width=0.185\linewidth]{figure1/000698_left_bbs.png}} &
   {\includegraphics[width=0.185\linewidth]{figure1/000698_left_our.png}} \\
  {\includegraphics[width=0.185\linewidth]{figure1/001908_left_img.jpg}} &
   {\includegraphics[width=0.185\linewidth]{figure1/001908_left_depth.png}} & {\includegraphics[width=0.185\linewidth]{figure1/001908_left_gt.png}} & {\includegraphics[width=0.185\linewidth]{figure1/001908_left_bbs.png}} &
   {\includegraphics[width=0.185\linewidth]{figure1/001908_left_our.png}} \\
    \footnotesize{Image} &\footnotesize{Depth} & \footnotesize{GT} & \footnotesize{BBSNet} & \footnotesize{Ours}\\
   \end{tabular}
   \end{center}
   \vspace{-5pt}
   \caption{Comparison of saliency prediction of a state-of-the-art RGB-D saliency detection model, \eg BBSNet \cite{fan2020bbs}, with ours.
}
   \label{fig:figure1}
\end{figure}

As RGB and depth data capture different information about the same scene, existing RGB-D saliency detection models \cite{dmra_iccv19,chen2018progressively,chen2019multi,chen2019three,zhao2019Contrast,ssf_rgbd,self_attention_rgbd,fan2020bbs,ji2020accurate,HDFNet-ECCV2020,zhang2020bilateral} focus on modeling the complementary information of the RGB image and depth data implicitly by using different fusion strategies.
Three main fusion strategies have been
widely studied:
early fusion \cite{qu2017rgbd,jing2020uc}, late fusion \cite{wang2019adaptive,han2017cnns,A2dele_cvpr2020} and cross-level fusion \cite{dmra_iccv19,chen2018progressively,chen2019multi,chen2019three,zhao2019Contrast,ssf_rgbd,self_attention_rgbd,fan2020bbs,ji2020accurate,HDFNet-ECCV2020,zhang2020bilateral,cmms_rgbd}. Although performance improvement can be achieved with effective fusion strategies,
there are
no constraints on the network design that force
it to learn complementary information between
the two modalities, and we cannot explicitly evaluate the contribution of depth data in those models \cite{Zhao2020IsDR}.

As a multi-modal learning task, a trained model should maximize the joint entropy of different modalities within the network capacity. Maximizing the joint entropy is also equal to the minimization of mutual information, which prevents a network from focusing on redundant information.
To \textit{explicitly} model the
complementary information between
the RGB image and depth data, we introduce a
multi-stage cascaded learning framework
via mutual information minimization.
Specifically, we introduce
mutual information minimization as regularizer (as shown in Fig.~\ref{fig:network_overview}) to achieve two main benefits: 1) explicitly modeling the redundancy between
appearance features and geometric features;
2) effectively fusing 
appearance features and geometric features with the mutual information minimization constraint.
The produced saliency maps in Fig.~\ref{fig:figure1} illustrate effectiveness of our solution.

















Furthermore, we find that
there is no
large-scale RGB-D saliency detection training set. In Table~\ref{tab:existing_rgbd_dataset} we compare the widely used RGB-D saliency datasets,
in terms of the size,
types of data,
the sources of depth data, and their roles (for training \enquote{Tr} or for testing \enquote{Te}) in RGB-D saliency detection. 
We note that the conventional training set for RGB-D saliency detection is a
combination of samples from the NJU2K \cite{NJU2000} dataset and NLPR \cite{peng2014rgbd}, which includes only 2,200 image pairs in total. Although another 800 training images from the DUT dataset \cite{dmra_iccv19} can serve as the third part of the training set, the total number of training images is 3,000, which is not big enough, and may lead to biased model.
In addition,
we observe that there are
similar backgrounds in the existing RGB-D saliency training set, \eg more than 10\% of the training dataset comes from the same scene with similar illumination conditions.
The lack of diversity in the
dataset may render models of poor generalization ability.
At the same time, we also notice that the largest testing set \cite{sip_dataset} contains only 1,000 image pairs, which may not be enough to fully evaluate the overall performance of the
deep RGB-D saliency detection models.


To provide an RGB-D saliency detection dataset for robust model training, and a sufficient size of
testing data for model evaluation, we contribute the largest
RGB-D saliency detection dataset, relabeled from Holo50K dataset \cite{hua2020holopix50k}, with 8,025 image pairs for training and 7,600 image pairs for testing. We provide not only binary annotations, but also
annotations for stereoscopic saliency detection, scribble and polygon annotations for weakly supervised RGB-D saliency detection, instance-level RGB-D saliency annotations and RGB-D saliency ranking.
Moreover, we contribute 5,000 unlabeled training images for semi-supervised or self-supervised RGB-D saliency detection.

Our main contributions are: 1)
We design a multi-stage cascaded learning framework via mutual information minimization for
RGB-D saliency detection
to \enquote{explicitly} model redundancy between the
RGB image and 
depth data.
    2) The mutual information minimization regularizer can be easily extend to other multi-modal learning pipelines to model the redundancy of multiple modalities.
3) We contribute the largest RGB-D saliency detection dataset, with a 15,625 labeled set and a 5,000 unlabeled set to achieve
fully-/weakly-/un-supervised RGB-D saliency detection.
4) We present new benchmarks for RGB-D saliency detection, and introduce baseline models for stereoscopic and weakly supervised RGB-D saliency detection.

\begin{figure*}[tbp]
\centering
\includegraphics[width=0.86\linewidth]{figures/iccv_overview.png}
\vspace{-5pt}
   \caption{Overview of the proposed multi-stage cascaded learning framework for RGB-D saliency detection.
We feed the RGB image and depth to the saliency encoders to extract saliency feature of each mode
with the mutual information regularizer term to push the features to be different from each other. Then, we fuse the lower dimensional feature of each mode ( and ) with raw image feature ( and ) to effectively model the complementary information of each mode and obtain our final prediction . The \enquote{DenseASPP} module is the dense atrous spatial pyramid pooling module from \cite{denseaspp}, and \enquote{DA} is the dual attention module from \cite{fu2019dual}.
}
   \label{fig:network_overview}
\end{figure*}



\begin{table}[t!]
  \centering
\scriptsize
  \renewcommand{\arraystretch}{1.2}
  \renewcommand{\tabcolsep}{1.2mm}
  \caption{\small Comparison with the widely used RGB-D datasets.
  }
\label{tab:existing_rgbd_dataset}
  \begin{tabular}{r|r|c|c|r}
  \hline
    \textbf{Dataset} & \textbf{Size} & \textbf{Type} & \textbf{Depth Source}                     &  \textbf{Role} \\
   \hline
   \hline
    NJU2K\cite{NJU2000}  &1,985  &Movie/Internet  & FujiW3 camera + optical flow  &Tr, Te \\
    DUT \cite{dmra_iccv19} & 1,200  & Indoor/Outdoor & Light-field cameras &Tr, Te \\
    NLPR \cite{peng2014rgbd} &1,000  & Indoor/Outdoor&  Microsoft Kinect &Tr, Te  \\
    SSB \cite{niu2012leveraging}   &1,000  & Internet & Stereo cameras  &Te\\
    SIP \cite{sip_dataset} & 929   & Person in outside & Huawei Mate10 &Te \\
    DES \cite{cheng2014depth}  &135  & Indoor& Microsoft Kinect &Te \\
    LFSD \cite{li2014saliency}   &80  & Indoor/Outdoor& Lytro Illum cameras &Te\\
  \hline
  \textbf{Ours} & \textbf{15,625}   & Indoor/Outdoor & Holopix Social Platform & Tr, Te \\
  \hline

  \end{tabular}
  \vspace{-4mm}
\end{table}


\section{Related Work}

\subsection{RGB-D saliency detection models}
For RGB-D saliency detection, one of the main focuses is to explore the complementary information between the RGB image and the depth data. The former provides appearance information of a scene, while the latter  introduces geometric information. Depending on how information from these two modalities is fused, existing RGB-D saliency detection models can be divided into three categories: early-fusion models \cite{qu2017rgbd,jing2020uc}, late-fusion models \cite{wang2019adaptive,han2017cnns,A2dele_cvpr2020} and cross-level fusion models \cite{dmra_iccv19,chen2018progressively,chen2019multi,chen2019three,zhao2019Contrast,ssf_rgbd,self_attention_rgbd,fan2020bbs,ji2020accurate,HDFNet-ECCV2020,zhang2020bilateral,cmms_rgbd,Li_2020_CMWNet}. 
The first solution directly concatenates the RGB image with its depth,
while
the late fusion models treat each mode (RGB and depth) separately, and then fusion is achieved at the output layer.
The above two solutions perform multi-modal fusion at the input or output layer, while the cross-level fusion models fuse RGB and depth in the feature space. Specifically, features of an RGB image and depth are gradually fused in different level of the network
\cite{HDFNet-ECCV2020,Li_2020_CMWNet,fan2020bbs,cmms_rgbd,Luo2020CascadeGN,chen2020eccv,ATSA,self_attention_rgbd,ssf_rgbd}.
Although the existing methods fuse the RGB image and depth data for multi-modal learning,
none of them \textit{explicitly} illustrate
how the network achieve effective multi-modal learning.
We propose a cross-level fusion model as shown in Fig.~\ref{fig:network_overview}. By designing the \enquote{Mutual information regularizer}, we aim to reduce redundancy of appearance features and geometric features for effective multi-modal learning.







\subsection{Multi-modal learning with RGB-D dataset}
The basic assumption for multi-modal learning is that there exits both common and diverse information in the separate modalities. For the RGB-D dataset, the RGB image and depth data share similar semantic information, which can be defined as the common parts. The RGB image encodes the appearance information, including intensity or color of the object, while the depth data encodes geometric information, showing the relative geometric localization of the objects. The difference between appearance information and geometric information is the diverse part of these two modalities. The main focus to achieve multi-modal learning for RGB-D data is through using different fusion strategies \cite{wang2018depthconv,real_time_rgbd_semantic,Luo2020CascadeGN,ASIF-Net}, \eg early fusion, late fusion or cross-level fusion. Different from conventional solutions, we introduce a multi-stage cascaded learning framework via mutual information minimization to reduce the feature redundancy of each modal. Although mutual information maximization \cite{Informax_book,TscDjoRubGelLuc20} is widely used in representation learning to produce a representation that is similar to the input, we take the mutual information minimization as a regularizer
to reduce the feature redundancy for effective multi-modal learning.


\subsection{RGB-D saliency datasets}
The widely used RGB-D saliency detection datasets include NJU2K \cite{NJU2000}, NLPR \cite{peng2014rgbd}, SSB \cite{niu2012leveraging}, DES \cite{cheng2014depth}, LFSD \cite{li2014saliency}, SIP \cite{sip_dataset}, DUT \cite{dmra_iccv19}, \etc, as shown in Table \ref{tab:existing_rgbd_dataset}. The typical training dataset
is the combination of 1,485 images from NJU2K \cite{NJU2000} and 700 images from NLPR \cite{peng2014rgbd}. Piao \etal~\cite{dmra_iccv19} introduces the DUT dataset, with 800 images
for training and 400 images for testing.
To prosper the RGB-D saliency detection task, we introduce the largest RGB-D saliency detection training and testing dataset, which will be introduced in Section \ref{the_new_dataset}.



\section{Proposed \ourmodel}
\label{our_method}

We introduce a multi-stage cascaded learning
framework in Fig.~\ref{fig:network_overview} to explicitly model 
complementary information for RGB-D saliency detection.


\subsection{Saliency encoder}
We denote our training dataset as , where  indexes the images and  is the size of the training set,  and  are the input RGB-D image pair and its
corresponding ground-truth (GT) saliency map.
We feed the training image pairs (RGB image  and
depth ) to the saliency encoder,
as illustrated in Fig.~\ref{fig:network_overview}, to extract appearance features  and geometric features  respectively, where  and  are the parameters of our RGB saliency encoder and depth saliency encoder respectively.

We build the saliency encoders upon the ResNet50 network \cite{ResHe2015}, which includes four convolutional stages . We add one additional convolutional layer of kernel size  after each  to reduce the channel dimension of  to , and obtain feature maps .
The final output of the RGB saliency encoder module is , and that of the depth saliency encoder is . Note that, the RGB saliency encoder and depth saliency encoder share the same network structure but not weights.









\subsection{Feature embedding}
\label{latent_feature_sec}


Given the output  from the RGB saliency encoder and  from the depth saliency encoder, we aim to map both the RGB feature and depth feature to a lower-dimensional feature space for feature embedding.
Specifically, we propose a multi-stage cascaded learning strategy to perform the complementary learning at each stage of the network. For the lower stages, we feed the RGB feature  and the depth feature  to two different  convolutional layers (\enquote{conv3x3} in Fig.~\ref{fig:network_overview}) to obtain feature maps of channel size  for both the RGB branch and depth branch. Then we adopt two fully connected layers (\enquote{fc} in Fig.~\ref{fig:network_overview}) to map the feature map of channel size  to two different lower-dimensional feature vectors  and  of size  respectively. The complementary learning related loss (which will be introduced in Section \ref{sub_sec_complementary} and \ref{obj_fun_sec}) is adopted to reduce the feature redundancy of RGB and depth at lower stages. 
At the highest stage, we first tile the lower dimensional feature vector  and  in spatial dimension. Then, we concatenate them with raw image feature\footnote{We define the concatenation of  as the raw RGB feature and the concatenation of  as raw depth feature} of the other mode
to obtain the  channel size feature map  and  for the RGB branch and depth branch respectively.


\begin{figure*}[t!]
   \begin{center}
   \begin{tabular}{ c@{ } c@{ } c@{ } c@{ } c@{ } c@{ } c@{ }}
   {\includegraphics[width=0.13\linewidth]{annotations/LOFvIAa5x5LPrYdrmx9_right_img.jpg}} 
   & {\includegraphics[width=0.13\linewidth]{annotations/LOFvIAa5x5LPrYdrmx9_right_depth.png}} 
   & {\includegraphics[width=0.13\linewidth]{annotations/LOFvIAa5x5LPrYdrmx9_right_gt.png}} 
   & {\includegraphics[width=0.13\linewidth]{annotations/LOFvIAa5x5LPrYdrmx9_right_gtinstance.png}}
   & {\includegraphics[width=0.13\linewidth]{annotations/LOFvIAa5x5LPrYdrmx9_right_rank.png}}
   & {\includegraphics[width=0.13\linewidth]{annotations/LOFvIAa5x5LPrYdrmx9_right_mscribble.jpg}}
   & {\includegraphics[width=0.13\linewidth]{annotations/LOFvIAa5x5LPrYdrmx9_right_poly.jpg}} \\
   {\includegraphics[width=0.13\linewidth]{annotations/LON2V-mBqmOPokhr-P2_right_img.jpg}} 
   & {\includegraphics[width=0.13\linewidth]{annotations/LON2V-mBqmOPokhr-P2_right_depth.png}} 
   & {\includegraphics[width=0.13\linewidth]{annotations/LON2V-mBqmOPokhr-P2_right_gt.png}} 
   & {\includegraphics[width=0.13\linewidth]{annotations/LON2V-mBqmOPokhr-P2_right_gtinstance.png}}
   & {\includegraphics[width=0.13\linewidth]{annotations/LON2V-mBqmOPokhr-P2_right_rank.png}}
   & {\includegraphics[width=0.13\linewidth]{annotations/LON2V-mBqmOPokhr-P2_right_mscribble.jpg}}
   & {\includegraphics[width=0.13\linewidth]{annotations/LON2V-mBqmOPokhr-P2_right_poly.jpg}} \\
    \footnotesize{(a)} &\footnotesize{(b)} & \footnotesize{(c)} & \footnotesize{(d)} &\footnotesize{(e)} & \footnotesize{(f)} & \footnotesize{(g)}\\
   \end{tabular}
   \end{center}
   \vspace{-5pt}
   \caption{Annotations of our new RGB-D saliency detection datasets: (a) the RGB image, (b) depth data and (c) binary ground truth,
(d) instance level annotation, (e) ranking based annotation, (f) scribble annotation and (g) polygon annotation. Our diverse annotations will facilitate developing different fully/weakly supervised RGB-D saliency detection.}
   \label{fig:dataset_annotation_all}
\end{figure*}

\subsection{Multi-modal learning}
\label{sub_sec_complementary}
After obtaining the feature embedding  and  for the
RGB image and depth data, we introduce a
mutual information minimization regularizer to explicitly reduce the redundancy between
these two modalities. Our basic assumption is that a good appearance saliency feature and geometric saliency feature pair should carry both common parts (semantic related) and different attributes (domain related). 
Mutual information  is used to measure the difference between
the entropy terms:

where  is the entropy,  and  are marginal entropies, and  is the joint entropy of  and . Intuitively, we have the Kullback–Leibler divergence (KL) of the two latent variable (or the conditional entropies) as:


where  is the cross-entropy.
We then sum Eq.~\ref{mutual_information}, Eq.~\ref{symetic_kl1} and Eq.~\ref{symetic_kl2}, and obtain:


Given the RGB image and depth data,  is non-negative, then
minimizing the mutual information can be achieved by minimizing:
.
Intuitively,
 measures the reduction of uncertainty in  when  is observed, or vice versa. As a multi-modal learning task, each mode should learn some new attributes of the task from other modalities. By minimizing , we can effectively explore the complementary attributes of both modalities.
Note that, although KL loss term was used in \cite{jing2020uc} as distribution similarity measure,
we use it to measure mode similarity for multi-modal learning.




\subsection{Saliency decoder}
With the mutual information as a regularizer, we achieve the feature redundancy constraint at the lower stages of the network, and obtain the refined RGB saliency feature  and refined depth saliency feature  at the highest stage. We then adopt one DenseASPP~\cite{denseaspp} module after  to obtain the RGB saliency prediction  with multi-scale context information. Similarly, we can obtain the depth saliency prediction .
The saliency decoder  (\enquote{Decoder} in Fig.~\ref{fig:network_overview}) takes the refined saliency features , , as well as the RGB saliency prediction  and depth saliency prediction  as input to produce our final prediction , where  is the parameter set of the saliency decoder.
Specifically,
we add a position attention module and a channel attention module \cite{fu2019dual} after  and  to obtain  and  respectively with discriminative features highlighted.
Then we concatenate  and , and
feed it to the DenseASPP~\cite{denseaspp} module to obtain our saliency prediction . To further fuse information from both modes, we concatenate ,  and  channel-wise, and feed it to a  convolutional layer
to achieve our final prediction .

\subsection{Objective function}
\label{obj_fun_sec}
We adopt the binary cross-entropy loss  as our objective function to train our multi-stage cascaded learning framework,
where the complementary constraint, as indicated in Eq.~\eqref{mutual_information}, pushes the saliency feature distribution of the RGB image to be apart from that of the depth data.
Our final objective function is:

and empirically we set .
As the range of the  is 10 times larger than that of the , we set its loss weight as  for balanced learning.








\section{\ourdataset~Dataset}
\label{the_new_dataset}
As shown in Table \ref{tab:existing_rgbd_dataset}, the exiting RGB-D saliency detection training dataset is not big enough, which may lead to models of poor generalization ability. Further, as the training dataset is a combination of samples from NJU2K \cite{NJU2000} and NLPR dataset \cite{peng2014rgbd}, different splits
of the training set often lead to inconsistent performance evaluation. Lastly, the small size of testing dataset may fails to full evaluate the RGB-D saliency detection models.
To boost 
RGB-D saliency detection, we contribute the largest RGB-D saliency detection dataset. We provide binary annotation, instance level annotation, ranking based annotation, weak annotation as shown in Fig.~\ref{fig:dataset_annotation_all}. The detailed analysis of the dataset is introduced in the supplementary material.



\subsection{Dataset annotation}
\label{dataset_annotation}
Our new \ourdataset~dataset is based on Holo50K \cite{hua2020holopix50k}, which is a stereo dataset, including scenarios from both indoor and outdoor. We first filter\footnote{We deleted the violent images.} the Holo50K dataset and then obtain
16,000 stereo image pairs for labelling (the candidate labeled set) and another 5,000 image pairs as the unlabeled set. Note that the stereo pairs in Holo50K dataset are directly captured by a stereo camera without rectification, we use a modified version of a SOTA off-the-shelf stereo matching algorithm \cite{zhong2020displacement} to compute the depth for both the candidate labeled set and unlabeled set with the left-right view images as input.


To provide annotations for the candidate labeled set, we firstly ask five \enquote{coarse} annotators to
label salient regions of each image (only the right view image is used) with scribble annotations.
Secondly, the \enquote{fine} annotators will segment the full scopes of salient objects and provide instance-level annotations.
Thirdly, we perform \enquote{majority voting} to obtain the binary GT saliency maps for our RGB-D saliency detection task. Note that, we delete those samples with no common salient regions, and obtain our final labeled dataset of size 15,625. 
Further, based on the scribble annotations and instance-level saliency maps, we rank each saliency instance according to the initial scribble annotations to form our RGB-D saliency ranking dataset. 

We also provide weak annotations for weakly-supervised RGB-D saliency detection, including scribble annotations and polygon annotations. We define the majority of the scribble annotations from multiple coarse annotators as the scribble annotations of our dataset. Specifically, we first obtain the instance with the majority of scribble. Then, we
define the scribble on the majority instance as
our scribble annotation.
We label the majority salient instance with polygons to form our polygon based annotations.













\begin{table*}[t!]
  \centering
  \scriptsize
  \renewcommand{\arraystretch}{0.9}
  \renewcommand{\tabcolsep}{0.30mm}
  \caption{Benchmarking results of three
leading handcrafted feature-based models and eighteen deep models () on six RGB-D saliency datasets.   denote the larger and smaller is better, respectively. Here, we adopt mean  and mean ~\cite{Fan2018Enhanced}.}
  \label{tab:BenchmarkResults}
  \begin{tabular}{lr|ccc|cccccc|cccccccccccc|c}
  \hline
&  &\multicolumn{3}{c|}{Early Fusion Models}&\multicolumn{6}{c|}{Late Fusion Models}&\multicolumn{12}{c|}{Cross-level Fusion Models}& \\
    & Metric &
   DF & DANet  & UCNet   & LHM  & DESM & CDB &
   A2dele & AFNet & CTMF & JLDCF & DMRA & PCF & MMCI   & TANet   & CPFP & S2MA & BBS-Net & CoNet   & HDFNet& BiaNet & CMWNet & \ourmodel \\
   &  & \cite{qu2017rgbd}        
   & \cite{DANet}       
   & \cite{jing2020uc}      
   & \cite{peng2014rgbd}   
   & \cite{cheng2014depth}                 
   & \cite{liang2018stereoscopic}  
   & \cite{A2dele_cvpr2020}
   & \cite{wang2019adaptive}   
   & \cite{han2017cnns} 
   & \cite{Fu2020JLDCF}  
   & \cite{dmra_iccv19}       
   & \cite{chen2018progressively}
   & \cite{chen2019multi}    
   & \cite{chen2019three}
   & \cite{zhao2019Contrast}  
   & \cite{self_attention_rgbd}   
   & \cite{fan2020bbs} 
   & \cite{ji2020accurate} 
   & \cite{HDFNet-ECCV2020} 
   & \cite{zhang2020bilateral}
& \cite{cmms_rgbd}
   & Ours \\
  \hline
  \multirow{4}{*}{\rotatebox{90}{\textit{NJU2K}}}&     & .763 & .897 & .897  & .514 & .665 & .632 & .873 & .822 & .849 & .902 & .886 & .877 & .858 & .879 & .878 & .894 & .921 & .911 & .908 & .915 & .903 & \textbf{.939}  \\
    &      & .653 & .877 & .886  & .328 & .550 & .498 & .867 & .827 & .779 & .885 & .873 & .840 & .793 & .841 & .850 & .865 & .902 & .903 & .892 & .903 & .881 & \textbf{.925}   \\
    &        & .700 & .926 & .930  & .447 & .590 & .572 & .913 & .867 & .846 & .935 & .920 & .895 & .851 & .895 & .910 & .914 & .938 & .944 & .936 & .934 & .923 & \textbf{.956}  \\
    &  & .140 & .046 & .043  & .205 & .283 & .199 & .051 & .077 & .085 & .041 & .051 & .059 & .079 & .061 & .053 & .053 & .035 & .036 & .038 & .039 & .046 & \textbf{.032}  \\ \hline
\multirow{4}{*}{\rotatebox{90}{\textit{SSB}}}&     & .757 & .892 & .903  & .562 & .642 & .615 & .876 & .825 & .848 & .903 & .835 & .875 & .873 & .871 & .879 & .890 & .908 & .896 & .900 & .904 & .905 & \textbf{.921}   \\
    &      & .617 & .857 & .884 & .378 & .519 & .489 & .874 & .806 & .758 & .873 & .837 & .818 & .813 & .828 & .841 & .853 & .883 & .877 & .870 & .879 & .872 & \textbf{.895}   \\
    &        & .692 & .915 & .938 & .484 & .579 & .561 & .925 & .872 & .841 & .936 & .879 & .887 & .873 & .893 & .911 & .914 & .928 & .939 & .931 & .926 & .928 & \textbf{.959}  \\
    &  & .141 & .048 & .039 & .172 & .295 & .166 & .044 & .075 & .086 & .040 & .066 & .064 & .068 & .060 & .051 & .051 & .041 & .040 & .041 & .043 & .043 & \textbf{.034}   \\ \hline
\multirow{4}{*}{\rotatebox{90}{\textit{DES}}}&     & .752 & .905 & .934 & .578 & .622 & .645 & .881 & .770 & .863 & .931 & .900 & .842 & .848 & .858 & .872 & .941 & .933 & .906 & .926 & .931 & .934 & \textbf{.953}  \\
    &      & .604 & .848 & .919 & .345 & .483 & .502 & .868 & .713 & .756 & .907 & .873 & .765 & .735 & .790 & .824 & .909 & .910 & .880 & .910 & .910 & .909 & \textbf{.926}  \\
    &        & .684 & .961 & .967 & .477 & .566 & .572 & .913 & .809 & .826 & .959 & .933 & .838 & .825 & .863 & .888 & .952 & .949 & .939 & .957 & .948 & .955 & \textbf{.970}  \\
    &  & .093 & .028 & .019 & .114 & .299 & .100 & .030 & .068 & .055 & .021 & .030 & .049 & .065 & .046 & .038 & .021 & .021 & .026 & .021 & .021 & .022 & \textbf{.015}  \\ \hline
\multirow{4}{*}{\rotatebox{90}{\textit{NLPR}}}&     & .806 & .908 & .920 & .630 & .572 & .632 & .887 & .799 & .860 & .925 & .899 & .874 & .856 & .886 & .888 & .916 & .930 & .900 & .923 & .925 & .917 & \textbf{.941}  \\
    &      & .664 & .850 & .891 & .427 & .430 & .421 & .871 & .755 & .740 & .894 & .865 & .802 & .737 & .819 & .840 & .873 & .896 & .859 & .894 & .894 & .877 & \textbf{.909}   \\
    &        & .757 & .945 & .951 & .560 & .542 & .567 & .933 & .851 & .840 & .955 & .940 & .887 & .841 & .902 & .918 & .937 & .950 & .937 & .955 & .948 & .939 & \textbf{.964} \\
    &  & .079 & .031 & .025 & .108 & .312 & .108 & .031 & .058 & .056 & .022 & .031 & .044 & .059 & .041 & .036 & .030 & .023 & .030 & .023 & .024 & .029 & \textbf{.019}  \\ \hline
\multirow{4}{*}{\rotatebox{90}{\textit{LFSD}}}&     & .791 & .845 & .864 & .557 & .722 & .520 & .831 & .738 & .796 & .862 & .847 & .794 & .787 & .801 & .828 & .837 & .864 & .842 & .854 & .845 & .876 & \textbf{.877}  \\
    &      & .679 & .826 & .855 & .396 & .612 & .376 & .829 & .736 & .756 & .848 & .845 & .761 & .722 & .771 & .811 & .806 & .843 & .834 & .835 & .834 & .862 & \textbf{.862}  \\
    &        & .725 & .872 & .901 & .491 & .638 & .465 & .872 & .796 & .810 & .894 & .893 & .818 & .775 & .821 & .863 & .855 & .883 & .886 & .883 & .871 & .900 & \textbf{.911}   \\
    &  & .138 & .082 & .066 & .211 & .248 & .218 & .076 & .134 & .119 & .070 & .075 & .112 & .132 & .111 & .088 & .094 & .072 & .077 & .077 & .085 & .066 & \textbf{.064}   \\ \hline
\multirow{4}{*}{\rotatebox{90}{\textit{SIP}}}&     & .653 & .878 & .875 & .511 & .616 & .557 & .826 & .720 & .716 & .880 & .806 & .842 & .833 & .835 & .850 & .872 & .879 & .868 & .886 & .883 & .867 & \textbf{.894}  \\
    &      & .465 & .829 & .867 & .287 & .496 & .341 & .827 & .702 & .608 & .873 & .811 & .814 & .771 & .803 & .821 & .854 & .868 & .855 & .875 & .873 & .851 & \textbf{.887}  \\
    &        & .565 & .914 & .914 & .437 & .564 & .455 & .887 & .793 & .704 & .918 & .844 & .878 & .845 & .870 & .893 & .905 & .906 & .915 & .923 & .913 & .900 & \textbf{.933}   \\
    &  & .185 & .054 & .051 & .184 & .298 & .192 & .070 & .118 & .139 & .049 & .085 & .071 & .086 & .075 & .064 & .057 & .055 & .054 & .047 & .052 & .062 & \textbf{.044}  \\ \hline
\end{tabular}
\end{table*}





\subsection{Dataset splitting}
We divide the labeled set into one training set with 8,025 samples and two different testing sets of size 4,600 and 3,000 respectively, namely the \enquote{Normal} and the \enquote{Difficult} sets. The training dataset is generated by randomly selecting
8,025 images from the labeled set. For the testing datasets, we intend to introduce two sets of different difficulties. Specifically, we rank the RGB images based on both global and interior contrast, and denote samples with low global contrast and high interior contrast as the difficult samples\footnote{Details about image global and interior contrast are introduced in the supplementary materials.}. Then we have a pool of 1,800 difficult samples  and 5,800 normal samples . We random select 30\% samples from  and 70\% samples from  to obtain our \enquote{Normal} testing set,
and the others as \enquote{Difficult} set.

\section{Experiments}
We compare our method
\ourmodel~with existing RGB-D saliency detection models,
and report the performance in Table \ref{tab:BenchmarkResults} \& \ref{tab:BenchmarkResults_DUTRGBD}. Furthermore, we retrain the state-of-the-art RGB-D saliency detection models on our new training dataset, and provide the performance of those models on our testing dataset in Table \ref{tab:BenchmarkResults_OurTestSet}.




\begin{table}[t!]
  \centering
  \scriptsize
  \renewcommand{\arraystretch}{1.0}
  \renewcommand{\tabcolsep}{0.7mm}
\caption{Model performance on DUT~\cite{dmra_iccv19} testing set.}
  \label{tab:BenchmarkResults_DUTRGBD}
  \begin{tabular}{r|cccccccc|c}
  \hline
    Metric & UCNet  & JLDCF &  A2dele & DMRA & CPFP & S2MA & CoNet  & HDFNet & \ourmodel \\    
   & \cite{jing2020uc}          
   & \cite{Fu2020JLDCF}      
   & \cite{A2dele_cvpr2020}
   & \cite{dmra_iccv19}
   & \cite{zhao2019Contrast}  
   & \cite{self_attention_rgbd}  
   & \cite{ji2020accurate} 
   & \cite{HDFNet-ECCV2020} 
   & Ours\\
  \hline
  
   & .907 &  .905 & .884 & .886 & .749 & .903 & .919 & .905  & \textbf{.928}  \\
       & .902 &  .884 & .889 & .883 & .695 & .881 & .911 & .889  & \textbf{.921}  \\
         & .931 &  .932 & .924 & .924 & .759 & .926 & .947 & .929  & \textbf{.959}  \\
      & .038 &  .043 & .043 & .048 & .100 & .044 & .033 & .040  & \textbf{.030}  \\ \hline
  \end{tabular}
\end{table}

\subsection{Setup}
\noindent\textbf{Dataset:} For fair comparisons with existing RGB-D saliency detection models, we follow the conventional training setting, in which the training set is a combination of 1,485 images from the NJU2K
dataset \cite{NJU2000} and 700 images from the NLPR dataset \cite{peng2014rgbd}. We then test the performance of our model and competing models on the NJU2K testing set, NLPR, testing set 
LFSD \cite{li2014saliency}, DES \cite{cheng2014depth}, SSB \cite{niu2012leveraging} SIP \cite{sip_dataset} and DUT \cite{dmra_iccv19} testing set.

\noindent\textbf{Metrics:} We evaluate the performance of the models on four golden evaluation metrics, \ie, Mean Absolute Error (), Mean F-measure (), Mean E-measure ()~\cite{Fan2018Enhanced} and S-measure ()~\cite{fan2017structure}, which are explained in detail in the supplementary materials.




\noindent\textbf{Training details:} Our model is implemented using the \textit{Pytorch} library.
The two saliency encoders share the same network structure, and are initialized with ResNet50 \cite{ResHe2015} trained on ImageNet, and the other newly added layers are randomly initialized. We resize all the images and ground truth to the same spatial size of  pixels. We set the maximum epoch as 100, and initial learning rate as 5e-5. We adopt the \enquote{step} learning rate decay policy, and set the decay size as 80 and decay rate as 0.1. The whole training takes 4.5 hours with batch size 5 on an NVIDIA GeForce RTX 2080Ti GPU for the conventional training (NJU2K-train+NLPR-train)
dataset, and 16 hours with our new training (\ourdataset-train)
dataset.

\subsection{Model comparison}
\noindent\textbf{Quantitative comparison:} We compare the performance of our \ourmodel~and state-of-the-art RGB-D saliency detection models, and report the performance in Table \ref{tab:BenchmarkResults}. Note that, we use the training set of NJU2K and NLPR as existing deep RGB-D saliency detection models. 
The consistently better performance of our model indicates the effectiveness of our solution.
Further, we observe that the performance gaps of current RGB-D saliency detection are very subtle, \eg, BBS-Net~\cite{fan2020bbs}, CoNet \cite{ji2020accurate}, HDFNet \cite{HDFNet-ECCV2020}, BiaNet \cite{zhang2020bilateral}, and CMWNet~\cite{zhang2020bilateral}, which demonstrates the
necessity of
larger and diverse training and testing datasets for model training and evaluation.





\noindent\textbf{Performance on DUT~\cite{dmra_iccv19} dataset:} Some existing RGB-D saliency detection approaches \cite{dmra_iccv19,self_attention_rgbd} fine-tune their models on the DUT training dataset \cite{dmra_iccv19} to evaluate their performance on the DUT testing set. To test our model on the DUT testing set, we follow the same training strategy.
In Table \ref{tab:BenchmarkResults_DUTRGBD}, all the models are trained with the conventional training set and then fine-tuned on the DUT training set. 
The consistently superior performance illustrates the superiority of our model.
Furthermore, since the current testing performance in Table \ref{tab:BenchmarkResults_DUTRGBD} is achieved in a train-retrain manner (train on the combination training set, and retrain on DUT training set \cite{dmra_iccv19}), we re-train these models with a combination of the conventional training set and DUT training set, and observe consistently worse performance. This observation tells us that inconsistent annotations may occur in the
above three training sets (\ie, NJU2K, NLPR and DUT). It also motivates us to collect a larger training dataset with consistent annotations for robust model training.


\begin{table*}[t!]
  \centering
  \scriptsize
  \renewcommand{\arraystretch}{0.9}
  \renewcommand{\tabcolsep}{0.55mm}
  \caption{Performance of the extra experiments.
  }
  \begin{tabular}{l|cccc|cccc|cccc|cccc|cccc|cccc}
  \hline
&\multicolumn{4}{c|}{NJU2K\cite{NJU2000}}&\multicolumn{4}{c|}{SSB \cite{niu2012leveraging}}&\multicolumn{4}{c|}{DES \cite{cheng2014depth}}&\multicolumn{4}{c|}{NLPR \cite{peng2014rgbd}}&\multicolumn{4}{c|}{LFSD \cite{li2014saliency}}&\multicolumn{4}{c}{SIP \cite{sip_dataset}} \\
    Method 
    &  &  &  & 
    &  &  &  & 
    &  &  &  & 
    &  &  &  & 
    &  &  &  & 
    &  &  &  &  \\
  \hline
Base & .910 & .900 & .935 & .035 & .890 & .870 & .917 & .043 & .926 & .915 & .959 & .018 & .920 & .898 & .942 & .024 & .842 & .835 & .880 & .077 & .879 & .876 & .917 & .049    \\
   K3 & .928 & .908 & .947 & \textbf{.032} & .909 & .892 & .939 & .036 & .934 & .922 & .964 & .018 & .925 & .904 & .956 & .022 & .869 & .845 & .898 & .067 & .885 & .879 & .919 & .047    \\
   K32 & .924 & .909 & .944 & .033 & .908 & .894 & .941 & .036 & .938 & .923 & .966 & .017 & .927 & .906 & .959 & .021 & .856 & .853 & .900 & .065 & .885 & .878 & .921 & .046   \\
   SS & .926 & .913 & .943 & .034 & .914 & .882 & .942 & .036 & .946 & .927 & .968 & .017 & .932 & .896 & .954 & .021 & .861 & .852 & .896 & .067 & .885 & .879 & .925 & .046    \\
   W0 & .918 & .907 & .944 & .033 & .892 & .877 & .923 & .042 & .934 & .924 & .964 & .017 & .924 & .900 & .945 & .023 & .843 & .836 & .881 & .076 & .884 & .878 & .916 & .048    \\
   W1 & .919 & .909 & .946 & \textbf{.032} & .905 & .886 & .937 & .037 & .938 & .927 & .971 & .016 & .923 & .903 & .956 & .022 & .857 & .853 & .891 & .071 & .887 & .882 & .921 & .045    \\
    \hline
     & .925 & .908 & .945 & .033 & .908 & .887 & .939 & .036 & .946 & .925 & .965 & .016 & .938 & .907 & .962 & .023 & .862 & .845 & .896 & .068 & .889 & .886 & .927 & .045    \\
 & .898 & .890 & .930 & .040 & .899 & .876 & .924 & .042 & .891 & .883 & .920 & .028 & .908 & .885 & .932 & .031 & .817 & .807 & .853 & .095 & .860 & .865 & .905 & .056    \\
    & .915 & .901 & .932 & .037 & .903 & .878 & .931 & .039 & .920 & .908 & .942 & .021 & .914 & .893 & .943 & .026 & .850 & .841 & .886 & .071 & .876 & .870 & .910 & .051    \\
    \hline
   \ourmodel & \textbf{.939} & \textbf{.925} & \textbf{956} & \textbf{.032} & \textbf{.921} & \textbf{.895} & \textbf{.959}& \textbf{034} & \textbf{.953} & \textbf{.926} & \textbf{.970} & \textbf{.015} & \textbf{.941} & \textbf{.909} & \textbf{.964} & \textbf{.019} & \textbf{.877} & \textbf{.860} & \textbf{.911} & \textbf{.064} & \textbf{.894} & \textbf{.887} & \textbf{.933} & \textbf{.044}    \\
   \hline
  \end{tabular}
  \label{tab:ablation_study_experiments}
\end{table*}

\noindent\textbf{Qualitative comparison:} We further visualize our prediction in Fig.~\ref{fig:figure1}. The qualitative comparisons demonstrate that with the proposed learning strategy, our model can effectively explore the two modalities for multi-modal learning.
More results are shown in the supplementary materials.


\noindent\textbf{Model size and running time:} Our model size is 84M, which is comparable with the state-of-the-art models, \eg model size of BBS-Net \cite{fan2020bbs} is 100M.
For inference, our model achieves 10 image/s, which is again comparable with the existing models.

\subsection{Ablation study}
We perform the following ablation studies to further analyse the components of our model. We also implement our baseline model without the proposed strategy to highlight the contribution of the mutual information minimization regularizer. Note that, all of these
experiments are trained with the conventional training dataset.


\noindent\textbf{The performance of the baseline model:} To test how our designed encoder and decoder in Fig. \ref{fig:network_overview} performs, we remove the \enquote{Mutual information regularizer} part from our framework\footnote{Detailed structure is shown in the supplementary materials}, and directly concatenate the RGB feature  with depth feature  and feed it to the decoder. The performance is shown as \enquote{Base} in Table \ref{tab:ablation_study_experiments}. We observe comparable performance of \enquote{Base} compared with existing RGB-D saliency detection models. The inferior performance of \enquote{Base} compared with our final results explains superior performance of the proposed solution of using mutual information as regularizer for redundancy constraining.

\noindent\textbf{The dimension of the feature space:} We set the dimension of lower-dimensional feature embedding ( and ) as . To test the impact of feature dimensions on the network performance, we set  and , and report their performance as \enquote{K3} and \enquote{K32} respectively in Table \ref{tab:ablation_study_experiments}. The experimental results demonstrates that our model achieves relative stable performance with different dimensions of the lower-dimensional feature, while the current dimension with  works the best.




\noindent\textbf{The structure of the \enquote{Mutual information regularizer} module as shown in Fig.~\ref{fig:network_overview}:} As discussed in Section \ref{latent_feature_sec}, the \enquote{Mutual information regularizer} module is composed of one  convolutional layers
and one fully connected layer. One may also achieve this directly from the output of the saliency encoder. Specifically, we can feed the RGB feature and depth feature to two fully connected layers to obtain  and  respectively. 
In Table \ref{tab:ablation_study_experiments}, we report the performance of our model with this simple setting, marked as \enquote{SS}. 
We observe
performance decreases, which indicates the
necessity of introducing more nonlinearity to effectively extract the feature representation of each mode.




\noindent\textbf{The weight of the mutual information regularizer:} The weight  of the mutual information regularization term controls the level of complementary information. We set  in this paper to achieve balanced training. We then test how the model performs with different
, and set  and  respectively. We show the performance of those variants in Table \ref{tab:ablation_study_experiments}, denoted by \enquote{W0} and \enquote{W1}. The inferior performance of \enquote{W0} indicates the effectiveness of our complementary information modeling strategy. Further, compared with our performance in Table \ref{tab:BenchmarkResults}, we observe relatively worse performance of \enquote{W1}, which inspires us to further investigate finding an optimal weight for the mutual information regularizer.



 


\subsection{Discussion}
\label{model_discussion}
\noindent\textbf{The effectiveness of mutual information minimization as regularizer:}
We computed the mean absolute cosine similarities of the highest stage feature embeddings ( and ) for \enquote{W0} from Table~\ref{tab:ablation_study_experiments} (without mutual information minimization as regularizer) and ours,
which are  and 
on the NLPR testing dataset. This clearly shows the advantage of our solution in extracting a less correlated feature for each mode. We visualize the learned feature embedding in the supplementary materials.


\noindent\textbf{The merging strategy:}
In this paper, we produce four different saliency maps as intermediate outputs, including the saliency prediction from both the RGB branch () and depth branch (), the feature embedding fusion branch (), and our final prediction  by fusing ,  and . As  has already been
included in
the complementary information of  and , we define our final prediction as  without the final fusion to obtain .
The performance is shown in Table~\ref{tab:ablation_study_experiments} \enquote{}. We observe 
inferior performance of \enquote{} compared with our final prediction. The main reason is that  and  are high-level feature embeddings of the RGB image and depth data, which mainly capture the semantic information. The direct merging of  and  will generate saliency prediction with less structure accuracy.





\begin{table*}[t!]
  \centering
  \scriptsize
  \renewcommand{\arraystretch}{1.0}
  \renewcommand{\tabcolsep}{0.55mm}
  \caption{Performance of the weakly supervised saliency detection baselines.
  }
  \begin{tabular}{l|cccc|cccc|cccc|cccc|cccc|cccc}
  \hline
&\multicolumn{4}{c|}{NJU2K\cite{NJU2000}}&\multicolumn{4}{c|}{SSB\cite{niu2012leveraging}}&\multicolumn{4}{c|}{NLPR~\cite{peng2014rgbd}}&\multicolumn{4}{c|}{SIP~\cite{sip_dataset}}&\multicolumn{4}{c|}{\ourdataset-Normal}&\multicolumn{4}{c}{\ourdataset-Difficult} \\
    Method 
    &  &  &  & 
    &  &  &  & 
    &  &  &  & 
    &  &  &  & 
    &  &  &  & 
    &  &  &  &  \\
  \hline
Scribble & .823 & .806 & .869 & .080 & .820 & .803 & .884 & .073 & .820 & .737 & .863 & .058 & .815 & .793 & .888 & .076 & .802 & .780 & .856 & .082 & .767 & .749 & .812 & .115    \\
   Polygon & .847 & .827 & .896 & .065 & .853 & .831 & .913 & .056 & .848 & .789 & .899 & .043 & .846 & .822 & .909 & .060 & .827 & .805 & .884 & .065 & .786 & .774 & .841 & .096  \\
\hline
  \end{tabular}
  \label{tab:weakly_saliency_baseline}
\end{table*}

\noindent\textbf{The contribution of depth data:} Saliency detection can be achieved merely with the RGB image. As discussed in Section \ref{sec:intro},
depth introduces useful geometric information for saliency detection. To verify this conclusion, we train our model (including only the encoder and decoder in Fig.~\ref{fig:network_overview}) with and without depth as input\footnote{The model with depth is an early-fusion model, where depth and RGB image are concatenated at input layer}. The performance is shown in Table \ref{tab:ablation_study_experiments} as \enquote{} and \enquote{} respectively. The superior performance of \enquote{} compared with \enquote{} explains contribution of the depth data for saliency detection. We also show examples explaining how depth contributes to saliency detection in the supplementary materials.

\noindent\textbf{Depth generation:} We generate our \ourdataset~dataset
with Holo50K~\cite{hua2020holopix50k}, where the stereo pairs are not strictly rectified, which may cause severe matching failures even with state-of-the-art stereo algorithms~\cite{zhong2020nipsstereo}. To solve this issue, we relax the horizontal search in stereo algorithms to both horizontal and vertical search but only keep the horizontal displacements as the stereo disparities. We use a modified stereo matching algorithm~\cite{zhong2020nipsflow} to generate the disparities / depth in our dataset. Further, stereo cameras are widely used in mobile devices, which makes it easier to obtain depth information for both indoor and outdoor.







\begin{table}[t!]
  \centering
  \scriptsize
  \renewcommand{\arraystretch}{1.0}
  \renewcommand{\tabcolsep}{0.5mm}
  \caption{Performance on the test sets of our new \ourdataset.}
  \label{tab:BenchmarkResults_OurTestSet}
  \begin{tabular}{lr|cccccccc|c}
  \hline
    & Metric & UCNet  & JLDCF &  A2dele & DMRA & CPFP & S2MA & CoNet  & BBS-Net & \ourmodel\\    
   &  & \cite{jing2020uc}          
   & \cite{Fu2020JLDCF}      
   & \cite{A2dele_cvpr2020}
   & \cite{dmra_iccv19}
   & \cite{zhao2019Contrast}  
   & \cite{self_attention_rgbd}  
   & \cite{ji2020accurate} 
   & \cite{fan2020bbs} 
   & Ours\\
  \hline
  \multirow{4}{*}{\rotatebox{90}{\textit{Normal}}}
     &    & .894 &  .894 & .833 & .782 & .795 & .877 & .820 & .902  & \textbf{.915}  \\
     &   & .883 &  .875 & .835 & .744 & .716 & .829 & .796 & .879  & \textbf{.893}  \\
    &      & .929 &  .919 & .882 & .812 & .801 & .881 & .850 & .923  & \textbf{.941}  \\
    &    & .036 &  .042 & .060 & .105 & .104 & .059 & .082 & .039  & \textbf{.033}  \\ \hline
    \multirow{4}{*}{\rotatebox{90}{\textit{Difficult}}}
     &    & .822 &  .845 & .787 & .743 & .770 & .828 & .779 & .853  & \textbf{.867}  \\
     &   & .814 &  .832 & .795 & .724 & .704 & .789 & .774 & .834  & \textbf{.852}  \\
    &      & .859 &  .870 & .838 & .775 & .776 & .836 & .813 & .876  & \textbf{.893}  \\
    &    & .079 &  .075 & .092 & .137 & .131 & .092 & .113 & .071  & \textbf{.064}  \\ \hline
\end{tabular}
\end{table}



\subsection{New benchmarks on \ourdataset}
We provide a new benchmark of state-of-the-art models trained with our new training dataset, and show performance in Table \ref{tab:BenchmarkResults_OurTestSet}. Further, with our rich annotations
as shown in Fig.~\ref{fig:dataset_annotation_all}, we discuss another three benchmarks for fully/weakly-supervised learning.




\noindent\textbf{Benchmark \#1: Re-train existing RGB-D saliency models with our new training dataset.}
We split our testing dataset to a moderate-level testing set (\enquote{Normal}) and a hard testing set (\enquote{Difficult}) with 4,600 image pairs and 3,000 pairs respectively.
To test how existing RGB-D saliency detection models perform on our new testing sets, we re-train existing RGB-D saliency detection models with our new training set, and show their performance on the new testing sets in Table \ref{tab:BenchmarkResults_OurTestSet}. The performance gap between those existing techniques illustrates effectiveness of our dataset in both model learning and evaluation.

\noindent\textbf{Benchmark \#2: Stereo saliency detection.}
As our RGB-D saliency dataset is constructed on a stereo dataset \cite{hua2020holopix50k}, we directly train a stereo image pair based saliency object detection model, where the depth is implicitly instead of explicitly obtained from the stereo image pairs.
Although there exist
some stereoscopic saliency detection models \cite{Stereoscopic_Videos_sal,Stereoscopic_depth_confidence,niu2012leveraging,Saliency_stereo}, all of them take both the RGB image and depth as input.
In this paper, similar to \cite{Intrinsic_stereo_video}, we design a real\footnote{We define the stereoscopic saliency models taking only the left and right view images as input as the \enquote{real} stereoscopic saliency models.} stereoscopic saliency detection model, and
provide a baseline
to manifest the potential of our dataset for stereoscopic saliency detection.
We train our stereoscopic saliency detection model on our new training dataset, where the left-right view images are taken as inputs and the GT saliency maps in the right view images are used as supervision. The same encoder and decoder as in Fig.~\ref{fig:network_overview} are adopted for our stereoscopic saliency detection model.
Specifically, we implicitly model the geometric information by using a cost volume between the saliency encoder for the right and left view images. The performance is shown in~Table \ref{tab:stereo_saliency_baseline}. We explained the architecture and the other stereo saliency datasets in detail in the supplementary material.





\begin{table}[t!]
  \centering
  \scriptsize
  \renewcommand{\arraystretch}{1.0}
  \renewcommand{\tabcolsep}{0.5mm}
  \caption{Performance of the stereo saliency detection baseline.
  }
  \begin{tabular}{ccc|ccc|ccc|ccc}
  \hline
\multicolumn{3}{c|}{NJU2K\cite{NJU2000}}&\multicolumn{3}{c|}{NJU400\cite{NJU400}}&\multicolumn{3}{c|}{\ourdataset-Normal}&\multicolumn{3}{c}{\ourdataset-Difficult} \\
     &  & 
    &  &  & 
    &  &  & 
    &  &  &  \\
  \hline
    .874 & .851 & .056 & .882 & .851 & .044 & .874 & .855 & .047 & .825 & .812 & .080  \\
   \hline
  \end{tabular}
  \label{tab:stereo_saliency_baseline}
\end{table}







\noindent\textbf{Benchmark \#3 and \#4: Scribble/Polygon as supervision.}
For scribble supervision,
we follow \cite{jing2020weakly} and use the smoothness loss and an auxiliary edge detection branch as a constraint to maintain structure information in the prediction.
We train our scribble supervised RGB-D saliency detection model by concatenating RGB and depth in the input layer, and feed the concatenanted feature to one  convolutional layer to adapt the model in \cite{jing2020weakly}. Performance of the scribble annotation based baseline model is shown in Table \ref{tab:weakly_saliency_baseline} \enquote{Scribble}.
The polygon label is generated after
majority voting. Fig.~\ref{fig:dataset_annotation_all} (g) shows that the polygon label covers a larger area with better structure information than scribbles.
We
train directly with polygon annotations as pseudo labels by adopting our model in Fig. \ref{fig:network_overview}, and provide performance of this baseline model in Table \ref{tab:weakly_saliency_baseline} \enquote{Polygon}.

\noindent\textbf{Benchmark analysis:} Our RGB-D saliency benchmark in Table \ref{tab:BenchmarkResults_OurTestSet} shows the superior performance of our method. Furthermore, the gap between state-of-the-art models illustrates the effectiveness of our new testing dataset in model evaluation. Our stereoscopic saliency benchmark in Table \ref{tab:stereo_saliency_baseline} introduces another solution to implicitly use the geometric information. Our two weakly-supervised baselines in Table \ref{tab:weakly_saliency_baseline} provide new options for weakly-supervised RGB-D saliency detection.







































































\section{Conclusion}
We proposed a multi-stage cascaded learning based RGB-D saliency detection framework that explicitly models complementary information between RGB images and depth data. 
By minimizing the mutual information between these two modalities during training, our model focuses on the diverse parts of each mode rather than the redundant information. 
In this fashion, our model is able to exploit the multi-modal information more effectively.
Further, we introduced the largest RGB-D saliency detection dataset with five types of annotations to prosper the development of fully-/weakly-/un-supervised RGB-D saliency detection tasks.
Four new benchmarks on 7 datasets and our new dataset demonstrate 
the superiority of our model compared to the existing RGB-D saliency detection techniques.








{\small
\bibliographystyle{ieee_fullname}
\bibliography{egbib}
}

\end{document}
