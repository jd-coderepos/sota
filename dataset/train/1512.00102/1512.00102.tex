

In this section, we provide a detailed description of the \SIFlong
method.  We show this method can enable operational use of the data
while maintaining information-theoretic data security, a protection
well above commonly used symmetric encryption techniques.

\subsection{Using Secret Sharing to Store a Set}
\label{ssec:background:set}

We draw from previous work~\cite{Ganger2001,Subbiah2005,Storer2009,Kroeger2013}
that proposed using secret sharing to create a secure, distributed
archive.  While the archive is usable for any type of data representable as a set, 
we use IP addresses as our exemplar.

\begin{figure}[t]
\begin{center}
\begin{adjustbox}{max totalsize={0.95\columnwidth}{0.95\columnwidth},center}
\includegraphics[width=0.95\columnwidth]{distribution.pdf}
\end{adjustbox}
 \end{center}
\caption{An example of share distribution to securely store an entry in a distributed archive.}
\label{fig:createList}
\end{figure}

A trusted user creates a new entry after externally determining a specific
IP is untrusted.  To securely store this element, it is split into 
shares using a  linear secret sharing scheme~\cite{Shamir1979}.
As shown in \reffig{fig:createList}, once the shares are created, they
are then distributed across  separate repositories to create the
archive holding the list of known bad IP addresses.  For example,
if there are 20 bad addresses, each of the  repositories would hold
20 shares unique to that repository.  The aggregate of these repositories
is an archive securely
storing 20 addresses. Once the shares have been distributed, the
user deletes the original data.  After this, the list exists solely as an abstract
concept and no single repository holds data recognizable as an
IP address (or component thereof) in isolation.  It is this unique
property that separates such solutions from systems using encryption
for data security.

When one of the repositories is inevitably compromised~\cite{Zaichkowsky2013}, 
the attacker learns nothing beyond the size of the set of elements.  Even additional 
compromises, as long as they number less than , do not let the attacker 
gain any additional information about the list contents. Ideally, the 
consortium is able to refresh the shares with new polynomials using 
techniques such as those by Herzberg~\etal~\cite{Herzberg1995} before the 
intruders have enough shares to reconstruct the data.

\subsection{Terminology}\label{ssec:sif:terminology}
We define the following terms:

\begin{itemize}
\item {\bf Element}---One entry in the set of data. We wish to simultaneously
protect and use these elements.
\item {\bf Share}---A resulting datum from using secret sharing to encode an
  element into  pieces.
\item {\bf Threshold}--- The number of shares, denoted as , needed to recover the original
  element.
\item {\bf Repository}---A remote server storing a unique set of shares.
\item {\bf Archive}---The collection of all  repositories. Together they
  create a system for securely storing and operating on data elements.
\end{itemize}

We perform all secret sharing operations over a finite field .
The data is a set  and is of size .
We split each data element  using a polynomial,
, of order .  The polynomial takes the form

where  is a coefficient chosen uniformly at random for
. 

We denote the list of shares given to Repository  as
, which is the vector of all of the polynomials
evaluated at .  We use the vector notation to represent
interpolation performed on all shares concurrently.

\subsection{Serial Lagrangian Interpolation}\label{ssec:sif:sli}

In traditional secret sharing,
given a set of  points,  and their
corresponding shares ( for ), we can use
Lagrangian interpolation to reconstruct the generating polynomial,
 as


where


This allows for recovery of the original element by evaluating
.

The archive stores the elements as the -intercepts of polynomials,
.  Here, we use the vector notation to denote operation 
over all elements in the set. In our example we
select a finite field able to represent .

To avoid reconstructing the polynomials in a single location, we
perform Lagrangian interpolation serially across a subset of  out
of  repositories.  The list of  repositories to be used is
provided as a part of the \SIF query and known to all of the
repositories involved.  Furthermore, the data of each repository,
, is private and known only to the repository owning
that data. Therefore, while all of the repositories can compute any of
the , only Repository  has access to . Thus it is possible to calculate  across three
repositories () as follows:





Nevertheless this calculation would reassemble the original secret in
a single location. To perform a set membership test while not exposing the
polynomial or query term, we must perturb both the query and the
polynomial reconstruction with a vector of nonces, or randomly generated
constants. This process is described in 
detail in the next section.

\subsection{Generalized Algorithm}\label{ssec:sif:general}

\begin{figure}[t]
\begin{center}
\begin{adjustbox}{max totalsize={0.95\columnwidth}{0.95\columnwidth},center}
\includegraphics[width=0.95\columnwidth]{membership.pdf}
\end{adjustbox}
\end{center}
\caption{An example of query algorithm for testing set membership.}
\label{fig:query}
\end{figure}

Once the shares have been distributed, we can now use the \SIF to
test for set membership. We present
the query algorithm a user at a repository would perform to determine if
the value  is present in the archive.  An example query round and 
user response for a  secret sharing can be seen in 
\reffig{fig:query}.

\begin{enumerate}
\item\label{enum:assemb:one} A user at repository  initiates the
  query with a unique transaction identifier .    then
  selects  repositories including itself and creates an ordered list of these
  repositories .  It generates a nonce
  vector () containing a different random pad for each element
	and calculates  as
  follows: 

\item\label{enum:assemb:two}   sends a membership test
  message () to the next repository listed in .
	In \reffig{fig:query},  sends the message to .
	

\item\label{enum:assemb:three}   calculates a nonced query
  term  and sends a separate message, , to the  repository.
	As  in \reffig{fig:query}, this message is sent from  to .






\item\label{enum:assemb:five}   for  receives the message from  and
   serially calculates its portion of  the interpolation
   polynomials
   
   and sends  to repository .

\item\label{enum:assemb:six}  : Calculates the final
  contribution to the interpolations
  
  and compares each component in  to each corresponding
  component of the query terms () received in message  in step~\ref{enum:assemb:three}. 
	If a match is found, the query response is set to true and false otherwise.

\item\label{enum:assemb:seven}   sends the results of the query to the 
	request originator  (). This final response is 
	seen as the gray arrow returning from  to the user at  in \reffig{fig:query}.
\end{enumerate}

The user  can determine if the value 
is in the set while no other user learns the value of . Additionally,
none of the secrets () are ever reconstructed in a
single location. 
In short, the \SIF protocol has
enabled  to actively query the data while maintaining
information-theoretic levels of data
protection. In \refsec{sec:analysis}, we will discuss these
protections in detail and explain some limitations around adversarial
collusion.
