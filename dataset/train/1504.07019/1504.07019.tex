

\documentclass[11pt,fleqn]{article}
\usepackage{fullpage}
\usepackage{amsthm}

\usepackage{amsmath,amssymb}
\usepackage[boxed]{algorithm}
\usepackage[noend]{algorithmic}
\usepackage{bbm}
\usepackage{graphicx}
\usepackage{subfigure}
\usepackage{xspace}
\usepackage{color}
\usepackage{afterpage}
\usepackage{ifpdf}
\ifpdf    \usepackage{hyperref}
\else    \usepackage[hypertex]{hyperref}
\fi


\renewcommand{\algorithmicrequire}{\textbf{Input:}}
\renewcommand{\algorithmicensure}{\textbf{Output:}}
\renewcommand{\algorithmiccomment}[1]{// {\em #1}}

\newtheorem{theorem}{Theorem}[section]
\newtheorem{lemma}[theorem]{Lemma}
\newtheorem{fact}[theorem]{Fact}
\newtheorem{corollary}[theorem]{Corollary}
\newtheorem{definition}[theorem]{Definition}
\newtheorem{proposition}[theorem]{Proposition}
\newtheorem{observation}[theorem]{Observation}
\newtheorem{claim}[theorem]{Claim}
\newtheorem{assumption}[theorem]{Assumption}
\newtheorem{remark}[theorem]{Remark}
\newtheorem{notation}[theorem]{Notation}
\newcommand{\proofof}[1]{Proof of #1}
\renewcommand{\qedhere}{}

\newcommand {\ignore} [1] {}
\def \opt   {\mathsf{opt}}                             \def \iter {10 \log_b \calD}

\newcommand{\rnote}[1]{\textcolor{red}{\textsf{#1 --Robi}\marginpar{\tiny\bf RK}}}

\DeclareMathOperator{\supp}{supp}
\DeclareMathOperator{\diam}{diam}
\DeclareMathOperator*{\EX}{{\mathbb E}}
\DeclareMathOperator{\polylog}{polylog}
\newcommand{\R}{{\mathbb R}}
\newcommand{\etal}{{\em et al.\ }\xspace}
\newcommand{\minn}[1]{\min\{{#1}\}}
\newcommand{\maxx}[1]{\max\{{#1}\}}
\providecommand{\ceil}[1]{\lceil #1 \rceil}
\providecommand{\abs}[1]{\lvert#1\rvert}
\providecommand{\card}[1]{\lvert#1\rvert}
\providecommand{\norm}[1]{\lVert#1\rVert}
\providecommand{\aset}[1]{\{#1\}}
\providecommand{\eqdef}{:=}

\def\compactify{\itemsep=0pt \topsep=0pt \partopsep=0pt \parsep=0pt}

\newcommand{\calP}{{\mathcal P}}
\newcommand{\calD}{{\mathcal D}}
\newcommand{\far}{{\text{far}}}
\newcommand{\near}{{\text{near}}}
\newcommand{\fin}{{\text{fin}}}
\newcommand{\longSub}{{\text{long}}}
\DeclareMathOperator{\texp}{Texp}



\title{Metric Decompositions of Path-Separable Graphs\thanks{The final publication of the paper is available at Springer via http://dx.doi.org/10.1007/s00453-016-0213-0.}}
\author{Lior Kamma\thanks{Work supported in part by a US-Israel BSF grant \#2010418
and an Israel Science Foundation grant \#897/13.
Email: \texttt{\{lior.kamma,robert.krauthgamer\}@weizmann.ac.il}
}
\\ The Weizmann Institute
\and Robert Krauthgamer\footnotemark[\value{footnote}]
\\ The Weizmann Institute
}

\begin{document}

\maketitle

\begin{abstract}
A prominent tool in many problems involving metric spaces is 
a notion of randomized low-diameter decomposition.
Loosely speaking, -decomposition refers to a probability distribution 
over partitions of the metric into sets of low diameter, 
such that nearby points (parameterized by )
are likely to be ``clustered'' together.
Applying this notion to the shortest-path metric in edge-weighted graphs,
it is known that -vertex graphs admit 
an -padded decomposition \cite{Bartal96},
and that excluded-minor graphs admit -padded decomposition 
\cite{KPR93,FT03,AGGNT14}.

We design decompositions to the family of -path-separable graphs,
which was defined by Abraham and Gavoille~\cite{AG06}
and refers to graphs that admit vertex-separators consisting 
of at most  shortest paths in the graph.

Our main result is that every -path-separable -vertex graph 
admits an -decomposition,
which refines the  bound for general graphs, 
and provides new bounds for families like bounded-treewidth graphs.
Technically, our clustering process differs from previous ones 
by working in (the shortest-path metric of) carefully chosen subgraphs.
\end{abstract}


\section{Introduction} \label{sec:intro}
In recent decades, the problem of decomposing a metric space into low-diameter subspaces has become a key step in the solution for many problems, including metric embeddings (e.g. \cite{Assouad83,Bartal96,Rao99,GKL03,FRT04,KLMN05}), distance oracles and routing schemes design (e.g. \cite{AP90,DSB97,Talwar04,CGMZ05,MN07}), graph sparsification (e.g. \cite{EGKRTT2010,KKN14}) and optimization problems, such as multicommodity cuts \cite{KPR93,GVY96,LR99}, -extension \cite{CKR04} and the traveling salesman problem \cite{Talwar04}.

Following the recent literature, we focus on \emph{randomized} decompositions, which loosely speaking refers to a probability distribution over partitions 
of a metric space into sets (called clusters) of low diameter, 
such that nearby points are more likely to be ``clustered'' together.
The formal definitions follows.

Let  be a metric space,
and denote the ball of radius  around  by . 
Let  be a partition of . Every  is called a {\em cluster},
and for every , let  denote the unique cluster 
 such that .
We will be using the following definition of Abraham \etal \cite{AGGNT14}.

\begin{definition}\label{def:beta} 
A metric space  is called {\em -decomposable} for  if for every  there is a probability distribution  over partitions of , satisfying the following properties.
\begin{enumerate}
\renewcommand{\theenumi}{(\alph{enumi})}
	\item \label{it:DiameterBound}
	Diameter Bound: For every  and , .
	\item \label{it:PadProb}
	Padding: For every  and , 
\end{enumerate}
\end{definition}

A slightly different definition that is common in the literature 
under the name {\em padded decomposition}, see e.g.\ \cite{KLMN05,LN2004}, 
is the special case of setting in~\ref{it:PadProb} ,
i.e., requiring that for every , with probability at least  
the entire ball  is contained in a single cluster of .
Our results provide constructions that satisfy Definition~\ref{def:beta}, 
and thus immediately apply also to the more common definition.

The metric spaces we study arise as shortest-path metrics in (certain) graphs. 
Specifically, given  
a connected graph with non-negative edge weights, let  denote the shortest-path metric induced on  by . Denote by  the ball of radius  around  in the metric space .
We say that a graph  is -decomposable if the metric space  is -decomposable.

Bartal~\cite{Bartal96} proved that for every -point metric space, , and that this bound is tight for general metric spaces, thus motivating an extensive research on restricted families of metric spaces. 
Notable progress has been made for families defined by topological restrictions, such as shortest-path metrics in graphs excluding a fixed minor \cite{KPR93,FT03,AGGNT14} or bounded-genus graphs \cite{LS10,AGGNT14} and geometric restrictions, such as a bounded doubling dimension \cite{GKL03} or hyperbolic structure \cite{KL06}.

In this paper we consider metrics induced by graphs of bounded ``path separability'', which is a blend of topological and geometric restrictions,
as defined below.

\paragraph{Shortest-Path Separators.} 
Given a graph  and a set , an {\em -flap} is a connected component of . We say that  is a (balanced) {\em vertex separator} if every -flap  has size . Vertex separators are widely used in divide-and-conquer algorithms. Thorup \cite{T04} observed that every planar graph has a vertex separator composed of three shortest paths \footnote{If we relax the balance constraint, and require that every -flap  has size , then two shortest paths suffice.}, and used this property to design distance and reachability oracles for planar graphs.
Abraham and Gavoille \cite{AG06} extended this notion and defined {\em path separability}. Intuitively, a graph is path separable if it has a vertex separator composed of a few shortest-paths.
\begin{notation}
For sets  and , we denote . 
\end{notation}

\begin{definition}\label{def:pathDec}\cite{AG06}
A graph  is called {\em -path separable} for  if there exists  
such that the following holds.
\begin{enumerate}
	\item There exist  such that  and every  is the union of  shortest-paths in .
	\item .
	\item  is a vertex separator, and every -flap is -path-separable.
\end{enumerate}
\end{definition}
Abraham and Gavoille showed that for every graph  there is a number  such that every graph  that excludes  as a minor is -path-separable under every edge weights .
Diot and Gavoille \cite{DG10} proved that every graph of treewidth  is -path-separable with every edge weights .

\subsection{Main Results}
\begin{theorem} \label{th:sepToDec}
Every -path-separable graph  is -decomposable.
\end{theorem}
We further note that if, in addition, for every subgraph  of , we can find a -path-separator in polynomial time then we can efficiently sample from the distribution guaranteed in Theorem~\ref{th:sepToDec}.

Combining our theorem with the result of Diot and Gavoille \cite{DG10} we get an upper bound for bounded treewidth graphs.
\begin{corollary}\label{cor:twToDec}
Every graph  of treewidth  with every edge weights  is -decomposable.
\end{corollary}

Previously, no bound was known for -path-separable graphs other than  due to Bartal \cite{Bartal96}.
For graphs of treewidth  the known upper bound is  due to \cite{AGGNT14}. 
Our decomposition provides a tradeoff between  and  
and matches or improves all other bounds when . 
It is conjectured that , 
which would be tight due to the lower bound of Bartal~\cite{Bartal96},
and our result provide partial evidence in favor of this conjecture.

Many known results ``interface'' the metric only through decompositions, 
and thus plugging in our decomposition bounds immediately yields new results for the aforementioned families of metric spaces.
For example, using a result from \cite{KLMN05} we conclude that 
every -vertex graph of treewidth  (with every edge weights ) 
can be embedded in a Hilbert space with distortion , which improves over the known bound  whenever .

\subsection{Techniques}
\paragraph{Carving Random Balls.} A common approach for constructing a decomposition of a metric  is to choose a sequence of {\em centers}  and corresponding radii , where the choice of centers and/or radii may involve randomization, and then define

Clearly, each  has diameter at most .
This approach goes back to \cite{Bartal96}, and has seen many useful variations, for example, randomly ordering the centers \cite{CKR04} or reducing the number of centers \cite{CKR04,GKL03}.
As it turns out, it is enough to bound the number of centers {\em locally}. More formally, for every  we  control the number of centers  that threaten  in the sense that .

\paragraph{Carving Balls in Subgraphs.} Inspired by ideas from \cite{AGGNT14}, we introduce in this paper another modification to the approach described theretofore. In addition to the above, for every center  we choose a corresponding subgraph  of , such that , and define 

Choosing the subgraphs and centers carefully allows us to reduce the number of centers that threaten a vertex  in two ways. The first and more obvious manner is by making sure that  for only a few indices . The second aspect is a bit more subtle. Since distances are considered in subgraphs of , they might be larger than the corresponding distances in the original graph , as demonstrated in Figure~\ref{f:sub}, thus reducing the number of threateners of a vertex .

\begin{figure*}[t]
  \begin{center}
    \subfigure[{Original unit weighted graph . 
      When all outer-cycle vertices are centers and all ,
      vertex  is threatened by  centers. 
    }] 
{\label{f:sub-a}\includegraphics[scale=0.4]{graphG}} \hspace{3pc}
    \ 
    \subfigure[If balls are carved in this subgraph,
       is threatened by only  centers.]
    {\label{f:sub-b}\includegraphics[scale=0.4]{GraphGi}}
  \end{center}
  \caption{When carving balls in a subgraph of , the number of threateners to a vertex is smaller.}
  \label{f:sub}
\end{figure*}


Note that we need to ensure that  is indeed a partition of , i.e. that the balls  cover all of .

\subsection{Preliminaries}
\paragraph{The Truncated Exponential Distribution.} Define the {\em truncated exponential} with parameters  and , 
denoted ,
to be distribution given by the probability density function 

Note that this is the pdf of an exponential random variable with mean  that is conditioned to be in the range .
\paragraph{-Nets.} Given a metric space , an {\em -net} of  is a set  satisfying
\begin{enumerate}
	\item Packing: For all distinct  we have .
	\item Covering: For every , there is some  such that .
\end{enumerate}
\section{Decomposing Path-Separable Graphs}
In this section we prove Theorem~\ref{th:sepToDec}. 
We present a procedure which, given a -path-separable graph  and a parameter , produces a random partition  of . 
The algorithm works in two phases. The first phase, presented in detail as Algorithm~\ref{alg:pPathCen}, constructs a sequence of centers.
This is performed deterministically and by recursion. 
The algorithm finds a path-separator  of  and chooses a -net on each path of the separator to serve as centers. For every center vertex, the algorithm chooses a corresponding subgraph of . 
The algorithm is then invoked recursively on every -flap.
The second phase, presented in detail in Algorithm~\ref{alg:pPathRec}, samples random radii and carves balls around the centers to obtain a partition of . The radii are all sampled independently at random from a truncated exponential distribution.

\afterpage{
\begin{algorithm}[ht]
\begin{algorithmic}[1]
\REQUIRE A -path-separable graph  and a parameter .
\ENSURE A sequence .
\STATE let ,  be as in Definition~\ref{def:pathDec}.
\FOR{ to }
\FORALL{paths }
\STATE let  be a -net of  in an arbitrary order.
\STATE let  be the sequence .
\ENDFOR
\ENDFOR
\STATE let  be the concatenation of  in that order.
\FORALL{connected components  of } \label{l:rec}
\STATE invoke Choose-Centers and append the output sequence to .
\ENDFOR
\RETURN .
\caption{Choose-Centers}
\label{alg:pPathCen}
\end{algorithmic}
\end{algorithm}

\begin{algorithm}[ht]
\begin{algorithmic}[1]
\REQUIRE A -path-separable graph  and a parameter .
\ENSURE A partition  of .
\STATE let  be the sequence returned by Choose-Centers. 
\STATE let .
\STATE let .
\FORALL{}
\STATE choose  independently at random.
\STATE let .
\STATE let .
\ENDFOR
\RETURN .
\caption{Decomposing Path-Separable Graphs}
\label{alg:pPathRec}
\end{algorithmic}
\end{algorithm}
}
\vspace{1pc}
Denote by  the union of all -nets throughout the execution of Algorithm~\ref{alg:pPathCen}. Note that  is independent of the random radii (in fact, it is constructed deterministically). For sake of clarity, for a center  let  be the subgraph, radius and ball corresponding to  respectively.
To prove that Algorithm~\ref{alg:pPathRec} produces a partition of , consider . During the execution of Algorithm~\ref{alg:pPathCen} there exists a subgraph  of  such that Algorithm~\ref{alg:pPathCen} is (recursively) invoked on  and  is in the path-separator  of  chosen by the algorithm (in fact,  is unique). 
Let  be the path in  such that , let  be the closest net point to , and let  be the respective subgraph, then . By the definition of a net, , and therefore  (recall  is the radius chosen for ). Therefore , and Algorithm~\ref{alg:pPathRec} indeed outputs a partition of .

To prove the diameter requirement, let , and let . Then there exists  such that  and therefore .

Next, we prove the padding property of the decomposition. 
Let , and let . Denote .
We say that  is {\em settled by } if  is the first ball (in order of execution) to have non-empty intersection with . Therefore,  iff  is settled by  and  for some .
Let  be the set of centers that threaten . In order to bound the size of  we consider the execution of Algorithm~\ref{alg:pPathCen}.
Consider first a single recursion level. Denote the current graph by , and let  and  be as in Definition~\ref{def:pathDec}. Let  and let  be some path in .  Consider the -net  picked by the algorithm. Since  is a shortest-path in , 

Let  be such that  and . Since  is a shortest path in , we get that . Therefore 

Since in every recursive call of Algorithm~\ref{alg:pPathCen}, the number of vertices in the input graph is reduced by at least a factor of , the depth of the recursion is at most .
Every recursion level contains exactly one subgraph that contains . Since the number of paths in each such subgraph is at most , we conclude that . For simplicity, let us further assume this inequality holds with equality, and denote .

Let  be the elements of  in the order in which the algorithm considers them.
Denote by  the event that , and for every , denote by  the event that  was not settled before  was considered.
\begin{lemma} \label{l:ind}
 for all .
\end{lemma}

\begin{proof}
Consider some . Conditioned on , there are three possible outcomes to the th round. Either  was not settled also in the th round, and then (for ) the final status of  is left to the following rounds, or  was settled in the th round, and in that case, either , (and thus  occurred), or .
The "bad" event  can therefore only occur (either in the th round or some time in the future) if either  and  or if .

Denote , , then 

Since , then . Substituting  we get that 

Similarly we get that 

The proof proceeds by induction over .
As previously noted,  Plugging \eqref{eq:cutBall} we get that

where the last inequality follows from the fact that . 
Next, let . Then as previously explained,

By plugging \eqref{eq:cutBall} and \eqref{eq:notSettle} into \eqref{eq:recursion} and using the induction hypothesis, we get that


which proves Lemma~\ref{l:ind}.
\end{proof}
To complete the proof of Theorem~\ref{th:sepToDec}, note that since  and for ,

By setting  we get that , thus proving the last part of Theorem~\ref{th:sepToDec}.



\paragraph{Acknowledgements.}
The authors thank Ittai Abraham, Alex Andoni, and Kunal Talwar for useful discussions during preliminary stages of this work.

\newcommand{\etalchar}[1]{}
\begin{thebibliography}{AGG{\etalchar{+}}14}

\bibitem[AG06]{AG06}
I.~Abraham and C.~Gavoille.
\newblock Object location using path separators.
\newblock In {\em Proceedings of the Twenty-fifth Annual ACM Symposium on
  Principles of Distributed Computing}, PODC '06, pages 188--197, 2006.

\bibitem[AGG{\etalchar{+}}14]{AGGNT14}
I.~Abraham, C.~Gavoille, A.~Gupta, O.~Neiman, and K.~Talwar.
\newblock Cops, robbers, and threatening skeletons: Padded decomposition for
  minor-free graphs.
\newblock In {\em Proceedings of the 46th Annual ACM Symposium on Theory of
  Computing}, STOC '14, pages 79--88, New York, NY, USA, 2014. ACM.

\bibitem[AP90]{AP90}
B.~Awerbuch and D.~Peleg.
\newblock Sparse partitions.
\newblock In {\em 31st Annual IEEE Symposium on Foundations of Computer
  Science}, pages 503--513, 1990.

\bibitem[Ass83]{Assouad83}
P.~Assouad.
\newblock Plongements lipschitziens dans {}.
\newblock {\em Bull. Soc. Math. France}, 111(4):429--448, 1983.

\bibitem[Bar96]{Bartal96}
Y.~Bartal.
\newblock Probabilistic approximation of metric spaces and its algorithmic
  applications.
\newblock In {\em 37th Annual Symposium on Foundations of Computer Science},
  pages 184--193. IEEE, 1996.

\bibitem[CGMZ05]{CGMZ05}
H.~T.-H. Chan, A.~Gupta, B.~M. Maggs, and S.~Zhou.
\newblock On hierarchical routing in doubling metrics.
\newblock In {\em 16th Annual ACM-SIAM Symposium on Discrete Algorithms}, pages
  762--771. SIAM, 2005.

\bibitem[CKR04]{CKR04}
G.~Calinescu, H.~J. Karloff, and Y.~Rabani.
\newblock Approximation algorithms for the 0-extension problem.
\newblock {\em SIAM J. Comput.}, 34(2):358--372, 2004.

\bibitem[DG10]{DG10}
E.~Diot and C.~Gavoille.
\newblock Path separability of graphs.
\newblock In {\em Frontiers in Algorithmics, 4th International Workshop,
  {FAW}}, pages 262--273, 2010.

\bibitem[DSB97]{DSB97}
B.~Das, R.~Sivakumar, and V.~Bharghavan.
\newblock Routing in ad hoc networks using a spine.
\newblock In {\em Sixth International Conference on Computer Communications and
  Networks.}, pages 1--20, 1997.

\bibitem[EGK{\etalchar{+}}14]{EGKRTT2010}
M.~Englert, A.~Gupta, R.~Krauthgamer, H.~Rהcke, I.~Talgam-Cohen, and
  K.~Talwar.
\newblock Vertex sparsifiers: New results from old techniques.
\newblock {\em SIAM Journal on Computing}, 43(4):1239--1262, 2014.

\bibitem[FRT04]{FRT04}
J.~Fakcharoenphol, S.~Rao, and K.~Talwar.
\newblock A tight bound on approximating arbitrary metrics by tree metrics.
\newblock {\em J. Comput. Syst. Sci.}, 69(3):485--497, 2004.

\bibitem[FT03]{FT03}
J.~Fakcharoenphol and K.~Talwar.
\newblock Improved decompositions of graphs with forbidden minors.
\newblock In {\em 6th International workshop on Approximation algorithms for
  combinatorial optimization}, pages 36--46, 2003.

\bibitem[GKL03]{GKL03}
A.~Gupta, R.~Krauthgamer, and J.~R. Lee.
\newblock Bounded geometries, fractals, and low-distortion embeddings.
\newblock In {\em 44th Annual IEEE Symposium on Foundations of Computer
  Science}, pages 534--543, October 2003.

\bibitem[GVY96]{GVY96}
N.~Garg, V.~V. Vazirani, and M.~Yannakakis.
\newblock Approximate max-flow min-(multi)cut theorems and their applications.
\newblock {\em SIAM Journal on Computing}, 25(2):235--251, 1996.

\bibitem[KKN14]{KKN14}
L.~Kamma, R.~Krauthgamer, and H.~L. Nguyen.
\newblock Cutting corners cheaply, or how to remove steiner points.
\newblock In {\em Proceedings of the Twenty-Fifth Annual ACM-SIAM Symposium on
  Discrete Algorithms}, pages 1029--1040, 2014.

\bibitem[KL06]{KL06}
R.~Krauthgamer and J.~R. Lee.
\newblock Algorithms on negatively curved spaces.
\newblock In {\em 47th Annual IEEE Symposium on Foundations of Computer
  Science}, pages 119--132. IEEE, 2006.

\bibitem[KLMN05]{KLMN05}
R.~Krauthgamer, J.~R. Lee, M.~Mendel, and A.~Naor.
\newblock Measured descent: {A} new embedding method for finite metrics.
\newblock {\em Geometric And Functional Analysis}, 15(4):839--858, 2005.

\bibitem[KPR93]{KPR93}
P.~Klein, S.~A. Plotkin, and S.~Rao.
\newblock Excluded minors, network decomposition, and multicommodity flow.
\newblock In {\em 25th Annual ACM Symposium on Theory of Computing}, pages
  682--690, May 1993.

\bibitem[LN04]{LN2004}
J.~R. Lee and A.~Naor.
\newblock Metric decomposition, smooth measures, and clustering, 2004.
\newblock Unpublished manuscript.
\newblock Available from: \url{http://www.cims.nyu.edu/~naor/homepage
  files/cluster.pdf}.

\bibitem[LR99]{LR99}
T.~Leighton and S.~Rao.
\newblock Multicommodity max-flow min-cut theorems and their use in designing
  approximation algorithms.
\newblock {\em J. ACM}, 46(6):787--832, 1999.

\bibitem[LS10]{LS10}
J.~R. Lee and A.~Sidiropoulos.
\newblock Genus and the geometry of the cut graph.
\newblock In {\em Proceedings of the Twenty-first Annual ACM-SIAM Symposium on
  Discrete Algorithms}, SODA '10, pages 193--201. Society for Industrial and
  Applied Mathematics, 2010.

\bibitem[MN07]{MN07}
M.~Mendel and A.~Naor.
\newblock Ramsey partitions and proximity data structures.
\newblock {\em J. Eur. Math. Soc.}, 9(2):253--275, 2007.

\bibitem[Rao99]{Rao99}
S.~Rao.
\newblock Small distortion and volume preserving embeddings for planar and
  {E}uclidean metrics.
\newblock In {\em Proceedings of the 15th Annual Symposium on Computational
  Geometry}, pages 300--306. ACM, 1999.

\bibitem[Tal04]{Talwar04}
K.~Talwar.
\newblock Bypassing the embedding: {A}lgorithms for low dimensional metrics.
\newblock In {\em Proceedings of the 36th Annual ACM Symposium on Theory of
  Computing}, pages 281--290, 2004.

\bibitem[Tho04]{T04}
M.~Thorup.
\newblock Compact oracles for reachability and approximate distances in planar
  digraphs.
\newblock {\em J. ACM}, 51(6), 2004.

\end{thebibliography}

\end{document}