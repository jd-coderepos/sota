\documentclass[a4paper,11pt]{article}
\usepackage{wrapfig}
\usepackage[dvips]{graphics}
\usepackage{latexsym}
\usepackage{graphicx}
\usepackage{epsfig}
\usepackage[linesnumbered,ruled,vlined]{algorithm2e}


\usepackage{algorithmic}

\renewcommand{\baselinestretch}{1.1}
\usepackage{amssymb}
\usepackage{amsfonts}
\usepackage{color}
\newcommand{\comment}[1]{{\red{({\em #1})}}}

\setlength{\parskip}{2mm}
\setlength{\parindent}{0pt}

\setlength{\textwidth}{16.5cm}
\setlength{\textheight}{23.cm}
\setlength{\topmargin}{-1cm}
\setlength{\oddsidemargin}{1pt}
\setlength{\parskip}{1.75mm}
\setlength{\parindent}{0pt}
\newcommand{\remove}[1]{}

\renewcommand{\baselinestretch}{1.1}
\newcommand{\IR}{\mathbb{R}}

\newtheorem{theorem}{Theorem}
\newtheorem{lemma}{Lemma}
\newtheorem{observation}{Observation}
\newtheorem{result}{Result}
\newtheorem{corollary}{Corollary}
\newtheorem{definition}{Definition}
\newtheorem{remark}{Remark}
\newenvironment{proof}{\noindent {\bf Proof:\,\ }}{\hfill\mbox{\
}\smallskip}
\title{Minimum Dominating Set for a Point Set in }
\author{Ramesh K. Jallu\thanks{Indian Institute of Technology Guwahati,
India}
\and Prajwal R. Prasad \thanks{National Institute of Technology Karnataka, India}
\and Gautam K. Das \footnotemark[1]}
\date{}
\begin{document}
\maketitle
\begin{abstract}
In this article, we consider the problem of computing minimum dominating set for a given set  
of  points in . Here the objective is to find a minimum cardinality 
subset  of  such that the union of the unit radius disks centered at the points in  covers 
all the points in . We first propose a simple 4-factor and 3-factor approximation algorithms in  
and  time respectively improving time complexities by a factor of  and 
 respectively over the best known result available in the literature 
[M. De, G.K. Das, P. Carmi and S.C. Nandy, {\it Approximation 
algorithms for a variant of discrete piercing set problem for unit disk}, Int. J. of Comp. Geom. and Appl., 
to appear]. Finally, we propose a very important shifting lemma, which is of independent 
interest and using this lemma we propose a -factor approximation algorithm and a PTAS for 
the minimum dominating set problem. 
\end{abstract}

{\bf Keywords:} minimum dominating set, unit disk graph, approximation algorithm.


\section{Introduction}
A minimum dominating set  for a set  of  points in  is defined as follows: (i)  
(ii) each point  is covered by at least one unit radius disk centered at a point in , and (iii) size of 
 is minimum. The {\it minimum dominating set} (MDS) problem for a point set  of size  in  
involves finding a minimum dominating set  for the set . We call this problem as a geometric version of 
MDS problem. The MDS problem for a point set can be modeled  
as an MDS problem in unit disk graph (UDG) as follows: A unit disk graph  for a set  of  
unit diameter disks in  is the intersection graph of the family of disks in  i.e., the vertex 
set  corresponds to the set  and two vertices are connected by an edge if the corresponding disks have 
common intersection. The minimum dominating set for the graph  is a minimum size subset  of  such 
that for each of the vertex  is either in  or adjacent to to a node in  in . Several people have done 
research on MDS problem because of its wide applications such as wireless networking,  
facility location problem, to name a few. Our interest in this problem arose from the following reason: 
suppose in a city we have a set  of  important locations (houses, etc.); the objective is to provide 
some emergency services (ambulance, fire station, etc.) to each of the locations in  so that 
each location is within a predefined distance of at least one service center. Note that positions of the 
emergency service centers are from the predefined set  of locations only. 


\subsection{Related Work}
The MDS problem can be viewed as a general set cover problem, but it is an NP-hard problem \cite{GJ79,J82} 
and not approximable within  for some constant  unless P = NP \cite{RS97}. Therefore 
-factor approximation algorithm is possible for MDS problem by applying the algorithm for 
general set cover problem \cite{chvatal79}. Some exciting results for the geometric version of MDS problem 
are available in the literature. 

In the {\it discrete unit disk cover} (DUDC) problem, two sets  and  of points in  are given, the 
objective is to choose minimum number of unit disks  centered at the points in  such that 
the union of the disks in  covers all the points in . Johnson \cite{J82} proved that 
the DUDC problem is NP-hard. Mustafa and Ray in 2010 \cite{MR10} proposed a -approximation 
algorithm for  (PTAS) for the DUDC problem using -net based local improvement approach. 
The fastest algorithm is obtained by setting  for a 3-factor approximation algorithm, which runs 
in ) time, where  and  are number of unit radius disks and number of points respectively \cite{DFLN12}. 
The high complexity of the PTAS leads to further research on constant factor 
approximation algorithms for the DUDC problem. A series of constant factor approximation algorithms for DUDC 
problem are available in the literature: 

\begin{itemize}
\item 108-approximation algorithm [C\u{a}linescu et al., 2004 \cite{CMWZ04}]
\item 72-approximation algorithm [Narayanappa and Voytechovsky, 2006 \cite{NV06}]
\item 38-approximation algorithm in O() time [Carmi et al., 2007 \cite{CKL07}]
\item 22-approximation algorithm in O() time [Claude et al., 2010 \cite{CDDDFLNS10}]
\item 18-approximation algorithm in O() time [Das et al., 2012 \cite{DFLN12}]
\item 15-approximation algorithm in O() time [Fraser and L\'{o}pez-Ortiz, 2012 \cite{FL12}]
\item -approximation algorithm in  time 
[Acharyya et al., 2013 \cite{ABD13}]
\end{itemize}

The DUDC problem is a geometric version of MDS problem for . Therefore all results for the DUDC problem 
are applicable to MDS problem.

The geometric version of MDS problem is known to be NP-hard \cite{CCJ90}. 
Nieberg and Hurink \cite{NH06} proposed -factor approximation algorithm 
for . The fastest algorithm is obtained by setting  for a
-approximation result, which runs in  time \cite{DDCN13}, which is not 
practical even for . Another PTAS for dominating set of arbitrary size disk graph is available 
in the literature proposed by Gibson and Pirwani \cite{GP10}. The running time of this PTAS is 
. 

Marathe et al. \cite{MBIRR95} proposed a 5-factor approximation algorithm for the MDS problem. 
Amb{\"u}hl et al. \cite{AEMN06} proposed 72-factor approximation algorithm for weighted dominating 
set (WDS) problem. In the WDS problem, each node has a positive weight and the objective is to find 
the minimum weight dominating set of the nodes in the graph. Huang et al. \cite{HGZW08}, Dai and Yu 
\cite{DY09}, and  Zou et al. \cite{ZWXLDWW11} improved the approximation factor for WDS problem to 
, , and  respectively. First, they proposed -factor 
( in \cite{HGZW08}, \cite{DY09}, and \cite{ZWXLDWW11} respectively) approximation 
algorithm for a subproblem and using the result of their corresponding sub-problems they proposed 
-factor approximation algorithms. The time complexity of their algorithms 
are , where  is the time complexity of the algorithm 
for the sub-problem and  is the number 
of times the sub-problem needs to be invoked to solve the original problem. The -factor 
approximation algorithm can be obtained by setting , but the time complexity becomes a
very high degree polynomial function in . Carmi et al. \cite{CKL08} proposed a 5-factor approximation 
algorithm of the MDS problem for arbitrary size disk graph. Fonseca et al. \cite{FFSM12} proposed a 
-factor approximation algorithm for the MDS problem in UDG which can be achieved in  
time, when the input is a graph with  vertices and  edges, and in  time, in the geometric 
version of the problem. The same set of authors also proposed a -factor approximation algorithm 
for the MDS problem in UDG which runs in  time \cite{FFSM12-1}. Recently, De at al. \cite{DDCN13} 
considered the geometric version of MDS problem and proposed 12-factor, 4-factor, and 3-factor approximation 
algorithms with running time , , and  respectively. They also 
proposed a PTAS with high degree polynomial running time. 

\subsection{Our Contribution}
In this paper, we consider the geometric version of MDS problem and propose a series of constant factor 
approximation algorithms. We first propose 4-factor and 3-factor approximation algorithms with running time 
 and  respectively improving the time complexities by a factor of  and 
 respectively over the best known result in the literature \cite{DDCN13}. Finally, we propose a new 
shifting strategy lemma. Using our shifting strategy lemma we propose -factor and 
-factor (i.e., PTAS) approximation algorithms for the MDS problem. The running time of 
proposed -factor and -factor approximation algorithms are  
and  respectively. Though the time complexity of the proposed PTAS is same as the PTAS proposed by 
De et al. \cite{DDCN13} in terms of  notation, but the constant involved in our PTAS is smaller than the 
same in \cite{DDCN13}.

\section{4-Factor Approximation Algorithm for the MDS Problem}\label{4factor}
In this section, a set  of  points in  is given inside a rectangular region . The 
objective is to find an MDS for . Here we propose a simple 4-factor approximation algorithm. The running time 
of our algorithm is , which is an improvement by a factor of  over the best known existing 
result \cite{DDCN13}. In order to obtain a 4-factor approximation algorithm, we consider a partition of 
 into regular hexagons of side length  (see Figure \ref{figure-2}(a)). We use {\it cell} 
to denote a regular hexagon of side length .

\begin{lemma} \label{lemma-1x}
 All points inside a single cell can be covered by an unit radius disk centered at any point inside that cell.
\end{lemma}

\begin{proof}
The lemma follows from the fact that the distance between any two points inside a regular hexagon of side length 
  is at most 1 (for demonstration see the Figure \ref{figure-2}(b)).
\end{proof}

\begin{figure}[!ht]
\begin{center} 
\includegraphics[]{fig1.eps}\\
\caption{(a) Regular hexagonal partition (b) single regular hexagon of side length  contained 
in an unit radius disk, and (c) a septa-hexagon}
\label{figure-2}
\end{center}
\vspace{-0.2in}
\end{figure}

\begin{definition}
A {\it septa-hexagon} is a combination of 7 adjacent cells such that one cell is inscribed by six other 
cells as shown in Figure \ref{figure-2}(c). 

For a point set , we use  to denote the set of unit radius disks centered at the points in .

Let  and  be two point sets such that . We use  to denote the set 
of points such that  and an unit radius disk centered at any point in 
 covers at least one point of .
\end{definition}


\subsection{Algorithm overview}
Let us consider a septa-hexagon . Recall that  is a combination of 7 cells (regular 
hexagon of side length ). Let  and . For the 
4-factor approximation algorithm, we first find minimum size subset  such that 
. Call this problem as {\it single septa-hexagon MDS} problem. 
Using the optimum (minimum size) solution of single septa-hexagon MDS problem, we present our main 4-factor 
approximation algorithm. The Lemma \ref{lemma-2x} gives an important feature to design optimum algorithm for
single septa-hexagon MDS problem. 


\begin{lemma} \label{lemma-2x}
 If is a minimum cardinality subset of  such that 
, then .
\end{lemma}

\begin{proof}
The septa-hexagon  has at most 7 non-empty cells. From Lemma \ref{lemma-1x}, we know that 
an unit radius disk centered at a point in a cell covers all points in that cell. Therefore one point 
from each of the non-empty cells is sufficient to cover all the points in . Thus the 
Lemma follows. 
\end{proof}


\begin{algorithm}[!ht]
\caption{Algorithm\_4\_Factor()}
\begin{algorithmic}[1]
\STATE {\bf Input:} A set  of  points and a septa-hexagon 

\STATE {\bf Output:} A set  such that . 

\STATE 
\IF{()}
  \STATE Choose one arbitrary point from each non-empty cell of  and add to .
  \STATE  /*  is at most 7 */
  \STATE  Let  and . 
  \FOR{()}
    \IF{()}
      \FOR {(Each possible combination of 5 points  of )}
	  \STATE Find  such that no point in  is covered by .
         
          \STATE Compute the farthest point Voronoi diagram of  \cite{BCKO08}
          \STATE Find a point  (if any) from  (using planar point location algorithm \cite{PS09}) 
	    such that the farthest point in  from  is less than or equal to 1. If such  exists, then 
	    set  and exit {\bf for} loop.
      \ENDFOR
    \ELSE
      \FOR {(Each possible combination of  points  of )}
	\IF{()}
	  \STATE Set  and exit from {\bf for} loop 
	\ENDIF
      \ENDFOR
    \ENDIF
  \ENDFOR
\ENDIF
\STATE Return 
\end{algorithmic}
\label{algo-4factor}
\end{algorithm}


\begin{lemma} \label{lemma-3x}
For a given set  of  points and a septa-hexagon , the Algorithm \ref{algo-4factor} computes an 
MDS for  using the points of  in  time.
\end{lemma}

\begin{proof}
 The optimality of the Algorithm \ref{algo-4factor} follows from the fact that Algorithm \ref{algo-4factor} 
 considers all possible set of sizes  (see Lemma \ref{lemma-2x}) as its solution and 
 reports minimum size solution.
 
 The line number 7 of the algorithm can be computed in  time as follows: (i) computation of the set 
  takes  time, (ii) computation of  can be done in  time using nearest 
 point Voronoi diagram of  in  time and for each point  apply planar point 
 location algorithm to find the nearest point in  in  time.
 
 The running time of the {\bf else} part in the line number 15 of the algorithm is at most  time. 
 The worst case running time of the algorithm comes from line numbers 9-14. The complexity of line numbers 11-13 
 is  time. Therefore the running time of the line numbers 9-14 is  time. 
 Thus the overall worst case running time of the proposed Algorithm \ref{algo-4factor} is .
\end{proof}


Let us consider a septa-hexagonal partition of  such that no point of  is on the boundary of 
any septa-hexagon and a 4 coloring scheme of it (see Figure \ref{fig:fig10}). Consider an unicolor 
septa-hexagon of color A (say). Its adjacent septa-hexagons are assigned colors B, C and D (say) such 
that opposite septa-hexagons are assigned the same color (see Figure \ref{fig:fig10}).  

\begin{figure}[ht]
\begin{center}
\includegraphics[height=3in]{figure-y.eps}
\caption{A septa-hexagonal partition and 4-coloring scheme}
\label{fig:fig10}
\end{center}
\end{figure}

\vspace{-0.1in}
\begin{lemma} \label{lemma-4x}
 If  and  are two same colored septa-hexagons, then 
  for any unit radius disk .
\end{lemma}

\begin{proof}
According to the 4-coloring scheme, size of the septa-hexagons, and no point of  is on the boundary 
of  and  the minimum distance between two points  
and ) is greater than 2 (see Figure \ref{fig:fig10}). Thus the lemma follows.
\end{proof}

\begin{theorem} \label{theorem-1y}
The 4-coloring scheme gives a 4-factor approximation algorithm for the MDS problem in  time, 
where  is the input size.
\end{theorem}

\begin{proof}
Let , and  be the sets of septa-hexagons of colors , and  respectively. 
Let  and  
for . By Lemma 
\ref{lemma-4x}, the pair () can be partitioned into  pairs () 
such that for each pair Algorithm \ref{algo-4factor} is applicable for solving the covering problem optimally 
to cover  using , where . Let  be the optimum solution for the set  
 () using the Algorithm \ref{algo-4factor}. If  is the optimum solution for the 
set , then . Therefore . Thus the 
approximation factor of the algorithm follows.

The time complexity result of the theorem follows from Lemma \ref{lemma-3x} and the fact that each point in 
 can participate in the Algorithm \ref{algo-4factor} at most constant number of times. 
\end{proof}

\section{3-Factor Approximation Algorithm for the MDS Problem}\label{3factor}
Given a set  of  points in a rectangular region , we wish to find an MDS for . 
Here we present a 3-factor approximation algorithm in  time for the MDS 
problem, which is an improvement by a factor of  over the best known result available in 
the literature \cite{DDCN13}. 

\begin{definition}
A {\it super-cell} is a combination of 15 regular hexagons of side length  arranged 
in three consecutive rows as shown in Figure \ref{fig:fig14}. 
\end{definition}

\begin{figure}[ht]
\begin{center}
\includegraphics[]{fig14.eps}
\caption{An example of a super-cell}
\label{fig:fig14}
\end{center}
\end{figure}

\subsection{Algorithm overview}
Let us consider a {\it super-cell} . Let  and . 
In order to obtain 3-factor approximation algorithm for the MDS problem, we first find a minimum 
size subset  such that . Call this problem 
as a {\it single super-cell MDS} problem. Using the optimum solution of single super-cell MDS problem, we 
present our main 3-factor approximation algorithm. 

\begin{lemma} \label{lemma-5x}
If  is the minimum cardinality subset of  such that 
, then .
\end{lemma}

\begin{proof}
The lemma follows from the Lemma \ref{lemma-1x} and the fact that the super-cell  
has at most 15 non-empty cells.
\end{proof}

We decompose a super-cell  into 3 regions namely , and  
(see Figure \ref{fig:fig16},  where , and  correspond to unshaded, 
light shaded, and dark shaded regions respectively). 

\begin{figure}[ht]
\begin{center}
\includegraphics[]{fig16.eps}
\caption{Decomposition of a super-cell}
\label{fig:fig16}
\end{center}
\end{figure}

\begin{lemma} \label{lemma-6x}
For any unit radius disk  and a super-cell , . 
\end{lemma}

\begin{proof}
The lemma follows from the fact that if  and  are two arbitrary points of  and  
respectively, then the Euclidean distance between  and  is greater than 2. 
\end{proof}

Let  and , where  is a super-cell. Our objective 
is to find a minimum cardinality set  such that .

Let , , and . 
A point on a boundary can be assigned to any set associated with that boundary. Let , 
, and . The Lemma \ref{lemma-6x} says that 
. 


\begin{algorithm}
\caption{Algorithm\_3\_Factor()}
\begin{algorithmic}[1]
\STATE {\bf Input:} A set  of  points and a super-cell 

\STATE {\bf Output:} A set  such that 

\STATE .
\STATE Find the sets ,and  as defined above.

\FOR {(Each possible combination  of  points in )}
  \IF {()}
	\STATE Let  and  be the subsets of  and  respectively such that no point in 
	 is covered by .
	
	\STATE Let  be the minimum size subset of  such that .
	  
	\STATE Let  be the minimum size subset of  such that .
	
	\IF{()}
	    \STATE Set 
	\ENDIF
  \ENDIF
\ENDFOR
\STATE Return 
\end{algorithmic}
\label{algo-3factor}
\end{algorithm}


\begin{lemma} \label{lemma-7x}
For a given set  of  points and a super-cell , the Algorithm \ref{algo-3factor} computes an 
MDS for  using the points of  in  time.
\end{lemma}

\begin{proof}
 In the case of selecting 3 points in  in line number 8 of the algorithm, we can choose one point from 
 each of the non-empty cells  of . Therefore, the worst case of line number 8 appears for the 
 case of choosing all possible combinations of two points in . This can be done in  using 
 the technique of the Algorithm \ref{algo-4factor} (line numbers 12-13). Similar analysis is applicable to line 
 number 9. Line numbers 6-7 and 10-12 can be implemented in  time.
 
 The worst case running time of the algorithm depends on the {\bf for} loop in the line number 5. In this 
 {\bf for} loop, we are choosing all possible 9 points from a set of  points in worst case. Therefore the
 time complexity of the Algorithm \ref{algo-3factor} is .  
 
 The optimality of the algorithm follows from the Lemma \ref{lemma-6x} and fact that Algorithm \ref{algo-3factor} 
 considers all possible combinations as its solution and reports minimum size solution.
 
 Note that Algorithm \ref{algo-3factor} checks {\bf if} condition in line number 6 because of the definition 
 of , and . 
\end{proof}

Let us consider a super-cell partition of  such that no point of  lies on the boundary and a 
3-coloring scheme (see Figure \ref{fig:fig15}). Consider an unicolor super-cell which has been assigned 
color A (say). Its adjacent super-cells are assigned colors B, and C alternately (see Figure~\ref{fig:fig15}). 

\begin{figure}[ht]
\begin{center}
\includegraphics[height=4cm,width=9cm]{figure-x.eps}
\caption{A super-cell partition and 3-coloring scheme}
\label{fig:fig15}
\end{center}
\end{figure}

\begin{lemma} \label{lemma-8x}
 If  and  are two same colored super-cells, then 
  for any unit radius disk .
\end{lemma}

\begin{proof}
The lemma follows from the following facts: (i) size of the super-cells  and  
(ii) no point of  on the boundary of  and , and (iii) the 3-coloring scheme.  
\end{proof}

\begin{theorem} \label{theorem-2y}
 The 3-coloring scheme gives a 3-factor approximation algorithm for the MDS problem in 
  time, where  is the input size.
\end{theorem}

\begin{proof}
The follows by the similar argument of Theorem \ref{theorem-1y}.   
\end{proof}

\section{Shifting Strategy and its Application to the MDS Problem} \label{ShiftingStrategy}
In this section, we first propose a shifting strategy for the MDS problem, which is 
a generalization of the shifting strategy proposed by Hochbaum and Maass \cite{HM85}. Next we propose 
-factor approximation algorithm and a PTAS algorithm for MDS problem using our shifting strategy.  

\subsection{The Shifting Strategy} \label{shifting-strategy}
Our shifting strategy is very similar to the shifting strategy in \cite{HM85}. We include a brief 
discussion here for completeness. Let a set  of  points be distributed inside an axis aligned 
rectangular region . Our objective is to find an MDS for . 

\begin{definition}
A {\it monotone chain}  with respect to line  is a chain of line segments 
such that any line perpendicular to  intersect it only once. We define the distance between two 
monotone chains  and  as the minimum Euclidean distance between any two points  and  
on the chains  and  respectively. A {\it monotone strip} denoted by  and is defined by the 
area bounded by any two monotone chains  and  such that the area is left closed and 
right open.
\end{definition}

Consider a set  of  monotone chains with respect to the line parallel to -axis 
from left to right dividing the region  such that distance between each pair of monotone chains is 
at least , where  and  are the left and right boundary of  respectively 
(see Figure \ref{shifting}). Let  be an -factor approximation algorithm, which provides a 
solution of any  consecutive monotone strips for the MDS problem. 

\begin{figure*}[!ht]
\begin{center}
\includegraphics[height=2.2in]{shifting.eps}
\end{center} 
\caption{Demonstration of shifting strategy}
\label{shifting}
\end{figure*}

\begin{theorem} \label{shift-factor}
We can design an -factor approximation algorithm for finding an MDS for .
\end{theorem}

\begin{proof}
The algorithm is exactly same as the algorithm proposed by Hochbaum and Maass \cite{HM85}. The approximation factor 
follows from exactly the same argument proved in the shifting lemma \cite{HM85}. 
\end{proof}


\subsection{-Factor Approximation Algorithm for the MDS Problem}
Here we propose a -factor approximation algorithm for MDS problem for a given 
set  of  points in  using shifting strategy discussed in Subsection \ref{shifting-strategy}. 

\begin{definition}
 A {\it duper-cell} is a combination of 30 cells (regular hexagon of side length ) as shown in 
 Figure \ref{figure-8}. A duper-cell  generates four monotone chains with respect to 
 vertical and horizontal lines along its boundary. See Figure \ref{figure-8}, where , and  
 are the monotone chains. We rename them as {\bf left, bottom, right}, and {\bf top} monotone chains.
\end{definition}

The basic idea is as follows: first optimally solve the subproblem {\it duper-cell} i.e., 
find an MDS for the set , where  is a duper-cell and then apply shifting 
strategy in both horizontal and vertical directions separately. The Lemma \ref{lemma-1x} leads 
to restriction on the size of the MDS, which is at most 30. Therefore an easy optimum solution 
for MDS can be obtained in  time. Here we propose a different technique for the MDS problem 
leading to lower time complexity as follows:

\begin{figure*}[!ht]
\begin{center}
\includegraphics[height=1.5in]{shift-1.eps}
\end{center} 
\caption{Demonstration of -factor approximation algorithm}
\label{figure-8}
\end{figure*}

We divide the duper-cell  into 2 groups  unshaded region () and shaded region () as shown in 
Figure \ref{figure-8}. Let  be the common boundary of the regions and two extended lines 
(see Figure \ref{figure-8}). Let  and  be two sets of points in the left (resp. right) of  such 
that each disk in  and  intersects . 

\begin{algorithm}[!ht]
\caption{MDS\_for\_duper-cell()}
\begin{algorithmic}[1]
\STATE {\bf Input:} A set  of  points and a duper-cell .
\STATE {\bf Output:} A set  for an MDS of .

\STATE Find  and  as described above. 
\STATE Let  and  be the set of points in  such that each 
disk in  and  covers at least one point in  and 
 respectively. 

\STATE 
\FOR {()}
  \STATE choose all possible  disks in  (resp. ) and for each combination 
  of  disks find  and  such that  and uncovered by that  disks, 
  and  and uncovered by that  disks.
  
  \STATE Call Algorithm \ref{algo-3factor} for finding an MDS for the sets  and  separately.

\ENDFOR
\STATE Return 
\end{algorithmic}
\label{algo-shifting}
\end{algorithm}

\begin{lemma} \label{lemma-9x}
An MDS for the set of points inside a duper-cell  can be computed optimally 
in  time, where  is the input size.
\end{lemma}

\begin{proof}
 The time complexity of line number 8 of the Algorithm \ref{algo-shifting} is  
 (see Lemma \ref{lemma-7x}). The line number 8 executes at most  time by the {\bf for} loop 
 in line number 6. Therefore the time complexity of the lemma follows.
 
 In the {\bf for} loop (line number 6 of the algorithm), we considered 
 all possible  () disks in  and  separately. Since 
 the number of cells that can intersect with such  disks is at most 9, therefore the range of  is correct. 
 For each combination of  disks, we considered all possible combinations to solve the problem 
 for  and  separately. Therefore the correctness of the algorithm follows. 
\end{proof}


\begin{theorem} \label{theorem-3y}
The shifting strategy discussed in Subsection \ref{shifting-strategy} gives a -factor 
approximation algorithm, which runs in  time for the MDS problem, where  is the 
input size.
\end{theorem}

\begin{proof}
The distance between the monotone chains {\bf left} and {\bf right} of  is greater than 8,  
the distance between the monotone chains {\bf bottom} and {\bf top} is 2, and the diameter  of the 
disks is 2. Now, if we apply shifting 
strategy in horizontal and vertical directions separately, then we get -factor 
i.e. -factor approximation algorithm in  time (see Lemma \ref{lemma-9x}) 
for the MDS problem. 
\end{proof}

\subsection{A PTAS for MDS Problem}
In this section, we present a -factor approximation algorithm in  time for 
a positive integer . Suppose a set  of  points within a rectangular region  is given. 
Consider a partition of  into regular hexagonal cells of side length . The idea 
of our algorithm is to solve the MDS problem optimally for the points inside regular hexagons (say ) 
such that the distance between {\bf left} and {\bf right} (resp. {\bf bottom} and {\bf top}) monotone chains 
is  (see Figure \ref{figure-9}) and using our proposed shifting strategy carefully 
(see Subsection \ref{shifting-strategy}). 

\begin{figure*}[!ht]
\begin{center}
\includegraphics[height=2in]{shift-2.eps}
\end{center} 
\caption{Demonstration of PTAS}
\label{figure-9}
\end{figure*}

To solve the MDS problem in  we further decompose  into four parts using 
the monotone chains  and  as shown in Figure \ref{figure-9}. The number of disks in the 
optimum solution intersecting the chain  with centers {\bf left} (resp. {\bf right}) side of 
 is at most  which is less than  and the 
number of disks in the optimum solution intersecting the chain  with centers {\bf bottom} 
(resp. {\bf top}) side of  is at most  which 
is less than . Next we apply recursive procedure to solve four independent sub-problems of 
size . If  is the running time of the recursive algorithm for the MDS problem 
for , then using the technique of \cite{DDCN13} we have the following recurrence 
relation: , which leads to the following theorem. 

\begin{theorem}\label{theorem-4y}
 For a given set  of  points in , the proposed algorithm produces an MDS of  
  in  time, whose size is at most , where  is a 
 positive integer and  is the optimum solution. 
\end{theorem}


\section{Conclusion}
In this paper, we proposed a series of constant factor approximation algorithms for the MDS 
problem for a given set  of  points. Here we used hexagonal partition very carefully.
We first presented a simple 4-factor and 3-factor approximation algorithms in  
and  time respectively, which improved the time complexities of best known 
result by a factor of  and  respectively \cite{DDCN13}. Finally, we proposed a 
very important shifting lemma and using this lemma we presented a -factor approximation 
algorithm and a PTAS for the MDS problem. Though the complexity of the proposed PTAS is same as that 
of the PTAS proposed by De et al. \cite{DDCN13} in terms of  notation, but the constant involved 
in our PTAS is smaller than the same in \cite{DDCN13}.

\bibliographystyle{abbrv}
\begin{thebibliography}{10}
\bibitem{ABD13}
R. Acharyya, M. Basappa and G.K. Das.
\newblock Unit disk cover problem in 2D.
\newblock In {\em ICCSA}, LNCS - 7972, pp. 73--85, 2013. 

\bibitem{AEMN06}
C.~Amb{\"u}hl, T.~Erlebach, M.~Mihal{\'a}k and M.~Nunkesser.
\newblock Constant-factor approximation for minimum-weight (connected)
  dominating sets in unit disk graphs.
\newblock In {\em APPROX-RANDOM}, pp. 3--14, 2006.

\bibitem{BCKO08}
M. de Berg, O. Cheong, M. van kreveld and M. H. Overmars.
\newblock {\em Computational Geometry: Algorithms and Applications (3. ed.).}
\newblock Springer-Verlag, 2008.

\bibitem{chvatal79}
V. Chv\'{a}tal.
\newblock A greedy heuristic for the set-covering problem. 
\newblock{\em Mathematics of Operations Research}, 4(3):233--235, 1979.

\bibitem{CMWZ04}
G.~C{\u{a}}linescu, I.~I. Mandoiu, P.~J. Wan and A.~Zelikovsky.
\newblock Selecting forwarding neighbors in wireless ad hoc networks.
\newblock {\em Mobile Network Applications}, 9(2):101--111, 2004.

\bibitem{CKL07}
P.~Carmi, M.~J. Katz and N.~Lev-Tov.
\newblock Covering points by unit disks of fixed location.
\newblock In {\em ISAAC}, pp. 644--655, 2007.

\bibitem{CKL08}
P.~Carmi, M.~J. Katz and N.~Lev-Tov.
\newblock Polynomial-time approximation schemes for piercing and covering with
  applications in wireless networks.
\newblock {\em Comput. Geom.}, 39(3):209--218, 2008.

\bibitem{CCJ90}
B.~N. Clark, C.~J. Colbourn and D.~S. Johnson.
\newblock Unit disk graphs.
\newblock {\em Discrete Mathematics}, 86(1-3):165--177, 1990.

\bibitem{CDDDFLNS10}
F.~Claude, G.~K. Das, R.~Dorrigiv, S.~Durocher, R.~Fraser, A.~L{\'o}pez-Ortiz,
  B.~G. Nickerson and A.~Salinger.
\newblock An improved line-separable algorithm for discrete unit disk cover.
\newblock {\em Discrete Math. Alg. and Appl.}, 2(1):77--88, 2010.

\bibitem{DDCN13}
M. De, G. K. Das, P. Carmi and S.C. Nandy.
\newblock Approximation algorithms for a variant 
of discrete piercing set problem for unit disks.
\newblock {\em Int. J. of Comput. Geom. and Appl.}, 2013 (to appear).

\bibitem{DFLN12}
G.~K. Das, R.~Fraser, A.~L{\'o}pez-Ortiz and B.~G. Nickerson.
\newblock On the discrete unit disk cover problem.
\newblock {\em Int. J. of Comput. Geom. and Appl.}, 22:407-419, 2012.

\bibitem{DY09}
D.~Dai and C.~Yu.
\newblock A 5+-approximation algorithm for minimum weighted
  dominating set in unit disk graph.
\newblock {\em Theor. Comput. Sci.}, 410(8-10):756--765, 2009.

\bibitem{FL12}
R.~Fraser and A.~L{\'o}pez-Ortiz.
\newblock The within-strip discrete unit disk cover problem.
\newblock In {\em CCCG}, pp. 53--58, 2012.

\bibitem{FFSM12}
G.D. da Fonseca, C. M. H. de Figueiredo, V. G. P. de S\'{a} and R. Machado.
\newblock Linear time approximation for dominating sets and independent dominating sets in unit disk graphs.
\newblock In {\em WAOA 2012}, LNCS 7846, pp. 82--92, 2013.

\bibitem{FFSM12-1}
G. D. da Fonseca, C. M. H. de Figueiredo, V. G. P. de S\'{a} and R. Machado.
\newblock Linear-time sub-5 approximation for dominating sets in unit disk graphs.
\newblock preprint, {\em arXiv:1204.3488}, 2012.

\bibitem{GJ79}
M.~R. Garey and D.~S. Johnson.
\newblock {\em Computers and Intractability: A Guide to the Theory of
  NP-Completeness}.
\newblock W. H. Freeman, 1979.

\bibitem{GP10}
M.~Gibson and I.~A. Pirwani.
\newblock Algorithms for dominating set in disk graphs: breaking the log{\it n}
  barrier.
\newblock In {\em ESA}, pp. 243--254, 2010.

\bibitem{HM85}
D.~S. Hochbaum and W.~Maass.
\newblock Approximation schemes for covering and packing problems in image
  processing and VLSI.
\newblock {\em J. ACM}, 32(1):130--136, 1985.

\bibitem{HGZW08}
Y.~Huang, X.~Gao, Z.~Zhang and W.~Wu.
\newblock A better constant-factor approximation for weighted dominating set in
  unit disk graph.
\newblock {\em Journal of Combinatorial Optimization}, 18:179--194, 2008.

\bibitem{J82}
D.~S. Johnson.
\newblock The np-completeness column: an ongoing guide.
\newblock {\em J. Algorithms}, 3(2):182--195, 1982.

\bibitem{MBIRR95}
M.~V. Marathe, H.~Breu, H.~B.~H. III, S.~S. Ravi and D.~J. Rosenkrantz.
\newblock Simple heuristics for unit disk graphs.
\newblock {\em Networks}, 25(2):59--68, 1995.

\bibitem{MR10}
N.~H. Mustafa and S.~Ray.
\newblock Improved results on geometric hitting set problems.
\newblock {\em Discrete {\&} Computational Geometry}, 44(4):883--895, 2010.

\bibitem{NH06}
T.~Nieberg and J.~Hurink.
\newblock A PTAS for the minimum dominating set problem in unit disk graphs.
\newblock In {\em WAOA}, pages 296--306, 2005.

\bibitem{NV06}
S.~Narayanappa and P.~Vojtechovsk{\'y}.
\newblock An improved approximation factor for the unit disk covering problem.
\newblock In {\em CCCG}, pp. 15--18, 2006.

\bibitem{PS09}
F. P. Preparata and M. I. Shamos.
\newblock {\em Computational Geometry: An Introduction}. 
\newblock Springer-Verlag, 2009.

\bibitem{RS97}
R. Raz and S. Safra.
\newblock A sub-constant error-probability low-degree test, and a
sub-constant error-probability PCP characterization of NP. 
\newblock In {\em STOC}, pp. 475--484, 1997.

\bibitem{ZWXLDWW11}
F.~Zou, Y.~Wang, X.~Xu, X.~Li, H.~Du, P.~J. Wan and W.~Wu.
\newblock New approximations for minimum-weighted dominating sets and
  minimum-weighted connected dominating sets on unit disk graphs.
\newblock {\em Theor. Comput. Sci.}, 412(3):198--208, 2011.

\end{thebibliography}
\end{document}
