

\usepackage{biograph}        \usepackage{afterpage}
\usepackage{array}
\usepackage{booktabs}

\usepackage{cite}
\usepackage{multiro                                                                                                                                                                                     w} 
\usepackage{rotating}

\usepackage{float}




\usepackage{color}
\usepackage{soul}

\usepackage{colortbl}

\interfootnotelinepenalty=10000



\def\rmd{{\rm d}}
\def\por1{\partial}
\DeclareMathOperator*{\argmax}{arg\,max}
\DeclareMathOperator*{\argmin}{arg\,min}
\DeclareMathOperator*{\submin}{\,min}
\newcolumntype{S}{>{\centering\arraybackslash} m{.4\linewidth} }

\makeatletter
  \newcommand\tinyv{\@setfontsize\tinyv{5pt}{7}}
\makeatother

\newlength{\hspacephantom}
\settowidth{\hspacephantom}{C = 0.00212}




\hyphenation{data-set data-sets met-hod}
\begin{document}
\DeclareGraphicsExtensions{.pdf,.jpeg,.png}

\title{Mean Deviation Similarity Indices: Efficient and Reliable Full-Reference Image Quality Evaluators}





\author{Hossein Ziaei Nafchi, Atena Shahkolaei, Rachid Hedjam, \textnormal{and} Mohamed Cheriet, \IEEEmembership{Senior Member,~IEEE} \thanks{Copyright (c) 2015 IEEE. Personal use of this material is permitted. However, permission to use this material for any other purposes must be obtained from the IEEE by sending a request to pubs-permissions@ieee.org.}
\thanks{H. Ziaei Nafchi, A. Shahkolaei and M. Cheriet are with the Synchromedia Laboratory for Multimedia Communication in Telepresence,
\'Ecole de technologie sup\'erieure, Montreal (QC), Canada H3C 1K3; Tel.: +1-514-396-8972; Fax: +1-514-396-8595;
Emails: hossein.zi@synchromedia.ca, atena.shahkolaei.1@ens.etsmtl.ca, mohamed.cheriet@etsmtl.ca}
\thanks{R. Hedjam is with the Department of Geography, McGill University, 805 Sherbrooke Street West, Montreal, QC H3A 2K6, Canada (email: rachid.hedjam@mcgill.ca)}
\thanks{Manuscript received ? ?, ?; revised ? ?, ?.}}



\markboth{}{}




\maketitle

\begin{abstract}
Applications of perceptual image quality assessment (IQA) in image and video processing, such as image acquisition, image compression, image restoration and multimedia communication, have led to the development of many IQA metrics. In this paper, two reliable full reference IQA models are proposed that utilize gradient similarity (GS), chromaticity similarity (CS), and deviation pooling (DP). By considering the shortcomings of the commonly used GS to model human visual system (HVS), a fusion technique is proposed to derive an HVS-based GS. We propose an efficient and effective formulation to calculate the joint similarity map of two chromatic channels for the purpose of measuring color changes. Unlike a commonly used formulation in the literature, the proposed CS map is shown to provide consistent quality predictions when used with the mean and the deviation pooling strategies. Motivated by a recent work that utilizes the standard deviation pooling, a general formulation of the DP is presented in this paper and used to compute a final score from the proposed GS and CS maps. The experimental results on five datasets of natural images and a synthetic dataset show that the proposed indices provide comparable or better quality predictions than the state-of-the-art IQA metrics in the literature, they are reliable and have low complexity. The MATLAB source codes of the proposed metrics are available to public. \end{abstract}


\begin{IEEEkeywords}
Image quality assessment, gradient similarity, chromaticity similarity, deviation pooling, synthetic image, Human visual system.
\end{IEEEkeywords}

\IEEEpeerreviewmaketitle




\maketitle


\section{Introduction}
\label{sec:intro}





\IEEEPARstart{A}{utomatic}
image quality assessment (IQA) plays a significant
role in numerous image and video processing applications. IQA is commonly used in image acquisition, image compression, image restoration, multimedia streaming, etc \cite{applications2011}. IQAs mimic the average quality predictions of human observers. Full reference IQAs (FR-IQAs), which fall within the scope of this paper, evaluate the perceptual quality of a distorted image with respect to its reference image. This quality prediction is an easy task for the human visual system (HVS) and the result of the evaluation is reliable. Automatic quality assessment, e.g. objective evaluation, is not an easy task because images may suffer from various types
and degrees of distortions. FR-IQAs can be employed to compare two images of the same dynamic range (usually low dynamic range) \cite{SSIM} or different dynamic ranges \cite{TMQI, FSITM}. This paper is dedicated to the IQA for low dynamic range images. 

Among IQAs, the conventional metric mean squared error (MSE) and its variations are widely used because of their simplicity. However, in many situations, MSE does not correlate with the human perception of image fidelity and quality \cite{scope2008, spm2009}. Because of this limitation, a number of IQAs have been proposed to provide better correlation with the HVS \cite{NQM, SSIM, MSSSIM, IFC, VIF, VSNR, RFSIM, MAD, SVR2010, CPSSIM, IWSSIM, FSIM, GS, SRSIM, GMSD, SFF, VSI}. In general, these better performing metrics measure structural information, luminance information and contrast information in the spatial and frequency domains.   

The most successful IQA models in the literature follow a top-down strategy \cite{Metrics2011}. They calculate a similarity map and use a pooling strategy that converts the values of this similarity map into a single quality score. For example, the luminance, contrast and structural information constitute a similarity map for the popular SSIM index \cite{SSIM}. SSIM then uses average pooling to compute the final similarity score. Different feature maps are used in the literature for calculation of this similarity map. Feature similarity index (FSIM) uses phase congruency and gradient magnitude features. GS \cite{GS} uses a combination of some designated gradient magnitudes and image contrast for this end, while the GMSD \cite{GMSD} uses only the gradient magnitude. SR\_SIM \cite{SRSIM} uses saliency features and gradient magnitude. VSI \cite{VSI} likewise benefits from saliency-based features and gradient magnitude. SVD based features \cite{SVD2006}, features based on the Riesz transform \cite{RFSIM}, features in the wavelet domain \cite{VSNR, wavelet2011, CWSSIM} and sparse features \cite{SFF} are used as well in the literature. Among these features, gradient magnitude is an efficient feature, as shown in \cite{GMSD}. In contrast, phase congruency and visual saliency features in general are not fast enough features to be used. Therefore, the features being used play a significant role in the efficiency of IQA models.   

As we mentioned earlier, the computation of the similarity map is followed by a pooling strategy. The state-of-the-art pooling strategies for perceptual image quality assessment (IQA) are based on the mean and the weighted mean \cite{SSIM, MSSSIM, IWSSIM, GS, FSIM, RFSIM}. They are robust pooling strategies that usually provide a moderate to high performance for different IQAs. Minkowski pooling \cite{spatial2006}, local distortion pooling \cite{spatial2006, pooling2009, MAD}, percentile pooling \cite{percentile2009} and saliency-based pooling \cite{SRSIM, VSI} are other possibilities. Recently, standard deviation (SD) pooling was also proposed and successfully used by GMSD \cite{GMSD}. The image gradients are sensitive to image distortions. Different local structures in a distorted image suffer from different degrees of degradations. This is the motivation
that the authors in \cite{GMSD} used to explore the standard variation
of the gradient-based local similarity map for overall image
quality prediction. In general, features that constitute the similarity map and the pooling strategy are very important factors in designing high performance IQA models. 



Here, we propose two IQA models called the mean deviation similarity index (MDSI) and MDSI that show very good compromise between prediction accuracy and model complexity. The proposed indices are efficient, effective and reliable at the same time. They also show consistent performance for both natural and synthetic images. The proposed metrics follow a top-down strategy. They use gradient magnitude to measure structural distortions and use chrominance features to measure color distortions. These two similarity maps are then combined to form a gradient-chromaticity similarity map. We then propose a novel deviation pooling strategy and use it to compute the final quality score. Both image gradient \cite{GSSIM, gradient2010, GS, FSIM, GMSD, VSI} and chrominance features \cite{FSIM, VSI} have been already used in the literature. The proposed MDSI utilizes the conventional gradient similarity, while the MDSI uses an HVS-based gradient similarity which is proposed in this paper. Also, both MDSI and MDSI use a new chromaticity similarity map which shows consistent performance when used with the different pooling strategies. The proposed index uses the summation over similarity maps to give independent weights to them. Also, less attention has been paid to the deviation pooling strategy, except for a special case of this type of pooling, namely, standard deviation pooling \cite{GMSD}. We therefore provide a general formulation for the deviation pooling strategy and show its power in the case of the proposed IQA models. The MATLAB source code of the proposed metrics is available at: https://dl.dropboxusercontent.com/u/74505502/MDSI.m and https://dl.dropboxusercontent.com/u/74505502/MDSIplus.m. In the following, the main contributions of the paper as well as its differences with respect to the previous works are briefly explained in sections \ref{A}, \ref{B}, and \ref{C}. 

\subsection{Features}
\label{A}

1) Unlike previous researches \cite{GSSIM, gradient2010, GS, FSIM, GMSD, VSI} that use a similar gradient similarity map, a new gradient similarity map is proposed in this paper which is more relevant to the human visual system (HVS). This statement is supported by visual examples and experimental results.

2) This paper proposes a new chormaticity similarity map with the following advantages over the previously used chromaticity similarity maps \cite{FSIM, VSI}. Its complexity is lower and is more robust to be used with the different pooling strategies.     

\subsection{Pooling}
\label{B}

1) Motivated by a previous study that proposed to use standard deviation pooling \cite{GMSD}, we propose a systematic and general formulation of the deviation pooling which has a comprehensive scope.    

\subsection{Results}
\label{C}

1) Two simple evaluation metrics are used in this paper to measure the reliability of IQA models.

2) The performance of IQA models on two datasets of ``synthetic images" and ``contrast distorted images" is compared.


\vspace{2mm}

The rest of the paper is organized as follows. The proposed mean deviation similarity indices are presented in section \ref{MDSI}. Extensive experimental results and discussion on five natural datasets and a synthetic dataset are provided in section \ref{results}. Section \ref{conclusion} presents our conclusions.             



\section{Mean Deviation Similarity Indices}
\label{MDSI}

The proposed IQA models use two similarity maps. Image gradient, which is sensitive to structural distortions, is used as the main feature to calculate the first similarity map. Then, color distortions are measured by a chromaticity similarity map. These similarity maps are combined and pooled by a proposed deviation pooling strategy. We use the following HVS optimized Gaussian color model to convert input RGB images:
 
According to \cite{invariance2001}, this Gaussian color model may be imperfect, but it is likely to offer accurate estimates of differential measurements. Therefore, image gradient computation in this color space should yield more accurate results. This color space has also been shown to provide a good decorrelation between three channels which is useful in measuring color distortions.    



\subsection{Gradient Similarity}


It is very common that gradient magnitude in the discrete domain is calculated on the basis of some operators that approximate derivatives of the image function using differences. These operators approximate vertical  and horizontal  gradients of an image  using convolution:  and . Where  and  are horizontal and vertical gradient operators and  denotes the convolution. The first derivative magnitude is defined as . The Sobel operator \cite{sobel}, the Scharr operator, and the Prewitt operator are common gradient operators that approximate first derivatives. Within the proposed IQA model, these operators perform almost the same. It is worth mentioning that we have used the more recent gradient operators of \cite{Farid} without observing improvement.     

The proposed IQA model MDSI uses the Prewitt operator to compute gradient magnitudes of luminance  channels of reference and distorted images,  and . From which, gradient similarity (GS) is computed by the following SSIM induced equation:
                    
where, parameter  is a constant to control numerical stability. The gradient similarity (GS) is widely used in the literature \cite{GSSIM, gradient2010, GS, FSIM, GMSD, VSI} and its usefulness to measure image distortions was extensively investigated in \cite{GMSD}. 




\begin{figure*}[htb]
\scriptsize
\begin{minipage}[b]{0.245\linewidth}
  \centering
  \centerline{\includegraphics[height=3.1cm]{I06.png}} 
\end{minipage}
\begin{minipage}[b]{0.245\linewidth}
  \centering
  \centerline{\includegraphics[height=3.1cm]{i06_24_5.png}}
\end{minipage}
\begin{minipage}[b]{0.245\linewidth}
  \centering
  \centerline{\includegraphics[height=3.1cm]{I06GS12.png}}
\end{minipage}
\begin{minipage}[b]{0.245\linewidth}
  \centering
  \centerline{\includegraphics[height=3.1cm]{I06GGG.png}} 
\end{minipage}
\\ \\
\begin{minipage}[b]{0.245\linewidth}
  \centering
  \centerline{\includegraphics[height=3.1cm]{I03.png}}
  \vspace{0.10cm}
\centerline{}
\end{minipage}
\begin{minipage}[b]{0.245\linewidth}
  \centering
  \centerline{\includegraphics[height=3.1cm]{i03_20_5.png}}
  \vspace{0.10cm}
\centerline{}
\end{minipage}
\begin{minipage}[b]{0.245\linewidth}
  \centering
  \centerline{\includegraphics[height=3.1cm]{I03GS12.png}}
  \vspace{0.10cm}
\centerline{GS}
\end{minipage}
\begin{minipage}[b]{0.245\linewidth}
  \centering
  \centerline{\includegraphics[height=3.1cm]{I03GGG.png}}
  \vspace{0.10cm}
\centerline{ (proposed)}
\end{minipage}
\caption{The difference between the conventional gradient similarity (GS) and the proposed HVS-based gradient similarity ().}
\label{GShvs}
\end{figure*}




In many scenarios, human visual system (HVS) disagrees with the judgments provided by the GS for structural distortions. In fact, in such a formulation, there is no difference between an added edge to or a removed edge from the distorted image with respect to the reference image. Also, GS does not take into account the ``edge color" itself. An extra edge in  bring less attention of HVS if its color is close to the relative pixels of that edge in . Likewise, HVS pays less attention to a removed edge from  that is replaced with pixels of the same or nearly the same color. In another scenario, suppose that edges are preserved in  but with different colors than in . In this case, GS is likely to fail at providing a good judgment ``on the edges". These shortcomings of the GS motivated us to propose an HVS-based GS map. 



\subsection{HVS-based Gradient Similarity} 

The aforementioned shortcomings of the conventional gradient similarity map (equation \ref{GS}) are mainly because  and  are computed independent of each other. In the following, we propose a fusion technique to include the correlation between  and  images into computation of the gradient similarity map.   

We fuse the luminance  channels of the  and  by a simple averaging:  = 0.5  ( + ). Two extra GS maps are computed as follows: 
                    
                    
where,  is the gradient magnitude of the fused image , and  is used for numerical stability. Please note that  and  can or can not be equal. The HVS-based gradient similarity () is computed by:
     

Fig. \ref{GShvs} shows two examples of the GS and . For the boat image (first row), GS and  are very similar except for the top of the image (sky). While GS indicates that edges in this region are highly distorted, the proposed map  truly shows that there is smaller difference between  and  in this region. For the hats image (second row), there is a considerable difference between GS and . In general,  highlighted more visible differences than the GS (the grooves on the wood, the grooves on the brims of the hats, blocks in the sky, etc). The experimental results of section \ref{results} also verify the usefulness of .              


\begin{figure*}[htb]
\scriptsize
\begin{minipage}[b]{0.197\linewidth}
  \centering
  \centerline{\includegraphics[height=2.35cm]{I17.png}} 
  \vspace{0.10cm}
\centerline{}
\end{minipage}
\begin{minipage}[b]{0.19\linewidth}
  \centering
  \centerline{\includegraphics[height=2.35cm]{i17_10_4.png}}
  \vspace{0.10cm}
\centerline{ (JPEG compression)}
\end{minipage}
\begin{minipage}[b]{0.19\linewidth}
  \centering
  \centerline{\fbox{\includegraphics[height=2.35cm]{GSs.png}}}
\vspace{0.10cm}
\centerline{GS}
\end{minipage}
\begin{minipage}[b]{0.19\linewidth}
  \centering
  \centerline{\fbox{\includegraphics[height=2.35cm]{CSs.png}}}
\vspace{0.10cm}
\centerline{CS}
\end{minipage}
\begin{minipage}[b]{.19\linewidth}
  \centering
  \centerline{\fbox{\includegraphics[height=2.35cm]{GCSs.png}}}
\vspace{0.10cm}
\centerline{GCS}
\end{minipage}
\\ \\
\begin{minipage}[b]{0.19\linewidth}
  \centering
  \centerline{\includegraphics[height=2.35cm]{I23.png}} 
  \vspace{0.10cm}
\centerline{}
\end{minipage}
\begin{minipage}[b]{0.19\linewidth}
  \centering
  \centerline{\includegraphics[height=2.35cm]{i23_18_5.png}}
  \vspace{0.10cm}
\centerline{ (color saturation)}
\end{minipage}
\begin{minipage}[b]{0.19\linewidth}
  \centering
  \centerline{\fbox{\includegraphics[height=2.35cm]{GS.png}}}
\vspace{0.10cm}
\centerline{GS}
\end{minipage}
\begin{minipage}[b]{0.19\linewidth}
  \centering
  \centerline{\fbox{\includegraphics[height=2.35cm]{CS.png}}}
\vspace{0.10cm}
\centerline{CS}
\end{minipage}
\begin{minipage}[b]{.19\linewidth}
  \centering
  \centerline{\fbox{\includegraphics[height=2.35cm]{GCS.png}}}
\vspace{0.10cm}
\centerline{GCS}
\end{minipage}
\caption{Complementary behavior of the gradient similarity (GS) and chromaticity similarity (CS) maps.}
\label{fig:out1}
\end{figure*}


\subsection{Chromaticity Similarity}
\label{Chromaticity}

For the case of color changes and especially when the structure of the distorted image remains unchanged, the gradient similarity (GS) and HVS-based GS may lead to inaccurate quality predictions. Therefore, previous researches such as \cite{FSIM, VSI} used a color similarity map to measure color differences. Let  and  denote two chromaticity channels regardless of the type of the color space. In \cite{FSIM, VSI}, for each channel a color similarity is computed and their result is combined as:
                    
where  is a constant to control numerical stability. In this paper, we propose a new formulation to calculate color similarity and show that it is more robust than the above color similarity since it can be used with different pooling strategies. The proposed formulation calculates a color similarity map using both chromaticity channels at once:    
                    


Similar to the CS in equation (\ref{CS}), the above joint color similarity () formulation gives equal weight to both chromaticity channels  and . The proposed  map has several advantages over CS. It is clear that  is more efficient than CS. CS needs 9 multiplications, 6 summations and 2 divisions, while  needs 7 multiplications, 6 summations and 1 division. Note that CS can also be computed through 10 multiplications, 6 summations and 1 division. Further, to show the robustness advantage of the proposed similarity map  over CS, we conducted the following experiment. Contrast distorted images of the CCID2014 dataset \cite{CCID2014} were chosen. This dataset contains 655 contrast distorted images of five types. Gamma transfer, convex and concave arcs, cubic and logistic functions, mean shifting, and a compound function are used to generate these five types of distortions. We analyzed the SRC performance of the CS and  as being two FR-IQA models for wide range of  values. Three pooling strategies were used in this experiment, e.g. mean pooling, mean absolute deviation (MAD) pooling and the standard deviation (SD) pooling. Fig. \ref{CSJCS} shows the SRC performance of the CS and  versus constant  for the three pooling strategies. From the plots in Fig. \ref{CSJCS}, the following conclusions can be drawn. Using the mean pooling, both CS and  perform almost the same. For MAD and SD poolings, the SRC performance of the  is higher than that for the CS. The highest SRC performance of the  for both MAD and SD poolings is higher than that for CS. For the three pooling strategies, peak SRC values of  are very close to each other. Also, there is a little variation between SRC values of the proposed  compared to the CS. Therefore, we can conclude that  is a more robust color similarity map than CS for a wider range of the pooling strategies. The results proof that these two color similarities, e.g. CS and , are essentially different and have different numerical behaviors. 


\begin{figure}[htb]
\scriptsize
\begin{minipage}[b]{0.99\linewidth}
  \centering
  \centerline{\includegraphics[height=5cm]{colorSRC11.png}} 
  \vspace{0.10cm}
\centerline{a) CS}
\end{minipage}
\\ \\
\begin{minipage}[b]{.99\linewidth}
  \centering
  \centerline{\includegraphics[height=5cm]{colorSRC22.png}}
  \vspace{0.10cm}
\centerline{b)  (proposed)}
\end{minipage}
\caption{The SRC performance of the two chromaticity similarities CS and  for different values of  and three pooling strategies.}
\label{CSJCS}
\end{figure}


The gradient similarity maps (GS or ) can be combined with the joint color similarity map  through the following summation scheme:       
 
 
where the parameter  adjusts the relative importance of the gradient and chromaticity similarity maps. An alternative combination scheme which is very popular in state-of-the-art is through multiplication in the form of , where the parameters  and  are used to adjust the relative importance of the two similarity maps. The proposed indices use the summation scheme for three reasons. First, slightly higher performances were achieved through the summation combination scheme within the context of the proposed indices that we refer to in the experimental results section with an example. Second advantage of the summation combination approach is its efficiency over the multiplication approach, e.g. two multiplications and one summation versus two powering and one multiplication. Finally, the optimization of equations (\ref{GJCS1} and \ref{GJCS2}) is easier than the above mentioned multiplication equation since they are linear and have one less parameter.  
        


In Fig \ref{fig:out1}, two examples are provided to show that these two similarity maps, e.g. GS and , are complementary. In the first example, there is a considerable difference between the gradient maps of the reference and the distorted images. Hence, the GS map is enough for a good judgment. However, this difference in the second example (second row) is trivial, which leads to a wrong prediction by using GS as the only similarity map. The examples in Fig \ref{fig:out1} show that the gradient similarity and chromaticity similarity are complementary. Please note that the same conclusion can be reached by replacing the GS with .       





\subsection{Deviation Pooling}


The motivation of using the deviation pooling is that HVS is sensitive to both magnitude and the spread of the distortions across the image. Other pooling strategies such as Minkowski pooling and percentile pooling adjust the magnitude of distortions or discard the less/non distorted pixels. These pooling strategies and the mean pooling do not take into account the spread of the distortions. A common wrong prediction by mean pooling is where it calculates the same quality scores for two distorted images of different type. In such cases, deviation pooling can provides good judgments over their quality through spread of the distortions. This is the reason why mean pooling have good inter-class quality prediction but its performance degrades for intra-class quality prediction. This statement can be verified from the experimental results provided in \cite{GMSD}. In the following, we propose the deviation pooling (DP) strategy and provide a general formulation of this pooling. 

DP for IQAs is rarely used in the literature, except the standard deviation used in GMSD \cite{GMSD}, which is a special case of DP. A deviation can be seen as the Minkowski distance of order  between vector \textbf{x} and its MCT:  
                    
where  indicates the type of deviation. Several researches shown that more emphasis on the severer distortions can lead to more accurate predictions. The Minkowski pooling \cite{spatial2006} and the percentile pooling \cite{percentile2009} are two examples. These pooling strategies follow a property of HVS that penalize severer distortions much more than the less distorted ones even though they constitute a small portion of total distortions. Hence, they try to moderate the weakness of the mean pooling through magnifying distortions \cite{spatial2006} or discarding the less/non distorted regions \cite{percentile2009}. The deviation pooling can be generalized to consider the aforementioned property of HVS:
                    
where,  adjusts the emphasis of the values in vector \textbf{x}, and MCT is calculated through \textbf{x} values. Furthermore, we propose to use power pooling in conjunction with the deviation pooling to control numerical behavior of the final quality scores:
                    
where,  is the power pooling applied on the final value of the deviation. The power pooling can be used to make an IQA model more linear versus the subjective scores or might be used for better visualization of the scores. Note that the above deviation pooling is equal to the Minkowski pooling \cite{spatial2006} when MCT ,  and . It is equal to the mean absolute deviation (MAD) to the power of  for  and equal to the standard deviation (SD) to the power of  for . The three parameters should be set according to the IQA model. For the two proposed indices MDSI and MDSI, we set ,  and . Therefore, the proposed IQA models can be written as:
                    
                    


In fact, GCS () places less emphasis on the less distorted regions in comparison with GCS (). These are regions that bring less attention to the HVS. Also, the values of GCS at the distorted regions become closer to the non-distorted and less-distorted parts. This moderates the possible overemphasis of GCS at the distorted regions as compared to GCS. Note that the extremely distorted and non-distorted regions remain unchanged. The global variations of GCS are then computed by mean absolute deviation. This is followed by power pooling. The power pooling, which is a part of the general formulation of deviation pooling, is used in this paper to increase linearity of the proposed metric.   






\begin{table*}[htb]
\caption{Performance comparison of the proposed IQA model MDSI and thirteen popular/competing indices on four benchmark datasets}
\scriptsize
\centering
\begin{tabular}{cccccccccccccccc}
\hline
                      &      & PSNR   & SSIM   & MSSSIM & VIF    & GS     & MAD             & IWSSIM & SR\_SIM         & FSIM           & GMSD            & SFF             & VSI             & MDSI             & MDSI                \\ \hline
                      & SRC  & 0.5531 & 0.7749 & 0.8542 & 0.7491 & 0.8504 & 0.8340          & 0.8559 & 0.8913 & 0.8840          & 0.8907          & 0.8767          & \textbf{0.8979} & \textbf{0.9070} & \textbf{0.9146}\\
TID                   & PCC  & 0.5734 & 0.7732 & 0.8451 & 0.8084 & 0.8422 & 0.8290          & 0.8579 & \textbf{0.8866} & 0.8738          & 0.8788          & 0.8817 & 0.8762          & \textbf{0.8891} & \textbf{0.9112}\\
2008                  & KRC  & 0.4027 & 0.5768 & 0.6568 & 0.5861 & 0.6596 & 0.6445          & 0.6636 & \textbf{0.7149} & 0.6946          & 0.7092          & 0.6882          & 0.7123 & \textbf{0.7254} & \textbf{0.7409}\\
                      & RMSE & 1.0994 & 0.8511 & 0.7173 & 0.7899 & 0.7235 & 0.7505          & 0.6895 & \textbf{0.6206} & 0.6468          & 0.6404          & 0.6333 & 0.6466          & \textbf{0.6143} & \textbf{0.5528}\\ \hline
                      & SRC  & 0.8058 & 0.8756 & 0.9133 & 0.9195 & 0.9108 & 0.9467          & 0.9213 & 0.9319          & 0.9310          & \textbf{0.9570} & \textbf{0.9627} & 0.9423          & 0.9536 & \textbf{0.9548}\\
\multirow{2}{*}{CSIQ} & PCC  & 0.7512 & 0.8613 & 0.8991 & 0.9277 & 0.8964 & 0.9500 & 0.9144 & 0.9250          & 0.9192          & \textbf{0.9541} & \textbf{0.9643} & 0.9279          & 0.9465 & \textbf{0.9504}         \\
                      & KRC  & 0.6084 & 0.6907 & 0.7393 & 0.7537 & 0.7374 & 0.7970          & 0.7529 & 0.7725          & 0.7690          & \textbf{0.8129} & \textbf{0.8288} & 0.7857          & 0.8056 & \textbf{0.8085}\\
                      & RMSE & 0.1733 & 0.1334 & 0.1149 & 0.0980 & 0.1164 & 0.0820 & 0.1063 & 0.0997          & 0.1034          & \textbf{0.0786} & \textbf{0.0695} & 0.0979          & 0.0847 & \textbf{0.0817}         \\ \hline
                      & SRC  & 0.8756 & 0.9479 & 0.9513 & 0.9636 & 0.9561 & \textbf{0.9669} & 0.9567 & 0.9618          & 0.9645 & 0.9603          & \textbf{0.9649} & 0.9524          & 0.9575 & \textbf{0.9647}         \\
\multirow{2}{*}{LIVE} & PCC  & 0.8723 & 0.9449 & 0.9489 & 0.9604 & 0.9512 & \textbf{0.9675} & 0.9522 & 0.9553          & 0.9613 & 0.9603          & \textbf{0.9632} & 0.9482          & 0.9543 & \textbf{0.9627}         \\
                      & KRC  & 0.6865 & 0.7963 & 0.8044 & 0.8282 & 0.8150 & \textbf{0.8421} & 0.8175 & 0.8299          & \textbf{0.8363} & 0.8268          & \textbf{0.8365} & 0.8058          & 0.8184 & 0.8338         \\
                      & RMSE & 13.359 & 8.9455 & 8.6188 & 7.6137 & 8.4327 & \textbf{6.9072} & 8.3472 & 8.0812          & 7.5296 & 7.6214          & \textbf{7.3460} & 8.6817          & 8.1666 & \textbf{7.3956}         \\ \hline
                      & SRC  & 0.6394 & 0.7417 & 0.7859 & 0.6769 & 0.7946 & 0.7807          & 0.7779 & 0.8073          & 0.8510          & 0.8044          & 0.8513 & \textbf{0.8965} & \textbf{0.8868} & \textbf{0.8894}\\
TID                   & PCC  & 0.4785 & 0.7895 & 0.8329 & 0.7720 & 0.8464 & 0.8267          & 0.8319 & 0.8663          & 0.8769 & 0.8590          & 0.8706          & \textbf{0.9000} & \textbf{0.8982} & \textbf{0.9090}\\
2013                  & KRC  & 0.4696 & 0.5588 & 0.6047 & 0.5147 & 0.6255 & 0.6035          & 0.5977 & 0.6406          & 0.6665 & 0.6339          & 0.6581          & \textbf{0.7183} & \textbf{0.7066} & \textbf{0.7110}\\
                      & RMSE & 1.0887 & 0.7608 & 0.6861 & 0.7880 & 0.6603 & 0.6976          & 0.6880 & 0.6193          & 0.5959 & 0.6346          & 0.6099          & \textbf{0.5404} & \textbf{0.5449} & \textbf{0.5166}\\ \hline
\end{tabular}
\label{results1}
\end{table*}




\begin{table*}[htb]
\caption{Overall performance comparison of the proposed IQA model MDSI and thirteen popular/competing indices over four benchmark datasets. The four datasets contain 52 distortion set.}
\scriptsize
\centering
\begin{tabular}{c|ccc|l|ccc|ccc|ccc}
\hline
\multirow{2}{*}{IQA model} & \multicolumn{4}{c|}{Weighted avg (6365 images)}                              & \multicolumn{3}{c|}{SRC (Distortions)}               & \multicolumn{3}{c|}{PCC (Distortions)}               & \multicolumn{3}{c}{KRC (Distortions)}                \\ \cline{2-14} 
                           & SRC               & PCC               & \multicolumn{2}{c|}{KRC}             & avg             & min              & std             & avg             & min              & std             & avg             & min              & std             \\ \hline
PSNR                       & 0.6684            & 0.5900            & \multicolumn{2}{c|}{0.4976}          & 0.8232          & +0.0766          & 0.1785          & 0.8282          & +0.1378 & 0.1807          & 0.6462          & +0.0514          & 0.1602          \\
SSIM \cite{SSIM}                       & 0.7944            & 0.8141            & \multicolumn{2}{c|}{0.6110}          & 0.8222          & -0.4141          & 0.2111          & 0.8413          & -0.4422          & 0.2081          & 0.6524          & -0.2930          & 0.1856          \\
MSSSIM \cite{MSSSIM}                     & 0.8421            & 0.8596            & \multicolumn{2}{c|}{0.6618}          & 0.8321          & -0.4099          & 0.2089          & 0.8544          & -0.4448          & 0.2052          & 0.6660          & -0.2873          & 0.1874          \\
VIF \cite{VIF}                        & 0.7651            & 0.8265            & \multicolumn{2}{c|}{0.6054}          & 0.8515          & -0.3099          & 0.1919          & 0.8728 & -0.3866          & 0.1984          & 0.6871          & -0.2200          & 0.1748          \\
GS \cite{GS}                         & 0.8454            & 0.8651            & \multicolumn{2}{c|}{0.6255}          & 0.8517          & -0.3578          & 0.1993          & 0.8688          & -0.3739          & 0.1964          & 0.6919          & -0.2496          & 0.1779          \\
MAD \cite{MAD}                       & 0.8408            & 0.8617            & \multicolumn{2}{c|}{0.6706}          & 0.7981          & -0.0575          & 0.2478          & 0.8179          & +0.0417          & 0.2257          & 0.6422          & -0.0487          & 0.2209          \\
IWSSIM \cite{IWSSIM}                     & 0.8406            & 0.8650            & \multicolumn{2}{c|}{0.6638}          & 0.8210          & -0.4196          & 0.2149          & 0.8465          & -0.4503          & 0.2097          & 0.6539          & -0.2910          & 0.1936  \\
RFSIM \cite{RFSIM}                      & 0.8411            & 0.8663            & \multicolumn{2}{c|}{0.6642}          & 0.8315          & -0.0204          & 0.2104          & 0.8540 & -0.1787          & 0.2000          & 0.6665          & -0.0191          & 0.1923          \\
SR\_SIM \cite{SRSIM}                   & 0.8660            & 0.8908            & \multicolumn{2}{c|}{0.7020}          & 0.8533          & -0.2053          & 0.1928          & 0.8721          & -0.3162          & 0.1970          & 0.6975          & -0.1451          & 0.1814          \\
FSIM \cite{FSIM}                      & 0.8595            & 0.8826            & \multicolumn{2}{c|}{0.6893}          & 0.8470          & -0.2748          & 0.1912          & 0.8703          & -0.1722          & 0.1724          & 0.6818          & -0.1947          & 0.1750          \\
FSIM \cite{FSIM}    & 0.8848            & 0.8929            & \multicolumn{2}{c|}{0.7103}          & 0.8752 & \textbf{+0.4679} & \textbf{0.1066} & 0.8959          & \textbf{+0.5488} & \textbf{0.0883} & 0.7071 & \textbf{+0.3488} & \textbf{0.1229} \\
GMSD \cite{GMSD}                      & 0.8678            & 0.8899            & \multicolumn{2}{c|}{0.7025}          & 0.8482          & -0.2948          & 0.2088          & 0.8740 & -0.3625          & 0.1979          & 0.6929          & -0.2091          & 0.1924          \\
SFF \cite{SFF}                        & 0.8875   & 0.8979   & \multicolumn{2}{c|}{0.7117} & 0.8519          & +0.1786 & 0.1527          & 0.8713          & +0.0786          & 0.1497          & 0.6893          & +0.1122          & 0.1571          \\
VSI \cite{VSI}                       & \textbf{0.9101}   & \textbf{0.9034}   & \multicolumn{2}{c|}{\textbf{0.7368}} & \textbf{0.8769} & +0.1713          & 0.1437 & \textbf{0.8970} & +0.4875          & 0.1086 & \textbf{0.7220} & +0.1228          & 0.1455          \\
MDSI                       & \textbf{0.9102}   & \textbf{0.9094}   & \multicolumn{2}{c|}{\textbf{0.7391}} & \textbf{0.8881} & \textbf{+0.4548} & \textbf{0.1077} & \textbf{0.9088} & \textbf{+0.6996} & \textbf{0.0815} & \textbf{0.7276} & \textbf{+0.3103} & \textbf{0.1274} \\
MDSI                       & \textbf{0.9145}   & \textbf{0.9219}   & \multicolumn{2}{c|}{\textbf{0.7476}} & \textbf{0.8909} & \textbf{+0.4448} & \textbf{0.1041} & \textbf{0.9107} & \textbf{+0.6922} & \textbf{0.0794} & \textbf{0.7303} & \textbf{+0.2899} & \textbf{0.1246} \\
\hline       
\end{tabular}
\label{results2}
\end{table*}

 
 
 





\section{Experimental results and discussion}
\label{results}

In the experiments, four standard datasets were used. The LIVE dataset \cite{LIVEweb} contains 29 reference images and 779 distorted images of five categories. The TID2008 \cite{TID2008} dataset contains 25 reference images and 1700 distorted images. For each reference image, 17 types of distortions of 4 degrees are available. CSIQ \cite{MAD} is another dataset that consists of 30 reference images; each is distorted using six different types of distortions at four to five levels of distortion. The large TID2013 \cite{TID2013} dataset contains 25 reference images and 3000 distorted images. For each reference image, 24 types of distortions of 5 degrees are available. In addition, 655 contrast distorted images of the CCID2014 dataset \cite{CCID2014} that was introduced in section \ref{Chromaticity} were used in the experiments. We also used the ESPL synthetic image database which contains 25 synthetic images of video games and animated movies. It contains 500 distorted images of 5 categories. Fig. \ref{fig:synthetic} shows an example of a reference and a distorted synthetic image. 






For objective evaluation, four popular evaluation metrics were used in the experiments: the Spearman Rank-order Correlation coefficient (SRC), the Pearson linear Correlation Coefficient (PCC), the Kendall Rank Correlation coefficient (KRC) and the Root Mean Square Error (RMSE). The SRC, PCC, and RMSE metrics measure prediction monotonicity, prediction accuracy, and prediction consistency, respectively. The KRC was used to evaluate the degree of similarity between quality scores and MOSs.


Fourteen state-of-the-art IQA models were chosen for comparison \cite{SSIM, MSSSIM, VIF, GS, MAD, IWSSIM, RFSIM, SRSIM, FSIM, GMSD, SFF, VSI} including the most recent indices in literature \cite{SFF, GMSD, VSI}. It should be noted that the three indices SFF \cite{SFF}, GMSD \cite{GMSD}, and VSI \cite{VSI} have shown superior performance over state-of-the-art indices.        



\begin{figure}[htb]
\scriptsize
\begin{minipage}[b]{0.99\linewidth}
  \centering
  \centerline{\includegraphics[height=2.8cm]{img2.jpg}} 
  \vspace{0.10cm}
\centerline{}
\end{minipage}
\\ \\
\begin{minipage}[b]{.99\linewidth}
  \centering
  \centerline{\includegraphics[height=2.8cm]{img2_gaussian_noise_4.jpg}}
  \vspace{0.10cm}
\centerline{ (Gaussian noise)}
\end{minipage}
\caption{An example of reference  and distorted  image in the ESPL synthetic images database \cite{ESPL}.}
\label{fig:synthetic}
\end{figure}


   


\subsection{Performance comparison}
\label{performance}

In Table \ref{results1}, the overall performance of fourteen IQA models on four benchmark datasets, e.g. TID2008, CSIQ, LIVE, and TID2013, is listed. For each dataset and evaluation metric, the top three IQA models are highlighted. MDSI is 15 times among the top indices, followed by MDSI/SFF (8 times), VSI (5 times), GMSD/MAD (4 times), SR\_SIM (3 times), and FSIM (1 times). MDSI and MDSI reveal the best performances for TID2008 according to all four evaluation metrics. On two datasets of CSIQ and LIVE, SFF and MAD are best performing indices, respectively. On the large dataset of TID2013, the indices MDSI, VSI and MDSI are the best performing indices. 

To provide a conclusion on the overall performance of these indices, weighted\footnote{The dataset size-weighted average is commonly used in the literature \cite{IWSSIM, SFF, GMSD, VSI}} overall performances on the four datasets (6365 images) are listed in Table \ref{results2}. It can be seen that MDSI has the best overall performance on the four datasets, while metrics MDSI, VSI and SFF are the second, third, and forth best, respectively. The overall PCC performance of the MDSI is more than 2\% higher than the competing metric VSI. At the same time, MDSI shows 1.5\% higher average PCC performance on distortion types compared to the VSI. 




\subsection{Performance comparison on contrast distorted images}
\label{Contrast} 

The CCID2014 dataset \cite{CCID2014} is a recently introduced dataset of contrast distorted images. In section \ref{Chromaticity}, two color similarity maps were compared on this dataset. Here, the performance of sixteen IQA models on this dataset is compared. The results are listed in Table \ref{resultsC}. The top three indices for each of the evaluation metrics are shown in boldface. The results show that VIF reveal the best performance among other indices, followed by MDSI, GMSD and SSIM. Only VIF and MDSI are in bold for different evaluation criteria. It can be seen that compared to the previous indices, VSI and FSIM, that use chromaticity similarity maps, the proposed indices MDSI and MDSI provide more accurate quality predictions on contrast distorted images.           



\begin{table}[htb]
\scriptsize
\centering
\caption{Performance comparison of the proposed IQA models MDSI and MDSI, and fourteen popular/competing indices on the contrast distorted dataset CCID2014 \cite{CCID2014}.}
\label{resultsC}
\begin{tabular}{c|cccc}
\hline
\multirow{2}{*}{IQA model}    & \multicolumn{4}{c}{CCID2014 (655 images)}                                                                                                             \\ \cline{2-5} 
                              & SRC                                 & PCC                                 & KRC                                 & RMSE                                \\ \hline
PSNR                          & 0.6743                              & 0.4112                              & 0.4834                              & 0.5961                              \\
SSIM \cite{SSIM}                          & \textbf{0.8174}                     & 0.8308                              & \textbf{0.6106}                     & 0.3640                              \\
MSSSIM \cite{MSSSIM}                       & 0.7770                              & 0.8278                              & 0.5845                              & 0.3668                              \\
VIF \cite{VIF}                           & \textbf{0.8349}                     & \textbf{0.8588}                     & \textbf{0.6419}                     & \textbf{0.3350}                     \\
GS \cite{GS}                            & 0.7768                              & 0.8073                              & 0.5711                              & 0.3859                              \\
MAD \cite{MAD}                           & 0.7451                              & 0.7516                              & 0.5490                              & 0.4313                              \\
IWSSIM \cite{IWSSIM}                        & 0.7811                              & 0.8342                              & 0.5898                              & 0.3606                              \\
RFSIM \cite{RFSIM}                         & 0.7304                              & 0.7936                              & 0.5368                              & 0.3979                              \\
SR\_SIM \cite{SRSIM}                       & 0.7363                              & 0.7834                              & 0.5372                              & 0.4064                              \\
FSIM \cite{FSIM}                          & 0.7658                              & 0.8201                              & 0.5707                              & 0.3741                              \\
FSIMc \cite{FSIM}                         & 0.7657                              & 0.8204                              & 0.5707                              & 0.3739                              \\
GMSD \cite{GMSD}                          & 0.8077                              & \textbf{0.8521}                     & 0.6100                              & \textbf{0.3422}                     \\
SFF \cite{SFF}                           & 0.6859                              & 0.7575                              & 0.5012                              & 0.4269                              \\
VSI \cite{VSI}                           & 0.7734                              & 0.8209                              & 0.5735                              & 0.3734                              \\
MDSI                          & 0.7866                              & 0.8313                              & 0.5906                              & 0.3634                              \\
MDSI & \textbf{0.8120} & \textbf{0.8530} & \textbf{0.6150} & \textbf{0.3413} \\ \hline
\end{tabular}
\end{table}




\subsection{Performance comparison on synthetic dataset}
\label{synthetic}

We evaluated the proposed IQA model on the recently introduced dataset of synthetic images \cite{ESPL}. In Table \ref{synthetic}, the prediction accuracy of fifteen IQA indices is evaluated according to the SRC, PCC, KRC, and RMSE metrics. The top three IQA indices for each evaluation metric are in boldface. It can be seen that MDSI and FSIM are among the top three indices on the basis of the overall performance for each of the four evaluation criteria. Statistical significance test shows that MDSI is significantly better than the other indices on ESPL dataset. On this dataset, MDSI has competitive performance in comparison with the best performing indices except for MDSI. 

The performance of MDSI and MDSI for the five available distortions in this dataset is not among the top three indices. It should be noted that the quality predictions by MDSI and MDSI are not far from these best performing indices for distortion types. The performance of the proposed metrics MDSI and MDSI on the synthetic dataset again proves their effectiveness and reliability.  


\begin{table}[htb]
\caption{Performance comparison of the proposed IQA models MDSI and MDSI, and fourteen popular/competing indices on the ESPL dataset \cite{ESPL}.}
\scriptsize
\centering
\begin{tabular}{c|cccc|cc}
\hline
\multirow{2}{*}{IQA model} & \multicolumn{4}{c|}{ESPL (500 synthetic images)}                                & \multicolumn{2}{c}{avg Distortions}                 \\ \cline{2-7} 
                           & SRC             & PCC             & KRC             & RMSE            & SRC             & PCC             \\ \hline
PSNR                       & 0.5802          & 0.5898          & 0.4010          & 11.2073         & 0.6842          & 0.7074          \\
SSIM \cite{SSIM}                       & 0.7963          & 0.8007          & 0.5893          & 8.3142          & 0.8290          & 0.8458          \\
MSSSIM \cite{MSSSIM}                     & 0.7247          & 0.7322          & 0.5208          & 9.4519          & 0.7790          & 0.8075          \\
VIF \cite{VIF}                       & 0.7488          & 0.7423          & 0.5565          & 9.2985          & 0.7943          & 0.8280          \\
GS \cite{GS}                         & 0.8727          & 0.8687          & 0.6803          & 6.8751          & 0.8412          & 0.8529          \\
MAD \cite{MAD}                        & 0.8624          & 0.8677 & 0.6720          & 6.8985          & \textbf{0.8696} & \textbf{0.8843} \\
IWSSIM \cite{IWSSIM}                     & 0.8270          & 0.8300          & 0.6221          & 7.7404          & \textbf{0.8651} & \textbf{0.8861} \\
RFSIM \cite{RFSIM}                      & 0.8246          & 0.8239 & 0.6231          & 7.8651          & 0.8268          & 0.8399 \\
SR\_SIM \cite{SRSIM}                    & \textbf{0.8802} & 0.8732          & \textbf{0.6932} & 6.7646 & \textbf{0.8667} & \textbf{0.8806} \\
FSIM \cite{FSIM}                       & 0.8760          & 0.8725          & 0.6848          & 6.7809          & 0.8444          & 0.8579  \\
FSIM \cite{FSIM}     & \textbf{0.8766} & \textbf{0.8738}          & \textbf{0.6853} & \textbf{6.7482} & 0.8447          & 0.8584  \\
GMSD \cite{GMSD}                       & 0.8209          & 0.8234          & 0.6178          & 7.8753          & 0.8433          & 0.8648    \\
SFF \cite{SFF}                        & 0.8127          & 0.8179          & 0.6127          & 7.9844          & 0.8015          & 0.8187     \\
VSI \cite{VSI}                        & 0.8717          & 0.8726          & 0.6765          & 6.7791          & 0.8429          & 0.8604      \\
MDSI                       & \textbf{0.8800} & \textbf{0.8791} & \textbf{0.6885} & \textbf{6.6146} & 0.8472          & 0.8620         \\
MDSI                       & 0.8749 & \textbf{0.8748} & 0.6816 & \textbf{6.7241} & 0.8417          & 0.8589          \\
\hline
\end{tabular}
\label{synthetic}
\end{table}




\begin{figure}[htb]
\scriptsize
\begin{minipage}[b]{0.49\linewidth}
  \centering
  \centerline{\includegraphics[height=3.5cm]{sizedFSIMcplot.png}} 
\end{minipage}
\begin{minipage}[b]{.49\linewidth}
  \centering
  \centerline{\includegraphics[height=3.5cm]{sizedGMSDplot.png}}
\end{minipage}
\\ \\
\begin{minipage}[b]{0.49\linewidth}
  \centering
  \centerline{\includegraphics[height=3.5cm]{sizedSFFplot.png}}
\end{minipage}
\begin{minipage}[b]{0.49\linewidth}
  \centering
  \centerline{\includegraphics[height=3.5cm]{sizedVSIplot.png}} 
\end{minipage}
\\ \\
\begin{minipage}[b]{.49\linewidth}
  \centering
  \centerline{\includegraphics[height=3.5cm]{sizedMDSIplot.png}}
\end{minipage}
\begin{minipage}[b]{0.49\linewidth}
  \centering
  \centerline{\includegraphics[height=3.5cm]{sizedMDSIplusplot.png}}
\end{minipage}
\caption{Scatter plots of quality scores against the subjective MOS on the TID2013 dataset for six IQA models.}
\label{scatter}
\end{figure}



\subsection{Visual comparison and statistical evaluation}
\label{significance}



\begin{table*}[htb]
\centering{
\tiny
\caption{The results of statistical significance test for ten IQA models on six datasets. The result of the F-test is equal to +1 if a metric is significantly better than another metric, it is equal to -1 if that metric is statistically inferior to another metric, and the result is equal to 0 if two metrics are statistically indistinguishable. The cumulative sum of individual tests for each metric is listed in the last column.}
\label{Ftest}
\begin{tabular}{|l|l|c|c|c|c|c|c|c|c|c|c||c|}
\hline
\multicolumn{2}{|c|}{TID2008} & 1           & 2           & 3           & 4           & 5           & 6           & 7           & 8           & 9           & 10 & sum         \\ \hline
1           & MSSSIM          & -           & \textbf{+1} & \textbf{+1} & -1          & -1          & -1          & -1          & -1          & -1          & -1 & -5          \\ \hline
2           & VIF             & -1          & -           & -1          & -1          & -1          & -1          & -1          & -1          & -1          & -1 & -9          \\ \hline
3           & MAD             & \textbf{+1} & \textbf{+1} & -           & -1          & -1          & -1          & -1          & -1          & -1          & -1 & -7          \\ \hline
4           & SR\_SIM         & \textbf{+1} & \textbf{+1} & \textbf{+1} & -           & \textbf{+1} & \textbf{+1} & \textbf{+1} & \textbf{+1} & -1          & -1 & \textbf{+5} \\ \hline
5           & FSIM           & \textbf{+1} & \textbf{+1} & \textbf{+1} & -1          & -           & -1          & -1          & 0           & -1          & -1 & -2          \\ \hline
6           & GMSD            & \textbf{+1} & \textbf{+1} & \textbf{+1} & -1          & \textbf{+1} & -           & -1          & \textbf{+1} & -1          & -1 & +1          \\ \hline
7           & SFF             & \textbf{+1} & \textbf{+1} & \textbf{+1} & -1          & \textbf{+1} & \textbf{+1} & -           & \textbf{+1} & -1          & -1 & +3          \\ \hline
8           & VSI             & \textbf{+1} & \textbf{+1} & \textbf{+1} & -1          & 0           & -1          & -1          & -           & -1          & -1 & -2          \\ \hline
9           & MDSI            & \textbf{+1} & \textbf{+1} & \textbf{+1} & \textbf{+1} & \textbf{+1} & \textbf{+1} & \textbf{+1} & \textbf{+1} & -           & -1 & \textbf{+7} \\ \hline
10          & MDSI           & \textbf{+1} & \textbf{+1} & \textbf{+1} & \textbf{+1} & \textbf{+1} & \textbf{+1} & \textbf{+1} & \textbf{+1} & \textbf{+1} & -  & \textbf{+9} \\ \hline
\end{tabular}
\hspace*{2 mm}
\begin{tabular}{|l|l|c|c|c|c|c|c|c|c|c|c||c|}
\hline
\multicolumn{2}{|c|}{CSIQ} & 1           & 2           & 3           & 4           & 5           & 6           & 7  & 8           & 9           & 10          & sum         \\ \hline
1         & MSSSIM         & -           & -1          & -1          & -1          & -1          & -1          & -1 & -1          & -1          & -1          & -9          \\ \hline
2         & VIF            & \textbf{+1} & \textbf{-}  & -1          & \textbf{+1} & \textbf{+1} & -1          & -1 & 0           & -1          & -1          & -2          \\ \hline
3         & MAD            & \textbf{+1} & \textbf{+1} & -           & \textbf{+1} & \textbf{+1} & -1          & -1 & \textbf{+1} & \textbf{+1} & 0           & \textbf{+4} \\ \hline
4         & SR\_SIM        & \textbf{+1} & -1          & -1          & -           & \textbf{+1} & -1          & -1 & -1          & -1          & -1          & -5          \\ \hline
5         & FSIM          & \textbf{+1} & -1          & -1          & -1          & -           & -1          & -1 & -1          & -1          & -1          & -7          \\ \hline
6         & GMSD           & \textbf{+1} & \textbf{+1} & \textbf{+1} & \textbf{+1} & \textbf{+1} & -           & -1 & \textbf{+1} & \textbf{+1} & \textbf{+1} & \textbf{+7} \\ \hline
7         & SFF            & \textbf{+1} & \textbf{+1} & \textbf{+1} & \textbf{+1} & \textbf{+1} & \textbf{+1} & -  & \textbf{+1} & \textbf{+1} & \textbf{+1} & \textbf{+9} \\ \hline
8         & VSI            & \textbf{+1} & 0           & -1          & \textbf{+1} & \textbf{+1} & -1          & -1 & -           & -1          & -1          & -2          \\ \hline
9         & MDSI           & \textbf{+1} & \textbf{+1} & -1          & \textbf{+1} & \textbf{+1} & -1          & -1 & \textbf{+1} & -           & -1          & +1          \\ \hline
10        & MDSI          & \textbf{+1} & \textbf{+1} & 0           & \textbf{+1} & \textbf{+1} & -1          & -1 & \textbf{+1} & \textbf{+1} & -           & \textbf{+4} \\ \hline
\end{tabular}
\newline
\vspace*{2 mm}
\newline
\begin{tabular}{|l|l|c|c|c|c|c|c|c|c|c|c||c|}
\hline
\multicolumn{2}{|c|}{LIVE} & 1           & 2           & 3           & 4           & 5           & 6           & 7           & 8           & 9           & 10          & sum         \\ \hline
1         & MSSSIM         & -           & -1          & -1          & -1          & -1          & -1          & -1          & 0           & -1          & -1          & -8          \\ \hline
2         & VIF            & \textbf{+1} & \textbf{-}  & -1          & \textbf{+1} & 0           & 0           & -1          & \textbf{+1} & \textbf{+1} & 0           & +2          \\ \hline
3         & MAD            & \textbf{+1} & \textbf{+1} & -           & \textbf{+1} & \textbf{+1} & \textbf{+1} & \textbf{+1} & \textbf{+1} & \textbf{+1} & \textbf{+1} & \textbf{+9} \\ \hline
4         & SR\_SIM        & \textbf{+1} & -1          & -1          & -           & -1          & -1          & -1          & \textbf{+1} & 0           & -1          & -4          \\ \hline
5         & FSIM          & \textbf{+1} & 0           & -1          & \textbf{+1} & -           & 0           & 0           & \textbf{+1} & \textbf{+1} & 0           & +3          \\ \hline
6         & GMSD           & \textbf{+1} & 0           & -1          & \textbf{+1} & 0           & -           & -1          & \textbf{+1} & \textbf{+1} & -1          & +1          \\ \hline
7         & SFF            & \textbf{+1} & \textbf{+1} & \textbf{-1} & \textbf{+1} & 0           & \textbf{+1} & -           & \textbf{+1} & \textbf{+1} & 0           & \textbf{+5} \\ \hline
8         & VSI            & 0           & -1          & -1          & -1          & -1          & -1          & -1          & -           & -1          & -1          & -8          \\ \hline
9         & MDSI           & \textbf{+1} & -1          & -1          & 0           & -1          & -1          & -1          & \textbf{+1} & -           & -1          & -4          \\ \hline
10        & MDSI          & \textbf{+1} & 0           & -1          & \textbf{+1} & 0 & \textbf{+1} & 0           & \textbf{+1} & \textbf{+1} & -           & \textbf{+4} \\ \hline
\end{tabular}
\hspace*{2 mm}
\begin{tabular}{|l|l|c|c|c|c|c|c|c|c|c|c||c|}
\hline
\multicolumn{2}{|c|}{TID2013} & 1           & 2           & 3           & 4           & 5           & 6           & 7           & 8           & 9           & 10 & sum         \\ \hline
1         & MSSSIM         & -           & \textbf{+1} & \textbf{+1} & -1          & -1          & -1          & -1          & -1          & -1          & -1 & -5          \\ \hline
2         & VIF            & -1          & \textbf{-}  & -1          & -1          & -1          & -1          & -1          & -1          & -1          & -1 & -9          \\ \hline
3         & MAD            & -1          & \textbf{+1} & -           & -1          & -1          & -1          & -1          & -1          & -1          & -1 & -7          \\ \hline
4         & SR\_SIM        & \textbf{+1} & \textbf{+1} & \textbf{+1} & -           & -1          & \textbf{+1} & -1          & -1          & -1          & -1 & -1          \\ \hline
5         & FSIM          & \textbf{+1} & \textbf{+1} & \textbf{+1} & \textbf{+1} & -           & \textbf{+1} & \textbf{+1} & -1          & -1          & -1 & +3          \\ \hline
6         & GMSD           & \textbf{+1} & \textbf{+1} & \textbf{+1} & -1          & -1          & -           & -1          & -1          & -1          & -1 & -3          \\ \hline
7         & SFF            & \textbf{+1} & \textbf{+1} & \textbf{+1} & \textbf{+1} & -1          & \textbf{+1} & -           & -1          & -1          & -1 & +1          \\ \hline
8         & VSI            & \textbf{+1} & \textbf{+1} & \textbf{+1} & \textbf{+1} & \textbf{+1} & \textbf{+1} & \textbf{+1} & -           & \textbf{+1} & -1 & \textbf{+7} \\ \hline
9         & MDSI           & \textbf{+1} & \textbf{+1} & \textbf{+1} & \textbf{+1} & \textbf{+1} & \textbf{+1} & \textbf{+1} & -1          & -           & -1 & \textbf{+5} \\ \hline
10        & MDSI          & \textbf{+1} & \textbf{+1} & \textbf{+1} & \textbf{+1} & \textbf{+1} & \textbf{+1} & \textbf{+1} & \textbf{+1} & \textbf{+1} & -  & \textbf{+9} \\ \hline
\end{tabular}
\newline
\vspace*{2 mm}
\newline
\begin{tabular}{|l|l|c|c|c|c|c|c|c|c|c|c||c|}
\hline
\multicolumn{2}{|c|}{CCID2014} & 1           & 2  & 3           & 4           & 5           & 6           & 7           & 8           & 9           & 10          & sum         \\ \hline
1           & MSSSIM           & -           & -1 & \textbf{+1} & \textbf{+1} & \textbf{+1} & -1          & \textbf{+1} & \textbf{+1} & -1          & -1          & +1          \\ \hline
2           & VIF              & \textbf{+1} & -  & \textbf{+1} & \textbf{+1} & \textbf{+1} & \textbf{+1} & \textbf{+1} & \textbf{+1} & \textbf{+1} & \textbf{+1} & \textbf{+9} \\ \hline
3           & MAD              & -1          & -1 & -           & -1          & -1          & -1          & -1          & -1          & -1          & -1          & -9          \\ \hline
4           & SR\_SIM          & -1          & -1 & \textbf{+1} & -           & -1          & -1          & \textbf{+1} & -1          & -1          & -1          & -5          \\ \hline
5           & FSIM            & -1          & -1 & \textbf{+1} & \textbf{+1} & -           & -1          & \textbf{+1} & 0           & -1          & -1          & -2          \\ \hline
6           & GMSD             & \textbf{+1} & -1 & \textbf{+1} & \textbf{+1} & \textbf{+1} & -           & \textbf{+1} & \textbf{+1} & \textbf{+1} & 0           & \textbf{+6} \\ \hline
7           & SFF              & -1          & -1 & \textbf{+1} & -1          & -1          & -1          & -           & -1          & -1          & -1          & -7          \\ \hline
8           & VSI              & -1          & -1 & \textbf{+1} & \textbf{+1} & 0           & -1          & \textbf{+1} & -           & -1          & -1          & -2          \\ \hline
9           & MDSI             & \textbf{+1} & -1 & \textbf{+1} & \textbf{+1} & \textbf{+1} & -1          & \textbf{+1} & \textbf{+1} & -           & -1          & +3          \\ \hline
10          & MDSI            & \textbf{+1} & -1 & \textbf{+1} & \textbf{+1} & \textbf{+1} & 0           & \textbf{+1} & \textbf{+1} & \textbf{+1} & -           & \textbf{+6} \\ \hline
\end{tabular}
\hspace*{2 mm}
\begin{tabular}{|l|l|c|c|c|c|c|c|c|c|c|c||c|}
\hline
\multicolumn{2}{|c|}{ESPL} & 1           & 2           & 3           & 4           & 5           & 6           & 7           & 8           & 9  & 10          & sum         \\ \hline
1         & MSSSIM         & -           & -1          & -1          & -1          & -1          & -1          & -1          & -1          & -1 & -1          & -9          \\ \hline
2         & VIF            & \textbf{+1} & -           & -1          & -1          & -1          & -1          & -1          & -1          & -1 & -1          & -7          \\ \hline
3         & MAD            & \textbf{+1} & \textbf{+1} & -           & -1          & -1          & \textbf{+1} & \textbf{+1} & -1          & -1 & -1          & -1          \\ \hline
4         & SR\_SIM        & \textbf{+1} & \textbf{+1} & \textbf{+1} & -           & 0           & \textbf{+1} & \textbf{+1} & \textbf{+1} & -1 & 0           & \textbf{+5} \\ \hline
5         & FSIM          & \textbf{+1} & \textbf{+1} & \textbf{+1} & \textbf{0}  & -           & \textbf{+1} & \textbf{+1} & \textbf{+1} & -1 & 0           & \textbf{+5} \\ \hline
6         & GMSD           & \textbf{+1} & \textbf{+1} & -1          & -1          & -1          & -           & \textbf{+1} & -1          & -1 & -1          & -3          \\ \hline
7         & SFF            & \textbf{+1} & \textbf{+1} & -1          & -1          & -1          & -1          & -           & -1          & -1 & -1          & -5          \\ \hline
8         & VSI            & \textbf{+1} & \textbf{+1} & \textbf{+1} & \textbf{-1} & -1          & \textbf{+1} & \textbf{+1} & -           & -1 & -1          & +1          \\ \hline
9         & MDSI           & \textbf{+1} & \textbf{+1} & \textbf{+1} & \textbf{+1} & \textbf{+1} & \textbf{+1} & \textbf{+1} & \textbf{+1} & -  & \textbf{+1} & \textbf{+9} \\ \hline
10        & MDSI          & \textbf{+1} & \textbf{+1} & \textbf{+1} & 0           & 0           & \textbf{+1} & \textbf{+1} & \textbf{+1} & -1 & -           & \textbf{+5} \\ \hline
\end{tabular} }
\end{table*}



To get a visual comparison, the scatter plots of the proposed IQA models MDSI and MDSI, and four competing IQA models on the TID2013 dataset are shown in Fig. \ref{scatter}. The logistic function suggested in \cite{statistical2006} was used to fit a curve on each plot:
                    
where , , ,  and  are fitting parameters computed by minimizing the mean square error between quality predictions  and subjective scores MOS. The linearity advantage of MDSI and MDSI plots is obvious among other models thanks to the use of the power pooling. 









The reported results in Tables \ref{results1}, \ref{resultsC}, and \ref{synthetic} show the difference between different IQA models. As suggested in \cite{LIVE}, we use F-test to decide weather a metric is statistically superior to another index. The F-test is based on the residuals between the quality scores given by an IQA model after applying nonlinear mapping of equation (\ref{equ:REG}), and the mean subjective scores MOS. The ratio of variance between residual errors of an IQA model to another model at 95\% significance level is used by F-test. The result of the test is equal to 1 if we can reject the null hypothesis and 0 otherwise. The results of F-test on four datasets are listed in Table \ref{Ftest}. In this Table, +1/-1 indicate that corresponding index is statistically superior/inferior to the other index being compared to. If the difference between two indices is not significant, the result is shown by 0. 










From the results we can see that MDSI is significantly better than the other indices on TID2008 and TID2013 datasets. Therefore, its sum value in the last column is +9 for these two datasets. MDSI also performs well on these two datasets. VSI is statistically superior to the other indices on the TID2013 dataset except for MDSI. On the LIVE dataset, only MAD is significantly better than MDSI. Also on the CSIQ dataset, only SFF and GMSD perform significantly better than MDSI. On the CCID2014 dataset, VIF is significantly better than the other indices, while the statistically indistinguishable indices MDSI and GMSD show promising results. On the ESPL dataset, MDSI is statistically superior to the other indices. The indices MDSI, FSIM and SR\_SIM are not statistically distinguishable. Considering all six datasets used in this experiment, with a minimum sum value of +4, the proposed index MDSI performs very well in comparison with the other indices. After MDSI, the two indices GMSD and MDSI, with minimum sum values of -3 and -4 respectively show better performance in comparison with the other indices on the six datasets. We can simply add the six sum values of each metric for the six datasets to have an overall comparison based on the statistical significance test. This score indicates how many times a metric is statistically superior to the other metrics. The results show that MDSI is the best performing index by a score of +37 (out of maximum +54), followed by MDSI (+23), GMSD (+9), SFF (+6), FSIM (0), SR\_SIM (-5), VSI (-6), MAD (-11), VIF (-16), and MSSSIM (-35). The results based on the statistical significance test verify that unlike other IQA models, the proposed metric MDSI is always among the best performing indices on different datasets.                        










\begin{table*}[htb]
\caption{Performance comparison of the thirteen IQA models on individual distortion types of four datasets in terms of SRC}
\scriptsize
\centering
\begin{tabular}{ccccccccccccccc}
\hline
                      &       & PSNR            & SSIM            & MSSSIM          & VIF             & GS              & MAD             & SR\_SIM         & FSIM           & GMSD            & SFF             & VSI             & MDSI             & MDSI                                \\ \hline
                      & AWN   & 0.9291          & 0.8671          & 0.8646          & 0.8989          & 0.9064          & 0.8842          & 0.9253          & 0.9101          & \textbf{0.9462} & 0.9066          & 0.9460 & \textbf{0.9495} & \textbf{0.9464}                     \\
                      & ANC   & \textbf{0.8981} & 0.7726          & 0.7730          & 0.8285          & 0.8175          & 0.8018          & 0.8570          & 0.8537          & 0.8684          & 0.8166          & 0.8705 & \textbf{0.8751} & \textbf{0.8723}                     \\
                      & SCN   & 0.9200          & 0.8515          & 0.8544          & 0.8829          & 0.9158          & 0.8911          & 0.9225          & 0.8900          & 0.9350 & 0.8982          & \textbf{0.9367} & \textbf{0.9443} & \textbf{0.9466}                     \\
                      & MN    & \textbf{0.8323} & 0.7767          & 0.8073          & \textbf{0.8450} & 0.7293          & 0.7373          & 0.7860          & 0.8094          & 0.7075          & \textbf{0.8185} & 0.7697          & 0.7959 & 0.7974                              \\
                      & HFN   & 0.9140          & 0.8634          & 0.8604          & 0.8969          & 0.8869          & 0.8876          & 0.9132          & 0.9040          & \textbf{0.9162} & 0.8977          & \textbf{0.9200} & \textbf{0.9183} & 0.9140                     \\
                      & IN    & \textbf{0.8968} & 0.7503          & 0.7629          & 0.8537 & 0.7965          & 0.2769          & 0.8277          & 0.8251          & 0.7637          & 0.7871          & \textbf{0.8741} & 0.8458 & \textbf{0.8570}                              \\
                      & QN    & 0.8808 & 0.8657          & 0.8706          & 0.7854          & 0.8841          & 0.8514          & 0.8502          & 0.8807          & \textbf{0.9049} & 0.8607          & 0.8748          & \textbf{0.8902} & \textbf{0.8870}                     \\
                      & GB    & 0.9149          & 0.9668 & \textbf{0.9673}          & 0.9650          & \textbf{0.9689} & 0.9319          & 0.9620          & 0.9551          & 0.9113          & \textbf{0.9675} & 0.9612          & 0.9453 & 0.9542                              \\
                      & DEN   & 0.9480 & 0.9254          & 0.9268          & 0.8911          & 0.9432          & 0.9253          & 0.9403          & 0.9330          & \textbf{0.9525}          & 0.9091          & 0.9484 & \textbf{0.9535} & \textbf{0.9513}                     \\
                      & JPEG  & 0.9189          & 0.9200          & 0.9265          & 0.9192          & 0.9284          & 0.9219          & 0.9386          & 0.9339          & \textbf{0.9507} & 0.9273          & \textbf{0.9541} & \textbf{0.9535} & 0.9475                     \\
                      & JP2K  & 0.8840          & 0.9468          & 0.9504          & 0.9516          & 0.9602          & 0.9511          & \textbf{0.9673} & 0.9589          & 0.9657          & 0.9571          & \textbf{0.9706} & \textbf{0.9688} & 0.9643                     \\
TID                   & JGTE  & 0.7685          & 0.8493          & 0.8475          & 0.8411          & 0.8512          & 0.8283          & 0.8543          & 0.8610          & 0.8403          & 0.8831 & \textbf{0.9216} & \multicolumn{1}{l}{\textbf{0.8958}} & \textbf{0.8991} \\
2013                  & J2TE  & 0.8883          & 0.8828          & 0.8889          & 0.8761          & \textbf{0.9182} & 0.8788          & 0.9166          & 0.8919          & 0.9136          & 0.8708          & \textbf{0.9228} & \multicolumn{1}{l}{\textbf{0.9179}} & 0.9157 \\
                      & NEPN  & 0.6863          & 0.7821          & 0.7968          & 0.7720          & 0.8130 & \textbf{0.8315} & 0.7975          & 0.7937          & \textbf{0.8140} & 0.7668          & 0.8060          & \multicolumn{1}{l}{0.8071} & \textbf{0.8159}          \\
                      & Block & 0.1552          & 0.5720          & 0.4801          & 0.5306          & \textbf{0.6418} & 0.2810          & 0.4731          & 0.5532          & \textbf{0.6625} & 0.1786          & 0.1713          & \multicolumn{1}{l}{0.5986} & \textbf{0.6949} \\
                      & MS    & 0.7671          & \textbf{0.7752}          & \textbf{0.7906} & 0.6272          & \textbf{0.7875} & 0.6457          & 0.6576          & 0.7487          & 0.7351          & 0.6654          & 0.7700          & \multicolumn{1}{l}{0.7696} & 0.7639          \\
                      & CTC   & 0.4400          & 0.3775          & 0.4634          & \textbf{0.8386} & \textbf{0.4857} & 0.1970          & 0.4705          & 0.4679          & 0.3235          & 0.4691          & \textbf{0.4754} & \multicolumn{1}{l}{0.4548} & 0.4448          \\
                      & CCS   & 0.0766          & -0.4141         & -0.4099         & -0.3135         & -0.3578         & -0.0575         &  -0.2053         & \textbf{0.8359} & -0.2948         & \textbf{0.8269} & \textbf{0.8100} & 0.7947 & 0.7991                              \\
                      & MGN   & \textbf{0.8905} & 0.7803          & 0.7786          & 0.8468          & 0.8348          & 0.8408          & 0.8778          & 0.8569          & 0.8886          & 0.8434          & \textbf{0.9117} & \textbf{0.9030} & 0.8823                     \\
                      & CN    & 0.8411          & 0.8566          & 0.8528          & 0.8947          & 0.9124          & 0.9064          & \textbf{0.9263} & 0.9135          & \textbf{0.9298} & 0.9007          & 0.9243          & \textbf{0.9274} & 0.9203                     \\
                      & LCNI  & 0.9145          & 0.9057          & 0.9068          & 0.9204          & 0.9563          & 0.9443          & \textbf{0.9608} & 0.9485          & \textbf{0.9629} & 0.9262          & 0.9564          & \textbf{0.9594} & 0.9568                     \\
                      & ICQD  & \textbf{0.9269} & 0.8542          & 0.8555          & 0.8414          & 0.8973          & 0.8745          & 0.8803          & 0.8815          & 0.9102 & 0.8795          & 0.8839          & \textbf{0.9163} & \textbf{0.9114}                     \\
                      & CHA   & \textbf{0.8872} & 0.8775          & 0.8784          & 0.8848 & 0.8823          & 0.8308          & 0.8754          & \textbf{0.8925}          & 0.8530          & 0.8789          & \textbf{0.8906} & 0.8820 & 0.8847                              \\
                      & SSR   & 0.9042          & 0.9461          & 0.9483          & 0.9353          & \textbf{0.9668} & 0.9567          & 0.9614          & 0.9576          & \textbf{0.9683} & 0.9522          & 0.9628          & \textbf{0.9686} & 0.9653                     \\ \hline
                      & AWN   & 0.9070          & 0.8107          & 0.8086          & 0.8797          & 0.8606          & 0.8388          & 0.8990          & 0.8758          & 0.9180 & 0.8731          & \textbf{0.9229} & \textbf{0.9321} & \textbf{0.9250}                     \\
                      & ANC   & 0.8995 & 0.8029          & 0.8054          & 0.8757          & 0.8091          & 0.8258          & 0.8954          & 0.8931          & 0.8977          & 0.8626          & \textbf{0.9118} & \textbf{0.9208} & \textbf{0.9125}                     \\
                      & SCN   & 0.9170 & 0.8145          & 0.8209          & 0.8698          & 0.8941          & 0.8678          & 0.9083          & 0.8711          & 0.9132          & 0.8939          & \textbf{0.9296} & \textbf{0.9319} & \textbf{0.9461}                     \\
                      & MN    & \textbf{0.8515} & 0.7795          & 0.8107          & \textbf{0.8698} & 0.7452          & 0.7336          & 0.7869          & 0.8264          & 0.7087          & \textbf{0.8365} & 0.7734          & 0.8104 & 0.8156                              \\
                      & HFN   & \textbf{0.9270} & 0.8729          & 0.8694          & 0.9075          & 0.8945          & 0.8864          & 0.9197          & 0.9156          & 0.9189          & 0.9119          & \textbf{0.9253} & \textbf{0.9207} & 0.9156                     \\
                      & IN    & 0.8724 & 0.6732          & 0.6907          & \textbf{0.8327} & 0.7235          & 0.0650          & 0.7667          & 0.7719          & 0.6611          & 0.7484          & \textbf{0.8298} & 0.7987 & \textbf{0.8262}                              \\
                      & QN    & 0.8696          & 0.8531          & 0.8589          & 0.7970          & \textbf{0.8800} & 0.8160          & 0.8364          & 0.8726          & \textbf{0.8875} & 0.8448          & 0.8731          & \textbf{0.8805} & 0.8775                     \\
TID                   & GB    & 0.8697          & 0.9544          & \textbf{0.9563}          & 0.9540          & \textbf{0.9600} & 0.9197          & 0.9549          & 0.9472          & 0.8968          & \textbf{0.9624} & 0.9529          & 0.9361 & 0.9469                              \\
2008                  & DEN   & 0.9416          & 0.9530          & 0.9582          & 0.9161          & \textbf{0.9725} & 0.9434          & 0.9668          & 0.9618          & \textbf{0.9752} & 0.9383          & 0.9693          & \textbf{0.9740} & 0.9716                     \\
                      & JPEG  & 0.8717          & 0.9252          & 0.9322          & 0.9168          & 0.9393          & 0.9275          & 0.9394          & 0.9294          & 0.9525 & 0.9323          & \textbf{0.9616} & \textbf{0.9573} & \textbf{0.9537}                     \\
                      & JP2K  & 0.8132          & 0.9625          & 0.9700          & 0.9709          & 0.9758          & 0.9707          & \textbf{0.9807} & 0.9780          & 0.9795          & 0.9772          & \textbf{0.9848} & \textbf{0.9824} & 0.9802                     \\
                      & JGTE  & 0.7516          & 0.8678          & 0.8681          & 0.8585          & 0.8790          & 0.8661          & \textbf{0.8881} & 0.8756          & 0.8621          & 0.8567          & \textbf{0.9160} & \textbf{0.8867} & 0.8863                     \\
                      & J2TE  & 0.8309          & 0.8577          & 0.8606          & 0.8501          & \textbf{0.8936} & 0.8394          & 0.8903          & 0.8555          & 0.8825          & 0.8386          & \textbf{0.8942} & 0.8909 & \textbf{0.8914}                     \\
                      & NEPN  & 0.5815          & 0.7107          & 0.7377          & 0.7619          & 0.7386          & \textbf{0.8287} & \textbf{0.7670}          & 0.7514          & 0.7601          & 0.6970          & \textbf{0.7699} & 0.7436 & 0.7584                              \\
                      & Block & 0.6193          & 0.8462          & 0.7546          & 0.8324          & \textbf{0.8862} & 0.7970          & 0.7787          & 0.8464          & \textbf{0.8967} & 0.5369          & 0.6295          & 0.8618 & \textbf{0.9037}                     \\
                      & MS    & 0.6957          & \textbf{0.7231} & \textbf{0.7338} & 0.5096          & \textbf{0.7190} & 0.5161          & 0.5727          & 0.6554          & 0.6486          & 0.5225          & 0.6714          & 0.6770 & 0.6679                              \\
                      & CTC   & 0.5859          & 0.5246          & 0.6381          & \textbf{0.8188} & \textbf{0.6691} & 0.2723          & 0.6483          & 0.6510          & 0.4659          & 0.6461          & \textbf{0.6557} & 0.6344 & 0.6240                              \\ \hline
                      & AWN  & 0.9363          & 0.8974          & 0.9471          & 0.9575          & 0.9440          & 0.9541          & 0.9628          & 0.9359          & \textbf{0.9676} & 0.9467          & 0.9636 & \textbf{0.9668} & \textbf{0.9669}                     \\
                      & JPEG  & 0.8881          & 0.9546          & 0.9634          & \textbf{0.9705} & 0.9632          & 0.9615          & \textbf{0.9671} & \textbf{0.9664} & 0.9653          & 0.9641          & 0.9618          & 0.9644 & 0.9593                              \\
\multirow{2}{*}{CSIQ} & JP2K  & 0.9362          & 0.9606          & 0.9684          & 0.9672          & 0.9648          & \textbf{0.9752} & \textbf{0.9773} & 0.9704          & 0.9718          & \textbf{0.9763} & 0.9694          & 0.9716 & 0.9740                              \\
                      & GPN  & 0.9339          & 0.8922          & 0.9331          & 0.9511          & 0.9387          & 0.9570 & 0.9520          & 0.9370          & 0.9503          & 0.9550          & \textbf{0.9638} & \textbf{0.9591} & \textbf{0.9653}                     \\
                      & GB    & 0.9291          & 0.9609          & 0.9712          & \textbf{0.9745}          & 0.9589          & 0.9682          & \textbf{0.9767} & 0.9729          & 0.9713          & \textbf{0.9751} & 0.9679          & 0.9694 & 0.9735                              \\
                      & GCD   & 0.8621          & 0.7922          & \textbf{0.9526} & 0.9345          & 0.9354          & 0.9207          & \textbf{0.9528}          & 0.9438          & 0.9039          & \textbf{0.9536} & 0.9504          & 0.9443 & 0.9475                              \\ \hline
                      & JP2K  & 0.8954          & 0.9614          & 0.9627          & 0.9696          & 0.9700          & 0.9677          & \textbf{0.9701} & \textbf{0.9724} & \textbf{0.9711} & 0.9672          & 0.9604          & 0.9677 & 0.9691                              \\
                      & JPEG  & 0.8809          & 0.9764          & 0.9815          & \textbf{0.9846} & 0.9778          & 0.9763          & \textbf{0.9823} & \textbf{0.9840} & 0.9782          & 0.9786          & 0.9761          & 0.9745 & 0.9754                              \\
LIVE                  & AWN  & 0.9854 & 0.9694          & 0.9733          & \textbf{0.9858} & 0.9774          & 0.9844          & 0.9810          & 0.9716          & 0.9737          & \textbf{0.9859} & 0.9835          & 0.9840 & \textbf{0.9872}                              \\
                      & GB    & 0.7823          & 0.9517          & 0.9542          & \textbf{0.9728} & 0.9518          & 0.9464          & 0.9660          & \textbf{0.9708} & 0.9567          & \textbf{0.9752} & 0.9527          & 0.9632 & 0.9660                              \\
                      & FF    & 0.8907          & \textbf{0.9556} & 0.9471          & \textbf{0.9650} & 0.9402          & \textbf{0.9569}          & 0.9465          & 0.9519          & 0.9416          & 0.9529 & 0.9430          & 0.9440 & 0.9476     \\
\hline
\rule{0pt}{2ex}                                            
Average & & 0.8232 & 0.8222 & 0.8321 & 0.8515 & 0.8517 & 0.7981 & 0.8533 & 0.8752 & 0.8482 & 0.8519 & \textbf{0.8769} & \textbf{0.8881} & \textbf{0.8909} \\
\hline 
\end{tabular}
\label{results3}
\end{table*}



\subsection{Performance comparison on individual distortions}
\label{individual}      

A good IQA model should perform not only accurate quality predictions for a whole dataset; it should provide good judgments over individual distortion types. We list in Table \ref{results2} the average SRC, PCC, and KRC values of sixteen IQA models for 52 sets of distortions available in the four datasets. The minimum value for each evaluation metric and standard deviation of these 52 values are also listed. These two evaluations indicate to the \textit{reliability} of an IQA model. An IQA model should provide good prediction accuracy for all of the distortion types. If a metric fails at assessing one or more types of distortions, that index can not be reliable.   

The proposed indices, MDSI and MDSI, have the best SRC, PCC, and KRC averages on distortion types. MDSI, MDSI and FSIM in the worst case perform better than the other IQA models, as can be seen in the \textit{min} column for each evaluation metric. This shows the reliability of the proposed indices. The negative \textit{min} values and near zero \textit{min} values in Table \ref{results2} indicate the unreliability of related models when dealing with some distortion types. The standard deviation of 52 values for each evaluation metric is another reliability factor. According to Table \ref{results2}, MDSI, FSIM and MDSI have the lowest variation. Therefore, we can conclude that indices MDSI, FSIM and MDSI are more reliable than the other indices.   

Furthermore, we have listed in Table \ref{results3} the SRC values of thirteen IQA models on 52 sets of distortions available in the four datasets. The top three indices for each distortion type are in boldface. In terms of SRC metric, MDSI is 26 times among the top three indices, followed by VSI (23 times), MDSI (20 times), GMSD (15 times), GS (13 times), SR\_SIM (12 times), VIF (11 times), SFF (10 times), PSNR (8 times), and FSIM (6 times). In terms of the PCC metric (corresponding table is not included in the paper), MDSI is 27 times among top three indices, followed by VSI (20 times), MDSI (18 times), GMSD (16 times), SR\_SIM (14 times), VIF (13 times), GS (11 times), PSNR (10 times), and FSIM (7 times). The results in terms of SRC and PCC indicate that MDSI, VSI, and MDSI perform better than the other indices on distortion types. The same conclusion could be drawn from Table \ref{results2}.       





\subsection{Mean deviation vs. standard deviation}
\label{vs1}

Considering the formulation of deviation pooling in equation (\ref{equ:DP}), we used the mean absolute deviation (MAD), e.g. , for the proposed metric. Standard deviation (SD), e.g. , is another option that can be used for deviation pooling. When SD was used with the proposed metric, its weighted averages for SRC, and PCC on four datasets were 0.8858, and 0.8804, respectively (see Table \ref{results2}). Also, its average/min for SRC and PCC on distortion types was 0.8683/0.2338, and 0.8966/0.5289, respectively. By a comparison with Table \ref{results2}, it can be seen that MAD (MDSI) clearly performs better than the SD within the proposed metric. On the synthetic images of the ESPL dataset, however, SD performs better than MAD. Using SD, its SRC, and PCC is 0.8888, and 0.8844, respectively. Also, its average in terms of SRC, and PCC for distortion types is 0.8638, and 0.8762, respectively. See Table \ref{synthetic} for a comparison. Despite this, the mean deviation shows consistent predictions for the different datasets in comparison to the standard deviation. Very similar conclusions can be reached by studying the effect of the MAD pooling and SD pooling on the MDSI with one exception. MAD pooling provides slightly better predictions for MDSI on the CCID2014 dataset than the SD pooling, while SD is a better choice for MDSI on this dataset.        


\subsection{Summation vs. Multiplication}
\label{vs2}
    
Two options for combination of the two similarity maps GS/ and  are summation and multiplication as explained in section \ref{Chromaticity}. Deciding weather one approach is superior to another for an index depends on many factors. These factors might be the pooling strategy being used, overall performance, performance on individual distortions, reliability, efficiency, simplicity, etc. In an experiment, the performance of the MDSI using the multiplication approach was examined. Based on the many set of parameters were tested, we found that  and  are good parameters to combine  and  via the multiplication scheme. The observation was that summation is a better choice for TID2013 dataset, a wrong choice for CSIQ and ESPL datasets, while both approaches show almost the same performance on TID2008, LIVE and CCID2014 datasets. Overall, the summation approach provides slightly better predictions for five natural datasets and their individual distortions. This experiment also shown that MDSI is more reliable through summation than multiplication based on the reliability measures introduced in this paper. Based on this experiment, the simplicity of the summation combination approach and its efficiency over multiplication, the former was used along with MDSI.






\subsection{Parameters}
\label{parameters}

The proposed IQA models MDSI and MDSI have only three and four parameters to be set, respectively. The three parameters of MDSI are ,  and , while MDSI has one more parameter . To further simplify the MDSI, we set . Therefore, MDSI has only two parameters to set, e.g.  and . For an example, we refer to the SSIM index \cite{SSIM} that also uses such a simplification. 

Before studying the impact of these parameters on the performance of MDSI and MDSI, we compared a number of competing indices in terms of the number of free parameters. The GMSD index has only one parameter. If we do not consider the parameters of phase congruency, FSIM has five free parameters in its model. SFF has also five parameters in its model. VSI has five free parameters in its model and four parameters to compute saliency features. Therefore, another advantage of the proposed indices is their small number of parameters. This makes their optimization easier. 

Fig. \ref{fig:parameters1} shows the weighted average SRC plot of MDSI for different values of  and . The plots show how MDSI performs by using different weights of gradient similarity and chromaticity similarity. It shows that  and some  are better choice of parameters for MDSI. MDSI shows acceptable performance for a wide range of parameters. For example, MDSI for any  and , has a greater weighted SRC average than 0.90. In the experiments,  is set to 0.70,  is set to 300 and therefore  is set to 600 (). Since both chromaticity channels are used by the proposed chromaticity similarity , some large values for  are used in this paper.      

\begin{figure}[htb]
\scriptsize
\begin{minipage}[b]{0.99\linewidth}
  \centering
  \centerline{\includegraphics[height=7cm]{optimizeChart1.png}} 
\end{minipage}
\caption{The weighted SRC performance of MDSI for different values of  and  on four datasets (TID2008, CSIQ, LIVE and TID2013).}
\label{fig:parameters1}
\end{figure}


For the purpose of studying the impact of the four parameters of MDSI, it is possible to simply set . Therefore, the same analysis of parameters can be done for MDSI by assuming two parameters  and . In Fig. \ref{fig:parameters2}, the impact of these two parameters on the performance of the MDSI is shown. Even though the parameters ,  and  are set approximately, it can be seen that MDSI is very robust under different setup of parameters. MDSI has greater weighted average SRC than 0.90 for any  and . Also, its performance is greater than 0.91 for  and . Note that many other possible setup of parameters are not included in this plot. In the experiments, we set , , , and . The analysis of parameters for MDSI and MDSI shows that the use of HVS-based gradient similarity can significantly increase the performance of MDSI. 

 

\begin{figure}[htb]
\scriptsize
\begin{minipage}[b]{0.99\linewidth}
  \centering
  \centerline{\includegraphics[height=7cm]{optimizeChart2.png}} 
\end{minipage}
\caption{The weighted SRC performance of MDSI for different values of  and  on four datasets (TID2008, CSIQ, LIVE and TID2013).}
\label{fig:parameters2}
\end{figure}




          




      


\subsection{Efficiency}
\label{efficiency}


Another very important factor of a good IQA model is its efficiency. The proposed indices have very low complexity. They first apply average filtering, downsampling and conversion to an opponent color space for both the reference and the distorted images. Then, the proposed indices calculate the gradient magnitudes of luminance channel, the chromaticity similarity map, and apply deviation pooling. All these steps are computationally efficient. Table \ref{time} lists the run times of sixteen IQA models when applied on images of size 384512 and 10801920. The experiments were performed on a Corei7 3.40GHz CPU with 16 GB of RAM. The IQA models were implemented in MATLAB 2013b running on Windows 7. It can be seen that MDSI and MDSI are among top seven fastest indices. The proposed indices are less than 2 times slower than the competing GMSD index. The reason for this is that GMSD only uses the luminance feature. Compared to the other competing indices, VSI, SFF, and FSIM, the proposed indices are about 3 to 9 times, 4 to 5 times, and 7 to 11 times faster, respectively. An interesting observation from the Table \ref{time} is that the ranking of indices might not be the same when they are tested on images of different size. For example, SSIM and GS perform slower than the proposed indices on smaller images, but faster on larger images. 


\begin{table}[htb]
\centering
\caption{Run time comparison of IQA models in terms of milliseconds}
\begin{tabular}{ccc}
\hline
IQA model                    & 384512 & 10801920 \\ \hline
{PSNR}    & 5.69 & 37.85                       \\
{GMSD} \cite{GMSD}    & 8.90 & 78.22                      \\
 {MDSI}    & 11.37 & 148.62                      \\
 {MDSI}    & 12.21 & 152.85                       \\
{SSIM} \cite{SSIM}    & 14.97 & 80.23                       \\
{SR\_SIM} \cite{SRSIM} & 17.02 & 100.06                       \\
{GS} \cite{GS}      & 17.13 & 82.21                       \\
{MSSSIM} \cite{MSSSIM}  & 52.16 & 413.70                       \\
{RFSIM} \cite{RFSIM}   & 62.74 & 178.99                       \\
{SFF} \cite{SFF}    & 64.22 & 588.57                       \\
{VSI} \cite{VSI}    & 106.87 & 492.85                      \\
{FSIM} \cite{FSIM}   & 142.53 & 587.42                     \\
{FSIM} \cite{FSIM}   & 145.02 & 590.84                     \\
{IWSSIM} \cite{IWSSIM} & 244.00 & 2538.43                      \\
{VIF} \cite{VIF}    & 635.22 & 6348.67                      \\
{MAD} \cite{MAD}    & 847.54 & 8452.50                     \\
\hline   
\end{tabular}
\label{time}
\end{table}



\afterpage{\clearpage}


\section{Conclusion}
\label{conclusion}

We propose two effective, efficient, and reliable full reference IQA models based on the new gradient and chromaticity similarities. The gradient similarity was used to measure local structural distortions. In a complementary way, a chromaticity similarity was proposed to measure color distortions. The proposed metrics, called MDSI and MDSI, use a novel deviation pooling to compute the quality score from the two similarity maps. Extensive experimental results on natural and synthetic benchmark datasets prove that the proposed indices are effective and reliable, have low complexity, and are fast enough to be used in real-time FR-IQA applications.  





\section*{Acknowledgments}
The authors thank the NSERC of Canada for their financial support under Grants RGPDD 451272-13 and RGPIN 138344-14. 






\bibliographystyle{IEEEtran}
\bibliography{egbib2}   
























\end{document}
