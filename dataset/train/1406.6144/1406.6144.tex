\documentclass[a4paper]{llncs}

\usepackage{amsmath}
\usepackage[T1]{fontenc}
\usepackage{amsfonts}
\usepackage{amssymb}
\usepackage{float}
\usepackage{url}
\usepackage{tikz}
\usetikzlibrary{automata}
\usetikzlibrary{arrows}
\usetikzlibrary{shapes}
\usetikzlibrary{decorations.pathmorphing}
\usetikzlibrary{fit}
\usepackage{algorithm}
\usepackage{algorithmic}
\usepackage{hyperref}
\usepackage[margin=1.5cm]{geometry}
\usepackage{multicol}


\setlength{\abovedisplayskip}{0pt}
\setlength{\belowdisplayskip}{0pt}
\setlength{\abovedisplayshortskip}{0pt}
\setlength{\belowdisplayshortskip}{0pt}
\allowdisplaybreaks[1]

\tikzstyle{every picture}=[
  >=stealth', shorten >=1pt, node distance=1.44cm,auto,bend angle=45,initial text=,
  every state/.style={inner sep=0.75mm, minimum size=1mm},font=\scriptsize,
]

\newcommand{\Intervalle}[2]{\{#1,\ldots,#2\}}
\newcommand{\Card}[1]{\mathrm{Card}(#1)}
\newcommand{\indef}{\bot}

\newcommand{\floMod}[2]{{\color{blue}#1}{\color{red}[#2]}}


\newcommand{\modif}[2]
  {{\color{gray}(#1)}{\color{red}#2}}
  

\bibliographystyle{splncs_srt}
\pagestyle{plain}


\setcounter{secnumdepth}{3}

\makeatletter
\newcommand{\subalign}[1]{\vcenter{\Let@ \restore@math@cr \default@tag
    \baselineskip\fontdimen10 \scriptfont\tw@
    \advance\baselineskip\fontdimen12 \scriptfont\tw@
    \lineskip\thr@@\fontdimen8 \scriptfont\thr@@
    \lineskiplimit\lineskip
    \ialign{\hfil&\crcr
      #1\crcr
    }}
}
\makeatother

    \everymath{\displaystyle}

\begin{document} 

  \title{Constrained Expressions and their Derivatives}
  
  \author{
    Jean-Marc Champarnaud \and
    Ludovic Mignot
    \and Florent Nicart
  } 

  \institute{
    LITIS, Universit\'e de Rouen, 76801 Saint-\'Etienne du Rouvray Cedex, France\\
     \email{\{jean-marc.champarnaud,ludovic.mignot,florent.nicart\}@univ-rouen.fr}\\
  }
  
  \maketitle
  


  \begin{abstract} 
This paper proposes an extension to classical regular expressions by the addition of two operators allowing the inclusion of boolean formulae from the zeroth order logic. These expressions are called \emph{constrained expressions}. 
The associated language is defined thanks to the notion of interpretation and of realization.

We show that the language associated when both interpretation and realization are fixed is stricly regular and can be not regular otherwise.

    Furthermore, we use an extension of Antimirov partial derivatives in order to solve the membership test in the general case.
Finally, we show that once the interpretation is fixed, the membership test of a word in the language denoted by a constrained expression can be undecidable whereas it is always decidable when the interpretation is not fixed. 
  \end{abstract}

\section{Introduction}\label{se:int}

	Regular expressions are a convenient formalism to denote in a finite and concise way regular languages that are potentially infinite. Based on only three simple operators (sum, catenation and iteration), they are extremely easy to manipulate and are widely used in numerous domains, such as pattern matching, specification or schema validation. However, their expressive power is restricted to the class of regular languages and many attempts have been made to extend the class of their denoted languages while trying to keep their simplicity.
  
  Several approaches exist in order to make this expressive power larger; \emph{e.g.} by adding new operators~\cite{CDJM13} or by modifying the way symbols are combined~\cite{Brz68,Semp00}. In the latter case, the expressive power of the so-called regular-like expressions is increased to the linear languages, that is, a strict subclass of context-free languages in the Chomsky hierarchy~\cite{Cho56}.
  Other concepts have been added to expressions, such as the mechanism of capturing variables~\cite{CSY03,SL12}.
  
Our approach, although it is based on the introduction of two new operators, is quite different
These two operators, respectively  and , establish a link with the first order logic without quantifiers (a.k.a. zeroth order logic), allowing us to easily describe non-regular languages, using both predicates and variables in order to evaluate  what we call \emph{constrained expressions}.

Given an expression  and a boolean formula , we define the expression  ( such that ) that denotes  when  is satisfied and the empty set otherwise. Given a word , based on symbols and variables, and an expression , we define the expression  denoting  if it is in  and the empty set otherwise. The addition of these two operators allows us to go beyond regular languages.
Also, it turns out that constrained expressions allow us to implement  the concept of comprehension over regular expressions. Recall that the comprehension axiom can be stated as follows: for any set , for any property , there exists a set  defined for any element  by   .  

    
  In our formalism, variables are used as a combination of the following two concepts.
  In~\cite{CSY03}, variables are used to formalize the notion of practical regular expressions (a.k.a. regex) with backreferences: in addition to the classical operators of regular expressions, variables can be used to submatch some parts of an expression. As an example, in the expression , the variable  is interpreted as a copy of the match of the expression ; therefore the language denoted by  is the set , that is, not a regular language.
  In~\cite{SL12}, variables are used not to extend the representation power of expressions (since the denoted languages are still regular), but to efficiently solve the submatching problem, that is, to split a word according to the subexpressions of an expression it is denoted by.
  Our formalism is a different application of the concept of variables: we match particular subexpressions that we can repeat, we filter \emph{a posteriori} and we obtain languages which are not necessarily regular. As an example, let us consider the expression . This expression denotes the set of the words that can be written  such that  (resp. , ) is only made of  (resp. , ) and such that the words ,  and  satisfy a property . As an example, if  is the property ",  and  admit the same length", then the language (under this interpretation) is the set .

  The aim of this paper is to define these new operators, and to show how to interpret them. We also show how to solve the membership problem, that is, to determine whether a given word belongs to the language denoted by a given constrained expression. 
  In order to perform this membership test, we present a mechanism that first associates to the variable some subword of the word to be matched, and then evaluates the boolean formula that may appear during the run.
  
  The membership test for classical regular expressions can be performed \emph{via} the computation of a finite state machine, an automaton. However, we cannot apply this technique here, since we deal with non-regular languages and therefore with infinite state machines. Nevertheless, we apply a well-known method, the expression derivation~\cite{Ant96,Brz64}, in order to reduce the membership problem of any word to the membership test of the empty word. Once this reduction made, we study the decidability of this problem.
  
  Section~\ref{se:pre} is a preliminary section where  we recall some basic definitions of formal language theory, such as languages or expression derivation. We also introduce the notion of zeroth-order logic that we use in the rest of the paper.
  Section~\ref{sec:cons expr lang and der} defines the constrained expressions and the different languages they may denote. We also define in this section the way we derive them.
  Section~\ref{sec:eps membership test} is devoted to illustrating the link between the empty word membership test and a satisfiability problem.
  In Section~\ref{sec:decidab}, we show that the satisfiability problem we use is decidable in the general case and that it is not in a particular subclass of our evaluations.
  
    

\section{Preliminaries}\label{se:pre}

  \subsection{Languages and Expressions}
  
  Let  be an alphabet. We denote by  the free monoid generated by  with  the catenation product and  its identity. Any element in  is called a \emph{word} and  the empty word. A \emph{language over}  is a subset of . 
  
  A language over  is \emph{regular} if and only if it belongs to the family  which is the smallest family containing all the subsets of  and closed under the following three operations:
  \begin{itemize}
    \item \emph{union}: ,
    \item \emph{catenation}: ,
    \item \emph{Kleene star}: .
  \end{itemize}
  
  A \emph{regular expression}  \emph{over}  is inductively defined as follows:
  
  where  is any symbol in  and  and  are any two regular expressions over . Parentheses can be omitted when there is no ambiguity.  The \emph{language denoted by}  is the language  inductively defined by:
  
  where  is any symbol in  and  and  are any two regular expressions over . It is well known that language denoted by a regular expression is regular.
  
  Given a word  and a language , the \emph{membership problem} is the problem defined by "Does  belong to ". Many methods exist in order to solve this problem: 
as far as regular languages, given by regular expressions, are concerned, a finite state machine, called an \emph{automaton}, can be constructed with a polynomial time complexity w.r.t. the size of the expression, that can decide with a polynomial time complexity w.r.t. the size of the expression if a word  belongs to the language denoted by the expression~\cite{Glu60,IY03,MY60,Tho68}. See~\cite{GH14} for an exhaustive study of these constructions and of their descriptional complexities.
  
  As far as regular expressions are concerned, the computation of a whole automaton is not necessary; the very structure of regular expressions is sufficient. 
Considering the residual  of  w.r.t. , that is, the set
  , the membership test  is equivalent to the membership test . This operation of quotient can be performed directly through regular expressions using the \emph{partial derivation}~\cite{Ant96}, which is an extension of the derivation~\cite{Brz64}.
  
  \begin{definition}[\cite{Ant96}]
    Let  be a regular expression over an alphabet . The \emph{partial derivative of}  w.r.t. a word  in  is the set  inductively defined as follows:  
  
  where  and  are any two regular expressions over  and where for any set  of regular expressions, for any regular expression ,  and for any word  in , .
  \end{definition}
  
  Furthermore, the membership test of  is syntactically computable. Considering the predicate , it can be checked that:
  
  where  and  are any two regular expressions over .
    
  Consequently, from these definitions, the membership test of  in  can be performed as follows:
  
  \begin{proposition}[\cite{Ant96}]
    Let  be a regular expression over an alphabet  and  be a word in . The following two conditions are equivalent:
    \begin{enumerate}
      \item 
      \item , .
    \end{enumerate}
  \end{proposition}
  
  Derivation and  partial derivation have already been used in order to perform the membership test over extensions of regular expressions~\cite{CCM11b,CCM12c,CCM14,CJM13}, expressions denoting non-necessarily regular languages~\cite{CDJM13}, guarded strings~\cite{ABM12} or even context-free grammars~\cite{MDS11}.  
  In the rest of this paper, 
we extend regular expressions by introducing new operators based on boolean formulae  
  in order to increase the expressive  power of expressions. 
Let us first recall some well-known definitions of logic.

  \subsection{Zeroth-Order Logic}
  
  The notion of constrained expression introduced in this paper is expressed through the formalism of zeroth-order logic, that is, first order logic without quantifiers (see primitive recursive arithmetic in~\cite{Sko67} for an example of the difference between the expressiveness of propositional logic and zeroth-order logic).
  
  More precisely, we consider two -indexed families  and  of disjoint sets, where for any integer  in ,  is a set of -ary function symbols and  is a set of -ary predicate symbols. The family   is combined with a set of variables in order to obtain a set of terms. This set of terms is combined with the family  in order to obtain boolean formulae.
  
  Given a set  of variables, a \emph{term}  \emph{over}  is inductively defined by:
  
  where  is any integer,  is any element in ,  are any  terms over . 
    We denote by  the set of the terms over .
    
    A \emph{subterm} of a term  is a term in the set  inductively computed as follows:
  
  where  is any element in ,  is any integer,  is any function symbol in  and , ,  are any  terms in .
     
    A \emph{boolean formula}  \emph{over}  is inductively defined by:
  
  where  and  are any two integers,  is any element in ,  are any  terms in ,  is any
    -ary boolean operator associated with a mapping from  to 
  and  are any  boolean formulae over . We denote by  the set of boolean formulae over .
  
  Given a formula  in , a term  in  and a symbol  in , we denote by  the \emph{substitution of}  by  in , which is the boolean formula inductively defined by:
  
  where  and  are any two integers,  is any element in ,  are any  terms in ,  is any
    -ary boolean operator associated with a mapping from  to ,  are any  boolean formulae over  and where for any term  in ,  is the \emph{substitution of}  by  in , which is the term inductively defined by:
  
  where  is any element in ,  is any integer,  is any function symbol in  and , ,  are any  terms in .
  
  \begin{example}\label{ex formule bool}
    Let us consider the two families  and  defined by
  
  and the set .        
        As an example, the set of boolean formulae over  contains the formulae:
    \begin{itemize}
      \item ,
      \item ,
      \item ,
      \item , where  is the ternary  boolean operator corresponding to the \emph{If-Then-Else}-like conditional expression, generally written .
    \end{itemize}
    \qed
  \end{example}
  
  After having defined the syntactic part of the logic formulae we use, we show how to evaluate these formulae, \emph{i.e.} how to define the semantics of the logic formulae. The boolean evaluation of a formula is performed in two steps. First, an \emph{interpretation} defines a domain and associates the function and predicate symbols with functions; then, each variable symbol is associated with a value from the domain by a \emph{realization}. 
  
  \begin{definition}[Interpretation]\label{def interpretation}
    Let  and  be two families of disjoint sets. An \emph{interpretation}  over  is a tuple  where:
    \begin{itemize}
      \item  is a set, called the \emph{interpretation domain} of ,
      \item  is a function:
        \begin{itemize}
          \item from  to ,
          \item and from  to  such that for any function symbol  in , for any two elements  and  in ,   , and such that for any  elements  in , there exists  in  such that   
        \end{itemize}
        called the \emph{interpretation function}.
    \end{itemize}
  \end{definition}
  
  \begin{definition}[Realization]
    Let  and  be two families of disjoint sets, and  an interpretation over . Let  be a set. An -\emph{realization}  over  is a function from  to .
  \end{definition}
  
  Once an interpretation  and a realization  given, a term can be evaluated as an element of the domain and a formula as a boolean \emph{via} the function , the \emph{-evaluation}:
  
  \begin{definition}[Term Evaluation]
    Let  and  be two families of disjoint sets and  an interpretation over . Let  be a set. Let  be an -realization over . Let  be a term in . The -\emph{evaluation} of  is the element  in  defined by:
    
    where  is any integer,  is any function symbol in , and  are any  elements in .
  \end{definition}
  
  \begin{definition}[Formula Evaluation]
    Let  and  be two families of disjoint sets and  an interpretation over . Let  be a set. Let  be an -realization over . Let  be a term in . Let  be a boolean formula in . The -\emph{evaluation} of  is the boolean  inductively defined by:
  
  where  is any integer,  is any predicate symbol in ,  are any  elements in ,  is any -ary boolean operator associated with a mapping  from  to  and  are any  boolean formulae over .
  \end{definition}
  
  
  \begin{example}\label{ex inter real}
    Let us consider Example~\ref{ex formule bool}. Let  be an alphabet. Let  be the interpretation  over  and  be the -realization over  defined by:
  
  where for any word  in ,  is the word defined by:
  
  Then:    
    
    \qed       
  \end{example}
  
 
\section{Constrained Expressions, their Languages and Derivatives}\label{sec:cons expr lang and der}

  In this section, zeroth-order logic is combined with classical regular expressions in order to define \emph{constrained expressions}. The language denoted by these expressions is not necessarily regular. We extend the membership problem for constrained expressions using partial derivatives and then show that it is equivalent to a satisfiability problem.

\subsection{Constrained Expressions and their Languages}

  Whereas regular expressions are defined over a unique symbol alphabet, constrained expressions deal with zeroth-order logic and therefore include function, predicate and variable symbols. Hence the notion of alphabet is extended to the notion of \emph{expression environment} in order to take into account
all these symbols.
  
  \begin{definition}[Expression Environment]
    An \emph{expression environment} is a -tuple  where:
    \begin{itemize}
      \item  is an alphabet, called the \emph{symbol alphabet},
      \item  is an alphabet, called the \emph{variable alphabet},
      \item  is a -indexed family of disjoint sets, called the \emph{family of predicate symbols},
      \item  is a -indexed family of disjoint sets, called the \emph{family of function symbols} such that  and .
    \end{itemize}
  \end{definition}
  
  Once this environment stated, we can syntactically define the set of \emph{constrained expressions}, by adding two new operators to regular operators: the first operator, , is based on the combination of an expression  and of a boolean formula , producing the expression ; the second operator, , links a word  composed of variable and letter symbols to an expression , producing the expression .
  Notice that the following definitions use extended boolean operators, such as intersection or negation. However, we will extend the membership test for expressions only using the sum operator.
  
    
  \begin{definition}[Constrained Expression]
    Let  be an expression environment. A \emph{constrained expression}  \emph{over}  is inductively defined by:
    
    where  is any integer,  is any -ary boolean operator associated with a mapping from  to ,  are any  constrained expressions over ,  is any word in  and  is a boolean formula in .
  \end{definition}  
\noindent  
  Parenthesis can be omitted when there is no ambiguity.
  
  Any boolean formula that appears in a constrained expression over an environment  is, by definition, a formula in . Furthermore, since we want a constrained expression to denote  a subset of , variable symbols in  have to be evaluated as words in . Moreover, classical symbols, like  or  in , have to be considered as -ary functions in the interpretation. All these considerations imply some specializations of the notions of
interpretation and realization, defined as follows.
  
  \begin{definition}[Expression Interpretation]
    Let  be an expression environment. An \emph{expression interpretation}  over  is an interpretation  over  satisfying the following three conditions:
    \begin{enumerate}
      \item ,
      \item for any symbol  in , , 
      \item .
    \end{enumerate}
  \end{definition} 
  
  The two new operators appearing in a constrained expression are used to extend the expressive power of regular expressions. The expression  denotes the set of words that  may denote whenever the formula  is satisfied. The expression  denotes the set of words that  may denote and that can be "matched" by . In order to perform this matching, we extend any -realization over an expression interpretation as a morphism from  to . 

    Let  be an expression environment. Let  be an expression interpretation over  and  be a -realization over . The domain of the realization  can be extended to  as follows. For any word  in ,  is the word in  inductively computed by:
      
    
  Using this extension, we can now formally define the different languages that a constrained expression may denote. We consider the following three cases where first both the interpretation and
the realization are fixed, then only the interpretation is fixed
and finally nothing is fixed.
  
  \begin{definition}[(I,r)-Language]\label{def i r lang}
    Let  be an expression environment. Let  be an expression interpretation over  and  be a -realization over . Let  be a constrained expression over .
    The -\emph{language denoted by}  is the language  inductively defined by:
    
    where  is any integer,  is any -ary boolean operator,  is the language operator associated with ,  are any  constrained expression over ,  is any word in  and  is any boolean formula in .
  \end{definition}
  
  We denote by  the set of the -realizations over an interpretation .
    
  \begin{definition}[I-Language]
    Let  be an expression environment. Let  be a constrained expression over . Let  be an expression interpretation over . The -\emph{language denoted by}  is the language  defined by:
    
  \end{definition}
  
  Given an expression environment , we denote by  the set of the expression interpretations over .
    
  \begin{definition}[Language]
    Let  be an expression environment. Let  be a constrained expression over . The \emph{language denoted by}  is the language  defined by:
    
  \end{definition}
  
  \begin{example}\label{ex exp cons lang}
    Let  be the expression environment defined by:
    \begin{itemize}
      \item ,
      \item ,
      \item ,
      \item , , .
    \end{itemize}    
    Let us consider the constrained expressions  and . 
    Let  be the expression interpretation defined by:
    \begin{itemize}
      \item ,
      \item ,
      \item , for any  in , 
      \item 
      \item .
    \end{itemize}
    
    In other words, the evaluation of an expression w.r.t.  considers that:
    \begin{itemize}
      \item  is true if and only if  is shorter than ,
      \item  is true if and only if ,
      \item  is a function that changes all the symbols in  that are different from  into a symbol .
    \end{itemize}
    
    By abuse of notation, let us syntactically apply the interpretation as follows:
    
    
    Let us consider the -languages denoted by these expressions:
    \begin{itemize}
      \item  is the set of words  with  and ,
      \item  is the set of words  with  and . 
    \end{itemize}
    
    As an example, the word  belongs to:
    \begin{itemize}
      \item  since it can be obtained by considering the realization  associating  with  and  with : 
    
that is equivalent to  ,
      \item  since it can be obtained by considering the realization  associating  with  and  with : 
    
    that is equivalent to  and finally to .
    \end{itemize}
    
    In other words, the word  belongs to  and to .
    
    Notice that the word  is not in the -language denoted by  that is equivalent to  nor in the -language denoted by  that is equivalent to  and finally to . 
  \end{example}
    \begin{example}\label{ex anbncn}
      Let us consider the expression environment  of Example~\ref{ex exp cons lang}.
      Let  be the constrained expression defined as follows:
      
      Let us consider an expression interpretation  that satisfies
      
      Then
      
    \end{example}

  \subsection{The -Language of a Constrained Expression is Regular}
  
  Whenever an interpretation and a realization are fixed, the language denoted by a constrained expression is a regular one. The proof is based on the computation of an equivalent regular expression.
  
  \begin{definition}[Regularization]\label{def i r reg exp}
    Let  be an expression environment and  be a constrained expression over . Let  be an expression interpretation over  and  be a -realization over . The -\emph{regularization} of  is the regular expression  inductively defined as follows:
    
    where  is any integer,  is any -ary boolean operator associated with a mapping from  to ,  are any  constrained expressions over ,  is any word in  and  is a boolean formula in .  
  \end{definition}
  
  \begin{proposition}
    Let  be an expression environment and  be a constrained expression over . Let  be an expression interpretation over  and  be a -realization over . Then:
    
  \end{proposition}
  \begin{proof}
    By induction over the structure of . According to Definition~\ref{def i r reg exp} and to Definition~\ref{def i r lang}:
    
    where  is any integer,  is any -ary boolean operator associated with a mapping from  to ,  is the language operator associated with ,  are any  constrained expressions over ,  is any word in  and  is a boolean formula in .
    \qed
  \end{proof}
  
 Once this regular
expression is computed, any classical membership test can be performed; hence:
  
  \begin{corollary}\label{cor i r lang rat}
    Let  be an expression environment and  be a constrained expression over . Let  be an expression interpretation over  and  be a -realization over . Let  be a word in . Then:
    
  \end{corollary}
  
  \subsection{Derivatives for Constrained Expressions}   
   
    \begin{remark}
From now on, the set of boolean operators is restricted to the sum.
  \end{remark}
  
 




We showed in the previous section that the language of any constrained expression with a fixed interpretation and realization is regular. However, whenever the realization or the interpretation is not given, the language denoted by a constrained expression is an infinite union of regular languages which is not necessarily regular. Thus, in order to perform the membership test, the notion of partial derivatives is extended to the case of constrained expressions.
The idea is the following: for any interpretation and realization, a syntactical test can be achieved by computing all the splits of the word for which the membership test is performed. Once these precomputations terminated, new constrained expressions are generated and the membership test has to be performed for the empty word. In fact, we show that it is equivalent to solving the logical part, that is, to determine the satisfiability of the new formulae.
   While deriving expressions, choices have to be made in order to fix a realization. As an example, deriving the expression , where  is a variable symbol, with respect to the symbol , implies that the variable symbol  is associated with a word starting with a, otherwise, the derivative would be empty. Consequently, such a realization transforms  in  and then associates the expression  with the expression . Deriving this expression w.r.t.  returns the expression  which is equivalent to .
  


  As a direct consequence, the partial derivation has to memorize the assumptions made during the computation. Therefore, a partial derivative needs to be a set of tuples composed of an expression and a set of assumptions, where an assumption is a tuple composed of a variable symbol  and a word : the realization associates the variable  with the word .    
These assumptions are needed to transform subexpressions of the initial expression. As an example, let us consider the expression . If assumptions are needed to perform the membership test while deriving , these assumptions have to be applied over  too via a substitution. Let us then extend the notion of substitution to words, to boolean formulae and to constrained expressions.



  Let  be an alphabet and let  and  be two words in . Let  be a symbol in . We denote by  the word obtained by substituting any occurrence of  in  by , that is:
  
    \begin{definition}[ Function]
     Let  be the function from  to  (that is, the set of functions over the empty set of variables) inductively defined for any word  as follows:
        
  \end{definition}
  For a boolean formula , we denote by  the boolean formula defined by:  
  
  Finally, for any constrained expression , we denote by  the expression:
  
 Let  be a subset of  satisfying the following two conditions:
 \begin{enumerate}
   \item \emph{Functional:} for any two distinct couples  and  in , ;
   \item \emph{Non-crossing: }for any two distinct couples  and  in ,  does not appear in .
 \end{enumerate}
 Since  is finite and therefore can be considered as ordered, we consider that  is ordered by an arbitrary lexicographic order from .
 We extend the substitution to couples in  as follows:
  
  


  Let us continue with the previous example with the expression . If we want to check that the word  belongs to the language denoted by this expression,  can be replaced by , and then the derivation of  w.r.t.  produces an expression that denotes . However, substituting  by a symbol is not sufficient in the general case. If we want to perform the membership test of the word , the derivation w.r.t.  has to memorize that the realization associates  with a word that \emph{starts with} the symbol . Then the variable  can be replaced by the word : the expression  is transformed into  when the derivative w.r.t.  is computed, producing the expression . Deriving w.r.t. , the assumption that  (the new , not the old one) is associated with a word that starts with  has to be made, replacing  by  and producing . Deriving it w.r.t. , a new assumption can be made: if the new  is replaced by , then the expression  is replaced by the word  and its derivation w.r.t.  will produce , proving  that the word  is denoted by .
  
  As a direct consequence, the partial derivation of a constrained expression will compute all the combinations of assumptions that can be made during the derivation.



Let  be an expression environment.
  We denote by  the set of the constrained expressions over . 
  Let us consider a word  in .
  Either , and therefore , or  with  in  and  in .
  If , then . 
  If , then the only derived term of  is  with no assumption of substitution made; therefore .
  If , then two assumptions have to be considered:
  \begin{itemize}
    \item if  starts with , then  can be substituted by , and then  becomes . In this case, the only derived term is  under the substitution ; therefore .
    \item if  equals , then all the occurrences of  have to be substituted by . 
    Thus  becomes . 
    Once this substitution made, there are no more occurrences of  in  and  has to be derived w.r.t. .
    Consequently, .
  \end{itemize} 
  More formally,
  \begin{definition}[Constrained Derivative of a word]\label{def deriv expr cont mot}
    Let  be an expression environment and let  be a word in . 
    Let  be a symbol in . 
    The \emph{constrained derivative of}  \emph{w.r.t.}  is the subset  of  inductively computed as follows:
    
  \end{definition}
  
Let us check that the sets that that appear in a derived term are functional and non-crossing: 
  \begin{lemma}\label{lem ens deriv part ok}
    Let  be an expression environment and let  be a word in . 
    Let  be a symbol in . 
    Then for any couple  in , it holds:
    
  \end{lemma}
  \begin{proof}
    The proof is done by induction over the length of the words.
    Obviously the condition holds for non inductive cases of Definition~\ref{def deriv expr cont mot}.
    Let  be a couple in .
    By induction over the length of , it can be shown that , since there is no occurrence of  in .
    By induction hypothesis,  is functional and non-crossing, and therefore so is .
    \qed
  \end{proof}
  
  Let us now extend the partial derivation to constrained expressions. 
  We first syntactically define the derivatives, and then we prove their existence.


  \begin{definition}[Constrained Derivative of a Constrained Expression]\label{def deriv expr cont exp}
    Let  be an expression environment and let  be a constrained expression over . Let  be a symbol in . The \emph{constrained derivative of}  \emph{w.r.t.}  is the subset  of  inductively computed as follows:
    
    where for any subset  of , for any expression  and for any formula ,
    
  \end{definition}
\begin{lemma}\label{lem union X dis}
    Let  be an expression environment and let  be a constrained expression over . Let  be a subset of . Let  be a symbol in . Then:
        
  
  \end{lemma}
  \begin{proof}
    Let us notice that according to Definition~\ref{def deriv expr cont mot}, there exists a symbol  in  and a word  in  satisfying  only if there exists a subexpression  of  such that  for some  in .    
    Furthermore, since applying  for some  over  (producing ) replaces any occurrence of  either by  or by , there exists no subexpression  of  such that  for some  in .    
    Consequently, the same symbol  cannot be the first component of both a tuple in  and of a tuple in .
    \qed
  \end{proof}
  
  \begin{lemma}
    Let  be an expression environment and let  be an expression over . 
    Let  be a symbol in . 
    Then for any couple  in , it holds that:
    
  \end{lemma}
  \begin{proof}
    The proof is done by induction over the structure of expressions.
    Basic cases are well defined from Lemma~\ref{lem ens deriv part ok}.
    The cases of the sum, catenation, star and "\emph{such that}"-operation leave the properties of the sets unchanged.
    Thus, let us consider the case of the "\emph{membership}"-operation.
    
    Let  and  be a couple in .
    Consider a couple  in .
    From Lemma~\ref{lem ens deriv part ok},  is a functional non-crossing set.
    Let  be a couple in .
    By induction hypothesis,   is a functional non-crossing set.
    Finally, from Lemma~\ref{lem union X dis},  is a functional non-crossing set.
    \qed
  \end{proof}
  
  \begin{corollary}
    The constrained derivation is well-defined.
  \end{corollary} 
  In the following, in the examples, we use the symbol  as a semantical equivalence between expressions or sets. As an example,  or .
  \begin{example}\label{ex deriv anbncn}
    Let us consider the expression  defined in Example~\ref{ex anbncn}.
    Let us set 
    
    Consequently
    
    Then
    
    Furthermore,
    
    Then
    
  \end{example}
  
  The following of this section is devoted to proving that the derivation can be used to perform the membership test. In fact, we show that to determine whether or not a word  is denoted by a constrained expression  is equivalent to determining whether or not  is denoted by one of the derived expressions from .
  
  We first model the fact that an assumption made through the derivation can be performed through a substitution without modifying the membership test: the main idea is that if a realization associates a word  with a symbol , the result is the same as if any occurrence of  is replaced by  and if another realization is considered, where  is associated with . Hence, we can transfer a symbol from the realization to the expression.
  
  \begin{definition}[Compatible Realization]
    Let  be an expression environment. Let  be an expression interpretation over  and  be a -realization over . Let  be a subset of . The realization  is said to be \emph{compatible with}  if and only if the following  two conditions hold:
    \begin{itemize}
      \item ,  for some word  in ,
      \item , .
    \end{itemize}
  \end{definition}
  
  \begin{definition}[Associated Realization]
    Let  be an expression environment. Let  be a subset of . Let  be an expression interpretation over  and  be a -realization over  compatible with .  The \emph{realization} \emph{-associated with}  is defined for any symbol  in  as follows:
    
  \end{definition}
  
    \begin{lemma}\label{lem r assoc egal r pour phi}
    Let  be an expression environment and let  be a boolean formula in . Let  be a subset of . Let  be an expression interpretation over  and  be a -realization over  compatible with . Let  be the realization -associated with . Then:
    
  \end{lemma}
  \begin{proof}
    We proceed in two steps.
    \begin{enumerate}
    \item\label{p1} Let us first show that for any term  in , .
    By induction over .
    \begin{enumerate}
      \item Suppose that .
        Then 
        
        
        Furthermore,
                
        Consequently,
        
      \item Suppose that , that  and that  . 
      By induction hypothesis, it holds that:
        
        Then:
        
    \end{enumerate}    
    \item Let us show now by induction over  that .    
    \begin{enumerate}
      \item If , then .
        Then
        
      \item Suppose that .
        Then .
        Then   
        
    \end{enumerate}
    \end{enumerate}    
    \qed
  \end{proof}
  
  
  \begin{lemma}\label{lem r assoc egal r}
    Let  be an expression environment and let  be a constrained expression over . Let  be a subset of . Let  be an expression interpretation over  and  be a -realization over  compatible with . Let  be the realization -associated with . Then:
        
  \end{lemma}
  \begin{proof}
    By induction over the structure of .
    \begin{enumerate}
      \item Let us suppose that . 
        By recurrence over the length of .
        \begin{enumerate}
          \item If  or ,  and then .
          \item If , then           
        
              
              Furthermore, 
              
               
               and then 
               
               
              
          \item Suppose that  with .
                Then        
        
        \end{enumerate}
        
      \item Let us suppose that .
        Then    
        
        
      \item Let us suppose that .
        According to Lemma~\ref{lem r assoc egal r pour phi}, .
        Hence if , .
        Otherwise,
        
      \item Let us suppose that .
        Then:
        
        
      \item Let us suppose that .
        Then:
        
      \item Let us suppose that .
        Then:
        
    \end{enumerate}
    \qed
  \end{proof}
  
  Let us now show that the partial derivation can be used to perform the membership test over constrained expressions whenever the realization is not fixed: if a word  belongs to the language of a constrained expression  whenever a realization  is considered, the partial derivation w.r.t.  always produces at least a tuple  where  denotes  when another realization is considered (the realization -associated with ). 
  
  \begin{proposition}\label{prop eq quot deriv part}
    Let  be an expression environment and let  be a constrained expression over . Let  be an expression interpretation over  and  be a -realization over . Let  be a word in  and  be a symbol in . Then the  following  two conditions are equivalent:
    \begin{itemize}
      \item 
      \item there exists a tuple  such that , where  is the realization -associated with .
    \end{itemize}
  \end{proposition}
  \begin{proof}
    By induction over the structure of . By definition,   .     
    \begin{enumerate}
      \item Whenever ,  and  are both empty.
      \item Let us suppose that . Three cases can occur.
      \begin{enumerate}
        \item Let us suppose that . If , .        
        Otherwise (if ), . Furthermore,
                
        Finally, since  is the realization -associated with ,
                 
        \item Let us suppose that  and that  for some . Let us denote by  the realization -associated with . Then
                
        As a direct consequence, 
                
        Finally, it holds by definition that .        
        \item Let us suppose that  and that . Then:
                
        As a direct consequence,
                
        According to induction hypothesis, there exists a tuple  belonging to  such that  with  the realization -associated with . Let  be the realization -associated with . Since there is no occurrence of  in  (since  is a derivated term of ), it holds that . Furthermore, it holds by definition that . Finally, 
      \end{enumerate}
      \item Let us suppose that . 
      Consider that there exists a tuple  such that , where  is the realization -associated with . 
      
      Equivalently, there exist  and   
      such that  and , where  is the realization -associated with .
      
      By definition, . Consequently  and . Let us denote by  (resp. ) the realization -associated (resp. -associated) with .
      
      Since  is the realization -associated with , and since according to Lemma~\ref{lem union X dis}, , the by definition, for any symbol  in , the following equality is satisfied:
            
    By definitions of  and ,  for any symbol  in :
            
    Hence, since ,
        
    Consequently,  is -associated with . Symmetrically,  is -associated with .
    
      Since  belongs to , there exists a tuple  such that , where  is the realization -associated with . By induction hypothesis . According to Lemma~\ref{lem r assoc egal r},   . Since  is -associated with , according to Lemma~\ref{lem r assoc egal r}, . Hence  and by induction hypothesis .      
      Finally, it holds that .
\item Let us suppose that . Then       
            
         .      
      By induction, it is equivalent to ,  where  is the realization  associated with .       
      Since , it is equivalent to ,  where  is the realization  associated with .
\item Let us suppose that . Then 
                    
      Moreover 
              
      According to Lemma~\ref{lem r assoc egal r},   .      
      Hence
              
      Finally consider that   .      
      By induction       
         where  is the realization -associated with .      
      Moreover    is equivalent to  according to Lemma~\ref{lem r assoc egal r}.
      
      Hence 
        
\item Let us suppose that . Then       
            
           .      
      By induction,       
         where  is the realization -associated with .      
      According to Lemma~\ref{lem r assoc egal r},       
       .      
      Hence 
        
\item Let us suppose that . Then      
            
         .      
      By induction,       
       ,  where  is the realization  associated with .      
      According to Lemma\ref{lem r assoc egal r pour phi}, .      
      Consequently, 
               
    \end{enumerate}    
    \qed
  \end{proof}
  
  As a direct consequence of Proposition~\ref{prop eq quot deriv part}, the partial derivation of a constrained expression w.r.t. a symbol is valid.
  
  \begin{theorem}\label{thm egal quot deriv part}
    Let  be an expression environment and let  be a constrained expression over . Let  be an expression interpretation over . Let  be a symbol in . Then the  following two conditions hold:
    \begin{enumerate}
      \item ,
      \item .
    \end{enumerate}
  \end{theorem}
  \begin{proof}
    Let  be a word in . 
    \begin{enumerate}
      \item By definition of ,   there exists a realization  such that .
      
      According to Proposition~\ref{prop eq quot deriv part},      
        there exists a tuple  such that , where  is the realization -associated with . As a direct conclusion, .
      
      Suppose that . Hence there exists a tuple  such that . By definition of , there exists a realization  such that . Let  be the realization defined for any symbol  in  as follows:
        
      
      As a direct consequence,  is the realization -associated with , and according to Proposition~\ref{prop eq quot deriv part}, since there exists a tuple  such that , where  is the realization -associated with , it holds that . By definition of , .
\item By definition of ,   there exists an interpretation  such that . We have shown that there exists an interpretation  such that   there exists an interpretation  such that , 
which is by definition of  equivalent to the fact that  . 
    \end{enumerate}
    \qed
  \end{proof}
  
  The partial derivation can be extended from symbols to words as follows: 
  
  \begin{definition}[Word Derivative]
    Let  be an expression environment and let  be a constrained expression over . Let  be a symbol in  and  be a word in . Then:
        
  \end{definition}  
  
  \begin{example}\label{ex deriv anbncn abc}
    Let us continue Example~\ref{ex deriv anbncn}.
    Let us set  and let us compute 
    
    Furthermore
    
    Then
    
    And consequently
    
    Finally, setting
    
    one can compute
    
  \end{example}
  
  \begin{theorem}
    Let  be an expression environment and let  be a constrained expression over . Let  be a word in . Then the following conditions hold:
    \begin{enumerate}
      \item ,
      \item .
    \end{enumerate}
  \end{theorem}
  \begin{proof}
    \ 
    \begin{enumerate}
      \item By recurrence over the length of . If , the condition is satisfied according to Theorem~\ref{thm egal quot deriv part}. Let  with  and . Then . According to Theorem~\ref{thm egal quot deriv part}, it holds that    that equals . By recurrence hypothesis, .       
      Consequently . Since by definition,  is equal to , then 
      
      \item Let  be a word in . By definition, it is equivalent to the fact that there exists an interpretation  such that . We have shown that there exists an interpretation  such that  if and only if there exists an interpretation  such that , which is equivalent by definition of  to .
    \end{enumerate}
    \qed
  \end{proof}
  
  \begin{corollary}
    Let  be an expression environment and let  be a constrained expression over . Let  be a word in . Then:
        
  \end{corollary} 
  
  \begin{example}\label{ex derivativ}
    Let us consider the expression  from Example~\ref{ex exp cons lang}. Let us compute the constrained derivative of  w.r.t. the symbol . Since the expression starts with a variable, an assumption has to be made and the process can be expressed as follows: 
    \begin{enumerate}
      \item Maybe the variable  starts with an . In this case, we replace all the occurrences of  by  except the first one, where the symbol  is erased by the derivation. Hence we get the tuple  with .
      \item Otherwise the variable  can be replaced by , and we try to derive the obtained expression . The catenation  implies that we have to make the assumption that  starts with the symbol . In this case, we replace all the occurrences of  by  except the first one, where the symbol  is erased by the derivation. Hence we get the tuple  with .
    \end{enumerate}    
    Hence:
        
        The constrained derivative of  w.r.t. the word  is obtained by computing the constrained derivative of  and  w.r.t. :   
        
        Hence
        
    \qed
  \end{example}
  
  \begin{example}
Let us continue Example~\ref{ex deriv anbncn abc}.
    Let us consider the expression  of Example~\ref{ex anbncn} and its derived term 
    
    w.r.t. .    
      Let us consider an expression interpretation  of Example~\ref{ex anbncn} that satisfies
      
      By considering a realization  that associates ,  and  with , one can check that
      
      Then
        
       Furthermore
       
      Therefore  and consequently .       
  \end{example}
  
  \section{Membership Test of  for Constrained Expressions}\label{sec:eps membership test}
  
  In this section, we consider the membership test for the empty word . We first consider the case where both the interpretation and the realization are fixed, which is a case where this test is decidable. Then we consider the two other cases and show that they are equivalent to a satisfiability problem.
  
  \subsection{-Language and }
  
  Corollary~\ref{cor i r lang rat} asserts that the -language denoted by a constrained expression is regular. Consequently, any membership test can be performed \emph{via} the regularization. However, this transformation may be avoided by directly and inductively computing the classical predicate function  and embedding the regularization in its computation.
  
  \begin{definition}[ Predicate]
    Let  be an expression environment and let  be a constrained expression over . Let  be  an expression interpretation over  and  be a -realization over . The boolean  is defined by:
        
  \end{definition}
  
  \begin{proposition}
    Let  be an expression environment and let  be a constrained expression over . Let  be  an expression interpretation over  and  be a -realization over . Then the boolean  is inductively computed as follows:
        
        where  is any integer,  is any -ary boolean operator,  is the boolean operator associated with ,  are any  constrained expression over ,  is any word in  and  is any boolean formula in .
  \end{proposition}
  \begin{proof}
    By induction over the structure of constrained expressions.
      \begin{multicols}{2}
        
      \end{multicols}
    \qed
  \end{proof}
  
  \subsection{General Cases}
  
  As was the case for derivating, the computation of the  predicate  needs assumptions to be made. However, we only need here to determine which variable symbols have to be transformed into the empty word. Since we need to "erase" several symbols at the same time, we define several notations to perform the corresponding substitutions.
  
  Let  be an expression environment.
We denote by  the set of the functions from  to . 
  Let . We denote by  the substitution defined by 
  
    


  Let  be a word in . We denote by  the subset  of .  
  We denote by  (resp. ) the -ary boolean operator True (resp. False).  
      Given a formula , we denote by  the formula inductively computed by:
        
    
  The general computation of the  predicate  takes into
account two aspects of an expression: first, it needs to consider the expression itself, in order to determine if  may appear; but secondly, it has to consider the fact that several formulae that appear in the expression have to be satisfied, otherwise the language may be empty. Hence we first compute a particular indicator set, made of 
tuples composed of a set of variable symbols that need to be erased and a formula that needs to be satisfied.
  
  \begin{definition}[]
    Let  be an expression environment and let  be a constrained expression over . We denote by  the subset of  inductively defined by:
    \begin{multicols}{2}
        
      \end{multicols}
        where for any two subsets ,  of ,  .
  \end{definition}
  
  Using this previous indicator set, it can be shown that the computation of the different  predicates is equivalent to different satisfiability problems. 
  
  \begin{theorem}\label{thm caract null i r}
    Let  be an expression environment and let  be a constrained expression over . Let  be  an expression interpretation over . Let  be a realisation in . Then the  following two conditions are equivalent:
    \begin{itemize}
        \item 
        \item there exists  in  such that the  following two conditions are satisfied:
          \begin{itemize}
            \item ,   ,
            \item .
          \end{itemize}
     \end{itemize}
  \end{theorem}
  \begin{proof}
    By induction over the structure of .
    
    Let us say that a tuple  in  satisfies the condition  if the  following two conditions are satisfied:
          \begin{itemize}
            \item ,   ,
            \item .
          \end{itemize}
          
    Hence the second condition of the equivalence can be rephrased as "there exists a tuple  in  that satisfies ", formally denoted by .
            
    \qed
  \end{proof} 
  
  
 
  \begin{definition}[ Predicate]
    Let  be an expression environment and let  be a constrained expression over . Let  be  an expression interpretation over . The boolean  is defined by:
        
  \end{definition}
  
  \begin{corollary}\label{cor nullIE lien sat}[of Theorem~\ref{thm caract null i r}]
    Let  be an expression environment and let  be a constrained expression over . Let  be  an expression interpretation over . Then the  following two conditions are equivalent :
    \begin{itemize}
        \item ,
        \item there exists  in  and  in  such that .
     \end{itemize}
  \end{corollary}
  
 
  
  \begin{definition}[ Predicate]
    Let  be an expression environment and let  be a constrained expression over . Let  be  an expression interpretation over . The boolean  is defined by:
        
  \end{definition}
  
  \begin{corollary}\label{cor nullE lien sat}[of Theorem~\ref{thm caract null i r}]
    Let  be an expression environment and let  be a constrained expression over . Then the  following two conditions are equivalent :
    \begin{itemize}
        \item 
        \item there exists  in ,  in  and  in  such that .
     \end{itemize}
  \end{corollary}
  
  \begin{example}
    Let us consider the expression  and its constrained derivative w.r.t.  (Example~\ref{ex derivativ}):
              
      In order to decide whether  belongs to , let us test whether there is an expression  in  such that .
      Let us consider the expression . 
Since epsilon should be matched,  has to be nullable and both  and  have to be realized as epsilon. 
      Consequently, the boolean formula, which has to be satisfied, is transformed into . This step is exactly what the set  computes:
        
        If there exists an interpretation  and a realization  associating  and  with  such that , then  belongs to . Finally, considering the same interpretation and a realization  associating  with  and  with ,  and then  belongs to the -language denoted by , \emph{i.e.} .
      \qed
  \end{example}
  
  \section{Decidability Considerations}\label{sec:decidab}
  
  The previous section (Corollary~\ref{cor nullIE lien sat} and Corollary~\ref{cor nullE lien sat}) shows that the membership test is equivalent to a classical satisfiability problem. Moreover, it is well-known that such a problem can be undecidable when the interpretation is fixed but there is no realization.
  
   \begin{theorem}
     Let  be an expression environment and  be a boolean formula in . Then there exists an interpretation  in  such that :
     
   \end{theorem}   
   \begin{proof}
      Let  be an integer. Let . Let  be a system of diophantine equations with  variables and  be a symbol in .      
     Let us consider the expression interpretation  such that  .
     Let . 
     Then there exists a solution  for  if and only if there exists a realization  that associates for any integer  in  the variable  with a word  of length  such that .
     The solvability of diophantine systems (a.k.a. the tenth Hilbert problem) has been proved to be undecidable by Matiyasevich~\cite{My93}. 
     Hence to determine whether or not "there exists a realization  in  such that " is undecidable.
     \qed
   \end{proof}
  
  However, given a boolean formula  in , to determine whether or not "there exists an interpretation  in  and a realization  in  such that " is decidable, only using propositional logic.
  
  \begin{theorem}\label{thm sat int exp avc rien est deci}
     Let  be an expression environment. Given a boolean formula  in , to determine whether or not there exists an interpretation  in  and a realization  in  such that  is decidable.
   \end{theorem} 
   
   The next subsections are devoted to proving Theorem~\ref{thm sat int exp avc rien est deci}. We first show that any boolean formula can be transformed into a propositional formula (which is a boolean formula with only -ary predicate symbols). Then we show that any formula is equisatisfiable to its propositionnal form whenever there exists an evaluation which is an injection. We finally show that any formula admits an equivalent formula such that an injection exists.
   
  \subsection{Propositionalisation}
  
  The propositionalisation of a boolean formula is performed by replacing any predicate by a unique symbol; in fact, any predicate appearing in the formula is considered as a new symbol.
  
  \begin{definition}[Propositionalisation]\label{def propositionalisation}
    Let  be an expression environment. The \emph{propositionalisation} of a boolean formula  in  is the transformation  inductively defined as follows:
        
        where  is any integer,  is any predicate symbol in ,  are any  elements in ,  is any -ary boolean operator associated with a mapping  from  to  and  are any  boolean formulae in . The symbol  is the \emph{propositional predicate symbol associated with} the term .
  \end{definition}
    
  
  \begin{definition}[Propositional Alphabet]\label{def propositional alphabet}
    Let  be an expression environment. The \emph{propositional alphabet} of a boolean formula  in  is the set  inductively defined as follows:
        
        where  is any integer,  is any predicate symbol in ,  are any  elements in ,  is any -ary boolean operator associated with a mapping  from  to  and  are any  boolean formulae in .
  \end{definition}
  
  \begin{proposition}
    Let  be an expression environment. Let  be a boolean formula in . Then:
            
    Furthermore,  is a finite set.
  \end{proposition}
  \begin{proof}
    Inductively deduced from Definition~\ref{def propositionalisation} and from Definition~\ref{def propositional alphabet}.
    \qed
  \end{proof}
  
  Two of the main interests of these propositional formulae are that \textbf{(I)} they do not need realization to be evaluated (since there is no variable symbols nor terms) and \textbf{(II)} their satisfiability is decidable, using truth tables for example.
  
  Let us now show that the Propositionalisation may produce an equisatisfiable formula.
  
  \subsection{Equisatisfiability of the Propositionalisation}
  
  Once the propositionalisation has been applied over a formula, it can be determined if the obtained formula is satisfiable. This leads to two cases corresponding to the following propositions.
  
   \begin{proposition}\label{prop si tphi contra phi aussi}
    Let  be an expression environment. Let  be a boolean formula in . Let  be an interpretation in  and  be a realization in . Let  be the expression interpretation over  such that for any symbol  in , .
    Then:
        
  \end{proposition}
  \begin{proof}
    By induction over the structure of .
    
    If , then .
    
    If , then 
        
    \qed
  \end{proof}
  
  \begin{corollary}\label{cor tphi contra phi aussi}
    Let  be an expression environment. Let  be a boolean formula in . Then:
    \begin{itemize}
      \item If  is a contradiction, so is .
      \item If  is a tautology, so is .
    \end{itemize}
  \end{corollary}
  
  However, the satisfiability of , when it is not a tautology, is not sufficient to conclude over the satisfiability of . 
  Indeed, 
  it can happen that two distinct predicates in  have to be evaluated differently while the associated predicates cannot be in . 
  As an example, consider the formulae  and . The formula  is satisfiable when  is true and   is not. However, whatever the realization considered,  and  will always be equi-evaluated. Let us formally define the notion of injection that separates two distinct terms while evaluating.  
    
  Let  be an expression environment. Let  be a boolean formula in . The set of the terms of  is the set  inductively defined by:
        
        where  is any integer,  is any predicate symbol in ,  are any  elements in ,  is any -ary boolean operator associated with a mapping  from  to  and  are any  boolean formulae in .
    
  \begin{definition}[Injection]
    Let  be an expression environment. Let  be a subset of . Let  be an interpretation in  and  be a realization in . The function  is said to be an injection of  in  if:
        
  \end{definition}
  
  Given that such an evaluation exists, let us show that the propositionalisation preserves the satisfiability.
  
  \begin{proposition}\label{prop si tphi sat phi aussi}
    Let  be an expression environment. 
    Let  be a boolean formula in . 
    Let  be an interpretation in  and  be a realization in  such that  is an injection of  in . 
    Let . 
    Let  be an expression interpretation over .    
    Let  be an expression interpretation over  satisfying the  following two conditions:
      \begin{itemize}
        \item for any function symbol  in , ,
        \item for any predicate symbol  in ,   .
      \end{itemize}
    
    Then:
           
  \end{proposition}
  \begin{proof} 
  \begin{enumerate}
    \item Let us show that  is an injection of  in . Let  be a term in . 
    \begin{enumerate}
      \item\label{it a prop si tphi} Let us show by induction over the structure of  that .
      \begin{enumerate}
        \item If  in , then .
        \item Let us suppose that  with  any -ary function symbol in  and  any  terms in . Then:
          
      \end{enumerate}  
      \item As a direct consequence of Item~\ref{it a prop si tphi}, since  is an injection of  in , so is .
    \end{enumerate}
    \item  Let us show by induction over  that .
    \begin{enumerate}
      \item If  with  a -ary predicate symbol in , then 
      
      \item Let us consider that . Then:
        
    \end{enumerate} 
    \end{enumerate}
    \qed
  \end{proof}
  
  \begin{corollary}\label{cor tphi sat phi aussi}
    Let  be an expression environment. Let  be a boolean formula in  such that there exists an injection of  in . Then:
        
  \end{corollary}
  
  \begin{example}\label{ex inj form}
    Let  be the expression environment defined by:
    \begin{itemize}
      \item ,
      \item ,
      \item ,
      \item , .
    \end{itemize}
    Let us consider the two boolean formulae defined by:
        
        The terms that appear in these two formulae are:
        
        Let  be the expression interpretation defined by:
    \begin{itemize}
      \item ,
      \item , for any  in , 
      \item 
      \item .
    \end{itemize} 
    Finally, let us consider a realization  that associates  with .  Then
        
        Consequently,  is an injection of  in  but it is not an injection of  in . Furthermore, it holds that
        
        Notice that  is what can we call an \emph{expression contradiction}, since it is a contradiction whenever the function  is interpreted as the catenation function, because of its associativity property. Consequently, for any expression interpretation  and any realization , .
  
  \noindent It is not the case for , since there exists an injection of its terms in . Let us show that  is satisfiable.
  
  \noindent First, we need to compute the formula   associated with . It contains two predicate symbols,  and . Then, let us consider an interpretation  such that  and . Consequently . From this interpretation, we can construct the interpretation  defined by:
    \begin{itemize}
      \item , for any  in , 
      \item ,
      \item ,      
      \item ,
      \item .
    \end{itemize} 
    Then:
        
        The existence of the injection allowed us to show that  was satisfiable \emph{via} the satisfiability of its propositionalised form. Notice that  is satisfiable too, since for any interpretation  satisfying  and , . However, since there is no injection due to the associativity of  in any expression interpretation, the satisfiability of  does not allow us to conclude about the satisfiability of the formula  (see the notion of normalization in the next subsection).  
  \qed
  \end{example}
  

  
  \subsection{Injections for non-Unary Alphabets \emph{via} the Normalization}\label{ssec normalization alph non unaire}
  
  In this subsection, we show that any formula can be transformed into an equivalent one where the set of terms can be evaluated by an injection. In fact, we compute a normal form that takes into account the associativity of the catenation and the identity element . Notice that we do not consider unary alphabets where the catenation is also commutative.
  
  \begin{definition}[Normalized Term]
    Let  be an expression environment. Let  be a term in . The term  is said to be \emph{normalized} if the  following two conditions are satisfied:
    \begin{itemize}
      \item any child of a concatenation node is not equal to ;
      \item the root of the left child of any concatenation node in  is not a concatenation node. 
    \end{itemize}
  \end{definition}   
  
  \begin{definition}[Normalization]
    Let  be an expression environment. The \emph{normalization} of a term  in  is the transformation  inductively defined as follows:
        
        where  is any symbol in ,  is any symbol in  and , ,  are any  terms in .   
  \end{definition}
  
  \begin{definition}[Left-Dot Level]
    Let  be an expression environment. Let  be a term in . The \emph{left-dot level}  is the integer inductively computed as follows: 
        
  \end{definition}
  
  
  \begin{proposition}\label{prop t prim norm}
    Let  be an expression environment. Let  be a term in . Then:
           
    
    Furthermore, whenever  is a normalized term, then .
  \end{proposition}
  \begin{proof}
    By induction over the structure of .
    \begin{enumerate}
      \item If , then  is normalized,  and then .
    
      \item If  with  any symbol in , by induction hypothesis it holds that for any integer  in ,  is normalized and if  is normalized, then . As a direct consequence,  is normalized and if  is normalized, since it implies that for any integer  in ,  is normalized, then .
    
      \item Suppose that . 
\begin{enumerate}
        \item If  (resp. ), then  (resp. ). By induction hypothesis it holds that  (resp. ) is normalized. As a consequence,  is normalized. Notice that in this case,  is not normalized.
\item Suppose that  with . Hence, . According to induction hypothesis,  is normalized and if  is normalized, then . Since , then  is normalized and if  is normalized, then .
\item Suppose that  with  any symbol in  and that . By recurrence over . 
        \begin{enumerate}
          \item If , then  with . Hence . According to induction hypothesis, for any integer  in ,  is normalized and if  is normalized, then . Since , then  is normalized (since the left child of its concatenation root is not a concatenation node). Furthermore, if  is normalized, since it implies that both  and  are normalized and that  and , it holds that . 
          \item Suppose that  with . Then . As a consequence, . Let us notice that . According to recurrence hypothesis,  is normalized. Notice that in this case,  is not normalized.
        \end{enumerate}
      \end{enumerate}
    \end{enumerate}
    \qed
  \end{proof}
  
  Let us show now that the normalization preserves the evaluation.
  
  \begin{proposition}\label{prop t tprim mem eval}
    Let  be an expression environment. Let  be a term in . Let  be an interpretation in  and  be a realization in . Then:
        
  \end{proposition}
  \begin{proof}
    By induction over the structure of .
    \begin{enumerate}
      \item If , then . Hence .    
      \item If  with  any symbol in , then . By induction hypothesis, it holds that for any integer  in , . Hence:
            
    Finally, using Definition~\ref{def interpretation}, it holds that .
    
    \item Suppose that .     
    \begin{enumerate}
      \item If  (resp. ), then  (resp. ). By induction hypothesis,  (resp. ). Then:
             
    \item  Suppose that  with  in . Then . By induction hypothesis, . Then:
            
    \item  Suppose that  with  any symbol in  and that . By recurrence over . 
    \begin{enumerate}
      \item  If , then  with . Hence .
     According to induction hypothesis, for any integer  in , .
     As a consequence, 
              
    \item Suppose that  with . Then . As a consequence, . Notice that . According to recurrence hypothesis,
    . Hence
    .
    \end{enumerate}
    \end{enumerate}
    \end{enumerate}    
    \qed
  \end{proof}
  
  Let  be an expression environment. We denote by  the set of normalized terms in .   
  
  \begin{definition}[Normalized Formula]
    Let  be an expression environment. Let  be a boolean formula in . The formula  is said to be \emph{normalized} if any of the terms appearing in it are normalized (\emph{i.e.} if it belongs to ).
  \end{definition} 
  
  
  \begin{definition}[Formula Normalization]
    Let  be an expression environment. The \emph{normalization} of a boolean formula  in  is the transformation  inductively defined as follows:
        
        where  is any integer,  is any predicate symbol in ,  are any  elements in ,  is any -ary boolean operator associated with a mapping  from  to  and  are any  boolean formulae over .
  \end{definition}
  
  \begin{proposition}
    Let  be an expression environment. Let  be a boolean formula in . Then:
         
  \end{proposition}
  \begin{proof}
    By induction over the structure of , this is a direct corollary of Proposition~\ref{prop t prim norm} as an inductive extension of the normalization.
    \qed
  \end{proof}
  
  \begin{proposition}\label{prop normalization preserv eval}
    Let  be an expression environment. Let  be a boolean formula in . Let  be an interpretation in  and  be a realization in . Then:
        
  \end{proposition}
  \begin{proof}
    By induction over the structure of , this is a direct corollary of Proposition~\ref{prop t tprim mem eval} as an inductive extension of the normalization.
    \qed
  \end{proof}
  
  \begin{example}
    Let us consider the formula  of Example~\ref{ex inj form}. Considering the catenation as right-associative, its normalized form is the formula , that is, a classical contradiction.
    \qed
  \end{example}
  
  Let us now show how to compute an injection from a set of normalized terms.
  
  \begin{definition}[Left, Right and Middle Word]
        A word is a \emph{left word} (resp. \emph{right word}, \emph{middle word}) of a term  in  if it belongs to the set  (resp. , ) computed as follows:
        
        where  is an element in ,  is a function symbol in  and  are any  terms in .
    
    A word  is a \emph{factor} of the term  if it is a factor of  where . 
  \end{definition}
  
  \begin{example}\label{ex factor term}
    Let us illustrate the notion of factor:
      \begin{itemize}
         \item The factors of  are .    
         \item The factors of  are .
         \item The factors of  are . 
     \end{itemize}
    \qed
  \end{example}
  
  
  \begin{definition}[ Function]
     Let  be the function from  to  defined for any term  as follows: 
        
  \end{definition}
  
  \begin{lemma}\label{lem exist mot separ}
    Let  be an expression environment such that . Let  be a finite subset of . Then there exists a word  in  such that for any term  in , for any two distinct terms  and  in , it holds:
        
  \end{lemma}
  \begin{proof} 
  Let  be such that  is the smallest integer such that any factor  of a term of  satisfies .
  Let  and .     
  
  If  is neither a subterm of  nor of , then ,  and thus .
    
  Let  be a subterm of  but not of . There exists a factor  of  and any factor  of  (that is, a factor of ) satisfies . Then .
    
  Suppose that  is a subterm of  and of . 
  
  \begin{enumerate}
    \item Suppose that . Then  since .
    \begin{enumerate}
    \item If , then . Hence .
    
    \item\label{1b} Suppose that  with . Then . Hence .
    
    \item\label{1c} Suppose that . Since  is normalized, then .
    \begin{enumerate}
    \item If , then . Since , then  and then .
    
    \item If , then .
    \begin{enumerate}
    \item If  then .
    
    \item Suppose that .  Either  and then  or , and   (Contradiction with the definition of ).
    \end{enumerate}
    \end{enumerate}
    \end{enumerate}
    
    \item  If  with . Then  and .
    \begin{enumerate}
    \item  Suppose that . See case~\ref{1b}.
    
    \item Suppose that . Then . Hence .
    
    \item\label{2c} Suppose that  with  and . Then  and then .
    
    \item Suppose that . Then  . Since , there exists  in  such that . According to induction hypothesis, , it holds that .
    \end{enumerate}
    \item  Suppose that . Then .
    \begin{enumerate}
    \item If , see case case~\ref{1c}.
    
    \item If , then . Hence . 
    
    \item If  with , see case~\ref{2c}.
    
    \item Suppose that . Consequently .
    
    \begin{enumerate}
    \item If  then .
    \begin{enumerate}
    \item If , then . According to induction hypothesis,  . Since , .
    
    \item\label{3d1b} If , then . Either  does not admit  as a prefix of a left word and then  or it does and then  admits  as a factor (contradiction with the definition of ).
    
    \item\label{3d1c} If , then the respective roots of the leftmost subterm of   and  are distinct. Hence .
    
    \end{enumerate}
    \item Suppose that .
    
    \begin{enumerate}
    \item If , see case~\ref{3d1b}.
    
    \item If , then since , it holds that . By induction hypothesis, . Finally, since , .
    
    \item If , then . Hence .
    \end{enumerate}
    \item Suppose that . Then  with .
    \begin{enumerate}
    \item If , see case~\ref{3d1c}.
    
    \item If  or if  with , the respective roots of the leftmost subterm of   and  are distinct. Hence .
    
    \item If , then  with . 
    Two cases can occur: either  or there exists  in  such that . 
    In the first (resp. second) case, it holds by induction that  (resp. ). Consequently, . 
    \end{enumerate}
    \end{enumerate}
    \end{enumerate}
    \end{enumerate}  
    \qed
  \end{proof}
  
  \begin{definition}[Term Index]
        Let us define the index  as the integer computed as follows:
        
        where for any term ,
        
    Let us define for any two terms  and  the set  computed as follows:
        
        where for any term ,
        
    Let us define for any term  the depth  inductively computed as follows:
        
  \end{definition}
  
    \begin{proposition}\label{prop exist inj cas binaire}
    Let  be an expression environment with . Let  be a finite subset of .
    There exist  an interpretation in  and  a realization in  such that the function  is an injection of  in .
  \end{proposition}
  \begin{proof}
    By recurrence over .
    
    \begin{enumerate}
      \item  If  then . Let us show that for any two distinct terms  and  in , for any interpretation  and for any realization , it holds that .     
    By recurrence over . Let  be any interpretation and  be any realization.
    
    \begin{enumerate}
      \item\label{it 1a prop exist inj} Suppose that . Then . Consequently, .
    
      \begin{enumerate}
        \item If , since , then . Hence, .
    
        \item If , then  with  and . Hence . As a consequence,   and then .
      \end{enumerate}
    
    \item  Suppose that . Then  with .
    
      \begin{enumerate}
        \item If , then it is symmetrically equivalent to item~\ref{it 1a prop exist inj}.
    
        \item Suppose that . Then  with . 
    If , then  and then .
    Otherwise, it holds . According to recurrence hypothesis,  and consequently . Consequently, .
      \end{enumerate}
    \end{enumerate}
    
    \item Suppose that .
    
    \begin{enumerate}
      \item Let  be a symbol in  such that  is a subterm of a term in . 
    
      \begin{enumerate}
        \item\label{it 2ai prop exist inj} According to Lemma~\ref{lem exist mot separ}, there exists  in  such that for any two terms  and  in , it holds: .

        \item It holds by recurrence hypothesis that there exists  an interpretation in  and  a realization in  such that the function  is an injection of  in . Let us consider the realization  defined for any symbol  in  as follows:
        
    Let us show that  is an injection of  in . 
    
    Let  and  be two terms in . According to Item~\ref{it 2ai prop exist inj}, . By definition of , . Since by construction of ,  and since , it holds that .
    \end{enumerate}
    \item Suppose that there is no subterm of a term in  that belongs to . Let  be a subterm in a term in  such that  are  terms in .  
    
    \begin{enumerate}
      \item\label{it 2bi prop exist inj} According to Lemma~\ref{lem exist mot separ}, there exists  in  such that for any two terms  and  in , it holds that: . 
        
      \item It holds by recurrence hypothesis that there exists  an interpretation in  and  a realization in  such that the function  is an injection of  in . 
    Let us denote by  the word  for any integer  in .
    Let us consider the interpretation  defined as follows:
    \begin{enumerate}
      \item for any predicate symbol  in , ,
      \item for any function symbol  in , ,
      \item for any  word  in :
            
    \item 
  \end{enumerate}
    
    Let us show that  is an injection of  in . 
    
        Let  and  be two terms in . According to Item~\ref{it 2bi prop exist inj}, . By definition of , . Since by construction of ,  and since , it holds that .
        \end{enumerate}
        \end{enumerate}
    \end{enumerate}
    \qed
  \end{proof} 
  
  
  
  \begin{proposition}
  There exists an expression environment  with  and a finite subset  of  such that for any interpretation  in , for any realization  in ,  is not an injection of  in .
  \end{proposition}
  \begin{proof}
    Let  and .
    By definition, both  and  are normalized.
    Since the catenation product is commutative for unary alphabets, for any interpretation  in , for any realization  in , , and therefore  is not an injection of  in .
  \qed
  \end{proof}
  
  \begin{example}
    Let us consider the terms  and  and their factors (Example~\ref{ex factor term}).
      \begin{itemize}
         \item The factors of  are .
         \item The factors of  are . 
     \end{itemize}
     Consider the word  which is not in . Consider a realization  associating  with . Let us substitute  with  in  and :
        
        The word  is neither a factor of  nor of . Consider an interpretation  where . Let us substitute  with  in  and :
        
        The word  is neither a factor of  nor of . Consider that the interpretation  satisfies . Let us substitute  with  in  and :
        
        Hence, since  and  are distinct, the function  is an injection of   in .     
    \qed
  \end{example}
  
  In conclusion, the following corollary holds from Proposition~\ref{prop exist inj cas binaire}, Proposition~\ref{prop normalization preserv eval} and Corollary~\ref{cor tphi sat phi aussi}:
  
  \begin{corollary}
    Let  be an expression environment such that . Let  be a boolean formula in . Then the  following two conditions are equivalent:
    \begin{itemize}
      \item  is satisfiable,
\item  is satisfiable.
    \end{itemize}
  \end{corollary}
  
  \begin{corollary}
    Let  be an expression environment and let  be a constrained expression over . Then the boolean  can be computed.
  \end{corollary}
  
  \begin{corollary}
    Let  be an expression environment,  be a constrained expression over  and  be a word in . The membership test of  in  is decidable.
  \end{corollary}
  


\section{Conclusion and Future Work}

  In this paper, we have extended the expressive power of regular expressions by the addition of two new operators involving the zeroth order boolean formulae leading to the notion of constrained expressions. We have presented a method in order to solve the membership problem in the general case where the interpretation is not fixed and when the alphabet is not unary.
  
  An interesting continuation would be to consider the case of unary alphabets by extending the normalization defined in Subsection~\ref{ssec normalization alph non unaire} with the commutativity of the catenation; indeed, as far as a unary alphabet is considered, two words commute. Hence, any term has to be sorted according to an order (\emph{e.g.} the lexicographic order). We conjecture that Proposition~\ref{prop exist inj cas binaire} still holds for unary case, considering the word  with  for any  in a term in  instead of  (see proof of Lemma~\ref{lem exist mot separ}).
  
   We have also shown that the membership problem can be undecidable when the interpretation is fixed. However, we can express a sufficient condition for the membership test to be decidable: whenever the interpretation  is fixed, if it can be decided if a formula  is satisfiable (\emph{e.g.} if there exists a realization  such that ), then (and trivially) the membership problem can be solved.
   Let us consider the following definition.
   
  \begin{definition}
    Let  be an expression environment, and  be an expression interpretation over .
    The interpretation  is \emph{decidable} if for any boolean formula  in , the existence of a realization  such that  is decidable.
  \end{definition}

  According to the previous definition, another perspective is to restrain the  operator and the denoted language 
  in order to embed a decidable interpretation in the predicate, yielding the notion of a decidable constrained expression.   
   As an example, the predicate of length equality is decidable, and the membership of an expression using it is decidable (\emph{e.g.}  is such an expression).
  It is an open question to determine if the decidability of an interpretation is decidable, and how it can be characterized.
  
\bibliography{D:/DocsSyncro/Recherche/Bibliographie/biblio}




\end{document}
