In this section we show that the problem~ is in~{\sf NC}, hence in~{\sf P}.

First, in Subsection~\ref{subsec:link-trace-refinement-bisim}, we establish a link between trace refinement and a notion of bisimulation between distributions that was studied in~\cite{Jans}.

Second, in Subsection~\ref{subsec:condition_bisimilar} we give a necessary and sufficient condition for the MDPs to be bisimilar.
It resembles the properties developed in~\cite{Jans}, but we rebuild a detailed proof from scratch, as the authors were unable to verify some of the technical claims made in~\cite{Jans}.

As corollaries, we show in Subsection~\ref{subsec:algo_conp_bisimilarity} that bisimulation between two MDPs can be decided in~{\sf coNP}, improving the exponential-time result from~\cite{Jans},
and in Subsection~\ref{subsec:algo_nc_trace_refinement} that the problem~ is in~{\sf NC}, hence in~{\sf P}.

\subsection{A Link between Trace Refinement and Bisimulation}\label{subsec:link-trace-refinement-bisim}

A \emph{local strategy} for an MDP  is a function~
that maps each state~ to a distribution~ over moves in~.
We call~ \emph{pure} if for all states~ there is a move~ such that .
For a subdistribution~, a local strategy~, and a label~,
define the \emph{successor} subdistribution  with

for all .
We often view a subdistribution~ as a row vector .
For a local strategy  and a label~,
define the \emph{transition matrix}  with
.
Viewing subdistributions~ as row vectors, we have:


For a trace-based strategy  and a trace ,
define the local strategy  with  for all .
We have the following lemma.
\begin{lem}\label{lem-subDis-as-matrix-prod}
Let  be an MDP\@.
Let  be a trace-based strategy.
Let  and .
Then:

\end{lem}
\begin{proof}
Let .
We have:

\end{proof}

The following lemma is based on the idea that, using Lemma~\ref{lem-subDis-as-matrix-prod}, we can ``slice'' a strategy into local strategies, and conversely we can compose local strategies to a strategy.
\begin{lem}\label{lem-link-trace-local}
Let  be an MDP\@.
Let .
Let  be subdistributions over~.
Then there is a strategy  with

if and only if there are local strategies  with

\end{lem}

\begin{proof}
We prove the two implications from the lemma in turn.
\begin{itemize}[align=left]
\item[``'':]
Let  be a strategy with

for all .
By Lemma~\ref{lem-trace-based-enough} we can assume that  is trace-based.
For all  define a local strategy  with .
Then we have for all :

\item[``'':]
Let  be local strategies with

for all .
Define a trace-based strategy~ such that
 for all .
(This condition need not completely determine~.)
We prove by induction on~ that

for all .
For  this is trivial.
For the step, we have:

\end{itemize}
\end{proof}

\noindent
Let    and  be two MDPs over
the same set~ of labels.
A \emph{bisimulation} is a relation~ such that whenever
  then
\begin{itemize}
\item ;
\item for all local strategies~ there exists a local strategy~ such that for all  we have ;
\item for all local strategies~ there exists a local strategy~ such that for all  we have .
\end{itemize}
As usual, a union of bisimulations is a bisimulation.
Denote by~ the union of all bisimulations, i.e.,  is the largest bisimulation.
We write  if .
In general, the set~ is uncountably infinite, so methods for computing state-based bisimulation (e.g., partition refinement) are not applicable.

Proposition~\ref{prop-link-trace-refinement-bisim} below establishes a link between trace refinement and bisimulation.
An intuitive interpretation of the proposition is that if  is an MDP and  an MC, then the best way of disproving bisimilarity between  and~ is to exhibit a sequence of local strategies in~ so that the resulting behaviour of~ cannot be matched by~.
Using Lemma~\ref{lem-link-trace-local} this sequence of local strategies can be assembled to a strategy for~, which then witnesses that .

\begin{prop}\label{prop-link-trace-refinement-bisim}
Let  be an MDP and  be an MC\@.
Then  if and only if .
\end{prop}

\begin{proof}
Let  and .
\begin{itemize}[align=left]
\item[``'':]
Let .
Hence .
We show that .
Let  be a strategy for~.
Let .
Let  be the subdistributions with
  for all~.
By Lemma~\ref{lem-link-trace-local} there exist local strategies  with
 for all~.
Since ,
there exist local strategies  for~ and subdistributions  with
 for all~ and
 for all~.
Since  is an MC, the local strategies~ are, in fact, irrelevant.
By Lemma~\ref{lem-link-trace-local} we have  for all~.
So we have:

Since  and~ were chosen arbitrarily, we conclude that .
\item[``'':]
Let .
We show .
Define a relation 
such that  if and only if there exist a strategy~ for~ and a trace~ with
 and .
We claim that  is a bisimulation.
To prove the claim, consider any  with .
Then there exist a strategy~ for~ and a trace~ with
 and .
Since , we have .
So we have:

This proves the first condition for  being a bisimulation.

For the rest of the proof assume .
Write  and .
Let  be a local strategy for~.
Let .
Define ,
   and  for an arbitrary (and unimportant as  is an MC) local strategy  for~.
For the second and the third condition of  being a bisimulation we need to prove
.
Define  such that
 for all .
By Lemma~\ref{lem-link-trace-local} there are local strategies 
such that  for all .
We also have ,
so again by Lemma~\ref{lem-link-trace-local} there is a strategy~ with
 for all . In particular, .
Similarly, we have .
Thus, .
Hence we have proved that  is a bisimulation.

Considering the empty trace, we see that .
Since , we also have , as desired. \qedhere
\end{itemize}
\end{proof}


\subsection{A Necessary and Sufficient Condition for Bisimilarity}\label{subsec:condition_bisimilar}

In the following we consider MDPs  and  over the same state space.
This is without loss of generality, since we might take the disjoint union of the state spaces.
Since  and~ differ only in the initial distribution, we will focus on~.


Let  with .
Assume the label set is .
For  and a local strategy~ we define a point  such that

For the reader's intuition, we remark that we will choose matrices  so that if two subdistributions  are bisimilar then .
(In fact, one can compute~ so that the converse holds as well, i.e.,  if and only if .)
It follows that, for subdistributions  and local strategies ,
if  holds for all  then .
Let us also remark that for fixed , the set  is a (bounded and convex) polytope.
As a consequence, if  then the polytopes  and~ must be equal.
In the next paragraph we define \emph{``extremal''} strategies~, which intuitively are local strategies such that  is a vertex of the polytope~.

Let  be a \emph{column} vector;
we denote column vectors in boldface.
We view~ as a ``direction''.
Recall that~ is the Dirac distribution on the state~.
A pure local strategy~ is \emph{extremal in direction~ with respect to~} if

for all states~ and all pure local strategies~.

By linearity, if~\eqref{eq-remove-v1} and~\eqref{eq-remove-v2} hold for all pure local strategies~ then~\eqref{eq-remove-v1} and~\eqref{eq-remove-v2} hold for all local strategies~.
We say a local strategy  is \emph{extremal with respect to~} if there is a direction~ such that  is extremal in direction~ with respect to~.

In the following we prove some facts about extremal local strategies that will be needed later.

\begin{lem}\label{lem-remove-v}
Let  be an MDP\@.
Let  with .
Let .
Let  be local strategies.
Suppose  is a direction in which  is extremal and
.
Then .
\end{lem}
\begin{proof}
We have:

With~\eqref{eq-remove-v1} it follows that for all  we have .
Hence by~\eqref{eq-remove-v2} we obtain  for all .
Thus:

\end{proof}

For a subdistribution~ define the bounded, convex polytope  with

Comparing two polytopes  and~ for subdistributions  will play a key role for deciding bisimulation.
First we prove the following lemma, which states that any vertex of the polytope~ can be obtained by applying an extremal local strategy.
Although this is intuitive, the proof is not very easy.

\begin{lem}\label{lem-vertex-extremal}
Let  be an MDP\@.
Let  with .
Let .
If  is a vertex of~ then there is an extremal local strategy~ with .
\end{lem}
\begin{proof}
Let  be a vertex of~.
Let  be a local strategy so that .
Since  is a vertex, we can assume that  is pure.
Since  is a vertex of~, there is a hyperplane  such that .
Let  be a normal vector of~.
Since , we have  or ; without loss of generality, say .
Since , we have for all  and all :


For all , redefine the pure local strategy~ so that all  and all local strategies~ satisfy .
Since ~and~ are finite, there is  such that all  and all \emph{pure} local strategies~  either satisfy  or .

Define

Consider the bounded, convex polytope  defined by

By an argument similar to the one above, there are a pure local strategy , a vertex  of~, and a vector  such that for all  and all , we have , and if  then .
By scaling down~ by a small positive scalar, we can assume that all  and all local strategies~ satisfy


Since , all  satisfy .
By~\eqref{eq-lem-vertex-extremal-suppD} all  satisfy .
Hence:


It remains to show that there is a direction~ in which  is extremal.
Take .
Let  and let  be a pure local strategy.
We consider two cases:
\begin{itemize}
\item
Assume .
Then there is  with , hence .
We have:

Hence~\eqref{eq-remove-v1} holds for~.
To show~\eqref{eq-remove-v2}, assume .
Then all terms in the computation above are equal, and .
By the definition of~, this implies .
Hence .
Hence~\eqref{eq-remove-v2} holds for~.
\item
Assume .
By the definition of~ it follows .
We have:

This implies~\eqref{eq-remove-v1} and~\eqref{eq-remove-v2} for~.
\end{itemize}
Hence,  is extremal in direction~.
\end{proof}

The following lemma states the intuitive fact that in order to compare the polytopes  and~, it suffices to compare the vertices obtained by applying extremal local strategies:

\begin{lem}\label{lem-check-extremal-enough}
Let  be an MDP\@.
Let  with .
Then for all  we have  if and only if for all extremal local strategies~ we have .
\end{lem}
\begin{proof}
We prove the two implications from the lemma in turn.
\begin{itemize}[align=left]
\item[``'':]
Suppose .
Let  be a local strategy that is extremal in direction~.
Since , there are  and~ such that
 and
.
We have:

So all inequalities are in fact equalities.
In particular, we have .
It follows:

\item[``'':]
Let  be a vertex of~.
By Lemma~\ref{lem-vertex-extremal} there exists an extremal local strategy~ with .
By the assumption we have .
Hence .
Since  is an arbitrary vertex of~, and  are bounded, convex polytopes, it follows .
The reverse inclusion is shown similarly. \qedhere
\end{itemize}
\end{proof}

\noindent
The following lemma shows that the alternation of quantifiers over local strategies can be replaced by quantifying over extremal local strategies:

\begin{lem}\label{lem-sim-pivot}
Let  be an MDP\@.
Let  with .
Let .
In the following let  range over local strategies, ~over extremal local strategies, and  over~.
Then

holds if and only the following holds:

\end{lem}

\begin{proof}
We have:


\end{proof}

The following proposition provides necessary and sufficient conditions for bisimilarity, which---as we will see---can be effectively checked.

\begin{prop}\label{prop-coNP-vector-space}
Let  be an MDP\@.
Let  with .
\begin{enumerate}
\item[(1)] Suppose that for all  with  we have .
    Then for all  with  we have  for all extremal local strategies~ and all .
\item[(2)] Suppose that  includes the column vector  (where the superscript~ denotes transpose) and that for all extremal local strategies~ and all  the columns of  are in the linear span of the columns of~.
    Then for all  with  we have .
\end{enumerate}
\end{prop}

\begin{proof}\leavevmode
\begin{itemize}[align=left]
\item[(1)]
Let  with .
By the definition of bisimulation and using~\eqref{eq-Succ-as-prod}, we obtain:
\begin{itemize}
\item for all local strategies~ there exists a local strategy~ such that for all  we have ;
\item for all local strategies~ there exists a local strategy~ such that for all  we have .
\end{itemize}
Using our assumption on~, we see that~\eqref{eq-lem-sim-pivot} from Lemma~\ref{lem-sim-pivot} holds for .
By Lemma~\ref{lem-sim-pivot} we have  for all extremal local strategies~ and all .
\item[(2)]
It suffices to show that the relation  defined by

is a bisimulation.
Let , i.e., .
Since  includes the column~, we have .
Since for all extremal local strategies~ and all  the columns of  are in the linear span of the columns of~, we have  for all extremal local strategies~ and all .
Lemma~\ref{lem-sim-pivot} implies that~\eqref{eq-lem-sim-pivot} holds for .
Using~\eqref{eq-Succ-as-prod} and the definition of~, we obtain:
\begin{itemize}
\item for all local strategies~ there exists a local strategy~ such that for all  we have ;
\item for all local strategies~ there exists a local strategy~ such that for all  we have .
\end{itemize}
Thus the relation~ is a bisimulation. \qedhere
\end{itemize}
\end{proof}

\subsection{A coNP Algorithm for Checking Bisimilarity of two MDPs}\label{subsec:algo_conp_bisimilarity}

Proposition~\ref{prop-coNP-vector-space}
suggests an algorithm for determining bisimilarity in a given MDP .
More concretely, we can compute a matrix~ such that for all subdistributions  we have  if and only if .
The algorithm initializes~ with the column vector~ and henceforth maintains the invariant that  implies .
\begin{itemize}
\item
If there exists an extremal local strategy~ and a label  such that a column of  is linearly independent of the columns of~, then add this column to~.
This maintains the invariant by Proposition~\ref{prop-coNP-vector-space}~(1).
Repeat.
\item
Otherwise (i.e., such  and~ do not exist) terminate.
Then by Proposition~\ref{prop-coNP-vector-space}~(2) we have .
Together with the invariant we get , as claimed.
\end{itemize}
This algorithm terminates because  can have at most  linearly independent columns.

Along these lines we can also prove the following theorem.

\begin{thm}\label{thm-coNP-result}
The problem that, given two MDPs~ and , asks whether  is in {\sf coNP}.
\end{thm}

\begin{proof}
Without loss of generality we assume
 and
.
Hence we wish to decide in {\sf NP} whether .

We proceed along the lines of the algorithm above.
Specifically, it follows from the arguments there that  holds if and only if the following condition~ holds:
\begin{quote}
There are  and  and 
and 
and  and pure local strategies  such that for all 
\begin{itemize}
\item  is extremal with respect to the matrix formed by the column vectors  and
\item  and
\item ,
\end{itemize}
and .
\end{quote}
It remains to argue that  can be checked in~{\sf NP}.
We can nondeterministically guess  and  and  and pure local strategies .
This determines .
All conditions in~ are straightforward to check in polynomial time, except the condition that for all  we have that  is extremal with respect to .
In the remainder of the proof, we argue that this can also be checked in polynomial time.

Let .
Let  be the matrix with columns .
We want to check that  is extremal with respect to~.
For all , compute in polynomial time the set  defined by

where  is a pure local strategy with  (it does not matter how  is defined for ).
We want to verify that~\eqref{eq-remove-v1}~and~\eqref{eq-remove-v2} holds for~.
Hence we need to find~ so that for all  and all  we have .
If such a vector~ exists, it can be scaled up by a large positive scalar so that we have:

Hence it suffices to check if there exists a vector~ that satisfies~\eqref{eq-v-lp}.
This amounts to a feasibility check of a linear program of polynomial size.
Such a check can be carried out in polynomial time~\cite{Khachiyan79}.
\end{proof}

\subsection{An NC Algorithm for Trace Refinement}\label{subsec:algo_nc_trace_refinement}

In the following we consider an MDP  and an MC .
Without loss of generality, we assume .
Similarly, we also assume that  is a restriction of~, and hence we write .
We may view subdistributions  as  in the natural way.
The following proposition is analogous to Proposition~\ref{prop-coNP-vector-space}.
The key difference is that the need for considering \emph{extremal} strategies has disappeared.
This is due to the fact that only one of the two models is nondeterministic.

\begin{prop}\label{prop-coNP-vector-space-MC}
Let  be an MDP
and  be an MC with .
Let  with .
In the following let  range over~ and  over~.
\begin{enumerate}
\item[(1)] Suppose that for all  with  we have .
    Then for all  with  we have  for all local strategies~ and all .
\item[(2)] Suppose that  includes the column vector  and that for all local strategies~ and all  the columns of  are in the linear span of the columns of~.
    Then for all  with  we have .
\end{enumerate}
\end{prop}

\begin{proof}
The proof is similar to but simpler than the proof of Proposition~\ref{prop-coNP-vector-space}.
For completeness, we give it explicitly.
\begin{itemize}[align=left]
\item[(1)]
Let .
By the definition of bisimulation and using~\eqref{eq-Succ-as-prod} we have  for all local strategies~ and all .
By our assumption on~, we have  for all local strategies~ and all .
\item[(2)]
It suffices to show that the relation  defined by

is a bisimulation.
Let , i.e., .
Since  includes the column~, we have .
Since for all local strategies~ and all  the columns of~ are in the linear span of the columns of~, we have  for all local strategies~ and all .
Using~\eqref{eq-Succ-as-prod} and the definition of~, we see that
for all local strategies~ and all  we have .
Thus the relation~ is a bisimulation. \qedhere
\end{itemize}
\end{proof}

\begin{cor}\label{cor-coNP-vector-space-MC}
Let  be the smallest column-vector space with  and  for all , all labels , and all local strategies~.
Then for all  and all  we have:

\end{cor}

Notice the differences to Proposition~\ref{prop-coNP-vector-space}: there we considered all extremal local strategies (potentially exponentially many), here we consider all local strategies (in general infinitely many).
However, we show that one can efficiently find few local strategies that span all local strategies.
This allows us to reduce (in logarithmic space) the bisimulation problem between an MDP and an MC to the bisimulation problem between two MCs,
which is equivalent to the trace-equivalence problem in MCs (by Proposition~\ref{prop-link-trace-refinement-bisim}).
The latter problem is known to be in~{\sf NC}~\cite{Tzeng96}.
Theorem~\ref{thm-MDP-MC} then follows with Proposition~\ref{prop-link-trace-refinement-bisim}.

\begin{thm}\label{thm-MDP-MC}
The problem~ is in~{\sf NC}, hence in~{\sf P}.
\end{thm}

\begin{proof}
Let  be an MDP
and  be an MC with .

Let  denote an arbitrary pure local strategy.
For each  and each  denote by~ the pure local strategy such that  and  for all .
Define

The vector space  from Corollary~\ref{cor-coNP-vector-space-MC} is the smallest vector space with
\begin{itemize}
\item
 and
\item , for all  and all .
\end{itemize}

\noindent
We have , where  is finite and  is infinite.
Every matrix in~ can be expressed as a linear combination of matrices from~:
Indeed, let  be a local strategy.
Then for all  we have:

So by linearity, the vector space~ is the smallest column-vector space such that
\begin{itemize}
\item
 and
\item , for all  and all .
\end{itemize}

\noindent
Define a finite set of labels , and for each  and each  a matrix

The matrix  is stochastic.
Define the MCs 
and 
such that  induces the transition matrices  for all .
The MCs  and~ are computable in logarithmic space.
Let  be the smallest column-vector space such that
\begin{itemize}
\item
 and
\item , for all  and all .
\end{itemize}
Since the matrices in~ and the matrices~ are scalar multiples of each other, we have .
It holds:

As mentioned in Section~\ref{sub-prel-trace-refine},
deciding whether  holds amounts to the trace-equivalence problem for MCs.
It follows from Tzeng~\cite{Tzeng96} that the latter is decidable in~{\sf NC}, hence in~{\sf P}.
\end{proof}

