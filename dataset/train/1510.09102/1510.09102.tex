In this section we show that the problem~${\sf MDP\sqsubseteq MC}$ is in~{\sf NC}, hence in~{\sf P}.

First, in Subsection~\ref{subsec:link-trace-refinement-bisim}, we establish a link between trace refinement and a notion of bisimulation between distributions that was studied in~\cite{Jans}.

Second, in Subsection~\ref{subsec:condition_bisimilar} we give a necessary and sufficient condition for the MDPs to be bisimilar.
It resembles the properties developed in~\cite{Jans}, but we rebuild a detailed proof from scratch, as the authors were unable to verify some of the technical claims made in~\cite{Jans}.

As corollaries, we show in Subsection~\ref{subsec:algo_conp_bisimilarity} that bisimulation between two MDPs can be decided in~{\sf coNP}, improving the exponential-time result from~\cite{Jans},
and in Subsection~\ref{subsec:algo_nc_trace_refinement} that the problem~${\sf MDP\sqsubseteq MC}$ is in~{\sf NC}, hence in~{\sf P}.

\subsection{A Link between Trace Refinement and Bisimulation}\label{subsec:link-trace-refinement-bisim}

A \emph{local strategy} for an MDP $\D = \tuple{Q,\mu_{0},\Label,\delta}$ is a function~$\alpha : Q \to \dists(\moves)$
that maps each state~$q$ to a distribution~$\alpha(q)\in \dists(\moves(q))$ over moves in~$q$.
We call~$\alpha$ \emph{pure} if for all states~$q$ there is a move~$\m$ such that $\alpha(q)(\m) = 1$.
For a subdistribution~$\mu \in \subdists(Q)$, a local strategy~$\alpha$, and a label~$a \in \Label$,
define the \emph{successor} subdistribution $\Succ(\mu, \alpha, a)$ with
\[
\Succ(\mu, \alpha, a)(q')
= \sum_{q \in Q} \mu(q) \cdot \sum_{\m \in \moves(q)} \alpha(q)(\m) \cdot \m(a, q')
\]
for all $q' \in Q$.
We often view a subdistribution~$d \in \subdists(Q)$ as a row vector $d \in {[0,1]}^Q$.
For a local strategy $\alpha$ and a label~$a$,
define the \emph{transition matrix} $\Delta_{\alpha}(a) \in {[0,1]}^{Q \times Q}$ with
$\Delta_{\alpha}(a)[q,q'] = \sum_{\m \in \moves(q)} \alpha(q)(\m) \cdot \m(a, q')$.
Viewing subdistributions~$\mu$ as row vectors, we have:
\begin{equation} \label{eq-Succ-as-prod}
\Succ(\mu, \alpha, a) = \mu \cdot \Delta_{\alpha}(a)
\end{equation}

For a trace-based strategy $\alpha : \Label^{*} \times Q \to \dists(\moves)$ and a trace $w \in \Label^*$,
define the local strategy $\alpha[w] : Q \to \dists(\moves)$ with $\alpha[w](q) = \alpha(w,q)$ for all $q \in Q$.
We have the following lemma.
\begin{lem}\label{lem-subDis-as-matrix-prod}
Let $\D = \tuple{Q,\mu_0,\Label,\delta}$ be an MDP\@.
Let $\alpha : \Label^{*} \times Q \to \dists(\moves)$ be a trace-based strategy.
Let $w \in \Label^*$ and $a \in \Label$.
Then:
\[ \subdis_{\D,\alpha}(w a) = \subdis_{\D,\alpha}(w) \cdot \Delta_{\alpha[w]}(a)
\]
\end{lem}
\begin{proof}
Let $q' \in Q$.
We have:
\begin{align*}
& \subdis_{\D,\alpha}(w a)(q') \\
& = \sum_{\rho \in \Path(w)} \Prob_{\D,\alpha}(\rho a q')
 && \text{definition of $\subdis$} \\
& = \sum_{q \in Q} \sum_{\rho \in \Path(w,q)} \Prob_{\D,\alpha}(\rho) \cdot
     \sum_{\m \in \moves(q)} \alpha(\rho)(\m) \cdot \m(a,q')
 && \text{definition of $\Prob$} \\
& = \sum_{q \in Q} \sum_{\rho \in \Path(w,q)} \Prob_{\D,\alpha}(\rho) \cdot
     \sum_{\m \in \moves(q)} \alpha(w,q)(\m) \cdot \m(a,q')
 && \text{$\alpha$ is trace-based} \\
& = \sum_{q \in Q} \subdis_{\D,\alpha}(w)(q) \cdot
     \sum_{\m \in \moves(q)} \alpha(w,q)(\m) \cdot \m(a,q')
 && \text{definition of $\subdis$} \\
& = \sum_{q \in Q} \subdis_{\D,\alpha}(w)(q) \cdot
     \sum_{\m \in \moves(q)} \alpha[w](q)(\m) \cdot \m(a,q')
 && \text{definition of $\alpha[w]$} \\
& = \sum_{q \in Q} \subdis_{\D,\alpha}(w)(q) \cdot
     \Delta_{\alpha[w]}(a)[q,q']
 && \text{definition of $\Delta_{\alpha[w]}(a)$} \\
& = \left( \subdis_{\D,\alpha}(w) \cdot \Delta_{\alpha[w]}(a) \right) (q')
\tag*{\qedhere}
\end{align*}
\end{proof}

The following lemma is based on the idea that, using Lemma~\ref{lem-subDis-as-matrix-prod}, we can ``slice'' a strategy into local strategies, and conversely we can compose local strategies to a strategy.
\begin{lem}\label{lem-link-trace-local}
Let $\D = \tuple{Q,\mu_0,\Label,\delta}$ be an MDP\@.
Let $w = a_1 a_2 \cdots a_n \in \Label^*$.
Let $\mu_1, \mu_2, \ldots, \mu_n$ be subdistributions over~$Q$.
Then there is a strategy $\alpha : \Path(\D) \to \dists(\moves)$ with
\[
 \mu_i = \subdis_{\D,\alpha}(a_1 a_2 \cdots a_i) \qquad \text{for all $i \in \{0, 1, \ldots, n\}$}
\]
if and only if there are local strategies $\alpha_0, \alpha_1, \ldots, \alpha_{n-1} : Q \to \dists(\moves)$ with
\[
 \mu_{i+1} = \Succ(\mu_i, \alpha_i, a_{i+1}) \qquad \text{for all $i \in \{0, 1, \ldots, n-1\}$}.
\]
\end{lem}

\begin{proof}
We prove the two implications from the lemma in turn.
\begin{itemize}[align=left]
\item[``$\Longrightarrow$'':]
Let $\alpha$ be a strategy with
$
 \mu_i = \subdis_{\D,\alpha}(a_1 a_2 \cdots a_i)
$
for all $i \in \{0, 1, \ldots, n\}$.
By Lemma~\ref{lem-trace-based-enough} we can assume that $\alpha$ is trace-based.
For all $i \in \{0, 1, \ldots, n-1\}$ define a local strategy $\alpha_i$ with $\alpha_i = \alpha[a_1 a_2 \cdots a_i]$.
Then we have for all $i \in \{0, 1, \ldots, n-1\}$:
\begin{align*}
\mu_{i+1}
& = \subdis_{\D,\alpha}(a_1 a_2 \cdots a_{i+1}) && \text{definition of $\alpha$} \\
& = \subdis_{\D,\alpha}(a_1 a_2 \cdots a_{i}) \cdot \Delta_{\alpha[a_1 a_2 \cdots a_i]}(a_{i+1})
    && \text{Lemma~\ref{lem-subDis-as-matrix-prod}} \\
& = \mu_i \cdot \Delta_{\alpha_i}(a_{i+1})
    && \text{definitions of $\alpha, \alpha_i$} \\
& = \Succ(\mu_i, \alpha_i, a_{i+1})
    && \text{by~\eqref{eq-Succ-as-prod}}
\end{align*}
\item[``$\Longleftarrow$'':]
Let $\alpha_0, \alpha_1, \ldots, \alpha_{n-1}$ be local strategies with
$
 \mu_{i+1} = \Succ(\mu_i, \alpha_i, a_{i+1})
$
for all $i \in \{0, 1, \ldots, n-1\}$.
Define a trace-based strategy~$\alpha$ such that
$\alpha[a_1 a_2 \cdots a_i] = \alpha_i$ for all $i \in \{0, 1, \ldots, n-1\}$.
(This condition need not completely determine~$\alpha$.)
We prove by induction on~$i$ that
$
 \mu_i = \subdis_{\D,\alpha}(a_1 a_2 \cdots a_i)
$
for all $i \in \{0, 1, \ldots, n\}$.
For $i=0$ this is trivial.
For the step, we have:
\begin{align*}
\mu_{i+1}
& = \Succ(\mu_i, \alpha_i, a_{i+1}) && \text{definition of $\alpha_i$} \\
& = \mu_i \cdot \Delta_{\alpha_i}(a_{i+1}) && \text{by~\eqref{eq-Succ-as-prod}} \\
& = \subdis_{\D,\alpha}(a_1 a_2 \cdots a_i) \cdot \Delta_{\alpha_i}(a_{i+1}) && \text{induction hypothesis} \\
& = \subdis_{\D,\alpha}(a_1 a_2 \cdots a_i) \cdot \Delta_{\alpha[a_1 a_2 \cdots a_i]}(a_{i+1}) && \text{definition of $\alpha$} \\
& = \subdis_{\D,\alpha}(a_1 a_2 \cdots a_{i+1}) && \text{Lemma~\ref{lem-subDis-as-matrix-prod}}\qedhere
\end{align*}
\end{itemize}
\end{proof}

\noindent
Let   $\D = \tuple{Q_\D,\mu^{\D}_{0},\Label,\deltaD}$ and $\E = \tuple{Q_\E,\mu^{\E}_{0},\Label,\deltaE}$ be two MDPs over
the same set~$\Label$ of labels.
A \emph{bisimulation} is a relation~$\R \subseteq \subdists(Q_\D) \times \subdists(Q_\E)$ such that whenever
 $\mu_\D \mathrel{\R} \mu_\E$ then
\begin{itemize}
\item $\norm{\mu_\D} = \norm{\mu_\E}$;
\item for all local strategies~$\alpha_\D$ there exists a local strategy~$\alpha_\E$ such that for all $a \in \Label$ we have $\Succ(\mu_\D, \alpha_\D, a) \mathrel{\R} \Succ(\mu_\E, \alpha_\E, a)$;
\item for all local strategies~$\alpha_\E$ there exists a local strategy~$\alpha_\D$ such that for all $a \in \Label$ we have $\Succ(\mu_\D, \alpha_\D, a) \mathrel{\R} \Succ(\mu_\E, \alpha_\E, a)$.
\end{itemize}
As usual, a union of bisimulations is a bisimulation.
Denote by~$\mathord{\sim}$ the union of all bisimulations, i.e., $\mathord{\sim}$ is the largest bisimulation.
We write $\D \sim \E$ if $\mu^\D_{0} \sim \mu^{\E}_{0}$.
In general, the set~$\sim$ is uncountably infinite, so methods for computing state-based bisimulation (e.g., partition refinement) are not applicable.

Proposition~\ref{prop-link-trace-refinement-bisim} below establishes a link between trace refinement and bisimulation.
An intuitive interpretation of the proposition is that if $\D$ is an MDP and $\C$ an MC, then the best way of disproving bisimilarity between $\D$ and~$\C$ is to exhibit a sequence of local strategies in~$\D$ so that the resulting behaviour of~$\D$ cannot be matched by~$\C$.
Using Lemma~\ref{lem-link-trace-local} this sequence of local strategies can be assembled to a strategy for~$\D$, which then witnesses that $\D \not\sqsubseteq \C$.

\begin{prop}\label{prop-link-trace-refinement-bisim}
Let $\D$ be an MDP and $\C$ be an MC\@.
Then $\D \sim \C$ if and only if $\D \sqsubseteq \C$.
\end{prop}

\begin{proof}
Let $\D = \tuple{Q_\D,\mu^{\D}_{0},\Label,\delta^{\D}}$ and $\C = \tuple{Q_\C,\mu^{\C}_{0},\Label,\delta^{\C}}$.
\begin{itemize}[align=left]
\item[``$\Longrightarrow$'':]
Let $\D \sim \C$.
Hence $\mu^{\D}_{0} \sim \mu^{\C}_{0}$.
We show that $\D \sqsubseteq \C$.
Let $\alpha^{\D}$ be a strategy for~$\D$.
Let $w = a_1 a_2 \cdots a_n \in \Label^*$.
Let $\mu^{\D}_{0}, \mu^{\D}_{1}, \ldots, \mu^{\D}_{n}$ be the subdistributions with
 $\mu^{\D}_{i} = \subdis_{\D,\alpha^{\D}}(a_1 a_2 \cdots a_i)$ for all~$i$.
By Lemma~\ref{lem-link-trace-local} there exist local strategies $\alpha^{\D}_0, \alpha^{\D}_1, \ldots, \alpha^{\D}_{n-1}$ with
$\mu^{\D}_{i+1} = \Succ(\mu^{\D}_{i}, \alpha^{\D}_i, a_{i+1})$ for all~$i$.
Since $\mu^{\D}_{0} \sim \mu^{\C}_{0}$,
there exist local strategies $\alpha^{\C}_0, \alpha^{\C}_1, \ldots, \alpha^{\C}_{n-1}$ for~$\C$ and subdistributions $\mu^{\C}_1, \mu^{\C}_2, \ldots, \mu^{\C}_n$ with
$\mu^{\C}_{i+1} = \Succ(\mu^{\C}_{i}, \alpha^{\C}_i, a_{i+1})$ for all~$i$ and
$\mu^{\D}_i \sim \mu^{\C}_i$ for all~$i$.
Since $\C$ is an MC, the local strategies~$\alpha^{\C}_i$ are, in fact, irrelevant.
By Lemma~\ref{lem-link-trace-local} we have $\mu^{\C}_i = \subdis_{\C}(a_1 a_2 \cdots a_i)$ for all~$i$.
So we have:
\begin{align*}
\Tr_{\D,\alpha^{\D}}(w)
& = \norm{\subdis_{\D,\alpha^{\D}}(w)} && \text{by~\eqref{eq-subdis=trace-probability}} \\
& = \norm{\mu^{\D}_n} && \mu^{\D}_n = \subdis_{\D,\alpha^{\D}}(w) \\
& = \norm{\mu^{\C}_n} && \mu^{\D}_n \sim \mu^{\C}_n \\
& = \norm{\subdis_{\C}(w)} && \mu^{\C}_n = \subdis_{\C}(w) \\
& = \Tr_{\C}(w) && \text{by~\eqref{eq-subdis=trace-probability}}
\end{align*}
Since $\alpha^{\D}$ and~$w$ were chosen arbitrarily, we conclude that $\D \sqsubseteq \C$.
\item[``$\Longleftarrow$'':]
Let $\D \sqsubseteq \C$.
We show $\mu^{\D}_{0} \sim \mu^{\C}_{0}$.
Define a relation $\R \subseteq \subdists(Q_\D) \times \subdists(Q_\C)$
such that $\mu^{\D} \mathrel{\R} \mu^{\C}$ if and only if there exist a strategy~$\alpha^{\D}$ for~$\D$ and a trace~$w$ with
$\mu^{\D} = \subdis_{\D,\alpha^{\D}}(w)$ and $\mu^{\C} = \subdis_{\C}(w)$.
We claim that $\R$ is a bisimulation.
To prove the claim, consider any $\mu^{\D}, \mu^{\C}$ with $\mu^{\D} \mathrel{\R} \mu^{\C}$.
Then there exist a strategy~$\alpha^{\D}$ for~$\D$ and a trace~$w$ with
$\mu^{\D} = \subdis_{\D,\alpha^{\D}}(w)$ and $\mu^\C = \subdis_{\C}(w)$.
Since $\D \sqsubseteq \C$, we have $\Tr_{\D,\alpha^{\D}}(w) = \Tr_{\C}(w)$.
So we have:
\begin{align*}
\norm{\mu^{\D}}
& = \norm{\subdis_{\D,\alpha^{\D}}(w)} && \mu^{\D} = \subdis_{\D,\alpha^{\D}}(w) \\
& = \Tr_{\D,\alpha^{\D}}(w) && \text{by~\eqref{eq-subdis=trace-probability}} \\
& = \Tr_{\C}(w)       && \text{as argued above} \\
& = \norm{\subdis_{\C}(w)} && \text{by~\eqref{eq-subdis=trace-probability}} \\
& = \norm{\mu^{\C}} && \mu^{\C} = \subdis_{\C}(w)
\end{align*}
This proves the first condition for $\R$ being a bisimulation.

For the rest of the proof assume $w = a_1 a_2 \cdots a_n$.
Write $\mu^{\D}_n = \mu^{\D}$ and $\mu^{\C}_n = \mu^{\C}$.
Let $\alpha^{\D}_n$ be a local strategy for~$\D$.
Let $a_{n+1} \in \Label$.
Define $\mu^{\D}_{n+1} = \Succ(\mu^{\D}_n, \alpha^{\D}_n, a_{n+1})$,
   and $\mu^{\C}_{n+1} = \Succ(\mu^{\C}_n, \alpha^{\C}_n, a_{n+1})$ for an arbitrary (and unimportant as $\C$ is an MC) local strategy $\alpha^{\C}_n$ for~$\C$.
For the second and the third condition of $\R$ being a bisimulation we need to prove
$\mu^{\D}_{n+1} \mathrel{\R} \mu^{\C}_{n+1}$.
Define $\mu^{\D}_1, \mu^{\D}_2, \ldots, \mu^{\D}_{n-1}$ such that
$\mu^{\D}_i = \subdis_{\D,\alpha^{\D}}(a_1 a_2 \cdots a_i)$ for all $i \in \{0, 1, \ldots, n\}$.
By Lemma~\ref{lem-link-trace-local} there are local strategies $\alpha^{\D}_0, \alpha^{\D}_1, \ldots, \alpha^{\D}_{n-1}$
such that $\mu^{\D}_{i+1} = \Succ(\mu^{\D}_i, \alpha^{\D}_i, a_{i+1})$ for all $i \in \{0, 1, \ldots, n-1\}$.
We also have $\mu^{\D}_{n+1} = \Succ(\mu^{\D}_n, \alpha^{\D}_n, a_{n+1})$,
so again by Lemma~\ref{lem-link-trace-local} there is a strategy~$\beta^{\D}$ with
$\mu^{\D}_i = \subdis_{\D,\beta^{\D}}(a_1 a_2 \cdots a_i)$ for all $i \in \{0, 1, \ldots, n+1\}$. In particular, $\mu^{\D}_{n+1} = \subdis_{\D,\beta^{\D}}(w a_{n+1})$.
Similarly, we have $\mu^{\C}_{n+1} = \subdis_{\C}(w a_{n+1})$.
Thus, $\mu^{\D}_{n+1} \mathrel{\R} \mu^{\C}_{n+1}$.
Hence we have proved that $\R$ is a bisimulation.

Considering the empty trace, we see that $\mu^{\D}_0 \mathrel{\R} \mu^{\C}_0$.
Since $\R \subseteq \mathord{\sim}$, we also have $\mu^{\D}_0 \sim \mu^{\C}_0$, as desired. \qedhere
\end{itemize}
\end{proof}


\subsection{A Necessary and Sufficient Condition for Bisimilarity}\label{subsec:condition_bisimilar}

In the following we consider MDPs $\D = \tuple{Q,\mu^{\D}_{0},\Label,\delta}$ and $\E = \tuple{Q,\mu^{\E}_{0},\Label,\delta}$ over the same state space.
This is without loss of generality, since we might take the disjoint union of the state spaces.
Since $\D$ and~$\E$ differ only in the initial distribution, we will focus on~$\D$.


Let $B \in \Reals^{Q \times k}$ with $k \ge 1$.
Assume the label set is $\Label = \{a_1, \ldots, a_{|\Label|}\}$.
For $\mu \in \subdists(Q)$ and a local strategy~$\alpha$ we define a point $p(\mu, \alpha) \in \Reals^{{|\Label|} \cdot k}$ such that
\[
p(\mu, \alpha) \ = \ \big( \mu \Delta_{\alpha}(a_1) B \quad
                 \mu \Delta_{\alpha}(a_2) B \quad
                 \cdots \quad
                 \mu \Delta_{\alpha}(a_{|\Label|}) B \big).
\]
For the reader's intuition, we remark that we will choose matrices $B \in \Reals^{Q \times k}$ so that if two subdistributions $\mu_\D, \mu_\E$ are bisimilar then $\mu_\D B = \mu_\E B$.
(In fact, one can compute~$B$ so that the converse holds as well, i.e., $\mu_\D \sim \mu_\E$ if and only if $\mu_\D B = \mu_\E B$.)
It follows that, for subdistributions $\mu_\D, \mu_\E$ and local strategies $\alpha_\D, \alpha_\E$,
if $\Succ(\mu_\D, \alpha_\D, a) \sim \Succ(\mu_\E, \alpha_\E, a)$ holds for all $a \in \Label$ then $p(\mu_\D,\alpha_\D) = p(\mu_\E,\alpha_\E)$.
Let us also remark that for fixed $\mu \in \subdists(Q)$, the set $P_\mu = \{p(\mu,\alpha) \mid \alpha \text{ is a local strategy}\} \subseteq \Reals^{{|\Label|} \cdot k}$ is a (bounded and convex) polytope.
As a consequence, if $\mu_\D \sim \mu_\E$ then the polytopes $P_{\mu_\D}$ and~$P_{\mu_\E}$ must be equal.
In the next paragraph we define \emph{``extremal''} strategies~$\halpha$, which intuitively are local strategies such that $p(\mu,\halpha)$ is a vertex of the polytope~$P_{\mu}$.

Let $\vec{v} \in \Reals^{{|\Label|} \cdot k}$ be a \emph{column} vector;
we denote column vectors in boldface.
We view~$\vec{v}$ as a ``direction''.
Recall that~$d_q$ is the Dirac distribution on the state~$q$.
A pure local strategy~$\halpha$ is \emph{extremal in direction~$\vec{v}$ with respect to~$B$} if
\begin{align}
 & p(d_q, \alpha) \vec{v} \le p(d_q, \halpha) \vec{v}  \label{eq-remove-v1} \\
 & p(d_q, \alpha) \vec{v}  =  p(d_q, \halpha) \vec{v} \text{\quad implies\quad}
   p(d_q, \alpha)          =  p(d_q, \halpha) \label{eq-remove-v2}
\end{align}
for all states~$q \in Q$ and all pure local strategies~$\alpha$.

By linearity, if~\eqref{eq-remove-v1} and~\eqref{eq-remove-v2} hold for all pure local strategies~$\alpha$ then~\eqref{eq-remove-v1} and~\eqref{eq-remove-v2} hold for all local strategies~$\alpha$.
We say a local strategy $\halpha$ is \emph{extremal with respect to~$B$} if there is a direction~$\vec{v}$ such that $\halpha$ is extremal in direction~$\vec{v}$ with respect to~$B$.

In the following we prove some facts about extremal local strategies that will be needed later.

\begin{lem}\label{lem-remove-v}
Let $\D = \tuple{Q,\mu_{0},\Label,\delta}$ be an MDP\@.
Let $B \in \Reals^{Q \times k}$ with $k \ge 1$.
Let $\mu \in \subdists(Q)$.
Let $\alpha, \halpha$ be local strategies.
Suppose $\vec{v} \in \Reals^{{|\Label|} \cdot k}$ is a direction in which $\halpha$ is extremal and
$p(\mu, \alpha) \vec{v} = p(\mu, \halpha) \vec{v}$.
Then $p(\mu, \alpha) = p(\mu, \halpha)$.
\end{lem}
\begin{proof}
We have:
\begin{align*}
\sum_{q \in Q} \mu(q) \cdot p(d_q, \alpha) \vec{v}
& = p(\mu, \alpha) \vec{v} && \text{definition of~$p$} \\
& = p(\mu, \halpha) \vec{v} && \text{assumption on~$\halpha$} \\
& = \sum_{q \in Q} \mu(q) \cdot p(d_q, \halpha) \vec{v} && \text{definition of~$p$}
\end{align*}
With~\eqref{eq-remove-v1} it follows that for all $q \in \Supp(\mu)$ we have $p(d_q, \alpha) \vec{v} = p(d_q, \halpha) \vec{v}$.
Hence by~\eqref{eq-remove-v2} we obtain $p(d_q, \alpha) = p(d_q, \halpha)$ for all $q \in \Supp(\mu)$.
Thus:
\[
p(\mu, \alpha)
= \sum_{q \in Q} \mu(q) \cdot p(d_q, \alpha)
= \sum_{q \in Q} \mu(q) \cdot p(d_q, \halpha)
= p(\mu, \halpha)
\qedhere
\]
\end{proof}

For a subdistribution~$\mu$ define the bounded, convex polytope $P_{\mu} \subseteq \Reals^{{|\Label|} \cdot k}$ with
\[
P_{\mu} = \{p(\mu, \alpha) \mid \alpha : Q \to \dists(\moves)\}.
\]
Comparing two polytopes $P_{\mu_\D}$ and~$P_{\mu_\E}$ for subdistributions $\mu_\D, \mu_\E$ will play a key role for deciding bisimulation.
First we prove the following lemma, which states that any vertex of the polytope~$P_\mu$ can be obtained by applying an extremal local strategy.
Although this is intuitive, the proof is not very easy.

\begin{lem}\label{lem-vertex-extremal}
Let $\D = \tuple{Q,\mu_{0},\Label,\delta}$ be an MDP\@.
Let $B \in \Reals^{Q \times k}$ with $k \ge 1$.
Let $\mu \in \subdists(Q)$.
If $x \in P_{\mu}$ is a vertex of~$P_{\mu}$ then there is an extremal local strategy~$\halpha$ with $x = p(\mu, \halpha)$.
\end{lem}
\begin{proof}
Let $x \in P_{\mu}$ be a vertex of~$P_{\mu}$.
Let $\alpha_1 : Q \to \dists(\moves)$ be a local strategy so that $x = p(\mu, \alpha_1)$.
Since $x$ is a vertex, we can assume that $\alpha_1$ is pure.
Since $x$ is a vertex of~$P_{\mu}$, there is a hyperplane $H \subseteq \Reals^{{|\Label|} \cdot k}$ such that $\{x\} = P_{\mu} \cap H$.
Let $\vec{v}_1 \in \Reals^{{|\Label|} \cdot k}$ be a normal vector of~$H$.
Since $\{x\} = P_{\mu} \cap H$, we have $x \vec{v}_1 = \max_{y \in P_{\mu}} y \vec{v}_1$ or $x \vec{v}_1 = \min_{y \in P_{\mu}} y \vec{v}_1$; without loss of generality, say $x \vec{v}_1 = \max_{y \in P_{\mu}} y \vec{v}_1$.
Since $\{x\} = P_{\mu} \cap H$, we have for all $q \in \Supp(\mu)$ and all $\alpha$:
\begin{equation} \label{eq-lem-vertex-extremal-suppD}
p(d_q, \alpha) \vec{v}_1 = p(d_q, \alpha_1) \vec{v}_1 \quad \text{implies} \quad p(d_q, \alpha) = p(d_q, \alpha_1).
\end{equation}

For all $q \in Q \setminus \Supp(\mu)$, redefine the pure local strategy~$\alpha_1(q)$ so that all $q \in Q$ and all local strategies~$\alpha$ satisfy $p(d_q, \alpha) \vec{v}_1 \le p(d_q, \alpha_1) \vec{v}_1$.
Since $Q$~and~$\moves$ are finite, there is $\varepsilon > 0$ such that all $q \in Q$ and all \emph{pure} local strategies~$\alpha$  either satisfy $p(d_q, \alpha) \vec{v}_1 = p(d_q, \alpha_1) \vec{v}_1$ or $p(d_q, \alpha) \vec{v}_1 \le p(d_q, \alpha_1) \vec{v}_1 - \varepsilon$.

Define
\[
 \Sigma = \left\{ \alpha : Q \to \dists(\moves) \mid \forall\, q \in Q: p(d_q, \alpha) \vec{v}_1 = p(d_q, \alpha_1) \vec{v}_1 \right\}.
\]
Consider the bounded, convex polytope $P_2 \subseteq \Reals^{{|\Label|} \cdot k}$ defined by
\[
P_2 = \left\{
\sum_{q \in Q} p(d_q, \alpha) \; \middle\vert \;
\alpha \in \Sigma\right\} .
\]
By an argument similar to the one above, there are a pure local strategy $\halpha \in \Sigma$, a vertex $x_2 = \sum_{q \in Q} p(d_q, \halpha)$ of~$P_2$, and a vector $\vec{v}_2 \in \Reals^{{|\Label|} \cdot k}$ such that for all $q \in Q$ and all $\alpha \in \Sigma$, we have $p(d_q, \alpha) \vec{v}_2 \le p(d_q, \halpha) \vec{v}_2$, and if $p(d_q, \alpha) \vec{v}_2 = p(d_q, \halpha) \vec{v}_2$ then $p(d_q, \alpha) = p(d_q, \halpha)$.
By scaling down~$\vec{v}_2$ by a small positive scalar, we can assume that all $q \in Q$ and all local strategies~$\alpha$ satisfy
\begin{equation} \label{eq-small-epsilon}
 \left\lvert p(d_q, \alpha) \vec{v}_2 \right\rvert \le \frac{\varepsilon}{3}.
\end{equation}

Since $\halpha \in \Sigma$, all $q \in Q$ satisfy $p(d_q, \halpha) \vec{v}_1 = p(d_q, \alpha_1) \vec{v}_1$.
By~\eqref{eq-lem-vertex-extremal-suppD} all $q \in \Supp(\mu)$ satisfy $p(d_q, \halpha) = p(d_q, \alpha_1)$.
Hence:
\[
p(\mu, \halpha)
= \sum_{q \in \Supp(\mu)} \mu(q) p(d_q, \halpha)
= \sum_{q \in \Supp(\mu)} \mu(q) p(d_q, \alpha_1)
= p(\mu, \alpha_1) = x
\]

It remains to show that there is a direction~$\vec{v}$ in which $\halpha$ is extremal.
Take $\vec{v} = \vec{v}_1 + \vec{v}_2$.
Let $q \in Q$ and let $\alpha$ be a pure local strategy.
We consider two cases:
\begin{itemize}
\item
Assume $p(d_q, \alpha) \vec{v}_1 = p(d_q, \alpha_1) \vec{v}_1$.
Then there is $\beta \in \Sigma$ with $\alpha(q) = \beta(q)$, hence $p(d_q, \alpha) = p(d_q, \beta)$.
We have:
\begin{align*}
p(d_q, \alpha) \vec{v}
& = p(d_q, \beta) \vec{v} && p(d_q, \alpha) = p(d_q, \beta) \\
& = p(d_q, \beta) \vec{v}_1 + p(d_q, \beta) \vec{v}_2 && \text{definition of~$\vec{v}$} \\
& = p(d_q, \alpha_1) \vec{v}_1 + p(d_q, \beta) \vec{v}_2 && \beta \in \Sigma \\
& = p(d_q, \halpha) \vec{v}_1 + p(d_q, \beta) \vec{v}_2 && \halpha \in \Sigma \\
& \le p(d_q, \halpha) \vec{v}_1 + p(d_q, \halpha) \vec{v}_2 && \text{definition of $\halpha$} \\
& = p(d_q, \halpha) \vec{v} && \text{definition of~$\vec{v}$}
\end{align*}
Hence~\eqref{eq-remove-v1} holds for~$\halpha$.
To show~\eqref{eq-remove-v2}, assume $p(d_q, \alpha) \vec{v} = p(d_q, \halpha) \vec{v}$.
Then all terms in the computation above are equal, and $p(d_q, \beta) \vec{v}_2 = p(d_q, \halpha) \vec{v}_2$.
By the definition of~$\halpha$, this implies $p(d_q, \beta) = p(d_q, \halpha)$.
Hence $p(d_q, \alpha) = p(d_q, \beta) = p(d_q, \halpha)$.
Hence~\eqref{eq-remove-v2} holds for~$\halpha$.
\item
Assume $p(d_q, \alpha) \vec{v}_1 \ne p(d_q, \alpha_1) \vec{v}_1$.
By the definition of~$\varepsilon$ it follows $p(d_q, \alpha) \vec{v}_1 \le p(d_q, \alpha_1) \vec{v}_1 - \varepsilon$.
We have:
\begin{align*}
p(d_q, \alpha) \vec{v}
& = p(d_q, \alpha) \vec{v}_1 + p(d_q, \alpha) \vec{v}_2 && \text{definition of~$\vec{v}$} \\
& \le p(d_q, \alpha_1) \vec{v}_1 - \varepsilon + p(d_q, \alpha) \vec{v}_2 && \text{as argued above} \\
&  =  p(d_q, \halpha) \vec{v}_1 - \varepsilon + p(d_q, \alpha) \vec{v}_2 && \halpha \in \Sigma \\
& \le p(d_q, \halpha) \vec{v}_1 - \varepsilon + \frac{\varepsilon}{3} && \text{by~\eqref{eq-small-epsilon}} \\
& \le p(d_q, \halpha) \vec{v}_1 + p(d_q, \halpha) \vec{v}_2 - \varepsilon + \frac{\varepsilon}{3} + \frac{\varepsilon}{3} && \text{by~\eqref{eq-small-epsilon}} \\
& < p(d_q, \halpha) \vec{v}_1 + p(d_q, \halpha) \vec{v}_2 && \varepsilon > 0 \\
& = p(d_q, \halpha) \vec{v} && \text{definition of~$\vec{v}$}
\end{align*}
This implies~\eqref{eq-remove-v1} and~\eqref{eq-remove-v2} for~$\halpha$.
\end{itemize}
Hence, $\halpha$ is extremal in direction~$\vec{v}$.
\end{proof}

The following lemma states the intuitive fact that in order to compare the polytopes $P_{\mu_\D}$ and~$P_{\mu_\E}$, it suffices to compare the vertices obtained by applying extremal local strategies:

\begin{lem}\label{lem-check-extremal-enough}
Let $\D = \tuple{Q,\mu_{0},\Label,\delta}$ be an MDP\@.
Let $B \in \Reals^{Q \times k}$ with $k \ge 1$.
Then for all $\mu_\D, \mu_\E \in \subdists(Q)$ we have $P_{\mu_\D} = P_{\mu_\E}$ if and only if for all extremal local strategies~$\halpha$ we have $p(\mu_\D, \halpha) = p(\mu_\E, \halpha)$.
\end{lem}
\begin{proof}
We prove the two implications from the lemma in turn.
\begin{itemize}[align=left]
\item[``$\Longrightarrow$'':]
Suppose $P_{\mu_\D} = P_{\mu_\E}$.
Let $\halpha$ be a local strategy that is extremal in direction~$\vec{v}$.
Since $P_{\mu_\D} = P_{\mu_\E}$, there are $\alpha_{\E}$ and~$\alpha_{\D}$ such that
$p(\mu_\D, \halpha) = p(\mu_\E, \alpha_\E)$ and
$p(\mu_\E, \halpha) = p(\mu_\D, \alpha_\D)$.
We have:
\begin{align*}
p(\mu_\D, \halpha) \vec{v}
& = p(\mu_\E, \alpha_\E) \vec{v}
 && p(\mu_\D, \halpha) = p(\mu_\E, \alpha_\E) \\
& \le p(\mu_\E, \halpha) \vec{v}
 && \text{$\halpha$ is extremal in direction~$\vec{v}$} \\
& = p(\mu_\D, \alpha_\D) \vec{v}
 && \text{$p(\mu_\E, \halpha) = p(\mu_\D, \alpha_\D)$} \\
& \le p(\mu_\D, \halpha) \vec{v}
 && \text{$\halpha$ is extremal in direction~$\vec{v}$}
\end{align*}
So all inequalities are in fact equalities.
In particular, we have $p(\mu_\D, \halpha) \vec{v} = p(\mu_\D, \alpha_\D) \vec{v}$.
It follows:
\begin{align*}
p(\mu_\D, \halpha)
& = p(\mu_\D, \alpha_\D) && \text{Lemma~\ref{lem-remove-v}} \\
& = p(\mu_\E, \halpha)  && \text{definition of~$\alpha_\D$}
\end{align*}
\item[``$\Longleftarrow$'':]
Let $x$ be a vertex of~$P_{\mu_\D}$.
By Lemma~\ref{lem-vertex-extremal} there exists an extremal local strategy~$\halpha$ with $x = p(\mu_\D, \halpha)$.
By the assumption we have $p(\mu_\D, \halpha) = p(\mu_\E, \halpha)$.
Hence $x = p(\mu_\D, \halpha) = p(\mu_\E, \halpha) \in P_{\mu_\E}$.
Since $x$ is an arbitrary vertex of~$P_{\mu_\D}$, and $P_{\mu_\D}, P_{\mu_\E}$ are bounded, convex polytopes, it follows $P_{\mu_\D} \subseteq P_{\mu_\E}$.
The reverse inclusion is shown similarly. \qedhere
\end{itemize}
\end{proof}

\noindent
The following lemma shows that the alternation of quantifiers over local strategies can be replaced by quantifying over extremal local strategies:

\begin{lem}\label{lem-sim-pivot}
Let $\D = \tuple{Q,\mu_{0},\Label,\delta}$ be an MDP\@.
Let $B \in \Reals^{Q \times k}$ with $k \ge 1$.
Let $\mu_\D, \mu_\E \in \subdists(Q)$.
In the following let $\alpha_\D, \alpha_\E$ range over local strategies, $\halpha$~over extremal local strategies, and $a$ over~$\Label$.
Then
\begin{equation}
\begin{aligned}
 &\forall \alpha_\D \exists \alpha_\E \forall a : \mu_\D \Delta_{\alpha_\D}(a) B = \mu_\E \Delta_{\alpha_\E}(a) B \\
 \quad \land \quad &
 \forall \alpha_\E \exists \alpha_\D \forall a : \mu_\D \Delta_{\alpha_\D}(a) B = \mu_\E \Delta_{\alpha_\E}(a) B
\end{aligned}
\label{eq-lem-sim-pivot}
\end{equation}
holds if and only the following holds:
\[
\forall \halpha \forall a: \mu_\D \Delta_{\halpha}(a) B = \mu_\E \Delta_{\halpha}(a) B
\]
\end{lem}

\begin{proof}
We have:
\begin{align*}
& \forall \alpha_\D \exists \alpha_\E \forall a : \mu_\D \Delta_{\alpha_\D}(a) B = \mu_\E \Delta_{\alpha_\E}(a) B \\
\quad\Longleftrightarrow\quad &
\forall \alpha_\D \exists \alpha_\E : p(\mu_\D,\alpha_\D) = p(\mu_\E,\alpha_\E) && \text{definition of~$p$} \\
\quad\Longleftrightarrow\quad &
P_{\mu_\D} \subseteq P_{\mu_\E}
\intertext{
It follows:
}
\eqref{eq-lem-sim-pivot} \quad\Longleftrightarrow\quad &
P_{\mu_\D} = P_{\mu_\E} \\
\quad\Longleftrightarrow\quad &
\forall \halpha \forall a: \mu_\D \Delta_{\halpha}(a) B = \mu_\E \Delta_{\halpha}(a) B
&& \text{Lemma~\ref{lem-check-extremal-enough}}
\qedhere
\end{align*}

\end{proof}

The following proposition provides necessary and sufficient conditions for bisimilarity, which---as we will see---can be effectively checked.

\begin{prop}\label{prop-coNP-vector-space}
Let $\D = \tuple{Q,\mu_{0},\Label,\delta}$ be an MDP\@.
Let $B \in \Reals^{Q \times k}$ with $k \ge 1$.
\begin{enumerate}
\item[(1)] Suppose that for all $\mu_\D, \mu_\E \in \subdists(Q)$ with $\mu_\D \sim \mu_\E$ we have $\mu_\D B = \mu_\E B$.
    Then for all $\mu_\D, \mu_\E \in \subdists(Q)$ with $\mu_\D \sim \mu_\E$ we have $\mu_\D \Delta_{\halpha}(a) B = \mu_\E \Delta_{\halpha}(a) B$ for all extremal local strategies~$\halpha$ and all $a \in \Label$.
\item[(2)] Suppose that $B$ includes the column vector $\vec{1} = {(1 \ 1 \cdots 1)}^T$ (where the superscript~$T$ denotes transpose) and that for all extremal local strategies~$\halpha$ and all $a \in \Label$ the columns of $\Delta_{\halpha}(a) B$ are in the linear span of the columns of~$B$.
    Then for all $\mu_\D, \mu_\E \in \subdists(Q)$ with $\mu_\D B = \mu_\E B$ we have $\mu_\D \sim \mu_\E$.
\end{enumerate}
\end{prop}

\begin{proof}\leavevmode
\begin{itemize}[align=left]
\item[(1)]
Let $\mu_\D, \mu_\E \in \subdists(Q)$ with $\mu_\D \sim \mu_\E$.
By the definition of bisimulation and using~\eqref{eq-Succ-as-prod}, we obtain:
\begin{itemize}
\item for all local strategies~$\alpha_\D$ there exists a local strategy~$\alpha_\E$ such that for all $a \in \Label$ we have $\mu_\D \Delta_{\alpha_\D}(a) \sim \mu_\E \Delta_{\alpha_\E}(a)$;
\item for all local strategies~$\alpha_\E$ there exists a local strategy~$\alpha_\D$ such that for all $a \in \Label$ we have $\mu_\D \Delta_{\alpha_\D}(a) \sim \mu_\E \Delta_{\alpha_\E}(a)$.
\end{itemize}
Using our assumption on~$B$, we see that~\eqref{eq-lem-sim-pivot} from Lemma~\ref{lem-sim-pivot} holds for $\mu_\D, \mu_\E$.
By Lemma~\ref{lem-sim-pivot} we have $\mu_\D \Delta_{\halpha}(a) B = \mu_\E \Delta_{\halpha}(a) B$ for all extremal local strategies~$\halpha$ and all $a \in \Label$.
\item[(2)]
It suffices to show that the relation $\mathord{\sim_B} \subseteq \subdists(Q) \times \subdists(Q)$ defined by
\[
 \mu_\D \sim_B \mu_\E \quad \Longleftrightarrow \quad \mu_\D B = \mu_\E B
\]
is a bisimulation.
Let $\mu_\D \sim_B \mu_\E$, i.e., $\mu_\D B = \mu_\E B$.
Since $B$ includes the column~$\vec{1}$, we have $\norm{\mu_\D} = \norm{\mu_\E}$.
Since for all extremal local strategies~$\halpha$ and all $a \in \Label$ the columns of $\Delta_{\halpha}(a) B$ are in the linear span of the columns of~$B$, we have $\vec{0}^T = (\mu_\D - \mu_\E) B = (\mu_\D - \mu_\E) \Delta_{\halpha}(a) B$ for all extremal local strategies~$\halpha$ and all $a \in \Label$.
Lemma~\ref{lem-sim-pivot} implies that~\eqref{eq-lem-sim-pivot} holds for $\mu_\D, \mu_\E$.
Using~\eqref{eq-Succ-as-prod} and the definition of~$\mathord{\sim_B}$, we obtain:
\begin{itemize}
\item for all local strategies~$\alpha_\D$ there exists a local strategy~$\alpha_\E$ such that for all $a \in \Label$ we have $\Succ(\mu_\D, \alpha_\D, a) \sim_B \Succ(\mu_\E, \alpha_\E, a)$;
\item for all local strategies~$\alpha_\E$ there exists a local strategy~$\alpha_\D$ such that for all $a \in \Label$ we have $\Succ(\mu_\D, \alpha_\D, a) \sim_B \Succ(\mu_\E, \alpha_\E, a)$.
\end{itemize}
Thus the relation~$\mathord{\sim_B}$ is a bisimulation. \qedhere
\end{itemize}
\end{proof}

\subsection{A coNP Algorithm for Checking Bisimilarity of two MDPs}\label{subsec:algo_conp_bisimilarity}

Proposition~\ref{prop-coNP-vector-space}
suggests an algorithm for determining bisimilarity in a given MDP $\D = \tuple{Q,\mu_{0},\Label,\delta}$.
More concretely, we can compute a matrix~$B$ such that for all subdistributions $\mu_\D, \mu_\E$ we have $\mu_\D \sim \mu_\E$ if and only if $\mu_\D B = \mu_\E B$.
The algorithm initializes~$B$ with the column vector~$\vec{1}$ and henceforth maintains the invariant that $\mu_\D \sim \mu_\E$ implies $\mu_\D B = \mu_\E B$.
\begin{itemize}
\item
If there exists an extremal local strategy~$\halpha$ and a label $a \in \Label$ such that a column of $\Delta_{\halpha}(a) B$ is linearly independent of the columns of~$B$, then add this column to~$B$.
This maintains the invariant by Proposition~\ref{prop-coNP-vector-space}~(1).
Repeat.
\item
Otherwise (i.e., such $\halpha$ and~$a$ do not exist) terminate.
Then by Proposition~\ref{prop-coNP-vector-space}~(2) we have $\mu_\D B = \mu_\E B \ \Longrightarrow \ \mu_\D \sim \mu_\E$.
Together with the invariant we get $\mu_\D \sim \mu_\E \ \Longleftrightarrow \ \mu_\D B = \mu_\E B$, as claimed.
\end{itemize}
This algorithm terminates because $B$ can have at most $|Q|$ linearly independent columns.

Along these lines we can also prove the following theorem.

\begin{thm}\label{thm-coNP-result}
The problem that, given two MDPs~$\D$ and $\E$, asks whether $\D \sim \E$ is in {\sf coNP}.
\end{thm}

\begin{proof}
Without loss of generality we assume
$\D = \tuple{Q,\mu_{\D},\Label,\delta}$ and
$\E = \tuple{Q,\mu_{\E},\Label,\delta}$.
Hence we wish to decide in {\sf NP} whether $\mu_{\D} \not\sim \mu_{\E}$.

We proceed along the lines of the algorithm above.
Specifically, it follows from the arguments there that $\mu_\D \not\sim \mu_\E$ holds if and only if the following condition~$\Cond$ holds:
\begin{quote}
There are $k \in \{1, 2, \ldots, |Q|\}$ and $\vec{b}_0 = \vec{1}$ and $\vec{b}_1, \ldots, \vec{b}_{k-1} \in \Reals^Q$
and $i_0, i_1, \ldots, i_{k-1} \in \{0, 1, \ldots, k-2\}$
and $a_1, a_2, \ldots, a_{k-1} \in \Label$ and pure local strategies $\halpha_1, \halpha_2, \ldots, \halpha_{k-1}$ such that for all $j \in \{1, 2, \ldots, k-1\}$
\begin{itemize}
\item $\halpha_j$ is extremal with respect to the matrix formed by the column vectors $\vec{b}_0, \vec{b}_1, \ldots, \vec{b}_{j-1}$ and
\item $i_j < j$ and
\item $\vec{b}_j = \Delta_{\halpha_j}(a_j) \vec{b}_{i_j}$,
\end{itemize}
and $\mu_{\D} \vec{b}_{k-1} \ne \mu_{\E} \vec{b}_{k-1}$.
\end{quote}
It remains to argue that $\Cond$ can be checked in~{\sf NP}.
We can nondeterministically guess $k \le |Q|$ and $i_0, i_1, \ldots, i_{k-1} \le k-2$ and $a_1, a_2, \ldots, a_{k-1} \in \Label$ and pure local strategies $\halpha_1, \halpha_2, \ldots, \halpha_{k-1}$.
This determines $\vec{b}_1, \ldots, \vec{b}_{k-1}$.
All conditions in~$\Cond$ are straightforward to check in polynomial time, except the condition that for all $j \in \{1, 2, \ldots, k-1\}$ we have that $\halpha_j$ is extremal with respect to $\vec{b}_0, \vec{b}_1, \ldots, \vec{b}_{j-1}$.
In the remainder of the proof, we argue that this can also be checked in polynomial time.

Let $j \in \{1, 2, \ldots, k-1\}$.
Let $B \in \Reals^{Q \times j}$ be the matrix with columns $\vec{b}_0, \vec{b}_1, \ldots, \vec{b}_{j-1}$.
We want to check that $\halpha_j$ is extremal with respect to~$B$.
For all $q \in Q$, compute in polynomial time the set $\eqmoves(q) \subseteq \moves(q)$ defined by
\[
\eqmoves(q) = \{ \m \in \moves(q) \mid p(d_q, \alpha_{q,\m}) = p(d_q, \halpha_j) \},
\]
where $\alpha_{q,\m}$ is a pure local strategy with $\alpha_{q,\m}(q)(\m) = 1$ (it does not matter how $\alpha_{q,\m}(q')$ is defined for $q' \ne q$).
We want to verify that~\eqref{eq-remove-v1}~and~\eqref{eq-remove-v2} holds for~$\halpha_j$.
Hence we need to find~$\vec{v} \in \Reals^{|\Label| \cdot j}$ so that for all $q \in Q$ and all $\m \in \moves(q) \setminus \eqmoves(q)$ we have $p(d_q, \alpha_{q,\m}) \vec{v} < p(d_q, \halpha_j) \vec{v}$.
If such a vector~$\vec{v}$ exists, it can be scaled up by a large positive scalar so that we have:
\begin{equation} \label{eq-v-lp}
 p(d_q, \alpha_{q,\m}) \vec{v} + 1 \le p(d_q, \halpha_j) \vec{v} \quad \forall\, q \in Q \quad \forall\, \m \in \moves(q) \setminus \eqmoves(q)
\end{equation}
Hence it suffices to check if there exists a vector~$\vec{v}$ that satisfies~\eqref{eq-v-lp}.
This amounts to a feasibility check of a linear program of polynomial size.
Such a check can be carried out in polynomial time~\cite{Khachiyan79}.
\end{proof}

\subsection{An NC Algorithm for Trace Refinement}\label{subsec:algo_nc_trace_refinement}

In the following we consider an MDP $\D = \tuple{Q,\mu^{\D}_{0},\Label,\delta}$ and an MC $\C = \tuple{Q_\C,\mu^{\C}_{0},\Label,\delta_\C}$.
Without loss of generality, we assume $Q_\C \subseteq Q$.
Similarly, we also assume that $\delta_\C$ is a restriction of~$\delta$, and hence we write $\C = \tuple{Q_\C,\mu^{\C}_{0},\Label,\delta}$.
We may view subdistributions $\mu_\C \in \subdists(Q_\C)$ as $\mu_\C \in \subdists(Q)$ in the natural way.
The following proposition is analogous to Proposition~\ref{prop-coNP-vector-space}.
The key difference is that the need for considering \emph{extremal} strategies has disappeared.
This is due to the fact that only one of the two models is nondeterministic.

\begin{prop}\label{prop-coNP-vector-space-MC}
Let $\D = \tuple{Q,\mu^{\D}_{0},\Label,\delta}$ be an MDP
and $\C = \tuple{Q_\C,\mu^{\C}_{0},\Label,\delta}$ be an MC with $Q_\C \subseteq Q$.
Let $B \in \Reals^{Q \times k}$ with $k \ge 1$.
In the following let $\mu_\D$ range over~$\subdists(Q)$ and $\mu_\C$ over~$\subdists(Q_\C)$.
\begin{enumerate}
\item[(1)] Suppose that for all $\mu_\D, \mu_\C$ with $\mu_\D \sim \mu_\C$ we have $\mu_\D B = \mu_\C B$.
    Then for all $\mu_\D, \mu_\C$ with $\mu_\D \sim \mu_\C$ we have $\mu_\D \Delta_{\alpha}(a) B = \mu_\C \Delta_{\alpha}(a) B$ for all local strategies~$\alpha$ and all $a \in \Label$.
\item[(2)] Suppose that $B$ includes the column vector $\vec{1} = {(1 \ 1 \cdots 1)}^T$ and that for all local strategies~$\alpha$ and all $a \in \Label$ the columns of $\Delta_{\alpha}(a) B$ are in the linear span of the columns of~$B$.
    Then for all $\mu_\D, \mu_\C$ with $\mu_\D B = \mu_\C B$ we have $\mu_\D \sim \mu_\C$.
\end{enumerate}
\end{prop}

\begin{proof}
The proof is similar to but simpler than the proof of Proposition~\ref{prop-coNP-vector-space}.
For completeness, we give it explicitly.
\begin{itemize}[align=left]
\item[(1)]
Let $\mu_\D \sim \mu_\C$.
By the definition of bisimulation and using~\eqref{eq-Succ-as-prod} we have $\mu_\D \Delta_{\alpha}(a) \sim \mu_\C \Delta_{\alpha}(a)$ for all local strategies~$\alpha$ and all $a \in \Label$.
By our assumption on~$B$, we have $\mu_\D \Delta_{\alpha}(a) B = \mu_\C \Delta_{\alpha}(a) B$ for all local strategies~$\alpha$ and all $a \in \Label$.
\item[(2)]
It suffices to show that the relation $\mathord{\sim_B} \subseteq \subdists(Q) \times \subdists(Q_\C)$ defined by
\[
 \mu_\D \sim_B \mu_\C \quad \Longleftrightarrow \quad \mu_\D B = \mu_\C B
\]
is a bisimulation.
Let $\mu_\D \sim_B \mu_\C$, i.e., $\mu_\D B = \mu_\C B$.
Since $B$ includes the column~$\vec{1}$, we have $\norm{\mu_\D} = \norm{\mu_\C}$.
Since for all local strategies~$\alpha$ and all $a \in \Label$ the columns of~$\Delta_{\alpha}(a) B$ are in the linear span of the columns of~$B$, we have $\vec{0}^T = (\mu_\D - \mu_\C) B = (\mu_\D - \mu_\C) \Delta_{\alpha}(a) B$ for all local strategies~$\alpha$ and all $a \in \Label$.
Using~\eqref{eq-Succ-as-prod} and the definition of~$\mathord{\sim_B}$, we see that
for all local strategies~$\alpha$ and all $a \in \Label$ we have $\Succ(\mu_\D, \alpha, a) \sim_B \Succ(\mu_\C, \alpha, a)$.
Thus the relation~$\mathord{\sim_B}$ is a bisimulation. \qedhere
\end{itemize}
\end{proof}

\begin{cor}\label{cor-coNP-vector-space-MC}
Let $\mathcal{V} \subseteq \Reals^Q$ be the smallest column-vector space with $\vec{1} \in \mathcal{V}$ and $\Delta_\alpha(a) \vec{u} \in \mathcal{V}$ for all $\vec{u} \in \mathcal{V}$, all labels $a \in \Label$, and all local strategies~$\alpha$.
Then for all $\mu_\D \in \subdists(Q)$ and all $\mu_\C \in \subdists(Q_\C)$ we have:
\[
\mu_\D \sim \mu_\C \quad \Longleftrightarrow\quad \mu_\D \vec{u} = \mu_\C \vec{u}\ \text{ for all } \vec{u} \in \mathcal{V}
\]
\end{cor}

Notice the differences to Proposition~\ref{prop-coNP-vector-space}: there we considered all extremal local strategies (potentially exponentially many), here we consider all local strategies (in general infinitely many).
However, we show that one can efficiently find few local strategies that span all local strategies.
This allows us to reduce (in logarithmic space) the bisimulation problem between an MDP and an MC to the bisimulation problem between two MCs,
which is equivalent to the trace-equivalence problem in MCs (by Proposition~\ref{prop-link-trace-refinement-bisim}).
The latter problem is known to be in~{\sf NC}~\cite{Tzeng96}.
Theorem~\ref{thm-MDP-MC} then follows with Proposition~\ref{prop-link-trace-refinement-bisim}.

\begin{thm}\label{thm-MDP-MC}
The problem~${\sf MDP\sqsubseteq MC}$ is in~{\sf NC}, hence in~{\sf P}.
\end{thm}

\begin{proof}
Let $\D = \tuple{Q,\mu^{\D}_{0},\Label,\delta}$ be an MDP
and $\C = \tuple{Q_\C,\mu^{\C}_{0},\Label,\delta}$ be an MC with $Q_\C \subseteq Q$.

Let $\alpha_0$ denote an arbitrary pure local strategy.
For each $q \in Q$ and each $\m \in \moves(q)$ denote by~$\alpha_{q,\m}$ the pure local strategy such that $\alpha_{q,\m}(q)(\m) = 1$ and $\alpha_{q,\m}(q') = \alpha_0(q')$ for all $q' \in Q \setminus \{q\}$.
Define
\begin{align*}
\Sigma &= \{ \alpha_0 \} \cup \{ \alpha_{q,\m} \mid q \in Q, \ \m \in \moves(q) \} && \text{and} \\
 \M &= \{ \Delta_{\alpha}(a) \in \Reals^{Q \times Q} \mid \alpha \in \Sigma,\ a \in \Label\}
 && \text{and} \\
 \M_\infty &= \left\{ \Delta_\alpha(a) \in \Reals^{Q \times Q} \;\middle\vert\; \alpha \text{ is a local strategy, $a \in \Label$}\right\}.
\end{align*}
The vector space $\V \subseteq \Reals^Q$ from Corollary~\ref{cor-coNP-vector-space-MC} is the smallest vector space with
\begin{itemize}
\item
$\vec{1} = {(1 \ 1 \cdots 1)}^T \in \V$ and
\item $M \vec{u} \in \V$, for all $\vec{u} \in \V$ and all $M \in \M_\infty$.
\end{itemize}

\noindent
We have $\M \subseteq \M_\infty$, where $|\M|$ is finite and $|\M_\infty|$ is infinite.
Every matrix in~$\M_\infty$ can be expressed as a linear combination of matrices from~$\M$:
Indeed, let $\alpha$ be a local strategy.
Then for all $a \in \Label$ we have:
\[
 \Delta_\alpha(a) = \Delta_{\alpha_0}(a)
  + \sum_{q \in Q}
  \left( - \Delta_{\alpha_0}(a) +
   \sum_{\m \in \moves(q)}
      \alpha(q)(\m) \cdot \Delta_{\alpha_{q,\m(q)}}(a)
  \right)
\]
So by linearity, the vector space~$\V$ is the smallest column-vector space such that
\begin{itemize}
\item
$\vec{1} = {(1 \ 1 \cdots 1)}^T \in \V$ and
\item $M \vec{u} \in \V$, for all $\vec{u} \in \V$ and all $M \in \M$.
\end{itemize}

\noindent
Define a finite set of labels $\Label' = \{b_{\alpha, a} \mid \alpha \in \Sigma, \ a \in \Label \}$, and for each $\alpha \in \Sigma$ and each $a \in \Label$ a matrix
\[
 \Delta'(b_{\alpha, a}) = \frac{1}{|\Sigma|} \Delta_\alpha(a).
\]
The matrix $\sum_{b \in \Label'} \Delta'(b)$ is stochastic.
Define the MCs $\D' = \tuple{Q,\mu^{\D}_{0},\Label',\delta'}$
and $\C' = \tuple{Q,\mu^{\C}_{0},\Label',\delta'}$
such that $\delta'$ induces the transition matrices $\Delta'(b)$ for all $b \in \Label'$.
The MCs $\D'$ and~$\C'$ are computable in logarithmic space.
Let $\V' \subseteq \Reals^{Q}$ be the smallest column-vector space such that
\begin{itemize}
\item
$\vec{1} = {(1 \ 1 \cdots 1)}^T \in \V$ and
\item $\Delta'(b) \vec{u} \in \V$, for all $\vec{u} \in \V$ and all $b \in \Label'$.
\end{itemize}
Since the matrices in~$\M$ and the matrices~$\Delta'(b)$ are scalar multiples of each other, we have $\V = \V'$.
It holds:
\begin{align*}
\D \sqsubseteq \C \quad
& \Longleftrightarrow\quad \D \sim \C \text{ in~$\D$}
 && \text{Proposition~\ref{prop-link-trace-refinement-bisim}} \\
& \Longleftrightarrow\quad \mu^\D_{0} \sim \mu^{\C}_{0} \text{ in~$\D$}
 && \text{definition} \\
& \Longleftrightarrow\quad \forall\, \vec{u} \in \V : \mu^{\D}_{0} \vec{u} = \mu^{\C}_{0} \vec{u}
 && \text{Corollary~\ref{cor-coNP-vector-space-MC}} \\
& \Longleftrightarrow\quad \forall\, \vec{u} \in \V' : \mu^{\D}_{0} \vec{u} = \mu^{\C}_{0} \vec{u}
 && \V = \V' \\
& \Longleftrightarrow\quad \mu^\D_{0} \sim \mu^{\C}_{0} \text{ in~$\D'$}
 && \text{Corollary~\ref{cor-coNP-vector-space-MC}} \\
& \Longleftrightarrow\quad \D' \sim \C' \text{ in~$\D'$}
 && \text{definition} \\
& \Longleftrightarrow\quad \D' \sqsubseteq \C'
 && \text{Proposition~\ref{prop-link-trace-refinement-bisim}}
\end{align*}
As mentioned in Section~\ref{sub-prel-trace-refine},
deciding whether $\D' \sqsubseteq \C'$ holds amounts to the trace-equivalence problem for MCs.
It follows from Tzeng~\cite{Tzeng96} that the latter is decidable in~{\sf NC}, hence in~{\sf P}.
\end{proof}

