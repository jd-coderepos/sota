\documentclass{llncs}



\usepackage{xspace}
\usepackage{color}
\usepackage{epsfig}
\usepackage{times}
\usepackage{gensymb}
\usepackage[english]{babel}
\usepackage{graphicx}
\usepackage{amssymb}
\usepackage{multirow}
\usepackage{amsmath}


\title{Increasing-Chord Graphs On Point Sets\thanks{Work partially supported by the Australian Research Council (grant DE140100708).}}

\date{}

\author{Hooman Reisi Dehkordi, Fabrizio Frati, Joachim Gudmundsson
\institute{
 School of Information Technologies -- Monash University\\
\email{hooman.dehkordi@monash.edu}\\
 School of Information Technologies -- The University of Sydney\\
\email{\{fabrizio.frati,joachim.gudmundsson\}@sydney.edu.au}}}

\newcommand{\image}[1]{\includegraphics[scale=0.3]{#1}}
\newcommand{\plot}[1]{\includegraphics[scale=0.55]{#1}}
\newcommand{\remove}[1]{}
\newcommand{\rephrase}[3]{\noindent\textbf{#1 #2}.~\emph{#3}}
\newcommand{\red}[1]{\color{red} #1 \color{black}\xspace}

\newtheorem{open}{Problem}

\renewenvironment{proof}
{{\bf Proof:}}{\hspace*{\fill}\par\vspace{2mm}}

\begin{document}

\maketitle

\begin{abstract}
We tackle the problem of constructing increasing-chord graphs spanning point sets. We prove that, for every point set  with  points, there exists an increasing-chord planar graph with  Steiner points spanning . Further, we prove that, for every convex point set  with  points, there exists an increasing-chord graph with  edges (and with no Steiner points) spanning .
\end{abstract}

\section{Introduction} \label{se:introduction}

A {\em proximity graph} is a geometric graph that can be constructed from a point set by connecting points that are ``close'', for some local or global definition of proximity. Proximity graphs constitute a topic of research in which the areas of graph drawing and computational geometry nicely intersect. A typical graph drawing question in this topic asks to characterize the graphs that can be represented as a certain type of proximity graphs. A typical computational geometry question asks to design an algorithm to construct a proximity graph spanning a given point set.

Euclidean minimum spanning trees and Delaunay triangulations are famous examples of proximity graphs. Given a point set , a {\em Euclidean minimum spanning tree} (MST) of  is a geometric tree with  as vertex set and with minimum total edge length; the {\em Delaunay triangulation} of  is a triangulation  such that no point in  lies inside the circumcircle of any triangle of . From a computational geometry perspective, given a point set  with  points, an MST of  with maximum degree five exists~\cite{MonmaS92} and can be constructed in  time~\cite{bcko-cgta-08}; also, the Delaunay triangulation of  exists and can be constructed in  time~\cite{bcko-cgta-08}. From a graph drawing perspective, every tree with maximum degree five admits a representation as an MST~\cite{MonmaS92} and it is NP-hard to decide whether a tree with maximum degree six admits such a representation~\cite{EadesW96}; also, characterizing the class of graphs that can be represented as Delaunay triangulations is a deeply studied question, which still eludes a clear answer; see, e.g.,~\cite{dv-aptg-96,ds-gtcidr-96}. Refer to the excellent survey by Liotta~\cite{gd-handbook} for more on proximity graphs.

While proximity graphs have constituted a frequent topic of research in graph drawing and computational geometry, they gained a sudden peak in popularity even outside these communities in 2004, when Papadimitriou {\em et al.}~\cite{conf/mobicom/RaoPSS03} devised an elegant routing protocol that works effectively in all the networks that can be represented as a certain type of proximity graphs, called {\em greedy graphs}. For two points  and  in the plane, denote by  the straight-line segment having  and  as end-points, and by  the length of . A geometric path  is {\em greedy} if , for every . A geometric graph  is {\em greedy} if, for every ordered pair of vertices  and , there exists a greedy path from  to  in . A result related to our paper is that, for every point set , the Delaunay triangulation of  is a greedy graph~\cite{PapadimitriouR05}.

In this paper we study {\em self-approaching} and {\em increasing-chord graphs}, that are types of proximity graphs defined by Alamdari {\em et al.}~\cite{acglp-sag-12}. A geometric path  is {\em self-approaching} if, for every three points , , and  in this order on  from  to  (possibly , , and  are internal to segments of ), it holds that . A geometric graph  is {\em self-approaching} if, for every ordered pair of vertices  and ,  contains a self-approaching path from  to ; also,  is {\em increasing-chord} if, for every pair of vertices  and ,  contains a path between  and  that is self-approaching both from  to  and from  to ; thus, an increasing-chord graph is also self-approaching. The study of self-approaching and increasing-chord graphs is motivated by their relationship with greedy graphs (a self-approaching graph is also greedy), and by the fact that such graphs have a small geometric dilation, namely at most 5.3332~\cite{ikl-sac-99} (self-approaching graphs) and at most 2.094~\cite{r-cic-94} (increasing-chord graphs).

Alamdari {\em et al.} showed: (i) how to test in linear time whether a path in  is self-approaching; (ii) a characterization of the class of self-approaching trees; and (iii) how to construct, for every point set  with  points in , an increasing-chord graph that spans  and uses  Steiner points.

In this paper we focus our attention on the problem of constructing increasing-chord graphs spanning given point sets in . We prove two main results.

\begin{itemize}
\item We show that, for every point set  with  points, there exists an increasing-chord planar graph with  Steiner points spanning . This answers a question of Alamdari {\em et al.}~\cite{acglp-sag-12} and improves upon their result (iii) above, since our increasing-chord graphs are planar and contain increasing-chord paths between every pair of points, including the Steiner points (which is not the case for the graphs in~\cite{acglp-sag-12}). It is interesting that our result is achieved by studying Gabriel triangulations, which are proximity graphs strongly related to Delaunay triangulations (a Gabriel triangulation of a point set  is a subgraph of the Delaunay triangulation of ). It has been proved in~\cite{acglp-sag-12} that Delaunay triangulations are not, in general, self-approaching.
\item We show that, for every convex point set  with  points, there exists an increasing-chord graph that spans  and that has  edges (and no Steiner points).
\end{itemize}

\section{Definitions and Preliminaries} \label{se:preliminaries}

A {\em geometric graph}  consists of a point set  in the plane and of a set  of straight-line segments (called {\em edges}) between points in . A geometric graph is {\em planar} if no two of its edges cross. A planar geometric graph partitions the plane into connected regions called {\em faces}. The bounded faces are {\em internal} and the unbounded face is the {\em outer face}. A geometric planar graph is a {\em triangulation} if every internal face is delimited by a triangle and the outer face is delimited by a convex polygon.

Let , , and  be points in the plane. We denote by  the angle defined by a clockwise rotation around  bringing  to coincide with .

A {\em convex combination} of a set of points  is a point  where  and  for each . The {\em convex hull}  of  is the set of points that can be expressed as a convex combination of the points in . A {\em convex point set}  is such that no point is a convex combination of the others. Let  be a convex point set and  be a directed straight line not orthogonal to any line through two points of . Order the points in  as their projections appear on ; then, the {\em minimum point} and the {\em maximum point} of  with respect to  are the first and the last point in such an ordering. We say that  is  {\em one-sided with respect to } if the minimum and the maximum point of  with respect to  are consecutive along the border of . See Fig.~\ref{fig:one-sided-x}. A {\em one-sided convex point set} is a convex point set that is one-sided with respect to some directed straight line . The proof of our first lemma shows an algorithm to construct an increasing-chord planar graph spanning a one-sided convex point set.


\begin{figure}[tb]
\begin{center}
\mbox{\includegraphics[scale=0.32]{One-Sided.eps}}
\caption{A convex point set that is one-sided with respect to a directed straight line .}
\label{fig:one-sided-x}
\end{center}
\end{figure}


\begin{lemma} \label{lemma:half_convex_planar_graph}
Let  be any one-sided convex point set with  points. There exists an increasing-chord planar graph spanning  with  edges.
\end{lemma}

\begin{proof}
Assume that  is one-sided with respect to the positive -axis . Such a condition can be met after a suitable rotation of the Cartesian axes. Let  be the points in , ordered as their projections appear on .

We show by induction on  that an increasing-chord planar graph  spanning  exists, in which all the edges on the border of  are in . If  then the graph with a single edge  is an increasing-chord planar graph spanning . Next, assume that  and let  be a point with largest -coordinate in  (possibly  or ). Point set  is convex, one-sided with respect to , and has  points. By induction, there exists an increasing-chord planar graph  spanning  in which all the edges on the border of  are in . Let  be the graph obtained by adding vertex  and edges  and  to . We have that  is planar, given that  is planar and that edges  and  are on the border of . Further, all the edges on the border of  are in . Moreover,  contains an increasing-chord path between every pair of points in , by induction; also,  contains an increasing-chord path between  and every point  in , as one of the two paths on the border of  connecting  and  is both - and -monotone, and hence increasing-chord by the results in~\cite{acglp-sag-12}. Finally,  is a maximal outerplanar graph, hence it has  edges.
\end{proof}

The {\em Gabriel graph} of a point set  is the geometric graph that has an edge  between two points  and  if and only if the closed disk whose diameter is  contains no point of  in its interior or on its boundary. A {\em Gabriel triangulation} is a triangulation that is the Gabriel graph of its point set . We say that a point set  {\em admits} a Gabriel triangulation if the Gabriel graph of  is a triangulation. A triangulation is a Gabriel triangulation if and only if every angle of a triangle delimiting an internal face is acute~\cite{gs-nsagva-69}. See~\cite{gs-nsagva-69,gd-handbook,ms-pgg-80} for more properties about Gabriel graphs.

In Section~\ref{se:steiner} we will prove that every Gabriel triangulation is increasing-chord. A weaker version of the converse is also true, as proved in the following.

\begin{lemma} \label{lemma:gabriel_edge_increasing_chord}
Let  be a set of points and let  be an increasing-chord graph spanning . Then all the edges of the Gabriel graph of  are in .
\end{lemma}

\begin{proof}
Suppose, for a contradiction, that there exists an increasing-chord graph  and an edge  of the Gabriel graph of  such that . Then, consider any increasing-chord path  in . Since , it follows that . Assume w.l.o.g. that , , and  appear in this clockwise order on the boundary of triangle . Since the closed disk with diameter  does not contain any point in its interior or on its boundary, it follows that . If , then , a contradiction to the assumption that  is increasing-chord. If , then the altitude of triangle  incident to  hits  in a point . Hence, , a contradiction to the assumption that  is increasing-chord which proves the lemma.
\end{proof}

\section{Planar Increasing-Chord Graphs with Few Steiner Points} \label{se:steiner}

We show that, for any point set , one can construct an increasing-chord planar graph  such that  and . Our result has two ingredients. The first one is that Gabriel triangulations are increasing-chord graphs. The second one is a result of Bern {\em et al.}~\cite{beg-pgmg-94} stating that, for any point set , there exists a point set  such that , , and  admits a Gabriel triangulation. Combining these two facts proves our main result. The proof that Gabriel triangulations are increasing-chord graphs consists of two parts. In the first one, we prove that geometric graphs having a {\em -path} between every pair of points are increasing-chord. In the second one, we prove that in every Gabriel triangulation there exists a -path between every pair of points.


We introduce some definitions. The {\em slope} of a straight-line segment  is the angle spanned by a clockwise rotation around  that brings  to coincide with the positive -axis. Thus, if  is the slope of , then  is also the slope of , . A straight-line segment  is a {\em -edge} if its slope is in the interval . Also, a geometric path  is a {\em -path} from  to  if  is a -edge, for every . Consider a point  on a -path  from  to . Then, the subpath  of  from  to  is also a -path. Moreover, denote by  the closed wedge with an angle of  incident to  and whose delimiting lines have slope  and ; then  is contained in  (see Fig.~\ref{fig:wedges}). We have the following:

\begin{figure}[tb]
\begin{center}
\mbox{\includegraphics[scale=0.35]{Wedge-M.eps}}
\caption{Wedge  contains path .}
\label{fig:wedges}
\end{center}
\end{figure}

\begin{lemma} \label{le:theta-path-approaching}
Let  be a -path from  to . Then,  is increasing-chord.
\end{lemma}

\begin{proof}
Lemma 3 in~\cite{ikl-sac-99} states the following (see also~\cite{aaiklr-gsac-01}): A curve  with end-points  and  is self-approaching from  to  if and only if, for every point  on , there exists a closed wedge with an angle of  incident to  and containing the part of  between  and .
By definition of -path, for every point  on , the closed wedge  with an angle of  incident to  and whose delimiting lines have slope  and  contains the subpath  of  from  to . Hence, by Lemma 3 in~\cite{ikl-sac-99},  is self-approaching from  to .
An analogous proof shows that  is self-approaching from  to , given that  is a -path from  to .
\remove{We prove that  is self-approaching from  to . The proof that  is self-approaching from  to  is analogous, as  is a -path from  to .

Consider any three points , , and  on  in this order from  to . Assume that , , and  appear in this counter-clockwise order along triangle , the case in which they appear in clockwise order , , and  is analogous. In order to prove that  it suffices to prove that . Since  is in  and since  is in , we have that , where  is the intersection point of  with the line delimiting  with slope . Moreover, , given that  is the supplementary  of the angle obtained by counter-clockwise rotating  until it coincides with the line delimiting  with slope . The latter angle is smaller than or equal to , given that it is not larger than the angle of wedge . This proves that  is self-approaching from  to , and hence it proves the lemma.
}
\end{proof}

\remove{
}
We now prove that Gabriel triangulations contain -paths.

\begin{lemma} \label{le:gabriel-is-theta}
Let  be a Gabriel triangulation on a point set . For every two points , there exists an angle  such that  contains a -path from  to .
\end{lemma}

\begin{proof}
Consider any two points . Clockwise rotate  of an angle  so that  and . Observe that, if there exists a -path from  to  after the rotation, then there exists a -path from  to  before the rotation.

A -path  in  is {\em maximal} if there is no  such that  is a -edge. For every maximal -path  in ,  lies on the border of . Namely, assume the converse, for a contradiction. Since  is a Gabriel triangulation, the angle between any two consecutive edges incident to an internal vertex of  is smaller than , thus there is a -edge incident to .  This contradicts the maximality of .  A maximal -path  is {\em high} if either (a)  and , or (b)  intersects the vertical line through  at a point above , for some . Symmetrically, a maximal -path  is {\em low} if either (a)  and , or (b)  intersects the vertical line through  at a point below , for some . High and low -paths starting at  can be defined analogously. The proof of the lemma consists of two main claims.

{\bf Claim 1.} If a maximal -path  starting at  and a maximal -path  starting at  exist such that  and  are both high or both low, for some , then there exists a -path in  from  to .

{\bf Claim 2.} For some , there exist a maximal -path  starting at  and a maximal -path  starting at  that are both high or both low.

Observe that Claims 1 and 2 imply the lemma.

We now prove Claim~1. Suppose that  contains a maximal high -path  starting at  and a maximal high -path  starting at , for some . If  and  share a vertex , then the subpath of  from  to  and the subpath of  from  to  form a -path in  from  to . Thus, it suffices to show that  and  share a vertex. For a contradiction assume the converse. Let  and  be the end-vertices of  and  different from  and , respectively. Recall that  and  lie on the border of . Denote by  and  the vertical half-lines starting at  and , respectively, and directed towards increasing -coordinates; also, denote by  and  the intersection points of  and  with the border of , respectively. Finally, denote by  the curve obtained by clockwise following the border of  from  to .

Assume that , as in Fig.~\ref{fig:gabriel}(a). Path  starts at  and passes through a point  on  (possibly ), given that  . Path  starts at  and either passes through a point  on , or ends at a point  on , depending on whether  or , respectively. Since  is -monotone and lies in , it follows that  and  are above or on ; also,  is below  given that  is a high path. It follows  and  intersect, hence they share a vertex given that  is planar.

\begin{figure}[tb]
\begin{center}
\begin{tabular}{c c c}
\mbox{\includegraphics[scale=0.45]{Gabriel-1.eps}} & \hspace{5mm}
\mbox{\includegraphics[scale=0.45]{Gabriel-3.eps}} & \hspace{5mm}
\mbox{\includegraphics[scale=0.45]{Gabriel-4.eps}}\\
(a) \hspace{5mm} & \hspace{5mm} (b) & \hspace{5mm} (c)
\end{tabular}
\caption{Paths  and  intersect if: (a) , (b) , and (c) .}
\label{fig:gabriel}
\end{center}
\end{figure}


Analogously, if , then  and  share a vertex.

If , then  is a -path from  to .

Next, if , as in Fig.~\ref{fig:gabriel}(b), then the end-points of  and  alternate along the boundary of the region  that is the intersection of , of the half-plane to the right of , and of the half-plane to the left of . Since  and  are -monotone, they lie in , thus they intersect, and hence they share a vertex.

Finally, assume that , as in Fig.~\ref{fig:gabriel}(c). Let  be the clockwise order of the points along , starting at  and ending at . By the assumption  we have . We prove that  is a -edge. Suppose, for a contradiction, that  is not a -edge. Since the slope of  is larger than  and smaller than , it is either larger than  and smaller than , or it is larger than  and smaller than . First, assume that the slope of  is larger than  and smaller than , as in Fig.~\ref{fig:gabriel-2}(a). Since the slope of  is between  and , it follows that  is below the line composed of  and , which contradicts the assumption that  is on . Second, if the slope of  is larger than  and smaller than , then we distinguish two further cases. In the first case, represented in Fig.~\ref{fig:gabriel-2}(b), the slope of  is larger than , hence  is below the line composed of  and , which contradicts the assumption that  is on . In the second case, represented in Fig.~\ref{fig:gabriel-2}(c), the slope of  is in the interval . It follows that the slope of  is in the interval ; since the slope of  is smaller than the one of , we have that  is not a -path. This contradiction proves that  is a -edge. However, this contradicts the assumption that  is a maximal -path, and hence concludes the proof of Claim~1.

\begin{figure}[tb]
\begin{center}
\begin{tabular}{c c c}
\mbox{\includegraphics[scale=0.45]{Gabriel-5.eps}} & \hspace{5mm}
\mbox{\includegraphics[scale=0.45]{Gabriel-6.eps}} & \hspace{5mm}
\mbox{\includegraphics[scale=0.45]{Gabriel-7.eps}}\\
(a) \hspace{5mm} & \hspace{5mm} (b) & \hspace{5mm} (c)
\end{tabular}
\caption{Illustration for the proof that  is a -edge.}
\label{fig:gabriel-2}
\end{center}
\end{figure}



We now prove Claim~2.  First, we prove that, for {\em every}  in the interval , there exists a maximal -path starting at  that is low or high. Indeed, it suffices to prove that there exists a -edge incident to , as such an edge is also a -path starting at , and the existence of a -path starting at  implies the existence of a maximal -path starting at . Consider a straight-line segment  that is the intersection of a directed half-line incident to  with slope  and of a disk of arbitrarily small radius centered at . If  is internal to , then consider the two edges  and  of  that are encountered when counter-clockwise and clockwise rotating  around , respectively. Then,  or  is a -edge, as the angle spanned by a clockwise rotation bringing  to coincide with  is smaller than , given that  is a Gabriel triangulation, and  is encountered during such a rotation. If  is outside , which might happen if  on the boundary of , then assume that the slope of  is in the interval  (the case in which the slope of  is in the interval  is analogous). Then, the angle spanned by a clockwise rotation bringing  to coincide with  is  at most . Since  is in interior or on the boundary of , an edge  of  is encountered during such a rotation, hence  is a -edge. An analogous proof shows that, for {\em every}  in the interval , there exists a maximal -path starting at  that is low or high.

Second, we prove that, for {\em some} , there exist a maximal low -path {\em and} a maximal high -path both starting at . All the maximal -paths (all the maximal -paths) starting at  are low (resp. high), given that every edge on these paths has slope in the interval  (resp. ). Thus, let  be the smallest constant in the interval  such that a maximal high -path exists. We prove that there also exists a maximal low -path starting at . Consider an arbitrarily small . By assumption, there exists no high -path. Hence, from the previous argument there exists a low -path . If  is sufficiently small, then no edge of  has slope in the interval . Thus every edge of  has slope in the interval , hence  is a maximal low -path starting at .

Since there exist a maximal high -path starting at , a maximal low -path starting at , and a maximal -path starting at  that is low or high, it follows that there exist a maximal -path  starting at  and a maximal -path  starting at  that are both high or both low. This proves Claim~2 and hence the lemma.
\end{proof}



Lemma~\ref{le:theta-path-approaching} and Lemma~\ref{le:gabriel-is-theta} immediately imply the following.

\begin{corollary}\label{cor:gabriel-is-increasing}
Any Gabriel triangulation is increasing-chord.
\end{corollary}

We are now ready to state the main result of this section.

\begin{theorem} \label{th:steiner}
Let  be a point set with  points.
One can construct in  time an increasing-chord planar graph  such that  and .
\end{theorem}
\begin{proof}
Bern, Eppstein, and Gilbert~\cite{beg-pgmg-94} proved that, for any point set , there exists a point set  with  and  such that  admits a Gabriel triangulation . Both  and  can be computed in  time~\cite{beg-pgmg-94}. By Corollary~\ref{cor:gabriel-is-increasing},  is increasing-chord, which concludes the proof.
\end{proof}

We remark that  Steiner points are not always enough to augment a point set  to a point set that admits a Gabriel triangulation. Namely, consider any point set  with  points that admits no Gabriel triangulation. Construct a point set  out of  copies of  placed ``far apart'' from each other, so that any triangle with two points in different copies of  is obtuse. Then, a Steiner point has to be added inside the convex hull of each copy of  to obtain a point set that admits a Gabriel triangulation.


\section{Increasing-Chord Convex Graphs with Few Edges}

In this section we prove the following theorem;

\begin{theorem} \label{th:convex-non-planar}
For every convex point set  with  points, there exists an increasing-chord geometric graph  such that .
\end{theorem}

The main idea behind the proof of Theorem~\ref{th:convex-non-planar} is that any convex point set  can be decomposed into some one-sided convex point sets  (which by Lemma~\ref{lemma:half_convex_planar_graph} admit increasing-chord spanning graphs with linearly many edges) in such a way that every two points of  are part of some  and that  is small. In order to perform such a decomposition, we introduce the concept of {\em balanced -partition}.

Let  be a convex point set and let  be a directed straight line not orthogonal to any line through two points of . See Fig.~\ref{fig:sets}. Let  and  be the minimum and maximum point of  with respect to , respectively. Let  be composed of those points in  that are encountered when clockwise walking along the boundary of  from  to , where  and . Analogously, let  be composed of those points in  that are encountered when clockwise walking along the boundary of  from  to , where  and .

\begin{figure}[tb]
\begin{center}
\mbox{\includegraphics[scale=0.3]{Sets.eps}}
\caption{Subsets  and  of a point set  determined by a directed straight line .}
\label{fig:sets}
\end{center}
\end{figure}

Let  and  be two directed straight lines not orthogonal to any line through two points of , where the clockwise rotation that brings  to coincide with  is at most . The {\em -partition} of  partitions  into subsets , , , and
. Note that every point in  is contained in one of , , , and . A {\em -partition} of  is {\em balanced} if  and . We now argue that, for every point set , a balanced -partition of  always exists, even if  is arbitrarily prescribed.

\begin{lemma} \label{le:partition}
Let  be a convex point set and let  be a directed straight line not orthogonal to any line through two points of . Then, there exists a directed straight line  that is not orthogonal to any line through two points of  such that the -partition of  is balanced.
\end{lemma}

\begin{proof}
Denote by  the points of  encountered when clockwise walking on the boundary of  from  to . Also, denote by  the points of  encountered when clockwise walking on the boundary of  from  to .

Initialize  to be a directed straight line coincident with . When , we have , , , and . We now clockwise rotate  until it is opposite to  (that is, parallel and pointing in the opposite direction). As we rotate , sets  and  change, hence sets , , , and  change as well. When  is opposite to , we have , , , and . We will argue that there is a moment during such a rotation of  in which the corresponding -partition of  is balanced. Assume that at any time instant during the rotation of  the following hold (see Figs.~\ref{fig:rotating}(a)--(b)):

\begin{itemize}
\item  (possibly  is empty);
\item  (possibly  is empty);
\item  (possibly  is empty);
\item  (possibly  is empty); and
\item  and  are the minimum and maximum point of  w.r.t. , respectively.
\end{itemize}


\begin{figure}[tb]
\begin{center}
\begin{tabular}{c c}
\mbox{\includegraphics[scale=0.3]{RotatingLine.eps}} & \hspace{5mm}
\mbox{\includegraphics[scale=0.35]{Slopes.eps}}\\
(a) \hspace{5mm} & \hspace{5mm} (b)
\end{tabular}
\caption{(a) Sets , , , and  at a certain time instant during the rotation of . (b) The slope of  with respect to the slopes of the lines orthogonal to , to , to , and to .}
\label{fig:rotating}
\end{center}
\end{figure}

The assumption is indeed true when  starts moving, with  and .

As we keep on clockwise rotating , at a certain moment  becomes orthogonal to  or to  (or to both if  and  are parallel). Thus, as we keep on clockwise rotating , sets , , , and  change. Namely:

If  becomes orthogonal first to  and then to , then as  rotates clockwise after the position in which it is orthogonal to , we have

\begin{itemize}
\item ;
\item  (possibly  is empty);
\item  (possibly  is empty);
\item  (possibly  is empty); and
\item  and  are the minimum and maximum point of  w.r.t. , respectively.
\end{itemize}


If  becomes orthogonal first to  and then to , then as  rotates clockwise after the position in which it is orthogonal to , we have that  and  stay unchanged, that  passes from  to , and that  and  are the minimum and maximum point of  w.r.t. , respectively.


If  becomes orthogonal to  and  simultaneously, then as  rotates clockwise after the position in which it is orthogonal to  , we have that  passes from  to , that  passes from  to , and that  and  are the minimum and maximum point of  w.r.t. , respectively.

Observe that:
\begin{enumerate}
\item whenever sets , , , and  change, we have that  and  change at most by two;
\item when  starts rotating we have that , and when  stops rotating we have that ;
\item when  starts rotating we have that  , and when  stops rotating we have that ; and
\item  holds at any time instant.
\end{enumerate}

By continuity, there is a time instant in which  and , or in which  and . This completes the proof of the lemma.
\end{proof}

We now show how to use Lemma~\ref{le:partition} in order to prove Theorem~\ref{th:convex-non-planar}.

Let  be any point set. Assume that no two points of  have the same -coordinate. Such a condition is easily met after rotating the Cartesian axes. Denote by  a vertical straight line directed towards increasing -coordinates. Each of  and  is convex and one-sided with respect to . By Lemma~\ref{lemma:half_convex_planar_graph}, there exist increasing-chord graphs  and  with  and . Then, graph  has less than  edges and contains an increasing-chord path between every pair of vertices in  and between every pair of vertices in . However,  does not have increasing-chord paths between any pair  of vertices such that  and .

We now present and prove the following claim. Consider a convex point set  and a directed straight line  not orthogonal to any line through two points of . Then, there exists a geometric graph  that contains an increasing-chord path between every point in  and every point in , such that .

The application of the claim with  and  provides a graph  that contains an increasing-chord path between every pair  of vertices such that  and . Thus, the union of  and  is an increasing-chord graph with  edges spanning . Therefore, the above claim implies Theorem~\ref{th:convex-non-planar}.

We show an inductive algorithm to construct . Let  be the number of edges that  has as a result of the application of our algorithm on a point set  and a directed straight-line . Also, let , where the maximum is among all point sets  with  points and among all the directed straight-lines  that are not orthogonal to any line through two points of .

Let  be any convex point set with  points and let  be any directed straight line not orthogonal to any line through two points of . By Lemma~\ref{le:partition}, there exists a directed straight line not orthogonal to any line through two points of  and such that the -partition of  is balanced.

Let , let , let , and let .

Point set  is convex and one-sided with respect to . By Lemma~\ref{lemma:half_convex_planar_graph} there exists an increasing-chord graph  with  edges. Analogously, by Lemma~\ref{lemma:half_convex_planar_graph} there exists an increasing-chord graph  with  edges.

Hence, there exists a graph  with  edges containing an increasing-chord path between every point in  and every point in , and between every point in  and every point in . However,  does not have an increasing-chord path between any point in  and any point in , and does not have an increasing-chord path between any point in  and any point in .

By Lemma~\ref{le:partition}, it holds that  and . By definition, we have . Analogously, it holds that . Hence, . This proves the claim and hence Theorem~\ref{th:convex-non-planar}.


\section{Conclusions}

We considered the problem of constructing increasing-chord graphs spanning point sets. We proved that, for every point set , there exists a planar increasing-chord graph  with  and . We also proved that, for every convex point set , there exists an increasing-chord graph  with .

Despite our research efforts, the main question on this topic remains open:

\begin{problem}
Is it true that, for every (convex) point set , there exists an increasing-chord planar graph ?
\end{problem}

One of the directions we took in order to tackle this problem is to assume that the points in  lie on a constant number of straight lines. While a simple modification of the proof of Lemma~\ref{lemma:half_convex_planar_graph} allows us to prove that an increasing-chord planar graph always exists spanning a set of points lying on two straight lines, it is surprising and disheartening that we could not prove a similar result for sets of points lying on three straight lines. The main difficulty seems to lie in the construction of planar increasing-chord graphs spanning sets of points lying on the boundary of an acute triangle.

Gabriel graphs naturally generalize to higher dimensions, where empty balls replace empty disks. In Section~\ref{se:steiner} we showed that, for points in , every Gabriel triangulation is increasing-chord. Can this result be generalized to higher dimensions?

\begin{problem}
Is it true that, for every point set  in , any Gabriel triangulation of  is increasing-chord?\end{problem}

Finally, it would be interesting to understand if increasing-chord graphs with few edges can be constructed for any (possibly non-convex) point set:

\begin{problem}
Is it true that, for every point set , there exists an increasing-chord graph  with ?
\end{problem}

\bibliographystyle{splncs_srt}
\bibliography{main}

\end{document}
