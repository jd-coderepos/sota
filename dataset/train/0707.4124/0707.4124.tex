

\documentclass[11pt,a4paper]{article}
\usepackage{amsfonts,amsmath,amsthm,epsfig,verbatim,amssymb,amscd,eucal,bbm}

\oddsidemargin=0pt \evensidemargin=0pt \topmargin=-20pt
\textwidth=430pt \textheight=660pt








\numberwithin{equation}{section}
\newtheorem{prop}{Proposition}[section]
\newtheorem{defn}[prop]{Definition}
\newtheorem{cor}[prop]{Corollary}
\newtheorem{lem}[prop]{Lemma}
\newtheorem{thm}[prop]{Theorem}
\newtheorem{example}[prop]{Example}
\newtheorem{rem}[prop]{Remark}

 \newcommand{\eps}{\varepsilon}
\def\Blackboardfont{\mathbb}
 \newcommand{\mrm}[1]{\text{\rm #1}}
 \newcommand{\To}{\longrightarrow}
 \newcommand{\h}{\mathcal{H}}
 \newcommand{\s}{\mathcal{S}}
 \newcommand{\A}{\mathcal{A}}
 \newcommand{\D}{\mathcal{D}}
 \newcommand{\cR}{\mathcal{R}}
 \newcommand{\J}{\mathcal{J}}
 \newcommand{\fE}{\mathfrak{E}}
 \newcommand{\M}{\mathcal{M}}
 \newcommand{\W}{\mathcal{W}}
 \newcommand{\X}{\mathcal{X}}
 \newcommand{\BOP}{\mathbf{B}}
 \newcommand{\BH}{\mathbf{B}(\mathcal{H})}
 \newcommand{\KH}{\mathcal{K}(\mathcal{H})}
 \newcommand{\R}{\mathbb{R}}
 \newcommand{\1}{\mathbbm{1}}
 \newcommand{\C}{\mathbb{C}}
 \newcommand{\E}{\mathbb{E}}
\newcommand{\p}{\mathbb{P}}
  \newcommand{\Z}{\mathbb{Z}}
 \newcommand{\N}{\mathbb{N}}
 \newcommand{\Complex}{\mathbb{C}}
 \newcommand{\Field}{\mathbb{F}}
 \newcommand{\RPlus}{\Real^{+}} 
 \newcommand{\Polar}{\mathcal{P}_{\s}}
 \newcommand{\Poly}{\mathcal{P}(E)}
 \newcommand{\EssD}{\mathcal{D}}
 \newcommand{\Lom}{\mathcal{L}}
 \newcommand{\States}{\mathcal{T}}
 \newcommand{\abs}[1]{\left\vert#1\right\vert}
 \newcommand{\set}[1]{\left\{#1\right\}}
 \newcommand{\seq}[1]{\left<#1\right>}
 \newcommand{\norm}[1]{\left\Vert#1\right\Vert}
 \newcommand{\essnorm}[1]{\norm{#1}_{\ess}}
 \def\eref#1{(\ref{#1})}
\begin{document}

\noindent

\title{\bf Services within a Busy Period of an M/M/1 Queue and Dyck
  Paths}




\author{Moez {\sc Draief}
\thanks{LIAFA, Universit{\'e} Paris 7, case 7014,
2 place Jussieu, 75251 Paris Cedex 05, France.
E-mail: {\tt Moez.Draief@liafa.jussieu.fr}
and {\tt  Jean.Mairesse@liafa.jussieu.fr}.}
\and
Jean {\sc Mairesse}\,\footnotemark[1]
}

\maketitle

\begin{abstract}

We analyze the service times of customers in a stable M/M/1 queue in
equilibrium depending on their position in a busy period.
We give the law of the service of a customer at the beginning, at the end,
or in the middle of the busy period. It enables as a by-product to prove
that the process of instants of beginning of services is not Poisson.
We then proceed to a more precise analysis. We consider a family of
polynomial generating series associated with Dyck paths of length 
and we show that they provide the correlation function of the
successive services in a busy period with  customers.

\end{abstract}

\renewcommand\abstractname{R\'esum\'e}
\begin{abstract}
On s'int\'eresse \`a l'analyse des temps de service des clients
d'une file M/M/1 stable et en \'equilibre selon
leur position dans une p\'eriode d'activit\'e.
On donne la loi d'un service sachant que le client se
trouve au d\'ebut, \`a la fin ou au milieu de la p\'eriode d'activit\'e. Ceci
permet, au passage, de prouver que le processus des instants de
d\'ebut de service n'est pas un processus de
Poisson. On m\`ene ensuite une \'etude plus fine.
On exhibe une famille de s\'eries g\'en\'eratrices
polyn\^omiales associ\'ees aux chemins de Dyck de longueur   et on
montre qu'il s'agit de la fonction de corr\'elation des diff\'erents services dans une
p\'eriode d'activit\'e comportant  clients.
\end{abstract}


\smallskip

{\noindent\bf Keywords:} M/M/1 queue, busy period, Dyck paths.

\smallskip

{\noindent\bf AMS classification (2000):} 60K25, 68R05.




\section{Introduction}

The M/M/1//FIFO queue (or M/M/1 queue) is the queue with a
Poissonian arrival stream, exponential services, a single server,
an unlimited buffer capacity, and a First-In-First-Out service
discipline. It can be argued that the M/M/1 queue is
the most elementary and the most studied system in queueing theory,
see for instance \cite{cohe82,taka,robe,ScWe}. Quoting \cite{ScWe},
``most likely, any book with {\em queueing} in the title has something
to say on the subject''.

Let  be the intensity of the Poisson arrival process and let
 be the parameter of the exponential service times. Assume that
the stability condition  holds and consider the queue
in equilibrium. Our objective is to get precise information on the
distribution of the service of a customer based on its position  in
the busy period.


First of all, recall that the distribution of the first, respectively
last,  service is an
exponential of parameter , respectively .
We are then able
to compute the distribution of a service in the ``middle''
of a busy period (i.e. neither at the beginning nor at the end).
As a by-product, we also get the distribution of the duration between
two successive beginning of services. Since it is not an exponential,
we conclude that the point process of
the instants of beginning of services is not Poisson (as opposed to the point process
of completion of services).

Then we study the service time of the -th customer in a busy period of length 
(i.e. containing  customers). Consider a busy period conditionned to be of length ,
and let  be the corresponding embedded queue-length excursion.
Its trajectories are equiprobable and it is easy to see that they are in bijection with Dyck paths of length .
If we condition  to be associated with a given Dyck path  of length  then we observe
that the law of the service time of the -th customer is equal to the convolution product of  exponentials
of parameter  where  is the length of the intersection of  with the line .
By summing over Dyck paths of length , we get an expression for the joint law of the services in a busy period of length .
Then using elementary properties of Dyck paths, we obtain results on services within a busy period somewhat difficult to obtain
by direct probabilistic arguments (Section 4). The correlation function of the services is a natural generating polynomial of
Dyck paths following a simple integral recursion (Section 5).


Using the combinatorial properties of lattice paths to study the busy period of simple queues is classical,
see \cite{Flajolet,Guillemin, taka2} and references therein.
In these articles, quantities such that the area swept by the queue-length process during a busy period are studied, with a much more involved
combinatorial analysis than what is presented below for the sequence of services within a busy period. This should come as no surprise.
The area and related quantities, are derived by counting in a Dyck path the number of ascents and descents of a given vertical coordinate
(Dyck paths are lattice paths in , see Section 3). On the contrary, the sequence of services is derived by counting in a Dyck path
the number of ascents of a given horizontal coordinate (roughly speaking). This is in essence like working with generating polynomials of Dyck paths in
{\em non-commuting} variables. It is therefore hopeless to get as precise information.


\section{In the Middle of the Busy Period}
Given a positive real random variable  with law , denote its Laplace transform by
.
We write   for the conditional
law of  given an event . The corresponding Laplace transform
is denoted . The convolution product of two probability distributions  and  is denoted by
. The
indicator function of a subset  of a set is denoted by .
It is convenient to denote by  the exponential
distribution of parameter  defined by .
Recall that .

\medskip

We consider an  queue with the following notations.
Let  be the arrival Poisson process of intensity
. Let  be the inter-arrival times, with
. Denote by  the service
times of the customers. The sequence  is
i.i.d. and .
We assume that the stability condition  is satisfied,
and we consider the queue in equilibrium. Let  be the
queue-length process, where  is the number of customers either in
service or in the buffer at time .


The state of the server can be described as an alternating sequence of
idle and busy periods. A {\em busy period}
is a maximal period during which . An idle period is a maximal
period during which .
The {\em length}  of a busy period  (not to be confused with its
duration) is the number of customers served during the busy period.
Throughout, when we consider a generic busy period , we denote for
simplicity by  and  respectively the service times and the inter-arrival
times of the different customers in the busy period.

\begin{lem}\label{le-easy}
Let  be  the event that a generic busy period consists
of  customers, then

\end{lem}
\noindent
The justification is easy.

The durations of successive busy periods and idle
periods are independent random variables.
The duration of an idle period is clearly distributed as
. The distribution of a busy period is more
complex. The next results can be found for instance in \cite[Chapter
II.2.2]{cohe82} or \cite[Chapter 1.2]{taka}.
The probability that a busy period 
consists of  customers is given by

where  is the -th Catalan number, see \S \ref{se-dyck}.
Let  be the conditional law of the duration of a busy period, given that the length
of the busy
period is .
The Laplace
transform of  is given by

Hence,  is the distribution of the sum of  i.i.d. r.v.'s of law
.

\medskip

Given two independent random variables 
and , where ,
recall that

Using elementary arguments based on the memoryless property of the
exponential distribution, we get:

Furthermore, remarking that  and
using \eref{eq-recall}, it follows that:

Our goal is now to derive the law of a service in the {\em middle} of
, i.e. of a service which is neither the first nor the last one
(assuming that ).

Let  be the service of a generic customer numbered 
and let  be the busy period it belongs to. Define the events

Clearly the four events are disjoint and . Since the lengths of successive busy periods
are i.i.d., we obtain immediately that

Now using (\ref{eq-length}), we get 
and . It follows that

Clearly,  and . We deduce that

That is,

After simplification of the above expression, we obtain the
Laplace transform of the conditional law of  on the event
 :


As a by-product, we can prove that the process of
instants of beginning of services is not a Poisson process,
in contrast
with the process of completion of services (departure instants) which
is Poisson of intensity  according to Burke Theorem
\cite{burk,reic}.
Let us detail the argument. Let  be the difference between the
instants of beginning of services of two generic successive customers
numbered  and . Let  be the busy period of  and let  be the first idle period following
 (. Using \eref{eq-basic},
\eref{eq-basic2}, we get immediately that

and

Since we have just computed , we
deduce the Laplace transform of :

We check on this expression that  and we have
. In particular, we have , where  is a generic inter-departure time.


\section{Dyck Paths}
\label{se-dyck}

The Catalan numbers  are defined by

The generating function of these numbers is given by

The first Catalan numbers are
. They appear in many combinatorial
contexts see for instance \cite{GKPa,stan2}. In particular, 
is the number of Dyck paths of length . A {\em Dyck path}  of
length  is a  path in the lattice  which begins
at the origin  ends at  and with steps of type
 or . Denote by  the set of Dyck paths of
length , observe that  is a singleton whose element is
the unique Dyck path of length .

We now define a family of polynomials
related to Dyck paths.
Let  and let
 be the line , for  and denote by
 the length of the intersection of  with 
(equivalently  is the number of lattice points common
to  and ). We introduce two polynomials 
and  defined by

Let   and  be the two families
of polynomials defined by .
Clearly  and  are homogeneous polynomials of degree 
over the  variables .

\section{The Law of the Services in a Busy Period}
Recall that the queue-length process  is a continuous time Birth-and-Death process on  with
generator  such that . Let  denote the Markov chain embedded at its jump instants.
More precisely, let  be the point process obtained as the superposition  of the arrival and departure processes and let 
 be its points with the convention . Then we set . The transition matrix of  is given by

and  otherwise.

A busy period corresponds to an excursion of  from  to its first return to .
With the same numbering convention as in Section ,
the generic busy period  consists of  customers if and only if

On this event, the (random) path with  successive edges
 is a (random) Dyck path of
length . We call it the Dyck path {\em associated with} 
(see  Figure \ref{bijec}). On the event , all Dyck
paths appear with the same probability (the probability of a given
trajectory  depends only on the number of increasing and
decreasing jumps, see (\ref{transition})). On the event that
 and that the associated Dyck path is , the
power of  in  is the number of customers which
join the system between the the -th and the -th
departures. Combining these observations with the fact that the
time between successive transitions of  are
independent r.v.'s of law  as long as the
queue is non empty, we get :

\begin{thm}\label{main}
Given that the length of the busy period is ,
the conditional density of the random vector
 representing the service
times of the successive customers is

where  is the Dyck polynomial of
degree  defined in Section 3.
\end{thm}
\begin{figure}[h]

\caption{Dyck path associated with a busy period.}
\label{bijec}
\end{figure}
A direct computation of the Laplace transform leads to the following :
\begin{cor}\label{Laplace trans}
Consider a random vector
.
Its Laplace transform is given by

where , and  is defined in Section 3.
\end{cor}
Let us paraphrase the above results in a somewhat more intuitive way.
In a busy period of length , the conditional law of
 is the same as the law of
 that we now describe.
The law of  is an  independent of
. Let  be a
r.v. uniformly distributed over . Conditionally on , the r.v.'s 
are independent and distributed as the sum of  random variables of law
, where  is the exponent of 
in . This is illustrated in Table 1.

We now exploit the correspondance with Dyck paths.

Let  be the set of Dyck paths of length  where the
first return to the axis , after the origin
, occurs at the point , . Clearly, the sets  
are disjoint and . Furthermore


A consequence of the above is the very classical identity on Catalan
numbers :

Let  be defined by , going back to Corollary \ref{Laplace trans}, we have

where . We also define , then

\begin{prop}\label{Rec}
On the event , we have

for, ,

\end{prop}
\begin{center}
\begin{figure}[hbt]

\caption{The mapping }
\label{Psi}
\end{figure}
\end{center}
On Table 1, one notices a simple relation between the laws of
 and , which is actually always true :
\begin{prop}
Let  be a generic busy period, for  we have

\end{prop}
\begin{center}
\begin{tabular}{|c|c|c|c|c|c|c|c|c|}
\hline
&&&&&&&& \\
&& 1 & 2 & 3 & 4 & 5 & 6 & 7  \\
&&&&&&&& \\ \hline
  & & & 2 & 2 & 1 & & & \\
 & & 2 & 2 & 1 & & &&  \\
 & & 3 & 2 & & & &&  \\
  & & 1 & & & & & & \\ \hline
  & & & 5 & 5 & 3 & 1 &&  \\
 & & 5 & 5 & 3 & 1 & &&  \\
 & &7 &5 &2 & & && \\
 & &9 &5 & & & & & \\
&  & 1 & & & & && \\ \hline
   & & &14  &14  &9  &4  &1& \\
 & &14  &14  &9  &4  &1 &&  \\ \hline
   & & & 42 & 42 & 28 & 14 &5 &1  \\
 & & 42 & 42 & 28& 14 &5 &1 &  \\  \hline
  \end{tabular}

The table should be read as follows. For instance, on , the law of  is .
\end{center}

\begin{proof}
The mapping  is defined in Figure
\ref{Psi}. It is clearly an involution, hence a bijection.
More formally, given a Dyck path  such that
 then  is
defined by . In view
of Corollary \ref{Laplace trans}, it completes the proof.
\end{proof}

\section{Dyck Paths Polynomials}
We go back to the family of polynomials  defined in Section
\ref{se-dyck}. We are going to use Theorem \ref{main} to give nice expressions for the 's.
Let  be the event that a generic busy period consists of
 customers.
Let  be borelians of ,

Let  and for
, let  and . Using Lemma \ref{le-easy}, we have

Then, using theorem \ref{main}, we get

Simple manipulations of formula (\ref{iterint1}) then yield :
\begin{lem}\label{recursion}
The polynomials  satisfy the following equations

and

\end{lem}
For completeness, here is a direct proof of (\ref{recpoly}) without using Theorem \ref{main}.

Let  be the set of all Dyck paths of length  starting
with  steps of type  followed by one step of type 
and define the polynomial  such that

Clearly, we have

\begin{figure}[h]

\caption{Proof of the equality (\ref{extension}):  Paths contributing to  (Left) and to  (Right).}
\label{extend}
\end{figure}
Hence, we get

With the help of Figure \ref{extend}, we notice that

\begin{figure}[hbt]

\caption{The volumes of the gray areas are  (left) and
   (right).}
\label{Volume}
\end{figure}

It leads to


This result can also be proved using the theory of species presented in \cite{Leroux}.
Finally, using (\ref{recpoly2}), the polynomials  can be interpreted as volumes. We give a
representation of this in Figure \ref{Volume} for  and .

\paragraph{\bf Conclusion.}
Here are several other simple models of queues for which the queue-length process is a Birth-and-Death process: the M/M/K/ queue,
the M/M/ queue, or the M/M/K/L queue (). In each case, if the generic busy period is of length ,
we can associate with it a Dyck path of length . However, the different Dyck paths of length  are not equiprobable anymore.
Hence, we do not get a simple formula for the joint law of the services as in Theorem \ref{main}.
\begin{thebibliography}{1}

\bibitem{Leroux}
F.~Bergeron, G.~Labelle and P.~Leroux,
\newblock Combinatorial species and tree-like structures.
\newblock Cambridge University Press, 1998.

\bibitem{burk}
P.~Burke,
\newblock The output of a queueing system.
\newblock Operations Research 4 (1956) 699-704.

\bibitem{cohe82}
J.W.~Cohen,
\newblock The single server queue. 2nd edition.
\newblock North-Holland, Amsterdam, 1982.


\bibitem{Flajolet}
P.~Flajolet and F.~Guillemin,
\newblock The formal theory of Birth-and-Death processes, lattice path combinatorics, and continued fractions.
\newblock Advances in Applied Probability 32 (2000) 750-778.

\bibitem{GKPa}
R.~Graham, D.~Knuth, and O.~Patashnik,
\newblock Concrete mathematics: a foundation for computer science. 2nd edition.
\newblock Addison-Wesley, 1994.

\bibitem{Guillemin}
F.~Guillemin and D.~Pinchon,
\newblock On the area swept under the occupation process of an M/M/1 queue in a busy period. Queueing Systems Theory
Appl. 29 (1998), no. 2-4, 383--398.

\bibitem{reic}
E.~Reich,
\newblock Waiting times when queues are in tandem.
\newblock Ann. Math. Stat. 28 (1957) 527-530.

\bibitem{robe}
P.~Robert,
\newblock R\'eseaux et files d'attente: m\'ethodes probabilistes.
\newblock Number~35 in Math\'ematiques \& Applications. Springer, 2000.

\bibitem{ScWe}
A.~Schwartz and A.~Weiss,
\newblock Large deviations for performance analysis. Queues,
  communications, and computing.
\newblock Chapman \& Hall, London, 1995.


\bibitem{stan2}
R.~Stanley,
\newblock Enumerative Combinatorics, Vol. 2.
\newblock Number~62 in Cambridge Studies in Advanced Mathematics. Cambridge
  University Press, 1999.

\bibitem{taka}
L.Tak\'{a}cs,
\newblock Introduction to the theory of queues.
\newblock University Texts in the Mathematical Sciences. Oxford University
  Press, 1962.

\bibitem{taka2}
L.~Tak\'{a}cs,
\newblock Queueing methods in the theory of random graphs,
\newblock Probability and Stochastics Series, CRC, Boca Raton, FL, (1995) 45-78.

\end{thebibliography}

\end{document}
