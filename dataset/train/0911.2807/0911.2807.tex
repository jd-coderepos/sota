\documentclass{llncs}


\usepackage{amsmath}
\usepackage{amssymb}
\usepackage {graphicx}
\begin{document}

\title{On the Minimum Size of a Contraction-Universal Tree}
\author{Olivier Bodini}
\institute{\obeylines LIP, \'Ecole Normale Sup\'erieure de Lyon,
46 All\'ee d'Italie, 69364 Lyon Cedex 05, France.} \maketitle
\pagestyle{myheadings} \markboth{\hfill{\sc WG (2002)}} {{\sc WG
(2002)}\hfill}

\begin{abstract}

A tree  is -\textit{universal} for the class of trees
if for every tree  of size ,  can be obtained from
 by successive contractions of edges. We prove that a
-universal tree for the class of trees has at least  edges where  is the Euler's
constant and we build such a tree with less than  edges for a
fixed constant 
\end{abstract}



\section{Introduction}





What is the minimum size of an object in which every object of
size  embeds? Issued from the category theory, questions of
this kind appeared in graph theory. For instance, R. Rado~
\cite{Ra} proved the existence of an "initial countable graph".
Recently, Z. F\"{u}redi and P. Komj\`{a}th \cite{FK} studied a
connected question.

We use here the following definition~: given a sub-class  of
graphs (trees, planar graphs, etc.), a graph  is
-\textit{universal} for  if for every graph  of size 
in  is a minor of  i.e. it can be obtained from
 by successive contractions or deletions of edges.

Inspired by the Robertson and Seymour work \cite{RS} on graph
minors, P. Duchet asked whether a polynomial bound in  could be
found for the size of a -universal tree for the class of trees.
We give here a positive sub-quadratic answer.

From an applied point of view, such an object would possibly
allows us to define a tree from the representation of its
contraction.

The main results of this paper are the following theorems which
give bounds for the minimum size of a -universal tree for the
class of trees~:





\begin{theorem}\label{th1} A -universal tree for the class of trees has
at least  edges where  is
the Euler's constant.
\end{theorem}


\begin{theorem} \label{th2} There exists a -universal tree  for the
class of trees with less than  edges for a fixed constant 
\end{theorem}
Our proof follows a recursive
construction where large trees are obtained by some amalgamation
process involving simpler trees. With this method, the constant
 could be reduced to 1.88... but it seems difficult to improve
this value.





We conclude the paper with related open questions.





\section{Terminology}





Our graphs are undirected and simple (with neither loops nor
multiple edges). We denote by  a graph (its vertex set is
 and its edge set is  (a subset of the family of all
the -subsets of cardinality 2)). Referring to C. Thomassen
\cite{Th}, we recall some basic definitions that are useful for
our purpose:

We denote by  the path of size 

If  is a vertex then  the \textit{degree} of  is the number of edges incident to


Let  be an edge of , the graph denoted by  is the
graph on the vertex set of , whose edge set is the edge set of
 without . We call classically this operation
\textit{deletion}.

Let  be an edge of , we name
\textit{contraction of }\textit{ along } the graph denoted by , with  where  is a new
vertex and  the edge set which contains all the edges of the sub-graph
 on  and all the edges of the form  for
 or  belonging to .

We say that  is a \textit{minor} of  if and only if we can
obtain it from  by successively deleting and~/or contracting
edges, in an other way, we can define the set  of minors of
 by the recursive formula~:








The notion of minor induces a partial order on graphs. We write
 to mean " is a minor of ".





For technical reasons, we prefer to use the size of a tree (edge number)
rather than its order (vertex number).

Finally, let us recall that, a graph  is -\textit{universal} \textit{for a sub-class } of graphs if for
every element  of  with  edges is a minor of .





\section{A Lower Bound}





In this section, we prove that a -universal tree  for
the trees has asymptotically at least  edges. We use the
fact that  has to contain all spiders of size  as
minors. A \textit{spider }\textit{ on a vertex } is a tree
such that . We denote the spider constituted by paths
of lengths  by 
(Fig.1).


\begin{figure}[htbp]
\centerline{\includegraphics[width=1.75in,height=0.79in]{unitrdes1.eps}}
\label{fig1}
\begin{center}
Fig.1. 
\end{center}
\end{figure}


\begin{definition} Let  be a tree, we denote by  the subtree of  with ,
where  is the set of the leaves of . Also, we denote by
 the -th iteration of .
\end{definition}




\begin{lemma}\label{lm1}  involves that .
 Moreover, if for all ,
 then  is a vertex.
Otherwise, put  the first value such that , we have
 excepted for
, in this last case we have .
\end{lemma}




\begin{proof} This just follows from an observation.\qed
\end{proof}





\begin{lemma}\label{lm2} For every tree ,  has at least  leaves.
\end{lemma}

\begin{proof} Trivial.\qed
\end{proof}



\begin{theorem}\label{th3} A -universal tree  for the class of trees
has at least  edges.
\end{theorem}




\begin{proof}A -universal tree  for the class of trees has to contain as
minors all spiders of size . So, for all  it contains as
minors the spiders  where we have  times the letter . By the lemma
\ref{lm1}, for all ,  and if  is odd, .
Moreover, it is clear that the terminal edges of the  constitute a partition of . By the lemma
\ref{lm2}, this involves that  has at least

 edges if  is even and  edges if  is odd. An easy
 calculation proves that these values are always equal to .\qed
\end{proof}




\begin{proof} (of the theorem \ref{th1}) it follows from the usual estimate
 and the inequality
.\qed
\end{proof}

What the above proof shows, in fact, is the following :





\begin{corollary}A minimum -universal spider for the class of
spiders has  edges.
\end{corollary}




\begin{proof}The spider  is clearly a -universal spider of size
 for the class of
spiders, and by theorem \ref{th3} it is a minimum value.\qed
\end{proof}




\section{The Main Stem}





In the sequel, we deal with\textit{ rooted graph}, i.e. graph 
where we can distinguish a special vertex denoted by ,
called the \textit{root}. Conventionally, any contracted graph
 of same rooted graph  will be rooted at the unique
vertex which is the image of the root under the contraction
mapping, we say in this case that the rooted graph  is a
\textit{rooted contraction} of . Note that, the contraction
operator suffices to obtain all minor trees of a tree. So, we can
now define the following new notion for sub-classes of rooted
trees~: a rooted tree  is\textit{ strongly -universal
for a sub-classes }\textit{ of rooted trees} if for every
rooted tree \textit{ in } of size  is a rooted
contraction of . The concept of root is introduced to
avoid problems with graph isomorphisms that, otherwise would
greatly impede our inductive proof.

For every edge  of a tree , the forest  has two connected
components. We call \textit{-branch}, denoted by , the connected component of
 which does not contain , we define the root of
 as 

A\textit{ main stem }of a rooted tree of size  is defined as a
path  which is issued from the root and such that for all
-branches  with , we have
 (Fig.2).

\begin{figure}[htbp]
\centerline{\includegraphics[width=1.64in,height=1.06in]{unitrdes2.eps}}
\label{fig2}
\begin{center}
Fig.2. A main stem in bold
\end{center}
\end{figure}



The following lemma suggests the procedure which will be used to find a
sub-quadratic upper bound for universal trees. Roughly speaking, it endows
every tree with some recursive structure constructed with the help of main
stems.





\begin{lemma} Every rooted tree has a main stem.
\end{lemma}




\begin{proof}By induction on the size of the rooted tree. Let  be a rooted tree, if  has
one or two edges, it is trivial. Otherwise let us consider the
sub-graph , which is a forest. We
choose a connected component  with maximum size and we
denote by  the unique vertex of  which is adjacent to
. Tree , rooted in , has, by the induction
hypothesis, a main stem  Then the path  is a main stem of .\qed
\end{proof}





\begin{remark} A tree may possess in general several main stems. Let
us notice also that a main stem is not necessarily one of the
longest paths which contain the root.
\end{remark}




\section{The Upper Bound}





We need some new definitions. A \textit{rooted brush} (Fig.3) is a
rooted tree such that the vertices of degree greater than 2 are on
a same path  issued from the root.

\begin{figure}[htbp]
\centerline{\includegraphics[width=1.35in,height=1.22in]{unitrdes3.eps}}
\label{fig3}
\begin{center}
Fig.3. A rooted brush
\end{center}
\end{figure}

A \textit{rooted comb}  (Fig.4) is a rooted brush with  and , .

\begin{figure}[htbp]
\centerline{\includegraphics[width=1.35in,height=1.13in]{unitrdes4.eps}}
\label{fig4}
\begin{center}
Fig.4. A rooted comb
\end{center}
\end{figure}



The \textit{length of a rooted comb} corresponds to the length of the longest path  issued from the root
which contains all vertices of degree greater than 2.





To obtain an upper bound, we consider two building processes : the
first one, a brushing , maps rooted trees with a main stem
into rooted brushes, the second one, a ramifying , consists
in obtaining a sequence of rooted trees, assuming that we have an
increasing sequence of rooted combs. We note  the -th
element of the sequence. These building processes will possess the
following fundamental property:





\begin{property} \label{pr1} Let  a rooted tree with a
main stem  and  a sequence of rooted combs :



\end{property}



\begin{lemma} If building processes verify the property \ref{pr1} and
if for all , the rooted comb  is strongly -universal
for the class of rooted brushes then the rooted tree  is strongly
-universal for the class of rooted trees.
\end{lemma}




\begin{proof}It is just an interpretation of the property.\qed
\end{proof}




We now establish the existence of building processes which satisfy
property \ref{pr1}.

\textbf{Brushing}  (Fig.5). Let  be a rooted tree with a
main stem . We are going to associate a rooted brush 
with it, denoted  of the same size
built from the same main stem  with the following
process: every -branch  connected to the main stem by
edge  is replaced by a path of length  connected by the same edge.

\begin{figure}[htbp]
\centerline{\includegraphics[width=2.92in,height=1.48in]{unitrdes5.eps}}
\label{fig5}
\begin{center}
Fig.5.
\end{center}
\end{figure}


\textbf{Ramifying }\textbf{.} For the second building
process we work in two steps :





\textbf{First step.} Given rooted trees  with
disjoint vertex sets, we build another rooted tree , denoted
, in the following way~:











\noindent
and .

If , conventionally .


Prosaically, from a path  of size  and from  rooted trees , we
build a rooted tree joining a branch  to the vertex 
of  (Fig.6).

\begin{figure}[htbp]
\centerline{\includegraphics[width=1.73in,height=1.19in]{unitrdes6.eps}}
\label{fig6}
\begin{center}
Fig.6. A rooted comb 
\end{center}
\end{figure}

\textbf{Second step.} By convention, .

We are going to construct rooted trees  in the following
way~:\\ , , and   if .

We can now define ~:







\begin{lemma} The building processes described above verify the
property \ref{pr1}.
\end{lemma}




\begin{proof} First, note that 
is an increasing sequence. We prove the lemma by recurrence on the
size  of . When  or , this is trivial. We
suppose the property is verified for  with size . Let
 be a rooted tree of size  with a stem , we note
 the edges of  issued from  which do
not belong to . To each -branch of  with  corresponds by  a
-branch (it is a path of same size) in . So there exists  distinct -branches  in  that we can respectively contract to obtain each
-branch with  in . By recurrence hypothesis, we have for  and we have also . So
each -branch of  is a minor contraction of . By associativity of contraction map,
we have .\qed
\end{proof}




In this phase, we determine a sequence of rooted combs  such that the rooted combs
 are strongly -universal for the rooted brushes.

In order to achieve this result, we define  as the set of
functions ~: 
satisfying the following property~:








\begin{lemma}  is not empty, it contains the following function
, defined for  by~:






\noindent
where  is the 2-valuation of  (i.e. the
greatest power of 2 dividing .
\end{lemma}


\begin{proof}The verification is obvious.\qed
\end{proof}





\begin{lemma}\label{lm:b} For every sequence 
of functions such that  for  and  for all  and , the rooted comb defined by  where  designs the path of size , for
 is strongly -universal for the rooted brushes.
\end{lemma}




\begin{proof} By induction on  :  is strongly
1-universal for the rooted brushes.

Suppose that  has all rooted brushes with  edges
as rooted contractions.

We consider two cases depending on the shape of a rooted brush  of size
 :

\hspace{1.7cm} case 1 \hspace{4cm} case 2
\begin{figure}[htbp]
\centerline{\includegraphics[width=3.80in,height=2.05in]{unitrdes7.eps}}
\end{figure}


Brushes of case 1 are clearly rooted contractions of the rooted
comb  (, so . Let us study case~2 :  is by induction hypothesis
a rooted contraction of the rooted comb ,
moreover  Finally, by the property of , there exists , such that  has more than 
edges. Linking these two points, we can conclude that the rooted
brush  is always a rooted contraction of the rooted comb
.\qed
\end{proof}





The rooted comb built as in lemma \ref{lm:b} will be said to be
\textit{associated to the sequence}  and denoted by .





\begin{theorem} A minimum strongly -universal rooted brush for the
rooted brushes has  edges.
\end{theorem}




\begin{proof} Proceeding as for theorem \ref{th1}, we obtain, mutatis mutandis, that a
-universal brush for the brushes has at least 
edges. This order of magnitude is precisely the size of the
strongly -universal rooted comb  for the class of
rooted brushes.\qed
\end{proof}





We have this immediate corollary~:


\begin{corollary} A minimum -universal brush for the brushes has
 edges.
\end{corollary}




By convention, we put  (tree reduced in a vertex)

We define .





As before, we will say that the tree built in such a way is\textit{ recursively associated to the sequence } and denoted by
.





Thus, we have :





\begin{theorem} The rooted tree  is strongly -universal
for the class of rooted trees.
\end{theorem}




We now analyze the size of .





\begin{proposition} Let  be a
sequence of functions such that  for . The
size of a -universal tree constructed from the sequence is
given by the following recursive formula :

 and 
\end{proposition}




\begin{proof}It derives from the following observation~:\\
\noindent  edges constitute the main stem, we have to add  edges to link branches to the main stem and  edges for the branches.\qed
\end{proof}



\begin{theorem} There is a sequence of functions  such that  and  where  is the unique
positive solution of the equation .
\end{theorem}




\begin{proof}We take the following sequence of functions~:\\
 if  and  even,  if  odd and . It is clear that, if  is a power
of 2, the comb  is strongly -universal for the
brushes.

In fact, the function  takes the value  when  is not a power of 2, otherwise it is equal to .
Thanks to this remark and with , (the sequence of sizes is increasing), we obtain
.
Thus, in evaluating the sums and reorganizing the terms, we obtain :






\noindent
with







\\
Now  when  and  by definition
of \\ So , hence .\qed
\end{proof}


\begin{remark} We observe that , where  is the positive root of .
\end{remark}

\bigskip


Theorem \ref{th2} then follows since any rooted tree which is
strongly -universal for the rooted trees is also clearly
-universal for the class of trees.





\section{Conclusion and Related Questions}





When using the sequence  of lemma \ref{lm:b}, the induction step leads to
involved expressions that do not allow us to find the asymptotic
behavior of the corresponding term . A computer simulation
gives that such a -universal tree for the trees has less than
 edges. In any case, the constructive approach we
proposed here, seems to be hopeless to reach the asymptotic best
size of a -universal tree for the trees.





\begin{conjecture} The minimal size of a -universal tree for the
trees is .
\end{conjecture}




As a possible way to prove such a conjecture, it would be
interesting to obtain an explicit effective coding of a tree of
size  using a list of contracted edges taken in a -universal
tree for the trees.

A variant of our problem consists in determining a minimum tree
which contains as a sub{\-}tree every tree of size  This is
closely related to a well known still open conjecture due to
Erd\"{o}s and S\"{o}s (see \cite{ES}).




\begin{thebibliography}{9}
\bibitem{Ra} R. Rado, Universal graphs and universal functions, \textit{Acta Arith.,} \textbf{9} (1964),
331-340.
\bibitem{FK} Z. F\"{u}redi and P. Komj\`{a}th, Nonexistence of universal graphs
without some trees, \textit{Combinatorica, }\textbf{17}, (2)
(1997), 163-171.

\bibitem{RS} N. Robertson and P.D. Seymour, series of papers on Graph
minors, \textit{Journal of combinarotics, serie B, }(1983-...).

\bibitem{Th} C. Thomassen, Embeddings and Minors, chapter 5 in \textit{Handbook of Combinatorics, }( R. Graham, M.
Gr\"{o}tschel and L. Lov\`{a}sz, eds.), Elsevier Science B.V., 1995,
301-349.

\bibitem{ES} P. Erd\"{o}s and T. Gallai, On maximal paths and circuits of graphs,
\textit{Acta Math. Sci. Hungar. }\textbf{10} (1959), 337-356.
\end{thebibliography}




\end{document}
