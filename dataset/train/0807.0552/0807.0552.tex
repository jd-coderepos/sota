\documentclass[english,11pt]{article}
\pdfoutput=1
\usepackage{amsthm, amssymb, fullpage, graphicx, subfigure, charter,hyperref}

\newtheorem{theorem}{Theorem}
\newtheorem{lemma}{Lemma}
\newtheorem{corollary}{Corollary}
\newtheorem{observation}{Observation}
\newcommand{\R}{\mathbb R}
\newcommand{\bigc}{\mathcal C}
\newcommand{\bigt}{\mathcal T}
\newcommand{\bigw}{\mathcal W}
\newcommand{\tr}{{\mathcal T}} 
\let\newbar\bar
\renewcommand{\bar}[1]{\textstyle\widetilde{\scriptstyle{#1}}}
\newcommand{\closure}{\mathrm{Cl}}

\title{Decomposition of Multiple Coverings into More Parts}
\author{Greg Aloupis\thanks{Universit\'e Libre de Bruxelles (ULB), CP212, Bld. du Triomphe, 1050 Bruxelles, Belgium. Supported by the \emph{Communaut\'e Fran\c caise de Belgique}. E-mail: {\tt \{greg.aloupis,jcardin,secollet,slanger\}@ulb.ac.be}.} \and Jean Cardinal\footnotemark[1] \and  S\'ebastien Collette\footnotemark[1]~\thanks{Charg\'e de Recherches du FRS-FNRS.} \and Stefan Langerman\footnotemark[1]~\thanks{Chercheur Qualifi\'e du FRS-FNRS.} \and David Orden\thanks{Universidad de Alcal\'a, Spain. E-mail: {\tt \{david.orden,pedro.ramos\}@uah.es}.}  \and Pedro Ramos\footnotemark[4]}
\date{}

\begin{document}
\maketitle
\sloppy

\begin{abstract}
We prove that for every centrally symmetric convex polygon , there exists a constant  such that any -fold covering of the plane by
translates of  can be decomposed into  coverings. This improves on a quadratic upper bound proved by Pach and T\'oth (SoCG'07). The question is motivated by a sensor network problem, in which a region has to be monitored by sensors with limited battery lifetime.
\end{abstract}

\section{Introduction}

A collection of subsets of the plane forms a {\em -fold covering} if any point in the plane is covered by at least  subsets. We consider the following problem (see Figure~\ref{fig:decomp}):\\
{\em Given a convex planar body , does there exist a function  such that any -fold covering of the plane by translates of  can be decomposed into  disjoint (1-fold) coverings?}\medskip

This problem, first raised by Pach and T\'oth in 1980 (see~\cite{Pa80} and references therein), is  a classical question in discrete geometry and remains largely open. In fact it is not even known whether there exists a constant  such that any -fold covering  can be decomposed into {\em two} coverings. A survey of the literature can be found in the book of Brass, Moser, and Pach~\cite{RPDG}.\medskip

\begin{figure}[htb]
\begin{center}
\includegraphics[scale=.5,angle=-90]{decomposition.pdf}
\end{center}
\caption{\label{fig:decomp}A -fold covering of a rectangle by hexagons that can be decomposed into three coverings.}
\end{figure}

Mani and Pach proved that 33-fold coverings by unit disks can be decomposed into two coverings~\cite{MP86}. Tardos and T\'{o}th recently proved that, for triangles, any 43-fold covering can be decomposed into two coverings~\cite{TT07}. 

For the case of centrally symmetric convex polygons, the problem proved to be challenging. The existence of a function 
was conjectured in 1980~\cite{Pa80}, and a few years later, resolved positively~\cite{Pach86} by Pach. Only twenty years later was it shown that  is at most quadratic in .

\begin{theorem}[Pach and T\'{o}th~\cite{PT07}]
\label{we-beat-pach}
Given a centrally symmetric convex polygon , there exists a constant  such that every -fold covering of the plane by translates of  can be decomposed into  coverings.
\end{theorem}

\noindent  In addition to the above, a lower bound of   was also given.\medskip

The main result  in this paper is an improvement of the bound in Theorem~\ref{we-beat-pach}, from  to ; thus, the upper and lower bounds now asymptotically match.

\paragraph{Related Work.} 
Coverings with other families of convex shapes have also been studied. For instance, indecomposable coverings of the plane by strips and rectangles were given by Pach, Tardos and T\'{o}th~\cite{PTT07}. The problem for arbitrary disks remains open, although a negative result for the dual problem was proved in~\cite{PTT07}: for any , there exists a point set such that for any 2-coloring of this set, an open disk containing  points of the same color can be found. Set-theoretic investigations of infinite-fold coverings can be found in~\cite{settheo}.

Note that covering decompositions can be seen as colorings of geometric hypergraphs. In these hypergraphs, vertices are the convex bodies in the covering, and every point in the plane corresponds to a hyperedge, defined as the set of bodies containing that point.
The assignment of colors to the vertices of this graph, such that every hyperedge contains all  colors, yields a suitable decomposition.
A recent study of such problems and of their dual, including colorings of hypergraphs induced by halfspaces, halfplanes,  disks, and pseudo-disks, is contained in~\cite{ACCLS08}.\footnote{Using the notation of~\cite{ACCLS08}, the main result of this paper is that .}\medskip

Other definitions of proper colorings of geometric hypergraphs have been studied, such as \emph{conflict-free} colorings~\cite{shakharcf}. 
Here the problem is to find a coloring such that every hyperedge contains at least one vertex with a \emph{unique} color.
Variants of this notion have also been analyzed, e.g., \emph{-fault-tolerant} conflict-free colorings where the conflict-free property must be true even if we were to remove any  vertices in a hyperedge~\cite{faulttolerant}. \emph{-conflict-free} colorings~\cite{jithamilton} require  vertices with unique colors in every hyperedge. 

\paragraph{Applications to sensor networks.}
Consider a planar region monitored by sensors. Each sensor is represented as a point, which is said to monitor every other point contained in a polygonal region around it. Sensors are assumed to have limited lifetime, but can be switched on at any chosen time. Such models of limited-lifetime sensors have been studied in other contexts~\cite{othersensors}. Our results imply that a region can be monitored for  units of time, provided that every point is covered by at least  sensors. This involves partitioning the set of sensors into  subsets, each covering the region. Sensors in the 
-th subset are switched on at time . 

\paragraph{Problem modification.}
We now slightly modify the statement of the problem.  Let  denote a centrally symmetric polygon  centered at point .  Notice that  covers a point  if and only if  contains .
 
The problem involves a set of translates of  that covers every point of the plane  at least  times. 
This is geometrically equivalent to a point set  such that any translate of  in the plane contains at least  points of .    
Note, of course that  must be infinite, as must be  in the original problem.

The decomposition of translates into  covers is equivalent to a coloring of  such that every translate of  in the plane will contain  colors.
We strengthen the problem statement, by relaxing the condition that every translate in the plane contains sufficiently many points.  That is, we say that if a translate contains enough points, it will contain  colors. This allows us to consider finite point sets as well. 
We thus prove the following result.

\begin{theorem}
\label{thm:rephrased}
Given a centrally symmetric convex polygon , there exists a constant  such that for every planar point set  and every , 
 can be -colored so that any translate of  containing at least  points will also contain at least one point of each color.
\end{theorem}

For simplicity of exposition, we assume general position: no two
points in  have the same slope as an edge of . This assumption
can be removed by applying an infinitesimal perturbation to the
points. Also, we assume  is locally finite; 
every compact region contains a finite number of points.

\paragraph{Overview.}
We start by giving a sketch of the complete proof before going into details. The original problem is transformed as follows.

The problem of coloring a (possibly infinite) point set with respect to translates of a polygon (the strengthened statement presented in Theorem~\ref{thm:rephrased}) is  shown to be equivalent to coloring a finite point set  with respect to a finite set of wedges determined by  (see Section~\ref{sec:prel}). 
In other words,  the problem is now to color a set of points such that every wedge containing a sufficient number of  points  will also contain  colors.  Our goal is to show that .
 
We will restrict to color points inside certain \emph{witness} wedges, which have the property that any wedge containing at least  points will contain a witness. Witnesses will contain at least  points. This is why we will define the \emph{level curve} which bounds the union of such minimal wedges for a fixed pair of bounding directions  (see Section~\ref{sec:prel}).


If the level curves did not intersect, coloring the points would be straightforward. It is the intersections of these curves that make the problem non-trivial, and forbids us to restrict to witnesses on level curves only.  Since  is small with respect to the point set, intuitively one can imagine that level curves tend not to venture too ``deep'' into a point set.  In other words, a typical wedge will not reach far into the set before collecting  points.
In Section~\ref{sec:truncation} we define a polygonal region that is deep enough so that the complexity of level curve intersections  within the region is manageable.  Our construction of this region will be such that we will be able to restrict to considering witness wedges within.

To enable us to reduce our problem to circular arc coloring, in Section~\ref{sec:wedges} we define a parametrization which maps the set of witness wedges to the boundary of a circle.  This is directly tied to a mapping of points in  to circular arcs, i.e., intervals on the boundary of the circle (Section~\ref{sec:inter}). Our mapping is such that a position  on the circle will belong to an interval corresponding to point  if and only if the witness wedge represented by  contains .  As every witness wedge contains at least  points, every position on the circle belongs to at least  intervals. The key property of the parametrization is that every point in  is mapped to at most two intervals. 

Thus, the problem is reduced to -coloring arcs on an -covered circle (with certain geometric constraints for the arcs), so that every position on the circle is covered by at least one interval of each color. In Section~\ref{sec:coloring} we give an algorithm for  this circular arc coloring problem. 

 Note that the reductions and the transformations of the problem are constructive; thus our algorithm to color circular arcs yields a simple polynomial algorithm for the original problem. 

\section{Reduction to Wedges}\label{sec:prel}

Let  be a closed, convex,  centrally symmetric
-gon, with vertices  in counterclockwise
order. Throughout the paper, indices are taken modulo . 
The set of indices between  and  in counterclockwise order is denoted by .\medskip


We first  reduce the problem to coloring a finite set of points with respect to {\em wedges} instead of coloring a possibly infinite set with respect to polygons. 
This idea is also used in~\cite{PT07,TT07}.


We consider a tiling of the plane, with squares of side
, where  is half of the smallest distance
between non-consecutive edges of .
Let  be a translate of .
By construction, any intersection of  with a  square is a
wedge
with boundary directions parallel to two consecutive edges of 
(see Figure~\ref{fig:reduc}).
A wedge bounded by rays parallel to  and
 will be called  {\em type} , or alternatively an {\em -wedge}. 
The closed -wedge with apex  is denoted by . 

\begin{figure}[htb]
\begin{center}
\includegraphics[scale=.7]{reduction.pdf}
\end{center}
\caption{\label{fig:reduc} Reduction of the problem with centrally symmetric polygons to wedges in a square.}
\end{figure}

The number of squares that  intersects is bounded by
a constant  that only depends on . Therefore if  contains at least  points,  by the
pigeonhole principle  contains at least  points within one square. 


We will restrict to considering a single square and
the  wedges defined by . Hence the problem reduces to coloring (independently) the finite bounded point set  in each square, i.e., we will seek a -coloring of each square such that any -wedge
containing at least  points will contain all  colors.\medskip

We now define the notion of \emph{level curves} for wedges. This notion extends the 
definition of {\em boundary points} in~\cite{PT07} and~\cite{Pach86}, which are the points found on the first level.
We associate a curve with each -wedge. 
Let  be the set of apices of all -wedges containing  points. Formally,

\noindent where  is the closure operator. We define  as the boundary of .   Accordingly, the closed region that includes the complement of   will be denoted  (i.e. the intersection of the two regions is ).\medskip

Note that  is a monotone staircase polygonal path, with edge directions
parallel to those of its corresponding -wedge. Since  is in general
position,  for any  that is not a vertex of ,
 contains exactly  points. More precisely, we have the following.
\begin{observation}
\label{obs:rorrp1}
For all ,  contains either  or  points of .
\end{observation}
\noindent The curves  for a square are illustrated in Figure~\ref{fig:curves}. A key property of  is the following.
\begin{observation}
Any -wedge containing at least  points of  contains an -wedge whose apex belongs to .
\end{observation}
We conclude that it is sufficient to color points in the union of 
all  (in other words, in the union of regions to the ``left'' of each ).   Handling the complexity of the intersections of these curves is the next problem that we deal with.



\begin{figure}[htb]
\begin{center}
\includegraphics[angle=-90, scale=.5]{curves.pdf}
\end{center}
\caption{\label{fig:curves} The curves of , when  is an axis-parallel square.}
\end{figure}

\section{Restriction to High-Depth Region}
\label{sec:truncation}
We will show that in order to determine the witness wedges that we must  color, it is not necessary to consider complete level curves. At the expense of a constant factor to , we restrict to the portion of the level curves inside a polygon . Inside this polygon, only few intersections between level curves can occur, which simplifies the coloring task.\medskip 

Let  be the oriented line with
direction  going through a point of~
and such that the closed halfplane to its left
contains exactly  points.
Let  be the closed halfplane to the right of .
Denote by   the intersection of the  halfplanes defined by :

We assume : this will be shown true later for the
values of  that we will use (by the well-known center point theorem, it is true as long as ). Note that not all lines 
appear on the boundary of  (see Figure~\ref{fig:vis}).

\begin{lemma}
\label{lem:vis}
For all  there is a vertex  of  such that
 for all .
\end{lemma}
\begin{proof}
Let  be the oriented line parallel to  that is tangent to~ and such that  is contained in the closed halfplane to the {\em right} of .
Then for 
 
the wedge~ contains~. Therefore,  for all .
Note  that a vertex of  may have multiple labels  (see Figure~\ref{fig:vis}).
\end{proof}

\begin{lemma}\label{lem:twowedges}
Let  be a point contained in two wedges  and
 that  contain at most  and  points of  respectively, with . 
Then for all , the oriented line
with direction  through  has at most  points
of  strictly to its left.
\end{lemma}
\begin{proof}
It suffices to observe that the halfplane to the left of the line is contained in the union of  wedges  and .
(See Figure~\ref{fig:twowedges}.)
\end{proof}

\begin{figure}[htb]
\begin{center}
\subfigure[\label{fig:vis}Definition of the points .]{\includegraphics[scale=1]{vis.pdf}}\hskip 1cm\subfigure[\label{fig:twowedges}Two wedges containing the same point .]{\includegraphics[scale=.5]{twowedges.pdf}}
\end{center}
\caption{Construction of , and illustration of Lemma~\ref{lem:twowedges}.}
\end{figure}

We now show that if two level curves have an intersection in , then they must have antipodal indices, that is,  and . 
We actually prove the stronger statement that the regions  do not have any intersection
in , unless they have antipodal indices.
Note that by Lemma~\ref{lem:vis}, 
.

\begin{lemma}
\label{lem:antipodal}
If  and , then .
\end{lemma}
\begin{proof}
Assume by symmetry that  and suppose the two regions intersect
at point . Consider the two wedges  and . Since  is contained 
in , they both contain at most  points. 
By Lemma~\ref{lem:twowedges}, for all , the oriented line
with direction  through  has at most  points of  strictly to its left.
This contradicts the fact that .
\end{proof}

We proceed to show that in fact only one pair of level curves can intersect inside  (a related statement was proved by Pach~\cite{Pach86}). 
This is illustrated in Figure~\ref{fig:cprimes}.\bigskip

\begin{figure}[htb]
\begin{center}
\subfigure[A case where the regions  (in dark) do not intersect.]{\includegraphics[angle=-90,scale=.4]{cprimes.pdf}}
\hspace{2cm}
\subfigure[Regions  and  may intersect in .]{\includegraphics[angle=-90,scale=.4]{cprimes2.pdf}}
\end{center}
\caption{\label{fig:cprimes}Illustration of Lemmas~\ref{lem:antipodal} and~\ref{lem:oneinter}.}
\end{figure}

\begin{lemma}
\label{lem:oneinter}
At most one pair of regions  intersect in .
\end{lemma}
\begin{proof}
By contradiction, suppose that  and 
, with .

First, let us suppose that , and focus on the case . Trivially, . Thus Lemma~\ref{lem:twowedges} implies that for all , the oriented line with direction  through~ has at most~ points of~ strictly to its left, contradicting . The case  works analogously.

On the other hand, if , then we claim that  and a similar argument leads to a contradiction. 
In order to prove the claim, consider the rays from  parallel to  for . Then,  implies that  lies (counterclockwise) between either the pair of rays parallel to  and to , or the pair parallel to  and to . Given , we have  in the first case, and  in the second case (see Figure~\ref{fig:pflemoneinter}).
\end{proof}

\begin{figure}[htb]
\begin{center}
\includegraphics[scale=.4, angle=-90]{pflemoneinter.pdf}
\end{center}
\caption{\label{fig:pflemoneinter}Proof of Lemma~\ref{lem:oneinter}.}
\end{figure}

\begin{lemma}\label{lem:C_and_T}
If   intersects the interior of
, then it intersects the boundary of  at exactly two distinct lines.
We denote the lines by  and , so that ,  and  appear in counterclockwise order on the boundary of .
\end{lemma}
\begin{proof}

Take any point  on . We have  and , and each of the common
supporting lines of those two wedges  properly intersects  only once. 
So each of the two wedges complementary to the union of  and  contains at least one
intersection of  with the boundary of . 
This implies that   and .
Note that since  the property is valid for all , intersections occur only on the lines  and .
\end{proof}

\noindent Lemma~\ref{lem:C_and_T} implies that every  intersecting  is such that .
Let  be the portion of  contained in :


\begin{lemma}
(i) The curve  is connected. 
(ii) If , then  is empty for . 
(iii) If , then  is empty for .
\end{lemma}

\begin{proof}
Statement (i) follows directly from the fact that 
is an unbounded curve and intersects  at most twice
(Lemma~\ref{lem:C_and_T}). 
Statements (ii) and (iii) follow from Lemma~\ref{lem:C_and_T} and Lemma~\ref{lem:antipodal}.
\end{proof}

\begin{observation}
\label{obs:witnessvi}
If  is empty, then any -wedge  for 
contains at least  points of . In particular,  and .
\end{observation}

The combinatorial properties described in this section lay the foundations for the definition of a set of witness wedges in Section~\ref{sec:wedges}.

\section{Witness Wedges}
\label{sec:wedges}

We now describe a set of wedges, parameterized by a real number
 with apex at point  and  
. We abbreviate .
This set of wedges is such that any -wedge containing at
least  points contains a witness wedge .
Thus it suffices to color only those witness wedges.

The wedge  will have its apex on  for  if  is not empty.
More precisely, we let ,  be a parametrization of , 
where  and 
 for .
If  is empty, then we distinguish three cases (see Figure~\ref{fig:param}):
\begin{itemize}
\item [A.] If there is a  such that  then 
   for .
\item [B.] If there is a  such that  then 
   for .
\item [C.] Otherwise,  for .
\end{itemize}
We define  as the concatenation of the functions :


\begin{figure}[htb]
\begin{center}
\includegraphics[scale=.7]{param.pdf}
\end{center}
\caption{\label{fig:param}Definition of , when  is empty.}
\end{figure}

\begin{lemma}
\label{lem:witnesses}
For any wedge  that contains at least  points of ,
there is a value  such that 
and  contains at least  points.
\end{lemma}
\begin{proof}
Since  contains at least  points, it must intersect . Thus it contains a wedge
 such that . 

First suppose that  is not empty. Then , otherwise
 cannot contain enough points. 
If , then 
is contained in  and contains at least  points. 

Now suppose  is empty and refer to the three cases above. 
In case C, from Observation~\ref{obs:witnessvi},  and
 contains at least  points. Since in that case  for , any value of  in  will work. 
In case A, note that  wedges  and  have the same apex 
,  contains at most  points,
 has  points on its left, and  is in the
union of  and the halfplane left of .
This implies that both  and the halfplane to the left of the oriented
line of direction  through its apex have at most 
points. Thus  has its apex to the right
of that line, which implies . 
Because  is empty, .
Case B is identical.
\end{proof}

It is natural to view the range  as a counterclockwise
parametrization of the points on a unit circle. Thus in what follows,
the real parameter  will be viewed modulo , and an interval
 is the set of points on the circle on a counterclockwise walk
from  to . 

\section{Reduction to Intervals}
\label{sec:inter}

Our goal is to color the points of  with  colors such that any witness wedge  contains at least one point of each color.
For each point  in , we consider the set  of witness wedges containing :


\begin{lemma}
\label{lem:oneinterval}
For any point , if , where  appears
before  (that is,   and 
), then  for all .
\end{lemma}
\begin{proof}
There are several cases to consider. If , then either
, or  lies on  between 
and . Since  is  monotone in all directions between
 and , the wedge  contains the
intersection of  and .


In the second case, . 
Then by Lemma~\ref{lem:antipodal}, . Also, because
 and  are in  and by the same lemma, we know
 and . This implies that the
counterclockwise bounding ray  of  intersects the clockwise
bounding ray  of  and  is in the closed wedge  right
of  and left of  (see Figure~\ref{fig:lemma8}). Then for any point  in the closed wedge
 opposite to , the wedges  for
.  

\begin{figure}[htb]
\begin{center}
\includegraphics{lemma8.pdf}
\end{center}
\caption{\label{fig:lemma8}Second case of Lemma~\ref{lem:oneinterval}.}
\end{figure}

In the remainder of this proof, we will show that  for
all  where  is an integer or an integer plus
. Then the lemma follows by applying the first case for every
other value of .
In fact, it will suffice to show that 
and  (or symmetrically that 
 and ) and
apply the lemma again on the subrange  (or ).

If  lies on , then so does 
. 
So  contains at most  points. This
implies, by the same argument as above, that the ray  intersects
the clockwise bounding ray of  and that ray
 intersects the counterclockwise bounding ray of 
. Therefore, 
 
(and so ), which implies 
 and
. 
The case where  lies on  is covered symmetrically.

If  is empty then . 
Furthermore, if  is defined according to case A,
 thus 
.
If  is defined according to case B, then either 
 and we are done, or  which
we treat below. 

Finally, we are left with the case where  is defined according
to case C or it is defined according to case B and . By
symmetry, we also assume that  is either defined according to
case C or it is defined according to case A and .
Note that in this case, the entire portion of the boundary of 
between  and  is inside . This implies again that 
.\end{proof}

As a consequence, a point defines either an interval, or a pair of intervals, the corresponding wedges of which are of two types  and .

\begin{corollary}
\label{cor:interval}
 is either an interval, or a pair of intervals , such that  for  and  for , where  is such that  and  intersect in .
\end{corollary}
\begin{proof}
From Lemma~\ref{lem:oneinterval},  cannot consist of more than two intervals, since otherwise we can find two points  and  satisfying the conditions of Lemma~\ref{lem:oneinterval} in two distinct intervals.

Now first suppose  that no pair  intersect in . Then again the statement is a direct consequence of Lemma~\ref{lem:oneinterval}. Otherwise suppose that . Then we must show that  and . For contradiction, let  be contained strictly in the interior of . Then there are again two points  and  satisfying the conditions of Lemma~\ref{lem:oneinterval}, a contradiction.
\end{proof}

\section{Coloring}
\label{sec:coloring}

We give an algorithm for coloring the points with  colors so that all wedges  contain all  colors. In the following, we say that a point  {\em covers} a point  whenever . We proceed by iteratively removing a covering of , that is, a subset of , the elements of which collectively cover the circle . We use a greedy algorithm to select such a subset; we iteratively expand the cover for , by selecting a new point that covers the largest interval starting from . Every point in a cover is assigned the same color. By repeating this  times, we ensure that all  colors are represented in each of the wedges , and thus by Lemma~\ref{lem:witnesses}, in all wedges containing at least  points. The key property of the algorithm is that it only requires .

A formal description of the algorithm follows. We suppose, without loss of generality, that only the pair  may intersect in .\medskip

\noindent \begin{minipage}{\textwidth}
\noindent {\bf Coloring Algorithm}\\

\noindent for  to  do:
\begin{enumerate}
  \item , 
  \item while  do:
  \begin{enumerate}
     \item find  such that  is maximized
     \item 
     \item 
  \end{enumerate}
  \item assign color  to all points in 
  \item 
\end{enumerate}
~
\end{minipage}

When every set  is a simple interval, this algorithm greedily colors circular arcs. The following lemma states that in that case, no point on a circle is covered more than a constant number of times per iteration (see Figure~\ref{fig:arcs}).

\begin{figure}[htb]
\begin{center}
\includegraphics[scale=.4, angle=-90]{arcs.pdf}
\end{center}
\caption{\label{fig:arcs}Covering the circle  by circular arcs.}
\end{figure}

\begin{lemma}
\label{lem:coloring}
Suppose that no pair  intersect in , and that there are enough points to perform  iterations of the coloring algorithm. Let  be the set of points colored by the algorithm after the iteration . Then every point of  is covered at most  times by points of .
\end{lemma}
\begin{proof}
It is sufficient to prove that no point of  is covered more than three times by points of .
Note that if no pair of curves intersect, then from Corollary~\ref{cor:interval}, every set  is an interval. Hence  is a greedy covering of the circle by intervals (i.e., circular arcs).

Let  be the last interval chosen by the algorithm, and consider . Suppose that a point  is covered by more than two points of . Let  and  be the first and the last points chosen, respectively, that cover . The remaining intervals that cover  either extend further than  and should have been chosen instead of , or do not extend further than , in which case  should have been chosen instead. In both cases, we have a contradiction. Hence the points of  do not cover any point of  more than twice. The last interval  can cover some points of the circle a third time. Therefore, every point of  is covered at most three times by points of .
\end{proof}

In the general case, a point  might correspond to two intervals on opposite regions of the circle . We show that the following similar property holds.
\begin{lemma}
\label{lem:gen-coloring}
Suppose that there are enough points to perform  iterations of the above algorithm, and let  be the set of points colored by the algorithm after the iteration . Then every point of  is covered at most  times by points of .
\end{lemma}
\begin{proof}
We consider that  and  intersect in . Otherwise, the statement is implied by Lemma~\ref{lem:coloring}. By Lemma~\ref{lem:oneinter}, only one such pair can intersect. Without loss of generality, we also assume that  and  are both orthogonal staircases going from top left to bottom right. This setting can always be enforced by symmetry and affine transformation of the points. We assume that  is, at some point, above , which might cause a point between the two curves to generate one interval on each (see Figure~\ref{fig:twointervals}).

We will prove our statement by induction on the number of iterations. Let us show that after the iteration , no point of  is covered more than  times. The induction hypothesis is that this is true for the iterations 0 to , where iteration  corresponds to the initial situation. The base case  is trivial.

Consider a point , and the corresponding point  on  . Suppose that this point was covered  times in the previous iterations (thus by points of colors 1 to ). By the induction hypothesis, .
We consider the set of points  selected by the algorithm at the iteration . The sets  start by covering the wedges of type 0, corresponding to points on . Let  be the first point of , in order of selection, that also covers . By Corollary~\ref{cor:interval},  only covers two types of wedges, 0 and . Let  be the horizontal projection of  on .

Let  be the next point selected by the greedy algorithm. If it covers the point , then from Corollary~\ref{cor:interval}, it cannot cover any point on . Otherwise, since the algorithm is greedy, the point is associated with an interval that intersects  and that has the farthest right endpoint. Geometrically,  is the lowest point to the left of the vertical line  through . Let  be the projection of  onto .
   
Two cases can occur. First, if  is below , then  is covered at most once, by .
On the other hand, if  is above , then  covers . 

By the  induction hypothesis,   contains at most  colored points. By Observation~\ref{obs:rorrp1} and since ,   contains at least  points. Hence  contains at least  uncolored points (including  and ). Also, since the algorithm is greedy,  and  do not contain  uncolored points, otherwise they would have been selected by the algorithm. Hence the orthogonal rectangle  with opposite vertices  and   contains at least  uncolored points (see Figure~\ref{fig:proof}).

\begin{figure}[htb]
\begin{center}
\subfigure[\label{fig:twointervals}The point  is associated with two intervals , and .]{\includegraphics[angle=-90, scale=.5]{twointervals.pdf}}
\hspace{1.5cm}
\subfigure[\label{fig:proof}Definition of the region .]{\includegraphics[angle=-90, scale=.5]{interference.pdf}}
\end{center}
\caption{Illustration of the proof of Lemma~\ref{lem:gen-coloring}.}
\end{figure}

Since  is covered  times and ,   can contain at most  uncolored points.  is included in , thus from the previous observation on , there are at most  uncolored points that are both to the right of  and above . These, together with  and , are the only points that may cover  after we have covered the interval . Hence after we have covered the interval , the points in  cannot be covered more than  times. On the other hand, the points in  cannot be covered more than  times.

A similar reasoning holds when the algorithm starts to cover points in the interval . We have to replace  by , since the points on both sides can already be covered  times. Thus after the  iteration , no point is covered more than  times, which concludes the proof.
\end{proof}



\begin{corollary}
For , the coloring algorithm finds a -coloring of the points in  such that all wedges  for  contain all  colors.
\end{corollary}

Note that with this choice of , by the well-known center point theorem,   is never empty. By Lemma~\ref{lem:witnesses}, this concludes the proof of Theorem~\ref{thm:rephrased}, with , hence for any . By duality, as the polygon  is symmetric, this implies the following.

\begin{theorem}
\label{thm:main}
Given a centrally symmetric convex polygon , there exists a constant  such that every -fold covering of the plane by translates of  can be decomposed into  coverings.
\end{theorem}

\bibliography{coverings}
\bibliographystyle{plain}

\end{document}
