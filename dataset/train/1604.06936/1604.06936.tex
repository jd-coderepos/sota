\documentclass{amsart}
\usepackage{amssymb,amsmath,amsfonts}
\usepackage{amsaddr}
\usepackage{geometry} 
\usepackage{multicol}
\usepackage{graphicx}
\usepackage{enumerate,enumitem}
\usepackage{verbatim,setspace}
\usepackage{pdfsync}
\usepackage{gastex}
\usepackage[lowtilde]{url}
\usepackage{xcolor}

\newtheorem{theorem}{Theorem}
\newtheorem{corollary}[theorem]{Corollary}
\newtheorem{lemma}[theorem]{Lemma}
\newtheorem{proposition}[theorem]{Proposition}
\newtheorem{conjecture}[theorem]{Conjecture}
\newtheorem{remarkthm}[theorem]{Remark}
\newtheorem{definition}[theorem]{Definition}
\newenvironment{remark}{\begin{remarkthm}\normalfont\quad}{\end{remarkthm}}
\renewcommand{\le}{\leqslant}
\renewcommand{\ge}{\geqslant}
\newcommand{\cmp}{\overline}
\newcommand{\ol}{\overline}
\newcommand{\eps}{\varepsilon}
\newcommand{\emp}{\emptyset}
\newcommand{\rhoR}{R}
\newcommand{\Sig}{\Sigma}
\newcommand{\sig}{\sigma}
\newcommand{\noin}{\noindent}
\newcommand{\Bbf}{\mathbf{B}_{\mathrm{bf}}}
\newcommand{\Vbf}{\mathbf{W}^{\le 5}_{\mathrm{bf}}}
\newcommand{\Wbf}{\mathbf{W}^{\ge 6}_{\mathrm{bf}}}
\newcommand{\e}[1]{\hat{#1}}
\newcommand{\etc}{\mbox{\it etc.}}
\newcommand{\ie}{\mbox{\it i.e.}}
\newcommand{\eg}{\mbox{\it e.g.}}
\newcommand{\FigureDirectory}{FIGS}
\newcommand{\inv}[1]{\mbox{}}
\newcommand{\stress}[1]{{\fontfamily{cmtt}\selectfont #1}}
\newcommand{\tid}{\mbox{{\bf 1}}}
\newcommand{\cA}{{\mathcal A}}
\newcommand{\cB}{{\mathcal B}}
\newcommand{\cC}{{\mathcal C}}
\newcommand{\cD}{{\mathcal D}}
\newcommand{\cG}{{\mathcal G}}
\newcommand{\cL}{{\mathcal L}}
\newcommand{\cI}{{\mathcal I}}
\newcommand{\cM}{{\mathcal M}}
\newcommand{\cN}{{\mathcal N}}
\newcommand{\cO}{{\mathcal O}}
\newcommand{\cP}{{\mathcal P}}
\newcommand{\cR}{{\mathcal R}}
\newcommand{\cS}{{\mathcal S}}
\newcommand{\cT}{{\mathcal T}}
\newcommand{\cW}{{\mathcal W}}
\newcommand{\gL}{{\mathcal L}}
\newcommand{\gR}{{\mathcal R}}
\newcommand{\gJ}{{\mathcal J}}
\newcommand{\gH}{{\mathcal H}}
\newcommand{\gD}{{\mathcal D}}
\newcommand{\Dm}{{\mathcal D}^{\mathrm{M}}}
\newcommand{\Dpm}{{\mathcal D}^{\mathrm{PM}}}
\newcommand{\Dnm}{{\mathcal D}^{\mathrm{NM}}}
\newcommand{\one}{{\mathbf 1}}
\newcommand{\Lra}{{\hspace{.1cm}\Leftrightarrow\hspace{.1cm}}}
\newcommand{\lra}{{\hspace{.1cm}\leftrightarrow\hspace{.1cm}}}
\newcommand{\la}{{\hspace{.1cm}\leftarrow\hspace{.1cm}}}
\newcommand{\ra}{{\rightarrow}}
\newcommand{\raL}{{\mathbin{\sim_L}}}
\newcommand{\lraL}{{\mathbin{\approx_L}}}
\newcommand{\tr}{{transformation}}
\newcommand{\timg}{\mbox{rng}}
\newcommand{\tdom}{\mbox{dom}}
\newcommand{\floor}[1]{\lfloor #1 \rfloor}
\newcommand{\qedb}{\hfill}

\begin{document}
\title{Syntactic complexity of bifix-free regular languages}
\author{Marek Szyku{\l}a}
\address{Institute of Computer Science, University of Wroc{\l}aw,\\
Joliot-Curie 15, PL-50-383 Wroc{\l}aw, Poland}
\email{msz@cs.uni.wroc.pl}
\author{John Wittnebel}
\address{David R. Cheriton School of Computer Science, University of Waterloo,\\
Waterloo, ON, Canada N2L 3G1}
\email{jkwittnebel@hotmail.com}

\begin{abstract}
We study the properties of syntactic monoids of bifix-free regular languages.
In particular, we solve an~open problem concerning syntactic complexity:
We prove that the cardinality of the syntactic semigroup of a~bifix-free language with state complexity  is at most  for .
The main proof uses a~large construction with the method of injective function.
Since this bound is known to be reachable, and the values for  are known, this completely settles the problem.
We also prove that  is the minimal size of the alphabet required to meet the bound for .
Finally, we show that the largest transition semigroups of minimal DFAs which recognize bifix-free languages are unique up to renaming the states.

\medskip\noin
{\textsc{Keywords:}} bifix-free, prefix-free, regular language, suffix-free, syntactic complexity, transition semigroup
\end{abstract}

\maketitle
\section{Introduction}

The \emph{syntactic complexity}~\cite{BrYe11}  of a~regular language  is defined as the size of its syntactic semigroup~\cite{Pin97}.
It is known that this semigroup is isomorphic to the transition semigroup of the quotient automaton  and of a~minimal deterministic finite automaton accepting the language.
The number  of states of  is the \emph{state complexity} of the language~\cite{Yu01}, and it is the same as the \emph{quotient complexity}~\cite{Brz10} (number of left quotients) of the language.
The \emph{syntactic complexity of a~class} of regular languages is the maximal syntactic complexity of languages in that class expressed as a~function of the quotient complexity~.

Syntactic complexity is related to the Myhill equivalence relation \cite{Myh57}, and it counts the number of classes of non-empty words in a~regular language which act distinctly.
It provides a~natural bound on the time and space complexity of algorithms working on the transition semigroup.
For example, a~simple algorithm checking whether a~language is \emph{star-free} just enumerates all transformations and verifies whether none of them contains a~non-trivial cycle \cite{McSe71}.

Syntactic complexity does not refine state complexity, but used as an~additional measure it can distinguish particular subclasses of regular languages from
the class of all regular languages, whereas state complexity alone cannot.
For example, the state complexity of basic operations in the class of star-free languages is the same as in the class of all regular languages (except the reversal, where the tight upper bound is  see \cite{BrSz15Aperiodic}).

Finally, the largest transition semigroups play an~important role in the study of \emph{most complex} languages~\cite{Brz13} in a~given subclass.
These are languages that meet all the upper bounds on the state complexities of Boolean operations, product, star, and reversal, and also have maximal syntactic semigroups and most complex atoms~\cite{BrTa14}.
In particular, the results from this paper enabled the study of most complex bifix-free languages \cite{FeSz17ComplexityOfBifixFree}.

A~language is \emph{prefix-free} if no word in the language is a~proper prefix of another word in the language.
Similarly, a~language is \emph{suffix-free} if there is no word that is a~proper suffix of another word in the language.
A~language is \emph{bifix-free} if it is both prefix-free and suffix-free.
Prefix-, suffix-, and bifix-free languages are important classes of codes, which have numerous applications in such fields as cryptography and data compression.
Codes have been studied extensively; see~\cite{BPR09} for example.

Syntactic complexity has been studied for a~number of subclasses of regular languages (e.g.,~\cite{BrLi15,BLL12,BLY12,BrSz15Aperiodic,HoKo04,IvNa14}).
For bifix-free languages, the lower bound  for the syntactic complexity for  was established in~\cite{BLY12}. The values for  were also determined.

The problem of establishing tight upper bound on syntactic complexity can be quite challenging, depending on the particular subclass.
For example, it is easy for prefix-free languages and right ideals, while much more difficult for suffix-free languages and left ideals.
The case of bifix-free languages studied in this paper requires an~even more involved proof, as the structure of a~maximal transition semigroup is more complicated.

Our main contributions in this paper are as follows:
\begin{enumerate}
\item We prove that  is also an~upper bound for syntactic complexity for .
To do this, we apply the general method of injective function (cf. \cite{BrSz14a} and~\cite{BrSz15SyntacticComplexityOfSuffixFree}).
The construction here is much more involved than in the previous cases and uses a~number of tricks for ensuring injectivity.
\item We prove that the transition semigroup meeting this bound is unique for every .
\item We refine the witness DFA meeting the bound by reducing the size of the alphabet to , and we show that it cannot be any smaller.
\item Using a~dedicated algorithm, we verify by computation that two semigroups  and  (defined below) are the unique largest transition semigroups of a~minimal DFA of a~bifix-free language, respectively for  and  (whereas they coincide for ).
\end{enumerate}
In summary, for every  we have determined the syntactic complexity, the unique largest semigroups, and the minimal sizes of the alphabets required; this completely solves the problem for bifix-free languages.

These results have been announced in~\cite{SzWi17SyntacticComplexityOfBifixFree} with only proof ideas.


\section{Preliminaries}

Let  be a~non-empty finite alphabet, and let  be a~language.
If  is a~word,  denotes the \emph{left quotient} or simply quotient of  by , which is defined by .
The number of quotients of  is its \emph{quotient complexity}~\cite{Brz10} . 
From the Myhill-Nerode Theorem, a~language is regular if and only if the set of all quotients of the language is finite.
We denote the set of quotients of regular  by , where  by convention.

A~\emph{deterministic finite automaton (DFA)} is a~tuple , where  is a~finite non-empty set of \emph{states},  is a~finite non-empty \emph{alphabet},  is the \emph{transition function},  is the \emph{initial} state, and  is the set of \emph{final} states.
We extend  to a~function  as usual.

The \emph{quotient DFA} of a~regular language  with  quotients is defined by
, where  if and only if , and .
Without loss of generality, we assume that .
Then , where  if , and  is the set of subscripts of quotients in .
A~state  is \emph{empty} if its quotient  is empty.
The quotient DFA of  is isomorphic to each complete minimal DFA of .
The number of states in the quotient DFA of  (the quotient complexity of ) is therefore equal to the state complexity of .

In any DFA , each letter  induces a~transformation on the set  of  states.
We let  denote the set of all  transformations of ; then  is a~monoid under composition. 
The \emph{image} of  under transformation  is denoted by , and the \emph{image} of a~subset  is .
If  are transformations, their composition is denoted by  and defined by .
The identity transformation is denoted by , and we have  for all .
By , where  and , we denote a~\emph{semiconstant} transformation that maps all the
states from  to  and behaves as the identity function for the states in .
A~\emph{constant} transformation is the semiconstant transformation , where .
A~\emph{unitary} transformation is , for some distinct ; this is denoted by  for simplicity.

The \emph{transition semigroup} of  is the semigroup of all transformations generated by the transformations induced by .
Since the transition semigroup of a~minimal DFA of a~language  is isomorphic to the syntactic semigroup of ~\cite{Pin97}, the syntactic complexity of  is equal to the cardinality of the transition semigroup of .

The \emph{underlying digraph} of a~transformation  is the digraph , where .
We identify a~transformation with its underlying digraph and use usual graph terminology for transformations:
The \emph{in-degree} of a~state  is the cardinality .
A~\emph{cycle} in  is a~cycle in its underlying digraph of length at least 2.
A~\emph{fixed point} in  is a~self-loop in its underlying digraph.
The \emph{orbit} of a~state  in  is a~connected component containing  in its underlying digraph, that is, the set .
Note that every orbit contains either exactly one cycle or one fixed point.
The \emph{distance} in  from a~state  to a~state  is the length of the path in the underlying digraph of  from  to , that is, , and is undefined if no such path exists.
If a~state  does not lie in a~cycle, then the \emph{tree} of  is the underlying digraph of  restricted to the states  such that there is a~path from  to .

\subsection{Bifix-free languages and semigroups}

Let , where , be a~minimal DFA accepting a~bifix-free language , and let  be its transition semigroup. We also define  (the set of the ``middle'' non-special states).

The following properties of bifix-free languages, slightly adapted to our terminology, are well known~\cite{BLY12}:
\begin{lemma}\label{lem:bifix-free}
A~minimal DFA  of a~bifix-free languages  satisfies the following properties:
\begin{enumerate}
\item There is an~empty state, which is  by convention.
\item There exists exactly one final quotient, which is , and whose state is  by convention, so .
\item For , if , then .
\item In the underlying digraph of every transformation of , there is a~path starting at  and ending at .
\end{enumerate}
\end{lemma}
Items~(1) and~(2) are sufficient and necessary conditions for a~prefix-free language, items~(1) and~(3) are sufficient and necessary conditions for a~suffix-free, and item~(4) follows from item~(3).
Following \cite{BrSz15SyntacticComplexityOfSuffixFree}, we say that an (unordered) pair  of distinct states in  is \emph{colliding} (or  \emph{collides} with ) in  if there is a~transformation  such that  and  for some .
A~pair of states is \emph{focused by} a~transformation  if  maps both states of the pair to a~single state .
We then say that  is \emph{focused to the state }.
By Lemma~\ref{lem:bifix-free}(3), it follows that if  is colliding in , then there is no transformation  that focuses .
Hence, in the case of bifix-free languages, colliding states can be mapped to a~single state only if this state is .
In contrast to suffix-free languages, we do not consider the pairs from  being colliding, as they cannot be focused.

For  we define the set of transformations

In~\cite{BLY12} it was shown that the transition semigroup  of a~minimal DFA of a~bifix-free
language must be contained in .
It contains all transformations  which fix , map  to , and do not focus any
pair which is colliding from .
The condition of fixing the empty state  is obvious. State  must be always mapped to  because the language must be prefix-free, and focusing state  with any other state is forbidden because the language must be suffix-free.
For ,  is not a~semigroup, because compositions of its transformations may violate the condition about focusing state .
For example, the transformations  and  are in , but their composition  focuses the pair , thus is not in .

Since  is not a~semigroup, no transition semigroup of a~minimal DFA of a~bifix-free language can contain all transformations from .
Therefore, its cardinality is not a~tight upper bound on the syntactic complexity of bifix-free languages.
A~lower bound on the syntactic complexity was established in~\cite{BLY12}.
We study the following two semigroups that play an~important role for bifix-free languages.

\subsubsection{Semigroup .}

For  we define the semigroup:

The name of this semigroup follows from \cite{BLY12}, and the superscript denotes that it is, as we will show, the largest syntactic semigroup of a~bifix-free language when .

The following remark summarizes the transformations of  (illustrated in Fig.~\ref{fig:Wbf_transformations}):
\begin{remark}\label{rem:Wbf_transformations}
 contains all transformations that:
\begin{enumerate}[align=left,leftmargin=*]
\item[\bf(Type~1)] map  to , and  into , or
\item[\bf(Type~2)] map  to ,  to , and  into , or
\item[\bf(Type~3)] map  to a~state , and  into .\qedb
\end{enumerate}
\end{remark}
They are the only transformations in , and we will be referring to these three types of transformations.
\begin{figure}[ht]
\unitlength 8pt\scriptsize
\gasset{Nh=2.5,Nw=2.5,Nmr=1.25,ELdist=0.3,loopdiam=1.5}
\begin{center}\begin{picture}(14,16)(0,-4)
\node[Nframe=n](name)(1,10){(1):}
\node(0)(2,4){0}\imark(0)
\node(1)(6,8){}
\node[Nframe=n,Nw=2,Nh=2](dots)(6,4){}
\node[Nw=3.5,Nh=11.5,Nmr=1.25,dash={.5 .25}{.25}](QM)(6,4){}
\node(n-3)(6,0){-}
\node(n-2)(10,4){-}\rmark(n-2)
\node(n-1)(10,0){-}
\drawedge[curvedepth=-6,sxo=-.2,exo=.2](0,n-1){}
\drawedge(n-2,n-1){}
\drawloop[loopangle=270](n-1){}
\drawloop[loopangle=0,syo=3](QM){}
\drawedge(QM,n-2){}
\drawedge(QM,n-1){}
\end{picture}\begin{picture}(14,14)(0,-4)
\node[Nframe=n](name)(1,10){(2):}
\node(0)(2,4){0}\imark(0)
\node(1)(6,8){}
\node[Nframe=n,Nw=2,Nh=2](dots)(6,4){}
\node[Nw=3.5,Nh=11.5,Nmr=1.25,dash={.5 .25}{.25}](QM)(6,4){}
\node(n-3)(6,0){-}
\node(n-2)(10,4){-}\rmark(n-2)
\node(n-1)(10,0){-}
\drawedge[curvedepth=8,sxo=-.5,exo=.5](0,n-2){}
\drawedge(n-2,n-1){}
\drawloop[loopangle=270](n-1){}
\drawloop[loopangle=0,syo=3](QM){}
\drawedge(QM,n-1){}
\end{picture}\begin{picture}(14,14)(0,-4)
\node[Nframe=n](name)(1,10){(3):}
\node(0)(2,4){0}\imark(0)
\node(1)(6,8){}
\node[Nframe=n,Nw=2,Nh=2](dots)(6,4){}
\node[Nw=3.5,Nh=11.5,Nmr=1.25,dash={.5 .25}{.25}](QM)(6,4){}
\node(n-3)(6,0){-}
\node(n-2)(10,4){-}\rmark(n-2)
\node(n-1)(10,0){-}
\drawedge(0,dots){}
\drawedge(n-2,n-1){}
\drawloop[loopangle=270](n-1){}
\drawedge(QM,n-2){}
\drawedge(QM,n-1){}
\end{picture}\end{center}
\caption{The three types of transformations in  from Remark~\ref{rem:Wbf_transformations}.}\label{fig:Wbf_transformations}
\end{figure}

The cardinality of  is .

\begin{proposition}\label{pro:Wbf_unique}
 is the unique maximal transition semigroup of a~minimal DFA  of a~bifix-free language in which there are no colliding pairs of states.
\end{proposition}
\begin{proof}
Since for any pair  there is the transformation  in the semigroup, the pair  cannot be colliding.
Therefore, there are no colliding pairs in .

Let  be a~transition semigroup in which there are no colliding pairs of states.
Consider .
If  then  as is a~transformation of Type~1.
If  then  as is a~transformation of Type~2.
If , then , as otherwise  would be a~colliding pair, so  is a~transformation of Type~3 from.
Therefore,  is a~subsemigroup of , and so  is unique maximal.
\end{proof}

In~\cite{BLY12} it was shown that for , there exists a~witness DFA of a~bifix-free language whose transition semigroup is  over an~alphabet of size  (and 18 if ).
Now we slightly refine the witness from \cite[Proposition~31]{BLY12} by reducing the size of the alphabet to , and then we show that it cannot be any smaller.
\begin{definition}[Bifix-free witness]
For , let , where  and  contains the following letters:
\begin{enumerate}
\item , for , inducing the transformations ,
\item , for every transformation of Type~(2) that is different from ,
\item , for every transformation of Type~(3) that is different from  for some state .
\end{enumerate}
Altogether, we have .
For  two letters suffice, since the transformation of  is induced by , where  and .
\end{definition}

\begin{proposition}
The transition semigroup of  is .
\end{proposition}
\begin{proof}
Consider a~transformation  of Type~1.
Let  be the states that are mapped to  by .
If , then , where  is the transformation induced by some , and  is the transformation induced by some  that maps  in the same way as .
If , then let  be the state such that .
Let  be the transformation induced by  that maps the states from  to  and  in the same way as .
Then , since , , and for  we know that .
Hence, we have all transformations of Type~1 in .

It remains to show how to generate the two missing transformations of Type~2 and Type~3 that do not have the corresponding generators  and , respectively.
Let , which is induced by a~.
Consider the transformation .
Then , where  is of Type~1.
Consider the transformation .
Then , where  is of Type~1.
\end{proof}

\begin{proposition}\label{pro:Wbf_alphabet_lower_bound}
For , at least  generators are necessary to generate .
\end{proposition}
\begin{proof}
Consider a~transformation  of Type~(2) that is different from .
If  were the composition of two transformations from , then either  maps  to , or  maps  into .
Since neither is the case,  must be a~generator.
There are  such generators.

Consider a~transformation  of Type~(3) that is different from  for some .
Note that to generate  a~transformation of Type~(3) must be used, but the composition of such a~transformation with any other transformation from  maps every state from  to .
Hence,  must be used as a~generator, and there are  such generators.

Consider a~transformation  of Type~(1) of the form  for some .
Note that to generate , transformations of Type~(3) cannot be used because  is not mapped into  if .
Let , where  are generators.
Since a~transformation of Type~(2) does not map  to ,  cannot be of Type~(2), and so must be of Type~(1).
Moreover , as otherwise  would map a~state  to .
Hence, , and for every selection of  there exists a~different .
There are  such generators.
\end{proof}

\subsubsection{Semigroup .}

For  we define the semigroup

The name of this semigroup follows from \cite{BLY12}, and the superscript denotes that it is, as we will show, the largest syntactic semigroup of a~bifix-free language when .
In this semigroup, there are all transformations from  that do not focus any pair.
By taking only such transformations, we are allowed to have all pairs of states possibly colliding.

\begin{proposition}\label{pro:Vbf_unique}
 is the unique maximal transition semigroup of a~minimal DFA  of a~bifix-free language in which all pairs of states from  are colliding.
\end{proposition}
\begin{proof}
Let  be two distinct states.
Then  is colliding because of the transformation .
Therefore, all pairs of states from  are colliding.

Let  be a~transition semigroup with all colliding pairs of states.
Consider a~transformation .
Then for every distinct , we have  or , as otherwise  would be focused.
By definition of , there are all such transformations  in .
Therefore,  is unique maximal.
\end{proof}

In~\cite{BLY12} it was shown that for  there exists a~DFA for a~bifix-free language
whose transition semigroup is  over an~alphabet of size .
We prove that this is an~alphabet of minimal size that generates this transition semigroup.
\begin{proposition}\label{pro:Vbf_alphabet_lower_bound}
To generate  at least  generators must be used.
\end{proposition}
\begin{proof}
First we show that the composition of any two transformations  maps a~state different from  to state .
Suppose that  does not map any state to .
If , then .
If , then some state  must be mapped either to  or to , and again .

Consider all transformations  that map  onto , hence they must be bijections between these states.
There are  such transformations, and since they cannot be generated by compositions, they must be generators.
\end{proof}

\section{Upper bound on the syntactic complexity of bifix-free languages}

Our main result shows that the lower bound  on the syntactic complexity of bifix-free languages is also an~upper bound for .

We consider a~minimal DFA , where  is the only final state, and  is the empty state, recognizing an~arbitrary bifix-free language.
Let  be the transition semigroup of .
We will show that  is not larger than .

Note that the semigroups  and  share the set , and in both of them , , and  play the role of the initial, final, and empty state, respectively.
When we say that a~pair of states from  is \emph{colliding} we always mean that it is colliding in .

First, we state the following lemma, which generalizes some arguments that we use frequently in the proof of the main theorem.
\begin{lemma}\label{lem:orbits}
Let  and  be transformations.
Suppose that:
\begin{enumerate}
\item All states from  whose mapping is different in  and  belong to , where  is either an~orbit in  or is the tree of a~state in .
\item All states from  whose mapping is different in  and  belong to , where  is either an~orbit in  or is the tree of a~state in .
\item The transformation , for some , focuses a~colliding pair whose states are in .
\end{enumerate}
Then either  or .
In particular, if  and  are both orbits or both trees rooted in a~state mapped by  to , then .
\end{lemma}
\begin{proof}
First observe that if  then , since by~(1) state  is mapped in the same way by  as by .
Also, , since if  would be in , then , because  is an~orbit or a~tree and  is reachable from .
Hence, for any , where  or , by a~simple induction we obtain .
The same claim holds symmetrically for .

Let  be the colliding pair that is focused by  from~(3).
Suppose that .
Since , we know that .
By the claim above for , , and .
But this means that  focuses , hence  and  cannot be both present in .

So it must be that , since they are orbits or trees we have either  or .
\end{proof}

\begin{theorem}\label{thm:bifix-free_upper_bound}
For , the syntactic complexity of the class of bifix-free languages with  quotients is .
\end{theorem}
\begin{proof}
We construct an~injective mapping .
Since  will be injective, this will prove that .

The mapping  is defined by 23 (sub)cases covering all possibilities for a~transformation .
Let  be a~transformation of , and  be the assigned transformation .
In every (sub)case we prove \emph{external injectivity}, which is that there is no other
transformation  that fits in one of the previous (sub)cases and results in the same , and
we prove \emph{internal injectivity}, which is that no other transformation  that fits  the same (sub)case results in the same .
All states and variables related to  are always marked by a~hat.

In every (sub)case we observe some properties of the defined transformations :
Property~(a) always says that a~colliding pair is focused by a~transformation of the form .
Property~(b) describes the orbits and trees of states which are mapped differently by  and ; this is often for a~use of Lemma~\ref{lem:orbits}.
Property~(c) concerns the existence of cycles in .

See the appendix for a~list and a~map of all (sub)cases.

\textbf{Supercase~1}: .\\
We take .
The internal and external injectivities are obvious.

\noindent For all the remaining cases let .
Note that all  with  fit in Supercase~1, and by the property of suffix-freeness,  for some .
Let  be the largest integer such that .

Then  is either  or , and we have two supercases covering these situations.

\textbf{Supercase~2}:  and .\\
Here we have the chain

Within this supercase, we will always assign transformations  focusing a~colliding pair.
Because the transformations assigned in Supercase~1 belong to , they do not focus any colliding pair, which makes them different from the transformations assigned now in Supercase~2.
Also, we will always have .
We have the following cases covering all possibilities for :

\textbf{Case~2.1}:  has a~cycle.\\
Let  be the minimal state among the states that appear in cycles of , that is,

Let  be the transformation illustrated in Fig.~\ref{fig:case2.1} and defined by:
\begin{center}
  , ,\\
   for ,\\
   for the other states .
\end{center}
\begin{figure}[ht]
\unitlength 10pt\small
\gasset{Nh=2.5,Nw=2.5,Nmr=1.25,ELdist=0.3,loopdiam=1.5}
\begin{center}\begin{picture}(28,13)(0,-4)
\node[Nframe=n](name)(0,7){\normalsize}
\node(0)(2,0){0}\imark(0)
\node(p)(8,0){}
\node[Nframe=n](pdots)(14,0){}
\node(pt^k)(20,0){}
\node(n-1)(26,0){-}
\node(n-2)(26,4){-}\rmark(n-2)
\node(z)(12,4){}
\node(r)(14,7){}
\node[Nframe=n](rdots)(16,4){}
\drawedge(0,p){}
\drawedge(p,pdots){}
\drawedge(pdots,pt^k){}
\drawedge(pt^k,n-1){}
\drawedge(n-2,n-1){}
\drawloop[loopangle=270](n-1){}
\drawedge[curvedepth=1](z,r){}
\drawedge[curvedepth=1](r,rdots){}
\drawedge[curvedepth=1](rdots,z){}
\end{picture}
\begin{picture}(28,13)(0,-4)
\node[Nframe=n](name)(0,7){\normalsize}
\node(0')(2,0){0}\imark(0')
\node(p')(8,0){}
\node[Nframe=n](pdots')(14,0){}
\node(pt^k')(20,0){}
\node(n-1')(26,0){-}
\node(n-2')(26,4){-}\rmark(n-2')
\node(z')(12,4){}
\node(r')(14,7){}
\node[Nframe=n](rdots')(16,4){}
\drawedge[linecolor=red,dash={.5 .25}{.25},curvedepth=-3](0',n-1'){}
\drawedge[linecolor=red,dash={.5 .25}{.25}](pdots',p'){}
\drawedge[linecolor=red,dash={.5 .25}{.25}](pt^k',pdots'){}
\drawedge[linecolor=red,dash={.5 .25}{.25},curvedepth=3.5](p',r'){}
\drawedge[curvedepth=1](z',r'){}
\drawedge[curvedepth=1](r',rdots'){}
\drawedge[curvedepth=1](rdots',z'){}
\drawloop[loopangle=270](n-1'){}
\drawedge(n-2',n-1'){}
\end{picture}\end{center}
\caption{Case~2.1.}\label{fig:case2.1}
\end{figure}

Let  be the state from the cycle of  such that . 
We observe the following properties:
\begin{enumerate}
\item[(a)] Pair  is a~colliding pair focused by  to state  in the cycle, which is the smallest state of all states in cycles. 
This is the only colliding pair which is focused by  to a~state in a~cycle.

\noindent\textit{Proof}: Note that  collides with any state in a~cycle of , in particular, with .
The property follows because  differs from  only in the mapping of states  () and , and the only state mapped to a~cycle is .

\item[(b)] All states from  whose mapping is different in  and  belong to the same orbit in  of a~cycle.
Hence, all colliding pairs that are focused by  consist only of states from this orbit.

\item[(c)]  has a~cycle.

\item[(d)] For each  with , there is precisely one state  colliding with  and mapped by  to , and that state is .

\noindent\textit{Proof}: Clearly  satisfies this condition. Suppose that . Since  is the only state mapped to  by  and not by , it follows that . So  and  are focused to  by ; since they collide, this is a~contradiction.
\end{enumerate}

\textit{Internal injectivity}:
Let  be any transformation that fits in this case and results in the same ; we will show that .
From~(a), there is the unique colliding pair  focused to a~state in a~cycle, hence .
Moreover,  and  are not in this cycle, so  and , which means that .
Since there is no state  such that , the only state mapped to  by  is , hence .
From~(d) for , state  is uniquely determined, hence .
Finally, for  there is no state colliding with  and mapped to , hence .
Since the other transitions in  are defined exactly as in  and , we have .

\textbf{Case~2.2}:  has no cycles, but .\\
Let  be the transformation illustrated in Fig.~\ref{fig:case2.2} and defined by:
\begin{center}
  , ,\\
   for ,\\
   for the other states .
\end{center}
\begin{figure}[ht]
\unitlength 10pt\small
\gasset{Nh=2.5,Nw=2.5,Nmr=1.25,ELdist=0.3,loopdiam=1.5}
\begin{center}\begin{picture}(28,10)(0,-4)
\node[Nframe=n](name)(0,4){\normalsize}
\node(0)(2,0){0}\imark(0)
\node(p)(8,0){}
\node[Nframe=n](pdots)(14,0){}
\node(pt^k)(20,0){}
\node(n-1)(26,0){-}
\node(n-2)(26,4){-}\rmark(n-2)
\drawedge(0,p){}
\drawedge(p,pdots){}
\drawedge(pdots,pt^k){}
\drawedge(pt^k,n-1){}
\drawedge(n-2,n-1){}
\drawloop[loopangle=270](n-1){}
\end{picture}
\begin{picture}(28,10)(0,-4)
\node[Nframe=n](name)(0,4){\normalsize}
\node(0')(2,0){0}\imark(0')
\node(p')(8,0){}
\node[Nframe=n](pdots')(14,0){}
\node(pt^k')(20,0){}
\node(n-1')(26,0){-}
\node(n-2')(26,4){-}\rmark(n-2')
\drawedge[curvedepth=-3,linecolor=red,dash={.5 .25}{.25}](0',n-1'){}
\drawedge[linecolor=red,dash={.5 .25}{.25}](pdots',p'){}
\drawedge[linecolor=red,dash={.5 .25}{.25}](pt^k',pdots'){}
\drawloop(p'){}
\drawedge(n-2',n-1'){}
\drawloop[loopangle=270](n-1'){}
\end{picture}\end{center}
\caption{Case~2.2.}\label{fig:case2.2}
\end{figure}

We observe the following properties:
\begin{enumerate}
\item[(a)]  is a~colliding pair focused by  to a~fixed point of in-degree 2.
This is the only pair among all colliding pairs focused to a~fixed point.

\noindent\textit{Proof}: This follows from the definition of , since any colliding pair focused by  contains  (), and only  is mapped to , which is a~fixed point.
Also, no state except  can be mapped to  by  because this would violate suffix-freeness; so only  and  are mapped by  to , and  has in-degree 2.

\item[(b)] All states from  whose mapping is different in  and  belong to the same orbit in  of a~fixed point.

\item[(c)]  does not have any cycles, but has a~fixed point  with in-degree .

\item[(d)] For each  with , there is precisely one state  colliding with  and mapped to , and that state is .

This follows exactly like Property~(d) from Case~2.1.
\end{enumerate}

\textit{External injectivity}:
Here  does not have a~cycle in contrast to the transformations of Case~2.1.

\textit{Internal injectivity}:
Let  be any transformation that fits in this case and results in the same .
From~(a) there is the unique colliding pair  focused to the fixed point , hence  and .
Then, from~(d), for  state  is uniquely defined, hence .
Since the other transitions in  are defined exactly as in  and , we have .

\textbf{Case~2.3}:  does not fit in any of the previous cases, but there exist at least two fixed points of in-degree~1.\\
Let the two smallest valued fixed points of in-degree 1 be the states  and , that is,


Let  be the transformation illustrated in Fig.~\ref{fig:case2.3} and defined by
\begin{center}
  , , , ,\\
   for the other states .
\end{center}
\begin{figure}[ht]
\unitlength 10pt\small
\gasset{Nh=2.5,Nw=2.5,Nmr=1.25,ELdist=0.3,loopdiam=1.5}
\begin{center}\begin{picture}(28,10)(0,-4)
\node[Nframe=n](name)(0,4){\normalsize}
\node(0)(2,0){0}\imark(0)
\node(p)(14,0){}
\node(n-1)(26,0){-}
\node(n-2)(26,4){-}\rmark(n-2)
\node(f1)(10,4){}
\node(f2)(18,4){}
\drawedge(0,p){}
\drawedge(p,n-1){}
\drawedge(n-2,n-1){}
\drawloop[loopangle=270](n-1){}
\drawloop(f1){}
\drawloop(f2){}
\end{picture}
\begin{picture}(28,10)(0,-4)
\node[Nframe=n](name)(0,4){\normalsize}
\node(0')(2,0){0}\imark(0')
\node(p')(14,0){}
\node(n-1')(26,0){-}
\node(n-2')(26,4){-}\rmark(n-2')
\node(f1')(10,4){}
\node(f2')(18,4){}
\drawedge[curvedepth=-3,linecolor=red,dash={.5 .25}{.25}](0',n-1'){}
\drawedge(n-2',n-1'){}
\drawloop[loopangle=270](n-1'){}
\drawedge[curvedepth=1,linecolor=red,dash={.5 .25}{.25}](f1',f2'){}
\drawedge[curvedepth=1,linecolor=red,dash={.5 .25}{.25}](f2',f1'){}
\drawedge[linecolor=red,dash={.5 .25}{.25}](p',f2'){}
\end{picture}\end{center}
\caption{Case~2.3.}\label{fig:case2.3}
\end{figure}

We observe the following properties:
\begin{enumerate}
\item[(a)]  is a~colliding pair focused by  to .
This is the only pair among all colliding pairs that are focused.

\item[(b)] All states from  whose mapping is different in  and  belong to the same orbit of a~cycle in .

\item[(c)]  has exactly one cycle, namely , and it is of length 2.
Moreover, one state in the cycle, which is , has in-degree~1, and the other one, which is , has in-degree~2.
\end{enumerate}

\textit{External injectivity}:
To see that  is distinct from the transformations of Case~2.1, observe that in  the only colliding pair is focused to , which lies in a~cycle but is not the smallest state of the states of cycles.
On the other hand, from~(a) of Case~2.1 the transformations of that case have only one colliding pair focused to a~state in a~cycle, and this is the smallest state from the states of cycles.

Since  has a~cycle, it is different from the transformations of Case~2.2.

\textit{Internal injectivity}:
Let  be any transformation that fits in this case and results in the same .
From~(c), there is a~single state in the unique cycle that has in-degree 2 and this is . Hence , and so .
From~(a), the unique focused colliding pair is , so  and .
Hence , , , and .
Since the other transitions in  are defined exactly as in  and , we have .

\textbf{Case~2.4}:  does not fit in any of the previous cases, but there exists  of in-degree  such that .\\
Let  be the smallest state among the states satisfying the conditions and with the largest  such that .
Since  and  does not have a~cycle,  and  are well defined.
We know that , and  has in-degree .
Within this case we have the following subcases covering all possibilities for :

\textbf{Subcase~2.4.1}:  and .\\
Let  be the transformation illustrated in Fig.~\ref{fig:subcase2.4.1} and defined by
\begin{center}
  , ,\\
   for the other states .
\end{center}
\begin{figure}[ht]
\unitlength 10pt\small
\gasset{Nh=2.5,Nw=2.5,Nmr=1.25,ELdist=0.3,loopdiam=1.5}
\begin{center}\begin{picture}(28,10)(0,-4)
\node[Nframe=n](name)(0,4){\normalsize}
\node(0)(2,0){0}\imark(0)
\node(p)(14,0){}
\node(n-1)(26,0){-}
\node(n-2)(26,4){-}\rmark(n-2)
\node(x)(6,4){}
\node(xt)(10,4){}
\node[Nframe=n](xdots)(14,4){}
\node(xt^ell)(18,4){}
\drawedge(0,p){}
\drawedge(p,n-1){}
\drawedge(n-2,n-1){}
\drawloop[loopangle=270](n-1){}
\drawedge(x,xt){}
\drawedge(xt,xdots){}
\drawedge(xdots,xt^ell){}
\drawedge(xt^ell,n-1){}
\end{picture}
\begin{picture}(28,10)(0,-4)
\node[Nframe=n](name)(0,4){\normalsize}
\node(0')(2,0){0}\imark(0')
\node(p')(14,0){}
\node(n-1')(26,0){-}
\node(n-2')(26,4){-}\rmark(n-2')
\node(x')(6,4){}
\node(xt')(10,4){}
\node[Nframe=n](xdots')(14,4){}
\node(xt^ell')(18,4){}
\drawedge[curvedepth=-3,linecolor=red,dash={.5 .25}{.25}](0',n-1'){}
\drawedge(n-2',n-1'){}
\drawloop[loopangle=270](n-1'){}
\drawedge[linecolor=red,dash={.5 .25}{.25}](p,xt^ell'){}
\drawedge(x',xt'){}
\drawedge(xt',xdots'){}
\drawedge(xdots',xt^ell'){}
\drawedge(xt^ell',n-1'){}
\end{picture}\end{center}
\caption{Subcase~2.4.1.}\label{fig:subcase2.4.1}
\end{figure}

We observe the following properties:
\begin{enumerate}
\item[(a)]  is a~colliding pair focused by  to .

\item[(b)]  is the only state from  whose mapping is different in  and , and  is mapped to a~state mapped to .

\item[(c)]  does not have any cycles.
\end{enumerate}

\textit{External injectivity}:
Since  does not have any cycles,  is different from the transformations of Case~2.1 and Case~2.3.

From~(a), we have a~focused colliding pair in the orbit of . Thus,  is different from the transformations of Case~2.2, where all states in focused colliding pairs are in the orbit of a~fixed point different from  (Property~(b) of Case~2.2).

\textit{Internal injectivity}:
Let  be any transformation that fits in this subcase and results in the same .
From~(b), all colliding pairs that are focused contain .
If there are at least two such pairs, then  is uniquely determined as the unique common state.
If there is only one such pair, then by~(a) it is , and  is determined as the state of in-degree , since  has in-degree .
Hence, , and since the other transitions in  are defined exactly as in  and , we have .

\textbf{Subcase~2.4.2}: , , and  has in-degree .\\
Let  be the smallest state different from  and such that .
Note that  has in-degree , as otherwise, it would contradict the choice of  since there would be a~state satisfying the conditions for  with a~larger .
Also, , as otherwise we would choose  as .
Let  be the transformation illustrated in Fig.~\ref{fig:subcase2.4.2} and defined by
\begin{center}
  , ,\\
  , ,\\
   for the other states .
\end{center}
\begin{figure}[ht]
\unitlength 10pt\small
\gasset{Nh=2.5,Nw=2.5,Nmr=1.25,ELdist=0.3,loopdiam=1.5}
\begin{center}\begin{picture}(28,14)(0,-4)
\node[Nframe=n](name)(0,8){\normalsize}
\node(0)(2,0){0}\imark(0)
\node(p)(14,0){}
\node(n-1)(26,0){-}
\node(n-2)(26,4){-}\rmark(n-2)
\node(x)(10,4){}
\node(xt)(18,4){}
\node(y)(10,8){}
\drawedge(0,p){}
\drawedge(p,n-1){}
\drawedge(n-2,n-1){}
\drawloop[loopangle=270](n-1){}
\drawedge(x,xt){}
\drawedge(xt,n-1){}
\drawedge(y,xt){}
\end{picture}
\begin{picture}(28,14)(0,-4)
\node[Nframe=n](name)(0,8){\normalsize}
\node(0')(2,0){0}\imark(0')
\node(p')(14,0){}
\node(n-1')(26,0){-}
\node(n-2')(26,4){-}\rmark(n-2')
\node(x')(10,4){}
\node(xt')(18,4){}
\node(y')(10,8){}
\drawedge[curvedepth=-3,linecolor=red,dash={.5 .25}{.25}](0',n-1'){}
\drawedge(n-2',n-1'){}
\drawloop[loopangle=270](n-1'){}
\drawedge[curvedepth=6,linecolor=red,dash={.5 .25}{.25}](p',y'){}
\drawedge[linecolor=red,dash={.5 .25}{.25}](xt',x'){}
\drawedge[linecolor=red,dash={.5 .25}{.25}](x',y'){}
\drawedge(y',xt'){}
\end{picture}\end{center}
\caption{Subcase~2.4.2.}\label{fig:subcase2.4.2}
\end{figure}

We observe the following properties.
\begin{enumerate}
\item[(a)]  is a~colliding pair focused by  to .

\item[(b)] All states from  whose mapping is different in  and  belong to the same orbit of a~cycle of length  in .

\item[(c)]  contains exactly one cycle, namely .
Furthermore,  has in-degree  and is preceded in this cycle by  of in-degree .
\end{enumerate}

\textit{External injectivity}:
To see that  is different from the transformations of Case~2.1, observe that by~(a) we have a~colliding pair focused to , which is from a~cycle, but is not the smallest state from the states in cycles since .

On the other hand, in Case~2.1 all colliding pairs focused to a~state in a~cycle are focused to the smallest state of all states in cycles (Property~(a) of Case~2.1). In Case~2.3, the transformation has a~cycle, but this cycle has length .

Since  has a~cycle, it is different from the transformations of Case~2.2 and Subcase~2.4.1 (recall that fixed points have not been defined as cycles).

\textit{Internal injectivity}:
Let  be any transformation that fits in this subcase and results in the same .
From~(c), in  we have a~unique cycle of length 3, and this cycle is .
Since  is uniquely determined as the state of in-degree 2 preceded in the cycle by the state of in-degree 1, we have .
Then also  and . State  is the only state outside the cycle mapped to , hence .
We have , , and .
Since the other transitions in  are defined exactly as in  and , we have .

\textbf{Subcase~2.4.3}: , , and  has in-degree .\\
We split the subcase into two subsubcases: (i)  and (ii) .
Let  be the transformation illustrated in Fig.~\ref{fig:subcase2.4.3} and defined by
\begin{center}
  , ,\\
  ,\\
   (i),  (ii),\\
   for the other states .
\end{center}
\begin{figure}[ht]
\unitlength 10pt\small
\gasset{Nh=2.5,Nw=2.5,Nmr=1.25,ELdist=0.3,loopdiam=1.5}
\begin{center}\begin{picture}(28,10)(0,-4)
\node[Nframe=n](name)(0,4){\normalsize}
\node(0)(2,0){0}\imark(0)
\node(p)(14,0){}
\node(n-1)(26,0){-}
\node(n-2)(26,4){-}\rmark(n-2)
\node(x)(10,4){}
\node(xt)(18,4){}
\drawedge(0,p){}
\drawedge(p,n-1){}
\drawedge(n-2,n-1){}
\drawloop[loopangle=270](n-1){}
\drawedge(x,xt){}
\drawedge(xt,n-1){}
\end{picture}
\begin{picture}(28,11)(0,-4)
\node[Nframe=n](name)(0,4){\normalsize}
\node(0')(2,0){0}\imark(0')
\node(p')(14,0){}
\node(n-1')(26,0){-}
\node(n-2')(26,4){-}\rmark(n-2')
\node(x')(10,4){}
\node(xt')(18,4){}
\drawedge[curvedepth=-3,linecolor=red,dash={.5 .25}{.25}](0',n-1'){}
\drawedge(n-2',n-1'){}
\drawloop[loopangle=270](n-1'){}
\drawedge[linecolor=red,dash={.5 .25}{.25}](p',x'){}
\drawedge[linecolor=red,dash={.5 .25}{.25}](xt',x'){}
\drawedge[curvedepth=2,linecolor=red,dash={.1 .1}{.1}](x',n-2'){(i)}
\drawedge[curvedepth=-.5,linecolor=red,dash={.1 .1}{.1},ELside=r](x',n-1'){(ii)}
\end{picture}\end{center}
\caption{Subcase~2.4.3.}\label{fig:subcase2.4.3}
\end{figure}

We observe the following properties:
\begin{enumerate}
\item[(a)]  is a~colliding pair focused by  to .
Both states from this pair have in-degree 0.

\item[(b)] All states from  whose mapping is different in  and  are from the orbit of ,
and  and  are the only such states that are not mapped to  nor to .

\item[(c)]  does not have any cycles.
\end{enumerate}

\textit{External injectivity}:
Since  does not have any cycles, it is different from the transformations of Case~2.1, Case~2.3, and Subcase~2.4.2.

By~(b) all colliding pairs that are focused have states from the orbit of , whereas the transformations of Case~2.2 focus a~colliding pair to a~fixed point.

Let  be a~transformation that fits in Subcase~2.4.1 and results in the same .
By Lemma~\ref{lem:orbits}, the orbits from Properties~(b) for both  and  must be the same, so .
But in , to  only states of in-degree 0 are mapped, whereas to  state  is mapped, which has in-degree at least 1.

\textit{Internal injectivity}:
Let  be any transformation that fits in this subcase and results in the same .
From~(a) and~(b),  is the unique colliding pair focused to a~state different from ; hence .
The pair is focused to , hence .
If  is mapped to , then we have subsubcase~(i) and  is the smaller state in the colliding pair.
If  is mapped to , then we have subsubcase~(ii) and  is the larger state in the colliding pair.
Hence  and . We have  and .
Since the other transitions in  are defined exactly as in  and , we have .

\textbf{Subcase 2.4.4}: .\\
Let  be the transformation illustrated in Fig.~\ref{fig:subcase2.4.4} and defined by
\begin{center}
  , ,\\
   for the other states .
\end{center}
\begin{figure}[ht]
\unitlength 10pt\small
\gasset{Nh=2.5,Nw=2.5,Nmr=1.25,ELdist=0.3,loopdiam=1.5}
\begin{center}\begin{picture}(28,10)(0,-4)
\node[Nframe=n](name)(0,4){\normalsize}
\node(0)(2,0){0}\imark(0)
\node(p)(14,0){}
\node(n-1)(26,0){-}
\node(n-2)(26,4){-}\rmark(n-2)
\node(x)(6,4){}
\node(xt)(10,4){}
\node[Nframe=n](xdots)(14,4){}
\node(xt^ell)(18,4){}
\drawedge(0,p){}
\drawedge(p,n-1){}
\drawedge(n-2,n-1){}
\drawloop[loopangle=270](n-1){}
\drawedge(x,xt){}
\drawedge(xt,xdots){}
\drawedge(xdots,xt^ell){}
\drawedge(xt^ell,n-2){}
\end{picture}
\begin{picture}(28,10)(0,-4)
\node[Nframe=n](name)(0,4){\normalsize}
\node(0')(2,0){0}\imark(0')
\node(p')(14,0){}
\node(n-1')(26,0){-}
\node(n-2')(26,4){-}\rmark(n-2')
\node(x')(6,4){}
\node(xt')(10,4){}
\node[Nframe=n](xdots')(14,4){}
\node(xt^ell')(18,4){}
\drawedge[curvedepth=-3,linecolor=red,dash={.5 .25}{.25}](0',n-1'){}
\drawedge(n-2',n-1'){}
\drawloop[loopangle=270](n-1'){}
\drawedge(x',xt'){}
\drawedge(xt',xdots'){}
\drawedge(xdots',xt^ell'){}
\drawedge(xt^ell',n-2'){}
\drawedge[curvedepth=-.5,linecolor=red,dash={.5 .25}{.25}](p',n-2'){}
\end{picture}\end{center}
\caption{Subcase~2.4.4.}\label{fig:subcase2.4.4}
\end{figure}

We observe the following properties:
\begin{enumerate}
\item[(a)]  is a~colliding pair focused by  to .

\item[(b)]  is the only state from  whose mapping is different in  and .

\item[(c)]  does not contain any cycles.
\end{enumerate}

\textit{External injectivity}:
Since  does not contain any cycles, it is different from the transformations of Case~2.1, Case~2.3, and Subcase~2.4.2.

From~(b), all focused colliding pairs contain  and so are mapped to  in .
Hence,  is different from the transformations of Case~2.2, Subcase~2.4.1, and Subcase~2.4.3.

\textit{Internal injectivity}:
Let  be any transformation that fits in this subcase and results in the same .
If there are two focused colliding pairs, then  is uniquely determined as the common state in these pairs.
If there is only one such pair, then  is the state of in-degree , as the other state is , which has in-degree .
Hence, . We have  and .
Since the other transitions in  are defined exactly as in  and , we have .

\textbf{Subcase 2.4.5}: .\\
Let  be the transformation illustrated in Fig.~\ref{fig:subcase2.4.5} and defined by
\begin{center}
  , ,\\
   for the other states .
\end{center}
\begin{figure}[ht]
\unitlength 10pt\small
\gasset{Nh=2.5,Nw=2.5,Nmr=1.25,ELdist=0.3,loopdiam=1.5}
\begin{center}\begin{picture}(28,11)(0,-4)
\node[Nframe=n](name)(0,6){\normalsize}
\node(0)(2,0){0}\imark(0)
\node(p)(14,0){}
\node(n-1)(26,0){-}
\node(n-2)(26,4){-}\rmark(n-2)
\node(x)(6,4){}
\node(xt)(10,4){}
\node[Nframe=n](xdots)(14,4){}
\node(xt^ell)(18,4){}
\drawedge(0,p){}
\drawedge(p,n-1){}
\drawedge(n-2,n-1){}
\drawloop[loopangle=270](n-1){}
\drawedge(x,xt){}
\drawedge(xt,xdots){}
\drawedge(xdots,xt^ell){}
\drawloop(xt^ell){}
\end{picture}
\begin{picture}(28,11)(0,-4)
\node[Nframe=n](name)(0,6){\normalsize}
\node(0')(2,0){0}\imark(0')
\node(p')(14,0){}
\node(n-1')(26,0){-}
\node(n-2')(26,4){-}\rmark(n-2')
\node(x')(6,4){}
\node(xt')(10,4){}
\node[Nframe=n](xdots')(14,4){}
\node(xt^ell')(18,4){}
\drawedge[curvedepth=-3,linecolor=red,dash={.5 .25}{.25}](0',n-1'){}
\drawedge(n-2',n-1'){}
\drawloop[loopangle=270](n-1'){}
\drawedge(x',xt'){}
\drawedge(xt',xdots'){}
\drawedge(xdots',xt^ell'){}
\drawloop(xt^ell'){}
\drawedge[linecolor=red,dash={.5 .25}{.25}](p',xt^ell'){}
\end{picture}\end{center}
\caption{Subcase~2.4.5.}\label{fig:subcase2.4.5}
\end{figure}

We observe the following properties:
\begin{enumerate}
\item[(a)]  is a~colliding pair focused by  to the fixed point , which has in-degree at least .

\item[(b)]  is the only state from  whose mapping is different in  and .

\item[(c)]  does not contain any cycles.
\end{enumerate}

\textit{External injectivity}:
Since  does not contain any cycles, it is different from the transformations of Case~2.1, Case~2.3, and Subcase~2.4.2.

Let  be a~transformation that fits in Case~2.2 and results in the same .
By Lemma~\ref{lem:orbits}, the orbits from Properties~(b) for both  and  must be the same, so .
But  has in-degree at least 3, whereas  has in-degree 2, which yields a~contradiction.

Since the orbits from Properties~(b) of the transformations of Subcase~2.4.1, Subcase~2.4.3, and Subcase~2.4.4 contain , by Lemma~\ref{lem:orbits} they are different from .

\textit{Internal injectivity}:
Let  be any transformation that fits in this subcase and results in the same .
By Lemma~\ref{lem:orbits}, the orbits from Properties~(b) for both  and  must be the same, so we obtain that .
If , then by~(b) , and also , as otherwise  and  would not result in the same .
Then,  is a~colliding pair because of .
But , so this colliding pair is focused by .
Hence, it must be that .

\textbf{Case 2.5}:  does not fit in any of the previous cases.\\
First we observe that there exists exactly one fixed point , and every state  is mapped either to  or to :
All transformations that fit in Supercase~2 and have a~cycle or with  are covered in Case~2.1 or~2.2.
If there are two fixed points of in-degree  then  is covered in Case~2.3.
If there is a~state  such that , then, since there are no cycles, there exists such a~state of in-degree , thus  is covered in Case~2.5.
Hence, every state  must either be a~fixed point or be mapped to  or , and there can be at most one fixed point.
If there is no fixed point, then  (transformation of Type~3) and so it falls into Supercase~1.

\textbf{Subcase~2.5.1}: There are at least two states from all  such that  for all .\\
Assume that .
Let  be the transformation illustrated in Fig.~\ref{fig:subcase2.5.1} and defined by
\begin{center}
  , ,\\
   for ,\\
  ,\\
   for the other states .
\end{center}
\begin{figure}[ht]
\unitlength 10pt\small
\gasset{Nh=2.5,Nw=2.5,Nmr=1.25,ELdist=0.3,loopdiam=1.5}
\begin{center}\begin{picture}(28,11)(0,-4)
\node[Nframe=n](name)(0,6){\normalsize}
\node(0)(2,0){0}\imark(0)
\node(p)(14,0){}
\node(n-1)(26,0){-}
\node(n-2)(26,4){-}\rmark(n-2)
\node(f)(8,4){}
\node(r1)(14,4){}
\node[Nframe=n](rdots)(18,4){}
\node(ru)(22,4){}
\drawedge(0,p){}
\drawedge(p,n-1){}
\drawedge(n-2,n-1){}
\drawloop[loopangle=270](n-1){}
\drawloop(f){}
\drawedge[curvedepth=-.2](r1,n-1){}
\drawedge[curvedepth=0,exo=.2](rdots,n-1){}
\drawedge[curvedepth=0,exo=.5](ru,n-1){}
\end{picture}
\begin{picture}(28,11)(0,-4)
\node[Nframe=n](name)(0,6){\normalsize}
\node(0')(2,0){0}\imark(0')
\node(p')(14,0){}
\node(n-1')(26,0){-}
\node(n-2')(26,4){-}\rmark(n-2')
\node(f')(8,4){}
\node(r1')(14,4){}
\node[Nframe=n](rdots')(18,4){}
\node(ru')(22,4){}
\drawedge[curvedepth=-3,linecolor=red,dash={.5 .25}{.25}](0',n-1'){}
\drawedge(n-2',n-1'){}
\drawloop[loopangle=270](n-1'){}
\drawloop(f'){}
\drawedge[linecolor=red,dash={.5 .25}{.25}](p',f'){}
\drawedge[linecolor=red,dash={.5 .25}{.25}](r1',rdots'){}
\drawedge[linecolor=red,dash={.5 .25}{.25}](rdots',ru'){}
\drawedge[curvedepth=-2,linecolor=red,dash={.5 .25}{.25}](ru',r1'){}
\end{picture}\end{center}
\caption{Subcase~2.5.1.}\label{fig:subcase2.5.1}
\end{figure}

We observe the following properties:
\begin{enumerate}
\item[(a)]  is a~colliding pair focused by  to the fixed point .
This is the only colliding pair that is focused by .

\item[(c)]  contains exactly one cycle.
\end{enumerate}

\textit{External injectivity}:
Since  has a~cycle, it is different from the transformations of Case~2.2, Subcase~2.4.1, Subcase~2.4.3, Subcase~2.4.4, and Subcase~2.4.5.

From~(a) and~(c),  has a~cycle and focuses a~colliding pair to a~state whose orbit is not the orbit of a~cycle.
Hence,  is different from the transformations of Case~2.1, Case~2.3, and of Subcase~2.4.2,
where all colliding pairs that are focused by these transformations have states from the orbit of a~cycle (Properties~(b) of these (sub)cases).

\textit{Internal injectivity}:
Let  be any transformation that fits in this subcase and results in the same .
By~(a),  is the unique colliding pair that is focused to the fixed point , so  and .
Also, there is exactly one cycle formed by the states , so .
It follows that , , and  for all .
Since the other transitions in  are defined exactly as in  and , we have .

\textbf{Subcase~2.5.2}:  does not fit in Subcase~2.5.1.\\
Because , we know that  contains at least three states.
Since  does not fit in Subcase~2.5.1, we have at least two states from all  such that .
Assume that .
Let  be the transformation illustrated in Fig.~\ref{fig:subcase2.5.2} and defined by
\begin{center}
  , ,\\
   for ,\\
  ,\\
   for the other states .
\end{center}
\begin{figure}[ht]
\unitlength 10pt\small
\gasset{Nh=2.5,Nw=2.5,Nmr=1.25,ELdist=0.3,loopdiam=1.5}
\begin{center}\begin{picture}(28,11)(0,-4)
\node[Nframe=n](name)(0,6){\normalsize}
\node(0)(2,0){0}\imark(0)
\node(p)(14,0){}
\node(n-1)(26,0){-}
\node(n-2)(26,4){-}\rmark(n-2)
\node(f)(8,4){}
\node(q1)(14,4){}
\node[Nframe=n](qdots)(18,4){}
\node(qv)(22,4){}
\drawedge(0,p){}
\drawedge(p,n-1){}
\drawedge(n-2,n-1){}
\drawloop[loopangle=270](n-1){}
\drawloop(f){}
\drawedge[curvedepth=-3,exo=1](q1,n-2){}
\drawedge[curvedepth=-2](qdots,n-2){}
\drawedge[curvedepth=0](qv,n-2){}
\end{picture}
\begin{picture}(28,11)(0,-4)
\node[Nframe=n](name)(0,6){\normalsize}
\node(0')(2,0){0}\imark(0')
\node(p')(14,0){}
\node(n-1')(26,0){-}
\node(n-2')(26,4){-}\rmark(n-2')
\node(f')(8,4){}
\node(q1')(14,4){}
\node[Nframe=n](qdots')(18,4){}
\node(qv')(22,4){}
\drawedge[curvedepth=-3,linecolor=red,dash={.5 .25}{.25}](0',n-1'){}
\drawedge(n-2',n-1'){}
\drawloop[loopangle=270](n-1'){}
\drawloop(f'){}
\drawedge[linecolor=red,dash={.5 .25}{.25}](p',f'){}
\drawedge[curvedepth=2,linecolor=red,dash={.5 .25}{.25}](q1',qv'){}
\drawedge[linecolor=red,dash={.5 .25}{.25}](qdots',q1'){}
\drawedge[linecolor=red,dash={.5 .25}{.25}](qv',qdots'){}
\end{picture}\end{center}
\caption{Subcase~2.5.2.}\label{fig:subcase2.5.2}
\end{figure}

We observe the following properties:
\begin{enumerate}
\item[(a)]  is a~colliding pair focused by  to the fixed point .
This is the only colliding pair that is focused by .

\item[(c)]  contains exactly one cycle.
\end{enumerate}

\textit{External injectivity}:
In the same way as in Subcase~2.5.1,  is different from the transformations of Cases~2.1--2.4.

Now suppose that the same transformation  is obtained in Subcase~2.5.1.
Since the unique cycles in both subcases go in opposite directions w.r.t. the ordering of the states, if they are equal then they must be of length .
But then, since , we have at least one state in  being mapped to  in , and also in .
But since  is also obtained in Subcase~2.5.1, there are no such states besides , , and , which yields a~contradiction.

\textit{Internal injectivity}:
Let  be any transformation that fits in this subcase and results in the same .
It follows in the same way as in~Subcase~2.5.1, that we have , , and  for all .
Since the other transitions in  are defined exactly as in  and , we have .

\textbf{Supercase 3:}  and .\\
Here we have the chain

We will always assign transformations  such that  together with  generate a~transformation that focuses a~colliding pair, which distinguishes such transformations  from those of Supercase~1.
Moreover, we will always have , to distinguish  from the transformations of Supercase~2.

For all the cases of Supercase~3, let  be all the states such that , for all .
Without loss of generality, we assume that .

In contrast to Supercase~2, we have an~additional difficulty in constructions of , which is that no state can be mapped to  except state .
On the other hand, the chains going through a~state  and ending in  are of length at most  (i.e.\ they contain at most  states including ).
Otherwise, if there is such a~chain of length at least , then there would exist a~state  such that , which means that the pair  is colliding because of  and focused by  to , contradicting suffix-freeness.
This fact will allow give us more knowledge about the transformation , helping to construct a~suitable .
In particular, when , all states  have in-degree .

We have the following cases covering all possibilities for :

\textbf{Case 3.1}:  and  has a~cycle.\\
Let  be the minimal among the states that appear in cycles of , that is,

Let  be the transformation illustrated in Fig.~\ref{fig:case3.1} and defined by
\begin{center}
  , ,\\
   for ,\\
   for the other states .
\end{center}
\begin{figure}[ht]
\unitlength 10pt\small
\gasset{Nh=2.5,Nw=2.5,Nmr=1.25,ELdist=0.3,loopdiam=1.5}
\begin{center}\begin{picture}(28,12)(0,-2)
\node[Nframe=n](name)(0,9){\normalsize}
\node(0)(2,7){0}\imark(0)
\node(p)(14,7){}
\node(n-1)(26,0){-}
\node(n-2)(26,7){-}\rmark(n-2)
\node(z)(12,-1){}
\node(r)(14,2){}
\node[Nframe=n](rdots)(16,-1){}
\node(q1)(18,4){}
\node[Nframe=n](qdots)(20.5,4){}
\node(qv)(23,4){}
\drawedge(0,p){}
\drawedge(p,n-2){}
\drawedge(n-2,n-1){}
\drawloop[loopangle=270](n-1){}
\drawedge[curvedepth=.5](q1,n-2){}
\drawedge[curvedepth=.6,sxo=-.5,exo=1.5](qdots,n-2){}
\drawedge[curvedepth=0](qv,n-2){}
\drawedge[curvedepth=1](z,r){}
\drawedge[curvedepth=1](r,rdots){}
\drawedge[curvedepth=1](rdots,z){}
\end{picture}
\begin{picture}(28,12)(0,-1)
\node[Nframe=n](name)(0,9){\normalsize}
\node(0')(2,7){0}\imark(0')
\node(p')(14,7){}
\node(n-1')(26,0){-}
\node(n-2')(26,7){-}\rmark(n-2')
\node(z')(12,-1){}
\node(r')(14,2){}
\node[Nframe=n](rdots')(16,-1){}
\node(q1')(18,4){}
\node[Nframe=n](qdots')(20.5,4){}
\node(qv')(23,4){}
\drawedge[curvedepth=3,linecolor=red,dash={.5 .25}{.25}](0',n-2'){}
\drawedge[linecolor=red,dash={.5 .25}{.25}](p',r'){}
\drawedge(n-2',n-1'){}
\drawloop[loopangle=270](n-1'){}
\drawedge[curvedepth=-.2,linecolor=red,dash={.5 .25}{.25}](q1',p'){}
\drawedge[curvedepth=-.3,syo=.5,linecolor=red,dash={.5 .25}{.25}](qdots',p'){}
\drawedge[curvedepth=-.8,linecolor=red,dash={.5 .25}{.25}](qv',p'){}
\drawedge[curvedepth=1](z',r'){}
\drawedge[curvedepth=1](r',rdots'){}
\drawedge[curvedepth=1](rdots',z'){}
\end{picture}\end{center}
\caption{Case~3.1.}\label{fig:case3.1}
\end{figure}

Let  be the state from the cycle of  such that . 
We observe the following properties:
\begin{enumerate}
\item[(a)]  is a~colliding pair focused by  to state  in the cycle, which is the smallest state in a~cycle. 
This is the only colliding pair which is focused to a~state in a~cycle.

\item[(b)] All states from  whose mapping is different in  and  belong to the tree of , and so to the orbit of a~cycle.

\item[(c)]  has a~cycle.
\end{enumerate}

\textit{Internal injectivity}:
Let  be any transformation that fits in this case and results in the same ; we will show that .
From~(a), there is the unique colliding pair  focused to a~state in a~cycle, hence .
Moreover,  and  are not in the cycle, whereas  and  are, so  and .
Since there is no state  such that , the only states mapped to  by  are , hence  for all .
We know that , and  for all .
Since the other transitions in  are defined exactly as in  and , we know that .

\textbf{Case~3.2}:  does not fit into any of the previous cases, , and there exists a~state  such that .\\
Let  be the smallest state among the states satisfying the conditions and with the largest  such that .
By the conditions of the case and since  does not have a~cycle,  is well-defined, and  and it is finite.

Note that , because  collides with .
We have , and  has in-degree 0.
Also note that, since , all  are of in-degree 0, because otherwise  would violate suffix-freeness.
We have the following subcases in this case that cover all possibilities for :

\textbf{Subcase~3.2.1}:  and .\\
We have the following two subsubcases: (i) there exists  such that , and (ii) there is no such .
Let  be the transformation illustrated in Fig.~\ref{fig:subcase3.2.1} and defined by
\begin{center}
  , ,\\
   (i),  (ii),\\
   for ,\\
   for the other states .
\end{center}
\begin{figure}[ht]
\unitlength 10pt\small
\gasset{Nh=2.5,Nw=2.5,Nmr=1.25,ELdist=0.3,loopdiam=1.5}
\begin{center}\begin{picture}(28,12)(0,-2)
\node[Nframe=n](name)(0,9){\normalsize}
\node(0)(2,7){0}\imark(0)
\node(p)(14,7){}
\node(n-1)(26,0){-}
\node(n-2)(26,7){-}\rmark(n-2)
\node(x)(2,0){}
\node(xt)(6,0){}
\node[Nframe=n](xdots)(10,0){}
\node(xtl)(14,0){}
\node(q1)(18,4){}
\node[Nframe=n](qdots)(20.5,4){}
\node(qv)(23,4){}
\drawedge(0,p){}
\drawedge(p,n-2){}
\drawedge(n-2,n-1){}
\drawloop[loopangle=270](n-1){}
\drawedge[curvedepth=.5](q1,n-2){}
\drawedge[curvedepth=.6,sxo=-.5,exo=1.5](qdots,n-2){}
\drawedge[curvedepth=0](qv,n-2){}
\drawedge(x,xt){}
\drawedge(xt,xdots){}
\drawedge(xdots,xtl){}
\drawedge(xtl,n-1){}
\end{picture}
\begin{picture}(28,12)(0,-1)
\node[Nframe=n](name)(0,9){\normalsize}
\node(0')(2,7){0}\imark(0')
\node(p')(14,7){}
\node(n-1')(26,0){-}
\node(n-2')(26,7){-}\rmark(n-2')
\node(q1')(18,4){}
\node[Nframe=n](qdots')(20.5,4){}
\node(qv')(23,4){}
\node(x')(2,0){}
\node(xt')(6,0){}
\node[Nframe=n](xdots')(10,0){}
\node(xtl')(14,0){}
\drawedge[curvedepth=3,linecolor=red,dash={.5 .25}{.25}](0',n-2'){}
\drawedge(n-2',n-1'){}
\drawloop[loopangle=270](n-1'){}
\drawedge[curvedepth=-.2,linecolor=red,dash={.5 .25}{.25}](q1',p'){}
\drawedge[curvedepth=-.3,syo=.5,linecolor=red,dash={.5 .25}{.25}](qdots',p'){}
\drawedge[curvedepth=-.8,linecolor=red,dash={.5 .25}{.25}](qv',p'){}
\drawedge[curvedepth=-3,linecolor=red,dash={.5 .25}{.25}](p',xtl'){}
\drawedge(x',xt'){}
\drawedge(xt',xdots'){}
\drawedge(xdots',xtl'){}
\drawloop[ELpos=80,linecolor=red,dash={.1 .1}{.1}](xtl'){(i)}
\drawedge[linecolor=red,dash={.1 .1}{.1}](xtl',n-1'){(ii)}
\end{picture}\end{center}
\caption{Subcase~3.2.1.}\label{fig:subcase3.2.1}
\end{figure}

We observe the following properties:
\begin{enumerate}
\item[(a)]  is a~colliding pair focused by  to .

\item[(b)] All states from  whose mapping is different in  and  belong to the tree of ,
which is either a~fixed point (i) or a~state mapped to  (ii).

\item[(c)]  does not contain any cycles.
\end{enumerate}

\textit{External injectivity}:
Since  does not have any cycles, it is different from the transformations of Case~3.1.

\textit{Internal injectivity}:
Let  be any transformation that fits in this subcase and results in the same ; we will show that .
By Lemma~\ref{lem:orbits} the trees from~(b) of  and  must be the same, so .
Also, the subsubcase is determined by  and thus it is the same for both  and .

Consider all colliding pairs focused by  to  that do not contain .
All of them contain , so if there are two or more such pairs, then .
Suppose that there is only one such pair .
Note that , as this is the length of a~longest path ending at .
Also, only states  are mapped to state , and they all have in-degree 0.
If , then  is distinguished from , since to  there is mapped  of in-degree ; hence .
Consider .
Let  be the set of states that are mapped either to  or to ; then .
The smallest state in  is either a~state  or  (by the choice of ).
If the subsubcase is (i), then the smallest state in  is  and so is mapped to , while in subsubcase (ii) it is  mapped to .
Hence, the smallest state distinguishes  from , and we have  and .
Then also  for all , since these are precisely the states mapped to .
Summarizing, we know that , , , and .
Since the other transitions in  are defined exactly as in  and , we have .

\textbf{Subcase~3.2.2}: , , and  has in-degree at least .\\
Let  be the smallest state such that  and .
Note that  and  has in-degree 0.
Let  be the transformation illustrated in Fig.~\ref{fig:subcase3.2.2} and defined by
\begin{center}
  , ,\\
  , ,\\
   for all ,\\
   for the other states .
\end{center}
\begin{figure}[ht]
\unitlength 10pt\small
\gasset{Nh=2.5,Nw=2.5,Nmr=1.25,ELdist=0.3,loopdiam=1.5}
\begin{center}\begin{picture}(28,12)(0,-2)
\node[Nframe=n](name)(0,9){\normalsize}
\node(0)(2,7){0}\imark(0)
\node(p)(14,7){}
\node(n-1)(26,0){-}
\node(n-2)(26,7){-}\rmark(n-2)
\node(x)(8,0){}
\node(xt)(14,0){}
\node(y)(8,4){}
\node(q1)(18,4){}
\node[Nframe=n](qdots)(20.5,4){}
\node(qv)(23,4){}
\drawedge(0,p){}
\drawedge(p,n-2){}
\drawedge(n-2,n-1){}
\drawloop[loopangle=270](n-1){}
\drawedge[curvedepth=.5](q1,n-2){}
\drawedge[curvedepth=.6,sxo=-.5,exo=1.5](qdots,n-2){}
\drawedge[curvedepth=0](qv,n-2){}
\drawedge(x,xt){}
\drawedge(xt,n-1){}
\drawedge(y,xt){}
\end{picture}
\begin{picture}(28,12)(0,-1)
\node[Nframe=n](name)(0,9){\normalsize}
\node(0')(2,7){0}\imark(0')
\node(p')(14,7){}
\node(n-1')(26,0){-}
\node(n-2')(26,7){-}\rmark(n-2')
\node(q1')(18,4){}
\node[Nframe=n](qdots')(20.5,4){}
\node(qv')(23,4){}
\node(x')(8,0){}
\node(xt')(14,0){}
\node(y')(8,4){}
\drawedge[curvedepth=3,linecolor=red,dash={.5 .25}{.25}](0',n-2'){}
\drawedge(n-2',n-1'){}
\drawloop[loopangle=270](n-1'){}
\drawedge[curvedepth=-.2,linecolor=red,dash={.5 .25}{.25}](q1',p'){}
\drawedge[curvedepth=-.3,syo=.5,linecolor=red,dash={.5 .25}{.25}](qdots',p'){}
\drawedge[curvedepth=-.8,linecolor=red,dash={.5 .25}{.25}](qv',p'){}
\drawedge(y',xt'){}
\drawedge[linecolor=red,dash={.5 .25}{.25}](p',y'){}
\drawedge[linecolor=red,dash={.5 .25}{.25}](xt',x'){}
\drawedge[linecolor=red,dash={.5 .25}{.25}](x',y'){}
\end{picture}\end{center}
\caption{Subcase~3.2.2.}\label{fig:subcase3.2.2}
\end{figure}

We observe the following properties:
\begin{enumerate}
\item[(a)]  is a~colliding pair focused by  to .
Note that in contrast to the previous cases, the focusing transformation here is  instead of .

\item[(b)] All states from  whose mapping is different in  and  belong to the same orbit of a~cycle.

\item[(c)]  contains exactly one cycle, namely .
\end{enumerate}

\textit{External injectivity}:
Since all colliding pairs focused by  must belong to the orbit from~(b), and the smallest state in the cycle of the orbit from~(b) is  of in-degree 1,  does not map a~colliding pair to it and thus it is different from the transformations of Case~3.1.

Since  has a~cycle, it is different from the transformations of Subcase~3.3.1.

\textit{Internal injectivity}:
Let  be any transformation that fits in this subcase and results in the same ; we will show that .
All colliding pairs that are focused have states from the orbit of the cycle from Property~(b), hence .
Since  and  are the smallest states in the cycle, we have , , and .
Since  has in-degree 0 in ,  is the only state outside the cycle that is mapped to  in ; hence .
Also, all states mapped to  by  are precisely the states ; hence  for all .
We know that , , , , and .
Since the other transitions in  are defined exactly as in  and , we have .

\textbf{Subcase~3.2.3}: , , and  has in-degree .\\
We split the subcase into the following two subsubcases: (i)  or ; (ii)  and .
Let  be the transformation illustrated in Fig.~\ref{fig:subcase3.2.3} and defined by
\begin{center}
  , ,\\
  ,\\
   (i),  (ii),\\
   for all ,\\
   for the other states .
\end{center}
\begin{figure}[ht]
\unitlength 10pt\small
\gasset{Nh=2.5,Nw=2.5,Nmr=1.25,ELdist=0.3,loopdiam=1.5}
\begin{center}\begin{picture}(28,12)(0,-2)
\node[Nframe=n](name)(0,9){\normalsize}
\node(0)(2,7){0}\imark(0)
\node(p)(14,7){}
\node(n-1)(26,0){-}
\node(n-2)(26,7){-}\rmark(n-2)
\node(x)(8,0){}
\node(xt)(14,0){}
\node(q1)(18,4){}
\node[Nframe=n](qdots)(20.5,4){}
\node(qv)(23,4){}
\drawedge(0,p){}
\drawedge(p,n-2){}
\drawedge(n-2,n-1){}
\drawloop[loopangle=270](n-1){}
\drawedge[curvedepth=.5](q1,n-2){}
\drawedge[curvedepth=.6,sxo=-.5,exo=1.5](qdots,n-2){}
\drawedge[curvedepth=0](qv,n-2){}
\drawedge(x,xt){}
\drawedge(xt,n-1){}
\end{picture}
\begin{picture}(28,13)(0,-2)
\node[Nframe=n](name)(0,9){\normalsize}
\node(0')(2,7){0}\imark(0')
\node(p')(14,7){}
\node(n-1')(26,0){-}
\node(n-2')(26,7){-}\rmark(n-2')
\node(q1')(18,4){}
\node[Nframe=n](qdots')(20.5,4){}
\node(qv')(23,4){}
\node(x')(8,0){}
\node(xt')(14,0){}
\drawedge[curvedepth=3,linecolor=red,dash={.5 .25}{.25}](0',n-2'){}
\drawedge(n-2',n-1'){}
\drawloop[loopangle=270](n-1'){}
\drawedge[curvedepth=-.2,linecolor=red,dash={.5 .25}{.25}](q1',p'){}
\drawedge[curvedepth=-.3,syo=.5,linecolor=red,dash={.5 .25}{.25}](qdots',p'){}
\drawedge[curvedepth=-.8,linecolor=red,dash={.5 .25}{.25}](qv',p'){}
\drawedge[linecolor=red,dash={.5 .25}{.25}](p',x'){}
\drawedge[linecolor=red,dash={.5 .25}{.25}](xt',x'){}
\drawloop[linecolor=red,dash={.1 .1}{.1}](x'){(i)}
\drawedge[linecolor=red,dash={.1 .1}{.1},ELside=r,curvedepth=-2](x',n-1){(ii)}
\end{picture}\end{center}
\caption{Subcase~3.2.3.}\label{fig:subcase3.2.3}
\end{figure}

We observe the following properties:
\begin{enumerate}
\item[(a)]  is a~colliding pair focused by  to .

\item[(b)] All states from  whose mapping is different in  and  belong to the same tree of ,
which is either a~fixed point (i) or a~state mapped to  (ii).

\item[(c)]  does not contain any cycles.
\end{enumerate}

\textit{External injectivity}:
Since  does not have any cycles, it is different from the transformations of Case~3.1 and Subcase~3.2.2.

Let  be a~transformation that fits in Subcase~3.2.1 and results in the same .
By Lemma~\ref{lem:orbits}, the trees from~(b) of both  and  must be the same, so .
It follows that the subsubcases, which are determined by , are the same for both  and .
Note that  has in-degree 2 in , one of the states from this pair (i.e.\ ) has in-degree , and the other one () has in-degree at least 1.
If the subsubcase is~(i), then  has in-degree at least 1, and so both the states have in-degree at least 1, which yields a~contradiction.
If the subsubcase is~(ii), then  has in-degree 0, and so both the states have in-degree 0, which yields a~contradiction.

\textit{Internal injectivity}:
Let  be any transformation that fits in this subcase and results in the same ; we will show that .
By Lemma~\ref{lem:orbits} we know that , and so also .
The subsubcase for both  and  is determined by  and so must be the same.
If the subsubcase is~(i), then  has in-degree  or it is smaller than ; hence  and .
If the subsubcase is~(ii), then both  and  have in-degree  and  is larger than ; hence again  and .
Also,  as these are precisely all the states mapped to  by .
We know that , , , and  for all .
Since the other transitions in  are defined exactly as in  and , we have .

\textbf{Subcase~3.2.4}: .\\
Let  be the transformation illustrated in Fig.~\ref{fig:subcase3.2.4} and defined by
\begin{center}
  , ,\\
   for ,\\
  ,\\
   for ,\\
   for the other states .
\end{center}
\begin{figure}[ht]
\unitlength 10pt\small
\gasset{Nh=2.5,Nw=2.5,Nmr=1.25,ELdist=0.3,loopdiam=1.5}
\begin{center}\begin{picture}(28,12)(0,-2)
\node[Nframe=n](name)(0,9){\normalsize}
\node(0)(2,7){0}\imark(0)
\node(p)(14,7){}
\node(n-1)(26,0){-}
\node(n-2)(26,7){-}\rmark(n-2)
\node(x)(2,0){}
\node(xt)(6,0){}
\node[Nframe=n](xdots)(10,0){}
\node(xtl)(14,0){}
\node(q1)(18,4){}
\node[Nframe=n](qdots)(20.5,4){}
\node(qv)(23,4){}
\drawedge(0,p){}
\drawedge(p,n-2){}
\drawedge(n-2,n-1){}
\drawloop[loopangle=270](n-1){}
\drawedge[curvedepth=.5](q1,n-2){}
\drawedge[curvedepth=.6,sxo=-.5,exo=1.5](qdots,n-2){}
\drawedge[curvedepth=0](qv,n-2){}
\drawedge(x,xt){}
\drawedge(xt,xdots){}
\drawedge(xdots,xtl){}
\drawloop(xtl){}
\end{picture}
\begin{picture}(28,16)(0,-5)
\node[Nframe=n](name)(0,9){\normalsize}
\node(0')(2,7){0}\imark(0')
\node(p')(14,7){}
\node(n-1')(26,0){-}
\node(n-2')(26,7){-}\rmark(n-2')
\node(q1')(18,4){}
\node[Nframe=n](qdots')(20.5,4){}
\node(qv')(23,4){}
\node(x')(2,0){}
\node(xt')(6,0){}
\node[Nframe=n](xdots')(10,0){}
\node(xtl')(14,0){}
\drawedge[curvedepth=3,linecolor=red,dash={.5 .25}{.25}](0',n-2'){}
\drawedge(n-2',n-1'){}
\drawloop[loopangle=270](n-1'){}
\drawedge[curvedepth=6.5,sxo=.5,linecolor=red,dash={.5 .25}{.25}](q1',x'){}
\drawedge[curvedepth=7,sxo=.5,exo=-.5,linecolor=red,dash={.5 .25}{.25}](qdots',x'){}
\drawedge[curvedepth=7.5,sxo=.5,exo=-1,linecolor=red,dash={.5 .25}{.25}](qv',x'){}
\drawedge[linecolor=red,dash={.5 .25}{.25}](p',xtl'){}
\drawedge[linecolor=red,dash={.5 .25}{.25}](xtl',xdots'){}
\drawedge[linecolor=red,dash={.5 .25}{.25}](xdots',xt'){}
\drawedge[linecolor=red,dash={.5 .25}{.25}](xt',x'){}
\drawedge[linecolor=red,dash={.5 .25}{.25}](x',p'){}
\end{picture}\end{center}
\caption{Subcase~3.2.4.}\label{fig:subcase3.2.4}
\end{figure}

We observe the following properties:
\begin{enumerate}
\item[(a)]  is a~colliding pair focused by  to .

\item[(b)] All states from  whose mapping is different in  and  belong to the same orbit of a~cycle.

\item[(c)]  contains exactly one cycle, namely . 
\end{enumerate}

\textit{External injectivity}:
Let  be a~transformation that fits in Case~3.1 and results in the same .
Then  must have the cycle , since it exists in  and the construction of Case~3.1 does not introduce any new cycles.
But then  and . Since  collides with ,  and  cannot be both in .

Since  has a~cycle, it is different from the transformations of Subcase~3.2.1 and Subcase~3.2.3.

Now let  be a~transformation that fits in Subcase~3.2.2 and results in the same .
Since  contains exactly one cycle, it must be that  and .
We have the following three possibilities:
If , , and , then  focuses the colliding pair ; hence  and  cannot be both in .
If , then we have a~contradiction with that  has in-degree 1 and  has in-degree 2.
Finally, suppose that , , and .
Then  must have in-degree , and there is  (and ).
But  is a~colliding pair because of , and it is focused to  by ; hence  and  cannot be both in .

\textit{Internal injectivity}:
Let  be any transformation that fits in this subcase and results in the same ; we will show that .
By~(c), we know that  and .

First suppose that .
Then also , , , and so on for the states of the cycle.
We know that  for all .
Hence, , ,  for all , and .
Since the other transitions in  are defined exactly as in  and , we have .

Now suppose that .
So  for some .
Note that  collides with all states , and  collides with all states .
If , then there exists  with  that is different from  and collides with .
But then  focuses both these states to .
Finally consider .
If  then , which is a~colliding pair because of  that is focused by  to .
On the other hand, if , then , and so  is a~colliding pair because of  that is focused by  to .
Hence,  and  cannot be both in .

\textbf{Case~3.3}:  does not fit into any of the previous cases, , and there exist at least two fixed points of in-degree 1.\\
Let the two smallest fixed points of in-degree 1 be the states  and , that is,


Let  be the transformation illustrated in Fig.~\ref{fig:case3.3} and defined by
\begin{center}
  , , , ,\\
   for ,\\
   for the other states .
\end{center}
\begin{figure}[ht]
\unitlength 10pt\small
\gasset{Nh=2.5,Nw=2.5,Nmr=1.25,ELdist=0.3,loopdiam=1.5}
\begin{center}\begin{picture}(28,12)(0,-2)
\node[Nframe=n](name)(0,9){\normalsize}
\node(0)(2,7){0}\imark(0)
\node(p)(14,7){}
\node(n-1)(26,0){-}
\node(n-2)(26,7){-}\rmark(n-2)
\node(q1)(18,4){}
\node[Nframe=n](qdots)(20.5,4){}
\node(qv)(23,4){}
\node(f1)(8,0){}
\node(f2)(14,0){}
\drawedge(0,p){}
\drawedge(p,n-2){}
\drawedge(n-2,n-1){}
\drawloop[loopangle=270](n-1){}
\drawedge[curvedepth=.5](q1,n-2){}
\drawedge[curvedepth=.6,sxo=-.5,exo=1.5](qdots,n-2){}
\drawedge[curvedepth=0](qv,n-2){}
\drawloop(f1){}
\drawloop(f2){}
\end{picture}
\begin{picture}(28,14)(0,-1)
\node[Nframe=n](name)(0,9){\normalsize}
\node(0')(2,7){0}\imark(0')
\node(p')(14,7){}
\node(n-1')(26,0){-}
\node(n-2')(26,7){-}\rmark(n-2')
\node(q1')(18,4){}
\node[Nframe=n](qdots')(20.5,4){}
\node(qv')(23,4){}
\node(f1')(8,0){}
\node(f2')(14,0){}
\drawedge[curvedepth=3,linecolor=red,dash={.5 .25}{.25}](0',n-2'){}
\drawedge(n-2',n-1'){}
\drawloop[loopangle=270](n-1'){}
\drawedge[curvedepth=-.2,linecolor=red,dash={.5 .25}{.25}](q1',p'){}
\drawedge[curvedepth=-.3,syo=.5,linecolor=red,dash={.5 .25}{.25}](qdots',p'){}
\drawedge[curvedepth=-.8,linecolor=red,dash={.5 .25}{.25}](qv',p'){}
\drawedge[curvedepth=1,linecolor=red,dash={.5 .25}{.25}](f1',f2'){}
\drawedge[curvedepth=1,linecolor=red,dash={.5 .25}{.25}](f2',f1'){}
\drawedge[curvedepth=0,linecolor=red,dash={.5 .25}{.25}](p',f2'){}
\end{picture}\end{center}
\caption{Case~3.3.}\label{fig:case3.3}
\end{figure}

We observe the following properties:
\begin{enumerate}
\item[(a)]  is a~colliding pair focused by  to .

\item[(b)] All states from  whose mapping is different in  and  belong to the same orbit of a~cycle.

\item[(c)]  contains exactly one cycle, namely .
\end{enumerate}

\textit{External injectivity}:
Since  is the only colliding pair that is focused by  to a~state in a~cycle, and  is not the minimal state in the cycle,  is different from the transformations of Case~3.1.

Since  has a~cycle, it is different from the transformations of Subcase~3.2.1 and Subcase~3.2.3.
Also, since  has exactly one cycle of length 2, it is different from the transformations of Subcase~3.2.2 and Subcase~3.2.4, which have a~cycle of length at least 3.

\textit{Internal injectivity}:
Let  be any transformation that fits in this case and results in the same ; we will show that .
From~(c), we know that , and since  has in-degree 1 and  has in-degree 2 in , we have  and .
Also , as only  and  are mapped to .
Then  for all , since these are precisely the states mapped to  in .
Hence , , , , and  for all .
Since the other transitions in  are defined exactly as in  and , we have .

\textbf{Case~3.4}:  does not fit into any of the previous cases and .\\
In , there is neither a~cycle (covered by Case~3.1) nor a~state  such that  (covered by Case~3.2).
Hence, because , there must be a~fixed point  of in-degree 1.
Because of Case~3.3, there is exactly one such fixed point.

Let  be all the states from  such that .
Let  be all the states from  such that .
All states  and  have in-degree 0 (covered by Case~3.2), and they are all the states besides .
Because , we know that .
We have the following subcases that cover all possibilities for :

\textbf{Subcase~3.4.1}: .\\
Let  be the transformation illustrated in Fig.~\ref{fig:subcase3.4.1} and defined by
\begin{center}
  , ,\\
   for ,\\
  ,\\
   for ,\\
   for the other states .
\end{center}
\begin{figure}[ht]
\unitlength 10pt\small
\gasset{Nh=2.5,Nw=2.5,Nmr=1.25,ELdist=0.3,loopdiam=1.5}
\begin{center}\begin{picture}(28,14)(0,-2)
\node[Nframe=n](name)(0,13){\normalsize}
\node(0)(2,10){0}\imark(0)
\node(p)(14,10){}
\node(n-1)(26,0){-}
\node(n-2)(26,10){-}\rmark(n-2)
\node(q1)(17,7){}
\node[Nframe=n](qdots)(20,7){}
\node(qv)(23,7){}
\node(r1)(17,3){}
\node[Nframe=n](rdots)(20,3){}
\node(ru)(23,3){}
\node(f)(8,3){}
\drawedge(0,p){}
\drawedge(p,n-2){}
\drawedge(n-2,n-1){}
\drawloop[loopangle=270](n-1){}
\drawedge[curvedepth=.5](q1,n-2){}
\drawedge[curvedepth=.6,sxo=-.5,exo=1.5](qdots,n-2){}
\drawedge[curvedepth=0](qv,n-2){}
\drawedge[curvedepth=-.5](r1,n-1){}
\drawedge[curvedepth=-.6,sxo=-.5,exo=1.5](rdots,n-1){}
\drawedge(ru,n-1){}
\drawloop(f){}
\end{picture}
\begin{picture}(28,15)(0,-2)
\node[Nframe=n](name)(0,13){\normalsize}
\node(0')(2,10){0}\imark(0')
\node(p')(14,10){}
\node(n-1')(26,0){-}
\node(n-2')(26,10){-}\rmark(n-2')
\node(q1')(17,7){}
\node[Nframe=n,Nh=2,Nw=2,Nmr=1](qdots')(20,7){}
\node(qv')(23,7){}
\node(r1')(17,3){}
\node[Nframe=n](rdots')(20,3){}
\node(ru')(23,3){}
\node(f')(8,3){}
\drawedge[curvedepth=3,linecolor=red,dash={.5 .25}{.25}](0',n-2'){}
\drawedge(n-2',n-1'){}
\drawloop[loopangle=270](n-1'){}
\drawedge[linecolor=red,dash={.5 .25}{.25}](p',f'){}
\drawloop(f){}
\drawedge[curvedepth=-1.2,linecolor=red,dash={.5 .25}{.25}](q1',qdots'){}
\drawedge[curvedepth=-1.2,linecolor=red,dash={.5 .25}{.25}](qdots',qv'){}
\drawedge[curvedepth=-1.2,linecolor=red,dash={.5 .25}{.25}](qv',q1'){}
\drawedge[curvedepth=-.8,exo=.5,linecolor=red,dash={.5 .25}{.25}](r1',qv'){}
\drawedge[curvedepth=-.5,exo=.5,linecolor=red,dash={.5 .25}{.25}](rdots',qv'){}
\drawedge[linecolor=red,dash={.5 .25}{.25}](ru',qv'){}
\end{picture}\end{center}
\caption{Subcase~3.4.1.}\label{fig:subcase3.4.1}
\end{figure}

We observe the following properties:
\begin{enumerate}
\item[(a)]  is a~colliding pair focused by  to .
This is the only colliding pair that is focused by  to a~fixed point.

\item[(c)]  contains exactly one cycle, namely .
\end{enumerate}

\textit{External injectivity}:
Observe that all states in the unique cycle have in-degree 1 except possibly .
Thus, no colliding pair of states is focused to the smallest state  in the cycle.
This distinguishes  from the transformations of Case~3.1.

Since  has a~cycle, it is different from the transformations of Subcase~3.2.1 and Subcase~3.2.3.
Also,  is different from the transformations of Subcase~3.2.2, Subcase~3.2.4, and Case~3.3, which do not focus a~colliding pair to a~fixed point, because the orbits from their Properties~(b) do not have a~fixed point.

\textit{Internal injectivity}:
Let  be any transformation that fits in this subcase and results in the same ; we will show that .
By~(c), we know that  for all .
Then all states mapped by  to  must be , hence  for all .
By~(a) and since the fixed point is distinguished in the colliding pair, we obtain that  and .
We know that , ,  and  for all .
Since the other transitions in  are defined exactly as in  and , we have .

\textbf{Subcase~3.4.2}: .\\
We have . Let  be the transformation illustrated in Fig.~\ref{fig:subcase3.4.2} and defined by
\begin{center}
  , ,\\
  ,\\
   for ,\\
   for other states .
\end{center}
\begin{figure}[ht]
\unitlength 10pt\small
\gasset{Nh=2.5,Nw=2.5,Nmr=1.25,ELdist=0.3,loopdiam=1.5}
\begin{center}\begin{picture}(28,14)(0,-2)
\node[Nframe=n](name)(0,13){\normalsize}
\node(0)(2,10){0}\imark(0)
\node(p)(14,10){}
\node(n-1)(26,0){-}
\node(n-2)(26,10){-}\rmark(n-2)
\node(q1)(17,7){}
\node(r1)(17,3){}
\node[Nframe=n](rdots)(20,3){}
\node(ru)(23,3){}
\node(f)(8,3){}
\drawedge(0,p){}
\drawedge(p,n-2){}
\drawedge(n-2,n-1){}
\drawloop[loopangle=270](n-1){}
\drawedge[curvedepth=.3](q1,n-2){}
\drawedge[curvedepth=-.5](r1,n-1){}
\drawedge[curvedepth=-.6,sxo=-.5,exo=1.5](rdots,n-1){}
\drawedge[curvedepth=-0](ru,n-1){}
\drawloop(f){}
\end{picture}
\begin{picture}(28,15)(0,-2)
\node[Nframe=n](name)(0,13){\normalsize}
\node(0')(2,10){0}\imark(0')
\node(p')(14,10){}
\node(n-1')(26,0){-}
\node(n-2')(26,10){-}\rmark(n-2')
\node(q1')(17,7){}
\node(r1')(17,3){}
\node[Nframe=n](rdots')(20,3){}
\node(ru')(23,3){}
\node(f')(8,3){}
\drawedge[curvedepth=3,linecolor=red,dash={.5 .25}{.25}](0',n-2'){}
\drawedge(n-2',n-1'){}
\drawloop[loopangle=270](n-1'){}
\drawedge[linecolor=red,dash={.5 .25}{.25}](p',f'){}
\drawloop(f'){}
\drawedge[curvedepth=-4.5,linecolor=red,dash={.5 .25}{.25}](r1',p'){}
\drawedge[curvedepth=-4.5,linecolor=red,dash={.5 .25}{.25}](rdots',p'){}
\drawedge[curvedepth=-4.5,eyo=.5,linecolor=red,dash={.5 .25}{.25}](ru',p'){}
\drawedge[linecolor=red,dash={.5 .25}{.25}](q1',f'){}
\end{picture}\end{center}
\caption{Subcase~3.4.2.}\label{fig:subcase3.4.2}
\end{figure}

We observe the following properties:
\begin{enumerate}
\item[(a)]  is a~colliding pair focused by  to .

\item[(b)] All states from  whose mapping is different in  and  belong to the same orbit of the fixed point .

\item[(c)]  does not contain any cycles.
\end{enumerate}

\textit{External injectivity}:
Since  does not have any cycles, it is different from the transformations of Case~3.1, Subcase~3.2.2, Subcase~3.2.4, Case~3.3, and Subcase~3.4.1.

Let  be a~transformation that fits in Subcase~3.2.1 and results in the same .
By Lemma~\ref{lem:orbits}, the orbits from Properties~(b) for both  and  must be the same, so the subsubcase for  is~(i), and necessarily .
We know that the states  and  are mapped to  and have in-degree at least 1.
This contradicts with that  and  are the only two states mapped to , and  has in-degree 0.

Let  be a~transformation that fits in Subcase~3.2.3 and results in the same .
By Lemma~\ref{lem:orbits}, the orbits from Properties~(b) for both  and  must be the same, so the subsubcase for  is~(i), and necessarily .
So , but this is a~colliding pair because of , which is focused to  by ; hence,  and  cannot be both present in .

\textit{Internal injectivity}:
Let  be any transformation that fits in this subcase and results in the same ; we will show that .
Lemma~\ref{lem:orbits}, the orbits from Properties~(b) for both  and  must be the same, so we obtain that .
So we have .
Since  and  have in-degree 0, and  and  have in-degree at least , we have  and .
Then  for all , as these are precisely the states mapped to .
We know that , , , and  for all .
Since the other transitions in  are defined exactly as in  and , we have .

\textbf{Subcase~3.4.3}: .\\
Let  be the transformation illustrated in Fig.~\ref{fig:subcase3.4.3} and defined by
\begin{center}
  , ,\\
  ,\\
   for ,\\
   for other states .
\end{center}
\begin{figure}[ht]
\unitlength 10pt\small
\gasset{Nh=2.5,Nw=2.5,Nmr=1.25,ELdist=0.3,loopdiam=1.5}
\begin{center}\begin{picture}(28,14)(0,-2)
\node[Nframe=n](name)(0,9){\normalsize}
\node(0)(2,7){0}\imark(0)
\node(p)(14,7){}
\node(n-1)(26,0){-}
\node(n-2)(26,7){-}\rmark(n-2)
\node(f)(8,0){}
\node(r1)(17,3){}
\node[Nframe=n](rdots)(20,3){}
\node(ru)(23,3){}
\drawedge(0,p){}
\drawedge(p,n-2){}
\drawedge(n-2,n-1){}
\drawloop[loopangle=270](n-1){}
\drawedge[curvedepth=-.5](r1,n-1){}
\drawedge[curvedepth=-.6,sxo=-.5,exo=1.5](rdots,n-1){}
\drawedge[curvedepth=0](ru,n-1){}
\drawloop(f){}
\end{picture}
\begin{picture}(28,12)(0,-1)
\node[Nframe=n](name)(0,9){\normalsize}
\node(0')(2,7){0}\imark(0')
\node(p')(14,7){}
\node(n-1')(26,0){-}
\node(n-2')(26,7){-}\rmark(n-2')
\node(f')(8,0){}
\node(r1')(17,3){}
\node[Nframe=n](rdots')(20,3){}
\node(ru')(23,3){}
\drawedge[curvedepth=3,linecolor=red,dash={.5 .25}{.25}](0',n-2'){}
\drawedge(n-2',n-1'){}
\drawloop[loopangle=270](n-1'){}
\drawedge[linecolor=red,dash={.5 .25}{.25}](p',f'){}
\drawloop(f'){}
\drawedge[linecolor=red,dash={.5 .25}{.25}](r1',p'){}
\drawedge[curvedepth=2.7,sxo=.5,eyo=.5,linecolor=red,dash={.5 .25}{.25}](rdots',f'){}
\drawedge[curvedepth=3,linecolor=red,dash={.5 .25}{.25}](ru',f'){}
\end{picture}\end{center}
\caption{Subcase~3.4.3.}\label{fig:subcase3.4.3}
\end{figure}

We observe the following properties:
\begin{enumerate}
\item[(a)]  is a~colliding pair focused by  to .

\item[(b)] All states from  whose mapping is different in  and  belong to the same orbit of the fixed point , which has in-degree .

\item[(c)]  does not contain any cycles.
\end{enumerate}

\textit{External injectivity}:
Since  does not have any cycles, it is different from the transformations of Case~3.1, Subcase~3.2.2, Subcase~3.2.4, Case~3.3, and Subcase~3.4.1.

Let  be a~transformation that fits in Subcase~3.2.1 and results in the same .
By Lemma~\ref{lem:orbits}, the orbits from Properties~(b) for both  and  must be the same, so the subsubcase for  must be~(i), and necessarily .
We know that the states  and  are mapped by  to  and have in-degree at least , because  and in subsubcase~(i) there exists some  mapped by  to .
On the other hand, all states mapped to  (except  itself) are  and , where all the states  have in-degree 0, which yields a~contradiction.

To distinguish  from the transformations of Subcase~3.2.3 and of Subcase~3.4.2, observe that if they focus a~colliding pair to a~fixed point, then this fixed point have in-degree 3, but  focuses a~colliding pair to the fixed point  of in-degree at least 4.

\textit{Internal injectivity}:
Let  be any transformation that fits in this subcase and results in the same ; we will show that .
By Lemma~\ref{lem:orbits}, the orbits from Properties~(b) for both  and  must be the same, so we obtain that .
We have , as this is the unique state of in-degree 1 that is mapped to .
Then  as this is the unique state mapped to .
All states of in-degree 0 that mapped to  are precisely ; hence  for all .
We know that , , and  for all .
Since the other transitions in  are defined exactly as in  and , we have .

\textbf{Case~3.5}: .\\
Let  be all the states from  such that .
We split the case into the following three subcases covering all possibilities for :

\textbf{Subcase~3.5.1}:  and  has in-degree .\\
Let  be the transformation illustrated in Fig.~\ref{fig:subcase3.5.1} and defined by
\begin{center}
  , ,\\
   for ,\\
   for the other states .
\end{center}
\begin{figure}[ht]
\unitlength 10pt\small
\gasset{Nh=2.5,Nw=2.5,Nmr=1.25,ELdist=0.3,loopdiam=1.5}
\begin{center}\begin{picture}(28,11)(0,-1)
\node[Nframe=n](name)(0,9){\normalsize}
\node(0)(2,7){0}\imark(0)
\node(p)(8,7){}
\node[Nframe=n](pdots)(14,7){}
\node(pt^k)(20,7){}
\node(n-2)(26,7){-}\rmark(n-2)
\node(n-1)(26,2){-}
\drawedge(0,p){}
\drawedge(p,pdots){}
\drawedge(pdots,pt^k){}
\drawedge(pt^k,n-2){}
\drawedge(n-2,n-1){}
\drawloop[loopangle=270](n-1){}
\end{picture}
\begin{picture}(28,10)(0,0)
\node[Nframe=n](name)(0,9){\normalsize}
\node(0')(2,7){0}\imark(0')
\node(p')(8,7){}
\node[Nframe=n](pdots')(14,7){}
\node(pt^k')(20,7){}
\node(n-2')(26,7){-}\rmark(n-2')
\node(n-1')(26,2){-}
\drawedge[curvedepth=3,linecolor=red,dash={.5 .25}{.25}](0',n-2'){}
\drawedge(n-2',n-1'){}
\drawloop[loopangle=270](n-1'){}
\drawloop[loopangle=270,linecolor=red,dash={.5 .25}{.25}](p'){}
\drawedge[linecolor=red,dash={.5 .25}{.25}](pdots',p'){}
\drawedge[linecolor=red,dash={.5 .25}{.25}](pt^k',pdots'){}
\end{picture}\end{center}
\caption{Subcase~3.5.1.}\label{fig:subcase3.5.1}
\end{figure}

We observe the following properties:
\begin{enumerate}
\item[(a)] Pair  is a~colliding pair focused by  to .

\item[(b)] All states from  whose mapping is different in  and  belong to the orbit of the fixed point , which has in-degree 2.
\end{enumerate}

\textit{External injectivity}:
Since the orbits from Properties~(b) for the transformations of Case~3.1, Subcase~3.2.2, Subcase~3.2.4, and Case~3.3 have cycles, and the orbit from~(b) of this subcase has a~fixed point, by Lemma~\ref{lem:orbits}  is different from these transformations.
Similarly, the orbits from Properties~(b) for the transformations of Subcase~3.2.1, Subcase~3.2.3, Subcase~3.4.2, and Subcase~3.4.3 have a~fixed point of in-degree at least 3 or they are orbits of , so by Lemma~\ref{lem:orbits}  is different from these transformations.

Let  be a~transformation that fits in Subcase~3.4.1 and results in the same .
Since  is the only colliding pair that is focused to a~fixed point, it must be that  and .
States  form a~cycle in , and since it is in a~different orbit from that from~(b), the cycle must be also present in .
Hence, states  collide with , and, in particular,  is a~colliding pair focused to  by , and so  and  cannot be both present in .

\textit{Internal injectivity}:
This follows exactly in the same way as in Case~2.2.

\textbf{Subcase~3.5.2}:  and  has in-degree at least .\\
Let  be the smallest state such that  and .\\
Let  be the transformation illustrated in Fig.~\ref{fig:subcase3.5.2} and defined by
\begin{center}
  , ,\\
  ,\\
   for ,\\
   for the other states .
\end{center}
\begin{figure}[ht]
\unitlength 10pt\small
\gasset{Nh=2.5,Nw=2.5,Nmr=1.25,ELdist=0.3,loopdiam=1.5}
\begin{center}\begin{picture}(28,11)(0,-1)
\node[Nframe=n](name)(0,9){\normalsize}
\node(0)(2,7){0}\imark(0)
\node(p)(8,7){}
\node[Nframe=n](pdots)(14,7){}
\node(pt^k)(20,7){}
\node(y)(20,2){}
\node(n-2)(26,7){-}\rmark(n-2)
\node(n-1)(26,2){-}
\drawedge(0,p){}
\drawedge(p,pdots){}
\drawedge(pdots,pt^k){}
\drawedge(pt^k,n-2){}
\drawedge(y,pt^k){}
\drawedge(n-2,n-1){}
\drawloop[loopangle=270](n-1){}
\end{picture}
\begin{picture}(28,11)(0,-1)
\node[Nframe=n](name)(0,9){\normalsize}
\node(0')(2,7){0}\imark(0')
\node(p')(8,7){}
\node[Nframe=n](pdots')(14,7){}
\node(pt^k')(20,7){}
\node(y')(20,2){}
\node(n-2')(26,7){-}\rmark(n-2')
\node(n-1')(26,2){-}
\drawedge[curvedepth=3,linecolor=red,dash={.5 .25}{.25}](0',n-2'){}
\drawedge(n-2',n-1'){}
\drawloop[loopangle=270](n-1'){}
\drawedge[linecolor=red,dash={.5 .25}{.25}](y',n-1'){}
\drawedge[linecolor=red,dash={.5 .25}{.25}](p',y'){}
\drawedge[linecolor=red,dash={.5 .25}{.25}](pdots',p'){}
\drawedge[linecolor=red,dash={.5 .25}{.25}](pt^k',pdots'){}
\end{picture}\end{center}
\caption{Subcase~3.5.2.}\label{fig:subcase3.5.2}
\end{figure}

We observe the following properties:
\begin{enumerate}
\item[(a)] Pair  is a~colliding pair focused by  to .

\item[(b)] All states from  whose mapping is different in  and  belong to the tree of  in , where  is mapped to .
\end{enumerate}

\textit{External injectivity}:
Since the orbits from Properties~(b) for the transformations of Case~3.1, Subcase~3.2.2, Subcase~3.2.4, and Case~3.3 have cycles, and the orbit from~(b) of this subcase is the orbit of , by Lemma~\ref{lem:orbits}  is different from these transformations.
Similarly, the orbits from Properties~(b) for the transformations of Subcase~3.2.1~(i), Subcase~3.2.3~(i), Subcase~3.4.2, Subcase~3.4.3, and Subcase~3.5.1 have a~fixed point from , so by Lemma~\ref{lem:orbits}  is different from these transformations.
Since the transformations of Subcase~3.4.1 focus a~colliding pair to a~fixed point, they are also different from .

Let  be a~transformation from Subcase~3.2.1~(ii) that results in the same .
By Lemma~\ref{lem:orbits}, the trees from Properties~(b) for both  and  must be the same, and so it must be that .
First observe that , because otherwise  and  for some  would form a~colliding pair because of , which is focused by  to .
So  must be another state mapped by  to , and so also by .
It follows that all states  are mapped by  in the same way as by .
But then  and , so the colliding pair  is focused by , which yields a~contradiction.

Let  be a~transformation from Subcase~3.2.3~(ii) that results in the same .
By Lemma~\ref{lem:orbits}, the trees from Properties~(b) for both  and  must be the same, and so .
But  and  are the only states mapped to  in , and they both have in-degree 0, whereas  is also mapped to  in  and has in-degree 1, which yields a~contradiction.

\textit{Internal injectivity}:
Let  be any transformation that fits in this subcase and results in the same ; we will show that .
By Lemma~\ref{lem:orbits}, the trees from Property~(b) must be the same, so .
Since in  all the states besides  that are mapped to  are also mapped to  in , it follows that  and .
Note that for , , the distance in  from  and from  to  is .
Hence, if  then .

\textbf{Subcase~3.5.3}: .\\
We define  to be the largest distance in  from a~state  to some state , that is,

Notice that , because otherwise, when  for some , the state  would be colliding with  and the pair  would be focused by  to  (see also the observation at the beginning of Supercase~3).
Define

that is,  is the smallest state w.r.t. the ordering of the states among the furthest states from some .
Let  be that state , which is the first state  in the path from .
Notice that if all  have in-degree 0, then  and .

Let  be the transformation illustrated in Fig.~\ref{fig:subcase3.5.3} and defined by
\begin{center}
  , ,\\
   for ,\\
   for ,\\
  ,\\
  , for the other states .
\end{center}
\begin{figure}[ht]
\unitlength 10pt\small
\gasset{Nh=2.5,Nw=2.5,Nmr=1.25,ELdist=0.3,loopdiam=1.5}
\begin{center}\begin{picture}(28,15)(0,-1)
\node[Nframe=n](name)(0,13){\normalsize}
\node(0)(2,11){0}\imark(0)
\node(p)(8,11){}
\node[Nframe=n](pdots)(14,11){}
\node(pt^k)(20,11){}
\node(q1)(14,6){}
\node[Nframe=n](qdots1)(16.5,6){}
\node(qi)(19,6){}
\node[Nframe=n](qdots2)(21.5,6){}
\node(qv)(24,6){}
\node(x)(8,2){}
\node[Nframe=n](xdots)(14,2){}
\node(n-2)(26,11){-}\rmark(n-2)
\node(n-1)(26,2){-}
\drawedge(0,p){}
\drawedge(p,pdots){}
\drawedge(pdots,pt^k){}
\drawedge(pt^k,n-2){}
\drawedge(n-2,n-1){}
\drawloop[loopangle=270](n-1){}
\drawedge(x,xdots){}
\drawedge[curvedepth=-2,exo=.2](xdots,qi){}
\drawedge[curvedepth=.6](q1,n-2){}
\drawedge[curvedepth=.3,sxo=-1](qdots1,n-2){}
\drawedge[curvedepth=.2,sxo=-.5](qi,n-2){}
\drawedge[curvedepth=.1](qdots2,n-2){}
\drawedge[curvedepth=0](qv,n-2){}
\end{picture}
\begin{picture}(28,14)(0,0)
\node[Nframe=n](name)(0,14){\normalsize}
\node(0')(2,11){0}\imark(0')
\node(p')(8,11){}
\node[Nframe=n](pdots')(14,11){}
\node(pt^k')(20,11){}
\node(q1')(14,6){}
\node[Nframe=n,Nh=2,Nw=2,Nmr=1](qdots1')(16.5,6){}
\node(qi')(19,6){}
\node[Nframe=n,Nh=2,Nw=2,Nmr=1](qdots2')(21.5,6){}
\node(qv')(24,6){}
\node[Nframe=n](xdots')(14,2){}
\node(x')(8,2){}
\node(n-2')(26,11){-}\rmark(n-2')
\node(n-1')(26,2){-}
\drawedge[curvedepth=3,linecolor=red,dash={.5 .25}{.25}](0',n-2'){}
\drawedge(n-2',n-1'){}
\drawloop[loopangle=270](n-1'){}
\drawedge[linecolor=red,dash={.5 .25}{.25}](pdots',p'){}
\drawedge[linecolor=red,dash={.5 .25}{.25}](pt^k',pdots'){}
\drawedge[linecolor=red,dash={.5 .25}{.25}](p',x'){}
\drawedge(x',xdots'){}
\drawedge[curvedepth=-2,exo=.2](xdots',qi'){}
\drawedge[curvedepth=-1.5,linecolor=red,dash={.5 .25}{.25}](q1',qdots1'){}
\drawedge[curvedepth=-1.5,linecolor=red,dash={.5 .25}{.25}](qdots1',qi'){}
\drawedge[curvedepth=-1.5,linecolor=red,dash={.5 .25}{.25}](qi',qdots2'){}
\drawedge[curvedepth=-1.5,linecolor=red,dash={.5 .25}{.25}](qdots2',qv'){}
\drawedge[curvedepth=-2,linecolor=red,dash={.5 .25}{.25}](qv',q1'){}
\end{picture}\end{center}
\caption{Subcase~3.5.3.}\label{fig:subcase3.5.3}
\end{figure}

We observe the following properties:
\begin{enumerate}
\item[(a)]  is a~colliding pair focused by  to .

\item[(b)] All states from  whose mapping is different in  and  belong to the same orbit of a~cycle (if ) or a~fixed point (if ).

\item[(d)] Every longest path in  from some state not in a~cycle to the first reachable  contain both  and , and this  is .

\textit{Proof}: If such a~path contains , then it does not contain , and so would exist also in .
But then, by the choice of , its length could be at most , whereas the path from  to  is of length .
Thus, every such a~path contain  and so , since  has in-degree 1, and ends in .
\end{enumerate}

\textit{External injectivity}:
Let  be a~transformation that fits in Case~3.1 and results in the same .
By Lemma~\ref{lem:orbits}, the orbits from Properties~(b) for both  and  are the same.
Let  be the state mapped to  in the path in  from  to .
If , then by the construction of  in Case~3.1, all states in the tree of  are mapped in  in the same way as in .
Hence,  is focused by  to , which yields a~contradiction.
If , then , since to  only the states  are mapped, which have in-degree 0, and  has in-degree 1.
Hence , , and  for some . However,  is a~colliding pair because of  that is focused by  to , which yields a~contradiction.

Let  be a~transformation that fits in Subcase~3.2.1 and results in the same .
By Lemma~\ref{lem:orbits}, the orbits from Properties~(b) for both  and  are the same, so necessarily the subsubcase for  must be~(i) and .
Since  has in-degree  in , it cannot be , because  has in-degree at most 2 (only  is mapped to  and, in the case , additionally a~state ).
Thus  and  are mapped in  in the same way as in .
But  is a~colliding pair because of , which is focused by  to , which yields a~contradiction.

Let  be a~transformation that fits in Subcase~3.2.2 and results in the same .
By Lemma~\ref{lem:orbits}, the orbits from Properties~(b) for both  and  are the same, so necessarily  and .
Observe that among the states mapped by  to a~state in the cycle , only  can have in-degree larger than 0.
It follows that , and we obtain a~contradiction exactly as for Case~3.1.

Let  be a~transformation that fits in Subcase~3.2.3 and results in the same .
By Lemma~\ref{lem:orbits}, the orbits from Properties~(b) for both  and  are the same, so necessarily the subsubcase for  must be~(i) and .
Since  has in-degree 1 in , it has in-degree 0 in , so it cannot be .
Therefore , but then we obtain a~contradiction exactly as for Case~3.1.

Let  be a~transformation that fits in Subcase~3.2.4 and results in the same .
By Lemma~\ref{lem:orbits}, the orbits from Properties~(b) for both  and  are the same, so necessarily  is the cycle formed by all states .
But  is a~colliding pair because of , which is focused by  to ; this yields a~contradiction.

Let  be a~transformation that fits in Case~3.3 and results in the same .
By Lemma~\ref{lem:orbits}, the orbits from Properties~(b) for both  and  are the same, so necessarily  and .
Then , and again we obtain a~contradiction exactly as for Case~3.1.

Let  be a~transformation that fits in Subcase~3.4.1 and results in the same .
In  there is exactly one orbit of a~fixed point from  and exactly one orbit of a~cycle.
But neither of them cannot be the orbit from~(b) of this subcase, since  and states  have in-degree 0 in  so they cannot be ; this yields a~contradiction.

Let  be a~transformation that fits in either Subcase~3.4.2 or Subcase~3.4.3 and results in the same .
By Lemma~\ref{lem:orbits}, the orbits from Properties~(b) for both  and  are the same, so necessarily  and .
Then , as  is the only state with non-zero in-degree in  that is mapped to .
So also .
But there is another state mapped by  to  ( or , depending on the subcase), and it is mapped to  also by .
However, this contradicts that  has in-degree 0 in .

Let  be a~transformation that fits in Subcase~3.5.1 and results in the same .
By Lemma~\ref{lem:orbits}, the orbits from Properties~(b) for both  and  are the same, so necessarily  and .
Consider the following path , which contains all the states from  that are mapped differently in  and :

Consider the second path in , which contains all the states from  that are mapped differently in  and :

Let  be the first common state in these paths;  exists since both paths end up in .
Note that  reverses the second path.
We consider all possibilities for , depending on where it occurs in the first chain:
\begin{itemize}
\item .
Then  for some , so  is a~colliding pair because of , which is focused by  to .
\item  for .
Then  so , since  is in the second path and  reverses it.
Also, , since  does not belong to the second path.
But then  is a~colliding pair because of , which is focused by  to .
\item .
Since in  only state  is mapped to  and , it must be that , as otherwise .
Therefore .
But  has in-degree 1 in  from the conditions of Subcase~3.5.1, so it has in-degree 0 in , which yields a~contradiction with in-degree 1 of  in .
\item  is a~state in the path in  from  to .
Then  for some .
Remind that , so .
Since , the distance in  from  to  is at least .
It follows that there is a~state  from the first chain such that .
However, we also know that , hence  cannot be in .
\end{itemize}
We obtained a~contradiction in every case, so  and  cannot be both in .

Let  be a~transformation that fits in Subcase~3.5.2 and results in the same .
However, by Lemma~\ref{lem:orbits}, the orbits from Properties~(b) for both  and  must be the same, but for  this is an~orbit of .

\textit{Internal injectivity}:
Let  be any transformation that fits in this subcase and results in the same ; we will show that .
By Lemma~\ref{lem:orbits}, the orbits from~(b) must be the same for both  and , hence  and the sets of  states are the same.

By~(d), both  and  are in every longest path to the first reachable , so .
Without loss of generality, state  occurs not later than , that is, we have  for some .
Since the path from  to  is the same in both  and , we have  for all .

Consider the following path  in :


First suppose that  does not contain .
Then also no state  for  would be in this path:
let  be such the state with the smallest ; then  would also be in this path, which is a~contradiction.
Hence, by the construction of , this path is also present in .
By the choice of , the distance in  from  to  is not smaller than the length of this path.
So we have , which yields  (because ).
Now observe that since in  state  is reachable from , we have the following path in :

where  is the smallest possible.
Then, by the construction of , we have the following path in :

This path has length at least .
Hence, there exists a~state  in this path such that .
This means that  is a~colliding pair because of , which is focused by  to .

There remains the case where  contain .
Since  must occur before  in , we have  for some .

We claim that  for all , which also implies .
We use induction on :
This holds for , and also for , because  and the in-degree of  is 1 in .
For  assume that  for all .
Suppose for a~contradiction that .
If , then ,
because in  among the states mapped to , only  is mapped differently than in .
Then, however,  is a~colliding pair because of  that is focused by  to .
If , then, dually, ,
because in , among the states mapped to , only  is mapped differently than in .
Then, however,  is a~colliding pair because of  that is focused by  to .
Hence, the claim follows.

Suppose that .
Since the path in  from  to  occurs also in  and , and is of length , we have .
Note that .
So there exists a~state .
But this state collides with  because of , and the pair  is focused by  to .

Finally, if , then  for all , and  for all .
Since the other transitions in  are defined exactly as in  and , we have .
\end{proof}

\section{Uniqueness of maximal semigroups}

Here we show that  for  and  for  (whereas  for ) have not only the maximal sizes, but are also the unique largest semigroups up to renaming the states in a~minimal DFA  of a~bifix-free language.

\begin{theorem}\label{thm:uniqueness}
If , and the transition semigroup  of a~minimal DFA  of a~bifix-free language has at least one colliding pair, then

\end{theorem}
\begin{proof}
Assume that there is a~colliding pair  with .
Since , there must be at least three states .
Let  be the transformation illustrated in Fig.~\ref{fig:uniqueness} and defined by:
\begin{center}
, , , , ,\\
, for the other states .
\end{center}
\begin{figure}[ht]
\unitlength 10pt\small
\gasset{Nh=2.5,Nw=2.5,Nmr=1.25,ELdist=0.3,loopdiam=1.5}
\begin{center}\begin{picture}(28,8)(0,-2)
\node(0)(2,0){0}\imark(0)
\node(p1)(6,4){}
\node(p2)(10,0){}
\node(r1)(14,4){}
\node(r2)(18,4){}
\node(r3)(22,4){}
\node(n-1)(26,0){-}
\node(n-2)(26,4){-}\rmark(n-2)
\drawedge[curvedepth=-3](0,n-1){}
\drawedge(n-2,n-1){}
\drawloop[loopangle=270](n-1){}
\drawedge(p1,p2){}
\drawloop(p2){}
\drawedge(r1,p2){}
\drawedge[curvedepth=1](r2,r3){}
\drawedge[curvedepth=1](r3,r2){}
\end{picture}\end{center}
\caption{The transformation  in the proof of Theorem~\ref{thm:uniqueness}.}\label{fig:uniqueness}
\end{figure}

Let  be the injective function from the proof of Theorem~\ref{thm:bifix-free_upper_bound}.
We will show that  does not belong to .

Since  is focused by  to ,  is different from the transformations of Supercase~1.
Since , it is also different from the transformations of Supercase~3.

To see that it is different from all transformations of Supercase~2, notice that only the transformations of Case~2.1, Case~2.3, Subcase~2.4.2, Subcase~2.5.1, and Subcase 2.5.2 have a~cycle.
The transformations of Case~2.1, Case~2.3, and Subcase~2.4.2 have a~cycle with a~state with in-degree at least 2, whereas the single cycle  in  has both states of in-degree 1.
In the transformations of Subcase~2.5.1 and Subcase~2.5.2, there is only one fixed point from , and it has in-degree 2, whereas the single fixed point  in  has in-degree 3.

Thus, since  is injective and ,  but , it follows that , so .
\end{proof}

\begin{corollary}
For , the transition semigroup  is the unique largest transition semigroup of a~minimal DFA of a~bifix-free language.
\end{corollary}
\begin{proof}
From Theorem~\ref{thm:uniqueness}, a~transition semigroup that has a~colliding pair cannot be largest.
From Proposition~\ref{pro:Wbf_unique},  is the unique maximal transition semigroup that does not have colliding pairs of states.
\end{proof}

The following theorem solves the remaining cases of small semigroups:
\begin{theorem}
For , the largest transition semigroup of minimal DFAs of bifix-free languages is  and it is unique.
For , the largest transition semigroup of minimal DFAs of bifix-free languages is  and it is unique.
For ,  is the unique largest transition semigroup of minimal DFAs of bifix-free languages.
\end{theorem}
\begin{proof}
We have verified this with the help of computation, basing on the idea of conflicting pairs of transformations from~\cite[Theorem~20]{BLY12}.
The idea of the algorithm is described as follows, and the program is available at~\cite{SzWi18SyntacticComplexityOfBifixFreeArxiv}.

We say that two transformations  \emph{conflicts} if they cannot be both present in the transition semigroup of a~minimal DFA  of a~bifix-free language, or they imply that all pairs of states from  are either colliding or focused.
In the latter case, by Proposition~\ref{pro:Vbf_unique} and Proposition~\ref{pro:Wbf_unique} we know that a~transition semigroup containing these transformations must be a~subsemigroup of  or , respectively.
Hence, we know that two conflicting transformations cannot be present in a~transition semigroup of size at least  which is different from  and .
Given a~set of transformations , the \emph{graph of conflicts} is the graph , where there is an~edge  if and only if  conflicts with .

Given an~, our algorithm is as follows:
We keep a~subset  of transformations that can potentially be present in a~largest transition semigroup.
Starting with , we iteratively compute , where  is obtained from  by removing some transformations.
This is done for  until we obtain .
If  then the algorithm fails.

Given , we compute  by checking every transformation  and estimating how many pairwise non-conflicting transformations can we add to the set .
Let  be the set of all transformations that do not conflict with .
The maximal number of pairwise non-conflicting transformations in  is the size of a~largest independent set in .
We only compute an~upper bound for it, since the problem is computationally hard.
Let  be a~maximal matching in the graph of conflicts of ; this can be computed by a~simple greedy algorithm in  time.
Then  is an~upper bound for the size of a~largest independent set in , and so  is an~upper bound for the cardinality of a~maximal transition semigroup containing  that is different from  and .
If this bound is smaller than , then we do not take  into ; otherwise we keep .

When , all transformations are rejected, which means that there are no transformations that can be present in a~transition semigroup of size at least  which is different from  and , so there are no such semigroups.

For , two iterations were sufficient, and we obtained , , and ; the computation took less than one minute.
\end{proof}

Since the largest transition semigroups are unique, from Propositions~\ref{pro:Wbf_alphabet_lower_bound} and~\ref{pro:Vbf_alphabet_lower_bound} we infer the sizes of the alphabets required in order to meet the bound for the syntactic complexity.
\begin{corollary}
To meet the bound for the syntactic complexity of bifix-free languages,  letters are required and sufficient for , and  letters are required and sufficient for .
\end{corollary}



\section{Conclusions}

We have solved the problem of syntactic complexity of bifix-free languages and identified the largest semigroups for every number of states .
In the main theorem, we used the method of injective function (cf.~\cite{BrSz14a,BrSz15SyntacticComplexityOfSuffixFree}) with new techniques and tricks for ensuring injectivity (in particular, Lemma~\ref{lem:orbits} and the constructions in Supercase~3). This stands as a~universal method for solving similar problems concerning the maximality of semigroups.
Our proof required an~extensive analysis of 23 (sub)cases and much more complicated injectivity arguments than those for suffix-free (12 cases), left ideals (5 subcases) and two-sided ideals (8 subcases).
The difficulty of applying the method grows quickly when the characterization of the class of languages gets more involved.

It may be surprising that we need a~witness with  (for ) letters to meet the bound for syntactic complexity of bifix-free languages, whereas in the case of prefix- and suffix-free languages only  and five letters suffice, respectively (see \cite{BLY12,BrSz15SyntacticComplexityOfSuffixFree}).

Finally, our results enabled establishing the existence of most complex bifix-free languages (\cite{FeSz17ComplexityOfBifixFree,FeSz18ComplexityOfBifixFree}).

\section*{Acknowledgments}

We are grateful to anonymous reviewers for their careful reading of the proof.
This work was supported in part by the National Science Centre, Poland under project numbers 2014/15/B/ST6/00615 and 2017/25/B/ST6/01920 (Marek Szyku{\l}a).

\bibliographystyle{plainurl}
\providecommand{\noopsort}[1]{}
\begin{thebibliography}{10}

\bibitem{BPR09}
J.~Berstel, D.~Perrin, and C.~Reutenauer.
\newblock {\em Codes and Automata}.
\newblock Cambridge University Press, 2009.

\bibitem{Brz10}
J.~A. Brzozowski.
\newblock Quotient complexity of regular languages.
\newblock {\em J. Autom. Lang. Comb.}, 15(1/2):71--89, 2010.

\bibitem{Brz13}
J.~A. Brzozowski.
\newblock In search of the most complex regular languages.
\newblock {\em Int. J. Found. Comput. Sc.}, 24(6):691--708, 2013.

\bibitem{BrLi15}
J.~A. Brzozowski and B.~Li.
\newblock Syntactic complexity of \mbox{}- and \mbox{}-trivial languages.
\newblock {\em Internat. J. Found. Comput. Sci.}, 16(3):547--563, 2005.

\bibitem{BLL12}
J.~A. Brzozowski, B.~Li, and D.~Liu.
\newblock Syntactic complexities of six classes of star-free languages.
\newblock {\em J. Autom. Lang. Comb.}, 17:83--105, 2012.

\bibitem{BLY12}
J.~A. Brzozowski, B.~Li, and Y.~Ye.
\newblock Syntactic complexity of \mbox{prefix-,} \mbox{{suffix-,}
  \mbox{bifix-,} and factor-free} regular languages.
\newblock {\em Theoret. Comput. Sci.}, 449:37--53, 2012.

\bibitem{BrSz14a}
J.~A. Brzozowski and M.~Szyku{\l}a.
\newblock Upper bounds on syntactic complexity of left and two-sided ideals.
\newblock In A.~M. Shur and M.~V. Volkov, editors, {\em DLT 2014}, volume 8633
  of {\em LNCS}, pages 13--24. Springer, 2014.

\bibitem{BrSz15Aperiodic}
J.~A. Brzozowski and M~Szyku{\l}a.
\newblock Large aperiodic semigroups.
\newblock {\em International Journal of Foundations of Computer Science},
  26(07):913--931, 2015.

\bibitem{BrSz15SyntacticComplexityOfSuffixFree}
J.~A. Brzozowski and M.~Szyku{\l}a.
\newblock Upper bound on syntactic complexity of suffix-free languages.
\newblock In J.~Shallit, editor, {\em DCFS 2015}, volume 9118 of {\em LNCS},
  pages 33--45. Springer, 2015.

\bibitem{BrTa14}
J.~A. Brzozowski and H.~Tamm.
\newblock Theory of \'atomata.
\newblock {\em Theoret. Comput. Sci.}, 539:13--27, 2014.

\bibitem{BrYe11}
J.~A. Brzozowski and Y.~Ye.
\newblock Syntactic complexity of ideal and closed languages.
\newblock In Giancarlo Mauri and Alberto Leporati, editors, {\em DLT 2011},
  volume 6795 of {\em LNCS}, pages 117--128. Springer, 2011.

\bibitem{FeSz17ComplexityOfBifixFree}
R.~Ferens and M.~Szyku{\l}a.
\newblock {Complexity of Bifix-Free Regular Languages}.
\newblock In {\em CIAA}, volume 10329 of {\em LNCS}, pages 270--287. Springer,
  2017.

\bibitem{FeSz18ComplexityOfBifixFree}
R.~Ferens and M.~Szyku{\l}a.
\newblock Complexity of bifix-free regular languages.
\newblock {\em Theoretical Computer Science}, 2018.

\bibitem{HoKo04}
M.~Holzer and B.~K\"onig.
\newblock On deterministic finite automata and syntactic monoid size.
\newblock {\em Theoret. Comput. Sci.}, 327:319--347, 2004.

\bibitem{IvNa14}
S.~Iv{\'{a}}n and J.~Nagy{-}Gy{\"{o}}rgy.
\newblock {On nonpermutational transformation semigroups with an application to
  syntactic complexity}.
\newblock \url{http://arxiv.org/abs/1402.7289}, 2014.

\bibitem{McSe71}
R.~McNaughton and S.~A. Papert.
\newblock {\em Counter-free automata (M.I.T. Research Monograph No. 65)}.
\newblock The MIT Press, 1971.

\bibitem{Myh57}
J.~Myhill.
\newblock Finite automata and representation of events.
\newblock {\em Wright Air Development Center Technical Report}, 57--624, 1957.

\bibitem{Pin97}
J.-E. Pin.
\newblock Syntactic semigroups.
\newblock In G.~Rozenberg and A.~Salomaa, editors, {\em Handbook of Formal
  Languages, vol.~1: Word, Language, Grammar}, pages 679--746. Springer, 1997.

\bibitem{SzWi17SyntacticComplexityOfBifixFree}
M.~Szyku{\l}a and J.~Wittnebel.
\newblock {Syntactic Complexity of Bifix-Free Languages}.
\newblock In {\em CIAA}, volume 10329 of {\em LNCS}, pages 76--88. Springer,
  2017.

\bibitem{SzWi18SyntacticComplexityOfBifixFreeArxiv}
M.~Szyku{\l}a and J.~Wittnebel.
\newblock {Syntactic complexity of bifix-free languages}.
\newblock \url{https://arxiv.org/abs/1604.06936}, 2018.

\bibitem{Yu01}
S.~Yu.
\newblock State complexity of regular languages.
\newblock {\em J. Autom. Lang. Comb.}, 6:221--234, 2001.

\end{thebibliography}
\newpage
\section*{Appendix: Map of the (sub)cases in the proof of Theorem~\ref{thm:bifix-free_upper_bound}}

{\setstretch{1.05}
\begin{enumerate}[leftmargin=*,widest=\textbf{Supercase~1}]
\item[\textbf{Supercase~1}:]
.
\end{enumerate}

Let , and  be the largest integer such that .
\begin{enumerate}[leftmargin=*,widest=\textbf{Supercase~1}]
\item[\textbf{Supercase~2}:]
 and .
\begin{enumerate}[leftmargin=*,widest=\textbf{Case~2.1}]
\item[\textbf{Case~2.1}:]  has a~cycle.
\item[\textbf{Case~2.2}:]  has no cycles and .
\item[\textbf{Case~2.3}:]  does not fit in any of the previous cases, and there exist at least two fixed points of in-degree 1.
\item[\textbf{Case~2.4}:]  does not fit in any of the previous cases, and there exists  of in-degree  such that .
Let  be the smallest state such a~state with the largest  such that .
\begin{enumerate}[leftmargin=*,widest=\textbf{Subcase~2.4.1}]
\item[\textbf{Subcase~2.4.1}:]  and .
\item[\textbf{Subcase~2.4.2}:] , , and  has in-degree .
\item[\textbf{Subcase~2.4.3}:] , , and  has in-degree .
\item[\textbf{Subcase 2.4.4}:] .
\item[\textbf{Subcase 2.4.5}:] .
\end{enumerate}
\item[\textbf{Case 2.5}:]  does not fit in any of the previous cases.
\begin{enumerate}[leftmargin=*,widest=\textbf{Subcase~2.5.1}]
\item[\textbf{Subcase~2.5.1}:] There are at least two states  from  such that  for all .
\item[\textbf{Subcase~2.5.2}:]  does not fit in Subcase~2.5.1.
\end{enumerate}
\end{enumerate}
\medskip
\item[\textbf{Supercase 3}:]  and .
\begin{enumerate}[leftmargin=*,widest=\textbf{Case~3.1}]
\item[\textbf{Case~3.1}:]  and  has a~cycle.
\item[\textbf{Case~3.2}:]  does not fit into any of the previous cases, , and there exists a~state  such that .
Let  be the smallest state such a~state with the largest  such that .
\begin{enumerate}[leftmargin=*,widest=\textbf{Subcase~3.2.1}]
\item[\textbf{Subcase~3.2.1}:]  and .
\item[\textbf{Subcase~3.2.2}:] , , and  has in-degree at least .
\item[\textbf{Subcase~3.2.3}:] , , and  has in-degree .
\item[\textbf{Subcase~3.2.4}:] .
\end{enumerate}
\item[\textbf{Case~3.3}:]  does not fit into any of the previous cases, , and there exist at least two fixed points of in-degree 1.
\item[\textbf{Case~3.4}:]  does not fit into any of the previous cases and .
\begin{enumerate}[leftmargin=*,widest=\textbf{Subcase~3.4.1}]
\item[\textbf{Subcase~3.4.1}:] .
\item[\textbf{Subcase~3.4.2}:] .
\item[\textbf{Subcase~3.4.3}:] .
\end{enumerate}
\item[\textbf{Case~3.5}:] .
\begin{enumerate}[leftmargin=*,widest=\textbf{Subcase~3.5.1}]
\item[\textbf{Subcase~3.5.1}:]  and  has in-degree .
\item[\textbf{Subcase~3.5.2}:]  and  has in-degree at least .
\item[\textbf{Subcase~3.5.3}:] .
\end{enumerate}
\end{enumerate}
\end{enumerate}}



\begin{figure}[htb]
\unitlength 7.2pt\scriptsize
\gasset{Nh=2.5,Nw=2.5,Nmr=1.25,ELdist=0.5,loopdiam=1.5}
\begin{center}\begin{picture}(28,10)(0,-1)
\node[Nframe=n](name)(2,7){Case 2.1:}
\node(0)(2,0){0}\imark(0)
\node(p)(8,0){}
\node[Nframe=n](pdots)(14,0){}
\node(pt^k)(20,0){}
\node(n-1)(26,0){-}
\node(n-2)(26,4){-}\rmark(n-2)
\node(z)(12,4){}
\node(r)(14,7){}
\node[Nframe=n](rdots)(16,4){}
\drawedge(0,p){}
\drawedge(p,pdots){}
\drawedge(pdots,pt^k){}
\drawedge(pt^k,n-1){}
\drawedge(n-2,n-1){}
\drawloop[loopangle=270](n-1){}
\drawedge[curvedepth=1](z,r){}
\drawedge[curvedepth=1](r,rdots){}
\drawedge[curvedepth=1](rdots,z){}
\drawedge[linecolor=red,dash={.5 .25}{.25},curvedepth=-2.5](0,n-1){}
\drawedge[linecolor=red,dash={.5 .25}{.25},curvedepth=3.5](p,r){}
\drawedge[linecolor=red,dash={.5 .25}{.25},curvedepth=-1.5](pdots,p){}
\drawedge[linecolor=red,dash={.5 .25}{.25},curvedepth=-1.5](pt^k,pdots){}
\end{picture}\begin{picture}(28,10)(-2,-1)
\node[Nframe=n](name)(2,7){Case 2.2:}
\node(0)(2,0){0}\imark(0)
\node(p)(8,0){}
\node[Nframe=n](pdots)(14,0){}
\node(pt^k)(20,0){}
\node(n-1)(26,0){-}
\node(n-2)(26,4){-}\rmark(n-2)
\drawedge(0,p){}
\drawedge(p,pdots){}
\drawedge(pdots,pt^k){}
\drawedge(pt^k,n-1){}
\drawedge(n-2,n-1){}
\drawloop[loopangle=270](n-1){}
\drawedge[linecolor=red,dash={.5 .25}{.25},curvedepth=-2.5](0,n-1){}
\drawedge[linecolor=red,dash={.5 .25}{.25},curvedepth=-1.5](pdots,p){}
\drawedge[linecolor=red,dash={.5 .25}{.25},curvedepth=-1.5](pt^k,pdots){}
\drawloop[linecolor=red,dash={.5 .25}{.25}](p){}
\end{picture}\end{center}

\begin{center}\begin{picture}(28,12)(0,-1)
\node[Nframe=n](name)(2,7){Case 2.3:}
\node(0)(2,0){0}\imark(0)
\node(p)(14,0){}
\node(n-1)(26,0){-}
\node(n-2)(26,4){-}\rmark(n-2)
\node(f1)(10,4){}
\node(f2)(18,4){}
\drawedge(0,p){}
\drawedge(p,n-1){}
\drawedge(n-2,n-1){}
\drawloop[loopangle=270](n-1){}
\drawloop(f1){}
\drawloop(f2){}
\drawedge[linecolor=red,dash={.5 .25}{.25},curvedepth=-2.5](0,n-1){}
\drawedge(n-2,n-1){}
\drawloop[loopangle=270](n-1){}
\drawedge[linecolor=red,dash={.5 .25}{.25},curvedepth=1](f1,f2){}
\drawedge[linecolor=red,dash={.5 .25}{.25},curvedepth=1](f2,f1){}
\drawedge[linecolor=red,dash={.5 .25}{.25}](p,f2){}
\end{picture}\begin{picture}(28,12)(-2,-1)
\node[Nframe=n](name)(2,7){Subcase 2.4.1:}
\node(0)(2,0){0}\imark(0)
\node(p)(14,0){}
\node(n-1)(26,0){-}
\node(n-2)(26,4){-}\rmark(n-2)
\node(x)(6,4){}
\node(xt)(10,4){}
\node[Nframe=n](xdots)(14,4){}
\node(xt^ell)(18,4){}
\drawedge(0,p){}
\drawedge(p,n-1){}
\drawedge(n-2,n-1){}
\drawloop[loopangle=270](n-1){}
\drawedge(x,xt){}
\drawedge(xt,xdots){}
\drawedge(xdots,xt^ell){}
\drawedge(xt^ell,n-1){}
\drawedge[linecolor=red,dash={.5 .25}{.25},curvedepth=-2.5](0,n-1){}
\drawedge[linecolor=red,dash={.5 .25}{.25}](p,xt^ell){}
\end{picture}\end{center}

\begin{center}\begin{picture}(28,13)(0,-1)
\node[Nframe=n](name)(2,8){Subcase 2.4.2:}
\node(0)(2,0){0}\imark(0)
\node(p)(14,0){}
\node(n-1)(26,0){-}
\node(n-2)(26,4){-}\rmark(n-2)
\node(x)(10,4){}
\node(xt)(18,4){}
\node(y)(10,8){}
\drawedge(0,p){}
\drawedge(p,n-1){}
\drawedge(n-2,n-1){}
\drawloop[loopangle=270](n-1){}
\drawedge(x,xt){}
\drawedge(xt,n-1){}
\drawedge(y,xt){}
\drawedge[linecolor=red,dash={.5 .25}{.25},curvedepth=-2.5](0,n-1){}
\drawedge[linecolor=red,dash={.5 .25}{.25},curvedepth=5,syo=.5](p,y){}
\drawedge[linecolor=red,dash={.5 .25}{.25},curvedepth=1](xt,x){}
\drawedge[linecolor=red,dash={.5 .25}{.25}](x,y){}
\end{picture}\begin{picture}(28,13)(-2,-1)
\node[Nframe=n](name)(2,8){Subcase 2.4.3:}
\node(0)(2,0){0}\imark(0)
\node(p)(14,0){}
\node(n-1)(26,0){-}
\node(n-2)(26,4){-}\rmark(n-2)
\node(x)(10,4){}
\node(xt)(18,4){}
\drawedge(0,p){}
\drawedge(p,n-1){}
\drawedge(n-2,n-1){}
\drawloop[loopangle=270](n-1){}
\drawedge(x,xt){}
\drawedge(xt,n-1){}
\drawedge[linecolor=red,dash={.5 .25}{.25},curvedepth=-2.5](0,n-1){}
\drawedge[linecolor=red,dash={.5 .25}{.25}](p,x){}
\drawedge[linecolor=red,dash={.5 .25}{.25},curvedepth=-1](xt,x){}
\drawedge[linecolor=red,dash={.1 .1}{.1},curvedepth=3](x,n-2){(i)}
\drawedge[linecolor=red,dash={.1 .1}{.1},curvedepth=-.2,ELside=r,ELdist=.2](x,n-1){(ii)}
\end{picture}\end{center}

\begin{center}\begin{picture}(28,11)(0,-1)
\node[Nframe=n](name)(2,7){Subcase 2.4.4:}
\node(0)(2,0){0}\imark(0)
\node(p)(14,0){}
\node(n-1)(26,0){-}
\node(n-2)(26,4){-}\rmark(n-2)
\node(x)(6,4){}
\node(xt)(10,4){}
\node[Nframe=n](xdots)(14,4){}
\node(xt^ell)(18,4){}
\drawedge(0,p){}
\drawedge(p,n-1){}
\drawedge(n-2,n-1){}
\drawloop[loopangle=270](n-1){}
\drawedge(x,xt){}
\drawedge(xt,xdots){}
\drawedge(xdots,xt^ell){}
\drawedge(xt^ell,n-2){}
\drawedge[linecolor=red,dash={.5 .25}{.25},curvedepth=-2.5](0,n-1){}
\drawedge[linecolor=red,dash={.5 .25}{.25},curvedepth=-.5](p,n-2){}
\end{picture}\begin{picture}(28,11)(-2,-1)
\node[Nframe=n](name)(2,7){Subcase 2.4.5:}
\node(0)(2,0){0}\imark(0)
\node(p)(14,0){}
\node(n-1)(26,0){-}
\node(n-2)(26,4){-}\rmark(n-2)
\node(x)(6,4){}
\node(xt)(10,4){}
\node[Nframe=n](xdots)(14,4){}
\node(xt^ell)(18,4){}
\drawedge(0,p){}
\drawedge(p,n-1){}
\drawedge(n-2,n-1){}
\drawloop[loopangle=270](n-1){}
\drawedge(x,xt){}
\drawedge(xt,xdots){}
\drawedge(xdots,xt^ell){}
\drawloop(xt^ell){}
\drawedge[linecolor=red,dash={.5 .25}{.25},curvedepth=-2.5](0,n-1){}
\drawedge[linecolor=red,dash={.5 .25}{.25}](p,xt^ell){}
\end{picture}\end{center}

\begin{center}\begin{picture}(28,13)(0,-2)
\node[Nframe=n](name)(2,7){Subcase 2.5.1:}
\node(0)(2,0){0}\imark(0)
\node(p)(14,0){}
\node(n-1)(26,0){-}
\node(n-2)(26,4){-}\rmark(n-2)
\node(f)(8,4){}
\node(r1)(14,4){}
\node[Nframe=n](rdots)(18,4){}
\node(ru)(22,4){}
\drawedge(0,p){}
\drawedge(p,n-1){}
\drawedge(n-2,n-1){}
\drawloop[loopangle=270](n-1){}
\drawloop(f){}
\drawedge[curvedepth=-.2](r1,n-1){}
\drawedge[curvedepth=0,exo=.2](rdots,n-1){}
\drawedge[curvedepth=0,exo=.5](ru,n-1){}
\drawedge[linecolor=red,dash={.5 .25}{.25},curvedepth=-2.5](0,n-1){}
\drawedge[linecolor=red,dash={.5 .25}{.25}](p,f){}
\drawedge[linecolor=red,dash={.5 .25}{.25}](r1,rdots){}
\drawedge[linecolor=red,dash={.5 .25}{.25}](rdots,ru){}
\drawedge[linecolor=red,dash={.5 .25}{.25},curvedepth=-2](ru,r1){}
\end{picture}\begin{picture}(28,13)(-2,-2)
\node[Nframe=n](name)(2,7){Subcase 2.5.2:}
\node(0)(2,0){0}\imark(0)
\node(p)(14,0){}
\node(n-1)(26,0){-}
\node(n-2)(26,4){-}\rmark(n-2)
\node(f)(8,4){}
\node(q1)(14,4){}
\node[Nframe=n](qdots)(18,4){}
\node(qv)(22,4){}
\drawedge(0,p){}
\drawedge(p,n-1){}
\drawedge(n-2,n-1){}
\drawloop[loopangle=270](n-1){}
\drawloop(f){}
\drawedge[curvedepth=-3,exo=1](q1,n-2){}
\drawedge[curvedepth=-2,sxo=-1](qdots,n-2){}
\drawedge[curvedepth=0](qv,n-2){}
\drawedge[linecolor=red,dash={.5 .25}{.25},curvedepth=-2.5](0,n-1){}
\drawedge[linecolor=red,dash={.5 .25}{.25}](p,f){}
\drawedge[linecolor=red,dash={.5 .25}{.25},curvedepth=2](q1,qv){}
\drawedge[linecolor=red,dash={.5 .25}{.25}](qdots,q1){}
\drawedge[linecolor=red,dash={.5 .25}{.25}](qv,qdots){}
\end{picture}\end{center}
\caption{Map of the (sub)cases of Supercase~2 in the proof of Theorem~\ref{thm:bifix-free_upper_bound}.}
\end{figure}



\begin{figure}[htb]
\unitlength 7.2pt\scriptsize
\gasset{Nh=2.5,Nw=2.5,Nmr=1.25,ELdist=0.5,loopdiam=1.5}
\begin{center}\begin{picture}(28,9.5)(0,-1)
\node[Nframe=n](name)(2,8.5){Case 3.1:}
\node(0)(2,6){0}\imark(0)
\node(p)(14,6){}
\node(n-1)(26,0){-}
\node(n-2)(26,6){-}\rmark(n-2)
\node(z)(12,-1){}
\node(r)(14,2){}
\node[Nframe=n](rdots)(16,-1){}
\node(q1)(18,3){}
\node[Nframe=n](qdots)(20.5,3){}
\node(qv)(23,3){}
\drawedge(0,p){}
\drawedge(p,n-2){}
\drawedge(n-2,n-1){}
\drawloop[loopangle=270](n-1){}
\drawedge[curvedepth=.4](q1,n-2){}
\drawedge[curvedepth=.4,sxo=-1,exo=1.5](qdots,n-2){}
\drawedge(qv,n-2){}
\drawedge[curvedepth=1](z,r){}
\drawedge[curvedepth=1](r,rdots){}
\drawedge[curvedepth=1](rdots,z){}
\drawedge[linecolor=red,dash={.5 .25}{.25},curvedepth=2](0,n-2){}
\drawedge[linecolor=red,dash={.5 .25}{.25}](p,r){}
\drawedge[linecolor=red,dash={.5 .25}{.25}](q1,p){}
\drawedge[linecolor=red,dash={.5 .25}{.25},curvedepth=-.4,sxo=1,exo=-1.5](qdots,p){}
\drawedge[linecolor=red,dash={.5 .25}{.25},curvedepth=-.4](qv,p){}
\end{picture}\begin{picture}(28,9.5)(-2,-1)
\node[Nframe=n](name)(2,8.5){Subcase 3.2.1:}
\node(0)(2,6){0}\imark(0)
\node(p)(14,6){}
\node(n-1)(26,0){-}
\node(n-2)(26,6){-}\rmark(n-2)
\node(x)(2,0){}
\node(xt)(6,0){}
\node[Nframe=n](xdots)(10,0){}
\node(xtl)(14,0){}
\node(q1)(18,3){}
\node[Nframe=n](qdots)(20.5,3){}
\node(qv)(23,3){}
\drawedge(0,p){}
\drawedge(p,n-2){}
\drawedge(n-2,n-1){}
\drawedge[curvedepth=.4](q1,n-2){}
\drawedge[curvedepth=.4,sxo=-1,exo=1.5](qdots,n-2){}
\drawedge(qv,n-2){}
\drawedge(x,xt){}
\drawedge(xt,xdots){}
\drawedge(xdots,xtl){}
\drawedge[ELdist=.2](xtl,n-1){(ii)}
\drawedge[linecolor=red,dash={.5 .25}{.25},curvedepth=2](0,n-2){}
\drawedge[linecolor=red,dash={.5 .25}{.25}](q1,p){}
\drawedge[linecolor=red,dash={.5 .25}{.25},curvedepth=-.4,sxo=1,exo=-1.5](qdots,p){}
\drawedge[linecolor=red,dash={.5 .25}{.25},curvedepth=-.4](qv,p){}
\drawedge[linecolor=red,dash={.5 .25}{.25},curvedepth=-3](p,xtl){}
\drawloop[linecolor=red,dash={.1 .1}{.1},ELpos=80,ELdist=.2](xtl){(i)}
\end{picture}\end{center}

\begin{center}\begin{picture}(28,11)(0,-1)
\node[Nframe=n](name)(2,8.5){Subcase 3.2.2:}
\node(0)(2,6){0}\imark(0)
\node(p)(14,6){}
\node(n-1)(26,0){-}
\node(n-2)(26,6){-}\rmark(n-2)
\node(x)(8,0){}
\node(xt)(14,0){}
\node(y)(8,4){}
\node(q1)(18,3){}
\node[Nframe=n](qdots)(20.5,3){}
\node(qv)(23,3){}
\drawedge(0,p){}
\drawedge(p,n-2){}
\drawedge(n-2,n-1){}
\drawedge[curvedepth=.4](q1,n-2){}
\drawedge[curvedepth=.4,sxo=-1,exo=1.5](qdots,n-2){}
\drawedge(qv,n-2){}
\drawedge(x,xt){}
\drawedge(xt,n-1){}
\drawedge(y,xt){}
\drawedge[linecolor=red,dash={.5 .25}{.25},curvedepth=2](0,n-2){}
\drawedge[linecolor=red,dash={.5 .25}{.25}](q1,p){}
\drawedge[linecolor=red,dash={.5 .25}{.25},curvedepth=-.4,sxo=1,exo=-1.5](qdots,p){}
\drawedge[linecolor=red,dash={.5 .25}{.25},curvedepth=-.4](qv,p){}
\drawedge[linecolor=red,dash={.5 .25}{.25}](p,y){}
\drawedge[linecolor=red,dash={.5 .25}{.25},curvedepth=1](xt,x){}
\drawedge[linecolor=red,dash={.5 .25}{.25}](x,y){}
\end{picture}\begin{picture}(28,11)(-2,-1)
\node[Nframe=n](name)(2,8.5){Subcase 3.2.3:}
\node(0)(2,6){0}\imark(0)
\node(p)(14,6){}
\node(n-1)(26,0){-}
\node(n-2)(26,6){-}\rmark(n-2)
\node(x)(8,0){}
\node(xt)(14,0){}
\node(q1)(18,3){}
\node[Nframe=n](qdots)(20.5,3){}
\node(qv)(23,3){}
\drawedge(0,p){}
\drawedge(p,n-2){}
\drawedge(n-2,n-1){}
\drawedge[curvedepth=.4](q1,n-2){}
\drawedge[curvedepth=.4,sxo=-1,exo=1.5](qdots,n-2){}
\drawedge(qv,n-2){}
\drawedge(x,xt){}
\drawedge(xt,n-1){}
\drawedge[linecolor=red,dash={.5 .25}{.25},curvedepth=2](0,n-2){}
\drawedge[linecolor=red,dash={.5 .25}{.25}](q1,p){}
\drawedge[linecolor=red,dash={.5 .25}{.25},curvedepth=-.4,sxo=1,exo=-1.5](qdots,p){}
\drawedge[linecolor=red,dash={.5 .25}{.25},curvedepth=-.4](qv,p){}
\drawedge[linecolor=red,dash={.5 .25}{.25}](p,x){}
\drawedge[linecolor=red,dash={.5 .25}{.25},curvedepth=-1](xt,x){}
\drawloop[linecolor=red,dash={.1 .1}{.1}](x){(i)}
\drawedge[linecolor=red,dash={.1 .1}{.1},ELdist=.2,curvedepth=-2](x,n-1){(ii)}
\end{picture}\end{center}

\begin{center}\begin{picture}(28,11.5)(0,-2)
\node[Nframe=n](name)(2,7.5){Subcase 3.2.4:}
\node(0)(2,5){0}\imark(0)
\node(p)(14,5){}
\node(n-1)(26,0){-}
\node(n-2)(26,5){-}\rmark(n-2)
\node(x)(2,0){}
\node(xt)(6,0){}
\node[Nframe=n](xdots)(10,0){}
\node(xtl)(14,0){}
\node(q1)(18,2){}
\node[Nframe=n](qdots)(20.5,2){}
\node(qv)(23,2){}
\drawedge(0,p){}
\drawedge(p,n-2){}
\drawedge(n-2,n-1){}
\drawloop[loopangle=270](n-1){}
\drawedge[curvedepth=.4](q1,n-2){}
\drawedge[curvedepth=.4,sxo=-1,exo=1.5](qdots,n-2){}
\drawedge(qv,n-2){}
\drawedge(x,xt){}
\drawedge(xt,xdots){}
\drawedge(xdots,xtl){}
\drawloop[loopangle=270](xtl){}
\drawedge[linecolor=red,dash={.5 .25}{.25},curvedepth=2](0,n-2){}
\drawedge[linecolor=red,dash={.5 .25}{.25},curvedepth=5.6,sxo=1,exo=-.2](q1,x){}
\drawedge[linecolor=red,dash={.5 .25}{.25},curvedepth=5.7,sxo=1,exo=-1](qdots,x){}
\drawedge[linecolor=red,dash={.5 .25}{.25},curvedepth=5.8,sxo=1,exo=-1.8](qv,x){}
\drawedge[linecolor=red,dash={.5 .25}{.25}](p,xtl){}
\drawedge[linecolor=red,dash={.5 .25}{.25},curvedepth=1](xtl,xdots){}
\drawedge[linecolor=red,dash={.5 .25}{.25},curvedepth=1](xdots,xt){}
\drawedge[linecolor=red,dash={.5 .25}{.25},curvedepth=1](xt,x){}
\drawedge[linecolor=red,dash={.5 .25}{.25}](x,p){}
\end{picture}\begin{picture}(28,11.5)(-2,-2)
\node[Nframe=n](name)(2,7.5){Case 3.3:}
\node(0)(2,5){0}\imark(0)
\node(p)(14,5){}
\node(n-1)(26,0){-}
\node(n-2)(26,5){-}\rmark(n-2)
\node(q1)(18,2){}
\node[Nframe=n](qdots)(20.5,2){}
\node(qv)(23,2){}
\node(f1)(8,0){}
\node(f2)(14,0){}
\drawedge(0,p){}
\drawedge(p,n-2){}
\drawedge(n-2,n-1){}
\drawloop[loopangle=270](n-1){}
\drawedge[curvedepth=.4](q1,n-2){}
\drawedge[curvedepth=.4,sxo=-1,exo=1.5](qdots,n-2){}
\drawedge(qv,n-2){}
\drawloop[loopangle=270](f1){}
\drawloop[loopangle=270](f2){}
\drawedge[linecolor=red,dash={.5 .25}{.25},curvedepth=2](0,n-2){}
\drawedge[linecolor=red,dash={.5 .25}{.25}](q1,p){}
\drawedge[linecolor=red,dash={.5 .25}{.25},curvedepth=-.4,sxo=1,exo=-1.5](qdots,p){}
\drawedge[linecolor=red,dash={.5 .25}{.25},curvedepth=-.4](qv,p){}
\drawedge[linecolor=red,dash={.5 .25}{.25},curvedepth=1](f1,f2){}
\drawedge[linecolor=red,dash={.5 .25}{.25},curvedepth=1](f2,f1){}
\drawedge[linecolor=red,dash={.5 .25}{.25},curvedepth=0](p,f2){}
\end{picture}\end{center}

\begin{center}\begin{picture}(28,15.5)(0,-1)
\node[Nframe=n](name)(2,12.5){Subcase 3.4.1:}
\node(0)(2,10){0}\imark(0)
\node(p)(14,10){}
\node(n-1)(26,0){-}
\node(n-2)(26,10){-}\rmark(n-2)
\node(q1)(18,7){}
\node[Nframe=n,Nw=1.5,Nh=1.5](qdots)(20.5,7){}
\node(qv)(23,7){}
\node(r1)(18,3){}
\node[Nframe=n,Nw=1.5,Nh=1.5](rdots)(20.5,3){}
\node(ru)(23,3){}
\node(f)(8,3){}
\drawedge(0,p){}
\drawedge(p,n-2){}
\drawedge(n-2,n-1){}
\drawloop[loopangle=270](n-1){}
\drawedge[curvedepth=.4](q1,n-2){}
\drawedge[curvedepth=.4,sxo=-1,exo=1.5](qdots,n-2){}
\drawedge(qv,n-2){}
\drawedge[curvedepth=-.4](r1,n-1){}
\drawedge[curvedepth=-.4,sxo=-1,exo=1.5](rdots,n-1){}
\drawedge(ru,n-1){}
\drawloop(f){}
\drawedge[linecolor=red,dash={.5 .25}{.25},curvedepth=2](0,n-2){}
\drawedge[linecolor=red,dash={.5 .25}{.25}](p,f){}
\drawedge[linecolor=red,dash={.5 .25}{.25},sxo=-.5,curvedepth=-1](q1,qdots){}
\drawedge[linecolor=red,dash={.5 .25}{.25},sxo=-.5,curvedepth=-1](qdots,qv){}
\drawedge[linecolor=red,dash={.5 .25}{.25},curvedepth=-1.5](qv,q1){}
\drawedge[linecolor=red,curvedepth=-.5,exo=1,dash={.5 .25}{.25}](r1,qv){}
\drawedge[linecolor=red,curvedepth=-.5,exo=.7,dash={.5 .25}{.25}](rdots,qv){}
\drawedge[linecolor=red,curvedepth=-.5,exo=0,dash={.5 .25}{.25}](ru,qv){}
\end{picture}\begin{picture}(28,15.5)(-2,-1)
\node[Nframe=n](name)(2,12.5){Subcase 3.4.2:}
\node(0)(2,10){0}\imark(0)
\node(p)(14,10){}
\node(n-1)(26,0){-}
\node(n-2)(26,10){-}\rmark(n-2)
\node(q1)(17,7){}
\node(r1)(17,3){}
\node[Nframe=n](rdots)(20,3){}
\node(ru)(23,3){}
\node(f)(8,3){}
\drawedge(0,p){}
\drawedge(p,n-2){}
\drawedge(n-2,n-1){}
\drawloop[loopangle=270](n-1){}
\drawedge[curvedepth=.3](q1,n-2){}
\drawedge[curvedepth=-.4](r1,n-1){}
\drawedge[curvedepth=-.4,sxo=-1,exo=1.5](rdots,n-1){}
\drawedge(ru,n-1){}
\drawloop(f){}
\drawedge[linecolor=red,dash={.5 .25}{.25},curvedepth=2](0,n-2){}
\drawedge[linecolor=red,dash={.5 .25}{.25}](p,f){}
\drawedge[linecolor=red,dash={.5 .25}{.25},curvedepth=-3.5,eyo=-.9,sxo=.3](r1,p){}
\drawedge[linecolor=red,dash={.5 .25}{.25},curvedepth=-3,eyo=-.7](rdots,p){}
\drawedge[linecolor=red,dash={.5 .25}{.25},curvedepth=-2.7,eyo=-.4](ru,p){}
\drawedge[linecolor=red,dash={.5 .25}{.25}](q1,f){}
\end{picture}\end{center}

\begin{center}\begin{picture}(28,10.5)(0,-1)
\node[Nframe=n](name)(2,8.5){Subcase 3.4.3:}
\node(0)(2,6){0}\imark(0)
\node(p)(14,6){}
\node(n-1)(26,0){-}
\node(n-2)(26,6){-}\rmark(n-2)
\node(f)(8,0){}
\node(r1)(18,3){}
\node[Nframe=n,Nw=2,Nh=2](rdots)(20.5,3){}
\node(ru)(23,3){}
\drawedge(0,p){}
\drawedge(p,n-2){}
\drawedge(n-2,n-1){}
\drawloop[loopangle=270](n-1){}
\drawedge[curvedepth=-.4](r1,n-1){}
\drawedge[curvedepth=-.4,sxo=-1,exo=1.5](rdots,n-1){}
\drawedge(ru,n-1){}
\drawloop(f){}
\drawedge[linecolor=red,dash={.5 .25}{.25},curvedepth=2.5](0,n-2){}
\drawedge[linecolor=red,dash={.5 .25}{.25}](p,f){}
\drawedge[linecolor=red,dash={.5 .25}{.25}](r1,p){}
\drawedge[linecolor=red,dash={.5 .25}{.25},curvedepth=2,sxo=1](rdots,f){}
\drawedge[linecolor=red,dash={.5 .25}{.25},curvedepth=2,eyo=-.5](ru,f){}
\end{picture}\begin{picture}(28,10.5)(-2,-1)
\node[Nframe=n](name)(2,8.5){Subcase 3.5.1:}
\node(0)(2,6){0}\imark(0)
\node(p)(8,6){}
\node[Nframe=n](pdots)(14,6){}
\node(pt^k)(20,6){}
\node(n-2)(26,6){-}\rmark(n-2)
\node(n-1)(26,0){-}
\drawedge(0,p){}
\drawedge(p,pdots){}
\drawedge(pdots,pt^k){}
\drawedge(pt^k,n-2){}
\drawedge(n-2,n-1){}
\drawloop[loopangle=270](n-1){}
\drawedge[linecolor=red,dash={.5 .25}{.25},curvedepth=2.5](0,n-2){}
\drawedge[linecolor=red,dash={.5 .25}{.25},curvedepth=1](pdots,p){}
\drawedge[linecolor=red,dash={.5 .25}{.25},curvedepth=1](pt^k,pdots){}
\drawloop[linecolor=red,dash={.5 .25}{.25},loopangle=270](p){}
\end{picture}\end{center}

\begin{center}\begin{picture}(28,12)(0,-1)
\node[Nframe=n](name)(2,9.5){Subcase 3.5.2:}
\node(0)(2,7){0}\imark(0)
\node(p)(8,7){}
\node[Nframe=n](pdots)(14,7){}
\node(pt^k)(20,7){}
\node(y)(20,2){}
\node(n-2)(26,7){-}\rmark(n-2)
\node(n-1)(26,2){-}
\drawedge(0,p){}
\drawedge(p,pdots){}
\drawedge(pdots,pt^k){}
\drawedge(pt^k,n-2){}
\drawedge(y,pt^k){}
\drawedge(n-2,n-1){}
\drawloop[loopangle=270](n-1){}
\drawedge[linecolor=red,dash={.5 .25}{.25},curvedepth=2.5](0,n-2){}
\drawedge[linecolor=red,dash={.5 .25}{.25}](y,n-1){}
\drawedge[linecolor=red,dash={.5 .25}{.25},curvedepth=-1,sxo=-1](p,y){}
\drawedge[linecolor=red,dash={.5 .25}{.25},curvedepth=1](pdots,p){}
\drawedge[linecolor=red,dash={.5 .25}{.25},curvedepth=1](pt^k,pdots){}
\end{picture}\begin{picture}(28,12)(-2,0)
\node[Nframe=n](name)(2,10.5){Subcase 3.5.3:}
\node(0)(2,8){0}\imark(0)
\node(p)(8,8){}
\node[Nframe=n](pdots)(14,8){}
\node(pt^k)(20,8){}
\node[Nw=2,Nh=2](q1)(14,3){}
\node[Nframe=n,Nw=2,Nh=2](qdots1)(16.5,3){}
\node[Nw=2,Nh=2](qi)(19,3){}
\node[Nframe=n,Nw=2,Nh=2](qdots2)(21.5,3){}
\node[Nw=2,Nh=2](qv)(24,3){}
\node(x)(8,0){}
\node[Nframe=n](xdots)(14,0){}
\node(n-2)(26,8){-}\rmark(n-2)
\node(n-1)(26,0){-}
\drawedge(0,p){}
\drawedge(p,pdots){}
\drawedge(pdots,pt^k){}
\drawedge(pt^k,n-2){}
\drawedge(n-2,n-1){}
\drawloop[loopangle=270](n-1){}
\drawedge(x,xdots){}
\drawedge[curvedepth=-1.5,exo=.5](xdots,qi){}
\drawedge[curvedepth=.6](q1,n-2){}
\drawedge[curvedepth=.3,sxo=-1](qdots1,n-2){}
\drawedge[curvedepth=.2,sxo=-.5](qi,n-2){}
\drawedge[curvedepth=.1](qdots2,n-2){}
\drawedge[curvedepth=0](qv,n-2){}
\drawedge[linecolor=red,dash={.5 .25}{.25},curvedepth=2.5](0,n-2){}
\drawedge[linecolor=red,dash={.5 .25}{.25},curvedepth=1](pdots,p){}
\drawedge[linecolor=red,dash={.5 .25}{.25},curvedepth=1](pt^k,pdots){}
\drawedge[linecolor=red,dash={.5 .25}{.25}](p,x){}
\drawedge[linecolor=red,dash={.5 .25}{.25},curvedepth=-1.5](q1,qdots1){}
\drawedge[linecolor=red,dash={.5 .25}{.25},curvedepth=-1.5](qdots1,qi){}
\drawedge[linecolor=red,dash={.5 .25}{.25},curvedepth=-1.5](qi,qdots2){}
\drawedge[linecolor=red,dash={.5 .25}{.25},curvedepth=-1.5](qdots2,qv){}
\drawedge[linecolor=red,dash={.5 .25}{.25},curvedepth=-2.5,sxo=1,exo=-1](qv,q1){}
\end{picture}\end{center}
\caption{Map of the (sub)cases of Supercase~3 in the proof of Theorem~\ref{thm:bifix-free_upper_bound}.}
\end{figure}

\end{document}
