\documentclass[a4, 11pt]{article}
\usepackage{wakaiki_singlecolumn_Eng}
\newcommand{\diag}{\mathop{\rm diag}\nolimits}
\newcommand{\Int}{\mathop{\rm Int~\!}}
\newcommand{\Cl}{\mathop{\rm Cl~\!}}
\usepackage{graphicx, subcaption}      

\renewcommand{\labelenumi}{(\alph{enumi})}

\title{
Quantized Feedback Stabilization of Sampled-Data Switched Linear Systems
}
\author{Masashi Wakaiki and Yutaka Yamamoto
\thanks{M. Wakaiki and Y. Yamamoto are with the Department of Applied Analysis and Complex
Dynamical Systems, Graduate School of Informatics, Kyoto University, Kyoto
606-8501, Japan
(e-mail:{\tt  \ wakaiki@acs.i.kyoto-u.ac.jp};
{\tt \ yy@i.kyoto-u.ac.jp}).}}
\begin{document}
\maketitle
\thispagestyle{empty}
\pagestyle{empty}

\begin{abstract}                We propose a stability analysis method for sampled-data switched linear
systems with quantization.
The available information to the controller
is limited: the quantized state and switching signal at
each sampling time.
Switching between sampling times can produce
the mismatch of the modes between the plant and the controller.
Moreover, the coarseness of quantization makes the trajectory wander
around, not approach, the origin.
Hence the trajectory may leave the desired neighborhood
if the mismatch leads to instability of the closed-loop system.
For the stability of the switched systems,
we develop a sufficient condition characterized by the {\em total mismatch time}.
The relationship between the mismatch time and the dwell time 
of the switching signal is also discussed.
\end{abstract}



\section{Introduction}
In this paper, we consider a
sampled-data switched linear system with a memoryless quantizer 
in Fig.~\ref{fig:WSDSLS}.
The available information to the controller is only
the quantized state and switching signal at each sampling time.
We then raise the questions:
{\em
What conditions are needed for the stability of the closed-loop system
under such imcomplete information?
If the system is stable, how close can the trajectories get to the origin?}

Switched systems and quantized control have been studied extensively but separately;
see, e.g., \cite{Liberzon2003Book, Lin2009} for switched systems
and \cite{Ishii2002Book, Nair2007} for quantized control.
Few works examine 
the state behavior of a switched system with quantization and
the effect of switching between sampling times. 
Recently, \cite{Liberzon2014} has proposed an
encoding and control strategy that achieves
sampled-data quantized state feedback stabilization of switched systems.
This strategy is rooted in
the non-switched case in \cite{Liberzon2003}.
In \cite{Liberzon2014}, the input of the controller is a
discrete-valued and discrete-time signal, whereas
the controller generates 
a continuous-valued and continuous-time output signal.
In contrast, here we consider a controller whose {\em output as well as input} are
{\em discrete-valued} and {\em discrete-time} signals.


 \begin{figure}[b]
 \centering
 \includegraphics[width = 7cm,bb= 60 175 700 460,clip]{quantized_SDsystem_verC.pdf}
 \caption{Sampled-data switched system with quantization, where  is a sampling period}
 \label{fig:WSDSLS}
 \end{figure}

\cite{Ishii2002Book,Ishii2004} have studied the stability analysis
of a sampled-data non-switched system of a memoryless quantizer.
Since a memoryless quantizer does not give an accurate value of the state near the origin, 
asymptotic stability cannot be generally achieved. 
However, 
such a quantizer is useful because of the simplicity in implementation.
\cite{Ishii2004} have developed a sufficient condition
for a non-switched system to be quadratic attractive. The authors have also
provided a randomized algorithm to verify this stability property in a computationally efficient way.
In the present paper, we use this algorithm and a scheduling function 
with the revisitation property introduced in \cite{Liberzon2004}.
The combined method constructs  a common Lyapunov function guaranteeing
the quadratic attractiveness of each subsystem.


We face two challenges in the stability analysis of the switched system in Fig.~\ref{fig:WSDSLS}.
First, since only at sampling times we know which subsystem is active, 
we do not always use the feedback gain designed for the subsystem active at
the present time.
Therefore
the closed-loop system may become unstable when switching occurs between sampling times.
Second, after arriving at a certain neighborhood of the origin,
the trajectory may not approach the origin anymore due to the coarseness of quantization. 
This implies that the trajectory can leave
the desired neighborhood if switching makes the system unstable.

This paper is organized as follows. 
In Section~2, we state the switched system and 
the information structure together with basic assumptions.
In Section~3, we first investigate the growth rate of the common
Lyapunov function when switching occurs in a sampling interval.
Next we develop
a stability analysis method for the sampled-data switched system 
by using
the {\em total mismatch time}, the total time when the modes 
mismatch between the plant and the controller.
In Section~4, we briefly discuss the relationship between the mismatch time
and the dwell time of the switching signal.
Section~5 concludes this paper.

\noindent
{\bf Notation} \\
We denote by  the set of non-negative integers
.
For a set , , 
, and  are its closure,
interior, and boundary, respectively.

Let  denote the transpose of .
The Euclidean norm of  is defined by
.
For , its Euclidean induced norm is
defined by  and
equals the largest singular value of .
Let  and  denote
the largest and the smallest eigenvalue of .

Let  be a sampling period. For ,
we define  by


\section{Sampled-data Switched Systems with Quantization}
\subsection{Switched systems}
Consider the continuous-time switched linear system

where  is the state and 
 is the control input.
For a finite index set , the mapping 
 is right-continuous and piecewise constant.
We call  \textit{switching signal} and the discontinuities of  
\textit{switching times}.

We assume that all subsystems are stabilizable and that
only finitely many switches occur on any finite interval:
\begin{assumption}
\label{ass:system}
For every ,  is stabilizable, i.e., 
there exists  such that
 is Hurwitz. 
Furthermore, every sampling interval has 
at most one switch.
\end{assumption}

\subsection{Quantized sampled-data system}
Let  be a sampling period.
The sampler  is given by

and
the zero-th hold  by



We now state the definition of a memoryless quantizer  given in \cite{Ishii2004}.
For an index set ,
the partition  
of  is 
said to be \textit{finite} if for every bounded set , there exists a
finite subset  of  such that
.
We define
the quantizer  
with respect to the finite partition 
by


The second assumption is that  if  is close to the origin. 
\begin{assumption}
\label{ass:quantization_near_origin}
If  contains the origin, then .
\end{assumption}

Let  be the output of the zero-th hold whose input is 
the quantized state at sampling times, i.e.,

Note that in Fig. \ref{fig:WSDSLS}, the control input  is given by


Let  be positive-definite and 
define the quadratic Lyapunov function
 for .
Its time derivative  along the trajectory of \eqref{eq:SLS} with \eqref{eq:control_input}
is given by

if  is not a switching time or a sampling time.

For  with , 
we also define  and  by

Then
 and  are the time derivatives of 
along the trajectories of the systems  and , 
respectively.

Every individual mode is assumed to be 
stable in the following sense with
a common Lyapunov function:
\begin{assumption}
\label{ass:subsystem_QA}
Consider 
the following sampled-data non-switched systems with quantization:

Let  be a positive number and
suppose that  and  satisfy . Then
there exists a positive-definite matrix 
such that for all , every trajectory  of the system
\eqref{eq:subsystem_QSDS} with 
satisfies

or  for ,
where  and 
are given by

\end{assumption}

Assumption \ref{ass:subsystem_QA} implies the followings:
If we have no switches, then the
common Lyapunov function  decreases
at a certain rate until
. 
Furthermore,

as well as
 are invariant sets.

The objective of the present paper is to
find switching conditions for the switched system in Fig.~\ref{fig:WSDSLS} 
to arrive at some neighborhood of the origin and remain there. 
We also determine how small the neighborhood is.

\begin{remark}
{\bf (a)}
Let  be
the closed ball in  with center at 0 and radius . 
The ellipsoid  is the \textit{smallest} level set of 
\textit{containing} , whereas  is
the \textit{largest} level set of  \textit{contained in} . 

\noindent
{\bf (b)}
In the non-sampled case,
the existence of common Lyapunov functions is a sufficient condition 
for stability under arbitrary switching; see, e.g., \cite{Liberzon2003Book, Lin2009}.
For sampled-data switched systems, however,
such functions do not guarantees the stability because
a switch within a sampling interval may make the closed-loop system unstable. 

\noindent
{\bf (c)}
For plants with a single mode,
\cite{Ishii2004}
proposed a randomized algorithm for the computation of  in 
Assumption~\ref{ass:subsystem_QA}.
Combining the algorithm with a scheduling function that has the 
revisitation property in \cite{Liberzon2004},
we can efficiently compute the desired common Lyapunov function.
Since this is an immediate consequence of the above two works,
we omit the details.

\noindent
{\bf (d)}
Assumption \ref{ass:subsystem_QA} does not cover
the trajectory after switches even without mode mismatch.
If the mode changes  
at the switching times  and  
on a sampling interval ,
then \eqref{eq:dotVp_bound} holds only for .
In Assumption \ref{ass:system}, we therefore assume that 
at most one switch occurs on
a sampling interval.
\end{remark}




\section{Stabilization with Limited Information}
\subsection{Upper bounds of }
Assumption~\ref{ass:subsystem_QA} gives an upper bound \eqref{eq:dotVp_bound}
of , i.e.,  when we use the feedback gain designed for
the currently active subsystem. 
In this subsection,
we will
find an upper bound of , i.e.,
 when intersample switching leads to the
mismatch of the modes between the plant and the feedback gain.
To this end, we investigate the state behavior in sampling intervals.


Let us first examine the relationship among 
the original state , the sampled state ,
and the sampled quantized state .

The partition  is finite. 
Moreover, 
Assumption \ref{ass:quantization_near_origin} shows that
if  () for some sequence ,
then  for all .
Hence
there exists  such that

for  and .
We also define  by 


The next result gives an upper bound on
the norm of the sampled state  with
the original state .

\begin{lemma}
\label{lem:alpha1_bound}
Consider the swithced system \eqref{eq:SLS} with
\eqref{eq:control_input}, where  has finitely many switching times
on every finite interval.
Suppose that

and define  by

Then 
we have

for all  with 
.
\end{lemma}
\begin{proof}
It suffices to prove \eqref{eq:alpha1_bound} for
 and 
.

Let  denote the state-transition matrix of 
the switched system \eqref{eq:SLS}
for .
If a switch does not occur,  is given by 
.
If  are switching times on an interval 
and if we define  and , then
we have

Since

and since , it follows that

This leads to

Let  be switching times on the interval .
Since  for ,
we obtain

It is obvious that the equation above holds in the non-switched case.
Since  when , 
if follows from \eqref{eq:alpha0_bound} that

Substituting \eqref{eq:state_map_bound} and \eqref{eq:int_state_map_bound}
into \eqref{eq:x0_bound},
we obtain

Thus if \eqref{eq:alpha_0_condition} holds, we derive
\eqref{eq:alpha1_bound}.
\end{proof}


Let us next develop an upper bound on the norm of
the error  due to sampling.
To this end, 
we show the following proposition:
\begin{proposition}
\label{prop:state_transition_bound}
Let  be the state-transition map of 
the switched system \eqref{eq:SLS}
as above. Then

\end{proposition}


\begin{proof}
Let us first show the case without switching, i.e.,

Define the partial sum  of  by

Then for 

If we let , we obtain \eqref{eq:no_switching_matrix_expo_diff_bound}.

We now prove \eqref{eq:Phi_1_diff_bound}
in the switched case.
Let  be the switching times in the interval .
Let  and .
Then \eqref{eq:Phi_1_diff_bound} is equivalent to


We have already shown the case , i.e., the non-switched case.
The general case follows by induction. For ,

Hence if \eqref{eq:matrix_expo_diff_bound} holds with
 in place of , then

Thus we obtain \eqref{eq:matrix_expo_diff_bound}.
\end{proof}

\begin{lemma}
\label{lem:beta1_bound}
Consider the switched system \eqref{eq:SLS} with \eqref{eq:control_input}, 
where  has finitely many switching times
on every finite interval.
Define  by

Then we have

for all  with 
.
\end{lemma}
\begin{proof}
As in the proof of Lemma \ref{lem:alpha1_bound}, it suffices to prove
\eqref{eq:beta1_bound} for all 
 and 
.

By \eqref{eq:state_t}, we obtain

This leads to


Proposition \ref{prop:state_transition_bound} provides
the following upper bound on the first term of the right side of 
\eqref{eq:xt_x0_dif_bound}:


Since a
calculation similar to \eqref{eq:state_map_bound} shows that
. 
Hence as in \eqref{eq:int_state_map_bound},



We obtain \eqref{eq:beta1_bound} by
combining \eqref{eq:first_term_upper}
with \eqref{eq:int_Phit_tau}.
\end{proof}

Similarly to \eqref{eq:alpha0_bound},
to each  with , 
there correspond a positive number
 such that

for .

We are now in a position to obtain upper bounds on
the norm of  
the error 
 due to
sampling and quantization by using
the original state .
\begin{theorem}
\label{thm:alpha_beta_bound}
Consider the switched system \eqref{eq:SLS} with \eqref{eq:control_input}, 
where  has finitely many switching times
on every finite interval.
Define  and  as in Lemmas 
\ref{lem:alpha1_bound} and \ref{lem:beta1_bound}.
If  are defined by

then  satisfy

for all  with 
.
\end{theorem}
\begin{proof}
It follows from \eqref{eq:beta1_bound} and \eqref{eq:beta2_bound} that

Thus the second inequality \eqref{eq:beta_bound} holds.
\end{proof}

An upper bound on 
can be obtained as follows.



Since  satisfies

we see from \eqref{eq:beta_bound} that

for all  with 
.
Fast sampling and fine quantization make
the upper bound \eqref{eq:second_bound_Vpq} small.

Define  by

Then we obtain

for  with  and
for  with

and .





\subsection{Stability analysis with total mismatch time}
Let us analyze the stability of 
the switched system \eqref{eq:SLS} with \eqref{eq:control_input} by the two
upper bounds \eqref{eq:dotVp_bound} and 
\eqref{eq:dotVpq_bound} of .
Note that
the former bound \eqref{eq:dotVp_bound} 
is for the case , while
the latter \eqref{eq:dotVpq_bound} for the case .
It is therefore useful to define the following characterization of the 
switching signal:
\begin{definition}
For , 
we define the total mismatch time  by

\end{definition}
More explicitly, 
the length of an interval means its Lebesgue measure.
We shall not, however, use any measure theory because
 has only finitely many discontinuities
on every interval.

Define  and  by
 
First we study the state behavior when it is outside of
.
The following lemma suggests that every trajectory with its initial state in
 goes into 
if the total mismatch time  is sufficiently small; see Fig. \ref{fig:Lemma_Explain}.
\begin{lemma}
\label{lem:tor0}
Let Assumptions \ref{ass:system}, \ref{ass:quantization_near_origin}, and
\ref{ass:subsystem_QA}
hold, and let
 satisfy

If  achieves 

for , then 
there exists  such that for every

and ,
 and
 for all .
\end{lemma}
\begin{proof}
First we show that the trajectory  does not leave
 without belonging 
to . That is, 
there does not exist 
such that 


Assume, to reach a contradiction, \eqref{eq:x(TR0)_outside} and
\eqref{eq:x(t)_0_TR0} hold for some .
Recall that

for .
It follows from
\eqref{eq:dotVp_bound} and \eqref{eq:dotVpq_bound} that
4pt]
\dot V_{p,q}(x(t),q_x(t)) \leq D_P V(x(t)).
\end{array}

\label{eq:VTR0_bound_by_V0}
V(x(T_{R})) 
\leq 
\exp \big(D_P \mu(T_{R},0)-C_P(T_{R} - \mu(T_{R},0))\big) 
V(x(0)).

D_P \mu(t,0)-C_P(t - \mu(t,0))
\leq
\left(
\left(
C_P + 
D_P
\right) L - 
C_P
\right)t
\label{eq:CD_bound0}

V(x(T_{R})) < V(x(0)) < R^2 \lambda_{\max} (P).

\label{eq:x(T_0)_def}
x(T_0) \in \partial \underline{\mathcal{E}}_P(r),~
x(T_0+\varepsilon) \not\in \underline{\mathcal{E}}_P(r)
~~~~ (0 <  \varepsilon < \delta).

\label{eq:a_cond}
a^2r^2 \lambda_{\min}(P)
< R^2\lambda_{\max}(P)

\label{eq:ba_bound}
b(a) =
\frac{2\log a}{C_P + D_P}.

\label{eq:mu_b_L_condition}
\mu(t,T_0) \leq b(a) + L(t-T_0)

V(x(t)) \leq 
\exp\big( 
\left(
\left(
C_P+ D_P \right)L - 
C_P
\right) (t - T_0)
\big) 
\exp\big(
\left(C_P  + D_P\right)b(a)
\big) V(x(T_0) ) \label{eq:Vt_VT0}

\label{eq:Lyapunov_bound_ar0}
\exp\big(
\left(C_P + D_P\right)b(a)
\big) V(x(T_0) ) 
\leq a^2 \lambda_{\min}(P).

\label{eq:mu_0T} 
\mu(t,0) < \frac{1}{n}t \qquad (t > 0).
 
\label{eq:mu_T0T}
\mu(t,T_0) < T_s + \frac{1}{n}(t-T_0) 
\qquad (t > T_0).

a = \exp 
\left(
\frac{T_s(C_P + D_P)}{2}
\right),

\mu(0,t) \geq \frac{1}{n}t - \left( \frac{T_s}{n} + \varepsilon \right).

\label{eq:lowerbound2}
\mu(T_0,t) \geq T_s + \frac{1}{n}(t-T_0) - \left( \frac{T_s}{n} + \varepsilon \right).

A_1 = 
\frac{1}{6}
\begin{bmatrix}
1 & -2\\ -3 & 2
\end{bmatrix}, \quad
B_1 = 
\frac{1}{6}
\begin{bmatrix}
-4 \\ 3
\end{bmatrix}, \\
A_2 = 
\begin{bmatrix}
1 & -5\\ 1 & 2
\end{bmatrix}, \quad
B_2 = 
\begin{bmatrix}
1 \\ -1
\end{bmatrix}.

\label{eq:state_feedback_gain_Ex}
K_1 = \begin{bmatrix}
1.38 &  -1.86
\end{bmatrix},\quad
K_2 = \begin{bmatrix}
-2.80 &  3.77
\end{bmatrix}.

\int^{\infty}_{0} \big( x(t)^{\top}x(t) + u(t)^2 \big) dt.

Q_i(x_i) =
\begin{cases}
 \frac{-\xi_0 (\kappa^n + \kappa^{n+1})}{2}
&  (-\xi_0 \kappa^{n+1} \leq x_i < -\xi_0\kappa^n)\\ 
0
& (-\xi_0 \leq x_i \leq \xi_0) \\
\frac{\xi_0 (\kappa^n + \kappa^{n+1})}{2}
& (\xi_0\kappa^n < x \leq \xi_0\kappa^{n+1}),
\end{cases}

P = 
\begin{bmatrix}
2.9171 & 0.3489 \\ 
0.3489 & 3.6256
\end{bmatrix}.

(A_{\sigma(t)},B_{\sigma(t)},K_{\sigma([t]^-)})
=
(A_{1},B_{1},K_{1})
~\text{or}~
(A_{2},B_{2},K_{2}).

(A_{\sigma(t)},B_{\sigma(t)},K_{\sigma([t]^-)})
=
(A_{1},B_{1},K_{2})
~\text{or}~
(A_{2},B_{2},K_{1}).

\mu(0,t) \leq \sum_{k=1}^{m} ([t_k]^-+T_s - t_k)
< mT_s \leq \frac{1}{n}t.

\mu(T_0, t) \leq [T_0]^- +T_s - T_0 < T_s,

\label{eq:xi_def}
\xi_k = (t_{k+1} - t_k ) - nT_s

t - T_0 &= 
(t - t_m) + \sum_{k=0}^{m-1}(t_{k+1} - t_{k}) - (T_0 - t_0) \notag \\
&=
(t - t_m) + \sum_{k=0}^{m-1}(\xi_{k} + nT_s) - (T_0 - t_0) \notag\\
&=
mnT_s + (t-t_m) + \sum_{k=0}^{m-1}\xi_{k} - (T_0 -t_0).
\label{eq:t-T_0}

\label{eq:case1}
(t-t_m) + \sum_{k=0}^{m-1}\xi_k \geq T_0 -t_0.

\mu(T_0, t) &\leq  ([T_0]^-+T_s - T_0)
+ \sum_{k=1}^{m} ([t_k]^-+T_s - t_k) \\
&< (m+1)T_s \leq T_s + \frac{1}{n}(t-T_0),

\label{eq:case2}
(t-t_m) + \sum_{k=0}^{m-1}\xi_k < T_0 -t_0.

t_k &= (t_k - t_{k-1}) + \dots + (t_1 - t_0) \\
&= \sum_{\ell = 0}^{k-1} (\xi_{\ell} + nT_s)
\leq  \sum_{\ell = 0}^{m-1}\xi_{\ell} + knT_s 
< T_0 + knT_s.

\label{eq:t_k_bound}
t_0+knT_s \leq t_k < T_0+knT_s

\label{eq:t0-T0}
[t_0]^- = [T_0]^- < t_0 \leq T_0 < [T_0]^- + T_s, 

\mu([t_k]^-, [t_{k+1}]^-) 
&=
\mu([t_k]^-, [t_k]^-+T_s) \notag \\
&\leq
\mu(t_0, [t_0]^-+T_s) \notag \\
&=
[t_0]^- +T_s - t_0
\label{eq:t_k-t_k+1_bound}

t < (T_0 - t_0) + t_m -\sum_{k=0}^{m-1} \xi_{k}
= T_0 + mnT_s.

t_0 + mnT_s \leq t_m < t < T_0 + mnT_s.

\label{eq:t-t_m_bound}
\mu([t_m]^-,t) = t - t_m< T_0 - t_0.

m < \frac{t-t_0}{nT_s}.

\mu(T_0,t) 
&= \mu(T_0, [t_1]^-) + \sum_{k=1}^{m-1} \mu([t_k]^-,[t_{k+1}]^-)
+ \mu([t_m]^-,t) \notag \\ 
&< ([T_0]^-+T_s - T_0) + (m-1)([t_0]^-+T_s - t_0)+ (T_0 - t_0) \notag \\
&< \frac{t-t_0}{n}\frac{[t_0]^-+T_s - t_0}{T_s}
< \frac{t-[t_0]^-}{n}.
\label{eq:mu(T_0,t)_case2_bound}

T_s + \frac{t-T_0}{n}- \frac{t-[t_0]^-}{n}
= T_s - \frac{T_0 - [t_0]^-}{n} 
> T_s - \frac{T_s}{n} \geq 0.

\mu(0,t) = m\left(T_s - \frac{\varepsilon}{m}\right) = mT_s -\varepsilon =
\frac{1}{n}t - \left( \frac{T_s}{n} + \varepsilon \right).

T_0+ knT_s + \frac{\varepsilon}{2(m+1)}
= [T_0]^- + knT_s + \frac{\varepsilon}{m+1}.

\mu(T_0,t) 
&= \left(T_s - \frac{\varepsilon}{2(m+1)}\right) + 
m\left(T_s - \frac{\varepsilon}{m+1} \right) \\
&\geq (m+1)T_s - \varepsilon \\
&= T_s + \frac{1}{n}(t-T_0) 
- \left( \frac{T_s}{n} + \varepsilon\right),

which is the desired inequality \eqref{eq:lowerbound2}.
\hfill 





\bibliographystyle{plain}
\begin{thebibliography}{10}
\providecommand{\natexlab}[1]{#1}
\providecommand{\url}[1]{\texttt{#1}}
\providecommand{\urlprefix}{URL }
\expandafter\ifx\csname urlstyle\endcsname\relax
  \providecommand{\doi}[1]{doi:\discretionary{}{}{}#1}\else
  \providecommand{\doi}{doi:\discretionary{}{}{}\begingroup
  \urlstyle{rm}\Url}\fi

\bibitem[{Hespanha and Morse(1999)}]{Hespanha1999CDC}
Hespanha, J.P. and Morse, A.S. (1999).
\newblock Stability of swithched systems with average dwell-time.
\newblock In \emph{Proc. 38th IEEE CDC}.

\bibitem[{Ishii et~al.(2004)}]{Ishii2004}
Ishii, H., Ba\c{s}ar, T., and Tempo, R. (2004).
\newblock Randomized algorithms for quadratic stability of quantized
  sampled-data systems.
\newblock \emph{Automatica}, 40, 839--846.

\bibitem[{Ishii and Francis(2002)}]{Ishii2002Book}
Ishii, H. and Francis, B.A. (2002).
\newblock \emph{Limited Data Rate in Control Systems with Networks}.
\newblock Lecture Notes on Control and Information Science, Vol. 275, Berlin:
  Springer.

\bibitem[{Liberzon(2003{\natexlab{a}})}]{Liberzon2003}
Liberzon, D. (2003{\natexlab{a}}).
\newblock On stabilization of linear systems with limited information.
\newblock \emph{IEEE Trans. Automat. Control}, 48, 304--307.

\bibitem[{Liberzon(2003{\natexlab{b}})}]{Liberzon2003Book}
Liberzon, D. (2003{\natexlab{b}}).
\newblock \emph{Switching in Systems and Control}.
\newblock Birkh\"auser, Boston.

\bibitem[{Liberzon(2014)}]{Liberzon2014}
Liberzon, D. (2014).
\newblock Finite data-rate feedback stabilization of switched and hybrid linear
  systems.
\newblock \emph{Automatica}, 50, 409--420.

\bibitem[{Liberzon and Tempo(2004)}]{Liberzon2004}
Liberzon, D. and Tempo, R. (2004).
\newblock {Common Lyapunov functions and gradient algorithm}.
\newblock \emph{IEEE Trans. Automat. Control}, 49, 990--994.

\bibitem[{Lin and Antsaklis(2009)}]{Lin2009}
Lin, H. and Antsaklis, P.J. (2009).
\newblock {Stability and stabilizability of switched linear systems: a survey
  of recent results}.
\newblock \emph{IEEE Trans. Automat. Control}, 54, 308--322.

\bibitem[{Nair et~al.(2007)}]{Nair2007}
Nair, G.N., Fagnani, F., Zampieri, S., and Evans, R.J. (2007).
\newblock Feedback control under data rate constraints: An overview.
\newblock \emph{Proc. IEEE}, 96, 108--137.

\bibitem[{Zhai et~al.(2001)}]{Zhai2001}
Zhai, G., Hu, B., Yasuda, K., and Michel, A.N. (2001).
\newblock Stability analysis of switched systems with stable and unstable
  subsystems: An average dwell time approach.
\newblock \emph{Int. J. Systems Science}, 32, 1055--1061.

\end{thebibliography}



\end{document}
