\def\d{\delta}
\def\e{\epsilon}

\label{section-Dekking}
In this section, we give a proof of  the following theorem due to Dekking~\cite{dekk:94}.

\begin{theorem}[Transducts of morphic sequences are morphic]\label{the:transducts_preserve_morphic}
  If  is a transducer with input alphabet  
  and  is a morphic sequence, then  is morphic or finite.
\end{theorem}

This proof will proceed by \emph{annotating} entries in the original sequence  
with information about what state the transducer is in upon reaching that entry.
This allows us to construct a new morphism which produces the transduced sequence  as output. 
After proving this theorem, we will show that this process of annotation preserves -substitutivity.

\begin{figure}[h!]
  \centering
  \includegraphics{fst_double.pdf}
  \caption{A transducer that doubles every other letter.}
  \label{fig:double}
\end{figure}

\begin{example}\label{example-transducer}
  To illustrate several points in this section,
  we will consider the Fibonacci morphism (, ) 
  and the transducer which doubles every other letter, shown in Figure~\ref{fig:double}. 
\end{example}

\subsection{Transducts of morphic sequences are morphic}



We show in Lemma~\ref{lem:state_annotated_is_morphic} that transducts of morphic sequences
are morphic.
In order to prove this, we  also need several lemmas about transducers which are of independent interest. 
The approach here is adapted from a result in Allouche and Shallit~\cite{allo:shal:2003}; it is  attributed in that book  to Dekking. 
We repeat it here partly for the convenience of the reader, but mostly because there are some details of the proof which 
are used in the analysis of the substitutivity property.

\begin{definition}[, ]
  Given a transducer  and a word ,
  we define  to be . Note that .
  Further, we define  by
  .
  \label{definition-tau}
\end{definition}


\begin{example}\label{ann_ex:tau_and_theta'}
  Recall the transducer  from Figure~\ref{fig:double}. Let  be the identity, 
  and let  be the transposition  and . 
  For this transducer, 
   if   is even and  if   is odd.
We have .
  In this notation,
  
\end{example}

Next, we show that  is finite.

\begin{lemma}\label{lem:finite_annotation_exists}
  For any  transducer  and any morphism ,  
  there are natural numbers  and  so that for all , 
   for all .
\end{lemma}

\begin{proof}
  Let . Define  by 
  .
When we write  on the right, here is what we mean.  Suppose that .
  Then  is short for the composition .
  Recall the notation  from Definition~\ref{definition-tau}; we thus have  for  the individual letters .
  Consider .   We define its \emph{orbit} as 
  the infinite sequence  of elements of  given by
  .
Since each of the  belongs to the finite set ,  the orbit of  is eventually periodic. 
  Let  be the preperiod length and  be the period length. The
  periodicity implies that   for each  and for all .


  Let  and . Since , we can write it as .
  We prove that 
  . 
  Note that
  .
We got this  by breaking  into individual letters, then using the fact that  is a morphism, and finally using 
  the fact that .
  Finally we know by  that for individual letters,
  .
  So  , as desired.
\end{proof}

\begin{definition}[]
  Given a  transducer  and a morphism , we find  and  as in Lemma~\ref{lem:finite_annotation_exists}
  just above and define .
\label{def-Theta}
\end{definition}

\begin{example}  We continue with Example~\ref{example-transducer}.
  As the proof in Lemma~\ref{lem:finite_annotation_exists} demonstrates, to find the  and  for our transducer and the
   Fibonacci morphism, we only need to  find the common period of   and .
Using what we saw in Example~\ref{ann_ex:tau_and_theta'} above,  we can  take  and . 
  Therefore,  and . 
  We also note that  and , as we will need these later.
\end{example}

\begin{lemma}
\begin{enumerate}[(i)]
\item Given  and , the set 
  is finite.
  \item 
 If   , then .
\item  If   , then for all ,  .
\end{enumerate}
\label{lemma-for-welldefinedness}
\end{lemma}

\begin{proof}
Part (i) comes from the fact
that  each of the  coordinates of  comes from the finite set .
For (ii), we calculate:

Part (iii) uses  as follows:

\end{proof}



\begin{definition}[]\label{Def-hbar}
  Given a  transducer  and a morphism , 
  let  be as in Lemma~\ref{lemma-for-welldefinedness}(i).  Define
  the morphism  as follows.
  For for all , whenever , let
  
\end{definition}
By repeated use of Lemma~\ref{lemma-for-welldefinedness},   is well-defined.
Notice that  for all . 

\begin{lemma}\label{lem:hannotate}
  For all , all 
  and all natural numbers , if ,
  then 
   
  In particular, for , the first component of the  term
  in  is . 
\end{lemma}

\begin{proof} By induction on .
  For , the claim is trivial.   Assume that it holds for .
  Let ,
  and  for ,
  let .  Thus
  .
  Then:
  
  For , we have
  
  Concatenating the  sequences 
  for  completes our induction step.
\end{proof}


\begin{lemma}\label{lem:state_annotated_is_morphic}
  Let  be a  transducer, 
  let  be a morphism prolongable on the letter ,   
  and write  as  .
 Let  be from Definition~\ref{def-Theta}.  Using this,
 let   be  from
 Lemma~\ref{lemma-for-welldefinedness}(i), and
   from Definition~\ref{Def-hbar}.    Then
  \begin{enumerate}
  \item[(i)]   is prolongable on  
. 
  \item[(ii)]   Let    be the coding 
.
    Then  is well-defined.
\item[(iii)]  The image under  of 
is
  
This sequence  is morphic in the alphabet .
  \end{enumerate}
\end{lemma}


\begin{proof}
For (i),   write  as .
Using the fact that  for all , we see that

This verifies the prolongability.

For (ii): if ,
  then  and  are the first component of  and are thus equal.



We turn to (iii).
Taking  in Lemma~\ref{lem:hannotate} shows that   is

The image of this sequence under the coding  is 

In view of the   functions' definition (Def.~\ref{definition-tau}), we obtain  in
 (\ref{eq-z}).
By definition,    is   morphic. 
\end{proof}

This is most of the work required to prove
Theorem~\ref{the:transducts_preserve_morphic},
 the main result of this section. 
 
\begin{proof}[Theorem~\ref{the:transducts_preserve_morphic}]  
  Since  is morphic there is a morphism , a coding , 
  and an initial letter  so that .   We are to show that 
  is morphic.    Since  is computable by a transducer, we have
  , where  is the composition of transducers from Definition~\ref{definition-composition-transducers}.
  It is thus sufficient to show that given a transducer , the sequence  is morphic.
  
  By Lemma~\ref{lem:state_annotated_is_morphic}, the sequence 
   is morphic. 
  The output function of  is a morphism  .  By 
  Corollary~\ref{cor:close:pure},
 is morphic or finite.   But   is exactly ; indeed, the definition of  is  
  basically the same as the definition of .  This proves the theorem.
\end{proof}



\subsection{Substitutivity of transducts}

We are also interested in analyzing the -substitutivity of transducts. 
We claim that if a sequence  is -substitutive, then  is also -substitutive for all . 

As a first step, we show that annotating a morphism does not change -substitutivity.
\begin{definition}
  Let  be an alphabet and  any set.
  Let  be a word.
  We call  the \emph{set of annotations}.
  We write  for the word ,
  that is, the word obtained by \emph{dropping the annotations}.

  A morphism 
  is an \emph{annotation} of 
  if  for all , .
\end{definition}

Note that the morphism  from Definition~\ref{Def-hbar} is an annotation of  in this sense.
Then from the following proposition it follows that if  is -substitutive,
then the sequence  in Lemma~\ref{lem:state_annotated_is_morphic} is also -substitutive. 

\begin{proposition}
If  is an -substitutive morphic sequence with morphism  and  is any set of annotations, then any annotated morphism  also has an infinite fixpoint  which is also -substitutive.
\end{proposition}

The proof of this proposition is in two lemmas: first that the eigenvalues of 
the morphism are preserved by the annotation process, and second that if  is the
 dominant eigenvalue for , then no greater eigenvalues are introduced for .

\begin{lemma}\label{lem:h:oh}
  All eigenvalues for  are also eigenvalues for any annotated version  of .
\end{lemma}

\begin{proof}
  Let  be the incidence matrix of . 
  Order the elements of the annotated alphabet  lexicographically. 
  Then the incidence matrix of , call it , can be thought of 
  as a block matrix where the blocks have size  and there are  such blocks in . 
  Note that by the definition of annotation, the row sum in each row of the  block of  is .
  To simplify the notation, for the rest of this proof we write  for  and  for .
  Suppose  is a column eigenvector for  with eigenvalue . 
  Consider
  . 
  This  is a ``block vector":  the first  entries are , the second  entries are , and so on,
  for a total of  entries. We claim that  is a column eigenvector for  with eigenvalue .

  Consider the product of row  of  with . This is 
  .
Now  .
  So  is the row sum of the  block of  and hence is . Therefore, row  of  times  is ,
  since  is an eigenvector of . Finally we note that 
  the th entry of  is  by its definition. Hence multiplying  by  multiplies the th entry of  by  for all . 

  We have shown  that  is a column eigenvector of  with eigenvalue , so the (column) eigenvalues of  are all present in . However, since a matrix and its transpose have the same eigenvalues, the (column) qualification on the eigenvalues is unnecessary.
  \qed
\end{proof}

If  is an annotation of , then we have 


\begin{lemma}\label{lem:oh:h}
  Let  be morphisms such that 
  is an annotation of .
  Then every eigenvalue of  with a non-negative eigenvector is also an eigenvalue for .
\end{lemma}

\begin{proof}
  Let  be the incidence matrix of 
  and  be the incidence matrix of .
  Let  be an eigenvalue of  with corresponding eigenvector ,
  that is,  and .
  We define a vector  as follows:
  .
  We show that .
  Let , then:
    
  Hence .
  If  it follows that  is an eigenvalue of .
  Note that if  is non-negative, then . This proves the claim.
\end{proof}

\begin{corollary}\label{corollary:dominant}
  Let  be morphisms such that 
  is an annotation of .
  Then the dominant eigenvalue for  coincides with the dominant eigenvalue for .
\end{corollary}

\begin{proof}
  By Lemma~\ref{lem:h:oh} every eigenvalue of  is an eigenvalue of .
  Thus the dominant eigenvalue of  is greater or equal to that of .
  By Theorem~\ref{thm:perron}, the dominant eigenvalue of a non-negative matrix is a real number 
  and its corresponding eigenvector is non-negative.
  By Lemma~\ref{lem:h:oh}, every eigenvalue of  with a non-negative eigenvector is also an eigenvalue of . 
  Thus the dominant eigenvalue of  is also greater or equal to that of .
  Hence the dominant eigenvalues of  and  must be equal.
\end{proof}




\begin{theorem}
  Let  and  be  multiplicatively independent real numbers.
 If  is a -substitutive sequence and  is an -substitutive sequence,
  then  and  have
 no common non-erasing transducts except for the ultimately periodic sequences. 
\label{theorem-apply-Durand}
\end{theorem}

\begin{proof} 
Let  and  be morphisms whose fixed points are  and , respectively.
  By the proof of Theorem~\ref{the:transducts_preserve_morphic}, 
   is a morphic image of an annotation   
  of , and also of an annotation  of .  
  The morphisms must be non-erasing, by the assumption in this theorem.
By  Corollary~\ref{corollary:dominant} and Theorem~\ref{thm:morphism}, 
   is both - and -substitutive.
  By Durand's Theorem~\ref{theorem-Durand},  is eventually periodic.
\end{proof}

\subsection{Example}

We conclude the section with an example of Theorem~\ref{the:transducts_preserve_morphic}
and the lemmas in this section.

\begin{example}[]
  We saw the Fibonacci sequence  in Example~\ref{example-Fibonacci}:
  
  We conclude our series of examples pertaining to this sequence
  and the transducer 
  which doubles every other letter (see Example~\ref{example-transducer} and Figure~\ref{fig:double}).
  We want to exhibit ,
  following the recipe of Lemma~\ref{lem:state_annotated_is_morphic}.  First,  some examples of how  works:
  

  It turns out that only a few elements from this  end up appearing in the expressions for
  : 
  It is convenient to abbreviate some of the elements of :  
  Let us use  as an element of , and  also
  write 
   as , 
   as ,
   as 
  and  as .
  It turns out that we do not need to exhibit  in full because only eight points are reachable
  from .
  We may take  to be 
  

  The fixpoint of this morphism starting with  starts as 
  Turning to the coding , recall that the set  of states of  is . 
  Let us abbreviate the elements of  the same way we did
  with .   
  It is not hard to check that , ,   
  , and
  .
  Then the state-annotated
  sequence  from Lemma~\ref{lem:state_annotated_is_morphic} is
  


  Recall that  
  in our transducer doubles whatever letter it sees while in state  and copies whatever letter it sees while in state .
  That is, , and .
  Thus when we apply the morphism  to the sequence , we get
  
  As we saw in the proof of Theorem~\ref{the:transducts_preserve_morphic}, this
  sequence 
  
  is exactly .  
\end{example}
