\def\d{\delta}
\def\e{\epsilon}

\label{section-Dekking}
In this section, we give a proof of  the following theorem due to Dekking~\cite{dekk:94}.

\begin{theorem}[Transducts of morphic sequences are morphic]\label{the:transducts_preserve_morphic}
  If $M = (\Sigma, \Delta, Q, q_0, \delta, \lambda)$ is a transducer with input alphabet $\Sigma$ 
  and $x \in \Sigma^\omega$ is a morphic sequence, then $M(x)$ is morphic or finite.
\end{theorem}

This proof will proceed by \emph{annotating} entries in the original sequence $x$ 
with information about what state the transducer is in upon reaching that entry.
This allows us to construct a new morphism which produces the transduced sequence $M(x)$ as output. 
After proving this theorem, we will show that this process of annotation preserves $\alpha$-substitutivity.

\begin{figure}[h!]
  \centering
  \includegraphics{fst_double.pdf}
  \caption{A transducer that doubles every other letter.}
  \label{fig:double}
\end{figure}

\begin{example}\label{example-transducer}
  To illustrate several points in this section,
  we will consider the Fibonacci morphism ($h(a) = ab$, $h(b) = a$) 
  and the transducer which doubles every other letter, shown in Figure~\ref{fig:double}. 
\end{example}

\subsection{Transducts of morphic sequences are morphic}



We show in Lemma~\ref{lem:state_annotated_is_morphic} that transducts of morphic sequences
are morphic.
In order to prove this, we  also need several lemmas about transducers which are of independent interest. 
The approach here is adapted from a result in Allouche and Shallit~\cite{allo:shal:2003}; it is  attributed in that book  to Dekking. 
We repeat it here partly for the convenience of the reader, but mostly because there are some details of the proof which 
are used in the analysis of the substitutivity property.

\begin{definition}[$\tau_w$, $\Xi(w)$]
  Given a transducer $M = (\Sigma, \Delta, Q, q_0, \delta, \lambda)$ and a word $w \in \Sigma^*$,
  we define $\tau_w \in Q^Q$ to be $\tau_w(q) = \d(q, w)$. Note that $\tau_{wv} = \tau_v \circ \tau_w$.
  Further, we define $\Xi: \Sigma^*\to  (Q^Q)^\omega$ by
  $\Xi(w) = (\tau_w, \tau_{h(w)}, \tau_{h^2(w)}, \ldots, \tau_{h^n(w)}, \ldots)$.
  \label{definition-tau}
\end{definition}


\begin{example}\label{ann_ex:tau_and_theta'}
  Recall the transducer $M$ from Figure~\ref{fig:double}. Let $id: Q \to Q$ be the identity, 
  and let $\nu: Q \to Q$ be the transposition $\nu(s) = t$ and $\nu(t) = s$. 
  For this transducer, 
  $\tau_w = id$ if  $|w|$ is even and $\tau_w = \nu$ if  $|w|$ is odd.
We have $\Xi(a) = (\tau_a, \tau_{ab}, \tau_{aba}, \tau_{abaab}, \tau_{abaababa}, \ldots)$.
  In this notation,
  \begin{align*}
    \Xi(a) & = (\nu, id, \nu, \nu, id, \nu, \nu, id, \nu, \nu, \ldots) &
    \Xi(b) & =  (\nu, \nu, id, \nu, \nu, id, \ldots) &
    \Xi(\e) & =  (id, id, id, id, \ldots)
  \end{align*}
\end{example}

Next, we show that $\set{\,\Xi(w) : w \in \Sigma^*\,}$ is finite.

\begin{lemma}\label{lem:finite_annotation_exists}
  For any  transducer $M$ and any morphism $h:\Sigma\to\Sigma^*$,  
  there are natural numbers $p \geq 1$ and $n \geq 0$ so that for all $w \in \Sigma^*$, 
  $\tau_{h^i(w)} = \tau_{h^{i+p}(w)}$ for all $i \geq n$.
\end{lemma}

\begin{proof}
  Let $\Sigma = \{1, 2, \ldots, s\}$. Define $H: (Q^Q)^{s} \to (Q^Q)^{s}$ by 
  $H(f_1, f_2, \ldots, f_{s}) =  (f_{h(1)}, f_{h(2)}, \ldots, f_{h(s)})$.
When we write $f_{h(i)}$ on the right, here is what we mean.  Suppose that $h(i) = v_0\cdots v_j$.
  Then $f_{h(i)}$ is short for the composition $f_{v_j} \circ f_{v_{j-1}} \circ \cdots \circ f_{v_1} \circ f_{v_0}$.
  Recall the notation $\tau_w$ from Definition~\ref{definition-tau}; we thus have $\tau_i$ for  the individual letters $i\in\Sigma$.
  Consider $T_0 = (\tau_1, \tau_2, \ldots, \tau_{s})$.   We define its \emph{orbit} as 
  the infinite sequence $(T_i)_{i\in\omega}$ of elements of $ (Q^Q)^{s}$ given by
  $T_i =  H^i(T_0) =  H^i(\tau_1, \ldots \tau_{s}) =  (\tau_{h^i(1)}, \ldots, \tau_{h^i(s)})$.
Since each of the $T_i$ belongs to the finite set $(Q^Q)^{s}$,  the orbit of $T_0$ is eventually periodic. 
  Let $n$ be the preperiod length and $p$ be the period length. The
  periodicity implies that $(*)$ $\tau_{h^i(j)} = \tau_{h^{i+p}(j)}$ for each $j \in \Sigma$ and for all $i \geq n$.


  Let $w \in \Sigma^*$ and $i \geq n$. Since $w \in \Sigma^*$, we can write it as $w = \sigma_1\sigma_2 \cdots \sigma_m$.
  We prove that 
  $\tau_{h^i(w)} = \tau_{h^{i+p}(w)}$. 
  Note that
  $\tau_{h^i(w)} =  \tau_{h^i(\sigma_1 \cdots \sigma_m)} =  \tau_{h^i(\sigma_1)h^i(\sigma_2) \cdots h^i(\sigma_m)}
  =  \tau_{h^i(\sigma_n)}\circ\cdots\circ\tau_{h^i(\sigma_1)}$.
We got this  by breaking $w$ into individual letters, then using the fact that $h$ is a morphism, and finally using 
  the fact that $\tau_{u v} = \tau_u\circ\tau_v$.
  Finally we know by $(*)$ that for individual letters,
  $\tau_{h^i(\sigma_j)} = \tau_{h^{i+p}(\sigma_j)}$.
  So  $\tau_{h^i(w)} = \tau_{h^{i+p}(w)}$, as desired.
\end{proof}

\begin{definition}[$\Theta(w)$]
  Given a  transducer $M$ and a morphism $h$, we find $p$ and $n$ as in Lemma~\ref{lem:finite_annotation_exists}
  just above and define $\Theta(w) = (\tau_w, \tau_{h(w)}, \ldots, \tau_{h^{n+p-1}(w)})$.
\label{def-Theta}
\end{definition}

\begin{example}  We continue with Example~\ref{example-transducer}.
  As the proof in Lemma~\ref{lem:finite_annotation_exists} demonstrates, to find the $p$ and $n$ for our transducer and the
   Fibonacci morphism, we only need to  find the common period of  $\Xi(a)$ and $\Xi(b)$.
Using what we saw in Example~\ref{ann_ex:tau_and_theta'} above,  we can  take $n = 0$ and $p = 3$. 
  Therefore, $\Theta(a) = (\nu, id, \nu)$ and $\Theta(b) = (\nu, \nu, id)$. 
  We also note that $\Theta(\e) = (id, id, id)$ and $\Theta(ab) = (id, \nu, \nu)$, as we will need these later.
\end{example}

\begin{lemma}
\begin{enumerate}[(i)]
\item Given $M$ and $h$, the set $A = \{\, \Theta(w) : w \in \Sigma^*\,\}$
  is finite.
  \item 
 If   $\Theta(w) = \Theta(y)$, then $\Theta(h(w)) = \Theta(h(y))$.
\item  If   $\Theta(w) = \Theta(y)$, then for all $u\in\Sigma^*$,  $\Theta(wu) = \Theta(y u)$.
\end{enumerate}
\label{lemma-for-welldefinedness}
\end{lemma}

\begin{proof}
Part (i) comes from the fact
that  each of the $n+p$ coordinates of $\Theta(w)$ comes from the finite set $Q^Q$.
For (ii), we calculate:
$$\begin{array}{lcl@{\qquad}l}
  \Theta(h(w)) & = & (\tau_{h(w)}, \tau_{h^2(w)}, \tau_{h^3(w)}, \ldots ,\tau_{h^{n+p}(w)}) \\
  &=& (\tau_{h(w)}, \tau_{h^2(w)}, \tau_{h^3(w)}, \ldots,  \tau_{h^{n+p-1}(w)} ,\tau_{h^{n}(w)})  
& \mbox{by Lemma~\ref{lem:finite_annotation_exists}}
\\
 & = & (\tau_{h(y)}, \tau_{h^2(y)}, \tau_{h^3(y)}, \ldots,   \tau_{h^{n+p-1}(y)}, \tau_{h^{n}(y)}) = \Theta(h(y))
 & \mbox{since $\Theta(w) = \Theta(y)$} \end{array}
$$
Part (iii) uses $\Theta(w) = \Theta(y)$ as follows:
$$\begin{array}{lcl@{\qquad}l}
\Theta(w u) & = & (\tau_{wu}, \tau_{h(wu)}, \tau_{h^2(wu)}, \ldots ,\tau_{h^{n+p-1}(wu)})  \\
  & = & (\tau_u\circ \tau_{w},  \tau_{h(u)}\circ \tau_{h(w)}, \tau_{h^2(u)}\circ \tau_{h^2(w)}, \ldots ,\tau_{h^{n+p-1}(u)}\circ \tau_{h^{n+p-1}(w)})  \\
  & = & (\tau_u\circ \tau_{y},  \tau_{h(u)}\circ \tau_{h(y)}, \tau_{h^2(u)}\circ \tau_{h^2(y)}, \ldots ,\tau_{h^{n+p-1}(u)}\circ \tau_{h^{n+p-1}(y)}) 
  = \Theta(y u)
\end{array}
$$
\end{proof}



\begin{definition}[$\overline{h}$]\label{Def-hbar}
  Given a  transducer $M$ and a morphism $h$, 
  let $A$ be as in Lemma~\ref{lemma-for-welldefinedness}(i).  Define
  the morphism $\overline{h}: \Sigma \times A \to (\Sigma \times A)^*$ as follows.
  For for all $\sigma \in \Sigma$, whenever $h(\sigma) = s_1s_2s_3\cdots s_\ell$, let
  $$\overline{h}((\sigma, \Theta(w))) \;\;=\;\; 
    (s_1, \Theta(h w))\;\; (s_2, \Theta((hw)s_1))
    \;\; (s_3, \Theta((hw)s_1s_2))
    \;\; \cdots \;\; (s_{\ell}, \Theta((h w)s_1s_2\cdots s_{\ell-1}))$$
\end{definition}
By repeated use of Lemma~\ref{lemma-for-welldefinedness}, $\overline{h}$  is well-defined.
Notice that $|\overline{h}(\sigma, a)| = |h(\sigma)|$ for all $\sigma$. 

\begin{lemma}\label{lem:hannotate}
  For all $\sigma\in\Sigma$, all $w\in \Sigma^*$
  and all natural numbers $n$, if $h^n(\sigma) = s_1 s_2 \cdots s_{\ell}$,
  then 
  $$\overline{h}^n((\sigma, \Theta(w))) \;\;=\;\;
  (s_1, \Theta(h^n w))\;\; (s_2, \Theta((h^n w)s_1)) \;\; (s_3, \Theta((h^n w)s_1s_2)) \;\cdots\; (s_{\ell}, \Theta((h^n w)s_1s_2\cdots s_{\ell-1})).$$ 
  In particular, for $1\leq i \leq \ell$, the first component of the $i^{th}$ term
  in $h^n(\sigma,\Theta(w))$ is $s_i$. 
\end{lemma}

\begin{proof} By induction on $n$.
  For $n = 0$, the claim is trivial.   Assume that it holds for $n$.
  Let $h^n(\sigma) = s_1 s_2 \cdots s_{\ell}$,
  and  for $1\leq i \leq \ell$,
  let $h(s_i) = t^i_1 t^i_2 \cdots t^i_{k_i}$.  Thus
  $
  h^{n+1}(\sigma) =
  h(s_1 s_2\cdots s_{\ell}) = t^1_1 t^1_2 \cdots t^1_{k_1} t^2_1 t^2_2 \cdots t^2_{k_2}
  t^\ell_1 t^\ell_2 \cdots t^\ell_{k_\ell}
  $.
  Then:
  $$\begin{array}{clcl}
& \overline{h}(\overline{h}^n(\sigma, \Theta(w))) 
  \;=\; \hbar(s_1, \Theta((h^n w)))\;\; \hbar(s_2, \Theta((h^n w)s_1))
  \;\; \hbar(s_3, \Theta((h^n w)s_1s_2))\\
  & \hspace{1.5in}\cdots\;\; \hbar(s_{\ell},
   \Theta((h^n w)s_1s_2\cdots s_{\ell-1}))
  \end{array}
  $$
  For $1\leq i \leq \ell$, we have
  $$\begin{array}{clcl}
   & \hbar(s_i, \Theta((h^n w)s_1\cdots s_{i-1})) \\
   = &   (t^i_1, \Theta((h h^n w)h(s_1\cdots s_{i-1})))
   \quad (t^i_2, \Theta((hh^nw)h(s_1\cdots s_{i-1})t^i_1))
\\
   & \hspace{.5in}\cdots\quad 
  (t^i_{k_i}, \Theta(hh^nw)h(s_1\cdots s_{i-1}) t^i_1 t^i_2\cdots t^i_{k_i-1}))  \\
  = & (t^i_1, \Theta((h^{n+1}w) t^1_1 t^1_2 \cdots t^1_{k_1} \cdots t^{i-1}_1 t^{i-1}_2 \cdots t^{i-1}_{k_{i-1}}))
  \quad 
  (t^i_2, \Theta((h^{n+1}w) t^1_1 t^1_2 \cdots
   t^1_{k_1} \cdots t^{i-1}_1 t^{i-1}_2 \cdots t^{i-1}_{k_{i-1}} t^i_1))
   \\
   & \hspace{.5in} \cdots \quad 
   (t^i_{k_i}, \Theta((h^{n+1}w) t^1_1 t^1_2 \cdots
   t^1_{k_i} \cdots t^{i-1}_1 t^{i-1}_2 \cdots t^{i-1}_{k_{i-1}} t^i_1 \cdots t^i_{k_i -1}))
  \end{array}
  $$
  Concatenating the  sequences $\hbar(s_i, \Theta((h^n w)s_1\cdots s_{i-1}))$
  for $i = 1, \ldots, \ell$ completes our induction step.
\end{proof}


\begin{lemma}\label{lem:state_annotated_is_morphic}
  Let $M = (\Sigma, \Delta, Q, q_0, \delta, \lambda)$ be a  transducer, 
  let $h$ be a morphism prolongable on the letter $x_1$,   
  and write $h^\omega(x_1)$ as  $x = x_1x_2x_3\cdots x_n \cdots$.
 Let $\Theta$ be from Definition~\ref{def-Theta}.  Using this,
 let  $A$ be  from
 Lemma~\ref{lemma-for-welldefinedness}(i), and
  $\overline{h}$ from Definition~\ref{Def-hbar}.    Then
  \begin{enumerate}
  \item[(i)]  $\overline{h}$ is prolongable on  
$(x_1, \Theta(\e))$. 
  \item[(ii)]   Let   $c: \Sigma \times A \to \Sigma \times Q$ be the coding 
$c(\sigma, \Theta(w)) = (\sigma, \tau_w(q_0))$.
    Then $c$ is well-defined.
\item[(iii)]  The image under $c$ of $\overline{h}^\omega((x_1, \Theta(\e))$
is
  \begin{equation}
    \begin{array}{lcl}
  z & \;\;=\;\; & (x_1, \d(q_0, \e))\;\;
  (x_2, \d(q_0, x_1))\;\;
  (x_3, \d(q_0, x_1x_2))\;\;\cdots\;\; (x_n, \d(q_0, x_1x_2\cdots x_{n-1}))
  \;\;\cdots
  \end{array}
  \label{eq-z}
  \end{equation}
This sequence $z$ is morphic in the alphabet $\Sigma \times Q$.
  \end{enumerate}
\end{lemma}


\begin{proof}
For (i),   write $h(x_1)$ as $ x_1 x_2 \cdots x_\ell$.
Using the fact that $h^i(\e) = \e$ for all $i$, we see that
$$ 
\begin{array}{lcl}
\overline{h}((x_1, \Theta(\e))) & = & (x_1, \Theta(\e)) 
\quad
  (x_2, \Theta(x_1)) \quad \cdots 
  \quad (x_{\ell}, \Theta(x_1,\ldots, x_{\ell -1})).
\end{array}
$$
This verifies the prolongability.

For (ii): if $\Theta(w) = \Theta(u)$,
  then $\tau_w$ and $\tau_u$ are the first component of $\Theta(w)$ and are thus equal.



We turn to (iii).
Taking $w = \e$ in Lemma~\ref{lem:hannotate} shows that  $\overline{h}^\omega((x_1, \Theta(\e))$ is
$$(x_1, \Theta(\e)) \quad (x_2, \Theta(x_1))
\quad (x_3, \Theta(x_1 x_2)) \quad \cdots \quad
(x_m, \Theta(x_1 x_2\cdots x_{m-1})) \quad \cdots .
$$
The image of this sequence under the coding $c$ is 
$$\begin{array}{lcl}
  (x_1, \tau_{\e}(q_0)) \quad (x_2, \tau_{x_1}(q_0))
  \quad (x_3, \tau_{x_1 x_2}(q_0))\quad \cdots\quad
  (x_m, \tau_{x_1 x_2\cdots x_{m-1}}(q_0)) \quad \cdots .
\end{array}
$$
In view of the  $\tau$ functions' definition (Def.~\ref{definition-tau}), we obtain $z$ in
 (\ref{eq-z}).
By definition,   $z$ is   morphic. 
\end{proof}

This is most of the work required to prove
Theorem~\ref{the:transducts_preserve_morphic},
 the main result of this section. 
 
\begin{proof}[Theorem~\ref{the:transducts_preserve_morphic}]  
  Since $x$ is morphic there is a morphism $h: \Sigma' \to (\Sigma')^*$, a coding $c: \Sigma' \to \Sigma$, 
  and an initial letter $x_1 \in \Sigma'$ so that $x = c(h^\omega(x_1))$.   We are to show that $M(c(h^\omega(x_1)))$
  is morphic.    Since $c$ is computable by a transducer, we have
  $x = (M \circ c)(h^\omega(x_1))$, where $\circ$ is the composition of transducers from Definition~\ref{definition-composition-transducers}.
  It is thus sufficient to show that given a transducer $M$, the sequence $M(h^\omega(x_1))$ is morphic.
  
  By Lemma~\ref{lem:state_annotated_is_morphic}, the sequence 
  $$z = (x_1, \d(q_0, \e)) \quad (x_2, \d(q_0, x_1))\quad
  (x_3, \d(q_0, x_1x_2))\quad \cdots\quad (x_n, \d(q_0, x_1x_2\cdots x_{n-1}))\quad\cdots$$ is morphic. 
  The output function of $M$ is a morphism  $\lambda: \Sigma\times Q \to \Delta^*$.  By 
  Corollary~\ref{cor:close:pure},
$\lambda(z)$ is morphic or finite.   But  $\lambda(z)$ is exactly $M(x)$; indeed, the definition of $M(x)$ is  
  basically the same as the definition of $\lambda(z)$.  This proves the theorem.
\end{proof}



\subsection{Substitutivity of transducts}

We are also interested in analyzing the $\alpha$-substitutivity of transducts. 
We claim that if a sequence $x$ is $\alpha$-substitutive, then $M(x)$ is also $\alpha$-substitutive for all $M$. 

As a first step, we show that annotating a morphism does not change $\alpha$-substitutivity.
\begin{definition}
  Let $\Sigma$ be an alphabet and $A$ any set.
  Let $w \; = \; (b_1,a_1) \;\; (b_2,a_2) \;\;  \ldots \;\;  (b_k,a_k) \in (\Sigma \times A)^*$ be a word.
  We call $A$ the \emph{set of annotations}.
  We write $\floor{w}$ for the word $b_1b_2\ldots b_k$,
  that is, the word obtained by \emph{dropping the annotations}.

  A morphism $\overline{h} : (\Sigma \times A) \to (\Sigma \times A)^*$
  is an \emph{annotation} of $h : \Sigma \to \Sigma^*$
  if $h(b) = \floor{\overline{h}(b,a)}$ for all $b\in\Sigma$, $a \in A$.
\end{definition}

Note that the morphism $\overline{h}$ from Definition~\ref{Def-hbar} is an annotation of $h$ in this sense.
Then from the following proposition it follows that if $x$ is $\alpha$-substitutive,
then the sequence $z$ in Lemma~\ref{lem:state_annotated_is_morphic} is also $\alpha$-substitutive. 

\begin{proposition}
If $x = h^\omega(\sigma)$ is an $\alpha$-substitutive morphic sequence with morphism $h: \Sigma \to \Sigma^*$ and $A$ is any set of annotations, then any annotated morphism $\overline{h}: \Sigma \times A \to (\Sigma \times A)^*$ also has an infinite fixpoint $\overline{h}^\omega((\sigma,a))$ which is also $\alpha$-substitutive.
\end{proposition}

The proof of this proposition is in two lemmas: first that the eigenvalues of 
the morphism are preserved by the annotation process, and second that if $\alpha$ is the
 dominant eigenvalue for $h$, then no greater eigenvalues are introduced for $\overline{h}$.

\begin{lemma}\label{lem:h:oh}
  All eigenvalues for $h$ are also eigenvalues for any annotated version $\overline{h}$ of $h$.
\end{lemma}

\begin{proof}
  Let $M = (m_{i,j})_{i,j \in \Sigma}$ be the incidence matrix of $h$. 
  Order the elements of the annotated alphabet $\Sigma \times A$ lexicographically. 
  Then the incidence matrix of $\overline{h}$, call it $N = (n_{i,j})_{i,j \in \Sigma \times A}$, can be thought of 
  as a block matrix where the blocks have size $|A| \times |A|$ and there are $|\Sigma| \times |\Sigma|$ such blocks in $N$. 
  Note that by the definition of annotation, the row sum in each row of the $(a,b)$ block of $N$ is $m_{a,b}$.
  To simplify the notation, for the rest of this proof we write $J$ for $|\Sigma|$ and $K$ for $|A|$.
  Suppose $v = (v_1, v_2, \ldots, v_{J})$ is a column eigenvector for $M$ with eigenvalue $\alpha$. 
  Consider
  $\overline{v} = (v_1, \ldots, v_1, v_2, \ldots, v_2, \ldots, v_n, \ldots v_n)$. 
  This  is a ``block vector":  the first $K$ entries are $v_1$, the second $K$ entries are $v_2$, and so on,
  for a total of $K\cdot J$ entries. We claim that $\overline{v}$ is a column eigenvector for $N$ with eigenvalue $\alpha$.

  Consider the product of row $k$ of $N$ with $\overline{v}$. This is 
  $\sum_{j = 1}^{K\cdot J} n_{k,j}\overline{v}_j =  \sum_{b = 1}^{J} v_b\cdot(\sum_{j=1}^{K}n_{k,Kb + j})$.
Now  $k = Ka + r$.
  So $\sum_{j=1}^{K} n_{k, Kb + j}$ is the row sum of the $(a, b)$ block of $N$ and hence is $m_{a,b}$. Therefore, row $k$ of $N$ times $\overline{v}$ is $\sum_{b = 1}^{J} v_b m_{a, b} = \alpha v_a$,
  since $v$ is an eigenvector of $M$. Finally we note that 
  the $k$th entry of $\overline{v}$ is $v_a$ by its definition. Hence multiplying $\overline{v}$ by $N$ multiplies the $k$th entry of $\overline{v}$ by $\alpha$ for all $k$. 

  We have shown  that $\overline{v}$ is a column eigenvector of $N$ with eigenvalue $\alpha$, so the (column) eigenvalues of $M$ are all present in $N$. However, since a matrix and its transpose have the same eigenvalues, the (column) qualification on the eigenvalues is unnecessary.
  \qed
\end{proof}

If $\overline{h}$ is an annotation of $h$, then we have 
\begin{align}
  \occ{b'}{h(b)} = \sum_{a'\in A} \occ{\pair{b'}{a'}}{\,\overline{h}(\pair{b}{a})\,}
  &&\text{for all $b,b'\in\Sigma$ and $a \in A$} \label{eq:sub:sum}
\end{align}

\begin{lemma}\label{lem:oh:h}
  Let $h,\overline{h}$ be morphisms such that $\overline{h} : (\Sigma \times A) \to (\Sigma \times A)^*$
  is an annotation of $h : \Sigma \to \Sigma^*$.
  Then every eigenvalue of $\overline{h}$ with a non-negative eigenvector is also an eigenvalue for $h$.
\end{lemma}

\begin{proof}
  Let $M = (m_{i,j})_{i,j \in \Sigma}$ be the incidence matrix of $h$
  and $N = (n_{i,j})_{i,j \in \Sigma\times A}$ be the incidence matrix of $\overline{h}$.
  Let $r$ be an eigenvalue of $N$ with corresponding eigenvector $v = (v_{\pair{b}{a}})_{\pair{b}{a} \in \Sigma\times A}$,
  that is, $Nv = rv$ and $v\ne 0$.
  We define a vector $w = (w_b)_{b \in \Sigma}$ as follows:
  $w_b = \sum_{a \in A} v_{\pair{b}{a}}$.
  We show that $Mw = rw$.
  Let $b' \in \Sigma$, then:
  \begin{align*}
    (Mw)_{b'} 
    &= \sum_{b \in \Sigma} M_{b',b}w_b 
    = \sum_{b \in \Sigma} \left( M_{b',b} \sum_{a \in A} v_{\pair{b}{a}} \right)\\
    &= \sum_{b \in \Sigma} \sum_{a \in A} M_{b',b} v_{\pair{b}{a}} 
    \stackrel{\text{by \eqref{eq:sub:sum}}}{=} \sum_{b \in \Sigma} \sum_{a \in A} \left(\sum_{a' \in A} N_{\pair{b'}{a'},\pair{b}{a}}\right) v_{\pair{b}{a}} 
    \\
    &= \sum_{a' \in A} \sum_{b \in \Sigma} \sum_{a \in A} N_{\pair{b'}{a'},\pair{b}{a}} v_{\pair{b}{a}} 
    \stackrel{Nv = rv}{=} \sum_{a' \in A} r v_{\pair{b'}{a'}} 
    =  r \sum_{a' \in A} v_{\pair{b'}{a'}} 
    =  r w_{b'}
  \end{align*}  
  Hence $Mw = rw$.
  If $w \ne 0$ it follows that $r$ is an eigenvalue of $M$.
  Note that if $v$ is non-negative, then $w \ne 0$. This proves the claim.
\end{proof}

\begin{corollary}\label{corollary:dominant}
  Let $h,\overline{h}$ be morphisms such that $\overline{h} : (\Sigma \times A) \to (\Sigma \times A)^*$
  is an annotation of $h : \Sigma \to \Sigma^*$.
  Then the dominant eigenvalue for $h$ coincides with the dominant eigenvalue for $\overline{h}$.
\end{corollary}

\begin{proof}
  By Lemma~\ref{lem:h:oh} every eigenvalue of $h$ is an eigenvalue of $\overline{h}$.
  Thus the dominant eigenvalue of $\overline{h}$ is greater or equal to that of $h$.
  By Theorem~\ref{thm:perron}, the dominant eigenvalue of a non-negative matrix is a real number $\alpha > 1$
  and its corresponding eigenvector is non-negative.
  By Lemma~\ref{lem:h:oh}, every eigenvalue of $\overline{h}$ with a non-negative eigenvector is also an eigenvalue of $h$. 
  Thus the dominant eigenvalue of $h$ is also greater or equal to that of $\overline{h}$.
  Hence the dominant eigenvalues of $h$ and $\overline{h}$ must be equal.
\end{proof}




\begin{theorem}
  Let $\alpha$ and $\beta$ be  multiplicatively independent real numbers.
 If $v$ is a $\alpha$-substitutive sequence and $w$ is an $\beta$-substitutive sequence,
  then $v$ and $w$ have
 no common non-erasing transducts except for the ultimately periodic sequences. 
\label{theorem-apply-Durand}
\end{theorem}

\begin{proof} 
Let $h_v$ and $h_w$ be morphisms whose fixed points are $v$ and $w$, respectively.
  By the proof of Theorem~\ref{the:transducts_preserve_morphic}, 
  $x$ is a morphic image of an annotation  $\hbar_v$ 
  of $h_v$, and also of an annotation $\hbar_w$ of $h_w$.  
  The morphisms must be non-erasing, by the assumption in this theorem.
By  Corollary~\ref{corollary:dominant} and Theorem~\ref{thm:morphism}, 
  $x$ is both $\alpha$- and $\beta$-substitutive.
  By Durand's Theorem~\ref{theorem-Durand}, $x$ is eventually periodic.
\end{proof}

\subsection{Example}

We conclude the section with an example of Theorem~\ref{the:transducts_preserve_morphic}
and the lemmas in this section.

\begin{example}[]
  We saw the Fibonacci sequence  in Example~\ref{example-Fibonacci}:
  $$x = abaababaabaababaababaabaababaabaababaaba\cdots$$
  We conclude our series of examples pertaining to this sequence
  and the transducer $M$
  which doubles every other letter (see Example~\ref{example-transducer} and Figure~\ref{fig:double}).
  We want to exhibit $\hbar$,
  following the recipe of Lemma~\ref{lem:state_annotated_is_morphic}.  First,  some examples of how $\hbar$ works:
  $$\begin{array}{l@{\hspace{.4in}}l}
    (b, \Theta(a)) \mapsto (a, \Theta(ab)) & (a, \Theta(\e)) \mapsto (a, \Theta(\e))(b, \Theta(a)) \\ 
    (b, \Theta(ab))  \mapsto (a, \Theta(aba)) = (a, \Theta(b)) & (a, \Theta(a)) \mapsto (a, \Theta(ab))(b, \Theta(aba)) = (a, \Theta(ab))(b, \Theta(b))
  \end{array}$$

  It turns out that only a few elements from this $A$ end up appearing in the expressions for
  $\hbar(\sigma, \Theta(w))$: 
  It is convenient to abbreviate some of the elements of $\Sigma\times A$:  
  Let us use $x$ as an element of $\set{a,b}$, and  also
  write 
  $(x,\Theta(\e))$ as $x_0$, 
  $(x,\Theta(a))$ as $x_1$,
  $(x,\Theta(b))$ as $x_2$
  and $(x,\Theta(ab))$ as $x_3$.
  It turns out that we do not need to exhibit $\hbar$ in full because only eight points are reachable
  from $a_0$.
  We may take $\hbar$ to be 
  $$\begin{array}{l@{\hspace{.4in}}l@{\hspace{.4in}}l@{\hspace{.4in}}l}
    a_0 \mapsto a_0b_1 &  a_1  \mapsto  a_2 b_3 &  a_2  \mapsto a_3b_2 &  a_3   \mapsto a_1b_0 \\
    b_0  \mapsto a_0 &  b_1  \mapsto  a_2  & b_2  \mapsto a_3 &  b_3  \mapsto a_1
  \end{array}$$

  The fixpoint of this morphism starting with $a_0$ starts as $$ y = \hbar^\omega(a_0) = a_0 \ b_1 \ a_2 \ a_3 \ b_2 \ a_1 \ b_0 \ a_3 \ a_2 \ b_3 \ a_0 \ a_1 \ b_0 \ a_3 \ b_2 \ a_1 \ a_0 \ b_1 \ a_2 \ b_3 \ a_0 \ a_1 \ b_0 \ a_3 \ a_2 \ b_3 \ a_0 \ b_1 \  \cdots$$
  Turning to the coding $c$, recall that the set $Q$ of states of $M$ is $\set{s,t}$. 
  Let us abbreviate the elements of $\Sigma\times Q$ the same way we did
  with $\Sigma\times A$.   
  It is not hard to check that $c(\sigma_0) = \sigma_s$, $c(\sigma_1) = \sigma_t$,   
  $c(\sigma_2) = \sigma_s$, and
  $c(\sigma_3) = \sigma_t$.
  Then the state-annotated
  sequence $z$ from Lemma~\ref{lem:state_annotated_is_morphic} is
  $$z = c(y) = 
  a_s \ b_t \ a_s \ a_t \ b_s \ a_t \ b_s \ a_t \ a_s \ b_t \ a_s \ a_t \ b_s \ a_t \ b_s \ a_t \ a_s \ b_t \ a_s \ b_t \ a_s \ a_t \ b_s \ a_t \ a_s \ b_t \ a_s \ b_t \  \cdots$$


  Recall that $\lambda:\Sigma\times Q \to \Delta^* = \Sigma^*$ 
  in our transducer doubles whatever letter it sees while in state $s$ and copies whatever letter it sees while in state $t$.
  That is, $\lambda(x_s) = xx$, and $\lambda(x_t) = x$.
  Thus when we apply the morphism $\lambda$ to the sequence $z$, we get
  $$\lambda(z) = aa \ b \ aa \ a \ bb \ a \ bb \ a \ aa \ b \ aa \ a \ bb \ a \ bb \ a \ aa \ b \ aa \ b \ aa \ a \ bb \ a \ aa \ b \ aa \ b \  \cdots$$
  As we saw in the proof of Theorem~\ref{the:transducts_preserve_morphic}, this
  sequence 
  $$aa  b  aa  a  bb  a  bb  a  aa  b  aa  a  bb  a  bb  a  aa  b  aa  b  aa  a  bb  a  aa  b  aa  b   \cdots$$
  is exactly $M(x)$.  
\end{example}
