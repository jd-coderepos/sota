\documentclass{llncs}

\usepackage[usenames]{color}
\usepackage{array}
\usepackage{graphicx}
\usepackage[all]{xy}
\xyoption{dvips}
\usepackage{amssymb,amsmath}
\usepackage[mathcal]{euscript}
\usepackage{proof}

\usepackage{paper}


\title{A linear-non-linear model for a computational call-by-value lambda calculus\0ex]}
\newcommand{\mycaption}[1]{\caption{#1}\vspace{-5ex}}

\author{Peter Selinger\inst{1} \and Beno\^{\i}t Valiron\inst{2}}

\institute{
  Dalhousie University,
  \email{selinger@mathstat.dal.ca}
  \and
  University of Ottawa,
  \email{bvali087@uottawa.ca}
}

\begin{document}

\maketitle

\begin{abstract}
  We give a categorical semantics for a call-by-value linear lambda
  calculus. Such a lambda calculus was used by Selinger and Valiron as
  the backbone of a functional programming language for quantum
  computation. One feature of this lambda calculus is its linear type system,
  which includes a duplicability operator ``'' as in linear logic.
  Another main feature is its call-by-value reduction strategy,
  together with a side-effect to model probabilistic measurements. The
  ``'' operator gives rise to a comonad, as in the linear logic
  models of Seely, Bierman, and Benton. The side-effects give rise to
  a monad, as in Moggi's computational lambda calculus.  It is this
  combination of a monad and a comonad that makes the present paper
  interesting. We show that our categorical semantics is sound and
  complete.
\end{abstract}



\section{Introduction}

In the last few years, there has been some interest in the semantics
of quantum programming languages.
{\cite{selinger04quantum}} gave a denotational
semantics for a flow-chart language, but this language did not include
higher-order types. Several authors defined quantum lambda calculi
{\cite{tonder04lambda,selinger05lambda}} as well as quantum process
algebras {\cite{GaySJ:comqp,LalJor}}, which had higher-order features
and a well-defined operational semantics, but lacked denotational
semantics. \cite{valiron06fully} gave a categorical model for a
higher-order quantum lambda calculus, but omitted all the non-linear
features (i.e., classical data).  Meanwhile, Abramsky and Coecke
{\cite{abramsky04categorical,coecke04informationflow}}
developed categorical axiomatics for Hilbert spaces, but there is
no particular language associated with these models.

In this paper, we give the first categorical semantics of an
unabridged quantum lambda calculus, which is a version of the language
studied in {\cite{selinger05lambda}}. 

For the purposes of the present paper, an understanding of the precise
mechanics of quantum computation is not required. We will focus
primarily on the type system and language, and not on the structure of
the actual ``built-in'' quantum operations (such as unitary operators
and measurements). In this sense, this paper is about the semantics of
a generic call-by-value linear lambda calculus, which is parametric on
some primitive operations that are not further explained. It should be
understood, however, that the need to support primitive quantum
operations motivates particular features of the type system, which we
briefly explain now.

The first important language feature is linearity. This arises from
the well-known {\em no-cloning} property of quantum computation, which
asserts that quantum data cannot be duplicated {\cite{WZ82}}. So if
 is a variable representing a quantum bit, and  is a
variable representing a classical bit, then it is legal to write
, but not . 
In order to keep track of duplicability at higher-order types we use
a type system based on linear logic. We use the duplicability operator
``'' to mark classical types.
In the categorical semantics, this
operator gives rise to a comonad as in the work of {\cite{Seely89}}
and {\cite{benton92linear}}.
Another account of mixing copyable and non-copyable data is~\cite{coecke06quantum},
where the copyability is internal to objects.

A second feature of quantum computation is its probabilistic nature.
Quantum physics has an operation called measurement, which converts
quantum data to classical data, and whose outcome is inherently
probabilistic. Given a quantum state , a
measurement will yield output  with probability  and
 with probability .  To model this probabilistic effect
in our call-by-value setting, our semantics requires a computational
monad in the sense of {\cite{moggi91notions}}. The coexistence of the
computational monad  and the duplicability comonad  in the same
category is what makes our semantics interesting and novel. It differs
from the work of \cite{benton96linear}, who considered a monad and a
comonad one two different categories, arising from a single
adjunction.

The computational aspects of linear logic have been extensively
explored by many authors, including \cite{bierman93intuitionistic,benton92linear,benton93term,abramsky93computational,wadler92substitute}. However, these works contain explicit
lambda terms to witness the structural rules of linear logic, for
example, . By contrast, in our
language, structural rules are implicit at the term level, so that
 is regarded as a subtype of  and one writes . As we have shown in {\cite{selinger05lambda}}, linearity
information can automatically be inferred by the type checker.  This
allows the programmer to program as in a non-linear language. 

This use of subtyping is the main technical complication in our proof
of well-definedness of the semantics. This is because one has to show
that the denotation is independent of the choice of a potentially
large number of possible derivations of a given typing judgment. We
are forced to introduce a Church-style typing system, and to prove
that the semantics finally does not depend on the additional type
annotations.

Another technical choice we made in our language concerns the relation
between the exponential  and the pairing operation. Linear
logic only requires  and not the opposite implication. However, in our programming
language setting, we find it natural to identify a classical pair of
values with a pair of classical values, and therefore we will have an
isomorphism .

The plan of the paper is the following. First, we describe the lambda
calculus and equational axioms we wish to consider. Then, we develop a
categorical model, called linear category for duplication, which is
inspired by \cite{bierman93intuitionistic} and \cite{moggi91notions}.
We then show that the language is an internal language for the
category, thus obtaining soundness and completeness. 



\section{The language}
\label{sec:language}

We will describe a linear typed lambda calculus with higher-order
functions and pairs. The language is designed to manipulate both
classical data, which is duplicable, and quantum data, which is
non-duplicable. For simplicity, we assume the language is strictly
linear, and not affine linear as in {\cite{selinger05lambda}}. This
means duplicable values are both copyable and discardable, whereas
non-duplicable values must be used once, and only once.
\vspace{-2ex}

\subsection{The type system}

The set of types is given as follows: .

Here  ranges over type constants. While the remainder of this
paper does not depend on the choice of type constants, 
in our main application~\cite{selinger05lambda}
this is
intended to include a type  of quantum bits, and a type 
of classical bits.  stands for
functions from  to ,  for pairs,  for the
unit type, and  for duplicable objects of types . We
denote  with  's 
by .

The intuitive definition of  is the key to the spirit in
which we want the language to be understood: The  on
 is understood as specifying a property, rather than
additional structure, on the elements of . Therefore, we will have
. Whether or not a given value of type  is also of
type  should be something that is inferred, rather than specified
in the code.

Since a term of type  can always be seen as a term of type
, we equip the type system with a subtyping relation as follows:
Provided that ,
\vspace{-2ex}

This relation encapsulates the main properties terms should satisfy
with respect to duplicability.


\subsection{Terms}

The language consists of the following typed
terms, divided into \define{values} on the one hand, and general terms, or
\define{computations}, on the other. Both share a subset of the values, the
\define{core values}.
2ex]
  {\it Value}\ V,W &::= & U\bor \prodterm{V,W} \bor 
      \lettermx{x^A}{V}{W}\bor
      \nletprodterm{x^A,y^B}{n}{V}{W}\bor\\&&
      \letunitterm{V}{W},
      \
where  is an integer,  ranges over a set of constant terms, 
over a set of term variables and  over a set of constant
types. We abbreviate  by ,
 by  and
we omit numerical indexes when they are null.

Note that the above terms carry Church-style type annotations, as well as integer superscripts; we call these terms \define{indexed terms}. We also define a notion
of \define{untyped terms} as terms with no index:

The erasure operation  is defined
as the operation of removing the types and integers attached to a
given indexed term.
If , we say that  is an \define{indexation
of }

Finally, we define an -equivalence on terms, denoted by
, in the usual way (see for 
example \cite{barendregt84lambda}).


\subsection{Duplicable pairs and pairs of duplicable elements}

Before we formally present the type system, let us informally motivate
our choice of typing rules. One non-obvious choice we had to make is
for the interaction of pairs and duplicability.  Unlike previous works
with comonads \cite{bierman93intuitionistic,benton93term}, we want to
think of the type  as a type of pairs of elements
of type  and : we want to use the same operation to access the
components as one would use for a pair of type , without
having to use a dereliction operation.

This immediately raises a concern: consider a pair of elements
 of type . Are  and 
duplicable? In the usual linear logic interpretation, they are not. Having a infinite supply of pair of shoes does not mean
one has an infinite supply of right shoes: we cannot discard the left
shoes. On the other hand, in our interpretation of ``classical'' data as residing in ``classical'' memory and therefore being duplicable, if the string
 is duplicable, then so should be the elements  and .
In other words, we want the duplication to ``permeate'' the pairing.

The choice of such a ``permeable'' pairing is more
or less forced on us by our desire to have no explicit term syntax for
structural rules. Consider the following untyped terms, 
which can be typed if  is of type :

First, we expect these two terms to be axiomatically
equal. Term~\eqref{eq:permeable2} should be of type
, regardless of the permeability
of the pairing: if  is duplicable, so should be
. Now, consider the term~\eqref{eq:permeable1} with a
non-permeable pairing. In the naive type system,  ends up being of type , and the variables  and  in the final
recombination end up being respectively of type  and
. It is not possible to make  of the duplicable type
. 

We therefore choose a permeable pairing, which will be reflected,
albeit subtly, in the typing rule  and  in
the following section.


\subsection{Typing judgments}
\label{sec:typjudg}

A typing judgment
is a tuple , where  is an indexed term,  is a
type, and  is a typing context.
To each constant term  we assign a type .
A \define{valid typing judgment} is a typing judgment that can be
derived from the typing rules in Table~\ref{table:typrules}.
We use the notation  for a context where all variables
have a type of the form . Finally, when we write a context
, we assume the contexts  and  to be
disjoint.



\begin{table*}[t]
\def\mynl{\\\
\infer[({\it ax}_1)]{\bang\Delta, \typ{x}{A} 
  \entail x^B:B}{A \subtype B}

\infer[({\it ax}_2)]{\bang\Delta \entail c^B:B}{A_c \subtype B}

\infer[({\it app})]
      {\Gamma_1, \Gamma_2, \bang{\Delta} \entail M N : B}{
        \Gamma_1, \bang{\Delta} \entail M : A \loli B &
        \Gamma_2, \bang{\Delta} \entail N : A
      }

\infer[(\lambda_1)]
      { \Delta \entail \lambda^0 x^A.M : A \loli B}
      { \Delta, \typ{x}{A}\entail M : B}

\infer[(\lambda_2)]
      { \bang{\Delta} \entail \lambda^{n+1} x^A.M : 
        \nbang{n+1}{(A \loli B)}}
      { \bang{\Delta}, \typ{x}{A} \entail M : B}

\infer[(\putype.I)]{
  \bang\Delta \entail \puterm^n : \nbang{n}{\putype}
}{}

\infer[(\otimes.I)]{
  \bang{\Delta},\Gamma_1,\Gamma_2
  \entail
  \langle M_1,M_2\rangle^n : \nbang{n}{({A_1}\otimes{A_2})}
}{
  \bang{\Delta},\Gamma_1 \entail M_1 : \nbang{n}{A_1} &
  \bang{\Delta},\Gamma_2 \entail M_2 : \nbang{n}{A_2}
}

\infer[(\putype.E)]{
  \bang\Delta,\Gamma_1,\Gamma_2\entail
  \letunitterm{M}{N}:A
}{
  \bang\Delta,\Gamma_1\entail M:\putype
  &
  \bang\Delta,\Gamma_2\entail N:A
}

\infer[(\otimes.E)]{
  \bang\Delta, \Gamma_1, \Gamma_2 
  \entail
  \nletprodterm{x_1^{A_1},x_2^{A_2}}{n}{M}{N} :A
}{
  \bang{\Delta},
  \Gamma_1\entail M:\nbangp{n}{A_1\otimes A_2}
  &
  \bang{\Delta},
  \Gamma_2,x_1:\nbang{n}{A_1},x_2{:}\nbang{n}{A_2}\entail N:A
}
-1ex]}
\resizebox*{4in}{!}{
\begin{minipage}{\textwidth}



\end{minipage}}
\tablenl
\mycaption{Axiomatic equivalence axioms}
\label{table:eqaxrules}
\end{table*}




\begin{lemma}
  The equivalences of
  Table~\ref{table:eqaxderived} are 
  derivable. \qed
\end{lemma}


\begin{table*}[t]
\resizebox{4.2in}{!}{
\begin{minipage}{\textwidth}
-1ex]
  (\axoletlambB)\quad&
  \bang\Delta\entail\lettermx{x^{\bang{C}}}{V}{\lambda y.M}
  &&\eqax
  \lambda^{n+1} y.\lettermx{x^{\bang{C}}}{V}{M}
  &&:\nbangp{n+1}{A\loli B}
  \-1ex]
  (\axolettensA)\quad&
  \Delta\entail\prodterm{\lettermx{-}{M}{N},V} &&\eqax
  \lettermx{-}{M}{\prodterm{N,V}}&&:\nbangp{n}{A\tensor B}
  \-1ex]
  (\axoletappA)\quad&
  \Delta\entail (\lettermx{-}{M}{N})V &&\eqax
  \lettermx{-}{M}{NV}&&:B  

    \def\t{\tensor}\def\u{\putype}
    \begin{array}{c}
    \begin{array}{c}
    \xymatrix@C=10ex@R=6ex{
        LA\t LB
        \ar[dd]_{\cohmap{L}_{A,B}}
        \ar[r]^<(0.2){\comonoidmult_A\t\comonoidmult_B}
      & (LA\t LA)\t(LB\t LB)
        \ar[d]^{\swapmap}
        \\
      & (LA\t LB)\t(LA\t LB)
        \ar[d]^{\cohmap{L}_{A,B}\t\cohmap{L}_{A,B}}
        \\
        L(A\t B) 
        \ar[r]_<(0.2){\comonoidmult_{(A\t B)}}
      & L(A\t B)\t L(A\t B)
    }
    \end{array}
    \ 
    \begin{array}{c}
    \xymatrix@C=2ex@R=4ex{
         \u
         \ar[r]^<(0.3){\lambda^{-1}_{\u}}
         \ar[d]_{\cohmap{L}_{\u}}
       & \u\t\u 
         \ar[d]^{\cohmap{L}_{\u}\t\cohmap{L}_{\u}}
         \\
         L\u
         \ar[r]_<(0.3){\comonoidmult_{\u}}
       & L\u\t L\u
    }
    \ 
    \xymatrix@C=5ex@R=4ex{
        LA\t LB
        \ar[r]^<(0.3){\comonoidunit_A\t\comonoidunit_B}
        \ar[d]_{\cohmap{L}_{A,B}}
      & \u\t\u
        \ar[d]^{\lambda_{\u}}
        \\
        L(A\t B)
        \ar[r]_<(0.3){\comonoidunit_{A\t B}}
      & \u
    }
    \\
    \xymatrix@C=8ex@R=4ex{
      \u \ar[rr]^{\id}\ar[rd]_{\cohmap{L}_{\u}} && \u;\\
      &L\u\ar[ru]_{\comonoidunit_{\u}} &
    }
    \end{array}
    \\
    \textrm{ and 
      are monoidal natural transformations}.
    \end{array}
    
    \def\t{\tensor}\def\u{\putype}
    \begin{array}{cc}
      \begin{array}{c}
    \xymatrix@R=4ex{
        LA \ar[r]^{\comonoidmult_A}
        \ar[dd]_{\comult_A}
      & LA\t LA
        \ar[d]^{\comult_A\t\comult_A}
        \\
      & L^2A\t L^2A
        \ar[d]^{\cohmap{L}_{LA,LA}}
        \\
        L^2A
        \ar[r]_{L\comonoidmult_A}
      & L(LA\t LA),
    }
    \ 
    \xymatrix@R=12ex{
        LA \ar[r]^{\comonoidunit_A}
        \ar[d]_{\comult_A}
      & \u
        \ar[d]^{\cohmap{L}_{\u}}
        \\
        L^2A
        \ar[r]_{L\comonoidunit_A}
      & L\u;
    }
      \end{array}
    &
    \begin{array}{c}
    \xymatrix@R=5ex{
        LA \ar[r]^{\comult_A}
        \ar[d]_{\comonoidmult_A}
      & L^2A
        \ar[d]_{\comonoidmult_{LA}},
        \\
        LA\t LA \ar[r]_{\comult_A\t\comult_A}
      & L^2A\t L^2A
    }
    \xymatrix@C=2ex@R=5ex{
       LA \ar[rr]^{\comult_A}
       \ar[dr]_{\comonoidunit_A}
     &&L^2A.
       \ar[dl]^{\comonoidunit_{LA}}
       \\
       &\u&
    }
    \end{array}
    \\
    \textrm{ and  are -coalgebra maps.}
    &
    \textrm{ us a comonoid morphism.}
    \end{array}
    \begin{array}{ll}
    \Strength_1:TA\tensor TB\xrightarrow{\sigma_{TA,TB}}
    TB\tensor TA\xrightarrow{t_{TB,A}}
    T(TB\tensor A)
    \xrightarrow{(\sigma_{TB,A};t_{A,B})^*}
    T(A\tensor B),
    \\
    \Strength_2:TA\tensor TB\xrightarrow{t_{TA,B}}
    T(TA\tensor B)
    \xrightarrow{(\sigma_{TA,B};t_{B,A})^*}
    T(B\tensor A) \xrightarrow{T\sigma_{B,A}}
    T(A\tensor B).
  \end{array}
  \Phi:\cat{C}(A, B\loli C) \xrightarrow{\quad\cong\quad}
  \cat{C}(A\tensor B, TC).
  
\begin{array}{llrll}
  \alpha_{A,B,C} &=& x:A{\tensor}(B{\tensor}C)&\entail
  \letterm\prodterm{y,z}=x\interm
  \letterm\prodterm{t,u}=z\interm
  \prodterm{\prodterm{y,t},u}&:(A{\tensor}B){\tensor} C
  \\
  \lambda_A &=& x:\tensunit\tensor A &\entail
  \letprodterm{y,z}{x}{\letunitterm{y}{z}}&:A
  \\
  \rho_A&=& x:A\tensor\tensunit &\entail
  \letprodterm{y,z}{x}{\letunitterm{z}{y}}&:A
  \\
  \sigma_{A,B}&=&x:A\tensor B&\entail
  \letprodterm{y,z}{x}{\prodterm{z,y}}&:B\tensor A
  \\
  \unit_A &=& x:A&\entail\lambdau.x&:\comptype A
  \\
  \mult_A &=& x:\comptype(\comptype A)&\entail
  \lambdau.(x\puterm)\puterm&: \comptype A
  \\
  \strength_{A,B} &=& z:A\tensor(\comptype B)&\entail
  \letprodterm{x,y}{z}{\lambdau.\prodterm{x,y\puterm}}&:
  \comptype(A\tensor B)
  \\
  \counit_A&=& x:\bang{A}&\entail x^A&:A
  \\
  \comult_A&=& x:\bang{A}&\entail x^{\nbang{2}{A}}&:\nbang{2}{A}
  \\
  \cohmap{\bang{}}_{A,B} &=&
  z:\bang{A}\tensor\bang{B}&\entail
  \letprodterm{x,y}{z}{\prodterm{x,y}}&: \bangp{A\tensor B}
  \\
  \cohmap{\bang{}}_{\putype} &=&
  z:\putype&\entail \letunitterm{z}{\puterm}&:\bang{\putype}
  \\
  \comonoidmult_A&=&x:\bang{A}&\entail
  \prodterm{x,x}&:\bang{A}\tensor\bang{A}
  \\
  \comonoidunit_A&=&x:\bang{A}&\entail
  \puterm&:\putype
\end{array}

\begin{array}{lcrll}
  (x:A\entail V:B)\tensor(y:C\entail W:D)
  &=& z:A\tensor B&\entail \letprodterm{x,y}{z}{\prodterm{V,W}}
  &:C\tensor D
  \\
  (x:A\entail V:B)\loli(y:C\entail W:D) &=& z:B\loli C&\entail
  \lambda x.(\lettermx{y}{zV}{W})&:A\loli D
  \\
  (x:A\entail V:\comptype B)^*&=&
  y:\comptype A&\entail \lambdau.\lettermx{x}{(y\puterm)}{(V\puterm)}
  &:\comptype B
  \\
  \Phi_{A,B,C} \left(x:A\entail V:B\loli C\right) & = &
  t:A\tensor B&\entail \lambda\puterm.\letprodterm{x,y}{t}{Vy}
  &:\putype\loli C
\end{array}
-1ex]}
\resizebox{4.6in}{!}{
\begin{minipage}{6.5in}
Interpretation of core values:
\vspace{-2ex}

\vspace{-1.5ex}
Interpretation of extended values:




\vspace{-1.5ex}
Interpretation of computations:
First, if  is a core value, .




\end{minipage}}
\tablenl
\mycaption{Interpretation of values and computations.}
\label{table:modelvaluejudg}
\label{table:modelcorevaluejudg}
\label{table:modelcompjudg}
\end{table*}




As we already noted in Section~\ref{sec:typjudg}, a valid typing
judgment does not have a unique typing tree {\em per se}. However the
following result holds:




\begin{theorem}
  Given a valid typing judgment with two typing derivations 
  and , for any interpretation  we have
   (and
   if the typing
  judgment is a value).
\end{theorem}

\begin{proof}
  The proof is done by showing that given any typing judgment
   with denotation  one can factor  as
  , where  is the denotation of ,
  where  and  is the set of
  dummy variables.\qed
\end{proof}

\begin{definition}\label{def:typjudgdenot}\rm
  Given a interpretation  of the language in a category
  , we define the denotation of a valid typing judgment
   with typing derivation  to be
   and
   if 
  is a value.
\end{definition}



\begin{lemma}\label{lem:valuemonad}
  Suppose that  is a valid typing judgment where
   is a value. Then .
\end{lemma}

\begin{proof}
  Proof by induction on , using
  Lemma~\ref{lem:strongmonadmonoidal}, the bifunctoriality of
   and the equations for strong monadicity in
  Definition~\ref{def:strongmonad}.\qed
\end{proof}

\subsection{Soundness of the denotation}

The axiomatic equivalence and the categorical semantics are two faces
of the same coin.
Indeed, as we will prove in this section, two terms
in the same axiomatic equivalence class have the same
denotation. A corollary is that the indexation of terms does not
influence the denotation. This proves semantically the fact that it is
safe to work with untyped terms. An alternate justification of this
fact is of course the operational semantics, which was given in
{\cite{selinger05lambda}}.

\begin{lemma}\label{lem:denottypecast}
  Suppose .
  Then
  .
  If  is a value, from Lemma~\ref{lem:typecastvalue}, 
   is a value. Then
  .
\end{lemma}

\begin{proof}
  Proof by structural induction on .\qed
\end{proof}


\begin{lemma}[Substitution]\label{lem:substitution}
  Given two valid typing judgments  and , the typing judgment
   is valid. Let  be
  
  and  be ,
  in the case where  is a value. Then they are defined by\\
  \resizebox*{4.85in}{!}{
  }
\end{lemma}

\begin{proof}
  Proof by induction on , using Lemma~\ref{lem:duplicvalue},
  Lemma~\ref{lem:valuemonad} and the naturality of .\qed
\end{proof}



\begin{theorem}\label{the:aximpliesdenot}
  If  then
   (and  if  is a value) for every interpretation .
\end{theorem}

\begin{proof}
  Proof by induction on , using
  Lemmas~\ref{lem:denottypecast} and~\ref{lem:substitution}.\qed
\end{proof}


\begin{corollary}\label{cor:erasuredenot}
  If  and if  are valid
  typing judgments, then  (and
   if  is a value).
\end{corollary}

\begin{proof}
  Corollary of Theorems~\ref{the:eraseimpliesax}
  and~\ref{the:aximpliesdenot}.\qed
\end{proof}



\subsection{Completeness}


The category  being a linear category for duplication, one
can interpret the language in it. This section states that the defined
lambda-calculus is an \define{internal language} of linear categories
for duplication.


Since the category  is a monoidal category, one can
w.l.o.g. generalize the notion of pairing to finite tensor products of
terms. Then the following results are true:



\begin{lemma}\label{lem:yaqcomplete}
  In , a valid typing judgment  has for computational denotation
  . If  is a value, the value
  interpretation is
  .
\end{lemma}

\begin{proof}
  Proof by structural induction on  and .\qed
\end{proof}

\begin{theorem}
  In ,  being the identity, one has
   and
  .
\end{theorem}

\begin{proof}
  Corollary of Lemma~\ref{lem:yaqcomplete}\qed
\end{proof}

\section{Towards a denotational model of quantum lambda calculus}

As noted in the introduction, this paper is mostly concerned with the
categorical requirements for modeling a generic call-by-value linear
lambda calculus, i.e., its type system (which includes subtyping) and
equational laws. We have not yet specialized the language to a
particular set of built-in operators, for example, those that are
required for quantum computation.

However, since the quantum lambda calculus {\cite{selinger05lambda}}
is the main motivation behind our work, we will comment very briefly
on what additional properties would be required to interpret its
primitives. The quantum lambda calculus is obtained by instantiating
and extending the call-by-value language of this paper with the
following primitive types, constants, and operations:

\begin{center}
  \begin{tabular}{ll}
    Types: & ,  \\
    Constants: & , \\
    & , , \\
    Operations: & 
  \end{tabular}
\end{center}

Here,  ranges over a set of built-in unitary operations. In the
intended semantics, , while  is
empty.  creates a new qubit, and  measures a
qubit. 

The denotational semantics of these operations is already
well-understood in the absence of higher-order types. They can all be
interpreted in the category  of superoperators from
{\cite{selinger04quantum}}.  The part that is not yet well-understood
is how these features interact with higher-order types. 

In light of our present work, we can conclude that a model of the
quantum lambda calculus consists of a linear category for duplication
, such that the associated category of
computations  contains the category  of
{\cite{selinger04quantum}} as a full monoidal subcategory.  To
construct an actual instance of such a model is still an open problem.

\section{Conclusion}

We have developed a call-by-value, computational lambda-calculus for
manipulating duplicable and non-duplicable data, together with an
axiomatic equivalence relation on typed terms. We use a subtyping
relation in order to have implicit promotion, dereliction, copying and
discarding. 
Then we developed categorical model for the language, inspired by
the work of \cite{bierman93intuitionistic} and \cite{moggi91notions}.
We finally showed that the model is sound and complete with respect to
the axiomatic equivalence.




\begin{thebibliography}{10}

\bibitem{abramsky93computational}
Abramsky, S.:
\newblock Computational interpretations of linear logic.
\newblock Theoretical Computer Science \textbf{111} (1993)  3--57

\bibitem{abramsky04categorical}
Abramsky, S., Coecke, B.:
\newblock A categorical semantics of quantum protocols.
\newblock In: Proceedings of LICS'04. (2004)  415--425

\bibitem{barendregt84lambda}
Barendregt, H.P.:
\newblock The Lambda-Calculus, its Syntax and Semantics.
\newblock North Holland (1984)

\bibitem{benton94mixed}
Benton, N.:
\newblock A mixed linear and non-linear logic: Proofs, terms and models
  (extended abstract).
\newblock In: Proceedings of CSL'94, Selected Papers. Volume 933 of Lecture
  Notes in Computer Science. (1994)  121--135

\bibitem{benton93term}
Benton, N., Bierman, G., de~Paiva, V.C.V., Hyland, M.:
\newblock A term calculus for intuitionistic linear logic.
\newblock In: Proceedings of TLCA'93. Volume 664 of Lecture Notes in Computer
  Science. (1993)  75--90

\bibitem{benton92linear}
Benton, N., Bierman, G., Hyland, M., de~Paiva, V.C.V.:
\newblock Linear lambda-calculus and categorical models revisited.
\newblock In: Proceedings of CSL'92, Selected Papers. Volume 702 of Lecture
  Notes in Computer Science. (1992)

\bibitem{benton96linear}
Benton, N., Wadler, P.:
\newblock Linear logic, monads and the lambda calculus.
\newblock In: Proceedings of LICS'96. (1996)  420--431

\bibitem{bierman93intuitionistic}
Bierman, G.:
\newblock On Intuitionistic Linear Logic.
\newblock PhD thesis, Computer Science Department, Cambridge University (1993)

\bibitem{coecke04informationflow}
Coecke, B.:
\newblock Quantum information-flow, concretely, abstractly.
\newblock  \cite{qpl04}  57--73

\bibitem{coecke06quantum}
Coecke, B., Pavlovic, D.:
\newblock Quantum measurements without sums.
\newblock In Chen, G., Kauffman, L., Lomonaco, S.J., eds.: Mathematics of
  Quantum Computation and Technology.
\newblock Chapman \& Hall (2007)  559--596

\bibitem{GaySJ:comqp}
Gay, S.J., Nagarajan, R.:
\newblock Communicating quantum processes.
\newblock In: Proceedings of POPL'05, ACM Press (2005)

\bibitem{LalJor}
Lalire, M., Jorrand, P.:
\newblock A process algebraic approach to concurrent and distributed
  computation: operational semantics.
\newblock  \cite{qpl04}  109--126

\bibitem{maclane98categories}
Mac~Lane, S.:
\newblock Categories for the Working Mathematician.
\newblock Springer Verlag (1998)

\bibitem{moggi91notions}
Moggi, E.:
\newblock Notions of computation and monads.
\newblock Information and Computation \textbf{93} (1991)  55--92

\bibitem{schalk04what}
Schalk, A.:
\newblock What is a model for linear logic.
\newblock Manuscript (2004)

\bibitem{Seely89}
Seely, R.A.G.:
\newblock *-autonomous categories and cofree coalgebras.
\newblock Contemporary Mathematics \textbf{92} (1989)

\bibitem{qpl04}
Selinger, P., ed.:
\newblock Proceedings of QPL'04.
\newblock TUCS General Publication No 33, Turku Centre for Computer Science
  (2004)

\bibitem{selinger04quantum}
Selinger, P.:
\newblock Towards a quantum programming language.
\newblock Mathematical Structures in Computer Science \textbf{14} (2004)
  527--586

\bibitem{selinger05lambda}
Selinger, P., Valiron, B.:
\newblock A lambda calculus for quantum computation with classical control.
\newblock Mathematical Structures in Computer Science \textbf{16} (2006)
  527--552

\bibitem{valiron06fully}
Selinger, P., Valiron, B.:
\newblock On a fully abstract model for a quantum linear functional language.
\newblock In: Preliminary proceedings of QPL'06. (2006)  103--115

\bibitem{tonder04lambda}
van Tonder, A.:
\newblock A lambda calculus for quantum computation.
\newblock SIAM Journal of Computing \textbf{33} (2004)  1109--1135

\bibitem{wadler92substitute}
Wadler, P.:
\newblock There's no substitute for linear logic.
\newblock Manuscript, presented at MFPS'92 (1992)

\bibitem{WZ82}
Wootters, W.K., Zurek, W.H.:
\newblock A single quantum cannot be cloned.
\newblock Nature \textbf{299} (1982)  802--803

\end{thebibliography}



\end{document}
