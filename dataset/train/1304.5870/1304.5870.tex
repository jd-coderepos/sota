\documentclass[11pt,a4paper]{article}
\usepackage{amssymb,amsmath,amsthm}
\usepackage{graphicx,color}
\usepackage{xspace}



\tolerance2000
\newtheorem{theorem}{Theorem}
\newtheorem{observation}{Observation}
\newtheorem{proposition}{Proposition}
\newtheorem{corollary}{Corollary}
\newtheorem{lemma}{Lemma}
\newtheorem{claim}{Claim}
\newtheorem{define}{Definition}


\DeclareMathOperator{\operatorClassP}{P}
\newcommand{\classP}{\ensuremath{\operatorClassP}}
\DeclareMathOperator{\operatorClassXP}{XP}
\newcommand{\classXP}{\ensuremath{\operatorClassXP}}
\DeclareMathOperator{\operatorClassNP}{NP}
\newcommand{\classNP}{\ensuremath{\operatorClassNP}}
\DeclareMathOperator{\operatorClassPH}{PH}
\newcommand{\classPH}{\ensuremath{\operatorClassPH}}
\DeclareMathOperator{\operatorClassFPT}{FPT}
\newcommand{\classFPT}{\ensuremath{\operatorClassFPT}}
\DeclareMathOperator{\operatorClassW}{W}
\newcommand{\classW}[1]{\ensuremath{\operatorClassW[#1]}}
\newcommand{\classWhier}{\ensuremath{\operatorClassW}}
\DeclareMathOperator{\operatorClassM}{M}
\newcommand{\classM}[1]{\ensuremath{\operatorClassM[#1]}}
\newcommand{\classMhier}{\ensuremath{\operatorClassM}}



\newcommand{\rnote}[1]{{\bf \textcolor[rgb]{1.00,0.00,0.00}{Rajesh}: #1}}

\newcommand{\dakc}{\textsc{Directed Anchored -Core}\xspace}
\newcommand{\DAKC}{\textsc{Dir-AKC}\xspace}
\newcommand{\PDAKC}{\textsc{-Dir-AKC}\xspace}
\newcommand{\sat}{\textsc{Restricted-Planar-3-SAT}\xspace}
\newcommand{\psc}{\textsc{Partial Set Cover}\xspace}



\newcommand*\samethanks[1][\value{footnote}]{\footnotemark[#1]}




\begin{document}


\title{Parameterized Complexity of the Anchored -Core Problem for Directed Graphs\footnote{This work is supported by the European Research Council (ERC) via grant Rigorous Theory of Preprocessing, reference
267959 and by NSF CAREER award 1053605, NSF grant CCF-1161626, ONR YIP award N000141110662, DARPA/AFOSR grant FA9550-12-1-0423, a University
of Maryland Research and Scholarship Award (RASA) and a Summer International Research Fellowship from the University of
Maryland.
}}

\author{Rajesh Chitnis\thanks{Department of Computer Science , University of Maryland at College Park, USA. Email: {\tt rchitnis@cs.umd.edu.}}
\and
Fedor V. Fomin\thanks{Department of Informatics, University of Bergen, PB 7803, 5020 Bergen, Norway. Email: {\tt{\{fedor.fomin, petr.golovach\}@ii.uib.no}}}
\addtocounter{footnote}{-1}
\and 
Petr A. Golovach\footnotemark
}




\date{}

\maketitle


\begin{abstract} Motivated by the study of  unraveling processes in social networks,
Bhawalkar,   Kleinberg,   Lewi,   Roughgarden,  and Sharma [ICALP 2012] introduced the \textsc{Anchored -Core} problem,
where the task is  for a given graph  and integers , and  to find an induced subgraph   with at least 
vertices (the core) such that all but at most  vertices (called anchors) of   are of
degree at least . In this paper, we extend the notion of -core to directed graphs and provide a number of new
algorithmic and complexity results for the directed version of the problem. We show that
\begin{itemize}
\item
The decision version of the problem is   \classNP-complete for every   even if the input graph is restricted to be a planar directed acyclic graph of maximum degree at
most .
\item The problem is fixed parameter tractable (\classFPT) parameterized by the size of the core  for  , and \classW1-hard for .
\item When the maximum degree of the graph is at most ,   the
 problem is  \classFPT\  parameterized by  if .
\end{itemize}
\end{abstract}



\section{Introduction}\label{sec:intro}
The anchored -core problem can be explained by the following illustrative example. We want
to organize a workshop on Theory of Social Networks. We send invitations to most  distinguished researchers in the area and
received many  replies of the following nature: ``Yes, in theory, I would be happy to come but my final decision  depends on
how many   people I know will be there."  Thus  we have a list of  tentative participants, but some of them can cancel their
participation and we are afraid that the cancellation process
may escalate. On the other hand, we also have  limited funds to reimburse travel expenses for
a small number of participants, which we believe, will guarantee their participation. Thus we want to ``anchor" a small subset
of  participants whose guaranteed participation would prevent the unraveling process, and by
fixing a small group   we hope to minimize the number of cancellations, or equivalently,
maximize the number of participants, or the core.

Unraveling processes are common for social networks where the behavior of an individual is often influenced by the actions of
her/his friends. New events occur quite often in social networks: some examples are usage of a particular cell phone brand,
adoption of a new drug within the medical profession, or the rise of a political movement in an unstable society. To estimate
whether these events or ideas spread extensively or die out soon, one has  to model and study the dynamics of \emph{influence
propagation} in social networks. Social networks are generally represented by making use of  undirected or directed graphs,
where the edge set represents  the relationship between individuals in the network. Undirected graph  model works fine for
some networks, say Facebook, but the nature of interaction on some social networks such as Twitter is asymmetrical:  the fact
that  user  follows  user  does not imply that that user  also follows .\footnote{The first author follows LeBron
James on Twitter (and so do 8,017,911 other people), but he only follows 302 people with the first author not being one of
them.}  In this case, it is more appropriate to model interactions in  the network by \textbf{directed} graphs. We add a
directed edge  if  follows .


In this work we are interested in the  model of \emph{user engagement}, where each individual with less than  people to
follow (or equivalently whose in-degree is less than ) drops out of the network. This
process can be contagious, and may affect even those individuals who initially were linked to more than  people, say follow
on Twitter. An extreme example of this was given by Schelling (see page 17 of ~\cite{schelling2006micromotives}): consider a
directed path on  vertices and let . The left-endpoint has in-degree zero, it drops out and now the in-degree of its
only out-neighbor in the path becomes zero and it drops out as well. It is not hard to see that this way the whole network
eventually drops out as the result of a \emph{cascade of iterated withdrawals}. In general at the end of all the iterated
withdrawals the remaining engaged individuals form a unique maximal induced subgraph whose minimum in-degree is at least .
This is called as the \emph{-core} and is a well-known concept in the theory of social networks. It was introduced by
Seidman~\cite{seidman-k-core} and also been studied in various social sciences
literature~\cite{chwe1999structure,chwe2000communication}.

\medskip
\noindent\textbf{Preventing Unraveling:} The unraveling process described above in Schelling's example of a directed path can
be  highly undesirable in many scenarios.  How can one attempt to prevent this unraveling? In Schelling's example it is easy
to see: if we ``buy" the left end-point person into being engaged then the whole path becomes engaged. In general we overcome
the issue of unraveling by allowing some ``anchors":  these are the vertices that remain engaged irrespective of their
payoffs. This can be achieved by giving them extra incentives or discounts. The hope is that with a few \emph{anchors} we can
now ensure a large subgraph remains engaged. This subgraph is  called as the \emph{anchored -core}: each non-anchor vertex
in this induced subgraph must have in-degree at least  while the anchored vertices can have arbitrary in-degrees. The
problem of identifying -cores in a network also has the following  game-theoretical interpretation   introduced by
Bhawalkar et al.~\cite{BhawalkarKLRS12}: each user in the social network pays a cost of  to remain engaged. On the other
hand, he/she receives a profit of one from every neighbor who is engaged. The ``network effects" come into play, and an
individual decides to remain engaged if has non-negative payoff, i.e., it has at least  in-neighbors who are engaged. The
-core can be viewed as the unique maximal equilibrium in this model.


Bhawalkar et al.~\cite{BhawalkarKLRS12} introduced  the \textsc{Anchored -Core} problem for (undirected) graphs. In the
\textsc{Anchored -Core} problem the input is an undirected graph  and integers , and the task is to find  an
induced subgraph  of maximum size with all vertices but at most  (which are anchored)
to be of degree at least . In this work we extend the
notion of {anchored -core} to directed graphs.  We are interested in the case, when  in-degrees of all but  vertices of
 are at least  .  More formally, we study the following parameterized version of the problem.
\begin{center}
\noindent\framebox{\begin{minipage}{4.50in} { \textsc{Directed Anchored -Core}  (\DAKC)}\\
\emph{Input}: A directed graph  and integers . \\
\emph{Parameter~1}: .\\
\emph{Parameter~2}: .\\
\emph{Parameter~3}: .\\
\emph{Question}:  Do there exist sets of vertices  such that , , and every
 satisfies ?
\end{minipage}}
\end{center}
We will call the set  as the \emph{anchors}, the graph  as the \emph{anchored -core}. Note that the undirected
version of \textsc{Anchored -Core} problem can be modeled by the directed version: simply replace each edge  by
arcs  and . Keeping the parameters  unchanged it is now easy to see that the two instances are
equivalent.



\medskip
\noindent\textbf{Parameterized Complexity:} We are mainly interested in the parameterized complexity of  \textsc{Anchored
-Core}. For the general background, we refer to the books by Downey and Fellows~\cite{downey-fellows-book},  Flum and
Grohe~\cite{flum-grohe-book} and Niedermeier~\cite{niedermeier-book}. Parameterized complexity is basically a two dimensional
framework for studying the computational complexity of a problem. One dimension is the input size  and another one is a
parameter . A problem is said to be \emph{fixed parameter tractable} (or \classFPT) if it can be solved in time  for some function . A problem is said to be in \classXP, if it can be solved in time  for some
function . The -hierarchy is a collection of computational complexity classes: we omit the technical
definitions here. The following relation is known amongst the classes in the -hierarchy:
. It is widely believed that , and hence if a
problem is hard for the class  (for any ) then it is considered to be fixed-parameter intractable.

\medskip
\noindent\textbf{Previous Results:} Bhawalkar et al.~\cite{BhawalkarKLRS12} initiated the algorithmic  study of
\textsc{Anchored -Core}  on undirected graphs and obtained an interesting dichotomy result:  the decision version of the
problem is solvable in polynomial time for  and is  \classNP-complete for all . For , they also
studied the problem from the viewpoint of parameterized complexity and approximation algorithms.
The current set of authors~\cite{ChitnisFG13} improved and generalized these results by showing that for  the problem
remains \classNP-complete even on planar graphs.


\medskip\noindent\textbf{Our Results:} In this paper we provide a number of new results on the algorithmic complexity of
\textsc{Directed Anchored -Core}  (\DAKC).
We start (Section~\ref{sec:defs}) by showing that that the decision version of \DAKC{} is \classNP-complete for every   even if the input graph is restricted to be a planar directed acyclic graph (DAG) of maximum degree at most . Note
that this shows that the directed version is in some sense strictly harder than the undirected version since it is known be in
\classP\ if , and \classNP-complete if ~\cite{BhawalkarKLRS12}. The \classNP-hardness result for \DAKC
motivates us to make a more refined analysis of the \DAKC problem via the paradigm of parameterized complexity. In
Section~\ref{sec:saved}, we  obtain the following dichotomy result: \DAKC is \classFPT \, parameterized by  if , and
\classW1-hard if . This fixed-parameter intractability result parameterized by  forces us to consider the
complexity on special classes of graphs such as bounded-degree directed graphs or directed acyclic graphs. In
Section~\ref{sec:bound-deg}, for graphs of degree upper bounded by , we show that the \DAKC problem is FPT
parameterized by  if . In particular, it implies that \DAKC is FPT parameterized by  for
directed graphs of maximum degree at most four. We complement these results by showing in Section~\ref{sec:concl} that  if  and , then \DAKC is \classW2-hard when parameterized by the number of anchors  even for
DAGs, but the problem is \classFPT\ when parameterized by  for DAGs of maximum degree at most . Note that we
can always assume that , and hence any \classFPT\ result with parameter  implies \classFPT\ result with parameter
 as well. On the other side, any hardness result with respect to  implies the same hardness with respect to .

\section{Preliminaries}\label{sec:defs}
We consider finite directed and undirected graphs without loops or multiple arcs. The vertex set of a (directed) graph  is
denoted by  and its edge set (arc set for a directed graph) by . The subgraph of  induced by a subset
 is denoted by . For  by  we denote the graph . For a directed
graph , we denote by  the undirected graph with the same set of vertices such that  if and only if
. We say that  is the \emph{underlying} graph of .

Let  be a directed graph. For a vertex , we say that  is an \emph{in-neighbor} of  if . The
set of all in-neighbors of  is denoted by . The \emph{in-degree} . Respectively,  is an
\emph{out-neighbor} of  if , the set of all out-neighbors of  is denoted by , and the
\emph{out-degree} . The \emph{degree}  of a vertex  is the sum , and the
\emph{maximum degree} of  is . A vertex  of  is called a \emph{source},
and if  , then  is a \emph{sink}. Observe that isolated vertices are sources and sinks simultaneously.



Let  be a directed graph. For , it is said that  can be \emph{reached} (or \emph{reachable}) from  if
there is a directed  path in . Respectively, a vertex  can be reached from a set  if
 can be reached from some vertex . Notice that each vertex is reachable from itself. We denote by 
( respectively) the set of vertices that can be reached from a vertex  (a set  respectively).
Let  denote the set of all vertices  such that  can be reached from .

For two non-adjacent vertices  of a directed graph , a set  is said to be a
\emph{ separator} if . An  separator  is \emph{minimal} if no proper subset 
is a  separator.

The notion of important separators was introduced by Marx~\cite{Marx06} and generalized for directed graphs
in~\cite{ChitnisHM12}. We need a special variant of this notion. Let  be a directed graph, and let  be non-adjacent
vertices of . An minimal  separator is an \emph{important  separator} if there is no  separator  with
 and . The following lemma is a variant of Lemma~4.1 of~\cite{ChitnisHM12}.
Notice that to obtain it, we should replace the directed graph in Lemma~4.1 of~\cite{ChitnisHM12} by the graph obtained from
it by reversing direction of all arcs.


\begin{lemma}[\cite{ChitnisHM12}]\label{lem:imp-sep}
Let  be a directed graph with  vertices, and let  be non-adjacent vertices of . Then for every , there
are at most  important  separators of size at most . Furthermore, all these separators can be enumerated in time
.
\end{lemma}


As further we are interested in the parameterized complexity of \DAKC, we show first \classNP-hardness of the problem. 


\begin{theorem}\label{thm:NPc}
For any , \DAKC is \classNP-complete, even for planar DAGs of maximum degree at
most .
\end{theorem}


\begin{proof}
We reduce {\sc Satisfiability}:
\begin{center}
\noindent\framebox{\begin{minipage}{4.50in}{Satisfiability}\\
\emph{Input}: Sets of Boolean variables  and clauses . \\
\emph{Question}: Can the formula  be satisfied?
\end{minipage}}
\end{center}
It is known (see e.g.~
\cite{DahlhausJPSY94}) that this problem remains \classNP-hard even if each clause contains at most 3 literals (notice that clauses of size one or two are allowed), each variable
is used in at most 3 clauses: at least once in positive and at least once in negation, and the graph that correspond to a
boolean formula is planar. Consider an instance of {\sc Satisfiability} with  variables  and  clauses
 that satisfies these restrictions on planarity and the number of occurrences of the variables. We construct the graph 
as follows.


\begin{figure}[ht]
\centering\scalebox{0.7}{\begin{picture}(0,0)\includegraphics{fig1.pdf}\end{picture}\setlength{\unitlength}{3947sp}\begingroup\makeatletter\ifx\SetFigFont\undefined \gdef\SetFigFont#1#2#3#4#5{\reset@font\fontsize{#1}{#2pt}\fontfamily{#3}\fontseries{#4}\fontshape{#5}\selectfont}\fi\endgroup \begin{picture}(7859,3190)(530,-4282)
\put(5675,-4192){\makebox(0,0)[lb]{\smash{{\SetFigFont{12}{14.4}{\rmdefault}{\mddefault}{\updefault}{\color[rgb]{0,0,0}}}}}}
\put(4331,-2101){\makebox(0,0)[lb]{\smash{{\SetFigFont{12}{14.4}{\rmdefault}{\mddefault}{\updefault}{\color[rgb]{0,0,0}}}}}}
\put(3761,-2311){\makebox(0,0)[lb]{\smash{{\SetFigFont{12}{14.4}{\rmdefault}{\mddefault}{\updefault}{\color[rgb]{0,0,0}}}}}}
\put(4851,-2311){\makebox(0,0)[lb]{\smash{{\SetFigFont{12}{14.4}{\rmdefault}{\mddefault}{\updefault}{\color[rgb]{0,0,0}}}}}}
\put(5691,-1341){\makebox(0,0)[lb]{\smash{{\SetFigFont{12}{14.4}{\rmdefault}{\mddefault}{\updefault}{\color[rgb]{0,0,0}}}}}}
\put(5421,-1761){\makebox(0,0)[lb]{\smash{{\SetFigFont{12}{14.4}{\rmdefault}{\mddefault}{\updefault}{\color[rgb]{0,0,0}}}}}}
\put(4384,-3046){\makebox(0,0)[lb]{\smash{{\SetFigFont{12}{14.4}{\rmdefault}{\mddefault}{\updefault}{\color[rgb]{0,0,0}}}}}}
\put(5377,-3731){\makebox(0,0)[lb]{\smash{{\SetFigFont{12}{14.4}{\rmdefault}{\mddefault}{\updefault}{\color[rgb]{0,0,0}}}}}}
\end{picture} }
\caption{Construction of  for .
\label{fig:NPh}}
\end{figure}



\begin{itemize}
\item For each ,
\begin{itemize}
\item add vertices  and add arcs ;
\item add a set of  vertices  and draw an arc from each of them to ;
\item for each vertex , add  vertices and draw an arc from each of them to , denote the set of these
     vertices .
\end{itemize}
\item For each ,
\begin{itemize}
\item add a vertex , and for each literal  ( respectively) in the clause , join the
    vertex  ( respectively) with  by an arc;
\item add a set of  vertices  and draw an arc from each of them to ;
\item for each vertex , add  vertices and draw an arc from each of them to , denote the set of these
     vertices .
\end{itemize}
\end{itemize}
Notice that if , then . The construction of  is shown in Fig.~\ref{fig:NPh}. We set
 and . It is straightforward to see that  is acyclic. Because
each variable  is used at most 2 times in positive and at most 2 times in negations,  for all , and . 
Because the graph of the boolean formula is a subcubic planar graph,  
 is planar.

We claim that all clauses  can be satisfied if and only if there are a set 
and an induced subgraph  of  such
that , , and for every , we have .

Suppose that we have a YES-instance of {\sc Satisfiability} and consider a truth assignment of  such that
all clauses are satisfied. We construct  by including all the vertices  in
this set, and for each , if , then  is included in  and  is
included otherwise. Clearly, . Let . Consider . If
 for , then  has  in-neighbors in . If  for ,
then  has  in-neighbors in  and either  or  is an in-neighbor of  as well. If 
for , then  has  in-neighbors in . Finally, if  for some
, then  has  in-neighbors in . As the clause  is satisfied, it contains a literal 
or  that has the value . Then by the construction of , the corresponding vertex  or
 respectively is in , and  has one in-neighbor in . It remains to observe that
.

Assume now there are a set 
and an induced subgraph  of  such
that ,  and for every  we have .

Let  and
. 
 We claim that
 and . To show it, observe that any vertex  is in  if and only if  as
. Because , at least  vertices of  are in . Since , we conclude that
exactly  vertices of  are in  and . Moreover, .

Let  for some  and assume that  is adjacent to . If , then  has
at most  in-neighbors in , a contradiction. Hence, . By the same arguments we
conclude that . Then we have exactly  elements of  in
. Consider a pair of vertices  for . If
, then  has at most  in-neighbors in , a contradiction. Therefore, for each
, exactly one vertex from the pair   is in . For , we set the
variable  if the vertex , and  otherwise.

It remains to prove that we have a satisfying truth assignment. Consider a clause  for . The vertex
 has  in-neighbors in  that are vertices of . Hence, it has at least one in-neighbor in . It can be
either a vertex  or  that correspond to a literal in . It is sufficient to observe that if , then the literal , and if  , then the literal  by our
assignment.
\end{proof}


We conclude this section by the simple observation that \DAKC is in \classXP\ when parameterized by the number of anchors .
For a directed graph  with  vertices, we can consider all the at most  possibilities to
choose the anchors, and then recursively delete non-anchor vertices that have the in-degree at most . Trivially, if we
obtain a directed graph with at least  vertices for some selection of the anchors, we have a solution and otherwise we can
answer NO.

\section{\DAKC parameterized by the size of the core}
\label{sec:saved}
\vskip-2mm
In this section we consider the \DAKC problem for fixed  when  is a parameter and obtain the following dichotomy: If
 then the \DAKC problem is FPT parameterized by , otherwise for  it is \classW1-hard parameterized by .

\begin{theorem}\label{thm:fpt-d-1}
For , the \DAKC problem is solvable in time  on digraphs with  vertices.
\end{theorem}

\begin{proof}
The proof is constructive, and we describe an \classFPT~ algorithm for the problem. Without loss of generality, we assume that
.

We apply the following preprocessing rule reducing the  instance to  an acyclic graph. Let   be strongly
connected components of . By making use of Tarjan's algorithm~\cite{Tarjan72}, the sets   can be found  in
linear time.  Let  be the set of vertices reachable from strongly connected
components. Then every  satisfies . If , then we select in  any
arbitrary  vertices . In this case we  output the set of anchors  and
graph . Otherwise, if ,  we set  and  and consider a new instance of
\DAKC  with the graph  and the parameter .

To see that the rule is safe, it is sufficient to observe that a set of anchors  and a subgraph  of size at least 
is a solution of the obtained instance if and only if  is a solution for the original problem. Let us
remark that the preprocessing rule can be easily  performed in  time .

From now we can assume that  has no strongly connected components, i.e.,  is a directed acyclic graph. Denote by
 the set of sources of . If , then set . In this case, we output the pair .
The pair  is a solution because every vertex  satisfies . It remains to consider
the case when . For , let . Then . Without loss
of generality, we can assume that every anchored vertex is from . Indeed, if  is an anchor, then each vertex of 
can be included in a solution. Hence for every anchor , we can delete anchor from  and anchor
, if it is not yet anchored.
Since we  can choose anchors only from , we are able to reduce the problem to {\sc Partial Set Cover}.

\begin{center}
\noindent\framebox{\begin{minipage}{4.50in}{\psc }\\
\emph{Input }: A collection  of subsets of a finite -element set  and positive integers . \\
\emph{Parameter}: .\\
\emph{Question}: Are there at most  subsets , , covering at least
 elements of , i.e., ?
\end{minipage}}
\end{center}

Bl\"{a}ser~\cite{Blaser03} showed that \psc if \classFPT parameterized by  and can be solved in time .
For \DAKC, 
we consider the collection of subsets  of . If we can select at most 
subsets  such that  , we return the solution with anchors
 and . Otherwise, we return a NO-answer.

Because our preprocessing can be done in time  and  {\sc Partial Set Cover} is solvable in time , we conclude that the total running time is .
\end{proof}

Now we complement Theorem~\ref{thm:fpt-d-1} by showing that for ,  \DAKC becomes hard
parameterized by the core size.

\begin{theorem}\label{thm:w-saved}
For any fixed , the \DAKC problem is \classW1-hard parameterized by , even for DAGs.
\end{theorem}

\begin{proof}
We reduce from the {\sc -Clique} problem which is known to be \classW1-hard~\cite{downey-fellows-book}:

\begin{center}
\noindent\framebox{\begin{minipage}{4.50in}{-Clique}\\
\emph{Input}: A undirected graph   and a positive integer . \\
\emph{Parameter}: \\
\emph{Question}: Is there a clique of size  in ?
\end{minipage}}
\end{center}


From a given graph  we construct a directed graph  as follows.
\begin{itemize}
\item Construct a copy of .
\item For each edge , construct a new vertex  and join   with   the copy of 
    by arcs  and .
\item Construct  vertices , and for each , join  with  by
    arcs.
\end{itemize}
It is straightforward to see that  is a directed acyclic graph. We say that the vertex  for  is a
\emph{subdivision} vertex, and we say that   is a \emph{branch} vertex. Let  and
. Let . We claim that  has a clique of size  if and only if there is a
set of at most  vertices  such that there exists an an induced subgraph  of  with at least 
vertices,  and for any  we have  .

Suppose that  is a clique in  of size . We let  and define . Notice that
 and each vertex of  has two in-neighbors in  and  in-neighbors in . We conclude
that  has  vertices and  for any  satisfies .

Assume now that there is a set of at most  vertices  such that there exists an induced subgraph  of
 with at least  vertices,  and for any  we have  . Since
subdivision vertices of  are sinks, we can assume that  contains only \emph{branch} vertices and vertices from , as
otherwise we can replace an anchor  that is a subdivision vertex of  by an arbitrary branch vertex or a vertex of
. Because branch vertices of  and the vertices of  are sources, any such vertex  is in  if and only if . Hence,  has at most  sources of  and at least  subdivision vertices. If there is a vertex
 such that , then each subdivision vertex  has at most  in-neighbors and  cannot contain
subdivision vertices. Therefore  and  has at most  branch vertices. It remains to observe that a
subdivision vertex  has  in-neighbors in  if and only if . Then the claim follows.
\end{proof}

\section{\DAKC on graphs of bounded degree}
\label{sec:bound-deg} 
In this section we show that \DAKC problem is \classFPT\ parameterized by  if
.

In our algorithms we need to check the existence of solutions for \DAKC that have bounded size. It can be observed that if we
are interested in solutions  such that , then for every positive , we can express this problem
in the first order logic. It was proved by Seese~\cite{Seese96} that any graph problem expressible in the first-order logic
can be solved in linear time on (directed) graphs of bounded degree. Later this result was extended for much more rich graph
classes (see~\cite{DvorakKT10} ). These meta theorems are very general, but do not provide
good upper bounds for running time for particular problems. Hence, we give the following lemma.
Our algorithms use the random separation technique due to Cai et al.~\cite{CaiCC06} (which is a variant of the color coding
method introduced by Alon et al.~\cite{AlonYZ95}) .


\begin{lemma}\label{lem:bounded}
There is a randomized algorithm with running time  that for an instance of \DAKC with an -vertex
directed graph of maximum degree at most  and a positive integer , either returns a solution  with  or gives the answer
that there is no solution with . Furthermore, the algorithm can be derandomized, and the deterministic variant
runs in time  .
\end{lemma}

\begin{proof}
Consider an instance of \DAKC with an -vertex directed graph  of maximum degree at most . We assume that . For given , to decide if  contains a solution of size at  most , we do the following.

We color each vertex of  uniformly at random with probability  by one of  two colors, say red or
blue. Let   be the set of vertices  colored red. Observe that if there is a solution  with , then with
probability at least  all vertices of  are colored red and with probability at least 
all 
in- and out-neighbors of the vertices of  that are outside of  are colored blue. Using this observation, we assume that
 is the union of some weakly connected components of  the graph  induced by red vertices.

In time  we find all weakly connected components of . If there is a component  with at least 
vertices of in-degree at most  (in ), then we discard this component as it cannot be a part of any solution. Denote by
 the remaining components. For , let ,  and
.

Thus everything boils down to the problem of finding a set  such that  and
.  But this is  the well known {\sc Knapsack} problem, which  is solvable in time  by dynamic
programming. If we obtain a solution , then we output , where  and .
Otherwise, we return a NO-answer. Notice that this algorithm can also find a solution   with .


It remains to observe that for any positive number , there is a constant  such that after running our
randomized algorithm  times, we either find a solution  or can claim that with
probability  that it does not exist.


This algorithm can be derandomized by the  technique proposed by Alon et al.~\cite{AlonYZ95}: replace the random colorings by
a family of at most  hash functions which are known to be constructible  in time .
\end{proof}


Our next aim is to prove that for  the \DAKC problem is \classFPT\ when parameterized by the number of anchors
.

\begin{lemma}\label{lem:bound-deg-anchors}
Let  be a positive integer. If , then the \DAKC problem can be solved in time  for -vertex directed graphs of maximum degree at most .
\end{lemma}

\begin{proof}
Suppose  is a solution for the \DAKC problem. Let us observe that because , for every vertex ,  we have . Recall that for any directed graph, the sum of in-degrees equals the sum of
out-degrees. Then

Since for every vertex ,  , we have that

On the other hand,  , and we arrive at

Hence, . Using this observation, we can solve the \DAKC problem as follows. If
, then we return a NO-answer. If , we apply Lemma~\ref{lem:bounded} for , and
solve that problem in time  .
\end{proof}


Now we show that if  then the \DAKC problem is FPT parameterized by . Due the space restrictions we only sketch the proof of the following lemma.

\begin{lemma}\label{lem:bound-deg-saved}
Let  be a positive integer. If , then the \DAKC problem can be solved in time  for -vertex directed graphs of maximum degree at most .
\end{lemma}

\begin{proof}
We describe an \classFPT~ algorithm. Consider an instance of the \DAKC problem. Without loss of generality we assume that .

We apply the following preprocessing rule. Suppose that  has a (weakly) connected component  such that for any , . If , then we choose a set  of  vertices arbitrary in
. Then we return a YES-answer, as the anchors  and  is a solution. Otherwise, if , we let  and . Now we consider a new instance of the problem with the graph  and the
parameter . To see that the rule is safe, it is sufficient to observe that a set of anchors  and a subgraph  of
size at least  is a solution of the obtained instance if and only if  and  is a solution for the
original problem. From now we assume that  has no such components.

\medskip

We need the following claim.

\medskip
\noindent
{\bf Claim A.} {\it
If an instance of the \DAKC problem has a core with at least  vertices, then it has a solution
 with the following property: there is a vertex   reachable in  from any vertex of .
Moreover, for each vertex  of , there is a path from  to  with all vertices except  in  .
}



\begin{proof}[Proof of Claim~A]
Let   be a solution with the set of anchors  and such that .

We show that , i.e., all vertices of  are reachable from the anchors. To obtain a contradiction,
suppose that there is a vertex  such that . Let , i.e.,  is the set of
vertices from which we can reach . Clearly, . Therefore,  for .
Notice that for a vertex ,  by the definition. Hence,  for . Because the sum of in-degrees equals the sum of out-degrees, for  every vertex , we have that
. Then  is a component of  such that for every ,
, but such components are excluded by the preprocessing; a contradiction.

Observe now that if , then  and thus . Hence, by adding at most 
(maybe multiple) arcs from  to , joining the vertices  of degrees  with
vertices of degrees  , we  can  transform  into a disjoint union of  directed Eulerian graphs.
Since , each of these directed Eulerian graphs contains at least one vertex of . Thus  the set of arcs
of  can be covered by at most  arc-disjoint directed walks,   each walk starting from a vertex of  and never
coming back to . Because  for , we have that . Then there is a walk  with at least  arcs.  Let  be the first vertex of  and let  be the last
vertex of the walk. The walk  visits  only once,   and all other vertices of  are visited at most  times. We
conclude  that  has at least  vertices.

Let  and let . Consider . Since
, .  For any , the in-neighbors of  in  are in  by the
construction and, therefore, . It remains to observe that to select at most  anchors, we take .
\end{proof}

Using  Claim~A, we proceed with our algorithm. We try to find a solution such that  has at most  vertices by applying  Lemma~\ref{lem:bounded}. It takes time . If we obtain a
solution, then we return it and stop. Otherwise, we conclude that every core  contains at least 
vertices. By Claim~A, we can search for a solution    with a  non-anchor vertex  which is reachable from
all other vertices of  by directed paths  avoiding . Notice that since  is a non-anchor vertex, we have that
. We try  at most  possibilities for all possible choices of  , and solve our problem for each choice.
Clearly, if we get a YES-answer for one of the choices, we return it and stop. Otherwise, if we fail,  we return a NO-answer.

From now we assume that we already selected . We denote by  the graph obtained from  by adding  an artificial source
vertex  joined by arcs with all the vertices  with . Observe that .

\medskip

Suppose that  is a solution with the set of anchors  such that  is reachable in  from any
vertex of  by a path with all inner vertices in . Denote by  the set , i.e.,  contains vertices that have in-neighbors outside . We
need a chain of claims about the structure of  in .


\medskip
\noindent
{\bf Claim~B.} {\it
.
}


\begin{proof}[Proof of Claim~B]
Let ,  and . Clearly,  Observe that  for ,   for  and
 for . Hence, . If  for , then . It
follows that  and . We have . Consider a vertex .
It has at least one in-neighbor outside  in  and . Then  and  . We conclude that
 and .
\end{proof}


\medskip
\noindent
{\bf Claim~C.} {\it
There is  an  separator  in  of size at most  such that
 .
}

\begin{proof}[Proof of Claim~C]
Let , i.e., the
set containing all anchors that are in , and for each non-anchor vertex of  containing all its
in-neighbors outside of . Consider a directed -path  in . Let  be the first vertex in  that is in
 and let  be its predecessor in . If , then . If , then  as  has no
non-anchor vertices with in-degree at most  in . Then . We conclude that each -path contains a vertex
of , i.e., this set is an  separator.

Observe that  by the definition of  and the fact that  can be reached from any
vertex of  in this graph by a path with all inner vertices in .

It remains to show that .  By Claim~B, . A
vertex  has at least one out-neighbor in  because  is reachable from . Then  has
at most  in-neighbors outside . Hence .
\end{proof}

Now we can prove the following claim about important  separators in .

\medskip
\noindent
{\bf Claim~D.} {\it
There is an important  separator  of size at most  in  such that .
}



\begin{proof}[Proof of Claim~D]
By Claim~C, there is an  separator  in  of size at most  such that . Notice that  not necessary a minimal separator, but there is a minimal  separator
. Clearly, .

We show that  .  
Because , we have that . Also if an anchor 
is in  , then . Let . If , then .
If  , then by Claim~C,  has an out-neighbor  and in this case we have .

It remains to observe that there is an important  separator  such that  and
. Therefore, .
\end{proof}

The next step of our algorithm is to check all important  separators in  of size at most  . By
Lemma~\ref{lem:imp-sep}, there are at most  important  separators and they can be listed in time
. For each important  separator , we consider the set of vertices  and decide whether there is a solution such that . If we have a solution for some , then we return
a YES-answer and stop. Otherwise, if we fail  to find such a solution for all important separators, we use
Claim~D  to deduce that there is no solution.

\medskip
From now on, we assume that an important  separator  is given and that  .
In what follows, we describe a procedure of finding  a solution with  .


\medskip
Denote by  the set . We need the following observation.



\medskip
\noindent
{\bf Claim~E.} {\it
Set  contains at most  vertices.
}


\begin{proof}[Proof of Claim~E]
Let . Let ,  and
. Clearly,

Observe that  for ,   for  and 
for . Hence, .

Recall that  is obtained from  by joining  with all vertices of in-degree at most . Since  is an 
separator, if for , , then . Hence,  and .
If for for , , then . We conclude that .
\end{proof}

Recall   that  set  contains vertices of  that have in-neighbors outside of . If , then it has at least  in-neighbors in  and at least one in-neighbor outside . Notice that
 because . Hence, . Because ,
. By Claim~C, , and by
Claim~E, .
We consider  all at most
 possibilities to select  . For each choice of
, we guess the arcs that join the vertices that are outside  with the vertices of
 and delete them. Denote the graph obtained from  by . Recall that from each vertex  of
, there is a directed path to  that avoids . Hence,  has at least one out-neighbor in 
and at most  in-neighbors in . Also  has at least  in-neighbors in , and we delete at most 
arcs. Therefore, for  we choose at most  arcs out of at most  arcs. We can upper bound the number of
possibilities for  by , and the total number of possibilities for  is
.

Observe that  is a solution for the new instance of \DAKC, where  is replaced by  for a correct guess of the
deleted arcs. Also each solution for the new instance provides a solution for the graph , because if we put deleted arcs
back, then we can only increase in-degrees. Hence, we can check for each possible choice of the set of deleted arcs, whether
the new instance has a solution. If for some choice we obtain a solution, then we return a YES-answer. Otherwise, if we fail
for all choices, then we return a NO-answer. Further we assume that  is given.


Denote by  the graph obtained from  by the addition of a vertex  joined by arcs with all the vertices .
Now . By the choice of ,
 and, therefore, . Also  is an  separator in
 by Claim~C.

Now we can prove the following.

\medskip
\noindent
{\bf Claim~F.} {\it
There is an important  separator  of size at most  in  such that  is a solution for the instance of the \DAKC problem for the graph .
}


\begin{proof}[Proof of Claim~F]
Let . It was already observed that  is an  separator in  of size
at most . Then there is a minimal  separator . Clearly, .

As before in the proof of Claim~D, we show that  . Because for any vertex
 of , there is a directed  path with all inner vertices in , . Because  we have . Also if   is in  , then . Let . Trivially, if , then . If  , then  has an out-neighbor
 and . Then there is an important  separator  such that
 and . Therefore, , and .

It remains to observe that  is adjacent to all vertices of  with in-degrees at most  and  is an 
separator. It immediately follows that for any  vertex , . Then
 is a solution.
\end{proof}

The final step of our algorithm is to enumerate all important  separators  of size at most  in , which
number by Lemma~\ref{lem:imp-sep} is at most , and for each , check whether  is a solution. Recall that all these separators can be listed in time . We return a YES-answer
if we obtain a solution for some important separator, and a NO-answer otherwise.

To complete the proof, let us observe that each step of the algorithm runs either in polynomial or \classFPT~ time.
Particularly, the preprocessing is done in time . Then we check the existence of a solution of a bounded size in
time . Further we consider at most  possibilities to choose . For each , we consider
at most  important  separators . Recall, that they can be listed in time
 for some constant . Then for each , we have  at most   possibilities to construct , and it can be done in time .
Finally, there are at most  important  separators  and they can be listed in time  for
some . We conclude that the total running time is  for some constant .
\end{proof}

Combining Lemmas~\ref{lem:bound-deg-anchors} and \ref{lem:bound-deg-saved}, we obtain the following theorem.

\begin{theorem}\label{thm:bound-deg}
Let  be a positive integer. If , then the \DAKC problem can be solved in time  for -vertex directed graphs of maximum degree at most .
\end{theorem}

Theorems~\ref{thm:fpt-d-1} and \ref{thm:bound-deg} give the next corollary.

\begin{corollary}\label{cor:fpt-delta}
The \DAKC problem can be solved in time   for -vertex directed graphs of maximum degree at most
.
\end{corollary}


\section{Conclusions}\label{sec:concl}
We proved that \DAKC is \classNP-complete even for planar DAGs of maximum degree at most . It was also shown  that \DAKC
is \classFPT~ when parameterized by  for directed graphs of maximum degree at most
 whenever . It is natural to ask whether the problem is \classFPT~ for other values . This question
is interesting even for the special case  and .

For the special case of  directed acyclic graphs (DAGs) we understand  the complexity of the problem much better.
Theorem~\ref{thm:w-saved} showed that \DAKC on DAGs is \classW{1}-hard parameterized by  for every fixed , when
the degree of the graph is not bounded. We now show the following theorem  that gives
\classW{2}-hardness of \DAKC when parameterized  by the number of anchors  (recall that we can always assume that ).

\begin{theorem}\label{thm:w-hardness-dags-bounded-degree}
For any  and any positive ,  \DAKC  is \classW{2}-hard (even on DAGs) when parameterized by
the number of anchors  on graphs of maximum degree at most .
\end{theorem}


\begin{proof}
First, we prove the claim for  and . We reduce from the {\sc -Set Cover} problem which is known to be
\classW2-hard~\cite{downey-fellows-book}:

\begin{center}
\noindent\framebox{\begin{minipage}{4.50in}{\textsc{-Set Cover} }\\
\emph{Input }: A collection  of subsets of a finite -element set  and a positive integer . \\
\emph{Parameter}: \\
\emph{Question}: Are there at most  subsets  such that these sets cover , i.e., ?
\end{minipage}}
\end{center}

\begin{figure}[ht]
\centering\scalebox{0.7}{\begin{picture}(0,0)\includegraphics{fig2.pdf}\end{picture}\setlength{\unitlength}{3947sp}\begingroup\makeatletter\ifx\SetFigFont\undefined \gdef\SetFigFont#1#2#3#4#5{\reset@font\fontsize{#1}{#2pt}\fontfamily{#3}\fontseries{#4}\fontshape{#5}\selectfont}\fi\endgroup \begin{picture}(3248,1986)(518,-2164)
\put(3751,-1636){\makebox(0,0)[lb]{\smash{{\SetFigFont{12}{14.4}{\rmdefault}{\mddefault}{\updefault}{\color[rgb]{0,0,0}}}}}}
\put(751,-2086){\makebox(0,0)[lb]{\smash{{\SetFigFont{12}{14.4}{\rmdefault}{\mddefault}{\updefault}{\color[rgb]{0,0,0}}}}}}
\put(2251,-2086){\makebox(0,0)[lb]{\smash{{\SetFigFont{12}{14.4}{\rmdefault}{\mddefault}{\updefault}{\color[rgb]{0,0,0}}}}}}
\put(3751,-2086){\makebox(0,0)[lb]{\smash{{\SetFigFont{12}{14.4}{\rmdefault}{\mddefault}{\updefault}{\color[rgb]{0,0,0}}}}}}
\put(1501,-361){\makebox(0,0)[lb]{\smash{{\SetFigFont{12}{14.4}{\rmdefault}{\mddefault}{\updefault}{\color[rgb]{0,0,0}}}}}}
\put(3001,-361){\makebox(0,0)[lb]{\smash{{\SetFigFont{12}{14.4}{\rmdefault}{\mddefault}{\updefault}{\color[rgb]{0,0,0}}}}}}
\put(751,-1636){\makebox(0,0)[lb]{\smash{{\SetFigFont{12}{14.4}{\rmdefault}{\mddefault}{\updefault}{\color[rgb]{0,0,0}}}}}}
\put(2251,-1636){\makebox(0,0)[lb]{\smash{{\SetFigFont{12}{14.4}{\rmdefault}{\mddefault}{\updefault}{\color[rgb]{0,0,0}}}}}}
\end{picture} }
\caption{Construction of  for  and .
\label{fig:W2h-1}}
\end{figure}


Let . We construct the directed graph  as follows (see Fig.~\ref{fig:W2h-1}).
\begin{itemize}
\item For , assume that  and
\begin{itemize}
\item construct a vertex  and  vertices ;
\item construct arcs .
\end{itemize}
\item For , assume that  is included in the sets  and
\begin{itemize}
\item construct a vertex  and  vertices ;
\item construct arcs ;
\item join  with  by a directed path  of length .
\end{itemize}
\item For  and , if , then construct an arc .
\end{itemize}
It is straightforward to see that  is a directed acyclic graph of maximum degree at most 3. We set . We claim
that  can be covered by at most  sets if and only if there is a set of at most  vertices  such that there exists
an induced subgraph  of  with at least  vertices,  and for any , .

Notice that  are the sources of ,  are the sinks, and .
Observe also that  can be reached from  if and only if .

Suppose that  can be covered by at most  sets say . Let  and
. It is straightforward to see that for any vertex , . Because  is covered,
all vertices  are in  and, therefore, . It remains to observe
that  and we conclude that  is a solution of our instance of \DAKC.

Assume now that  is a solution of the \DAKC problem. Without loss of generality we can assume that that each 
is a source of . Otherwise,  for some source , and we can replace  by  in  (or delete it
if  already). Let . We show that 
cover . To obtain a contradiction, assume that there is an element  such that . Then the vertex  is not reachable from . Hence, the vertices of  are not reachable from . It
follows that . We have that . Because  for
 and each  is included in at most  sets for , . Therefore,  because  has  vertices; a
contradiction.

Now we prove \classW{2}-hardness for  and . We reduce from an instance of the \DAKC problem with 
and . Consider an instance of this problem with a directed acyclic graph  and positive integers . Assume
that  and . We construct the graph  as follows (see Fig.~\ref{fig:W2h-2}).

\begin{figure}[ht]
\centering\scalebox{0.7}{\begin{picture}(0,0)\includegraphics{fig3.pdf}\end{picture}\setlength{\unitlength}{3947sp}\begingroup\makeatletter\ifx\SetFigFont\undefined \gdef\SetFigFont#1#2#3#4#5{\reset@font\fontsize{#1}{#2pt}\fontfamily{#3}\fontseries{#4}\fontshape{#5}\selectfont}\fi\endgroup \begin{picture}(4899,2715)(439,-2464)
\put(2476,-136){\makebox(0,0)[lb]{\smash{{\SetFigFont{12}{14.4}{\rmdefault}{\mddefault}{\updefault}{\color[rgb]{0,0,0}}}}}}
\put(5026,-136){\makebox(0,0)[lb]{\smash{{\SetFigFont{12}{14.4}{\rmdefault}{\mddefault}{\updefault}{\color[rgb]{0,0,0}}}}}}
\put(1051,-136){\makebox(0,0)[lb]{\smash{{\SetFigFont{12}{14.4}{\rmdefault}{\mddefault}{\updefault}{\color[rgb]{0,0,0}}}}}}
\put(526, 14){\makebox(0,0)[lb]{\smash{{\SetFigFont{12}{14.4}{\rmdefault}{\mddefault}{\updefault}{\color[rgb]{0,0,0}}}}}}
\put(1201,-2386){\makebox(0,0)[lb]{\smash{{\SetFigFont{12}{14.4}{\rmdefault}{\mddefault}{\updefault}{\color[rgb]{0,0,0}}}}}}
\put(2626,-2386){\makebox(0,0)[lb]{\smash{{\SetFigFont{12}{14.4}{\rmdefault}{\mddefault}{\updefault}{\color[rgb]{0,0,0}}}}}}
\put(5176,-2386){\makebox(0,0)[lb]{\smash{{\SetFigFont{12}{14.4}{\rmdefault}{\mddefault}{\updefault}{\color[rgb]{0,0,0}}}}}}
\end{picture} }
\caption{Construction of  for .
\label{fig:W2h-2}}
\end{figure}

\begin{itemize}
\item Construct a copy of  and denote its vertices by .
\item For each , construct a set of  vertices  and join  vertices of this set with 
    by arcs.
\item For each , join each vertex of  with all vertices of  by arcs.
\end{itemize}
Clearly,  is a directed acyclic graph. We let  and . Let also . Notice that for
each ,  as maximum degree of  is 3. For , . Hence maximum degree of  is at most . We now claim that there is a set of at most  vertices
 such that there exists an an induced subgraph  of  with at least  vertices,  and
for any ,   if and only if there is a set of at most  vertices  such that there exists an an induced subgraph  of  with at least  vertices,  and for any
,  .

Suppose that our original instance of \DAKC has a solution . We let  and . Then each
vertex  has  in-neighbors in . It remains to observe that each vertex  of  from
 has at least one in-neighbor in  and  in-neighbors in . Therefore, .

Assume now that  is a solution for the constructed instance of \DAKC with  and . If
, then we claim that . To prove it, suppose that  and consider the smallest index  such that there is . Clearly, .
The vertex  has in-neighbors only in . By the choice of ,  has at most  vertices of , because
they can be only anchors and . Then , a contradiction.

Then if ,  and  as  and . This
contradicts our assumption about size of . Hence, at least  anchors are in  and . Let
 and . If , then  and  has at most  in-neighbors
from  in . Then  has at least one in-neighbor in  and .
\end{proof}


The case of  the complexity of parameterization by  on DAGs is left open. However we can show that
\DAKC is \classFPT on DAGs of maximum degree , when parameterized by .

\begin{theorem}\label{thm:dags}
For any positive integers  and , \DAKC can be solved in time  for
-vertex DAGs of maximum degree at most .
\end{theorem}

\begin{proof}
Consider an instance of \DAKC with an -vertex directed acyclic graph . Without loss of generality we can assume that
.

We apply Lemma~\ref{lem:bounded} for . In time  we either obtain a solution or conclude
that for any solution ,  has size at least . If we obtain a solution, we return it. Suppose that we got a
NO-answer. If , then we return a NO-answer. Otherwise,  we select a sink  using the fact that any directed
acyclic graph has at least one such vertex. Observe that we can assume that  is not an anchor in any solution. Also if 
is included in a solution  of size at least , then  is a solution of size at least , because  is not joined
by arcs with other vertices of . Then we solve the instance  of \DAKC recursively.

As each step is done in time  and the number of steps is at most , the claim follows.
\end{proof}

Let us remark that this result can be easily extended for any class of directed acyclic graphs  such that the
corresponding class of underlaying graphs  has (locally) bounded expansion by making use of  the
results by Dvorak et al.~\cite{DvorakKT10}.
Finally, what happens when the input graph is planar? We know that the problem is \classNP-complete on planar graphs for fixed
 and maximum degree . Is the problem \classFPT\ on planar directed graphs when parameterized by the size of the
core ?



\begin{thebibliography}{10}

\bibitem{AlonYZ95}
{\sc N.~Alon, R.~Yuster, and U.~Zwick}, {\em Color-coding}, J. ACM, 42 (1995),
  pp.~844--856.

\bibitem{BhawalkarKLRS12}
{\sc K.~Bhawalkar, J.~M. Kleinberg, K.~Lewi, T.~Roughgarden, and A.~Sharma},
  {\em Preventing unraveling in social networks: The anchored k-core problem},
  in ICALP '12, vol.~7392 of Lecture Notes in Computer Science, 2012,
  pp.~440--451.

\bibitem{Blaser03}
{\sc M.~Bl{\"a}ser}, {\em Computing small partial coverings}, IPL., 85 (2003),
  pp.~327--331.

\bibitem{CaiCC06}
{\sc L.~Cai, S.~M. Chan, and S.~O. Chan}, {\em Random separation: A new method
  for solving fixed-cardinality optimization problems}, in IWPEC '06, vol.~4169
  of Lecture Notes in Computer Science, 2006, pp.~239--250.

\bibitem{ChitnisFG13}
{\sc R.~H. Chitnis, F.~V. Fomin, and P.~A. Golovach}, {\em Preventing
  unraveling in social networks gets harder}, in AAAI '13, AAAI Press, 2013.

\bibitem{ChitnisHM12}
{\sc R.~H. Chitnis, M.~Hajiaghayi, and D.~Marx}, {\em Fixed-parameter
  tractability of directed multiway cut parameterized by the size of the
  cutset}, in SODA '12, SIAM, 2012, pp.~1713--1725.

\bibitem{chwe1999structure}
{\sc M.~Chwe}, {\em {Structure and Strategy in Collective Action 1}}, American
  Journal of Sociology, 105 (1999), pp.~128--156.

\bibitem{chwe2000communication}
\leavevmode\vrule height 2pt depth -1.6pt width 23pt, {\em {Communication and
  Coordination in Social Networks}}, The Review of Economic Studies, 67 (2000),
  pp.~1--16.

\bibitem{DahlhausJPSY94}
{\sc E.~Dahlhaus, D.~S. Johnson, C.~H. Papadimitriou, P.~D. Seymour, and
  M.~Yannakakis}, {\em The complexity of multiterminal cuts}, SIAM J. Comput.,
  23 (1994), pp.~864--894.

\bibitem{downey-fellows-book}
{\sc R.~G. Downey and M.~R. Fellows}, {\em Parameterized Complexity},
  Springer-Verlag, 1999.

\bibitem{DvorakKT10}
{\sc Z.~Dvorak, D.~Kr{\'a}l, and R.~Thomas}, {\em Deciding first-order
  properties for sparse graphs}, in FOCS, IEEE Computer Society, 2010,
  pp.~133--142.

\bibitem{flum-grohe-book}
{\sc J.~Flum and M.~Grohe}, {\em Parameterized Complexity Theory}, Texts in
  Theoretical Computer Science. An EATCS Series, Springer-Verlag, Berlin, 2006.

\bibitem{Marx06}
{\sc D.~Marx}, {\em Parameterized graph separation problems}, Theor. Comput.
  Sci., 351 (2006), pp.~394--406.

\bibitem{niedermeier-book}
{\sc R.~Niedermeier}, {\em Invitation to fixed-parameter algorithms}, vol.~31
  of Oxford Lecture Series in Mathematics and its Applications, Oxford
  University Press, Oxford, 2006.

\bibitem{schelling2006micromotives}
{\sc T.~Schelling}, {\em {Micromotives and Macrobehavior}}, WW Norton, 2006.

\bibitem{Seese96}
{\sc D.~Seese}, {\em Linear time computable problems and first-order
  descriptions}, Mathematical Structures in Computer Science, 6 (1996),
  pp.~505--526.

\bibitem{seidman-k-core}
{\sc S.~Seidman}, {\em {Network Structure and Minimum Degree}}, Social
  networks, 5 (1983), pp.~269--287.

\bibitem{Tarjan72}
{\sc R.~E. Tarjan}, {\em Depth-first search and linear graph algorithms}, SIAM
  J. Comput., 1 (1972), pp.~146--160.

\end{thebibliography}

\end{document}
