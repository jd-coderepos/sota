\documentclass{llncs}

\usepackage{microtype}
\usepackage{cite}







\usepackage{amsmath}
\usepackage{amssymb}

\usepackage{graphicx}

\usepackage{subfig}
\usepackage{wrapfig}

\spnewtheorem{observation}[theorem]{Observation}{\bfseries}{\itshape}
\spnewtheorem*{uremark}{Remark}{\bfseries}{\rm}



\usepackage[usenames,dvipsnames]{color}


\newcommand{\Vbest}{V_{\mathrm{best}}}
\newcommand{\Vapp}{V_{\mathrm{apx}}}
\newcommand{\wapp}{w_{\mathrm{apx}}}
\newcommand{\wopt}{w_{\mathrm{opt}}}
\let\eps\varepsilon
\intextsep 5pt

\newcommand{\A}{\mathcal{A}}

\renewcommand{\theenumi}{\alph{enumi}}
\renewcommand{\labelenumi}{\theenumi)}

\DeclareMathOperator{\conv}{CH}
\DeclareMathOperator{\dist}{dist}
\DeclareMathOperator{\width}{width}




\title{How to Cover a Point Set with a V-Shape of Minimum Width\thanks{Work on this paper has been supported by grant No.~2006/194
    from the U.S.-Israel Binational Science Foundation and by NSF
    Grant CCF-08-30691.  Work by Boris Aronov has also been supported
    by NSA MSP Grant H98230-10-1-0210.  An extended abstract of this
    paper appeared in the \emph{Proceedings of the Algorithms and Data
      Stuctures Symposium (WADS'11)} \cite{AD11}.}}

\author{Boris Aronov \and Muriel Dulieu}

\institute{Department of Computer Science and Engineering, Polytechnic 
 Institute of NYU, Brooklyn, NY~11201-3840, USA;
 \email{aronov@poly.edu}, \email{mdulieu@gmail.com}}




\begin{document}
\maketitle
\pagestyle{plain}
\thispagestyle{plain}
\begin{abstract}
  A balanced V-shape is a polygonal region in the plane contained in
  the union of two crossing equal-width strips.  
  It is delimited by two pairs of parallel rays that emanate from two points
  , , are contained in the strip boundaries, and are mirror-symmetric with respect to the line .
   The width of a balanced V-shape is the width of the strips.
   
  We first present an  time algorithm to compute, given
  a set of ~points , a minimum-width balanced V-shape covering
  .
We then describe a PTAS for computing a -approximation of
  this V-shape in time .
  A much simpler constant-factor approximation algorithm is also described.
\end{abstract}


\section{Introduction}
\label{sec:introduction}

\paragraph*{Motivation.}
The problem we consider in this paper was motivated by the following
curve reconstruction question: One is given a set of points sampled
from a curve in the plane.  The sample is noisy in the sense that
the points lie near the curve, but not exactly on it.  One would like to
reconstruct the original curve from this data.  Clearly one has to
make some assumptions about the point set and the curve: If the curve is
``too wiggly'' or the noise is too large, little can be done.  One
approach is to assume that the curve is smooth and the sample points
lie not too far from it; see \cite{curve-noisy,curve-dey} and
references therein.\footnote{See \cite{alt-guibas-survey} for a detailed survey of different
  notions of measuring similarity between geometric objects;
  is there a sensible (and relevant for our purposes) notion of
  closeness between a discrete unordered point set and a curve?}
Roughly speaking, one can then approximate a stretch of a curve by an
elongated rectangle (or strip) whose width is determined both by the
curvature of the curve and the amount of noise.  Refining this
approximation allows one to reconstruct the location of the curve and
its normal vector.

Complications arise when a curve makes a sharp turn, as it does not
have a well-defined direction near the point of turn.  It has been
suggested \cite{curve-noisy,proj-clustering} that one approach to handle this situation
is to replace fitting the set of points corresponding to a smooth arc of a
curve with a strip by fitting with a wedge-like shape that we call a ``balanced
V-shape;'' perhaps one might incorporate it in an algorithm such as
that of
\cite{Funke-Ramos}.  It is meant to model one thickened turn in a 
piecewise-linear curve; refer to the figure and precise
definitions below.

In this paper, we construct a slower exact algorithm for identifying a V-shape that best
fits a given set of points in the plane, then a
faster constant-factor approximation algorithm, and finally a
considerably more involved algorithm that produces a
-approximation, for any positive~.

The problem we solve is a new representative of a widely studied class
of problems, namely \emph{geometric optimization} or \emph{fitting}
questions; see
\cite{coreset-survey,random-opt-survey,alg-opt-survey,eff-alg-opt-survey}
and references therein.  Generally, the problem is to find a shape
from a given class that best fits a given set of points.  Classical
examples of such problems are linear regression in statistics, the
computation of the width of a point set (which constructs a
minimum-width strip covering the set), computing a minimum enclosing
ball, cylinder, or ellipsoid, a minimum-width spherical or
cylindrical shell, or a small number of strips of minimum width,
covering the point set; see~\cite{chan-apx-all,coreset-survey}.

Previous work most closely related to our problem is that of Glozman,
Kedem, and Shpitalnik \cite{GKS}.  They compute a double-ray center
for a planar point set .  A~\emph{double-ray center} is a pair of
rays emanating from a common apex, minimizing the Hausdorff distance
between  and the double ray.  While the shape they consider is not
exactly a V-shape, it is similar enough to be used for the same
purpose.  The exact algorithm they present runs in  time, however, in contrast to our near-quadratic-time
algorithm; here  is the inverse Ackermann function.

Another paper closely related to our problem is that of Agarwal,
Procopiuc, and Varadarajan \cite{2-line-center}.
It concerns the
2-line-center problem studied extensively in the past; see the
references in \cite{2-line-center}.  The goal is to cover a given set
of points by two strips of minimum possible width.  One
application is fitting \emph{two} lines to a point set.  There had
been several previously known near-quadratic-time exact algorithms for
the problem.  An -time 6-approximation algorithm, and an
-time
-approximation algorithm were presented in
\cite{2-line-center}.  A V-shape covering a point set is a special
case of covering a point set by two strips, so some of the
tools from \cite{2-line-center} apply to our problem as well.




\paragraph*{Problem statement and results.}
\begin{wrapfigure}[9]{r}{0.47\textwidth}
  \scalebox{0.4}{\begin{picture}(0,0)\includegraphics{V1.pdf}\end{picture}\setlength{\unitlength}{3947sp}\begingroup\makeatletter\ifx\SetFigFont\undefined \gdef\SetFigFont#1#2#3#4#5{\reset@font\fontsize{#1}{#2pt}\fontfamily{#3}\fontseries{#4}\fontshape{#5}\selectfont}\fi\endgroup \begin{picture}(6869,4268)(2754,-7397)
\put(5176,-3931){\makebox(0,0)[lb]{\smash{{\SetFigFont{20}{24.0}{\rmdefault}{\mddefault}{\updefault}{\color[rgb]{0,0,0}}}}}}
\put(4336,-5716){\makebox(0,0)[lb]{\smash{{\SetFigFont{20}{24.0}{\rmdefault}{\mddefault}{\updefault}{\color[rgb]{0,0,0}}}}}}
\put(7681,-5686){\makebox(0,0)[lb]{\smash{{\SetFigFont{20}{24.0}{\rmdefault}{\mddefault}{\updefault}{\color[rgb]{0,0,0}}}}}}
\put(6646,-3931){\makebox(0,0)[lb]{\smash{{\SetFigFont{20}{24.0}{\rmdefault}{\mddefault}{\updefault}{\color[rgb]{0,0,0}}}}}}
\put(8206,-3541){\makebox(0,0)[lb]{\smash{{\SetFigFont{20}{24.0}{\rmdefault}{\mddefault}{\updefault}{\color[rgb]{0,0,0}}}}}}
\put(3316,-3736){\makebox(0,0)[lb]{\smash{{\SetFigFont{20}{24.0}{\rmdefault}{\mddefault}{\updefault}{\color[rgb]{0,0,0}}}}}}
\put(6076,-4696){\makebox(0,0)[lb]{\smash{{\SetFigFont{20}{24.0}{\rmdefault}{\mddefault}{\updefault}{\color[rgb]{0,0,0}}}}}}
\put(6076,-7261){\makebox(0,0)[lb]{\smash{{\SetFigFont{20}{24.0}{\rmdefault}{\mddefault}{\updefault}{\color[rgb]{0,0,0}}}}}}
\end{picture} }  
\end{wrapfigure}
In this paper, we focus on the class of polygonal regions in the plane
that we call balanced V-shapes.   
A \emph{balanced V-shape} has 
two \emph{vertices}  and  and is delimited by two pairs of
parallel rays. 
One pair of 
parallel rays emanate from  and  on one side of
the line  and the other pair of rays emanate from  and  on
the other side of , symmetrically with respect to   
(see the above figure).
In particular, a balanced V-shape is completely contained in the union
of two crossing strips of equal width.  Its \emph{width} is the width
of the strips.

Consider a point set  of  points in the plane.  
We describe, in Section~\ref{sec:algorithm}, an  time 
algorithm that computes a balanced V-shape with minimum width covering . 

Our algorithm actually identifies a particular type of V-shapes that
we call ``canonical'' (see below for definitions) and enumerates all
minimum-width canonical V-shapes covering~; as some degenerate
-point sets have  such V-shapes (see Section~\ref{sec:lower-bound}), this approach will probably not yield
a subquadratic algorithm.  This leaves open the problem of how quickly
one can identify just \emph{one} minimum-width V-shape covering .

In Section~\ref{sec:13-approx}, we present an  algorithm
that constructs a V-shape covering  with width at most 13 times the
minimum possible width.  In Section~\ref{sec:one+eps}, we show how to
construct a -approximation in time , starting with the 13-approximation
obtained earlier.

\section{Reduction to canonical V-shapes}
\label{sec:canonical}

In the remainder of this paper, for simplicity of presentation and 
unless noted otherwise, we assume
that the points of  are \emph{in general position}: no three points are
collinear and no two pairs of points define parallel lines. 
All algorithms can be adapted to degenerate inputs without asymptotic slowdown.


We will find it convenient to consider a larger class of objects,
namely V-shapes. A (\emph{not necessarily balanced}) \emph{V-shape} (refer to the
figure below) is a polygonal
\begin{wrapfigure}[9]{r}{0.45\textwidth}
  \scalebox{0.4}{\begin{picture}(0,0)\includegraphics{V2.pdf}\end{picture}\setlength{\unitlength}{3947sp}\begingroup\makeatletter\ifx\SetFigFont\undefined \gdef\SetFigFont#1#2#3#4#5{\reset@font\fontsize{#1}{#2pt}\fontfamily{#3}\fontseries{#4}\fontshape{#5}\selectfont}\fi\endgroup \begin{picture}(6494,3878)(2829,-7292)
\put(4801,-4561){\makebox(0,0)[lb]{\smash{{\SetFigFont{20}{24.0}{\rmdefault}{\mddefault}{\updefault}{\color[rgb]{0,0,0}}}}}}
\put(3451,-4111){\makebox(0,0)[lb]{\smash{{\SetFigFont{20}{24.0}{\rmdefault}{\mddefault}{\updefault}{\color[rgb]{0,0,0}}}}}}
\put(6226,-4411){\makebox(0,0)[lb]{\smash{{\SetFigFont{20}{24.0}{\rmdefault}{\mddefault}{\updefault}{\color[rgb]{0,0,0}}}}}}
\put(4726,-5986){\makebox(0,0)[lb]{\smash{{\SetFigFont{20}{24.0}{\rmdefault}{\mddefault}{\updefault}{\color[rgb]{0,0,0}}}}}}
\put(7426,-5836){\makebox(0,0)[lb]{\smash{{\SetFigFont{20}{24.0}{\rmdefault}{\mddefault}{\updefault}{\color[rgb]{0,0,0}}}}}}
\put(7501,-4186){\makebox(0,0)[lb]{\smash{{\SetFigFont{20}{24.0}{\rmdefault}{\mddefault}{\updefault}{\color[rgb]{0,0,0}}}}}}
\put(6076,-7156){\makebox(0,0)[lb]{\smash{{\SetFigFont{20}{24.0}{\rmdefault}{\mddefault}{\updefault}{\color[rgb]{0,0,0}}}}}}
\put(5551,-5236){\makebox(0,0)[lb]{\smash{{\SetFigFont{20}{24.0}{\rmdefault}{\mddefault}{\updefault}{\color[rgb]{0,0,0}}}}}}
\end{picture} }  
\end{wrapfigure}
region similar to a balanced V-shape except that the widths of its two
arms need not be the same.  More formally, a \emph{V-shape}~ is a
polygonal region bounded by two pairs of parallel rays emanating from
its two \emph{vertices}  and .  One pair of parallel rays
(\emph{left rays}  and ) lies on the left side of the
directed line , while the other pair (\emph{right rays}  and
) lies on its right side.  The \emph{inner rays}  emanate
from , while \emph{outer rays}  emanate from .   is the \emph{inner boundary} of , while  is its
\emph{outer boundary}.  The \emph{left arm} of , , is its
portion on the left of ; i.e., it is the region bounded by rays
 and  and segment .  The \emph{width} of the left arm,
, is the distance between  and .  The right
arm and its width are defined analogously.  The \emph{width} of ,
, is the larger of the widths of its two arms.   is
contained in the union of two strips  and :  is
delimited by the lines containing  and , respectively; we
refer to  and  as the \emph{left} and \emph{right}
\emph{strip} of , respectively.

A minimum-width balanced V-shape can be obtained from a minimum-width 
V-shape by widening the narrower arm until the widths of the arms are equal.

In the remainder of the paper, the -point set  is fixed.  To
avoid trivial cases, we assume that .  By the general
position assumption, all points of  cannot be collinear, nor can
 be covered by a V-shape of zero width. 
We need not consider V-shapes with all points in one strip as according
to Lemma~\ref{lem:no-empty-strip}, 
such a V-shape does not have minimum width.
\begin{lemma}
  \label{lem:no-empty-strip}
  In a positive-width minimum-width V-shape  covering , it is
  not possible that one of the strips already contains  in its
  entirety.
\end{lemma}
\begin{proof}
  Suppose  covers .  Let  be its width.  We argue
  that there is a V-shape covering  of width , for any
  positive , so  does not have minimum width.  Indeed, let
   be the median line of .  Cut  by  into two
  parallel strips of width .  They cover .  They do not form a
  V-shape, but they can be approximated arbitrarily closely by a
  V-shape near , by placing its vertex~ sufficiently far to the
  left of  along ,  at the rightmost point of , and the boundary rays near-parallel to . \hfill 
\end{proof}

Unless otherwise stated, the only particular V-shapes we will be interested in are the ones we call canonical. A V-shape is \emph{canonical}, if the bounding rays of each arm pass through exactly
three points of ; more precisely if ,
for  (recall that, by our general position assumption,
); in addition, we require that each arm
of a canonical V-shape covering~ is locally of minimum width, i.e.,
neither arm can be narrowed by an infinitesimal motion.

The reason why we consider only canonical V-shapes is that 
at least one minimum-width V-shape covering  is canonical (see
Lemma~\ref{CanonicalV-shape} below), so we can 
confine the search to canonical V-shapes and 
discard any non-canonical V-shapes considered by our algorithm.


\begin{lemma}
\label{CanonicalV-shape}
At least one minimum-width V-shape covering  is canonical.
\end{lemma}

\begin{proof}

In order to prove this lemma, we first explain why we can assume
that .  We then discuss how
the boundary points may be positioned on the arms.


By Lemma~\ref{lem:no-empty-strip}, in no minimum-width covering V-shape
one strip covers~ completely.
Hence in the following we disregard this
possibility.

We present a sequence of transformations, starting with a minimum-width
V-shape covering~, which do not
increase its width and end in a canonical V-shape.  We begin by
translating its outer boundary in the direction of first  and
then  to ensure that each of the outer rays contains a point of ;
this point might be .  Now translating the inner boundary, first in
the direction opposite to that of , and then that of , we
guarantee that each of ,  contains a point of .  (By
Lemma~\ref{lem:no-empty-strip}, an outer ray  cannot meet its
corresponding inner ray  without meeting a point of , as we
started with a minimum-width V-shape.)  Now consider an arm (say,
left) of the resulting V-shape.  We will further transform it so that
.  We have so far ensured that each of
 and  contains at least one point.  If exactly one point is
present on each left ray,  and  can be rotated so that the
width of  shrinks.  
This process stops either when  collapses to a line (in which
case it is easy to check that  contains two points of  and
 contains at least one) or when three points 
lie in , say  and  on one ray and  on the
other.  In the latter case, unless the angles  and  are acute, the rotation can proceed, further narrowing .
Corollary~\ref{Cor:angle} follows from this last condition.


Repeating the process with the right arm, we arrive at a covering
V-shape whose width is no larger than that of the original V-shape,
with the property that (a)~it satisfies Corollary~\ref{Cor:angle} if there is no zero-width arm,
(b)~each bounding ray contains a point of , and (c)~ and ; i.e., the resulting covering V-shape is canonical and as good or
better in terms of width.  Hence, indeed, it is sufficient to examine
only canonical V-shapes.  \hfill 

\end{proof}

\begin{corollary}
  \label{Cor:angle}
  Let  be a point set and  be a minimum-width canonical V-shape
  covering it.  Let  be the points on the
  boundary of a non-zero-width arm of , with  and  on one
  ray and  on the other.  Then the angles 
  and  are acute.
\end{corollary}



All canonical minimum-width V-shapes fall into the following three
categories: \begin{description}\itemsep 0pt \parsep 0pt \parskip 0pt
\item[both-outer] Each outer ray contains exactly two points of , and each
  inner ray contains at least one. 
\item[inner-outer] On one arm of the V-shape, the inner ray contains exactly two points of ; on the other
  arm, the outer ray contains exactly two points. The other rays contain at least one point of .
\item[both-inner] Each inner ray contains exactly two points of  and each outer 
  ray contains at least one.
\end{description}

\section{Computing a canonical minimum-width V-shape}
\label{sec:algorithm}

To find a canonical minimum-width V-shape covering , we will search
independently for the best solution for each of the three types identified
above and output the V-shape that minimizes the width.  Let  be the
convex hull of .

\paragraph*{V-shapes of both-outer type.}




\iffalse
Two points of  lying on a common outer ray determine an edge of the
convex hull  of .  We compute  in 
time \cite{ComputationalGeometry} and test all pairs of edges ,
 of .  For each such pair, we find a canonical V-shape of
minimum width whose outer rays ,  contain , ,
respectively, if one exists.

For a pair of edges ,  from  we do the following:
Compute the pair of rays , 
emanating from a common vertex .  Let  be the ray bisecting the
angle between  and .  
Call the points on the  side of , , and , .
(We argue in appendix~\ref{sec:obey-bisectors} that a minimum width V-shape 
with  and  on its boundary has  in its left arm and  in its right arm.)





\begin{wrapfigure}[10]{l}{0.55 \textwidth}
\scalebox{0.38}{\begin{picture}(0,0)\includegraphics{V3.pdf}\end{picture}\setlength{\unitlength}{3947sp}\begingroup\makeatletter\ifx\SetFigFont\undefined \gdef\SetFigFont#1#2#3#4#5{\reset@font\fontsize{#1}{#2pt}\fontfamily{#3}\fontseries{#4}\fontshape{#5}\selectfont}\fi\endgroup \begin{picture}(7917,3823)(2491,-6917)
\put(8056,-5686){\makebox(0,0)[lb]{\smash{{\SetFigFont{20}{24.0}{\rmdefault}{\mddefault}{\updefault}{\color[rgb]{0,0,0}}}}}}
\put(6556,-6781){\makebox(0,0)[lb]{\smash{{\SetFigFont{20}{24.0}{\rmdefault}{\mddefault}{\updefault}{\color[rgb]{0,0,0}}}}}}
\put(4201,-5611){\makebox(0,0)[lb]{\smash{{\SetFigFont{20}{24.0}{\rmdefault}{\mddefault}{\updefault}{\color[rgb]{0,0,0}}}}}}
\put(6676,-3421){\makebox(0,0)[lb]{\smash{{\SetFigFont{20}{24.0}{\rmdefault}{\mddefault}{\updefault}{\color[rgb]{0,0,0}}}}}}
\put(2506,-4006){\makebox(0,0)[lb]{\smash{{\SetFigFont{20}{24.0}{\rmdefault}{\mddefault}{\updefault}{\color[rgb]{0,0,0}}}}}}
\put(10171,-3781){\makebox(0,0)[lb]{\smash{{\SetFigFont{20}{24.0}{\rmdefault}{\mddefault}{\updefault}{\color[rgb]{0,0,0}}}}}}
\put(5626,-4441){\makebox(0,0)[lb]{\smash{{\SetFigFont{20}{24.0}{\rmdefault}{\mddefault}{\updefault}{\color[rgb]{0,0,0}}}}}}
\put(7231,-4471){\makebox(0,0)[lb]{\smash{{\SetFigFont{20}{24.0}{\rmdefault}{\mddefault}{\updefault}{\color[rgb]{0,0,0}}}}}}
\end{picture} }
\end{wrapfigure}



Our algorithm proceeds as follows: for every pair of edges ,
we compute the point furthest from  in  and the point
furthest from  in  in  time, thereby determining the shape and width
of a candidate canonical minimum-width V-shape;
refer to the figure on the left. Examining all  candidate pairs we find the
best both-outer canonical V-shape in  time.
\fi

Consider a covering V-shape~ with outer rays 
\begin{wrapfigure}[8]{r}{0.5 \textwidth}
  \scalebox{0.38}{\begin{picture}(0,0)\includegraphics{V3.pdf}\end{picture}\setlength{\unitlength}{3947sp}\begingroup\makeatletter\ifx\SetFigFont\undefined \gdef\SetFigFont#1#2#3#4#5{\reset@font\fontsize{#1}{#2pt}\fontfamily{#3}\fontseries{#4}\fontshape{#5}\selectfont}\fi\endgroup \begin{picture}(7917,3823)(2491,-6917)
\put(8056,-5686){\makebox(0,0)[lb]{\smash{{\SetFigFont{20}{24.0}{\rmdefault}{\mddefault}{\updefault}{\color[rgb]{0,0,0}}}}}}
\put(6556,-6781){\makebox(0,0)[lb]{\smash{{\SetFigFont{20}{24.0}{\rmdefault}{\mddefault}{\updefault}{\color[rgb]{0,0,0}}}}}}
\put(4201,-5611){\makebox(0,0)[lb]{\smash{{\SetFigFont{20}{24.0}{\rmdefault}{\mddefault}{\updefault}{\color[rgb]{0,0,0}}}}}}
\put(6676,-3421){\makebox(0,0)[lb]{\smash{{\SetFigFont{20}{24.0}{\rmdefault}{\mddefault}{\updefault}{\color[rgb]{0,0,0}}}}}}
\put(2506,-4006){\makebox(0,0)[lb]{\smash{{\SetFigFont{20}{24.0}{\rmdefault}{\mddefault}{\updefault}{\color[rgb]{0,0,0}}}}}}
\put(10171,-3781){\makebox(0,0)[lb]{\smash{{\SetFigFont{20}{24.0}{\rmdefault}{\mddefault}{\updefault}{\color[rgb]{0,0,0}}}}}}
\put(5626,-4441){\makebox(0,0)[lb]{\smash{{\SetFigFont{20}{24.0}{\rmdefault}{\mddefault}{\updefault}{\color[rgb]{0,0,0}}}}}}
\put(7231,-4471){\makebox(0,0)[lb]{\smash{{\SetFigFont{20}{24.0}{\rmdefault}{\mddefault}{\updefault}{\color[rgb]{0,0,0}}}}}}
\end{picture} }
\end{wrapfigure}
containing edges  of , respectively; refer to the figure
on the right.  Let  be the bisector of the angle . 
Notice that  is not minimal unless its width is given by the
largest distance from a point in  to its closest outer ray.
Therefore, we can assume that points of  left of  belong to the
left arm of  and points right of ---to its right arm.



Thus, given , , and  it is sufficient to determine the
furthest point from~ to the left of  and the furthest point
from  to the right of .  The larger distance determines the
width of .  This can be accomplished by building a data structure
 on  that supports the following queries: Given a
halfplane~ and a direction~, return an extreme point of  in direction .   queries are sufficient to enumerate all
choices of  and identify the best both-outer-type V-shape.
 can be constructed in  time and supports
logarithmic-time queries, resulting in total running time of
.

 is constructed as follows: We build the arrangement
 of lines dual to points of .  Cells of  correspond
to different ways to partition  by a line.  We construct a directed
spanning tree  of the cells of , starting with the bottommost
cell and allowing only arcs from a cell~ to a cell immediately
above~ and sharing an edge with it; we use  to denote
the convex hull of the set of points whose dual lines lie below .  Using  as the
history tree, we store the convex hull  for every face , in a fully persistent version~\cite{Bob} of the semi-dynamic
convex hull data structure of~\cite{Preparata}.  We also preprocess
 for point location.  Given a query (say, upper) half-plane  and
direction~, we locate the face  of  containing the point
dual to the bounding line of  and consult the data structure
associated with  and storing  to find the extreme
point of  in direction~, all in logarithmic time.  


\iffalse
The cells of 
correspond to different ways of splitting  with a line .
Namely, consider the face  containing .  The lines
of  below  correspond to points of  above ; let this
set of points be .
Now consider a directed version  of
the adjacency graph of the faces of : Two faces  and  in
 are joined by a \emph{directed} edge  if and only if 
and  share an edge and  lies below the line containing this
edge, while  lies above it.  The graph  has a unique
source---the bottommost face  of .  It is easy to
check that every face can be reached by a path from .
Construct a spanning tree  of , starting at . 

We now consider the semi-dynamic convex hull algorithm
\cite{Preparata}, which uses an AVL-tree based data structure for
storing convex hulls, querying them in  time, and
supporting logarithmic-time insertion operations.  Since this is a
pointer-based data structure with bounded in-degree, we can make it
fully persistent \cite{Bob}.  We now use  as the history
tree---starting with  and ,
we follow , for , constructing  from  by
adding the single point corresponding to the line separating  from
 to the semi-dynamic convex hull data structure storing . 

We also preprocess  for logarithmic-time point location, in
 time and  space, and associate with each face~
the version of the semi-dynamic convex hull data structure
corresponding to .  The queries are now answered as follows:
Given a line , we use point location in  to identify the
face  containing .  We then access the version of the
semi-dynamic convex hull structure  corresponding to face
 and find the extreme point in direction  in logarithmic
time.

This concludes our description of our data structure and the
algorithm; the running time is ; the space used is
.
\fi

\paragraph{V-shapes of inner-outer type.}




In this section, we describe how to find a minimum-width canonical
V-shape covering  and having exactly one edge of , say~,
on its outer boundary; it
contains two points of  on the inner bounding ray of its other arm.
We handle each choice of  independently, in  time,
yielding overall  running time.

\begin{wrapfigure}[15]{r}{0.45\textwidth}
 \includegraphics[scale=0.85]{inner-outer}
\end{wrapfigure}

Having fixed an edge  of , consider a (minimum-width
canonical) V-shape~ covering~ that has  on its boundary. For
ease of description, suppose ,  contains two
points , while both  and  contain at least one point
of  each, denoted  and , respectively; see the figure on the right.


Let  be the line containing , and  be the line
containing .  Set  and .  We
observe that
\begin{enumerate} \itemsep 0pt \parsep 0pt \parskip 0pt
\item  is the set of points of  at distance at most
   from ;
\item  is an edge of ;
\item  is contained in a supporting line  of
   (which must also be a supporting line to  for 
   to cover ) parallel to ; this line lies on
  the same side of  as ;\footnote{If , we have  and .}
   and
\item .
\end{enumerate}
Our algorithm enumerates all choices for the point , in order of
decreasing distance from . 
  For the current choice of , it
maintains (the boundary of)
, say as an AVL~tree, and, for each edge  of
, 
the distance from  to the furthest point of  to the
right of (i.e., ``beyond'') .
Edges with distances are
stored in a min-heap; the minimum such distance gives the minimum
width for  for the current choice of ; the larger of the two widths
determines the width of the current V-shape.  We record the best width
of any V-shape encountered in the process.

The algorithm is initialized with the set  containing the two points of  furthest from  (the case where  contains 
only one point treated by the both-outer case as the zero-width strip  can be rotated until it 
contains one edge of ).  A generic step of the
algorithm involves moving the current point  from  to .  We
update the convex hull of  by computing the supporting tangents
from  to the old hull, in  time.  For the two new hull
edges , , we compute the corresponding supporting lines
 of , using a suitable balanced-tree
representation of , also in logarithmic time.  We add the new
edges with the corresponding widths to the min-heap, after removing
from it the entries of all the eliminated edges of .  The
root of the min-heap yields the best width for  for the current
partition .
The algorithm requires presorting points by distance from  and
then a linear number of balanced-search-tree and heap operations
(since the number of edges inserted is less than  and each cannot
be deleted more than once), for a total running time of 
for a fixed , as claimed.

Working through the entire set  (except for the endpoints of ), in
order of decreasing distance from , growing  and shrinking
, we obtain a sequence of fewer than~ V-shapes which include
all the canonical minimum-width V-shapes covering  with  on its
outer boundary and two other points of  lying on the opposite arm's
inner boundary (it may include non-canonical V-shapes as well, but it
is not difficult to check  
that every combination  examined
by the algorithm yields a valid V-shape covering , which is
sufficient for our purposes). 


To summarize, inner-outer type V-shapes can be handled in total
time .

\paragraph{V-shapes of both-inner type.}
Now a covering V-shape~ has points  of  on its inner ray
 and points  on its inner ray ; refer to Firgure~\ref{V4}; 
points  are in convex position, in this counterclockwise order.  
\begin{figure}
\centering
\scalebox{0.5}{\begin{picture}(0,0)\includegraphics{V11.pdf}\end{picture}\setlength{\unitlength}{3947sp}\begingroup\makeatletter\ifx\SetFigFont\undefined \gdef\SetFigFont#1#2#3#4#5{\reset@font\fontsize{#1}{#2pt}\fontfamily{#3}\fontseries{#4}\fontshape{#5}\selectfont}\fi\endgroup \begin{picture}(5405,3193)(3679,-6467)
\put(8566,-5326){\makebox(0,0)[lb]{\smash{{\SetFigFont{20}{24.0}{\rmdefault}{\mddefault}{\updefault}{\color[rgb]{0,0,0}}}}}}
\put(6886,-6331){\makebox(0,0)[lb]{\smash{{\SetFigFont{20}{24.0}{\rmdefault}{\mddefault}{\updefault}{\color[rgb]{0,0,0}}}}}}
\put(5716,-4771){\makebox(0,0)[lb]{\smash{{\SetFigFont{20}{24.0}{\rmdefault}{\mddefault}{\updefault}{\color[rgb]{0,0,0}}}}}}
\put(6676,-4201){\makebox(0,0)[lb]{\smash{{\SetFigFont{20}{24.0}{\rmdefault}{\mddefault}{\updefault}{\color[rgb]{0,0,0}}}}}}
\put(6976,-5686){\makebox(0,0)[lb]{\smash{{\SetFigFont{20}{24.0}{\rmdefault}{\mddefault}{\updefault}{\color[rgb]{0,0,0}}}}}}
\put(6181,-5851){\makebox(0,0)[lb]{\smash{{\SetFigFont{20}{24.0}{\rmdefault}{\mddefault}{\updefault}{\color[rgb]{0,0,0}}}}}}
\end{picture} }
\caption{Empty wedge defined by  and .}
\label{V4}
\end{figure}
It is known~\cite{EmptyPolygons} that there are at most  such
\emph{wedges}  determined by a quadruple of points
 and empty of points of ; note that  completely
determines~, and, given~, one can construct the
corresponding~ in  time,
so it is sufficient to enumerate all empty wedges~.







For a pair , we compute all pairs , so that
 is an empty wedge.  
Let  be the set of all points of  lying to the left of
the directed line .  


\begin{observation}
  , in the above notation, is an empty wedge if and only
  if line  supports  and separates segment 
  from  (and  are in this counterclockwise order).
\end{observation}

Now enumerating all ~pairs  for a fixed choice of  can be
done in time , as follows.
While handling V-shapes of both-outer type
we constructed a data structure  which, for a given line (here
), produces a balanced search tree storing the convex hull of the
points of  lying to one side of the line (here ).  Using
, we find the point  of  closest to the line  and
traverse the boundary of  in both directions from ,
to list all  edges  of  satisfying the conditions
of the above observation.  
Since all such edges are consecutive, it is sufficient to examine
 edges of .
Repeating the procedure for all choices of  and recalling that
the number of empty wedges is at most quadratic, we deduce that the
enumeration algorithm runs in time .



\section{Maximum number of canonical minimum-width V-shapes}
\label{sec:lower-bound}

How far is our algorithm from optimality?  In Figure~\ref{fig:lower},
starting with the vertex set of two congruent regular -gons, for a
suitably large~, we sketch a construction of a set of  points
with  distinct covering minimum-width V-shapes.  The idea
is that a minimum-width covering V-shape would consist essentially of
two independently chosen minimum-width strips, each covering one
-gon.  The point set is highly degenerate.  However, perturbing it
slightly yields a point set with  canonical V-shapes with
width arbitrarily close to minimum possible.  This is an indication
that any algorithm that explicitly enumerates canonical covering
V-shapes may have to spend  time on this input, thus it
is unlikely that any algorithm taking our approach can run much
faster.  On the other hand, for this specific input one can encode the
 optimal V-shapes in  space.  This leaves open
the possibility that identifying a single minimum-width covering
V-shape may still be possible in subquadratic worst-case time.

\begin{figure}
  \centering
  \includegraphics[scale=0.8]{lower}
  \caption{A sketch of a construction of a set with many minimum-width
    covering V-shapes.  Points below and above the two regular
    -gons are added to raise the minimum width of any covering
    V-shape to that of the -gons.}
  \label{fig:lower}
\end{figure}


\section{A 13-approximation algorithm}
\label{sec:13-approx}

Given a set of points , let  be the minimum value such that~
can be covered by a V-shape of width~.
We present an algorithm that computes a V-shape covering  of width at most  in time .
For this purpose, we use the  time 6-approximation algorithm for the 2-line-center problem  presented by Agarwal, Procopiuc, and Varadarajan \cite{2-line-center}.
Recall that the 2-line-center problem is the following: Given a set  of 
points in , cover  by two congruent strips of minimum
width.  We start with the following observation which follows from the
fact that the union of the two strips of any V-shape covering  contains .

\begin{observation}
  \label{Obs:width}
  If  is the width of two congruent strips of minimum width covering , .
\end{observation}


Our 13-approximation algorithm proceeds as follows.  Use the 6-approximation
algorithm of \cite{2-line-center} to compute two congruent
strips of width  that cover , with .
(It is possible that the two strips just computed are such
that a V-shape defined by them contains .  In this case we return
that V-shape.  This clearly produces a 6-approximation, due to
Observation~\ref{Obs:width}.  In the remainder of this section, we
will assume that this is not the case, in other words, one of the two
strips has points of  on both sides of it.)
Find the median lines  and  of
the strips.  For all points in each strip, project them orthogonally
onto 
and  respectively (the points in the intersection of the
strips are duplicated and projected onto both  and ).
Let  be the resulting set of projected points.  Compute an exact
minimum width V-shape  covering  (see Section~\ref{sec:two-lines})
in  time.  The desired approximate V-shape  is
obtained by widening  by  in all directions.



\begin{theorem}
This algorithm computes a 13-approximation of a minimum-width V-shape covering~.
\end{theorem}
\begin{proof}
  Let  be a minimum-width covering V-shape of ,  --- a
  minimum-width covering V-shape of , and  --- the approximate
  covering V-shape computed by the algorithm.
  As the points of  have been moved by a distance
  of at most  to form , .
  Since  is a widened version of , it contains the points
  of .  Moreover,  by
  Observation~\ref{Obs:width}. \hfill 
\end{proof}

\begin{uremark}
  Using the -approximation
  algorithm of \cite{2-line-center} in place of their 6-approximation
  algorithm in our procedure, we can attain any approximation factor
  larger than 
  three for the minimum-width V-shape.  The running time remains , with the constant of proportionality depending on the
  quality of the approximation.  We do not discuss this extension further,
  since we present our own -approximation algorithm for the
  problem in Section~\ref{sec:one+eps}.
\end{uremark}

\subsection{Minimum-width V-shape for points on two lines}
\label{sec:two-lines}

We now describe how to compute the minimum-width V-shape~ covering
a given point set~ contained in the union of two lines , 
in the plane.  Put .

Let  and .  If  is
covered by a zero-width -shape, which is easy to check, we are
done.  From now on we assume that this is not the case, i.e.,
that ,  do not already form a V-shape
containing , so  separates some two points of  and/or 
separates some two points of .  The convex hull  has
three or four vertices.  Moreover, by reasoning similar to that of
Section~\ref{sec:algorithm}, the outer boundary of  
contains two, three, or four vertices of  (in the case where 
an outer ray is contained in  or , we consider only
the extreme points).  Before describing
how we handle these cases, we need a technical lemma.

\begin{lemma}
  \label{lem:cut}
  Given a line partitioning  into ,  and given
  their convex hulls , , the minimum-width
  canonical V-shape~ of  containing  in one strip and  in
  the other can be computed in constant time; some points
  of  might lie in both strips of .
\end{lemma}

\begin{proof}
  The convex hulls  and  have at most four
  vertices each. It must be the case that the
  boundary of one arm of  contains an edge  of  or
  an outer common tangent of  and , and the
  other arm boundary contains an edge  of  or an
  outer common tangent to  and .  There is
  a constant number of possible pairs of such edges.  Let  be
  the minimum-width strip covering a set  and parallel to .  For
  each such pair of edges , check whether 
  and  form a V-shape.  Return the canonical V-shape of
  minimum width among all V-shapes so generated. \hfill 
\end{proof}

Now we consider the different types of canonical V-shapes
covering~ and describe how to find a minimum-width V-shape of each
type.

\emph{Case 1: An outer bounding ray of  contains an edge  of .}
Let  be the line containing .  For all points  of ,
draw a line  through  and parallel to~.  Apply
Lemma~\ref{lem:cut} to (the partition induced by) .  This can be
implemented to run in overall time .

\smallskip

In the remaining cases, each of the outer rays of  contains
precisely one vertex of  and each inner ray contains two
points of .

\emph{Case 2: An inner ray of  lies on  or .}
Suppose an inner ray of  is contained in .
Draw two lines parallel to  and very close to it,
one to the left of , one to the right of .
Apply Lemma~\ref{lem:cut} to each of these two lines.

\emph{Case 3: Point  lies between the two arms of .}
Draw the two lines passing through  and bisecting the angles
between  and .  
Apply Lemma~\ref{lem:cut} to each of these two
lines.

\emph{Case 4: Point  is inside one arm of .}
For each pair of consecutive points  on  or
on  not separated by , apply Lemma~\ref{lem:cut} to the
perpendicular bisector of the segment .

Now we argue that the last procedure returns the best minimum-width
V-shape  of  with two points on its outer boundary and  in
one of its arms, correctly handling case~4 and thereby concluding our description.

Let  and  be the vertices of .  For ease of presentation,
rotate the entire picture so that  lies below ; refer to figure~\ref{V8}.
Let  be the point of  on , ,  be the points on , with  closer to  than .
Similarly let  be the point on , ,  be the points on , with  closer to  than . As  and  don't intersect between the two arms of ,  and  lie on one line,  and  lie on the other line.
Let  and  lie on , and  and  lie on
, without loss of generality.


\begin{figure}
\centering
\subfloat[Three boundary points on ,.]{\label{V8} \scalebox{0.4}{\begin{picture}(0,0)\includegraphics{V8b.pdf}\end{picture}\setlength{\unitlength}{3947sp}\begingroup\makeatletter\ifx\SetFigFont\undefined \gdef\SetFigFont#1#2#3#4#5{\reset@font\fontsize{#1}{#2pt}\fontfamily{#3}\fontseries{#4}\fontshape{#5}\selectfont}\fi\endgroup \begin{picture}(5775,7361)(3452,-7906)
\put(6766,-2821){\makebox(0,0)[lb]{\smash{{\SetFigFont{20}{24.0}{\rmdefault}{\mddefault}{\updefault}{\color[rgb]{0,0,0}}}}}}
\put(7699,-3884){\makebox(0,0)[lb]{\smash{{\SetFigFont{20}{24.0}{\rmdefault}{\mddefault}{\updefault}{\color[rgb]{0,0,0}}}}}}
\put(6301,-2536){\makebox(0,0)[lb]{\smash{{\SetFigFont{20}{24.0}{\rmdefault}{\mddefault}{\updefault}{\color[rgb]{0,0,0}}}}}}
\put(5026,-3886){\makebox(0,0)[lb]{\smash{{\SetFigFont{20}{24.0}{\rmdefault}{\mddefault}{\updefault}{\color[rgb]{0,0,0}}}}}}
\put(5401,-4141){\makebox(0,0)[lb]{\smash{{\SetFigFont{20}{24.0}{\rmdefault}{\mddefault}{\updefault}{\color[rgb]{0,0,0}}}}}}
\put(6451,-4261){\makebox(0,0)[lb]{\smash{{\SetFigFont{20}{24.0}{\rmdefault}{\mddefault}{\updefault}{\color[rgb]{0,0,0}}}}}}
\put(4293,-5373){\makebox(0,0)[lb]{\smash{{\SetFigFont{20}{24.0}{\rmdefault}{\mddefault}{\updefault}{\color[rgb]{0,0,0}}}}}}
\put(5371,-4635){\makebox(0,0)[lb]{\smash{{\SetFigFont{20}{24.0}{\rmdefault}{\mddefault}{\updefault}{\color[rgb]{0,0,0}}}}}}
\put(5957,-5356){\makebox(0,0)[lb]{\smash{{\SetFigFont{20}{24.0}{\rmdefault}{\mddefault}{\updefault}{\color[rgb]{0,0,0}}}}}}
\put(6691,-7770){\makebox(0,0)[lb]{\smash{{\SetFigFont{20}{24.0}{\rmdefault}{\mddefault}{\updefault}{\color[rgb]{0,0,0}}}}}}
\put(3488,-4730){\makebox(0,0)[lb]{\smash{{\SetFigFont{20}{24.0}{\rmdefault}{\mddefault}{\updefault}{\color[rgb]{0,0,0}}}}}}
\put(3580,-5568){\makebox(0,0)[lb]{\smash{{\SetFigFont{20}{24.0}{\rmdefault}{\mddefault}{\updefault}{\color[rgb]{0,0,0}}}}}}
\put(5626,-4426){\makebox(0,0)[lb]{\smash{{\SetFigFont{20}{24.0}{\rmdefault}{\mddefault}{\updefault}{\color[rgb]{0,0,0}}}}}}
\put(5956,-4321){\makebox(0,0)[lb]{\smash{{\SetFigFont{20}{24.0}{\rmdefault}{\mddefault}{\updefault}{\color[rgb]{0,0,0}}}}}}
\end{picture} }}
\subfloat[ Four boundary points on ]{\label{V9} \scalebox{0.4}{\begin{picture}(0,0)\includegraphics{V9.pdf}\end{picture}\setlength{\unitlength}{3947sp}\begingroup\makeatletter\ifx\SetFigFont\undefined \gdef\SetFigFont#1#2#3#4#5{\reset@font\fontsize{#1}{#2pt}\fontfamily{#3}\fontseries{#4}\fontshape{#5}\selectfont}\fi\endgroup \begin{picture}(5775,7361)(3452,-7906)
\put(5616,-2600){\makebox(0,0)[lb]{\smash{{\SetFigFont{20}{24.0}{\rmdefault}{\mddefault}{\updefault}{\color[rgb]{0,0,0}}}}}}
\put(6451,-4261){\makebox(0,0)[lb]{\smash{{\SetFigFont{20}{24.0}{\rmdefault}{\mddefault}{\updefault}{\color[rgb]{0,0,0}}}}}}
\put(4293,-5373){\makebox(0,0)[lb]{\smash{{\SetFigFont{20}{24.0}{\rmdefault}{\mddefault}{\updefault}{\color[rgb]{0,0,0}}}}}}
\put(8366,-1263){\makebox(0,0)[lb]{\smash{{\SetFigFont{20}{24.0}{\rmdefault}{\mddefault}{\updefault}{\color[rgb]{0,0,0}}}}}}
\put(5852,-3696){\makebox(0,0)[lb]{\smash{{\SetFigFont{20}{24.0}{\rmdefault}{\mddefault}{\updefault}{\color[rgb]{0,0,0}}}}}}
\put(5507,-3981){\makebox(0,0)[lb]{\smash{{\SetFigFont{20}{24.0}{\rmdefault}{\mddefault}{\updefault}{\color[rgb]{0,0,0}}}}}}
\put(5371,-4635){\makebox(0,0)[lb]{\smash{{\SetFigFont{20}{24.0}{\rmdefault}{\mddefault}{\updefault}{\color[rgb]{0,0,0}}}}}}
\put(5957,-5356){\makebox(0,0)[lb]{\smash{{\SetFigFont{20}{24.0}{\rmdefault}{\mddefault}{\updefault}{\color[rgb]{0,0,0}}}}}}
\put(6691,-7770){\makebox(0,0)[lb]{\smash{{\SetFigFont{20}{24.0}{\rmdefault}{\mddefault}{\updefault}{\color[rgb]{0,0,0}}}}}}
\put(6708,-2867){\makebox(0,0)[lb]{\smash{{\SetFigFont{20}{24.0}{\rmdefault}{\mddefault}{\updefault}{\color[rgb]{0,0,0}}}}}}
\put(4786,-4140){\makebox(0,0)[lb]{\smash{{\SetFigFont{20}{24.0}{\rmdefault}{\mddefault}{\updefault}{\color[rgb]{0,0,0}}}}}}
\put(3580,-5568){\makebox(0,0)[lb]{\smash{{\SetFigFont{20}{24.0}{\rmdefault}{\mddefault}{\updefault}{\color[rgb]{0,0,0}}}}}}
\put(3488,-4730){\makebox(0,0)[lb]{\smash{{\SetFigFont{20}{24.0}{\rmdefault}{\mddefault}{\updefault}{\color[rgb]{0,0,0}}}}}}
\end{picture} }}
\newline
\subfloat[Four boundary points on .]{\label{V10} \scalebox{0.4}{\begin{picture}(0,0)\includegraphics{V10b.pdf}\end{picture}\setlength{\unitlength}{3947sp}\begingroup\makeatletter\ifx\SetFigFont\undefined \gdef\SetFigFont#1#2#3#4#5{\reset@font\fontsize{#1}{#2pt}\fontfamily{#3}\fontseries{#4}\fontshape{#5}\selectfont}\fi\endgroup \begin{picture}(5775,7361)(3452,-7906)
\put(3944,-5040){\makebox(0,0)[lb]{\smash{{\SetFigFont{20}{24.0}{\rmdefault}{\mddefault}{\updefault}{\color[rgb]{0,0,0}}}}}}
\put(4786,-4140){\makebox(0,0)[lb]{\smash{{\SetFigFont{20}{24.0}{\rmdefault}{\mddefault}{\updefault}{\color[rgb]{0,0,0}}}}}}
\put(6451,-4261){\makebox(0,0)[lb]{\smash{{\SetFigFont{20}{24.0}{\rmdefault}{\mddefault}{\updefault}{\color[rgb]{0,0,0}}}}}}
\put(5371,-4635){\makebox(0,0)[lb]{\smash{{\SetFigFont{20}{24.0}{\rmdefault}{\mddefault}{\updefault}{\color[rgb]{0,0,0}}}}}}
\put(5957,-5356){\makebox(0,0)[lb]{\smash{{\SetFigFont{20}{24.0}{\rmdefault}{\mddefault}{\updefault}{\color[rgb]{0,0,0}}}}}}
\put(6691,-7770){\makebox(0,0)[lb]{\smash{{\SetFigFont{20}{24.0}{\rmdefault}{\mddefault}{\updefault}{\color[rgb]{0,0,0}}}}}}
\put(3488,-4730){\makebox(0,0)[lb]{\smash{{\SetFigFont{20}{24.0}{\rmdefault}{\mddefault}{\updefault}{\color[rgb]{0,0,0}}}}}}
\put(3580,-5568){\makebox(0,0)[lb]{\smash{{\SetFigFont{20}{24.0}{\rmdefault}{\mddefault}{\updefault}{\color[rgb]{0,0,0}}}}}}
\put(7699,-3884){\makebox(0,0)[lb]{\smash{{\SetFigFont{20}{24.0}{\rmdefault}{\mddefault}{\updefault}{\color[rgb]{0,0,0}}}}}}
\put(6318,-2552){\makebox(0,0)[lb]{\smash{{\SetFigFont{20}{24.0}{\rmdefault}{\mddefault}{\updefault}{\color[rgb]{0,0,0}}}}}}
\put(5611,-4411){\makebox(0,0)[lb]{\smash{{\SetFigFont{20}{24.0}{\rmdefault}{\mddefault}{\updefault}{\color[rgb]{0,0,0}}}}}}
\put(5941,-4306){\makebox(0,0)[lb]{\smash{{\SetFigFont{20}{24.0}{\rmdefault}{\mddefault}{\updefault}{\color[rgb]{0,0,0}}}}}}
\end{picture} }}
\caption{}
\end{figure}

The three points of  on one arm boundary cannot lie on the same
line , as they form a triangle.  Therefore either each line  contains
three boundary points belonging to three different boundary rays, or one
line contains four boundary points from four boundary rays, and the
other line contains two boundary points from the two inner rays.

We consider all possible cases: 
\begin{enumerate} \itemsep 0pt \parsep 0pt \parskip 0pt
\item  contains , , and , and  contains , , and  ( contains , , and , and  contains , , and  is a symmetric case).
\item  contains four boundary points,  contains two boundary points.
\item  contains two boundary points,  contains four boundary points.
\end{enumerate}
In each case, we prove either that the configuration
of the points is impossible, or that the angles
 and  (see
figure~\ref{V8}) between  and the inner boundary rays are
acute.  When the angles are acute, the perpendicular bisector of 
separates the points of  belonging to each arm of .  Therefore
Lemma~\ref{lem:cut} can be applied and our handling of case~4 is justified.

We consider the cases in turn:
\begin{enumerate} \itemsep 0pt \parsep 0pt \parskip 0pt
\item  contains , , and , and  contains
  , , and 
(see figure~\ref{V8}).
As the angle  is acute, its opposite angle  is acute. 
What is left to prove is that  is acute as well.
The angle  is acute, hence  is obtuse,
and so is .
But the angle  is acute, hence  does not intersect  in the right arm of , so  and  intersect in the left arm.
More precisely  intersects the segment .
The opposite angle of  is smaller than the angle , which is acute.
Therefore so is .
\item  contains four boundary points,  contains two boundary points.
Let  be the line containing ,  be the line containing , and  be the line containing  (see figure~\ref{V9}).
By Corollary~\ref{Cor:angle}, as the angle  is acute,  forms an acute angle  with .
Similarly, as the angle  is acute,  forms an acute angle  with .
But as ,  and  are supplementary, a contradiction.
\item  contains two boundary points,  contains four boundary points (see figure~\ref{V10}).
By Corollary~\ref{Cor:angle}, the angles  and  are acute, therefore their opposite angles  and  are acute as well.
\end{enumerate}





\section{A -approximation algorithm}
\label{sec:one+eps}

In this section we describe how to construct, given a point set 
and a real number , a V-shape  covering~, with
, where  is the width of a
minimum-width V-shape covering .

We start by recalling the notion of an anchor pair used in
\cite{2-line-center}.  Given a V-shape~ covering , fix one of
the strips of , say .  We say that a pair of points  is an \emph{anchor pair}, if .  Lemma~3.3 in~\cite{2-line-center} describes how to identify
at most~11 pairs of points in~, such that, for any two-strip cover
of~, at least one of the pairs is an anchor pair for one of the
strips; the algorithm requires  time.  As covering by a
V-shape is a special case of covering by two strips, the definition
and the algorithm apply here as well.

We show how to, given a potential anchor pair~, construct a
-approximation of the minimum-width V-shape covering  for
which  is an anchor pair.  More precisely, below we prove
\begin{lemma}
  \label{lem:fixed-anchor-pair}
  Given a potential anchor pair , we can construct, in time
  , a V-shape
  covering , of width at most  times the minimum width of
  any V-shape covering  for which  is an anchor pair.
\end{lemma}

Applying this procedure at most~11 times, we obtain our desired
approximation algorithm:
\begin{theorem}
  A V-shape covering  and of width at most  can be
  constructed in time .
\end{theorem}

We first prove that it is sufficient to consider those V-shapes~
with anchor pair , for which the strip containing  has one
of a small set of fixed directions.  Setting  and
, we
prove the following

\begin{lemma}
  \label{lem:rotate-strip}
  Let V-shape  cover , and let  be an anchor pair for
  .  Rotating  by an angle at most  does not
  increase the width of the V-shape by more than a factor of
  , and the angle between  and the direction of the
  rotated strip cannot exceed .
\end{lemma}

\begin{proof}
  Put .  Let  be the minimum bounding box of .  More precisely, it is the shortest rectangle cut out of 
  by two lines perpendicular to  and containing ;
  refer to Figure~\ref{fig:rotated}.  Let  and  be the
  length (along the axis of ) and width of , respectively.
\begin{figure}
 \centering
    \includegraphics[width=0.5\textwidth]{rotate}
    \caption{Rotation by  does not change the width by much.}
    \label{fig:rotated}
\end{figure}
Let  be the minimal parallel strip containing , whose
  direction is  away from that of  (there are two choices for , corresponding to
  rotating clockwise and counterclockwise; only one is shown; the
  argument applies to both cases).  
  Then
  
  since .  Now
  replace  by  to obtain a new V-shape  covering~.
  Its width is , as
  claimed.

  Observe that in the above construction, the angle between  and
  the direction of  cannot exceed
   \hfill 
  \end{proof} 

We conclude that enumerating all V-shapes that contain  in their
strip  and whose directions are (a)~at most  away from
that of  and (b) spaced at most  apart, would yield a
V-shape whose existence is claimed in
Lemma~\ref{lem:fixed-anchor-pair}.  The number of directions to be
tested is at most .

Given a candidate anchor pair , the algorithm proceeds by
starting with the direction .  Since we need not consider V-shapes
whose width is larger than the approximate width  computed in
Section~\ref{sec:13-approx} (this is where the 13-approximation
algorithm is used to bootstrap our -approximation), we
replace  by the smaller  in the
definition of  above and by the larger  in the
definition of , thereby erring on the conservative side in
each case.  Having computed (conservative estimates of)  and
, we enumerate the  directions of the form
, where  is the direction
of  and  is an integer ranging from  to .  It remains to
explain how to deal with one such direction~.

\begin{lemma}
  \label{lem:apx-one-direction}
  One can compute a canonical V-shape  covering  with one arm in
  given direction  and width at most  times the
  minimum width of any such V-shape, in time .
\end{lemma}

\begin{proof}
  We use an approach similar to that of the inner-outer case of our
  exact algorithm with a slight twist.  


  Let  be a line in direction~ supporting~.  We
  again let  be the furthest point from  in 
  and let .  When  is fixed, the minimum-width
  V-shape is determined by the minimum-width strip  covering 
  and not ``splitting'' , i.e., such that it does not have points of
   on both sides of it.  It is easy to ensure that  does not
  split  by observing that a direction of  lying between the
  directions of the common outer tangents to  and  is
  never useful.  Depending on the side where the lines supporting
  these tangents cross, a minimal strip  covering  and lying
  in the range between them either crosses  (and therefore~) or
  completely covers  (and therefore~).  In the former case,
   and  do not form a legal V-shape covering  and in the
  latter they form a covering V-shape with one empty strip, which 
never yields minimum width by reasoning as in Lemma~\ref{lem:no-empty-strip}. 


  The width of the resulting V-shape is the maximum of 
  and (the restricted) .  The algorithm proceeds by processing points 
  in order of decreasing distance to , keeping track of
   and a \emph{coreset} for , which is a subset of
   with the property that its directional width, in every
  direction, is at least  that of  (and, expanding the
  corresponding minimal strip containing the subset by
  a factor of , we get a strip covering ).  Chan~\cite{chan04}, in
  Theorem~3.7 and remarks in Section 3.4, describes a streaming
  algorithm that maintains an -size coreset at an
  amortized cost of  per insertion.
  For a fixed , we go through the coreset (after computing its
  convex hull, if necessary), and determine the narrowest strip
  covering it and satisfying our angle constraints.  The maximum of
  that and  gives the width of the minimum-width
  V-shape whose boundary passes through~.\footnote{More precisely,  lies on the boundary of  and may not even appear
    on the boundary of . However, as before, all V-shapes we
    examine are valid and cover~, and the desired approximating
    V-shape is among them, which is sufficient.}
  The amortized cost per
  point is dominated by the  cost of
  insertion.  Together with presorting points by distance from ,
  the total cost is then . \hfill 
\end{proof}

Combining Lemmas
\ref{lem:rotate-strip}~and~\ref{lem:apx-one-direction} yields the
procedure claimed in Lemma~\ref{lem:fixed-anchor-pair} and thereby
completes our description of the -approximation algorithm.



\section{Concluding remarks}\label{sec:conclusion}

As mentioned in the introduction, this work was inspired by research
on curve fitting, in the situations where a curve takes a sharp turn.
Besides the exact and approximate versions of the problem studied
above, it would be natural to investigate a variant that can handle 
a small number of outliers.  A natural ``peeling''
approach to the problem would be to eliminate the points defining the
optimal V-shape found by our exact algorithm and trying again.
However, it is easy to construct an example of a point set in which
removing a single point \emph{not} appearing on the boundary of the
minimum-width covering V-shape significantly reduces the width of the
optimum V-shape.


Are there natural assumptions (perhaps in the spirit of ``realistic
input models''~\cite{realistic-input-models} or in the form of
requiring reasonable sampling density) that would be relevant for the
curve-fitting problem, and that would make finding the minimum-width covering
V-shape easier?

Returning to the problem studied in the paper, is it possible to find
an exact minimum-width covering V-shape in subquadratic time?  Is the
problem \textsc{3sum}-hard?  

Is it possible to speed up the approximation algorithm, improving the
dependence of its running time on ?
Is time  achievable?

Finally, we would like to point out that there are other
``reasonable'' definitions for a V-shape, if the goal is to
approximate a sharp turn of a curve: One can imagine defining a
V-shape as the Minkowski sum of a disk with the union of two rays
emanating from a common point as in~\cite{GKS}.  
The width of the
V-shape would be the diameter of the disk.  Can the exact algorithm
from~\cite{GKS} be sped up?  Is there a faster approximation
algorithm?  Is this version of the problem better suited for curve
fitting?

\section*{Acknowledgments}

We thank Piyush Kumar for suggesting the problem studied in this
paper, Pankaj K. Agarwal for extremely useful pointers, Alfredo Hubard
for several helpful conversations, and Sariel Har-Peled for sharing
his wisdom and dispelling some confusion about approximation.



\begin{thebibliography}{99}
\bibitem{coreset-survey}
P. K. Agarwal, S. Har-Peled, and K. R. Varadarajan,
Geometric approximation via coresets,
in
\emph{Current Trends in Combinatorial and Computational Geometry},
(J.~E.~Goodman, J. Pach, and E. Welzl, eds.), MSRI Publications,
Volume 52, Cambridge University Press, New York, 2005, pp.~1--30.

\bibitem{2-line-center}
P. K. Agarwal, C. M. Procopiuc, and K. R. Varadarajan,
A (1 + )-approximation algorithm for 2-line-center,
\emph{Comput. Geom. Theory Appl.}, 26(2):119--128, 2003. 
 
\bibitem{random-opt-survey}
P. K. Agarwal and S. Sen,
Randomized algorithms for geometric optimization problems,
in
\emph{Handbook of Randomized Computation}, J. Pardalos,
S. Rajasekaran, J.~Reif, and J. Rolim, eds., Kluwer Academic Press,
The Netherlands, 2001, pp.~151--201.

\bibitem{alg-opt-survey}
P. K. Agarwal and M. Sharir,
Algorithmic techniques for geometric optimization,
in \emph{Computer Science Today: Recent Trends and Developments},
Lecture Notes in Computer Science, vol. 1000 (J. van Leeuwen, ed.),
Springer-Verlag, Berlin, 1995, pp.~234--253. 
	 
\bibitem{eff-alg-opt-survey}
P. K. Agarwal and M. Sharir,
Efficient algorithms for geometric optimization,
\emph{ACM Computing Surveys} 30:412--458, 1998.
	
\bibitem{alt-guibas-survey}
H. Alt and L. J. Guibas, 
Discrete geometric shapes: matching, interpolation, and approximation,
Chapter 3 in \emph{Handbook of Computational Geometry}, J.-R. Sack and
J. Urrutia, eds., 1999, pp.~121--153.

\bibitem{AD11} B. Aronov and M. Dulieu, 
How to cover a point set with a V-shape of minimum width,
\emph{Proc. Algorithms Data Stuctures Symp. (WADS'11)}, 2011, to appear.

 \bibitem{ComputationalGeometry}
 M. de Berg, O. Cheong, M. van Kreveld, and M. Overmars, 
 \emph{Computational Geometry: Algorithms and Applications},
 3rd ed., Springer-Verlag, 2008.

\bibitem{realistic-input-models}
M. de Berg, M. Katz, A.F. van der Stappen, and J. Vleugels, 
Realistic input models for geometric algorithms,
\emph{Algorithmica},
34(1):81--97, 2008.

\bibitem{chan-apx-all}
T. M. Chan,
Approximating the diameter, width, smallest enclosing cylinder, and
minimum-width annulus, 
\emph{Inter. J. Comput. Geom. Appl.}, 12:67--85, 2002.

\bibitem{chan04}
T. M. Chan,
Faster core-set constructions and data-stream algorithms in fixed dimensions,
\emph{Comput. Geom. Theory Appl.}, 35:20--35, 2006.


\bibitem{curve-noisy}
S.-W. Cheng, S. Funke, M. Golin, P. Kumar, S.-H. Poon, and E. Ramos, 
Curve reconstruction from noisy samples,
\emph{Comput. Geom. Theory Appl.},
31(1--2):63--100, 2005.

\bibitem{curve-dey} 
T. K. Dey, \emph{Curve and Surface Reconstruction:
  Algorithms with Mathematical Analysis}, Cambridge Monographs on
Applied and Computational Mathematics, Vol.~23, Cambridge Univ. Press, 2007.

\bibitem{Bob}
J. R. Driscoll, N. Sarnak, D. D. Sleator, and R. E. Tarjan, 
Making data structures persistent,
\emph{J. Computer System Sci.},
38(1):86--124, 1989.


\bibitem{EOS86} H. Edelsbrunner, J. O'Rourke, and R. Seidel,
Constructing arrangements of lines and hyperplanes with applications,
\emph{SIAM J. Comput.}, 15:341-363, 1986.

\bibitem{Funke-Ramos}
S. Funke and E. A. Ramos,
Reconstructing a collection of curves with corners and endpoints,
\emph{Proc. 12th ACM-SIAM Symp. Discr. Algorithms (SODA~2001)}, 2001, pp.~344--353.

\bibitem{GKS} A. Glozman, K. Kedem, and G. Shpitalnik, 
Computing a double-ray center for a planar point set,
\emph{Inter. J. Comp. Geom. Appl.}, 2(9):103--123, 1999.

\bibitem{proj-clustering} A. Mhatre and P. Kumar,
Projective clustering and its application to surface reconstruction:
extended abstract, 
\emph{Proc. 22nd Annu. Symp. Comput. Geom.},
2006, pp.~477--478.




 \bibitem{Piyush-private}
 P. Kumar, private communication, 2010.

\bibitem {EmptyPolygons}
R. Pinchasi, R. Radoi\v{c}i\'c, and M. Sharir,
On empty convex polygons in a planar point set,
\emph{J. Combinat. Theory, Series~A},
113:385--419, 2006.

\bibitem{Preparata}
F. P. Preparata,
An optimal real time algorithm for planar convex hulls,
\emph{Comm. ACM},
22:402--405, 1979.
\end{thebibliography}




\end{document}
