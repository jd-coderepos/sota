\label{sec:pollang}

In this section we take a closer look at the anatomy of policy schemes and policy specification mechanisms,
with the intention of introducing a standard terminology for discussing and comparing them.
We discuss the different \emph{control levels} at which a scheme can operate, and how
these levels taken together, in combination with assignment of labels in a system,
form what we generally refer to as a policy. 
We further use this anatomy as an
overlay to shed new light on the ``dimensions of declassification'' introduced by
Sabelfeld and Sands \cite{Sabelfeld:Sands:JCS}, and resolve some perceived unclarities and
ambiguities among the dimensions. We categorise existing policy specification mechanisms in 
the literature accordingly.
Finally we derive a framework for reasoning about the kinds of \emph{invariants} a
mechanism can express in a generated policy scheme, as a starting point for comparing 
expressiveness between different mechanisms. 



\subsection{The anatomy of dynamic policies}
\label{sec:pollang:anatomy}

\newcommand{\orderings}{\ensuremath{\mathcal{F}}}
\newcommand{\dynamicpol}{\ensuremath{\delta}}
\newcommand{\metapol}{\ensuremath{\mu}}


A dynamic policy scheme can be understood, discussed and classified in terms of a \emph{hierarchy of control}, 
consisting of the following control levels:
\begin{itemize}
  \item Level 0 control -- \orderings{}, a \emph{set} of possible \emph{flow relations} between the information
        sources and sinks available in the system.
  \item Level 1 control -- \dynamicpol{}, a \emph{determining function} selecting which flow relation in \orderings{} 
        should be active. We refer to the arguments to this function as the \emph{discriminator}.
\item Level 2 control -- \metapol{}, a \emph{meta policy} controlling the way in which the current flow relation 
        may be changed.\end{itemize}
The control levels allow us to be explicit about what it is that makes a policy dynamic: 
the possibility to define a determining function \emph{and} have an input to this function
that changes during program execution.
Arguably, one could have a meta-meta policy which in turn controls the meta policy.
However, with no loss of generality we group all further abstractions in the 
\emph{meta policy} class, since the meta policy can also be taken to control itself.


As a concrete example, consider a scheme consisting of two security labels, \emph{Secret} and \emph{Public}.
Potentially the set \orderings{} could contain all four possible flow relations that involve these 
two security labels, but in this example it contains only two:
one in which \emph{Public} information may flow to \emph{Secret} but not vice-versa; 
and one in which information may also flow from \emph{Secret} to \emph{Public}.
The former flow relation is the default, and the latter is available when using
a special ``declassify'' operator.
The determining function \dynamicpol{} then decides whether information may be released from 
\emph{Secret} to \emph{Public} or not, based on the current value of its argument, . In
this example,  can range over boolean values, indicating whether ``declassify'' is used
or not. The meta-policy simply always allows transitions back and
forth between true and false.





Formally we characterise the determining function as:

Here  is the information used to determine which flow relation is currently active.
~could range over boolean values as in our simple example; it could be partial information from the current program state, as is the case in Paragon~\cite{Paragon} or the work by Chong and Myers~\cite{Chong:Myers:CCS04}.
~may also consist of other information such as lexical location in the source code (as is done by Boudol and Matos~\cite{Boudol:Matos:On}) or `asynchronous' information external to the program (e.g. the acts-for hierarchy among principals used by Hicks et al.~\cite{Hicks+:Dynamic}).

In turn, the meta policy controls the changes between flow relations caused by the determining function.
Strictly speaking, a meta policy aims to control the transitions between flow relations, and does not
require the determining function as a ``proxy''. For example, a meta policy could, again based on some 
(meta) information  such as program state, impose a constraint on the progress of flow relations:

Here the pair  indicates that transition from flow 
relation  to  is permitted. In this way, it would be easy to for example define a \metapol{}
specifying the meta policy that the flow relation between information only becomes more liberal.
However in the surveyed literature, we find that the meta policy is typically defined in terms of
controlling how the information  used by \dynamicpol{} may change, rather than the resulting
flow relation determined by \dynamicpol{} -- simply an extra level of indirection.
That is, a meta policy instead typically has the characterisation:

Note that the determining function does not solely serve as a level of indirection for the meta policy:
the meta policy specifies how the current flow relation \emph{may} be changed, but it is the determining function that specifies what the current flow relation \emph{is}.


To demonstrate that the control hierarchy functions as a terminology for policy specification mechanisms, 
we show how various mechanisms from literature fit onto these levels:
\begin{itemize}

  \item The programming language Jif~\cite{Myers:POPL99} has a declassification function which can be
        described as temporarily changing , making the intent clear such that \dynamicpol{} provides
        a flow relation which allows for the declassification.
        The relation between declassification and dynamic policies is discussed in more detail in 
        Section~\ref{sec:declassification}.
        The decision to declassify is restricted by components such as authority and declassification
        robustness~\cite{Zdancewic:Myers:CSFW01}, which constitute . Both \dynamicpol{} and \metapol{}
        are pre-determined.
        The flow relations in \orderings{} are determined by the labels used, in combination with the
        \emph{acts-for} hierarchy. If changes to the acts-for hierarchy are allowed, 
        as per Hicks et al \cite{Hicks+:Dynamic}, then this hierarchy is also included in .

  \item The programming language Paragon~\cite{Paragon} allows for the specification of Paralocks policies (see Section~\ref{sec:semantics:survey}) in a Java-like language.
        Hence the current flow relation is determined by the lock state, 
        which constitutes~. In turn, these locks are information in the program state and are protected by
        locks themselves\footnote{We observe that the possibility to place locks on policies is part of the 
        specification mechanism used in Paragon, but not included in the Paralocks specification language.}, 
        making that second set of locks determine the meta-policy, i.e. when the first set of locks can be 
        opened and closed. The labels include specification of how they are interpreted with respect to
        changes in the lock state, meaning that the set of flow relations, and the behaviour of \dynamicpol{}
        (as well as \metapol{}) can be customised for each specific policy scheme.

  \item The programming language \Rx{}~\cite{Swamy+:Managing} incorporates the \textit{RT} 
        framework~\cite{Li:Mitchell:Winsborough:Design} to specify a flow relation among \emph{roles}.
        The active flow relation is determined by the set of memberships and delegations specified on
        each role, which form~.
        Roles also carry labels which specify who can observe the current members of a role, hence
        forming the set  determining on which secret data the decision to change flow relation
        (add or remove memberships and delegations) may depend.
  
  \item Matos and Cederquist~\cite{Matos2014} present a security condition for distributed computations 
        (Distributed Non-interference). The default flow relation between security labels can be relaxed
        using the lexical construct from earlier work on the non-disclosure property~\cite{Boudol:Matos:On},
        making  the locality in the code.
        These scopes with more liberal flow relations may only be entered when the node on which the 
        computation runs allows for it. Each node specifies its own regulation on the allowed added flows,
        making  the locality in the network.
  
\end{itemize}

We note that the apparently clear separation offered by the levels of control is not necessarily mirrored 
in actual specification mechanisms. As noted, in Paragon each data source and sink is annotated with a
security label that specifies not only a static behaviour, but also how that label should be interpreted
in different lock states. In other words, the information on what the flow relations are and how they are
determined is distributed across all the labels in a program, not cleanly as a single determining function.
This does not mean that Paragon could not still be understood and described in terms of the levels of control.


Another observation is that the higher levels of control, \dynamicpol{} and \metapol{}, have two components
where control can be exercised: in the definition of \dynamicpol{}, resp. \metapol{}, and in the argument
to the respective function. To contrast these two possibilities, \Rx{} allows control over the argument
provided to \metapol{} (who can observe the members of a role), but \metapol{} itself is fixed.
This is opposed to the meta-policy by Matos and Cederquist where the argument to \metapol{} is fixed to
be the node on which the computation runs, but \metapol{} itself can be defined by the policy designer.

\subsection{Rethinking the Dimensions of Declassification}
\label{sec:pollang:reclassify}

Looking at the literature, it is clear that the four ``dimensions of declassification'' introduced
by Sabelfeld and Sands \cite{Sabelfeld:Sands:JCS} -- ``what'', ``who'' ``where'' and ``when'' --
are significantly different from each other in nature.
In particular aspects of the ``what'' dimension are largely orthogonal from the other three,
while many uses of ``where'' and ``when'' often coincide. There are also
different aspects grouped within the same dimension that are so disparate as to be incomparable.
For example, the ``where'' dimension intends to cover both \emph{code locality} and \emph{level locality},
which are only remotely related at best.

We propose that these dimensions should be discussed for each of our identified dynamic policy levels 
\emph{individually}. We identify that not every dimension is relevant for every policy level, as 
summarised in Table~\ref{tbl:decldimensions}.
This insight resolves some of the confusion in the declassification dimensions,
showing that by taking the anatomy of a policy into account while discussing its 
``declassification' dimensions, we arrive at a clearer framework for discussing and comparing 
security conditions.

We present a short summary of each of the declassification dimensions and discuss them with respect to the policy anatomy.

\begin{table}
\begin{center}
  \begin{tabular}{@{}|l|*{4}{c}|@{}}
    \cline{2-5}
    \multicolumn{1}{ c|  }{} & What & Who & Where & When \\ 
    \hline
    \orderings{} & + & - &  Level locality only & - \\
    \dynamicpol{} & - & + & + & + \\
    \metapol{} & - & + & + & + \\ \hline
  \end{tabular}
\end{center}
\vspace*{0.7ex}
\caption{Revisiting the declassification dimensions; + indicates that a dimension concern can be addressed by that policy component, - that it cannot.}
\label{tbl:decldimensions}
\end{table}

\para{What} -- 
Policies can dictate that only parts of a secret may be released (e.g. the last digits of a credit card). 
In addition this dimension covers quantitative release which is better characterised by ``how much'' and can be achieved using an information-theoretic approach (e.g. \cite{Clark+:ENTCS}).
Although the decision to e.g. increase the amount of information that may be released comes from different components, the possibility to \emph{express} this dimension only exists naturally in the ordering between information itself, i.e.
when specifying flow relations.


\para{Who} -- 
This dimension is concerned with being able to express who controls the release.
In particular, it is sensible to prevent an attacker from abusing the release mechanism, 
as is the motivation for robust declassification~\cite{Zdancewic:Myers:CSFW01}.
Since the decision to declassify can be controlled both by the determining function 
and the meta policy, this dimension can be addressed on either level.
By nature this dimension talks about control over flow relations, and therefore 
is not relevant on the level of the flow relation itself.


\para{Where} --
This dimension is split into two different forms of locality: \emph{level} locality and \emph{code} locality.

Level locality addresses the concern where information may flow relative to the security levels of the system.
This dimension is particularly present in \emph{intransitive non-interference}~\cite{Rushby:92}, which is exemplified by a policy which allows information to flow from security level {\it Secret} to {\it Declassify} and from {\it Declassify} to {\it Public}, but not directly from {\it Secret} to {\it Public}.
This can be expressed in a flow relation using downgrading relations (Mantel~\cite{Mantel:FME01}),
but could also be addressed by the \dynamicpol{} and \metapol{} controls on the ordering.
The latter is achieved by changing between flow relations such that only one of the two flows is
permitted at any specific time (this essentially requires \emph{time-transitive flows},
see Section~\ref{sec:semantics:dimensions}).

Code locality allows policies to describe where syntactically in the code information may be released.
One could sort of view this as level locality except the information should not pass through the 
{\it Declassify} level but through a lexical declassification construct in the program's code.
Similar for the Who dimension, code locality is concerned with controlling which flow relation is active, 
and can therefore only be addressed by \dynamicpol{} or \metapol{}.

\para{When} --
A policy can dictate that information may only be released after (or before) a certain time has passed.
This temporal restriction can be based on various elements, such as real time, the size of the secret or relative to other events in the system.
Although the original presentation of this dimension splits it into various classes, all temporal controls need to be addressed by \dynamicpol{} and \metapol{} as they concern the decision to change the ordering of information.



When we now reclassify policy mechanisms by the levels first and the dimensions second, 
the classification becomes clear and unambiguous.
We briefly show this for the examples used in Section~\ref{sec:pollang:anatomy}.
\footnote{None of the considered examples addresses the ``what'' dimension, or 
support intransitive flows in the flow relations, thus we do not discuss this dimension further.}

For Jif, the discriminator for the determining function \dynamicpol{} is given by a combination
of the declassification construct, which concerns the ``where'' dimension (both code and level locality), and
the acts-for hierarchy, which concerns the ``who'' dimension.
As a meta-policy, authority provides a meta control on the decision to
declassify in the ``who'' dimension.
Robustness does so as well, and in addition partially addresses the ``what'' dimension by limiting what information can be declassified.
For Paragon, both \dynamicpol{} and \metapol{} are regulated by locks which concern the ``when'' 
dimension: information flows are allowed relative to the actions of opening and closing locks.
Paragon also has a lexically scoped version of opening a lock, which works in the ``where'' dimension
(code locality).
Implementations in Jif and Paragon can combine the programming and policy language to encode
requirements in other dimensions~\cite{Askarov:Sabelfeld:ESORICS2005,Paragon,paragontut},
but these are not a natural part of the policy language.

Both \dynamicpol{} and \metapol{} in \Rx{} use the ``who'' dimension: who is a member of a role
determines the active flow relation, and who can view the current members of a role decides what
policy change can be made.
The security framework for distributed non-interference by Matos and Cederquist finds a fit in
the ``where'' dimension for \dynamicpol{} as the lexical flow construct determines where in the code the additional
flows are allowed. The meta policy also fits in that dimension, as it determines where in the network each flow construct is allowed.

\subsection{The expressiveness of policy languages}
\label{sec:pollang:expressivity}

Different policy specification mechanisms offer a variety of expressiveness, from the
simplest fixed two-level systems, up to full policy specification \emph{languages}
like those found in Jif \cite{jif}, \Rx{}~\cite{Swamy+:Managing} or Paragon \cite{Paragon}.
We can have intuitive ideas regarding the relative expressiveness of such mechanisms,
but what are the measures by which we can compare them formally? In this section we speculate
on how a formal framework for comparison could be constructed.



Montagu et al \cite{Montagu13} construct a framework for comparing ``label models'',
or \emph{policy schemes} in our terminology. We argue that such an approach is too simple
for comparing expressiveness of full policy specification languages.

Consider a policy scheme which has three labels called ,  and .
      By default, the flow relation consists of the three flows 
      , 
      and , so the labels form a strict hierarchy
      of security levels. The policy scheme also allows data to be declassified from 
      to . When declassifying, the flow relation is then the same three flows from before,
      with  added. These are the only two flow relations
      possible. We refer to this scheme as TSP.

A second scheme has three labels simply called ,  and . Any of the six possible
    flows between two labels can be allowed or not independently of other flows. In other words,
    all  conceivable flow relations involving these three labels are possible, and a
    programmer can freely change between them. We refer to this scheme as ABC.

A first attempt at comparing expressiveness could look at the possibility to \emph{embed} one
\emph{policy scheme} in the other\footnote{This is the comparison done by Montagu et al \cite{Montagu13}.}.
That is, the embedding scheme should contain at least the same set of flow relations as the embedded scheme.
In this case we could embed TSP in ABC using
,  and , and use the corresponding
two matching flow relations. We could then claim that the second is at least as
expressive as the first, by virtue of having at least as many labels allowing at least
the same flow relations. However, such an attempt misses an important aspect of expressiveness.
If we were to use ABC in place of TSP, what (other than regimen) stops us from making
one of the ``other'' flow relations active? In particular, we have no guarantees that we will
not use a flow relation in which , proxying as , can flow to other labels.
ABC is certainly more \emph{flexible} than TSP -- but when embedding, added
flexibility is not a good thing. Restrictions matter!

For a policy scheme, the degree of flexibility is already fixed, so there is never any room
for expressiveness. Truly, expressiveness should be compared at the level of policy specification
mechanisms. Consider mechanisms  and : we have that  is at least as expressive as 
 if, for every possible policy scheme that  can generate,  can generate a scheme that
can embed it -- \emph{including restrictions}. But how can we express restrictions formally?

The examples used so far show the need for restrictions at the level of what flow relations are
possible. However, not all such restrictions are equally important. For our example above, the
fact that when using ABC we could end up in contexts where the flow relation allows \emph{fewer}
flows than any of the ones matching those of TSP, is arguably acceptable -- the system would
still be secure. But the fact that we could end up in a context where , representing ,
can flow at all is not acceptable, as it means that using ABC we cannot give the same security
guarantees \emph{by construction}. Hence, what matters is the ability to express
\emph{invariants} over flow relations -- specifically, invariants that concern the \emph{absence}
of some (set of) flows.

In this work we identify two principal forms of invariants that we want the ability to express.
The first form are the \emph{invariants over sets of flow relations}. Such invariants can be
global, for example ``no flow from  to any other level is ever allowed'';
or \emph{conditional}, for example ``flows from  to  are not
allowed, except when declassifying''. The second principal form are the
\emph{invariants over sequences of flow relations}. A simple example could be the
Gradual Release property \cite{Askarov:Sabelfeld:Gradual} that the policy may only
change to become more liberal over time. Another more complicated example is a strong
Chinese Wall property stating that if a flow  has ever
been allowed at any point, then  may not be allowed at
any point in the future~\cite{brewer1989chinese}. 

To formalise these notions we first observe that given some starting state  and the set of possible
transitions as given by the range of \footnote{If we also know how the argument  to  may change
over time, we can have better precision than considering the whole range of . As argued, this could
be accomplished using yet another level of meta-policy that governs how  may change, or baking this into
 itself.}, we can enumerate all possible sequences
of discriminators by iteratively applying all possible transitions. Let 
be the set of all such sequences. A global invariant over sets of reachable flow relations is then a property 
such that 
 holds. A \emph{conditional} invariant adds a filter  such that

holds.  An invariant over \emph{sequences} of flow relations is a property  such that
 holds. We can easily imagine a conditional version of invariants over sequences too, with a similar filter
based on the domain of , however we have not identified any compelling examples.

\begin{framed}
\centerline{\bf Comparing expressiveness of policy languages}
\noindent
A policy specification mechanism  is at least as restrictive
as another mechanism , if for every possible policy scheme that  can generate,
 can generate a scheme that can embed it, including enforcing the same (negative)
invariants.
\end{framed}


The kinds of invariants we have categorised here capture the majority of all conceivable invariants
that we may want a policy scheme to enforce, however, there are more complex invariants that
cannot be expressed in these terms. Our invariants are essentially \emph{safety properties}, 
and not all desired invariants can be expressed in terms of these. 
Our framework of invariants should thus be seen as a starting point for
formalising comparisons between policy specification mechanisms, not a completed journey,
and the formalisation of further points in the space of invariants is an open research question.

























