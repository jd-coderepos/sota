\documentclass[letterpaper]{article}
\usepackage{aaai18}
\usepackage{times}
\usepackage{helvet}
\usepackage{courier}
\usepackage{url}
\usepackage{graphicx}


\usepackage{amsmath}
\usepackage{mathrsfs}
\usepackage{amsfonts}
\usepackage{amsthm}
\usepackage{stmaryrd}
\usepackage{xcolor}
\usepackage{colortbl}
\usepackage{cases}
\usepackage{subcaption}
\usepackage{multirow}
\usepackage{chngcntr}
\usepackage{enumitem}
\usepackage{mathtools}
\usepackage{booktabs}
\usepackage{lipsum} 

\usepackage{trackchanges}

\usepackage{algorithm}
\usepackage{algorithmic}

\frenchspacing  

\newcommand{\theHalgorithm}{\arabic{algorithm}}

\newcommand{\pnorm}[1]{\| #1 \|_p} 
\newcommand{\panorm}[1]{| #1 |^p} 
\newcommand{\dP}{\textrm{d} P}


\newcommand{\cX}{\mathcal{X}}
\newcommand{\cS}{\mathcal{S}}
\newcommand{\cM}{\mathcal{M}}
\newcommand{\cF}{\mathcal{F}}
\newcommand{\cA}{\mathcal{A}}
\newcommand{\cE}{\mathcal{E}}
\newcommand{\cB}{\mathcal{B}}
\newcommand{\cZ}{\mathcal{Z}}
\newcommand{\cP}{\mathcal{P}}
\newcommand{\cR}{\mathcal{R}}
\newcommand{\cT}{\mathcal{T}}
\newcommand{\cG}{\mathcal{G}}
\newcommand{\cQ}{\mathcal{Q}}
\newcommand{\cL}{\mathcal{L}}

\newcommand{\bR}{\mathbb{R}}
\newcommand{\bZ}{\mathbb{Z}}
\newcommand{\bN}{\mathbb{N}}
\newcommand{\bT}{\mathbb{T}}
\newcommand{\bG}{\mathbb{G}}
\newcommand{\bF}{\mathbb{F}}

\newcommand{\bx}{\mathbf{x}}

\newcommand{\lpar}{\left (}
\newcommand{\rpar}{\right )}
\newcommand{\lbar}{\left \vert}
\newcommand{\rbar}{\right \vert}

\newcommand{\zerovec}{\mathbf{0}}
\newcommand{\intset}[1]{[#1]}

\newcommand{\indic}[1]{\mathbb{I}\left [ #1 \right ]}


\newcommand{\dotp}[2]{#1 \cdot #2}

\DeclareMathOperator*{\expect}{{\huge \mathbb{E}}}
\newcommand{\expects}{\expect\nolimits}
\newcommand{\bigexpect}[1]{\bE \left [ #1 \right ]}
\newcommand{\Var}{\mathbb{V}}

\newcommand{\norm}[1]{\left \| #1 \right \|}
\newcommand{\bignorm}[1]{\big \| #1 \big \|}
\newcommand{\infnorm}[1]{\left \| #1 \right \|_{\infty}}

\newtheorem{defn}{Definition}
\newtheorem{prop}{Proposition}
\newtheorem{lem}{Lemma}
\newtheorem{rem}{Remark}
\newtheorem{assum}{Assumption}
\newtheorem{cor}{Corollary}
\newtheorem{thm}{Theorem}
\newtheorem{example}{Example}
\newtheorem{notation}{Notation}
\newtheorem{obs}{Observation}

\newcommand{\etal}{\textit{et al.}}

\newcommand{\xn}{x_{1:n}}
\newcommand{\xln}{x_{<n}}
\newcommand{\xt}{x_{1:t}}
\newcommand{\at}{a_{1:t}}

\newcommand{\logl}{\mathcal{L}}

\newcommand{\floor}[1]{\left \lfloor {#1} \right \rfloor}
\newcommand{\ceil}[1]{\left \lceil {#1} \right \rceil}

\newcommand{\cbar}{\, | \,}
\newcommand{\cdbar}{\, \| \,}
\newcommand{\csemi}{\, ; \,}

\newcommand{\KL}{\textrm{KL}}
\newcommand{\TV}{\textrm{TV}}
\newcommand{\K}{\textrm{K}}
\def \grad {\nabla}

\DeclareMathOperator*{\argmin}{arg\,min}
\DeclareMathOperator*{\argmax}{arg\,max}

\newcommand{\eqnref}[1]{(\ref{eqn:#1})}


\def \cfo {\textsc{c51}}
\def \dqn {\textsc{dqn}}
\def \td {\textsc{td}()}
\def \qrtd {\textsc{qrtd}}
\def \qrdqn {\textsc{qr-dqn}}

\def \Rmax {R_{\textsc{max}}}
\def \Ddef {\overset{D}{:=}}
\def \Deq {\overset{D}{=}}

\def \cTpi {\cT^\pi}
\def \ditwo {d_{\infty,2}}
\def \dip {\bar d_p}
\def \Vrange {B}
\def \qcZ {\cZ_{Q}}

\def \Vmax {V_{\textsc{max}}}
\def \Vmin {V_{\textsc{min}}}

\def \bu {\mathbf{u}}

\newcommand{\gamename}[1]{\textsc{#1}}

\usepackage{verbatim}
\usepackage{cancel}
\usepackage{thmtools,thm-restate}

\newcommand{\citet}[1]{\citeauthor{#1} (\citeyear{#1})}

\setlength{\pdfpagewidth}{8.5in}  \setlength{\pdfpageheight}{11in}  \pdfinfo{
/Title (Distributional Reinforcement Learning with Quantile Regression)
/Author (Will Dabney, Mark Rowland, Marc G. Bellemare, Remi Munos)
/Keywords (Reinforcement learning, Temporal difference learning, distributional RL)
}
\setcounter{secnumdepth}{0}  
\begin{document}


\title{Distributional Reinforcement Learning with Quantile Regression}
\author{Will Dabney\\ DeepMind \And Mark Rowland\\ University of Cambridge\thanks{Contributed during an internship at DeepMind.} \And Marc G. Bellemare\\ Google Brain \And R\'emi Munos\\ DeepMind}

\nocopyright

\maketitle
\begin{abstract}
In reinforcement learning an agent interacts with the environment by taking actions and observing the next state and reward. When sampled probabilistically, these state transitions, rewards, and actions can all induce randomness in the observed long-term return. Traditionally, reinforcement learning algorithms average over this randomness to estimate the value function. In this paper, we build on recent work advocating a distributional approach to reinforcement learning in which the distribution over returns is modeled explicitly instead of only estimating the mean. That is, we examine methods of learning the \textit{value distribution} instead of the value function. We give results that close a number of gaps between the theoretical and algorithmic results given by \citet{c51}. First, we extend existing results to the approximate distribution setting. Second, we present a novel distributional reinforcement learning algorithm consistent with our theoretical formulation. Finally, we evaluate this new algorithm on the Atari 2600 games, observing that it significantly outperforms many of the recent improvements on , including the related distributional algorithm .
\end{abstract}

\section{Introduction}

In reinforcement learning, the \emph{value} of an action  in state  describes the expected return, or discounted sum of rewards, obtained from beginning in that state, choosing action , and subsequently following a prescribed policy. Because knowing this value for the \emph{optimal policy} is sufficient to act optimally, it is the object modelled by classic value-based methods such as SARSA \cite{rummery94online} and Q-Learning \cite{watkins1992q}, which use Bellman's equation \cite{bellman57dynamic} to efficiently reason about value.

Recently, \citet{c51} showed that the distribution of the random returns, whose expectation constitutes the aforementioned value, can be described by the distributional analogue of Bellman's equation, echoing previous results in risk-sensitive reinforcement learning \cite{heger1994consideration,morimura10parametric,chow2015risk}. In this previous work, however, the authors argued for the usefulness in modeling this \emph{value distribution} in and of itself. Their claim was asserted by exhibiting a distributional reinforcement learning algorithm, , which achieved state-of-the-art on the suite of benchmark Atari 2600 games \cite{bellemare13arcade}.

One of the theoretical contributions of the  work was a proof that the distributional Bellman operator is a contraction in a maximal form of the Wasserstein metric between probability distributions. In this context, the Wasserstein metric is particularly interesting because it does not suffer from disjoint-support issues \cite{wgan} which arise when performing Bellman updates. Unfortunately, this result does not directly lead to a practical algorithm: as noted by the authors, and further developed by \citet{bellemare17cramer}, the Wasserstein metric, viewed as a loss, cannot generally be minimized using stochastic gradient methods. 

This negative result left open the question as to whether it is possible to devise an online distributional reinforcement learning algorithm which takes advantage of the contraction result. Instead, the  algorithm first performs a heuristic projection step, followed by the minimization of a KL divergence between projected Bellman update and prediction. The work therefore leaves a theory-practice gap in our understanding of distributional reinforcement learning, which makes it difficult to explain the good performance of . Thus, the existence of a distributional algorithm that operates end-to-end on the Wasserstein metric remains an open question.

In this paper, we answer this question affirmatively. By appealing to the theory of quantile regression \cite{qrbook}, we show that there exists an algorithm, applicable in a stochastic approximation setting, which can perform distributional reinforcement learning over the Wasserstein metric. Our method relies on the following techniques:
\begin{itemize}
    \item We ``transpose'' the parametrization from : whereas the former uses  fixed locations for its approximation distribution and adjusts their probabilities, we assign fixed, uniform probabilities to  adjustable locations;
    \item We show that \emph{quantile regression} may be used to stochastically adjust the distributions' locations so as to minimize the Wasserstein distance to a target distribution.
    \item We formally prove contraction mapping results for our overall algorithm, and use these results to conclude that our method performs distributional RL end-to-end under the Wasserstein metric, as desired.
\end{itemize}

The main interest of the original distributional algorithm was its state-of-the-art performance, despite still acting by maximizing expectations. One might naturally expect that a direct minimization of the Wasserstein metric, rather than its heuristic approximation, may yield even better results. We derive the Q-Learning analogue for our method (), apply it to the same suite of Atari 2600 games, and find that it achieves even better performance. By using a smoothed version of quantile regression, \emph{Huber quantile regression}, we gain an impressive  median score increment over the already state-of-the-art .

\section{Distributional RL}

We model the agent-environment interactions by a Markov decision process (MDP) () \cite{puterman94markov}, with  and  the state and action spaces,  the random variable reward function,  the probability of transitioning from state  to state  after taking action , and  the discount factor. A policy  maps each state  to a distribution over .

For a fixed policy , the \textit{return}, , is a random variable representing the sum of discounted rewards observed along one trajectory of states while following . Standard RL algorithms estimate the expected value of , the \textit{value function}, 

Similarly, many RL algorithms estimate the action-value function,


The -greedy policy on  chooses actions uniformly at random with probability  and otherwise according to . 

In distributional RL the distribution over returns (i.e. the probability law of ), plays the central role and replaces the value function. We will refer to the value distribution by its random variable. When we say that the value function is the mean of the value distribution we are saying that the value function is the expected value, taken over all sources of intrinsic randomness \cite{goldstein1981intrinsic}, of the value distribution. This should highlight that the value distribution is not designed to capture the uncertainty in the estimate of the value function \cite{dearden98bayesian,engel05reinforcement}, that is the \textit{parametric uncertainty}, but rather the randomness in the returns intrinsic to the MDP.

Temporal difference (TD) methods significantly speed up the learning process by incrementally improving an estimate of  using dynamic programming through the \textit{Bellman operator} \cite{bellman57dynamic},

Similarly, the value distribution can be computed through dynamic programming using a \textit{distributional Bellman operator} \cite{c51},

where  denotes equality of probability laws, that is the random variable  is distributed according to the same law as .

\begin{figure}[t]
\begin{center}
\includegraphics[width=.4\textwidth]{c51_projection.pdf}
\end{center}
\caption{Projection used by  assigns mass inversely proportional to distance from nearest support. Update minimizes KL between projected target and estimate.\label{fig:c51proj}}
\end{figure}

The  algorithm models  using a discrete distribution supported on a ``comb'' of fixed locations  uniformly spaced over a predetermined interval. The parameters of that distribution are the probabilities , expressed as logits, associated with each location . Given a current value distribution, the  algorithm applies a projection step  to map the target  onto its finite element support, followed by a Kullback-Leibler (KL) minimization step (see Figure~\ref{fig:c51proj}).  achieved state-of-the-art performance on Atari 2600 games, but did so with a clear disconnect with the theoretical results of \citet{c51}. We now review these results before extending them to the case of approximate distributions.

\subsection{The Wasserstein Metric}

The -Wasserstein metric , for , also known as the Mallows metric \cite{bickel81asymptotic} or the Earth Mover's Distance (EMD) when  \cite{levina2001earth}, is an integral probability metric between distributions.
The -Wasserstein distance is characterized as the  metric on inverse cumulative distribution functions (inverse CDFs) \cite{muller1997integral}. That is, the -Wasserstein metric between distributions  and  is given by,\footnote{For , .}

where for a random variable , the inverse CDF  of  is defined by

where  is the CDF of . Figure~\ref{fig:wmin} illustrates the 1-Wasserstein distance as the area between two CDFs.

Recently, the Wasserstein metric has been the focus of increased research due to its appealing properties of respecting the underlying metric distances between outcomes \cite{wgan,bellemare17cramer}. 
Unlike the Kullback-Leibler divergence, the Wasserstein metric is a true probability metric and considers both the probability of and the distance between various outcome events. These properties make the Wasserstein well-suited to domains where an underlying similarity in outcome is more important than exactly matching likelihoods.

\subsection{Convergence of Distributional Bellman Operator}

In the context of distributional RL, let  be the space of action-value distributions with finite  moments:

Then, for two action-value distributions , we will use the maximal form of the Wasserstein metric introduced by \cite{c51},

It was shown that  is a metric over value distributions. Furthermore, the distributional Bellman operator  is a contraction in , a result that we now recall.
\begin{lem}[Lemma 3, \citeauthor{c51} \citeyear{c51}]\label{lem:wasserstein_contraction_operator}
 is a -contraction: for any two ,

\end{lem}
Lemma \ref{lem:wasserstein_contraction_operator} tells us that  is a useful metric for studying the behaviour of distributional reinforcement learning algorithms, in particular to show their convergence to the fixed point . Moreover, the lemma suggests that an effective way in practice to learn value distributions is to attempt to minimize the Wasserstein distance between a distribution  and its Bellman update , analogous to the way that TD-learning attempts to iteratively minimize the  distance between  and . 

Unfortunately, another result shows that we cannot in general minimize the Wasserstein metric (viewed as a loss) using stochastic gradient descent.
\begin{thm}[Theorem 1, \citeauthor{bellemare17cramer} \citeyear{bellemare17cramer}]\label{thm:biased_gradients}
Let  be the empirical distribution derived from samples  drawn from a Bernoulli distribution . Let  be a Bernoulli distribution parametrized by , the probability of the variable taking the value . Then the minimum of the expected sample loss is in general different from the minimum of the true Wasserstein loss; that is,

\end{thm}
This issue becomes salient in a practical context, where the value distribution must be approximated. Crucially, the  algorithm is not guaranteed to minimize any -Wasserstein metric. This gap between theory and practice in distributional RL is not restricted to . \citet{morimura10parametric} parameterize a value distribution with the mean and scale of a Gaussian or Laplace distribution, and minimize the KL divergence between the target  and the prediction . They demonstrate that value distributions learned in this way are sufficient to perform risk-sensitive Q-Learning. However, any theoretical guarantees derived from their method can only be asymptotic; the Bellman operator is at best a non-expansion in KL divergence.

\begin{figure}[t]
\begin{center}
\includegraphics[width=.48\textwidth]{w1_min.pdf}
\end{center}
\caption{1-Wasserstein minimizing projection onto  uniformly weighted Diracs. Shaded regions sum to form the 1-Wasserstein error.\label{fig:wmin}}
\end{figure}


\section{Approximately Minimizing Wasserstein}
Recall that  approximates the distribution at each state by attaching variable (parametrized) probabilities  to fixed locations . Our approach is to ``transpose'' this parametrization by considering fixed probabilities but variable locations. 
Specifically, we take uniform weights, so that  for each .

Effectively, our new approximation aims to estimate \emph{quantiles} of the target distribution. Accordingly, we will call it a \emph{quantile distribution}, and let  be the space of quantile distributions for fixed . We will denote the cumulative probabilities associated with such a distribution (that is, the discrete values taken on by the CDF) by , so that  for . We will also write  to simplify notation.

Formally, let  be some parametric model. A quantile distribution  maps each state-action pair  to a uniform probability distribution supported on . That is, 

where  denotes a Dirac at .

Compared to the original parametrization, the benefits of a parameterized quantile distribution are threefold. First, (1) we are not restricted to prespecified bounds on the support, or a uniform resolution, potentially leading to significantly more accurate predictions when the range of returns vary greatly across states. This also (2) lets us do away with the unwieldy projection step present in , as there are no issues of disjoint supports. Together, these obviate the need for domain knowledge about the bounds of the return distribution when applying the algorithm to new tasks. Finally, (3) this reparametrization allows us to minimize the Wasserstein loss, without suffering from biased gradients, specifically, using \emph{quantile regression}.


\subsection{The Quantile Approximation}

It is well-known that in reinforcement learning, the use of function approximation may result in instabilities in the learning process \cite{tsitsiklis97analysis}. Specifically, the Bellman update projected onto the approximation space may no longer be a contraction. In our case, we analyze the distributional Bellman update, projected onto a parameterized quantile distribution, and prove that the combined operator is a contraction.

\subsubsection{Quantile Projection}

We are interested in quantifying the projection of an arbitrary value distribution  onto , that is


Let  be a distribution with bounded first moment and  a uniform distribution over  Diracs as in \eqnref{definition_quantile_distribution}, with support . Then

\begin{restatable}{lem}{wonemidpoint}\label{w1_midpoint}
For any  with  and cumulative distribution function  with inverse , the set of  minimizing

is given by

In particular, if  is the inverse CDF, then  is always a valid minimizer, and if  is continuous at , then  is the unique minimizer.
\end{restatable}

These \textit{quantile midpoints} will be denoted by  for .
Therefore, by Lemma~\ref{w1_midpoint}, the values for  that minimize  are given by . Figure~\ref{fig:wmin} shows an example of the quantile projection  minimizing the -Wasserstein distance to .\footnote{We save proofs for the appendix due to space limitations.}

\subsection{Quantile Regression}

The original proof of Theorem \ref{thm:biased_gradients} only states the \emph{existence} of a distribution whose gradients are biased. As a result, we might hope that our quantile parametrization leads to unbiased gradients. Unfortunately, this is not true.
\begin{restatable}{prop}{biasedgradients}\label{prop:biased_transpose_gradients}
Let  be a quantile distribution, and  the empirical distribution composed of  samples from . Then for all , there exists a  such that

\end{restatable}

However, there is a method, more widely used in economics than machine learning, for unbiased stochastic approximation of the quantile function. \textit{Quantile regression}, and \textit{conditional quantile regression}, are methods for approximating the quantile functions of distributions and conditional distributions respectively \cite{qrbook}. These methods have been used in a variety of settings where outcomes have intrinsic randomness \cite{koenker2001quantile}; from food expenditure as a function of household income \cite{engel1857productions}, to studying value-at-risk in economic models \cite{taylor1999quantile}.

The quantile regression loss, for quantile , is an asymmetric convex loss function that penalizes overestimation errors with weight  and underestimation errors with weight . For a distribution , and a given quantile , the value of the quantile function  may be characterized as the minimizer of the \emph{quantile regression loss}

More generally, by Lemma \ref{w1_midpoint} we have that the minimizing values of  for  are those that minimize the following objective:


In particular, this loss gives unbiased sample gradients. As a result, we can find the minimizing  by stochastic gradient descent. 

\subsubsection{Quantile Huber Loss}
The quantile regression loss is not smooth at zero; as , the gradient of Equation~\ref{eqn:qr_loss} stays constant. We hypothesized that this could limit performance when using non-linear function approximation. To this end, we also consider a modified quantile loss, called the \textit{quantile Huber loss}.\footnote{Our quantile Huber loss is related to, but distinct from that of \citet{aravkin2014sparse}.} This quantile regression loss acts as an asymmetric squared loss in an interval  around zero and reverts to a standard quantile loss outside this interval.

The Huber loss is given by \cite{huber1964robust},

The quantile Huber loss is then simply the asymmetric variant of the Huber loss,

For notational simplicity we will denote , that is, it will revert to the standard quantile regression loss.

\subsection{Combining Projection and Bellman Update}

We are now in a position to prove our main result, which states that the combination of the projection implied by quantile regression with the Bellman operator is a contraction. The result is in -Wasserstein metric, i.e. the size of the largest gap between the two CDFs.
\begin{restatable}{prop}{Winftycontract}
Let  be the quantile projection defined as above, and when applied to value distributions gives the projection for each state-value distribution. For any two value distributions  for an MDP with countable state and action spaces,

\end{restatable}
We therefore conclude that the combined operator  has a unique fixed point , and the repeated application of this operator, or its stochastic approximation, converges to . Because , we conclude that convergence occurs for all . Interestingly, the contraction property does not directly hold for ; see Lemma \ref{lem:dpnocontraction} in the appendix.

\section{Distributional RL using Quantile Regression}
We can now form a complete algorithmic approach to distributional RL consistent with our theoretical results. That is, approximating the value distribution with a parameterized quantile distribution over the set of quantile midpoints, defined by Lemma~\ref{w1_midpoint}. Then, training the location parameters using quantile regression (Equation~\ref{eqn:qr_loss}).

\subsection{Quantile Regression Temporal Difference Learning}

Recall the standard TD update for evaluating a policy ,

TD allows us to update the estimated value function with a single unbiased sample following . Quantile regression also allows us to improve the estimate of the quantile function for some target distribution, , by observing samples  and minimizing Equation~\ref{eqn:qr_loss}.

Furthermore, we have shown that by estimating the quantile function for well-chosen values of  we can obtain an approximation with minimal 1-Wasserstein distance from the original (Lemma~\ref{w1_midpoint}). Finally, we can combine this with the distributional Bellman operator to give a target distribution for quantile regression. This gives us the quantile regression temporal difference learning () algorithm, summarized simply by the update,

where  is a quantile distribution as in \eqnref{definition_quantile_distribution}, and  is the estimated value of  in state . It is important to note that this update is for each value of  and is defined for a single sample from the next state value distribution. In general it is better to draw many samples of  and minimize the expected update. A natural approach in this case, which we use in practice, is to compute the update for all pairs of (). Next, we turn to a control algorithm and the use of non-linear function approximation.

\subsection{Quantile Regression }

Q-Learning is an off-policy reinforcement learning algorithm built around directly learning the optimal action-value function using the Bellman optimality operator \cite{watkins1992q},


The distributional variant of this is to estimate a state-action value distribution and apply a distributional Bellman optimality operator,

Notice in particular that the action used for the next state is the greedy action with respect to the mean of the next state-action value distribution.

For a concrete algorithm we will build on the  architecture \cite{mnih15nature}. We focus on the minimal changes necessary to form a distributional version of . Specifically, we require three modifications to . First, we use a nearly identical neural network architecture as , only changing the output layer to be of size , where  is a hyper-parameter giving the number of quantile targets. Second, we replace the Huber loss used by \footnote{ uses gradient clipping of the squared error that makes it equivalent to a Huber loss with .},  with , with a quantile Huber loss (full loss given by Algorithm~\ref{alg:wdqn}). Finally, we replace RMSProp \cite{tieleman2012lecture} with Adam \cite{kingma2014adam}. We call this new algorithm quantile regression  ().


\begin{algorithm}[ht]
\caption{Quantile Regression Q-Learning}\label{alg:wdqn}
\begin{algorithmic}
\REQUIRE 
\INPUT , 
\STATE \textcolor{gray}{\# Compute distributional Bellman target}
\STATE 
\STATE 
\STATE 
\STATE \textcolor{gray}{\# Compute quantile regression loss (Equation~\ref{eqn:huber_quantile})}
\OUTPUT 
\end{algorithmic}
\end{algorithm}

Unlike ,  does not require projection onto the approximating distribution's support, instead it is able to expand or contract the values arbitrarily to cover the true range of return values. As an additional advantage, this means that  does not require the additional hyper-parameter giving the bounds of the support required by . The only additional hyper-parameter of  not shared by  is the number of quantiles , which controls with what resolution we approximate the value distribution. As we increase ,  goes from  to increasingly able to estimate the upper and lower quantiles of the value distribution. It becomes increasingly capable of distinguishing low probability events at either end of the cumulative distribution over returns. 

\begin{figure*}[ht]
\begin{center}
\includegraphics[width=\textwidth]{gridworld_exp.pdf}
\end{center}
\caption{(a) Two-room windy gridworld, with wind magnitude shown along bottom row. Policy trajectory shown by blue path, with additional cycles caused by randomness shown by dashed line. (b, c) (Cumulative) Value distribution at start state , estimated by MC, , and by , . (d, e) Value function (distribution) approximation errors for  and . \label{fig:windy_gw}}
\end{figure*}

\section{Experimental Results}

In the introduction we claimed that learning the distribution over returns had distinct advantages over learning the value function alone. We have now given theoretically justified algorithms for performing distributional reinforcement learning,  for policy evaluation and  for control. In this section we will empirically validate that the proposed distributional reinforcement learning algorithms: (1) learn the true distribution over returns, (2) show increased robustness during training, and (3) significantly improve sample complexity and final performance over baseline algorithms.

\subsubsection{Value Distribution Approximation Error}

We begin our experimental results by demonstrating that  actually learns an approximate value distribution that minimizes the -Wasserstein to the ground truth distribution over returns. Although our theoretical results already establish convergence of the former to the latter, the empirical performance helps to round out our understanding.

We use a variant of the classic windy gridworld domain \cite{sutton98reinforcement}, modified to have two rooms and randomness in the transitions. Figure~\ref{fig:windy_gw}(a) shows our version of the domain, where we have combined the transition stochasticity, wind, and the doorway to produce a multi-modal distribution over returns when anywhere in the first room. Each state transition has probability  of moving in a random direction, otherwise the transition is affected by wind moving the agent northward. The reward function is zero until reaching the goal state , which terminates the episode and gives a reward of . The discount factor is .

We compute the ground truth value distribution for optimal policy , learned by policy iteration, at each state by performing  Monte-Carlo (MC) rollouts and recording the observed returns as an empirical distribution, shown in Figure~\ref{fig:windy_gw}(b). Next, we ran both  and  while following  for  episodes. Each episode begins in the designated start state (). Both algorithms started with a learning rate of . For  we used  and drop  by half every  episodes.

Let  be the MC estimated distribution over returns from the start state , similarly  its mean. In Figure~\ref{fig:windy_gw} we show the approximation errors at  for both algorithms with respect to the number of episodes. In (d) we evaluated, for both  and , the squared error, , and in (e) we show the -Wasserstein metric for , , where  and  are the expected returns and value distribution at state  estimated by the algorithm. As expected both algorithms converge correctly in mean, and  minimizes the -Wasserstein distance to .


\subsection{Evaluation on Atari 2600}

\begin{figure*}[ht]
\begin{center}
\includegraphics[width=\textwidth]{combined_atari.pdf}
\end{center}
\caption{Online evaluation results, in human-normalized scores, over 57 Atari 2600 games for 200 million training samples. (Left) Testing performance for one seed, showing median over games. (Right) Training performance, averaged over three seeds, showing percentiles (10, 20, 30, 40, and 50) over games.\label{fig:wdqn_test}}
\end{figure*}

We now provide experimental results that demonstrate the practical advantages of minimizing the Wasserstein metric end-to-end, in contrast to the  approach. We use the 57 Atari 2600 games from the Arcade Learning Environment (ALE) \cite{bellemare13arcade}. Both  and  build on the standard  architecture, and we expect both to benefit from recent improvements to  such as the dueling architectures \cite{wang2016dueling} and prioritized replay \cite{schaul16prioritized}. However, in our evaluations we compare the pure versions of  and  without these additions. We present results for both a strict quantile loss,  (-), and with a Huber quantile loss with  (-). 

We performed hyper-parameter tuning over a set of five training games and evaluated on the full set of 57 games using these best settings (, , and ).\footnote{We swept over  in ();  in ();  ()} As with  we use a target network when computing the distributional Bellman update. We also allow  to decay at the same rate as in , but to a lower value of , as is common in recent work \cite{c51,wang2016dueling,vanhasselt16deep}.

Out training procedure follows that of \citet{mnih15nature}'s, and we present results under two evaluation protocols: \textit{best agent} performance and \textit{online} performance. In both evaluation protocols we consider performance over all 57 Atari 2600 games, and transform raw scores into \textit{human-normalized scores} \cite{vanhasselt16deep}.

\begin{table}[ht]
\begin{center}
\begin{tabular}{ l | r | r | r | r }
\multicolumn{1}{c}{} & \mbox{\textbf{Mean}} & \mbox{\textbf{Median}} & \mbox{\textbf{human}} & \mbox{\textbf{DQN}} \\
\hline
\textsc{dqn}  &   228\% & 79\% & 24 & 0 \\
\textsc{ddqn}   &   307\% & 118\% & 33 & 43 \\
\textsc{Duel.}   &   373\% & 151\% & 37 & 50 \\
\textsc{Prior.}   &   434\% & 124\% & 39 & 48 \\
\textsc{Pr. Duel.}   &   592\% & 172\% & 39 & 44 \\
\hline
\hline
   &   701\% & 178\% & 40 & 50 \\
\hline
-   &   881\% & 199\% & 38 & 52 \\
-   &   \textbf{\textcolor{blue}{915\%}} & \textbf{\textcolor{blue}{211\%}} & \textbf{\textcolor{blue}{41}} & \textbf{\textcolor{blue}{54}} \\
\end{tabular}
\end{center}
\caption{Mean and median of \textit{best} scores across 57 Atari 2600 games, measured as percentages of human baseline \cite{nair15massively}.}
\label{fig:perc_scores}
\end{table}


\subsubsection{Best agent performance}

To provide comparable results with existing work we report test evaluation results under the best agent protocol. Every one million training frames, learning is frozen and the agent is evaluated for  frames while recording the average return. Evaluation episodes begin with up to  random no-ops \cite{mnih15nature}, and the agent uses a lower exploration rate (). As training progresses we keep track of the best agent performance achieved thus far.

Table~\ref{fig:perc_scores} gives the best agent performance, at  million frames trained, for , , , Double  \cite{vanhasselt16deep}, Prioritized replay \cite{schaul16prioritized}, and Dueling architecture \cite{wang2016dueling}. We see that  outperforms all previous agents in mean and median human-normalized score. 

\subsubsection{Online performance}
In this evaluation protocol (Figure~\ref{fig:wdqn_test}) we track the average return attained during each testing (left) and training (right) iteration. For the testing performance we use a single seed for each algorithm, but show online performance with no form of early stopping. For training performance, values are averages over three seeds. Instead of reporting only median performance, we look at the distribution of human-normalized scores over the full set of games. Each bar represents the score distribution at a fixed percentile (th, th, th, th, and th). The upper percentiles show a similar trend but are omitted here for visual clarity, as their scale dwarfs the more informative lower half.

From this, we can infer a few interesting results. (1) Early in learning, most algorithms perform worse than random for at least  of games. (2)  gives similar improvements to sample complexity as prioritized replay, while also improving final performance. (3) Even at  million frames, there are  of games where all algorithms reach less than  of human. This final point in particular shows us that all of our recent advances continue to be severely limited on a small subset of the Atari 2600 games.

\section{Conclusions}
The importance of the distribution over returns in reinforcement learning has been (re)discovered and highlighted many times by now. In \citet{c51} the idea was taken a step further, and argued to be a central part of approximate reinforcement learning. However, the paper left open the question of whether there exists an algorithm which could bridge the gap between Wasserstein-metric theory and practical concerns.

In this paper we have closed this gap with both theoretical contributions and a new algorithm which achieves state-of-the-art performance in Atari 2600.
There remain many promising directions for future work. Most exciting will be to expand on the promise of a richer policy class, made possible by action-value distributions. We have mentioned a few examples of such policies, often used for risk-sensitive decision making. However, there are many more possible decision policies that consider the action-value distributions as a whole.

Additionally,  is likely to benefit from the improvements on  made in recent years. For instance, due to the similarity in loss functions and Bellman operators we might expect that  suffers from similar over-estimation biases to those that Double  was designed to address \cite{vanhasselt16deep}. A natural next step would be to combine  with the non-distributional methods found in Table~\ref{fig:perc_scores}.

\section*{Acknowledgements}
The authors acknowledge the vital contributions of their colleagues at DeepMind. Special thanks to Tom Schaul, Audrunas Gruslys, Charles Blundell, and Benigno Uria for their early suggestions and discussions on the topic of quantile regression. Additionally, we are grateful for feedback from David Silver, Yee Whye Teh, Georg Ostrovski, Joseph Modayil, Matt Hoffman, Hado van Hasselt, Ian Osband, Mohammad Azar, Tom Stepleton, Olivier Pietquin, Bilal Piot; and a second acknowledgement in particular of Tom Schaul for his detailed review of an previous draft.

\begin{thebibliography}{}

\bibitem[\protect\citeauthoryear{Aravkin \bgroup et al\mbox.\egroup
  }{2014}]{aravkin2014sparse}
Aravkin, A.~Y.; Kambadur, A.; Lozano, A.~C.; and Luss, R.
\newblock 2014.
\newblock {Sparse Quantile Huber Regression for Efficient and Robust
  Estimation}.
\newblock {\em arXiv}.

\bibitem[\protect\citeauthoryear{Arjovsky, Chintala, and Bottou}{2017}]{wgan}
Arjovsky, M.; Chintala, S.; and Bottou, L.
\newblock 2017.
\newblock {Wasserstein Generative Adversarial Networks}.
\newblock In {\em Proceedings of the 34th International Conference on Machine
  Learning (ICML)}.

\bibitem[\protect\citeauthoryear{Bellemare \bgroup et al\mbox.\egroup
  }{2013}]{bellemare13arcade}
Bellemare, M.~G.; Naddaf, Y.; Veness, J.; and Bowling, M.
\newblock 2013.
\newblock {The Arcade Learning Environment: An Evaluation Platform for General
  Agents.}
\newblock {\em Journal of Artificial Intelligence Research} 47:253--279.

\bibitem[\protect\citeauthoryear{Bellemare \bgroup et al\mbox.\egroup
  }{2017}]{bellemare17cramer}
Bellemare, M.~G.; Danihelka, I.; Dabney, W.; Mohamed, S.; Lakshminarayanan, B.;
  Hoyer, S.; and Munos, R.
\newblock 2017.
\newblock {The Cramer Distance as a Solution to Biased Wasserstein Gradients}.
\newblock {\em arXiv}.

\bibitem[\protect\citeauthoryear{Bellemare, Dabney, and Munos}{2017}]{c51}
Bellemare, M.~G.; Dabney, W.; and Munos, R.
\newblock 2017.
\newblock {A Distributional Perspective on Reinforcement Learning}.
\newblock {\em Proceedings of the 34th International Conference on Machine
  Learning (ICML)}.

\bibitem[\protect\citeauthoryear{Bellman}{1957}]{bellman57dynamic}
Bellman, R.~E.
\newblock 1957.
\newblock {\em {Dynamic Programming}}.
\newblock Princeton, NJ: Princeton University Press.

\bibitem[\protect\citeauthoryear{Bickel and
  Freedman}{1981}]{bickel81asymptotic}
Bickel, P.~J., and Freedman, D.~A.
\newblock 1981.
\newblock {Some Asymptotic Theory for the Bootstrap}.
\newblock {\em The Annals of Statistics}  1196--1217.

\bibitem[\protect\citeauthoryear{Chow \bgroup et al\mbox.\egroup
  }{2015}]{chow2015risk}
Chow, Y.; Tamar, A.; Mannor, S.; and Pavone, M.
\newblock 2015.
\newblock {Risk-Sensitive and Robust Decision-Making: a CVaR Optimization
  Approach}.
\newblock In {\em Advances in Neural Information Processing Systems (NIPS)},
  1522--1530.

\bibitem[\protect\citeauthoryear{Dearden, Friedman, and
  Russell}{1998}]{dearden98bayesian}
Dearden, R.; Friedman, N.; and Russell, S.
\newblock 1998.
\newblock Bayesian {Q}-learning.
\newblock In {\em Proceedings of the National Conference on Artificial
  Intelligence}.

\bibitem[\protect\citeauthoryear{Engel, Mannor, and
  Meir}{2005}]{engel05reinforcement}
Engel, Y.; Mannor, S.; and Meir, R.
\newblock 2005.
\newblock {Reinforcement Learning with Gaussian Processes}.
\newblock In {\em Proceedings of the International Conference on Machine
  Learning (ICML)}.

\bibitem[\protect\citeauthoryear{Engel}{1857}]{engel1857productions}
Engel, E.
\newblock 1857.
\newblock {Die Productions-und Consumtionsverh{\"a}ltnisse des K{\"o}nigreichs
  Sachsen}.
\newblock {\em Zeitschrift des Statistischen Bureaus des K{\"o}niglich
  S{\"a}chsischen Ministeriums des Innern} 8:1--54.

\bibitem[\protect\citeauthoryear{Goldstein, Misra, and
  Courtage}{1981}]{goldstein1981intrinsic}
Goldstein, S.; Misra, B.; and Courtage, M.
\newblock 1981.
\newblock {On Intrinsic Randomness of Dynamical Systems}.
\newblock {\em Journal of Statistical Physics} 25(1):111--126.

\bibitem[\protect\citeauthoryear{Heger}{1994}]{heger1994consideration}
Heger, M.
\newblock 1994.
\newblock {Consideration of Risk in Reinforcement Learning}.
\newblock In {\em Proceedings of the 11th International Conference on Machine
  Learning},  105--111.

\bibitem[\protect\citeauthoryear{Huber}{1964}]{huber1964robust}
Huber, P.~J.
\newblock 1964.
\newblock {Robust Estimation of a Location Parameter}.
\newblock {\em The Annals of Mathematical Statistics} 35(1):73--101.

\bibitem[\protect\citeauthoryear{Kingma and Ba}{2015}]{kingma2014adam}
Kingma, D., and Ba, J.
\newblock 2015.
\newblock {Adam: A Method for Stochastic Optimization}.
\newblock {\em Proceedings of the International Conference on Learning
  Representations}.

\bibitem[\protect\citeauthoryear{Koenker and
  Hallock}{2001}]{koenker2001quantile}
Koenker, R., and Hallock, K.
\newblock 2001.
\newblock {Quantile Regression: An Introduction}.
\newblock {\em Journal of Economic Perspectives} 15(4):43--56.

\bibitem[\protect\citeauthoryear{Koenker}{2005}]{qrbook}
Koenker, R.
\newblock 2005.
\newblock {\em {Quantile Regression}}.
\newblock {Cambridge University Press}.

\bibitem[\protect\citeauthoryear{Levina and Bickel}{2001}]{levina2001earth}
Levina, E., and Bickel, P.
\newblock 2001.
\newblock {The Earth Mover's Distance is the Mallows Distance: Some Insights
  from Statistics}.
\newblock In {\em The 8th IEEE International Conference on Computer Vision
  (ICCV)}.
\newblock IEEE.

\bibitem[\protect\citeauthoryear{Mnih \bgroup et al\mbox.\egroup
  }{2015}]{mnih15nature}
Mnih, V.; Kavukcuoglu, K.; Silver, D.; Rusu, A.~A.; Veness, J.; Bellemare,
  M.~G.; Graves, A.; Riedmiller, M.; Fidjeland, A.~K.; Ostrovski, G.; et~al.
\newblock 2015.
\newblock {Human-level Control through Deep Reinforcement Learning}.
\newblock {\em Nature} 518(7540):529--533.

\bibitem[\protect\citeauthoryear{Morimura \bgroup et al\mbox.\egroup
  }{2010}]{morimura10parametric}
Morimura, T.; Hachiya, H.; Sugiyama, M.; Tanaka, T.; and Kashima, H.
\newblock 2010.
\newblock {Parametric Return Density Estimation for Reinforcement Learning}.
\newblock In {\em Proceedings of the Conference on Uncertainty in Artificial
  Intelligence (UAI)}.

\bibitem[\protect\citeauthoryear{M{\"u}ller}{1997}]{muller1997integral}
M{\"u}ller, A.
\newblock 1997.
\newblock {Integral Probability Metrics and their Generating Classes of
  Functions}.
\newblock {\em Advances in Applied Probability} 29(2):429--443.

\bibitem[\protect\citeauthoryear{Nair \bgroup et al\mbox.\egroup
  }{2015}]{nair15massively}
Nair, A.; Srinivasan, P.; Blackwell, S.; Alcicek, C.; Fearon, R.; De~Maria, A.;
  Panneershelvam, V.; Suleyman, M.; Beattie, C.; and Petersen, S. e.~a.
\newblock 2015.
\newblock {Massively Parallel Methods for Deep Reinforcement Learning}.
\newblock In {\em ICML Workshop on Deep Learning}.

\bibitem[\protect\citeauthoryear{Puterman}{1994}]{puterman94markov}
Puterman, M.~L.
\newblock 1994.
\newblock {\em {M}arkov {D}ecision {P}rocesses: Discrete stochastic dynamic
  programming}.
\newblock John Wiley \& Sons, Inc.

\bibitem[\protect\citeauthoryear{Rummery and Niranjan}{1994}]{rummery94online}
Rummery, G.~A., and Niranjan, M.
\newblock 1994.
\newblock {On-line Q-learning using Connectionist Systems}.
\newblock Technical report, Cambridge University Engineering Department.

\bibitem[\protect\citeauthoryear{Schaul \bgroup et al\mbox.\egroup
  }{2016}]{schaul16prioritized}
Schaul, T.; Quan, J.; Antonoglou, I.; and Silver, D.
\newblock 2016.
\newblock {Prioritized Experience Replay}.
\newblock In {\em Proceedings of the International Conference on Learning
  Representations (ICLR)}.

\bibitem[\protect\citeauthoryear{Sutton and
  Barto}{1998}]{sutton98reinforcement}
Sutton, R.~S., and Barto, A.~G.
\newblock 1998.
\newblock {\em {Reinforcement Learning: An Introduction}}.
\newblock MIT Press.

\bibitem[\protect\citeauthoryear{Taylor}{1999}]{taylor1999quantile}
Taylor, J.~W.
\newblock 1999.
\newblock {A Quantile Regression Approach to Estimating the Distribution of
  Multiperiod Returns}.
\newblock {\em The Journal of Derivatives} 7(1):64--78.

\bibitem[\protect\citeauthoryear{Tieleman and
  Hinton}{2012}]{tieleman2012lecture}
Tieleman, T., and Hinton, G.
\newblock 2012.
\newblock Lecture 6.5: Rmsprop.
\newblock {\em COURSERA: Neural Networks for Machine Learning} 4(2).

\bibitem[\protect\citeauthoryear{Tsitsiklis and {Van
  Roy}}{1997}]{tsitsiklis97analysis}
Tsitsiklis, J.~N., and {Van Roy}, B.
\newblock 1997.
\newblock {An Analysis of Temporal-Difference Learning with Function
  Approximation}.
\newblock {\em IEEE Transactions on Automatic Control} 42(5):674--690.

\bibitem[\protect\citeauthoryear{van Hasselt, Guez, and
  Silver}{2016}]{vanhasselt16deep}
van Hasselt, H.; Guez, A.; and Silver, D.
\newblock 2016.
\newblock {Deep Reinforcement Learning with Double Q-Learning}.
\newblock In {\em Proceedings of the AAAI Conference on Artificial
  Intelligence}.

\bibitem[\protect\citeauthoryear{Wang \bgroup et al\mbox.\egroup
  }{2016}]{wang2016dueling}
Wang, Z.; Schaul, T.; Hessel, M.; Hasselt, H.~v.; Lanctot, M.; and de~Freitas,
  N.
\newblock 2016.
\newblock {Dueling Network Architectures for Deep Reinforcement Learning}.
\newblock In {\em Proceedings of the International Conference on Machine
  Learning (ICML)}.

\bibitem[\protect\citeauthoryear{Watkins and Dayan}{1992}]{watkins1992q}
Watkins, C.~J., and Dayan, P.
\newblock 1992.
\newblock Q-learning.
\newblock {\em Machine Learning} 8(3):279--292.

\end{thebibliography}
\bibliographystyle{aaai}

\newpage

\section*{Appendix}

\subsection{Proofs}

\wonemidpoint*

\begin{proof}
For any , the function  is convex, and has subgradient given by

so the function  is also convex, and has subgradient given by

Setting this subgradient equal to  yields

and since  is the identity map on , it is clear that  satisfies Equation \ref{eq:subgradeqn}. Note that in fact any  such that  yields a subgradient of , which leads to a multitude of minimizers if  is not continuous at .
\end{proof}

\biasedgradients*

\begin{proof}
Write , with . We take  to be of the same form as . Specifically, consider  given by

supported on the set ,
and take .
Then clearly the unique minimizing  for  is given by taking . However, consider the gradient with respect to  for the objective

We have

In the event that the sample distribution  has an atom at , then the optimal transport plan pairs the atom of  at  with this atom of , and gradient with respect to  of  is . If the sample distribution  does not contain an atom at , then the left-most atom of  is greater than  (since  is supported on . In this case, the gradient on  is negative. Since this happens with non-zero probability, we conclude that 

and therefore  cannot be the minimizer of .
\end{proof}

\Winftycontract*

\begin{proof}
We assume that instantaneous rewards given a state-action pair are deterministic; the general case is a straightforward generalization. Further, since the operator  is a -contraction in , it is sufficient to prove the claim in the case . In addition, since Wasserstein distances are invariant under translation of the support of distributions, it is sufficient to deal with the case where  for all . The proof then proceeds by first reducing to the case where every value distribution consists only of single Diracs, and then dealing with this reduced case using Lemma \ref{lem:1DiracCase}.

We write  and , for some functions . Let  be a state-action pair, and let  be all the state-action pairs that are accessible from  in a single transition, where  is a (finite or countable) indexing set. Write  for the probability of transitioning from  to , for each . We now construct a new MDP and new value distributions for this MDP in which all distributions are given by single Diracs, with a view to applying Lemma \ref{lem:1DiracCase}. The new MDP is of the following form. We take the state-action pair , and define new states, actions, transitions, and a policy , so that the state-action pairs accessible from  in this new MDP are given by , and the probability of reaching the state-action pair  is . Further, we define new value distributions  as follows. For each  and , we set:

The construction is illustrated in Figure \ref{fig:TransformedMDP}.
\begin{figure}
    \centering
    \includegraphics[keepaspectratio,width=0.36\textwidth]{MDP1alt.pdf}
    \vspace{0.25cm}
    \rule{0.47\textwidth}{0.4pt}


    \includegraphics[keepaspectratio,width=0.36\textwidth]{MDP2alt.pdf}
    \caption{Initial MDP and value distribution  (top), and transformed MDP and value distribution  (bottom).}
    \label{fig:TransformedMDP}
\end{figure}

Since, by Lemma \ref{lem:WinftyIsMaxQuantileDiff}, the  distance between the 1-Wasserstein projections of two real-valued distributions is the max over the difference of a certain set of quantiles, we may appeal to Lemma \ref{lem:1DiracCase} to obtain the following:


Now note that by construction,  (respectively, ) has the same distribution as  (respectively, ), and so

Therefore, substituting this into the Inequality \ref{eq:ndiracsto1dirac}, we obtain

Taking suprema over the initial state  then yields the result.
\end{proof}

\subsection{Supporting results}

\begin{lem}\label{lem:1DiracCase}Consider an MDP with countable state and action spaces.
Let  be value distributions such that each state-action distribution ,  is given by a single Dirac. Consider the particular case where rewards are identically  and , and let . Denote by  the projection operator that maps a probability distribution onto a Dirac delta located at its \textsuperscript{th} quantile. Then

\end{lem}
\begin{proof}
Let  and  for each state-action pair , for some functions . Let  be a state-action pair, and let  be all the state-action pairs that are accessible from  in a single transition, with  a (finite or countably infinite) indexing set.
To lighten notation, we write  for  and  for . Further, let the probability of transitioning from  to  be , for all .

Then we have

Now consider the \textsuperscript{th} quantile of each of these distributions, for  arbitrary. Let  be such that  is equal to this quantile of , and let  such that  is equal to this quantile of . Now note that

We now show that 

is impossible, from which it will follow that

and the result then follows by taking maxima over state-action pairs .
To demonstrate the impossibility of \eqref{eq:contradictthis}, 
without loss of generality we take . 

We now introduce the following partitions of the indexing set . Define:

and observe that we clearly have the following disjoint unions:

If \eqref{eq:contradictthis} is to hold, then we must have . Therefore, we must have . But if this is the case, then since  is the \textsuperscript{th} quantile of , we must have

and so consequently

from which we conclude that the \textsuperscript{th} quantile of  is less than , a contradiction. Therefore \eqref{eq:contradictthis} cannot hold, completing the proof.
\end{proof}

\begin{lem}\label{lem:WinftyIsMaxQuantileDiff}
For any two probability distributions  over the real numbers, and the Wasserstein projection operator  that projects distributions onto support of size , we have that

\end{lem}
\begin{proof}
By the discussion surrounding Lemma \ref{w1_midpoint}, we have that  for . Therefore, the optimal coupling between  and  must be given by  for each . This immediately leads to the expression of the lemma.
\end{proof}

\subsection{Further theoretical results}

\begin{lem}\label{lem:dpnocontraction}
The projected Bellman operator  is in general not a non-expansion in , for .
\end{lem}
\begin{proof}
Consider the case where the number of Dirac deltas in each distribution, , is equal to , and let . We consider an MDP with a single initial state, , and two terminal states,  and . We take the action space of the MDP to be trivial, and therefore omit it in the notation that follows. Let the MDP have a  probability of transitioning from  to , and  probability of transitioning from  to . We take all rewards in the MDP to be identically . Further, consider two value distributions,  and , given by:

Then note that we have

and so

We now consider the projected backup for these two value distributions at the state . We first compute the full backup:

Appealing to Lemma \ref{w1_midpoint}, we note that when projected these distributions onto two equally-weighted Diracs, the locations of these Diracs correspond to the 25\% and 75\% quantiles of the original distributions. We therefore have

and we therefore obtain

completing the proof.
\end{proof}

\subsection{Notation}
Human-normalized scores are given by \cite{vanhasselt16deep},

where ,  and  represent the per-game raw scores for the agent, human baseline, and random agent baseline.

\begin{table}[ht]
\centering
\caption{Notation used in the paper}
\label{my-label}
\begin{tabular}{lll}
  Symbol        &  Description of usage\\
  \hline \\
                & Reinforcement Learning \\
  \hline
           &  MDP (, , , , ) \\
           &  State space of MDP \\
           &  Action space of MDP \\
  ,     &  Reward function, random variable reward \\
             &  Transition probabilities,  \\
        &  Discount factor,  \\
   & States \\
   & Actions \\
      &   Rewards\\
           & Policy \\
       & (dist.) Bellman operator \\
           & (dist.) Bellman optimality operator \\
  ,   & Value function, state-value function \\
  ,   & Action-value function \\
        & Step-size parameter, learning rate \\
      & Exploration rate, -greedy \\
   & Adam parameter \\
        & Huber-loss parameter  \\
    & Huber-loss with parameter  \\
  \hline \\
                & Distributional Reinforcement Learning \\
  \hline
  ,   & Random return, value distribution \\
    & Monte-Carlo value distribution under policy  \\
           & Space of value distributions \\
   & Fixed point of convergence for  \\
      & Instantiated return sample \\
             & Metric order \\
           & -Wasserstein metric \\
           & Metric order  \\
          & maximal form of Wasserstein \\
          & Projection used by \\
     & -Wasserstein projection \\
     & Quantile regression loss \\
   & Huber quantile loss \\
   & Probabilities, parameterized probabilities \\
   & Cumulative probabilities with \\
   & Midpoint quantile targets\\
        & Sample from unit interval \\
      & Dirac function at \\
        & Parameterized function \\
             & Bernoulli distribution \\
       & Parameterized Bernoulli distribution \\
          & Space of quantile (value) distributions \\
      & Parameterized quantile (value) distribution \\
             & Random variable over \\
   & Random variable samples\\
     & Empirical distribution from -Diracs
\end{tabular}
\end{table}

\begin{figure*}[t]
\begin{center}
\includegraphics[width=\textwidth]{atari_game_curves.pdf}
\end{center}
\caption{Online training curves for , , and  on 57 Atari 2600 games. Curves are averages over three seeds, smoothed over a sliding window of 5 iterations, and error bands give standard deviations.\label{fig:all_games}}
\end{figure*}

\newpage

\begin{figure*}
\small
\centering
\begin{tabular}{ l | r|r|r|r|r|r| r }
  \textbf{\textsc{games}}  &  \textbf{\textsc{random}}  &  \textbf{\textsc{human}}  &  \textbf{\textsc{dqn}}  &   \textbf{\textsc{prior.}} \textbf{\textsc{duel.}}  &  \textbf{\textsc{c51}} & \textbf{\textsc{qr-dqn-0}} & \textbf{\textsc{qr-dqn-1}}\\
\hline
Alien & 227.8 & 7,127.7 & 1,620.0 & 3,941.0 & 3,166 & \textbf{\textcolor{blue}{9,983}} & 4,871 \\
Amidar & 5.8 & 1,719.5 & 978.0 & 2,296.8 & 1,735 & \textbf{\textcolor{blue}{2,726}} & 1,641 \\
Assault & 222.4 & 742.0 & 4,280.4 & 11,477.0 & 7,203 & 19,961 & \textbf{\textcolor{blue}{22,012}} \\
Asterix & 210.0 & 8,503.3 & 4,359.0 & 375,080.0 & 406,211 & \textbf{\textcolor{blue}{454,461}} & 261,025 \\
Asteroids & 719.1 & \textbf{\textcolor{blue}{47,388.7}} & 1,364.5 & 1,192.7 & 1,516 & 2,335 & 4,226 \\
Atlantis & 12,850.0 & 29,028.1 & 279,987.0 & 395,762.0 & 841,075 & \textbf{\textcolor{blue}{1,046,625}} & 971,850 \\
Bank Heist & 14.2 & 753.1 & 455.0 & \textbf{\textcolor{blue}{1,503.1}} & 976 & 1,245 & 1,249 \\
Battle Zone & 2,360.0 & 37,187.5 & 29,900.0 & 35,520.0 & 28,742 & 35,580 & \textbf{\textcolor{blue}{39,268}} \\
Beam Rider & 363.9 & 16,926.5 & 8,627.5 & 30,276.5 & 14,074 & 24,919 & \textbf{\textcolor{blue}{34,821}} \\
Berzerk & 123.7 & 2,630.4 & 585.6 & 3,409.0 & 1,645 & \textbf{\textcolor{blue}{34,798}} & 3,117 \\
Bowling & 23.1 & \textbf{\textcolor{blue}{160.7}} & 50.4 & 46.7 & 81.8 & 85.3 & 77.2 \\
Boxing & 0.1 & 12.1 & 88.0 & 98.9 & 97.8 & 99.8 & \textbf{\textcolor{blue}{99.9}} \\
Breakout & 1.7 & 30.5 & 385.5 & 366.0 & 748 & \textbf{\textcolor{blue}{766}} & 742 \\
Centipede & 2,090.9 & 12,017.0 & 4,657.7 & 7,687.5 & 9,646 & 9,163 & \textbf{\textcolor{blue}{12,447}} \\
Chopper Command & 811.0 & 7,387.8 & 6,126.0 & 13,185.0 & \textbf{\textcolor{blue}{15,600}} & 7,138 & 14,667 \\
Crazy Climber & 10,780.5 & 35,829.4 & 110,763.0 & 162,224.0 & 179,877 & \textbf{\textcolor{blue}{181,233}} & 161,196 \\
Defender & 2,874.5 & 18,688.9 & 23,633.0 & 41,324.5 & 47,092 & 42,120 & \textbf{\textcolor{blue}{47,887}} \\
Demon Attack & 152.1 & 1,971.0 & 12,149.4 & 72,878.6 & \textbf{\textcolor{blue}{130,955}} & 117,577 & 121,551 \\
Double Dunk & -18.6 & -16.4 & -6.6 & -12.5 & 2.5 & 12.3 & \textbf{\textcolor{blue}{21.9}} \\
Enduro & 0.0 & 860.5 & 729.0 & 2,306.4 & \textbf{\textcolor{blue}{3,454}} & 2,357 & 2,355 \\
Fishing Derby & -91.7 & -38.7 & -4.9 & \textbf{\textcolor{blue}{41.3}} & 8.9 & 37.4 & 39.0 \\
Freeway & 0.0 & 29.6 & 30.8 & 33.0 & 33.9 & \textbf{\textcolor{blue}{34.0}} & \textbf{\textcolor{blue}{34.0}} \\
Frostbite & 65.2 & 4,334.7 & 797.4 & \textbf{\textcolor{blue}{7,413.0}} & 3,965 & 4,839 & 4,384 \\
Gopher & 257.6 & 2,412.5 & 8,777.4 & 104,368.2 & 33,641 & \textbf{\textcolor{blue}{118,050}} & 113,585 \\
Gravitar & 173.0 & \textbf{\textcolor{blue}{3,351.4}} & 473.0 & 238.0 & 440 & 546 & 995 \\
H.E.R.O. & 1,027.0 & 30,826.4 & 20,437.8 & 21,036.5 & \textbf{\textcolor{blue}{38,874}} & 21,785 & 21,395 \\
Ice Hockey & -11.2 & \textbf{\textcolor{blue}{0.9}} & -1.9 & -0.4 & -3.5 & -3.6 & -1.7 \\
James Bond & 29.0 & 302.8 & 768.5 & 812.0 & 1,909 & 1,028 & \textbf{\textcolor{blue}{4,703}} \\
Kangaroo & 52.0 & 3,035.0 & 7,259.0 & 1,792.0 & 12,853 & 14,780 & \textbf{\textcolor{blue}{15,356}} \\
Krull & 1,598.0 & 2,665.5 & 8,422.3 & 10,374.4 & 9,735 & 11,139 & \textbf{\textcolor{blue}{11,447}} \\
Kung-Fu Master & 258.5 & 22,736.3 & 26,059.0 & 48,375.0 & 48,192 & 71,514 & \textbf{\textcolor{blue}{76,642}} \\
Montezuma's Revenge & 0.0 & \textbf{\textcolor{blue}{4,753.3}} & 0.0 & 0.0 & 0.0 & 75.0 & 0.0 \\
Ms. Pac-Man & 307.3 & \textbf{\textcolor{blue}{6,951.6}} & 3,085.6 & 3,327.3 & 3,415 & 5,822 & 5,821 \\
Name This Game & 2,292.3 & 8,049.0 & 8,207.8 & 15,572.5 & 12,542 & 17,557 & \textbf{\textcolor{blue}{21,890}} \\
Phoenix & 761.4 & 7,242.6 & 8,485.2 & \textbf{\textcolor{blue}{70,324.3}} & 17,490 & 65,767 & 16,585 \\
Pitfall! & -229.4 & \textbf{\textcolor{blue}{6,463.7}} & -286.1 & 0.0 & 0.0 & 0.0 & 0.0 \\
Pong & -20.7 & 14.6 & 19.5 & 20.9 & 20.9 & \textbf{\textcolor{blue}{21.0}} & \textbf{\textcolor{blue}{21.0}} \\
Private Eye & 24.9 & \textbf{\textcolor{blue}{69,571.3}} & 146.7 & 206.0 & 15,095 & 146 & 350 \\
Q*Bert & 163.9 & 13,455.0 & 13,117.3 & 18,760.3 & 23,784 & 26,646 & \textbf{\textcolor{blue}{572,510}} \\
River Raid & 1,338.5 & 17,118.0 & 7,377.6 & \textbf{\textcolor{blue}{20,607.6}} & 17,322 & 9,336 & 17,571 \\
Road Runner & 11.5 & 7,845.0 & 39,544.0 & 62,151.0 & 55,839 & \textbf{\textcolor{blue}{67,780}} & 64,262 \\
Robotank & 2.2 & 11.9 & \textbf{\textcolor{blue}{63.9}} & 27.5 & 52.3 & 61.1 & 59.4 \\
Seaquest & 68.4 & 42,054.7 & 5,860.6 & 931.6 & \textbf{\textcolor{blue}{266,434}} & 2,680 & 8,268 \\
Skiing & -17,098.1 & \textbf{\textcolor{blue}{-4,336.9}} & -13,062.3 & -19,949.9 & -13,901 & -9,163 & -9,324 \\
Solaris & 1,236.3 & \textbf{\textcolor{blue}{12,326.7}} & 3,482.8 & 133.4 & 8,342 & 2,522 & 6,740 \\
Space Invaders & 148.0 & 1,668.7 & 1,692.3 & 15,311.5 & 5,747 & \textbf{\textcolor{blue}{21,039}} & 20,972 \\
Star Gunner & 664.0 & 10,250.0 & 54,282.0 & \textbf{\textcolor{blue}{125,117.0}} & 49,095 & 70,055 & 77,495 \\
Surround & -10.0 & 6.5 & -5.6 & 1.2 & 6.8 & \textbf{\textcolor{blue}{9.7}} & 8.2 \\
Tennis & -23.8 & -8.3 & 12.2 & 0.0 & 23.1 & \textbf{\textcolor{blue}{23.7}} & 23.6 \\
Time Pilot & 3,568.0 & 5,229.2 & 4,870.0 & 7,553.0 & 8,329 & 9,344 & \textbf{\textcolor{blue}{10,345}} \\
Tutankham & 11.4 & 167.6 & 68.1 & 245.9 & 280 & \textbf{\textcolor{blue}{312}} & 297 \\
Up and Down & 533.4 & 11,693.2 & 9,989.9 & 33,879.1 & 15,612 & 53,585 & \textbf{\textcolor{blue}{71,260}} \\
Venture & 0.0 & 1,187.5 & 163.0 & 48.0 & \textbf{\textcolor{blue}{1,520}} & 0.0 & 43.9 \\
Video Pinball & 16,256.9 & 17,667.9 & 196,760.4 & 479,197.0 & \textbf{\textcolor{blue}{949,604}} & 701,779 & 705,662 \\
Wizard Of Wor & 563.5 & 4,756.5 & 2,704.0 & 12,352.0 & 9,300 & \textbf{\textcolor{blue}{26,844}} & 25,061 \\
Yars' Revenge & 3,092.9 & 54,576.9 & 18,098.9 & \textbf{\textcolor{blue}{69,618.1}} & 35,050 & 32,605 & 26,447 \\
Zaxxon & 32.5 & 9,173.3 & 5,363.0 & \textbf{\textcolor{blue}{13,886.0}} & 10,513 & 7,200 & 13,112
\end{tabular}
\caption{Raw scores across all games, starting with 30 no-op actions. Reference values from \citet{wang2016dueling} and \citet{c51}.\label{fig:atari_sota}}
\end{figure*}

\end{document}