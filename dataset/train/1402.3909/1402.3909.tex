



In this section we design a faster algorithm to find $q$-representative family for product families. Our main technical tool is a generalization of {\em $n$-$p$-$q$-separating collection} defined in~\cite{FominLS13} to compute  $q$-representative families of an arbitrary family. In fact we design a {\em family} of $n$-$p$-$q$-separating collections of various sizes governed by a parameter $0<x<1$.  The construction of generalized $n$-$p$-$q$-separating collection  is similar to the proof given in~\cite{FominLS13}. However, the new construction  requires some additional ideas and the proof is slightly more involved. Finally, we combine two $n$-$p$-$q$-separating collections obtained with different parameters to obtain the desired algorithm for product families. 

\subsection{Generalized $n$-$p$-$q$-separating collection}

We start with the formal definition of {\em generalized $n$-$p$-$q$-separating collection}.

\begin{definition}
\label{def:twincollection}
A generalized $n$-$p$-$q$-separating collection ${\cal C}$ is a tuple $({\cal F}, \chi, \chi')$, where ${\cal F}$ is a family of sets over a universe $U$ of size $n$, $\chi$ is a function from 
$\bigcup_{p'\leq p}{U\choose p'}$ to $2^{\cal F}$ and $\chi'$ is a function from $\bigcup_{q'\leq q}{U\choose q'}$ to $2^{\cal F}$ such that the following properties are satisfied
\begin{enumerate}
 \item for every $A\in \bigcup_{p'\leq p}{U\choose p'}$ and $F \in \chi(A)$, $A \subseteq F$,
 \item for every $B\in \bigcup_{q'\leq q}{U\choose q'}$ and $F \in \chi'(B)$, $F\cap B=\emptyset$, 
 \item for every pairwise disjoint sets $A_1\in {U \choose p_1},A_2\in {U \choose p_1},\cdots, A_r\in {U \choose p_r}$ and $B \in {U \choose q}$ such that $p_1+\cdots+p_r=p$, 
$\exists F\in \chi(A_1)\cap\chi(A_2)\ldots\chi(A_r)\cap \chi'(B)$.
\end{enumerate}
The size of  $({\cal F},\chi, \chi')$ is $|{\cal F}|$, the $(\chi,p')$-degree of $({\cal F},\chi,\chi')$ for $p'\leq p$ is $\max_{A \in {U \choose p'}} |\chi(A)|$, 
and the $(\chi',q')$-degree of $({\cal F},\chi, \chi')$ for $q'\leq q$ is $\max_{B \in {U \choose q'}} |\chi'(B)|$. 
\end{definition}




A {\em construction} of generalized separating collections is a data structure, that given $n$, $p$ and $q$ initializes and outputs a family ${\cal F}$ of sets over the universe $U$ of size $n$. 
After the initialization one can query the data structure by giving it a set  $A \in \bigcup_{p'\leq p}{U \choose p'}$ or $B\in \bigcup_{q'\leq q}{U \choose q'}$, the data structure 
then outputs a family  $\chi(A) \subseteq 2^{\cal F}$ or $\chi'(B)\subseteq 2^{\cal F}$ respectively. Together the tuple ${\cal C}= ({\cal F},\chi, \chi')$ computed by the data structure 
should form a {\em generalized} $n$-$p$-$q$-{\em separating collection}.

We call the time the data structure takes to initialize and output ${\cal F}$ the {\em initialization time}. 
The {\em $(\chi,p')$-query time}, $p'\leq p$, of the data structure is the maximum time the data structure uses to compute $\chi(A)$ over all $A \in {U \choose p'}$. Similarly, the 
{\em $(\chi',q')$-query time}, $q'\leq q$, of the data structure is the maximum time the data structure uses to compute $\chi'(B)$ over all $B \in {U \choose q'}$.
The initialization time of the data structure and the size of ${\cal C}$ are functions of  $n$, $p$ and $q$. The initialization time is denoted by 
$\tau_I(n,p,q)$, size of ${\cal C}$ is denoted by $\zeta(n,p,q)$. The $(\chi,p')$-query time and $(\chi,p')$-degree of 
$\cal C$, $p'\leq p$, are functions of 
$n,p',p,q$ and is denoted by ${Q_{(\chi,p')}}(n,p,q)$ and $\Delta_{(\chi,p')}(n,p,q)$ respectively. Similarly, the $(\chi',q')$-query time and $(\chi',q')$-degree of ${\cal C}$, $q'\leq q$,  are functions of 
$n,q',p,q$ and are denoted by ${Q_{(\chi',q')}}(n,p,q)$ and $\Delta_{(\chi',q')}(n,p,q)$ respectively.  
We are now ready to state the main technical  tool of this subsection.


\begin{lemma}
\label{lem:twin_sep_coll_construction}
Given a constant $x$ such that $0<x<1$, there is a construction of generalized $n$-$p$-$q$- separating collection with the following parameters
\begin{itemize} 
\setlength\itemsep{-.7mm}
\item size, $\zeta(n,p,q) \leq 2^{\cO(\frac{p+q}{\log\log\log(p+q)})}\cdot \frac{1}{x^p(1-x)^q}\cdot (p+q)^{\cO(1)} \cdot \log n$
\item initialization time, $\tau_I(n,p,q) \leq  2^{\cO(\frac{p+q}{\log\log\log(p+q)})}\cdot \frac{1}{x^p(1-x)^q}\cdot (p+q)^{\cO(1)} \cdot n\log n$
\item $(\chi,p')$-degree, $\Delta_{(\chi,p')}(n,p,q) \leq  2^{\cO(\frac{p+q}{\log\log\log(p+q)})}\cdot \frac{1}{x^{p-p'}(1-x)^q}\cdot (p+q)^{\cO(1)} \cdot \log n$
\item $(\chi,p')$-query time, $Q_{(\chi,p')}(n,p,q) \leq  2^{\cO(\frac{p+q}{\log\log\log(p+q)})}\cdot \frac{1}{x^{p-p'}(1-x)^q}\cdot (p+q)^{\cO(1)} \cdot \log n$
\item $(\chi',q')$-degree, $\Delta_{(\chi',q')}(n,p,q) \leq  2^{\cO(\frac{p+q}{\log\log\log(p+q)})}\cdot \frac{1}{x^{p}(1-x)^{q-q'}}\cdot (p+q)^{\cO(1)} \cdot \log n$
\item $(\chi',q')$-query time, $Q_{(\chi',q')}(n,p,q) \leq  2^{\cO(\frac{p+q}{\log\log\log(p+q)})}\cdot \frac{1}{x^{p}(1-x)^{q-q'}}\cdot (p+q)^{\cO(1)} \cdot \log n$
\end{itemize}
\end{lemma}

We first give a road map to prove Lemma~\ref{lem:twin_sep_coll_construction}. 
The proof of  Lemma~\ref{lem:twin_sep_coll_construction} uses three auxiliary lemmata. 
\begin{enumerate}
\item[(a.)] {\bf Existential Proof (Lemma~\ref{lem:twin_sep_coll_brute_force}}). This lemma shows that there is indeed a 
generalized $n$-$p$-$q$-separating collection with the required sizes, degrees and query time. Essentially, it shows that if we form a family  ${\cal F}=\{F_1,\ldots,F_t\}$ of sets of $U$ such that each $F_i$ is a random subset of $U$ where each element  is inserted into $F_i$ with probability $x$, then ${\cal F}$ has the desired sizes, degrees and query time. Thus, this also gives a brute force algorithm to design the family $\cal F$ by just guessing the family of desired size and then checking whether it is indeed  a generalized $n$-$p$-$q$-separating collection. 
\item[(b.)]  {\bf Universe Reduction (Lemma~\ref{lem:twinreduceUniverse}).} The construction obtained in Lemma~\ref{lem:twin_sep_coll_brute_force} 
has only one drawback that the initialization time is much larger than claimed in Lemma~\ref{lem:twin_sep_coll_construction}. To overcome this lacuna, we do not apply the construction in Lemma~\ref{lem:twin_sep_coll_brute_force} directly. 
We first prove a Lemma~\ref{lem:twinreduceUniverse} which helps us in reducing the universe size to $(p+q)^2$. This is done using the  known construction of $k$-perfect hash families of size $(p+q)^{\cO(1)} \log n$. 
Lemma~\ref{lem:twinreduceUniverse} alone  can not reduce the universe size sufficiently, that we can apply the construction of Lemma~\ref{lem:twin_sep_coll_brute_force}. 
\item[(c.)] {\bf Splitting Lemma (Lemma~\ref{lem:splitSolution}).} We give a splitter type construction in Lemma~\ref{lem:splitSolution} that when applied with 
Lemma~\ref{lem:twinreduceUniverse} makes the universe and other parameters small enough that we can apply the construction given in 
Lemma~\ref{lem:twin_sep_coll_brute_force}. In this construction we consider all the ``consecutive partitions''  of the universe into $t$ parts, assume that the sets $A\cup B$, $A=\cup_{i=1}^r A_i$, are distributed uniformly into $t$ parts and then use this information to obtain a construction of generalized separating collections in each part and then take the product of these collections to obtain a collection for the original instance.
\end{enumerate}

We start with an existential proof. 


\begin{lemma}\label{lem:twin_sep_coll_brute_force}
Given $0<x<1$, there is a construction of generalized $n$-$p$-$q$-separating collections with 
\begin{itemize}\setlength\itemsep{-.7mm}
\item size $\zeta(n,p,q) =\cO \left(\frac{1}{x^{p}(1-x)^q} \cdot (p^2+q^2+1)\log n \right)$, 
\item initialization time $\tau_I(n,p,q) = \cO({2^n \choose \zeta(n,p,q)} \cdot \frac{1}{x^p(1-x)^q} \cdot n^{\cO(p+q)})$,
\item $(\chi,p')$-degree for $p'\leq p$, $\Delta_{(\chi,p')}(n,p,q) = \cO\left(\frac{1}{x^{p-p'}}\cdot\frac{(p^2+q^2+1)}{(1-x)^q} \cdot \log n\right)$
\item $(\chi,p')$-query time ${Q_{(\chi,p')}}(n,p,q) = \cO(\frac{1}{x^{p}(1-x)^q} \cdot n^{\cO(1)}).$
\item $(\chi',q')$-degree $\Delta_{(\chi',q')}(n,p,q)=\cO\left(\frac{1}{x^{p}(1-x)^{q-q'}}\cdot(p^2+q^2+1) \cdot \log n\right)$
\item $(\chi',q')$-query time ${Q_{(\chi',q')}}(n,p,q)=\cO(\frac{1}{x^{p}(1-x)^q} \cdot n^{\cO(1)}).$
\end{itemize}
\end{lemma}

\begin{proof}
We start by giving a randomized algorithm that with positive probability constructs a generalized $n$-$p$-$q$-separating collection ${\cal C} = ({\cal F},\chi, \chi')$ with the desired size and degree parameters. 
We will then discuss how to deterministically compute such a ${\cal C}$ within the required time bound. Set $t = \frac{1}{x^p(1-x)^q} \cdot (p^2+q^2+1)\log n$ and construct the family 
${\cal F} = \{F_1, \ldots, F_t\}$ as follows. Each set $F_i$ is a random subset of $U$, where each element of $U$ is inserted into $F_i$ with probability $x$. Distinct elements are inserted (or not) into $F_i$ 
independently, and the construction of the different sets in ${\cal F}$ is also independent. For each  $A \in \bigcup_{p'\leq p}{U\choose p'}$ we set $\chi(A) = \{F \in {\cal F}~:~A \subseteq F\}$ and 
for each $B\in \bigcup_{q'\leq q}{U\choose q'}$ we set $\chi'(B)=\{F\in {\cal F}~:~F\cap B=\emptyset\}$.

The size of ${\cal F}$ is within the required bound by construction. We now argue that with positive probability 
$({\cal F},\chi, \chi')$ is indeed a generalized $n$-$p$-$q$-separating collection, and that the degrees of ${\cal C}$ 
is within the required bounds as well. For fixed sets $A \in {U \choose p}$, $B \in {U\setminus A \choose q}$, and integer $i \leq t$, we consider the probability that $A \subseteq F_i$ and $B \cap F_i = \emptyset$. 
This probability is $x^{p}(1-x)^q$. Since each $F_i$ is constructed independently from the other sets in ${\cal F}$, the probability that {\em no} $F_i$ satisfies $A \subseteq F_i$ and $B \cap F_i = \emptyset$ is
\begin{align*} \left(1 - x^p(1-x)^q\right)^t \leq e^{-(p^2+q^2+1)\log n} = \frac{1}{n^{p^2+q^2+1}}.\end{align*}
For a fixed $A_1,A_2,\ldots,A_r$ and $B$ (choices in condition $3$), the probability that no $F_i$ in $\chi(A_1)\cap\chi(A_2)\cap \cdots \cap\chi(A_r)\cap \chi'(B)$ is equal to the 
probability that no $F_i$ in $\chi(A_1\cup A_2\cdots \cup A_r)\cap \chi'(B)$  (since $\chi(A')$ contains all 
the sets in ${\cal F}$ that contains $A'$ and $\chi'(B)$ contains all the sets in ${\cal F}$ that are disjoint from $B$). 
Hence the probability that condition $3$ fails is upper bounded by 
$$Y\cdot\frac{1}{n^{p^2+q^2+1}}$$
where $Y$ is the number of choices for $A_1,\ldots,A_r$ and $B$ in condition $3$. We upper bound $Y$ as follows.
There are ${n \choose p}$ choices for $A_1\cup\cdots\cup A_r$ and ${n \choose q}$ choices for $B$. 
For each choice of $A_1\cup\cdots\cup A_r$ there are at most $r^p$ choices of making $A_1,\ldots,A_r$ with some of them being empty as well. Note that $r\leq p$. 
Therefore the number of possible choices of sets $A_1,A_2,\ldots,A_r$ and $B$ in condition $3$ is upper bounded by ${n\choose p}{n\choose q}p^p\leq n^{2p+q}\leq n^{p^2+q^2}$. 
Hence the probability that condition $3$ in Definition~\ref{def:twincollection} fails is at most $\frac{1}{n}$.

We also need to upper bound the maximum degree of ${\cal C}$. For every $A \in  {U \choose p'}$, $|\chi(A)|$ is a random variable. For a fixed $A \in  {U \choose p'}$ and $i \leq t$ the probability 
that $A \subseteq F_i$ is exactly $x^{p'}$. Hence $|\chi(A)|$ is the sum of $t$ independent  $0/1$-random variables that each take value $1$ with probability $x^{p'}$. Hence the expected value of $|\chi(A)|$ is 
$$E[|\chi(A)|] = t \cdot x^{p'} = \frac{1}{x^{p-p'}(1-x)^q}\cdot (p^2+q^2+1)\log n$$ 
For every $B\in {U\choose q'}$, $|\chi'(B)|$ is also a random variable. For a fixed $B \in  {U \choose q'}$ and $i \leq t$ the probability that $A \cap F_i=\emptyset$ is exactly $(1-x)^{q'}$. 
Hence the expected value of $|\chi'(B)|$ is,
$$E[|\chi'(B)|] = t \cdot (1-x)^{q'} = \frac{1}{x^{p}(1-x)^{q-q'}}\cdot (p^2+q^2+1)\log n.$$
Standard Chernoff bounds~\cite[Theorem 4.4]{mitzenmacher2005probability} show that the probability that for any $A\in{U\choose p'}$, $|\chi(A)|$ 
is at least $6E[|\chi(A)|]$ is upper bounded by $2^{-6E[|\chi(A)|]} \leq \frac{1}{n^{p^2+q^2+1}}$. 
Similarly the probability that for any $B\in{U \choose q'}$, $|\chi'(B)|$ is at least $6E[|\chi'(B)|]$ is upper bounded by $2^{-6E[|\chi'(B)|]} \leq \frac{1}{n^{p^2+q^2+1}}$.  
There are  $\sum_{p'\leq p}{n \choose p'}\leq{n^{p^2}}$ choices for $A\in\bigcup_{p'\leq p}{U\choose p'}$ and $\sum_{q'\leq q}{n \choose q'}\leq{n^{q^2}}$ choices for $B\in\bigcup_{q'\leq q}{U\choose q'}$. 
Hence the union bound yields that the probability that there exists an $A\in\bigcup_{p'\leq p}{U\choose p'}$ such that $|\chi(A)|  > 6E[|\chi(A)|]$ or there exists $B\in \bigcup_{q'\leq q}{U\choose q'}$ such 
that $|\chi'(B)|  > 6E[|\chi'(B)|]$ is upper bounded by $\frac {1}{n}$. Thus ${\cal C}$ is a family of $n$-$p$-$q$-separating collections with the desired size and degree parameters with probability at least 
$1 - \frac{2}{n} > 0$. The degenerate case that  $1 - \frac{2}{n} \leq 0$ is handled 
by the family ${\cal F}$ containing all (at most four) subsets of $U$. 

To construct ${\cal F}$ within the stated initialization time bound, it is sufficient to try all families ${\cal F}$ of size $t$ and for each of the ${2^n \choose \zeta(n,p,q)}$ 
guesses, test whether it is indeed a family of $n$-$p$-$q$-separating collections in time $\cO(t \cdot n^{\cO(p+q)}) = \cO(\frac{1}{x^p(1-x)^q} \cdot n^{\cO(p+q)})$.

For the queries, we need to give an algorithm that given $A$, computes $\chi(A)$ (or $\chi'(A)$), under the assumption that ${\cal F}$ has already has been computed in the initialization step. 
This is easily done within the stated running time bound by going through every set $F \in {\cal F}$, checking whether $A \subseteq F$ (or $A\cap F=\emptyset$), and if so, inserting $F$ into $\chi(A)$ ($\chi'(A)$). 
This concludes the proof.
\end{proof}
We will now work towards improving the time bounds of Lemma~\ref{lem:twin_sep_coll_brute_force}. 
 To that end we will need a construction of {\em $k$-perfect hash functions} by Alon et al.~\cite{AlonYZ}
\begin{definition} 
A family of functions $f_1, \ldots, f_t$ from a universe $U$ of size $n$ to a universe of size $r$ is a $k$-perfect family of hash functions if for every set $S \subseteq U$ such that $|S|=k$ 
there exists an $i$ such that the restriction of $f_i$ to $S$ is injective.
\end{definition}
Alon et al.~\cite{AlonYZ} give very efficient constructions of $k$-perfect families of hash functions from a universe of size $n$ to a universe of size $k^2$.
\begin{proposition}[\cite{AlonYZ}]\label{prop:hashFun} 
For any universe $U$ of size $n$ there is a $k$-perfect family $f_1, \ldots, f_t$ of hash functions from $U$ to 
$[k^2]$ 
with $t = \cO(k^{\cO(1)} \cdot \log n)$. 
Such a family of hash functions can be constructed in time $\cO(k^{\cO(1)}n \log n)$. 
\end{proposition}

\begin{lemma}\label{lem:twinreduceUniverse} If there is a construction of generalized $n$-$p$-$q$-separating collections $(\hat{\cal F},\hat{\chi},\hat{\chi}')$ with initialization time $\tau_I(n,p,q)$, size $\zeta(n,p,q)$, 
$(\hat{\chi},p')$-query time ${Q_{(\hat{\chi},p')}}(n,p,q)$, $(\hat{\chi}',q')$-query time ${Q_{(\hat{\chi}',q')}}(n,p,q)$, 
$(\hat{\chi},p')$-degree $\Delta_{(\hat{\chi},p')}(n,p,q)$, and $(\hat{\chi}',q')$-degree $\Delta_{(\hat{\chi}',q')}(n,p,q)$  
then there is a construction of generalized $n$-$p$-$q$-separating collections 
with following parameters.
\begin{itemize} \item 
$\zeta'(n,p,q) \leq \zeta\left((p+q)^2,p,q\right) \cdot  (p+q)^{\cO(1)} \cdot \log n$,
\item 
$\tau_I'(n,p,q) = \cO\left(\tau_I\left((p+q)^2,p,q\right) + \zeta\left((p+q)^2,p,q\right) \cdot (p+q)^{\cO(1)} \cdot n \log n\right)$,
\item 
$\Delta'_{(\chi,p')}(n,p,q) \leq \Delta_{(\hat{\chi},p')}\left((p+q)^2,p,q\right) \cdot  (p+q)^{\cO(1)} \cdot \log n$, 
\item 
${Q'_{(\chi,p')}}(n,p,q) = \cO\left(\left({Q_{(\hat{\chi},p')}}\left((p+q)^2,p,q\right) + \Delta_{(\hat{\chi},p')}\left((p+q)^2,p,q\right) \right) \cdot (p+q)^{\cO(1)} \cdot \log n\right)$,
\item 
$\Delta'_{(\chi',q')}(n,p,q)\leq \Delta_{(\hat{\chi}',q')}\left((p+q)^2,p,q\right) \cdot  (p+q)^{\cO(1)} \cdot \log n$,
\item 
${Q'_{(\chi',q')}}(n,p,q)=\cO\left(\left({Q_{(\hat{\chi}',q')}}\left((p+q)^2,p,q\right) + \Delta_{(\hat{\chi}',q')}\left((p+q)^2,p,q\right) \right) \cdot (p+q)^{\cO(1)} \cdot \log n\right)$
\end{itemize}
\end{lemma}
\begin{proof}
We give a construction of generalized $n$-$p$-$q$-separating collections with initialization time, query time, size and degree $\tau_I'$, ${Q}'$, $\zeta'$ and $\Delta'$ respectively using the 
construction with initialization time, query time, size and degree $\tau_I$, ${Q}$, $\zeta$ and $\Delta$ as a black box. 
 
We first describe the initialization of the data structure. Given $n$, $p$, and $q$, we construct using Proposition~\ref{prop:hashFun} a $(p+q)$-perfect family $f_1, \ldots f_t$ 
of hash functions from the universe $U$ to $[(p+q)^2]$. The construction takes time $\cO((p+q)^{\cO(1)}n \log n)$ and $t \leq  (p+q)^{\cO(1)} \cdot \log n$. 
We will store these hash functions in memory. We use the following notations.
\begin{itemize}
 \item For a set $S \subseteq U$ and $T\subseteq [(p+q)^2]$,  \\ $f_i(S)=\{f_i(s) ~:~ s \in S\}$ and $f_i^{-1}(T)=\{s \in U ~:~ f(s) \in T\}$. 
 \item For a family ${\cal Z}$ of sets over $U$ and family ${\cal W}$ of sets over $[(p+q)^2]$,\\ $f_i({\cal Z})=\{f_i(S) ~:~ S \in {\cal Z}\}$ and $f_i^{-1}({\cal W})=\{f_i^{-1}(T) ~:~ T \in {\cal W}\}$.
\end{itemize}


 We first use the given black box construction for $(p+q)^2$-$p$-$q$-separating collections $(\hat{\cal F}, \hat{\chi},\hat{\chi}')$ over the universe $[(p+q)^2]$. 
We run the initialization algorithm of this construction and store the family $\hat{\cal F}$ in memory. We then set
\begin{align*} {\cal F} = \bigcup_{i\leq t} f_i^{-1}(\hat{\cal F}). \end{align*}

 We spent $\cO((p+q)^{\cO(1)}n \log n)$ time to construct a $(p+q)$-perfect family of hash functions,  $\cO(\tau_I((p+q)^2,p,q))$ to construct $\hat{\cal F}$ of size $\zeta((p+q)^2,p,q)$, 
and $\cO(\zeta((p+q)^2,p,q) \cdot (p+q)^{\cO(1)} \cdot n \log n)$ time to construct ${\cal F}$ from $\hat{\cal F}$ and the family of perfect hash functions. 
Thus the upper bound on $\tau_I'(n,p,q)$ follows. Furthermore,  $|{\cal F}| \leq |\hat{\cal F}| \cdot  (p+q)^{\cO(1)} \cdot \log n$, yielding the claimed bound for $\zeta'$.
 
 We now define $\chi(A)$ for every $A \in \bigcup_{p'\leq p}{U\choose p'}$ and describe the query algorithm. For every $A \in \bigcup_{p'\leq p}{U\choose p'}$ we let
\begin{align*} \chi(A) = \bigcup_{\substack{i\leq t \\ |f_i(A)|=|A|}} f_i^{-1}(\hat{\chi}(f_i(A))). \end{align*}
Since $\forall\;\hat{F} \in \hat{\chi}(f_i(A))$, $f_i(A) \subseteq \hat{F}$, it follows that $A \subseteq F$ for every $F \in \chi(A)$. Furthermore we can bound $|\chi(A)|$ for any $A\in \bigcup_{p'\leq p}{U\choose p'}$, 
as follows 
\begin{align*} |\chi(A)| \leq \sum_{\substack{i\leq t \\ |f_i(A)|=|A|}} |\hat{\chi}(f_i(A))| \leq \Delta_{(\hat{\chi},p')}((p+q)^2,p,q) \cdot  (p+q)^{\cO(1)} \cdot \log n.\end{align*}
Thus the claimed bound for $\Delta'_{(\chi,p')}$ follows. 
Similar way we define $\chi'(B)$ for every $B\in\bigcup_{q'\leq q}{U\choose q'}$ as 
\begin{align*} \chi'(B) = \bigcup_{\substack{i\leq t \\ |f_i(A)|=|A|}} f_i^{-1}(\hat{\chi}'(f_i(A))). \end{align*}
\begin{align*} |\chi'(B)| \leq \sum_{\substack{i\leq t \\ |f_i(A)|=|A|}} |\hat{\chi}'(f_i(A))| \leq \Delta_{(\hat{\chi}',q')}((p+q)^2,p,q) \cdot  (p+q)^{\cO(1)} \cdot \log n.\end{align*}
To compute $\chi(A)$ for any $A\in\bigcup_{p'\leq p} {U \choose p'}$, we go over every $i \leq t$ and check whether $f_i$ is injective on $A$. This takes time $\cO((p+q)^{\cO(1)} \cdot \log n)$. 
For each $i$ such that $f_i$ is injective on $A$, we compute $f_i(A)$ and then $\hat{\chi}(f_i(A))$ in time $\cO({Q_{(\chi,p')}}((p+q)^2,p,q))$. Then we compute $f_i^{-1}(\hat{\chi}(f_i(A)))$  in time 
$\cO(|\hat{\chi}(f_i(A))|\cdot (p+q)^{\cO(1)}) =\cO(\Delta_{(\chi,p')}((p+q)^2,p,q)\cdot (p+q)^{\cO(1)})$ and add this set to $\chi(A)$. As we need to do this $\cO((p+q)^{\cO(1)} \cdot \log n)$ times, the total time 
to compute $\chi(A)$ is upper bounded by $\cO(({Q_{(\chi,p')}}((p+q)^2,p,q) + \Delta_{(\chi,p')}((p+q)^2,p,q)) \cdot (p+q)^{\cO(1)} \cdot \log n)$, 
yielding the claimed upper bound on ${Q'_{(\chi,p')}}$. Similar way we can bound ${Q'_{(\chi',q')}}$.

It remains to argue that $({\cal F},\chi,\chi')$ is in fact a generalized $n$-$p$-$q$-separating collection. For any $r$, consider pairwise disjoint sets $A_1 \in {U \choose p_1},\ldots,A_r \in {U \choose p_r}$, and 
$B \in {U \choose q}$ such that $p_1+\ldots+p_r=p$. We need to show that $\exists F\in \chi(A_1)\cap\cdots\cap\chi(A_r)\cap\chi'(B)$. 
Since $f_1, \ldots, f_t$ is a $(p+q)$-perfect family of hash functions, there is an $i$ such that $f_i$ is injective on $A_1\cup\cdots \cup A_r \cup B$. 
Since $(\hat{\cal F}, \hat{\chi},\hat{\chi}')$ is a $(p+q)^2$-$p$-$q$-separating collection,  
$\exists \hat{F}\in \hat{\chi}(f_i(A_1))\cap\cdots\hat{\chi}(f_i(A_r))\cap \hat{\chi}'(f_i(B))$. Since $f_i$ is injective on $A_1,\ldots,A_r$ and $B$, $f_i^{-1}(\hat{F})\in \chi(A_1)\cap\cdots \chi(A_r) \cap\chi'(B)$. 
This concludes the proof.
\end{proof}
We now give a {\em splitting lemma}, which allows us to reduce the problem of finding generalized $n$-$p$-$q$-separating collections to the same problem, but with much smaller values for $p$ and $q$. To that end we need some definitions.


\begin{definition}
A {\em partition} of $U$ is a family ${\cal U}_P = \{U_1, U_2, \ldots U_t\}$ of sets over $U$ such that $\forall i\neq j,\;U_i \cap U_j = \emptyset$ and $U = \bigcup_{i \leq t} U_i$. 
Each of the sets $U_i$ are called the {\em parts} of the partition. A {\em consecutive partition} of $\{1,\ldots,n\}$ is a partition ${\cal U}_P = \{U_1, U_2, \ldots U_t\}$ of $\{1,\ldots,n\}$ 
such that for every integer $i \leq t$ and integers $1 \leq x \leq y \leq z$, if $x \in U_i$ and $z \in U_i$ then $y \in U_i$ as well. 
\end{definition}
\begin{proposition}
Let $\mathscr{P}_t^{n}$ denote the collection of all consecutive partitions of $\{1,\ldots,n\}$ with exactly $t$ parts. Let 
${\cal Z}_{s,t}^p$ be the set of all $t$-tuples $(p_1,p_2, \ldots, p_t)$ of integers such that $\sum_{i \leq t} p_i = p$ and $0 \leq p_i \leq s$ for all $i$. 
Then for every $t$, $|\mathscr{P}_t^{n}| = {n+t-1 \choose t-1}$ and $|{\cal Z}_{s,t}^p| \leq {p+t-1 \choose t-1}$.
\end{proposition}
\begin{lemma}\label{lem:splitSolution} For any $p$, $q$ let $s = \lfloor (\log (p+q))^2 \rfloor$ and 
$t = \lceil \frac{p+q}{s} \rceil$. If there is a construction of generalized $n$-$p$-$q$-separating collections  
with initialization time $\tau_I(n,p,q)$, query times ${Q_{({\chi},p')}}(n,p,q)$ and ${Q_{({\chi}',q')}}(n,p,q)$, 
producing a generalized $n$-$p$-$q$-separating collection with size $\zeta(n,p,q)$, $({\chi},p')$-degree 
$\Delta_{({\chi},p')}(n,p,q)$ and $({\chi}',q')$-degree $\Delta_{({\chi}',q')}(n,p,q)$ then there is a construction 
of generalized $n$-$p$-$q$-separating collection with following parameters 
\begin{itemize}\item 
$\zeta'(n,p,q) \leq |\mathscr{P}_t^n| \cdot 
\sum_{\substack{(p_1,\ldots,p_t) \in {\cal Z}_{s,t}^{p}}} \prod_{i \leq t} \zeta(n,p_i,s-p_i)$,
\item 
$\tau_I'(n,p,q) = \cO\Big(\big(\sum_{\substack{\hat{p} \leq s}} \tau_I(n,\hat{p},s-\hat{p})\big) + \zeta'(n,p,q) \cdot n^{\cO(1)}\Big)$,
\item 
$\Delta_{(\chi,p')}'(n,p,q) \leq |\mathscr{P}_t^n|\cdot |{\cal Z}_{s,t}^p|\cdot
\max_{\substack{(p_1,\ldots,p_t) \in {\cal Z}_{s,t}^{p} \\ p_1'\leq p_1,\ldots, p_t'\leq p_t \\ p_1'+\ldots+p_t'=p' }} \prod_{i \leq t}  \Delta_{({\chi},p_i')}(n,p_i,s-p_i)$,
\item 
$\Delta'_{(\chi',q')}(n,p,q) \leq |\mathscr{P}_t^n|\cdot |{\cal Z}_{s,t}^p|\cdot
\max_{\substack{(p_1,\ldots,p_t) \in {\cal Z}_{s,t}^{p} \\ q_1'\leq s-p_1,\ldots, q_t'\leq s-p_t \\ q_1'+\ldots+q_t'=q'}} \prod_{i \leq t}  \Delta_{({\chi}',q_i')}(n,p_i,s-p_i)$,
\item 
${Q'_{(\chi,p')}}(n,p,q) =\cO \Big(\Delta'_{(\chi,p')}(n,p,q) \cdot n^{\cO(1)} + |\mathscr{P}_{t}^n|\cdot |{\cal Z}_{s,t}^p| \cdot t\cdot \big( \sum_{\substack{ \hat{p}'\leq \hat{p}\leq s \\ \hat{p}-\hat{p}'\leq p-p' \\ s-\hat{p}\leq q }} Q_{(\chi_{\hat{p}},\hat{p}')}(n,\hat{p},s-\hat{p})\big)  \Big)$
\item 
${Q'_{(\chi',q')}}(n,p,q) =\cO \Big(\Delta'_{(\chi,p')}(n,p,q) \cdot n^{\cO(1)} + |\mathscr{P}_{t}^n|\cdot |{\cal Z}_{s,t}^p| \cdot t\cdot \big( \sum_{\substack{ \hat{q}'\leq \hat{q}\leq s \\ \hat{q}-\hat{q}'\leq q-q' \\ s-\hat{q} \leq p }} Q_{(\chi_{\hat{q}},\hat{q}')}(n,s-\hat{q},\hat{q})\big)  \Big)$
\end{itemize}
\end{lemma}
\begin{proof}
Set $s = \lfloor (\log (p+q))^2 \rfloor$, $t = \lceil \frac{p+q}{s} \rceil$ and $\tilde{q} = st - p$. We will give 
a construction of generalized $n$-$p$-$\tilde{q}$-separating collections with initialization time, query time, size and degree 
within the claimed bounds above. In the construction we will be using the construction with initialization time 
$\tau_I$, query times ${Q_{({\chi},p')}}$ and ${Q_{({\chi}',q')}}$, size $\zeta$, and degrees $\Delta_{({\chi},p')}$ 
and $\Delta_{({\chi}',q')}$ as a black box. Since  $\tilde{q} \geq q$, a $n$-$p$-$\tilde{q}$-separating collection 
is also a $n$-$p$-$q$-separating collection. We may assume without loss of generality that $U = \{1, \ldots, n\}$.
 
Our algorithm runs  for every $0\leq\hat{p}\leq s$, the initialization of  the given construction of  generalized 
$n$-$\hat{p}$-$(s-\hat{p})$-separating collections. We will refer by 
$({\cal F}_{\hat{p}}, {\chi}_{\hat{p}}, {\chi}'_{\hat{p}})$ to the generalized separating collection constructed for 
$\hat{p}$. For each $\hat{p}$ the initialization of the construction outputs the family ${\cal F}_{\hat{p}}$. 
 
We need to define a few operations on families of sets. For families of sets  ${\cal A}$, ${\cal B}$ over $U$ 
and subset $U' \subseteq U$ we define
\begin{eqnarray*}
{\cal A} \sqcap U' &=& \{A \cap U'~:~A \in {\cal A}\} \\ 
{\cal A} \circ {\cal B} &=& \{A \cup B ~:~A \in {\cal A} \wedge B \in {\cal B}\} 
\end{eqnarray*}
We now define ${\cal F}$ as follows.
\begin{align}\label{eqn:defineFSplit} 
{\cal F} = \bigcup_{\substack{\{U_1,\ldots,U_{t}\} \in \mathscr{P}_{t}^n \\ (p_1,\ldots,p_t) \in {\cal Z}_{s,t}^{p}}} 
(\hat{\cal F}_{p_1} \sqcap U_1) \circ (\hat{\cal F}_{p_2} \sqcap U_2) \circ \ldots \circ (\hat{\cal F}_{p_t} \sqcap U_{t}) 
\end{align}
It follows directly from the definition of  ${\cal F}$ that $|{\cal F}|$ is within the claimed bound for 
$\zeta'(n,p,q)$. For the initialization time, the algorithm spends 
$\cO\left(\sum_{\substack{\hat{p} \leq s}} \tau_I(n,\hat{p},s-\hat{p})\right)$ time to initialize the constructions 
of the generalized $n$-$\hat{p}$-$(s-\hat{p})$-separating collections for all $\hat{p}\leq s$ together. Now the 
algorithm can output the entries of ${\cal F}$ one set at a time by using~\eqref{eqn:defineFSplit}, spending 
$n^{\cO(1)}$ time per output set. Hence the time bound for $\tau'_I(n,p,q)$ follows. 

For every set $A \in \bigcup_{p'\leq p}{U \choose p'}$ we define $\chi(A)$ as follows.
\begin{align}\label{eqn:defineChiSplit} 
\chi(A) = \bigcup_{\substack{\{U_1,\ldots,U_{t}\} \in \mathscr{P}_{t}^n\\(p_1,\ldots,p_t) \in {\cal Z}_{s,t}^{p}~\mbox{\scriptsize such that}\\ \forall U_i~:~|U_i \cap A| \leq p_i}} 
\Big[({\chi}_{p_1}(A \cap U_1) \sqcap U_1) \circ ({\chi}_{p_2}(A \cap U_2) \sqcap U_2) \circ \ldots \\
\nonumber ... \circ ({\chi}_{p_t}(A \cap U_t) \sqcap U_t)\Big] 
\end{align}
Now we show that $\chi(A)\subseteq {\cal F}$. From the definition of generalized $n$-${p_i}$-$(s-{p_i})$-separating 
collections $(\hat{\cal F}_{{p_i}},\chi_{{p_i}},\chi'_{{p_i}})$, each family $\chi_{p_i}(A\cap U_i)$ in~
\eqref{eqn:defineChiSplit} is a subset of $\hat{\cal F}_{p_i}$. This implies that 
$\chi_{p_i}(A\cap U_i)\sqcap U_i\subseteq \hat{\cal F}_{p_i} \sqcap U_i$. Hence $\chi(A)\subseteq {\cal F}$.   
Similar way we can define $\chi'(B)$ for any $B\in\bigcup_{q'\leq q}{U\choose q'}$ as
\begin{align}\label{eqn:defineChiprimeSplit} 
\chi'(B) = \bigcup_{\substack{\{U_1,\ldots,U_{t}\} \in \mathscr{P}_{t}^n\\(p_1,\ldots,p_t) \in {\cal Z}_{s,t}^{p}~\mbox{\scriptsize such that} \\ \forall U_i~:~|U_i \cap B| \leq s-p_i }} 
\Big[({\chi}'_{p_1}(B \cap U_1) \sqcap U_1) \circ ({\chi}'_{p_2}(B \cap U_2) \sqcap U_2) \circ \ldots \\
\nonumber ... \circ ({\chi}'_{p_t}(B \cap U_t) \sqcap U_t)\Big] 
\end{align}
Similar to the proof of $\chi(A)\subseteq {\cal F}$, we can show that $\chi'(B)\subseteq {\cal F}$. 
It follows directly from the definition of  $\chi(A)$ and $\chi'(B)$ that $|\chi(A)|$ and $|\chi'(B)|$ is within the 
claimed bound for $\Delta_{(\chi,p')}'(n,p,q)$ and $\Delta_{(\chi',q')}'(n,p,q)$ respectively. We now describe how 
queries $\chi(A)$ can be answered, and analyze how much time it takes. Given $A$ we will compute $\chi(A)$ using  ~\eqref{eqn:defineChiSplit}. Let $|A|=p'$. 
For each $\{U_1,\ldots,U_{t}\} \in \mathscr{P}_{t}^n$ and $(p_1,\ldots,p_t) \in {\cal Z}_{s,t}^{p}$ such that $p_i' = |U_i \cap A| \leq p_i$ for all $i \leq t$, 
we proceed as follows. First we compute ${\chi}_{p_i}(A \cap U_i)$ for each $i \leq t$, spending in total 
$\cO(\sum_{i \leq t} Q_{(\chi_{p_i},p_i')}(n,p_i,s-p_i))$ time. Now we add each set in 
$({\chi}_{p_1}(A \cap U_1) \sqcap U_1) \circ ({\chi}_{p_2}(A \cap U_2) \sqcap U_2) \circ \ldots \circ ({\chi}_{p_t}(A \cap U_t) \sqcap U_t)$ 
to $\chi(A)$, spending $n^{\cO(1)}$ time per set that is added to $\chi(A)$, yielding the bound below, 
\begin{align*}
Q'_{(\chi,p')}(n,p,q) \leq \cO\Big(\Delta'_{(\chi,p')}(n,p,q) \cdot n^{\cO(1)} + \sum_{\substack{\{U_1,\ldots,U_{t}\} \in \mathscr{P}_{t}\\(p_1,\ldots,p_t) \in {\cal Z}_{s,t}^{p}~\mbox{\scriptsize such that}\\ \forall U_i~:~p_i' = |U_i \cap A| \leq p_i}} \big[\sum_{i \leq t} Q_{\chi_{p_i},p_i'}(n,p_i,s-p_i)\big]\Big) \\
\leq \cO\Big(\Delta'_{(\chi,p')}(n,p,q) \cdot n^{\cO(1)} + |\mathscr{P}_{t}^n|\cdot |{\cal Z}_{s,t}^p| \cdot 
\max_{\substack{(p_1,\ldots,p_t) \in {\cal Z}_{s,t}^p\\ p_1'\leq p_1,\cdots,p_t'\leq p_t~\mbox{\scriptsize such that}\\p_1'+\cdots +p_t'=p'}} \big( \sum_{i \leq t} Q_{(\chi_{p_i},p_i')}(n,p_i,s-p_i)\big)  \Big) \\
\leq \cO\Big(\Delta'_{(\chi,p')}(n,p,q) \cdot n^{\cO(1)} + |\mathscr{P}_{t}^n|\cdot |{\cal Z}_{s,t}^p| \cdot t\cdot 
\big( \sum_{\substack{ \hat{p}'\leq \hat{p}\leq s \\ \hat{p}-\hat{p}'\leq p-p' \\ s-\hat{p}\leq q }} Q_{(\chi_{\hat{p}},\hat{p}')}(n,\hat{p},s-\hat{p})\big)  \Big) \\
\end{align*}
By doing similar analysis, we get required bound for $Q'_{(\chi',q')}$.  
We now need to argue that $({\cal F}, \chi,\chi')$ is in fact a generalized  $n$-$p$-$\tilde{q}$-separating collection. For any 
$r$, consider pairwise disjoint sets $A_1\in {U \choose p_1},\ldots,A_r\in {U \choose p_r}$ and 
$B \in {U \choose \tilde{q}}$ such that $p_1+\cdots+p_r=p$. Let $A=A_1\cup\cdots\cup A_r$. There exists a 
consecutive partition $\{U_1, \ldots, U_t\} \in \mathscr{P}_t^n$ of $U$ such that for every $i \leq t$ we have that 
$|(A \cup B) \cap U_i| = \frac{p+\tilde{q}}{t} = s$. For each $i \leq t$ set $p_i =  |A \cap U_i|$ and 
$q_i = |B \cap U_i| = s - p_i$. For every $i \leq t$ the tuple $({\cal F}_{p_i}, {\chi}_{p_i},\chi'_{p_i})$ form 
a $n$-$p_i$-$q_i$-separating collection. Hence 
$\exists F_i \in \chi_{p_i}(A_1\cap U_i)\cap\ldots\cap\chi_{p_i}(A_r\cap U_i) \cap \chi'_{p_i}(B\cap U_i)$ 
because $|A_1\cap U_i|+\ldots+|A_r\cap U_i|=p_i$, $|B\cap U_i|=q_i$ and 
$({\cal F}_{p_i}, {\chi}_{p_i},\chi'_{p_i})$ is a $n$-$p_i$-$q_i$-separating collection. That is 
$F_i \in \chi_{p_i}(A_j\cap U_i)$ for all $j\leq r$ and $F_i \in \chi'_{p_i}(B\cap U_i)$. Let 
$F=\bigcup_{i\leq t}F_i\cap U_i$. By construction of $\chi$ and $\chi'$, $F \in \chi(A_j)$ for all $j\leq r$  and 
$F \in \chi'(B)$. Hence $F \in \chi(A_1)\cap\ldots\cap \chi(A_r)\cap\chi'(B)$. This completes the proof
\end{proof}

Now we are ready to prove the Lemma~\ref{lem:twin_sep_coll_construction}. We restate the lemma for easiness of presentation. 

\medskip 



{{\bf Lemma~\ref{lem:twin_sep_coll_construction}} \em 
Given a constant $x$ such that $0<x<1$, there is a construction of generalized $n$-$p$-$q$- separating collection with the following parameters
\begin{itemize} \item size, $\zeta(n,p,q) \leq 2^{\cO(\frac{p+q}{\log\log\log(p+q)})}\cdot \frac{1}{x^p(1-x)^q}\cdot (p+q)^{O(1)} \cdot \log n$
\item initialization time, $\tau_I(n,p,q) \leq  2^{O(\frac{p+q}{\log\log\log(p+q)})}\cdot \frac{1}{x^p(1-x)^q}\cdot (p+q)^{O(1)} \cdot n\log n$
\item $(\chi,p')$-degree, $\Delta_{(\chi,p')}(n,p,q) \leq  2^{O(\frac{p+q}{\log\log\log(p+q)})}\cdot \frac{1}{x^{p-p'}(1-x)^q}\cdot (p+q)^{O(1)} \cdot \log n$
\item $(\chi,p')$-query time, $Q_{(\chi,p')}(n,p,q) \leq  2^{O(\frac{p+q}{\log\log\log(p+q)})}\cdot \frac{1}{x^{p-p'}(1-x)^q}\cdot (p+q)^{O(1)} \cdot \log n$
\item $(\chi',q')$-degree, $\Delta_{(\chi',q')}(n,p,q) \leq  2^{O(\frac{p+q}{\log\log\log(p+q)})}\cdot \frac{1}{x^{p}(1-x)^{q-q'}}\cdot (p+q)^{O(1)} \cdot \log n$
\item $(\chi',q')$-query time, $Q_{(\chi',q')}(n,p,q) \leq  2^{O(\frac{p+q}{\log\log\log(p+q)})}\cdot \frac{1}{x^{p}(1-x)^{q-q'}}\cdot (p+q)^{O(1)} \cdot \log n$
\end{itemize}}




\begin{proof}
The structure of the proof is as follows. We first create a collection using Lemma~\ref{lem:twin_sep_coll_construction}. 
Then we apply Lemma~\ref{lem:twinreduceUniverse} and obtain another construction. From here onwards we keep applying Lemma~\ref{lem:splitSolution} and Lemma~\ref{lem:twinreduceUniverse} in phases until we achieve the 
required bounds on size, degree, query and intializitaion time. 

We first apply Lemma~\ref{lem:twin_sep_coll_brute_force} and get a construction of $n$-$p$-$q$-twin separating collections with the following parameters.
\begin{itemize}\setlength\itemsep{-.7mm}
\item size, $\zeta^1(n,p,q) =\cO \left(\frac{1}{x^{p}(1-x)^q} \cdot (p^2+q^2+1)\log n \right)$, 
\item initialization time, $\tau_I^1(n,p,q) = \cO({2^n \choose \zeta(n,p,q)} \cdot \frac{1}{x^p(1-x)^q} \cdot n^{\cO(p+q)})$,
\item $(\chi,p')$-degree for $p'\leq p$, $\Delta^1_{(\chi,p')}(n,p,q) = \cO\left(\frac{1}{x^{p-p'}}\cdot\frac{(p^2+q^2+1)}{(1-x)^q} \cdot \log n\right)$
\item $(\chi,p')$-query time ${Q^1_{(\chi,p')}}(n,p,q) = \cO(\frac{1}{x^{p}(1-x)^q} \cdot n^{\cO(1)})=\cO(2^n n^{\cO(1)})$
\item $(\chi',q')$-degree for $q'\leq q$, $\Delta^1_{(\chi',q')}(n,p,q)=\cO\left(\frac{1}{x^{p}(1-x)^{q-q'}}\cdot(p^2+q^2+1) \cdot \log n\right)$
\item $(\chi',q')$-query time, ${Q^1_{(\chi',q')}}(n,p,q)=\cO(\frac{1}{x^{p}(1-x)^q} \cdot n^{\cO(1)})=\cO(2^n n^{\cO(1)})$
\end{itemize}
We apply Lemma~\ref{lem:twinreduceUniverse} to this construction to get a new construction with the following parameter.
\begin{itemize} \item size, $\zeta^2(n,p,q)=\cO\left(\frac{1}{x^{p}(1-x)^q} \cdot  (p+q)^{\cO(1)} \cdot \log n\right)$ 
\item initialization time, 
\begin{eqnarray*}
\tau_I^2(n,p,q) &=& \cO\left(\tau_I^1\left((p+q)^2,p,q\right) + \zeta^1\left((p+q)^2,p,q\right) \cdot (p+q)^{\cO(1)} \cdot n \log n\right)\\
&=& \cO\left(
\frac{2^{2^{(p+q)^2}}}{x^p(1-x)^q} \cdot (p+q)^{\cO(p+q)} + \left(\frac{1}{x^{p}(1-x)^q} \cdot (p+q)^{\cO(1)} \cdot n \log n \right)\right)\\
&=&\cO\left( \frac{(p+q)^{\cO(p+q)}}{x^{p}(1-x)^q}\left({2^{2^{(p+q)^2}}}+n \log n\right)\right)
\end{eqnarray*}
\item $(\chi,p')$-degree, $\Delta_{(\chi,p')}^2(n,p,q) =\cO\left( \frac{1}{x^{p-p'}{(1-x)^q}} \cdot  (p+q)^{\cO(1)} \cdot \log n\right)$
\item $(\chi,p')$-query time, $Q_{(\chi,p')}^2(n,p,q)=\cO\left(\left(2^{(p+q)^2} + \frac{1}{x^{p-p'}{(1-x)^q}}\right)   (p+q)^{\cO(1)} \cdot \log n\right)$
\item $(\chi',q')$-degree, $\Delta_{(\chi',q')}^2(n,p,q) =\cO\left( \frac{1}{x^{p}(1-x)^{q-q'}} \cdot  (p+q)^{\cO(1)} \cdot \log n\right)$
\item $(\chi,q')$-query time, $Q_{(\chi',q')}^2(n,p,q)=\cO\left(\left(2^{(p+q)^2} + \frac{1}{x^{p}(1-x)^{q-q'}}\right)   (p+q)^{\cO(1)} \cdot \log n\right)$
\end{itemize}
We apply Lemma~\ref{lem:splitSolution} to this construction. Recall that in Lemma~\ref{lem:splitSolution} 
we set $s=\lfloor(\log (p+q))^2 \rfloor$ and $t = \lceil \frac{p+q}{s} \rceil$.
\begin{eqnarray*}
 \zeta^3(n,p,q) &\leq& |\mathscr{P}_t^{n}| \cdot 
\sum_{(p_1,\ldots,p_t) \in {\cal Z}_{s,t}^{p}} \prod_{i \leq t} \zeta^2(n,p_i,s-p_i)\\
&\leq& n^{\cO(t)}\cdot |{\cal Z}_{s,t}^{p}| \cdot \max_{(p_1,\ldots,p_t) \in {\cal Z}_{s,t}^{p}} \prod_{i \leq t} \zeta^2(n,p_i,s-p_i)\\
&\leq& n^{\cO(t)}\cdot (p+q)^{\cO(t)}\cdot \frac{1}{x^{p}(1-x)^{q+s}}\cdot s^{\cO(t)}\cdot (\log n)^{\cO(t)}\\
&\leq& n^{\cO(\frac{p+q}{\log^2(p+q)})}\cdot \frac{1}{x^{p}(1-x)^{q}} \qquad\qquad\quad \left(\mbox{Because} \left(\frac{1}{1-x}\right)^s\in n^{\cO(t)}.\right)\\
\end{eqnarray*}
\begin{eqnarray*}
\tau_I^3(n,p,q) &=&\cO\left(\left(\sum_{\hat{p} \leq s} \tau_I^2(n,\hat{p},s-\hat{p})\right) + \zeta^3(n,p,q) \cdot n^{\cO(1)}\right)\\
&=&\cO\left(\left(\sum_{\hat{p} \leq s} \frac{s^{\cO(s)}}{x^{\hat{p}}(1-x)^{s-\hat{p}}}\left({2^{2^{s^2}}}+n \log n\right)\right) + \zeta^3(n,p,q) \cdot n^{\cO(1)}\right)\\
&=& \cO\left(\frac{(\log(p+q))^{\cO(\log^2(p+q))}}{x^{p}(1-x)^{q}}\left({2^{2^{\log^4(p+q)}}}+n \log n\right) + n^{\cO(\frac{p+q}{\log^2(p+q)})}\cdot \frac{1}{x^{p}(1-x)^{q}}\right)\\
\end{eqnarray*}
\begin{eqnarray*}
\Delta_{(\chi,p')}^3(n,p,q) &\leq& |\mathscr{P}_t^n|\cdot |{\cal Z}_{s,t}^{p}| \cdot 
\max_{\substack{(p_1,\ldots,p_t) \in {\cal Z}_{s,t}^{p} \\ p_1'\leq p_1,\ldots, p_t'\leq p_t \\ p_1'+\ldots+p_t'=p'}} \prod_{i \leq t}  \Delta_{(\chi,p')}^2(n,p_i,s-p_i)\\
&\leq& n^{\cO(t)}\cdot (p+q)^{\cO(t)} \cdot\frac{1}{x^{p-p'}(1-x)^{q+s}}\cdot s^{\cO(t)}\cdot (\log n)^{\cO(t)}\\
&\leq& n^{\cO(\frac{p+q}{\log^2(p+q)})}\cdot \frac{1}{x^{p-p'}(1-x)^{q}}\\
\end{eqnarray*}
\begin{eqnarray*}
\Delta_{(\chi',q')}^3(n,p,q) &\leq& |\mathscr{P}_t^n|\cdot |{\cal Z}_{s,t}^{p}| \cdot 
\max_{\substack{(p_1,\ldots,p_t) \in {\cal Z}_{s,t}^{p} \\ q_1'\leq s-p_1,\ldots, q_t'\leq s-q_t \\ q_1'+\ldots+q_t'=q'}} \prod_{i \leq t}  \Delta_{(\chi',q_i')}^2(n,p_i,s-p_i)\\
&\leq& n^{\cO(t)}\cdot (p+q)^{\cO(t)}\cdot \frac{1}{x^{p}(1-x)^{q+s-q'}}\cdot s^{\cO(t)}\cdot (\log n)^{\cO(t)}\\
&\leq& n^{\cO(\frac{p+q}{\log^2(p+q)})}\cdot \frac{1}{x^{p}(1-x)^{q-q'}}\\
\end{eqnarray*}
\begin{eqnarray*}
Q_{(\chi,p')}^3(n,p,q) &\leq& \cO\left(\Delta_{(\chi,p')}^3(n,p,q) \cdot n^{\cO(1)} + 
|\mathscr{P}_{t}^{n}|\cdot |{\cal Z}_{s,t}^{p}| \cdot t \cdot 
\sum_{\substack{ \hat{p}'\leq \hat{p}\leq s \\ \hat{p}-\hat{p}'\leq p-p' \\ s-\hat{p}\leq q }} Q_{(\chi,\hat{p}')}^2(n,\hat{p},s-\hat{p}) \right)\\
 &\leq& \cO\left(\Delta_{(\chi,p')}^3(n,p,q) \cdot n^{\cO(1)} + n^{\cO(t)} \cdot 
\sum_{\substack{ \hat{p}'\leq \hat{p}\leq s \\ \hat{p}-\hat{p}'\leq p-p' \\ s-\hat{p}\leq q }} \left(2^{s^2}+\frac{1}{x^{\hat{p}-\hat{p}'}(1-x)^{s-\hat{p}}}\right)s^{\cO(1)} \log n \right)\\
 &\leq& \cO\left(\frac{n^{\cO(\frac{p+q}{\log^2(p+q)})}}{x^{p-p'}(1-x)^{q}} + n^{\cO(t)} \cdot s^{\cO(1)}\cdot \log n
\left(2^{s^2}+\frac{1}{x^{p-p'}(1-x)^{q}}\right)  \right)\\
 &\leq& \cO\left(\frac{n^{\cO(\frac{p+q}{\log^2(p+q)})}}{x^{p-p'}(1-x)^{q}}  \right)\\
\end{eqnarray*}
Similar way we can bound $Q_{(\chi',q')}^3$ as,
\begin{eqnarray*}
 Q_{(\chi',q')}^3(n,p,q) &\leq& \cO\left(\frac{n^{\cO(\frac{p+q}{\log^2(p+q)})}}{x^{p}(1-x)^{q-q'}}  \right)
\end{eqnarray*}



We apply Lemma~\ref{lem:twinreduceUniverse} to this construction to get a new construction with the following parameters.
\begin{itemize} \item  size, 
$\zeta^4(n,p,q) \leq 2^{\cO(\frac{p+q}{\log(p+q)})}\cdot \frac{1}{x^{p}(1-x)^{q}} \cdot  (p+q)^{\cO(1)} \cdot \log n$,
 \item  initialization time, 
\begin{eqnarray*}
\tau_I^4(n,p,q) &\leq& \cO\left(\tau_I^3\left((p+q)^2,p,q\right) + \zeta^3\left((p+q)^2,p,q\right) \cdot (p+q)^{\cO(1)} \cdot n \log n\right)\\
&\leq& 2^{2^{\log^4(p+q)}}\cdot\frac{(\log(p+q))^{\cO(\log^2(p+q))}}{x^p(1-x)^q} + \frac{2^{\cO(\frac{p+q}{\log (p+q)})}}{x^p(1-x)^q} \cdot (p+q)^{\cO(1)}n\log n\\
\end{eqnarray*}
\item $(\chi,p')$-degree, 
\begin{eqnarray*}
\Delta_{(\chi,p')}^4(n,p,q) &\leq& \Delta_{(\chi,p')}^3\left((p+q)^2,p,q\right) \cdot  (p+q)^{\cO(1)} \cdot \log n \\
&\leq& \frac{2^{\cO(\frac{p+q}{\log (p+q)})}}{x^{p-p'}(1-x)^q} \cdot  (p+q)^{\cO(1)} \cdot \log n
\end{eqnarray*}
\item $(\chi',q')$-degree, 
\begin{eqnarray*}
\Delta_{(\chi',q')}^4(n,p,q) &\leq& \Delta_{(\chi',q')}^3\left((p+q)^2,p,q\right) \cdot  (p+q)^{\cO(1)} \cdot \log n \\
&\leq& \frac{2^{\cO(\frac{p+q}{\log (p+q)})}}{x^{p}(1-x)^{q-q'}} \cdot  (p+q)^{\cO(1)} \cdot \log n
\end{eqnarray*}
\item $(\chi,p')$-query time, 
\begin{eqnarray*}
Q_{(\chi,p')}^4(n,p,q) &\leq& \cO\left(\left( Q_{(\chi,p')}^3\left((p+q)^2,p,q\right) + \Delta_{(\chi,p')}^3\left((p+q)^2,p,q\right) \right) \cdot (p+q)^{\cO(1)} \cdot \log n\right)\\
&\leq& \frac{2^{\cO(\frac{p+q}{\log (p+q)})}}{x^{p-p'}(1-x)^q}\cdot (p+q)^{\cO(1)}\log n
\end{eqnarray*}
\item $(\chi',q')$-query time, 
\begin{eqnarray*}
Q_{(\chi',q')}^4(n,p,q) 
&\leq& \frac{2^{\cO(\frac{p+q}{\log (p+q)})}}{x^{p}(1-x)^{q-q'}}\cdot (p+q)^{\cO(1)}\log n 
\end{eqnarray*}
\end{itemize}
We apply Lemma~\ref{lem:splitSolution} to this construction by setting $s=\lfloor(\log (p+q))^2 \rfloor$ and 
$t = \lceil \frac{p+q}{s} \rceil$.
\begin{itemize}\item size,
\begin{eqnarray*}
\zeta^5(n,p,q) &\leq& |\mathscr{P}_t^n| \cdot 
\sum_{(p_1,\ldots,p_t) \in {\cal Z}_{s,t}^{p} } \prod_{i \leq t} \zeta^4(n,p_i,s-p_i)\\
&\leq& n^{\cO(t)}\cdot(p+q)^{\cO(t)}\cdot s^{\cO(t)} \cdot 2^{\cO(\frac{st}{\log s})}\cdot (\log n)^{\cO(t)}\cdot \frac{1}{x^p(1-x)^{q+s}}\\ 
&\leq& n^{\cO(\frac{p+q}{\log^2(p+q)})} \cdot 2^{\cO(\frac{p+q}{\log\log(p+q)})} \frac{1}{x^p(1-x)^q}
\end{eqnarray*}
\item initialization time,
\begin{eqnarray*}
\tau_I^5(n,p,q) &\leq& \cO\left(\left(\sum_{\hat{p} \leq s} \tau_I^4(n,\hat{p},s-\hat{p})\right) + \zeta^5(n,p,q) \cdot n^{\cO(1)}\right)\\
&\leq&\cO\left(s\frac{2^{2^{\log^4s}}\cdot (\log s)^{\cO(\log^2s)}}{x^p(1-x)^q}+ \frac{2^{\cO(\frac{s}{\log s})}}{x^p(1-x)^q}\cdot n\log n + n^{\cO(\frac{p+q}{\log^2(p+q)})} \cdot  \frac{2^{\cO(\frac{p+q}{\log\log(p+q)})}}{x^p(1-x)^q}
\right)\\
&\leq&\cO\left(s\frac{2^{2^{\log^4s}}\cdot (\log s)^{\cO(\log^2s)}}{x^p(1-x)^q} + n^{\cO(\frac{p+q}{\log^2(p+q)})} \cdot  \frac{2^{\cO(\frac{p+q}{\log\log(p+q)})}}{x^p(1-x)^q}
\right)\\
&\leq&\cO\left(\frac{2^{2^{\log^4s}}\cdot (s)^{\cO(s)}}{x^p(1-x)^q} + n^{\cO(\frac{p+q}{\log^2(p+q)})} \cdot  \frac{2^{\cO(\frac{p+q}{\log\log(p+q)})}}{x^p(1-x)^q}
\right)\\
&\leq&\cO\left(\frac{2^{2^{(2\log\log(p+q))^4}}\cdot (\log(p+q))^{\cO((\log(p+q))^2)}} {x^p(1-x)^q} 
+ n^{\cO(\frac{p+q}{\log^2(p+q)})} \cdot  \frac{2^{\cO(\frac{p+q}{\log\log(p+q)})}}{x^p(1-x)^q}
\right)
\end{eqnarray*}
 \item $(\chi,p')$-degree, 
\begin{eqnarray*}
\Delta_{(\chi,p')}^5(n,p,q) &\leq& |\mathscr{P}_t^n|\cdot |{\cal Z}_{s,t}^{p}| \cdot  
\max_{\substack{(p_1,\ldots,p_t) \in {\cal Z}_{s,t}^{p} \\ p_1'\leq p_1,\ldots, p_t'\leq p_t \\ p_1'+\ldots+p_t'=p' }} \prod_{i \leq t}  \Delta_{(\chi,p_i')}^4(n,p_i,s-p_i)\\
&\leq& n^{\cO(t)}\cdot (p+q)^{\cO(t)} \cdot \frac{2^{\cO(\frac{st}{\log s})}}{x^{p-p'}(1-x)^{q+s}}\cdot s^{\cO(t)}\cdot (\log n)^{\cO(t)}\\
&\leq& n^{\cO(\frac{p+q}{\log^2(p+q)})} \cdot 2^{\cO(\frac{p+q}{\log\log(p+q)})} \cdot \frac{1}{x^{p-p'}(1-x)^q}
\end{eqnarray*}
\item $(\chi',q')$-degree, 
\begin{eqnarray*}
\Delta_{(\chi',q')}^5(n,p,q) 
&\leq& n^{\cO(\frac{p+q}{\log^2(p+q)})} \cdot 2^{\cO(\frac{p+q}{\log\log(p+q)})} \cdot \frac{1}{x^{p}(1-x)^{q-q'}}
\end{eqnarray*}
\item $(\chi,p')$-query time, 
\begin{eqnarray*}
Q_{(\chi,p')}^5(n,p,q) &\leq& \cO\left(\Delta_{(\chi,p')}^5(n,p,q) \cdot n^{\cO(1)} + |\mathscr{P}_{t}^{n}|\cdot |{\cal Z}_{s,t}^{p}| \cdot   
\max_{\substack{\hat{p}'\leq \hat{p} \leq s}} Q_{(\chi,\hat{p}')}^4(n,\hat{p},s-\hat{p}) \right)\\
&\leq& n^{\cO(\frac{p+q}{\log^2(p+q)})} \cdot 2^{\cO(\frac{p+q}{\log\log(p+q)})} \cdot \frac{1}{x^{p-p'}(1-x)^q}
\end{eqnarray*}
\item $(\chi',q')$-query time,
\begin{eqnarray*}
 Q_{(\chi',q')}^5(n,p,q) 
&\leq& n^{\cO(\frac{p+q}{\log^2(p+q)})} \cdot 2^{\cO(\frac{p+q}{\log\log(p+q)})} \cdot \frac{1}{x^{p}(1-x)^{q-q'}}
\end{eqnarray*}
\end{itemize}
We apply Lemma~\ref{lem:twinreduceUniverse} to this construction to get a new construction with the following parameters.
\begin{itemize} \item size, 
\begin{eqnarray*}
\zeta^6(n,p,q) &\leq& \zeta^5\left((p+q)^2,p,q\right) \cdot  (p+q)^{\cO(1)} \cdot \log n\\
&\leq & 2^{\cO(\frac{p+q}{\log\log(p+q)})}\cdot \frac{(p+q)^{\cO(1)}}{x^p(1-x)^q}\cdot \log n
\end{eqnarray*}
\item initialization time, 
\begin{eqnarray*}
\tau_I^6(n,p,q) &\leq& \cO\left(\tau_I^5\left((p+q)^2,p,q\right) + \zeta^5\left((p+q)^2,p,q\right) \cdot (p+q)^{\cO(1)} \cdot n \log n\right)\\
&=&\cO\left(\frac{2^{2^{(2\log\log(p+q))^4}}\cdot (\log(p+q))^{\cO((\log(p+q))^2)}} {x^p(1-x)^q} + 2^{\cO(\frac{p+q}{\log\log(p+q)})}\cdot \frac{(p+q)^{\cO(1)}}{x^p(1-x)^q}\cdot n\log n \right)
\end{eqnarray*}
\item $(\chi,p')$-degree, 
\begin{eqnarray*}
\Delta_{(\chi,p')}^6(n,p,q) &\leq& \Delta_{(\chi,p')}^5\left((p+q)^2,p,q\right) \cdot  (p+q)^{\cO(1)} \cdot \log n \\
&\leq& \cO\left( 2^{\cO(\frac{p+q}{\log\log(p+q)})} \cdot \frac{1}{x^{p-p'}(1-x)^q} \cdot (p+q)^{\cO(1)} \cdot \log n\right)
\end{eqnarray*}
\item $(\chi,p')$-query time,
\begin{eqnarray*}
Q_{(\chi,p')}^6(n,p,q) &\leq& \cO\left(\left(Q_{(\chi,p')}^5\left((p+q)^2,p,q\right) + \Delta_{(\chi,p')}^5\left((p+q)^2,p,q\right) \right) \cdot (p+q)^{\cO(1)} \cdot \log n\right)\\
&\leq& \cO\left( 2^{\cO(\frac{p+q}{\log\log(p+q)})} \cdot \frac{1}{x^{p-p'}(1-x)^q} \cdot (p+q)^{\cO(1)} \cdot \log n\right)
\end{eqnarray*}
\item $(\chi',q')$-degree, 
\begin{eqnarray*}
\Delta_{(\chi',q')}^6(n,p,q)&=&\Delta_{(\chi',q')}^5\left((p+q)^2,p,q\right) \cdot  (p+q)^{\cO(1)} \cdot \log n\\
&\leq& \cO\left( 2^{\cO(\frac{p+q}{\log\log(p+q)})} \cdot \frac{1}{x^{p}(1-x)^{q-q'}} \cdot (p+q)^{\cO(1)} \cdot \log n\right)
\end{eqnarray*}
\item $(\chi',q')$-query time, 
\begin{eqnarray*}
Q_{(\chi',q')}^6(n,p,q)&=&\cO\left(\left(Q_{(\chi',q')}^5\left((p+q)^2,p,q\right) + \Delta_{(\chi',q')}^5\left((p+q)^2,p,q\right) \right) \cdot (p+q)^{\cO(1)} \cdot \log n\right)\\
&\leq& \cO\left( 2^{\cO(\frac{p+q}{\log\log(p+q)})} \cdot \frac{1}{x^{p}(1-x)^{q-q'}} \cdot (p+q)^{\cO(1)} \cdot \log n\right)
\end{eqnarray*}
\end{itemize}
We apply Lemma~\ref{lem:splitSolution} to this construction by setting $s=\lfloor(\log (p+q))^2 \rfloor$ and 
$t = \lceil \frac{p+q}{s} \rceil$.
\begin{itemize}\item size,
\begin{eqnarray*}
\zeta^7(n,p,q) &\leq& |\mathscr{P}_t^n| \cdot 
\sum_{(p_1,\ldots,p_t) \in {\cal Z}_{s,t}^{p} } \prod_{i \leq t} \zeta^6(n,p_i,s-p_i)\\
&\leq& n^{O(t)}\cdot(p+q)^{\cO(t)}\cdot s^{\cO(t)} \cdot 2^{\cO(\frac{st}{\log\log s})}\cdot (\log n)^{\cO(t)}\cdot \frac{1}{x^p(1-x)^{q+s}}\\ 
&\leq& n^{\cO(\frac{p+q}{\log^2(p+q)})} \cdot 2^{\cO(\frac{p+q}{\log\log\log(p+q)})} \frac{1}{x^p(1-x)^q}
\end{eqnarray*}
\item initialization time,
\begin{eqnarray*}
 \tau_I^7(n,p,q) &\leq& \cO\left(\left(\sum_{\hat{p} \leq s} \tau_I^6(n,\hat{p},s-\hat{p})\right) + \zeta^7(n,p,q) \cdot n^{\cO(1)}\right)\\
&\leq& 2^{2^{(2\log\log (s))^4}}\cdot \frac{(\log s )^{\cO(\log^2(s))}}{x^p(1-x)^q} + n^{\cO(\frac{p+q}{\log^2(p+q)})} \cdot 2^{\cO(\frac{p+q}{\log\log\log(p+q)})} \frac{1}{x^p(1-x)^q}\\
&\leq& n^{\cO(\frac{p+q}{\log^2(p+q)})} \cdot 2^{\cO(\frac{p+q}{\log\log\log(p+q)})} \frac{1}{x^p(1-x)^q}\\
\end{eqnarray*}
($\because 2^{2^{(2\log \log s)^4}} , (\log s )^{\cO(\log^2(s))} \leq 2^{\cO(\frac{p+q}{\log \log\log(p+q)})}$.
This inequality holds because $\log\log 2^{2^{(2\log \log s)^4}}$ is upper bounded by a polynomial function in 
$\log\log\log(p+q)$ where as $\log\log 2^{\cO(\frac{p+q}{\log \log\log(p+q)})}$ is lower bounded by a polynomial 
function in $\log(p+q)$. Similarly $\log (\log s )^{\cO(\log^2(s))}$ is upper bounded by a polynomial function in 
$\log(p+q)$ where as $\log 2^{\cO(\frac{p+q}{\log \log\log(p+q)})}$ is lower bounded by a polynomial function in $(p+q)$)
\item $(\chi,p')$-degree, 
\begin{eqnarray*}
\Delta_{(\chi,p')}^7(n,p,q) &\leq& |\mathscr{P}_t^n|\cdot |{\cal Z}_{s,t}^{p}|\cdot 
\max_{\substack{(p_1,\ldots,p_t) \in {\cal Z}_{s,t}^{p} \\ p_1'\leq p_1,\ldots, p_t'\leq p_t \\ p_1'+\ldots+p_t'=p'  }} \prod_{i \leq t}  \Delta_{(\chi,p_i')}^6(n,p_i,s-p_i)\\
&\leq& n^{\cO(t)}\cdot(p+q)^{\cO(t)}\cdot s^{\cO(t)} \cdot 2^{\cO(\frac{st}{\log\log s})}\cdot (\log n)^{\cO(t)}\cdot \frac{1}{x^{p-p'}(1-x)^{q+s}}\\
&\leq& n^{\cO(\frac{p+q}{\log^2(p+q)})} \cdot 2^{\cO(\frac{p+q}{\log\log\log(p+q)})} \cdot \frac{1}{x^{p-p'}(1-x)^{q}}
\end{eqnarray*}
\item $(\chi',q')$-degree, 
\begin{eqnarray*}
\Delta_{(\chi',q')}^7(n,p,q) 
&\leq& n^{\cO(\frac{p+q}{\log^2(p+q)})} \cdot 2^{\cO(\frac{p+q}{\log\log\log(p+q)})} \cdot \frac{1}{x^{p}(1-x)^{q-q'}}
\end{eqnarray*}
\item $(\chi,p')$-query time, 
\begin{eqnarray*}
Q_{(\chi,p')}^7(n,p,q) &\leq& \cO\left(\Delta_{(\chi,p')}^7(n,p,q) \cdot n^{\cO(1)} + |\mathscr{P}_{t}^{n}|\cdot |{\cal Z}_{s,t}^{p}|\cdot t \cdot 
\max_{\hat{p}'\leq \hat{p} \leq s } Q_{(\chi,\hat{p}')}^6(n,\hat{p},s-\hat{p}) \right)\\
&\leq& n^{\cO(\frac{p+q}{\log^2(p+q)})}\cdot  2^{\cO(\frac{p+q}{\log\log\log(p+q)})} \cdot \frac{1}{x^{p-p'}(1-x)^q}\log n 
\end{eqnarray*}
\item $(\chi',q')$-query time,
\begin{eqnarray*}
 Q_{(\chi',q')}^7(n,p,q) 
&\leq& n^{\cO(\frac{p+q}{\log^2(p+q)})}\cdot  2^{\cO(\frac{p+q}{\log\log\log(p+q)})} \cdot \frac{1}{x^{p}(1-x)^{q-q'}}\log n 
\end{eqnarray*}
\end{itemize}
We apply Lemma~\ref{lem:twinreduceUniverse} to this construction to get a new construction with the following parameters.
\begin{itemize} \item size, 
\begin{eqnarray*}
\zeta^8(n,p,q) &\leq& \zeta^7\left((p+q)^2,p,q\right) \cdot  (p+q)^{\cO(1)} \cdot \log n\\
&\leq & 2^{\cO(\frac{p+q}{\log\log\log(p+q)})}\cdot \frac{1}{x^p(1-x)^q}\cdot (p+q)^{\cO(1)} \cdot \log n
\end{eqnarray*}
\item initialization time, 
\begin{eqnarray*}
\tau_I^8(n,p,q) &\leq& \cO\left(\tau_I^7\left((p+q)^2,p,q\right) + \zeta^7\left((p+q)^2,p,q\right) \cdot (p+q)^{\cO(1)} \cdot n \log n\right)\\
&\leq & 2^{\cO(\frac{p+q}{\log\log\log(p+q)})}\cdot \frac{1}{x^p(1-x)^q}\cdot (p+q)^{\cO(1)} \cdot n\log n
\end{eqnarray*}
\item $(\chi,p')$-degree, 
\begin{eqnarray*}
\Delta_{(\chi,p')}^8(n,p,q) &\leq& \Delta_{(\chi,p')}^7\left((p+q)^2,p,q\right) \cdot  (p+q)^{\cO(1)} \cdot \log n \\
&\leq & 2^{\cO(\frac{p+q}{\log\log\log(p+q)})}\cdot \frac{1}{x^{p-p'}(1-x)^q}\cdot (p+q)^{\cO(1)} \cdot \log n
\end{eqnarray*}
\item $(\chi,p')$-query time,
\begin{eqnarray*}
Q_{(\chi,p')}^8(n,p,q) &\leq& \cO\left(\left(Q_{(\chi,p')}^7\left((p+q)^2,p,q\right) + \Delta_{(\chi,p')}^7\left((p+q)^2,p,q\right) \right) \cdot (p+q)^{\cO(1)} \cdot \log n\right)\\
&\leq & 2^{\cO(\frac{p+q}{\log\log\log(p+q)})}\cdot \frac{1}{x^{p-p'}(1-x)^q}\cdot (p+q)^{\cO(1)} \cdot \log n
\end{eqnarray*}
\item $(\chi',q')$-degree, 
\begin{eqnarray*}
\Delta_{(\chi',q')}^8(n,p,q)&=&\Delta_{(\chi',q')}^7\left((p+q)^2,p,q\right) \cdot  (p+q)^{\cO(1)} \cdot \log n\\
&\leq & 2^{\cO(\frac{p+q}{\log\log\log(p+q)})}\cdot \frac{1}{x^{p}(1-x)^{q-q'}}\cdot (p+q)^{\cO(1)} \cdot \log n
\end{eqnarray*}
\item $(\chi',q')$-query time, 
\begin{eqnarray*}
Q_{(\chi',q')}^8(n,p,q)&=&\cO\left(\left(Q_{(\chi',q')}^7\left((p+q)^2,p,q\right) + \Delta_{(\chi',q')}^7\left((p+q)^2,p,q\right) \right) \cdot (p+q)^{\cO(1)} \cdot \log n\right)\\
&\leq & 2^{\cO(\frac{p+q}{\log\log\log(p+q)})}\cdot \frac{1}{x^{p}(1-x)^{q-q'}}\cdot (p+q)^{\cO(1)} \cdot \log n
\end{eqnarray*}
\end{itemize}
The final construction satisfies all the claimed bounds. This concludes the proof. 
\end{proof}
