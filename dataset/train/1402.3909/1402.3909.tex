



In this section we design a faster algorithm to find -representative family for product families. Our main technical tool is a generalization of {\em ---separating collection} defined in~\cite{FominLS13} to compute  -representative families of an arbitrary family. In fact we design a {\em family} of ---separating collections of various sizes governed by a parameter .  The construction of generalized ---separating collection  is similar to the proof given in~\cite{FominLS13}. However, the new construction  requires some additional ideas and the proof is slightly more involved. Finally, we combine two ---separating collections obtained with different parameters to obtain the desired algorithm for product families. 

\subsection{Generalized ---separating collection}

We start with the formal definition of {\em generalized ---separating collection}.

\begin{definition}
\label{def:twincollection}
A generalized ---separating collection  is a tuple , where  is a family of sets over a universe  of size ,  is a function from 
 to  and  is a function from  to  such that the following properties are satisfied
\begin{enumerate}
 \item for every  and , ,
 \item for every  and , , 
 \item for every pairwise disjoint sets  and  such that , 
.
\end{enumerate}
The size of   is , the -degree of  for  is , 
and the -degree of  for  is . 
\end{definition}




A {\em construction} of generalized separating collections is a data structure, that given ,  and  initializes and outputs a family  of sets over the universe  of size . 
After the initialization one can query the data structure by giving it a set   or , the data structure 
then outputs a family   or  respectively. Together the tuple  computed by the data structure 
should form a {\em generalized} ---{\em separating collection}.

We call the time the data structure takes to initialize and output  the {\em initialization time}. 
The {\em -query time}, , of the data structure is the maximum time the data structure uses to compute  over all . Similarly, the 
{\em -query time}, , of the data structure is the maximum time the data structure uses to compute  over all .
The initialization time of the data structure and the size of  are functions of  ,  and . The initialization time is denoted by 
, size of  is denoted by . The -query time and -degree of 
, , are functions of 
 and is denoted by  and  respectively. Similarly, the -query time and -degree of , ,  are functions of 
 and are denoted by  and  respectively.  
We are now ready to state the main technical  tool of this subsection.


\begin{lemma}
\label{lem:twin_sep_coll_construction}
Given a constant  such that , there is a construction of generalized --- separating collection with the following parameters
\begin{itemize} 
\setlength\itemsep{-.7mm}
\item size, 
\item initialization time, 
\item -degree, 
\item -query time, 
\item -degree, 
\item -query time, 
\end{itemize}
\end{lemma}

We first give a road map to prove Lemma~\ref{lem:twin_sep_coll_construction}. 
The proof of  Lemma~\ref{lem:twin_sep_coll_construction} uses three auxiliary lemmata. 
\begin{enumerate}
\item[(a.)] {\bf Existential Proof (Lemma~\ref{lem:twin_sep_coll_brute_force}}). This lemma shows that there is indeed a 
generalized ---separating collection with the required sizes, degrees and query time. Essentially, it shows that if we form a family   of sets of  such that each  is a random subset of  where each element  is inserted into  with probability , then  has the desired sizes, degrees and query time. Thus, this also gives a brute force algorithm to design the family  by just guessing the family of desired size and then checking whether it is indeed  a generalized ---separating collection. 
\item[(b.)]  {\bf Universe Reduction (Lemma~\ref{lem:twinreduceUniverse}).} The construction obtained in Lemma~\ref{lem:twin_sep_coll_brute_force} 
has only one drawback that the initialization time is much larger than claimed in Lemma~\ref{lem:twin_sep_coll_construction}. To overcome this lacuna, we do not apply the construction in Lemma~\ref{lem:twin_sep_coll_brute_force} directly. 
We first prove a Lemma~\ref{lem:twinreduceUniverse} which helps us in reducing the universe size to . This is done using the  known construction of -perfect hash families of size . 
Lemma~\ref{lem:twinreduceUniverse} alone  can not reduce the universe size sufficiently, that we can apply the construction of Lemma~\ref{lem:twin_sep_coll_brute_force}. 
\item[(c.)] {\bf Splitting Lemma (Lemma~\ref{lem:splitSolution}).} We give a splitter type construction in Lemma~\ref{lem:splitSolution} that when applied with 
Lemma~\ref{lem:twinreduceUniverse} makes the universe and other parameters small enough that we can apply the construction given in 
Lemma~\ref{lem:twin_sep_coll_brute_force}. In this construction we consider all the ``consecutive partitions''  of the universe into  parts, assume that the sets , , are distributed uniformly into  parts and then use this information to obtain a construction of generalized separating collections in each part and then take the product of these collections to obtain a collection for the original instance.
\end{enumerate}

We start with an existential proof. 


\begin{lemma}\label{lem:twin_sep_coll_brute_force}
Given , there is a construction of generalized ---separating collections with 
\begin{itemize}\setlength\itemsep{-.7mm}
\item size , 
\item initialization time ,
\item -degree for , 
\item -query time 
\item -degree 
\item -query time 
\end{itemize}
\end{lemma}

\begin{proof}
We start by giving a randomized algorithm that with positive probability constructs a generalized ---separating collection  with the desired size and degree parameters. 
We will then discuss how to deterministically compute such a  within the required time bound. Set  and construct the family 
 as follows. Each set  is a random subset of , where each element of  is inserted into  with probability . Distinct elements are inserted (or not) into  
independently, and the construction of the different sets in  is also independent. For each   we set  and 
for each  we set .

The size of  is within the required bound by construction. We now argue that with positive probability 
 is indeed a generalized ---separating collection, and that the degrees of  
is within the required bounds as well. For fixed sets , , and integer , we consider the probability that  and . 
This probability is . Since each  is constructed independently from the other sets in , the probability that {\em no}  satisfies  and  is

For a fixed  and  (choices in condition ), the probability that no  in  is equal to the 
probability that no  in   (since  contains all 
the sets in  that contains  and  contains all the sets in  that are disjoint from ). 
Hence the probability that condition  fails is upper bounded by 

where  is the number of choices for  and  in condition . We upper bound  as follows.
There are  choices for  and  choices for . 
For each choice of  there are at most  choices of making  with some of them being empty as well. Note that . 
Therefore the number of possible choices of sets  and  in condition  is upper bounded by . 
Hence the probability that condition  in Definition~\ref{def:twincollection} fails is at most .

We also need to upper bound the maximum degree of . For every ,  is a random variable. For a fixed  and  the probability 
that  is exactly . Hence  is the sum of  independent  -random variables that each take value  with probability . Hence the expected value of  is 
 
For every ,  is also a random variable. For a fixed  and  the probability that  is exactly . 
Hence the expected value of  is,

Standard Chernoff bounds~\cite[Theorem 4.4]{mitzenmacher2005probability} show that the probability that for any ,  
is at least  is upper bounded by . 
Similarly the probability that for any ,  is at least  is upper bounded by .  
There are   choices for  and  choices for . 
Hence the union bound yields that the probability that there exists an  such that  or there exists  such 
that  is upper bounded by . Thus  is a family of ---separating collections with the desired size and degree parameters with probability at least 
. The degenerate case that   is handled 
by the family  containing all (at most four) subsets of . 

To construct  within the stated initialization time bound, it is sufficient to try all families  of size  and for each of the  
guesses, test whether it is indeed a family of ---separating collections in time .

For the queries, we need to give an algorithm that given , computes  (or ), under the assumption that  has already has been computed in the initialization step. 
This is easily done within the stated running time bound by going through every set , checking whether  (or ), and if so, inserting  into  (). 
This concludes the proof.
\end{proof}
We will now work towards improving the time bounds of Lemma~\ref{lem:twin_sep_coll_brute_force}. 
 To that end we will need a construction of {\em -perfect hash functions} by Alon et al.~\cite{AlonYZ}
\begin{definition} 
A family of functions  from a universe  of size  to a universe of size  is a -perfect family of hash functions if for every set  such that  
there exists an  such that the restriction of  to  is injective.
\end{definition}
Alon et al.~\cite{AlonYZ} give very efficient constructions of -perfect families of hash functions from a universe of size  to a universe of size .
\begin{proposition}[\cite{AlonYZ}]\label{prop:hashFun} 
For any universe  of size  there is a -perfect family  of hash functions from  to 
 
with . 
Such a family of hash functions can be constructed in time . 
\end{proposition}

\begin{lemma}\label{lem:twinreduceUniverse} If there is a construction of generalized ---separating collections  with initialization time , size , 
-query time , -query time , 
-degree , and -degree   
then there is a construction of generalized ---separating collections 
with following parameters.
\begin{itemize} \item 
,
\item 
,
\item 
, 
\item 
,
\item 
,
\item 

\end{itemize}
\end{lemma}
\begin{proof}
We give a construction of generalized ---separating collections with initialization time, query time, size and degree , ,  and  respectively using the 
construction with initialization time, query time, size and degree , ,  and  as a black box. 
 
We first describe the initialization of the data structure. Given , , and , we construct using Proposition~\ref{prop:hashFun} a -perfect family  
of hash functions from the universe  to . The construction takes time  and . 
We will store these hash functions in memory. We use the following notations.
\begin{itemize}
 \item For a set  and ,  \\  and . 
 \item For a family  of sets over  and family  of sets over ,\\  and .
\end{itemize}


 We first use the given black box construction for ---separating collections  over the universe . 
We run the initialization algorithm of this construction and store the family  in memory. We then set


 We spent  time to construct a -perfect family of hash functions,   to construct  of size , 
and  time to construct  from  and the family of perfect hash functions. 
Thus the upper bound on  follows. Furthermore,  , yielding the claimed bound for .
 
 We now define  for every  and describe the query algorithm. For every  we let

Since , , it follows that  for every . Furthermore we can bound  for any , 
as follows 

Thus the claimed bound for  follows. 
Similar way we define  for every  as 


To compute  for any , we go over every  and check whether  is injective on . This takes time . 
For each  such that  is injective on , we compute  and then  in time . Then we compute   in time 
 and add this set to . As we need to do this  times, the total time 
to compute  is upper bounded by , 
yielding the claimed upper bound on . Similar way we can bound .

It remains to argue that  is in fact a generalized ---separating collection. For any , consider pairwise disjoint sets , and 
 such that . We need to show that . 
Since  is a -perfect family of hash functions, there is an  such that  is injective on . 
Since  is a ---separating collection,  
. Since  is injective on  and , . 
This concludes the proof.
\end{proof}
We now give a {\em splitting lemma}, which allows us to reduce the problem of finding generalized ---separating collections to the same problem, but with much smaller values for  and . To that end we need some definitions.


\begin{definition}
A {\em partition} of  is a family  of sets over  such that  and . 
Each of the sets  are called the {\em parts} of the partition. A {\em consecutive partition} of  is a partition  of  
such that for every integer  and integers , if  and  then  as well. 
\end{definition}
\begin{proposition}
Let  denote the collection of all consecutive partitions of  with exactly  parts. Let 
 be the set of all -tuples  of integers such that  and  for all . 
Then for every ,  and .
\end{proposition}
\begin{lemma}\label{lem:splitSolution} For any ,  let  and 
. If there is a construction of generalized ---separating collections  
with initialization time , query times  and , 
producing a generalized ---separating collection with size , -degree 
 and -degree  then there is a construction 
of generalized ---separating collection with following parameters 
\begin{itemize}\item 
,
\item 
,
\item 
,
\item 
,
\item 

\item 

\end{itemize}
\end{lemma}
\begin{proof}
Set ,  and . We will give 
a construction of generalized ---separating collections with initialization time, query time, size and degree 
within the claimed bounds above. In the construction we will be using the construction with initialization time 
, query times  and , size , and degrees  
and  as a black box. Since  , a ---separating collection 
is also a ---separating collection. We may assume without loss of generality that .
 
Our algorithm runs  for every , the initialization of  the given construction of  generalized 
---separating collections. We will refer by 
 to the generalized separating collection constructed for 
. For each  the initialization of the construction outputs the family . 
 
We need to define a few operations on families of sets. For families of sets  ,  over  
and subset  we define

We now define  as follows.

It follows directly from the definition of   that  is within the claimed bound for 
. For the initialization time, the algorithm spends 
 time to initialize the constructions 
of the generalized ---separating collections for all  together. Now the 
algorithm can output the entries of  one set at a time by using~\eqref{eqn:defineFSplit}, spending 
 time per output set. Hence the time bound for  follows. 

For every set  we define  as follows.
p_1,\ldots,p_t) \in {\cal Z}_{s,t}^{p}~\mbox{\scriptsize such that}\\ \forall U_i~:~|U_i \cap A| \leq p_i}} 
\Big[({\chi}_{p_1}(A \cap U_1) \sqcap U_1) \circ ({\chi}_{p_2}(A \cap U_2) \sqcap U_2) \circ \ldots \\
\nonumber ... \circ ({\chi}_{p_t}(A \cap U_t) \sqcap U_t)\Big] 
\label{eqn:defineChiprimeSplit} 
\chi'(B) = \bigcup_{\substack{\{U_1,\ldots,U_{t}\} \in \mathscr{P}_{t}^n\
Similar to the proof of , we can show that . 
It follows directly from the definition of   and  that  and  is within the 
claimed bound for  and  respectively. We now describe how 
queries  can be answered, and analyze how much time it takes. Given  we will compute  using  ~\eqref{eqn:defineChiSplit}. Let . 
For each  and  such that  for all , 
we proceed as follows. First we compute  for each , spending in total 
 time. Now we add each set in 
 
to , spending  time per set that is added to , yielding the bound below, 
p_1,\ldots,p_t) \in {\cal Z}_{s,t}^{p}~\mbox{\scriptsize such that}\\ \forall U_i~:~p_i' = |U_i \cap A| \leq p_i}} \big[\sum_{i \leq t} Q_{\chi_{p_i},p_i'}(n,p_i,s-p_i)\big]\Big) \\
\leq \cO\Big(\Delta'_{(\chi,p')}(n,p,q) \cdot n^{\cO(1)} + |\mathscr{P}_{t}^n|\cdot |{\cal Z}_{s,t}^p| \cdot 
\max_{\substack{(p_1,\ldots,p_t) \in {\cal Z}_{s,t}^p\\ p_1'\leq p_1,\cdots,p_t'\leq p_t~\mbox{\scriptsize such that}\\p_1'+\cdots +p_t'=p'}} \big( \sum_{i \leq t} Q_{(\chi_{p_i},p_i')}(n,p_i,s-p_i)\big)  \Big) \\
\leq \cO\Big(\Delta'_{(\chi,p')}(n,p,q) \cdot n^{\cO(1)} + |\mathscr{P}_{t}^n|\cdot |{\cal Z}_{s,t}^p| \cdot t\cdot 
\big( \sum_{\substack{ \hat{p}'\leq \hat{p}\leq s \\ \hat{p}-\hat{p}'\leq p-p' \\ s-\hat{p}\leq q }} Q_{(\chi_{\hat{p}},\hat{p}')}(n,\hat{p},s-\hat{p})\big)  \Big) \\

\tau_I^2(n,p,q) &=& \cO\left(\tau_I^1\left((p+q)^2,p,q\right) + \zeta^1\left((p+q)^2,p,q\right) \cdot (p+q)^{\cO(1)} \cdot n \log n\right)\\
&=& \cO\left(
\frac{2^{2^{(p+q)^2}}}{x^p(1-x)^q} \cdot (p+q)^{\cO(p+q)} + \left(\frac{1}{x^{p}(1-x)^q} \cdot (p+q)^{\cO(1)} \cdot n \log n \right)\right)\\
&=&\cO\left( \frac{(p+q)^{\cO(p+q)}}{x^{p}(1-x)^q}\left({2^{2^{(p+q)^2}}}+n \log n\right)\right)

 \zeta^3(n,p,q) &\leq& |\mathscr{P}_t^{n}| \cdot 
\sum_{(p_1,\ldots,p_t) \in {\cal Z}_{s,t}^{p}} \prod_{i \leq t} \zeta^2(n,p_i,s-p_i)\\
&\leq& n^{\cO(t)}\cdot |{\cal Z}_{s,t}^{p}| \cdot \max_{(p_1,\ldots,p_t) \in {\cal Z}_{s,t}^{p}} \prod_{i \leq t} \zeta^2(n,p_i,s-p_i)\\
&\leq& n^{\cO(t)}\cdot (p+q)^{\cO(t)}\cdot \frac{1}{x^{p}(1-x)^{q+s}}\cdot s^{\cO(t)}\cdot (\log n)^{\cO(t)}\\
&\leq& n^{\cO(\frac{p+q}{\log^2(p+q)})}\cdot \frac{1}{x^{p}(1-x)^{q}} \qquad\qquad\quad \left(\mbox{Because} \left(\frac{1}{1-x}\right)^s\in n^{\cO(t)}.\right)\\

\tau_I^3(n,p,q) &=&\cO\left(\left(\sum_{\hat{p} \leq s} \tau_I^2(n,\hat{p},s-\hat{p})\right) + \zeta^3(n,p,q) \cdot n^{\cO(1)}\right)\\
&=&\cO\left(\left(\sum_{\hat{p} \leq s} \frac{s^{\cO(s)}}{x^{\hat{p}}(1-x)^{s-\hat{p}}}\left({2^{2^{s^2}}}+n \log n\right)\right) + \zeta^3(n,p,q) \cdot n^{\cO(1)}\right)\\
&=& \cO\left(\frac{(\log(p+q))^{\cO(\log^2(p+q))}}{x^{p}(1-x)^{q}}\left({2^{2^{\log^4(p+q)}}}+n \log n\right) + n^{\cO(\frac{p+q}{\log^2(p+q)})}\cdot \frac{1}{x^{p}(1-x)^{q}}\right)\\

\Delta_{(\chi,p')}^3(n,p,q) &\leq& |\mathscr{P}_t^n|\cdot |{\cal Z}_{s,t}^{p}| \cdot 
\max_{\substack{(p_1,\ldots,p_t) \in {\cal Z}_{s,t}^{p} \\ p_1'\leq p_1,\ldots, p_t'\leq p_t \\ p_1'+\ldots+p_t'=p'}} \prod_{i \leq t}  \Delta_{(\chi,p')}^2(n,p_i,s-p_i)\\
&\leq& n^{\cO(t)}\cdot (p+q)^{\cO(t)} \cdot\frac{1}{x^{p-p'}(1-x)^{q+s}}\cdot s^{\cO(t)}\cdot (\log n)^{\cO(t)}\\
&\leq& n^{\cO(\frac{p+q}{\log^2(p+q)})}\cdot \frac{1}{x^{p-p'}(1-x)^{q}}\\

\Delta_{(\chi',q')}^3(n,p,q) &\leq& |\mathscr{P}_t^n|\cdot |{\cal Z}_{s,t}^{p}| \cdot 
\max_{\substack{(p_1,\ldots,p_t) \in {\cal Z}_{s,t}^{p} \\ q_1'\leq s-p_1,\ldots, q_t'\leq s-q_t \\ q_1'+\ldots+q_t'=q'}} \prod_{i \leq t}  \Delta_{(\chi',q_i')}^2(n,p_i,s-p_i)\\
&\leq& n^{\cO(t)}\cdot (p+q)^{\cO(t)}\cdot \frac{1}{x^{p}(1-x)^{q+s-q'}}\cdot s^{\cO(t)}\cdot (\log n)^{\cO(t)}\\
&\leq& n^{\cO(\frac{p+q}{\log^2(p+q)})}\cdot \frac{1}{x^{p}(1-x)^{q-q'}}\\

Q_{(\chi,p')}^3(n,p,q) &\leq& \cO\left(\Delta_{(\chi,p')}^3(n,p,q) \cdot n^{\cO(1)} + 
|\mathscr{P}_{t}^{n}|\cdot |{\cal Z}_{s,t}^{p}| \cdot t \cdot 
\sum_{\substack{ \hat{p}'\leq \hat{p}\leq s \\ \hat{p}-\hat{p}'\leq p-p' \\ s-\hat{p}\leq q }} Q_{(\chi,\hat{p}')}^2(n,\hat{p},s-\hat{p}) \right)\\
 &\leq& \cO\left(\Delta_{(\chi,p')}^3(n,p,q) \cdot n^{\cO(1)} + n^{\cO(t)} \cdot 
\sum_{\substack{ \hat{p}'\leq \hat{p}\leq s \\ \hat{p}-\hat{p}'\leq p-p' \\ s-\hat{p}\leq q }} \left(2^{s^2}+\frac{1}{x^{\hat{p}-\hat{p}'}(1-x)^{s-\hat{p}}}\right)s^{\cO(1)} \log n \right)\\
 &\leq& \cO\left(\frac{n^{\cO(\frac{p+q}{\log^2(p+q)})}}{x^{p-p'}(1-x)^{q}} + n^{\cO(t)} \cdot s^{\cO(1)}\cdot \log n
\left(2^{s^2}+\frac{1}{x^{p-p'}(1-x)^{q}}\right)  \right)\\
 &\leq& \cO\left(\frac{n^{\cO(\frac{p+q}{\log^2(p+q)})}}{x^{p-p'}(1-x)^{q}}  \right)\\

 Q_{(\chi',q')}^3(n,p,q) &\leq& \cO\left(\frac{n^{\cO(\frac{p+q}{\log^2(p+q)})}}{x^{p}(1-x)^{q-q'}}  \right)

\tau_I^4(n,p,q) &\leq& \cO\left(\tau_I^3\left((p+q)^2,p,q\right) + \zeta^3\left((p+q)^2,p,q\right) \cdot (p+q)^{\cO(1)} \cdot n \log n\right)\\
&\leq& 2^{2^{\log^4(p+q)}}\cdot\frac{(\log(p+q))^{\cO(\log^2(p+q))}}{x^p(1-x)^q} + \frac{2^{\cO(\frac{p+q}{\log (p+q)})}}{x^p(1-x)^q} \cdot (p+q)^{\cO(1)}n\log n\\

\Delta_{(\chi,p')}^4(n,p,q) &\leq& \Delta_{(\chi,p')}^3\left((p+q)^2,p,q\right) \cdot  (p+q)^{\cO(1)} \cdot \log n \\
&\leq& \frac{2^{\cO(\frac{p+q}{\log (p+q)})}}{x^{p-p'}(1-x)^q} \cdot  (p+q)^{\cO(1)} \cdot \log n

\Delta_{(\chi',q')}^4(n,p,q) &\leq& \Delta_{(\chi',q')}^3\left((p+q)^2,p,q\right) \cdot  (p+q)^{\cO(1)} \cdot \log n \\
&\leq& \frac{2^{\cO(\frac{p+q}{\log (p+q)})}}{x^{p}(1-x)^{q-q'}} \cdot  (p+q)^{\cO(1)} \cdot \log n

Q_{(\chi,p')}^4(n,p,q) &\leq& \cO\left(\left( Q_{(\chi,p')}^3\left((p+q)^2,p,q\right) + \Delta_{(\chi,p')}^3\left((p+q)^2,p,q\right) \right) \cdot (p+q)^{\cO(1)} \cdot \log n\right)\\
&\leq& \frac{2^{\cO(\frac{p+q}{\log (p+q)})}}{x^{p-p'}(1-x)^q}\cdot (p+q)^{\cO(1)}\log n

Q_{(\chi',q')}^4(n,p,q) 
&\leq& \frac{2^{\cO(\frac{p+q}{\log (p+q)})}}{x^{p}(1-x)^{q-q'}}\cdot (p+q)^{\cO(1)}\log n 

\zeta^5(n,p,q) &\leq& |\mathscr{P}_t^n| \cdot 
\sum_{(p_1,\ldots,p_t) \in {\cal Z}_{s,t}^{p} } \prod_{i \leq t} \zeta^4(n,p_i,s-p_i)\\
&\leq& n^{\cO(t)}\cdot(p+q)^{\cO(t)}\cdot s^{\cO(t)} \cdot 2^{\cO(\frac{st}{\log s})}\cdot (\log n)^{\cO(t)}\cdot \frac{1}{x^p(1-x)^{q+s}}\\ 
&\leq& n^{\cO(\frac{p+q}{\log^2(p+q)})} \cdot 2^{\cO(\frac{p+q}{\log\log(p+q)})} \frac{1}{x^p(1-x)^q}

\tau_I^5(n,p,q) &\leq& \cO\left(\left(\sum_{\hat{p} \leq s} \tau_I^4(n,\hat{p},s-\hat{p})\right) + \zeta^5(n,p,q) \cdot n^{\cO(1)}\right)\\
&\leq&\cO\left(s\frac{2^{2^{\log^4s}}\cdot (\log s)^{\cO(\log^2s)}}{x^p(1-x)^q}+ \frac{2^{\cO(\frac{s}{\log s})}}{x^p(1-x)^q}\cdot n\log n + n^{\cO(\frac{p+q}{\log^2(p+q)})} \cdot  \frac{2^{\cO(\frac{p+q}{\log\log(p+q)})}}{x^p(1-x)^q}
\right)\\
&\leq&\cO\left(s\frac{2^{2^{\log^4s}}\cdot (\log s)^{\cO(\log^2s)}}{x^p(1-x)^q} + n^{\cO(\frac{p+q}{\log^2(p+q)})} \cdot  \frac{2^{\cO(\frac{p+q}{\log\log(p+q)})}}{x^p(1-x)^q}
\right)\\
&\leq&\cO\left(\frac{2^{2^{\log^4s}}\cdot (s)^{\cO(s)}}{x^p(1-x)^q} + n^{\cO(\frac{p+q}{\log^2(p+q)})} \cdot  \frac{2^{\cO(\frac{p+q}{\log\log(p+q)})}}{x^p(1-x)^q}
\right)\\
&\leq&\cO\left(\frac{2^{2^{(2\log\log(p+q))^4}}\cdot (\log(p+q))^{\cO((\log(p+q))^2)}} {x^p(1-x)^q} 
+ n^{\cO(\frac{p+q}{\log^2(p+q)})} \cdot  \frac{2^{\cO(\frac{p+q}{\log\log(p+q)})}}{x^p(1-x)^q}
\right)

\Delta_{(\chi,p')}^5(n,p,q) &\leq& |\mathscr{P}_t^n|\cdot |{\cal Z}_{s,t}^{p}| \cdot  
\max_{\substack{(p_1,\ldots,p_t) \in {\cal Z}_{s,t}^{p} \\ p_1'\leq p_1,\ldots, p_t'\leq p_t \\ p_1'+\ldots+p_t'=p' }} \prod_{i \leq t}  \Delta_{(\chi,p_i')}^4(n,p_i,s-p_i)\\
&\leq& n^{\cO(t)}\cdot (p+q)^{\cO(t)} \cdot \frac{2^{\cO(\frac{st}{\log s})}}{x^{p-p'}(1-x)^{q+s}}\cdot s^{\cO(t)}\cdot (\log n)^{\cO(t)}\\
&\leq& n^{\cO(\frac{p+q}{\log^2(p+q)})} \cdot 2^{\cO(\frac{p+q}{\log\log(p+q)})} \cdot \frac{1}{x^{p-p'}(1-x)^q}

\Delta_{(\chi',q')}^5(n,p,q) 
&\leq& n^{\cO(\frac{p+q}{\log^2(p+q)})} \cdot 2^{\cO(\frac{p+q}{\log\log(p+q)})} \cdot \frac{1}{x^{p}(1-x)^{q-q'}}

Q_{(\chi,p')}^5(n,p,q) &\leq& \cO\left(\Delta_{(\chi,p')}^5(n,p,q) \cdot n^{\cO(1)} + |\mathscr{P}_{t}^{n}|\cdot |{\cal Z}_{s,t}^{p}| \cdot   
\max_{\substack{\hat{p}'\leq \hat{p} \leq s}} Q_{(\chi,\hat{p}')}^4(n,\hat{p},s-\hat{p}) \right)\\
&\leq& n^{\cO(\frac{p+q}{\log^2(p+q)})} \cdot 2^{\cO(\frac{p+q}{\log\log(p+q)})} \cdot \frac{1}{x^{p-p'}(1-x)^q}

 Q_{(\chi',q')}^5(n,p,q) 
&\leq& n^{\cO(\frac{p+q}{\log^2(p+q)})} \cdot 2^{\cO(\frac{p+q}{\log\log(p+q)})} \cdot \frac{1}{x^{p}(1-x)^{q-q'}}

\zeta^6(n,p,q) &\leq& \zeta^5\left((p+q)^2,p,q\right) \cdot  (p+q)^{\cO(1)} \cdot \log n\\
&\leq & 2^{\cO(\frac{p+q}{\log\log(p+q)})}\cdot \frac{(p+q)^{\cO(1)}}{x^p(1-x)^q}\cdot \log n

\tau_I^6(n,p,q) &\leq& \cO\left(\tau_I^5\left((p+q)^2,p,q\right) + \zeta^5\left((p+q)^2,p,q\right) \cdot (p+q)^{\cO(1)} \cdot n \log n\right)\\
&=&\cO\left(\frac{2^{2^{(2\log\log(p+q))^4}}\cdot (\log(p+q))^{\cO((\log(p+q))^2)}} {x^p(1-x)^q} + 2^{\cO(\frac{p+q}{\log\log(p+q)})}\cdot \frac{(p+q)^{\cO(1)}}{x^p(1-x)^q}\cdot n\log n \right)

\Delta_{(\chi,p')}^6(n,p,q) &\leq& \Delta_{(\chi,p')}^5\left((p+q)^2,p,q\right) \cdot  (p+q)^{\cO(1)} \cdot \log n \\
&\leq& \cO\left( 2^{\cO(\frac{p+q}{\log\log(p+q)})} \cdot \frac{1}{x^{p-p'}(1-x)^q} \cdot (p+q)^{\cO(1)} \cdot \log n\right)

Q_{(\chi,p')}^6(n,p,q) &\leq& \cO\left(\left(Q_{(\chi,p')}^5\left((p+q)^2,p,q\right) + \Delta_{(\chi,p')}^5\left((p+q)^2,p,q\right) \right) \cdot (p+q)^{\cO(1)} \cdot \log n\right)\\
&\leq& \cO\left( 2^{\cO(\frac{p+q}{\log\log(p+q)})} \cdot \frac{1}{x^{p-p'}(1-x)^q} \cdot (p+q)^{\cO(1)} \cdot \log n\right)

\Delta_{(\chi',q')}^6(n,p,q)&=&\Delta_{(\chi',q')}^5\left((p+q)^2,p,q\right) \cdot  (p+q)^{\cO(1)} \cdot \log n\\
&\leq& \cO\left( 2^{\cO(\frac{p+q}{\log\log(p+q)})} \cdot \frac{1}{x^{p}(1-x)^{q-q'}} \cdot (p+q)^{\cO(1)} \cdot \log n\right)

Q_{(\chi',q')}^6(n,p,q)&=&\cO\left(\left(Q_{(\chi',q')}^5\left((p+q)^2,p,q\right) + \Delta_{(\chi',q')}^5\left((p+q)^2,p,q\right) \right) \cdot (p+q)^{\cO(1)} \cdot \log n\right)\\
&\leq& \cO\left( 2^{\cO(\frac{p+q}{\log\log(p+q)})} \cdot \frac{1}{x^{p}(1-x)^{q-q'}} \cdot (p+q)^{\cO(1)} \cdot \log n\right)

\zeta^7(n,p,q) &\leq& |\mathscr{P}_t^n| \cdot 
\sum_{(p_1,\ldots,p_t) \in {\cal Z}_{s,t}^{p} } \prod_{i \leq t} \zeta^6(n,p_i,s-p_i)\\
&\leq& n^{O(t)}\cdot(p+q)^{\cO(t)}\cdot s^{\cO(t)} \cdot 2^{\cO(\frac{st}{\log\log s})}\cdot (\log n)^{\cO(t)}\cdot \frac{1}{x^p(1-x)^{q+s}}\\ 
&\leq& n^{\cO(\frac{p+q}{\log^2(p+q)})} \cdot 2^{\cO(\frac{p+q}{\log\log\log(p+q)})} \frac{1}{x^p(1-x)^q}

 \tau_I^7(n,p,q) &\leq& \cO\left(\left(\sum_{\hat{p} \leq s} \tau_I^6(n,\hat{p},s-\hat{p})\right) + \zeta^7(n,p,q) \cdot n^{\cO(1)}\right)\\
&\leq& 2^{2^{(2\log\log (s))^4}}\cdot \frac{(\log s )^{\cO(\log^2(s))}}{x^p(1-x)^q} + n^{\cO(\frac{p+q}{\log^2(p+q)})} \cdot 2^{\cO(\frac{p+q}{\log\log\log(p+q)})} \frac{1}{x^p(1-x)^q}\\
&\leq& n^{\cO(\frac{p+q}{\log^2(p+q)})} \cdot 2^{\cO(\frac{p+q}{\log\log\log(p+q)})} \frac{1}{x^p(1-x)^q}\\

\Delta_{(\chi,p')}^7(n,p,q) &\leq& |\mathscr{P}_t^n|\cdot |{\cal Z}_{s,t}^{p}|\cdot 
\max_{\substack{(p_1,\ldots,p_t) \in {\cal Z}_{s,t}^{p} \\ p_1'\leq p_1,\ldots, p_t'\leq p_t \\ p_1'+\ldots+p_t'=p'  }} \prod_{i \leq t}  \Delta_{(\chi,p_i')}^6(n,p_i,s-p_i)\\
&\leq& n^{\cO(t)}\cdot(p+q)^{\cO(t)}\cdot s^{\cO(t)} \cdot 2^{\cO(\frac{st}{\log\log s})}\cdot (\log n)^{\cO(t)}\cdot \frac{1}{x^{p-p'}(1-x)^{q+s}}\\
&\leq& n^{\cO(\frac{p+q}{\log^2(p+q)})} \cdot 2^{\cO(\frac{p+q}{\log\log\log(p+q)})} \cdot \frac{1}{x^{p-p'}(1-x)^{q}}

\Delta_{(\chi',q')}^7(n,p,q) 
&\leq& n^{\cO(\frac{p+q}{\log^2(p+q)})} \cdot 2^{\cO(\frac{p+q}{\log\log\log(p+q)})} \cdot \frac{1}{x^{p}(1-x)^{q-q'}}

Q_{(\chi,p')}^7(n,p,q) &\leq& \cO\left(\Delta_{(\chi,p')}^7(n,p,q) \cdot n^{\cO(1)} + |\mathscr{P}_{t}^{n}|\cdot |{\cal Z}_{s,t}^{p}|\cdot t \cdot 
\max_{\hat{p}'\leq \hat{p} \leq s } Q_{(\chi,\hat{p}')}^6(n,\hat{p},s-\hat{p}) \right)\\
&\leq& n^{\cO(\frac{p+q}{\log^2(p+q)})}\cdot  2^{\cO(\frac{p+q}{\log\log\log(p+q)})} \cdot \frac{1}{x^{p-p'}(1-x)^q}\log n 

 Q_{(\chi',q')}^7(n,p,q) 
&\leq& n^{\cO(\frac{p+q}{\log^2(p+q)})}\cdot  2^{\cO(\frac{p+q}{\log\log\log(p+q)})} \cdot \frac{1}{x^{p}(1-x)^{q-q'}}\log n 

\zeta^8(n,p,q) &\leq& \zeta^7\left((p+q)^2,p,q\right) \cdot  (p+q)^{\cO(1)} \cdot \log n\\
&\leq & 2^{\cO(\frac{p+q}{\log\log\log(p+q)})}\cdot \frac{1}{x^p(1-x)^q}\cdot (p+q)^{\cO(1)} \cdot \log n

\tau_I^8(n,p,q) &\leq& \cO\left(\tau_I^7\left((p+q)^2,p,q\right) + \zeta^7\left((p+q)^2,p,q\right) \cdot (p+q)^{\cO(1)} \cdot n \log n\right)\\
&\leq & 2^{\cO(\frac{p+q}{\log\log\log(p+q)})}\cdot \frac{1}{x^p(1-x)^q}\cdot (p+q)^{\cO(1)} \cdot n\log n

\Delta_{(\chi,p')}^8(n,p,q) &\leq& \Delta_{(\chi,p')}^7\left((p+q)^2,p,q\right) \cdot  (p+q)^{\cO(1)} \cdot \log n \\
&\leq & 2^{\cO(\frac{p+q}{\log\log\log(p+q)})}\cdot \frac{1}{x^{p-p'}(1-x)^q}\cdot (p+q)^{\cO(1)} \cdot \log n

Q_{(\chi,p')}^8(n,p,q) &\leq& \cO\left(\left(Q_{(\chi,p')}^7\left((p+q)^2,p,q\right) + \Delta_{(\chi,p')}^7\left((p+q)^2,p,q\right) \right) \cdot (p+q)^{\cO(1)} \cdot \log n\right)\\
&\leq & 2^{\cO(\frac{p+q}{\log\log\log(p+q)})}\cdot \frac{1}{x^{p-p'}(1-x)^q}\cdot (p+q)^{\cO(1)} \cdot \log n

\Delta_{(\chi',q')}^8(n,p,q)&=&\Delta_{(\chi',q')}^7\left((p+q)^2,p,q\right) \cdot  (p+q)^{\cO(1)} \cdot \log n\\
&\leq & 2^{\cO(\frac{p+q}{\log\log\log(p+q)})}\cdot \frac{1}{x^{p}(1-x)^{q-q'}}\cdot (p+q)^{\cO(1)} \cdot \log n

Q_{(\chi',q')}^8(n,p,q)&=&\cO\left(\left(Q_{(\chi',q')}^7\left((p+q)^2,p,q\right) + \Delta_{(\chi',q')}^7\left((p+q)^2,p,q\right) \right) \cdot (p+q)^{\cO(1)} \cdot \log n\right)\\
&\leq & 2^{\cO(\frac{p+q}{\log\log\log(p+q)})}\cdot \frac{1}{x^{p}(1-x)^{q-q'}}\cdot (p+q)^{\cO(1)} \cdot \log n

\end{itemize}
The final construction satisfies all the claimed bounds. This concludes the proof. 
\end{proof}
