\documentclass{article}

\usepackage[numbers]{natbib}
\usepackage[final]{neurips_2021}
\usepackage[utf8]{inputenc} \usepackage[T1]{fontenc}    \usepackage{hyperref}       \usepackage{url}            \usepackage{booktabs}       \usepackage{amsfonts}       \usepackage{nicefrac}       \usepackage{microtype}      \usepackage{xcolor}         \usepackage{amsmath}
\usepackage{subfig}
\usepackage{graphicx}
\usepackage{titlesec}
\titlespacing*{\section}{0pt}{0.25\baselineskip}{0.25\baselineskip}
\titlespacing*{\subsection}{0pt}{0.2\baselineskip}{0.2\baselineskip}

\title{Real-Time Monitoring of User Stress, Heart Rate, \\ and Heart Rate Variability on Mobile Devices}

\author{Peyman Bateni and Leonid Sigal \\
  Beam AI, University of British Columbia, Vector Institute, CIFAR AI Chair  \\
  Correspondence to \texttt{pbateni@beamhealth.ai} \\
}

\begin{document}

\maketitle

\vspace{-0.15in}
\begin{abstract}
Stress is considered to be the epidemic of the 21st-century \cite{Fink2016_StressEpidemic}. Yet, mobile apps cannot directly evaluate the impact of their content and services on user stress. We introduce the Beam AI SDK to address this issue. Using our SDK, apps can monitor user stress through the selfie camera in real-time. Our technology extracts the user’s pulse wave by analyzing subtle color variations across the skin regions of the user’s face. The user’s pulse wave is then used to determine stress (according to the Baevsky Stress Index), heart rate, and heart rate variability. We evaluate our technology on the UBFC dataset, the MMSE-HR dataset, and Beam AI's internal data. Our technology achieves 99.2\%, 97.8\% and 98.5\% accuracy for heart rate estimation on each benchmark respectively, a nearly twice lower error rate than competing methods. We further demonstrate an average Pearson correlation of 0.801 in determining stress and heart rate variability, thus producing commercially useful readings to derive content decisions in apps. Our SDK is available for use\footnote{Visit \url{beamhealth.ai} for more information. The Beam AI SDK (iOS) files can be accessed at \url{github.com/beamai/BeamAISDK-iOS}. SDK access keys can be obtained by signing up at \url{dashboard.beamhealth.ai}.}.
\end{abstract}

\begin{figure}[!b]
    \centering
    \includegraphics[width=\textwidth]{figures/SDKOverview.png}
    \caption{{\bf Overview of the Beam AI SDK.} First, the user’s pulse wave is extracted by processing subtle color variations across the skin regions of the user’s face. The user’s pulse is then processed by the SDK's proprietary peak detection algorithm which produces the inter-beat intervals used to determine the user's stress (according to Baevsky Stress Index), heart rate and heart rate variability.}
    \label{fig:sdk-overview}
\end{figure}

\section{Introduction}

It is estimated that over \\downarrow\downarrow\downarrow\rho\uparrowtt\{\text{IBI}_i\}\{\text{IBI}_i\}\text{mod}\text{amp}3.92 * \text{SDNN}(\{\text{IBI}_i\})\downarrow\downarrow\downarrow\rho\uparrow\{\hat{y}_i\}\{y_i\}\{\hat{y_i}\}\{y_i\}\downarrow\downarrow\downarrow\rho\uparrow\downarrow\downarrow\downarrow\rho\uparrow$ \\
        \midrule
        Heart Rate {\small (Beats Per Minute)} & 1.046 & 1.48\% & 1.471 & 0.959 \\
        Heart Rate Variability {\small (SDNN ms)} & 12.003 & 13.52\% & 16.444 & 0.781 \\ 
        Stress {\small (Baevsky Stress Index)} & 0.171 & 31.61\% & 0.243 & 0.850 \\
    \end{tabular}
    \vspace{0.05in}
    \caption{Evaluating stress, heart rate, and heart rate variability estimation from the subject's face on Beam AI's internal data. Ground truth readings are provided by a Polar H10 chest.}
    \label{tab:beam-ai-internal-benchmark}
    \vspace{-0.2in}
\end{table}

\textbf{Benchmark:} We continue to use the UBFC \cite{Bobbia17_UBFC} video dataset for this section. Please refer to Section \ref{sec:ubfc} for details on the video dataset and the synchronized pulse wave signal.

\textbf{Ground Truth Signal:} When comparing to prior literature in Section \ref{sec:ubfc}, we follow the procedure of Liu et al. \cite{EfficientPhys_Liu2021} for generating ground truth heart rate readings to assure consistency in comparison. However, after direct examination, it's clear that there are limited but non-zero instances where the standard peak detector of Scipy \cite{2020SciPy-NMeth} generates false pulse peaks. This prompted us to manually examine the pulse wave signal for every video in the UBFC benchmark and identify the peaks by hand \footnote{This was completed by Peyman Bateni without official direct medical training to perform peak detection.}. This ensures that they are accurately localized and can be used for stress and HRV evaluation. We then employ these hand-verified peaks to extract inter-beat intervals that are then used to calculate ground truth heart rate (beats per minute), heart rate variability (SDNN ms), and stress (according to the Baevsky Stress Index).

\textbf{Predictions by the Beam AI SDK:} For each video in the benchmark, the Python implementation of the ``Pulse Processor'' in our SDK estimates a high-quality pulse wave signal. We then generate the inter-beat intervals using an equivalent Python implementation of the  ``Inter-Beat Interval Processor'' from the SDK. The output inter-beat intervals are then directly used to estimate heart rate, heart rate variability, and stress for each video using the metric definitions inside the ``Biometric Estimator''.

\textbf{Results:} As demonstrated in Table \ref{tab:ubfc-eval-with-manual-data}, we achieve an MAE of 0.318 beats per minute (MAPE of 0.32\%) on heart rate estimation, achieving near-perfect pulse estimation on the majority of videos in the benchmark. Furthermore, we achieve an MAE of 11.125 ms (MAPE 20.26\%) with a high correlation score of 0.841, demonstrating the ability to produce commercially useful heart rate variability estimates that strongly correlate with the increases and decreases in the ground truth heart rate variability. Lastly, we achieve an MAE of 0.973 (MAPE of 44.43\%) on stress estimation. Despite a comparatively larger error rate, we demonstrate a strong correlation with ground truth readings, achieving a Pearson correlation score of 0.730.

\subsection{Evaluation on Beam AI's Internal Data}
\label{sec:iphone-data}

\begin{figure}[t]
    \centering
    \subfloat[heart rate Estimation over 60s Windows]{{\includegraphics[width=0.88\textwidth]{figures/pulse.pdf} }} \\
    \vspace{-0.1in}
    \subfloat[HRV Estimation over 60s Windows]{{\includegraphics[width=0.88\textwidth]{figures/hrv.pdf} }} \\
    \vspace{-0.1in}
    \subfloat[Stress Estimation over 60s Windows]{{\includegraphics[width=0.88\textwidth]{figures/stress.pdf} }}
    \\
    \vspace{-0.1in}
    \subfloat[Heart Rate Estimation Compared to Apple Watch over 20s Windows]{{\includegraphics[width=0.9\textwidth]{figures/applewatch.pdf} }}
    \caption{Continuous stress, heart rate and heart rate variability monitoring on Beam AI's internal data consisting of a 20-minute passive recording of Peyman Bateni on an iPhone 13 device. A sliding 60-second time window is used to estimate each value. Note that there is a 10-second minimum window for estimating values (shown in dark grey) and values before the 60s mark do not have a complete 60-second window (shown in light grey). In (d), we compare heart rate estimation over shorter 20-second windows to provide a comparison to measurements from a series 7 Apple Watch.}
    \label{fig:beam-ai-data}
    \vspace{-0.2in}
\end{figure}

\textbf{Benchmark:} We further evaluate the Beam AI SDK directly on Beam AI's internal data consisting of a 20-minute recording of a single subject (Peyman Bateni) on an iPhone 13 device. In the first half of the recording, the subject was holding the iPhone 13 device in a natural pose and using the device. In the second half, the device was placed next to the computer at the subject's workstation where he then proceeded to work on the computer for the remainder of the recording. The entire recording encompasses 19 minutes and 20 seconds and covers a range of movements when sitting and changes in lighting due to the large monitor in front of the subject.

\textbf{Ground Truth Signal:} A Polar H10 chest strap is worn by the subject during the recording from which, the subject's pulse in the form of inter-beat intervals is recorded using the EliteHRV iOS app \cite{elite_hrv_2022}. The inter-beat interval data is synchronized according to the recording's per frame time stamps and used to estimate stress, heart rate and heart rate variability over a moving 60s window according to their respective formulas from Section \ref{sec:technology}. Note that the 60s window is not fully complete during the first minute of the recording where naturally a 60s window is not available. As a result, there are no estimates for the first 10s and, in the 50s thereafter, we use the largest window size available (i.e. 10s window at 10th second, 30s window as 30th second, etc.).

\textbf{Predictions by the Beam AI SDK:} A pulse wave signal was extracted and recorded by the Beam AI SDK during the recording. This signal was saved and then post-processed using a Python implementation of our ``Inter-Beat Interval Processor'' module, resulting in a set of inter-beat intervals that are similarly grouped over 60s windows, with the exception of the first minute, where the same adaptive window strategy is used for measurements between the 10s and the 60s mark. The inter-beat intervals over each window are then subsequently used to produce continuous readings for the subject's heart rate, heart rate variability, and stress.

\textbf{Results:} We report overall performance on Beam AI's internal data in Table \ref{tab:beam-ai-internal-benchmark} and also provide graphs of continuous estimates over time for heart rate, heart rate variability and stress in Figure \ref{fig:beam-ai-data}. As shown, our technology achieves strong results, with an MAE of 1.046 beats per minute, 12.003 ms, and 0.171 for heart rate, heart rate variability and stress respectively. Furthermore, as indicated by the Pearson correlations achieved and shown in Figure \ref{fig:beam-ai-data}, we are able to estimate values that strongly correlate with the gold-standard ground truth from the Polar H10 chest-strap monitoring device.

\textbf{Comparison to Apple Watch:} We further evaluate our technology accuracy as compared to a series 7 Apple Watch that was worn simultaneously during the recording and used to extract heart rate values at every 5s intervals. These values were then interpolated to produce per-second heart rate estimates. Unfortunately, inter-beat intervals and continuously updated heart rate variability estimates are not available for third-party usage on the Apple Watch, and accordingly we cannot compare those. The results are shown in Figure \ref{fig:beam-ai-data}-d. Here, we reduce the window size for our estimates from 60s to the 20s window that is believed to be used by the Apple Watch. As shown, we are able to produce heart rate estimates that strongly correlate with the Apple Watch measurements. Overall, we achieve an MAE of 1.959 beats per minute whereas the Apple Watch achieves an MAE of 1.399 beats per minute when compared to the Polar H10 device. This indicates that we are approximately 0.6 beats per minute less accurate than the Apple Watch on average during seated phone usage.

\begin{table}[t]
    \centering
    \tabcolsep=1.67cm
    \begin{tabular}{lc}
        Model & Time To Process 1 Frame \\
        \midrule
        Beam AI & \textbf{0.5 ms} \\
        \midrule
        EfficientPhys-T1 \cite{EfficientPhys_Liu2021} & 30.0 ms \\
        TS-CAN \cite{TSCAN_Liu2020} & 6.0 ms \\
        EfficientPhys-C \cite{EfficientPhys_Liu2021} & 4.0 ms \\
        POS \cite{POS_Wang2016} & 2.7 ms \\
        CHROM \cite{CHROM_DeHaan2013} & 2.8 ms \\
        ICA \cite{ICA_Poh2011} & 3.1 ms \\
    \end{tabular}
    \vspace{0.05in}
    \caption{Processing speed on devices. Note that the Beam AI SDK was evaluated on an iPhone 13 device while baselines are reference run-times on an ARM CPU \cite{EfficientPhys_Liu2021, TSCAN_Liu2020}.}
    \label{tab:processing-speed}
    \vspace{-0.2in}
\end{table}

\subsection{Processing Speed}
\label{sec:processing-speed}

Inference on mobile devices is best done on the device as it preserves user privacy, can operate in real-time and reduces the rate of frame loss. However, this requires very efficient models to be able to run real-time processing, especially at high framerates. Table \ref{tab:processing-speed} compares the processing speed of the Beam AI SDK with competing methods. As shown, the Beam AI SDK takes 0.5 ms to process one frame, a near 6x improvement over the fastest competing methods. This enables the Beam AI SDK to run smoothly at 120fps while using limited computational resources on the device.

\section{Discussion}
\label{sec:discussion}

We introduce the Beam AI SDK to enable smartphone apps to monitor user stress in real-time. We provide two sample apps (Beam AI Lite and Beam AI Browser) on App Store to demonstrate some applications of real-time stress monitoring inside apps. We further establish the empirical efficacy of the Beam AI SDK by validating the underlying technology on UBFC \cite{Bobbia17_UBFC}, MMSE-HR \cite{Zhang16_MMSE} and Beam AI's internal data. We demonstrate nearly twice better accuracy as compared to competing methods while running up to six times faster on mobile devices.

\textbf{Acknowledgements:} We have conducted experiments with publicly available datasets and privately collected data. Our experiments are consistent with best published practices in the academic domain \cite{TSCAN_Liu2020, EfficientPhys_Liu2021, MetaPhys_Liu2021, ICA_Poh2011, POS_Wang2016}. However, we have not conducted medical grade testing with strict medical studies and guidelines to validate our measurements. For this reason, we cannot make any claims on medical reliability of our measurements or their relevance for any sort of medically-relevant diagnostics. This is something we will explore for future iterations of our application and deployments. In addition, this is why we have clear messaging whenever a recording is in progress in our demo apps that ``Beam AI is not medically approved and should not be used for health decisions'' as shown in Figure \ref{fig:screenshots}. We strongly recommend interested developers maintain the necessary disclaimer messaging when using the Beam AI SDK for applications that are intended for or can be mistaken for medical usage.

\textbf{Future Studies:} We are undertaking an extensive empirical study with a large set of participants in Vancouver. This will extensively evaluate our technology during diverse phone usage (such as video replay, gaming, texting, emailing, browsing, and social networking) in different lighting and motion settings. We will report these results publicly once the study completes.

\textbf{Future Directions:} The Beam AI SDK is currently available on iOS only. We will expand support to other mobile operating systems, cross-platform development frameworks, and desktop operating systems in the future. We are also developing improved core technologies for a more robust extraction of the user pulse wave in noisy environments.

\bibliographystyle{plainnat}
\bibliography{main}

\end{document}