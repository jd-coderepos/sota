\documentclass{article}
\usepackage{latexsym,amssymb,amsmath}
\usepackage[pdftex]{graphicx}
\usepackage{pifont}
\newtheorem{lemma}{Lemma}
\newtheorem{proposition}{Proposition}
\newtheorem{definition}{Definition}
\newtheorem{theorem}{Theorem}
\newtheorem{corollary}{Corollary}
\newcommand{\acro}[1]{\ensuremath{\mathcal{#1}}}
\newcommand{\agents}{N}
\newcommand{\gr}{Gr}
\newcommand{\img}{\mathbf{image}}
\newcommand{\themap}{\mathbf{f}}
\newcommand{\tuple}[1]{\langle #1\rangle}
\newcommand{\coal}[1]{[#1]}
\newcommand{\todo}[1]{\textbf{TODO:} #1}
\newenvironment{proof}[1][Proof]{\begin{trivlist}
\item[\hskip \labelsep {\bfseries #1}]}{\unskip\nobreak\hfil\penalty50
   \hskip2em\hbox{}\nobreak\hfil
   \ding{111}\parfillskip=0pt \finalhyphendemerits=0
    \medskip\goodbreak\noindent\ignorespacesafterend\end{trivlist}}
\begin{document}
\title{Completeness of Epistemic Coalition Logic with Group Knowledge} 
\author{Thomas {\AA}gotnes and Natasha Alechina}
\date{\today}

\maketitle

\begin{abstract}
  Coalition logic is one of the most popular logics for multi-agent
  systems. While epistemic extensions of coalition logic have received
  much attention, existence of their complete axiomatisations
  has so far been an open problem. In this paper we settle several of
  those problems. We prove completeness for epistemic coalition logic
  with common knowledge, with distributed knowledge, and with both
  common and distributed knowledge, respectively.
\end{abstract}


\section{Introduction}

Logics of coalitional ability such as \emph{Coalition Logic}
(\acro{CL}) \cite{pauly:2002a}, \emph{Alternating-time Temporal Logic}
(\acro{ATL}) \cite{Alur2002}, and STiT logics \cite{Belnap88stit}, are
arguably some of the most studied logics in multi-agent systems in
recent years. Many different variants of these logics have been
proposed and studied, but so far meta-logical results have focused
more on computational expressiveness and expressive power and less on
completeness, with Goranko's and van Drimmelen's completeness proof
for \acro{ATL} \cite{goranko:2006a}, Pauly's completeness proof for
\acro{CL} \cite{pauly:2002a} and Broersen and colleagues' completeness
proofs for different variants of STiT logic
\cite{Broersen//:07a,BroersenDALT2008,BroersenHerzigTroquardJANCL2009}
being notable exceptions.

The main construction in coalitional ability logics is of the form
, where  is a set of agents and  a formula,
intuitively meaning that  is \emph{effective} for , or that
 can make  come true no matter what the other agents do.  One
of the most studied extension of basic coalitional ability logics is
adding \emph{knowledge} operators of the type found in \emph{epistemic
  logic} \cite{Fagin1995,Meyer1995}: both \emph{individual} knowledge
operators  where  is an agent, and different types of
\emph{group} knowledge operators ,  and  where  is a
group of agents, standing for everybody-knows, common knowledge and
distributed knowledge, respectively. Combining coalitional ability
operators and epistemic operators in general and group knowledge
operators in particular lets us express many potentially interesting
properties of multi-agent systems, such as \cite{Hoek2003a}:
\begin{itemize}
\item :  can communicate her
  knowledge of  to 
\item : common knowledge in  of
   is sufficient for  to ensure that 
\item : distributed knowledge in  of
   is necessary for  to ensure that 
\item :  can cooperate to make
  distributed knowledge explicit
\end{itemize}

In this paper we study a complete axiomatisation of variants of
\emph{epistemic coalition logic} (\acro{ECL}), extensions of coalition
logic with individual knowledge and different combinations of common
knowledge and distributed knowledge. Coalition logic, the next-time
fragment of \acro{ATL}, is one of the most studied coalitional ability
logics, and this paper settles a key open problem: completeness of its
epistemic variants.

While epistemic coalitional ability logics have been studied to a
great extent, we are not aware of any published completeness results
for such logics with all epistemic operators. \cite{Hoek2003a} gives
some axioms of \acro{ATEL}, \acro{ATL} extended with epistemic
operators, but does not attempt to prove completeness\footnote{In an
  abstract of a talk given at the LOFT workshop in 2004
  \cite{goranko:2004}, the authors propose a full axiomatisation of
  \acro{ATEL} with individual knowledge and common knowledge operators
  with similar axioms. However, neither a completeness proof nor the result
  itself was published, and a proof indeed does not exist as explained
  to us in personal communication (Valentin Goranko).}.  Broersen and
colleagues \cite{BroersenDALT2008,BroersenHerzigTroquardJANCL2009}
prove completeness of variants of STiT logic that include individual
knowledge operators, but not group knowledge operators, and
\cite{BroersenHerzigTroquardJANCL2009} concludes that adding group
operators is an important challenge.

The rest of the paper is organised as follows. In the next section we
first give a brief review of coalition logic, and how it is extended
with epistemic operators. We then, in each of the three following
sections, consider basic epistemic coalition logic with individual
knowledge operators extended with common knowledge, with distributed
knowledge, and with both common and distributed knowledge,
respectively. For each of these cases we show a completeness
result. For the common knowledge case we also show a filtration
result. We conclude in Section \ref{sec:conclusions}.


\section{Background}

We define several extensions of propositional logic, and the usual
derived connectives, such as  for , will be used.

\subsection{Coalition Logic}

Assume a set  of atomic propositions, and a finite set
 of agents. A \emph{coalition} is a set 
of agents. We sometimes abuse notation and write a singleton coalition
 as .

The language of coalition logic (\acro{CL}) is defined by the
following grammar:

where  and .

A \emph{coalition model} is a tuple

where 
\begin{itemize}
\item  is a non-empty set of \emph{states};
\item  is a \emph{valuation function}, assigning a set  to each state ;
\item  assigns a \emph{truly playable effectivity function}  to each state
.
\end{itemize}
An \emph{effectivity function} \cite{pauly:2002a} over  and a
set of states  is a function  that maps any coalition  to a set of sets of states . An
effectivity function is \emph{truly playable}
\cite{pauly:2002a,goranko:2011} iff it satisfies the following
conditions:
\begin{description}
\item[E1]  (Liveness)
\item[E2]  (Safety)
\item[E3]  (-maximality)
\item[E4]  (outcome monotonicity)
\item[E5]   and 
, where  (superadditivity)
\item[E6] , where  is
the \emph{non-monotonic core} of the empty coalition, namely

\end{description}
An effectivity function that only satisfies E1-E5 is called
\emph{playable}. On finite domains an effectivity function is playable
iff it is truly playable \cite{goranko:2011}, because on finite domains E6
follows from E1-E5.

An \acro{CL} formula is interpreted in a state in a coalition model as follows:
\begin{description}
\item[]  iff 
\item[]  iff 
\item[]  iff 
\item[]  iff 
\end{description}
where .

Figure \ref{fig:cl-ax} shows an axiomatisation  of coalition logic
which is sound and complete wrt. all coalition models
\cite{pauly:2002a}. The following \emph{monotonicity rule} is
derivable, and will be useful later: .
\begin{figure}[h]
  \centering
\begin{description}
\item[Prop] Classical propositional logic
\item[G1] 
\item[G2] 
\item[G3] 
\item[G4] 
\item[G5] , if 
\item[MP] 
\item[RG] 
\end{description}
  \caption{: axiomatisation of \acro{CL}.}
  \label{fig:cl-ax}
\end{figure}


\subsection{Adding Knowledge Operators}

Epistemic extensions of coalition logic were first proposed in
\cite{Hoek2003a}\footnote{In that paper for \acro{ATL}; \acro{CL} is a
  fragment of \acro{ATL}.}. They are obtained by extending the
language with \emph{epistemic operators}, and the models with
\emph{epistemic accessibility relations}.

An epistemic accessibility relation for agent  over a set of states
 is a binary relation . We will assume
that epistemic accessibility relations are equivalence relations. An
\emph{epistemic coalition model}, henceforth often called simply a
\emph{model}, is a tuple
 where  is a
coalition model and  is an epistemic accessibility relation
over  for each agent .

Epistemic operators come in two types: individual knowledge operators
, where  is an agent, and group knowledge operators  and
 where  is a coalition for expressing \emph{common knowledge}
and \emph{distributed knowledge}, respectively. Formally, the language
of \acro{CLCD} (\emph{coalition logic with common and distributed
  knowledge}), is defined by extending coalition logic with all of
these operators:

where , ,  and
. When  is a coalition, we
write  as a shorthand for 
(everyone in  knows ).

The languages of the logics \acro{CLK}, \acro{CLC} and \acro{CLD} are
the restrictions of this language with no  and no 
operators, no  operators, and no  operators, respectively.

The interpretation of these languages in an (epistemic coalition)
model  is defined by adding the following clauses to the definition
for \acro{CL}:
\begin{description}
\item[]  iff 
\item[]  iff 
\item[]  iff 
\end{description}
where  denotes the transitive closure of the relation .  We
use  to denote the fact that  is \emph{valid},
i.e., that  for all  and states  in .

\subsubsection{Some Auxiliary Definitions}

The following are some auxiliary concepts that will be useful in the
following.

A \emph{pseudomodel} is a tuple  where  is a model and:
\begin{itemize}
\item  is an equivalence relation for each 
\item For any , 
\item For any , ,  implies that 
\end{itemize}
The interpretation of a \acro{CLCD} formula in a state of a
pseudomodel is defined as for a model, except for the case for 
which is interpreted by the  relation:
\begin{description}
\item[]  iff 
\end{description}

An \emph{epistemic model} is a model without the  function, i.e., a
tuple . An \emph{epistemic
  pseudomodel} is a pseudomodel without the  function, i.e., a
tuple .

Finally, a \emph{playable (pseudo)model} is a (pseudo)model where only 
conditions E1-E5 on  hold.


\section{Coalition Logic with Common Knowledge}
\label{sec:clc}

In this section we consider the logic \acro{CLC}, extending coalition
logic with individual knowledge operators and common knowledge. We
first prove a completeness result, and then show that \acro{CLC}
admits filtrations.

\subsection{Completeness}

The axiomatisation  is shown in Figure \ref{fig:clc-ax}. It extends
 with standard axioms and rules for individual and common knowledge
(see, e.g., \cite{Fagin1995}).

\begin{figure}[h]
  \centering
\begin{description}
\item[Prop] Classical propositional logic
\item[G1] 
\item[G2] 
\item[G3] 
\item[G4] 
\item[G5] , if 
\item[MP] 
\item[RG] 
\item[K] 
\item[T] 
\item[4] 
\item[5] 
\item[C1] 
\item[C2] 
\item[RN] 
\item[RC] 
\end{description}
  \caption{: axiomatisation of \acro{CLC}.}
  \label{fig:clc-ax}
\end{figure}

It is easy to show that  is sound wrt. all models.
\begin{lemma}[Soundness]
  For any -formula , .
\end{lemma}
In the remainder of this section we show that  also is complete.

\begin{theorem}
\label{th:clc}
Any -consistent formula is satisfied in some model.
\end{theorem}
\begin{proof}
  We define a canonical playable model  as follows:
\begin{description}
\item  is the set of all maximally  consistent sets of formulas
\item  iff 
  \item  iff 
\item :  iff 
\end{description}
The conditions on  (that it is an equivalence relation) and on  (that it satisfies E1-E5)
hold in . The proof for  is obvious and the proof for  is standard.
The intuition of cause is that a formula belongs to a state
 in a model iff it is true there (truth lemma).  However, the
canonical model is in general \emph{not} guaranteed to satisfy
every consistent formula in the \acro{CLC} language; the case of  in the
truth lemma does not necessarily hold.  Therefore we are going to
transform  by filtration into a finite model for a given  consistent formula .
Note that since  is consistent, it will belong to at least one
 in .

Let  be the set of subformulas of  closed under single
negations and the condition that  for all .  We are going to filtrate  through
. The resulting model  is constructed as follows:
\begin{description}
\item  is  where . We will omit the subscript  in what follows for
readability.
\item  iff 
\item  (where ).
Again we will omit the subscript for readability.
\item  iff  
where  and  is
a conjunction of all formulas in .
\end{description}

We now prove by induction on the size of  that for every ,  iff .

\begin{description}
\item[case ] trivial
\item[case booleans] trivial

\item[case]  
assume . The latter means there is a 
such that  and . By the inductive hypothesis
. Since  is deductively closed wrt  and ,
also .  means that  and  contain the same  formulas from , 
hence .


Assume . Then for all 
such that , . This means by the IH that  for all .
Assume by contradiction that . Then , where  is the conjunction of all formulas
in , is consistent with . If we write  for the dual of the  modality, this is
equivalent to:  is consistent. By forcing choices, 
 is consistent. By the distributivity of  over
, 
 is consistent. So for some  with ,
 is consistent. We claim that . If this is the case, we have a contradiction,
since we assumed that  for all .

Proof of the claim: if  is consistent, then . Suppose not ,
that is there is a formula  such that  and  or vice versa. Then we have  is consistent, but since  is an S5 modality, this is impossible.
Same for the case when  and .

\item[case] 

   iff  iff
   iff (by the IH)  iff(*) 
  iff(**)  iff (since ) .

  Proof of (*): assume  contains  states,
   contain  and 
  contain . Clearly,  is provably
  equivalent to . Consider . It
  is provably equivalent to .  Since for every  such that
  ,  is provably
  equivalent to ,

is provably equivalent to 

which in turn is provably equivalent to

which in turn is equivalent to  hence to . So in , . 

Proof of (**): since we defined  to hold iff
, it suffices to show the
case that .  The direction to the left is immediate:
if  then 
by definition. For the other direction assume that , i.e., there is some  such that
 and . It is easy to see that  implies that
, and by the monotonicity rule it
follows that .


\item[case]  
The proof is similar to \cite{hvdetal.del:2007}. First we show that in , if , then 
iff every state on every  path from  contains .

Suppose . The proof is by induction on the length of the path. If the path is of 0 length, then clearly by deductive closure
and by  we have . We also have  by the assumption.
IH: if , then every state on every  path of length  from  contains  \emph{and }.
Inductive step: let us prove this for paths of length . Suppose we have a path . By the IH, . Since  is deductively closed and 
, we have . Since 
and the definition of ,  and hence by reflexivity .

For the other direction, suppose that every state on every  path from  contains . Prove that
. Let  be the set of all  such that  every state on every  path from  
contains . Note that each  is/corresponds to a finite set of formulas so we can write its conjunction .
Consider a formula 

Similarly to \cite{hvdetal.del:2007} it can be proved that , 
and . And from that follows that 
hence .

Now we prove that  iff .  iff every
state on every  path from  contains  iff 
for every  reachable from  by a  path,  iff . 
 
\end{description}
\end{proof}

It is obvious that in ,  are equivalence relations. So
what remains to be proved is that  satisfies E1-E6. Since 
is finite, it suffices to show E1-E5, which for finite sets of states
entail E6.

\begin{proposition}
 satisfies E1-E5.
\end{proposition}
\begin{proof}
\begin{description}
\item[E1] Note that  is an empty disjunction, namely .

 iff (by definition of )
 iff . Since 
 satisfies , .

\item[E2]  iff 
 iff 
. Since  satisfies ,
.

\item[E3] Let . Then . Note that
 is the complement of , since .
Since  satisfies E3, this means that . Hence .
 
\item[E4] Let  and . Clearly . 
Hence . Since  , we have .
Since  satisfies E4,   so .


\item[E5]  Let  and  and . 
So  and  and since  satisfies E5,
. Note that 

which is in turn the same as

since , . 
\end{description}
\end{proof}

\begin{corollary}
  For any -formula ,   iff .
\end{corollary}


\section{Completeness of Coalition Logic with Distributed Knowledge}
\label{sec:cld}

In this section we consider the logic \acro{CLD}, extending coalition
logic with individual knowledge operators and distributed knowledge.

The axiomatisation  is shown in Figure \ref{fig:cld-ax}. It extends
 with standard axioms and rules for individual and distributed
knowledge (see, e.g., \cite{Fagin1995}).

\begin{figure}[h]
  \centering
\begin{description}
\item[Prop] Classical propositional logic
\item[G1] 
\item[G2] 
\item[G3] 
\item[G4] 
\item[G5] , if 
\item[MP] 
\item[RG] 
\item[K] 
\item[T] 
\item[4] 
\item[5] 
\item[RN] 
\item[DK] 
\item[DT] 
\item[D4] 
\item[D5] 
\item[D1] 
\item[D2] , if 
\end{description}
  \caption{: axiomatisation of \acro{CLD}.}
  \label{fig:cld-ax}
\end{figure}

As usual, soundness can easily be shown.
\begin{lemma}[Soundness]
  For any -formula , .
\end{lemma}
In the remainder of this section we show that  also is complete.

For a set of formulae , let  and .

\begin{definition}[Canonical Playable Pseudomodel]
  The canonical playable pseudomodel  for \acro{CLD} is defined as
  follows:
  \begin{itemize}
  \item  is the set of maximal consistent sets.
  \item  iff 
  \item  iff  whenever 
  \item 
  \item  iff 
  \end{itemize}
\end{definition}

\begin{lemma}[Pseudo Truth Lemma]
  .
\end{lemma}
\begin{proof}
  The proof is by induction on . The epistemic cases are exactly
  as for standard normal modal logic. The case for coalition operators
  is exactly as in \cite{pauly:2002a}.
\end{proof}

It is easy to check that  are equivalence relations and E1-E5 hold
for .

\begin{lemma}[Finite Pseudomodel]\label{lemma:cld}
Every -consistent formula  has a finite pseudomodel where E1-E6
hold.
\end{lemma}
\begin{proof}
The proof is exactly as in Theorem~\ref{th:clc}, namely the
construction of , but starting with a Canonical Playable Pseudomodel
rather than Canonical Playable Model; the definition of 
contains the clause
 

We add the following condition to the closure:  iff .

We define  to be a pseudomodel instead of a model, by adding the clause:
    

We show that  is indeed a pseudomodel:
    \begin{itemize}
    \item : this follows from the fact that  iff  for any  and ,
      which holds because of the  axiom
      and the new closure condition above.
   \item : this holds
      by definition.
\end{itemize}

We add a case for  to the inductive
proof. This case is proven in exactly the same way as the  case: the definitions of  and  are of
exactly the same form (in particular,  is also an 
modality). The proof that E1-E6 hold in the resulting pseudomodel is
the same as in the proof of Theorem~\ref{th:clc} for .
\end{proof}


We are now going to transform the pseudomodel into a proper
model; it is a well-known technique for dealing with distributed
knowledge. In fact, we can make direct use of a corresponding existing
result for epistemic logic with distributed knowledge, and extend it
with the coalition operators/effectivity functions. We here give the
more general result for the language with also common knowledge, which
will be useful later.

\begin{theorem}[\cite{Fagin1995}]
  \label{th:yi}
  If  is an epistemic pseudomodel, then there is
  an epistemic model  and a
  surjective (onto) function  such that
 for every  and formula ,  iff .
\end{theorem}
\begin{proof}
  This result is directly obtained from the completeness proof for
  \acro{ELCD} sketched in \cite[p. 70]{Fagin1995}. For a more detailed proof
  (for a more general language), see \cite[Theorem 9]{yi}.
\end{proof}

\begin{theorem}
  \label{th:sat-d}
  If a formula is satisfied in a finite pseudomodel, then it is satisfied 
in a model.
\end{theorem}
\begin{proof}
  Let  be a finite pseudomodel such that .  Let  be the epistemic pseudomodel
  underlying , and let  and  be as in Theorem \ref{th:yi}.  Let
   for any set . Finally, let 
  where  is defined as follows:
  \begin{itemize}
  \item For :

\item for :

  \end{itemize}

  Two things must be shown: that  is a proper model, and that it satisfies .

  Since  is an epistemic model, to show that  is a model all
  that remains to be shown is that  is truly playable. We now show
  that that follows from true playability of .
  \begin{description}
\item[E1] Note that  iff . 

For ,
 iff (by definition of )
 iff 
 which is impossible since  satisfies E1.
Note that in particular this proves . 

For ,  iff  and we'll see that this is impossible below.

\item[E2] Note that .

For , 
 iff (by definition of )  
and since  and ,  holds. 
Note that in particular this proves .

For ,  iff  and this was proved above.

\item[E3]  follows immediately
from the definition for .

\item[E4]  is monotonic by definition for . 

For , assume  and . Then . Since for  we already have
monotonicity and , . So .

\item[E5]  and 
, where 

For :
 
Let , , , . This means that for some , ,
such that  and , 
 and , so since  satisfies E5, . Since
 and , , so .

For  ( has to be ): , , prove .
We have , .  Assume by contradiction ,
then  which means . This together with
 and E5 for  gives , that is
. The latter contradicts .

\item[E6] Suppose . We claim that for
  every  such that , .

  Assume by contradiction that there exists a  such
  that  for some  s.t. . By the definition of , this means that there is a
   such that  and . Since  and  it follows that . But it is a property
  of the  function that that implies that , and
  this contradicts the assumption that  and .

\end{description}

In order to show that  satisfies , we show that  iff  for  and any
, by induction in . All cases except  are exactly as in the proof of Theorem \ref{th:yi} (see
\cite[Theorem 9]{yi} for a detailed inductive proof).

For the case that , the inductive hypothesis is that
for all  with , and any  with ,  iff .  Given that
for every  there is a unique  such that , we can
state this as , or
.

First consider .   iff . Consider . By the inductive hypothesis,
.   holds iff  iff .

 iff  iff (*)  iff (as above)  iff .

Explanation of (*): one direction E3, the other direction E5 and E1.
\end{proof}

\begin{corollary}
  For any -formula ,   iff .
\end{corollary}


\section{Completeness of Coalition Logic with both Common and Distributed Knowledge }
\label{sec:clcd}

In this section we consider the logic \acro{CLCD}, extending coalition
logic with operators for individual knowledge, common knowledge and
distributed knowledge.

The axiomatisation  is shown in Figure \ref{fig:clcd-ax}. It
extends  with standard axioms and rules for individual, common and
distributed knowledge.

\begin{figure}[h]
  \centering
\begin{description}
\item[Prop] Classical propositional logic
\item[G1] 
\item[G2] 
\item[G3] 
\item[G4] 
\item[G5] , if 
\item[MP] 
\item[RG] 
\item[K] 
\item[T] 
\item[4] 
\item[5] 
\item[RN] 
\item[C1] 
\item[C2] 
\item[RN] 
\item[RC] 
\item[DK] 
\item[DT] 
\item[D4] 
\item[D5] 
\item[D1] 
\item[D2] , if 
\end{description}
  \caption{: axiomatisation of \acro{CLCD}.}
  \label{fig:clcd-ax}
\end{figure}

As usual, soundness can easily be shown.
\begin{lemma}[Soundness]
  For any -formula , .
\end{lemma}
In the remainder of this section we show that  also is complete.

\begin{theorem}
  Any -consistent formula is satisfied in a finite pseudomodel.
\end{theorem}
\begin{proof}
  The proof is identical to the proof of Lemma \ref{lemma:cld}, with the
addition of the inductive clause  as in the proof of
Theorem \ref{th:clc}.
\end{proof}

We can now use the same approach as in the case of \acro{CLD}.

\begin{theorem}
  If a \acro{CLCD} formula is satisfied in a finite pseudomodel, it is
  satisfied in a model.
\end{theorem}
\begin{proof}
  The proof goes exactly like the proof of Theorem \ref{th:sat-d},
  using Theorem \ref{th:yi}.  The definition of the model  is
  identical to the definition in Theorem \ref{th:sat-d}, as is the proof 
that it is a
  proper model. For the last part of the proof, i.e., showing that
   satisfies , note that the last clause in Theorem
  \ref{th:yi} holds for epistemic logic with both distributed and
  common knowledge. Thus, the proof is completed by only adding the
  inductive clause for , which is done in exactly the same
  way as in Theorem \ref{th:sat-d}.
\end{proof}

\begin{corollary}
  For any -formula ,   iff .
\end{corollary}


\section{Conclusions}
\label{sec:conclusions}

This papers solves several hitherto open problems, namely proving
completeness of Coalition Logic extended with group knowledge
modalities. The axioms for the epistemic modalities are the same as in
the absence of the Coalition Logic axioms, however the completeness
proofs require non-trivial combinations of techniques. The next step
would be to look at complete axiomatisations of logics resulting from
imposing some conditions on the interaction of coalitional ability and
group knowledge (such as the examples in the Introduction), and
obtaining results on the complexity of satisfiability problem for
\acro{CLC}, \acro{CLD} and \acro{CLCD}.

\bibliographystyle{plain}
\bibliography{bib}



\end{document}
