
\subsection{Simple Topologies}

\label{sec:simpletop}

Figure~\ref{fig:layered} presents the three simple topologies considered in our experiments.  
The layered topology (Figure~\ref{fig:layered}(a)) is inspired  by the Akamai network~\cite{sitaraman2014overlay}.  
The chain topology and the tree topology (Figures~\ref{fig:layered}(b) and (c)) are used to clarify the 
role of caching as opposed to routing in network performance.  Under the chain and tree topologies, 
all requests have a single path from the requester to the custodian, assumed to be in the root of the tree.



The reference setup considered  in our experiments is shown in Table \ref{tab:experiments}.  
The request arrival rate for the -th content is given by , for , where  is referred to as the \emph{Zipf parameter}. 
Parameters are varied according to our experimental goals.  
For each parameter setting we simulated 10 runs to obtain the mean and the  95\% confidence interval 
of the metrics of interest, represented by
 a line and a shaded region in the plots that follow, respectively.  
  



\setlength{\tabcolsep}{0pt} 
\begin{figure}[h]
\begin{tabular}{c@{\hskip 0pt}c@{\hskip 0pt}c}
	\includegraphics[scale=0.22]{Slide4.jpg} & 
\hspace{-0.3in}		\includegraphics[scale=0.22]{Slide5.jpg} &
\hspace{-0.5in}	\includegraphics[scale=0.22]{treedepth.jpg} \\
(a) & \hspace{-0.3in}	(b) & \hspace{-0.5in}	(c) 
\end{tabular}
\vspace{-0.1in}
\caption{Hierarchical topologies: (a) layered; (b) chain and (c) tree. }
\label{fig:layered} \label{fig:chain}
\label{fig:treedepthtopology}
\end{figure}

\begin{comment}
\begin{table}[h]
\centering
\begin{tabular}{c |c|c|c|c|c|c|c}
\hline
 Experiment  & Nodes & Cache & Zipf & Topology & Files & Custodian & Exploration \\            &        &  Size & parameter & &  &Cost & Rate \\
 \ref{sub:exp}& 24 & 10 & 0.8 & layered & 100 & 100 & variable\\
 \ref{sub:space_diversity}&  & 10 & 0.8 & tree & 100 & 100 &0.05 \\
 \ref{sub:cost_aware}& 12 & 10 & 0.8 & layered & 100 & variable &0.05\\
 \ref{sub:locality}& 12 & 10 & 0.8 & layered & 100 & 100 &0.05 \\
 \ref{sub:size}& 12 & variable & 0.8 & layered & 100 & 100 &0.05\\
  Zipf& 12 & 10 & variable & layered & 100 & 100 &0.05\\
 \hline
 \hline

\hline
\end{tabular}
\vspace{-0.1in}
\caption{Parameters used in experiments}
\label{tab:experiments}
\end{table}

\end{comment}


\setlength{\tabcolsep}{2pt} 


\begin{table}[h]
\centering
\footnotesize
\begin{tabular}{c|c|c|c|c|c|c}
\hline
  Number of nodes, & Cache & Zipf & Topology & Number of files, & Custodian & Exploration \\            
        &  size,  & parameter,  &  &   &cost & rate \\
  \hline
12 & 10 & 0.8 & layered & 100 & 100 &0.05 \\
 \hline
 \hline

\hline
\end{tabular}
\vspace{-0.1in}
\caption{Reference setup for experiments}
\label{tab:experiments}
\vspace{-0.2in}
\end{table}







 
 \subsubsection{Optimal Exploration Rate}
 \label{sub:exp}
 
 
 Next, we study the impact of the exploration rate on the metrics of interest.  To this aim,
 we consider a network of 12 nodes, and vary the exploration rate.  
 The other parameters are shown in Table~\ref{tab:experiments}.  
 
Figure \ref{fig:exploration}(a) shows how the mean download time varies as a function of the exploration rate.
 We consider a layered topology with  exogenous arrivals occuring at every node. 
Recall that exploration is necessary under MEC and Q-LFU in order to learn the distribution of cost-to-go.
Recall that the cost-to-go  is used for routing decisions by Q-routing and for caching decisions by MEC, i.e.,
when caching is performed using LFU or LRU, the cost-to-go is used only for routing decisions.  

 According to Figure~\ref{fig:exploration}(a), the optimal exploration rate is roughly 0.1 for MEC and 0.2 for Q-LFU.  
 Figure \ref{fig:exploration}(b) evaluates the mean download cost of MEC in more details, with a more fine grained set of exploration rates.
 The minimum mean download cost occurs when the exploration rate equals roughly 0.05.
 
 
Note that Q-LRU  
  is not sensitive to the exploration rate.
We believe that the estimates of the cost-to-go are not very helpful under LRU due to the high content churn 
    associated to such a policy,
    which causes high variability in the distribution of items in the caches over time.  
    This variability  precludes the quick convergence and gains of Q-routing.   
    
    



 
 \begin{figure}[h]
\begin{center}
\begin{tabular}{cc}
    \includegraphics[width=0.48\textwidth]{sim_exploration_24nodes_100files_layered.jpg}
& 
    \includegraphics[width=0.48\textwidth]{sim_exploration_12nodes_100files_layered.jpg} \\
    (a) three strategies & (b) MEC
    \end{tabular}
\end{center}
\vspace{-0.1in}
\caption{ Impact of exploration rate  }
\label{fig:exploration} \label{fig:exploration_onlymes}
\end{figure}
 
\subsubsection{MEC Increases Space Diversity}
\label{sub:space_diversity}


Next, we indicate that MEC increases space diversity, and as a consequence decreases mean download time.  
 To this aim, we consider a tree topology with  levels and  nodes.   The remaining  parameters are shown in Table~\ref{tab:experiments}.
Figure \ref{fig:treedepth} shows how the mean download time varies as a function fo the tree depth.   As the tree depth increases,  the mean download time decresaes more significantly under MEC as opposed to other strategies.  This indicates that MEC, by  exploring space diversity, stores different replicas of distinct items over the network, allowing requests to opportunistically download content from within the network and avoiding requests to the custodian, assumed to be associated to the most costly link in the system.

\begin{figure}[h]
\begin{center}
	\includegraphics[scale=0.2]{sim_treedepth_63nodes_100files_tree_new.jpg}
\end{center}
\vspace{-0.2in}
\caption{Impact of tree depth on mean download time  }
\label{fig:treedepth}
\end{figure}


\subsubsection{MEC is Cost-aware}
\label{sub:cost_aware}




Figure \ref{fig:custodian_cost}(a) shows the impact of the costs of accesses to the  
custodian on the mean download time. We consider the reference setup of Table \label{tab:experiments}, with custodian cost varying between 0 and 100.  
Although the mean download time increases linearly with respect to the cost for all the strategies considered, the slope of  MEC is smaller than its counterparts.  This indicates that MEC, by being cost-aware, can gracefully handle changes in  custodian costs.  Figure~\ref{fig:custodian_cost}(b) shows how the number of server hits varies as a function of the custodian cost.  While MEC adjusts itself to account for cost changes, the other strategies considered are cost-oblivious and maintain the same server hit rate irrespective of server costs.

\begin{figure}[h]
\begin{center}
\begin{tabular}{cc}
    \includegraphics[width=0.48\textwidth]{sim_custodian_cost12nodes_100files_layered.jpg}
& 
    \includegraphics[width=0.48\textwidth]{sim_custodian_cost_6nodes_100files_10csize_layered_leafs.jpg} \\
    (a) & (b)
    \end{tabular}
\end{center}
\vspace{-0.1in}
\caption{ Impact of custodian cost }
\label{fig:custodian_cost} \label{fig:custodian_cost_serverhits}
\vspace{-0.3in}
\end{figure}


\begin{comment}

\subsubsection{MEC is Competitive with Q-LFU}

in unfavorable environments


\subsubsection{Impact of Routing}

(tree vs layered)


\subsubsection{Impact of Topology }

(tree vs chain)
  
  
\end{comment}

\subsubsection{Impact of Request  Locality}
\label{sub:locality}
Next, our goal is to study the impact of request locality on mean download time. 
We consider the reference setup shown in Table~\ref{tab:experiments}. 
In Figures~\ref{fig:arrivals on every node}(a) and ~\ref{fig:arrivals on every node}(b) we consider the cases
  where exogenous requests arrive at every node and only at the leaves, respectively. 
As Figure \ref{fig:arrivals on every node}(a) shows, MEC is better that Q-LRU, and slight better than Q-LFU,  
when there are exogenous arrivals at every node.
 In contrast,  Figure \ref{fig:arrivals only on edges}(b) shows that LFU is slightly better than MEC when there are 
 exogenous arrivals only at the edge.    This is because, due to symmetry, the cost-to-go associated 
 to items which are not stored in the leaves  is roughly the same for  all nodes.  
 Therefore, the gains of MEC due to heterogenous cost-to-go values are not leveraged.  


\begin{figure}[h!]
\begin{center}
\begin{tabular}{cc}
    \includegraphics[width=0.45\textwidth]{sim_cache_size6nodes_100files_layered.jpg}
& 
    \includegraphics[width=0.45\textwidth]{sim_cache_size6nodes_100files_layered_only_leafs.jpg} \\
    (a) arrivals  at every node & (b) arrivals only at the edge (leaves)
    \end{tabular}
\end{center}
\vspace{-0.1in}
\caption{Impact of request locality}
\label{fig:arrivals on every node} \label{fig:arrivals only on edges}
\end{figure}

\begin{comment}

\subsubsection{Impact of Cache Size}
\label{sub:size}

\begin{figure}[h]
\centering
\includegraphics[scale=0.6]{mesh_10_10-crop.jpg}
\vspace{-0.1in}
\caption{Comparison of different caching strategies on a 10-node mesh network. Plotted is the total cost to get all items from all clients agains the cache size of the nodes in the network.}
\label{fig:mesh_10_10}
\end{figure}


\end{comment}



