

\documentclass[12pt]{article}



\usepackage{graphicx}
\usepackage{euscript}\usepackage{picins}\usepackage{pifont}\usepackage{amsmath}\usepackage{amssymb}\usepackage{mathabx}\usepackage{xcolor}\usepackage{paralist}\usepackage{caption}\usepackage{xspace}\usepackage[cm]{fullpage}\usepackage{caption}\usepackage{mleftright}

\usepackage[amsmath,thmmarks]{ntheorem}\theoremseparator{.}

\usepackage{titlesec}
\titlelabel{\thetitle. }

\usepackage{hyperref}\hypersetup{breaklinks,ocgcolorlinks,
   colorlinks=true,linkcolor=[rgb]{0.45,0.0,0.0},citecolor=[rgb]{0,0,0.45}
}


\usepackage[hypcap=true]{caption}



\newtheorem{theorem}{Theorem}[section]

\newtheorem{lemma}[theorem]{Lemma}

\theoremstyle{remark}\theoremheaderfont{\sf}\theorembodyfont{\upshape}\newtheorem{defn}[theorem]{Definition}

\newtheorem*{remark:unnumbered}[theorem]{Remark}\newtheorem*{remarks}[theorem]{Remarks}\newtheorem{remark}[theorem]{Remark}

\theoremheaderfont{\em}\theorembodyfont{\upshape}\theoremstyle{nonumberplain}\theoremseparator{}\theoremsymbol{\myqedsymbol}\newtheorem{proof}{Proof:}\newtheorem{proofNoColon}{Proof}




\newcommand{\aftermathA}{\par\vspace{-\baselineskip}}

\providecommand{\si}[1]{#1}

\newcommand{\myqedsymbol}{\rule{2mm}{2mm}}
\newcommand{\Quote}[2]{
   \begin{center}
{\footnotesize
          \begin{quote}
              {#1}\\
              \hspace*{2cm}-- {#2.}
          \end{quote}
       }
       \index{quotation!#2}
\end{center}
   \bigskip
}

\renewcommand{\Re}{\mathbb{R}}

\newcommand{\ts}{\hspace{0.6pt}}\newcommand{\MakeBig}{\rule[-.2cm]{0cm}{0.4cm}}
\newcommand{\MakesBig}{\rule[0.0cm]{0.0cm}{0.32cm}} \newcommand{\MakeSBig}{\rule[0.0cm]{0.0cm}{0.35cm}} \newcommand{\etal}{\textit{et~al.}\xspace}


\newcommand{\HLinkShort}[2]{\hyperref[#2]{#1\ref*{#2}}}
\newcommand{\HLink}[2]{\hyperref[#2]{#1~\ref*{#2}}}
\newcommand{\HLinkPage}[2]{\hyperref[#2]{#1~\ref*{#2}}}
\newcommand{\HLinkPageOnly}[1]{\hyperref[#1]{Page~\refpage*{#1}}}

\newcommand{\HLinkSuffix}[3]{\hyperref[#2]{#1\ref*{#2}{#3}}}
\newcommand{\HLinkPageSuffix}[3]{\hyperref[#2]{#1\ref*{#2}#3}}

\newcommand{\apndlab}[1]{\label{apnd:#1}}
\newcommand{\apndref}[1]{\HLink{Appendix}{apnd:#1}}
\newcommand{\apndrefpage}[1]{\HLinkPage{Appendix}{apnd:#1}}


\newcommand{\seclab}[1]{\label{sec:#1}}
\newcommand{\secref}[1]{\HLink{Section}{sec:#1}}

\newcommand{\lemlab}[1]{\label{lemma:#1}}
\newcommand{\lemref}[1]{\HLink{Lemma}{lemma:#1}}

\newcommand{\thmlab}[1]{{\label{theo:#1}}}
\newcommand{\thmref}[1]{\HLink{Theorem}{theo:#1}}

\providecommand{\deflab}[1]{\label{def:#1}}
\newcommand{\defref}[1]{\HLink{Definition}{def:#1}}

\newcommand{\figlab}[1]{\label{fig:#1}}
\newcommand{\figref}[1]{\HLink{Figure}{fig:#1}}

\newcommand{\atgen}{\symbol{'100}} \newcommand{\SarielThanks}[1]{\thanks{Department of Computer Science; University of Illinois; 201
      N. Goodwin Avenue; Urbana, IL, 61801, USA; {\tt
         sariel\atgen{}illinois.edu}; {\tt
         \url{http://www.illinois.edu/\string~sariel/}.} #1}}

\newcommand{\PPicCapLab}[4][]{
   \settowidth{\ppicwd}{\includegraphics[#1]{{#2}}} \begin{minipage}{\ppicwd}
       {\includegraphics[#1]{{#2}}}\vspace{-0.3cm}\captionof{figure}{}
       \figlab{#4}
   \end{minipage}}

\newlength{\ppicwd}

\newcommand{\BallSet}{\mathcal{B}}\newcommand{\BallSetA}{\mathcal{C}}

\newcommand{\SepSet}{S}

\newcommand{\sphere}{\mathsf{s}}

\renewcommand{\th}{th\xspace}

\newcommand{\ball}{\mathsf{b}}\newcommand{\ballA}{\mathsf{b'}}


\newcommand{\DSet}{\EuScript{D}}\newcommand{\PntSet}{{P}}\newcommand{\disk}{\mathsf{d}}\newcommand{\diskA}{\mathsf{f}}\newcommand{\diskB}{\mathsf{u}}\newcommand{\diskC}{\mathsf{v}}\newcommand{\pnt}{\mathsf{p}}\newcommand{\pntA}{\mathsf{q}}\newcommand{\pntB}{\mathsf{t}}\newcommand{\AreaX}[1]{\mathrm{area}\pth{#1}}

\newcommand{\Graph}{{G}}

\newcommand{\Vertices}{{V}}\newcommand{\Edges}{{E}}

\newcommand{\Vd}{v_d}\newcommand{\constD}{c_d}\newcommand{\constDbl}{\ell_d}\newcommand{\constDblTwo}{\ell_2}

\newcommand{\VolX}[1]{\mathrm{vol}\pth{#1}}
\newcommand{\ballX}[1]{\mathrm{ball}\pth{#1}}

\newcommand{\Holder}{H\"older\xspace}

\newcommand{\distX}[2]{\left\| {#1} - {#2} \right\|}

\newcommand{\pth}[1]{\mleft({#1}\mright)}

\newcommand{\Cone}{\psi}\definecolor{blue25}{rgb}{0,0,0.55}\newcommand{\emphic}[2]{\textcolor{blue25}{\textbf{\emph{#1}}}\index{#2}}

\newcommand{\emphi}[1]{\emphic{#1}{#1}}

\newcommand{\Ex}[2][\!]{\mathop{\mathbf{E}}#1\pbrcx{#2}}
\newcommand{\Prob}[1]{\mathop{\mathbf{Pr}}\!\pbrcx{#1}}
\newcommand{\cardin}[1]{\left| {#1} \right|}\newcommand{\norm}[1]{\left\| {#1} \right\|}
\newcommand{\pbrcx}[1]{\left[ {#1} \right]}


\newcommand{\DSetSmall}{\DSet_{\leq \ell}}
\newcommand{\DSetLarge}{\DSet_{> \ell}}

\newcommand{\diskY}[2]{\mathrm{disk} \pth{#1, #2}}
\newcommand{\rad}{\rho}\newcommand{\areaX}[1]{\mathrm{area}\pth{#1}}









\begin{document}


\title{A Simple Proof of the Existence of a Planar Separator}

\author{Sariel Har-Peled\SarielThanks{Work on this paper was partially supported by a Work
      on this paper was partially supported by a NSF AF awards
      CCF-1421231, CCF-1217462, and CCF-0915984. }}

\date{\today}

\maketitle

\begin{abstract}
    We provide a simple proof of the existence of a planar separator
    by showing that it is an easy consequence of the circle packing
    theorem. We also reprove other results on separators, including:
    \smallskip
    \begin{compactenum}[\qquad(A)]
        \item There is a simple cycle separator if the planar graph is
        triangulated. Furthermore, if each face has at most  edges
        on its boundary, then there is a cycle separator of size
        .

        \item For a set of  balls in , that are -ply,
        there is a separator, in the intersection graph of the balls,
        of size .

        \item The  nearest neighbor graph of a set of  points in
         contains a separator of size .
    \end{compactenum}
    \smallskip The new proofs are (arguably) significantly\footnote{Or
       insignificantly, or not at all. I am willing to support all
       sides of this argument. The skeptical reader can replace the
       above sentence by ``The new proofs are newer than the older
       proofs.''} simpler than previous proofs.
\end{abstract}



\section{Introduction}


The \emphi{planar separator theorem} is a fundamental result about
planar graphs \cite{u-tpg-51,lt-stpg-79}. Informally, it states that
one can remove  vertices from a
planar graph with  vertices and break it into ``significantly''
smaller parts. It is widely used in algorithms to facilitate efficient
divide and conquer schemes on planar graphs.  For further details on
planar separators and their applications, see Wikipedia
(\url{http://en.wikipedia.org/wiki/Planar_separator_theorem}).


Here, we present a simple proof of the planar separator theorem.  Most
of the main ingredients of the proof are present in earlier work on
this problem; see Miller \etal \cite{mttv-sspnng-97}, Smith and
Wormald \cite{sw-gsta-98}, and Chan \cite{c-ptasp-03}. Furthermore,
the constants in the separator we get are inferior to known
constructions \cite{ast-ps-94}. See \thmref{separator} for the exact
statement.


Nevertheless, the new proof is relatively self contained and
(arguably) simpler than previous proofs.  We also reprove some of the
other results of Miller \etal \cite{mttv-sspnng-97} and Miller
\cite{m-fsscs-86}. Again, arguably, our proofs are simpler (but the
constants are inferior).

\section{Proof of the planar separator theorem}

\subsection{The proof}
\seclab{proof}

Given a planar graph  it is known that it can be drawn
in the plane as a \emphi{kissing graph}; that is, every vertex is a
disk, and an edge in  implies that the two corresponding disks
touch (this is known as Koebe's theorem or the cycle packing theorem,
see \cite{pa-cg-95}). Furthermore, all these disks are interior
disjoint.

Let  be the set of disks realizing  as a kissing graph,
and let  be the set of centers of these disks.  Let 
be the smallest radius disk containing  of the points of
, where . To simplify the
exposition, we assume that  is of radius  and it is centered
in the origin. Randomly pick a number  and consider the
circle  of radius  centered at the origin.  Let  be the set
of all disks in  that intersect . We claim that, in
expectation,  is a good separator.

\begin{lemma}
    \lemlab{separates}The separator  breaks  into two subgraphs with at most
     vertices in each connected component.
\end{lemma}



\begin{proof}
    The circle  breaks the graph into two components: (i) the
    disks with centers inside , and (ii) the disks with centers
    outside .

    \parpic[r]{\PPicCapLab[scale=0.97]{figs/double}{}{double:cover}}

    Clearly, the corresponding vertices in  are disconnected
    once we remove . Furthermore, a disk of radius  can be
    covered by  disks of radius , as depicted in
    \figref{double:cover}. As such, the disk of radius  at the
    origin can contain at most  points of  inside it,
    as a disk of radius  can contain at most  points of
    . We conclude that there are at least  disks of
     with their centers outside , and, by construction,
    there are at least  disks of  with centers inside
    . As such, once  is removed, no connected component of the
    graph  can be of size larger than .
\end{proof}


\begin{lemma}
    \lemlab{small}We have , where
    .
\end{lemma}

\begin{proof}
    Let  be a parameter to be specified shortly. We split
     into two sets:  and  of all disks
    of diameter  and , respectively.

    \parpic[r]{\includegraphics{figs/ring}}

    Consider the ring
    , and
    observe that any disk  of  that intersects
    , must contain inside it a disk of radius  that is
    fully contained in . As such,  covers an area of size
    at least  of this ring. The area of 
    is
    
    As such, the number of disks of  that intersect 
    is
    
    As , we have .

    Consider a disk  of radius  centered
    at .  The circle  intersects  if and only
    if , and as 
    is being picked uniformly from , the probability for that
    is at most .  As such, since
    , we have that the expected number of
    disks of  that intersect  is at most .
    Adding the two quantities together, we have that the expected
    number of disks intersecting  is bounded by
    , which is , for
    .
\end{proof}

Now, putting \lemref{separates} and \lemref{small} together implies
the following.

\begin{theorem}\thmlab{separator}Let  be a planar graph with 
    vertices. There exists a set  of  vertices of
    , such that removing  from  breaks it into
    several connected components, each one of them contains at most
     vertices.
\end{theorem}

\subsection{Remarks}
\begin{remark}
\begin{figure}[t]\centerline{\begin{tabular}{ccccc}
      \includegraphics[page=1]{figs/hex_2}&\qquad&
    \includegraphics[page=2]{figs/hex_2}&\qquad\quad&
                    \includegraphics[page=3]{figs/hex_2}\\
      (A) && (B) && (C)
    \end{tabular}}
    \caption{How to cover a disk of radius  by  disks of radius
       .}
    \figlab{hex}
\end{figure}(A) The constant in \lemref{small} can be improved by working a bit
harder and using the Cauchy-Schwarz inequality.  For the sake of
completeness, we provide the proof in \apndref{better:c}.

    (B) The main difference between the proof of \thmref{separator}
    and the work of Miller \etal \cite{mttv-sspnng-97}, is that they
    found the cycle  by lifting the disks to the sphere, using
    conformal mapping to recenter the resulting caps on the sphere
    around the center point of the centers of the caps. Our direct
    packing argument avoids these stages. We also avoid using the
    Cauchy-Schwarz inequality.

    (C) As suggested by G\"u\si{nter} Rote, one can improve the
    constant of \thmref{separator} to  (instead of ) by
    using a tiling that uses only  disks instead of , see
    \figref{hex}. It is easy to verify that  disks are needed for
    such a cover.
\end{remark}






\section{Extensions}

\subsection{Weighted version}

\begin{lemma}\lemlab{separator:w}Let  be a planar graph with 
    vertices, and assume that the vertices have non-negative weights
    assigned to them, with total weight . There exists a set  of
     vertices of , such that removing  from
     breaks it into several connected components, each one of
    them contains a set of vertices of total weight at most .
\end{lemma}

\begin{proof}
    The proof of \thmref{separator} goes through, with the minor
    modification that that  is picked to be the smallest disk,
    such that the total weight of the centers of the disks it covers
    is .
\end{proof}

Note, that if there is a vertex in the graph with weight ,
then the returned separator could be this single vertex, which is a
legal answer (as the weight of the remaining graph is sufficiently
small).

\subsection{Cycle separators}

A planar graph  is \emphi{maximal} if one can not add edges to
it without violating its planarity. Any drawing of a maximal planar
graph is a triangulation; that is, every face is a triangle.  But
then, in the realization of the graph as a kissing graph of disks, a
face of the complement of the union of the disks has three touching
disks as its boundary. 

\parpic[r]{\includegraphics{figs/cycle}}

In particular, consider the separating cycle
, and two disks  and  that intersect it
consecutively along . Let  be interval on  between
 and . The interval  belong to a
single face of the complement of the union of disks, and in
particular, this face has both  and  on its
boundary. As such, the vertices of  that corresponds to
 and  are connected by an edge. That is, the
resulting separator is a cycle in . Since  intersects a
disk along an interval (or not at all), it follows that this cycle is
simple.  Thus, we get the following.


\begin{theorem}[\cite{m-fsscs-86}]
    \thmlab{separator:2}Let  be a maximal planar graph with
     vertices. There exists a set  of 
    vertices of , such that removing  from  breaks
    it into several connected components, each one of them contains at
    most  vertices. Furthermore  is a simple cycle
    in .
\end{theorem}

\subsubsection{Cycle separator if the graph is not triangulated.}


\begin{lemma}[\cite{m-fsscs-86}]
    \lemlab{separator:2:x}Let  be a connected planar graph with
     vertices, where the \th face has  vertices on its
    boundary, and let .  Then, there exists a set
     of  vertices of , such that removing
     from  breaks it into several connected components,
    each one of them contains at most  vertices. Furthermore
     is a cycle in .

    In particular, if the maximum face degree in  is , then
    the separator size is .
\end{lemma}
\begin{proof}
    The idea to fill in the faces of  so that they are all
    triangulated.

    So, consider a cycle  (not necessarily simple -- an edge might
    be traversed twice) with  vertices that forms the boundary of a
    single face in the given embedding of . Next, we build a
    graph having  as its outer boundary, as follows -- it has
     copies of  one inside the other, where the \th copy
     is connected to the  and  copies, in the natural
    way, where a vertex is connect to its copies. Drawn in the plane,
    this results in a grid like construction.  We also triangulate the
    inner most copy  in an arbitrary fashion, and every
    quadrilateral face is triangulated in an arbitrary fashion. The
    resulting graph  has  vertices, and has the
    property that the any path between any two vertices of  in
    , the corresponding shortest path in  is shorter (or
    of the same length). See \figref{fill:in} for an example.

    \parpic[r]{\begin{minipage}{5cm}
           \includegraphics{figs/grid_face}
           \captionof{figure}{}
           \figlab{fill:in}
       \end{minipage}}

    We repeat this fill-in process for all the faces of , and
    let  be the resulting graph.  is still planar,
    and clearly the number of resulting vertices in the new graph is
    . Observe that , as every
    vertex  incident on a face , can be charged to an edge
    adjacent to both  and . Clearly, if done in a consistent
    fashion, an edge would be charged at most twice, and the maximum
    number of edges in a planar graph is  by Euler's formula.

    In particular, if the maximum value of  is , then maximum
    of  is , as can be easily verified.

    Now, we assign weight zero to all the newly introduced vertices in
    , and assign weight one for the original vertices (that
    appear in ). The graph  is a fully triangulated
    planar graph and it has  vertices.  By \lemref{separator:w},
    there is separator providing the desired partition, and the number
    of vertices on this separator is . Since
     is triangulated, the separator is a simple cycle in
    . We now replace portions of it that uses the face grids
    by the appropriate paths along the original boundary of the
    faces. Clearly, the resulting cycle in  has the same
    number of vertices, provide the same quality of separation (or
    better, since some vertices migrated to the separator), as
    desired.
\end{proof}

Miller's result is somewhat stronger than \lemref{separator:2:x}, as
he assumes the graph is -connected, and can ensure that in this
case the separator is a \emph{simple} cycle.


\subsection{Ball systems that are -ply}

A set of balls  in  is \emphi{-ply}, if no point
of  is contained in more than  balls of . 

\begin{defn}
    \deflab{doubling:constant}The \emphi{doubling constant} of a metric space is the smallest
    number of balls of the same radius needed to cover a ball of twice
    the radius (formally, we take the maximum such number over all
    possible balls to be covered). The doubling constant of  is
     \cite{v-cbseb-05}.
\end{defn}


\begin{theorem}[\cite{mttv-sspnng-97}]\thmlab{sep:k:ply}Let  be a set of  balls that is -ply in
    . Then, there exists a sphere  that intersects  balls of . Furthermore, the number of
    balls of  that are completely inside (resp. outside)
     is .
\end{theorem}

\begin{proof}
    Let  be the set of centers of the balls of . As
    above, let  be the smallest ball containing
     points of . As above, assume that
     is centered at the origin and has radius . Let
     be a random sphere centered at the origin with radius
     picked randomly from the range .

    Now, arguing as above, there are at most  points of  inside , and
    as such, at least  points of  outside . As such
     is a good separator for the balls.

    As for the expected number of balls intersecting , let
     be the volume of a ball of radius  in , where
     is a constant that depends on the dimension. As above, we
    clip the balls of  to the ball of radius  centered at
    the origin, replacing every lens, by a an appropriate ball of the
    same volume. Let  denote the radius of the \th such
    ball , for . By the -ply property, we
    have that
    
    where  denotes a ball of radius  in . As
    before, the probability of the \th ball to intersect 
    is bounded by .  Let  be the set of balls of
     that intersects . We have, by \Holder's
    inequality, that
    
    as desired.
\end{proof}


\subsection{Separators for the \th nearest neighbor graph}

Let  be a set of  points in , and let  be a
parameter. The \emphi{\th nearest neighbor graph}  is the graph, where two points  are connected by an edge , if 
is the \th nearest neighbor of  in  (or  is
the \th nearest neighbor of ), for .

\begin{theorem}[\cite{mttv-sspnng-97}]
    Let  be a set of  points in , and let  be a
    parameter. The \th nearest neighbor graph
     has a separator of size
    , such that each connected component has at
    most  vertices, where
     is the doubling constant of , see
    \defref{doubling:constant}.
\end{theorem}

\begin{proof}
    We follow the proof of Miller \etal \cite{mttv-sspnng-97}.  A
    point  is an \emphi{-client} of , if  is the \th nearest neighbor of , for
    .  If  is a -client of , then create a
    ball of radius  centered at .  Let
     be the resulting set of  balls. The key observation
    is that this set of balls is -ply -- which we reprove here
    using a standard argument.

    We claim that every point  can serve at most
     clients. To this end, cover the sphere of directions around
     with cones with angular diameter at most . It
    is easy to verify that at most  such cones are
    needed.


    \parpic[r]{\includegraphics{figs/blocking}}

    The key observation is now that for any two points  that belong to the same cone  of , it
    must be that ,
    assuming that  is closer to  than , as an easy
    geometric argument shows. That is, if 
    are the  closest points to  in , then
    these are the only points of  that might be
    -clients of . It follows that  can have at most  -clients, and as such its degree in  is . That is, the maximum degree of a vertex in  is
    .

    To see why this implies that the set of balls  is
    -ply, consider any point , insert it into
    , and observe that the degree of  in the graph
     bounds the number of balls of  that cover
    it. By the above, this is , as desired.

    By \thmref{sep:k:ply}, there are  balls of
    , such that their removal breaks the intersection graph
    of  into connected components each of size at most
    . Clearly, the corresponding
    set of points of  is the desired separator of .
\end{proof}

\subsection{Separator for  vertices in a planar graph}

Our purpose here is to show that in a triangulated planar graph, there
is always a cycle of size  that its removal separates
(roughly)  vertices from remainder of the graph. To this end, we
need the following.

\begin{lemma}\lemlab{packing:r}Let  be a set of  balls in  that are interior
    disjoint, and let  be some prespecified integer number.
    Let  be the smallest ball that contains  centers of the
    balls of . Then  intersects at most
     balls of . Furthermore,
     intersects at most 
    balls of , where  is the doubling constant of
    , see \defref{doubling:constant}.
\end{lemma}
\begin{proof}
    Assume  is of radius one and it is centered at the
    origin. Consider the ball , and observe that it can be
    covered by  balls of radius one, and let
     be this set of balls. As such,  contains at
    most  centers of balls of . Any
    other ball of  that intersect  must be radius at
    least , as its center is at distance at least  from the
    origin.

    It is easy to verify that such a ball  must contain fully
    at least one ball of . Indeed, consider the segment
    connecting the center of  with the origin, and consider
    the point on this segment on . Clearly, this
    point must be covered by one of the balls of , and this
    ball is fully contained in .
\end{proof}

\begin{lemma}
    Let  be a planar graph with  vertices, and let 
    be an integer number which is sufficiently large. There exists a
    set of vertices  of size ,
    such that  is disconnected into two sets
    of vertices,  and , such that , where  is a constant (see
    \defref{doubling:constant}).  Furthermore, if  is
    triangulated then  is a cycle in the graph.
\end{lemma}

\begin{proof}
    Let  be the realization of  as a kissing graph
    of interior disjoint disks.  Let  be the smallest disk
    containing  centers of , and assume that
    it is of radius one and centered at the origin.
    \lemref{packing:r} implies that  intersects at most
     disks of , and let 
    be this set of balls. Now consider the circle  centered at
    the origin of radius , where  is picked randomly and
    uniformly from the range . Let  be the set of
    disks of  that intersects . 

    Now, by the analysis of \lemref{small}, the expected number of
    disks of , and thus of  that intersects 
    is
    .
    This implies that the number of disks strictly inside  is at
    least
    , if
    . Similarly, it is easy to argue
    that  contains at most  disks of .
\end{proof}


\section{Conclusions}

This write-up demonstrates that the planar separator theorem is an
easy consequence of the circle packing theorem, originally proved by
Paul Koebe in 1936 \cite{k-kdka-36}.  The circle packing theorem is
thus the ``true'' magic -- converting a topological property (a graph
being planar) into a packing property (i.e., disks touching each
other).

\paragraph{An open problem.}

The current algorithmic proofs of the circle packing theorem build an
evolving discrete structure that keeps improving after each iteration,
till in the limit it converges to the desired packing.  Specifically,
there is no finite algorithm that computes the realization of a planar
graph as a circle packing.

It seems unlikely that a finite algorithm is possible because of
numerical issues. However, a much weaker version is sufficient for the
planar separator theorem. In particular, can one find for a planar
graph a set of disks, such that two vertices are connected if and only
if their respective disks intersect (in their interiors), and no point
in the plane is contained in more than, say,  disks of this set,
where  is some universal constant (thus, we allow disks to
intersect even if their corresponding vertices are not connected in
the planar graph). We leave the development of such a finite
construction algorithm as an open problem for further research.






\section*{Acknowledgments}
The author thanks Mark \si{de} Berg, Timothy Chan, Robert Krauthgamer,
G\"u\si{nter} Rote, and Christian Sommer for useful comments on the
manuscript. The idea of using a ring area argument, in the proof of
\lemref{small}, came about during discussions with Mark \si{de}
Berg. G\"u\si{nter} Rote suggested the elegant tilling\footnote{Which
   is of course well known and not new.} depicted in \figref{hex}.






\providecommand{\CNFX}[1]{ {\em{\textrm{(#1)}}}}
  \providecommand{\tildegen}{{\protect\raisebox{-0.1cm}{\symbol{'176}\hspace{-0.03cm}}}}
  \providecommand{\SarielWWWPapersAddr}{http://sarielhp.org/p/}
  \providecommand{\SarielWWWPapers}{http://sarielhp.org/p/}
  \providecommand{\urlSarielPaper}[1]{\href{\SarielWWWPapersAddr/#1}{\SarielWWWPapers{}/#1}}
  \providecommand{\Badoiu}{B\u{a}doiu}
  \providecommand{\Barany}{B{\'a}r{\'a}ny}
  \providecommand{\Bronimman}{Br{\"o}nnimann}  \providecommand{\Erdos}{Erd{\H
  o}s}  \providecommand{\Gartner}{G{\"a}rtner}
  \providecommand{\Matousek}{Matou{\v s}ek}
  \providecommand{\Merigot}{M{\'{}e}rigot}
  \providecommand{\CNFSoCG}{\CNFX{SoCG}}
  \providecommand{\CNFCCCG}{\CNFX{CCCG}}
  \providecommand{\CNFFOCS}{\CNFX{FOCS}}
  \providecommand{\CNFSODA}{\CNFX{SODA}}
  \providecommand{\CNFSTOC}{\CNFX{STOC}}
  \providecommand{\CNFBROADNETS}{\CNFX{BROADNETS}}
  \providecommand{\CNFESA}{\CNFX{ESA}}
  \providecommand{\CNFFSTTCS}{\CNFX{FSTTCS}}
  \providecommand{\CNFIJCAI}{\CNFX{IJCAI}}
  \providecommand{\CNFINFOCOM}{\CNFX{INFOCOM}}
  \providecommand{\CNFIPCO}{\CNFX{IPCO}}
  \providecommand{\CNFISAAC}{\CNFX{ISAAC}}
  \providecommand{\CNFLICS}{\CNFX{LICS}}
  \providecommand{\CNFPODS}{\CNFX{PODS}}
  \providecommand{\CNFSWAT}{\CNFX{SWAT}}
  \providecommand{\CNFWADS}{\CNFX{WADS}}
\begin{thebibliography}{MTTV97}

\bibitem[AST94]{ast-ps-94}
\href{http://www.math.tau.ac.il/~nogaa/}{N.~{Alon}}, P.~Seymour, and R.~Thomas.
\newblock Planar separators.
\newblock {\em SIAM J. Discrete Math.}, 2(7):184--193, 1994.

\bibitem[Cha03]{c-ptasp-03}
\href{http://www.math.uwaterloo.ca/~tmchan/}{T.~M.~{Chan}}.
\newblock Polynomial-time approximation schemes for packing and piercing fat
  objects.
\newblock {\em J. Algorithms}, 46(2):178--189, 2003.

\bibitem[Koe36]{k-kdka-36}
P.~Koebe.
\newblock Kontaktprobleme der konformen {Abbildung}.
\newblock {\em Ber. Verh. S{\"a}chs. Akademie der Wissenschaften Leipzig,
  Math.-Phys. Klasse}, 88:141--164, 1936.

\bibitem[LT79]{lt-stpg-79}
R.~J. Lipton and R.~E. Tarjan.
\newblock A separator theorem for planar graphs.
\newblock {\em SIAM J. Appl. Math.}, 36:177--189, 1979.

\bibitem[Mil86]{m-fsscs-86}
G.~L. Miller.
\newblock {Finding small simple cycle separators for 2-connected planar
  graphs}.
\newblock {\em J. Comput. Sys. Sci.}, 32(3):265--279, 1986.

\bibitem[MTTV97]{mttv-sspnng-97}
G.~L. Miller, S.~H. Teng, W.~P. Thurston, and S.~A. Vavasis.
\newblock Separators for sphere-packings and nearest neighbor graphs.
\newblock {\em \href{http://www.acm.org/jacm/}{J. Assoc. Comput. {Mach.}}}, 44(1):1--29, 1997.

\bibitem[PA95]{pa-cg-95}
\href{http://www.math.nyu.edu/~pach}{J.~{Pach}} and \href{http://www.cs.duke.edu/~pankaj}{P.~K.~{Agarwal}}.
\newblock {\em Combinatorial Geometry}.
\newblock John Wiley \& Sons, 1995.

\bibitem[SW98]{sw-gsta-98}
W.~D. Smith and N.~C. Wormald.
\newblock Geometric separator theorems and applications.
\newblock In {\em Proc. 39th Annu. IEEE Sympos. Found. Comput. Sci.\CNFFOCS},
  pages 232--243, 1998.

\bibitem[Ung51]{u-tpg-51}
P.~Ungar.
\newblock A theorem on planar graphs.
\newblock {\em J. London Math. Soc.}, 26:256--262, 1951.

\bibitem[{Ver}05]{v-cbseb-05}
J.-L. {Verger-Gaugry}.
\newblock Covering a ball with smaller equal balls in {}.
\newblock {\em \href{http://link.springer.com/journal/454}{Discrete Comput. {}Geom.}}, 33(1):143--155, 2005.

\end{thebibliography}



\appendix

\section{Proof of \lemref{small} with a better constant}
\apndlab{better:c}


\begin{proof}
    Consider a disk  of  of radius  centered at
    .  If  is fully contained in  (the
    disk of radius  centered at the origin), then the circle 
    intersects  if and only if , and as  is being picked uniformly from
    , the probability for that is at most . For reasons that would become clear shortly, we set  and  in this case.

    \vspace{-0.3cm}
    \parpic[r]{\includegraphics{figs/diskify}}
    \vspace{0.3cm}

    Otherwise, if  is not fully contained in  then
    the set  is a ``lens''. Consider a
    disk  of the same area as  contained inside
     and tangent to its boundary. Clearly, if 
    intersects  then it also intersects , see
    figure on the right. Furthermore, the radius of  is
    , and, by
    the above, the probability that  intersects  (and
    thus ) is at most .

    Observe that as the disks of  are interior disjoint, we
    have that .  Now, by
    linearity of expectation and the Cauchy-Schwarz inequality, we
    have that
    
\end{proof}

\end{document}
