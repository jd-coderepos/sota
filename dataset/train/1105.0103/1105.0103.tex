

\documentclass[12pt]{article}



\usepackage{graphicx}
\usepackage{euscript}\usepackage{picins}\usepackage{pifont}\usepackage{amsmath}\usepackage{amssymb}\usepackage{mathabx}\usepackage{xcolor}\usepackage{paralist}\usepackage{caption}\usepackage{xspace}\usepackage[cm]{fullpage}\usepackage{caption}\usepackage{mleftright}

\usepackage[amsmath,thmmarks]{ntheorem}\theoremseparator{.}

\usepackage{titlesec}
\titlelabel{\thetitle. }

\usepackage{hyperref}\hypersetup{breaklinks,ocgcolorlinks,
   colorlinks=true,linkcolor=[rgb]{0.45,0.0,0.0},citecolor=[rgb]{0,0,0.45}
}


\usepackage[hypcap=true]{caption}



\newtheorem{theorem}{Theorem}[section]

\newtheorem{lemma}[theorem]{Lemma}

\theoremstyle{remark}\theoremheaderfont{\sf}\theorembodyfont{\upshape}\newtheorem{defn}[theorem]{Definition}

\newtheorem*{remark:unnumbered}[theorem]{Remark}\newtheorem*{remarks}[theorem]{Remarks}\newtheorem{remark}[theorem]{Remark}

\theoremheaderfont{\em}\theorembodyfont{\upshape}\theoremstyle{nonumberplain}\theoremseparator{}\theoremsymbol{\myqedsymbol}\newtheorem{proof}{Proof:}\newtheorem{proofNoColon}{Proof}




\newcommand{\aftermathA}{\par\vspace{-\baselineskip}}

\providecommand{\si}[1]{#1}

\newcommand{\myqedsymbol}{\rule{2mm}{2mm}}
\newcommand{\Quote}[2]{
   \begin{center}
{\footnotesize
          \begin{quote}
              {#1}\\
              \hspace*{2cm}-- {#2.}
          \end{quote}
       }
       \index{quotation!#2}
\end{center}
   \bigskip
}

\renewcommand{\Re}{\mathbb{R}}

\newcommand{\ts}{\hspace{0.6pt}}\newcommand{\MakeBig}{\rule[-.2cm]{0cm}{0.4cm}}
\newcommand{\MakesBig}{\rule[0.0cm]{0.0cm}{0.32cm}} \newcommand{\MakeSBig}{\rule[0.0cm]{0.0cm}{0.35cm}} \newcommand{\etal}{\textit{et~al.}\xspace}


\newcommand{\HLinkShort}[2]{\hyperref[#2]{#1\ref*{#2}}}
\newcommand{\HLink}[2]{\hyperref[#2]{#1~\ref*{#2}}}
\newcommand{\HLinkPage}[2]{\hyperref[#2]{#1~\ref*{#2}$_\text{p\pageref{#2}}$}}
\newcommand{\HLinkPageOnly}[1]{\hyperref[#1]{Page~\refpage*{#1}$_\text{p\pageref{#1}}$}}

\newcommand{\HLinkSuffix}[3]{\hyperref[#2]{#1\ref*{#2}{#3}}}
\newcommand{\HLinkPageSuffix}[3]{\hyperref[#2]{#1\ref*{#2}#3$_\text{p\pageref{#2}}$}}

\newcommand{\apndlab}[1]{\label{apnd:#1}}
\newcommand{\apndref}[1]{\HLink{Appendix}{apnd:#1}}
\newcommand{\apndrefpage}[1]{\HLinkPage{Appendix}{apnd:#1}}


\newcommand{\seclab}[1]{\label{sec:#1}}
\newcommand{\secref}[1]{\HLink{Section}{sec:#1}}

\newcommand{\lemlab}[1]{\label{lemma:#1}}
\newcommand{\lemref}[1]{\HLink{Lemma}{lemma:#1}}

\newcommand{\thmlab}[1]{{\label{theo:#1}}}
\newcommand{\thmref}[1]{\HLink{Theorem}{theo:#1}}

\providecommand{\deflab}[1]{\label{def:#1}}
\newcommand{\defref}[1]{\HLink{Definition}{def:#1}}

\newcommand{\figlab}[1]{\label{fig:#1}}
\newcommand{\figref}[1]{\HLink{Figure}{fig:#1}}

\newcommand{\atgen}{\symbol{'100}} \newcommand{\SarielThanks}[1]{\thanks{Department of Computer Science; University of Illinois; 201
      N. Goodwin Avenue; Urbana, IL, 61801, USA; {\tt
         sariel\atgen{}illinois.edu}; {\tt
         \url{http://www.illinois.edu/\string~sariel/}.} #1}}

\newcommand{\PPicCapLab}[4][]{
   \settowidth{\ppicwd}{\includegraphics[#1]{{#2}}} \begin{minipage}{\ppicwd}
       {\includegraphics[#1]{{#2}}}\vspace{-0.3cm}\captionof{figure}{}
       \figlab{#4}
   \end{minipage}}

\newlength{\ppicwd}

\newcommand{\BallSet}{\mathcal{B}}\newcommand{\BallSetA}{\mathcal{C}}

\newcommand{\SepSet}{S}

\newcommand{\sphere}{\mathsf{s}}

\renewcommand{\th}{th\xspace}

\newcommand{\ball}{\mathsf{b}}\newcommand{\ballA}{\mathsf{b'}}


\newcommand{\DSet}{\EuScript{D}}\newcommand{\PntSet}{{P}}\newcommand{\disk}{\mathsf{d}}\newcommand{\diskA}{\mathsf{f}}\newcommand{\diskB}{\mathsf{u}}\newcommand{\diskC}{\mathsf{v}}\newcommand{\pnt}{\mathsf{p}}\newcommand{\pntA}{\mathsf{q}}\newcommand{\pntB}{\mathsf{t}}\newcommand{\AreaX}[1]{\mathrm{area}\pth{#1}}

\newcommand{\Graph}{{G}}

\newcommand{\Vertices}{{V}}\newcommand{\Edges}{{E}}

\newcommand{\Vd}{v_d}\newcommand{\constD}{c_d}\newcommand{\constDbl}{\ell_d}\newcommand{\constDblTwo}{\ell_2}

\newcommand{\VolX}[1]{\mathrm{vol}\pth{#1}}
\newcommand{\ballX}[1]{\mathrm{ball}\pth{#1}}

\newcommand{\Holder}{H\"older\xspace}

\newcommand{\distX}[2]{\left\| {#1} - {#2} \right\|}

\newcommand{\pth}[1]{\mleft({#1}\mright)}

\newcommand{\Cone}{\psi}\definecolor{blue25}{rgb}{0,0,0.55}\newcommand{\emphic}[2]{\textcolor{blue25}{\textbf{\emph{#1}}}\index{#2}}

\newcommand{\emphi}[1]{\emphic{#1}{#1}}

\newcommand{\Ex}[2][\!]{\mathop{\mathbf{E}}#1\pbrcx{#2}}
\newcommand{\Prob}[1]{\mathop{\mathbf{Pr}}\!\pbrcx{#1}}
\newcommand{\cardin}[1]{\left| {#1} \right|}\newcommand{\norm}[1]{\left\| {#1} \right\|}
\newcommand{\pbrcx}[1]{\left[ {#1} \right]}


\newcommand{\DSetSmall}{\DSet_{\leq \ell}}
\newcommand{\DSetLarge}{\DSet_{> \ell}}

\newcommand{\diskY}[2]{\mathrm{disk} \pth{#1, #2}}
\newcommand{\rad}{\rho}\newcommand{\areaX}[1]{\mathrm{area}\pth{#1}}









\begin{document}


\title{A Simple Proof of the Existence of a Planar Separator}

\author{Sariel Har-Peled\SarielThanks{Work on this paper was partially supported by a Work
      on this paper was partially supported by a NSF AF awards
      CCF-1421231, CCF-1217462, and CCF-0915984. }}

\date{\today}

\maketitle

\begin{abstract}
    We provide a simple proof of the existence of a planar separator
    by showing that it is an easy consequence of the circle packing
    theorem. We also reprove other results on separators, including:
    \smallskip
    \begin{compactenum}[\qquad(A)]
        \item There is a simple cycle separator if the planar graph is
        triangulated. Furthermore, if each face has at most $d$ edges
        on its boundary, then there is a cycle separator of size
        $O\bigl( \sqrt{ d n} \,\bigr)$.

        \item For a set of $n$ balls in $\Re^d$, that are $k$-ply,
        there is a separator, in the intersection graph of the balls,
        of size $O\pth{ k^{1/d}n^{1-1/d}}$.

        \item The $k$ nearest neighbor graph of a set of $n$ points in
        $\Re^d$ contains a separator of size $O\pth{
           k^{1/d}n^{1-1/d}}$.
    \end{compactenum}
    \smallskip The new proofs are (arguably) significantly\footnote{Or
       insignificantly, or not at all. I am willing to support all
       sides of this argument. The skeptical reader can replace the
       above sentence by ``The new proofs are newer than the older
       proofs.''} simpler than previous proofs.
\end{abstract}



\section{Introduction}


The \emphi{planar separator theorem} is a fundamental result about
planar graphs \cite{u-tpg-51,lt-stpg-79}. Informally, it states that
one can remove $O\pth{\! \sqrt{n}\ts \MakesBig }$ vertices from a
planar graph with $n$ vertices and break it into ``significantly''
smaller parts. It is widely used in algorithms to facilitate efficient
divide and conquer schemes on planar graphs.  For further details on
planar separators and their applications, see Wikipedia
(\url{http://en.wikipedia.org/wiki/Planar_separator_theorem}).


Here, we present a simple proof of the planar separator theorem.  Most
of the main ingredients of the proof are present in earlier work on
this problem; see Miller \etal \cite{mttv-sspnng-97}, Smith and
Wormald \cite{sw-gsta-98}, and Chan \cite{c-ptasp-03}. Furthermore,
the constants in the separator we get are inferior to known
constructions \cite{ast-ps-94}. See \thmref{separator} for the exact
statement.


Nevertheless, the new proof is relatively self contained and
(arguably) simpler than previous proofs.  We also reprove some of the
other results of Miller \etal \cite{mttv-sspnng-97} and Miller
\cite{m-fsscs-86}. Again, arguably, our proofs are simpler (but the
constants are inferior).

\section{Proof of the planar separator theorem}

\subsection{The proof}
\seclab{proof}

Given a planar graph $\Graph=(V,E)$ it is known that it can be drawn
in the plane as a \emphi{kissing graph}; that is, every vertex is a
disk, and an edge in $\Graph$ implies that the two corresponding disks
touch (this is known as Koebe's theorem or the cycle packing theorem,
see \cite{pa-cg-95}). Furthermore, all these disks are interior
disjoint.

Let $\DSet$ be the set of disks realizing $\Graph$ as a kissing graph,
and let $\PntSet$ be the set of centers of these disks.  Let $\disk$
be the smallest radius disk containing $n/10$ of the points of
$\PntSet$, where $n= \cardin{\PntSet} = \cardin{V}$. To simplify the
exposition, we assume that $\disk$ is of radius $1$ and it is centered
in the origin. Randomly pick a number $x \in [1,2]$ and consider the
circle $C_x$ of radius $x$ centered at the origin.  Let $S$ be the set
of all disks in $\DSet$ that intersect $C_x$. We claim that, in
expectation, $S$ is a good separator.

\begin{lemma}
    \lemlab{separates}The separator $S$ breaks $\Graph$ into two subgraphs with at most
    $(9/10) n$ vertices in each connected component.
\end{lemma}



\begin{proof}
    The circle $C_x$ breaks the graph into two components: (i) the
    disks with centers inside $C_x$, and (ii) the disks with centers
    outside $C_x$.

    \parpic[r]{\PPicCapLab[scale=0.97]{figs/double}{}{double:cover}}

    Clearly, the corresponding vertices in $\Graph$ are disconnected
    once we remove $S$. Furthermore, a disk of radius $2$ can be
    covered by $9$ disks of radius $1$, as depicted in
    \figref{double:cover}. As such, the disk of radius $2$ at the
    origin can contain at most $9n/10$ points of $\PntSet$ inside it,
    as a disk of radius $1$ can contain at most $n/10$ points of
    $\PntSet$. We conclude that there are at least $n/10$ disks of
    $\DSet$ with their centers outside $C_x$, and, by construction,
    there are at least $n/10$ disks of $\DSet$ with centers inside
    $C_x$. As such, once $S$ is removed, no connected component of the
    graph $\Graph \setminus S$ can be of size larger than $(9/10)n$.
\end{proof}


\begin{lemma}
    \lemlab{small}We have $\Ex{\MakeSBig\! \cardin{S}} \leq 11 \sqrt{n}$, where
    $n = \cardin{V}$.
\end{lemma}

\begin{proof}
    Let $\ell < 1$ be a parameter to be specified shortly. We split
    $\DSet$ into two sets: $\DSetSmall$ and $\DSetLarge$ of all disks
    of diameter $\leq \ell$ and $>\ell$, respectively.

    \parpic[r]{\includegraphics{figs/ring}}

    Consider the ring
    $R = \diskY{0}{x+\ell} \,\setminus \, \diskY{0}{x-\ell}$, and
    observe that any disk $\diskA$ of $\DSetLarge$ that intersects
    $C_x$, must contain inside it a disk of radius $\ell/2$ that is
    fully contained in $R$. As such, $\diskA$ covers an area of size
    at least $\alpha = \pi (\ell/2)^2$ of this ring. The area of $R$
    is
    \begin{math}
        \beta = \pi \pth{\pth{x+\ell}^2 - \pth{x-\ell}^2} = 4\pi x
        \ell.
    \end{math}
    As such, the number of disks of $\DSetLarge$ that intersect $C_x$
    is
    \begin{math}
        \leq \beta/\alpha = 4\pi x \ell/(\pi \ell^2/4) = 16x /\ell.
    \end{math}
    As $\Ex{x}=3/2$, we have $\Ex{\beta/\alpha} = 24/\ell$.

    Consider a disk $\diskB_i \in \DSetSmall$ of radius $r_i$ centered
    at $\pnt_i$.  The circle $C_x$ intersects $\diskB_i$ if and only
    if $x \in [\norm{\pnt_i} - r_i, \norm{\pnt_i} + r_i]$, and as $x$
    is being picked uniformly from $[1,2]$, the probability for that
    is at most $2r_i/|2-1| = 2r_i \leq \ell$.  As such, since
    $\cardin{\DSetSmall} \leq n$, we have that the expected number of
    disks of $\DSetSmall$ that intersect $C_x$ is at most $n \ell$.
    Adding the two quantities together, we have that the expected
    number of disks intersecting $C_x$ is bounded by
    $n \ell + 24/\ell$, which is $\leq 2\sqrt{24 n}$, for
    $\ell = 1/\sqrt{24 n}$.
\end{proof}

Now, putting \lemref{separates} and \lemref{small} together implies
the following.

\begin{theorem}\thmlab{separator}Let $\Graph = (\Vertices,\Edges)$ be a planar graph with $n$
    vertices. There exists a set $S$ of $11 \sqrt{n}$ vertices of
    $\Graph$, such that removing $S$ from $\Graph$ breaks it into
    several connected components, each one of them contains at most
    $(9/10)n$ vertices.
\end{theorem}

\subsection{Remarks}
\begin{remark}
\begin{figure}[t]\centerline{\begin{tabular}{ccccc}
      \includegraphics[page=1]{figs/hex_2}&\qquad&
    \includegraphics[page=2]{figs/hex_2}&\qquad\quad&
                    \includegraphics[page=3]{figs/hex_2}\\
      (A) && (B) && (C)
    \end{tabular}}
    \caption{How to cover a disk of radius $2$ by $7$ disks of radius
       $1$.}
    \figlab{hex}
\end{figure}(A) The constant in \lemref{small} can be improved by working a bit
harder and using the Cauchy-Schwarz inequality.  For the sake of
completeness, we provide the proof in \apndref{better:c}.

    (B) The main difference between the proof of \thmref{separator}
    and the work of Miller \etal \cite{mttv-sspnng-97}, is that they
    found the cycle $C_x$ by lifting the disks to the sphere, using
    conformal mapping to recenter the resulting caps on the sphere
    around the center point of the centers of the caps. Our direct
    packing argument avoids these stages. We also avoid using the
    Cauchy-Schwarz inequality.

    (C) As suggested by G\"u\si{nter} Rote, one can improve the
    constant of \thmref{separator} to $7/8$ (instead of $9/10$) by
    using a tiling that uses only $7$ disks instead of $9$, see
    \figref{hex}. It is easy to verify that $7$ disks are needed for
    such a cover.
\end{remark}






\section{Extensions}

\subsection{Weighted version}

\begin{lemma}\lemlab{separator:w}Let $\Graph = (\Vertices,\Edges)$ be a planar graph with $n$
    vertices, and assume that the vertices have non-negative weights
    assigned to them, with total weight $W$. There exists a set $S$ of
    $4 \sqrt{n}$ vertices of $\Graph$, such that removing $S$ from
    $\Graph$ breaks it into several connected components, each one of
    them contains a set of vertices of total weight at most $(9/10)W$.
\end{lemma}

\begin{proof}
    The proof of \thmref{separator} goes through, with the minor
    modification that that $\disk$ is picked to be the smallest disk,
    such that the total weight of the centers of the disks it covers
    is $\geq W/10$.
\end{proof}

Note, that if there is a vertex in the graph with weight $\geq W/10$,
then the returned separator could be this single vertex, which is a
legal answer (as the weight of the remaining graph is sufficiently
small).

\subsection{Cycle separators}

A planar graph $\Graph$ is \emphi{maximal} if one can not add edges to
it without violating its planarity. Any drawing of a maximal planar
graph is a triangulation; that is, every face is a triangle.  But
then, in the realization of the graph as a kissing graph of disks, a
face of the complement of the union of the disks has three touching
disks as its boundary. 

\parpic[r]{\includegraphics{figs/cycle}}

In particular, consider the separating cycle
$C_k$, and two disks $\diskA$ and $\diskA'$ that intersect it
consecutively along $C_x$. Let $I$ be interval on $C_x$ between
$\diskA \cap C_x$ and $\diskA'\cap C_x$. The interval $I$ belong to a
single face of the complement of the union of disks, and in
particular, this face has both $\diskA$ and $\diskA'$ on its
boundary. As such, the vertices of $\Graph$ that corresponds to
$\diskA$ and $\diskA'$ are connected by an edge. That is, the
resulting separator is a cycle in $\Graph$. Since $C_x$ intersects a
disk along an interval (or not at all), it follows that this cycle is
simple.  Thus, we get the following.


\begin{theorem}[\cite{m-fsscs-86}]
    \thmlab{separator:2}Let $\Graph = (\Vertices,\Edges)$ be a maximal planar graph with
    $n$ vertices. There exists a set $\SepSet$ of $4 \sqrt{n}$
    vertices of $\Graph$, such that removing $S$ from $\Graph$ breaks
    it into several connected components, each one of them contains at
    most $(9/10)n$ vertices. Furthermore $\SepSet$ is a simple cycle
    in $\Graph$.
\end{theorem}

\subsubsection{Cycle separator if the graph is not triangulated.}


\begin{lemma}[\cite{m-fsscs-86}]
    \lemlab{separator:2:x}Let $\Graph = (\Vertices,\Edges)$ be a connected planar graph with
    $n$ vertices, where the $i$\th face has $d_i$ vertices on its
    boundary, and let $N = \sum_{i} d_i^2$.  Then, there exists a set
    $\SepSet$ of $4 \sqrt{N}$ vertices of $\Graph$, such that removing
    $S$ from $\Graph$ breaks it into several connected components,
    each one of them contains at most $(9/10)n$ vertices. Furthermore
    $\SepSet$ is a cycle in $\Graph$.

    In particular, if the maximum face degree in $\Graph$ is $d$, then
    the separator size is $O\pth{\!\sqrt{nd}}$.
\end{lemma}
\begin{proof}
    The idea to fill in the faces of $\Graph$ so that they are all
    triangulated.

    So, consider a cycle $C$ (not necessarily simple -- an edge might
    be traversed twice) with $k$ vertices that forms the boundary of a
    single face in the given embedding of $\Graph$. Next, we build a
    graph having $C_1= C$ as its outer boundary, as follows -- it has
    $k$ copies of $C$ one inside the other, where the $i$\th copy
    $C_i$ is connected to the $i-1$ and $i+1$ copies, in the natural
    way, where a vertex is connect to its copies. Drawn in the plane,
    this results in a grid like construction.  We also triangulate the
    inner most copy $C_k$ in an arbitrary fashion, and every
    quadrilateral face is triangulated in an arbitrary fashion. The
    resulting graph $\Graph_C$ has $k^2$ vertices, and has the
    property that the any path between any two vertices of $C$ in
    $\Graph_C$, the corresponding shortest path in $C$ is shorter (or
    of the same length). See \figref{fill:in} for an example.

    \parpic[r]{\begin{minipage}{5cm}
           \includegraphics{figs/grid_face}
           \captionof{figure}{}
           \figlab{fill:in}
       \end{minipage}}

    We repeat this fill-in process for all the faces of $\Graph$, and
    let $\Graph'$ be the resulting graph. $\Graph'$ is still planar,
    and clearly the number of resulting vertices in the new graph is
    $N= \sum_{i} d_i^2$. Observe that $\sum_i d_i \leq 6n$, as every
    vertex $v$ incident on a face $r$, can be charged to an edge
    adjacent to both $v$ and $f$. Clearly, if done in a consistent
    fashion, an edge would be charged at most twice, and the maximum
    number of edges in a planar graph is $3n -6$ by Euler's formula.

    In particular, if the maximum value of $d_i$ is $d$, then maximum
    of $N = \sum_i d_i^2$ is $O(nd)$, as can be easily verified.

    Now, we assign weight zero to all the newly introduced vertices in
    $\Graph'$, and assign weight one for the original vertices (that
    appear in $\Graph$). The graph $\Graph'$ is a fully triangulated
    planar graph and it has $N$ vertices.  By \lemref{separator:w},
    there is separator providing the desired partition, and the number
    of vertices on this separator is $\leq 4 \sqrt{N}$. Since
    $\Graph'$ is triangulated, the separator is a simple cycle in
    $\Graph'$. We now replace portions of it that uses the face grids
    by the appropriate paths along the original boundary of the
    faces. Clearly, the resulting cycle in $\Graph$ has the same
    number of vertices, provide the same quality of separation (or
    better, since some vertices migrated to the separator), as
    desired.
\end{proof}

Miller's result is somewhat stronger than \lemref{separator:2:x}, as
he assumes the graph is $2$-connected, and can ensure that in this
case the separator is a \emph{simple} cycle.


\subsection{Ball systems that are $k$-ply}

A set of balls $\BallSet$ in $\Re^d$ is \emphi{$k$-ply}, if no point
of $\Re^d$ is contained in more than $k$ balls of $\BallSet$. 

\begin{defn}
    \deflab{doubling:constant}The \emphi{doubling constant} of a metric space is the smallest
    number of balls of the same radius needed to cover a ball of twice
    the radius (formally, we take the maximum such number over all
    possible balls to be covered). The doubling constant of $\Re^d$ is
    $\constDbl \leq 2^{O(d)}$ \cite{v-cbseb-05}.
\end{defn}


\begin{theorem}[\cite{mttv-sspnng-97}]\thmlab{sep:k:ply}Let $\BallSet$ be a set of $n$ balls that is $k$-ply in
    $\Re^d$. Then, there exists a sphere $\sphere$ that intersects $4
    k^{1/d} n^{1-1/d}$ balls of $\BallSet$. Furthermore, the number of
    balls of $\BallSet$ that are completely inside (resp. outside)
    $\sphere$ is $\geq n/(\constDbl+1)$.
\end{theorem}

\begin{proof}
    Let $\PntSet$ be the set of centers of the balls of $\BallSet$. As
    above, let $\ball$ be the smallest ball containing
    $n/(1+\constDbl)$ points of $\PntSet$. As above, assume that
    $\ball$ is centered at the origin and has radius $1$. Let
    $\sphere$ be a random sphere centered at the origin with radius
    $x$ picked randomly from the range $[1,2]$.

    Now, arguing as above, there are at most $\pth{\constDbl
       /(\constDbl + 1)}n $ points of $\PntSet$ inside $\sphere$, and
    as such, at least $(1- \constDbl/(\constDbl + 1 )) =
    n/(\constDbl+1)$ points of $\PntSet$ outside $\sphere$. As such
    $\sphere$ is a good separator for the balls.

    As for the expected number of balls intersecting $\sphere$, let
    $\Vd r^d$ be the volume of a ball of radius $r$ in $\Re^d$, where
    $\Vd$ is a constant that depends on the dimension. As above, we
    clip the balls of $\BallSet$ to the ball of radius $2$ centered at
    the origin, replacing every lens, by a an appropriate ball of the
    same volume. Let $\rad_i$ denote the radius of the $i$\th such
    ball $\ballA_i$, for $i=1,\ldots, n$. By the $k$-ply property, we
    have that
    \begin{align*}
        \sum_i \rad_i^d=\frac{1}{\Vd} \pth{\sum_i \Vd \rad_i^d}\leq \frac{k}{\Vd} \VolX{\MakeSBig \ballX{2}}\leq k 2^d,
    \end{align*}
    where $\ballX{2}$ denotes a ball of radius $2$ in $\Re^d$. As
    before, the probability of the $i$\th ball to intersect $\sphere$
    is bounded by $2\rad_i$.  Let $\SepSet$ be the set of balls of
    $\BallSet$ that intersects $\sphere$. We have, by \Holder's
    inequality, that
    \begin{align*}
        \Ex{\cardin{\SepSet} \MakeBig\! }&=\sum_i \Prob{ \ballA_i \cap \sphere \ne \emptyset \MakeBig }\leq \sum_i 2\rad_i = 2 \sum_i 1 \cdot \rad_i\leq 2 \pth{\sum_{i=1}^n 1^{d/(d-1)}}^{(d-1)/d} \pth{\sum_{i=1}^n
           \rad_i^d}^{1/d}\\ &\leq 2 n^{1-1/d} \pth{k 2^d }^{1/d} \leq 4 n^{1-1/d} k^{1/d},
    \end{align*}
    as desired.
\end{proof}


\subsection{Separators for the $k$\th nearest neighbor graph}

Let $\PntSet$ be a set of $n$ points in $\Re^d$, and let $k$ be a
parameter. The \emphi{$k$\th nearest neighbor graph} $\Graph_k =
(\PntSet, \Edges)$ is the graph, where two points $\pnt, \pntA \in
\PntSet$ are connected by an edge $\pnt \pntA \in \Edges$, if $\pntA$
is the $i$\th nearest neighbor of $\pnt$ in $\PntSet$ (or $\pnt$ is
the $i$\th nearest neighbor of $\pntA$), for $i \leq k$.

\begin{theorem}[\cite{mttv-sspnng-97}]
    Let $\PntSet$ be a set of $n$ points in $\Re^d$, and let $k$ be a
    parameter. The $k$\th nearest neighbor graph
    $\Graph_k = (\PntSet, \Edges)$ has a separator of size
    $O(k^{1/d} n^{1-1/d})$, such that each connected component has at
    most $\pth{\constDbl/(\constDbl +1)} n$ vertices, where
    $\constDbl$ is the doubling constant of $\Re^d$, see
    \defref{doubling:constant}.
\end{theorem}

\begin{proof}
    We follow the proof of Miller \etal \cite{mttv-sspnng-97}.  A
    point $\pntA \in \PntSet$ is an \emphi{$i$-client} of $\pnt \in
    \PntSet$, if $\pnt$ is the $i$\th nearest neighbor of $\pntA$, for
    $i\leq k$.  If $\pntA$ is a $k$-client of $\pnt$, then create a
    ball of radius $\distX{\pnt}{\pntA}$ centered at $\pntA$.  Let
    $\BallSet$ be the resulting set of $n$ balls. The key observation
    is that this set of balls is $O(k)$-ply -- which we reprove here
    using a standard argument.

    We claim that every point $\pnt \in \PntSet$ can serve at most
    $O(k)$ clients. To this end, cover the sphere of directions around
    $\pnt$ with cones with angular diameter at most $30^\degree$. It
    is easy to verify that at most $c = 2^{O(d-1)}$ such cones are
    needed.


    \parpic[r]{\includegraphics{figs/blocking}}

    The key observation is now that for any two points $\pntA,\pntB
    \in \PntSet$ that belong to the same cone $\Cone$ of $\pnt$, it
    must be that $\distX{\pntA}{\pntB} \leq \distX{\pnt}{\pntB}$,
    assuming that $\pntA$ is closer to $\pnt$ than $\pntB$, as an easy
    geometric argument shows. That is, if $\pntA_1,\ldots, \pntA_k$
    are the $k$ closest points to $\pnt$ in $\PntSet \cap \Cone$, then
    these are the only points of $\PntSet \cap \Cone$ that might be
    $k$-clients of $\pnt$. It follows that $\pnt$ can have at most $c
    k$ $k$-clients, and as such its degree in $\Graph_k$ is $\leq c k
    + k$. That is, the maximum degree of a vertex in $\Graph_k$ is
    $O(k)$.

    To see why this implies that the set of balls $\BallSet$ is
    $k$-ply, consider any point $\pnt \in \Re^d$, insert it into
    $\PntSet$, and observe that the degree of $\pnt$ in the graph
    $\Graph_{k+1}$ bounds the number of balls of $\BallSet$ that cover
    it. By the above, this is $O(k)$, as desired.

    By \thmref{sep:k:ply}, there are $4 k^{1/d} n^{1-1/d}$ balls of
    $\BallSet$, such that their removal breaks the intersection graph
    of $\BallSet$ into connected components each of size at most
    $\pth{ \constDbl/(\constDbl +1)} n$. Clearly, the corresponding
    set of points of $\PntSet$ is the desired separator of $\Graph_k$.
\end{proof}

\subsection{Separator for $r$ vertices in a planar graph}

Our purpose here is to show that in a triangulated planar graph, there
is always a cycle of size $O(\sqrt{r})$ that its removal separates
(roughly) $r$ vertices from remainder of the graph. To this end, we
need the following.

\begin{lemma}\lemlab{packing:r}Let $\BallSet$ be a set of $n$ balls in $\Re^d$ that are interior
    disjoint, and let $r> 0 $ be some prespecified integer number.
    Let $\ball$ be the smallest ball that contains $r$ centers of the
    balls of $\BallSet$. Then $\ball$ intersects at most
    $\pth{\constDbl}^2 \pth{r + 1} $ balls of $\BallSet$. Furthermore,
    $2\ball$ intersects at most $\pth{\constDbl}^3 \pth{r + 1} $
    balls of $\BallSet$, where $\constDbl$ is the doubling constant of
    $\Re^d$, see \defref{doubling:constant}.
\end{lemma}
\begin{proof}
    Assume $\ball$ is of radius one and it is centered at the
    origin. Consider the ball $4\ball$, and observe that it can be
    covered by $\pth{\constDbl}^2$ balls of radius one, and let
    $\BallSetA$ be this set of balls. As such, $4\ball$ contains at
    most $\pth{\constDbl}^2 r$ centers of balls of $\BallSet$. Any
    other ball of $\BallSet$ that intersect $\ball$ must be radius at
    least $3$, as its center is at distance at least $4$ from the
    origin.

    It is easy to verify that such a ball $\ball'$ must contain fully
    at least one ball of $\BallSetA$. Indeed, consider the segment
    connecting the center of $\ball'$ with the origin, and consider
    the point on this segment on $\partial 4\ball$. Clearly, this
    point must be covered by one of the balls of $\BallSetA$, and this
    ball is fully contained in $\ball'$.
\end{proof}

\begin{lemma}
    Let $\Graph$ be a planar graph with $n$ vertices, and let $r > 0$
    be an integer number which is sufficiently large. There exists a
    set of vertices $\SepSet$ of size $\leq 4 \constDblTwo \sqrt{r}$,
    such that $\Graph \setminus \SepSet$ is disconnected into two sets
    of vertices, $X$ and $Y$, such that $r/2\constDblTwo \leq
    \cardin{X} \leq r$, where $\constDblTwo$ is a constant (see
    \defref{doubling:constant}).  Furthermore, if $\Graph$ is
    triangulated then $\SepSet$ is a cycle in the graph.
\end{lemma}

\begin{proof}
    Let $\BallSet$ be the realization of $\Graph$ as a kissing graph
    of interior disjoint disks.  Let $\disk$ be the smallest disk
    containing $r/\constDblTwo$ centers of $\BallSet$, and assume that
    it is of radius one and centered at the origin.
    \lemref{packing:r} implies that $2\disk$ intersects at most
    $r\pth{\constDblTwo}^2$ disks of $\BallSet$, and let $\BallSetA$
    be this set of balls. Now consider the circle $C_x$ centered at
    the origin of radius $x$, where $x$ is picked randomly and
    uniformly from the range $[1,2]$. Let $\SepSet$ be the set of
    disks of $\BallSetA$ that intersects $C_x$. 

    Now, by the analysis of \lemref{small}, the expected number of
    disks of $\BallSetA$, and thus of $\BallSet$ that intersects $C_x$
    is
    $\leq 4 \sqrt{ \cardin{\BallSetA}} \leq 4
    \constDblTwo\sqrt{r}$.
    This implies that the number of disks strictly inside $C_x$ is at
    least
    $r/\constDblTwo - 4 \constDblTwo\sqrt{r} \geq r/2\constDblTwo$, if
    $r \geq 64 \pth{\constDblTwo}^4$. Similarly, it is easy to argue
    that $C_x$ contains at most $r$ disks of $\BallSet$.
\end{proof}


\section{Conclusions}

This write-up demonstrates that the planar separator theorem is an
easy consequence of the circle packing theorem, originally proved by
Paul Koebe in 1936 \cite{k-kdka-36}.  The circle packing theorem is
thus the ``true'' magic -- converting a topological property (a graph
being planar) into a packing property (i.e., disks touching each
other).

\paragraph{An open problem.}

The current algorithmic proofs of the circle packing theorem build an
evolving discrete structure that keeps improving after each iteration,
till in the limit it converges to the desired packing.  Specifically,
there is no finite algorithm that computes the realization of a planar
graph as a circle packing.

It seems unlikely that a finite algorithm is possible because of
numerical issues. However, a much weaker version is sufficient for the
planar separator theorem. In particular, can one find for a planar
graph a set of disks, such that two vertices are connected if and only
if their respective disks intersect (in their interiors), and no point
in the plane is contained in more than, say, $c$ disks of this set,
where $c$ is some universal constant (thus, we allow disks to
intersect even if their corresponding vertices are not connected in
the planar graph). We leave the development of such a finite
construction algorithm as an open problem for further research.






\section*{Acknowledgments}
The author thanks Mark \si{de} Berg, Timothy Chan, Robert Krauthgamer,
G\"u\si{nter} Rote, and Christian Sommer for useful comments on the
manuscript. The idea of using a ring area argument, in the proof of
\lemref{small}, came about during discussions with Mark \si{de}
Berg. G\"u\si{nter} Rote suggested the elegant tilling\footnote{Which
   is of course well known and not new.} depicted in \figref{hex}.






\providecommand{\CNFX}[1]{ {\em{\textrm{(#1)}}}}
  \providecommand{\tildegen}{{\protect\raisebox{-0.1cm}{\symbol{'176}\hspace{-0.03cm}}}}
  \providecommand{\SarielWWWPapersAddr}{http://sarielhp.org/p/}
  \providecommand{\SarielWWWPapers}{http://sarielhp.org/p/}
  \providecommand{\urlSarielPaper}[1]{\href{\SarielWWWPapersAddr/#1}{\SarielWWWPapers{}/#1}}
  \providecommand{\Badoiu}{B\u{a}doiu}
  \providecommand{\Barany}{B{\'a}r{\'a}ny}
  \providecommand{\Bronimman}{Br{\"o}nnimann}  \providecommand{\Erdos}{Erd{\H
  o}s}  \providecommand{\Gartner}{G{\"a}rtner}
  \providecommand{\Matousek}{Matou{\v s}ek}
  \providecommand{\Merigot}{M{\'{}e}rigot}
  \providecommand{\CNFSoCG}{\CNFX{SoCG}}
  \providecommand{\CNFCCCG}{\CNFX{CCCG}}
  \providecommand{\CNFFOCS}{\CNFX{FOCS}}
  \providecommand{\CNFSODA}{\CNFX{SODA}}
  \providecommand{\CNFSTOC}{\CNFX{STOC}}
  \providecommand{\CNFBROADNETS}{\CNFX{BROADNETS}}
  \providecommand{\CNFESA}{\CNFX{ESA}}
  \providecommand{\CNFFSTTCS}{\CNFX{FSTTCS}}
  \providecommand{\CNFIJCAI}{\CNFX{IJCAI}}
  \providecommand{\CNFINFOCOM}{\CNFX{INFOCOM}}
  \providecommand{\CNFIPCO}{\CNFX{IPCO}}
  \providecommand{\CNFISAAC}{\CNFX{ISAAC}}
  \providecommand{\CNFLICS}{\CNFX{LICS}}
  \providecommand{\CNFPODS}{\CNFX{PODS}}
  \providecommand{\CNFSWAT}{\CNFX{SWAT}}
  \providecommand{\CNFWADS}{\CNFX{WADS}}
\begin{thebibliography}{MTTV97}

\bibitem[AST94]{ast-ps-94}
\href{http://www.math.tau.ac.il/~nogaa/}{N.~{Alon}}, P.~Seymour, and R.~Thomas.
\newblock Planar separators.
\newblock {\em SIAM J. Discrete Math.}, 2(7):184--193, 1994.

\bibitem[Cha03]{c-ptasp-03}
\href{http://www.math.uwaterloo.ca/~tmchan/}{T.~M.~{Chan}}.
\newblock Polynomial-time approximation schemes for packing and piercing fat
  objects.
\newblock {\em J. Algorithms}, 46(2):178--189, 2003.

\bibitem[Koe36]{k-kdka-36}
P.~Koebe.
\newblock Kontaktprobleme der konformen {Abbildung}.
\newblock {\em Ber. Verh. S{\"a}chs. Akademie der Wissenschaften Leipzig,
  Math.-Phys. Klasse}, 88:141--164, 1936.

\bibitem[LT79]{lt-stpg-79}
R.~J. Lipton and R.~E. Tarjan.
\newblock A separator theorem for planar graphs.
\newblock {\em SIAM J. Appl. Math.}, 36:177--189, 1979.

\bibitem[Mil86]{m-fsscs-86}
G.~L. Miller.
\newblock {Finding small simple cycle separators for 2-connected planar
  graphs}.
\newblock {\em J. Comput. Sys. Sci.}, 32(3):265--279, 1986.

\bibitem[MTTV97]{mttv-sspnng-97}
G.~L. Miller, S.~H. Teng, W.~P. Thurston, and S.~A. Vavasis.
\newblock Separators for sphere-packings and nearest neighbor graphs.
\newblock {\em \href{http://www.acm.org/jacm/}{J. Assoc. Comput. {Mach.}}}, 44(1):1--29, 1997.

\bibitem[PA95]{pa-cg-95}
\href{http://www.math.nyu.edu/~pach}{J.~{Pach}} and \href{http://www.cs.duke.edu/~pankaj}{P.~K.~{Agarwal}}.
\newblock {\em Combinatorial Geometry}.
\newblock John Wiley \& Sons, 1995.

\bibitem[SW98]{sw-gsta-98}
W.~D. Smith and N.~C. Wormald.
\newblock Geometric separator theorems and applications.
\newblock In {\em Proc. 39th Annu. IEEE Sympos. Found. Comput. Sci.\CNFFOCS},
  pages 232--243, 1998.

\bibitem[Ung51]{u-tpg-51}
P.~Ungar.
\newblock A theorem on planar graphs.
\newblock {\em J. London Math. Soc.}, 26:256--262, 1951.

\bibitem[{Ver}05]{v-cbseb-05}
J.-L. {Verger-Gaugry}.
\newblock Covering a ball with smaller equal balls in {$\Re^n$}.
\newblock {\em \href{http://link.springer.com/journal/454}{Discrete Comput. {}Geom.}}, 33(1):143--155, 2005.

\end{thebibliography}



\appendix

\section{Proof of \lemref{small} with a better constant}
\apndlab{better:c}


\begin{proof}
    Consider a disk $\diskB_i$ of $\DSet$ of radius $r_i$ centered at
    $\pnt_i$.  If $\diskB_i$ is fully contained in $\diskA_2$ (the
    disk of radius $2$ centered at the origin), then the circle $C_x$
    intersects $\diskB_i$ if and only if $x \in [\norm{\pnt_i} - r_i,
    \norm{\pnt_i} + r_i]$, and as $x$ is being picked uniformly from
    $[1,2]$, the probability for that is at most $2r_i/|2-1| =
    2r_i$. For reasons that would become clear shortly, we set $\rad_i
    = r_i$ and $\diskC_i = \diskB_i$ in this case.

    \vspace{-0.3cm}
    \parpic[r]{\includegraphics{figs/diskify}}
    \vspace{0.3cm}

    Otherwise, if $\diskB_i$ is not fully contained in $\diskA_2$ then
    the set $L_i = \diskB_i \cap \diskA_2$ is a ``lens''. Consider a
    disk $\diskC_i$ of the same area as $L_i$ contained inside
    $\diskA_2$ and tangent to its boundary. Clearly, if $C_x$
    intersects $\diskB_i$ then it also intersects $\diskC_i$, see
    figure on the right. Furthermore, the radius of $\diskC_i$ is
    $\rad_i = \sqrt{\areaX{ \diskB_i \cap \diskA_2} /\pi}$, and, by
    the above, the probability that $C_x$ intersects $\diskC_i$ (and
    thus $\diskB_i$) is at most $2\rad_i$.

    Observe that as the disks of $\DSet$ are interior disjoint, we
    have that $\sum_i \rad_i^2 = \sum_i \areaX{\diskB_i \cap
       \diskA_2}/\pi \leq \AreaX{\diskA_2}/\pi = 4$.  Now, by
    linearity of expectation and the Cauchy-Schwarz inequality, we
    have that
    \begin{align*}\Ex{\cardin{S} \MakeBig\! }&=\Ex{\MakeBig\! \cardin{ \DSet \cap C_x}}=\sum_i \Prob{ \diskB_i \cap C_x \ne \emptyset \MakeBig }\leq \sum_i \Prob{ \diskC_i \cap C_x \ne \emptyset \MakeBig }\leq \sum_i 2\rad_i = 2 \sum_i 1 \cdot \rad_i\\& \leq 2 \sqrt{\sum_{i=1}^n 1^2} \sqrt{\sum_{i=1}^n \rad_i^2}\leq 2 \sqrt{n} \sqrt{4} = 4 \sqrt{n}. 
    \end{align*}
\end{proof}

\end{document}
