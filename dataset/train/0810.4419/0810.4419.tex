\documentclass{llncs}
\usepackage{amsmath,amsfonts,amssymb,stmaryrd,cmll}
\usepackage{graphicx}
\usepackage[nohug,midshaft]{diagrams}
\usepackage{tikz}
\usepackage[english]{babel}
\selectlanguage{english}



\hyphenation{co-mo-noid hir-scho-witz}
\DeclareMathOperator{\ob}{ob}
\DeclareMathOperator{\mor}{mor}
\newarrow{Eqto}=====




\newcommand{\pgph}{\paragraph{}}
\newcommand{\intro}[1]{\emph{#1}}
\newcommand{\bigcat}[1]{\mathsf{#1}}
\newcommand{\cat}[1]{\mathbf{#1}}
\newcommand{\fonc}[1]{\mathsf{#1}}
\newcommand{\dbr}[1]{\left \llbracket {#1} \right \rrbracket}
\newcommand{\cd}[2][]{\vcenter{\hbox{\xymatrix#1{#2}}}}
\newcommand{\Formulae}{\mathcal{F}}
\newcommand{\commentthis}[1]{}

\newcommand{\ie}{i.e.,}
\newcommand{\eg}{e.g.,}

\newcommand{\bsig}{\mathcal{K}} \newcommand{\model}{\mathcal{L}} \newcommand{\bbig}{\cat{Bbg}} \newcommand{\T}{\fonc{T}} \newcommand{\theory}{\mathcal{T}}
\newcommand{\sig}{\Sigma}
\newcommand{\A}{\mathcal{A}}
\newcommand{\C}{\mathcal{C}}
\newcommand{\D}{\mathcal{D}}
\newcommand{\M}{\mathcal{M}}
\newcommand{\N}{\mathcal{N}}
\newcommand{\V}{\mathcal{V}}
\newcommand{\E}{\mathcal{E}}
\newcommand{\X}{\mathcal{X}}
\renewcommand{\S}{\mathcal{S}}
\newcommand{\Sig}{\bigcat{SMCSig}} \newcommand{\Set}{\bigcat{Set}}
\newcommand{\Setp}{\bigcat{Setp}}
\newcommand{\SMCCat}{\bigcat{SMCCat}}
\newcommand{\occ}{\xi}

\newcommand{\ty}{\textsf{ty}}
\newcommand{\Nat}{\mathbb{N}} \newcommand{\smc}{\textsc{smc}}
\newcommand{\imll}{\textsc{imll}}
\newcommand{\fst}{\mathtt{i}_1}
\newcommand{\snd}{\mathtt{i}_2}
\newcommand{\abs}{\mathtt{abs}}
\newcommand{\app}{\mathtt{app}}
\newcommand{\sexp}{\emph{sexp}}
\newcommand{\loc}{\mathit{loc}}
\newcommand{\prnt}{\mathit{prnt}}
\newcommand{\link}{\mathit{link}}
\newcommand{\ctrl}{\mathit{ctrl}}
\newcommand{\var}{\mathit{var}}
\newcommand{\prt}{\mathit{prt}}
\newcommand{\lnk}{\mathit{lnk}}
\newcommand{\aux}{\mathit{aux}}
\newcommand{\pri}{\mathit{pri}}
\newcommand{\bin}{\mathit{bin}}
\newcommand{\MonL}{\fonc{Mon}(\Lambda)}
\newcommand{\ModL}{\fonc{Mod}(\Lambda)}
\newcommand{\LC}{\fonc{LC}}
\newcommand{\LG}{\fonc{LG}}
\newcommand{\preLG}{\fonc{preLG}}
\newcommand{\zero}{\mathbf{0}}
\newcommand{\un}{1}
\newcommand{\paral}{\mid}
\newcommand{\thg}{{\mathord{\text{--}}}}
\newcommand{\shape}[2]{[#1, #2]}
\newcommand{\transl}[1]{\llbracket #1 \rrbracket}
\newcommand{\rond}{\circ}
\newcommand{\id}{\mathrm{id}}
\newcommand{\iso}{\cong}
\newcommand{\impll}{\multimap}
\newcommand{\tens}{\otimes}
\newcommand{\bigtens}{\bigotimes}
\newcommand{\Bigtens}[1]{\displaystyle{\bigtens_{#1}}}
\newcommand{\Bigcup}[1]{\displaystyle{\bigcup_{#1}}}
\newcommand{\Coprod}[1]{\displaystyle{\coprod_{#1}}}
\newcommand{\card}[1]{|#1|}
\newcommand{\eqdef}{\stackrel{\mathtt{def}}{=}}
\newcommand{\ens}[1]{\{ #1 \}}
\newcommand{\inv}[1]{#1^{-1}}
\newcommand{\alt}{\mathrel{|}}
\newcommand{\name}[1]{\ulcorner #1 \urcorner}
\newcommand{\op}[1]{#1^{\mathit{op}}}

\title{Binding bigraphs\\as symmetric monoidal closed theories}

\author{Tom Hirschowitz\inst{1} \and Aur\'elien Pardon\inst{2}}
\institute{CNRS, Universit\'e de Savoie
  \email{tom.hirschowitz@univ-savoie.fr} \and ENS Lyon
  \email{aurelien.pardon@ens-lyon.fr}}

\begin{document}
\maketitle

\begin{abstract}
  \commentthis{Milner's bigraphs are a general framework for reasoning about
    distributed and concurrent programming languages. Notably, it has
    been designed to encompass both the pi-calculus and the Ambient
    calculus.

    This paper is only concerned with bigraphical syntax: given what
    we here call a bigraphical signature K, Milner constructs a (pre-)
    category of bigraphs BBig(K), whose main features are (1) the
    presence of relative pushouts (RPOs), which makes them
    well-behaved w.r.t. bisimulations, and that (2) the so-called
    structural equations become equalities. Examples of the latter
    are, e.g., in pi and Ambients, renaming of bound variables,
    associativity and commutativity of parallel composition, or scope
    extrusion for restricted names. Bigraphs follow a scoping
    discipline ensuring that, roughly, bound variables never escape
    their scope.

    Here, we reconstruct bigraphs using a standard categorical tool:
    symmetric monoidal closed (SMC) theories.  SMC structure is the
    categorical counterpart of intuitionistic multiplicative linear
    logic (IMLL): through the Curry-Howard-Lambek correspondence, an
    SMC signature corresponds to a set of IMLL axioms, and the free
    SMC category over a signature corresponds to IMLL proofs under the
    corresponding axioms, modulo cut elimination.

    Our construction refines the folklore view that bigraphs
    correspond to compact closed structure: SMC structure provides the
    expected input/output duality, but restricts it in order to
    enforce scoping.  Furthermore, it elucidates the slightly
    mysterious status of so-called edges in bigraphs. Finally, the
    obtained category is also considerably larger than the category of
    bigraphs, notably encompassing in the same framework terms and a
    flexible form of higher-order contexts.  We view this as a gain in
    modularity: we keep the same closed bigraphs, but provide more
    ways of cutting them into pieces, thanks to richer interfaces.  }

  We reconstruct Milner's~\cite{Milner:bigraphs} category of abstract
  binding bigraphs  over a signature  as the
  free (or initial) \emph{symmetric monoidal closed} (\smc) category  generated by a derived theory .  The
  morphisms of  are essentially proof nets from the
  Intuitionistic Multiplicative fragment (\imll) of Linear
  Logic~\cite{ll}.

  Formally, we construct a faithful, essentially injective on objects
  functor , which is surjective on
  closed bigraphs (i.e., bigraphs without free names or sites).  The
  functor is not full, which we view as a gain in modularity: we
  maintain the scoping discipline for whole programs (bound names
  never escape their scope) but allow more program fragments,
  including a large class of binding contexts, thanks to richer
  interfaces.  Possible applications include bigraphical programming
  languages~\cite{Damgaard:Matching} and Rathke and Soboci\'{n}ski's
  derived labelled transition systems~\cite{modularLTS}.
  \end{abstract}

\section{Overview}
Milner's (binding) bigraphs~\cite{Milner:bigraphs} are a general
framework for reasoning about distributed and concurrent programming
languages, designed to encompass both the
-calculus~\cite{Milner:pi} and the Ambient
calculus~\cite{Ambients}.  We are here only concerned with bigraphical
syntax: given what we call a \emph{bigraphical signature} ,
Milner constructs a \emph{pre-category}, and then a category , whose objects are \emph{bigraphical interfaces}, and whose
morphisms are bigraphs.

Its main features are (1) the presence of \emph{relative pushouts}
(RPOs) in the pre-category, which makes it well-behaved w.r.t.\
bisimulations, and that (2) in both the pre-category and the category,
the so-called \emph{structural} equations become equalities. Examples
of the latter are, e.g., in  and Ambients, renaming of bound
variables, associativity and commutativity of parallel composition, or
scope extrusion for restricted names. Also, bigraphs follow a scoping
discipline ensuring that, roughly, bound variables never escape their
scope.

In this paper, we reconstruct bigraphs using standard algebraic tools.
To explain this, let us quickly review the notion of a
\emph{many-sorted algebraic theory}, which is central in universal
algebra. It is specified by first giving a \emph{signature}---a set of
sorts  and a set  of operations with arities---together with
a set of \emph{equations} over that signature. For example, the theory
for monoids is specified by taking only one sort , and operations
 and , together with the usual
associativity and unitality equations. We may equally well view this
signature as given by a graph

with vertices being the objects of the free category with (strict)
finite products generated by .

In this paper, we use the same kind of theories, but replacing from
the start finite products with \smc\ structure.  \smc{} structure is
the categorical counterpart of \imll{}~\cite{ll,Trimble:phd}: through
the Curry-Howard-Lambek correspondence, an \smc{} signature amounts to
a set of \imll{} axioms, and the free \smc{} category 
over a signature  has as morphisms \imll{} proofs under the
corresponding axioms, modulo cut elimination.  We use an isomorphic
presentation of , essentially due to
Trimble~\cite{Trimble:phd}, in which morphisms are very much like
\imll{} \emph{proof nets}~\cite{ll}: they are kind of graphs, whose
\emph{correctness} is checked by (a mild generalisation of) the
well-known Danos-Regnier criterion~\cite{Danos:mll}.

Here, we translate any bigraphical signature  into an \smc{}
theory , and consider the free \smc\ category  generated by  as an alternative
category of bigraphs over .  To compare 
with Milner's , we construct a faithful functor . This functor is moreover
essentially injective on objects (i.e., two objects with the same
image are isomorphic), so that it is essentially an embedding.

However, our functor  is not full: even between bigraphical
interfaces,  contains morphisms which would be
ill-scoped according to Milner's scope rule. Neither is 
surjective on objects:  has many more objects
beyond bigraphical interfaces. This could be perceived as negative at
first sight, but it actually represents a gain in
modularity. Informally, even though our category is much more
versatile, it still prevents bound variables to escape their
scope. More formally, our functor is full on whole programs, i.e.,
bigraphs with no sites nor open names. Namely, it induces an
isomorphism on closed terms, i.e., of hom-sets  where  is the unit of tensor
product, and  is a particular object representing terms\footnote{We
  cheat a little here, see the actual result
  Lemma~\ref{lemma:closed}.}.  So, our additional interfaces and
morphisms allow just as many whole programs as Milner's, but more
program fragments. Notably, it contains both the equivalent of terms
and a kind of multi-hole, higher-order, binding contexts, all
cohabiting happily.

In passing, our functor fully elucidates the status of so-called
\emph{edges} in bigraphs: we translate differently \emph{free} edges
(used for name restriction, much like  in the -calculus) and
\emph{bound} edges (used for linking so-called \emph{binding ports} to
their peers). In the former case, we translate the edge into a 
node (which may also be understood as representing the private name in
question); in the latter case, we simply remove the edge, and rely on
our use of directed graphs to represent the flow from the binding port
to its peers.

Finally, our algebraic approach directly extends to what
Debois~\cite{Debois:phd} calls discrete sortings, which amount to a
many-sorted variant of bigraphs. Given a set  of sorts, instead of
only two sorts  and  for terms and variables (or names), one
starts with sorts  and  for each , which directly
allows to specify a typed signature.

\paragraph{Future work}
A possible application of our work is as an alternative representation
for bigraphical programming
languages~\cite{Damgaard:Matching}.  For example,
on the graphical side, the extensive litterature on efficient
correctness criteria for proof nets applies directly.  On the
syntactic side, \imll{} proofs provide an essentially algebraic
representation, i.e., one avoiding the use of variable binding and the
associated
trickery~\cite{PfenningElliott:hoas,PittsAM:newaas,hofmann:presheaf}.

Another possible application is as a handy foundation for Rathke and
\linebreak Soboci\'{n}ski's~\cite{modularLTS} work on deconstructing labelled
transition systems, which involves second-order (binding) contexts.

As to future research directions, we here only handle abstract
bigraphs, which do not have RPOs. We thus should generalise our
approach to deal with \emph{concrete} bigraphs, be it in the form of
Milner's original pre-category or in Sassone and Soboci\'{n}ski's
\emph{G-categories}~\cite{Sobocinski:grpos}, and then try to construct
the needed RPOs (or GRPOs).  

Another natural research direction from this paper concerns the
dynamics of bigraphs.  Our hope is that Bruni et
al.'s~\cite{Bruni:ccdc} very modular approach to dynamics may be
revived, and work better with \smc\ structure than with cartesian
closed structure. Specifically, with \smc\ structure, there is no
duplication at the static level, which might simplify matters.

\paragraph{Related work}
The construction of the free \smc\ category generated by an \smc\
theory is essentially due to Trimble~\cite{Trimble:phd}, followed by
others~\cite{Tan:phd,Hughes:freestar,Lamarche:nets}.  The construction
we use is a variant of Hughes'~\cite{Hughes:freestar} construction,
defined in our joint work with Richard Garner~\cite{GHP}.  It was
known that \smc\ (or cartesian closed) structure precisely represents
various kinds of variable binding (see, e.g.,
Barber~\cite{Barber:action}).  So we do not claim originality for the
use of \smc{} structure to recover bigraphs.

Sassone and Soboci\'{n}ski~\cite{Sobocinski:graphslics} share our goal of
categorically reconstructing bigraphs. They obtain very satisfactory
results for pure bigraphs as a bicategory of cospans over a particular
category of graphs. But their approach still has to scale up to deal
with binding.

Damgaard and Birkedal~\cite{Birkedal} axiomatise the category of
bigraphs as an equational theory over a term language with variable
binding. Our work may be seen as an essentially algebraic counterpart
of theirs (which relies on variable binding and the trickery evoked
above).  Moreover, they do not recognise the special status of bound
edges, so our construction is not a mere reformulation of theirs in
categorical terms.

Milner~\cite{Milner:bigraphs2} and Debois~\cite{Debois:phd}
propose other extensions of Milner's original scope condition, the
latter subsuming the former~\cite[Section~6.4]{Debois:phd}.  Debois
sees binding as a \emph{sorting} over the category of pure bigraphs
generated by , where  forgets the binding information.
His construction, starting from a `no bound name escapes its scope'
predicate, is reminiscent from Hyland and Tan's \emph{double glueing}
construction~\cite{Tan:phd}. However, his construction is not known to
satisfy any universal property, and an efficient (e.g., algebraic)
representation of it still has to be found. 

Finally, Grohmann and Miculan propose \emph{directed
  bigraphs}~\cite{grohmann} as a more flexible framework for
bigraphs. Hildebrandt~\cite{Hildebrandt:choco} proposes an algebraic
approach to them, using \emph{compact closed} categories instead of
\smc{} categories. This line of work does not handle scope directly:
one has to resort to a sorting in the above sense, with the same
inconveniences.

\paragraph{Summary of contributions}
In view of the above, our contribution thus mainly lies in working out
the details of the encoding of bigraphs as \smc{} theories, plus
unraveling and simplifying the status of edges. Our task is made
easier by Hughes' economic presentation of the free \smc{} category.

Furthermore, previous work using \smc{} structure as a representation
for binding mostly lead to conservative extensions, i.e., full and
faithful functors. Our work emphasises that Milner's \emph{ad hoc}
condition leaves room for generalisation, and provides a new,
canonical condition, which benefits from efficient criteria from
linear logical literature.


\paragraph{Structure of the paper}
Section~\ref{sec:smctheories} recalls the construction of the free
\smc{} category generated by an \smc\ theory, including a
specialisation to the case where a sort is equipped with a commutative
monoid object structure.  Section~\ref{sec:big} reviews bigraphs,
defining along the way our translation of bigraphical signatures. In
Section~\ref{sec:trans}, we construct our functor from bigraphs to the
corresponding free \smc\ category, and show that it is an isomorphism
on closed terms.


\section{Symmetric monoidal closed theories}\label{sec:smctheories}
This section reviews \smc{} theories and their free models; see our
note~\cite{GHP} for details. The constructions are essentially due to
Trimble~\cite{Trimble:phd}, but reworked using our extensions to
Hughes'~\cite{Hughes:freestar} presentation.

\subsection{Signatures}
We should start our overview with the definition of \smc{} categories:
these are symmetric monoidal categories, i.e., categories with a
tensor product  on objects and morphisms, symmetric in the
sense that , such that  has a
right adjoint , for each . We do not give further
details, since we are interested in describing the free such category,
which happens to be easier. Knowing that there is a category 
of \smc{} categories and strictly structure-preserving functors should
be enough to grasp the following.

As sketched in the introduction, a \emph{signature}  is given
by a set  of sorts, equipped with a graph whose vertices are
\imll{} formulae over , as defined by the grammar:

We think of each edge  of the graph as specifying an operation 
of type .
A morphism of signatures  is a 
function , equipped with a morphism of graphs, whose 
vertex component is ``'', i.e., it sends 
any formula  to .
This defines a category  of signatures.

There is then a forgetful functor  sending
each \smc{} category  to the graph with
\begin{description}
\item[vertices:] formulae in , and
\item[edges :] morphisms  in ,
  where  is defined inductively to send each syntactic
  connective to the corresponding function on .
\end{description}

Trimble~\cite{Trimble:phd} constructs an \smc{} category 
from any signature , which extends to a functor , left adjoint to : for any \smc\ category  and natural
transformation , there is a unique \smc\
functor  such that  decomposes as
\begin{diagram}[inline]
  X & \rTo^{\eta} & US (X) & \rTo^{U (f^*)} & U(\C).
\end{diagram}

\subsection{The free \smc{} category}
How does  look like?  Its objects are \imll{} formulae in
 and morphisms are kind of proof-nets with some
\emph{cells} representing operations.  A morphism from  to  in
 thus consists of
\begin{itemize}
\item a finite set  of cells labelled by
  operations  in ,
    \item and \emph{wires} connecting the \emph{ports} together.
\end{itemize}
For example in the -calculus, the operations  and
 correspond to cells:
\begin{center}
\includegraphics{example_get}
\end{center}
We always use the flat edge (or base) of the polygon to denote the domain. 
The orientation of a port corresponds to its sign, see below. 

A port is a leaf occurrence (atomic or ) in , , or in some
, .
Equivalently, a port is a leaf occurrence in the formula

(or  when  is empty).
In a formula, a port is \emph{positive} when it lies to the left of an even
number of 's, and \emph{negative} otherwise\footnote{The sign of a port
in  is directly apparent viewing  is a classical LL formula, see the next
paragraph.}.
The sign of a port in a morphism is its sign in the formula~\eqref{eq:ports}.

Wires are oriented, with sources the \emph{negative} ports of the
morphism, and targets in its \emph{positive} ports.  For each sort , the wires must induce a bijection between ports labelled 
(which we call  ports).  A negative  port can be wired to any
positive port.

An example of morphism is presented in Fig.~\ref{freesig:example:ii}, where
 has sorts  and operations , , ,  and .
\begin{figure}[ht] \centering
    \includegraphics{simple_example2}
    \caption{A morphism of .}
    \label{freesig:example:ii}
\end{figure}

\paragraph{Correctness}
But, crucially, not all such graphs qualify as morphisms of : they have to satisfy a correctness criterion essentially
due to Danos-Regnier~\cite{Danos:mll}, which goes as follows.
The formula~\eqref{eq:ports} may be written using the connectives
of \emph{classical} linear logic, defined by the grammar:

We have removed , now encoded as ; some
classical formulae are not expressible in \imll{}, such as , or
. The de Morgan dual  of  is defined as usual.

A \emph{switching} of a classical formula is its abstract syntax tree,
where exactly one argument edge of each  has been removed. A
\emph{switching} of a candidate morphism  is a graph obtained by
gluing along ports the wires of  with a switching of the
formula~\eqref{eq:ports}. The candidate is then \emph{correct} iff all
its switchings are acyclic and connected.

Equivalently, since~\eqref{eq:ports} is ultimately a , one may
separately glue  with switchings of , , and each
. Notably, a cell of type  for
some sort  is switched by connecting  to a switching of .

In Fig.~\ref{freesig:super_example} is pictured one switching (among 64) of
the graph underlying the morphism in Fig.~\ref{freesig:example:ii}. 
The displayed connectives are those of the formula~\eqref{eq:ports} translated
into classical linear logic. 
\begin{figure}[ht] \centering
    \includegraphics{simple_example2_swi}
    \caption{A connected and acyclic switching of the morphism in Fig.~\ref{freesig:example:ii}.}
    \label{freesig:super_example}
\end{figure}

\paragraph{Rewiring} 
Finally, morphisms are quotiented by Trimble \emph{rewiring}: a
morphism \emph{rewires} to another by changing the target of an edge
from some negative  port, as soon as this preserves
correctness. Rewiring is the smallest equivalence relation generated
by this relation.  
In Fig.~\ref{freesig:example:ii}, the dotted wire starting from the
negative  port can be rewired to any positive port because all switchings
would remain trees. 

\subsection{Theories}\label{subsec:co:monoids}
That gives the construction for signatures. We now extend it to \smc\
theories: define a theory  to be given by a signature ,
together with a set  of equations between morphisms in , for each \imll\ formulae .  The free \smc\
category  generated by such a theory is defined in our
note~\cite{GHP} to be the quotient of  by the
equations. Constructing  graphically is more direct than
could have been feared: we first define the binary predicate  relating two morphisms
 in  as soon as each  decomposes as
\begin{diagram}
  C & \rTo^{\iso} & I \tens C & \rTo^{\name{g_i} \tens C} & (A \impll
  B) \tens C & \rTo^{f} & D
\end{diagram}
with a common , with , and where 
is the currying of . Then, we take the smallest generated
equivalence relation, prove it stable under composition, and quotient
 accordingly, which yields the free \smc\ category  generated by the theory .

\paragraph{Commutative monoid objects}
We finally slightly extend the results of our note~\cite{GHP} to
better handle the special case of commutative monoids objects. This
will be useful in our translation of bigraphs, where the sort  of
terms has a commutative monoid structure given by parallel composition
and . Assume a theory  where a sort  is equipped
with two operations  and  as in~(\ref{eq:monoid}), with
equations making it into a commutative monoid object ( is
associative and commutative,  is its unit).  Further assume that
 and  do not occur in other equations.

Let  be the result of removing the operations  and  in
.  We define a relaxed version of our morphisms where each
negative  port is connected to a positive one,
but not necessarily bijectively.  This defines a category isomorphic
to , in which the operations  and  are built into
the linking.  The isomorphism is pictured in Fig.~\ref{monoid_to_c}.
\begin{figure}[ht] \centering
    \includegraphics{monoid_to_c}
    \caption{Contracting  cells and deleting  cells.}
    \label{monoid_to_c}
\end{figure}

\section{Binding bigraphs and the translation of signatures}\label{sec:big}
We now proceed to recall some definitions from Milner~\cite{Milner:bigraphs}, along which we give our translation of
bigraphical signatures  into \smc\ theories . We
then turn to our translation from the corresponding category of
bigraphs to the free model .

\subsection{Signatures}
\label{bigraph:sig}
\begin{definition}
  A \intro{bigraphical (binding) signature} is a 4-uple  where  is a set of \intro{controls},  are maps providing a \emph{binding} and a
  \emph{free} arity for each control and  is a set
  of \emph{atomic} controls.
\end{definition}
We fix such a bigraphical signature  for the rest of the paper. 
This signature can be translated into a \smc\ signature  over two
sorts , standing for terms and variables (or names).
It consists of the following \emph{structural} operations, accounting for the
built-in structure of bigraphs:

plus, for all controls , a \emph{logical} operation
 
where  if  is atomic and  otherwise.

We call  the theory consisting
of the operations in , with the equations making
\begin{itemize}
\item  into a commutative monoid object, 
\item  into a cocommutative comonoid object ( is
  coassociative, cocommutative, and  is its unit), and
\item  and  annihilate each other, as in
	\begin{center}
	    \includegraphics{bigraph_nw} 
	\end{center}
\end{itemize}
We now proceed to describe the category  of
\emph{abstract binding bigraphs} over , which we relate in
Section~\ref{sec:trans} to the free model  of
.

\subsection{Interfaces}
We assume an infinite and totally ordered set  of \emph{names}. 

\begin{definition}
  A \intro{bigraphical (binding) interface} is a triple  where  is a finite ordinal,  a finite set of
  names and  a function
  called \intro{locality map}.
\end{definition}
A name  is said \emph{global} if  and \emph{local} or
\emph{located at } when .

Bigraphical interfaces are the objects of the category .
We define a function  from these objects to \imll\ formulas, \ie\
objects of , by:

  where  and for all , .
  The ordering on  induces a bijection between  and   leaves in the formula. 

In \cite{Milner:bigraphs2}, Milner presents a slight generalisation of binding
bigraphs, where names have multiple locality.
Some interfaces cannot be simply translated into \imll\ formulas as before,
\eg\ if  is located in  and  and  in  and , this dependency
cannot be expressed directly in an \imll\ formula.

\subsection{Place graph}
Let  and  be two finite ordinals. 
\begin{definition}
    A \intro{place graph}  is a pair where:
    \begin{itemize}
	\item  is a finite set of \emph{nodes},
	\item  is a function called \emph{control map} and
	\item  is an acyclic function
	    called \emph{parent map} whose image does not contain any
	    atomic node.
    \end{itemize}
\end{definition}
The ordinals  and  index respectively the \emph{sites} and \emph{roots}.
A node is said \emph{barren} if it has no preimage under the parent map (atomic
nodes are thus barren). 

The relation  over sites, roots and nodes defined by:

is a (strict) partial order. 
The maximal elements of  are the roots; the minimal elements are the
barren nodes (including atomic nodes) and the sites. 

\subsection{Link graph}
Let  and  be two finite sets of names. 
\begin{definition}
    A \intro{link graph}  is a tuple
    where:
    \begin{itemize}
	\item  is a finite set of \emph{nodes},
	\item  is a finite set of \emph{edges},
	\item  is a \emph{control map} and
	\item  is a function called the \emph{link
	    map} 
    \end{itemize}
    with  being the set of \emph{ports}, \ie\ the coproduct of binding ports
    defined by  and free ports . 
    Moreover,  must satisfy the \emph{binding rule}:
    \begin{quote}
	For all binding ports , . 
    \end{quote}
\end{definition}

This binding rule is not mentionned in the original
paper~\cite{Milner:bigraphs} about bigraphs whereas it is mandatory
for the scoping discipline to be stable under composition (it is added
in~\cite{Milner:bigraphs2}). 
Alternatively, we can only require  to not be a global name of ;
the scope rule (defined below) handles the case of local names of . 

We define the \emph{binders} of our link
graph to be the local names of  (located at a root) and the binding
ports (located at a node) .
Two distinct \emph{points} (\ie\, two elements of )  and  are
\emph{peers} when . 
An edge is \emph{idle} when it has no preimage under the link map. 

\subsection{Abstract binding bigraphs}
\label{bigraph:sec:def}
Let  and  be two
bigraphical interfaces. 
\begin{definition}
    A bigraph  is a tuple where: 
    \begin{itemize}
	\item  is a place graph,
	\item  is a link graph, 
	\item  satisfies the \emph{scope rule}:
	    \begin{quote}
		If  is a binder located at , then each of its peers is located at
		some . 
	    \end{quote}
    \end{itemize}
\end{definition}

\begin{example}
  An example of bigraph, roughly corresponding to the -calculus
  context  where a global variable  is not used, is given in Fig.~\ref{bigraph:milner} using a
  representation from~\cite{Milner:bigraphs} and another from~\cite{Hildebrandt:choco}.  Binding (resp. free) names and ports are
  pictured by  (resp. ).
\end{example}
\begin{figure}[ht!]\centering
    \includegraphics{milner_big}
    \caption{Two bigraphical representations of the same -term.}
    \label{bigraph:milner}
\end{figure}

The binding rule ensures that no binding port  is peer of a name in
, hence  has to be an edge. Moreover, by acyclicity of
, no two binding ports may be peers, hence edges are linked to
at most one binding port.  The set of edges may thus be decomposed
into a set of \emph{free} edges  (without binding port) and a set
of \emph{bound} edges  in one-to-one correspondence with  by
the link map: .

Finally, two bigraphs are \emph{lean-support} equivalent when after discarding
their idle edges, there is an isomorphism between their sets of nodes and edges
preserving the structure. 

\begin{definition}
    The category  of \emph{abstract binding bigraphs over
    } has bigraphical interfaces as objects and lean-support equivalence
    classes of bigraphs as morphisms. 
\end{definition}

The composition of two bigraphs
\begin{diagram}[inline,width=.7cm]
  U_1 & \rTo^{G} & U_2 & \rTo^{G'} & U_3
\end{diagram} is defined by taking the
coproduct of their nodes, edges and control maps and the composition of parent
and link maps (modulo some bijections on sets), forgetting the roots/sites from
.
Acyclicity of the parent map, and the binding and scope rules are preserved by
composition.

\section{Translation}\label{sec:trans}

We now want to show how a binding bigraph  over  can be translated into a morphism  in the free model  of the \smc\
theory .  We have already defined  on objects
in~\eqref{eq:T:ob}. Now, let  and . We will define the support  of  as the disjoint union
of:
\begin{itemize}\item a \emph{logical} support  containing a
   cell for every node whose control is  and a  cell for
  every free edge in , and
\item a \emph{structural} support  consisting of  and 
  cells, which we define below.
\end{itemize}
We then specify the graph  for each sort in 
separately, and for .  For example, the image by  of the
bigraph in Fig.~\ref{bigraph:milner} is the morphism in
Fig.~\ref{freesig:example:ii}. 

\subsection{Places}

First, since  has a commutative monoid object
structure, the representation of Section~\ref{subsec:co:monoids}
applies: we just have to define a function from negative  ports to
positive ones.  Now, for any set  labeled in formulae, denote by
 its set of positive  ports, and similarly for
. Now, considering each cell  to be labelled by
the formula , we have:
\begin{itemize}
\item , because each type of cell  has one positive
   port,
\item , where
   is the set of non-atomic nodes, because there is
  one negative  port for each non-atomic cell,
\item , because for each  there is a
  positive  port in ,
\item similarly, , and finally
\item . 
\end{itemize}
Our morphism  is thus defined on the sort  by
the function :
\begin{diagram}[width=2cm,height=0.8cm]
  \T (U)^+_t + C^+_t + \T (W)^-_t & 
  \rTo^{\iso} & \T (U)^+_t + C^+_t & \rTo^{\iso} & n + V \\
  \dDashto<{f_t} &&&& \dTo>{\prnt} \\
  \T (U)^-_t + C^-_t + \T (W)^+_t & \lTo^{\iso} & \T (V)^+_t + C^-_t & \lTo^{\iso} & m + V_{\mathit{na}} \text{.}
\end{diagram}



\subsection{Links}
The function  for  requires more work, and involves defining
the structural support .  Recall that the data is the function
.

We start with an informal description of  based on
Fig.~\ref{bigraph:t_e}, in which bold arrows come from binders.
First, we deal with points sent to edges.  There are two kinds of
edges.  

First, we understand each free edge  as the creation of a fresh
name, and each free point  in  sent to  as an
occurrence of this free name. Accordingly,  is replaced by its
 cell in , and each  becomes a  port in .  We hence link the  port of the  cell to each
corresponding , through a tree of  and  cells, as depicted in
the bottom row.

Second, we understand each bound edge  as an indirection to its
binding port , itself understood as a bound name. We
further understand each free peer  of  as an
occurrence of the bound name.  Accordingly, we completely forget about
,  becomes a  port in , and each 
becomes a  port in , hence we link  to each
corresponding , again through a tree of  and  cells.

Finally, points  not sent to an edge are sent to some name .  But each such  becomes a  port in  and each
such  becomes a  port in , hence we link  to each
, again using  and  cells.  This determines the structural
support , as well as .  Finally, for the  part , each
negative  port arises from a structural  cell. But in
the above each cell is generated by \emph{one}  port (the fresh
name or the binder). In the former case, we may safely link our 
port to any valid  port. In the latter, the binder occurs to the
left of a , whose right-hand side is a  port, to which we
safely link our  port.

\begin{figure*}[t]\centering
    \includegraphics{bigraph_t_e}
    \caption{Translation of .}
    \label{bigraph:t_e}
\end{figure*}

More formally, observe from our translation of signatures and
interfaces, plus the logical support  defined above, that:\begin{itemize}
\item each free edge in  corresponds to one  cell,
  hence to one port in ,
\item each binding port in  corresponds to one negative
  occurrence of  in the domain of some cell in , hence to one
  port in ,
\item each local name in  corresponds  
  one port in .
\end{itemize}
Thus, we have an isomorphism .
Similarly, free points in  correspond to ports in , i.e., .

We may thus define a first function  by:


We then encode this function by a forest of  and  cells  (as
pictured in Fig.~\ref{function_to_cwtree}), to obtain a function
, which qualifies as the  part of our morphism. \begin{figure}[ht]\centering
    \includegraphics{function_to_cwtree}
    \caption{Translation of a function using  and  cells.}
    \label{function_to_cwtree}
\end{figure}The rest follows similarly.




This defines a function from bigraphs to candidate morphisms
(respecting domain and codomain).  We now show that it extends to a
functor.
\subsection{The functor}

First, we prove that the image of a bigraph is correct, i.e., is a
proper morphism.

\begin{lemma}\label{lemma:connected}
    All switchings of  are connected. 
\end{lemma}
\begin{proof}
  Essentially an induction over the place ordering .
\end{proof}
The following seems known \cite{Soloviev:imll}:
\begin{lemma}
  Any switching of a morphism in  is acyclic iff it
  is connected.
\end{lemma}
\begin{proof}[sketch]
    One proves by induction on the domain and codomain formulae that the
    graph induced by the switching has one more vertex than it has edges.
\end{proof}

\begin{lemma}
    The map  is a functor. 
\end{lemma}
\begin{proof}[sketch]
    The equations of  defined in Section~\ref{bigraph:sig}
    ensure that  behaves well w.r.t.\ composition and lean-support
    equivalence. 
\end{proof}


One sees at once that  is not full. For example, the morphism in
Fig.~\ref{bigraph:iso_1} has no preimage -- any such preimage would
violate the scope rule for bigraphs.
\begin{figure}[!ht] \centering
    \includegraphics{bigraph_iso_1}
    \caption{A correct morphism violating the scope rule.}
    \label{bigraph:iso_1}
\end{figure}
This example reflects that it is not necessary to distinguish global and local
variables in a bigraph with only one site. 
Nevertheless, the notion of scope is preserved by  because closed morphisms
can actually be translated into bigraphs.
In , define 
the interfaces 
and .
\begin{lemma}\label{lemma:closed}
  The functor  induces an isomorphism .
\end{lemma}
\begin{proof}[sketch]
  For any morphism , we construct a candidate bigraph,
  and observe that it vacuously satisfies the binding rule.  Then, we
  proceed by contrapositive: assuming either that its parent map is
  cyclic or that it breaks the scope rule, we show that  was
  incorrect.
\end{proof}


All in all, we have 
\begin{theorem}\label{thm:conservative}
  The functor  is faithful, essentially
  injective on objects, and surjective on .
\end{theorem}
It is however not full and far from surjective on objects.

\bibliographystyle{splncs}
\bibliography{bib}

\appendix


\section{Proof of Lemma~\ref{lemma:connected}}
Consider a switching of .

Given a site or a node , we denote by  the 
negative  port corresponding to it in the switching. If  is
a root, then  denotes the positive  port of its
image.

Free ports of a node  (resp. local names of a site ) have
their image (a positive  port) connected to 
(resp. ) as shown in Fig.~\ref{bigraph:connect_2}.
Moreover, either one negative  port (corresponding to
a binding port) or the positive  port of the cell  is
connected to  by the switched formula.
\begin{figure*}[th] \centering
  \includegraphics{bigraph_connect_2}
  \caption{Domain, codomain and a node of a switching.}
  \label{bigraph:connect_2}
\end{figure*}

We now prove by induction that all binding ports (located at a
node or a root ) have their image connected to .  Let
 be a binding port, and  its image by  (a
negative  port).

If  has no peers (this is necessarily the case if  is a
barren node), then  is connected to a  cell whose 
port is connected to .

If  has peers, then  is connected, in the morphism, to
their translations through a tree of  cells.  But this tree is
heavily switched and only connects  to one
positive  port  (whose preimage is) located, thanks to the
scope rule, to a site or a node .

By induction  is connected to  and  is connected to
 through the (unswitched) parent map. 
Indeed, the parent map connects the  ports of cells between  and
, and these cells have their  ports connected thanks to the induction
hypothesis.
The port  and  are thus connected. 

Finally, we remark that:
\begin{itemize}
\item roots are connected to each other in the codomain's formula (by their 
  or  ports, see Fig.~\ref{bigraph:connect_2}), 
\item global variables of the domain are connected to a site (by the domain's formula, see Fig.~\ref{bigraph:connect_2}) and
\item remaining negative  ports (global variable
  of the codomain and  cells) are connected to the other
  positive ports by a switched tree of  cells or a
   cell.
\end{itemize}
We conclude that all ports of our switching are connected.

\section{Proof of Lemma~\ref{lemma:closed}}
  Consider any .  We have 
  and , which justifies our
  ``induces'' above.  We now define  such that .

  Let the set of nodes  be the set of logical cells in ; the
  control map  sends each  cell to .  
  
  The set of edges is the coproduct of binding  ports in the
  support of  and of  cells (where a  port is binding when
  it occurs to the left of a , e.g., a cell of type  has two binding ports).

  The parent map  is exactly the restriction
  of  to  ports.  The link map  is obtained from the restriction of  to  ports
  as follows. From any  port , following the tree of
  contractions towards its root leads to a maximal positive
   port in the support, which may be either a port from a 
  cell, or a binding port of a logical cell.  In each case, there is a
  corresponding edge .  Our link map sends each port  to
  . Since in each tree there is only one root, the binding rule
  is respected.

  We then prove that  is correct.  Suppose that the parent map
  contains a cycle, then any switching where, for all cells of the
  cycle, the two  ports are connected contains this cycle.  Suppose
  that the scope rule is not satisfied for a binder  and one of its
  peers . Then, in ,  is the root of a contraction tree with
   as a leaf: among the switchings connecting them, choose again
  one that connects both  ports of each logical cell: every logical
  cell then has a path to the root  (the  port in the codomain),
  which forms a cycle involving , , and , hence
  contradicting correctness of .  The binding rule is automatically
  satisfied because the codomain has no name.  An atomic node has no
  antecedent in the parent map because the corresponding cell in 
  has no positive  port.

\end{document}
