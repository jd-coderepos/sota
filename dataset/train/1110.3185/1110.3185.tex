\documentclass[a4paper,11pt]{article}  
\usepackage{amsmath,amssymb,fullpage}
\usepackage{algorithmic}

\newcommand{\fib}[2]{\ensuremath{{\rm fib}_{#1}(#2)}}
\newcommand{\Fib}[1]{\ensuremath{{\rm Fib}_{#1}}}
\newcommand{\Bez}{\ensuremath{{\rm B\acute{e}zout}}}
\newcommand{\conj}[1]{\ensuremath{{\overline{#1}}}}
\newcommand{\proof}{\noindent {\sc Proof:}~}
\newcommand{\foorp}{\hfill }
 
\newcommand{\ch}{\ensuremath{\mathcal{C}}}
\def\A{\ensuremath{\mathbb{A}}}
\def\G{\ensuremath{\mathcal{G}}}
\def\N{\ensuremath{\mathbb{N}}}
\def\Z{\ensuremath{\mathbb{Z}}}
\def\C{\ensuremath{\mathbb{C}}}
\def\Q{\ensuremath{\mathbb{Q}}}
\def\F{\ensuremath{\mathbb{F}}}
\def\bQ{\ensuremath{\overline{\mathbb{Q}}}}
\def\a{\ensuremath{\mathbf{a}}}
\def\SS{\ensuremath{\mathbb{S}}}
\def\lm{\ensuremath{{\mathsf{lm}}}}
\def\lt{\ensuremath{{\sf lt}}}
\def\lc{\ensuremath{{\sf lc}}}
\def\l{\ensuremath{\langle}}
\def\r{\ensuremath{\rangle}}
\def\prc{\ensuremath{\preccurlyeq}}

\newtheorem{Theo}{Theorem}
\newtheorem{Lem}{Lemma}
\newtheorem{Cor}{Corollary}
\newtheorem{Prop}{Proposition}
\newtheorem{Def}{Definition}
\newtheorem{Not}{Notation}
\title{Structure of lexicographic Gr\"obner bases in three variables
of ideals of dimension zero}
\author{X.~Dahan\footnote{Supported by the GCOE program ``Math-for-Industry''
of Ky\^ush\^u university}\bigskip
\\
Dep\textsuperscript{t} of Mathematics, Ky\^ush\^u university, Japan\\
{\tt dahan@math.kyushu-u.ac.jp}}
\date{}
\begin{document}
\maketitle

\begin{abstract}
We generalize  the structural theorem of Lazard in 1985,
from 2 variables to 3 variables.
We use the Gianni-Kalkbrener result to do this,
which implies some restrictions
inside which lies the case of a radical ideal.
\end{abstract}

\section{Introduction}
Let  be a zero-dimensional ideal of a polynomial
ring  over a N\oe therian domain .
The {\em lexicographic order}
, for which
,
is put on the monomials 
of  
Given a polynomial ,
the {\em leading monomial} of ,
denoted  is the largest
monomial for  occurring in .
The coefficient in  in front of 
is called the {\em leading coefficient} of ,
denoted .
It might also be convenient to define the
{\em leading term} of 
denoted  equal to .

The {\em ideal of leading terms} of  is the ideal
of  generated by the leading terms of elements
of ; it is equal to .
Since  is N\oe therian, there is a finite set of generators
of this ideal.
A {\em Gr\"obner basis} of 
is a finite set of elements in ,
 such that .

In our case, we will take  a field.
Note that then 
is equal to .
This last ideal being a monomial ideal, it admits a minimal basis
of monomials ; Then a Gr\"obner basis
 is {\em minimal} if  for all .
It is  {\em monic} if  for all .

From now on, the monomial order will always be assumed
to be  and th symbol  will be omitted
in  and .

\begin{Not}
Consider the rings  and .
Given , let 
be the leading coefficient of  for the lexicographic order 
on  and let  be the leading coefficient
of .

Furthermore, let  and  be the monomials
such that
.
\end{Not}
Moreover, we make the following  assumption:
\smallskip

\noindent {\bf Assumption:}
The ideal  will be supposed {\em zero-dimensional},
or, equivalently the -algebra  is supposed finite.
We are given a minimal and monic Gr\"obner basis 
 of ,
indexed in a way that .
 \smallskip

We recall some basic facts about the Gr\"obner basis : 
\begin{itemize}
\item   and  for some 
(we say that  is {\em pure power} of ).
\item Moreover, there exists  such that:
 is a pure power of  and
such that
 for ;
and  for .
\item {\bf Elimination property:} the set of polynomials
 is a minimal lexicographic
Gr\"obner basis of the zero-dimensional ideal .
\end{itemize}

In 1985, Lazard in~\cite{Laz85}  proves the following.
\begin{Theo}[D. Lazard]
Let  be a zero-dimensional ideal,
and  a minimal lexicographic Gr\"obner basis
of  for . Then:

\end{Theo}
It follows easily a factorization property of the polynomials
in such a Gr\"obner basis~\cite[Theorem~1~(i)]{Laz85}.
However, the formulation above
is more compact and handy, and is equivalent.
The main result of this paper is the following analogue
in the case of 3 variables:
\begin{Theo}\label{th:main}
Let ,  and   be defined as above.
Then, for all  such that the variable 
appears in the monomials  and  with the {\em same
exponent}, holds:

Furthermore, in the later case, for all , .  
\end{Theo}
The proof will occupy the next section.
There is one corollary to this theorem
in the context of ``stability of Gr\"obner bases under specialization'',
which generalizes the theorem of Gianni-Kalkbrener~\cite{Gi87,Ka87},
and improves the theorem of Becker~\cite{Be94}
(but  holds only with 3 variables).
\begin{Cor}\label{cor:main}
Let us assume   radical.
Let  be a root of ,
, ,
and  a polynomial
among the Gr\"obner basis.
Then, either , or .
This implies that: ,
and in particular, that  is a Gr\"obner basis.
\end{Cor}
\proof By Theorem~\ref{th:main},
we can write  with .
Hence, if , then .
Else, since , we get
. But ,
from which follows .
On the other hand, .
\foorp
\medskip

Gianni-Kalkbrener's result~\cite{Gi87,Ka87} concerns the easier
case where all the variables but the largest one for  are specialized.
\smallskip

\noindent {\bf Gianni-Kalkbrener.}
The map  is therein , 
 for 
a solution of the system .
For any  in the Gr\"obner basis 
such that ,
they show that either  or
,
which implies
.
\smallskip


Becker~\cite{Be94} has generalized partly this result to the case
of a map  that specializes the  lowest variables
for . Taking , this covers the case of Corollary~\ref{cor:main},
but is weaker: it does also say that  remains a Gr\"obner basis,
while assuming that for ,
 may be a term with a monomial
strictly smaller for  than the monomial in the term  (see the definition
of the integer  during the proof of Prop.~1  page~4
of~\cite{Be94}.
With the notations on the same page of~\cite{Be94} we see ;
Corollary~\ref{cor:main} above implies ).
It can not be said that: . 

Concerning  previous works, let us mention that Kalkbrener~\cite{Ka97}
has expanded Becker's result to the more general elimination monomial
orders. Still, staying in the purely lexicographic case,
it does not enhance  the theorem of Becker.

\section{Proof of Theorem~\ref{th:main}}
The main ingredient of the proof consists in generalizing two lemmas of Lazard.
These refers to Lemma~2, and Lemma~3 of~\cite{Laz85}.
We shall explain that a weaker form holds with a larger
number of variables.
The version of interest here concerns the case of 3 variables.
It is nonetheless easy to produce a version with an arbitrary number
of variables.
Let us first introduce some notations
for exponents:
\begin{Not}
Let  non zero, with leading monomial
. The 3 notations ,
 and  will denote
 and  respectively.

If  is among the Gr\"obner basis ,
the shortcuts 
will be used instead of 
\end{Not}

\begin{Prop}\label{prop:1}
Let  be such that

and .
Then  divides .
\end{Prop}
\proof
Let .
The multivariate division algorithm with respect
to  of  by  gives:

and  does not
divide any monomial occurring in .

By definition of , 
so that , hence 
holds:

By an elementary property of the lexicographic order ,
this implies  and therefore
.
Next, the equality  gives:

Again, property of lexicographic order
implies 
and if 
then .
We distinguish three cases;
in the first two ones the conclusion
of the theorem holds, and the third case
never happens.
\smallskip

{\em Case 1:} .
Then , and ,
this concludes the proof. 
\smallskip

{\em Case 2:} Else ,
and . Similarly,
this shows that ,
concluding the proof.
\smallskip

{\em Case 3:} Else 
and .
Since ,
necessarily .
On the other hand, 
implies that there exists 
such that .
Therefore,
,
and in this case ,
.
This means ,
and , which is impossible since 
the Gr\"obner basis is minimal. \foorp


\begin{Prop}\label{prop:2}
For any , the polynomial   of the
 the Gr\"obner basis  verifies:  divides .
\end{Prop}
\proof
Define,

Note that  is well-defined
because  and .
This also shows that  is well-defined.
By Proposition~\ref{prop:1}, 
divides . Let

By construction, . Furthermore, 
so its normal form modulo the Gr\"obner basis
of  is 0. The multivariate division equality
with respect to  of  by 
is written: .
If , then . The inequality  follows, which is possible 
only if .
Otherly said, .

It follows that
,
and that:

with   if 

and
  if .
However 
and 
imply that .
In particular 
and consequently .
By definition of ,
this gives:
.
Proposition~\ref{prop:1} then yields: .


To conclude, note that Lazard's Lemma~4
in~\cite{Laz85}
proves that Prop.~\ref{prop:2} is true for
. So we can proceed by induction
on  and assume that 
for . Applied in Equation~\eqref{eq:lc2a}:

Finally, .
\foorp
\medskip

This proves the first part of Theorem~\ref{th:main}.
The second part is based upon the previous proposition and the theorem
of Gianni-Kalkbrener.
The use of the later requires a restriction:
\begin{Prop}\label{prop:3}
Suppose there is an 
such that: ,
there is a root   of 
which is not a root of .
Then, 
and .
\end{Prop}
\proof Since , by Proposition~\ref{prop:2},
 as well. By Gianni-Kalkbrener,
this implies .
Furthermore,  ,
implying  is not zero. 
Let  be a root of this polynomial.
By Gianni-Kalkbrener, ,
showing that .
\foorp
\medskip

Note that if  is radical, all elements 
for which  verify the assumption
on the root 
of  Proposition~\ref{prop:3}. By an elementary use
of the Chinese remaindering theorem, we get the more general,
.
This proves the last part of Theorem~\ref{th:main}.

\section*{Conclusion}

It is likely that Theorem~\ref{th:main}
holds without the assumption  radical.
This assumption was set to allow the use
of Gianni-Kalkbrener's result. A proof
circumventing it must be found.
Also, some experiments shown  that the results presented
here are certainly true in the case of more than 3 variables.



\bibliographystyle{plain}
\bibliography{/Users/dahan/main}
\end{document}
