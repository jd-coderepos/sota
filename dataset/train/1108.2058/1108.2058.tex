\documentclass{llncs}

\usepackage{microtype}





\usepackage{latexsym,amssymb,amsmath}

\usepackage{cite}

\usepackage{wrapfig} 

\usepackage{graphicx}





\newtheorem{observation}{Observation}


\usepackage[usenames,dvipsnames]{color}
 \newcommand{\boris}[2][wrote]{*** \textcolor{blue}{\textsc{Boris #1:} #2} ***}
 \let\Boris\boris
 \newcommand{\muriel}[2][says]{*** \textcolor{WildStrawberry}{\textsc{Muriel #1:} #2} ***}
 \let\Muriel\muriel
 \newcommand{\ferran}[2][comments]{*** \textcolor{green}{\textsc{Ferran #1:} #2} ***}


\renewcommand{\P}{\mathcal{P}}
\newcommand{\C}{\mathcal{C}}
\DeclareMathOperator{\RG}{RG} \DeclareMathOperator{\RIG}{RIG} \DeclareMathOperator{\MNG}{MNG} \DeclareMathOperator{\GG}{GG} \DeclareMathOperator{\DG}{DG} \DeclareMathOperator{\WRG}{WRG} 





\usepackage{paralist}


\newenvironment{denseitems}{\list{}{\labelwidth3em\itemsep0pt\parsep0pt\topsep0.6ex}}{\endlist}



\title{Witness Rectangle Graphs\thanks{Research of B.A.\ has been partially supported by NSA MSP
    Grants H98230-06-1-0016 and H98230-10-1-0210.  Research of B.A.\
    and M.D. has also been supported by a grant from the U.S.-Israel
    Binational Science Foundation and by NSF Grant CCF-08-30691.
    Research by F.H. has been partially supported by projects MEC
    MTM2006-01267 and MTM2009-07242, and Gen.~Catalunya DGR
    2005SGR00692 and 2009SGR1040.
A preliminary version of this paper appeared in the \emph{Proceedings of
  the Algorithms and Data Structures Symposium (WADS'11)} \cite{ADH11}.}}


\author{Boris Aronov\inst{1}\and
  Muriel Dulieu\inst{1}
  \and
  Ferran Hurtado\inst{2}
}

\institute{Department of Computer Science and Engineering, Polytechnic 
 Institute of NYU, Brooklyn, NY~11201-3840, USA;
 \email{aronov@poly.edu}, \email{mdulieu@gmail.com}
 \and Departament de Matem\`{a}tica Aplicada~II,
    Universitat Polit\`{e}cnica de Catalunya (UPC),
    Barcelona, Spain. \email{ferran.hurtado@upc.edu}}


\intextsep 5pt plus 1pt minus 1pt
\headsep 8pt
\footnotesep 6pt
\textfloatsep 5pt plus 1.5pt minus 2pt


\makeatletter
\renewcommand\section{\@startsection{section}{1}{\z@}{-8\p@ \@plus -2\p@ \@minus -2\p@}{6\p@ \@plus 2\p@ \@minus 2\p@}{\normalfont\large\bfseries\boldmath
                        \rightskip=\z@ \@plus 4em\pretolerance=10000 }}
\makeatother

\clubpenalty=001
\widowpenalty=001
\begin{document}
\maketitle

\begin{abstract}
    In a \emph{witness rectangle graph} (WRG) on vertex point set 
    with respect to witness point set~ in the plane, two points
     in  are adjacent whenever the open isothetic rectangle with  and
     as opposite corners contains at least one point in
    .  WRGs are representative of a larger family of witness
    proximity graphs introduced in two previous papers.



    We study graph-theoretic properties of WRGs.  We prove that
    any WRG has at most two non-trivial connected components.  We
    bound the diameter of the non-trivial connected components of a
    WRG in both the one-component and two-component cases.  In the
    latter case, we prove that a graph is representable as a
    WRG if and only if each component is a connected co-interval graph, thereby
    providing a complete characterization of WRGs of this type.  We
    also completely characterize trees drawable as WRGs. 
    In addition, we prove that a WRG with no isolated vertices has domination number at most four.
    
    Moreover, we show that any combinatorial graph can be drawn as a WRG using a combination of 
    positive and negative witnesses.

    Finally, we conclude with some related results on the number of
    points required to stab all the rectangles defined by a set of
    ~points.
  \end{abstract}


\pagestyle{plain}

\section{Introduction}\label{section:introduction}

\emph{Proximity graphs} have been widely used in situations in which there is a need of expressing the fact that some objects in a given set---which are
assigned to nodes in the graph---are close, adjacent, or neighbors, according to some geometric, physical, or conceptual
criteria, which translates to edges being added to the corresponding graph.
In the geometric scenario the objects are often points and the goal is to analyze the shape or the structure of the set of spatial data they describe or represent. This situation arises in Computer Vision, Pattern Recognition, Geographic Information Systems, and Data Mining, among other fields. The paper \cite{JT92} is a survey from this viewpoint, and several related papers appear in \cite{T88}. In most proximity graphs, given a point set , the adjacency between two points  is decided by checking
whether in their \emph{region of influence} there is no
other point from , besides  and . One may say that the presence of another point
is considered an \emph{interference}. There are many variations,
depending on the choice of the family of influence regions \cite{JT92,Co06,Li08}.



Given a combinatorial graph , a \emph{proximity drawing} of  consists of
a choice of a point set  in the plane with , for a given criterion of neighborhood
for points, such that the corresponding proximity graph on  is
isomorphic to .  This question
belongs to the subject of \emph{graph drawing problems}, in which the emphasis is on geometrically representing graphs
with good readability properties and fulfilling some aesthetic criteria \cite{BETT98}. The main issues are to characterize the graphs that admit a certain kind of representation, and to design efficient algorithms for finding such a drawing,
whenever possible.

Proximity drawings have been studied extensively and utilized widely \cite{BLL94,Li08}. However, this  kind of representation is
somehow limited and there have been some attempts to expand the class, for example using weak proximity graphs \cite{BLW06}.
Another recently introduced generalization is the
concept of \emph{witness proximity graphs} \cite{ADH,ADH08}, in which the adjacency of points in a
given vertex set  is decided by the presence or
absence of points from a second
point set ---the \emph{witnesses}---in their region of
influence. This generalization includes the classic proximity graphs
as a particular case,
and offers both a stronger tool for neighborhood description and much more flexibility for graph representation purposes.

In the \emph{positive witness} version, there is an adjacency when
a witness point is covered by the region of influence.  In the \emph{negative witness} version,
two points are neighbors when they admit a region of influence free of any witnesses.
In both cases the decision is based on the presence or absence of witnesses
in the regions of influence, and a combination of both types of
witnesses may also be considered; refer to Section~\ref{sec:PlusMinusW}. Observe that by taking  playing a negative role, we recover the original proximity
graphs; so this is a proper generalization.
Witness graphs were introduced in \cite{ADH}, where the focus is on the generalization of \emph{Delaunay graphs}. The witness version
of \emph{Gabriel graphs} was studied in \cite{ADH08}, and a thorough exploration of this set of problems is the main topic
of the thesis \cite{thesis}.

In this paper, we consider a positive witness proximity
graph related to the rectangle-of-influence graph, the
witness rectangle graph.  The \emph{rectangle of influence graph}
, also named the Delaunay graph of a point set in the plane with respect to axis-parallel rectangles~\cite{Janos}, is usually studied as one of the basic proximity graphs
\cite{LLMW98,Li08}.
In ,  are adjacent when the \emph{rectangle
   they define} covers no third point from ;  is
the unique open isothetic rectangle with  and  at its opposite corners.
In other words, two points are adjacent to each other when they are rectangularly visible~\cite{RectangularVisibility1, RectangularVisibility2, RectangularVisibility3}.
Much effort has been devoted to the rectangle of influence drawability problem~\cite{LLMW98, RectangleDrawability1, RectangleDrawability2, RectangleDrawability3, RectangleDrawability4, RectangleDrawability5, RectangleDrawability6}.
The \emph{witness rectangle graph}~(WRG) of vertex point set  (or,
simply, \emph{vertices}) with respect to witness point set 
(\emph{witnesses}), denoted , is the graph with the vertex
set , in which two points  are adjacent when the
rectangle  contains at least one point of .
The graph  has no edges.  When  is
sufficiently large and appropriately distributed,  is
complete. 
Ichino and Slansky's \cite{MNG} \emph{mutual neighborhood graph} 
is precisely the negative-witness version of the witness rectangle 
graph. See further discussion in Section~\ref{sec:MNG}.
We also note that a
negative-witness version of this graph with  would be precisely
 discussed above;
in fact  is precisely the complement of
.     An example is shown in
Figure~\ref{fig:RGexample}. 

We show in this paper that the connected components of WRGs are geometric examples of graphs with small diameter; these have
  been attracting attention in the pure graph theory setting \cite{SmallDiameter1, SmallDiameter2}, and are far from being
  well understood, even for diameter two \cite{diameterTwo1,diameterTwo2,diameterTwo3}.
  We prove below that the maximum domination number of a WRG with no isolated vertices is four, which we find interesting, considering that the domination number of sufficiently large \emph{planar} graphs of diameter three is at most seven~\cite{Domination, Domination2}.

\begin{figure}[htbp!]
  \centering
  \includegraphics[scale=0.8]{BRGexample}
  \caption{Left: A set of points  and a witness set
    . Right: the witness rectangle graph .  In
    all our figures for WRGs solid dots denote vertices and dots with
    a cross denote (positive) witnesses.  }
  \label{fig:RGexample}
\end{figure}

Besides some computational issues, such as the construction of  for given sets 
and  in an output-sensitive manner, we study several graph-theoretic properties of WRGs: (a)~We
completely characterize trees drawable as WRGs (Theorem~\ref{tree}).
(b)~We argue that any WRG has zero, one, or two \emph{non-trivial} connected
components (see the definition below and Theorem~\ref{thm:rectangleStructural}). (c)~We prove
that the diameter of a single-component WRG is at most six, and that this bound is tight in the worst case
(Theorem~\ref{thm:rectangleStructural} and subsequent discussion). (d)~We prove that the diameter
of a (non-trivial) connected component of a two-component WRG is at
most three and this can be achieved in the worst case
(Theorem~\ref{thm:rectangleStructural}).
(e)~In the two-component case, we provide a complete characterization
of graphs representable as WRGs. Such graphs, disregarding
isolated vertices, are precisely disjoint unions of two connected co-interval
graphs (Theorem~\ref{staircase}).
This last result allows us to recognize in linear time if a combinatorial graph with two
non-trivial components can be drawn as a WRG, and to construct such a drawing if it exists.
(f)~We prove that the maximum domination number of a WRG with no isolated vertices is four.



In Section~\ref{sec:MNG}, we give a counterexample to a theorem of
Ichino and Slansky in~\cite{MNG}.
In Section~\ref{sec:PlusMinusW}, we show that any combinatorial graph can be drawn as a WRG with positive and negative witnesses, using a quadratic number of witnesses.
Finally, in Section~\ref{section:stabbing}, we present some related
results on stabbing rectangles defined by a set of points with other
points.  They can be interpreted as questions on ``blocking''
rectangular influences.

\paragraph*{Terminology and notation.}
Throughout the paper, we will work with finite point sets in the
plane, in which no two points lie on the same vertical or the same
horizontal line.

Hereafter, for a graph  we write  or  to
indicate that  are adjacent in , and generally use
standard graph terminology as in \cite{CL04}.
When we speak of a \emph{non-trivial connected component} of a graph,
we refer to a connected component with at least one edge (and at least
two vertices).

Given two graphs  and  with disjoint
vertex sets, their \emph{join} is the graph ~\cite{mathworld:join}.

\paragraph*{How to compute a witness rectangle graph.}

De Berg, Carlsson, and Overmars \cite{BCO92} generalized the notion of 
dominance, in a way that is closely related to witness rectangle graphs, by defining dominance pairs  of a set of points , with respect to 
a set  of so-called obstacle points.
More precisely,  is said to \emph{dominate  with respect to } if there is no point  such that  dominates  and  dominates .
Recall that  \emph{dominates}  if and only if ,
, and .

They prove the following theorem:




\begin{theorem}[De Berg, Carlsson, and Overmars \cite{BCO92}]
All dominance pairs in a set of points  with respect to a set of points  can be computed in time 
, where  and  is the number of answers.
\end{theorem}

Collecting all dominance pairs in a set of points  with respect to a set of points , and repeating the procedure after rotating the plane by , one obtains the negative version of the witness rectangle graph.
A simple modification of their algorithm yields the positive version:

\begin{corollary}
  Let  and  be two point sets in the plane.  The
  witness rectangle graph  with ~edges can be computed
  in  time, where .
\end{corollary}

\section{Structure of Witness Rectangle Graphs}\label{section:rectangles}

Let  be the witness rectangle graph of vertex set 
with respect to witness set .
We assume that the set of witnesses is \emph{minimal}, in the sense that
removing any witness from  changes .  Put
 and let  be the edge set of .  We
partition  into  and  according to the slope sign of the
edges when drawn as segments.  Slightly abusing the terminology we refer
to two edges of  (or two edges of ) as having \emph{the same
  slope} and an edge of  and an edge of  as having
\emph{opposite slopes}.

\begin{wrapfigure}[7]{r}{0pt}
\includegraphics[width=4.5cm]{cornerBay}
\end{wrapfigure}
Recall that the open isothetic rectangle (or \emph{box}, for short) defined
by two points  and  in the plane is denoted ; for an
edge  we also write  instead of .  Every edge ,
say in , defines four regions as in the figure on the right,
that we call \emph{(open) corners} and \emph{(closed) bays}.






\begin{observation} \label{obs:quadrants} Every  inside a
  corner of an edge  is adjacent to at least one endpoint of .
\end{observation}

Note that for any , , and , the graph 
is an induced subgraph of , so the class of graphs
representable as WRGs is closed under the operation of taking induced
subgraphs.

Two edges are \emph{independent} when they share no vertices and the
subgraph induced by their endpoints contains no third edge.  Below we
show that  cannot contain three pairwise independent edges, which
imposes severe constraints on the graph structure of .

\begin{lemma}
  \label{lem:2sameSlope}
  Two independent edges in  (respectively, ) cannot cross or
  share a witness.  The line defined by their witnesses is of negative
  (respectively, positive) slope.
\end {lemma}
\begin{proof} Let the two edges be , with 
  and .  A common witness would have  and  in its
  third quadrant and  and  in the first, implying  and
  , a contradiction.  If  and  cross, assume
  without loss of generality that .  Neither  nor 
  can be inside  (because of Observation~\ref{obs:quadrants})
  and hence , implying  or
  , a contradiction.  Finally, the second part of the
  statement is a direct consequence of
  Observation~\ref{obs:quadrants}. \hfill 
\end{proof}

\begin{lemma}
  \label{lem:2oppositeSlope}
  Two independent edges with opposite slopes must share a witness.
\end {lemma}
\begin{proof}
  Let  and  be independent.  Let  be a
  witness for  and let  be a witness for .  The points 
  and  are not in quadrants I~or~III of , as otherwise the two
  edges would not be independent.
If  is shared, we are done.  Otherwise it cannot be that  lies
  in quadrant~II of  while  lies in quadrant~IV, or vice versa.
  Therefore  and  are either both in quadrant~II or both in
  quadrant~IV of .  Assume, without loss of generality, the former
  is true.  The witness  is not outside of , as we would
  have  and/or  (assuming, without loss of
  generality, that ) and the edges would not be
  independent.  Therefore  is in , so  is a shared
  witness, as claimed. 
    \hspace*{0pt} \hfill 
\end{proof}




\begin{lemma}
  \label{lem:no3independentSameSlope}
  There are no three pairwise independent edges in  (or in
  ).
\end{lemma}
\begin{proof} Assume that three pairwise independent edges  in  are witnessed by , respectively, with
  .  Then, by Lemma~\ref{lem:2sameSlope},
  .  By the same lemma at least one endpoint of
   is in the second quadrant of  and at least one endpoint
  of  is in its fourth quadrant, contradicting their independence. \hfill 
\end{proof} 















\begin{lemma}
  \label{lem:no3independent}
  A witness rectangle graph does not contain three pairwise independent edges.
\end{lemma}

\begin{proof}
  By Lemma~\ref{lem:no3independentSameSlope}, two edges  and 
  of the three pairwise independent edges , , and  have
  opposite slopes.  By Lemma~\ref{lem:2oppositeSlope},  and 
  share a witness point .  Every quadrant of  contains one of
  the points , , , or , therefore both  and  must be
  adjacent to one of them, a contradiction.  \hfill 
\end{proof}





The preceding results allow a complete characterization of the trees that can
be realized as WRGs. An analogous result for rectangle-of-influence graphs was
given in \cite{LLMW98}.

\begin{theorem}
  \label{tree}
  A tree is representable as a witness rectangle graph if and only if it has no three
  independent edges.
\end{theorem}
\begin{proof}
\begin{figure}
\centering
  \includegraphics[width=1.8cm]{BRGtree1}\hfil
  \includegraphics[width=1.8cm]{BRGtree2}\hfil
  \includegraphics[width=2.7cm]{BRG3}\hfil
  \includegraphics[width=3.6cm]{BRG4}\st_R(n)=\max_{|P|=n} \min \{|W| : \RG^+(P,W)=K_{|P|}\}.


\begin{theorem} \label{thm:stabbingRectangles}
  The asymptotic value of  is .
\end{theorem}
\begin{proof}
  We first construct a set  of  points, no two of them with
  equal abscissa or ordinate, that admits a set of  pairwise openly-disjoint rectangles, whose boundary contains two
  points from , which will imply the lower bound.

  Start with a grid of size , then rotate the
  whole grid infinitesimally clockwise, and finally perturb the points
  very slightly, so that no point coordinate is repeated and there are
  no collinearities.  The desired set of rectangles contains 
  for every pair of points  that were neighbors in the
  original grid; refer to Figure~\ref{fig:pairwiseDisjointRectangles}.

\begin{figure}[htbp!]
  \centering
  \includegraphics[scale=0.55]{perturbedGrid4x4}\caption{Construction of a set  with a large number of rectangles
    with disjoint interiors, each one with two points from  on its
    boundary; every point of  participates in four rectangles, with
    the exception of those on the boundary of the configuration.}
  \label{fig:pairwiseDisjointRectangles}
\end{figure}

 \begin{figure}[htbp!]
  \centering
  \includegraphics[scale=0.9]{rectangleSufficient}
  \caption{Construction, for a point set , of a set  of positive
    witnesses such that .}
  \label{fig:pairwiseDisjointRectangles2}
\end{figure}


For the proof that  points suffice to stab all the rectangles refer to Theorem 6 in \cite{ADH}. A sketch of the proof is
illustrated in Figure~\ref{fig:pairwiseDisjointRectangles2}. The points that get only one witness form an ascending chain.
A descending chain would be used similarly and one of these alternative chains must have
 size.

\hfill 





\end{proof}
  












\begin{thebibliography}{99}



\bibitem{RectangularVisibility1} N. Alon, Z. Furedi, and M. Katchalski,
``Separating pairs of  points by standard boxes.''
\emph{European Journal of Combinatorics},
6, 205--210, 1985.

\bibitem{ADH11} B. Aronov, M. Dulieu, and F. Hurtado, ``Witness
  rectangle graphs,'' \emph{Proc. Algorithms Data Structures
    Symp. (WADS'11)}, 2011, to appear.

\bibitem{ADH} B. Aronov, M. Dulieu, and F. Hurtado, ``Witness
  (Delaunay) graphs,'' \emph{Computational Geometry Theory and Applications}, 44(6-7),
  329--344, Aug. 2011.
  


\bibitem{ADH08} B. Aronov, M. Dulieu, and F. Hurtado, ``Witness Gabriel
  graphs,''  \emph{Computational Geometry Theory and Applications}, to appear.


\bibitem{AK00} F. Aurenhammer and R. Klein, ``Voronoi
  diagrams,'' chapter~5 in \emph{Handbook of Computational Geometry},
  J. Sack and G. Urrutia, Editors, Elsevier Science Publishing,
  201--290, 2000.
  
  \bibitem{RectangleDrawability4} T.C.Biedl, A. Bretscher, and H. Meijer,
  ``Rectangle of Influence Drawings of Graphs without Filled 3-Cycles,''
  \emph{Graph Drawing},
  1731, 359--368, 1999.
  
    \bibitem{mathworld:DominationNumber} N. Bray, ``Domination number,'' From
  MathWorld--A Wolfram Web
  Resource, created by E.W. Weisstein. \textsl{http://mathworld.wolfram.com/DominationNumber.html}
  
     \bibitem{mathworld:DominatingSet} N. Bray, ``Dominating Set,'' From
  MathWorld--A Wolfram Web
  Resource, created by E.W. Weisstein. \textsl{http://mathworld.wolfram.com/DominatingSet.html}

\bibitem{BETT98}
G. Di Battista, P. Eades, R. Tamassia, and I.G. Tollis.
\newblock \emph{Graph Drawing: Algorithms for the Visualization of Graphs},
\newblock Prentice Hall, 1998.

\bibitem{BLL94} G. Di Battista, W. Lenhart, and G. Liotta, ``Proximity
  drawability: A survey,'' \emph{Proc. Graph Drawing'94}, Lect. Notes
  Comp. Sci. 894, pp.~328--339, 1995.

\bibitem{BLW06}  G. Di Battista, G. Liotta, and S. Whitesides,
``The strength of weak proximity,''
\emph{J. Discrete Algorithms} 4(3), 384--400, 2006.

\bibitem{BCO92}  M. de Berg, S. Carlsson, and M. Overmars,
``A general approach to dominance in the plane,''
\emph{J. Algorithms} 13(2), 274--296, 1992.

\bibitem{CL04} G. Chartrand and L. Lesniak,
\emph{Graphs and Digraphs}, 4th edition, Chapman \& Hall, 2004.

\bibitem{Janos} X. Chen, J. P\'ach, M. Szegedy, and G. Tardos,
``Delaunay Graphs of Point Sets in the Plane with Respect to Axis-Parallel Rectangles,''
\emph{Random Struct. Algorithms},
34(1), 11--23, 2009.


\bibitem{SmallDiameter2} F.~R.~K.~ Chung,
``Graphs with small diameter after edge deletion,''
\emph{Discrete Applied Mathematics},
37/38, 73--94, 1992.


\bibitem{Co06} S. Collette. ``Regions, Distances and Graphs'', PhD thesis, Universit\'{e} Libre de Bruxelles, 2006.

\bibitem{CKU99}  J. Czyzowicz, E. Kranakis, and J. Urrutia,
``Dissections, cuts, and triangulations,''
\emph{Proc. 11th Canadian Conf. Comput. Geometry},
pp.~154-157, 1999.

\bibitem{dillencourt1} M.B. Dillencourt,
``Toughness and Delaunay triangulations,''
\emph{Discrete Comput. Geometry}, 5(6), 575--601, 1990.

\bibitem{dillencourt2} M.B. Dillencourt,
``Realizability of Delaunay triangulations,''
\emph{Inf. Proc. Letters}, 33(6), 283--287, 1990.

\bibitem{dillencourt3} M.B. Dillencourt and W.D. Smith,
``Graph-theoretical conditions for inscribability and Delaunay
realizability,''
\emph{Discrete Math.}, 161(1-3), 63--77, 1996.

\bibitem{Domination2} M.~Dorfling, W. Goddard, and M.A. Henning,
``Domination in Planar Graphs with Small Diameter II,''
\emph{Ars Comb.},
78, 2006.

\bibitem{thesis} M. Dulieu, ``Witness proximity graphs,'' PhD thesis, Polytechnic Institute of
  NYU, Brooklyn, New York, 2012, in preparation.

\bibitem{DM41}  B. Dushnik and E.W. Miller,
``Partially ordered sets,''
\emph{American J. Math.} 63, 600--619, 1941.

\bibitem{RectangleDrawability5} H.A. ElGindy, G. Liotta, A. Lubiw, H. Meijer, and S. Whitesides,
``Recognizing Rectangle of Influence Drawable Graphs,''
\emph{Proc. Graph Drawing'94}, Lect. Notes Comp. Sci. 894, 
pp. 352--363, 1995.

\bibitem{diameterTwo1} P. Erd\H{o}s, S. Fajtlowicz, A.J. Hoffman,
``Maximum degree in graphs of diameter 2,''
\emph{Networks}, 10(1), 87--90, 2006.

\bibitem{Fo04} S. Fortune,
``Voronoi diagrams and Delaunay  triangulations,''
Chapter~23 in \emph{Handbook of Discrete and Computational Geometry},
2nd edition, J.E.  Goodman and J. O'Rourke, Editors,
CRC Press, 513--528, 2004.

\bibitem{SmallDiameter1} J.~Fox and B. Sudakov,
``Decompositions into Subgraphs of Small Diameter,''
\emph{Combinatorics, Probability \& Computing}, 
19(5--6), 753--774, 2010.

\bibitem{Domination} W.~Goddard and M.A.~Henning,
``Domination in Planar Graphs with Small Diameter,''
\emph{The Electronic Journal of combinatorics},
10(9), 1--25, 2003.


\bibitem{IGraph} M. Habib, R. McConnell, C. Paul, and L. Viennot,
``Lex-BFS and partition refinement, with applications to transitive orientation,
interval graph recognition and consecutive ones testing,''
\emph{Theoretical Computer Science}, 234(1--2), 59--84, 2000.


\bibitem{MNG} M. Ichino, J. Sklansky,
``The relative neighborhood graph for mixed feature variables'',
\emph{Pattern Recognition}, 18(2), 161--167, 1985.

\bibitem{JT92}  J.W. Jaromczyk and G.T. Toussaint,
``Relative neighborhood graphs and their relatives,''
\emph{Proc. IEEE}, 80(9), 1502--1517, 1992.


\bibitem{KM88}
M. Katchalski and A. Meir, ``On empty triangles determined by
points in the plane,'' \emph{Acta Math. Hungar.}, 51, 23--328, 1988.

\bibitem{RectangularVisibility2} J.M. Keil,
``Minimally covering a horizontally convex orthogonal polygon,''
\emph{Proceedings of the Second ACM Symposium on Computational Geometry},
43--51, 1986.

\bibitem{Li08} G. Liotta, Proximity Drawings. Chapter in
\emph{Handbook of Graph Drawing and Visualization}, R. Tamassia,
Editor, Chapman \& Hall/CRC Press, in preparation.


\bibitem{LLMW98}  G. Liotta, A. Lubiw, H. Meijer, and S. Whitesides,
``The rectangle of influence drawability problem,''
\emph{Computational Geometry Theory and Applications}, 10(1), 1--22,
1998.

\bibitem{PComm} R. McConnell, Personal communication, 2011.

\bibitem{diameterTwo2}  B.D. McKay, M. Miller, and J. \v{S}ir\'{a}\u{n},
``A note on large graphs of diameter two and given maximum degree,''
\emph{J. Combin. Theory Ser. B}, 74, 110--118,
1998.

\bibitem{diameterTwo3}  M. Miller and J. \v{S}ir\'{a}\u{n},
``Moore graphs and beyond: A survey of the degree/diameter problem,''
\emph{Electron. J. Combin.}, DS14, 61 pp., 2005.

\bibitem{RectangleDrawability2} K. Miura, T. Matsuno, and T. Nishizeki, 
`` Open Rectangle-of-Influence Drawings of Inner Triangulated Plane Graphs,''
\emph{Discrete \& Computational Geometry},
41(4), 643--670, 2009.

\bibitem{RectangleDrawability1} K. Miura and T. Nishizeki, 
``Rectangle-of-Influence Drawings of Four-Connected Plane Graphs,''
\emph{APVIS},
45, 75--80, 2005.

\bibitem{RectangularVisibility3} J.~I.~Munro, M.~H.~Overmars, and D. Wood,
``Variations on Visibility,''
\emph{Symposium on Computational Geometry},
291--299, 1987.

\bibitem{RectangleDrawability3} S. Sadasivam and H. Zhang,
``Closed Rectangle-of-Influence Drawings for Irreducible Triangulations,''
\emph{Computational Geometry},
44(1), 9--19, 2011.

\bibitem{SU07}  T. Sakai and J. Urrutia, ``Covering the convex
  quadrilaterals of point sets,''
\emph{Graphs and Combinatorics}, Vol. 23 Suppl. 1, 343--357, 2007.

\bibitem{T88} G.T. Toussaint, Editor,
\emph{Computational Morphology},
North-Holland, 1988.

\bibitem{RectangleDrawability6} M. Vaidya and H. Zhang,
``On Open Rectangle-of-Influence Drawings of planar Graphs,''
\emph{COCOA}, Publisher Springer, Series Lectures Notes in Computer Science,
5573, 123--134, 2009.

\bibitem{mathworld:join} E.W. Weisstein, ``Graph Join,'' From
  MathWorld--A Wolfram Web
  Resource, created by E.W. Weisstein. \textsl{http://mathworld.wolfram.com/GraphJoin.html}
  



\end{thebibliography}

\end{document}
