
In this section we characterize the transducts of `spiralling' sequences.
Proofs omitted in the text can be found in the appendix.


\begin{definition}\label{def:seq}
  For a function  we define the sequence~
  by 
  \begin{center}
  \end{center}
  For a sequence , we often speak of the -th \emph{block} of  
  to refer to the occurrence of the word  in .
\end{definition}

In the sequel we often write 
 to denote the sequence~.
We note that there is a one-to-one correspondence between functions ,
and infinite sequences over the alphabet~ that start with the letter~ and that
contain infinitely many occurrences of the letter~.
Every degree is of the form 
for some .

The following lemma is concerned with some basic operations on functions
that have no effect on the degree of 
(multiplication with a constant, and - and -shifts),
and others by which we may go to a lower degree
(taking subsequences, merging blocks).

\begin{lemma}\label{lem:basic}
  Let , and .
  It holds that:
  \begin{enumerate}
    \item 
      \label{item:lem:basic:mult}
      , for ,
    \item 
      ,
      \label{item:lem:basic:xshift}
    \item 
      ,
      \label{item:lem:basic:yshift}
    \item 
      , for ,
      \label{item:lem:basic:sub}
    \item 
      .
      \label{item:lem:basic:merge}
  \end{enumerate}
\end{lemma}
\begin{proof}Recall that, for any ,
  with the -th \emph{block} of  we mean the factor  of .
  \begin{enumerate}
    \item 
      We have  where  is an FST that replaces 
      every factor  by~.
      Clearly, the inverse operation is also FST-realisable, so .
    \item 
      An FST that prepends 
      
      to the incoming word
      witnesses .
      For the converse direction, let the FST remove the first  blocks from the input.
    \item 
      To see that , 
      consider an FST that removes a factor  from every block in .
      Conversely,  is witnessed by an FST that appends  to every block.
    \item 
      It is clear that the operation of selecting every -th block is realisable by an FST.
    \item 
      By Theorem~\ref{thm:transducts}.
    \qed
  \end{enumerate}
\end{proof}


\begin{definition}\label{def:periodic}
  Let  be a set.
  A function  
  is \emph{ultimately periodic}
  if for some integers ,  we have
   for all .
\end{definition}  

\begin{definition}\label{def:spiralling}
  A function  
  is called \emph{spiralling}
  if 
  \begin{enumerate}
    \item{}\label{def:spiralling:item:i}
      , and
    \item{}\label{def:spiralling:item:ii} 
      for every , the function  is ultimately periodic.
  \end{enumerate}
\end{definition}

Functions with the property \ref{def:spiralling:item:ii} in Definition~\ref{def:spiralling}
have been called `ultimately periodic reducible' by Siefkes \cite{sief:1971} (quotation from \cite{seif:mcna:1976}). 
Note that the identity function is spiralling.
Furthermore scalar products, and pointwise sums and products of spiralling functions 
are again spiralling. 
As a consequence also polynomials  are spiralling.

In the remainder of this section, we will characterise the transducts  of  for spiralling .
We will show that if such a transduct  is not ultimately periodic, 
then it is equivalent to a sequence  for a spiralling function , 
and moreover,  can be obtained from  by a `weighted product'.

\begin{lemma}\label{lem:transducts}
  Let  be a spiralling function.
  We have 
  if and only if 
   is of the form
  
  for some integers  and , 
  and finite words~ 
  such that  for all  and .
\end{lemma}

\begin{proof}
  Assume  and let  be an FST
  that transduces  to .
As  is spiralling, 
  there exist ,  such that 
   for every .
  Moreover, as a consequence of , there exists  such that 
   for every .

  For , let  be the state that the automaton  is in, before 
  reading the -th occurrence of  in the sequence 
  (i.e., the start of the block ). 
By the pigeonhole principle there exist 
  with  and 
  such that  and .
Then, for every  and , we have 
   and 
  .
  For , 
  we define .
  Note that .
  Then for all  and  we have: 
  
  where  and .
Using Lemma~\ref{lem:pumping} it follows by induction that
   for every .
  Hence
   for every  and .
  Then, again by Lemma~\ref{lem:pumping}, for every 
  there exist words~
  such that 
  
  for all .
  We conclude that 
  
  where .

  For the other direction, 
  assume  is of the form \eqref{eq:double:prod}
  for some  with , 
  and~ 
  such that  for all  and .
We define an FST  such that , where 
  
  Define  
  where ,
  and 
  -.5ex]
    &&&\text{and  and , otherwise,} \\
\pair{\delta}{\lambda}(q_j^h,1) & = \pair{q_{j'}^{{-}a_{j'}}}{1} && (j' = j+1 \bmod{m}) \,.
  
    \seq{\floor{\frac{n}{2}}} = 1 1 10 10 10^2 10^2 10^3 10^3 10^4 10^4 \cdots \,,
  
\underline{q_0} 1 q_2 1 q_1 1 q_0 0 q_1 1 q_0 0 
    q_1 1 q_0 0 q_1 0 q_2 1 q_1 0 q_2 0 
    \underline{q_0} 1 q_2 0 q_0 0 q_1 0 q_2 1 q_1 0 q_2 0 q_0 0 \cdots
  
    T(\seq{n}) = 1 \, 1 (01) \, 1 (10)^2 \, 1 (01)^3 \, 1 (10)^4 \, \cdots\,.
  
    \wof{\alpha}{f} \;=\; a_0 f(0) + a_1 f(1) + \cdots + a_{k-1} f(k-1) + b \,.
  
    (\wprod{\vec{\alpha}}{f})(0) & = \alpha_0 \cdot f \\
    (\wprod{\vec{\alpha}}{f})(n+1) & = (\wprod{\vec{\alpha}'}{\shift{|\alpha_0|-1}{f}})(n) && (n \in \nat)
  
    (\wprod{\vec{\alpha}}{f})(n) = \wof{\alpha_r}{\shift{t}{f}}
    &&&\text{where , and  with ,}\\
    &&&\text{and .}
  
    (\wprod{\vec{\alpha}}{f})(n) 
    = (\wprod{\vec{\alpha}'}{\shift{k_0}{f}})(n') 
    = \wof{\alpha_{r'+1}}{\shift{t'}{\shift{k_0}{f}}}
  
    \zip_k(f_0,f_1,\ldots,f_{k-1})(kn+i) = f_i(n) && (n \in \nat,\, i \in \nat_{<k}) \,.
  
    \wprod{\vec{\alpha}}{f} & = \zip_m(g_0,g_1,\ldots,g_{m-1}) && \text{where  for .}
  
    (\beta_i)_{s_m} = b_i
    &&
    (\beta_i)_{s_j + h} =
    \begin{cases}
      a_{i,h} & \text{if ,} \\  
      0 & \text{if }  
    \end{cases}
    && (j \in \nat_{<m},\, h \in \nat_{<k_i})\,.
  
    \pair{\delta}{\lambda}(q_{i,j}^{h},0) & = \pair{q_{i,j}^{h'}}{0^e} 
    && \text{where , }\\
    \pair{\delta}{\lambda}(q_{i,j}^{h},1) & = \pair{q_{i,j+1}^{h}}{\emptyword}
    && (j < k_i - 1) \\
    \pair{\delta}{\lambda}(q_{i,k_i-1}^{h},1) & = \pair{q_{i',0}^{h'}}{1 0^e} 
    && \text{where , \,,}
  
    \pair{\delta}{\lambda}(q_{i,0}^{b_i - e d_i},u) = \pair{q_{i',0}^{b_{i'} - e' d_{i'}}}{0^{\wof{\alpha_i}{\shift{n}{f}}-e} 1 0^{e'}}
\label{eq:lem:wprod:FST:i}
  
      h_0 & = b_i - e d_i \\
      h_{j+1} & = h_j + a_{i,j} \cdot f(n+j)  - e_j \cdot d_i && (j \lt k_i - 1) \\
      e_{j} & = \floor{\frac{h_j + a_{i,j} \cdot f(n+j)}{d_i}} \,.
  
    \pair{\delta}{\lambda}(q_{i,j}^{h_j},0^{f(n+j)} 1) & = \pair{q_{i,j+1}^{h_{j+1}}}{0^{e_{j}}}
    && \text{for , and} \\
    \pair{\delta}{\lambda}(q_{i,k_i-1}^{h_{k_i-1}},0^{f(n+k_i-1)} 1) & = \pair{q_{i',0}^{b_{i'}}}{0^{e_{k_i-1}} 1}
    && \text{where .}
  
    \sum_{j = 0}^{k_i-1} e_j 
    = \floor{\frac{h_0 + \sum_{j=0}^{k_i-1}a_{i,j} \cdot f(n+j)}{d_i}} 
= \wof{\alpha}{\shift{n}{f}} - e \,,
  
    \dpf{f,\vec{\alpha},\vec{p},\vec{c}}
    = \prod_{i = 0}^{\infty} \prod_{j = 0}^{m-1} p_j \, c_j^{\cyc{i}{j}}
    && \text{where} &&
    \cyc{i}{j} = (\wprod{\vec{\alpha}}{f})(mi + j) \,.
  
    \cycp{i}{j} &= 
    \begin{cases}
      \cyc{i}{j+1} & \text{if ,} \\
      \cyc{i+1}{0} & \text{if ,}
    \end{cases}
    \\
    \psi(i,j) & = (\wprod{\vec{\alpha}'}{\shift{\length{\alpha_0} - 1}{f}})(mi+j)
  
    \dpf{f,\vec{\alpha},\vec{p},\vec{c}} 
    & = \prod_{i=0}^\infty \prod_{j=0}^{m-1} p_j c_j^{\cyc{i}{j}} \\
& = p_0 c_0^{\cyc{0}{0}} \cdot \prod_{i=0}^\infty \prod_{j=0}^{m-1} p_{j+1} c_{j+1}^{\cycp{i}{j}} \\
    & = p_0 c_0^{\wof{\alpha_0}{f}} \cdot \prod_{i=0}^\infty \prod_{j=0}^{m-1} p_{j+1} c_{j+1}^{\psi(i,j)} \\
    & = p_0 c_0^{\wof{\alpha_0}{f}} \cdot \dpf{\shift{\length{\alpha_0}-1}{f},\vec{\alpha}',\vec{p}',\vec{c}'}
  
    \delta^\star(q,u_1u_2v) & = \delta^\star(\delta(q,u_1,u_2),u_2v)
    &&&& (\tup{q,u_1,u_2} \in D, v \in \two^*)\,.
  
\lambda^\star(q,u_1u_2v) & = \lambda(q,u_1,u_2) \cdot \lambda^\star(\delta(q,u_1,u_2),u_2v)
    &&&& (\tup{q,u_1,u_2} \in D, v \in \two^\infty)\,.
  
    \pair{\delta'}{\lambda'}(\pair{q}{v},a) &= \pair{\pair{q}{va}}{\emptyword} && \text{ with }\\
    \pair{\delta'}{\lambda'}(\pair{q}{u_1u_2v},a) &= \pair{\pair{q}{u_2va}}{\lambda(q,u_1,u_2)} && \text{, }\
  To make the FST  complete, we let  whenever  and 
  and neither of the two clauses above applies.
  It is straightforward to verify that 
  for all , 
   whenever  is defined.
  \qed
\end{proof}

\begin{theorem}\label{thm:transducts}
  Let  be spiralling, and .
  Then 
  if and only if 
  
  for some integer~, and a tuple of weights .
\end{theorem}

\begin{proof}
  One direction is by Lemma~\ref{lem:wprod:FST}.
  For the other, assume .
  If , then .
  Thus let .
  By Lemma~\ref{lem:disambiguate} there exist
  , , ,  and  
  with 
  such that , and
  fulfilling the conditions~\ref{item:no:ambiguities} and~\ref{item:no:empty:cycles} 
  of Lemma~\ref{lem:disambiguate}.
  We abbreviate .
  We will show that .
By Lemma~\ref{lem:wprod:preserve:spiralling}
  we have that the function~ is spiralling too.
  
By conditions~\ref{item:no:ambiguities} and~\ref{item:no:empty:cycles},
  for every ,
  there exists  such that  
  (where addition is modulo );
  let  be minimal with this property.
For , let  be minimal such that 
  and .
  Then by minimality of  and , we obtain
  \begin{enumerate}
    \item \label{item:thm:transducts:i}
       for every , and
    \item 
       for every 
  \end{enumerate}
  (with again addition computed modulo~).
  From \ref{item:thm:transducts:i} we moreover obtain
  \begin{enumerate}[resume]
    \item \label{item:thm:transducts:iii}
       for every  and .
  \end{enumerate}

  Next, we take a suffix  of  such that every occurrence of a block 
  has as a prefix .
  Let  be such that for 
  we have that  for all ;
  the existence of such an  follows from  being spiralling.
  To prove
  
  it suffices to show
  
  where
  
  where , and .
  Note that by the choice of , we have
   for all  and .

  It is clear how to construct an FST that transduces  to ,
  see Figure~\ref{fig:easy}.
  For , we define a \lfst~, as follows,
  and apply Lemma~\ref{lem:LFST2FST}.
  Let  and ,
  and define  by
  
  We now argue that .
  This follows from the following facts:
  \begin{enumerate}[label=(\alph*)]
    \item ,
    \item 
      for all , since by item~\ref{item:thm:transducts:iii} 
      we have .
      \qed
  \end{enumerate}
\end{proof}

\begin{lemma}\label{lem:one:weight}
  Let  be spiralling, and  with .
  Then we have 
  for some integer~, and a non-constant weight .
\end{lemma}
\begin{proof}Let  be a transduct of . 
  By Theorem~\ref{thm:transducts} we have
  
  for some~, and an -tuple of weights .
  By Lemma~\ref{lem:wprod:not:botdeg}, there exists  
  such that  is not constant.
Let  (), 
  and for  let .
  We define a weight  of length  where, for  and ,
   if , and  if , 
  and .
  Then we have
  .
\end{proof}

\begin{theorem}\label{thm:no:minimal:below}
  There is a non-atom, non-zero degree  that has no atom degree below it.
  Hence, non-zero transducts of  start an infinite~descending~chain. \end{theorem}

\begin{proof}
  We define the function  by .
  We show that the degree  has no atom degree below it.
  Let  with . 
  By Lemma~\ref{lem:one:weight} there is a non-constant weight 
  such that  where 
  .
Since  it follows that 
  
  By Lemma~\ref{lem:wprod:FST} we have that .
  Thus we have .
  Also 
  holds by Lemma~\ref{lem:basic}~\ref{item:lem:basic:sub},
  and by Lemma~\ref{lem:too:fast} we conclude .
  \qed
\end{proof}