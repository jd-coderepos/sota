\documentclass[envcountsame,envcountsect,10pt,oribibl]{llncs}

\usepackage{microtype}

\usepackage[usenames,dvipsnames]{xcolor}





\usepackage{amssymb,amstext,amsfonts,amsopn,stmaryrd}

\usepackage{tabularx}
\usepackage{array,multirow} 

\usepackage{comment}



\usepackage{graphics}
\usepackage{amsmath}
\usepackage{subfig}			 \usepackage{stackrel}
\usepackage{tikz}
\usepackage{mathrsfs}
\usepackage{enumerate}
\usepackage{amssymb}




\graphicspath{{./}}


\newcommand{\pname}[1]{\textnormal{\textsc{#1}}}
\newcommand{\cclass}[1]{\textnormal{\textsf{#1}}}

\newcommand{\HED}{\pname{-free Edge Deletion}}
\newcommand{\HEC}{\pname{-free Edge Completion}}
\newcommand{\HEE}{\pname{-free Edge Editing}}

\newcommand{\HDED}{\pname{-free Edge Deletion}}
\newcommand{\HDEC}{\pname{-free Edge Completion}}
\newcommand{\HDEE}{\pname{-free Edge Editing}}

\newcommand{\HBED}{\pname{-free Edge Deletion}}
\newcommand{\HBEC}{\pname{-free Edge Completion}}
\newcommand{\HBEE}{\pname{-free Edge Editing}}

\newcommand{\TED}{\pname{-free Edge Deletion}}
\newcommand{\TEE}{\pname{-free Edge Editing}}
\newcommand{\TDED}{\pname{-free Edge Deletion}}

\newcommand{\TDDED}{\pname{-diamond-free Edge Deletion}}
\newcommand{\TDDEE}{\pname{-diamond-free Edge Editing}}
\newcommand{\TODDED}{\pname{-diamond-free Edge Deletion}}
\newcommand{\TODDEE}{\pname{-diamond-free Edge Editing}}

\newcommand{\RED}{\pname{-free Edge Deletion}}
\newcommand{\REE}{\pname{-free Edge Editing}}
\newcommand{\RDED}{\pname{-free Edge Deletion}}

\newcommand{\PTED}{\pname{-free Edge Deletion}}
\newcommand{\PTEE}{\pname{-free Edge Editing}}
\newcommand{\PFED}{\pname{-free Edge Deletion}}
\newcommand{\PFEE}{\pname{-free Edge Editing}}
\newcommand{\CLED}{\pname{-free Edge Deletion}}
\newcommand{\CLEE}{\pname{-free Edge Editing}}
\newcommand{\TWKTED}{\pname{-free Edge Deletion}}
\newcommand{\TWKTEE}{\pname{-free Edge Editing}}

\newcommand{\DED}{\pname{Diamond-free Edge Deletion}}
\newcommand{\DEE}{\pname{Diamond-free Edge Editing}}

\newcommand{\KOKTEE}{\pname{-free Edge Editing}}

\newcommand{\HDBEC}{\pname{-free Edge Completion}}

\newcommand{\SLED}{\pname{-free Edge Deletion}}
\newcommand{\SLOED}{\pname{-free Edge Deletion}}
\newcommand{\SLLED}{\pname{-free Edge Deletion}}
\newcommand{\SLOLOED}{\pname{-free Edge Deletion}}
\newcommand{\KTED}{\pname{-free Edge Deletion}}
\newcommand{\SHED}{\pname{-free Edge Deletion}}
\newcommand{\THED}{\pname{-free Edge Deletion}}
\newcommand{\TOHED}{\pname{-free Edge Deletion}}
\newcommand{\HOED}{\pname{-free Edge Deletion}}
\newcommand{\TKTED}{\pname{-free Edge Deletion}}
\newcommand{\REC}{\pname{-free Edge Completion}}
\newcommand{\PIED}{\pname{ Edge Deletion}}
\newcommand{\PIEC}{\pname{ Edge Completion}}
\newcommand{\PIEE}{\pname{ Edge Editing}}
\newcommand{\PIVD}{\pname{ Vertex Deletion}}
\newcommand{\TSAT}{\pname{3-SAT}}
\newcommand{\VC}{\pname{Vertex Cover}}
\newcommand{\CH}{}

\newcommand{\CPVC}{\pname{Cubic Planar Vertex Cover}}
\newcommand{\cpvc}{\CPVC}

\newcommand{\CD}{\pname{-Free Edge Deletion}}
\newcommand{\ACD}{\pname{Annotated -Free Edge Deletion}}

\newcommand{\NP}{\cclass{NP}}
\newcommand{\NPC}{\cclass{NP-complete}}
\newcommand{\NPH}{\cclass{NP-hard}}
\newcommand{\FPT}{\cclass{FPT}}
\newcommand{\MCHD}{}

\newcommand{\PH}{}
\newcommand{\ph}{\PH}

\newcommand{\NOPH}{}
\newcommand{\noph}{\NOPH}





\DeclareMathOperator{\poly}{poly} \DeclareMathOperator{\diam}{diam} \DeclareMathOperator{\degree}{degree} 




\newcommand{\edges}[1][V]{\left[#1\right]^2} 

\newcommand{\symdiff}{\oplus} 



\newtheorem{construction}{Construction}
\newtheorem{ppt}{Property}
\newtheorem{observation}[lemma]{Observation}




\frontmatter          \pagestyle{plain}  



\newcommand{\defstage}[2]{\hfill\\\smallskip\noindent \begin{tabularx}{\textwidth}{|l X|}\hline \multicolumn{2}{|l|}{\textbf{#1}}\\&#2\\\hline \end{tabularx}}

\begin{document}
\mainmatter              \title{Parameterized Lower Bounds and Dichotomy Results for the NP-completeness of -free Edge Modification Problems}
\author{N. R. Aravind\inst{1} \and R. B. Sandeep\inst{1}\thanks{supported by TCS Research Scholarship} \and Naveen Sivadasan\inst{2}}
\institute{Department of Computer Science \& Engineering\\
Indian Institute of Technology Hyderabad, India\\
\email{\{aravind,cs12p0001\}\makeatletter@\makeatother iith.ac.in}
\and
TCS Innovation Labs, Hyderabad, India\\
\email{naveen\makeatletter@\makeatother atc.tcs.com}}

\maketitle              \begin{abstract}
For a graph , the \HED\ problem asks whether there exist at most  
edges whose deletion from the input graph  
results in a graph without any induced copy of . 
\HEC\ and \HEE\ are defined similarly where
only completion (addition) of edges are allowed in the former and 
both completion and deletion are allowed in the latter. 
We completely settle the classical complexities of these problems by proving that
\HED\ is \NPC\ if and only if  is a graph with at least two edges, 
\HEC\ is \NPC\ if and only if  is a graph with at least two non-edges and
\HEE\ is \NPC\ if and only if  is a graph with at least three vertices.
Additionally, we prove that, these \NPC\ problems cannot be 
solved in parameterized subexponential time, i.e., 
in time , unless Exponential Time Hypothesis fails.
Furthermore, we obtain implications on the incompressibility of these 
problems.
\end{abstract}
\section{Introduction}
\label{sec:introduction}

Edge modification problems are to test whether modifying at most  edges
makes the input graph satisfy certain properties. The three major edge modification
problems are edge deletion, edge completion and edge editing problems.
In edge deletion problems we are allowed to delete at most  edges from
the input graph. Similarly, in completion problems, it is allowed to
complete (add) at most  edges and in editing problems at most
 editing (deletion or completion) are allowed. Edge modification problems
comes under the broader category of graph modification problems which have
found applications in DNA physical mapping 
\cite{goldberg1995four},
numerical algebra \cite{rose1972graph}, circuit design \cite{el1988complexity}
and machine learning \cite{DBLP:journals/ml/BansalBC04}.

The focus of this paper is on -free edge modification problems,
in which we are allowed to modify at most  edges to make the input
graph devoid of any induced copy of , where  is any fixed graph.
Though these problems have been studied for four decades, a complete
dichotomy result on the classical complexities of these problems are not yet found.
We settle this by proving that \HED\ is \NPC\ if and only if  is a graph with at least two edges, 
\HEC\ is \NPC\ if and only if  is a graph with at least two non-edges and
 \HEE\ is \NPC\ if and only if  is a graph with at least three vertices.
As a bonus, we obtain the parameterized lower bounds for these \NPC\ problems.
We obtain that these \NPC\ problems cannot be solved in parameterized 
subexponential time (i.e., in time ), unless
Exponential Time Hypothesis (ETH) fails, where ETH is a widely believed complexity theoretic
assumption. Furthermore, we obtain implications on the incompressibility 
(non-existence of polynomial kernels) of these problems.

We build on our recent paper \cite{DBLP:conf/cocoa/AravindSS15},
in which we proved that \HED\ is \NPC\ if  has at least two edges and 
has a component with
maximum number of vertices which is a tree or a regular graph.
We also proved that these problems cannot be solved in parameterized 
subexponential time, unless ETH fails.



\paragraph{\textbf{Related Work:}}
In 1981, Yannakakis proved that \HED\ is \NPC\
if  is a cycle~\cite{DBLP:journals/siamcomp/Yannakakis81}. 
Later in 1988, El-Mallah and Colbourn proved that the problem
is \NPC\ if  is a path of at least two edges \cite{el1988complexity}.
Addressing the fixed parameter tractability of a generalized version of these problems,
Cai proved that \cite{DBLP:journals/ipl/Cai96}~\HED, \textsc{Completion} and \textsc{Editing}
are fixed parameter tractable, i.e., they can be solved in time
, for some function . 
Polynomial kernelizability of these problems
have been studied widely. 
Given an instance  of the problem the objective is to 
obtain in polynomial time an equivalent instance of size
polynomial in . Kratsch and Wahlstr{\"o}m gave the first result on the incompressibility of 
-free edge modification problems. They proved that \cite{kratsch2009two} for a certain graph 
on seven vertices, \HED\ and \HEE\ do not admit polynomial kernels, unless \PH. They use
polynomial parameter transformation from an \NPC\ problem and hence their results imply the NP-completeness
of these problems. Later, Cai and Cai proved that 
\HEE, \textsc{Deletion} and \textsc{Completion} do not admit polynomial kernels
if  is a path or a cycle with at least four edges, unless \PH~\cite{DBLP:journals/algorithmica/CaiC15}.
Further, they proved that \HEE\ and \textsc{Deletion} are incompressible if  is 3-connected but not complete, 
and \HEC\ is incompressible if  is 3-connected and has at least two non-edges,
unless \PH~\cite{DBLP:journals/algorithmica/CaiC15}.
Under the same assumption, 
it is proved that \HED\ and \HEC\ are incompressible if  is a tree on at least 7 vertices, which is not a star graph and
\HED\ is incompressible if  is the star graph , where ~\cite{cai2012polynomial}.
They also use polynomial parameter transformations and hence these problems
are \NPC. 



\paragraph{\textbf{Outline of the Paper:}}
Section~\ref{sec:prebasics} gives the notations and terminology used in the paper.
It also introduces a construction which is a modified version of the main
construction used in \cite{DBLP:conf/cocoa/AravindSS15}.
Section~\ref{sec:editing} settles the case of \HEE.
Section~\ref{sec:deletion} obtains results for \HED\ and \textsc{Completion}.
In the concluding section, we discuss the implications of our results on the 
incompressibility of -free edge modification problems. 
\section{Preliminaries and Basic Tools}
\label{sec:prebasics}

\paragraph{\textbf{Graphs:}}

For a graph ,  denotes the vertex set and 
 denotes the edge set. We denote the symmetric
difference operator by , i.e., for two sets  and ,
.
For a graph  and a set , 
denotes the graph . A component of a graph is largest
if it has maximum number of vertices. By  we denote .
The disjoint union of two graphs  and  is denoted by 
and the disjoint union of  copies of  is denoted by .
A simple path on  vertices is denoted by .
The graph -diamond is , the join of  and . 
Hence, -diamond is the diamond graph.
The minimum degree of a graph  is denoted by 
and the maximum degree is denoted by .
Degree of a vertex  in a graph  is denoted by .
We remove the subscript when there is no ambiguity.
We denote the complement of a graph  by .
For a graph  and a vertex set ,
 is the graph induced by  in .
A null graph is a graph without any edge.

For integers  and  such that , -degree graph is a graph in which 
every vertex has degree either  or . The set of vertices with degree 
is denoted by  and the set of vertices with degree  is denoted by . 
An -degree graph is called \emph{sparse} if  induces a graph with at most 
one edge and  induces a graph with at most one edge.


The context determines whether \HED\ denotes the classical problem or the parameterized
problem. This applies to \textsc{Completion} and \textsc{Editing} problems.
For the parameterized problems, 
we use  (the size of the solution being sought) as the parameter.
In this paper, edge modification implies either deletion, completion or editing.

\paragraph{\textbf{Technique for Proving Parameterized Lower Bounds:}}
Exponential Time Hypothesis (ETH) is a widely believed complexity theoretic
assumption that \TSAT\ 
cannot be solved in time , where 
 is the number of variables in
the \TSAT\ instance. 
A linear parameterized reduction is a polynomial time reduction
from a parameterized problem  to a parameterized problem 
such that for every instance  of , the reduction gives
an instance  such that . The following
result helps us to obtain parameterized lower bound under ETH.

\begin{proposition}[\cite{DBLP:books/sp/CyganFKLMPPS15}]
  \label{pro:lpr}
  If there is a linear parameterized reduction from a parameterized problem 
  to a parameterized problem  and if  does not admit a parameterized subexponential
  time algorithm, then  does not admit a parameterized subexponential time algorithm.
\end{proposition}

Two parameterized problems  and  are linear parameter equivalent if there is a linear 
parameterized reduction from  to  and there is a linear parameterized reduction from 
 to . We refer the book \cite{DBLP:books/sp/CyganFKLMPPS15} for various aspects of 
parameterized algorithms and complexity.
The following are some folklore observations.

\begin{proposition}
  \label{pro:folklore}
  \HED\ and \HBEC\ are linear parameter equivalent. Similarly,
  \HEE\ and \HBEE\ are linear parameter equivalent.
\end{proposition}

\begin{proposition}
  \label{pro:equivalence}
  \begin{enumerate}[(i)]
  \item\label{item:dc-equivalence} \HED\ is \NPC\ if and only if \HBEC\ is \NPC.
    Furthermore, \HED\ cannot be solved in parameterized subexponential time
    if and only if \HBEC\ cannot be solved in parameterized subexponential time.
  \item\label{item:ee-equivalence} \HEE\ is \NPC\ if and only if \HBEE\ is \NPC.
    Furthermore, \HEE\ cannot be solved in parameterized subexponential time
    if and only if \HBEE\ cannot be solved in parameterized subexponential time.
  \end{enumerate}
\end{proposition}

\begin{proposition}
  \label{pro:polynomial}
  \begin{enumerate}[(i)]
  \item\label{item:poly-deletion} \HED\ is polynomial time solvable if  is a graph with
    at most one edge.
  \item\label{item:poly-completion} \HEC\ is polynomial time solvable if  is a graph with
    at most one non-edge.
  \item\label{item:poly-editing} \HEE\ is polynomial time solvable if  is a graph with
  at most two vertices.
  \end{enumerate}
\end{proposition}

In this paper, we prove that these are the only polynomial time solvable 
-free edge modification problems.
For any fixed graph , the -free edge modification problems trivially belong to \NP. Hence, we may 
state that these problems are \NPC\ by proving their NP-hardness.
\subsection{Basic Tools}

The following construction is a slightly modified version of the main 
construction used in \cite{DBLP:conf/cocoa/AravindSS15}.
The modification is done to make it work for reductions of \textsc{Completion}
and \textsc{Editing} problems. 
The input of the construction is a tuple , where  and  are
graphs,  is a positive integer and . In the old construction
(Construction 1 in \cite{DBLP:conf/cocoa/AravindSS15}), for every copy  of 
in , we introduced  copies of  such that the intersection of every pair
of them is . In the modified construction given below, we do the same for every copy 
of  on a complete graph on . 



\begin{construction}
  \label{con:nonadj}
  Let  be an input to the construction, where  and  are graphs, 
  is a positive integer and  is a subset of vertices of .
  Label the vertices of  such that every vertex gets a unique label. Let the labelling be .
  Consider a complete graph  on . 
  For every subgraph (not necessarily induced)  with a vertex set  
  and an edge set  in  such that  is isomorphic to ,
  do the following:
  \begin{itemize}
    \item Give a labelling  for the vertices in  such that there is an isomorphism
       between  and  which maps every vertex  in  to a vertex  in 
      such that , i.e.,  if and only if .
    \item Introduce  sets of vertices , each of size .
    \item For each set , introduce an edge set  of size  among
      
       such that there is an isomorphism  between  and
       which preserves , i.e.,
      for every vertex , .
  \end{itemize}
  This completes the construction. Let the constructed graph be . 
\end{construction}

We remark that the complete graph  on  is not part of the constructed graph.
The complete graph is only used to find where we need to introduce new vertices and edges.
An example of the construction is shown in Figure~\ref{fig:cons}.
We use the terminology used in \cite{DBLP:conf/cocoa/AravindSS15}. We repeat it here for convenience.
Let  be a copy of  in . Then,  is called a \emph{base}.
Let  be the  sets of vertices introduced in the construction for the base .
Then, each  is called a \emph{branch} of  and the vertices in  are called 
the \emph{branch vertices} of . If  is a branch of , then
the vertex set of  is denoted by .
The vertex set of  in  is denoted by . The copy of  formed by , 
and  is denoted by . Since  is a fixed graph, the construction runs in polynomial time. 
The following two Lemmas are the generalized version of Lemma 2.3 and 3.5 of \cite{DBLP:conf/cocoa/AravindSS15}.

\begin{figure}[h]
  \centering
  \subfloat[Subfigure 1 list of figures text][]{
    \includegraphics[width=1.5in]{gdash.pdf}
    \label{fig:gdash}}
  \qquad
  \subfloat[Subfigure 2 list of figures text][. The vertices in  are blackened.]{
    \includegraphics[width=1.0in]{h.pdf}
    \label{fig:h}}
  \qquad
  \subfloat[Subfigure 2 list of figures text][Output of Construction~\ref{con:nonadj} with an input .]{
    \includegraphics[width=1.5in]{g.pdf}
    \label{fig:g}}
  \caption{An example of Construction~\ref{con:nonadj}}
  \label{fig:cons}
\end{figure}

\begin{lemma}
  \label{lem:con:nonadj-backward}
  Let  be obtained by Construction~\ref{con:nonadj} on 
  the input , where  and  are graphs,  is a positive integer and .
  Then, if  is a yes-instance of \HEE~(\textsc{Deletion}/\textsc{Completion}), 
  then  is a yes-instance of \HDEE~(\textsc{Deletion}/\textsc{Completion}), 
  where  is .
\end{lemma}
\begin{proof}
  Let  be a solution of size at most  of . For a contradiction, assume that 
  has an induced  with a vertex set . Hence there is a base  in  isomorphic to
   with the vertex set . Since there are  copies of  in , where each pair
  of copies of  has the intersection , and , operating with  cannot kill all the copies of 
   associated with . Therefore,
  since  induces an  in , there exists a branch  of  such that 
  induces  in , which is a contradiction.\qed
\end{proof}



\begin{lemma}
  \label{lem:degree}
  Let  be any graph and  be any integer. Let  be the set of vertices
  in  with degree more than .
  Let  be . Then, there is a linear parameterized reduction
  from \HDEE~(\textsc{Deletion}/\textsc{Completion}) to \HEE~(\textsc{Deletion}/\textsc{Completion}).
\end{lemma}
\begin{proof}
  Let  be an instance of \HDEE~(\textsc{Deletion}/\textsc{Completion}).
  Apply Construction~\ref{con:nonadj} on  to obtain .
  We claim that  is a yes-instance of \HDEE~(\textsc{Deletion}/\textsc{Completion})
  if and only if  is a yes-instance of \HEE~(\textsc{Deletion}/\textsc{Completion}).

  Let  be a solution of size at most  of . For a contradiction,
  assume that  has an induced  with a vertex set .
  Since a branch vertex has degree at most , every vertex in 
  with degree more than  in  must be from .
  Hence there is an induced  in , which is a contradiction.
  Lemma~\ref{lem:con:nonadj-backward} proves the converse.\qed
\end{proof}
\section{\HEE}
\label{sec:editing}

In this section, we prove that \HEE\ is \NPC\
if and only if  is a graph with at least three vertices.
We also prove that these problems cannot be solved in 
parameterized subexponential time unless ETH fails.
We use the following known results.

\begin{proposition}
  \label{pro:editing-base}
  The following problems are \NPC. Furthermore, they cannot be 
  solved in time , unless ETH fails.
  \begin{enumerate}[(i)]
  \item\label{item:editing-p3} \PTEE~\cite{komusiewicz2012cluster}.
\item\label{item:editing-p4} \PFEE~[Follows from the proof of the lower 
    bound of \textsc{-free Edge Editing} in \cite        
    {drange2015trivially}\footnote{We thank P{\aa}l Gr{\o}n{\aa}s Drange for pointing out this and sharing a complete proof of the same.}].
  \item\label{item:editing-cycle} \CLEE, for any fixed ~[Follows
    from the proof for the corresponding \textsc{Deletion} problems in \cite{DBLP:journals/siamcomp/Yannakakis81}].
  \item\label{item:editing-2k2} \TWKTEE~[(\ref{item:editing-cycle}) and 
    Proposition~\ref{pro:equivalence}(\ref{item:ee-equivalence})].
  \item\label{item:editing-diamond} \DEE~\cite{DBLP:journals/classification/BarthelemyB01}.
  \end{enumerate}
\end{proposition}

In our previous work \cite{DBLP:conf/cocoa/AravindSS15},
we proved that \RED\ is \NPC\ if  is a regular graph
with at least two edges. We also proved that these
\NPC\ problems cannot be solved in parameterized 
subexponential time, unless ETH fails.
We observe that the results for \RED\ follows for
\REE\ as well. The proofs are very similar except
that we use Construction~\ref{con:nonadj} instead of its
ancestor in \cite{DBLP:conf/cocoa/AravindSS15} and we
reduce from \textsc{Editing} problems instead of \textsc{Deletion}
problems. We can use \PTEE, \CLEE\ and \TWKTEE\ as the base cases instead of their
\textsc{Deletion} counterparts.
We skip the proof as it will 
be a repetition of that in \cite{DBLP:conf/cocoa/AravindSS15}.

\begin{lemma}
  \label{lem:editing-reg-temp}
  Let  be a regular graph with at least two edges.
  Then \REE\ is \NPC. Furthermore, the problem cannot be
  solved in time , unless ETH fails.
\end{lemma}

Now, we strengthen the above lemma by proving the same results for all 
regular graphs with at least three vertices.

\begin{lemma}
  \label{lem:editing-reg}
  Let  be a regular graph with at least three vertices.
  Then \REE\ is \NPC. Furthermore, the problem cannot be
  solved in time , unless ETH fails.
\end{lemma}
\begin{proof}
  If  has at least two edges then the statements follows from 
  Lemma~\ref{lem:editing-reg-temp}. Assume that
   has at most one edge and at least three vertices. It is 
  straight-forward to see that  must be the null graph.
  Then the complement of  is a complete graph with at least two edges. 
  Now, the statements follows from Proposition~\ref{pro:equivalence}(\ref{item:ee-equivalence})
  and Lemma~\ref{lem:editing-reg-temp}.\qed
\end{proof}

Having these results in hand, we use Lemma~\ref{lem:degree} to prove the 
dichotomy result and the parameterized lower bound of \HEE.
Given a graph  with at least three vertices, we introduce a method 
Editing-Churn() to 
obtain a graph  such that there is a linear parameterized reduction
from \HDEE\ to \HEE\ and  is a graph with at least three vertices
and is a regular graph or a  or a  or a diamond.

\defstage{Editing-Churn()}
{  is a graph with at least three vertices.
  \begin{enumerate}[Step 1:]
    \item\label{item:churn-ed1} If  is a regular graph, a , a  or a diamond, then return .
    \item\label{item:churn-ed2} If  is a graph in which the number of vertices with degree more than 
      is at most two, then let  and goto Step~\ref{item:churn-ed1}.
    \item\label{item:churn-ed3} Delete all vertices with degree  in  and go to Step~\ref{item:churn-ed1}.
  \end{enumerate}
}

\begin{observation}
  \label{obs:editing-churn}
  Let  be a graph with at least three vertices.
  Then Editing-Churn() returns a graph  which has
  at least three vertices and is a regular graph or a  or a 
  or a diamond. Furthermore, there is a linear parameterized reduction
  from \HDEE\ to \HEE.
\end{observation}
\begin{proof}
  At any stage of the method, we make sure that the graph has at least three vertices. 
  Let  be an intermediate graph obtained in the method such that it is neither a regular graph
  nor a  nor a  nor a diamond. If Step 2 is applicable to both  and ,
  then  hat at most four vertices. Hence  has either three or four vertices. 
  It is straight-forward to verify that a graph (with three or four vertices) or its complement,
  satisfying the condition in Step 2,
  is either a regular graph or a  or a  or a diamond, which is a contradiction.
  The linear parameterized reduction from \HDEE\ to \HEE\ follows from
  Proposition~\ref{pro:equivalence}(\ref{item:ee-equivalence}) and Lemma~\ref{lem:degree}.\qed
\end{proof}

\begin{theorem}
  \label{thm:editing}
  \HEE\ is \NPC\ if and only if  is a graph with 
  at least three vertices. Furthermore, these \NPC\
  problems cannot be solved in time , unless ETH fails.
\end{theorem}
\begin{proof}
  If  is a graph with at most two vertices, 
  the statements follows from Proposition~\ref{pro:polynomial}(\ref{item:poly-editing}).
  Let  be a graph with at least three vertices.
  Let  be the graph returned by Editing-Churn().
  By Observation~\ref{obs:editing-churn},  is either a regular graph or a 
  or a  or a diamond and there is a linear parameterized reduction from
  \HDEE\ to \HEE.
  Now, the statements follows from the lower bound results for these graphs 
  (\ref{pro:editing-base}(\ref{item:editing-p3}), (\ref{item:editing-p4}), (\ref{item:editing-diamond}) and
  Lemma~\ref{lem:editing-reg}).\qed
\end{proof}
\section{\HED}
\label{sec:deletion}

In this section, we prove that
\HED\ is \NPC\ if and only if  is a graph with 
at least two edges. We also prove that
these \NPC\ problems cannot be solved in parameterized subexponential time,
unless ETH fails. Then, from Proposition~\ref{pro:equivalence}(\ref{item:dc-equivalence}), we
obtain a dichotomy result for \HEC. We apply a technique similar to that we 
applied for \textsc{Editing} in the last section. 


\begin{proposition}
  \label{pro:deletion-base}
  The following problems are \NPC. Furthermore, they cannot be 
  solved in time , unless ETH fails.
  \begin{enumerate}[(i)]
  \item\label{item:deletion-p3} \PTED~\cite{komusiewicz2012cluster}.
  \item\label{item:deletion-diamond} \DED~\cite{DBLP:journals/disopt/FellowsGKNU11,DBLP:conf/ipec/SandeepS15}.
  \item\label{item:deletion-tree-reg} \HED, if  is a graph with at least two edges and
    has a largest component which is a regular graph or a tree~\cite{DBLP:conf/cocoa/AravindSS15}.
  \end{enumerate}
\end{proposition}

The following Lemma is a consequence of Lemma~\ref{lem:degree} and Proposition~\ref{pro:equivalence}(\ref{item:dc-equivalence}).

\begin{lemma}
  \label{lem:rotate}
  Let  be any graph. Then the following hold true:
  \begin{enumerate}[(i)]
  \item Let  be the subgraph of  obtained by removing all
    vertices with degree .
    Then there is a linear parameterized reduction from \HDED\ to \HED.
  \item Let  be the subgraph of  obtained by removing all
    vertices with degree .
    Then there is a linear parameterized reduction from \HDED\ to \HED. 
  \end{enumerate}
\end{lemma}
\begin{proof}
  The first part directly follows from Lemma~\ref{lem:degree} by setting .
  To prove the second part, consider the problem \HBEC. 
  Let  be the graph obtained by removing all vertices with degree  from . 
  Now, by Lemma~\ref{lem:degree},
  there is a linear parameterized reduction from \textsc{-free Edge Completion} to \HBEC.
  We observe that  is . Hence, by Proposition~\ref{pro:equivalence}(\ref{item:dc-equivalence}),
  there is a linear parameterized reduction from \HDED\ to \HED.\qed
\end{proof}

Given a graph , we keep on deleting either the minimum degree vertices or the 
maximum degree vertices by making sure that the resultant graph has at least 
two edges. We do this process until we obtain a graph
in which vertices with degree more than  induces a graph with 
at most one edge and vertices with degree less than  induces
a graph with at most one edge. We call this method Deletion-Churn.

\defstage{Deletion-Churn()}
{  is a graph with at least two edges.
  \begin{enumerate}[Step 1:]
    \item\label{item:churn-del1} If  is a graph in which the vertices with degree more than 
      induces a subgraph with at most one edge and the vertices with degree less than  induces
      a subgraph with at most one edge, then return .
    \item\label{item:churn-del2} If  is a graph in which the vertices with degree more than 
      induces a subgraph with at least two edges, then delete all vertices with degree  from  and goto Step 1.
    \item\label{item:churn-compl3} If  is a graph in which the vertices with degree less than 
      induces a subgraph with at least two edges, then delete all vertices with degree  from . Goto Step 1.
  \end{enumerate}
}

\begin{observation}
  \label{obs:deletion-churn}
  Let  be a graph with at least two edges. If the vertices with degree more than  
  induces a graph with at most one edge and the vertices with degree less than  induces
  a graph with at most one edge, then  is either regular graph or a forest or a sparse -degree graph.
\end{observation}
\begin{proof}
  Assume that  is not a regular graph.
  Since  has at least two edges and it satisfies the premises, .
  If , the premises imply that  is a forest. Assume that .
  Then we prove that  is a sparse -degree graph.
  For a contradiction, assume that there exists a vertex  such that .
  The premises imply that  has degree at most two, which is a contradiction.\qed
\end{proof}

\begin{lemma}
  \label{lem:deletion-churn}
  Let  be a graph with at least two edges. Then Deletion-Churn()
  returns a graph  such that:
  \begin{enumerate}[(i)]
  \item\label{item:deletion-churn-red} There is a linear parameterized reduction from \HDED\ to \HED.
  \item\label{item:deletion-churn-output}  has at least two edges and is either a regular graph
    or a forest or a sparse -degree graph.
  \end{enumerate}
\end{lemma}
\begin{proof}
  In every step, we make sure that there are at least two edges in the resultant graph.
  Now, the first part follows from Lemma~\ref{lem:rotate} and the second part follows
  from Observation~\ref{obs:deletion-churn}.\qed
\end{proof}

If the output of Deletion-Churn(),  is a regular graph or a forest,
we obtain from Proposition~\ref{pro:deletion-base}(\ref{item:deletion-tree-reg}) that
\HED\ is \NPC\ and cannot be solved in parameterized subexponential
time, unless ETH fails.
Therefore, the only graphs to be handled now is the sparse -degree graphs with 
at least two edges. We do that in the next two subsections.
\subsection{\TDDED}
\label{sec:tdiamond}

We recall that -diamond is the graph  and that 
2-diamond is the diamond graph. Clearly, -diamond is a sparse -degree graph.
In this subsection, we prove that \TDDED\ is \NPC. Further, we 
prove that the problem cannot be solved in parameterized subexponential time,
unless ETH fails. We use an inductive proof where the base case is \DED.
For the proof, we introduce a simple construction, which is 
given below.


\begin{figure}[h]
  \centering
  \includegraphics[width=1.0in]{diamond.pdf}
  \caption{A 2-diamond is isomorphic to a diamond graph.}
  \label{fig:diamond}
\end{figure}


\begin{construction}
  \label{con:diamond}
  Let  be an input to the construction.
  For every edge  in , introduced a clique  of 
  vertices such that every vertex in  is adjacent to both  and .
  This completes the construction. Let  be the resultant graph.
\end{construction}

\begin{lemma}
  \label{lem:tdiamond}
  For any , \TDDED\ is \NPC. Furthermore,
  the problem cannot be solved in time , unless
  ETH fails.
\end{lemma}
\begin{proof}
  The proof is by induction on . If , the problem is \DED\ and the 
  theorem follows from Proposition~\ref{pro:deletion-base}(\ref{item:deletion-diamond}). 
  Assume that  and that the statements hold true for .
  We give a reduction from \TODDED\ to \TDDED. 

  Let  be an instance of \TODDED. Apply Construction~\ref{con:diamond}
  on  to obtain . We claim that  is a yes-instance of \TODDED\
  if and only if  is a yes-instance of \TDDED. 

  Let  be a yes-instance of \TODDED. Let  be a solution of size 
  at most  of . We claim that  is a solution of .
  For a contradiction, assume that  has an induced -diamond on a 
  vertex set . Let  and  be the -degree vertices in 
  the -diamond induced by  in . Now there are three cases to be considered.
  
  Case 1: Both  and  are 
  from a clique  introduced in the construction. 

  We note that  and 
  are adjacent to  and  and all other vertices in . Hence the common neighborhood
  of  and  does not have an independent set of size at least , which is a contradiction.

  Case 2: Let  is from a clique  introduced in the construction and  be .

  The common neighborhood
  of  and  does not have an independent set of size at least , which is a contradiction.

  Case 3: Both  and  are from . The common neighborhood of  and  in 
  is constituted by  and the common neighbors of  a and  in .
  Since  is a clique, it can contribute at most one to the independent set of the 
  common neighborhood of  and . Hence, there should be an independent set of size at least 
  in the common neighborhood of  and  in . Since  is -diamond-free, this is a contradiction.

  Conversely, let  be a solution of size at most  of . We prove that  is -diamond-free.
  For a contradiction, assume that  has an induced -diamond on a vertex set .
  Let  and  be the -degree vertices of the -diamond induced by  in .
  Since there are  common neighbors of  and  () introduced by the construction, 
  there exists a common neighbor  such that 
  induces a -diamond in , which is a contradiction.\qed
\end{proof}

\subsection{Handling sparse -degree graphs}
\label{sec:sparselh}

We recall that for , 
every vertex of a sparse -degree graph  is either of degree 
or of degree  and that  induces a graph with at most one edge and 
induces a graph with at most one edge. We have already handled -diamond graphs.
We handle the rest of the sparse -degree graphs in this subsection.
Let  be any sparse -graph.
There are four cases to be handled:
\begin{description}
\item[Case 1:]  is an independent set;  is an independent set
\item[Case 2:]  induces a graph with one edge;  is an independent set
\item[Case 3:]  is an independent set;  induces a graph with one edge
\item[Case 4:]  induces a graph with one edge;  induces a graph with one edge
\end{description}

\begin{observation}
  \label{obs:sparselh}
  Let  be a sparse -graph with at least two edges. Then the following hold true:
  \begin{enumerate}[(i)]
  \item\label{item:sparselh-1} If , then  is a forest.
  \item\label{item:sparselh-2} If , then  and the equality holds only when  is a diamond.
  \end{enumerate}
\end{observation}
\begin{proof}
  To prove the first part, we observe that 
  has at most one edge. 
  To prove the second part, we observe that if  and if  is not a diamond, then ,
  which is a contradiction.\qed
\end{proof}

Since the case of forest is already handled in Proposition~\ref{pro:deletion-base}(\ref{item:deletion-tree-reg}),
we can safely assume that  and hence .
We start with handling Case 1.
We use a slightly modified 
version of Construction~\ref{con:nonadj}. 
We recall that, in Construction~\ref{con:nonadj}, with an input ,
For every copy  of  in  (a complete graph on ), we introduced  branches such that
each branch along with  form a copy of . 
In the modified
construction, in addition to this, we make every pair of vertices from different branches mutually adjacent.

\begin{construction}
  \label{con:adj}
  Let  be an input to the construction, where  and  are graphs, 
  is a positive integer and  is a subset of vertices of .
  Apply Construction~\ref{con:nonadj} on  to obtain .
  For every pair of vertices  such that  and , where ,
  make  and  adjacent.
  This completes the construction. Let the constructed graph be . 
\end{construction}

Now, we have a lemma similar to Lemma~\ref{lem:con:nonadj-backward}. 
We skip the proof as it is quite similar to that of Lemma~\ref{lem:con:nonadj-backward}. 

\begin{lemma}
  \label{lem:con:adj-backward}
  Let  be obtained by Construction~\ref{con:adj} on 
  the input , where  and  are graphs,  is a positive integer and .
  Then, if  is a yes-instance of \HED, then  is a yes-instance of \HDED, where  is .
\end{lemma}

\begin{lemma}
  \label{lem:lhbipartite}
  Let  be a sparse -graph, where  such that both  and 
  are independent sets.
  Then \HED\ is \NPC. Furthermore, the problem cannot be solved in time
  , unless ETH fails.
\end{lemma}
\begin{proof}
  We reduce from \PTED. Let  be such that
  ,  and  induces a  in .
  Since , such a subset of vertices does exist in .
  Let  be an instance of \PTED.
  Apply Construction~\ref{con:adj} on 
  to obtain . Let  be . 
  We claim that  is a yes-instance of \PTED\ if and only if 
   is a yes-instance of \HED.

  Let  be a yes-instance of \PTED. Let  be a solution
  of size at most  of . For a contradiction, assume that
   has an induced  on a vertex set . Let  and
   be the  and  respectively of the  induced by
   in .

  Claim 1:  is a subset of a single branch, say .

  Since 
  is an independent set in , 
   cannot span over multiple branches. Hence .
  Let . Consider the neighborhood of ,  in .
  Since the neighborhood of every vertex in  is triangle-free,  cannot 
  contain vertices from multiple branches. Further, since  is -free, 
   can have at most one vertex from . Let  is adjacent to vertices in . 
  We note that, by construction, 
   has at most  neighbors in . Therefore , which is a contradiction.
  Thus we obtained that .

  Claim 2: 

  Assume that .
  Since degree of  in  is ,  must have 
  edges to . Therefore,  must be the middle vertex of the 
   formed by  in . 

  Claim 1 and 2 imply that . Hence,
  there exists a branch, other than
  , say , such that . Since  is an independent set,
  no other branches can have vertices in . Therefore,
  . Let  be a vertex in .
  Since  is adjacent to all vertices in , .
  Hence  is a complete bipartite graph. Further, .
  It is straight-forward to verify that  does not have an independent set of size 
  , which is a contradiction.
  Lemma~\ref{lem:con:adj-backward} proves the converse.\qed
\end{proof}


Now we handle the cases in which  induces a graph with one edge. 

\begin{lemma}
  \label{lem:sparselh-vl}
  Let  be a sparse -graph with at least two edges 
  such that  induces a graph with one edge.
  Let  and  be the two adjacent vertices in .
  Let  be the graph induced by .
  Then, there is a linear parameterized reduction from \HDED\ to \HED.
\end{lemma}
\begin{proof}
  Let  be an instance of \HDED. Apply Construction~\ref{con:nonadj}
  on , where  is .
  Let  be the graph obtained from the construction. We claim that 
   is a yes-instance of \HDED\ if and only if  is a 
  yes-instance of \HED.

  Let  be a yes-instance of \HDED\ and let  be a solution of size
  at most  of . For a contradiction, assume that  has an induced
   with a vertex set . It is straight-forward to verify that 
  If a branch vertex  is in , then its neighbor in the same branch 
  must be in  and both acts as  and  in the 
  induced by  in . Hence  has an induced , which is a contradiction.
  Lemma~\ref{lem:con:nonadj-backward} proves the converse.\qed
\end{proof}

\begin{observation}
  \label{obs:sparselh-vl}
  Let  be a sparse -graph with at least two edges where 
  such that  induces a graph with one edge.
  Let  and  be the two adjacent vertices in .
  Let  be the graph induced by .
  Then  has at least two edges.
\end{observation}
\begin{proof}
  By Observation~\ref{obs:sparselh}(\ref{item:sparselh-2}), since  is not a diamond,
  . This implies that  is 
  nonempty. Now the observation follows from the fact that .\qed
\end{proof}

Now we handle Case 2, i.e.,  induces a graph with one edge and 
is an independent set.

\begin{lemma}
  \label{lem:vh1vl0}
  Let  be a sparse  graph where , 
  induces a graph with one edge and  is an independent set.
  Let  be not a -diamond.
  Let  and  be the two adjacent vertices in .
  Let  be . Let  be . 
  Then, there is a linear
  parameterized reduction from \HDED\ to \HED.
\end{lemma}
\begin{proof}
  For convenience, we give a reduction from \HDBEC\ to \HBEC. Then the statements follow 
  from Proposition~\ref{pro:equivalence}(\ref{item:dc-equivalence}).

  Let  be an instance of \HDBEC. Apply Construction~\ref{con:nonadj}
  on , where  is .
  Let  be the graph obtained from the construction. We claim that 
   is a yes-instance of \HDBEC\ if and only if  is a 
  yes-instance of \HBEC.

  Let  be a yes-instance of \HDBEC\ and let  be a solution of size
  at most  of . For a contradiction, assume that  has an induced
   with a vertex set . It is straight-forward to verify that 
  If a branch vertex  is in , then all its neighbors in the same branch are in  and
   acts as  of  in 
  induced by  in . Hence  has an induced , which is a contradiction.
  Lemma~\ref{lem:con:nonadj-backward} proves the converse.\qed  
\end{proof}

\begin{observation}
  \label{obs:vh1vl0}
  Let  be a sparse  graph where , 
  induces a graph with one edge and  is an independent set.
  Let  be not a -diamond, for .
  Let  and  be the two adjacent vertices in .
  Let  be . Let  be . 
  Then  has at least two edges and .
\end{observation}
\begin{proof}
  Follows from the facts that  and  is not a -diamond.\qed
\end{proof}

\begin{lemma}
  \label{lem:sparselh}
  Let  be a sparse -degree graph with at least two edges.
  Then \HED\ is \NPC. Furthermore, the problem cannot be solved in time
  , unless ETH fails.
\end{lemma}
\begin{proof}
  If  induces a graph with an edge, then we apply
  the technique used in Lemma~\ref{lem:sparselh-vl} and obtain
  a graph  with at least two edges. Similarly, if  is 
  not a -diamond and  induces a graph with an edge,
  then we apply the technique used in Lemma~\ref{lem:vh1vl0} to obtain
  a graph  with at least two edges. If the obtained graph 
  is not a sparse -degree graph, then we apply 
  Deletion-Churn() to obtain . We repeat this process until no more
  repetition is possible. Then, it is straight-forward to verify that
  we obtain a graph which is either a -diamond, or a graph handled in Lemma~\ref{lem:lhbipartite} or
  a regular graph or a forest with at least two edges.\qed
\end{proof}
\begin{comment}
We observe that in all the four cases, we either give a direct reduction and prove 
that the problem is \NPC\ or give a reduction from \HDED\ where  is a 
graph with at least two edges and has less number of vertices compared to .

\begin{theorem}
  \label{thm:}
  Let  be a sparse -graph with at least two edges.
  Then \HED\ is \NPC. Furthermore, the problem cannot be solved 
  in time , unless ETH fails.
\end{theorem}
\end{comment}
\subsection{Dichotomy Results}

We are ready to state the dichotomy results and the 
parameterized lower bounds for \HED\ and \HEC.

\begin{theorem}
  \label{thm:final}
  \HED\ is \NPC\ if and only if  is a graph with at least two edges.
  Furthermore, the problem cannot be solved in time .
  \HEC\ is \NPC\ if and only if  is a graph with at least two non-edges.
  Furthermore, the problem cannot be solved in time .
\end{theorem}
\begin{proof}
  Consider \HED. The statements follow from 
  Proposition~\ref{pro:polynomial}(\ref{item:poly-deletion}), Lemma~\ref{lem:deletion-churn}, 
  Proposition~\ref{pro:deletion-base}(\ref{item:deletion-tree-reg}) and Lemma~\ref{lem:sparselh}.
  Now the results for \HEC\ follows from Proposition~\ref{pro:equivalence}(\ref{item:dc-equivalence}).\qed
\end{proof}
\section{Concluding Remarks}

Our results have wide implications on the incompressibility of 
-free edge modification problems. Polynomial parameter transformation (PPT)
is a widely used technique to prove the incompressibility of 
problems. To prove the incompressibility of a problem
it is enough to to give a PPT from a problem which is already
known to be incompressible, under some complexity theoretic assumption.
All our reductions are
linear parameterized reductions and hence are polynomial
parameter transformations. The following lemma is a
direct consequence of Lemma~\ref{lem:degree}.

\begin{lemma}
  \label{lem:incompressibility}
  Let  be a graph and  be any integer. 
  Let  be obtained from  by deleting vertices with degree  or less.
  Then, if \HDEE~(\textsc{Deletion}/\textsc{Completion})
  is incompressible, then \HEE~(\textsc{Deletion}/\textsc{Completion}) is incompressible.
\end{lemma}

We give a simple example to show an implication of this lemma. Consider an 
-sunlet graph which is a graph in which 
a vertex with degree one is attached to each vertex of a cycle of  vertices.
From the incompressibility of \textsc{-free Edge Editing}, \textsc{Deletion} and \textsc{Completion},
for any , it follows that \textsc{-sunlet-free Edge Editing}, \textsc{Deletion} and \textsc{Completion}
are incompressible for any .

We believe that our result is a step towards a dichotomy result on the incompressibility of -free
edge modification problems. 
Another direction is to get a dichotomy result on the complexities of -free edge modification
problems where  is a finite set of graphs. 



\bibliographystyle{plain}
\bibliography{npc}


\end{document}
