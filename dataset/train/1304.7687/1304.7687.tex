

\documentclass[11pt]{article}
\usepackage{times}
\usepackage{enumerate}
\usepackage{comment}
\usepackage{url}
\usepackage{subfig}
\usepackage{amsmath}
\usepackage{amsthm}
\usepackage{amssymb}
\usepackage{wrapfig}
\usepackage{graphicx}
\graphicspath{{./Figs/}}
\usepackage{paralist}
\usepackage[active]{srcltx}
\usepackage{algorithm}
\usepackage{algorithmic}
\usepackage[a4paper,hmargin=1in,vmargin=1.1in]{geometry}
\usepackage{color}
\newenvironment{proof sketch}[1]{\noindent {\emph{Proof sketch of #1:}}}{\hfill \qed}
\newtheorem{theorem}{Theorem}
\newtheorem{proposition}{Proposition}
\newtheorem{lemma}{Lemma}
\newtheorem{definition}{Definition}
\newtheorem{fact}{Fact}
\newtheorem{coro}{Corollary}
\newtheorem{rem}{Remark}
\newtheorem{observation}{Observation}
\newtheorem{defn}{Definition}
\newtheorem{assumption}{Assumption}

\newcommand{\eqdf}{\triangleq}
\newcommand{\eps}{\varepsilon}
\newcommand{\NNN}{[N]}
\newcommand{\poly}{\textit{Poly}}

\newcommand{\dlp}{\textsc{d-lp}}
\newcommand{\plp}{\textsc{p-lp}}
\newcommand{\valp}{P}

\newcommand{\alg}{\textsc{alg}}
\newcommand{\opt}{\textsc{opt}}

\newcommand{\PR}{\Delta_t P}
\newcommand{\PRp}{\Delta_{t'} P}
\newcommand{\PRj}{\Delta_j P}
\newcommand{\PRaj}{\Delta_{a_j} P}
\newcommand{\PRbj}{\Delta_{b_j} P}
\newcommand{\PRa}{\Delta_{a_k} P}
\newcommand{\PRb}{\Delta_{b_k} P}
\newcommand{\PRx}[1]{\Delta_{#1} P}

\newcommand{\TS}{a_k}
\newcommand{\TE}{b_k}
\newcommand{\TSJ}{a_j}
\newcommand{\TEJ}{b_j}

\newcommand{\initt}{\text{\textsf{start}}}
\newcommand{\event}{\text{}}
\newcommand{\load}{\text{\textit{load}}}
\newcommand{\alive}{\text{\textit{Alive}}}
\newcommand{\dead}{\text{\textit{Dead}}}
\newcommand{\cut}{\text{\textit{cut}}}
\newcommand{\paths}{\text{\textit{paths}}_t}
\newcommand{\pathsp}{\text{\textit{paths}}_{t+1}}
\newcommand{\pathtt}[1]{\text{\textit{paths}}_{#1}}

\newcommand{\mainn}{\text{\textsc{Main}}}
\newcommand{\routealg}{\text{\textsc{Route}}}
\newcommand{\unroutealg}{\text{\textsc{UnRoute}}}
\newcommand{\depart}{\text{\textsc{Depart}}}
\newcommand{\reroutealg}{\text{\textsc{MakeFeasible}}}

\newcommand{\NNP}{{\mathbb{N}^+}}
\newcommand{\NN}{{\mathbb{N}}}
\newcommand{\ZZ}{{\mathbb{Z}}}

\begin{document}

\title{A Nonmonotone Analysis with the Primal-Dual Approach: online routing of virtual circuits with unknown durations}

\author{Guy Even\thanks{School of Electrical Engineering, Tel-Aviv
Univ., Tel-Aviv 69978, Israel.
\protect\url{{guy,medinamo}@eng.tau.ac.il}.\newline
M.M was partially funded
by the Israeli Ministry of Science and Technology.} \and Moti Medina
}
\date{}
\maketitle

\begin{abstract}
  We address the question of whether the primal-dual approach for the
  design and analysis of online algorithms can be applied to
  nonmonotone problems. We provide a positive answer by presenting a
  primal-dual analysis to the online algorithm of Awerbuch
  et al.~\cite{awerbuch2001competitive} for routing virtual circuits with unknown durations.
\end{abstract}



\section{Introduction}
The analysis of most online algorithms is based on a potential
function (see, for example,~\cite{AAP, azar1997line, aspnes1997line, awerbuch2001competitive} in the context of online routing).
Buchbinder and Naor~\cite{BNsurvey} presented a primal-dual approach for
analyzing online algorithms.  This approach replaces the need to find
the appropriate potential function by the task of finding an
appropriate linear programming formulation.

The primal-dual approach presented by Buchbinder and Naor has a
monotone nature.  Monotonicity means that: (1)~Variables and
constraints arrive in an online fashion. Once a variable or constraint
appears, it is never deleted. (2)~Values of variables, if updated, are
only increased.  We address the question of whether the primal-dual
approach can be extended to analyze nonmonotone
algorithms\footnote{The only instance we are aware of in which the
  primal-dual approach is applied to nonmonotone variables appears
  in~\cite{buchbinder2011frequency}. In this instance, the change in
  the dual profit, in each round, is at least a constant times the change in the
  primal profit. In general, this property does not hold in a nonmonotone setting.}.

An elegant example of nonmonotone behavior occurs in the problem of
online routing of virtual circuits with unknown durations. In the
problem of routing virtual circuits, we are given a graph with edge
capacities. Each request  consists of a
source-destination pair .  A request  is served by
allocating to it a path from  to . The goal is to serve the
requests while respecting the edge capacities as much as possible. In
the online setting, requests arrive one-by-one. Upon arrival of a
request , the online algorithm must serve . In the special
case of unknown durations, at each time step, the adversary may
introduce a new request or it may terminate an existing request.  When
a request terminates, it frees the path that was allocated to it, thus
reducing the congestion along the edges in the path.  The online
algorithm has no knowledge of the future; namely, no information about
future requests and no information about when existing requests will
end. Nonmonotonicity is expressed in this online problem in two ways:
(1)~Requests terminate thus deleting the demand to serve them. (2)~The
congestion of edges varies in a nonmonotone fashion; an addition of a
path increases congestion, and a deletion of a path decreases
congestion.

Awerbuch et al.~\cite{awerbuch2001competitive} presented an online
algorithm for online routing of virtual circuits when the requests
have unknown durations. In fact, their algorithm resorts to rerouting
to obtain a logarithmic competitive ratio for the load. Rerouting means
that the path allocated to a request is not fixed and the algorithm
may change this path from time to time. Hence, allowing rerouting
increases the nonmonotone characteristics of the problem.

We present an analysis of the online algorithm of Awerbuch et
al.~\cite{awerbuch2001competitive} for online routing of virtual circuits with
unknown durations. Our analysis uses the primal-dual approach, and hence
we show that the primal-dual approach can be applied in nonmonotone
settings.







\section{Problem Definition}
\subsection{Online Routing of Virtual Circuits with Unknown Durations}
Let  denote a directed or undirected graph.
Each edge  in  has a capacity .  A routing request  is a -tuple , where
\begin{inparaenum}[(i)]
\item
   are the source and the destination of the th
  routing request, respectively,
\item  is both the arrival time and the start time of the request, and
\item  is the departure time or end time of the request.
\end{inparaenum}
Let  denote the set of paths in  from  to .
A request  is served if it is allocated  a path in .

Let  denote the set .
The input consists of a sequence of events
.  We assume that time is discrete, and
event  occurs at time .
There are two types of events:
\begin{inparaenum}[(i)]
\item An \emph{arrival} of a request. When a request  arrives, we
  are given the source  and the destination . Note that the
  arrival time  simply equals the current time .
\item A \emph{departure} of a request. When a request  departs
  there is no need to serve it anymore (namely, the departure time  simply equals the  current time ).
\end{inparaenum}

The set of active requests at time  is denoted by  and is defined by



An \emph{allocation} is a sequence  of paths such that
 is a path from the source  to the destination  of request .
Let  denote the number of requests that are routed along edge  by allocation  at time , formally:

The \emph{load} of an edge  at text  is defined by

The \emph{load} of an allocation  at time  is defined by

The \emph{load} of an allocation  is defined by


An algorithm computes an allocation of paths to the requests, and
therefore we abuse notation and identify the algorithm with the
allocation that is computed by it. Namely,  denotes the
allocation computed by algorithm  for an input sequence
.

In the online setting, the events arrive one-by-one, and no information is known about an event before its arrival.
Moreover,
\begin{inparaenum}[(1)]
  \item the length  of the sequence of events is unknown; the input simply stops at some point,
  \item the departure time  is \emph{unknown} (and may even be determined later by the adversary),  and
  \item the online algorithm must allocate a path to the request as soon as the request arrives.
\end{inparaenum}

The \emph{competitive ratio} of an online algorithm  with respect to , and a sequence  is defined
by

where  is an allocation with minimum load.
The \emph{competitive ratio} of an online algorithm  is defined by


\noindent
Note that since every request has a unit demand, we may assume that  for every edge .
\subsection{Rerouting}
In the classical setting, a request  is served by a fixed single
path  throughout the duration of the request.  The term \emph{rerouting} means that we allow the allocation to change the path  that serves .  Thus, there are two extreme cases: (i)~no rerouting at all is permitted (classical
setting), and (ii)~total
flexibility in which, a new allocation can be computed in each time step.

Following the paper by Awerbuch et al.~\cite{awerbuch2001competitive}, we allow the online algorithm to reroute each request
at most  times. In the analysis of the competitive ratio,
we compare the load of the online algorithm with the load of an
optimal (splittable) allocation with total rerouting flexibility. Namely, the
optimal solution recomputes a minimum load allocation at each time step, and, in addition may serve a request by a convex combination of paths.


\section{The Online Algorithm \alg}\label{sec:gen}
In this section we present the online algorithm \alg\ that is listed in Algorithm~\ref{alg:alg}.
Thus algorithm is equivalent to the algorithm presented in~\cite{awerbuch2001competitive}.

\medskip
\noindent
The algorithm maintains the following variables.
\begin{enumerate}
  \item For every edge  a variable . The value of  is exponential in the load of edge .
  \item For every request  a  variable . The value of  is the complement of the ``weight'' of the path  allocated to  at the time the path was allocated.
  \item For every routing request , and for every path  a variable . The value of  indicates whether  is allocated to . That is, the value of  equals  if path  is allocated for request , and  otherwise.
\end{enumerate}

The algorithm \alg\ consists of the following  procedures:
(1)~\mainn, (2)~\routealg, (3)~\depart, (4)~\unroutealg, and (5)~\reroutealg.

The \mainn\ procedure begins with initialization.
For every ,  is initialized to , where .
For every ,  is initialized to zero.
For every , and for every path ,  is  initialized to zero.
Since the number of  and  variables is unbounded, their initialization is done in a ``lazy'' fashion; that is, upon arrival of the th request the corresponding variables are set to zero.


The main procedure \mainn\ proceeds as follows.
For every time step  , if
the event  is an arrival of a request, then the \routealg\ procedure is invoked.
Otherwise, if
the event  is a departure of a request, then the \depart\ procedure is invoked.

The \routealg\ procedure serves request  by allocating a
``lightest'' path  in the set  (recall that  denotes
the set of paths from the source  to the destination ). The
allocation is done by two actions. First, the allocation of  to request
 is indicated by setting . Second, the loads of the edges along  are updated by increasing the
variables  for .  The variable  equals the
``complement'' weight of the allocated path .  Note that this
complement is with respect to half the weight of the path before its
update.

The \depart\ procedure ``frees'' the path that is allocated for ,
by calling the \unroutealg\ procedure.  The \unroutealg\ procedure
frees  by nullifying  and , and by decreasing the
edge variables  for the edges along .  The freeing of 
decreases the load along the edges in . As a result of this
decrease, it may happen that a path allocated
to an alive request might be very heavy compared to a lightest path.
In such a case, the request should be rerouted. This is why the
 \reroutealg\ procedure is invoked after the \unroutealg\ procedure.

Rerouting is done by the \reroutealg\ procedure. This rerouting is done by freeing a path and then routing the request again.
Requests with improved alternative paths are rerouted.


The listing of the online algorithm \alg\ appears in Algorithm~\ref{alg:alg}.

\begin{algorithm}
\footnotesize
    \underline{\textbf{\mainn(})}
    \begin{algorithmic}[1]
     \STATE .
     \STATE, where .
     \STATE.
     \STATE \textbf{Upon } {arrival of event  }\textbf{do}
       \STATE \textbf{\quad if}  is an arrival of request  \textbf{then}
        \textbf{Call \routealg()}. \label{line:main5}
       \STATE \textbf{\quad else} ( is an departure of request )
        \textbf{Call \depart()}.
    \end{algorithmic}

    \smallskip
    \underline{r_k}
    \begin{algorithmic}[1]
        \STATE \label{state:min path} Find the ``lightest'' path: .
        \STATE \label{state:route z}.
        \STATE Route  along : .
        \STATE \textbf{for all}  \textbf{do}
            \STATE \quad \label{state:route x} where .
            \COMMENT {Update edge ``load''}
    \end{algorithmic}

    \smallskip
    \underline{r_k}
    \begin{algorithmic}[1]
        \STATE \textbf{Call} \unroutealg().
        \STATE \textbf{Call} \reroutealg().
    \end{algorithmic}

    \smallskip
    \underline{r_k}
    \begin{algorithmic}[1]
        \STATE Free variables: .
        \STATE \textbf{for all}  \textbf{do}
            \STATE \quad \label{state:unroute x} where .
            \COMMENT {Update edge ``load''}
    \end{algorithmic}

    \smallskip
    \underline{x,z}
    \begin{algorithmic}[1]
        \STATE { \textbf{if}  \textbf{then}}
            \STATE \textbf{\quad Call} \unroutealg().
            \STATE \textbf{\quad Call} \routealg(). \label{line:makefeas3}
    \end{algorithmic}
\caption{\alg: Online routing algorithm. The input consists of (1)~ a graph  where each  has capacity , and (2)~a sequence of events .
}\label{alg:alg}
\end{algorithm}



\section{Primal-Dual Analysis of \alg}~\label{sec:analysis} In this section we prove that the load on
every edge is always , and that each request is rerouted
at most  times.  We refer to an input sequence 
as \emph{feasible} if there is an allocation , such that for all
requests that are alive at time , it holds that .  The following theorem holds under the assumption that the input
sequence  is feasible. Note that the removal of this
assumption increases the competitive ratio only by a constant factor
by standard doubling techniques~\cite{awerbuch2001competitive}.

\begin{theorem}[\cite{awerbuch2001competitive}]~\label{thm:main result}
If the input sequence  is feasible and assuming that , then \alg\ is:
 \begin{enumerate}
   \item An -competitive online algorithm.
   \item Every request is rerouted at most  times.
 \end{enumerate}
\end{theorem}

We point out that the allocation computed by  is
\emph{nonsplittable} in the sense that at every given time each
request is served by a single path.  The optimal allocation, on the
other hand, is both totally
flexible and \emph{splittable}. Namely, the optimal allocation may reroute all
the requests in each time step, and, in addition, may serve a request
by a convex combination of paths.


\medskip \noindent
The rest of the proof is as follows.
We begin by formulating a packing and covering programs for our problem in Section~\ref{sec:lp}.
We then prove Lemma~\ref{lemma:x ub} in Section~\ref{sec:proofof5}.
We conclude the analysis with the proof of Theorem~\ref{thm:main result} in Section~\ref{sec:main}
\subsection{Formulation as an Online Packing Problem}\label{sec:lp}
For the sake of analysis, we define for every prefix of events  a \emph{primal} linear program  and its \emph{dual} linear program . The primal LP is a \emph{covering} LP, and the dual LP is a \emph{packing} LP.
The LP's appear in Figure~\ref{fig:LPframe}.

The variables of the LPs correspond to the variables maintained by
\alg, as follows.  The covering program  has a variable 
for every edge , and a variable  for every .  The packing program  has a variable  for
every request , and for every path .  The
variable  equals to the fraction of 's ``demand'' that is
routed along path .

The dual LP has three types of constraints: capacity constraints,
demand constrains, and sign constraints.  In the fractional setting
the load of an edge is defined by

The capacity constraint in the dual LP requires that the load of each
edge is at most one.  The demand constraints require that each request
 that is alive at time  is allocated a convex combination of
paths.

If the dual LP is feasible, then the
objective function of the dual LP simply equals the number of requests
that are alive at time step , i.e., .

The primal LP has two types of constraints: covering constraints and
sign constraints.  The covering constraints requires that for every
request  that is alive and for every path , the sum of
 and the ``weight'' of  is at least~.  Note that the sign
constraints apply only to the edge variables  whereas the request
variables  are free.


\begin{figure}
\centering
\begin{tabular}{| l |}
\hline
\begin{minipage}{0.67\textwidth}
     \begin{center}
         \\
        (I)
     \end{center}
    \end{minipage}
\\
\hline
\begin{minipage}{0.67\textwidth}
     \begin{center}
        
        \\forall e \in E~:x_{e}^{(t)} \leq 3\:.
    \valp_t \eqdf \sum_{r_k \in \alive_t} z_k^{(t)} +\sum_{e\in E}x_e^{(t)}\:.

      \PR =&  1- \frac 12 \cdot \sum_{e\in p_k}\frac{x_{e}^{(t)}}{c_e} + \sum_{e\in p_k} \frac{x_e^{(t)}}{4c_e} \nonumber\\
          =&1- \frac 14 \cdot \sum_{e\in p_k} \frac{x_e^{(t)}}{c_e}  \label{eqn:case 1}\\
          < & 1\:, \nonumber
    
      x_e ^{(t)} = \frac{1}{4m} \cdot \lambda_e^{\paths(e)}\:.
    
      \sum_{r_j \in \dead_t}\left(\PRaj +\PRbj \right) \leq 0\:. \label{eqn:canceling}
    
        \forall j\in [1,q]~:~\cut(\alpha_j) = \cut(\beta_{\pi(j)})\:.\label{eqn:mapping}
    
A \eqdf \{ (i,j) \mid \alpha_j<\alpha_i<\beta_j<\beta_i\}\:.

\PRx{a_j} = z_j^{(a_j+1)} +\sum_{e\in p_j} \frac{x_e^{(a_j)}}{4c_e}.

\PRx{b_j} = -z_j^{(b_j)} -\sum_{e\in p_j} \frac{x_e^{(b_j+1)}}{4c_e}.

       \sum_{r_j \in \dead_t}\left(\PRaj +\PRbj \right) =&\sum_{r_j \in \dead_t} \sum_e \frac{1}{4c_e}\cdot \left(x_e^{(a_j)} - x_e^{(b_j+1)}\right)\\
=&
\sum_{r_j \in \dead_t} \sum_e \frac{1}{4c_e}\cdot \left(x_e^{(a_j)} - x_e^{(b_{\pi(j)}+1)}\right),\\
\label{eq:local}
x_e^{(a_j)} - x_e^{(b_{\pi(j)}+1)} =0.
\sum_{e\in p_j}\frac{x_{e}^{(j-1)}}{c_e} < 1\:.
        \valp_{j-1}(x,z) < \frac {1}{4} + |\alive_{i-1}| = \frac {1}{2} + (|\alive_i| -1) <  |\alive_i|\:,
    \valp_t(x,z) = 1 + \sum_{j=0}^{t} \PRj < 1 + |\alive_t|\:.
t_2 \eqdf \min\{t \mid
x_e^{(t)} >3 \text{ and  is an original event}\}.

\alive_{\in e}(t_1,t_2) &\eqdf \{r_j \mid t_1 <  a_j < t_2 < b_j, e \in p_j\}.

      x_e^{(t_2)} =& x_e^{(t_1)}\cdot \left(1+
        \frac{1}{4c_e}\right)^{\delta_e}. \nonumber \label{eqn:gneq2}
    
        \left(1+\frac{1}{4c_e}\right)^{\delta_e} \geq 3\:.
      
       |\alive_{\in e}(t_1,t_2)| > 4\cdot c_e \label{eqn:bigdelta}\:.
    \label{eq:Praj}
  \PRaj<1-\frac{1}{4c_e}.

        \valp_{t_2} & = & \frac {1}{4m}\cdot m + \sum_{t=0}^{t_2-1} \PR  \nonumber\\
        & = & \frac 14+ \sum_{r_j \in \dead_{t_2}}( \PRaj +\PRbj) + \sum_{r_j \in \alive_{t_2}} \PRaj \nonumber\\
        & \leq & \frac 14+  \sum_{r_j \in \alive_{t_2}} \PRaj \nonumber\\
        & < & \frac 14+
        |\alive_{t_2}|
-
        \frac{|\alive_{\in e}(t_1,t_2)| }{4c_e}
\nonumber\\
& < &|\alive_{t_2}| \label{eqn: finaly}\:.
    
        \forall t  ~~\forall e\in E : x_e = \frac{1}{4m}\cdot \left(1+\frac{1}{4c_e}\right)^{\paths(e)}\:.
    
        \forall e\in E : \frac{1}{4m}\cdot \left(1+\frac{1}{4c_e}\right)^{\paths(e)} \leq 3\:.
    
\forall e\in E : \paths(e) \leq c_e \cdot 4 \log(12m)\:,
    
\sum_{e\in p_j} \frac{x_e}{c_e} \leq \sum_{e\in p^*} \frac{x_e}{c_e} \leq 3 \cdot \sum_{e\in p^*} \frac{1}{c_e}\:.

\sum_{e\in p} \frac{x_e}{c_e} \geq \frac {1}{4m} \cdot \sum_{e\in p} \frac{1}{c_e} \geq \frac {1}{4m} \cdot\sum_{e\in p^*} \frac{1}{c_e}.


It follows that the number of reroutes each request undergoes is bounded by
,
and the second part of the theorem follows.
\end{proof}
\begin{rem}
  Note that the first routing request will not be rerouted at all, the second routing request will be rerouted at most twice, and so on.
  In general, a routing request that arrives at time  will be rerouted at most  times.
\end{rem}

\section{Discussion}
We present a primal-dual analysis of an online algorithm in a
nonmonotone setting. Specifically, we analyze the online algorithm by
Awerbuch et al.~\cite{awerbuch2001competitive} for online routing of
virtual circuits with unknown durations.  We think that the main
advantage of this analysis is that it provides an alternative
explanation to the stability condition for rerouting that appears
in~\cite{awerbuch2001competitive}.  According to the primal-dual
analysis, rerouting is used simply to preserve the feasibility of the
solution of the covering LP.

Our analysis provides a small improvement compared
to~\cite{awerbuch2001competitive} in the following sense.  The optimal
solution in our analysis is both totally flexible (i.e., may
reroute every request in every time step) and splittable (i.e., may
serve a request using a convex combination of paths). The optimal
solution in the analysis of Awerbuch et
al.~\cite{awerbuch2001competitive} is only totally flexible and must
allocate a path to each request.

The primal-dual approach of Buchbinder and Naor~\cite{BNsurvey} is
based on bounding the change in the value of the primal solution by
the change in the dual solution (this is often denoted by ). The main technical challenge we encountered was that
this bound simply does not hold in our case. Instead, we use an
averaging argument to prove an analogous result (see Lemma~\ref{eqn:np}).


\bibliographystyle{alpha}
\bibliography{reroute}



\end{document}
