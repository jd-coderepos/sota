\documentclass{article}
\usepackage[T1]{fontenc}
\usepackage[utf8]{inputenc}
\usepackage[american]{babel}

\usepackage[nonatbib,final]{neurips_2018}

\usepackage{amsmath}
\usepackage{amsthm}
\usepackage{url}
\usepackage{booktabs}
\usepackage{amsfonts}
\usepackage{nicefrac}
\usepackage{microtype}
\usepackage{enumitem}
\usepackage{etoolbox}
\usepackage{amstext}
\usepackage{amssymb}
\usepackage{csquotes}
\usepackage{stmaryrd}
\usepackage{xparse}

\usepackage{xcolor}
\usepackage[]{todonotes}


\usepackage{appendix}
\usepackage{graphicx}
\usepackage{epstopdf}
\usepackage{relsize}
\usepackage{subcaption}

\usepackage[colorlinks,linkcolor={red!80!black},citecolor={green!80!black},urlcolor={blue!80!black},hypertexnames=false]{hyperref}

\usepackage[noabbrev,capitalize]{cleveref}
\crefformat{equation}{(#2#1#3)}
\crefrangeformat{equation}{(#3#1#4) to~(#5#2#6)}
\crefmultiformat{equation}{(#2#1#3)}{ and~(#2#1#3)}{, (#2#1#3)}{ and~(#2#1#3)}
\crefname{appsec}{Appendix}{Appendices}



\usepackage[backend=biber,style=numeric,maxcitenames=2,maxbibnames=8,sortcites,isbn=false,doi=false,firstinits=true]{biblatex}
\renewbibmacro{in:}{}


\addbibresource{refs.bib}


\DeclareMathOperator*{\argmin}{argmin}
\DeclareMathOperator*{\argmax}{argmax}

\DeclareMathOperator{\diag}{diag}
\newcommand{\distconv}{\xrightarrow{D}}
\newcommand{\wassconv}[1][]{\xrightarrow{{\mathcal W}_{#1}}}
\newcommand{\ud}{\mathrm d}
\newcommand{\dx}{\ud x}
\newcommand{\mudx}{\operatorname{\mu}(\dx)}
\DeclareMathOperator{\D}{\mathcal D}
\DeclareMathOperator{\E}{\mathbb E}
\DeclareMathOperator{\err}{\mathcal E}
\newcommand{\f}{\mathcal F}
\newcommand{\h}{\mathcal H}
\newcommand{\hs}{\mathrm{HS}}
\DeclareMathOperator{\N}{\mathcal N}
\newcommand{\bigO}{\mathcal O}
\newcommand{\bigOmega}{\Omega}
\newcommand{\littleO}{o}
\newcommand{\p}{\mathcal P}
\newcommand{\R}{\mathbb R}
\newcommand{\n}{\mathbb N}
\newcommand{\PP}{\mathbb P}
\newcommand{\QQ}{\mathbb Q}
\DeclareMathOperator{\ReLU}{ReLU}
\DeclareMathOperator{\range}{range}
\DeclareMathOperator{\spn}{span}
\DeclareMathOperator{\supp}{supp}
\DeclareMathOperator{\Tr}{Tr}
\DeclareMathOperator{\rank}{rank}
\DeclareMathOperator{\cond}{cond}
\newcommand{\tp}{^\mathsf{T}}
\newcommand{\x}{\mathcal X}
\newcommand{\y}{\mathcal Y}
\newcommand{\z}{\mathcal Z}
\newcommand{\lip}{\mathrm{Lip}}
\newcommand{\pushforward}{\scalebox{.85}{\#}}
\DeclareMathOperator{\sigmamin}{\sigma_{min}}

\DeclareMathOperator{\MMD}{MMD}
\DeclareMathOperator{\GCMMD}{GCMMD}
\DeclareMathOperator{\SMMD}{SMMD}
\DeclareMathOperator{\LipMMD}{LipMMD}
\newcommand{\optMMD}[1][\Psi]{\operatorname{\mathcal D_{\mathrm{MMD}}^{#1}}}
\NewDocumentCommand{\optLipMMD}{O{\Psi} O{\lambda}}{\operatorname{\mathcal D_{\mathrm{LipMMD}}^{#1,#2}}}
\NewDocumentCommand{\optGCMMD}{O{\Psi} O{\mu} O{\lambda}}{\operatorname{\mathcal D_{\mathrm{GCMMD}}^{#2,#1,#3}}}
\NewDocumentCommand{\optSMMD}{O{\Psi} O{\mu} O{\lambda}}{\operatorname{\mathcal D_{\mathrm{SMMD}}^{#2,#1,#3}}}
\DeclareMathOperator{\DGAN}{\D_{GAN}}
\DeclareMathOperator{\DWGANGP}{\D_{WGAN-GP}}
\DeclareMathOperator{\W}{\mathcal{W}}

\newcommand{\httpsurl}[1]{\href{https://#1}{\nolinkurl{#1}}}

\let\citep\parencite
\let\citet\textcite

\makeatletter
\newcommand\given{\@ifstar{\mathrel{}\middle|\mathrel{}}{\mid}}

\DeclareRobustCommand{\abs}{\@ifstar\@abs\@@abs}
\newcommand{\@abs}[1]{\left\lvert #1 \right\rvert}
\newcommand{\@@abs}[1]{\lvert #1 \rvert}

\DeclareRobustCommand{\norm}{\@ifstar\@norm\@@norm}
\newcommand{\@norm}[1]{\left\lVert #1 \right\rVert}
\newcommand{\@@norm}[1]{\lVert #1 \rVert}


\DeclareRobustCommand{\inner}{\@ifstar\@inner\@@inner}
\newcommand{\@inner}[1]{\left\langle #1 \right\rangle}
\newcommand{\@@inner}[1]{\langle #1 \rangle}

\newcommand{\newreptheorem}[2]{\newtheorem*{rep@#1}{\rep@title}\newenvironment{rep#1}[1]{\def\rep@title{#2 \ref*{##1}}\begin{rep@#1}}{\end{rep@#1}}}
\makeatother

\newtheorem{lem}{Lemma}
\newtheorem{theorem}{Theorem}
\newreptheorem{theorem}{Theorem}
\newtheorem{prop}[lem]{Proposition}
\newreptheorem{prop}{Proposition}
\newtheorem{example}{Example}


\newcommand\numberthis{\addtocounter{equation}{1}\tag{\theequation}}


\newlist{assumplist}{enumerate}{1}
\setlist[assumplist]{label=(\textbf{\Alph*})}
\Crefname{assumplisti}{Assumption}{Assumptions}

\newlist{assumplist2}{enumerate}{1}
\setlist[assumplist2]{label=(\textbf{\Roman*})}
\Crefname{assumplist2i}{Assumption}{Assumptions}

\newlist{proplist}{enumerate}{1}
\setlist[proplist]{label=({\roman*})}
\Crefname{proplisti}{Property}{Properties}


\title{On gradient regularizers for MMD GANs}

\author{
  Michael Arbel\thanks{These authors contributed equally.\vspace*{-5mm}}\\
  Gatsby Computational Neuroscience Unit\\University College London\\
  \texttt{michael.n.arbel@gmail.com}
  \And
  Danica J. Sutherland\footnotemark[1]\\
  Gatsby Computational Neuroscience Unit\\University College London\\
  \texttt{djs@djsutherland.ml}
  \And
  Miko{\l}aj Bi\'nkowski\\
  \phantom{xxxxx}Department of Mathematics\phantom{xxxxx}\\Imperial College London\\
  \texttt{mikbinkowski@gmail.com}
  \vspace*{-5mm}
  \And
  Arthur Gretton\\
  Gatsby Computational Neuroscience Unit\\University College London\\
  \texttt{arthur.gretton@gmail.com}
  \vspace*{-5mm}
}

\begin{document}
\maketitle

\begin{abstract}
We propose a principled method for gradient-based regularization of the critic of GAN-like models trained by adversarially optimizing the kernel of a Maximum Mean Discrepancy (MMD).
We show that controlling the gradient of the critic is vital to having a sensible loss function,
and devise a method to enforce exact, analytical gradient constraints
at no additional cost compared to existing approximate techniques based on additive regularizers.
The new loss function is provably continuous,
and experiments show that it stabilizes and accelerates training,
giving image generation models that outperform state-of-the art methods
on  CelebA and  unconditional ImageNet.
\end{abstract}

\section{Introduction}



There has been an explosion of interest in \emph{implicit generative models} (IGMs) over the last few years,
especially after the introduction of generative adversarial networks (GANs) \parencite{gans}.
These models allow approximate samples from a complex high-dimensional target distribution ,
using a model distribution , where estimation of likelihoods, exact inference, and so on are not tractable.
GAN-type IGMs have yielded very impressive empirical results,
particularly for image generation,
far beyond the quality of samples seen from most earlier generative models \parencite[e.g.][]{progressive-growing,dcgan,wgan-gp,munit,anime-gans}.

These excellent results, however, have depended on adding a variety of methods of regularization and other tricks to stabilize the notoriously difficult optimization problem of GANs \parencite{improved-gans,dcgan}.
Some of this difficulty is perhaps because
when a GAN is viewed as minimizing a discrepancy ,
its gradient

does not provide useful signal to the generator
if the target and model distributions are not absolutely continuous,
as is nearly always the case \parencite{towards-principled-gans}.

An alternative set of losses are the integral probability metrics (IPMs) \citep{Mueller97},
which can give credit to models   ``near'' to the target
distribution  \parencites{wgan}{Bottou:2017}[Section 4 of][]{GneRaf07}.
IPMs are defined in terms of a {\em critic function}: a
``well behaved'' function with  large amplitude
where  and  differ most.
The IPM is  the difference in the expected critic under  and ,
and is zero when the distributions agree.
The Wasserstein IPMs, whose critics are made smooth via a Lipschitz constraint,
have been particularly successful in IGMs \citep{wgan,wgan-gp,sinkhorn-igm}.
But the Lipschitz constraint must hold uniformly, which can be hard
to enforce. A popular approximation has been to apply a gradient constraint
only in expectation \citep{wgan-gp}:
the critic's gradient norm is constrained to be small
on points chosen uniformly between  and .

Another class of IPMs used as IGM losses
are the Maximum Mean Discrepancies (MMDs) \citep{mmd-jmlr}, as in \citep{gmmn,gen-mmd}. Here the critic
function is a member of a reproducing kernel Hilbert space
(except in \cite{coulomb-gan}, who learn a deep approximation to an RKHS critic).
Better performance can be obtained, however, when the MMD kernel is not based
directly on image pixels, but on learned features
of images. Wasserstein-inspired gradient regularization
approaches can be used on the MMD critic when learning these features:
\citep{mmd-gan} uses weight clipping \citep{wgan}, and \citep{Binkowski:2018,cramer-gan}
use a gradient penalty \citep{wgan-gp}.

The recent Sobolev GAN \citep{sobolev-gan}
uses a similar constraint on the expected gradient norm,
but phrases it as estimating a Sobolev IPM rather than loosely approximating Wasserstein.
This expectation can be taken over the same distribution as \citep{wgan-gp},
but other measures are also proposed,
such as .
A second recent approach, the spectrally normalized GAN \citep{Miyato:2018},
controls the Lipschitz constant of the critic by enforcing the spectral norms of the weight matrices to be 1.
Gradient penalties also benefit GANs based on -divergences \citep{NowBotRyo16}:
for instance, the spectral normalization technique of \citep{Miyato:2018} can be applied to the critic network of an -GAN.
Alternatively, a gradient penalty can be defined to approximate the
effect of blurring  and  with noise \citep{roth:regularization}, which
addresses the problem of non-overlapping support \parencite{towards-principled-gans}.
This approach has recently been shown to yield locally convergent optimization
in some cases with non-continuous distributions,
where the original GAN does not \parencite{Mescheder:2018}.


In this paper, we introduce a novel regularization
for the MMD GAN critic of \citep{cramer-gan,mmd-gan,Binkowski:2018},
which {\em directly targets generator performance},
rather than
adopting regularization methods intended to approximate Wasserstein distances \cite{wgan,wgan-gp}.
The new MMD regularizer derives from an approach widely used in semi-supervised
learning \parencite[][Section 2]{Bousquet:2004}, where the aim is to
define a classification function  which is positive on 
(the positive class) and negative on  (negative class),
in the absence of labels on many of the samples.
The decision boundary between the classes is assumed to be in a region
of low density for both  and :  should therefore
be flat where  and  have support (areas with
constant label), and have a larger slope in regions of low density.
\Textcite{Bousquet:2004} propose as their regularizer on  a sum of the
variance and a density-weighted gradient norm.

We adopt a related penalty on the MMD critic,
with the difference that we only apply the penalty on :
thus, the critic is flatter where  has high mass,
but does not vanish on the generator samples from  (which we optimize).
In excluding  from the critic function constraint,
we also avoid the concern raised by \citep{Miyato:2018}
that a critic depending on  will change with the current minibatch --
potentially leading to less stable learning.
The resulting discrepancy is no longer an integral probability metric:
it is asymmetric,
and the critic function class depends on the target  being approximated.

We first discuss in \cref{sec:igm-losses} how MMD-based losses can be used to learn implicit generative models, and how a naive approach could fail.
This motivates our new discrepancies, introduced in \cref{sec:new_discrepancies}.
\Cref{sec:experiments} demonstrates that these losses outperform state-of-the-art models for image generation.



















\section{Learning implicit generative models with MMD-based losses} \label{sec:igm-losses}
An IGM is a model  which aims to approximate a target distribution 
over a space .
We will define  by a \emph{generator} function ,
implemented as a deep network with parameters ,
where  is a space of latent codes, say .
We assume a fixed distribution on ,
say ,
and call  the distribution of .
We will consider learning
by minimizing a
discrepancy  between distributions,
with  and ,
which we call our {\em loss}.
We aim to minimize  with
stochastic gradient descent on an estimator of .























In the present work,
we will build losses 
based on the Maximum Mean Discrepancy,

an integral probability metric where the critic class is the unit ball within ,
the reproducing kernel Hilbert space with a kernel .
The optimization in \eqref{eq:mmd} admits a simple closed-form optimal critic,
.
There is also an unbiased, closed-form estimator of 
with appealing statistical properties \parencite{mmd-jmlr}~--~in particular, its sample complexity is \emph{independent} of the dimension of ,
compared to the exponential dependence \parencite{weed:wasserstein-rates} of the Wasserstein distance


The MMD is \emph{continuous in the weak topology}
for any bounded kernel with Lipschitz embeddings \parencite[Theorem 3.2(b)]{opt-est-probabilities},
meaning that if  converges in distribution to , , then .
( is continuous in the slightly stronger Wasserstein topology \citep[Definition 6.9]{Villani:2009};
 implies ,
and the two notions coincide if  is bounded.)
Continuity means the loss can provide better signal to the generator as  approaches ,
as opposed to e.g.\ Jensen-Shannon where the loss could be constant until suddenly jumping to 0 \parencite[e.g.][Example 1]{wgan}.
The MMD is also {\em strict}, meaning it is zero iff , for \emph{characteristic} kernels \parencite{SriFukLan11}.
The Gaussian kernel yields an MMD both continuous in the weak topology and strict.
Thus in principle, one need not conduct any alternating optimization in an IGM at all,
but merely choose generator parameters  to minimize .

Despite these appealing properties,
using simple pixel-level kernels leads to poor generator samples \parencite{gen-mmd,gmmn,opt-mmd,Bottou:2017}.
More recent MMD GANs \parencite{mmd-gan,cramer-gan,Binkowski:2018}
achieve better results by using a parameterized \emph{family} of kernels, ,
in the Optimized MMD loss
previously studied by \cite{kernel-choice-mmd,opt-est-probabilities}:










We primarily consider kernels defined by some fixed kernel 
on top of a learned low-dimensional representation ,
i.e.\ ,
denoted .
In practice,
 is a simple characteristic kernel, e.g.\ Gaussian,
and  is usually a deep network
with output dimension say  \citep{Binkowski:2018} or even  (in our experiments).
If  is powerful enough, this choice is sufficient;
we need not try to ensure each  is characteristic, as did \cite{mmd-gan}.
\begin{prop} \label{prop:optmmd:strict}
  Suppose ,
  with  characteristic
  and  rich enough that
  for any ,
  there is a  for which .\footnote{
     denotes the \emph{pushforward} of a distribution:
    if , then .}
  Then if , .
\end{prop}
\vspace*{-3mm}
\begin{proof}
  Let  be such that .
  Then, since  is characteristic,
  
\end{proof}
\vspace*{-3mm}
To estimate ,
one can conduct alternating optimization to estimate a 
and then update the generator according to ,
similar to the scheme used in GANs and WGANs.
(This form of estimator is justified by an envelope theorem \citep{envelope-thm},
although it is invariably biased \parencite{Binkowski:2018}.)
Unlike  or , fixing a  and optimizing the generator still yields a sensible distance .

Early attempts at minimizing  in an IGM, though, were unsuccessful \parencite[footnote 7]{opt-mmd}.
This could be because for some kernel classes,
 is stronger than Wasserstein or MMD.

\begin{figure}[t]
\centering
        \includegraphics[width=1.\linewidth]{figures/vector_fields.pdf}
        \caption{The setting of \cref{example:diracgan}.
(a, b): parameter-space gradient fields for the MMD and the SMMD (\cref{subsec:Scaled-MMD}); the horizontal axis is , and the vertical .
         (c): optimal MMD critics for  with different kernels.
         (d): the MMD and the distances of \cref{sec:new_discrepancies} optimized over .}
       \label{fig:mmd_vector_fields}
\end{figure}
\begin{example}[DiracGAN \citep{Mescheder:2018}] \label{example:diracgan}
We wish to model a point mass at the origin of ,
,
with any possible point mass,  for .
We use a Gaussian kernel of any bandwidth,
which can be written as 
with  for 
and
.
Then

\end{example}
Considering ,
even though ,
shows that the Optimized MMD
distance is not continuous in the weak or Wasserstein topologies.

This also causes optimization issues.
\Cref{fig:mmd_vector_fields} (a) shows gradient vector fields in parameter space,
.
Some sequences following  (e.g.\ A) converge to an optimal solution , but some (B) move in the wrong direction,
and others (C) are stuck because there is essentially no gradient.
\Cref{fig:mmd_vector_fields} (c, red) shows that the optimal  critic is very sharp near  and ;
this is less true for cases where the algorithm converged.


We can avoid these issues if we ensure a bounded Lipschitz critic:\footnote{\cite[Theorem 4]{mmd-gan} makes a similar claim to \cref{prop:optmmd:weakness}, but its proof was incorrect: it tries to uniformly bound , but the bound used is for a Wasserstein in terms of .}
\begin{prop} \label{prop:optmmd:weakness}
  Assume the critics
  
  are uniformly bounded and have a common Lipschitz constant:
   and
  .
In particular, this holds when  and

Then  is continuous in the weak topology:
  if , then .
\end{prop}
\begin{proof}
  The main result is \cite[Corollary 11.3.4]{dudley:analysis}.
To show the claim for , note that
  ,
  which since  is

\end{proof}

Indeed, if we put a box constraint on  \parencite{mmd-gan}
or regularize the gradient of the critic function \parencite{Binkowski:2018},
the resulting MMD GAN generally matches or outperforms WGAN-based models.
Unfortunately, though,
an additive gradient penalty doesn't substantially change the vector field of \cref{fig:mmd_vector_fields} (a),
as shown in \cref{fig:vector_fields_all} (\cref{appendix:diracgan-full}).
We will propose distances with much better convergence behavior.








\section{New discrepancies for learning implicit generative models}\label{sec:new_discrepancies}

Our aim here is to introduce a discrepancy
that can provide useful gradient information when used as an IGM loss.
Proofs of results in this section are deferred to \cref{appendix:proofs}.






\subsection{Lipschitz Maximum Mean Discrepancy} \label{sec:lipmmd}
\Cref{prop:optmmd:weakness} shows that an MMD-like discrepancy
can be continuous under the weak topology
even when optimizing over kernels,
if we directly restrict the critic functions to be Lipschitz.
We can easily define such a distance, which we call the Lipschitz MMD: for some ,

For a universal kernel , we conjecture that .
But for any  and ,
 is upper-bounded by ,
as \eqref{eq:LipMMD} optimizes over a smaller set of functions than \eqref{eq:wasserstein}.
Thus
 is also upper-bounded by ,
and hence is continuous in the Wasserstein topology.
It also shows excellent empirical behavior on \cref{example:diracgan}
(\cref{fig:mmd_vector_fields} (d), and \cref{fig:vector_fields_all} in \cref{appendix:diracgan-full}).
But estimating , let alone ,
is in general extremely difficult (\cref{sec:est-lipmmd}),
as finding  requires optimization in the input space.
Constraining the \emph{mean} gradient rather than the \emph{maximum},
as we will do next,
is far more tractable.

\subsection{Gradient-Constrained Maximum Mean Discrepancy}
We define the Gradient-Constrained MMD
for 
and using some measure  as

 denotes the squared  norm.
Rather than directly constraining the Lipschitz constant,
the second term 
encourages the function  to be flat where  has mass.
In experiments we use ,
flattening the critic near the target sample.
We add the first term following \cite{Bousquet:2004}:
in one dimension and with  uniform,

is then an RKHS norm with the kernel
,
which is also a Sobolev space.
The correspondence to a Sobolev norm is lost in higher dimensions \citep[][Ch. 10]{Wendland05},
but we also found the first term to be beneficial in practice.

We can exploit some properties of 
to compute \eqref{eq:SobolevMMD} analytically.
Call the difference in kernel mean embeddings
;
recall .
\begin{prop} \label{prop:Finite_rank_approx}
  Let . Define  with th entry ,
  and  with th entry\footnote{We use  to denote ;
    thus  stacks , \dots,  into one vector.
  } .
  Then under \cref{Moments,Growth,Differentiability,Integrability} in \cref{sec:proofs:distances},

  where  is the kernel matrix ,
   is the matrix of left derivatives \footnote{We use  to denote the partial derivative with respect to ,
    and  that for .
  }
  ,
  and  that of derivatives of both arguments .
\end{prop}
As long as  and  have integrable first moments, and  has second moments,
\cref{Moments,Growth,Differentiability,Integrability} are satisfied e.g.\ by a Gaussian or linear kernel on top of a differentiable .
We can thus estimate the GCMMD based on samples from , , and  by using the empirical mean  for .

This discrepancy indeed works well in practice:
\cref{appendix:sobolev-expt} shows that optimizing our estimate of

yields a good generative model on MNIST.
But the linear system of size  is impractical:
even on  images
and using a low-rank approximation,
the model took days to converge.
We therefore design a less expensive discrepancy in the next section.

The GCMMD is related to some discrepancies previously used in IGM training.
The Fisher GAN \citep{fisher-gan}
uses only the variance constraint
.
The Sobolev GAN \citep{sobolev-gan}
constrains ,
along with a vanishing boundary condition on 
to ensure a well-defined solution (although this was not used in the implementation,
and can cause very unintuitive critic behavior; see \cref{appendix:critic-vector-fields}).
The authors considered several choices of ,
including the WGAN-GP  measure \citep{wgan-gp} and mixtures .
Rather than enforcing the constraints in closed form as we do, though,
these models used additive regularization. We will compare to the Sobolev GAN in experiments.




















\subsection{Scaled Maximum Mean Discrepancy \label{subsec:Scaled-MMD}}

We will now derive a lower bound on the Gradient-Constrained MMD
which retains many of its attractive qualities
but can be estimated in time linear in the dimension .
\begin{prop} \label{prop:Sobolev_Upperbound}
  Make \cref{Moments,Growth,Differentiability,Integrability}.
  For any ,
,
where
  
\end{prop}
We then define the Scaled Maximum Mean Discrepancy based on this bound of \cref{prop:Sobolev_Upperbound}:

Because the constraint in the optimization of \eqref{eq:Scaled_MMD}
is more restrictive than in that of \eqref{eq:SobolevMMD},
we have that .
The Sobolev norm ,
and a fortiori the gradient norm under ,
is thus also controlled for the SMMD critic.
We also show in
\cref{appendix:gcmmd-smmd}
that  behaves similarly to 
on Gaussians.

If 
and ,
then .
Or if  is linear, ,
    then .
Estimating these terms based on samples from  is straightforward,
giving a natural estimator for the SMMD.

Of course,
if  and  are fixed,
the SMMD is simply a constant times the MMD,
and so behaves in essentially the same way as the MMD.
But optimizing the SMMD over a kernel family ,
,
gives a distance very different from  \eqref{eq:Optimized_MMD}.

\Cref{fig:mmd_vector_fields} (b) shows the vector field for the Optimized SMMD loss in \cref{example:diracgan},
using the WGAN-GP measure .
The optimization surface is far more amenable:
in particular the location ,
which formerly had an extremely small gradient that made learning effectively impossible,
now converges very quickly by first reducing the critic gradient until some signal is available.
\Cref{fig:mmd_vector_fields} (d) demonstrates that
, like  and 
but in sharp contrast to ,
is continuous with respect to the location  and provides a strong gradient towards 0.


We can establish that  is continuous in the Wasserstein topology under some conditions:
\begin{theorem}\label{thm:continuity_opt_SMMD}
Let ,
with  a fully-connected -layer network
with Leaky-ReLU activations
whose layers do not increase in width,
and  satisfying mild smoothness conditions 
(\cref{decreasing_dimensions,leaky_relu,Lichitz_kernel,Convexe_Hessian} in \cref{appendix:continuity_opt_smmd}).
Let  be the set of parameters where each layer's weight matrices have condition number .
If  has a density (\cref{full_support}),
then

Thus if , then ,
even if  is chosen to depend on  and .
\end{theorem}



\paragraph{Uniform bounds vs bounds in expectation}
Controlling 
does not necessarily imply a bound on ,
and so does not in general give continuity via \cref{prop:optmmd:weakness}.
\cref{thm:continuity_opt_SMMD} implies that when the network's weights are well-conditioned,
it is sufficient to only control ,
which is far easier in practice than controlling .


If we instead tried to directly controlled  with e.g.\ spectral normalization (SN) \citep{Miyato:2018},
we could significantly reduce the expressiveness of the parametric family.
In \cref{example:diracgan},
constraining  limits us to only .
Thus  is simply the  with an RBF kernel of bandwidth 1,
which has poor gradients when  is far from  (\cref{fig:mmd_vector_fields} (c), blue).
The Cauchy-Schwartz bound of \cref{prop:Sobolev_Upperbound}
allows jointly adjusting the smoothness of  and the critic ,
while SN must control the two independently.
Relatedly, limiting  by limiting the Lipschitz norm of each layer
could substantially reduce capacity,
while  need not be decomposed by layer.
Another advantage is that  provides a data-dependent measure of complexity as in \cite{Bousquet:2004}:
we do not needlessly prevent ourselves from using critics that behave poorly only far from the data.



\paragraph{Spectral parametrization}\label{par:parametrization}
When the generator is near a local optimum,
the critic might identify only one direction on which  and  differ.
If the generator parameterization is such that there is no local way for the generator to correct it,
the critic may begin to single-mindedly focus on this difference,
choosing redundant convolutional filters
and causing the condition number of the weights to diverge.
If this occurs, the generator will be motivated to fix this single direction
while ignoring all other aspects of the distributions,
after which it may become stuck.
We can help avoid this collapse by using a critic parameterization
that encourages diverse filters with higher-rank weight matrices.
\Textcite{Miyato:2018} propose to parameterize the weight matrices as
,
where  is the spectral norm of .
This parametrization works particularly well with ;
\cref{fig:celebA_scores_and_singular_values} (b) shows the singular values of the second layer of a critic's network (and \cref{fig:singular_values-full}, in \cref{appendix:smmd_vs_sn}, shows more layers), while \cref{fig:celebA_scores_and_singular_values} (d) shows the evolution of the condition number during training.
The conditioning of the weight matrix remains stable throughout training with spectral parametrization,
while it worsens through training in the default case.











\section{Experiments} \label{sec:experiments}

We evaluated unsupervised image generation on three datasets:
CIFAR-10 \parencite{cifar10} ( images, ),
CelebA \parencite{celeba} ( face images, resized and cropped to  as in \cite{Binkowski:2018}),
and the more challenging ILSVRC2012 (ImageNet) dataset \parencite{Russakovsky:2014} ( images, resized to ).
Code for all of these experiments is available at
\httpsurl{github.com/MichaelArbel/Scaled-MMD-GAN}.

\textbf{Losses} All models are based on a scalar-output critic network ,
except MMDGAN-GP where  as in \cite{Binkowski:2018}.
The WGAN and Sobolev GAN use a critic ,
while the GAN uses a discriminator .
The MMD-based methods use a kernel ,
except for MMDGAN-GP which uses a mixture of RQ kernels as in \cite{Binkowski:2018}.
Increasing the output dimension of the critic or using a different kernel didn't substantially change the performance of our proposed method.
We also consider SMMD with a linear top-level kernel, ;
because this becomes essentially identical to a WGAN (\cref{appendix:wgan-linear-kernel}),
we refer to this method as SWGAN.
SMMD and SWGAN use ; Sobolev GAN uses  as in \cite{sobolev-gan}.
We choose  and an overall scaling to obtain the losses:




\textbf{Architecture} For CIFAR-10, we used the CNN architecture proposed by \cite{Miyato:2018}
with a 7-layer critic and a 4-layer generator.
For CelebA, we used a 5-layer DCGAN discriminator and a 10-layer ResNet generator as in \parencite{Binkowski:2018}.
For ImageNet, we used a 10-layer ResNet for both the generator and discriminator.
In all experiments we used  filters for the smallest convolutional layer,
and double it at each layer (CelebA/ImageNet) or every other layer (CIFAR-10).
The input codes for the generator are drawn from .
We consider two parameterizations for each critic:
a standard one where the parameters can take any real value,
and a spectral parametrization (denoted SN-)
as above \parencite{Miyato:2018}.
Models without explicit gradient control
(SN-GAN, SN-MMDGAN, SN-MMGAN-L2, SN-WGAN)
fix , for spectral normalization;
others learn , using a spectral parameterization.

\textbf{Training}
All models were trained for  generator updates on a single GPU,
except for ImageNet where the model was trained on 3 GPUs simultaneously.
To limit communication overhead we averaged the MMD estimate on each GPU,
giving the block MMD estimator \parencite{b-test}.
We always used  samples per GPU from each of  and ,
and  critic updates per generator step.
We used initial learning rates of  for CIFAR-10 and CelebA,
 for ImageNet,
and decayed these rates using the KID adaptive scheme of \cite{Binkowski:2018}:
every  steps, generator samples are compared to those from  steps ago,
and if the relative KID test \parencite{3sample} fails to show an improvement three consecutive times,
the learning rate is decayed by .
We used the Adam optimizer \parencite{adam} with , .


\textbf{Evaluation}
To compare the sample quality of different models,
we considered three different scores
based on the Inception network \parencite{inception} trained for ImageNet classification,
all using default parameters in the implementation of \cite{Binkowski:2018}.
The \emph{Inception Score (IS)} \parencite{improved-gans}
is based on the entropy of predicted labels;
higher values are better.
Though standard, this metric has many issues,
particularly on datasets other than ImageNet \parencite{note-on-inception,fid,Binkowski:2018}.
The \emph{FID} \parencite{fid}
instead measures the similarity of samples from the generator and the target
as the Wasserstein-2 distance between Gaussians fit to
their intermediate representations.
It is more sensible than the IS and becoming standard,
but its estimator is strongly biased \parencite{Binkowski:2018}.
The \emph{KID} \parencite{Binkowski:2018}
is similar to FID,
but by using a polynomial-kernel MMD its estimates enjoy better statistical properties
and are easier to compare.
(A similar score was recommended by \cite{empirical-evaluation}.)





 \begin{figure}[p]
  \centering
    \includegraphics[width=1.\linewidth]{figures/celebA_analysis.pdf}
    \caption{The training process on CelebA.
      (a) KID scores. We report a final score for SN-GAN slightly before its sudden failure mode;
          MMDGAN and SN-MMDGAN were unstable and had scores around 100.
      (b) Singular values of the second layer, both early (dashed) and late (solid) in training.
      (c)  for several MMD-based methods.
      (d) The condition number in the first layer through training.
      SN alone does not control ,
      and SMMD alone does not control the condition number.
    }
    \label{fig:celebA_scores_and_singular_values}
 \end{figure}

\begin{table}[ht]
  \centering
  \caption{Mean (standard deviation) of score estimates, based on  samples from each model.} \label{tab:scores}
  \begin{subtable}[t]{\linewidth}
    \centering
    \caption{CIFAR-10 and CelebA.}
    \label{tab:celebA_cifar10_scores}
    \begin{tabular}{lllllll}
\toprule
{} & \multicolumn{3}{l}{CIFAR-10} & \multicolumn{3}{l}{CelebA} \\
Method &                                    IS &                                    FID &                       KID &                                    IS &                                    FID &                      KID \\
\midrule
WGAN-GP      &             6.90.2 &             31.10.2 &             22.21.1 &             2.70.0 &             29.20.2 &            22.01.0 \\
MMDGAN-GP-L2 &             6.90.1 &             31.40.3 &             23.31.1 &             2.60.0 &             20.50.2 &            13.01.0 \\
Sobolev-GAN  &             7.00.1 &             30.30.3 &             22.31.2 &     \textbf{2.90.0}&             16.40.1 &            10.60.5 \\
SMMDGAN      &             7.00.1 &             31.50.4 &             22.21.1 &             2.70.0 &             18.40.2 &            11.50.8 \\
SN-GAN       &     \textbf{7.20.1}&             26.70.2 &    \textbf{16.10.9} &             2.70.0 &             22.60.1 &            14.61.1 \\
SN-SWGAN     &     \textbf{7.20.1}&             28.50.2 &    \textbf{17.61.1} &             2.80.0 &             14.10.2 &            \phantom{0}7.70.5 \\
SN-SMMDGAN   &     \textbf{7.30.1}&     \textbf{25.00.3}&    \textbf{16.62.0} &             2.80.0 &     \textbf{12.40.2}&  \textbf{\phantom{0}6.10.4} \\
\bottomrule
\end{tabular}
   \end{subtable}
  \begin{subtable}[t]{\linewidth}
    \centering
    \caption{ImageNet.}
    \label{tab:imagenet_scores}
    \begin{tabular}{llll}
\toprule
Method &                                     IS &                                    FID &                       KID \\
\midrule
BGAN       &             10.70.4 &             43.90.3 &             47.01.1 \\
SN-GAN     &             \textbf{11.20.1} &             47.50.1 &             44.42.2 \\
SMMDGAN    &             10.70.2 &             38.40.3 &             39.32.5 \\
SN-SMMDGAN & 10.90.1 &  \mathsmaller{\pm} &  \mathsmaller{\pm} \\
\bottomrule
\end{tabular}
   \end{subtable}
\end{table}





\begin{figure}[p]
    \centering
    \begin{subfigure}[t]{0.30\textwidth}
        \centering
        \includegraphics[width=\linewidth]{samples/imagenet/img_79_5X5.jpg}
        \caption{Scaled MMD GAN with SN}
        \label{fig:imagenet_sn_smmd}
    \end{subfigure}
    \hfill
    \begin{subfigure}[t]{0.30\textwidth}
        \centering
        \includegraphics[width=\linewidth]{samples/imagenet/sngan.jpg}
        \caption{SN-GAN} \label{fig:imagenet_sngan}
    \end{subfigure}
    \hfill
    \begin{subfigure}[t]{0.30\textwidth}
        \centering
        \includegraphics[width=\linewidth]{samples/imagenet/bgan.jpg}
        \caption{Boundary Seeking GAN} \label{fig:imagenet_bgan}
    \end{subfigure}


    \begin{subfigure}[t]{0.30\textwidth}
        \centering
        \includegraphics[width=\linewidth]{samples/celebA_sn_smmd_resnet_5X5.jpg}
        \caption{Scaled MMD GAN with SN} \label{fig:celebA_sn_smmd}
    \end{subfigure}
    \hfill
    \begin{subfigure}[t]{0.30\textwidth}
        \centering
        \includegraphics[width=\linewidth]{samples/celebA_sn_swgan_resnet_5X5.jpg}
        \caption{Scaled WGAN with SN} \label{fig:celebA_sn_swgan}
    \end{subfigure}
    \hfill
    \begin{subfigure}[t]{0.30\textwidth}
        \centering
        \includegraphics[width=\linewidth]{samples/celebA_mmd_rq_5X5.jpg}
        \caption{MMD GAN with GP+L2} \label{fig:celebA_mmd_gp}
    \end{subfigure}
    \caption{Samples from various models. Top:  ImageNet; bottom:  CelebA.}
\label{fig:samples}
\end{figure}

\textbf{Results}
\cref{tab:celebA_cifar10_scores} presents the scores for models trained on both CIFAR-10 and CelebA datasets. On CIFAR-10,  SN-SWGAN and  SN-SMMDGAN performed comparably to SN-GAN. But on CelebA,
SN-SWGAN and SN-SMMDGAN dramatically outperformed the other methods with the same architecture in all three metrics. It also trained faster, and consistently outperformed other methods over multiple initializations (\cref{fig:celebA_scores_and_singular_values} (a)).
It is worth noting that SN-SWGAN far outperformed WGAN-GP on both datasets.
\cref{tab:imagenet_scores} presents the scores for SMMDGAN and SN-SMMDGAN trained on ImageNet,
and the scores of pre-trained models using BGAN \citep{began} and SN-GAN \citep{Miyato:2018}.\footnote{These models are courtesy of the respective authors and also trained at  resolution. SN-GAN used the same architecture as our model, but trained for  generator iterations; BS-GAN used a similar 5-layer ResNet architecture and trained for 74 epochs, comparable to SN-GAN.}
The proposed methods substantially outperformed both methods in FID and KID scores.
\cref{fig:samples} shows samples on ImageNet and CelebA;
\cref{appendix:additional-samples} has more.

\textbf{Spectrally normalized WGANs / MMDGANs}
To control for the contribution of the spectral parametrization to the performance,
we evaluated variants of MMDGANs, WGANs and Sobolev-GAN using spectral normalization
(in \cref{tab:sn_cifar10_scores}, \cref{appendix:smmd_vs_sn}).
WGAN and Sobolev-GAN led to unstable training and didn't converge at all (\cref{fig:score_per_iter_cifar10_sn}) despite many attempts.
MMDGAN converged on CIFAR-10 (\cref{fig:score_per_iter_cifar10_sn})  but was unstable on CelebA (\cref{fig:loss_celebA}).
The gradient control due to SN is thus probably too loose for these methods.
This is reinforced by \cref{fig:celebA_scores_and_singular_values} (c),
which shows that the expected gradient of the critic network
is much better-controlled by SMMD, even when SN is used.
We also considered variants of these models with a learned 
while also adding a gradient penalty and an  penalty on critic activations \citep[footnote 19]{Binkowski:2018}.
These generally behaved similarly to MMDGAN, and didn't lead to substantial improvements.
We ran the same experiments on CelebA,
but aborted the runs early when it became clear that training was not successful.


\textbf{Rank collapse}
We occasionally observed the failure mode for SMMD where the critic becomes low-rank,
discussed in \cref{par:parametrization},
especially on CelebA;
this failure was obvious even in the training objective.
\Cref{fig:celebA_scores_and_singular_values} (b) is one of these examples.
Spectral parametrization seemed to prevent this behavior.
We also found one could avoid collapse by reverting to an earlier checkpoint
and increasing the RKHS regularization parameter ,
but did not do this for any of the experiments here.


\section{Conclusion}
We studied gradient regularization for MMD-based critics in implicit generative models,
clarifying how previous techniques relate to the  loss.
Based on these insights,
we proposed the Gradient-Constrained MMD and its approximation the Scaled MMD,
a new loss function for IGMs that controls gradient behavior in a principled way
and obtains excellent performance in practice.

One interesting area of future study for these distances is their behavior
when used to diffuse particles distributed as  towards particles distributed as .
\Textcite[Appendix A.1]{sobolev-gan} began such a study for the Sobolev GAN loss; \cite{sobolev-descent} proved convergence and studied discrete-time approximations.

Another area to explore is the geometry of these losses,
as studied by \textcite{Bottou:2017},
who showed potential advantages of the Wasserstein geometry over the MMD.
Their results, though, do not address any distances based on optimized kernels;
the new distances introduced here
might have interesting geometry of their own.



\printbibliography

\clearpage
\appendix



\section{Proofs} \label{appendix:proofs}

We first review some basic properties of Reproducing Kernel Hilbert
Spaces. We consider here a separable RKHS  with basis ,
where  is either finite if  is finite-dimensional, or 
otherwise. We also assume that the reproducing kernel  is continuously
twice differentiable.

We use a slightly nonstandard notation for derivatives:
 denotes the th partial derivative of  evaluated at ,
and 
denotes .

Then the following reproducing properties hold for
any given function  in  \citep[Lemma 4.34]{SteChr08}:



We say that an operator  is Hilbert-Schmidt if 
is finite.  is called the Hilbert-Schmidt norm
of . The space of Hilbert-Schmidt operators itself a Hilbert space
with the inner product .
Moreover, we say that an operator  is trace-class if its trace
norm is finite, i.e. .
The outer product  for 
gives an  operator such that
 for all  in .

Given two vectors  and  in  and a Hilbert-Schmidt operator
 we have the following properties:
\begin{proplist}
  \item \label{HS_norm_rank_one}
    The outer product  is a Hilbert-Schmidt operator with Hilbert-Schmidt norm given by: .
  \item \label{HS_inner_rank_one}
    The inner product between two rank-one operators  and  is
    .
  \item \label{HS_inner_prod_rank_one}
    The following identity holds: .
\end{proplist}

Define the following covariance-type operators:

these are useful in that, using \eqref{eq:reproducing_prop} and \eqref{eq:reproducing_derivative},
.

\subsection{Definitions and estimators of the new distances} \label{sec:proofs:distances}

We will need the following assumptions about the distributions  and ,
the measure ,
and the kernel :
\begin{assumplist}
  \item \label{Moments}  and  have integrable first moments.
  \item \label{Growth}  grows at most linearly in : for all  in ,  for some constant .
  \item \label{Differentiability} The kernel  is twice continuously differentiable.
  \item \label{Integrability} The functions  and  for  are -integrable.
\end{assumplist}
When , \cref{Growth} is automatically satisfied by a  such as the Gaussian;
when  is linear, it is true for a quite general class of networks  \citep[Lemma 1]{Binkowski:2018}.

We will first give a form for the Gradient-Constrained MMD \eqref{eq:SobolevMMD}
in terms of the operator \eqref{eq:d-op}:
\begin{prop} \label{prop:dot_prod_expression}
Under \cref{Moments,Growth,Differentiability,Integrability},
the Gradient-Constrained MMD is given by
 
\end{prop}
\begin{proof}[Proof of \cref{prop:dot_prod_expression}]
Let  be a function in .
We will first express the squared -regularized Sobolev norm of 
\eqref{eq:sobolev-norm}
as a quadratic form in .
Recalling the reproducing properties
of \eqref{eq:reproducing_prop} and \eqref{eq:reproducing_derivative},
we have:

Using \cref{HS_inner_rank_one} and the operator \eqref{eq:d-op}, one further gets

Under \cref{Integrability}, and using \cref{lem:Bochner_interversion},
one can take the integral inside the inner product, which leads to
.
Finally, using \cref{HS_inner_prod_rank_one} it follows that


Under \cref{Moments,Growth}, \cref{lem:Bochner_interversion} applies,
and it follows that  is also Bochner integrable under  and .
Thus

where  is defined as this difference in mean embeddings.

Since  is symmetric positive definite,
its square-root  is well-defined and is also invertible.
For any ,
let ,
so that .
Note that for any , there is a corresponding .
Thus we can re-express the maximization problem in \eqref{eq:SobolevMMD} in terms of :

\end{proof}

\Cref{prop:dot_prod_expression}, though, involves inverting the infinite-dimensional operator 
and thus doesn't directly give us a computable estimator.
\Cref{prop:Finite_rank_approx} solves this problem in the case where  is a discrete measure:
\begin{repprop}{prop:Finite_rank_approx}
  Let  be an empirical measure of  points.
  Let  have th entry ,
  and  have th entry\footnote{We use  to denote ;
    thus  stacks , \dots,  into one vector.
  } .
  Then under \cref{Moments,Growth,Differentiability,Integrability},
  the Gradient-Constrained MMD is
  
  where  is the kernel matrix ,
   is the matrix of left derivatives ,
  and  that of derivatives of both arguments .
\end{repprop}

Before proving \cref{prop:Finite_rank_approx},
we note the following interesting alternate form.
Let  be the th standard basis vector for ,
and define  as the linear operator

Then ,
and .
Thus we can write




\begin{proof}[Proof of \cref{prop:Finite_rank_approx}]
\label{proof:Finite_rank_approx}

Let  be the solution to the regression problem :

Taking the inner product of both sides of \eqref{eq:Functional_Regression}
with  for each 
yields the following  equations:

Doing the same with  gives  equations:

From \eqref{eq:Functional_Regression}, it is clear that  is a linear
combination of the form:

where the coefficients 
and  satisfy the system of equations \eqref{eq:system_g} and \eqref{eq:system_grad_g}.
We can rewrite this system as

where ,  are the identity matrices of dimension , .
Since  and  must be positive semidefinite, an inverse exists.
We conclude by noticing that

\end{proof}

The following result was key to our definition of the SMMD in \cref{subsec:Scaled-MMD}.
\begin{repprop}{prop:Sobolev_Upperbound}
  Under \cref{Moments,Growth,Differentiability,Integrability},
  we have for all  that
  
  where
  .
\end{repprop}
\begin{proof}[Proof of \cref{prop:Sobolev_Upperbound}]
\label{proof:Sobolev_Upperbound}
The key idea here is to use the Cauchy-Schwarz inequality for the Hilbert-Schmidt inner product.
Letting ,
 is

 follows from the reproducing properties \eqref{eq:reproducing_prop}
and \eqref{eq:reproducing_derivative} and \cref{HS_inner_rank_one}.
 is obtained using \cref{HS_inner_prod_rank_one},
while  follows from the Cauchy-Schwarz inequality and \cref{HS_norm_rank_one}.
\end{proof}
\begin{lem}
\label{lem:Bochner_interversion}Under \cref{Integrability}, 
is Bochner integrable and its integral  is a trace-class symmetric
positive semi-definite operator with  invertible
for any positive . Moreover, for any Hilbert-Schmidt operator
 we have: .

Under \cref{Moments,Growth},  is Bochner integrable with
respect to any probability distribution  with finite first moment
and the following relation holds:

for all  in .
\end{lem}
\begin{proof}
The operator  is positive self-adjoint.
It is also trace-class,
as by the triangle inequality

By \cref{Integrability},
we have that 
which implies that  is -integrable in the Bochner sense
\parencite[Definition 1 and Theorem 2]{Retherford:1978}. Its integral  is trace-class and satisfies
.  This allows to have  for all Hilbert-Schmidt operators . Moreover, the integral preserves the symmetry and positivity. It follows that  is invertible.

The Bochner integrability of  under a distribution  with finite moment follows directly from \cref{Moments,Growth}, since . This allows us to write .
\end{proof}


\subsection{Continuity of the Optimized Scaled MMD in the Wasserstein topology}\label{appendix:continuity_opt_smmd}
To prove \cref{thm:continuity_opt_SMMD},
we we will first need some new notation.

We assume the kernel is ,
i.e.\ ,
where the representation function  is a network 
consisting of  fully-connected layers:

The intermediate representations  are of dimension ,
the weights  are matrices in ,
and biases  are vectors in .
The elementwise activation function  is given by
,
and for  the activation  is a leaky ReLU with
leak coefficient :



The parameter  is the concatenation of all the layer parameters:

We denote by  the set of all such possible parameters,
i.e.\ .
Define the following restrictions of :

 is the set of those parameters such that  have a small condition number,
.
 is the set of per-layer normalized parameters with a condition number bounded by .


Recall the definition of Scaled MMD, \cref{eq:Scaled_MMD},
where  and  is a probability measure:

The Optimized SMMD over the restricted set  is given by:

The constraint to  is critical to the proof.
In practice, using a spectral parametrization helps enforce this assumption, as shown in \cref{fig:celebA_scores_and_singular_values,fig:singular_values-full}.
Other regularization methods, like orthogonal normalization \cite{Brock:2016}, are also possible.


We will use the following assumptions:
\begin{assumplist2}
  \item \label{full_support}  is a probability distribution absolutely continuous with respect to the Lebesgue measure.
  \item \label{decreasing_dimensions} The dimensions of the weights are decreasing per layer:  for all .
  \item \label{leaky_relu} The non-linearity used is Leaky-ReLU, \eqref{eq:lReLU}, with leak coefficient .
  \item \label{Lichitz_kernel} The top-level kernel  is globally Lipschitz in the RKHS norm: there exists a positive constant  such that  for all  and  in .
  \item \label{Convexe_Hessian} There is some  for which  satisfies
  
\end{assumplist2}

\cref{full_support}
ensures that the points where  is not differentiable
are reached with probability  under .
This assumption can be easily satisfied
e.g.\ if we define  by adding Gaussian noise to .

\Cref{decreasing_dimensions} helps ensure that the span of  is never contained in the null space of .
Using Leaky-ReLU as a non-linearity, \cref{leaky_relu},
further ensures that
the network  is locally full-rank almost everywhere;
this might not be true with ReLU activations, where it could be always .
\cref{decreasing_dimensions,leaky_relu} can be easily satisfied by design of the network.


\cref{Lichitz_kernel,Convexe_Hessian} only depend on the top-level kernel  and are easy to satisfy in practice.
In particular, they always hold for a smooth translation-invariant kernel, such as the Gaussian,
as well as the linear kernel.

We are now ready to prove \cref{thm:continuity_opt_SMMD}.

\begin{reptheorem}{thm:continuity_opt_SMMD}
Under \cref{full_support,decreasing_dimensions,leaky_relu,Lichitz_kernel,Convexe_Hessian},

which implies that if ,
then .
\end{reptheorem}
\begin{proof}

Define the pseudo-distance corresponding to the kernel 

Denote by  the optimal transport metric
between  and  using the cost , given by

where  is the set of couplings with marginals  and .
By \cref{thm:mmd-w-upperbound},

Recall that  is Lipschitz,
,
so along with \cref{Lichitz_kernel} we have that

Thus

where  is the standard Wasserstein distance \eqref{eq:wasserstein},
and so


We have that

where the middle term is a  matrix
and the outer terms are vectors of length .
Thus \cref{Convexe_Hessian} implies that
,
and hence

so that


Using \cref{appendix:prop:pseudo_homogeneity}, we can write

with .
Then we have

where we used
.
But by \cref{lem:grad_essinf},
for Lebesgue-almost all ,
.
Using \cref{full_support},
this implies that


Thus for any ,

The desired bound on  follows immediately.
\end{proof}

\begin{lem} \label{thm:mmd-w-upperbound}
  Let  be the continuous kernel of an RKHS  defined on a Polish space , and define the corresponding pseudo-distance . Then the following  inequality holds for any distributions  and  on , including when the quantities are infinite:
  
\end{lem}
\begin{proof}
Let  and  be two probability distributions,
and let  be the set of couplings between them.
Let  be an optimal coupling,
which is guaranteed to exist \citep[Theorem 4.1]{Villani:2009};
by definition .
When  the inequality trivially holds, so assume that .

Take a sample  and a function  with  . By the Cauchy-Schwarz inequality,


Taking the expectation with respect to , we obtain

The right-hand side is just the definition of .
By Jensen's inequality, the left-hand side is lower-bounded by
 since  has marginals  and .
We have shown so far that for any  with ,

the result follows by taking the supremum over .
\end{proof}

\begin{lem}
  \label{appendix:prop:pseudo_homogeneity}
  Let .
  There exists a corresponding scalar  and
  ,
  defined by \eqref{eq:def:psi-kappa-1},
  such that for all ,
  
\end{lem}
\begin{proof}
Set , ,
and .
Note that the condition number is unchanged,
,
and ,
so .
It is also easy to see from \eqref{eq:lReLU} that

so that

\end{proof}

\begin{lem} \label{lem:grad_essinf}
Make \cref{decreasing_dimensions,leaky_relu}, and let .
Then the set of inputs for which any intermediate activation is exactly zero,

has zero Lebesgue measure.
Moreover, for any ,
 exists and

\end{lem}
\begin{proof}
First, note that the network representation at layer  is piecewise affine.
Specifically,
define  by, using \cref{leaky_relu},

it is undefined when any ,
i.e.\ when .
Let .
Then

and thus

where
, ,
and ,
so long as .

Because ,
we have  and ;
also, , .
Thus ,
and using \cref{decreasing_dimensions} with \cref{lem:min-sv}
gives .
In particular, each  is full-rank.



Next, note that  and 
each only depend on  through the activation patterns .
Letting  denote the full activation patterns up to level ,
we can thus write

There are only finitely many possible values for ;
we denote the set of such values as .
Then we have that

Because each  is of rank ,
each set in the union is either empty
or an affine subspace of dimension .
As each ,
each set in the finite union has zero Lebesgue measure,
and  also has zero Lebesgue measure.

We will now show that the activation patterns are piecewise constant,
so that  for all .
Because ,
we have ,
and in particular

Thus, take some ,
and find the smallest absolute value of its activations,
;
clearly .
For any  with ,
we know that for all  and ,

implying that
 as well as .
Thus for any point ,
.
Finally, we obtain

\end{proof}

\begin{lem} \label{lem:min-sv}
  Let , , with . Then
  .
\end{lem}
\begin{proof}
A more general version of this result can be found in \cite[Theorem 2]{Gungor:2007}; we provide a proof here for completeness.

  If  has a nontrivial null space,  and the inequality holds.
  Otherwise, let  denote .
  Recall that for  with ,
  
  Thus, as  for ,
  
\end{proof}


\subsubsection{When some of the assumptions don't hold} \label{sec:counterexamples}
Here we analyze through simple examples what happens when the condition number can be unbounded, and when \cref{decreasing_dimensions}, about decreasing widths of the network, is violated.
\paragraph{Condition Number:}\label{example_condition_number}
We start by a first example where the condition number can be arbitrarily high. We consider a two-layer network on , defined by

where . As  approaches  the matrix  becomes singular which means that its condition number blows up. We are interested in analyzing the behavior of the Lipschitz constant of  and the expected squared norm of its gradient under  as  approaches .

One can easily compute the squared norm of the gradient of  which is given by

Here , ,  and  are defined by \cref{eq:sets_examples_1} and are represented in \cref{fig:example_1_domains}:


\begin{figure}[ht]
\centering
        \includegraphics[width=0.7\linewidth]{figures/network_domains.png}
            \caption{Decomposition of  into 4 regions , ,  and  as defined in \cref{eq:sets_examples_1}. As  approaches , the area of sets  and  becomes negligible.}
    \label{fig:example_1_domains}
\end{figure}


It is easy to see that whenever  has a density, the probability of the sets  and  goes to  are . Hence one can deduce that  when . On the other hand, the squared Lipschitz constant of  is given by  which converges to . This shows that controlling the expectation of the gradient doesn't allow to effectively control the Lipschitz constant of .

\paragraph{Monotonicity of the dimensions:}\label{example_monotonicity_dimensions} We would like to consider a second example where \cref{decreasing_dimensions} doesn't hold. Consider the following two layer network defined by:

for . Note that  is a full rank matrix, but \cref{decreasing_dimensions} doesn't hold. Depending on the sign of the components of  one has the following expression for :

where  are defined by \cref{eq:sets_examples_2}


The squared Lipschitz constant is given by  while the expected squared norm of the gradient of  is given by:

Again the set  becomes negligible as  approaches  which implies that  . On the other hand  converges to .  Note that unlike in the first example in \cref{eq:example_1}, the matrix  has a bounded condition number. In this example, the columns of   are all in the null space of , which implies  for all , even though all matrices have full rank.


\section{DiracGAN vector fields for more losses} \label{appendix:diracgan-full}
\begin{figure}[ht]
\centering
   \includegraphics[width=\linewidth]{figures/vector_fields_all.png}
    \caption{Vector fields for different losses with respect to the generator parameter  and the feature representation parameter ;
    the losses use a Gaussian kernel, and are shown in \eqref{eq:losses_dirac_gan}.
    Following \cite{Mescheder:2018}, ,  and .
    The curves show the result of taking simultaneous gradient steps in 
    beginning from three initial parameter values.}
    \label{fig:vector_fields_all}
\end{figure}
\Cref{fig:vector_fields_all} shows parameter vector fields, like those in \cref{fig:critic_vector_fields}, for \cref{example:diracgan}
for a variety of different losses:

The squared MMD between  and  under a Gaussian kernel of bandwidth  and is given by .
MMD-GP-unif uses a gradient penalty as in \cite{Binkowski:2018} where each samples from  is obtained by first sampling  and  from  and  and then sampling uniformly between  and .
MMD-GP uses the same gradient penalty, but the expectation is taken under  rather than .
SN-MMD refers to MMD with spectral normalization; here this means that .
Sobolev-MMD refers to the loss used in \cite{sobolev-gan} with the quadratic penalty only.
 is defined by \cref{eq:SobolevMMD}, with .


\section{Vector fields of Gradient-Constrained MMD and Sobolev GAN critics} \label{appendix:critic-vector-fields}
\citet{sobolev-gan} argue that \emph{the gradient of the critic (...) defines a transportation plan
for moving the distribution mass} (from generated to reference distribution) and present
the solution of Sobolev PDE for 2-dimensional Gaussians. We observed that in this simple example
the gradient of the Sobolev critic can be very high outside of the areas of high density,
which is not the case with the Gradient-Constrained MMD.
\Cref{fig:critic_vector_fields}
presents critic gradients in both cases,
using  for both.
\begin{figure}[ht]
\centering
    \begin{subfigure}[t]{.48\linewidth}
        \centering
        \includegraphics[width=\linewidth]{figures/critic_vector_field_gcmmd.pdf}
        \caption{Gradient-Constrained MMD critic gradient.}
    \end{subfigure}
    ~
    \begin{subfigure}[t]{.48\linewidth}
        \centering
        \includegraphics[width=\linewidth]{figures/critic_vector_field_sobolev.pdf}
        \caption{Sobolev IPM critic gradient.}
    \end{subfigure}
    \caption{Vector fields of critic gradients between two Gaussians. The grey arrows show
normalized gradients, i.e. gradient directions, while the black ones are the actual gradients.
Note that for the Sobolev critic, gradients norms are orders of magnitudes higher on the right hand side
of the plot than in the areas of high density of the given distributions.}
    \label{fig:critic_vector_fields}
\end{figure}

This unintuitive behavior is most likely
related to the vanishing boundary condition, assummed by Sobolev GAN. Solving the actual Sobolev PDE,
we found that the Sobolev critic has very high gradients close to the boundary in order to match
the condition; moreover, these gradients point in opposite directions to the target distribution.


\section{An estimator for Lipschitz MMD} \label{sec:est-lipmmd}
We now describe briefly how to estimate the Lipschitz MMD in low dimensions.
Recall that

For ,
it is the case that

Thus we can approximate the constraint

by enforcing the constraint on a set of  points 
reasonably densely covering the region around the supports of  and ,
rather than enforcing it at every point in .
An estimator of the Lipschitz MMD
based on  and  is

By the generalized representer theorem,
the optimal  for \eqref{eq:lipmmd-approx-problem} will be of the form

Writing ,
the objective function is linear in ,

The constraints are quadratic, built from the following matrices,
where the  and  samples are concatenated together,
as are the derivatives with each dimension of the  samples:

Given these matrices, and letting

where  is the th standard basis vector in ,
we have that

Thus the optimization problem \eqref{eq:lipmmd-approx-problem}
is a linear problem with convex quadratic constraints,
which can be solved by standard convex optimization software.
The approximation is reasonable only if we can effectively cover the region of interest with densely spaced ;
it requires a nontrivial amount of computation even for the very simple 1-dimensional toy problem of \cref{example:diracgan}.

One advantage of this estimator, though,
is that finding its derivative with respect to the input points or the kernel parameterization
is almost free once we have computed the estimate,
as long as our solver has computed the dual variables  corresponding to the constraints in \eqref{eq:lipmmd-approx-problem}.
We just need to exploit the envelope theorem and then differentiate the KKT conditions,
as done for instance in \cite{Amos2017}.
The differential of \eqref{eq:lipmmd-approx-problem} ends up being,
assuming the optimum of \eqref{eq:lipmmd-approx-problem} is at  and ,



\section{Near-equivalence of WGAN and linear-kernel MMD GANs} \label{appendix:wgan-linear-kernel}
For an MMD GAN-GP with kernel ,
we have that

and the corresponding critic function is

Thus if we assume ,
as that is the goal of our critic training,
we see that the MMD becomes identical to the WGAN loss,
and the gradient penalty is applied to the same function.

(MMD GANs, however, would typically train on the unbiased estimator of ,
giving a very slightly different loss function.
\cite{Binkowski:2018} also applied the gradient penalty to  rather than the true critic .)

The SMMD with a linear kernel is thus analogous to applying the scaling operator to a WGAN;
hence the name SWGAN.


\section{Additional experiments}

\subsection{Comparison of Gradient-Constrained MMD to Scaled MMD} \label{appendix:gcmmd-smmd}

\Cref{fig:isolines} shows the behavior of the MMD, the Gradient-Constrained SMMD, and the Scaled MMD when comparing Gaussian distributions.
We can see that  and the Gradient-Constrained MMD behave similarly in this case,
and that optimizing the  and the Gradient-Constrained MMD is also similar.
Optimizing the MMD would yield an essentially constant distance.

\begin{figure}[p]
  \centering
  \includegraphics[width=\linewidth]{figures/Isolines_2_log}
  \includegraphics[width=\linewidth]{figures/kernel_full_sobolev.png}
  \caption{Plots of various distances between one dimensional Gaussians,
    where ,
    and the colors show .
    All distances use .
    Top left: MMD with a Gaussian kernel of bandwidth .
    Top right: MMD with bandwidth .
    Middle left: Gradient-Constrained MMD with bandwidth .
    Middle right: Gradient-Constrained MMD with bandwidth .
    Bottom left: Optimized SSMD, allowing any .
    Bottom right: Optimized Gradient-Constrained MMD.
}
  \label{fig:isolines}
\end{figure}

\subsection{IGMs with Optimized Gradient-Constrained MMD loss} \label{appendix:sobolev-expt}

We implemented the estimator of \cref{prop:Finite_rank_approx}
using the empirical mean estimator of ,
and sharing samples for .
To handle the large but approximately low-rank matrix system,
we used an incomplete Cholesky decomposition \parencite[Algorithm 5.12]{shawe-taylor-christianini}
to obtain 
such that .
Then the Woodbury matrix identity allows an efficient evaluation:

Even though only a small  is required for a good approximation,
and the full matrices , , and  need never be constructed,
backpropagation through this procedure is slow
and not especially GPU-friendly; training on CPU was faster.
Thus we were only able to run the estimator on MNIST,
and even that took days to conduct the optimization on powerful workstations.

The learned models, however, were reasonable.
Using a DCGAN architecture,
batches of size 64,
and a procedure that otherwise agreed with the setup of \cref{sec:experiments},
samples with and without spectral normalization
are shown in \cref{fig:mnist:sobolev:nosn,fig:mnist:sobolev:sn}.
After the points in training shown, however,
the same rank collapse as discussed in \cref{sec:experiments} occurred.
Here it seems that spectral normalization may have delayed the collapse,
but not prevented it.
\Cref{fig:mnist:loss} shows generator loss estimates through training,
including the obvious peak at collapse;
\cref{fig:mnist:kid} shows KID scores based on the MNIST-trained convnet representation \citep{Binkowski:2018},
including comparable SMMD models for context.
The fact that SMMD models converged somewhat faster than Gradient-Constrained MMD models here
may be more related to properties of
the estimator of \cref{prop:Finite_rank_approx}
rather than the distances;
more work would be needed to fully compare the behavior of the two distances.

\begin{figure}
  \begin{subfigure}{.2\linewidth}
    \includegraphics[width=\linewidth]{samples/mnist_sobolevmmd_dcgan.png}
    \caption{Without spectral normalization;  generator iterations.}
    \label{fig:mnist:sobolev:nosn}
  \end{subfigure}
  \begin{subfigure}{.2\linewidth}
    \includegraphics[width=\linewidth]{samples/mnist_sn_sobolevmmd_dcgan.png}
    \caption{With spectral normalization;  generator iterations.}
    \label{fig:mnist:sobolev:sn}
  \end{subfigure}
  \begin{subfigure}{.28\linewidth}
    \includegraphics[width=\linewidth]{figures/mnist-sobolev-losses.pdf}
    \caption{Generator losses.}
    \label{fig:mnist:loss}
  \end{subfigure}
  \begin{subfigure}{.28\linewidth}
    \includegraphics[width=\linewidth]{figures/mnist-sobolev-KID.pdf}
    \caption{KID scores.}
    \label{fig:mnist:kid}
  \end{subfigure}
  \caption{The MNIST models with Optimized Gradient-Constrained MMD loss.}
\end{figure}



\subsection{Spectral normalization and Scaled MMD} \label{appendix:smmd_vs_sn}
\Cref{fig:singular_values-full} shows the distribution of critic weight singular values, like \cref{fig:celebA_scores_and_singular_values}, at more layers.
\Cref{fig:score_per_iter_cifar10_sn,tab:sn_cifar10_scores}
show results for the spectral normalization variants considered in the experiments.
MMDGAN, with neither spectral normalization nor a gradient penalty, did surprisingly well in this case, though it fails badly in other situations.

\cref{fig:singular_values-full} compares the decay of singular values for layer of the critic's network at both early and later stages of training in two cases: with or without the spectral parametrization. The model was trained on CelebA using SMMD.
\cref{fig:score_per_iter_cifar10_sn} shows the evolution per iteration of Inception score, FID and KID for Sobolev-GAN, MMDGAN and variants of MMDGAN and WGAN using spectral normalization. It is often the case that this parametrization alone is not enough to achieve good results.


\begin{figure}[ht]

        \centering
        \includegraphics[width=\linewidth]{figures/singular_values.pdf}
        \caption{Singular values at different layers, for the same setup as \cref{fig:celebA_scores_and_singular_values}.}
       \label{fig:singular_values-full}
\end{figure}

\begin{figure}[ht]
	        \centering
        \includegraphics[width=\linewidth]{figures/singular_values_celebA_3.pdf}
        \caption{Evolution of various quantities per generator iteration on CelebA during training. 4 models are considered: (SMMDGAN, SN-SMMDGAN, MMDGAN, SN-MMDGAN). (a) Loss:  for SMMDGAN and SN-SMMDGAN, and  for MMDGAN and SN-MMDGAN. The loss saturates for MMDGAN (green); spectral normalization allows some improvement in loss, but training is still unstable (orange). SMMDGAN and SN-SMMDGAN both lead to stable, fast training (blue and red). (b) SMMD controls the critic complexity well, as expected (blue and red); SN has little effect on the complexity (orange). (c) Ratio of the highest singular value to the smallest for the first layer of the critic network: . SMMD tends to increase the condition number of the weights during training (blue), while SN helps controlling it (red). (d) KID score during training: Only variants using SMMD lead to stable training in this case.}
       \label{fig:loss_celebA}
\end{figure}


\begin{figure}[ht]
        \centering
        \includegraphics[width=\linewidth]{figures/cifar10_score_vs_iter_worst_methods.pdf}

        \caption{Evolution per iteration of different scores for variants of methods, mostly using spectral normalization, on CIFAR-10.}
       \label{fig:score_per_iter_cifar10_sn}
\end{figure}



\begin{table}[ht]
    \centering
 \caption{Mean (standard deviation) of score evaluations on CIFAR-10 for different methods using Spectral Normalization.}
    \label{tab:sn_cifar10_scores}
  \begin{tabular}{llll}
\toprule
{} &                                    IS &                                    FID &                       KID \\
Method          &                                       &                                        &                                        \\
\midrule
MMDGAN          &             5.50.0 &             73.90.1 &             39.41.5 \\
SN-WGAN         &             2.20.0 &            208.50.2 &            178.91.5 \\
SN-WGAN-GP      &             2.50.0 &            154.30.2 &            125.30.9 \\
SN-Sobolev-GAN  &             2.90.0 &            140.20.2 &            130.01.9 \\
SN-MMDGAN-GP    &             4.60.1 &             96.80.4 &             59.51.4 \\
SN-MMDGAN-L2    &             7.10.1 &             31.90.2 &             21.70.9 \\
SN-MMDGAN       &             6.90.1 &             31.50.2 &             21.71.0 \\
SN-MMDGAN-GP-L2 &             6.90.2 &             32.30.3 &             20.91.1 \\
SN-SMMDGAN      &  \mathsmaller{\pm} &  \mathsmaller{\pm} &  \mathsmaller{\pm} \\
\bottomrule
\end{tabular}
 \end{table}



\subsection{Additional samples} \label{appendix:additional-samples}
\cref{fig:imagenet_additional,fig:celebA_samples_additional} give extra samples from the models.

\begin{figure}
        \centering
        \includegraphics[width=\linewidth]{samples/imagenet/img_98.jpg}
        \caption{Samples from a generator trained on ImageNet dataset using Scaled MMD with Spectral Normalization: SN-SMMDGAN.  }
       \label{fig:imagenet_additional}
\end{figure}

\begin{figure}[ht!]
    \centering
    \begin{subfigure}[t]{0.48\textwidth}
        \centering
        \includegraphics[width=\linewidth]{samples/celebA_sngan_resnet.jpg}
        \caption{SNGAN} \label{fig:celebA_sngan:samples}
    \end{subfigure}
    ~
    \begin{subfigure}[t]{0.48\textwidth}
        \centering
        \includegraphics[width=\linewidth]{samples/celebA_sobolev_gan_resnet_61K.jpg}
        \caption{SobolevGAN} \label{fig:celebA_sobolev_gan:samples}
    \end{subfigure}
   \vspace{0cm}
    \begin{subfigure}[t]{0.48\textwidth}
        \centering
        \includegraphics[width=\linewidth]{samples/celebA_mmd_gp_l2_resnet.jpg}
        \caption{MMDGAN-GP-L2} \label{fig:celebA_mmd_gp_l:samples}
    \end{subfigure}
    ~
    \begin{subfigure}[t]{0.48\textwidth}
        \centering
        \includegraphics[width=\linewidth]{samples/celebA_sn_smmd_resnset.jpg}
        \caption{SN-SMMD GAN} \label{fig:celebA_sn_smmd:samples}
    \end{subfigure}
     \vspace{0cm}
    \begin{subfigure}[t]{0.48\textwidth}
        \centering
        \includegraphics[width=\linewidth]{samples/celebA_sn_swgan_resnet.jpg}
        \caption{SN SWGAN} \label{fig:celebA_sn_swgan:samples}
    \end{subfigure}
    ~
    \begin{subfigure}[t]{0.48\textwidth}
        \centering
        \includegraphics[width=\linewidth]{samples/celebA_smmd_resnet.jpg}
        \caption{SMMD GAN} \label{fig:celebA_smmd:samples}
    \end{subfigure}
    \caption{Comparison of samples from different models trained on CelebA with  resolution.}
    \label{fig:celebA_samples_additional}
\end{figure}

\end{document}
