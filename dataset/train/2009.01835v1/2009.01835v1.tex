\documentclass[runningheads]{llncs}
\usepackage{amsmath,amssymb} 


\usepackage{graphicx}
\usepackage{subcaption}
\usepackage{float}
\usepackage{caption}	\usepackage{lscape}                                         


\usepackage[lined,ruled,linesnumbered]{algorithm2e}
\usepackage{xcolor}
\usepackage{tikz}
\usetikzlibrary{fit,calc}

\usepackage{booktabs}                   \usepackage{multirow}
\usepackage{paralist}
\usepackage{enumitem}
\usepackage{colortbl}

\usepackage{bm}                          \usepackage{epsfig}                      \usepackage{graphicx}                  \usepackage{mathtools}

\usepackage{color}

\usepackage{comment}

\usepackage{url}  \usepackage[pagebackref=false,breaklinks=true,colorlinks=true,filecolor=blue,urlcolor=blue,linkcolor=red,bookmarks=false]{hyperref}
\usepackage[nocompress]{cite}

\usepackage{listings}

\usepackage{xspace}
\usepackage{setspace}

\usepackage{nicefrac}
\usepackage{microtype}
\usepackage[utf8]{inputenc} \usepackage[T1]{fontenc}    

 

\renewcommand{\baselinestretch}{1}       



\def\etal{et~al.}			  \def\eg{e.g.,~}               \def\ie{i.e.,~}               \def\etc{etc}                 \def\cf{cf.~}                 \def\viz{viz.~}               \def\vs{vs.~}                 

\setlength{\fboxsep}{0mm}
\DeclareMathAlphabet{\altmathcal}{OMS}{cmsy}{m}{n}


\newcommand*{\tikzmk}[1]{\tikz[remember picture,overlay,] \node (#1) {};\ignorespaces}
\newcommand{\boxit}[1]{\tikz[remember picture,overlay]{\node[yshift=3pt,fill=#1,opacity=.25,fit={(A)()}] {};}\ignorespaces}

\def\p{\hspace{-0.2mm}(\hspace{-0.2mm}p\hspace{-0.2mm})}
\def\fn{\altmathcal{F}}
\def\ij{i \rightarrow j}
\def\ji{j \rightarrow i}
\def\Hji{\altmathcal{H}_{\!\ji}}
\def\Hij{\bm{H}_{\ij}}

\def\Ii{ }
\def\Ij{ }
\def\Mi{ }
\def\Mj{ }
\def\pixel{ }

\def\flowcorr{ }
\def\edgecorr{ }
\def\flow{ }
\def\edgepred{ }
\def\homography{ }
\def\flowgray{}

\def\neighborF{ }
\def\neighborB{ }
\def\neighborNLfirst{ }
\def\neighborNLsecond{ }
\def\neighborNLthird{ }

\DeclareMathOperator*{\argmin}{\arg\!\min} 
\DeclareMathOperator*{\argmax}{\arg\!\max}

\newlength\paramargin
\newlength\figmargin
\newlength\secmargin
\newlength\figcapmargin

\setlength{\secmargin}{0.0mm}
\setlength{\paramargin}{0.0mm}
\setlength{\figmargin}{-3.0mm}
\setlength{\figcapmargin}{0.0mm}

\newcommand{\red}{\textcolor{red}}
\newcommand{\blue}{\textcolor{blue}}

\newcommand {\first}[1]{{\color{red}\textbf{#1}}}
\newcommand {\second}[1]{{\color{blue}\underline{#1}}}

\newcommand{\mpage}[2]
{
\begin{minipage}{#1\linewidth}\centering
#2
\end{minipage}
}

\newcommand{\mfigure}[2]
{
\begin{subfigure}{#1\linewidth}\centering
\includegraphics[width=\linewidth]{#2}
\end{subfigure}
}



\newcommand{\heading}[1]
{
\vspace{1mm}\noindent\textbf{#1}
}


\newcommand{\secref}[1]{Section~\ref{sec:#1}}
\newcommand{\figref}[1]{Figure~\ref{fig:#1}} 
\newcommand{\tabref}[1]{Table~\ref{tab:#1}}
\newcommand{\eqnref}[1]{\eqref{eq:#1}}
\newcommand{\thmref}[1]{Theorem~\ref{#1}}
\newcommand{\prgref}[1]{Program~\ref{#1}}
\newcommand{\algref}[1]{Algorithm~\ref{#1}}
\newcommand{\clmref}[1]{Claim~\ref{#1}}
\newcommand{\lemref}[1]{Lemma~\ref{#1}}
\newcommand{\ptyref}[1]{Property~\ref{#1}}

\long\def\ignorethis#1{}

\newcommand {\jiabin}[1]{{\color{magenta}\textbf{Jia-Bin: }#1}\normalfont}
\newcommand {\chen}[1]{{\color{green}\textbf{Chen: }#1}\normalfont}
\newcommand {\ayush}[1]{{\color{red}\textbf{Ayush: }#1}\normalfont}
\newcommand {\johannes}[1]{{\color{red}\textbf{J}\color{cyan}\textbf{o}\color{blue}\textbf{h}\color{magenta}\textbf{a}\color{red}\textbf{n}\color{cyan}\textbf{n}\color{blue}\textbf{e}\color{magenta}\textbf{s}\color{red}: #1}\normalfont\xspace}
\newcommand {\todo}{{\textbf{\color{red}[TO-DO]\_}}}
\def\newtext#1{\textcolor{blue}{#1}}
\def\modtext#1{\textcolor{red}{#1}}


\newcommand{\tb}[1]{\textbf{#1}}
\newcommand{\mb}[1]{\mathbf{#1}}

\newcommand{\jbox}[2]{
  \fbox{\begin{minipage}{#1}\hfill\vspace{#2}\end{minipage}}}

\newcommand{\jblock}[2]{\begin{minipage}[t]{#1}\vspace{0cm}\centering #2\end{minipage}}

\makeatletter
\newenvironment{jalgo}[1][htbp]
  {\renewcommand{\@algocf@start}{\@algoskip \begin{lrbox}{\algocf@algobox}\begin{minipage}{\dimexpr\columnwidth\relax}
  \setlength{\algowidth}{\hsize}\vbox\bgroup \hbox to\algowidth\bgroup\hbox to \algomargin{\hfill}\vtop\bgroup \ifthenelse{\boolean{algocf@slide}}{\parskip 0.5ex\color{black}}{}\let\@mathsemicolon=\;\def\;{\ifmmode\@mathsemicolon\else\@endalgoln\fi}\raggedright\AlFnt{}\ifthenelse{\boolean{algocf@slide}}{\IncMargin{\skipalgocfslide}}{}\@algoinsideskip }\renewcommand{\@algocf@finish}{\@algoinsideskip \egroup \hfill\egroup \ifthenelse{\boolean{algocf@slide}}{\DecMargin{\skipalgocfslide}}{}\egroup \end{minipage}
  \end{lrbox}\makebox[\linewidth][c]{\algocf@makethealgo}\@algoskip \setlength{\hsize}{\algowidth}\lineskip\normallineskip\setlength{\skiptotal}{\@defaultskiptotal}\let\;=\@mathsemicolon \let=\@emathdisplay }\begin{algorithm}[#1]}
  {\end{algorithm}}
\makeatother


 

\def\xi{\mathbf{x}_i}
\def\yi{\mathbf{y}_i}
 \graphicspath{{figure}, {example}}
 
\usepackage{ruler}
\usepackage[width=122mm,left=12mm,paperwidth=146mm,height=193mm,top=12mm,paperheight=217mm]{geometry}
\setcounter{secnumdepth}{1}
\setcounter{secnumdepth}{2}
\setcounter{secnumdepth}{3}

\begin{document}

\pagestyle{headings}
\mainmatter
\def\ECCVSubNumber{1715}  

\title{Flow-edge Guided Video Completion}

\titlerunning{ECCV-20 submission ID \ECCVSubNumber} 
\authorrunning{ECCV-20 submission ID \ECCVSubNumber} 
\author{Anonymous ECCV submission}
\institute{Paper ID \ECCVSubNumber}
\newcommand{\review}[2]{\vspace{-3.0mm}\paragraph{\color{blue}{\bf#1}}{\hskip -0.75em} : {{\color{blue}\emph{#2}}}}

\def\httilde{\mbox{\tt\raisebox{-.5ex}{\symbol{126}}}}






R1:

**1. Technical novelty.**
Our work identifies main issues in flow-based completion formulation and presents practical solutions (edge-guided flow completion, non-local neighbors, and gradient domain processing) to produce high-quality results.
The detailed ablation study validates the contributions of individual design choices. 




**2. Related work on flow completion.**
We discussed optimization based method [13] (L147-149) and CNN based method [40] (L165-167).
We will make the discussions more explicit. 




**3. Why flow completion network failed**

We experimented with several inpainting architectures for flow completion (L616-622) and found the results often contain visible seams along the hole and blurry motion boundaries.
The DFC-Net [40] also produced visible seams (Fig. 8) and did not preserve sharp motion boundary (Fig. 4 in the supp material). 
Our edge-guided interpolation explicitly addresses these issues by producing piecewise-smooth flow.



**4. Results with and without using the completed flow edge maps**


We did report the comparison in Table 2(b) (first row). 
*Diffusion* refers to not using the completed flow edge maps in flow completion.




**5. Adaptive non-local frame selection**
The simple fixed frame selection works well as DAVIS sequences are generally short (~ 80 frames).
We agree that an adaptive scheme may work better, particularly for longer sequences.
We will add this experiment in the revision.



**6. Seamless blending.**

Gradient domain processing can be applied as post-processing but cannot remove seams within the hole (L175-180).
The method in [10] is the first to apply seamless blending for video completion. 
However, it assumes static background and access to segmentation masks for dynamic foreground objects.
We will clarify.








**7. Memory efficiency.**

Our flow edge completion network operates *frame by frame* and thus supports higher resolution inputs (compared to methods that process an entire video [5, 36] or short clips [18, 20, 25, 40] at a time).









**8. Canny edge.**

Yes, we remove the edges of missing regions using the masks.



**9. Accuracy of the optical flow in interpolated areas?**

Our method produces plausible, piecewise-smooth flow fields in the missing regions.
We then use the inpainted flow maps to propagate contents to ensure that the completed video agrees with the inpainted flow.




**10. Experiments without using Gradient and Non-local**

Per request, we show the results without using 'Gradient' and 'non-local' below for the stationary masks setting:
||PSNR|SSIM|LPIPS |
|:-:|:-:|:-:|:-:|
|Stationary masks|28.28|0.9451|0.067|
|Object masks|39.29|0.9893|0.009|















**11. Figure 9**

We included 3 additional results in Fig. 3 of the supp. material.



**12. Writing**

We thank R1 for the detailed suggestion.
We will revise accordingly.




R2:

**1. Analysis on the effect of the flow quality**

We conducted detailed ablation study and validate the effectiveness of the proposed gradient domain processing, non-local temporal neighbors and edge-guided flow completion (L535-584). 

Table 2(b), Figure 8 and Figure 4 in the supplementary material shows the comparison between different flow completion methods.
Table 2(a), Figure 6 and Figure 2 in the supplementary material shows the effectiveness of the proposed gradient domain processing to generate seamless results.
Table 2(a), Figure 9 and Figure 3 in the supplementary material qualitatively shows the effectiveness of the proposed non-local temporal neighbor to transfers the correct contents from temporally distant frames.




**2. How to validate the quality of the completed flow edges?**

We don't evaluate the quality of the completed flow edge.
Instead, we directly evaluate the completed flow. 
We calculate the EPE between our completed flow and GT flow in Table 2(a).








R3:

We thank the reviewers for their thoughtful feedback. We are encouraged they found our idea to be reasonable (R1), sound(R2), and novel (R3). We are glad they found our representation to be clear and organized (R3), experiments to be comprehensive (R1), and paper to be detailed enough to reproduce (R1). We are pleased all reviewers recognizes the importance of piecewise smooth flow with sharp boundary, and that R2 recognizes the superiority of utilizing non-local neighbors for finding better explainable pixels. We address each reviewer's comments individually below and will incorporate all feedback.


**1. Clarification on the fusion method**

We first fill the color gradients of the missing regions by weighted-summing the gradient from local and non-local temporal neighbors.
For each frame, we then use the filled gradients, along with the boundary constraints along the hole to solve the missing colors using Poisson reconstruction.
With this process, indeed we cannot adopt new colors as is from other frames. 
Instead, we transfer the ``contents" (in the form of color gradients) from other frames while ensuring seamless blending along the hole boundaries.




\end{document}