\documentclass{article}

\usepackage{amsthm,amsmath,amssymb,sidecap,subfig,graphicx,hyperref}
\usepackage[left=2.5cm,top=3cm,right=3cm]{geometry}
\usepackage[lined,boxed,commentsnumbered]{algorithm2e}
{
\theoremstyle{definition}
\newtheorem{examp}{Example}
\newtheorem*{defi}{Definition}
}
\newtheorem{lem}{Lemma}
\newtheorem{theo}{Theorem} 
\newtheorem{coro}{Corollary}
\newtheorem{definition}{Definition}
\newtheorem{facts}{Facts}
\newtheorem{prop}{Proposition}
\newtheorem{remk}{Remark}
\newcommand\meetirrdiff[2]{\mathfrak{m}(#1,#2)}
\newcommand\downvertices[1]{Anc(#1)}
\newcommand\downn[2]{e_{#1,#2}}
\newcommand\downnn[2]{{#1}_{#2}}
\newcommand\down[1]{e_{#1}}

\newcommand{\keywords}[1]{\par\noindent{\small{\em Keywords\/}: #1}}
\newcommand\supportgraph[1]{#1}
\newcommand\predecessor[1]{prec(#1)}
\newcommand\successor[1]{{#1}^{+}}
\newcommand\shotvector[2]{sh_{#1}(#2)}
\newcommand\set[1]{\{ #1 \}}
\newcommand\oneshotvector[1]{sh_{#1}}
\newcommand\lop[1]{Br_G (#1,#1)}
\newcommand\comple[2]{\overline{#2}_{#1}}
\newcommand\coverrelations[1]{\mathfrak{C} (#1)}
\newcommand\labelmap[1]{\mathfrak{m}(#1)}
\newcommand\cfgequa[2]{\mathcal{E}_{#1,#2}}
\newcommand\nedges[2]{E(#1,#2)}
\newcommand\maxoperator[2]{max \{ #1,#2 \}}
\newcommand\joinlabel[2]{J_{#1}(#2)}
\newenvironment{proof2}{\hspace{4.ex}}{\hfill }
\def\firedf{\mathfrak{f}}
\def\restrictedf{\mathfrak{r}}
\def\claim{\textit{{Claim: }}}
\def\question{\emph{\textbf{Question. }}}
\def\observation{\textbf{\textit{Observation. }}}
\def\firstname{\emph{D-graph}}
\def\Proof{\emph{\textbf{Proof. } }}
\def\example{\emph{\textbf{Example. }}}
\def\secondname{\emph{\textbf{CFGLE}}}
\def\thirdname{\emph{\textbf{compatible }}}
\def\fourthname{\emph{\textbf{labeling}}}
\def\fifthname{\emph{\textbf{ex-ideal}}}
\def\sixthname{\emph{\textbf{ex-ideal}}} \def\uppersets{\emph{\textbf{}}}
\def\problem{\emph{\textbf{Problem }}}

\graphicspath{{converted_graphics/}{D:/PhD/CFGLattices/Images/}}

\begin{document}
\title{Lattices generated by Chip Firing Game models: criteria and recognition algorithm\thanks{This paper was partially sponsored by the Vietnamese National Foundation for Science and Technology Development (NAFOSTED)}} 

\author{Trung Van Pham and Thi Ha Duong Phan}
\maketitle
\begin{abstract}
It is well-known that the class of lattices generated by Chip Firing games (CFGs) is strictly included in the class of upper locally distributive lattices (ULD). However a necessary and sufficient criterion for this class is still an open question. In this paper we settle this problem by giving such a criterion.  This criterion provides a polynomial-time algorithm for constructing a CFG which generates a given lattice if such a CFG exists. Going further we solve the same problem on two other classes of lattices which are generated by CFGs on the classes of undirected graphs and directed acyclic graphs.\\
\hspace{6.ex}\textit{\textbf{Keywords.}} Abelian Sandpile model, Chip Firing Game, discrete dynamic model, lattice, Sandpile model, ULD lattice, linear programming
\end{abstract}
\section{Introduction}
The Chip Firing Game (CFG) is a discrete dynamical model which was first defined by A. Bj\"orner, L. Lov\'asz and W. Shor while studying the `balancing game' \cite{BL92,BLS91,BTW87,S86}. The model has various applications in many fields of science such as physics \cite{DRSV95,BTW87}, computer science \cite{BL92,BLS91,GMP98}, social science \cite{B97,B99} and mathematics \cite{B99,M97,M01}.

The model is a game which consists of a directed multi-graph  (also called \emph{support graph}), the set of \emph{configurations} on  and an \emph{evolution rule} on this set of configurations. Here, a configuration  on  is a map from the set  of vertices of  to non-negative integers. For each vertex  the integer  is regarded as the number of chips stored in . In a configuration , vertex  is \emph{firable} if  has at least one outgoing edge and  is at least the out-degree of . The \emph{evolution rule} is defined as follows. When  is firable in ,  can be transformed into another configuration  by moving one chip stored in  along each outgoing edge of . 
\begin{SCfigure} \centering
\includegraphics[bb=9 13 466 316,width=2.67in,height=1.77in,keepaspectratio]{image34}
  \caption{The number at each vertex indicates the number of chips stored there. The configuration at the bottom of the figure can be transformed into two new configurations since it has two firable vertices}
  \label{fig:image34}
\end{SCfigure}
We call this process \emph{firing }, and write . An \emph{execution} is a sequence of firing and is often written in the form . The set of configurations which can be obtained from  by a sequence of firing is called \emph{configuration space}, and denoted by . 

A CFG begins with an initial configuration . It can be played forever or reaches a unique fixed point where no firing is possible \cite{Eri93}. When the game reaches the unique fixed point,  is an \emph{upper locally distributive lattice} with the order defined by setting  if  can be transformed into  by a (possibly empty) sequence of firing  \cite{LP01}. A CFG is \emph{simple} if each vertex is fired at most once during any of its executions. Two CFGs are \emph{equivalent} if their generated lattices are isomorphic. Let  denote the class of lattices generated by CFGs. A well-known result is that   \cite{MVP01}, where  and  denote the classes of distributive lattices and upper locally distributive lattices, respectively. Despite of the results on inclusion, one knows little about the structure of , even an algorithm for determining whether a given ULD lattice is in  is unknown so far.

The Chip Firing Game has some important restrictions. An important restriction is the Abelian Sandpile model (ASM), a restriction of CFGs on undirected graphs  \cite{Mag03,BTW87,BLS91}\footnote{The term ``Abelian Sandpile model'' has been introduced in Dhar's earlier work in which the author focused on studying the algebraic property of the recurrent configurations of the model \cite{Dha90}. The definition for the Abelian Sandpile model we present here follows from the work of Magnien in which the author used the term again with a different definition \cite{Mag03}.}. This model has been extensively studied in recent years. In \cite{Mag03} the author studied the class of lattices generated by ASMs, denoted by , and showed that this class of lattices is strictly included in  and strictly includes the class of distributive lattices. As , the structure of  is little known. An algorithm for determining whether a given ULD lattice is in  is still open.

The goal of our study is to find conditions that completely characterize those classes of lattices. One of the most important discoveries in our study is pointing out a strong connection  between the objects which does not seem to be closely related. These objects are meet-irreducibles,  simple CFGs, firing vertices of a CFG, and systems of linear inequalities. In particular, we establish a one-to-one correspondence between the firing vertices of a simple CFG and the meet-irreducibles of the lattice generated by this CFG. Using this correspondence we achieve a necessary and sufficient condition for . By generalizing this correspondence to CFGs that are not necessarily simple, we also obtain a necessary and sufficient condition for . Both conditions provide polynomial-time algorithms that address the above computational problems. As an application of these conditions, we present in this paper a lattice in  that is smaller than the one shown in \cite{Mag03}.



In \cite{Mag03}, to prove  the author studied simple CFGs on directed acyclic graphs (DAGs) and showed that such a CFG is equivalent to a CFG on an undirected graph. It is natural to study CFGs on DAGs which are not necessarily simple. Again our method is applicable to this model and we show that any CFG on a DAG is equivalent to a simple CFG on a DAG. As a corollary, the class of lattices generated by CFGs on DAGs is strictly included in . We also give a necessary and sufficient condition for the class of lattices generated by this model.

Section \ref{second section} gives some preliminary definitions, notations and results on lattice and Chip Firing games. In Sections \ref{third section}, \ref{fourth section} and \ref{fifth section} we study the properties of three classes of lattices generated by CFGs on general graphs, undirected graphs and directed acyclic graphs, respectively. These sections are devoted to necessary and sufficient criteria for determining which class of lattices a given ULD lattice belongs to. In the conclusion we give some open problems which are currently in our interests.
\section{Preliminary definitions and previous results}
\label{second section}
\subsection{\small \textit{Notations and definitions}}
\hspace{1.ex}Let  be a finite partial order ( is equipped with a binary relation  which is transitive, reflexive and antisymmetric). A subset  of  is called an \emph{ideal} of  if for every  and  such that  we have . For ,  is an \emph{upper cover} of  if  and for every ,  implies that  or . If  is an upper cover of  then  is a \emph{lower cover} of y, and then we write . A finite partial order is often presented by a Hasse diagram in which for each cover  of , there is a curve that goes upward from  to . The lattice  is a \emph{lattice} if any two elements of  have a least upper bound (\emph{join}) and a greatest lower bound (\emph{meet}). When  is lattice, we have the following notations and denitions
\begin{itemize}
  \item  denote the minimum and the maximum of . 
  \item for every ,  and  denote the join and the meet of , respectively. 
  \item for ,  is a \emph{meet-irreducible} if it has exactly one upper cover. The element  is a \emph{join-irreducible} if  has exactly one lower cover. Let  and  denote the collections of the meet-irreducibles and the join-irreducibles of , respectively. Let  be given by  and . For , if  is a minimal element in  then we write . If  is a maximal element in  then we write , and  if  and . 
  \item The lattice  is a \emph{distributive lattice} if it satisfies one of the following equivalent conditions
       \begin{itemize}
         \item[1. ] for every , we have . 
         \item[2. ] for every , we have .
       \end{itemize}  
For a finite set ,  is a distributive lattice. A lattice generated in this way is called \emph{hypercube}. 
 \item for  satisfying ,  stands for set . If ,  denotes the join of all upper covers of . Note that if  is a meet-irreducible then  is the unique upper cover of . If ,  denotes the meet of all lower covers of . If  is a join-irreducible then  is the unique lower cover of . The lattice  is an \emph{upper locally distributive (ULD) lattice} \cite{M90,D40} if for every ,  implies the sublattice induced by  is a hypercube. By dual notion,  is a \emph{lower locally distributive (LLD) lattice} if for every ,  implies that the sublattice induced by  is a hypercube.
\end{itemize}  

Let  be a directed multi-graph. For ,  denotes the number of edges from  to . It follows that  is the number of loops at . For , the out-degree of , denoted by , is defined by  and the in-degree of , denoted by , is defined by . A vertex  of  is called \emph{sink} if it has no outgoing edge, \emph{i.e.} . A subset  of  is a \emph{closed component} if ,  is a strongly connected component and there is no edge going from  to a vertex outside of . A CFG, which is defined on a graph having no closed component, always reaches a unique fixed point, moreover its configuration space is a ULD lattice  \cite{BL92,LP01}. If  has a unique fixed point and  is isomorphic to a ULD lattice , we say  generates . Then we can identify the configurations of  with the elements of  (by an isomorphism).

\textbf{Remark. }Throughout this paper when  generates , the configurations in  are automatically identified with the elements of . All later arguments use this assumption.
\subsection{\small \textit{Previous results}}
\begin{theo}[Birkhoff \cite{B33}]
A lattice is distributive if and only if it is isomorphic to the lattice of the ideals of the order induced by its meet-irreducibles.
\end{theo}
\begin{lem}[Caspard \cite{C98}]
\label{condition on cover relation for ULD lattice}
A lattice  is upper locally distributive if and only if for any ,

\end{lem}
\begin{lem}[Latapy and Phan \cite{LP01}]
In a CFG reaching a unique fixed point, if two sequences of firing are starting at the same configuration and leading to the same configuration then for every , the number of times v fired in each sequences are the same, where  is the support on which the game is  defined. 
\end{lem}
In a  having a unique fixed point, for each  being a configuration in , the \emph{shotvector} of , denoted by , assigns each vertex  of  to the number of times  fired in any execution from the  configuration   to . Thus  is a map from  to . It follows from the above lemma that the shotvector of  is well-defined. For  we write  if for every , . It is known that  if and only if  can be transformed into  by a sequence of firing \cite{LP01}. 

Throughout the coming sections, we always work with a general finite ULD lattice . Recall that  denote the collections of the meet-irreducibles and the join-irreducibles of , respectively. The map  is given by  is the element in . All graphs are supposed to be directed multi-graphs. In a CFG if configuration  can be transformed into  by firing some vertex in the support graph then we denote this unique vertex by . All CFGs, which are considered in this paper, are assumed to be reaching a fixed point. To denote a CFG, a configuration space and a lattice generated by a CFG, we will use the common notation  since all of them are completely defined by  and .
\section{A necessary and sufficient condition for }
\label{third section}
Given a ULD lattice , is  in ? This question was asked in \cite{MVP01}. Up to now, there exists no good criterion for   that suggests a polynomial-time algorithm for this computational problem. In this section we address this problem by giving a necessary and sufficient condition for . We recall an important result in \cite{MVP01}
\begin{theo}[Magnien, Vuillon and Phan \cite{MVP01}]
\label{theorem of simple CFG}
Any CFG that reaches a unique fixed point is equivalent to a simple CFG
\end{theo}
From now until the end of this section, all CFGs are supposed to be simple. The following lemma is known in \cite{FK09}. Since it will play an important role in this paper and its proof is simple, it is presented here with a proof.
\begin{lem}[Felsner and Knauer \cite{FK09}]
\label{lemma of the square connection}
Let  be two elements of  such that . Let  denote . Then for any chain  in , there exists a chain  in  such that  for every . Moreover,   for every . 
\end{lem}
\begin{proof}
It's clear that . Since  is a ULD lattice, there exists a unique  such that  and . It follows easily that . If  then . Otherwise  repeat the previous argument starting with  until the index reaches . We obtain the sequence  which has the desired property.
\end{proof}
\begin{figure}[!h]
\centering
\includegraphics[bb=112 17 268 262,width=1.19in,height=1.88in,keepaspectratio]{image9}
\end{figure}
\begin{lem}
\label{lemma of relation}
Let  be a ULD lattice generated by  and let  denote the set of vertices which are fired in . For each ,  denotes the set of vertices which are fired to obtain .   Then
\begin{itemize}
  \item[1. ]  The map  determined by , where  are two elements in  such that  and , is well-defined. Furthermore  is a bijection.
  \item[2. ] For every , .
\end{itemize}  
\end{lem}
\begin{proof}\\
\begin{itemize}
\item[1. ] The map  is defined on whole  since for every , . To prove  is well-defined, it suffices to show that for each , if  then .  Let . By Lemma \ref{lemma of the square connection}, there exists  such that for every , we have . Therefore .

\hspace{3.ex} Clearly  is surjective. To prove  is bijective, it suffices to show that . Let  be an execution to obtain the fixed point. Since  and for every , , it follows that , therefore 
\item[2. ] Let  be an execution to obtain the fixed point. It's clear that , therefore  \linebreak . By the definition of , we have  and . It follows that  since  is bijective. 
\end{itemize}
\end{proof}

The lemma means that if  is generated by a CFG then each meet irreducible of  can be considered as a vertex of its support graph. It is an important point to set up a criterion for . For better understanding, we give an example for this correspondence. The CFG which is defined on the support graph and the initial configuration shown in Figure \ref{fig:image38} and Figure \ref{fig:image39}
\begin{figure}
\centering
\subfloat[Support graph]{\label{fig:image38}\includegraphics[bb=1 4 134 140,width=1.19in,height=1.22in,keepaspectratio]{image44}}
\quad
\subfloat[Initial configuration]{\label{fig:image39}\includegraphics[bb=9 11 189 199,width=1.27in,height=1.33in,keepaspectratio]{image39}}
\caption{An example of Chip firing game}
\end{figure}
\noindent generates the lattice represented in Figure \ref{fig:image16}.\begin{figure}
\centering
\subfloat[Cover relation labeled by firing vertices]{\label{fig:image16}\includegraphics[bb=13 7 214 268,width=1.45in,height=1.88in,keepaspectratio]{image16}}
\quad
\subfloat[Configurations labeled by fired vertices]{\label{fig:image19}\includegraphics[bb=4 0 249 292,width=1.58in,height=1.88in,keepaspectratio]{image19}}
\caption{firing-vertex labeling}
\end{figure}
\begin{figure}
\centering
\subfloat[Cover relation labeled by meet-irreducibles]{\label{fig:image17}\includegraphics[bb=13 11 222 270,width=1.52in,height=1.88in,keepaspectratio]{image17}}
\quad
\subfloat[Configurations labeled by meet-irreducibles]{\label{fig:image18}\includegraphics[bb=4 0 232 271,width=1.58in,height=1.88in,keepaspectratio]{image18}}
\caption{meet-irreducible labeling}
\end{figure}
\noindent In Figure \ref{fig:image16}, each  is labeled by the vertex which is fired in  to obtain . The lattice in Figure \ref{fig:image17} is the same as one in Figure \ref{fig:image16} but each  is labeled by . Figure \ref{fig:image19} shows the lattice in the way each configuration is presented by the set of vertices which are fired to obtain this configuration. In Figure \ref{fig:image18}, each configuration  is presented by .  Clearly the labelings in Figure \ref{fig:image16} and Figure \ref{fig:image17} are the same, the presentations in Figure \ref{fig:image19} and Figure \ref{fig:image18} are the same too with respect to the correspondence  defined by .

For each ,  denotes the collection of all minimal elements of  and  denotes the collection of all maximal elements of . 

Let us explain in a few words why we define these notations. Suppose that  is generated by . For a vertex  fired in the game, we consider all configurations in  which have enough chips stored at  in order that  can be fired. If we only care about the firability of , we only need to consider the collection  of all  minimal configurations of these configurations. The configurations, which are not greater than equal to any configuration in , do not have enough chips stored at  in order that  can be fired. We only need to consider the collection  of all maximal configurations of these configurations to know the firability of . Sets  are exactly , , respectively in . Note that  only depend on , not depend on the CFGs generating , even if there exists no such CFGs. 

For each  the system of linear inequalities  is given by

where  is an added variable. The collection of all variables of  is . It follows from the definitions of  and  that if  is in the collection of all variables of  then . Note that  if and only if there exists  such that  and .

\textbf{Remark.} When  is generated by some CFG, Lemma \ref{lemma of relation} means that each  can be regarded as a vertex of this CFG. The system of linear inequalities  describes the firability of  in the following meaning. In order that  can be fired,  receives at least  chips from its neighbors. Each  in  indicates the number of chips that  sends to  when it is fired. For each  when all vertices in  are fired, the game arrives at the configuration , and  receives  chips from its neighbors. The vertex  is not firable in each , therefore . Similarly  is firable in each , therefore . 
\begin{examp}
\label{first example}
We consider again the lattice presented in Figure \ref{fig:image17}.  We have  . Then 


\end{examp}

\begin{lem}
\label{lemma of solution}
If  then for every ,   has non-negative integral solutions. 
\end{lem}
\begin{proof}

Let  be a CFG that generates . If  then clearly it has a non-negative integral solution. Otherwise, let  be given by
 
where  is the map which is defined as in Lemma \ref{lemma of relation}. Note that since ,  cannot be fired at the beginning of the game, therefore .

We show that  is a solution of . Indeed let . By Lemma \ref{lemma of relation} the set of vertices which are fired to obtain  is . After firing all vertices in   receives  chips from its neighbors. Since  is firable in , it follows that . It remains to prove that for , we have . It follows from the definition of  and from Lemma \ref{lemma of relation} that  is not firable in . By a similar argument we have .
\end{proof}
\begin{theo}
\label{the condition of lattices induced by CFG}
 is in  if and only if for each ,  has non-negative integral solutions.
\end{theo}
\begin{proof}
 has been proved by Lemma \ref{lemma of solution}. It remains to show that  is also true. We are going to construct a graph  and an initial configuration  so that the game is simple and  is isomorphic to .

The set of vertices of  is , where  is distinct from  and will play a role as the sink of . The edges of  are constructed as follows. For each  let  be a solution of , where  is the collection of all variables in . Set  and for each  satisfying , and .

The constructing graph  has the following properties. The graph  is connected and has no closed component since each vertex  has at least one edge going from  to , and  has no outgoing edge. Thus any CFG on  reaches a fixed point. For each  we have  and . Note that in the formula of  if  then we set . The in-degree and the out-degree at each  vertex of  depend on the non-negative integral solutions  we choose, therefore they may be large. In fact the number of vertices of  is small, that is , whereas the number of edges of  is often very large. However this is not a problem of presenting  since a multi-graph is often represented by associating each pair  of vertices of  with a number that indicates the number of edges from  to .

We construct  as follows


We claim that  is simple. Indeed for the sake of contradiction we suppose that there exists at least one vertex in  which is fired more than once in an execution, say , to reach the fixed point  of . By the assumption,  are not pairwise distinct. Let  be the largest index such that  are pairwise distinct. Vertex , therefore, is in . We have  since  is fired more than once during the execution. To obtain  each vertex in  is fired exactly once, therefore . Since  is firable in , it follows that . It contradicts the fact that .

We claim that for every execution  of , there exists a chain  in  such that  for every . Note that if the chain exists then it is defined uniquely. We prove the claim by induction on . For ,  is firable in . It follows from the construction of  and  that only the vertices in  having indegree 0 are firable in , therefore . It implies that there exists  such that  and . The claim holds for . For , let  be the chain in  such that  for every . If  is not less than or equal to any element in  then there exists  such that . It follows from the definition of  that . It implies that after  have been fired,  receives less than  chips from its neighbors, therefore  is not firable in . It's a contradiction. If there exists  such that  then there exists  such that  and . Let . It suffices to show that . Indeed, we have . Since , , therefore .

Our next claim is that for any chain  in , there exists an execution  in , where  for every . We prove the claim by induction on . For , we have . It follows easily that , therefore  is firable in . By firing  in , we obtain . The claim holds for . For , let  be an execution in the game such that  for every .
Since  is in the set  and  is the collection of all minimal elements in this set,  there is  such that . Thus . The vertex  receives at least  chips from its neighbors after all vertices  have been fired. The vertex  is firable in  since . The claim follows.

It follows immediately from the above claims that  and  are isomorphic. This completes the proof.
\end{proof}
We establish a relation between  and the join-irreducibles of . The main result of \cite{MVP01} will follow easily from this relation.
\begin{prop}
\label{relation with join-irreducibles}
For each meet-irreducible  of , 
\end{prop}
\begin{proof}
For each , let  be given by .  Let  denote . First, we show that . To this end, we prove that  for every  satisfying .  Since  and , we have  and , therefore . Since , it follows that , hence . It remains to prove that   is a minimal element of . For a contradiction, we suppose that there exists  in  such that  and . It follows easily that , therefore there is a chain  in  of length . We have  since , therefore . It contradicts the fact that  is a join-irreducible. 

We are left with showing that . Let . There is a unique element  in  such that  and . It suffices to show that . For a contradiction, we suppose that . Then there exists  such that  and . Let  denote the infimum of  and . There exists  such that  and . Since , we have , therefore . It follows from  that , hence . It contradicts the fact that  is the infimum of  and .
\end{proof}
\begin{coro}
\label{uniqueness of join-irreducibles}
If  is a distributive lattice then for every meet-irreducible  of , we have . 
\end{coro}
\begin{proof}
For each , we define . Note that . For every ,  implies that   since if  then . In a distributive lattice, the cardinality of the meet-irreducibles is equal to the cardinality of the join-irreducibles, \emph{i.e} . It follows easily that for every , , therefore  by Proposition \ref{relation with join-irreducibles}.
\end{proof}
\begin{prop}
\label{lemma of the condition on distributive lattice}
If  is a distributive lattice then for each ,  has non-negative integral solutions.
\end{prop}
\begin{proof}
It follows from Corollary \ref{uniqueness of join-irreducibles} that . If  then , therefore the proposition holds. If ,  let  denote the unique element in . The system  of linear inequalities now becomes

The collection  of all variables in  is . Let  be given by


We claim that  is a solution of . By the definition of , it is straightforward to verify that . It remains to show that for every , . Indeed,  follows from the definition of . Since , we have , therefore .  
\end{proof}
We derive easily the following corollary from Theorem \ref{the condition of lattices induced by CFG} and Proposition \ref{lemma of the condition on distributive lattice}
\begin{coro}[\cite{MVP01}]Every distributive lattice is in .
\end{coro}
We close this section by presenting a polynomial time algorithm for determining whether a given ULD lattice is in . To do this we have to show that finding a non-negative integral solution of  can be done in polynomial time. It is well-known that the problem of deciding whether an integral system of linear inequalities has an integral solution is NP-complete. Fortunately the following shows that the problem is solvable in polynomial time when it is restricted to .
\begin{lem}
\label{find a solution}
Given , we can decide if it has a non-negative integral solution, and if so, find one, in polynomial time.
\end{lem}
\begin{proof}
Clearly the corresponding problem on  is solvable in polynomial time by using the known algorithms for linear programming. If  has no non-negative real solution then  has no non-negative integral solution. Otherwise let  be a non-negative real solution of . We are going to construct a non-negative integral solution  of  from . The map  is defined by  for every , and . We show that  is a non-negative integral solution of . By the definition of  it remains to show that  for any . Let  such that . Clearly .
\end{proof}

The lattice  can be input as a directed acyclic graph with the edges induced from the cover relation of , \emph{i.e. }  iff  holds in . Note that  and  can be found in  time by using search algorithms.
The algorithm is presented by the following pseudocode\\
\begin{algorithm}[H]
\SetKwInOut{Input}{Input}
\SetKwInOut{Output}{Output}

\Input{A ULD lattice  which is input as a acyclic graph with the edges defined by the cover relation}
\Output{\textbf{Yes} if  is in , \textbf{No} otherwise. If \textbf{Yes} then give a support graph  and an initial configuration  on  so that  is isomorphic to }
\;
\;
\For{}{
Construct  \;
\lIf{\text{ has no non-negative integral solutions }}{\textbf{Reject}}\;
\Else{
        Let  be a non-negative integral solution of \;
        Let  be the collection of all variables in \;
        \For{}{
                      Add  edges  to 
                     }
        Add  edges  to 
        }
}
Construct the initial configuration  by

\end{algorithm}
We can use the Karmarkar's algorithm \cite{K84} to find a non-negative integral solutions  of  that can be done as in the proof Lemma \ref{find a solution}. For each  the number of bits that are input to the algorithm is bounded by . We have to run the Karmarkar's algorithm  times. Hence the algorithm can be implemented to run in  time.
\section{A necessary and sufficient condition for }
\label{fourth section}
Abelian Sandpile model is the CFG model which is defined on connected undirected graphs \cite{BTW87}. In this model, the support graph is undirected and it has a distinguished vertex which is called \emph{sink} and never fires in the game even if it has enough chips. If we replace each undirected edge  in the support graph by two directed edges  and  and remove all out-edges of the sink then we obtain an CFG on directed graph which has the same behavior as the old one. For example, a CFG defined on the following undirected graph with sink 
\begin{center}
\includegraphics[bb=0 1 222 219,width=1.47in,height=1.44in,keepaspectratio]{image27}
\end{center}
is the same as one which is defined on the following graph
\begin{center}
\includegraphics[bb=1 3 242 254,width=1.38in,height=1.44in,keepaspectratio]{image43}
\end{center}
and the initial configuration is the same as the old one. Thus a  can be regarded as a CFG on a directed multi-graph. We give an alternative definition of  on directed multi-graphs as follows. A , where  is a directed multi-graph, is a  if  is connected,  has only one sink  and for any two distinct vertices  of , which are distinct from the sink, we have . Therefore in this model we will continue to work on directed multi-graphs. 

The lattice structure of this model was studied in \cite{Mag03}. The authors proved that the class of lattices induced by ASMs is strictly included in  and strictly includes the class of distributive lattices. To get the necessary and sufficient condition for , we used the important result from \cite{MVP01} which asserts that every CFG is equivalent to a simple CFG.  A difficulty of getting a necessary and sufficient criterion for  is that we do not know whether a similar assertion holds for the , {\emph i.e.} whether an ASM is equivalent to a simple ASM, therefore the argument in \cite{MVP01} does not seem to be transferable to ASM. Nevertheless, we overcome this difficulty by constructing a generalized correspondence between the firing vertices in a relation with their times of firing of a CFG  and the meet-irreducibles of the lattice generated by this CFG. Using this correspondence we achieve a necessary and sufficient condition for . This condition provides a polynomial-time algorithm for determining whether a given ULD lattice is in . We also give some other results which concern to this model. The following lemma shows that correspondence, it is a generalization of Lemma \ref{lemma of relation}.
\begin{lem}
\label{generalization of lemma of relation}
If  is generated by  then the map , determined by , where  are two configurations of  such that  and , is well-defined. Furthermore  is injective.  
\end{lem}
Note that games in Lemma \ref{lemma of relation} are supposed to be simple, whereas games in the above lemma are not necessarily simple. The lemma means that if each  is labeled by the pair of the vertex at which  is fired to obtain  and the number of times this vertex is fired to reach  from the initial configuration then the labeling is the same as  labeling  by . Let us give a concrete example to illustrate this concept. The CFG defined by the support graph  and the initial configuration , which are shown in Figure \ref{fig:image4546}, generates the lattice that is shown by Figure \ref{fig:image22} and Figure \ref{fig:image23}.
\begin{figure}[!h]
\centering
\subfloat[Support graph]{\label{fig:image45}\includegraphics[bb=2 2 258 204,width=1.55in,height=1.22in,keepaspectratio]{image45}}
\qquad \qquad
\subfloat[Initial configuration]{\label{fig:image46}\includegraphics[bb=10 14 182 150,width=1.55in,height=1.22in,keepaspectratio]{image46}}
\caption{A non-simple CFG}
\label{fig:image4546}
\end{figure}
\captionsetup[subfloat]{justification=centerfirst,singlelinecheck=false}
\begin{figure}[!h]
\centering
\subfloat[Cover relation labeled by firing vertex and times of firing]{\label{fig:image22}\includegraphics[bb=12 6 159 394,width=0.962in,height=2.54in,keepaspectratio]{image22}}
\qquad \qquad
\subfloat[Cover relation labeled by meet-irreducibles]{\label{fig:image23}\includegraphics[bb=13 4 174 433,width=0.951in,height=2.54in,keepaspectratio]{image23}}
\caption{Two ways of labeling}
\end{figure}
\captionsetup[subfloat]{justification=justified,singlelinecheck=true}In Figure \ref{fig:image22}, each  is labeled by the fired vertex and the number of times this vertex is fired to obtain . Figure \ref{fig:image23} shows the lattice in the way each  is labeled by . It is obvious that the labelings are the same with respect to the correspondence .\\

\noindent\emph{{Proof of Lemma \ref{generalization of lemma of relation}}}. To prove  is well-defined, it suffices to show that for  being two configurations of  such that , where , we have  \linebreak . Let  be an execution in . By Lemma \ref{lemma of the square connection}, there exists a chain  such that  for every . It is easy to see that . Let  denote . It remains to prove that . For each , we have , therefore . It implies that .
 

It follows immediately from the definition of  that  is a surjection from  to . Here  denotes the set , by convention,  if . For ,  is the number of times  is fired in any execution from  to , therefore  is the number of times the vertices are fired to reach . Thus  is the height of . To prove  is injective it suffices to show that  is also the height of . Let   be any chain of , where  is the height of . Clearly , and  for any two distinct indices  in . By Lemma \ref{condition on cover relation for ULD lattice} each  contains exactly one element, therefore .\hfill 

In the case of directed graphs the systems  of linear inequalities are solved independently to know whether  is in  since there is no requirement for relation between  and  on support graph. In the case of undirected graphs the condition  must be satisfied for any two vertices distinct from sink. Hence the systems of linear inequalities for  are constructed as follows.

For each  we define the system of linear inequalities  by replacing each variable  in  by  and  by . We give an example for this transformation. Consider the lattice shown in Figure \ref{fig:image17}. We have 

then 



For each ,  is a system of linear inequalities whose variables are a subset of . Let  denote the set of all variables in . The system  of linear inequalities is given by  

If  is generated by a simple CFG, say , then it follows from the correspondence established in Lemma \ref{lemma of relation} and the construction in Theorem \ref{the condition of lattices induced by CFG} that for ,  can be regarded as the number of directed edges from  to  in , where  are the corresponding vertices of , respectively. As the sufficient condition in Theorem \ref{the condition of lattices induced by CFG}, the following lemma shows a similar assertion for .
\begin{lem}
\label{sufficient condition for L(ASM)}
If  has non-negative integral solutions then .
\end{lem}
\begin{proof}
We construct the graph  whose set of vertices is  and the edges are defined as follows. Let  be a non-negative integral solution of . For each two distinct elements , if  then there are  edges connecting  to  in  and  edges connecting  to .  If  and  then there is no edge connecting  with  in . It follows immediately from the definition of  that  is well-defined. For each , there are  edges connecting  to . The initial configuration  for the game is defined by 

where  denotes the out-degree of  in . It's clear that  is the sink of the game. We can argue similarly as in the proof of Theorem \ref{the condition of lattices induced by CFG} that the game is simple and generates . 
\end{proof}
\begin{examp}
\label{second example ?}
We consider the system of linear inequalities of Example \ref{first example}. Then  is the following system

The map  defined by 

is a non-negative integral solution of . By the construction in the sketch of proof,  and the initial configuration are presented by Figure \ref{fig:image2526}.
\begin{figure}[h]
\centering
\subfloat[Support graph]{\label{fig:image25}\includegraphics[bb=0 1 224 231,width=1.08in,height=1.11in,keepaspectratio]{image25}}
\qquad \qquad
\subfloat[Initial configuration]{\label{fig:image26}\includegraphics[bb=11 9 234 242,width=1.08in,height=1.11in,keepaspectratio]{image26}}
\caption{A ASM solution}
\label{fig:image2526}
\end{figure}
Note that in the figure,  is presented by an undirected graph for a nice presentation. The sink of the game is in black. Doing a simple computation on the game, it is straightforward to verify that the lattice generated by  is isomorphic to . 

The following theorem shows that the condition that  has non-negative integral solutions is not only a sufficient condition but also a necessary condition of . 
\end{examp}
\begin{theo}
\label{necessary and sufficient condition for ASM}
 if and only if  has non-negative integral solutions.
\end{theo}
\begin{proof}
The direction  is already proved by Lemma \ref{sufficient condition for L(ASM)}. We are left with proving the direction . Let  be a  and generates . Let  denote the sink of the game. We define 
\noindent where  denotes the out-degree of  in . Let  be the injective map which is defined as in Lemma \ref{generalization of lemma of relation}. For , let  denote the first and second components of , respectively. 

We claim that for each  and each , we have . Indeed, let  be a chain of , where  is a non-negative integer. Note that . For a contradiction, we suppose that . It implies that the number of times  is fired in the execution (chain) is greater than or equal to . Hence there is a unique index  such that  and . It follows from the definition of  in Lemma \ref{generalization of lemma of relation} that . Since  is injective, it follows that . It contradicts the definition of . The claim follows. 

Our next claim is that for each  and each , if  then for every , we have

Indeed, the righ-hand side of \eqref{first equa} indicates the number of chips vertex  receives from its neighbors during an execution from  to . To reach ,  has been fired  times. It follows that the number of chips stored at  in  is

 is firable in , therefore . By a similar argument, the number of chips stored at  in  is

 is not firable in , therefore . It follows easily from  and  that  holds. 

Let  be given by
, where  is the collection of all variables of . The proof is completed by showing that  is a non-negative integral solution of . Since  is a ASM, it follows easily that for any two distinct elements , if  and  both are in  then .  It remains to show that for each ,  satisfies . If  then the assertion follows easily. If , it is straightforward to verify that  for any . We are left with proving  for any . For this purpose, we show that  for any . We have

where . There are two possibilities
\begin{itemize}
  \item[a. ]. It follows from \eqref{first equa} and  that 

\item[b. ] . It follows from the definition of  that 

\end{itemize}  
Therefore,  is a non-negative integral solution of .
\end{proof}
As Lemma \ref{find a solution}, the problem of finding a non-negative integral solution of  is solvable in polynomial time.
\begin{lem}
\label{find a solution for Omega}
Given , we can decide if it has a non-negative integral solution, and if so, find one, in polynomial time.
\end{lem}
\begin{proof}
Clearly the corresponding problem on  is solvable in polynomial time by using the known algorithms for linear programming. If  has no non-negative real solution then  has no non-negative integral solution. Otherwise let  be a non-negative real solution of , where  denotes the set of variables of . Let  be given by  if , and . Now we can use the same arguments as in Lemma \ref{find a solution} to argue that  is a non-negative integral solution of . This completes the proof.
\end{proof}

Lemma \ref{find a solution for Omega} implies a polynomial time algorithm for the problem of determining whether a given lattice is in , and construct a corresponding CFG if there exists one. We again use the Karmarkar's algorithm for finding a non-negative integral solution of . The number of variables of  is bounded by  and the number of bits, which are input to the algorithms for linear programming to find a non-negative integral solution of , is bounded by . Therefore the algorithm can be implemented to run in  time.\\
\begin{examp}
\label{old example}
Let  be the following lattice
\begin{center}
\includegraphics[bb=11 6 366 209,width=3.1in,height=1.77in,keepaspectratio]{image32}
\end{center}
This latttice was presented in \cite{Mag03} as an example of showing the class of lattices induced by ASM is strictly included in the class of lattices induced by CFG. We again present it here as an application of Theorem \ref{necessary and sufficient condition for ASM}.  The system  of linear inequalities is
\\
Using the algorithms for linear programming we know that the system has no non-negative  solution, therefore has no non-negative integral solution. Therefore the lattice is not in .
\end{examp}
\begin{examp}
\label{smaller example}
The game with the initial configuration presented in Figure \ref{fig:image30} generates the lattice presented in Figure \ref{fig:image31}. It is an example which is smaller than one presented in \cite{Mag03}. 
\begin{figure}[h]
\centering
\subfloat[Initial configuration]{\label{fig:image30}\includegraphics[bb=15 6 242 251,width=1.23in,height=1.33in,keepaspectratio]{image30}}
\qquad \qquad
\subfloat[Generating lattice]{\label{fig:image31}\includegraphics[bb=11 5 321 243,width=2.3in,height=1.77in,keepaspectratio]{image31}}
\caption{Smaller example}
\label{fig:image3031}
\end{figure}
\end{examp}
Note that the two lattices in Example \ref{old example} and Example \ref{smaller example} are generated only by simple CFGs. It is useful to give a sufficient condition for such lattices. The following proposition shows such a condition
\begin{prop} 
Let  be the undirected simple graph whose vertices are  and edges are defined by  if there are  such that ,  and . If  and  is a complete graph then  is generated only by simple CFGs.
\end{prop}
\begin{proof}
We assume otherwise that  is generated by a non-simple CFG, say . Let  be the map defined in Lemma \ref{generalization of lemma of relation}. Since the game is not simple, there is a vertex  of   such that  is fired more than once during an execution from  to the fixed point. Therefore there are two distinct meet-irreducibles  and  in  such that  and . Since  is complete, there are three distinct configurations  and  such that  and . By the definition of  we have  and . It implies that , a contradiction.
\end{proof}

In \cite{MVP01}, the authors proved that a general CFG  is always equivalent to a simple CFG. An arising question is that whether a ASM is equivalent to a simple ASM. The idea from the proof in \cite{MVP01} does not seem to be applicable to this model, whereas the answer follows easily from the proofs of Lemma \ref{sufficient condition for L(ASM)} and Theorem \ref{necessary and sufficient condition for ASM}
\begin{prop}
Any ASM is equivalent to a simple  ASM.
\end{prop}
\begin{proof}
Assume that . By Theorem \ref{necessary and sufficient condition for ASM}  has non-negative integral solutions. We consider the CFG that is constructed as in the proof of Lemma \ref{sufficient condition for L(ASM)}. It is a simple ASM and generates . This completes the proof.
\end{proof}
\section{CFGs on acyclic graphs}
\label{fifth section}
In \cite{Mag03} the author gave a strong relation between ASM and the simple CFGs on acyclic graphs (directed acyclic graphs). The author pointed out that a simple CFG on an acyclic graph is equivalent to a ASM. In this section we study CFGs on acyclic graphs that are not necessarily simple. We show that each CFG on an acyclic graph is equivalent to a simple CFG on an acyclic graph. As a corollary, every lattice generated by a CFG on an acyclic graph is in . We also give a necessary and sufficient criterion for lattices generated by CFGs on acyclic graphs.

\begin{lem}
\label{CFGs on acyclic graphs}
If  is generated by a CFG on an acyclic graph then  is acyclic, where  is the simple directed graph whose vertices are  and edges are defined by  if and only if .
\end{lem}
\begin{proof}
Let  be a CFG which generates , where  is an directed acyclic graph. Let  be the map which is defined in Lemma \ref{generalization of lemma of relation}. For each ,  denotes the collection of vertices  of  such that  there is a directed path from  to  in  ( could be equal to ). A sequence  is called a \emph{valid firing sequence} if there is an execution  in the game. Note that if such an execution exists then it is defined uniquely. 

We claim that for each  and each , we have . Indeed, let  be an execution, where  is the configuration which is obtained by firing  in . It is clear that  is a valid firing sequence. Since firing of the vertices not in  does not affect the firability of the remaining vertices, by removing all vertices not in  of the sequence, we get the sequence  which remains a valid firing sequence. There exists an execution  in the game. Since the number of occurrences of  in  is the same as the one in , it follows that , thus . Clearly we have , therefore . It follows from the definition of   that , hence .

Our next claim is that for each  and each , if  then . Indeed, let  be an execution. Since , there is  such that . Thus , and . The claim follows.

Since  is an acyclic graph, there exists a function  such that if  then . Let . We define  by . To prove  is acyclic, it suffices to show that for every , we have . From the definition of , there exists  such that . There are two possibilities
\begin{itemize}
  \item[a. ] .  It follows from the second claim that . 
  \item[b. ] . It follows from the first claim that .  Therefore .
\end{itemize}  
\end{proof}
We recall a result in \cite{Mag03}
\begin{theo}
\label{theorem of Magnien}\cite{Mag03} 
Let  be a simple CFG on an acyclic graph . Then  is equivalent to a ASM.
\end{theo}
Here, our main result of this section
\begin{theo}
\label{CFGs on acyclic graph including in ASM}
Any CFG on an acyclic graph is equivalent to a simple CFG on an acyclic graph, therefore equivalent to a ASM.
\end{theo}
\begin{proof}
Let  be a CFG such that  is an acyclic graph, and let  denote . By Theorem \ref{the condition of lattices induced by CFG} that for each ,  has non-negative integral solutions. Let  be the collection of all variables of  and  be a non-negative integral solution of . The function  defined by

is also a non-negative integral solution of . By using solutions ,  it follows from the construction of the CFG in the proof of Theorem \ref{the condition of lattices induced by CFG} that  is generated by a simple CFG on a graph, say , such that  and if  then  or , where  is the graph that is defined as in Lemma \ref{CFGs on acyclic graphs}. It follows directly from Lemma \ref{CFGs on acyclic graphs} that   is acyclic, so is . Theorem \ref{theorem of Magnien} implies that  is equivalent to a ASM.
\end{proof}
Using Lemma \ref{CFGs on acyclic graphs} and a similar argument as in the proof of Theorem \ref{CFGs on acyclic graph including in ASM} we obtain a necessary and sufficient criterion for the class of lattices generated by CFGs on acyclic graphs
\begin{coro}
Let . Then  is generated by a CFG on an acyclic graph if and only if  is acyclic.
\end{coro}
\begin{proof}
The necessary condition is proved by Lemma \ref{CFGs on acyclic graphs}. It remains to prove that the sufficient condition also hold. Let  be given as in Lemma \ref{CFGs on acyclic graphs}. By Theorem \ref{the condition of lattices induced by CFG} that for each ,  has non-negative integral solutions. We define non-negative integral solutions ,  for each , and a simple CFG on  that generates   as in Theorem \ref{CFGs on acyclic graph including in ASM}. Since  is acyclic  and since if  then  or , it follows that  is acyclic. This completes the proof.
\end{proof}
Let  denote the class of lattices generated by CFGs on acyclic graphs. Theorem \ref{CFGs on acyclic graph including in ASM} implies that . We consider the lattice shown in Figure \ref{fig:image16}. In this case,  is presented by the following figure
\begin{center}
\includegraphics[bb=0 2 150 155,width=1.2in,height=1.22in,keepaspectratio]{image33}
\end{center}
 is not acyclic, therefore the lattice is not in . From Example \ref{second example ?}, the lattice is in . It implies that . Furthermore the lattice presented in Figure \ref{fig:image37}
\begin{figure}[h]
\centering
\includegraphics[bb=4 4 107 151,width=1.24in,height=1.77in,keepaspectratio]{image37}
\caption{A lattice in }
\label{fig:image37}
\end{figure}
is generated by a CFG on acyclic graph but not a distributive lattice. Thus .
\section{Conclusion and perspectives}
\label{sixth section}
In this paper we have studied the properties of three classes of lattices generated by CFGs, that are ,  and . On algorithmic aspect we give a necessary and sufficient criterion for each studied class. These criteria provide the polynomial-time algorithms for determining which class of lattices a given ULD lattice belongs to. A relation between those classes of lattices is also pointed out by showing that  is situated strictly between the class of distributive lattices and . In other word, we obtain a finer chain of the studied classes of lattices, that is

where  is the class of lattices generated by MCFGs (Multating Chip Firing Game \cite{B97,H99,Mag03}).

It is interesting to investigate CFGs defined on the classes  of graphs that are studied widely in literature, for example the class of Eulerian directed graphs. This class is a close extension of the class of undirected graphs. Recall that a graph  is \emph{Eulerian} if it is connected and for each vertex of  its out-degree and in-degree are equal. We define a CFG on an Eulerian graph  as follows. We fix a vertex  of  which will play a role as the sink of the game. Then we remove all out-edges of . The resulting graph  remains a connected graph and has no closed component. The game is defined on this graph. Let  denote the class of lattices generated by CFGs on Eulerian graphs. It is clear that . The problem of determining which inclusion is strict remains to be done.  

It turns out to be interesting that a CFG defined on each studied class of graphs is equivalent to a simple CFG which again is defined on this class. This property implies that to study the lattices generated by CFGs defined on these classes of graphs, it is sufficient to study simple CFGs. As we saw in this paper this property is proved on different classes of graphs with different techniques. Thus it is not easy to know whether this property holds for other classes of graphs. In particular we still do not know whether this property holds for the Eulerian graphs. A  characterization of classes of graphs having this property remains to be done.

Finally, we are also interested in the following computational problem: Given a graph  and a ULD lattice , is  generated by a CFG on ?

So now, we have the practical criteria for the classes of lattices generated by CFGs defined on three classes of graphs which are studied widely in literature, they are acyclic graphs, undirected graphs, and directed graphs. We believe that our method presented here is not only applicable to these classes but also applicable to many other classes of graphs on which CFGs are defined.
\text{}\\

\textbf{Acknowledgement.} We would like to thank D. Dhar for the comments on the term ``Abelian Sandpile model'' that was used in the first version (in arXiv and also in journal). The more informations about  this term have been added to this version.
\pagebreak
\begin{thebibliography}{9}
\bibitem[Bi97]{B97} N. Biggs. Algebraic potential theory on graphs. \emph{The Bulletin of London Mathematical Society}, 29:641-682,1997
\bibitem[Bi99]{B99} N. Biggs. Chip firing and the critical group of a graph. \emph{Journal of Algebraic Combinatorics 1}, pages 304-328, 1999
\bibitem[Bir33]{B33} G. Birkhoff. On the combination of subalgebras. \emph{Proc. Camb. Phil. Soc}, 1933.
\bibitem[BL91]{BLS91} A. Bj\"orner, L. Lov\'asz and W. Shor. Chip-firing games on graphs. \emph{E. J. Combinatorics}, 12:283-291, 1991.
\bibitem[BL92]{BL92} A. Bj\"orner and L. Lov\'asz. Chip-firing games on directed graphs. \emph{J. Algebraic Combinatorics}, 1:304-328, 1992.
\bibitem[BTW87]{BTW87} P. Bak, C. Tang and K. Wiesenfeld.  Self-organized criticality: an explanation of the 1/f noise. \emph{Physics Review Letters}. 59(1): 381-384, 1987.
\bibitem[Ca98]{C98} N. Caspard. Etude structurell et algorithmique de classe de treillis obtenus par duplications. PhD thesis, Universit\'e Paris I, 1998.
\bibitem[Dha90]{Dha90} D. Dhar. Self-Organized Critical State of Sandpile Automaton Models, \emph{Physical Review Letters} 64(14), 1613-1616, 1990.
\bibitem[Di40]{D40} Robert P. Dilworth. Lattices with unique irreducible decompositions, \emph{Annals of Mathematics} 41 (4): 771-777,1940.
\bibitem[DRSV95]{DRSV95} D. Dhar, P. Ruelle, S. Sen and D. Verma. Algebraic aspects of sandpile models. \emph{Journal of {P}hysics A}. 28: 805-831, 1995.
\bibitem[Er93]{Eri93} K. Eriksson. Strongly convergent games and Coxeter group, 1993.
\bibitem[FK09]{FK09} S. Felsner and Kolja B. Knauer. ULD-Lattices and Delta-Bonds. \emph{Combinatorics, Probability \& Computing} 18(5): 707-724, 2009.
\bibitem[GMP98]{GMP98} E. Goles, M. Morvan and H. Phan. Lattice structure and convergence of a game of cards. \emph{Annals of Combinatorics}, 1998.
\bibitem[He99]{H99} J. Heuvel. Algorithmic aspects of a Chip Firing Game. \emph{London School of Economics, CDAM Research report}, 1999.
\bibitem[Ka84]{K84} N. Karmarkar. A new polynomial-time algorithm for linear programming. \emph{Combinatorica}, 4(4): 373-395, 1984.

\bibitem[LP01]{LP01} M. Latapy and H. D.  Phan. The lattice structure of Chip Firing Game. \emph{Physica D}, 115:69-82, 2001.

\bibitem[Ma03]{Mag03} C. Magnien. Classes of lattices induced by Chip Firing (and Sandpile) Dynamics. \emph{European Journal of Combinatorics}, 24(6):665-683, 2003.
\bibitem[Me01]{M01} C. Merino. The chip firing game and matroid complex. \emph{Discrete Mathematics and Theoretical Computer Science. Proceedings vol. AA}, pages 245-256, 2001.
\bibitem[Me97]{M97} C. Merino. Chip-firing and the Tutte polynomial. \emph{Annals of Combinatorics}, 1(3): 253-259, 1997.

\bibitem[Mo90]{M90} B. Monjardet. The consequences of Dilworth's work on lattices with unique irreductible decompositions, in: K. P. Bogart, R. Freese, J. Kung (Eds.), The Dilworth Theorems Selected Papers of Robert P. Dilworth, Birkhauser, Boston, 1990, pp. 192-201.
\bibitem[MVP01]{MVP01} C. Magnien, L. Vuillon and H. D. Phan. Characterisation of lattice induced by (extended) Chip Firing Games. \emph{The proceedings of DM-CCG}, 2001.

\bibitem[Sp86]{S86} J. Spencer. Balancing vectors in the max norm. \emph{Combinatorica}, 6:55-66, 1986.
\end{thebibliography}
Trung Van Pham\\
Department of Mathematics of Computer Science\\
Vietnam Institute of Mathematics\\
18 Hoang Quoc Viet road, Cau Giay district, Hanoi, Vietnam\\
Email: pvtrung@math.ac.vn\\
\text{}\\
Thi Ha Duong Phan\\
Department of Mathematics of Computer Science\\
Vietnam Institute of Mathematics\\
18 Hoang Quoc Viet road, Cau Giay district, Hanoi, Vietnam\\
Email: phanhaduong@math.ac.vn
\end{document}
