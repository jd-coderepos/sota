
\label{sec:newlp}

\newcommand{\vset}{{\mathcal{V}}}
Due to the integrality gap result in the previous section, there is no hope to improve the best known algorithms via the canonical LP relaxation. 
Hartke~\cite{Hartke2006} suggested adding the following constraints to narrow down the integrality gap of the LP. 
\[\sum_{ u\in P_v \cup (T_v \cap L_j)} x_u \leq 1 \mbox{ for all vertex $v \in V(T)$ and layer $L_j$ below the layer of $v$} \] 

We write the new LP with these constraints below: 
\vspace{0.1in}

\noindent
\framebox[\textwidth]{ 
\begin{minipage}[b]{0.99\textwidth}
\vspace{-1em}
\begin{eqnarray*} 
 \mbox{(LP')} \\
  &\max & \sum_{v \in V} y_v\\
  && \sum_{u \in P_v \cup (T_v \cap L_j)} x_u \leq 1 \mbox{ for all layer $j$ below vertex $v$} \\ 
  &&  y_v \leq \sum_{u \in P_v} x_u \mbox{ for all $v\in V$}\\
  && x_v, y_v \in [0,1] \mbox{ for all $v$ }  
\end{eqnarray*}
\end{minipage}
}

\begin{proposition} 
\label{prop:setyvalues}
Given the values $\set{x_v}_{v \in V(T)}$ that satisfy the first set of constraints, then the solution $(x,y)$ defined by $y_v = \sum_{u \in P_v} x_v$ is feasible for (LP') and at least as good as any other feasible $(x,y')$.  
\end{proposition} 

In this section, we study the power of this LP and provide three evidences that it may be stronger than (LP-1). 

\subsection{New properties of extreme points}
In this section, we show that Finbow et al.~tractable instances~\cite{FinbowG09} admit a polynomial time exact algorithm via (LP') (in fact, any optimal extreme point for (LP') is integral.) 
In contrast, we show that (LP-1) contains an extreme point that is not integral.   



We first present the following structural lemma. 
\begin{lemma}
\label{lem: extreme}  
Let $({\bf x},{\bf y})$ be an optimal extreme point for (LP') on instance $T$ rooted at $s$. Suppose $s$ has two children, denoted by $a$ and $b$.  
Then $x_a, x_b \in \set{0,1}$.  
\end{lemma}  

\begin{appendixproof}{sec:newlp}{\subsection{Proof of Lemma \ref{lem: extreme}}}
Suppose that $x_a, x_b \in (0,1)$. 
We will define two solutions $({\bf x'},{\bf y'})$ and $({\bf x''}, {\bf y''})$ and derive that $({\bf x},{\bf y})$ can be written as a convex combination of $({\bf x'},{\bf y'})$ and $({\bf x''}, {\bf y''})$, a contradiction.

First, we define $({\bf x}',{\bf y}')$ by setting $x'_b = 1, x'_a=0$. For each vertex $v \in T_b$, we set $x'_v = 0$. For each vertex $v \in T_a$, we define $x'_v = x_v/(1-x_a)$.
We verify that $x'$ is feasible for (LP'): For each $v \in T_a$ and any layer $L_j$ below $v$, $\sum_{u \in P_v} x'_u + \sum_{u \in T_v \cap L_j} x'_u = \frac{\paren{\sum_{u \in P_v} x_u} - x_a}{(1-x_a)} + \frac{\sum_{u \in T_v \cap L_j} x_u}{(1-x_a)} \leq \frac{\paren{\sum_{u \in P_v \cup (T_v \cap L_j)} x_u} -x_a}{(1-x_a)} \leq 1$ (the last inequality is due to the fact that ${\bf x}$ is feasible).
The constraint is obviously satisfied for all $v \in T_b$.  
For the root node $v =s$, we have $\sum_{u \in L_j} x'_u = \frac{\paren{\sum_{u\ \in (L_j \cap T_a)} x_u} - x_a}{(1-x_a)} \leq 1$.  

We define $({\bf x}'',{\bf y}'')$ analogously: $x''_b =0, x''_a =1$. For each vertex $v \in T_a$, we set $x''_v = 0$, and for each $v \in T_b$, we define $x''_v = x_v/ (1-x_b)$. 
It can be checked similarly that $({\bf x}'',{\bf y}'')$ is a feasible solution.  

\begin{claim} 
If ${\bf x}$ is an optimal extreme point, then $x_a + x_b =1$. 
\label{claim:extremepoint}
\end{claim} 

\begin{proof}Observe that, for each $v \in T_b$, $y'_v = 1$ and for each $v \in T_a$, $y'_v = \frac{y_v  - x_a}{1-x_a}$.  
The objective value of ${\bf x'}$ is $|T_b| + \sum_{v \in T_a} y'_v = |T_b| + \frac{1}{(1-x_a)} \sum_{v \in T_a} (y_v - x_a) = |T_b| + \frac{\sum_{v \in T_a} y_v}{(1-x_a)} - \frac{x_a}{(1-x_a)} |T_a|$.
Similarly, the objective value of solution ${\bf x''}$ is $|T_a| + \frac{1}{(1-x_b)} \sum_{v \in T_b} (y_v - x_b) = |T_a| + \frac{\sum_{v \in T_b} y_v}{(1-x_b)} - \frac{x_b}{(1-x_b)} |T_b|$.
 
Consider the convex combination $\frac{1-x_a}{(2-x_a-x_b)} {\bf x'} + \frac{1-x_b}{(2-x_a-x_b)} {\bf x''}$. 
This solution is feasible and has the objective value of 
$$\frac{1}{(2-x_a-x_b)}\cdot \paren{ (1-x_a -x_b) \paren{|T_a| + |T_b|}+ \sum_{v \in V(T)} y_v}$$
If $x_a + x_b <1$, we apply the fact that $|T_a| + |T_b| > \sum_{v \in V(T)} y_v$ to get the objective of strictly more than $\sum_{v \in V(T)} y_v$, contradicting the fact that $({\bf x}, {\bf y})$ is optimal.  
\end{proof} 

Finally, we define the convex combination by ${\bf z}= (1-x_a) {\bf x'}  + x_a {\bf x''}$.
It can be verified easily that $z_v = x_v$ for all $v \in V(T)$.  
\end{appendixproof} 

\vspace{0.1in} 

\noindent {\bf Finbow et al. Instances:} In this instance, the tree has degree at most $3$ and the root has degree $2$.  
Finbow et al.~\cite{FinbowG09} showed that this is polynomial time solvable. 

\begin{wrapfigure}[11]{r}{0.25\textwidth}
\centering
\includegraphics{figures/fig4}
\caption{Instance with a non-integral extreme point for (LP-1). Gray vertices: $x_v=1/2$; otherwise: $x_v = 0$.}
\label{fig:lp1badcase}
\end{wrapfigure}

\begin{theorem} 
Let $(T,s)$ be an input instance where $T$ has degree at most $3$ and $s$ has degree two. Let $(x,y)$ be a feasible fractional solution for (LP-3). 
Then there is a polynomial time algorithm that saves at least $\sum_{v \in V(T)} y_v$ vertices. 
\label{thm:finbowinst}
\end{theorem}  
\begin{appendixproof}{sec:newlp}{\subsection{Proof of Theorem \ref{thm:finbowinst}}}
We prove this by induction on the number of nodes in the tree that, for any tree $(T',s')$ that is a Finbow et al. instance, for any fractional solution $(x,y)$ for (LP'), there is an integral solution $(x',y')$ such that $\sum_{v \in T' \setminus \set{s'}} y'_v = \sum_{v \in T' \setminus \set{s'}} y_v$.  
Let $a$ and $b$ be the children of the root $s$.
From Lemma~\ref{lem: extreme}, assume w.l.o.g. that $x_a = 1$, so we have $\sum_{v \in T_a} y_v = |T_a|$. 
By the induction hypothesis, there is an integral solution $(x',y')$ for the subtree $T_b$ such that $\sum_{v \in T_b} y'_v = \sum_{v \in T_b \setminus \set{b}} y'_v = \sum_{v \in T_b} y_v$.  
The solution $(x',y')$ can be extended to the instance $T$ by defining $x'_a = 1$. 
This solution has the objective value of $|T_a| + \sum_{v \in T_b} y'_b = |T_a| + \sum_{v \in T_b} y_b$, completing the proof. 
\end{appendixproof}

\noindent {\bf Bad instance for (LP-1):} 
We show in Figure \ref{fig:lp1badcase} a Finbow et al. instance as well as a
solution for (LP-1) that is optimal and an extreme point, but not integral.

\begin{appendixextra}{sec:newlp}{\subsection{Analysis of Figure \ref{fig:lp1badcase}}}
\begin{claim}

The solution $(x,y)$ represented in Figure \ref{fig:lp1badcase}, with $y$ defined according to
Proposition \ref{prop:setyvalues}, is an extreme point of this instance for
(LP-1).

\end{claim}

\begin{proof}

Suppose (for contradiction) that $(x,y)$ is not an extreme point. Then, there
are distinct solutions $(x',y')$, $(x'',y'')$ and $\alpha \in (0,1)$ such that
$(x,y) = \alpha(x',y') + (1-\alpha)(x'',y'')$. Since $y_c = 1$ and $y'_c,
y''_c \leq 1$, then $y'_c = y''_c = 1$, and likewise, $y'_d = y''_d = 1$.
Combining that $x'_a + x'_c = y'_c = 1$ with $x'_a + x'_d = y'_d = 1$ and
$x'_c + x'_d \leq 1$, we conclude that $x'_a \geq 1/2$. Similarly, we get that
$x''_a \geq 1/2$, which implies that $x'_a = x''_a = 1/2$.

Similar reasoning using that $x'_a + x'_b \leq 1$ allows us to conclude that
$x'_b = x''_b = 1/2$, and thus, $(x',y') = (x'',y'') = (x,y)$, which
contradicts our assumption.
\end{proof}
\end{appendixextra}


\subsection{Rounding \texorpdfstring{$1/2$}{1/2}-integral Solutions}
\label{sec:half-int}
We say that the LP solution $(x,y)$ is $(1/k)$-integral if, for all $v$, we have $x_v = r_v/k$ for some integer $r_v
\in \{0,\ldots,k\}$.  
By standard LP theory, one can assume that the LP solution is $(1/k)$-integral for some polynomially large integer $k$. 



In this section, we consider the case when $k=2$ ($1/2$-integral LP solutions).  
From Theorem~\ref{thm: gap 1/k}, (LP-1) is not strong enough to obtain a $3/4+\epsilon$ approximation algorithm, for any $\epsilon > 0$. 
Here, we show a $5/6$ approximation algorithm based on rounding (LP').  

\begin{theorem} 
Given a solution $(x,y)$ for (LP') that is $1/2$-integral, there is a polynomial time algorithm that produces a solution of cost $5/6\,\sum_{v \in V(T)} y_v$.  
\label{thm:12intalgo}
\end{theorem} 

We believe that the extreme points in some interesting special cases will be $1/2$-integral. 

\vspace{0.1in} 

\noindent {\bf Algorithm's Description:} 
Initially, $\uset = \emptyset$. 
Our algorithm considers the layers $L_1,\ldots, L_n$ in this order.
When the algorithm looks at layer $L_j$, it picks a vertex $u_j$ and adds it to $\uset$, as follows. 
Consider $A_j \subseteq L_j$, where $A_j = \set{v \in L_j: x_v >0}$.
Let $A'_j \subseteq A_j$ contain vertices $v$ such that there is no ancestor of $v$ that belongs to $A_{j'}$ for some $j' <j$, and $A''_j = A_j \setminus A'_j$, i.e. for each $v \in A''_j$, there is another vertex $u \in A_{j'}$ for some $j' <j$ such that $u$ is an ancestor of $v$.
We choose the vertex $u_j$ based on the following rules: 
\begin{itemize} 
\item If there is only one $v \in A_j$, such that $v$ is not saved by $\uset$ so far, choose $u_j=v$. 

\item Otherwise, if $|A'_j| =2$, pick $u_j$ at random from $A'_j$ with uniform probability.
Similarly, if $|A''_j| =2$, pick $u_j$ at random from $A''_j$.  

\item Otherwise, we have the case $|A'_j| = |A''_j| = 1$. In this case, we pick vertex $u_j$ from $A'_j$ with probability $1/3$; otherwise, we take from $A''_j$.  
\end{itemize}   


\begin{appendixextra}{sec:newlp}{\subsection{Analysis of Theorem \ref{thm:12intalgo}}}
\textbf{Analysis: }    
Below, we argue that each vertex $v \in V(T): x_v >0$ is saved with probability at least $(5/6) y_v$. 
It is clear that this implies the theorem: Consider a vertex $v': x_{v'}=0$.
If $y_{v'}=0$, we are immediately done. Otherwise, consider the bottommost ancestor $v$ of $v'$ such that $x_v >0$. 
Since $y_v = y_{v'}$, the probability that $v'$ is saved is the same as that of $v$, which is at least $(5/6) y_v$.    

We analyze a number of cases.
Consider a layer $L_j$ such that $|A_j|=1$. 
Such a vertex $v \in A_j$ is saved with probability $1$. 

Next, consider a layer $L_j$ such that $|A'_j| =2$. 
Each vertex $v \in A'_j$ is saved with probability $1/2$ and $y_v = 1/2$. 
So, in this case, the probability of saving $v$ is more than $(5/6) y_v$.  

\begin{lemma}
\label{one-and-one} 
Let $L_j$ be the layer such that $|A'_j| = |A''_j| =1$. 
Then the vertex $u \in A'_j$ is saved with probability $2/3 \geq (5/6) y_u$ and vertex $v \in A''_j$ is saved with probability $5/6$. 
\end{lemma} 

\begin{proof} 
Let $v' \in A_{j'}$ be the ancestor of $v$ in some layer above $A_j$. 
The fact that $v$ has not been saved means that $v'$ is not picked by the algorithm, when it processed $A_{j'}$. 

We prove the lemma by induction on the value of $j$. 
For the base case, let $L_j$ be the first layer such that $|A'_j| = |A''_j| =1$.
This means that the layer $L_{j'}$ must have $|A'_{j'}| = 2$, and therefore the probability of $v'$ being saved is at least $1/2$. 
Vertex $u$ is not saved only if both $v'$ and $u$ are not picked, and this happens with probability $1/2 \cdot 2/3 = 1/3$.
Hence, vertex $u$ is saved with probability $2/3$ as desired.   
Consider now the base case for vertex $v$, which is not saved only if $v'$ is not saved and $u$ is picked by the algorithm among $\set{u,v}$.
This happens with probability $1/2 \cdot 1/3 = 1/6$, thus completing the proof of the base case.   


 For the purpose of induction, we now assume that, for all layer $L_i$ above $L_j$ such that $|A'_{i}| = |A''_i| =1$, the probability that the algorithm saves the vertex in $A'_i$ is at least $2/3$.
Since the vertex $u$ is not saved only if $v'$ is not saved, this probability is either $1/2$ or $1/3$ depending on the layer to which $v'$ belongs. If it is $1/3$, we are done; otherwise, the probability is at most $1/2 \cdot 2/3 = 1/3$.   
Now consider vertex $v$, which is not saved only if $v'$ is not saved and $u$ is picked at $L_j$. This happens with probability at most $1/2 \cdot 1/3 = 1/6$.  
\end{proof} 


\begin{lemma}
Let $L_j$ be a layer such that $A''_j = \set{u,v}$ (containing two vertices). Then each such vertex is saved with probability at least $5/6$. 
\end{lemma}

\begin{proof}

Let $u'$ and $v'$ be the ancestors of $u$ and $v$ in some sets $A'_{i}$ and $A'_k$ above the layer $L_j$.
There are the two possibilities: either both $u'$ and $v'$ are in layers with $|A'_i| = |A'_k| =2$ (maybe $i=k$); or $u'$ is in the layer with $|A'_i| = |A''_i| =1$. We remark that $u' \neq v'$: otherwise, the LP constraint for $v'$ and $L_j$ would not be satisfied.

For $u$ or $v$ to be unsaved, we need that both $u'$ and $v'$ are not saved by the algorithm.  
Otherwise, if, say, $u'$ is saved, $u$ is also saved, and the algorithm would have picked $v$.

\begin{align*}
P[u \text{ is not saved}] &= P[u \text{ not picked} \wedge u' \text{ is not saved} \wedge v' \text{ is not saved}] \\
                   &= P[u \text{ not picked}]\cdot P[u' \text{ is not saved} \wedge v' \text{ is not saved}] \\
                   &= \frac12 \cdot \frac14 
                    = \frac18\\
P[u \text{ is saved}] &=\frac78 \geq \frac56
\end{align*}

It must be that $P[u' \text{ burns} \wedge v' \text{ burns}] \leq 1/4$,
since either $u'$ and $v'$ are in different layers or they are in the same
layer. 
If they are in different layers, picking each of them is independent,
and the probability of neither being saved is at most $1/4$. 
If they are in
the same layer, one of them is necessarily picked, which implies that the
probability of neither being saved is $0$. In any case, the probability is at most $1/4$.

In the second case, at least one of the vertices $u'$, $v'$ is in a layer with one
2-special vertex. W.~l.~o.~g.~let $u'$ be in such a
layer. By Lemma~\ref{one-and-one}, we know that the probability that $u'$
is not saved is at most $1/3$. 
Therefore, 

\begin{align*}
P[u \text{ burns}] &= P[u \text{ not picked} \wedge u' \text{ burns} \wedge v' \text{ burns}] \\
                   &= P[u \text{ not picked}]\cdot P[u' \text{ burns} \wedge v' \text{ burns}] \\
                   &\leq P[u \text{ not picked}]\cdot P[u' \text{ burns}] \\
                   &\leq \frac12 \cdot \frac13 
                    = \frac16\\
P[u \text{ is saved}] &\geq \frac56
\end{align*}

The proof for both cases works analogously for $v$.
\end{proof}

\end{appendixextra}

\subsection{Ruling Out the Gap Instances in Section~\ref{sec:gap}}  
\label{sec:well-separable}

In this section, we show that the integrality gap instances for (LP-1) presented in the previous section admit a better than $(1-1/e)$ approximation via (LP').
To this end, we introduce the concept of well-separable LP solutions  and show an improved rounding algorithm for solutions in this class.  


Let $\eta \in (0,1)$. 
Given an LP solution $(x,y)$ for (LP-1) or (LP'), we say that a vertex $v$ is $\eta$-light if $\sum_{u \in P_v \setminus \set{v}} x_u < \eta$; if a vertex $v$ is not $\eta$-light, we say that it is $\eta$-heavy. 
A fractional solution is said to be $\eta$-separable if for all layer $j$, either all vertices in $L_j$ are $\eta$-light, or they are all $\eta$-heavy.  
For an $\eta$-separable LP solution $(x,y)$, each layer $L_j$ is either an $\eta$-light layer that contains only $\eta$-light vertices, or $\eta$-heavy layer that contains only $\eta$-heavy vertices.

\begin{observation} 
The LP solution in Section~\ref{sec:gap} is $\eta$-separable for all values of $\eta \in \set{1/k, 2/k, \ldots, 1}$.  
\end{observation}   

\begin{theorem}
If the LP solution $(x,y)$ is $\eta$-separable for some $\eta$, then there is an efficient algorithm that produces an integral solution of cost $(1-1/e + f(\eta))\sum_{v} y_v$, where $f(\eta)$ is some function depending only on $\eta$.
\label{thm:wellseparable}
\end{theorem}  

\vspace{0.1in} 

\textbf{Algorithm:}
Let $T$ be an input tree, and $(x,y)$ be a solution for (LP') on $T$ that is $\eta$-separable for some constant $\eta \in (0,1)$.
Our algorithm proceeds in two phases.  
In the first phase, it performs
randomized rounding independently for each $\eta$-light layer.
Denote by $V_1$ the (random) collection of vertices selected in this phase. 
Then,
in the second phase, our algorithm performs randomized rounding conditioned on the solutions in
the first phase. 
In particular, when we process each $\eta$-heavy layer $L_j$, let $\tilde L_j$ be the collection of vertices that have not yet been saved by $V_1$.
We sample one vertex $v \in \tilde L_j$ from the distribution $\left\{\frac{x_v}{x(\tilde L_j)}\right\}_{v \in \tilde L_j}$. 
Let $V_2$ be the set of vertices chosen from the second phase. 
This completes the description of our algorithm. 
\begin{appendixextra}{sec:newlp}{\subsection{Analysis of Theorem \ref{thm:wellseparable}}}

For notational simplification, we present the proof when $\eta = 1/2$. It will be relatively obvious that the proof can be generalized to work for any $\eta$.  
Now we argue that each terminal $t \in \xset$ is saved with probability at least $(1-1/e +\delta)y_t$ for some universal constant $\delta >0$ that depends only on $\eta$. 
We will need the following simple observation that follows directly by standard probabilistic analysis. 


\begin{proposition}
\label{prop:LP-frac}  
For each vertex $v \in V(T)$, the probability that $v$ is not saved is at most $\prod_{u \in P_v}(1-x_u) \geq 1- e^{-y_v}$. 
\end{proposition} 
 
We start by analyzing two easy cases.  

\begin{lemma} 
\label{newlp:specialcases}
Consider $t \in \xset$. If $y_t < 0.9$ or there is some ancestor $v \in P_t$ such that $x_v > 0.2$, then the probability that $v$ is saved by the algorithm is at least $(1-1/e+\delta) y_t$.  
\end{lemma}  
\begin{proof} 
First, let us consider the case where $y_t < 0.9$. The probability of $t$
being saved is at least $1-e^{-y_v}$, according to the straightforward analysis.
If $y_t < 0.9$, we have $1-e^{-y_t}/y_t > 1.04(1-1/e)y_t$ as desired. 

Consider now the second case when $x_v > 0.2$ for some ancestor $v \in P_t$. 
The bound used
typically in the analysis is only tight when the values are all small, and,
therefore, we get an advantage when one of the values is relatively big. In
particular,
\begin{align*}
\pr{}{t \text{ is saved}} &\geq 1 - \prod_{u \in P_t} (1-x_u) \\
&\geq 1 - (1-x_v)e^{-(y_t-x_v)} \\
&\geq 1 - (1-0.2)e^{-(y_t-0.2)} \\
&\geq 1.01(1-1/e)y_v
\end{align*}
\end{proof} 

From now on, we only consider those terminals $t \in\xset$ such that $y_t \geq 0.9$ and $x_v < 0.2$, for all $v \in
P_t$. 
We remark here that if the value of $\eta$ is not $1/2$, we can easily pick other suitable thresholds instead of $0.9$ and $0.2$.  

For each vertex $v \in V$, let $\xset_1 \subseteq \xset$ be the set of terminals that are saved by $V_1$, i.e. a vertex $t \in \xset_1$ if and only if $t$ is a descendant of some vertex in $V_1$.  
Let $\xset_2 \subseteq \xset \setminus \xset_1$ contain the set of terminals that are not saved by the first phase, but are saved by the second phase, i.e. $t \in \xset_2$ if and only if $t$ has some ancestor in $V_2$.  
\[\pr{V_1, V_2}{t \not \in \xset_1 \cup \xset_2} = \pr{V_1, V_2}{t \not \in \xset_1}
                                  \pr{V_1, V_2}{t \not \in \xset_2 : t \not \in \xset_1}\]

\newcommand{\avgsumgoodterm}[0]{\ensuremath{\sum_{i \in \lset_{t,2}}x_{v_{i,t}} x(\tilde L_{i,t})}}

For any terminal $t$, let $\sset'_t$ and $\sset''_t$ be the sets of ancestors of $t$ that are $\eta$-light and $\eta$-heavy respectively, i.e. ancestors in $\sset'_t$ and $\sset''_t$ are considered by the algorithm in Phase 1 and 2 respectively. 
By Proposition~\ref{prop:LP-frac}, we can upper bound the first term by $e^{-x(\sset'_t)}$. 
In the rest of this section, we show that the second term is upper bounded by $e^{-x(\sset''_t)} c$ for some $c <1$, and therefore $\pr{}{t \not \in \xset_1 \cup \xset_2} \leq c e^{-x(\sset'_t) - x(\sset''_t)} \leq c e^{-y_t}$, as desired.   

The following lemma is the main technical tool we need in the analysis.
We remark that this lemma is the main difference between (LP') and (LP-2).  
\begin{lemma}
\label{newlp:advantage}
Let $t \in \xset$ and  $L_j$ be a
layer containing some $\eta$-heavy ancestor of $t$.  
Then 
\[{\mathbb E}_{V_1} [x(\tilde L_j) \mid t \not \in \xset_1] \leq  \alpha \]
for $\displaystyle{\alpha = \frac12 + (1-e^{-1/2}) \leq 0.9}$ 
\end{lemma}

Intuitively, this lemma says that any terminal that is still not saved by the result of the first phase will have a relatively ``sparse'' layer above it.   
We defer the proof of this lemma to the next subsection. 
Now we proceed to complete the analysis.  


For each vertex $v$, denote by $\ell(v)$ the layer to which vertex $v$ belongs. 
For a fixed choice of $V_1$, we say that terminal $t$ is {\em partially protected by $V_1$} if 
$\sum_{ v \in \sset''_t} x_v x(\tilde L_{\ell(v)}) \leq C x(\sset''_t)$ (we will choose the value of $C \in (\alpha,1)$ later).
Let $\xset' \subseteq \xset \setminus \xset_1$ denote the subset of terminals that are partially protected by $V_1$.  

\begin{claim}
\label{improv:layer:alt:goodterm}
For any $t \in \xset$,
$\pr{V_1}{t \in \xset' \mid t \not \in \xset_1} \geq 1- \alpha/C$. 
\end{claim}

\begin{proof} 
By linearity of expectation and Lemma \ref{newlp:advantage}, 
\[\expect{V_1}{\sum_{v \in \sset''_t} x_v x(\tilde L_{\ell(v)}) \mid t \not \in \xset_1} = \sum_{v \in \sset''_t} x_v \expect{V_1}{x(\tilde L_{\ell(v)}) \mid t \not \in \xset_1} \leq \alpha x(\sset''_t)\]
Using Markov's inequality, 
\begin{align*}
&\pr{V_1}{\sum_{v \in \sset''_t} x_v x(\tilde L_{\ell(v)}) \leq C x(\sset''_t) \mid t \not \in \xset_1} \\
  &\quad\quad\quad\quad= 1 - \pr{}{\sum_{v \in \sset''_t} x_v x(\tilde L_{\ell(v)})   > C x(\sset''_t) \mid t \not \in \xset_1} \\
  &\quad\quad\quad\quad\geq 1 - \frac{\alpha x(\sset''_t)}{C x(\sset''_t)} \\
  &\quad\quad\quad\quad= 1 - \frac{\alpha }{C}
\end{align*}
\end{proof} 


We can now rewrite the probability of a terminal $t \in \xset$ not being saved by the solution after the second phase.
\begin{align*}
&\pr{V_1, V_2}{t \not \in \xset_2 \mid  t \not \in \xset_1 } \\
&= \pr{}{ t \in \xset' \mid  t \not \in \xset_1}\,\pr{}{t \not \in \xset_2\mid  t \in \xset'} 
        + \pr{}{t \not \in \xset' \mid t \not \in \xset_1}\,\pr{}{ t \not \in \xset_2 \mid t \not \in \xset'}\\
&\leq (1-\alpha/C) \pr{V_1, V_2}{t \not \in \xset_2  \mid t \in \xset'} 
       + \frac{\alpha}{C} \cdot e^{-x(\sset''_t)}
\end{align*}

The last inequality holds because $\pr{V_1, V_2}{t \not \in \xset_2 \mid t \not\in \xset'}$ is at most $e^{-x(\sset''_t)}$ from Proposition~\ref{prop:LP-frac}.
 
It remains to provide a better
upper bound for $\pr{}{t \not \in \xset_2 \mid t \in \xset'}$. 
Consider a vertex $v \in \sset''_t$ that is involved in the second phase rounding.  
We say that vertex $v$  is \emph{good} for $t$ and $V_1$ if $x(\tilde L_{\ell(v)}) \leq C'$ (we will choose the value $C' \in (C,1)$ later.)  
Denote by $\sset^{good}_t \subseteq \sset''_t$ the set of good ancestors of $t$. 
The following claim ensures that good ancestors have large LP-weight in total. 

\begin{claim}
For any node $t \in \xset'$, $x(\sset^{good}_t) =  \sum_{v \in \sset^{good}_t} x_{v} \geq (1-C/C') x(\sset''_t)$. 
\end{claim}

\begin{proof}
Suppose (for contradiction) that the fraction of good layers was less than $1-C/C'$. This means that $x(\sset''_t \setminus \sset^{good}_t) \geq C/C'$.
For each such $v \in \sset''_t \setminus \sset^{good}_t$, we have $x(\tilde L(v)) > C'$.  
Then,
$\sum_{v \in \sset''_t} x_v x(\tilde L_{\ell(v)}) > \sum_{v\in \sset''_t \setminus \sset^{good}_t} x_v C' \geq C$. 
This contradicts the assumption that $t$ is partially protected, and concludes our proof.
\end{proof}

Now the following lemma follows.  

\newcommand{\qqqq}{\quad\quad\quad\quad}
\begin{lemma}
\label{lem:partial}  
$\pr{V_1, V_2}{t \not \in \xset_2 \mid t \in \xset'} \leq  e^{-x(\sset''_t)} e^{-(1-\frac{C}{C'}) x(\sset''_t) \paren{\frac{1}{C'}-1}}$ 
\end{lemma}

\begin{proof} 
\begin{align*}
&\pr{V_1, V_2}{t \not \in \xset_2 \mid t \in \xset'} \\
&\qqqq=    \sum_{V^\prime_1: t \in \xset'} \pr{V_1}{V_1 = V^\prime_1}\,\pr{V_2}{t \not \in \xset_2 \mid V_1 = V^\prime_1} \\
&\qqqq\leq \sum_{V^\prime_1: t \in \xset'} \pr{V_1}{V_1 = V^\prime_1}\,
          \prod_{\text{bad }v \in \sset''_t}\paren{1-x_{v}}\,
          \prod_{\text{good }v \in \sset''_t}\paren{1-\frac{x_{v}}{C'}} \\
&\qqqq\leq \sum_{V^\prime_1: t \in \xset'} \pr{V_1}{V_1 = V^\prime_1}\,
          \prod_{\text{bad } v \in \sset''_t} e^{-x_{v}}\,
          \prod_{\text{good }v \in \sset''_t} e^{-x_{v}/C'} \\
&\qqqq\leq \sum_{V^\prime_1: t \in \xset'} \pr{V_1}{V_1 = V^\prime_1}\,
          e^{-x(\sset''_t)\frac{C}{C'} - (1-C/C') x(\sset''_t) / C'} \\
&\qqqq\leq e^{-x(\sset''_t)\frac{C}{C'} - (1-C/C')x(\sset''_t) / C'}\,
          \sum_{V^\prime_1: t \in \xset'} \pr{V_1}{V_1 = V^\prime_1} \\
&\qqqq\leq e^{-x(\sset''_t)\frac{C}{C'} - (1-C/C') x(\sset''_t) / C'} \\
&\qqqq\leq e^{-x(\sset''_t)} e^{-(1-C/C') x(\sset''_t) \paren{\frac1{C'}-1}}
\end{align*}
\end{proof} 
 
Now we choose the parameters $C$ and $C'$ such that $C = (1+\delta) \alpha$, $C' = (1+\delta)C$, and $(1+\delta)C' = 1$, where $(1+\delta)^3 = 1/\alpha$.
Notice that this choice of parameters satisfy our previous requirements that $\alpha < C < C' <1$. 
The above lemma then gives the upper bound of $e^{-x(\sset''_t)} e^{-\frac{\delta^2}{1+\delta} x(\sset''_t)}$, which is at most $e^{-(1+\delta^2/2) x(\sset''_t)}$.  
Since $\delta>0$ is a constant, notice that we do have an advantage over the standard LP rounding in this case.
Now we plug in all the parameters to obtain the final result.   

\begin{align*}
\pr{V_1, V_2}{t \not \in \xset_1 \cup \xset_2} 
&= \pr{V_1, V_2}{t \not\in \xset_1}\,\pr{V_1, V_2}{t \not \in \xset_2 \mid t \not\in \xset_1} \\
&\leq e^{-x(\sset'_t)}\,\paren{(1-\alpha/C) \, \pr{V_1, V_2}{t \not \in \xset_2 \mid t \in \xset'} + \frac{\alpha}{C} \,e^{-x(\sset''_t)}} \\
&\leq e^{-x(\sset'_t)}\,
        \paren{(1-\alpha/C) \, e^{-x(\sset''_t)} e^{-\frac{\delta^2}{2} x(\sset''_t)}  + \frac{\alpha}{C} \, e^{-x(\sset''_t)}} \\
&\leq e^{-y_t}\,\paren{(1-\alpha/C) \, e^{- \frac{\delta^2}{2}x(\sset''_t) } + \alpha/C } \\
& \leq  e^{-y_t} \paren{\frac{\delta}{1+\delta} e^{-(1+\delta^2/2)x(\sset''_t)} + \frac{1}{1+\delta}} 
\end{align*}

Since we assume that $y_t > 0.9$ and $x_v \leq 0.2$, we must have $x(\sset''_t) \geq 0.2$,
 and therefore the above term can be seen as $e^{-y_t} \cdot \delta'$ for some $\delta' <1$.
Overall, the approximation factor we get is $(1-\delta'/e)$ for some universal constant $\delta' \in (0,1)$.  

\subsubsection{Proof of Lemma~\ref{newlp:advantage}}
For each $u$, let $\eset_u$ denote the event that $u$ is not saved by $V_1$.  
First we break the expectation term into $\sum_{u \in L_j} x_u \pr{}{\eset_u \mid t \not \in \xset_1}$.  
Let $v \in L$ be the ancestor of $t$ in layer $L_j$.  
We break down the sum further based on the ``LP coverage'' of the least common ancestor between $u$ and $v$, as follows: 
\[\sum_{i=0}^{k/2} \sum_{u\in L_j: q'(lca(u,v))=i} x_u \pr{}{\eset_u \mid t \not\in \xset_1} \]

Here, $q'(u)$ denotes $k \cdot x(P_u)$; this term is integral since we consider the $1/k$-integral solution $(x,y)$.  
The rest of this section is devoted to upper bounding the term $\pr{}{\eset_u \mid t \not \in \xset_1}$. 
The following claim gives the bound based on the level $i$ to which the least common ancestor belongs.

\begin{claim} 
\label{newlp:advantage:probpair}
For each $u \in L_j$ such that $q'(lca(u,v)) = i$, 
\[\pr{}{\eset_u \mid t \not \in \xset_1} \leq e^{-\paren{1/2 -i/k}}\]
\end{claim}

\begin{proof}
First, we recall that $y_u \geq 1/2$ and $q'(u) \geq k/2$, since $u$ is in the $1/2$-heavy layer $L_j$.  
Let $w= lca(u,v)$ and $P'$ be the path that connects $w$ to $u$.
Moreover, denote by $S \subseteq P'$ the set of light vertices on the path $P'$, i.e. $S = \sset'_t \cap P'$.   
Notice that $x(S) \geq \sum_{a \in \sset'_t \cap P_u} x_a - \sum_{a \in P_w} x_a  \geq (1/2 - i/k)$.  

For each $w' \in S$, $\pr{}{w' \not \in V_1 \mid t \not \in \xset_1} \in \set{1-x_{w^\prime}, 1- x_{w^\prime}/(1-x_{v^\prime})}$ depending on whether there is a vertex $v^\prime$ in $P_v$ that shares a layer with $w'$. 
In any case, it holds that $\pr{}{w' \not \in V_1 \mid t \not \in \xset_1} \leq (1-x_{w^\prime})$.
This implies that 
\begin{align*}
\pr{}{\eset_u \mid t \not \in \xset_1 } &\leq \prod_{w' \in S} \pr{}{w' \not \in V_1 \mid t \not \in \xset_1}  \\
&\leq \prod_{w' \in S} (1-x_{w^\prime}) \\
&\leq \prod_{w' \in S} e^{-x_{w^\prime}} \\
&\leq e^{-(1/2-i/k)} 
\end{align*}
\end{proof} 

\begin{claim} 
\label{newlp:advantage:nelems}
Let $i$ be an integer and $L^\prime \subseteq L_j$ be the set of vertices $u$ such that $q'(lca(u,v))$ is at least $i$. Then $x(L^\prime) \leq (k-i)/k$.
\end{claim} 

\begin{proof} 
This claim is a consequence of Hartke's constraints.
Let $v'$ be the topmost ancestor of $v$ such that $q'(v') \geq i$. 
We remark that all vertices in $L'$ must be descendants of $v'$, so it must be that $\sum_{w \in P_{v'}} x_w + x(L^\prime) \leq 1$. 
The first term is $i/k$, implying that $x(L') \leq (k-i)/k$.
\end{proof} 

Let $L^i_j \subseteq L_j$ denote the set of vertices $u$ whose least common ancestor $lca(u,v)$ satisfies $q'(lca(u,v)) = i$.
As a consequence of Claim \ref{newlp:advantage:nelems}, $\sum_{i' \geq i} x(L^{i'}_j) \leq (k-i)/k$.  
Combining this inequality with Claim \ref{newlp:advantage:probpair}, we get that
\[\expect{}{x(\tilde L_j) \mid t \not \in \xset_1} \leq \sum_{i=0}^{k/2} x(L_j^i) e^{-1/2+i/k}\]

This term is maximized when $x(L_j^{k/2}) =1/2$ and $x(L_j^i) = 1/k$ for all other $i=0,1,\ldots, k/2-1$. 
This implies that 
\[\expect{}{x(\tilde L_j) \mid t \not \in \xset_1} \leq 1/2 + \sum_{i=0}^{k/2-1} e^{-1/2+i/k}/k\]

Finally, using some algebraic manipulation and the fact that $1+x \leq e^x$, we get  
\begin{align*}
\expect{}{x(\tilde L_j) \mid t \not \in \xset_1} 
&\leq 1/2 + \sum_{i=0}^{k/2-1} e^{-1/2+i/k}/k \\
&= 1/2 + \frac{1}{k}\, e^{-1/k}\,\frac{1-e^{-1/2}}{1-e^{-1/k}} \\
&= 1/2 + (1-e^{-1/2})\, \frac1{e^{1/k}}\, \frac{1/k}{1-e^{-1/k}} \\
&= 1/2 + (1-e^{-1/2})\, \frac{1/k}{e^{1/k}-1} \\
&\leq 1/2 + (1-e^{-1/2})
\end{align*}

\end{appendixextra}


\subsection{Integrality Gap for (LP')}
\label{sec:gap-newlp}  
In this section, we present an instance where (LP') has an integrality gap of
$5/6+\eps$, for any $\eps > 0$.
Interestingly, this instance admits an optimal $\frac{1}{2}$-integral LP solution.  

{\bf Gadget:}
The motivation of our construction is a simple gadget represented in
Figure \ref{12intgap:gadget}. 
In this instance, vertices are either {\em special} (colored gray) or {\em regular}. 
This gadget has three properties of our interest: 

\begin{itemize}

\item If we assign an LP-value of $x_v= 1/2$ to every special vertex, then this is a feasible LP solution that ensures $y_u = 1$ for all leaf $u$.  

\item For any integral solution $\uset$ that does not pick any vertex in the first layer of this gadget, at most $2$ out of $3$ leaves of the gadget are saved. 

\item Any pair of special vertices in the same layer do not have a common ancestor inside this gadget. 

\end{itemize}

\begin{wrapfigure}[13]{r}{0.25\textwidth}
\centering
\includegraphics{figures/fig2}
\caption{Gadget used to get $5/6$ integrality gap. Special vertices are colored gray.}
\label{12intgap:gadget}
\end{wrapfigure}

Our integrality gap instance is constructed by creating partially overlapping copies of this gadget.
We describe it formally below.   


\vspace{0.1in} 

\noindent {\bf Construction:}
The first layer of this instance, $L_1$, contains 4 nodes: two special nodes, which we
name $a(1)$ and $a(2)$, and two regular nodes, which we name $b(1)$ and $b(2)$.
We recall the definition of spider from Section~\ref{sec:good-gadget}.  


Let $\alpha = 5\left\lceil 1/\eps \right\rceil$. The nodes $b(1)$ and $b(2)$ are
the roots of two spiders. 
Specifically, the spider $Z_1$ rooted at $b(1)$ has $\alpha$ feet, with one foot per layer, in consecutive layers $L_2,\ldots, L_{\alpha+1}$. For each $j \in [\alpha]$, denote by $b'(1,j)$, the $j^{th}$ foot of spider $Z_1$.  
The spider $Z_2$, rooted at $b(2)$, has $\alpha^2$ feet, with one foot per layer, in layers $L_{\alpha+2}, \ldots, L_{\alpha^2+\alpha+1}$. 
For each $j \in [\alpha^2]$, denote by $b'(2,j)$, the $j^{th}$ foot of spider $Z_2$.  
All the feet of spiders $Z_1$ and $Z_2$ are special vertices. 

For each $j \in [\alpha]$, the node $b'(1,j)$ is also the root of spider $Z'_{1,j}$, with $\alpha^2$ feet, lying in the $\alpha^2$ consecutive layers $L_{2+\alpha + j \alpha^2},\ldots, L_{1+\alpha+ (j+1) \alpha^2}$ (one foot per layer).
For $j' \in [\alpha^2]$, let $b''(1,j,j')$ denote the $j'$-th foot of spider $Z'_{1,j}$ that lies in layer $L_{1+\alpha+j\alpha^2 +j'}$.  
Notice that we have $\alpha^3$ such feet of these spiders $\set{Z'_{1,j}}_{j=1}^{\alpha}$ lying in layers $L_{2+\alpha+\alpha^2}, \ldots, L_{1+\alpha+\alpha^2+\alpha^3}$. 
Similarly, for each $j \in [\alpha^2]$, the node $b'(2,j)$ is the root of spider $Z'_{2,j}$ with $\alpha^2$ feet, lying in consecutive layers $L_{2+\alpha+\alpha^3 + j\alpha^2}, \ldots, L_{1+\alpha+ \alpha^3 + (j+1)\alpha^2}$. 
We denote by $b''(2,j,j')$ the $j'$-th foot of this spider.  

The special node $a(1)$ is also the root of spider $W_1$ which has $\alpha + \alpha^3$ feet: The first $\alpha$ feet, denoted by $a'(1,j)$ for $j \in [\alpha]$, are aligned with the nodes $b'(1,j)$, i.e. for each $j \in [\alpha]$, the foot $a'(1,j)$ of spider $W_1$ is in the same layer as the foot $b'(1,j)$ of $Z_1$. 
For each $j \in [\alpha], j' \in [\alpha^2]$, we also have a foot $a''(1,j,j')$ which is placed in the same layer as $b''(1,j,j')$.
Similarly, the special node $a(2)$ is the root of spider $W_2$ having $\alpha^2 + \alpha^4$ feet. 
For $j \in [\alpha^2]$, spider $W_2$ has a foot $a'(2,j)$ placed in the same layer as $b'(2,j)$. 
For $j \in [\alpha^2], j' \in [\alpha^2]$, $W_2$ also has a foot $a''(2,j,j')$ in the layer of $b''(2,j,j')$.  
All the feet of both $W_1$ and $W_2$ are special vertices.

Finally, for $i \in \set{1,2}$, and $j \in [\alpha^i]$, each node $a'(i,j)$ has $\alpha^{5-i}$ children, which are leaves of the instance.  
For $j \in [\alpha], j' \in [\alpha^2]$, the nodes $b''(i,j,j')$, $a''(i,j,j')$ have $\alpha^{3-i}$ children each which are also leaves of the instance. 
The set of terminals $\xset$ is simply the set of leaves. 

\begin{proposition} 
We have $|\xset|  = 6 \alpha^5$.
Moreover, 
(i) the number of terminals in subtrees $T_{a(1)} \cup T_{b(1)}$ is $3 \alpha^5$, and (ii) the number of terminals in subtrees $T_{a(2)} \cup T_{b(2)}$ is $3 \alpha^5$.  
\label{prop:56intgapanalysis}
\end{proposition}  

\begin{appendixproof}{sec:newlp}{\subsection{Analysis of Proposition \ref{prop:56intgapanalysis}}}
Each node $a'(1,j)$ has $\alpha^4$ children, and there are $\alpha$ such nodes. 
Similarly, each node $a'(2,j)$ has $\alpha^3$ children. There are $\alpha^2$ such nodes. 
This accounts for $2 \alpha^5$ terminals. 

For $i \in \set{1,2}$, each node $a''(i,j,j')$ has $\alpha^{3-i}$ children. 
There are $\alpha^{i+2}$ such nodes. 
This accounts for another $2 \alpha^5$ terminals. 
Finally, there are $\alpha^{3-i}$ children of each $b''(i,j,j')$, and there are $\alpha^{2+i}$ such nodes. 



\end{appendixproof} 


\vspace{0.1in} 

\noindent{\bf Fractional Solution:} 
Our construction guarantees that any path from root to leaf contains $2$ special vertices: For a leaf child of $a'(i,j)$, its path towards the root must contain $a'(i,j)$ and $a(i)$. 
For a leaf child of $a''(i,j,j')$, its path towards the root contains $a''(i,j,j')$ and $a(i)$. 
For a leaf child of $b''(i,j,j')$, the path towards the root contains $b''(i,j,j')$ and $b'(i,j)$.   

  


\begin{lemma}
\label{lem:feas}  
For each special vertex $v$, for each layer $L_j$ below $v$, the set $L_j \cap T_v$ contains at most one special vertex. 
\end{lemma} 

\begin{appendixproof}{sec:newlp}{\subsection{Proof of Lemma \ref{lem:feas}}}
Each layer contains two special vertices of the form $\set{a'(i,j), b'(i',j')}$ or $\set{a''(i,j), b''(i',j')}$.
In any case, the least common ancestor of such two special vertices in the same layer is always the root $s$ (since one vertex is in $T_{a(i)}$, while the other is in $T_{b(i)}$)
This implies that, for any non-root vertex $v$, the set $L_j \cap T_v$ can contain at most one special vertex. 
\end{appendixproof}

Notice that, there are at most two special vertices per layer.  
We define the LP solution $x$, with $x_v =1/2$ for all special vertices $v$ and $x_v = 0$ for all other vertices. 
It is easy to verify that this is a feasible solution.  

We now check the constraint at $v$ and layer $L_j$ below $v$: If the sum $\sum_{u \in P_v} x_u = 0$, then the constraint is immediately satisfied, because $\sum_{ u \in L_j \cap T_v} x_u \leq 1$. 
If $\sum_{u \in P_v} x_u = 1/2$, let $v'$ be the special vertex ancestor of $v$. 
Lemma~\ref{lem:feas} guarantees that $\sum_{u \in L_j \cap T_v} x_u \leq \sum_{u \in L_j \cap T_{v'}} x_u \leq 1/2$, and therefore the constraint at $v$ and $L_j$ is satisfied. 
Finally, if $\sum_{u \in P_v} x_u = 1$, there can be no special vertex below $v$ and therefore $\sum_{u \in L_j \cap T_v} x_u = 0$.  

\vspace{0.1in} 

\noindent {\bf Integral Solution:} 
We argue that any integral solution cannot save more than $(1+5/\alpha) 5 \alpha^5$ terminals. The following lemma is the key to our analysis. 

\begin{lemma} 
Any integral solution $\uset: \uset \cap \set{a(1), b(1)} =\emptyset$ saves at most $(1+5/\alpha) 5 \alpha^5$ terminals.
\label{intgap56:int1}
\end{lemma} 
\begin{appendixproof}{sec:newlp}{\subsection{Proof of Lemma \ref{intgap56:int1}}}
Consider the set $Q= \set{a'(1,j)}_{j=1}^{\alpha} \cup \set{b'(1,j)}_{j=1}^{\alpha}$, and a collection of paths from $\set{a(1),b(1)}$ to vertices in set $Q$.  
These paths are contained in the layers $L_1, \ldots, L_{\alpha+1}$, so the strategy $\uset$ induces a cut of size at most $\alpha+1$ on them. 
This implies that at most $\alpha+1$ vertices (out of $2\alpha$ vertices in $Q$) can be saved by $\uset$. 
Let $\tilde Q \subseteq Q$ denote the set of vertices that have not been saved by $\uset$. We remark that $|\tilde Q| \geq \alpha - 1$.
We write $\tilde Q = \tilde Q_a \cup \tilde Q_b$ where $\tilde Q_a$ contains the set of vertices $a'(1,j)$ that are not saved, and $\tilde Q_b = \tilde Q \setminus \tilde Q_a$.  
For each vertex in $\tilde Q_a$, at least $\alpha^4 -1$ of its children cannot be saved, so we have at least $(\alpha^4 -1) |\tilde Q_a| \geq \alpha^4 |\tilde Q_a| - \alpha$ unsaved terminals that are descendants of $\tilde Q_{a}$.
If $|\tilde Q_b| \leq 3$, we are immediately done: We have $|\tilde Q_a| \geq \alpha-4$, so $(\alpha^4-1)(\alpha-4) \geq \alpha^5 - 5\alpha^4$ unsaved terminals.  
 
Consider the set 
$$R = \paren{\bigcup_{j \in [\alpha],j' \in [\alpha^2]} \set{a''(1,j,j')}} \cup \paren{\bigcup_{j: b'(1,j) \in \tilde Q_b} \bigcup_{j' \in [\alpha^2]} \set{b''(1,j,j')}}$$  
This set satisfies $|R| = \alpha^3 + |\tilde Q_b| \alpha^2$, and the paths connecting vertices in $R$ to $\tilde Q_b \cup \set{a(1)}$ lie in layers $L_1,\ldots, L_{\alpha^3 + \alpha^2 + \alpha+ 1}$.
So the strategy $\uset$ induced on these paths disconnects at most $\alpha^3+ \alpha^2 + \alpha+1$ vertices.
Let $\tilde R \subseteq R$ contain the vertices in $R$ that are not saved by $\uset$, so we have $|\tilde R| \geq (|\tilde Q_b|-1)  \alpha^2 - \alpha - 1$, which is at least $(|\tilde Q_b| - 2) \alpha^2$.  
Each vertex in  $\tilde R$ has $\alpha^2$ children. We will have $(\alpha^2-1)$ unsaved terminals for each such vertex, resulting in a total of at least $(\alpha^2 -1) (|\tilde Q_b|-2)\alpha^2 \geq \alpha^4 |\tilde Q_b| - 4 \alpha^4$ terminals that are descendants of $b(1)$.   

In total, by summing the two cases, at least $(\alpha^4 |\tilde Q_a| - \alpha) + (\alpha^4 |\tilde Q_b| - 4 \alpha^4) \geq (|\tilde Q_a| + |\tilde Q_b|)\alpha^4 - 5\alpha^4 \geq \alpha^5 - 5\alpha^4$ terminals are not saved by $\uset$, thus concluding the proof.  
\end{appendixproof} 

\begin{lemma} 
Any integral solution $\uset: \uset \cap \set{a(2), b(2)} =\emptyset$ saves at most $(1+5/\alpha) 5 \alpha^5$ terminals.
\label{intgap56:int2}
\end{lemma} 


Since nodes $a(1)$, $a(2)$, $b(1)$, $b(2)$ are in the first layer, it is only
possible to save one of them. Therefore, either Lemma \ref{intgap56:int1} or
Lemma \ref{intgap56:int2} apply, which concludes the analysis.
