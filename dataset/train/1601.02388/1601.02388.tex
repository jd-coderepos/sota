
\label{sec:newlp}

\newcommand{\vset}{{\mathcal{V}}}
Due to the integrality gap result in the previous section, there is no hope to improve the best known algorithms via the canonical LP relaxation. 
Hartke~\cite{Hartke2006} suggested adding the following constraints to narrow down the integrality gap of the LP. 
 

We write the new LP with these constraints below: 
\vspace{0.1in}

\noindent
\framebox[\textwidth]{ 
\begin{minipage}[b]{0.99\textwidth}
\vspace{-1em}

\end{minipage}
}

\begin{proposition} 
\label{prop:setyvalues}
Given the values  that satisfy the first set of constraints, then the solution  defined by  is feasible for (LP') and at least as good as any other feasible .  
\end{proposition} 

In this section, we study the power of this LP and provide three evidences that it may be stronger than (LP-1). 

\subsection{New properties of extreme points}
In this section, we show that Finbow et al.~tractable instances~\cite{FinbowG09} admit a polynomial time exact algorithm via (LP') (in fact, any optimal extreme point for (LP') is integral.) 
In contrast, we show that (LP-1) contains an extreme point that is not integral.   



We first present the following structural lemma. 
\begin{lemma}
\label{lem: extreme}  
Let  be an optimal extreme point for (LP') on instance  rooted at . Suppose  has two children, denoted by  and .  
Then .  
\end{lemma}  

\begin{appendixproof}{sec:newlp}{\subsection{Proof of Lemma \ref{lem: extreme}}}
Suppose that . 
We will define two solutions  and  and derive that  can be written as a convex combination of  and , a contradiction.

First, we define  by setting . For each vertex , we set . For each vertex , we define .
We verify that  is feasible for (LP'): For each  and any layer  below ,  (the last inequality is due to the fact that  is feasible).
The constraint is obviously satisfied for all .  
For the root node , we have .  

We define  analogously: . For each vertex , we set , and for each , we define . 
It can be checked similarly that  is a feasible solution.  

\begin{claim} 
If  is an optimal extreme point, then . 
\label{claim:extremepoint}
\end{claim} 

\begin{proof}Observe that, for each ,  and for each , .  
The objective value of  is .
Similarly, the objective value of solution  is .
 
Consider the convex combination . 
This solution is feasible and has the objective value of 

If , we apply the fact that  to get the objective of strictly more than , contradicting the fact that  is optimal.  
\end{proof} 

Finally, we define the convex combination by .
It can be verified easily that  for all .  
\end{appendixproof} 

\vspace{0.1in} 

\noindent {\bf Finbow et al. Instances:} In this instance, the tree has degree at most  and the root has degree .  
Finbow et al.~\cite{FinbowG09} showed that this is polynomial time solvable. 

\begin{wrapfigure}[11]{r}{0.25\textwidth}
\centering
\includegraphics{figures/fig4}
\caption{Instance with a non-integral extreme point for (LP-1). Gray vertices: ; otherwise: .}
\label{fig:lp1badcase}
\end{wrapfigure}

\begin{theorem} 
Let  be an input instance where  has degree at most  and  has degree two. Let  be a feasible fractional solution for (LP-3). 
Then there is a polynomial time algorithm that saves at least  vertices. 
\label{thm:finbowinst}
\end{theorem}  
\begin{appendixproof}{sec:newlp}{\subsection{Proof of Theorem \ref{thm:finbowinst}}}
We prove this by induction on the number of nodes in the tree that, for any tree  that is a Finbow et al. instance, for any fractional solution  for (LP'), there is an integral solution  such that .  
Let  and  be the children of the root .
From Lemma~\ref{lem: extreme}, assume w.l.o.g. that , so we have . 
By the induction hypothesis, there is an integral solution  for the subtree  such that .  
The solution  can be extended to the instance  by defining . 
This solution has the objective value of , completing the proof. 
\end{appendixproof}

\noindent {\bf Bad instance for (LP-1):} 
We show in Figure \ref{fig:lp1badcase} a Finbow et al. instance as well as a
solution for (LP-1) that is optimal and an extreme point, but not integral.

\begin{appendixextra}{sec:newlp}{\subsection{Analysis of Figure \ref{fig:lp1badcase}}}
\begin{claim}

The solution  represented in Figure \ref{fig:lp1badcase}, with  defined according to
Proposition \ref{prop:setyvalues}, is an extreme point of this instance for
(LP-1).

\end{claim}

\begin{proof}

Suppose (for contradiction) that  is not an extreme point. Then, there
are distinct solutions ,  and  such that
. Since  and , then , and likewise, .
Combining that  with  and
, we conclude that . Similarly, we get that
, which implies that .

Similar reasoning using that  allows us to conclude that
, and thus, , which
contradicts our assumption.
\end{proof}
\end{appendixextra}


\subsection{Rounding \texorpdfstring{}{1/2}-integral Solutions}
\label{sec:half-int}
We say that the LP solution  is -integral if, for all , we have  for some integer .  
By standard LP theory, one can assume that the LP solution is -integral for some polynomially large integer . 



In this section, we consider the case when  (-integral LP solutions).  
From Theorem~\ref{thm: gap 1/k}, (LP-1) is not strong enough to obtain a  approximation algorithm, for any . 
Here, we show a  approximation algorithm based on rounding (LP').  

\begin{theorem} 
Given a solution  for (LP') that is -integral, there is a polynomial time algorithm that produces a solution of cost .  
\label{thm:12intalgo}
\end{theorem} 

We believe that the extreme points in some interesting special cases will be -integral. 

\vspace{0.1in} 

\noindent {\bf Algorithm's Description:} 
Initially, . 
Our algorithm considers the layers  in this order.
When the algorithm looks at layer , it picks a vertex  and adds it to , as follows. 
Consider , where .
Let  contain vertices  such that there is no ancestor of  that belongs to  for some , and , i.e. for each , there is another vertex  for some  such that  is an ancestor of .
We choose the vertex  based on the following rules: 
\begin{itemize} 
\item If there is only one , such that  is not saved by  so far, choose . 

\item Otherwise, if , pick  at random from  with uniform probability.
Similarly, if , pick  at random from .  

\item Otherwise, we have the case . In this case, we pick vertex  from  with probability ; otherwise, we take from .  
\end{itemize}   


\begin{appendixextra}{sec:newlp}{\subsection{Analysis of Theorem \ref{thm:12intalgo}}}
\textbf{Analysis: }    
Below, we argue that each vertex  is saved with probability at least . 
It is clear that this implies the theorem: Consider a vertex .
If , we are immediately done. Otherwise, consider the bottommost ancestor  of  such that . 
Since , the probability that  is saved is the same as that of , which is at least .    

We analyze a number of cases.
Consider a layer  such that . 
Such a vertex  is saved with probability . 

Next, consider a layer  such that . 
Each vertex  is saved with probability  and . 
So, in this case, the probability of saving  is more than .  

\begin{lemma}
\label{one-and-one} 
Let  be the layer such that . 
Then the vertex  is saved with probability  and vertex  is saved with probability . 
\end{lemma} 

\begin{proof} 
Let  be the ancestor of  in some layer above . 
The fact that  has not been saved means that  is not picked by the algorithm, when it processed . 

We prove the lemma by induction on the value of . 
For the base case, let  be the first layer such that .
This means that the layer  must have , and therefore the probability of  being saved is at least . 
Vertex  is not saved only if both  and  are not picked, and this happens with probability .
Hence, vertex  is saved with probability  as desired.   
Consider now the base case for vertex , which is not saved only if  is not saved and  is picked by the algorithm among .
This happens with probability , thus completing the proof of the base case.   


 For the purpose of induction, we now assume that, for all layer  above  such that , the probability that the algorithm saves the vertex in  is at least .
Since the vertex  is not saved only if  is not saved, this probability is either  or  depending on the layer to which  belongs. If it is , we are done; otherwise, the probability is at most .   
Now consider vertex , which is not saved only if  is not saved and  is picked at . This happens with probability at most .  
\end{proof} 


\begin{lemma}
Let  be a layer such that  (containing two vertices). Then each such vertex is saved with probability at least . 
\end{lemma}

\begin{proof}

Let  and  be the ancestors of  and  in some sets  and  above the layer .
There are the two possibilities: either both  and  are in layers with  (maybe ); or  is in the layer with . We remark that : otherwise, the LP constraint for  and  would not be satisfied.

For  or  to be unsaved, we need that both  and  are not saved by the algorithm.  
Otherwise, if, say,  is saved,  is also saved, and the algorithm would have picked .



It must be that ,
since either  and  are in different layers or they are in the same
layer. 
If they are in different layers, picking each of them is independent,
and the probability of neither being saved is at most . 
If they are in
the same layer, one of them is necessarily picked, which implies that the
probability of neither being saved is . In any case, the probability is at most .

In the second case, at least one of the vertices ,  is in a layer with one
2-special vertex. W.~l.~o.~g.~let  be in such a
layer. By Lemma~\ref{one-and-one}, we know that the probability that 
is not saved is at most . 
Therefore, 



The proof for both cases works analogously for .
\end{proof}

\end{appendixextra}

\subsection{Ruling Out the Gap Instances in Section~\ref{sec:gap}}  
\label{sec:well-separable}

In this section, we show that the integrality gap instances for (LP-1) presented in the previous section admit a better than  approximation via (LP').
To this end, we introduce the concept of well-separable LP solutions  and show an improved rounding algorithm for solutions in this class.  


Let . 
Given an LP solution  for (LP-1) or (LP'), we say that a vertex  is -light if ; if a vertex  is not -light, we say that it is -heavy. 
A fractional solution is said to be -separable if for all layer , either all vertices in  are -light, or they are all -heavy.  
For an -separable LP solution , each layer  is either an -light layer that contains only -light vertices, or -heavy layer that contains only -heavy vertices.

\begin{observation} 
The LP solution in Section~\ref{sec:gap} is -separable for all values of .  
\end{observation}   

\begin{theorem}
If the LP solution  is -separable for some , then there is an efficient algorithm that produces an integral solution of cost , where  is some function depending only on .
\label{thm:wellseparable}
\end{theorem}  

\vspace{0.1in} 

\textbf{Algorithm:}
Let  be an input tree, and  be a solution for (LP') on  that is -separable for some constant .
Our algorithm proceeds in two phases.  
In the first phase, it performs
randomized rounding independently for each -light layer.
Denote by  the (random) collection of vertices selected in this phase. 
Then,
in the second phase, our algorithm performs randomized rounding conditioned on the solutions in
the first phase. 
In particular, when we process each -heavy layer , let  be the collection of vertices that have not yet been saved by .
We sample one vertex  from the distribution . 
Let  be the set of vertices chosen from the second phase. 
This completes the description of our algorithm. 
\begin{appendixextra}{sec:newlp}{\subsection{Analysis of Theorem \ref{thm:wellseparable}}}

For notational simplification, we present the proof when . It will be relatively obvious that the proof can be generalized to work for any .  
Now we argue that each terminal  is saved with probability at least  for some universal constant  that depends only on . 
We will need the following simple observation that follows directly by standard probabilistic analysis. 


\begin{proposition}
\label{prop:LP-frac}  
For each vertex , the probability that  is not saved is at most . 
\end{proposition} 
 
We start by analyzing two easy cases.  

\begin{lemma} 
\label{newlp:specialcases}
Consider . If  or there is some ancestor  such that , then the probability that  is saved by the algorithm is at least .  
\end{lemma}  
\begin{proof} 
First, let us consider the case where . The probability of 
being saved is at least , according to the straightforward analysis.
If , we have  as desired. 

Consider now the second case when  for some ancestor . 
The bound used
typically in the analysis is only tight when the values are all small, and,
therefore, we get an advantage when one of the values is relatively big. In
particular,

\end{proof} 

From now on, we only consider those terminals  such that  and , for all . 
We remark here that if the value of  is not , we can easily pick other suitable thresholds instead of  and .  

For each vertex , let  be the set of terminals that are saved by , i.e. a vertex  if and only if  is a descendant of some vertex in .  
Let  contain the set of terminals that are not saved by the first phase, but are saved by the second phase, i.e.  if and only if  has some ancestor in .  


\newcommand{\avgsumgoodterm}[0]{\ensuremath{\sum_{i \in \lset_{t,2}}x_{v_{i,t}} x(\tilde L_{i,t})}}

For any terminal , let  and  be the sets of ancestors of  that are -light and -heavy respectively, i.e. ancestors in  and  are considered by the algorithm in Phase 1 and 2 respectively. 
By Proposition~\ref{prop:LP-frac}, we can upper bound the first term by . 
In the rest of this section, we show that the second term is upper bounded by  for some , and therefore , as desired.   

The following lemma is the main technical tool we need in the analysis.
We remark that this lemma is the main difference between (LP') and (LP-2).  
\begin{lemma}
\label{newlp:advantage}
Let  and   be a
layer containing some -heavy ancestor of .  
Then 

for  
\end{lemma}

Intuitively, this lemma says that any terminal that is still not saved by the result of the first phase will have a relatively ``sparse'' layer above it.   
We defer the proof of this lemma to the next subsection. 
Now we proceed to complete the analysis.  


For each vertex , denote by  the layer to which vertex  belongs. 
For a fixed choice of , we say that terminal  is {\em partially protected by } if 
 (we will choose the value of  later).
Let  denote the subset of terminals that are partially protected by .  

\begin{claim}
\label{improv:layer:alt:goodterm}
For any ,
. 
\end{claim}

\begin{proof} 
By linearity of expectation and Lemma \ref{newlp:advantage}, 

Using Markov's inequality, 

\end{proof} 


We can now rewrite the probability of a terminal  not being saved by the solution after the second phase.


The last inequality holds because  is at most  from Proposition~\ref{prop:LP-frac}.
 
It remains to provide a better
upper bound for . 
Consider a vertex  that is involved in the second phase rounding.  
We say that vertex   is \emph{good} for  and  if  (we will choose the value  later.)  
Denote by  the set of good ancestors of . 
The following claim ensures that good ancestors have large LP-weight in total. 

\begin{claim}
For any node , . 
\end{claim}

\begin{proof}
Suppose (for contradiction) that the fraction of good layers was less than . This means that .
For each such , we have .  
Then,
. 
This contradicts the assumption that  is partially protected, and concludes our proof.
\end{proof}

Now the following lemma follows.  

\newcommand{\qqqq}{\quad\quad\quad\quad}
\begin{lemma}
\label{lem:partial}  
 
\end{lemma}

\begin{proof} 

\end{proof} 
 
Now we choose the parameters  and  such that , , and , where .
Notice that this choice of parameters satisfy our previous requirements that . 
The above lemma then gives the upper bound of , which is at most .  
Since  is a constant, notice that we do have an advantage over the standard LP rounding in this case.
Now we plug in all the parameters to obtain the final result.   



Since we assume that  and , we must have ,
 and therefore the above term can be seen as  for some .
Overall, the approximation factor we get is  for some universal constant .  

\subsubsection{Proof of Lemma~\ref{newlp:advantage}}
For each , let  denote the event that  is not saved by .  
First we break the expectation term into .  
Let  be the ancestor of  in layer .  
We break down the sum further based on the ``LP coverage'' of the least common ancestor between  and , as follows: 


Here,  denotes ; this term is integral since we consider the -integral solution .  
The rest of this section is devoted to upper bounding the term . 
The following claim gives the bound based on the level  to which the least common ancestor belongs.

\begin{claim} 
\label{newlp:advantage:probpair}
For each  such that , 

\end{claim}

\begin{proof}
First, we recall that  and , since  is in the -heavy layer .  
Let  and  be the path that connects  to .
Moreover, denote by  the set of light vertices on the path , i.e. .   
Notice that .  

For each ,  depending on whether there is a vertex  in  that shares a layer with . 
In any case, it holds that .
This implies that 

\end{proof} 

\begin{claim} 
\label{newlp:advantage:nelems}
Let  be an integer and  be the set of vertices  such that  is at least . Then .
\end{claim} 

\begin{proof} 
This claim is a consequence of Hartke's constraints.
Let  be the topmost ancestor of  such that . 
We remark that all vertices in  must be descendants of , so it must be that . 
The first term is , implying that .
\end{proof} 

Let  denote the set of vertices  whose least common ancestor  satisfies .
As a consequence of Claim \ref{newlp:advantage:nelems}, .  
Combining this inequality with Claim \ref{newlp:advantage:probpair}, we get that


This term is maximized when  and  for all other . 
This implies that 


Finally, using some algebraic manipulation and the fact that , we get  


\end{appendixextra}


\subsection{Integrality Gap for (LP')}
\label{sec:gap-newlp}  
In this section, we present an instance where (LP') has an integrality gap of
, for any .
Interestingly, this instance admits an optimal -integral LP solution.  

{\bf Gadget:}
The motivation of our construction is a simple gadget represented in
Figure \ref{12intgap:gadget}. 
In this instance, vertices are either {\em special} (colored gray) or {\em regular}. 
This gadget has three properties of our interest: 

\begin{itemize}

\item If we assign an LP-value of  to every special vertex, then this is a feasible LP solution that ensures  for all leaf .  

\item For any integral solution  that does not pick any vertex in the first layer of this gadget, at most  out of  leaves of the gadget are saved. 

\item Any pair of special vertices in the same layer do not have a common ancestor inside this gadget. 

\end{itemize}

\begin{wrapfigure}[13]{r}{0.25\textwidth}
\centering
\includegraphics{figures/fig2}
\caption{Gadget used to get  integrality gap. Special vertices are colored gray.}
\label{12intgap:gadget}
\end{wrapfigure}

Our integrality gap instance is constructed by creating partially overlapping copies of this gadget.
We describe it formally below.   


\vspace{0.1in} 

\noindent {\bf Construction:}
The first layer of this instance, , contains 4 nodes: two special nodes, which we
name  and , and two regular nodes, which we name  and .
We recall the definition of spider from Section~\ref{sec:good-gadget}.  


Let . The nodes  and  are
the roots of two spiders. 
Specifically, the spider  rooted at  has  feet, with one foot per layer, in consecutive layers . For each , denote by , the  foot of spider .  
The spider , rooted at , has  feet, with one foot per layer, in layers . 
For each , denote by , the  foot of spider .  
All the feet of spiders  and  are special vertices. 

For each , the node  is also the root of spider , with  feet, lying in the  consecutive layers  (one foot per layer).
For , let  denote the -th foot of spider  that lies in layer .  
Notice that we have  such feet of these spiders  lying in layers . 
Similarly, for each , the node  is the root of spider  with  feet, lying in consecutive layers . 
We denote by  the -th foot of this spider.  

The special node  is also the root of spider  which has  feet: The first  feet, denoted by  for , are aligned with the nodes , i.e. for each , the foot  of spider  is in the same layer as the foot  of . 
For each , we also have a foot  which is placed in the same layer as .
Similarly, the special node  is the root of spider  having  feet. 
For , spider  has a foot  placed in the same layer as . 
For ,  also has a foot  in the layer of .  
All the feet of both  and  are special vertices.

Finally, for , and , each node  has  children, which are leaves of the instance.  
For , the nodes ,  have  children each which are also leaves of the instance. 
The set of terminals  is simply the set of leaves. 

\begin{proposition} 
We have .
Moreover, 
(i) the number of terminals in subtrees  is , and (ii) the number of terminals in subtrees  is .  
\label{prop:56intgapanalysis}
\end{proposition}  

\begin{appendixproof}{sec:newlp}{\subsection{Analysis of Proposition \ref{prop:56intgapanalysis}}}
Each node  has  children, and there are  such nodes. 
Similarly, each node  has  children. There are  such nodes. 
This accounts for  terminals. 

For , each node  has  children. 
There are  such nodes. 
This accounts for another  terminals. 
Finally, there are  children of each , and there are  such nodes. 



\end{appendixproof} 


\vspace{0.1in} 

\noindent{\bf Fractional Solution:} 
Our construction guarantees that any path from root to leaf contains  special vertices: For a leaf child of , its path towards the root must contain  and . 
For a leaf child of , its path towards the root contains  and . 
For a leaf child of , the path towards the root contains  and .   

  


\begin{lemma}
\label{lem:feas}  
For each special vertex , for each layer  below , the set  contains at most one special vertex. 
\end{lemma} 

\begin{appendixproof}{sec:newlp}{\subsection{Proof of Lemma \ref{lem:feas}}}
Each layer contains two special vertices of the form  or .
In any case, the least common ancestor of such two special vertices in the same layer is always the root  (since one vertex is in , while the other is in )
This implies that, for any non-root vertex , the set  can contain at most one special vertex. 
\end{appendixproof}

Notice that, there are at most two special vertices per layer.  
We define the LP solution , with  for all special vertices  and  for all other vertices. 
It is easy to verify that this is a feasible solution.  

We now check the constraint at  and layer  below : If the sum , then the constraint is immediately satisfied, because . 
If , let  be the special vertex ancestor of . 
Lemma~\ref{lem:feas} guarantees that , and therefore the constraint at  and  is satisfied. 
Finally, if , there can be no special vertex below  and therefore .  

\vspace{0.1in} 

\noindent {\bf Integral Solution:} 
We argue that any integral solution cannot save more than  terminals. The following lemma is the key to our analysis. 

\begin{lemma} 
Any integral solution  saves at most  terminals.
\label{intgap56:int1}
\end{lemma} 
\begin{appendixproof}{sec:newlp}{\subsection{Proof of Lemma \ref{intgap56:int1}}}
Consider the set , and a collection of paths from  to vertices in set .  
These paths are contained in the layers , so the strategy  induces a cut of size at most  on them. 
This implies that at most  vertices (out of  vertices in ) can be saved by . 
Let  denote the set of vertices that have not been saved by . We remark that .
We write  where  contains the set of vertices  that are not saved, and .  
For each vertex in , at least  of its children cannot be saved, so we have at least  unsaved terminals that are descendants of .
If , we are immediately done: We have , so  unsaved terminals.  
 
Consider the set 
  
This set satisfies , and the paths connecting vertices in  to  lie in layers .
So the strategy  induced on these paths disconnects at most  vertices.
Let  contain the vertices in  that are not saved by , so we have , which is at least .  
Each vertex in   has  children. We will have  unsaved terminals for each such vertex, resulting in a total of at least  terminals that are descendants of .   

In total, by summing the two cases, at least  terminals are not saved by , thus concluding the proof.  
\end{appendixproof} 

\begin{lemma} 
Any integral solution  saves at most  terminals.
\label{intgap56:int2}
\end{lemma} 


Since nodes , , ,  are in the first layer, it is only
possible to save one of them. Therefore, either Lemma \ref{intgap56:int1} or
Lemma \ref{intgap56:int2} apply, which concludes the analysis.
