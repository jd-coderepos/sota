\documentclass[preprint]{sig-alternate}

\conferenceinfo{SAC'14,} {Mar. 24-28, 2014. Gyeongju, Korea}
\CopyrightYear{2014} 
\crdata{TBD} 
\clubpenalty=10000 
\widowpenalty = 10000 

\usepackage{comment,amsmath,amssymb,verbatim,url,bm}

\newtheorem{definition}{Definition}[section]
\newtheorem{theorem}{Theorem}[section]
\newtheorem{lemma}{Lemma}[section]
\newtheorem{cor}{Corollary}[section]
\newtheorem{pro}{Protocol}[section]
\newtheorem{con}{Construction}[section]
\date{\today}

\setpagenumber{1}
\pagenumbering{arabic}


\begin{document}

\toappear{Permission to make digital or hard copies of all or part of this work for personal or classroom use is granted without fee provided that copies are
not made or distributed for profit or commercial advantage and that copies
bear this notice and the full citation on the first page. To copy otherwise, to
republish, to post on servers or to redistribute to lists, requires prior specific permission and/or a fee.\\
This is a preprint version of our paper presented in SAC'14, March 24-28, 2014, Gyeongju, Korea.\\}

\title{Building Secure and Anonymous Communication Channel: Formal Model and its Prototype Implementation}

\author{
\numberofauthors{2}
\alignauthor
Keita Emura\\
       \affaddr{National Institute of Information and Communications Technology, Japan}\\
       \email{k-emura@nict.go.jp}
\alignauthor
Akira Kanaoka\\
       \affaddr{National Institute of Information and Communications Technology, and \\Toho University, Japan}\\
       \email{akira.kanaoka@is.sci.toho-u.ac.jp}
\and 
\alignauthor 
Satoshi Ohta\\
       \affaddr{National Institute of Information and Communications Technology, Japan}\\
       \email{sota@nict.go.jp}
\alignauthor Takeshi Takahashi\\
       \affaddr{National Institute of Information and Communications Technology, Japan}\\
       \email{takeshi\_takahashi@ieee.org}
}

\maketitle
\begin{abstract}
Various techniques need to be combined to realize anonymously authenticated communication.
Cryptographic tools enable anonymous user authentication while anonymous communication protocols hide users' IP addresses from service providers.
One simple approach for realizing anonymously authenticated communication is their simple combination, but this gives rise to another issue; how to build a secure channel.
The current public key infrastructure cannot be used since the user's public key identifies the user.
To cope with this issue, we propose a protocol that uses identity-based encryption for packet encryption without sacrificing anonymity, and group signature for anonymous user authentication.
Communications in the protocol take place through proxy entities that conceal users' IP addresses from service providers.
The underlying group signature is customized to meet our objective and improve its efficiency.
We also introduce a proof-of-concept implementation to demonstrate the protocol's feasibility.
We compare its performance to SSL communication and demonstrate its practicality, and conclude that the protocol realizes secure, anonymous, and authenticated communication between users and service providers with practical performance.
\end{abstract}




\category{C.2.0}{Computer-Communication Networks}{General}[Security and Protection]
\category{H.1.m}{Information System}{Models and Principles}[Miscellaneous]
\category{E.3}{Data}{Data Encryption}[Public key cryptosystems]

\terms{Security, Design, Theory}



\keywords{Anonymous Communication, Anonymous Authentication, Secure Channel, Identity-Based Encryption, Group Signature}



\section{Introduction}

Anonymity\footnote{This paper considers sender/prover anonymity and does not consider recipient anonymity.} is an important aspect of privacy, and systems that provide services to anonymous users are currently a topic of keen interest. 
Such systems can provide services to users without revealing their identity. 
To date, a great deal of studies have been reported on~\cite{SPA}, 
and many use cryptography as the important building block for constructing the systems, 
but these need further improvement before they can be used for actual services.
To realize secure, anonymous, and authenticated communication, these building blocks need to collaborate with each other.

\subsection{Research Background}


Several cryptographic primitives providing anonymity have been proposed. 
Among them is group signature~\cite{[ChaumH91]}, which enables a signer to anonymously prove signatures' validity.
A group manager (GM) that has a pair of a group public key, , and master secret key, , issues a secret signing key, , to a user , which computes a group signature,  (on certain messages), using . 
No user-dependent value is required in the verification phase; a verifier verifies  using only the corresponding . 
These approaches alone, however, cannot guarantee anonymity when applied to online communication.
For instance, let a signer compute a group signature and \emph{send} it to a verifier. 
The verifier can anonymously verify the signature's validity. 
However, there is a question of how to anonymously send the group signature to the verifier. 
Usually, a source IP address is included in a packet, and that reveals the identity of the sender, thus user anonymity is already infringed upon.
The situation remains the same regardless of the primitives we implement so long as direct communication between a sender (signer, prover, etc) and a receiver (verifier, etc) is required. 


The user's IP address is naturally visible in the IP packets sent from the user, and it cannot simply be erased or forged to enable bi-directional communication. 
One approach for this is using intermediate agents that send packets on behalf of the actual user terminal, and several such protocols have already been proposed~\cite{SPA}, including Tor~\cite{TorProject}. 
Nevertheless, another issue arises in the question of how we can assure user's legitimacy. 
We need to discern legitimate and illegitimate users to restrict unauthorized access to the channel.
One might think that only end-to-end authentication is needed, but it is hard to authenticate users without identifying them.
For instance, a server needs to send a response code to a user in basic authentication and the user needs to return a user ID and password.
That is, the server needs to identify the user.
Moreover, it seems to be hard to send a certain message back from the server to a user since the corresponding source IP address is generally required. 
Authentication by an intermediate agent (as in Tor~\cite{TorProject}) might be a solution to these problems.
The agent can authenticate a user and can hide the user's source IP address from the server. 
That notwithstanding, we still need to know how the server can directly authenticate end users. 


A simple approach to the anonymous authentication problems is just combining both cryptographic primitives and anonymous communication protocols as follows. 
Let a user compute an anonymously-authenticated token (e.g., group signature), 
and send it to a server via an anonymous channel (e.g., using Tor). 
Then, the server can directly authenticate the user without compromising anonymity. 
However, another problem arises is how we can establish a secure channel (i.e., flowed data is encrypted). 
If the server uses a user's public key (certified by a trusted Certificate Authority (CA) in a public key infrastructure (PKI)), 
then server identifies the user, since a certificate contains information on the key holder. 
The same problem arises even if symmetric key encryption is used. 
Assume that the server tries to exchange a secret key with an end user.
Since the server does not know who the actual end user is, the server does not know the user's public key for running a key exchange protocol. 

\subsection{Our Contribution}

We propose a protocol that realizes secure, anonymous, and authenticated communication.
The proposed protocol uses identity-based encryption (IBE) to encrypt content without identifying key holders\footnote{The conventional public key encryption (PKE) with certain non-interactive zero-knowledge (NIZK) proofs may also be applicable, where a user chooses a temporary public key for each session, and makes a NIZK proof of the corresponding secret key. We do not consider this construction anymore since the NIZK proof must be constructed from scratch by considering algebraic structures of the underlying PKE scheme, and this may lead to some difficulty of its implementation. }.
It can set arbitrary values on public keys, thus it can enable a user to select a temporary ID for each session, which the server can use as a public key.
It also uses group signature for anonymous user authentication. 
Communications in the protocol take place through proxy entities that conceal users' IP addresses from service providers (SPs). 

This paper gives the framework of the proposed protocol, gives a formal model and security definitions of the proposed protocol, points out the needlessness of the group signature's open capability for our usage, 
and then proposes an open-free variant of the Furukawa-Imai group signature scheme~\cite{[FurukawaI06]}. 
The modification can reduce its signature size by 50\% compared to the original scheme.
Note that if someone needs to identify an illegitimate user, we can add such a mechanism without relying on cryptographic techniques; e.g., an IP address managed by the proxy. 

We demonstrate the feasibility and practicality of the proposed protocol by introducing our proof-of-concept implementation.
The implementation uses the modified group signature scheme mentioned above.
It also uses the Boneh-Franklin IBE scheme~\cite{[BonehF03]} for its underlying IBE scheme. 

Note that, our protocol in this paper focuses on encrypted communication from the SP to users. 
It can easily be extended to interactive secure communication since SP is not anonymous to users and each user thus can simply use SP's public key for building a secure channel.

\subsection{Related Work}

There exist similar attempts to our approach.
Sudarsono et al.~\cite{[SudarsonoNNF10]} has considered an anonymous IEEE802.1X authentication system by using a group signature scheme.
They use group signatures as the client's digital certificate.
The means of sending such certificates over IP networks was, however, outside its scope. 
Lee et al.~\cite{anon-pass13} proposed an anonymous subscription service, called Anon-pass.
Their construction methodology is similar to group signatures, wherein a user proves the possession of signatures using zero-knowledge proofs.
Though Anon-pass does not consider end-to-end secure (encrypted) communication, our protocol does. 

Gilad and Herzberg~\cite{[GiladH13]} also considered how to distribute public keys using an anonymous service.
They consider two peers, a querier and a responder.
The querier specifies a random ephemeral public key that is not certified by the CA, 
and sends a query containing this public key to a responder via an anonymous service, like Tor. 
The responder replies with a response message encrypted by the (anonymous) querier's ephemeral public key.
However, a responder cannot check whether a public key is a valid key or a random value since this scheme gives no certification of the public key, and moreover the responder cannot detect even if the pubic key is replaced by an attacker. 
Moreover, no anonymous user authentication is considered in the Gilad-Herzberg system. 
In our protocol, the SP can be convinced that a public key (i.e., a temporary ID) will work, since arbitrary values can be public keys in IBE systems. 
Moreover, since a temporary ID is signed by group signature, 
we can prevent the key replacement attack and can achieve anonymous user authentication, simultaneously. 

Proxy re-encryption (PRE) (e.g.,~\cite{[LibertV11]}) 
is another candidate. 
In our context, first users compute re-encryption keys using their secret key and the SP public key, and the SP only computes ciphertext using its own public key. Then, the proxy can re-encrypt ciphertexts. 
However, the proxy needs to manage all re-encryption keys, and therefore it is difficult to assume that no private information is infringed on even if the proxy is corrupted after the communication. Moreover, there is a possibility that other user may decrypt unexpected ciphertexts, since the proxy manages many re-encryption keys (from the SP to each user). 
It is undesirable to generate re-encryption keys that can be used in an unexpected manner, even if the proxy is modeled as an honest-but-curious entity and always follows the protocol. 
In our protocol no unexpected user (including the proxy) can decrypt ciphertexts, since a unique temporary ID is assigned for each user and each session. Note that the key escrow problem happens as an outcome of IBE, where key generation center (KGC) can decrypt all ciphertexts. However, KGC is modeled as a trusted third party, whereas it is difficult to fully trust all proxies involved in the systems. 





\section{Framework}

Figure \ref{fig1.eps} depicts the framework of the proposed protocol, 
which has five roles; User, Proxy, Service Provider, GM, and KGC.
A {\bf User} wishes to communicate with an SP without revealing its identity.
The {\bf Proxy} assists communication between a User and SP by relaying packets without revealing the User's IP address.
We assume that it is honest-but-curious.
The {\bf SP} provides services to Users, but wishes to authenticate them.
It does not care about their identities but needs to confirm whether the user accessing it is legitimate.
The {\bf GM} manages a group key and issuer key, and issues a signing key for a user that is used for generating an anonymously-authenticated token.
We assume that the GM suitably authenticates a user before issuing the signing key. 
The {\bf KGC} generates a decryption key for a user. 
We assume that the KGC suitably authenticates a user before issuing the decryption key. 

\begin{figure}[t]
\centering
{\includegraphics[scale=.31]{fig1.eps}}
\caption{Framework of the Proposed Protocol}\label{fig1.eps}
\end{figure}

These roles need to collaborate with each other to realize the proposed secure anonymous authentication.
Their interaction sequence is as follows.
(1) A user (whose IP address is ) chooses a temporary ID , (2) computes a group signature  on , and (3) sends  to the proxy.
(4) The proxy associates  with this temporary ID, and (5) forwards  to the SP.
(6) The SP can directly authenticate the users by verifying the group signature without compromising anonymous communications.
(7) If the user is successfully verified, the SP encrypts content using TempID as the public key of IBE; otherwise it returns .
Here, we apply an IBE's property to establish a secure channel between the SP and an anonymous user, where arbitrary values can be a public key, and a ciphertext can be independently computed with the generation of the corresponding decryption key\footnote{This property is used in timed-release encryption~\cite{[CheonHKO08]} context, where an encryptor can control when ciphertexts will be decrypted.}. 
(8) The SP sends this IBE ciphertext to the proxy, which again (9) forwards it to the corresponding user. (10) Finally, the user decrypts the IBE ciphertext using the corresponding decryption key issued by the KGC. 
After relaying the (mutual) communication, the proxy immediately deletes the corresponding pair of . 
Therefore, no private information is infringed on even if the proxy is corrupted after the communication. 
Note that Figure~\ref{fig1.eps} explains one-pass communication.
The proxy can reuse the information of the pair  of a session so long as the session is alive, but it removes the information from its registry once the session is closed. 
In either case, the proxy immediately deletes the corresponding pair  after the session.
Moreover, we can easily extend one-proxy setting to multi-proxy setting, since all the proxy has to do is (1) manage , and (2) forwarding  to the next. 
The above framework is embodied as a concrete protocol in the following section.




\section{Authentication Protocol}

This section proposes a secure anonymous authentication protocol.\footnote{Note that our protocol achieves to send an anonymous token (group signature) and it is not an authentication protocol in the strict sense. Nevertheless, we can easily extend it to an authentication protocol (via the classical challenge-and-response methodology) as follows: first a SP sends a random nonce to a User (via the Proxy) and the User computes a group signature whose signed message contains the nonce. Therefore, we do not further consider the extension in this paper.}
It first defines the syntax of the protocol, 
and then gives its construction. 
We consider a scenario in which an SP is modeled as a server, which provides a service only for a legitimate user. 
That is, we can assume that the GM has authenticated a user before issuing a signing key, and the SP can judge that the user who can generate a valid group signature is a legitimate user. 

\subsection{Syntax and Security Definitions}

Let  and  be an identity space and message space, respectively, 
and , , and  stand for IP address of User, Proxy, and SP, respectively. 
For a set  and an element , }{\leftarrow}XxX{\sf GM.Setup}\lambdagpkik{\sf KGC.Setup}\lambdaparamsmsk{\sf Join}gpkiksk{\sf UserKeyGen}paramsmsk\mathtt{TempID}\in\mathcal{ID}dk_{\mathtt{TempID}}{\sf SendRequest}gpksk\mathtt{TempID}\mathtt{Adr_{src}}\mathtt{Adr_{dst}}\mathtt{Adr_{proxy}}\sigma\mathtt{TempID}\mathtt{Adr_{dst}}\mathtt{Adr_{proxy}}{\sf RelayRequest}\mathtt{Adr_{src}}\mathtt{Adr_{dst}}\mathtt{Tbl}\sigma\mathtt{TempID}(\sigma,\mathtt{TempID})\mathtt{Adr_{proxy}}\mathtt{Adr_{dst}}(\mathtt{TempID}, \mathtt{Adr_{src}})\mathtt{Tbl}{\sf ValidityCheck}gpk\sigma\mathtt{TempID}\sigma{\sf SendContent}gpk\sigma\mathtt{TempID}M\in\mathcal{M}\mathtt{Adr_{proxy}}C\sigmaC\mathtt{Adr_{proxy}}\sigma\bot{\sf RelayContent}C\mathtt{Tbl}C\mathtt{Adr_{src}}\mathtt{Tbl}(\mathtt{TempID}, \mathtt{Adr_{src}})\mathtt{Tbl}C{\sf GetContent}Cdk_{\mathtt{TempID}}M\sigma(gpk,ik)\leftarrow {\sf GM.Setup}\allowbreak(1^\lambda)(params,msk)\leftarrow {\sf KGC.Setup}(1^\lambda)sk\leftarrow \linebreak {\sf Join}(gpk,ik)\mathtt{TempID}\in\mathcal{ID}M\in\mathcal{M}(\mathtt{Adr_{src}},\mathtt{Adr_{dst}},\allowbreak \mathtt{Adr_{proxy}})(\sigma,\mathtt{TempID},\mathtt{Adr_{dst}})\leftarrow{\sf SendRequest}(gpk,sk,\mathtt{TempID},\allowbreak \mathtt{Adr_{src}},\allowbreak \mathtt{Adr_{dst}},\mathtt{Adr_{proxy}})(\sigma,\mathtt{TempID},\mathtt{Adr_{proxy}})\leftarrow{\sf RelayRequest}(\mathtt{Adr_{src}},\allowbreak \mathtt{Adr_{dst}},\allowbreak \mathtt{Tbl},\sigma, \mathtt{TempID})C\leftarrow {\sf SendContent}(gpk,\sigma,\allowbreak \mathtt{TempID},\allowbreak M,\allowbreak \mathtt{Adr_{proxy}}){\sf SendRequest}{\sf GetContent}{\sf SendRequest}\rightarrow {\sf RelayRequest} \rightarrow {\sf SendContent} \rightarrow \allowbreak{\sf RelayContent} \rightarrow {\sf GetContent}\mathcal{A}\mathtt{Adr_{src}}\mathcal{A}\mathcal{A}msk\mathcal{A}\mathtt{TempID}\mathtt{Adr_{src}}\mathtt{TempID}\mathcal{C}(gpk,ik)\leftarrow{\sf GM.Setup}(1^\lambda)(params,msk)\leftarrow {\sf KGC.Setup}(1^\lambda)sk_0,sk_1\leftarrow {\sf Join}(gpk,ik)gpksk_0sk_1(params,msk)\mathcal{A}\mathcal{C}\mathtt{Tbl}:=\emptyset\mathcal{A}{\sf SendRequest}(i,\mathtt{TempID})\allowbreak \in\{0,1\}\times\mathcal{ID}\mathcal{C}{\sf SendRequest}(gpk,\allowbreak sk_b,\mathtt{TempID},\allowbreak \mathtt{Adr_{src}},\allowbreak \mathtt{Adr_{dst}},\mathtt{Adr_{proxy}})\sigma{\sf SendRequest}\mathcal{A}\mathcal{A}{\sf RelayRequest}(\sigma,\mathtt{TempID})\mathcal{C}{\sf RelayRequest}(\mathtt{Adr_{src}},\allowbreak \mathtt{Adr_{dst}},\allowbreak \mathtt{Tbl},\allowbreak \sigma, \mathtt{TempID})\mathtt{Tbl}\mathcal{A}{\sf RelayContent}C\mathcal{C}{\sf RelayContent}(C,\allowbreak \mathtt{Tbl})\mathtt{Tbl}\mathcal{A}\mathtt{TempID}^\ast\in\mathcal{ID}\mathcal{C}\mathcal{C}b\stackrel{\, and runs 

\noindent and . 
 returns an arbitrary  to . 
 runs . 
Note that  can know the transcript of these algorithms executed by : , , and . 
 outputs . 
\end{enumerate}

\noindent
The protocol is said to have anonymity if  is negligible in .
\end{definition}

\noindent 
Next, we define semantic security which guarantees that no information of content  is revealed from the transcripts of algorithms. In this game, an adversary  is modeled as a malicious proxy. Moreover,  is allowed to obtain  in order to guarantee that no information of  is revealed even from the GM's viewpoint. 

\begin{definition}[Semantic Security]~
\begin{enumerate}
\setlength{\itemsep}{0em}\setlength{\parsep}{0em}
\item The challenger  runs  and , and gives , , and  to an adversary . 

\item  is allowed to issue the  query .  runs  and returns . 

\item  sends  and  to  as his choice, where  has not been sent as a  query.  flips a coin }{\leftarrow}\{0,1\}{\sf SendRequest}(gpk,\allowbreak sk^\ast,\mathtt{TempID}^\ast,\allowbreak \mathtt{Adr_{src}},\mathtt{Adr_{dst}},\mathtt{Adr_{proxy}})C^\ast\leftarrow {\sf SendContent}(gpk,\sigma,\allowbreak \mathtt{TempID},\allowbreak M^\ast_b,\mathtt{Adr_{proxy}})(\sigma,\mathtt{TempID}^\ast,\mathtt{Adr_{dst}})C^\ast\mathcal{A}\mathcal{A}{\sf UserKeyGen}\mathtt{TempID}\in\mathcal{ID}\mathtt{TempID}\neq \mathtt{TempID}^\ast\mathcal{C}{\sf UserKeyGen}(params,\allowbreak msk,\allowbreak \mathtt{TempID})dk_{\mathtt{TempID}}\mathcal{A}b^\prime\in\{0,1\}{\sf Adv}^{{\sf ss}}_{{\sf pro},\mathcal{A}}(\lambda):=|\Pr[b=b^\prime]-1/2|\lambda\mathcal{A}{\sf ValidityCheck}\mathcal{A}\mathcal{A}msk\mathcal{C}(gpk,ik)\leftarrow{\sf GM.Setup}(1^\lambda)(params,msk)\leftarrow {\sf KGC.Setup}(1^\lambda)gpkparamsmsk\mathcal{A}\mathcal{C}\mathcal{S}=\emptyset\mathcal{A}{\sf SendRequest}(i,\mathtt{TempID})sk_i\mathcal{C}sk_i\leftarrow {\sf Join}(gpk,ik)sk_i\mathcal{C}{\sf SendRequest}(gpk,\allowbreak sk_i,\allowbreak \mathtt{TempID},\allowbreak \mathtt{Adr_{src}},\allowbreak \mathtt{Adr_{dst}},\allowbreak \mathtt{Adr_{proxy}})\sigma\mathcal{A}\mathcal{C}(\sigma,\mathtt{TempID})\mathcal{S}\mathcal{A}(\sigma^\ast, \mathtt{TempID}^\ast)\mathcal{A}(\sigma^\ast, \mathtt{TempID}^\ast)\not\in\mathcal{S}{\sf ValidityCheck}(gpk,\sigma^\ast,\allowbreak \mathtt{TempID}^\ast)\allowbreak=1{\sf Adv}^{{\sf uf}}_{{\sf pro},\mathcal{A}}(\lambda):=\Pr[\mathcal{A}~\text{wins}]\lambda\mathcal{IBE}({\sf IBE.Setup},\allowbreak {\sf Extract}, {\sf IBE.Enc}, {\sf IBE.Dec})\mathcal{ID}\mathcal{M}\lambdaparamsmskparamsmskID\in\mathcal{ID}dk_{ID}paramsIDM\in\mathcal{M}C_{IBE}paramsC_{IBE}dk_{ID}M(params,\allowbreak msk)\leftarrow{\sf IBE.Setup}(1^\lambda)IDM\Pr[{\sf IBE.Dec}(params,\allowbreak {\sf IBE.Enc}(params,ID,M), {\sf Extract}(params,msk,ID))=M]=1\mathcal{GS}({\sf GS.Setup}, {\sf Join}, {\sf Sign}, {\sf Verify}){\sf Open}{\sf Judge}{\sf Judge}{\sf Open}{\sf Open}{\sf Judge}\lambdagpkikgpkikskgpkskM\sigmagpk\sigmaM1\sigmaM(gpk,ik)\allowbreak \leftarrow {\sf GS.Setup}(1^\lambda)sk\leftarrow {\sf GS.Join}(gpk,ik)\Pr[{\sf Verify}(gpk,\allowbreak {\sf Sign}(gpk,\allowbreak sk,M),M)=1]=1\mathtt{TempID}{\sf GM.Setup}(gpk,ik)\leftarrow{\sf GS.Setup}(1^\lambda)(gpk,\allowbreak ik){\sf KGC.Setup}(params,msk)\leftarrow{\sf IBE.Setup}(1^\lambda)(params,msk){\sf Join}sk\leftarrow {\sf GS.Join}(gpk,ik)sk{\sf UserKeyGen}dk_{\mathtt{TempID}}\leftarrow{\sf Extract}(params,msk,\mathtt{TempID})dk_{\mathtt{TempID}}{\sf SendRequest}\mathtt{TempID}\stackrel{\. Run , and send  to the proxy whose IP address is . 

\item[]: Append  to , and relays a pair  and  to the destination SP whose IP address is . 

\item[]: Output  if , and , otherwise. 

\item[]: Output  if . Otherwise, run , and send  to the proxy whose IP address is . 

\item[]: Relay  to a user whose IP address is  contained in . Moreover, remove  from . 



\item[]: Output the result of . 
\end{description}
\end{con}


Note that our the above construction only considers one-proxy setting, 
and therefore no anonymity is guaranteed from the viewpoint of the Proxy, 
since the Proxy directly relays communications between the User and SP. 
Note that this situation does not contradict our definition of anonymity (Def.~\ref{def:anon}). 
We can simply extend this protocol to a multi-proxy setting, where each Proxy relays  or  between the previous Proxy and the next Proxy. Then, anonymity is guaranteed even from the Proxies' point of view unless all Proxies collude with each other. 


\section{Group Signature}

The proposed secure anonymous authentication protocol uses a group signature scheme as its fundamental component. 
Though arbitrary group signature schemes could be used (i.e., by ignoring open functionality), it is beneficial to remove unnecessary functionality and improve performance efficiency, thus the proposed protocol in Section \ref{Sec:ProtocolConstruction} uses a group signature without open functionality. We call this an open-free group signature\footnote{A difference between ring signature~\cite{[RivestST01]} and open-free group signature is as follows. In ring signature schemes, a signer chooses a set of other members, and signs on behalf of the group of users. The anonymity of the signer cannot be revoked in contrast to group signature schemes. However, a signer needs to involve/choose other members when the signer signs, and therefore needs to know other members. This does not match our setting. }. 
This section defines the security of such signatures. 

\subsection{Defining Open-free Group Signature}\label{GS}

In this section, 
we redefine the security definitions of the Furukawa-Imai group signature scheme, anonymity, traceability, and non-frameability, to match the open-free variant. 
Anonymity guarantees that no adversary  can distinguish whether two signers of group signatures are the same or not, even if  has the corresponding signing keys. 
Usually, there are two kind of anonymity, CPA-anonymity and CCA-anonymity. 
In CCA-anonymity,  is allowed to issue open queries, where  sends , and is given the result of the  algorithm. 
Meanwhile, we do not have to consider these differences due to the open-free property. 

\begin{definition}[Anonymity]~
\begin{enumerate}
\setlength{\itemsep}{0em}\setlength{\parsep}{0em}
\item An adversary  with the security parameter  sends , , , and  to the challenger . 
\item  chooses }{\leftarrow}\{0,1\}\sigma^\ast\leftarrow {\sf Sign}(gpk,sk_b,M)\sigma^\ast\mathcal{A}\mathcal{A}b^\prime\in\{0,1\}\mathcal{GS}{\sf Adv}^{{\sf anon}}_{\mathcal{GS},\mathcal{A}}(\lambda):=|\Pr[b=b^\prime]-1/2|\lambda\mathcal{A}{\sf Open}{\sf Verify}\mathcal{A}\mathcal{C}(gpk,ik)\leftarrow{\sf GS.Setup}(1^\lambda)gpk\mathcal{A}\mathcal{A}(M,i)U_i\mathcal{C}{\sf GS.Join}sk_i\sigma\leftarrow {\sf Sign}(gpk,sk_i,M)\mathcal{A}U_i\mathcal{C}\sigma\leftarrow {\sf Sign}(gpk,sk_i,M)\mathcal{A}\mathcal{C}(\sigma,M)\mathcal{S}\mathcal{A}(\sigma^\ast,M^\ast)\mathcal{A}{\sf Verify}(gpk,\sigma^\ast,M^\ast)=1(\sigma^\ast,M^\ast)\not\in\mathcal{S}\mathcal{GS}{\sf Adv}^{{\sf un}}_{\mathcal{GS},\mathcal{A}}(\lambda):=\Pr[\mathcal{A}~\text{wins}]\lambda\mathcal{A}\mathcal{A}Uusksk{\sf GS.Join}uskuskusk{\sf GS.Join}{\sf GS.Join}{\sf Open}({\sf VK}, {\sf SK}){\sf SK}(\mathbb{G}_1,\mathbb{G}_2,\mathbb{G}_T)p\langle g_1\rangle=\mathbb{G}_1\langle g_2\rangle=\mathbb{G}_2e:\mathbb{G}_1\times\mathbb{G}_2\rightarrow \mathbb{G}_Ta,b\in\mathbb{Z}_pe(g_1^a,g_2^b)=e(g_1,g_2)^{ab}=e(g_1^b,g_2^a)e(g_1,g_2)\neq 1_{\mathbb{G}_T}1_{\mathbb{G}_T}\mathbb{G}_T\gamma\stackrel{\, and }{\leftarrow}\mathbb{G}_1W=g_2^\gammagpk=\linebreak(\mathbb{G}_1,\mathbb{G}_2,\mathbb{G}_T,e,g_1,g_2,h,W, e(g_1,g_2),e(g_1,W),H_3)ik\allowbreak =\gammaH_3:\{0,1\}^\ast\rightarrow \mathbb{Z}_pU_ix_i,y_i\stackrel{\, compute , and output . 

\item[{\sf Sign}:] Let . 
Choose }{\leftarrow}\mathbb{Z}_p\delta=\beta x-yT=A h^\betar_{x},r_{\delta}, r_\beta\stackrel{\, and compute , , , , and , and output . 

\item[{\sf Verify}:] Compute , and output 1 if  holds, and 0 otherwise. 
\end{description}
\end{con}

Compared to the original Furukawa-Imai scheme, 
we can reduce three DDH-hard group elements and three  elements. 
Accordingly, we can reduce the size of signature by 50\% compared to the original Furukawa-Imai group signature scheme. 






\section{Implementation and Discussion}

\subsection{Analysis on Proposed Protocol}

We can prove that our proposed protocol is secure if the underlying IBE scheme is IND-ID-CPA secure (like the Boneh-Franklin IBE scheme~\cite{[BonehF03]}) and the underlying group signature scheme is anonymous and unforgeable. 
We only give a sketch of proof of anonymity (other theorems can be similarly proved) here and omit the full proofs of the following theorems due to the page limitation. 

\begin{theorem}
Our protocol is anonymous if the underlying group signature scheme is anonymous. 
\end{theorem}

\begin{proof}[(Sketch)] Let  be an adversary who can break anonymity of our protocol. Then, we can construct an algorithm  that breaks anonymity of the underlying group signature scheme as follow. Let  be the challenger of the underlying group signature. 
 generates , , and , and generates all IBE-related values. Then  gives  to . In the challenge phase,  gets  from , forwards it to , and gets  from .  uses  as the output of the  algorithm, and similarly simulates other algorithms.  outputs  and  also outputs  as the guessing bit. Then,  can break anonymity of the group signature with the same advantage of . This contradicts that the underlying group signature is anonymous.
\end{proof}

\begin{theorem}
Our protocol is semantic secure if the underlying IBE scheme is IND-ID-CPA secure. 
\end{theorem}


\begin{theorem}
Our protocol is unforgeable if the underlying group signature scheme is unforgeable. 
\end{theorem}


\subsection{Analysis on Group Signature}

The remaining part is to show that the proposed open-free group signature scheme is anonymous and unforgeable. 
The proposed open-free group signature scheme is constructed from an (honest-verifier) zero-knowledge proof of knowledge 
by using the Fiat-Shamir conversion~\cite{[FiatS86]}. First, we explain the original proof of knowledge protocol as follows. 
A prover computes , and sends it to a verifier. The verifier sends a challenge value  to the prover. 
The prover computes , and sends it to the verifier. The verifier checks whether the verification equation 
holds or not. 
Next, we show that this 3-move protocol is zero-knowledge (this immediately leads to anonymity). 
The simulator chooses }{\leftarrow}\mathbb{G}\beta\stackrel{\, and computes . 
Note that  is chosen uniformly random. Therefore,  generated from the simulator is drawn from a distribution that is indistinguishable from the distribution output by any particular prover. 
For , the simulator chooses }{\leftarrow}\mathbb{Z}_pR=\frac{e(h,g_2)^{s_\delta}e(h,W)^{s_\beta}}{e(T,g_2)^{s_x}}\big{(}\frac{e(T,W)}{e(g_1,g_2)}\big{)}^{-c}(T,R,c,\allowbreak s_x,s_\delta,s_\beta)(T,R,c,s_x,s_\delta,s_\beta)(T,R,c^\prime,s^\prime_x,s^\prime_\delta,s^\prime_\beta)c\neq c^\prime\tilde{x}:=\frac{s_x-s^\prime_x}{c-c^\prime}\tilde{y}:=\frac{(s_x-s^\prime_x)(s_\beta-s^\prime_\beta)-(s_\delta-s^\prime_\delta)(c-c^\prime)}{(c-c^\prime)^2}\tilde{\beta}:=\frac{s_\beta-s^\prime_\beta}{c-c^\prime}\frac{e(T,W)}{e(g_1,g_2)}=\frac{e(h,g_2)^{\tilde{\beta}\tilde{x}-\tilde{y}}e(h,W)^{\tilde{\beta}}}{e(T,g_2)^{\tilde{x}}}\tilde{A}=T/h^{\tilde{\beta}}e(\tilde{A},g^{\tilde{x}}W)=e(g_1,g_2)e(h,g_2)^{-\tilde{y}}(\tilde{x},\tilde{y},\tilde{A})sk\mathtt{TempID}dk_{\mathtt{TempID}}dk_{\mathtt{TempID}}{\sf Get Content}\rightarrow\rightarrow\rightarrow\rightarrow{\sf GM.Setup}{\sf KGC.Setup}{\sf Join}{\sf UserKeyGen}{\sf RelayRequest}{\sf RelayContent}{\sf GM.Setup}{\sf KGC.Setup}{\sf Join}{\sf UserKeyGen}{\sf SendRequest}{\sf ValidityCheck}{\sf SendContent}{\sf GetContent}{\sf SendRequest}\mathtt{TempID}\mathbf{Acknowledgment}$: 
The authors would like to thank Dr. Goichiro Hanaoka and Dr. Miyako Ohkubo for their invaluable comments. 



\begin{thebibliography}{10}

\bibitem{openssl}
{OpenSSL}: {C}ryptography and {SSL/TLS} {T}oolkit.
\newblock Available at \url{http://www.openssl.org/}.

\bibitem{SPA}
{Selected Papers in Anonymity}.
\newblock Available at http://freehaven.net/anonbib/date.html.

\bibitem{Simpleproxy}
{Simpleproxy}: Crocodile group software.
\newblock Available at \url{http://www.crocodile.org/software.html}.

\bibitem{TEPLA}
{TEPLA}: {U}niversity of {T}sukuba {E}lliptic {C}urve and {P}airing {L}ibrary.
\newblock Available at
  \url{http://www.cipher.risk.tsukuba.ac.jp/tepla/index_e.html}.

\bibitem{TorProject}
{Tor Project}.
\newblock Available at https://www.torproject.org/.

\bibitem{[BarretoN05]}
P.~S. L.~M. Barreto and M.~Naehrig.
\newblock Pairing-friendly elliptic curves of prime order.
\newblock In {\em Selected Areas in Cryptography}, pages 319--331, 2005.

\bibitem{[BellareSZ05]}
M.~Bellare, H.~Shi, and C.~Zhang.
\newblock Foundations of group signatures: The case of dynamic groups.
\newblock In {\em CT-RSA}, pages 136--153, 2005.

\bibitem{[BonehB08]}
D.~Boneh and X.~Boyen.
\newblock Short signatures without random oracles and the {SDH} assumption in
  bilinear groups.
\newblock {\em J. Cryptology}, 21(2):149--177, 2008.

\bibitem{[BonehF03]}
D.~Boneh and M.~K. Franklin.
\newblock Identity-based encryption from the weil pairing.
\newblock {\em SIAM J. Comput.}, 32(3):586--615, 2003.

\bibitem{[ChaumH91]}
D.~Chaum and E.~van Heyst.
\newblock Group signatures.
\newblock In {\em EUROCRYPT}, pages 257--265, 1991.

\bibitem{[CheonHKO08]}
J.~H. Cheon, N.~Hopper, Y.~Kim, and I.~Osipkov.
\newblock Provably secure timed-release public key encryption.
\newblock {\em ACM Trans. Inf. Syst. Secur.}, 11(2), 2008.

\bibitem{[FiatS86]}
A.~Fiat and A.~Shamir.
\newblock How to prove yourself: Practical solutions to identification and
  signature problems.
\newblock In {\em CRYPTO}, pages 186--194, 1986.

\bibitem{[FurukawaI06]}
J.~Furukawa and H.~Imai.
\newblock An efficient group signature scheme from bilinear maps.
\newblock {\em IEICE Transactions}, 89-A(5):1328--1338, 2006.

\bibitem{[GiladH13]}
Y.~Gilad and A.~Herzberg.
\newblock Plug-and-play {IP} security - anonymity infrastructure instead of
  {PKI}.
\newblock In {\em ESORICS}, pages 255--272, 2013.

\bibitem{oakland2013-parrot}
A.~Houmansadr, C.~Brubaker, and V.~Shmatikov.
\newblock The parrot is dead: Observing unobservable network communications.
\newblock In {\em IEEE S\&P}, pages 65--79, 2013.

\bibitem{anon-pass13}
M.~Z. Lee, A.~M. Dunn, B.~Waters, E.~Witchel, and J.~Katz.
\newblock Anon-pass: Practical anonymous subscriptions.
\newblock In {\em IEEE S\&P}, pages 319--333, 2013.

\bibitem{[LibertPY12]}
B.~Libert, T.~Peters, and M.~Yung.
\newblock Group signatures with almost-for-free revocation.
\newblock In {\em CRYPTO}, pages 571--589, 2012.

\bibitem{[LibertV11]}
B.~Libert and D.~Vergnaud.
\newblock Unidirectional chosen-ciphertext secure proxy re-encryption.
\newblock {\em IEEE Trans. on Information Theory}, 57(3):1786--1802, 2011.

\bibitem{[MoghaddamLDG12]}
H.~M. Moghaddam, B.~Li, M.~Derakhshani, and I.~Goldberg.
\newblock Skype{M}orph: protocol obfuscation for {T}or bridges.
\newblock In {\em ACM CCS}, pages 97--108, 2012.

\bibitem{[RivestST01]}
R.~L. Rivest, A.~Shamir, and Y.~Tauman.
\newblock How to leak a secret.
\newblock In {\em ASIACRYPT}, pages 552--565, 2001.

\bibitem{[SudarsonoNNF10]}
A.~Sudarsono, T.~Nakanishi, Y.~Nogami, and N.~Funabiki.
\newblock Anonymous {IEEE}802.1{X} authentication system using group
  signatures.
\newblock {\em JIP}, 18:63--76, 2010.

\end{thebibliography}



\end{document}
