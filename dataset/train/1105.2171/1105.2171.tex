\documentclass[dvipdfm]{llncs}
\usepackage[latin1]{inputenc}
\usepackage{url,amsfonts,epsfig}
\usepackage{amsmath,latexsym,amssymb}
\usepackage{color,caption}
\usepackage{graphicx, subfigure}
\usepackage{footmisc}
\usepackage{algorithm,algorithmic}

\newtheorem{Theorem}{Theorem}
\newtheorem{Claim}{Claim}
\newtheorem{Corollary}{Corollary}
\newtheorem{Proposition}{Proposition}
\newtheorem{Lemma}{Lemma}
\newtheorem{Definition}{Definition}
\newtheorem{Conjecture}{Conjecture}
\newtheorem{Property}{Property}
\newtheorem{Observation}{Observation}
\newtheorem{Problem}{Problem}
\newcommand{\vect}[1]{\textbf{#1}}

\newcommand{\DS}{\displaystyle}
\newcommand{\TS}{\textstyle}
\newcommand{\F}{\mathcal F}
\newcommand{\T}{\mathcal T}
\newcommand{\M}[1]{\mathcal{#1}}
\newcommand{\dmin}{d_{min}}
\newcommand{\dmax}{d_{max}}
\newcommand{\PCGM}{PCG_{max}}
\newcommand{\PCGm}{PCG_{min}}



\title{On relaxing the constraints in pairwise compatibility graphs}
\titlerunning{On relaxing the constraints in pairwise compatibility graphs}  

\author{Tiziana Calamoneri \and Rossella Petreschi \and Blerina Sinaimeri}
\authorrunning{T. Calamoneri, R. Petreschi and B. Sinaimeri}   \institute{Department of Computer Science \\
    ``Sapienza'' University of Rome - Italy\\
  via Salaria 113, 00198 Roma, Italy.\\
\email{e-mail: \{calamo, petreschi, sinaimeri\}@di.uniroma1.it}
}


\begin{document}


\maketitle              


\begin{abstract}
A graph  is called a pairwise compatibility graph (PCG) if there exists an edge weighted tree  and two non-negative real numbers  and  such that each leaf  of  corresponds to a vertex  and there is an edge  if and only if  where  is the sum of the weights of the edges on the unique path from  to  in .  In this paper we analyze the class of PCG in relation with two particular subclasses resulting from the the cases where  (LPG) and  (mLPG). In particular, we show that the union of LPG and mLPG does not coincide with the whole class PCG, their intersection is not empty, and that neither of the classes LPG and mLPG is contained in the other.  Finally, as the graphs we deal with belong to the more general class of split matrogenic graphs,  we focus on this class of graphs for which we try to establish the membership to the PCG class. 



\end{abstract}

{\bf keywords:} PCG, leaf power graph, threshold graphs, matrogenic graphs.


\section{Introduction}

Given an edge weighted tree , let  and  be two nonnegative real numbers with . For any two leaves  and  of the tree , we denote by  the sum of the weights of the edges on the unique path from  to  in . Starting from ,  and , it can be easily constructed a \textsl{pairwise compatibility graph }of , i.e. a graph  where each vertex  corresponds to a leaf  of  and there is an edge  if and only if . We will denote such a graph  by . Consequently, we say that a graph  is a pairwise compatibility graph (PCG) if there exists an edge weighted tree  and two nonnegative real numbers   and  such that . Determine whether a graph  is a PCG seems in general difficult even if at the beginning, when the problem arose in a computational biology context \cite{Kal03}, it was conjectured that every graph was a PCG. Nowadays it is known that this conjecture is false \cite{YBR10}, while it is proved that some specific classes of graphs e.g., graphs with five nodes or less \cite{P02}, cliques and disjoint union of cliques \cite{B}, chordless cycles and single chord cycles \cite{YHR09} and some particular subclasses of bipartite graphs \cite{YBR10}, are PCG.


The pairwise compatibility concept is defined with respect to two bounds concerning  and . If we relax these conditions, requiring only that the distance between some pair of leaves is smaller than or equal to  (i.e. we set ) then we are considering a particular subclass of PCG graphs, namely the \textsl{leaf power} graphs (LPG). More formally, a graph  is a leaf power if there exists a tree  and a nonnegative number  such that there is an edge  in  if and only if for their corresponding leaves  we have  (see \cite{NRTh02}). Although there has been a lot of works on this class of graphs \cite{B}, a completely description of leaf power graphs is still unknown. 


To the best of our knowledge, nothing is known in literature concerning the subclass of PCG when the constraint concerns only the minimum distance, i.e. there is an edge in  if and only if the corresponding leaves are at a distances greater than  in the tree (observe that in this case we set ). In this paper we introduce this new concept and exploit the relations between the new defined class and the two known classes LPG and PCG.

The paper is organized as follows: in Section \ref{sec:preliminaries} we introduce some terminologies and recall some known concepts that we will use in the forthcoming work. Then, we define the new subclass of PCG, namely mLPG, characterized by the use of  only. Next, in Section \ref{sec:relations} we study the relations between the classes PCG, LPG and mLPG. In particular, we show that the union of LPG and mLPG does not coincide with the whole class PCG, their intersection is not empty, and neither of the classes LPG and mLPG is contained in the other. All the graphs we furnish as examples in Section \ref{sec:relations} are particular cases of the more general class of split matrogenic graphs. Hence, in Section \ref{sec:matrogenic} we focus on the class of split matrogenic graphs trying to determine if it belongs to the PCG class. We prove that many split matrogenic graphs are PCG. However, the membership to PCG class of one particular subclass of split matrogenic graph remains an open problem that is reported in the final Section \ref{sec:conclusion} together with some other open problems.
\section{Preliminaries}\label{sec:preliminaries}

In this section we introduce some definitions and some concepts that we use in the rest of  this paper. 

When we say that a {\em tree}  is {\em weighted}, we mean that it is edge weighted, that is each edge is assigned a number as its weight. In this paper we consider only weighted trees and graphs that are connected.  


A {\em caterpillar} is a tree in which all the vertices are within distance one of a central path which is called the {\em spine}.


A graph  is said to be\textit{ split} if there is a vertex partition \mbox{} such that the subgraphs induced by  and  are complete and stable, respectively.  


Given two split graphs  and   their \textit{composition}  is formed by taking the disjoint union of  and  and adding all the edges   such that  and . Observe that  is again a split graph.


A set  of edges is a {\em perfect matching} of dimension  of  onto  if and only if  and  are disjoint subsets of vertices of cardinality  and each vertex in  is adjacent to exactly one vertex in  and no two edges share a point. We say that the split graph  is a  {\em split matching }if the subset of edges in  not belonging to the clique forms a perfect matching. We denote by  the class of split matching graphs.


An \textit{antimatching} of dimension  of  onto  is a set of edges such that the non edges between  and  form a perfect matching. We say that the split graph  is a \textit{split antimatching }if the subset of edges in  not belonging to the clique forms an antimatching. We denote by  the class of split antimatching graphs.


A {\em cactus} is a connected graph in which any two simple cycles have at most one vertex in common. Equivalently, every edge in such a graph may belong to at most one cycle. We will denote by  the class of cacti with at least a cycle of length .


Given a connected graph  whose distinct vertex degrees are \mbox{}, we define , for any .  The sets  are usually referred as {\em boxes }and the sequence  is called the \textit{degree partition of}  \textit{into boxes}. 


Given a graph  with  degree partition ,  is a  {\em threshold graph} if and only if for all , , , we have  if and only if . We will denote by  the class of threshold graphs. 


\smallskip

Now, we introduce a new subclass of PCG, namely mLPG as follows: 

\begin{Definition}
A graph  is an mLPG if there exists a tree  and an integer  such that there is an edge  in  if and only if for their corresponding leaves  in  we have . 
\end{Definition}
Note that for the sake of simplicity and homogeneity of the paper, here we slightly abuse notation as these graphs are not power of trees.


\medskip
\noindent
In what follows we will often make use of the following simple observation.

\begin{Proposition}\label{prop:technical}
Let  be a  graph that does not belong to some class  from  then every graph  that contains  as an induced subgraph, does not belong to  either.
\end{Proposition}


Given two vertices  in a tree , we denote by  the unique path in  connecting the vertices  and . A {\em subtree induced by a set of leaves} of  is the minimal subtree of  which contains those leaves. We denote by  the subtree of a tree induced by three leaves  and . 

The following technical lemma will turn out to be very useful in the forthcoming results.

\begin{Lemma}\label{lem:technical} \cite{YBR10}
Let  be an edge weighted tree, and  and  be three leaves of 
such that  is the largest path in . Let  be a leaf of  other than  and
. Then, either  or .
\end{Lemma}


\begin{figure}[h]
	\centering
	\includegraphics[scale=0.4]{classes.eps}
\caption{Relationships between PCG, LPG and mLPG.}
\label{fig:classes}
\end{figure}



\section{Relationships between PCG, LPG and mLPG  }\label{sec:relations}

In this section we study the relationships between the classes of ,  and . First, in Subsection \ref{subsec:union} we show that the union of  and  does not contain the whole class of . Next, in Subsection \ref{subsec:intersection} we show that their intersection  is not empty, by proving that threshold graphs belong to both classes. Finally, in Subsection \ref{subsec:proper} we show that neither of the classes  and  is contained in the other one by providing for each of these classes a particular graph which is proper to it. These relations are graphically shown in Figure \ref{fig:classes}. 



\subsection{} \label{subsec:union}
In this subsection we prove that the  class does not coincide with the union of  and . Indeed, in \cite{YHR09} it is proved that any cycle is a PCG. Now, it is well-known (see, for example, \cite{B}) that LPG is a subclass of strongly chordal graphs and clearly cycles of length  are not strongly chordal, so they are not LPG. The following lemma states that cycles do not belong to mLPG, deducing that .

\begin{Lemma}
Let  be a cycle of length , then .
\end{Lemma}
\proof
Let  be the ordered vertices of a cycle  with . Suppose by contradiction that  and let 
be the leaf in  corresponding to the vertex , for any .  Let  be the first three consecutive vertices in  and consider the largest path in . As  (as ) then . Hence, the largest path must be one from  and . 

Suppose first the largest path is . Using Lemma \ref{lem:technical} with  we have that either  or , deducing that at least one between the  and  must be an edge in , a contradiction. 

If  is the largest path, we arrive at the same result  by taking this time . This concludes the proof. \qed

Easily, in view of Proposition \ref{prop:technical} the class  of cacti with at least one cycle of length  does not belong either to LPG or to mLPG. 

\subsection{}\label{subsec:intersection}

In this subsection we prove that the intersection of LPG and mLPG is not empty by showing that threshold graphs belong to . 

\begin{Theorem}\label{theo:threshold}
Let  be a threshold graph. Then  and it is polynomial to find the tree  and the values  associated to .
\end{Theorem}

\proof
Let  be a threshold graph on  vertices and let  be the degree partition of .  We consider an -leaf star with center a vertex , as the tree . 

To prove that , for each vertex  of , assign weight   to the edge  in  if . Define .  As for each , , , we have  if and only if  it is straightforward that  is a PCG of  with . 


On the other hand, to prove  for any  assign  to the edge  in  if .  Note that, as  we assign nonnegative weights to the edges of the star. Define . For any two vertices  and , we have that if  (meaning that ) then . Otherwise, if  (meaning that ) then . This concludes the proof. \qed


\subsection{ and }\label{subsec:proper}

Here we show that neither of the classes LPG and mLPG is contained in the other one by providing, for each of these classes a particular graph which is proper to it. 

\begin{Theorem} \label{theo:splitmatching}
Let  be a split matching graph. Then ,  and in this case it is polynomial to find the tree  and the value  associated to .
\end{Theorem}

The proof will follow immediately by the next two lemmas.

\begin{Lemma}\label{lem:matchingtree}
Let  be a split matching graph. Then  and it is polynomial to find the tree  and the value  associated to .
\end{Lemma}
\proof
Given a split matching graph  with , we associate a caterpillar tree  as in Figure \ref{fig:matching}. The leaves , corresponding to the vertices  of , are connected to the spine with edges of weight  and the leaves , corresponding to vertices , with edges of weight . It is clear that . Indeed, for any two  it holds that , for any two  we have  and for any  we have  (hence the edge ) and for any  with  we have  (hence the edge ).\qed
Note that this representation is not unique. Indeed, one can easily check that the binary tree  in Figure \ref{fig:matchingbinary} also is a pairwise compatibility tree of a split matching graph when . 

\begin{figure}[!ht]
  \begin{center}
 \subfigure[]{\label{fig:matching}	\includegraphics[scale=0.35]{matching.eps}}
 \subfigure[]{\label{fig:matchingbinary}	\includegraphics[scale=0.35]{matchingbinary.eps}}
  \end{center}
\caption{\footnotesize{(a) A pairwise compatibility caterpillar tree for a split matching graph. (b) A pairwise compatibility tree for a split matching graph.}}
\label{fig:mat}
\end{figure}



\begin{Lemma}\label{lem:matching2}
Let  be a split matching graph. Then .
\end{Lemma}
\proof
Given a split matching graph  with , we assume by contradiction . Then let  be three leaves of  corresponding to three vertices of , . Without loss of generality let  be the largest path in the subtree . Consider the vertex  in  associated to the leaf  in , with . From Lemma \ref{lem:technical} we deduce that either  or . The existence of the edge  in  implies , therefore one from  and  must be an edge in , a contradiction. \qed


Analogously, we can show that the set  is not empty.

\begin{Theorem} \label{theo:splitantimatching}
Let  be a split antimatching graph. Then ,   and in this case it is polynomial to find the tree  and the value  associated to .
\end{Theorem}

For the sake of brevity we omit the proof of this theorem, that follows using arguments similar to those in the proofs of Lemmas \ref{lem:matchingtree} and \ref{lem:matching2}. The tree  associated to a split antimatching graph is one from the ones depicted in Figure \ref{fig:anti}.

\begin{figure}[!ht]
  \begin{center}
 \subfigure[]{\label{fig:antimatching}	\includegraphics[scale=0.35]{antimatching.eps}}
 \subfigure[]{\label{fig:antimatchingbinary}	\includegraphics[scale=0.35]{antimatchingbinary.eps}}
  \end{center}
\caption{\footnotesize{(a) A pairwise compatibility caterpillar tree for a split antimatching graph. (b) A pairwise compatibility tree for a split antimatching graph.}}
\label{fig:anti}
\end{figure}


\section{Split Matrogenic Graphs}\label{sec:matrogenic}


In Section \ref{sec:relations}, when studying the relations among the three classes PCG, LPG and mLPG, we have dealt with threshold graphs, split matchings and split antimatchings. All these graphs are split matrogenic graphs (cfr. definition later). For this reason, it is natural to ask whether split matrogenic graphs are PCG or not. This section is devoted to answer this question.


\begin{Definition}
A {\em split matrogenic} graph is the composition of  split graphs  with  such that: either  is a split matching or  is a split antimatching or  (and  is called {\em stable graph}) or  (and  is called {\em clique graph}).
\end{Definition}

In order to make easier the exposition, we introduce two subclasses of split matrogenic graphs.

\begin{Definition}
Given a sequence of  split graphs  with , we say the graph  is a {\em split matching (antimatching) sequence} if each of the graphs  is either a split matching (antimatching), or a  stable graph or a clique graph. 
\end{Definition}


We first prove that split matching sequences and split antimatching sequences are PCG. In both these proofs, in the construction of the pairwise compatibility tree, we will make use of the constructions depicted in Figure \ref{fig:matchingbinary} and Figure  \ref{fig:antimatchingbinary}, respectively. Finally, we want to point out that a clique graph (a stable graph) can be considered both as a split matching or a split antimatching graph and in each case the pairwise compatibility tree is constructed in the same way, where only leaves  (respectively ) appear. In Figure \ref{fig:matchingAntimatchingEmpty} the pairwise compatibility tree is given for a stable graph  when it is considered first as a split matching graph and next as a split antimatching graph.


\begin{figure}[!ht]
  \begin{center}
 \subfigure[]{\label{fig:matchingbinaryEmpty} \includegraphics[scale=0.4]{matchingbinaryEmpty.eps}}
 \subfigure[]{\label{fig:antimatchingbinaryEmpty}\includegraphics[scale=0.4]{antimatchingbinaryEmpty.eps}}
  \end{center}
\caption{\footnotesize{ The pairwise compatibility tree for a stable graph  with  vertices when it is considered as (a) a split matching graph  (b) a split antimatching graph. }}
\label{fig:matchingAntimatchingEmpty}
\end{figure}



\begin{Theorem}\label{theo:matchingSequence}
Let  be a split matching sequence. Then  and it is polynomial to find the tree and the value  associated to .
\end{Theorem}
\proof
Let  be a  split matching sequence. For each graph  we define a tree  as shown in Figure \ref{fig:matchingsequenceBi} (where the leaves  () may not possibly appear if  is a stable (clique) graph). It is clear that  where  is a value to be defined later, but surely greater than or equal to . Indeed, let  be the leaves of  corresponding to vertices of  and  those corresponding to vertices of . For any two leaves  it holds that  and for any two  we have . Finally, for any two leaves  that correspond to an edge of the matching their distance is  and for any two leaves corresponding to a non edge  their distance is . 

\begin{figure}[!ht]
  \begin{center}
 \subfigure[]{\label{fig:matchingsequenceBi} \includegraphics[scale=0.4]{matchingSequence.eps}}
 \subfigure[]{\label{fig:matchingsequenceTree}\includegraphics[scale=0.4]{tree.eps}}
  \end{center}
\caption{\footnotesize{(a) The pairwise compatibility tree for the split matching graph .  (b) The pairwise compatibility tree for the split matching sequence . }}
\label{fig:matchingsequence}
\end{figure}


In order to prove that ,  we define a new tree  starting from the trees , simply by contracting all their roots to a single vertex as shown in Figure \ref{fig:matchingsequenceTree}. We claim that  where we set . In order to prove it, consider two graphs  and  with . Let   and  be four distinct leaves  corresponding to vertices in  and  respectively. 
Observe that the vertices in  are connected to all the other vertices in  as the distances in  are  and  (as ). Finally, any vertex in  is not connected to any vertex  and to any vertex  as in these cases the distances are  (as ) and . \qed



\noindent
Using similar arguments we prove the following result.


\begin{Theorem}\label{theo:antimatchingSequence}
Let  be a split antimatching sequence. Then   and it is polynomial to find the tree and the value  associated to .
\end{Theorem}
\proof

The proof follows the same lines of the proof of Theorem \ref{theo:matchingSequence}. Let  be a  split antimatching sequence. We will associate to each split antimatching graph  a tree  as depicted in Figure \ref{fig:antimatchingsequence}. 
We prove that  where  is a value to be defined later, but surely greater than or equal to . Indeed, let  be the leaves of  corresponding to vertices of  and  those corresponding to vertices of . For any two leaves  it holds that  and for any two  we have . Finally, for any two leaves  that correspond to an edge of the antimatching their distance is  and for any two leaves corresponding to a non edge  their distance is . 

We define the tree  starting from the trees , in the same way we have done in the previous theorem (see Figure \ref{fig:matchingsequenceTree}). Using the same arguments it is not difficult to check that  where we have set . 
Indeed, consider two graphs  and  with . Let   and  be four distinct leaves  corresponding to vertices in  and  respectively. 
Observe that the vertices in  are connected to all the other vertices in  as the distances in  are 
 and . Finally, any vertex in  is not connected to any vertex  and to any vertex  as in these cases the distances are  (as ) and . \qed

\begin{figure}[!ht]
  \begin{center}
\includegraphics[scale=0.4]{antimatchingSequence.eps}
  \end{center}
\caption{\footnotesize{The pairwise compatibility tree for the split antimatching graph . }}
\label{fig:antimatchingsequence}
\end{figure}



\begin{figure}[!ht]
  \begin{center}
\includegraphics[scale=0.34]{matrogenic.eps}
  \end{center}
\caption{\footnotesize{The pairwise compatibility tree for the split matrogenic graph . }}
\label{fig:matchingAntimatching}
\end{figure}


\begin{Theorem}
Let  be a split matrogenic graph such that for any split matching graph  and for any split antimatching graph  it holds  that . Then  and it is polynomial to find the tree and the values  associated to .
\end{Theorem}
\proof
Let . It is clear that if none of the graphs  is a split matching (a split antimatching) the proof trivially follows from Theorem \ref{theo:matchingSequence} (Theorem \ref{theo:antimatchingSequence}). Hence, let , , be the first occurrence of a split matching graph. Then, the graphs  and  are a split matching sequence and a split antimatching sequence, respectively.
Then, let  where the tree is constructed in the same way as in the proof of Theorem \ref{theo:matchingSequence} and  (recall that in the proof of Theorem \ref{theo:matchingSequence}  we only need  to be a value greater that ). Similarly, according to the Theorem  \ref{theo:antimatchingSequence},  and  (note that we choose to have )). We modify  in such a way that of the weights of the edges outcoming form the root start from value  instead of from  and the other edges are modified accordingly. This is not restrictive, as  results as if  was the composition of  split antimatching graphs whose the first  are empty graphs.



We construct the pairwise compatibility tree  by joining the roots of  and  with an edge of weight . We set  and . We modify the weights of the resulting tree increasing by  the weight of any edge incident to a leaf in .  Observe that in this way the distance of any two leaves in  is increased by . This means that two leaves correspond to vertices of an edge in  if and only if their distance is less than or equal to . Furthermore, the maximum distance of any two leaves in  is less than or equal to  meaning that they correspond to vertices of an edge in  if and only if their distance is greater than or equal to . 

We claim that  (recall that ). We have already shown that the pairwise compatibility constraints hold for any two leaves that correspond to two vertices of the same graph  or  . It remains to show it also holds for two leaves  where one corresponds to a vertex in  and the other to . To this purpose, let  and  be two distinct leaves in , connected to the root with edges of weight  and  corresponding to vertices of the clique and the stable graph of , respectively. Similarly let  be two distinct leaves in , connected to the root with edges of weight  and corresponding to vertices in the clique and the stable graph of , respectively. The following hold:

\begin{itemize}
\item[a)]  and as  and  then . Hence, the corresponding vertices of  in  are connected.
\item[b)]  and as  then .  Hence, the corresponding vertices of  in  are connected.
\item[c)]  and as  then . Hence, the corresponding vertices of  in  are not connected.
\item[d)]  and as  then . Hence, the corresponding vertices of  in  are not connected.
\end{itemize}

This, concludes the proof. \qed




It seems that the order of appearance of a matching or an antimatching sequence in a split matrogenic graph is somehow strictly related to the pairwise compatibility property.  Indeed, in spite our efforts, the following problem remains open.

\textit{Problem:} Let  be a split matrogenic graph such that for any split antimatching graph  and for any split matching graph  it holds  that . Is  a PCG ?

If this problem has an affermative answer then it should not be difficult to prove that all split matrogenic graphs are PCG. Otherwise, the separation between the split matrogenic graphs that are PCG and those that are not, would be perfectly known. 



\section{Conclusions and Open Problems}\label{sec:conclusion}


In this paper we analyze the class of PCG with particular attention to two particular subclasses resulting when the pairwise compatibility constriants are relaxed. Hence, we consider the sublasses LPG and mLPG, resulting from the the cases where  and , respectively. We study the relations between the classes PCG, LPG and mLPG. In particular, we show that the union of LPG and mLPG does not coincide with the whole class PCG, their intersection is not empty, and that neither of the classes LPG and mLPG is contained in the other.  The graphs considered in these considerations are particular cases of the more general class of split matrogenic graphs. Hence,  we attempt to determine whether the class of split  matrogenic graphs belongs to PCG class. We prove that many split matrogenic graphs are PCG. However, the membership to PCG class of one particular split matrogenic graph remains an open problem.  

It should be stressed that up to date, the pairwise compatibility has been investigated only for a few classes of graphs, thus determine this property for many graph classes remains an open problem.  It is worth to mention that in \cite{Kal03} is shown that the clique problem can be solved in polynomial time for the class of compatibility graphs if we are able to construct in polynomial time a weighed tree that generates the pairwise compatibility graph.  In view of this, it seems even more interesting to completely identify the Pairwise Compatibility graphs class.  






\begin{thebibliography}{99}

\begin{small}


\bibitem{B}
A. Brandst\"adt. : On Leaf Powers. Technical report, University of Rostock, (2010).

















\bibitem{Kal03}
P.E. Kearney, J. I. Munro and D. Phillips: Efficient generation of uniform samples from phylogenetic trees. In: Benson, G., Page, R.D.M. (eds.) WABI. Lecture Notes in Computer Science, vol. 2812, pp. 177--189. Springer, Berlin (2003).




\bibitem{NRTh02}
N. Nishimura, P. Ragde, D.M. Thilikos, On graph powers for leaf-labeled trees, J. Algorithms 42 (2002) 69-108.

\bibitem{P02}
Phillips, D.: Uniform sampling from phylogenetics trees. Masters Thesis, University of Waterloo (2002).

\bibitem{YBR10}
Nur Yanhaona, M., Bayzid, Md.S., Rahman, Md. S.: Discovering Pairwise compatibility graphs. J. Appl. Math. Comput. 30, 479--503 (2009).

\bibitem{YHR09}
Nur Yanhaona, M., Hossain, K.S.M. T. Rahman, M. S.: Pairwise compatibility graphs. J. Appl. Math. Comput. 30, 479--503 (2009).

\end{small}
\end{thebibliography}


\end{document}