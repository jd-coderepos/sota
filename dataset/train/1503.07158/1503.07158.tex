\documentclass[twocolumn]{autart}    \usepackage{cite}
\usepackage{psfrag}
\usepackage{graphics}
\usepackage{epsfig, latexsym, amsmath, color, amsfonts, amssymb,graphicx}
\usepackage{algorithm}
\usepackage{pdfsync}
\usepackage{booktabs}
\usepackage{algpseudocode}
\usepackage{amsmath}
\usepackage{graphics}
\usepackage{epsfig}
\usepackage{caption}
\usepackage{subfigure}

\newtheorem{appendixlemma}{Lemma}[section]
\newtheorem{theorem}{Theorem}[section]
\newtheorem{lemma}[theorem]{Lemma}
\newtheorem{corollary}[theorem]{Corollary}
\newtheorem{conjecture}[theorem]{Conjecture}
\newtheorem{proposition}[theorem]{Proposition}
\newtheorem{remark}[theorem]{Remark}
\newtheorem{comments}[theorem]{Comments}
\newtheorem{example}[theorem]{Example}
\newtheorem{definition}[theorem]{Definition}
\newtheorem{assumption}[theorem]{Assumption}
\newtheorem{problem}[theorem]{Problem}
\newtheorem{acknowledgment}[theorem]{acknowledgment}

\newcommand{\R}{{\rm  I\kern-2pt R}}
\renewcommand{\Re}{{\rm  I\kern-2pt R}}
\newcommand{\Rea}{{\mathbb{R}}}
\newcommand{\rank}{{\mathrm{rank}}}
\renewcommand{\span}{{\mathrm{span}}}
\newcommand{\Q}{{\mathbb{Q}}}
\newcommand{\Ze}{{\mathbb Z}}
\newcommand{\Pe}{{\mathbb P}}
\newcommand{\Ce}{{\mathbb C}}
\newcommand{\ie}{{\it i.e.}}
\newcommand{\eg}{{\it e.g.}}
\newcommand{\ea}{{\it et al}}
\newcommand{\D}{\displaystyle}
\newcommand{\Pre}{\textrm{Pre}}
\newcommand{\argmax}{\textrm{arg}\max}
\newcommand{\argmin}{\textrm{arg}\min}
\newcommand{\sub}{\textrm{sub}}
\newcommand{\UP}{\textrm{UP}}
\newcommand{\defeq}{\triangleq}
\newcommand{\x}{{\mathbf x}}
\renewcommand{\u}{{\mathbf u}}
\renewcommand{\b}{{\mathbf b}}
\newcommand{\Node}{\textrm{Node}}

\begin{document}

\begin{frontmatter}

\title{{Data-Driven Power Control for State Estimation:~A
Bayesian Inference Approach}\thanksref{footnoteinfo}}

\thanks[footnoteinfo]{The work of J.~Wu, Y.~Li and L.~Shi is
supported by a HK RGC GRF grant 618612. }

\author[Paestum]{Junfeng Wu}\ead{jfwu@ust.hk},
\author[Paestum]{Yuzhe Li}\ead{yliah@ust.hk},
\author[Rome]{Daniel E. Quevedo}\ead{dquevedo@ieee.org},
\author[Paestum]{Vincent Lau}\ead{eeknlau@ust.hk},
\author[Paestum]{Ling Shi}\ead{eesling@ust.hk}

\address[Paestum]{Department of Electronic and Computer Engineering, Hong Kong University of Science and Technology, Hong Kong.}
\address[Rome]{School of Electronic Engineering and Computer Science, University of Newcastle, Australia.}

\begin{keyword}
Kalman filtering, Transmission power control, State estimation, Packet losses, Bayesian
inference
\end{keyword}


\begin{abstract}
We consider sensor transmission power control for state
estimation, using a Bayesian inference approach.
A sensor node sends its local state estimate to a remote estimator over an
unreliable wireless communication channel with random data packet drops. As related to packet dropout rate, transmission power is chosen by the
sensor based on the relative importance of the local state estimate. The
proposed power controller is proved to preserve Gaussianity of
local estimate innovation, which enables us to obtain a closed-form
solution of the expected state estimation error covariance. Comparisons with
alternative non-data-driven controllers demonstrate performance
improvement using our approach.

\end{abstract}

\end{frontmatter}




\section{Introduction}\label{sec:intro}
{Wireless networked systems have a wide spectrum of applications in smart grid, environment monitoring,
intelligent transportation, etc.
State estimation is a key enabling technology where
the sensor(s) and the estimator communicate over a wireless network.
 Energy conservation is a crucial issue as most wireless sensors use on-board
 batteries which are difficult to replace and typically are expected to work for years without replacement.
Thus power control becomes crucial.
In this work, we consider sensor transmission power control for remote state
estimation over a packet-dropping network.
{Transmission power control in state estimation scenario has been considered
from different perspectives. Some works took transmission costs as constant.}
Shi \textit{et al.} \cite{shi2012optimal}
assumed sensors to have two energy
modes, allowing it to send
data to a remote estimator over an unreliable
channel either using a high or low transmission power
level. The optimal power controller is
to minimize the expected terminal estimation error at the remote estimator subject to an energy
constraint.
Similar works can also be found in~\cite{xu2004optimal,Imer05CDC}.
Meanwhile, some literature has taken channel conditions
into account.
Quevedo \textit{et al.}~\cite{queahl10} studied state estimation over fading channels.
They proposed a predictive control algorithm,
where power and cookbooks are determined in an online fashion based on
the undergoing estimation error covariance and the channel gain predictions.
{More related works can been
 seen in~\cite{leong2012power,nouleo14,quevedo2012kalman}.}

{An important issue which has not been taken seriously in most
works is that the transmission power assignment, as a tool to
control the accessibility of information to the receiver,
should be determined not only by the underlying channel condition and
the desired estimation performance, but also by the transmitted
information itself.
In~\cite{queahl10} and~~\cite{leong2012power},
the authors failed to associate transmission power with data to be sent.
The plant states are used to determine the transmission power in~\cite{GatsisACC13}.
In this case, lost packets signal the receiver of the state information. To avoid computation difficulty, the signaling information is discarded.}

In this paper, we focus on how to adapt the transmission power to the
measurements of plant state and how to exploit information contained in the lost packets. We propose a data-driven power controller, which utilizes
different transmission power levels to send the local estimate according to a quadratic function of a key parameter called ``incremental
innovation" which is evaluated by the sensor at each time slot.
By doing this, even when data dropouts occur, {the
remote estimator can utilize the additional signaling information
to refine the posterior probability density} of the estimation error by a Bayesian inference technique (see
\cite{box2011bayesian}), therefore deriving the
MMSE estimate. It compensates the
deteriorated estimation performance caused by packet losses.
{To facilitate analysis, we assume that a baseline power controller has already been
established based on different factors with regard to different settings, such as
the requirement of estimation performance as in~\cite{shi2012optimal}
or the channel conditions as in~\cite{queahl10,quevedo2012kalman,leong2012power}.
We are devoted to developing a power controller that
embellishes this baseline controller by adapting the transmission power
to the measurements such that the averaged power
with respect to all
possible values taken by the measurements does not exceed that of the baseline power controller.
The proposed power controller, driven by online measurements,
can run on top of non-data-driven power controllers, which results in
hierarchical power control mechanisms.
Then extension to a time-varying power baseline is established in
Section~\ref{subsec:vary-power-in-time}.} Note that a related controller
was first proposed in~\cite{Liyuzhe13CDC}, but as a special case of the
controller in this work. The main contributions of the present work
are summarized as follows.
\vspace{-3mm}
\begin{enumerate}
\item We propose a data-driven power control
strategy for state estimation with packet losses, which adapts the transmission power to the
measured plant states.

\item We prove that the proposed power controller preserves Gaussianity of the local innovation. It simplifies derivation of the MMSE estimate and leads to a closed-form expression of the expected state estimation error covariance.

\item {We present a tuning method for parameter design.
Despite of its sub-optimality, the controller is shown to perform
{not worse} than an
alternative non-data-driven one.}
\vspace{-3mm}
\end{enumerate}
The remainder of this paper is organized
as follows. In
Sections \ref{sec:problem-setup} and~\ref{sec:transmission-power-control}, we give mathematical
models of the considered system and introduce the data-driven transmission power controller. In Section~\ref{section:analysis}, we
present the MMSE estimate at the remote estimator and
{a sub-optimal power controller that minimizes an upper bound of the remote
estimation error}. In Section~\ref{sec:simulation}, comparisons with
alternative non-data-driven controllers demonstrate performance
improvement using our approach. Section~\ref{sec:conculsion} presents concluding remarks.

\textit{Notation}:  (and ) is the set of nonnegative
(and positive) integers. 
is the cone of  by  positive semi-definite
matrices. For a matrix ,  is the th
smallest nonzero eigenvalue. We abuse
notations  and , which are used, in case of a singular matrix , to denote the pseudo-determinant and
the Moore-Penrose pseudoinverse.
 is the Dirac delta function, i.e.,  equals to  when  and  otherwise. The notation 
represents the probability density function (pdf) of a random variable  taking value at .

\section{State Estimation using a Smart Sensor} \label{sec:problem-setup}
Consider a linear time-invariant (LTI) system:

where  ,  is the system state vector at time ,  is the measurement obtained by the sensor, the state noise  and observation noise  are zero-mean i.i.d. Gaussian noises with  (),  (), . The initial state  is a zero-mean Gaussian random vector with covariance  and is uncorrelated with  and .  is assumed to be detectable and  is assumed to be stabilizable. Furthermore, we assume  is Hurtwitz.\footnote{{Since we focus on remote state estimation in this paper, for any practically working systems (to be monitored alone),  has to be Hurwitz. Otherwise, the system state will go unbounded and there is no real sensing device which can track an unbounded state trajectory. Adding a control input to regulate the system state for an unstable  and studying its associated stability issue will be beyond the scope of this paper and will be left as our future work.}}

\begin{figure}[htbp]
\centering
  \includegraphics[width=7cm]{block-diagram-new}
  \caption{The system architecture.} \label{fig:system}
\end{figure}

\subsection{Sensor Local Estimate} \label{sec:local-state-estimate}
Hovareshti \textit{et al.}~\cite{hovareshti2007sensor} illustrated that utilization of the computation capabilities of wireless sensors may improve the system performance significantly.
Equipped with such
``smart sensors", the sensor locally runs a Kalman filter to produce the MMSE
estimate  of the state  based on all the measurements
collected up to time , i.e., , and then
transmits its local estimate to the remote estimator.
Denote the sensor's local MMSE state estimate, the corresponding estimation error and error covariance as ,  and , respectively, i.e., ,  and .
Standard Kalman filtering analysis suggests that
 these quantities can be calculated recursively (cf.,~\cite{andmoo79}),
where the recursion starts from  and .
Since 
converges to a steady-state value exponentially fast (cf.,~\cite{andmoo79}), we assume that the sensor's local Kalman filter has entered the steady state, that is,
,
{This assumption
simplifies our subsequent analysis and results, such as Theorem~\ref{theorem:form_P_k_on} and
Proposition~\ref{proposition: rank}.}


\subsection{Wireless Communication Model}


The data are sent to the remote estimator over
an Additive White Gaussian Noise (AWGN) channel using the Quadrature Amplitude Modulation
(QAM) whereby  is
quantized into  bits and mapped to one of  available QAM
symbols.\footnote{QAM is a common modulation scheme  widely used in IEEE 802.11g/n as well
as 3G and LTE systems, due to its high bandwidth efficiency.} For simplicity, the following assumptions are made:
\vspace{-3mm}
\begin{enumerate}
\item[A.1:]
The channel noise is independent of  and .

\item[A.2:]
 is large enough so that quantization effect is negligible when
analyzing the performance of the remote estimator.
\item[A.3:]\label{asmpt:assumption-SER-2-gamma}
The remote estimator can
detect symbol errors\footnote{In practice, symbol errors can be detected via a cyclic redundancy check (CRC) code.}. Only the
data arriving error-free are regarded as being successfully received; otherwise they are regarded as dropout.
\end{enumerate}
\vspace{-3.5mm}
These assumptions are commonly used in communication and control
{theories~(cf.,\cite{sinopoli2004kalman,fu2009Automatica,queahl10,leong2012power,GatsisACC13}).}
{For example, Fu and Souza~\cite{fu2009Automatica} demonstrated that the estimation quality improvement (in terms of reduction of the remote estimation error) achieved by
increasing the number  of the quantization bits is marginal when  is sufficiently large (in their example  only needs to be greater or equal to 4.} Based on A.3, the communication channel can be characterized by a random process , where\vspace{-1mm}

initialized with .
{Denote }
Let  be the transmission power for the QAM symbol
at time .
We adopt the wireless communication channel model used in~\cite{Liyuzhe13CDC}, and have

where  is given by 
 is the AWGN noise power spectral
density,  is the channel bandwidth, and  is a
constant that depends on the specific modulation being used.
To send local estimates to the remote estimator, the sensor chooses from a continuum of available power levels , see
Fig.~\ref{fig:system}. Note that different power levels lead to different
dropout rates, thereby affecting estimation performance.


\subsection{Remote State Estimation} \label{sec:remote-state-estim}
Define  as the information available to the remote estimator up to time , i.e.,

 Denote  and  as the remote estimator's own MMSE state estimate and the corresponding estimation error covariance, i.e.,
   and
  ,
{where expectations are taken with respect to a fixed
power controller.}
We
assume that the remote estimator feedbacks acknowledgements  before time . Such setups are common especially when the remote
estimator (gateway) is an energy-abundant device.
This energy asymmetry allows the estimator to trade energy cost for estimation
accuracy.



\section{Data-driven Transmission Power Control}\label{sec:transmission-power-control}
Our strategy uses
the measurements  to assign
transmission power level efficiently.
{As focusing on how the power controller utilize the
sensor's real-time data,
to simplify discussion, we assume a constant power
baseline  in this section.
We define

as a transmission power controller over the entire time horizon, where
 is a mapping from  and  to .
Before proceeding to study , let us first briefly explain the idea of data-driven power control mechanism.
Define  as the holding time since
the most recent time when the remote estimator received the data from the sensor, i.e.,

We interchange  with  when the underlying time index is clear from the context.
Define  as the incremental innovation in the sensor local state estimate compared to time , the previous reception instant, i.e.,

\begin{lemma} \label{lemma:z_k}

\begin{pf*}{Proof:}
{
The result follows from noting that

where the second equality holds because  is independent of , and
the last equality holds since 
}
\end{pf*}
\end{lemma}
Note that, if , then the sensor generates a local estimate,
 identical to the prediction . We would say that, for the remote estimator, the
 ``value" of information contained in  is null.
As  becomes larger,
 has an increasing drift
from the prediction  and the importance of the sensor sending  thereby
raises. Motivated by these observations, we define a stationary power controller, ,
as an increasing function
of . To fit the above observations, we introduce a quadratic function of  given by

where  is a weight matrix.
According to~Lemma~\ref{lemma:z_k},
the covariance of  is a function of .
Therefore we specify  for the index
of 
and construct the following controller:

In contrast to~\eqref{def:quadratic_schedule}, most non-data-driven transmission power controllers
(i.e.,~\cite{queahl10,leong2012power})
use a given power 
regardless of what value  takes.
Note that in~\eqref{def:quadratic_schedule} a constant term  is added
after .
If one sets , then the transmission
with the baseline power controller  is a special case of the proposed
transmission power controller.
As for ,
the transmission power is a constant  if ; otherwise it is adapted according to .
{Compared with a related controller proposed earlier
in~\cite{Liyuzhe13CDC},  in~\eqref{def:quadratic_schedule} is more general at least
from two aspects: 1) we introduce a weight matrix  to highlight
the roles of different entries of ; 2) it allows
the sensor to transmit using a standard power  even
if , which
includes a non-data-driven power transmission as a special case.}
As shown later in Lemma~\ref{lemma:still-gaussian}, given
,
 is zero-mean Gaussian with a
covariance  depending on , i.e.,

For convenience of our subsequent analysis, we define a new parameter  satisfying

where .
We now list the main
problems considered in the remainder of this work,
\begin{enumerate}
\item Under  defined in~\eqref{def:quadratic_schedule}, what is the
    MMSE estimate and its associated
    estimation error covariance?
\item What value should  (or ) take in order
to minimize, , the expected estimation error at the remote estimator?
\end{enumerate}
The solution to the first problem is presented in
Section~\ref{subsec:bayesian-reference-power-schedule}.
A sub-optimal solution to the second one is given
in Section~\ref{subsec:opt-bayesian-reference-schedule} in
view of the difficulty of the optimization problem.


{Before proceeding, we note that in previous works such as~\cite{GatsisACC13}
the difficulty of
using the information contained in lost packets, i.e., , when computing
the MMSE estimate of the plant state has been acknowledged. One typically discards such information as was done in~\cite{GatsisACC13} or resorts to approximations, e.g., treating a truncated Gaussian distribution as a Gaussian distribution as was done in~\cite{wu2013event}. These approaches either lead to conservative results (due to the unutilized information) or inaccurate results (due to approximations).
Our method, on other hand, makes use of the information contained in the event  to improve the estimation performance. The associated MMSE estimate, relying on no approximation techniques, is derived in a closed-form.}

\section{Main Results} \label{section:analysis}


\subsection{Preliminaries}\label{subsec:preliminaries}


For any  that is singular,
there exist
matrices 
such that

where  is unitary, whose columns
are right eigenvectors of ,
and , where 
is a diagonal matrix generated by the
corresponding nonzero eigenvalues
of . Let . Then 

Generally speaking, an -dimensioned random vector
,
does not have a pdf
with respect to the Lebesgue measure
on 
if some entries in  degenerate
to almost surely constant random variables.
To work with such vectors, one can instead consider Lebesgue measure in the -dimension affine subspace: , with respect to which
 has a pdf

where .
Without loss of generality, in the remainder of this paper,
for a random variable
 with a singular , the pdf of  means
the probability density on .
Note that the Moore-Penrose
pseudoinverse of  is unique and given
by

and that the
pseudo-determinant of  equals to the
product of all nonzero eigenvalues of .


Consider the power control law  defined in~\eqref{def:quadratic_schedule}.
In order to guarantee that  is always nonnegative for any value , the difference of
 and  needs to
be at least positive semi-definite, i.e.,
two conditions must be simultaneously satisfied, which
are   and  The following lemma
provides a necessary condition that  needs to satisfy.
\begin{lemma}\label{lemma:sigma-psi-property}
Suppose  and  satisfy  and . Then

and

where  is the image of .
\begin{pf*}{Proof:}
Since , it is true that
.
To verify \eqref{eqn:idential-rank1}, suppose that .
Then from~(\ref{eqn:eigen-decompostion}), ,
which contradicts with .
To prove~\eqref{eqn:same-image}, let us
denote  and  assume there is a set of vectors 
such that

Suppose . Then
there exists a vector in
 (without
loss of generality, let it be ),
and a vector 
where the operator  is the kernel of a matrix , such that .
It leads to the fact that
.
We in turn have

which contradicts with .
\end{pf*}
\end{lemma}
For convenience, denote
,  and

One has next lemma, the proof provided in the Appendix.
\begin{lemma}\label{lemma:rank-Phi}
The rank of  equals that of  (or ), i.e.,

\end{lemma}
\begin{example}
Two matrices are provided below
as a simple example for ,

We can verify that , ,
,~(\ref{eqn:idential-rank1}), and Lemma~\ref{lemma:sigma-psi-property} holds.
\end{example}












\subsection{MMSE State Estimate}\label{subsec:bayesian-reference-power-schedule}

In general, the posterior distribution of  fails to maintain Gaussianity without
analog-amplitude observations. The defect is
especially common for quantized Kalman filtering and Gaussian filters, where
it is tackled by Gaussian approximation \cite{andmoo79,kotecha2003gaussian,soi-kf-tsp06}.
By contrast,
the following lemma shows that,
using  in \eqref{def:quadratic_schedule},
the distribution of  conditioned on

is Gaussian. The proof, similar to that of Lemma 3.5 in~\cite{Liyuzhe13CDC}, is omitted.
\begin{lemma} \label{lemma:still-gaussian}
Under  defined in~\eqref{def:quadratic_schedule}, given ,
 follows a Gaussian distribution:

where  is given by the following recursion:

with . It is also true that, given
 and ,

\end{lemma}
\begin{proposition} \label{lemma:communication_rate}
Under  defined in~\eqref{def:quadratic_schedule}, given ,
the packet drop rate at time  is given by 
\end{proposition}
We denote the packet arrival rate
as  where
the subscript  is to
emphasize that it depends on  and .
To ensure that the averaged transmission power with respect to different values
taken by the measurement in  does not exceed
, i.e.,
, we require the following result.
\begin{lemma} \label{lemma:energy beta}
Under ~\eqref{def:quadratic_schedule}, given , the relation between  and , and  is given by

\begin{pf*}{Proof:}
From Lemma \ref{lemma:still-gaussian}, we know that

Under , we have:

\end{pf*}
\end{lemma}
With  defined in \eqref{def:quadratic_schedule}, the remote estimator computes  and  according to the following two theorems.
\begin{theorem}\label{theorem:mmse-proof}
Under ~\eqref{def:quadratic_schedule},
the remote estimator computes  as

where  is updated as  when .
\begin{pf*}{Proof:}
    When , the result is straightforward
    since  is the MMSE estimate of  given
     . Now consider . The tower rule gives
    
    Lemma~\ref{lemma:still-gaussian} leads to
    .
    \end{pf*}
\end{theorem}
\begin{theorem} \label{theorem:form_P_k_on}
Under ~\eqref{def:quadratic_schedule},
 at
the remote estimator is updated as

\begin{pf*}{Proof:}
    When  the result is straightforward.
    We only prove the case when .
    
    where the third equality is due to
    Lemma~\ref{lemma:z_k} and the last one
    is from Lemma~\ref{lemma:still-gaussian}.
\end{pf*}
\end{theorem}
\begin{remark}
{Under a baseline power controller with a constant power control , the remote estimator's estimate still obeys the recursion~\eqref{eqn:mmse-estimate}; however, the estimation error covariance is updated differently:  when , and  when ). Note that although the obtained estimates under the two power controllers are the same, their different estimation error covariance matrices suggest different confident levels with which the remote estimator trusts the obtained estimate: with the data-driven power controller, it is more convinced that the obtained estimate is close to the real state while less convinced with a non-data-driven power controller.}
\end{remark}
\subsection{{Selection of Design Parameters}}
\label{subsec:opt-bayesian-reference-schedule}
The performances of  for different
's are difficult to compare in general.
However, for  and
, there must exist a
real number 
such that  and
.
Observe that

which yields

In light of
(\ref{eqn:evolution_sigma_psi}),
we further have

According to Proposition~\ref{lemma:communication_rate},
it can be seen given  that  has an
upper bound:

Instead of minimizing ,
we minimize its upper bound which is equivalent to
minimize . Iterating
over time, one eventually needs to minimize  for
any  at any .
To this end, we propose to assign parameters of  in
\eqref{def:quadratic_schedule} as the solution to the following
optimization problem:
\begin{problem}\label{problem:optimization_online}
  
\end{problem}
The constraint is imposed by~(\ref{eqn:expected-power}). To solve Problem~\ref{problem:optimization_online}, we
first note that

However, for any matrix ,
 does not hold in general since  means 's
pseudo-determinant (in case  is singular). Fortunately, this property still holds for  and .
The proof is given in the Appendix.
\begin{lemma}\label{lemma:det-transform}
Suppose  and 
satisfy  and
.
Then 
\end{lemma}
From linear algebra,
,
and 
We simply write
 as ,
and denote the nonzero
eigenvalues of  by
. Then
Problem~\ref{problem:optimization_online}
can be recast as
\begin{problem}\label{problem:transform_optimization_online}
      
\end{problem}
\begin{lemma}\label{lemma:equal-eig}
Let  be the optimal solution to Problem~\ref{problem:transform_optimization_online}.
Then  satisfies

\begin{pf*}{Proof:}
Suppose that

is the optimal solution to
Problem~\ref{problem:transform_optimization_online}
but does not satisfy~\eqref{eqn:opt-eig-equal}.
We will show that there must exist another
vector, which is different from  and has
a smaller cost function~\eqref{eqn:objective-func-opt}.
Let  where
 is a positive constant.
Due to the fact that
 in  is the minimum eigenvalue of
 and the inequality of arithmetic
and geometric means, we have
 and 
the equalities simultaneously
satisfied when . Thus,  results in a smaller value
of~\eqref{eqn:objective-func-opt}, which
contradicts with the assumption and
completes the proof.
\end{pf*}
\end{lemma}
The following lemma is a result of Lemma~\ref{lemma:equal-eig}. Its proof is
presented in the Appendix.
\begin{lemma}\label{thm:opt-sol}
If , then the optimal
solution to Problem~\ref{problem:transform_optimization_online}
is  and

Otherwise, if , the optimizer is  and

\end{lemma}
Denote by  the transmission power associated with
the solution to Problem~\ref{problem:transform_optimization_online}. Then we have the following theorem. It can be readily
verified from Lemma~\ref{thm:opt-sol}.
\begin{theorem}\label{thm:opt-schedule}
If , then 
is given by

where
 with .
Otherwise, if 
 is given by

where .
\end{theorem}
\begin{remark}\label{remark:better-performance}
{A non-data-driven baseline power controller
with a constant power level  is
feasible to
Problem~\ref{problem:optimization_online}. Since
 is the optimal solution, it has not
worse state estimation performance
compared with the alternative non-data-driven
power controller. Numerical examples in
Section~\ref{sec:simulation} demonstrate performance improvements using
 compared with the non-data-driven power controller.}
\end{remark}
The following proposition shows that the rank of  can be calculated offline. The proof is given in the Appendix.
\begin{proposition}\label{proposition: rank}
Consider the  given in Theorem~\ref{thm:opt-schedule},
for any ,  can be calculated
as:

In particular, when , the dimension of , 
becomes a constant which is given by:

\end{proposition}

\subsection{Extension }\label{subsec:vary-power-in-time}
 In many
cases, the base-line power controller
changes over time with
respect to different
settings. For example, in~\cite{queahl10}, block fading channels were taken
into account. To deal with a time-varying channel power gain \footnote{
The term ``channel power gain" means the square of the magnitude of the complex channel.}, a predictive power control algorithm
was established, which
determines the transmission power level, bit rates and
codebooks used by the sensors.
{
The algorithm in~\cite{queahl10} requires
that the receiver (i.e, the remote estimator) runs a channel gain predictor, see e.g.,~\cite{linahl02}.}
{A key observation is that the data-driven controller proposed in the present work can be readily adapted to situations where the baseline controller provides time-varying power levels .\footnote{{ Following assumptions commonly made in the literature, see, e.g., \cite{queahl10,quevedo2012kalman}, in the sequel we shall assume that the channel gain  is available via the one-step ahead channel gain predictor.}}}
In fact, by solving Problem~\ref{problem:transform_optimization_online}
for a time-varying power
baseline , we obtain the optimal
solution  as
follows:~If , then 
is given by

and
.
Otherwise, if 
 is given by

and  In both cases, .
Note that ,  and  are calculated similar to ,
 and  given in Theorem~\ref{thm:opt-schedule}.
To reduce the sensor's computational load, the sensor only needs to
calculate the quadratic form ,
while the rest of the paraments are updated and then sent to the sensor by the estimator.
Note that calculating 
has a complexity of .}






\section{Simulation and Examples}
\label{sec:simulation}
{
Consider a system with parameters as follows:

}
{We first assume that 
has a constant power baseline
 and 
In Section~\ref{subsec:Comparison under Time-varying channel power gains},
a time-varying power baseline is considered.}




\begin{figure}[thp]
  \centering
  \includegraphics[width=7cm]{simulation_2}
  \caption{Empirical estimation covariance provided by  controllers  and  as a function of energy constraint .} \label{fig:simulation_2}
  \vspace{-1mm}
\end{figure}

\subsection{Comparison with Different Energy Constraints}
We compare our proposed schedule
 (denoted as )
with a constant baseline power controller
within the entire time horizon (denoted as ).
Define

as the empirical approximation (via 100000 Monte Carlo simulations) of the average expected state error covariance (denoted as ). We choose  as an approximation of .

Fig. \ref{fig:simulation_2} shows that  leads to a better system performance when compared to  under the same energy constraint.

\subsection{Comparison under Fading Channels }\label{subsec:Comparison under Time-varying channel power gains}
In practice, wireless communication channels
typically comprise fading often assumed to be Rayleigh \cite{rappaport1996wireless}, i.e., the channel power gain  is exponentially distributed with 
where  and  is the mean of .
{Truncated channel inversion transmit
power controllers have been studied in several works
\cite{quevedo2012kalman,leong2012power,goldsmith1997capacity}, where the transmission power is the inversion of ,
with a truncated boundary. In this subsection, we use the baseline power
determined by truncated channel gain inversion }
Denote the truncated channel inversion transmission power controller as :

where  and  are design parameters.
Consider the case of  and set . Based on the results in~\cite{leong2012power}, we can choose  to meet the energy constraint.
{Fig.~\ref{fig:simulation_3} suggests that  leads
to better system performance when compared with . Fig.~\ref{fig:simulation_4}
shows the comparison given a specific realization of channel power gains.
\begin{figure}[thp]
  \centering
  \includegraphics[width=7cm]{simulation_3}
  \caption{Comparison of  and  under Rayleigh fading.} \label{fig:simulation_3}
  \vspace{-1mm}
\end{figure}
\begin{figure}[thp]
  \centering
  \includegraphics[width=7cm]{simulation_4}
  \caption{Comparison of  and  given a specific realization of channel power gains.} \label{fig:simulation_4}
  \vspace{-1mm}
\end{figure}

\section{Conclusion}\label{sec:conculsion}
We proposed a data-driven transmission power controller for remote state
estimation, which adjusts the sensor's transmission power according to
its real-time measurements. Then we proved that the proposed power
 controller preserves Gaussianity of the incremental innovation and provided a closed-form
expression of the expected state estimation error covariance.
a tuning method for parameter design was presented to guarantee
that the data-driven power controller
not worse performance than the
alternative non-data-driven ones.
Comparisons were conducted to illustrate estimation performance improvement.

\section*{Appendix}




\textit{Proof of Lemma~\ref{lemma:rank-Phi}:~}
To verify the clain, it
suffices to show that .
Suppose that . Since ,
there must exist exactly  mutually orthogonal vectors

such that

Denote the unit vector with
only the th
entry being  by
, that is,

Since
 without loss of generality, let
.
As we assume that  is orthogonal to

it is true that .
Since  is nonsingular and ,
we have .
We then observe that

and

which contradicts with .
\hfill


\textit{Proof of Lemma~\ref{lemma:det-transform}:~}
By definition, it is easy to see that

Therefore we only need to prove

Observe that
 and  can be factorized
as
 and
,
where
 and 
are diagonal matrices generated
respectively
by the nonzero eigenvalues of  and
. For ,
 and  are the eigenvectors
associated with
 and
. In addition,

and . Then  can be
written as

where ,
 and
.
According to Lemma 4.3,
 is nonsingular, so
.
Since  from~\eqref{eqn:same-image},
there exists a
unitary matrix  such that . Thus,

which completes the proof. \hfill 

\textit{Proof of Lemma~\ref{thm:opt-sol}:~}
According to Lemma~\ref{lemma:equal-eig}, we
set . Logarithm does not change the
monotonicity of~\eqref{eqn:objective-func-opt}.
Problem~\ref{problem:transform_optimization_online}
is consequently transformed to\vspace{-2mm}

Substituting  into ~\eqref{eqn:derivative}
and taking derivative, it yields that the minimum
of~\eqref{eqn:derivative} is attained at
. Meanwhile
 needs to be nonnegative, so
the optimal solution to Problem~\ref{problem:optimization_online} is
\eqref{eqn:opt_sol1} if  or \eqref{eqn:opt_sol2} otherwise.
\hfill 


\textit{Proof of Proposition~\ref{proposition: rank}:~}
Consider a matrix  with
.
We have

which leads to the first assertion.
By the Cayley-Hamilton theorem,
we have

where  are coefficients of the
characteristic polynomial of .
When , we have

The last assertion follows from the reasoning used in~\eqref{eq:Image-eqn}.
\hfill 

\bibliographystyle{IEEETran}
\bibliography{reference1,dquevedo,sj_reference}

\end{document}
