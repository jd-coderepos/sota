\documentclass{llncs}
\usepackage{llncsdoc}
\usepackage{epsf}
\usepackage{amsfonts}
\usepackage{tabularx}
\usepackage{mathrsfs}
\usepackage{amsmath}
\usepackage{multirow}
\usepackage{graphicx}
\usepackage{mathtools}

\newcommand\nc{\newcommand}

\def\qed{\rule{1.5mm}{3mm}}

\nc{\crl}[2]{\begin{corollary}\label{crl:#1} #2 \end{corollary}}
\nc{\lem}[2]{\begin{lemma}\label{lem:#1} #2 \end{lemma}}
\nc{\prp}[2]{\begin{proposition}\label{prp:#1} #2 \end{proposition}}
\nc{\thm}[2]{\begin{theorem}\label{thm:#1} #2 \end{theorem}}
\nc{\cnj}[2]{\begin{conjecture}\label{cnj:#1} #2 \end{conjecture}}
\nc{\que}[2]{\begin{question}\label{que:#1} #2 \end{question}}
\nc{\pro}[2]{\begin{problem}\label{pro:#1} #2 \end{problem}}
\nc{\dfn}[2]{\begin{definition}\label{def:#1} #2 \end{definition}}

\nc{\llem}[3]{\begin{lemma}[#2]\label{lem:#1} #3 \end{lemma}}
\nc{\tthm}[3]{\begin{theorem}[#2]\label{thm:#1} #3 \end{theorem}}
\nc{\ddfn}[3]{\begin{definition}[#2]\label{def:#1} #3 \end{definition}}

\nc{\fig}[4]{\begin{figure}[hbt]
\centerline{
\epsfysize=#2in
\epsffile{#4}
}
\caption{#3}
\label{fig:#1}
\end{figure}}

\nc{\tbl}[3]{\begin{table}[hbt]
#3
\caption{#2}
\label{tab:#1}
\end{table}}

\newcommand{\tabincell}[2]{\begin{tabular}{@{}#1@{}}#2\end{tabular}}
\nc{\refc}[1]{Corollary~\ref{crl:#1}}
\nc{\reff}[1]{Figure~\ref{fig:#1}}
\nc{\refj}[1]{Conjecture~\ref{cnj:#1}}
\nc{\refl}[1]{Lemma~\ref{lem:#1}}
\nc{\refp}[1]{Proposition~\ref{prp:#1}}
\nc{\reft}[1]{Theorem~\ref{thm:#1}}
\nc{\refq}[1]{Question~\ref{que:#1}}
\nc{\refb}[1]{Problem~\ref{pro:#1}}
\nc{\reffct}[1]{Fact~\ref{fct:#1}}
\nc{\reftb}[1]{Table~\ref{tab:#1}}

\begin{document}

\title{ 
\Large \bf  Finding Two Edge-Disjoint Paths with Length Constraints
}
\author{
	Leizhen CAI\thanks{Partially supported by GRF grant CUHK410212 of the Research Grants Council of Hong Kong.} and Junjie YE}
\institute{Department of Computer Science and Engineering, The Chinese University of Hong Kong, Shatin, Hong Kong SAR, China \\
	\email{\{lcai,jjye\}@cse.cuhk.edu.hk}
}
\maketitle

\begin{abstract}
	We consider the problem of finding, for two pairs 
	and  of vertices in an undirected graphs, an -path 
	and an -path  such that  and  share no edges and
	the length of each  satisfies , where
	.
	
	We regard  and  as parameters and investigate the parameterized complexity
	of the above problem when at least one of  and  has a length constraint
	(note that  indicates that  has no length constraint).
	For the nine different cases of , we obtain FPT algorithms for seven of them.
	Our algorithms uses random partition backed by some structural results.
	On the other hand,  we prove that the problem admits no polynomial kernel for
	all nine cases unless . \\

	\noindent {\bf Keywords:} Edge-disjoint paths, random partition, parameterized complexity,
	kernelization.
\end{abstract}

\section{Introduction}

Disjoint paths in graphs are fundamental and have been studied extensively in the literature. 
Given  pairs of {\em terminal vertices}  for  in an undirected 
graph , the classical {\sc Edge-Disjoint Paths} problem asks whether  contains  
pairwise edge-disjoint paths  between  and  for all . 
The problem is NP-complete as shown by Itai et al.~\cite{itai1982complexity},
but is solvable in time  by network flow~\cite{orlin2013max} 
if all vertices  (resp.,  ) are the same vertex  (resp.,  ). 
When we regard  as a parameter, a celebrated result of Robertson and 
Seymour~\cite{robertson1995graph} on vertex-disjoint paths can be used to
obtain an FPT algorithm for {\sc Edge-Disjoint Paths}.
On the other hand, Bodlaender et al.~\cite{bodlaender2011kernel} have shown that 
{\sc Edge-Disjoint Paths} admits no polynomial kernel unless .  

In this paper, we study {\sc Edge-Disjoint Paths} with length constraints  on 
-paths  and focus on the problem for two pairs of terminal vertices.
The length constraints  indicate
that the length of  need to satisfy .
We regard  and  as parameters, and study the parameterized complexity 
of the following problem.

\begin{quote}
	{\sc Edge-Disjoint -Paths}
	
	{\sc Instance}: Graph , two pairs  and  of vertices.
	
	{\sc Question}: Does  contain -paths  for  
	such that  and  share no edge and the length of  satisfies ?
\end{quote}

There are nine different length constraints on two paths 
(note that {\sc Edge-Disjoint -Paths} puts no length constraint on two paths). 
For instance, {\sc Edge-Disjoint -Paths}
requires that  but  has no length constraint,
and {\sc Edge-Disjoint -Paths} requires that  and . \\

\noindent {\bf Related Work.} {\sc Edge-Disjoint -Paths} has been studied 
under the framework of classical complexity. 
Ohtsuki~\cite{ohtsuki1980two}, Seymour~\cite{seymour1980disjoint}, 
Shiloah~\cite{shiloach1980polynomial}, and Thomasssen~\cite{thomassen19802} independently 
gave polynomial-time algorithms for {\sc Edge-Disjoint -Paths}.
Tragoudas and Varol~\cite{tragoudas1997computing} proved the NP-completeness of
{\sc Edge-Disjoint -Paths},
and Eilam-Tzoreff~\cite{eilam1998disjoint} showed the NP-completeness of
{\sc Edge-Disjoint -Paths} even when  equals the -distance.
For {\sc Edge-Disjoint -Paths} with  or  
(same for  or ), we can easily establish its NP-completeness by 
reductions from the classical {\sc Hamiltonian Path} problem.

As for the parameterized complexity, there are a few results in connection with our
{\sc Edge Disjoint -Paths}.
Golovach and Thilikos~\cite{golovach2011paths} obtained an -time
algorithm
for {\sc Edge Disjoint Paths} when every path has length at most .
For a single pair  of vertices, Fomin et al.~\cite{fomin2014efficient}
gave the currently fastest -time algorithm for finding
an -path of length exactly , if it exists.
For the problem of finding an -path of length at least ,
Bodlaender~\cite{bodlaender1993linear} derived an -time algorithm,
Gabow and Nie~\cite{gabow2008finding} designed an -time algorithm,
and a recent FPT algorithm of Fomin et al.~\cite{fomin2014efficient} for cycles
can be adapted to yield a -time algorithm. \\

\noindent {\bf Our Contributions.} 
In this paper, we investigate the parameterized complexity of {\sc Edge-Disjoint -Paths}
for the nine different length constraints and have obtained FPT algorithms for seven of them 
(see \reftb{summary} for a summary).

In particular, we use random partition in an interesting way to obtain FPT algorithms for
{\sc Edge-Disjoint -Paths} and {\sc Edge-Disjoint -Paths}.
This is achieved by bounding the number of some special edges, called ``nearby-edges'',
in the two paths  and  by a function of  and  alone.
We also consider polynomial kernels and prove that all nines cases admit no polynomial kernel 
unless . \\

\tbl{summary}{Running times of FPT algorithms for {\sc Edge-Disjoint -Paths} with 
	length constraints  for .
	Note that , , and . }{
	\setlength{\extrarowheight}{4pt}{
		\begin{tabular}{|p{2.0cm}|p{2.5cm}|p{2.0cm}|p{2.0cm}|p{2.5cm}|}
			\hline
			\multicolumn{1}{|c|}{ Constraints } & \multicolumn{1}{c|}{  } & 
			\multicolumn{1}{c|}{  } & \multicolumn{1}{c|}{  } & 
			\multicolumn{1}{c|}{  } \\
			\hline
			\multicolumn{1}{|c|}{  } & 
			\multicolumn{1}{c|}{  } & & 
			\multicolumn{1}{c|}{ \multirow{2}{*}{  }} & 
			\multicolumn{1}{c|}{  } \\
			\cline{1 - 2}
			\cline{5 - 5}
			\multicolumn{1}{|c|}{  } & 
			\multicolumn{2}{c|}{  } & & 
			\multicolumn{1}{c|}{  } \\
			\hline
			\multicolumn{1}{|c|}{  } & 
			\multicolumn{2}{c|}{  }  & 
			\multicolumn{2}{c|}{ Open } \\
			\hline
		\end{tabular}
	}
	\vspace*{4pt}
}

\noindent {\bf Notation and Definitions.} All graphs in the paper are simple undirected connected graphs. 
For a graph , we use  and  to denote 
its vertex set and edge set respectively,
and  and , respectively, are numbers of vertices and edges of . 
For two vertices  and , the distance between  and  is denoted by . 

An instance  of a parameterized problem  
consists of two parts: an input  and a parameter . 
We say that a parameterized problem  is fixed-parameter tractable (FPT) if 
there is an algorithm solving every instance  in time  
for some computable function . 
A kernelization algorithm for a parameterized problem  maps an instance  
in time polynomial in  into a smaller instance  
such that  is a yes-instance iff  is a yes-instance 
and  for some computable function . 
Problem  has a polynomial kernel if  is a polynomial function. 

For simplicity, we write  for  
as the latter is  for any constant  
and we choose . 
In particular, . 

In the rest of the paper, we present FPT algorithms for seven cases in Section 2, 
and show the nonexistence of polynomial kernels in Section 3. 
We conclude with some open problems in Section 4. 

\section{FPT algorithms}

Random partition provides a natural tool for finding edge-disjoint -paths
in a graph : We randomly partition edges of  to form two graphs  and ,
and then independently find paths  in  (resp.,  in )
whose lengths satisfy  (resp., ).

When our problem satisfies the following two conditions, the above approach
yields a randomized FPT algorithm and can typically be derandomized by universal sets.

\begin{enumerate}
	\item Whenever  has a solution, the probability of `` contains required  and
	 contains required '' is bounded above by a function of  and  alone.
	
	\item It takes FPT time to find required paths  in  and  in .
\end{enumerate}

Indeed, straightforward applications of the above method yield FPT algorithms for
{\sc Edge-Disjoint -Paths} when 
for .

\thm{}{
	{\sc Edge-Disjoint -Paths} can be solved in  time
	for , and  time for
	 or .
}
\proof{
	Let . 
	We randomly color each edge by color 1 or 2 with probability 1/2 to
	define a random partition of edges.
	Denote by , , the graph consisting of edges of color . 
	Then for all three cases of , the probability that both  and
	 contain required paths is at least  when
	{\sc Edge-Disjoint -Paths} has a solution.
	
	We can use BFS starting from  to determine 
	whether  contains an -path of length at most  in time , 
	and an algorithm of Fomin et al.~\cite{fomin2014efficient} to determine 
	whether  contains an -path of length exactly  in time .
	Furthermore, we use a family of -universal sets of size ~\cite{naor1995splitters} for derandomization. 
	Therefore {\sc Edge-Disjoint -Paths} can be solved in time 
	
	for , and time 
	
	for  or .
	\qed} \\

For other cases of , a random edge partition of  does not,
unfortunately, gurantee condition (1) because of the possible existence
of a long path in a solution.
To handle such cases, we will compute some special edges and then use
random partition on such edges to ensure condition (1).
For this purpose, we call a vertex  a {\em nearby-vertex}
if , and call an edge a {\em nearby-edge}
if its two endpoints are both nearby-vertices.
We will show that there exists a solution where the number of nearby-edges
is bounded above by a polynomial in  and  alone, which enables us to
apply random partition to nearby-edges to ensure condition (1) and hence to
obtain FPT algorithms.
We note that such a clever way of applying random partition has been used by
Cygan et. al~\cite{cygan2014parameterized} in obtaining an Eulerian graph by deleting at most  edges.

In the next two subsections, we rely on random partition of nearby-edges
to obtain FPT algorithms to solve {\sc Edge-Disjoint -Paths}
for the following four cases of :
 and .

\subsection{One short and one unconstrained}

In this subsection, we use random partition on nearby-edges to obtain FPT algorithms
for {\sc Edge-Disjoint -Paths} when  is 
or .
To lay the foundation of our FPT algorithms, we first present the following
crucial property on the number of nearby-edges in a special solution.
Recall that a nearby-vertex  satisfies  and
both endpoints of a nearby-edge are nearby-vertices.

\lem{disjoint-paths-2}{
	Let  and  be two pairs of vertices in a graph , 
	 an -path of length at most , 
	and  a minimum-length -path edge-disjoint from . Then 
	\begin{enumerate}
		\item all edges in  are nearby-edges, and
		
		\item  contains at most  nearby-edges.
	\end{enumerate}
}
\proof{
	Statement 1 is obvious and we focus on Statement 2. 
	
	For a vertex  in , 
	we say that  is a {\em -near vertex} if there is a vertex  in  such that
	 contains a -path of length at most  that is edge-disjoint from .
	We call  a {\em -near vertex} when we want to emphasize the endpoint , 
	and refer to such a -path as a {\em -near -path}. 
	
	Let  be a nearby-vertex in . 
	Since , 
	there is an -path or a -path of length at most . 
	As  and  are vertices of ,  must be a -near vertex. 
	Therefore each nearby-vertex in  is a -near vertex,
	and we bound the number of -near vertices to prove this lemma. 
	
	Suppose to the contrary that  contains at least  -near vertices. 
	Then by pigeonhole principle, there exists a vertex  in  
	that has at least  -near vertices. 
	Sort these vertices along  from  to . 
	Let  and  be the first and last vertex respectively. 
	Then the -section of  has length at least . 
	Let  be the -walk concatenating the -near -path 
	and the -near -path. 
	Then  contains at most  edges and is edge-disjoint from  by the definition of -near path. 
	So we can replace the -section by  to obtain an -walk 
	that contains an -path shorter than , contradicting to the minimality of . 
	Therefore  contains at most  -near vertices and thus nearby-vertices, 
	which implies that  contains at most  nearby-edges. 
	\qed} \\

The above lemma lays the ground for an FPT algorithm based on random partition.
Let  be a random partition of nearby-edges,
and construct  and .
Note that whenever  admits a solution, it has a solution  such
that  is a minimum-length -path edge disjoint from .
Lemma 1 implies that  is inside  with probability ,
and  is inside  with probability .
This ensures that, with probability ,
 contains an -path of length at most  and,
with probability at least ,   contains an -path.
Therefore with probability ,
we will be able to find a solution for  by finding an -path
of length at most  in  and an -path in .
This paves the way for the following randomized FPT algorithm for
{\sc Edge-Disjoint -Paths}.
Note that the algorithm also works for {\sc Edge-Disjoint -Paths} once we change ``length '' to ``length '' in Step 3. \\

{\bf Algorithm 1: }

\begin{enumerate}
	\item Find all nearby-edges in  time by two rounds of BFS, one from
	 and the other from .
	
	\item Randomly color each nearby-edge by color 1 or 2 with probability 1/2,
	and color all remaining edges of  by color 2.
	Let  () be the graph consisting of edges of color .
	
	\item Find an -path  of length  in , and an
	-path  in .
	Return  as a solution if both  and  exist,
	and return ``No'' otherwise.
\end{enumerate}

Algorithm 1 solves {\sc Edge-Disjoint -Paths}
with probability  and runs in  time,
as the two tasks in Step 3 for  and  also take  time.
Let  be the number of nearby-edges and .
We can use -universal sets to derandomize our algorithm, and obtain
a deterministic FPT algorithm running in time


For {\sc Edge-Disjoint -Paths}, Step 3 takes more
time as it takes  time to find
an -path  of length .
Therefore our deterministic FPT algorithm for the problem takes time


\thm{disjoint-paths-none-1}{
	{\sc Edge-Disjoint -Paths} and {\sc Edge-Disjoint -Paths} can be solved in time  and  respectively.
}

\subsection{One short and one long}

Now we consider {\sc Edge-Disjoint -Paths} when  is
 or .
The main difficulty lies in the possibility that one path may be long, and we
overcome this obstacle by the following lemma similar to \refl{disjoint-paths-2}
to upper bound the number of nearby-edges in a special solution.
Again, the lemma enables us to use random partition on nearby-edges to
obtain FPT algorithms for both cases.

For an -path , a {\em -valid -path} is an -path that is edge-disjoint from  and has length at least .

\lem{disjoint-paths-3}{
	Let  and  be two pairs of vertices in a graph , 
	 an -path of length at most , and 
	 a -valid -path of minimum length. Then
	
	\begin{enumerate}
		\item all edges in  are nearby-edges, and
		
		\item at most  edges of  are nearby-edges.  
	\end{enumerate}
}
\proof{
	Statement (1) is obvious and we focus on Statement (2). 
	
	For path , let  denote the section containing the first  vertices, 
	and  the section containing the remaining vertices. 
	We show that  contains at most  nearby-edges,
	which implies Statement (2) as the remaining part of , i.e., , has  edges.
	
	Let  be a vertex in .
	We say that  is a {\em -near vertex} (resp., {\em -near vertex}) 
	if there is a vertex  in  (resp., ) such that
	 contains a -path of length at most  that is edge 
	disjoint from  and vertex-disjoint from  (except ).
	We refer to such a -path as a {\em -near -path} 
	(resp., {\em -near -path}).
	
	Consider a nearby-vertex  in .
	Since ,  contains either an -path or a 
	-path of length at most .
	If this path contains a vertex  of  
	such that the -section is edge-disjoint from  and vertex-disjoint from  except , 
	then  is a -near vertex, and otherwise  is a -near vertex.
	Therefore all nearby-vertices in  are -near or -near vertices,
	and we will put an upper bound on the number of nearby-vertices 
	in  by limiting the numbers of -near and -near vertices.
	
	We can use the proof of \refl{disjoint-paths-2} to show that 
	{\em  contains at most  -near vertices},
	and we now prove that
	{\em  contains at most  -near vertices}.
	
	Suppose to the contrary that  contains at least  -near vertices.
	Let  denote the -th -near vertex in  
	when we travel along  from its other endpoint to .
	Let , and denote by  a -near -path
	for some vertex  in .
	Denote by  the first -near vertex in  when we travel along  
	from  to .
	
	Since all internal vertices of the -section  of  is vertex-disjoint from ,
	we can replace  by  to obtain an -path 
	(see Figure \ref{fig:cases} for illustration).
	Clearly,  is edge-disjoint from  as both  and  are edge-disjoint from . 
	We show in two cases that  to contradict the minimality of .
	
	Note that the -section  of  is vertex-disjoint from  and 
	edge-disjoint from . 
	It follows that if the -section  of  is longer
	than , we can replace  in  by  to obtain an 
	-walk  that is edge-disjoint from  and shorter than .
	Since the first  vertices of  are distinct vertices, we can obtain from 
	a -valid -path shorter than .
	Therefore we may assume that  by the minimality of .
	
	\begin{figure}[hbt]
		\centerline{
			\includegraphics[scale=1.00]{cases.eps}
		}
		\caption{Two cases for the intersections of  and . Note that  may share internal vertices with . }
		\label{fig:cases}
	\end{figure}
	
	{\bf Case 1:} .
	Clearly  as  contains 
	more than  -near vertices.
	On the other hand,  and .
	Therefore  
	
	and hence . 
	
	{\bf Case 2:} .
	Clearly  as ,
	and we show that .  
	Since ,  contains at most  -near vertices.
	Therefore  and  contains at least  -near vertices,
	implying .
	
	Therefore  has at most  -near vertices.
	Together with at most  -near vertices in  and  vertices in ,
	we conclude that  contains at most  -near and -near vertices,
	and hence at most  nearby-vertices/edges.
	\qed} \\


The above lemma enables us to obtain a randomized FPT for {\sc Edge-Disjoint } 
by replacing Step 3 of Algorithm 1 as follows:

\noindent {\bf Step 3:} Find an -path  of length  in , and an
-path  of length  in .
Return  as a solution if both  and  exist,
and return ``No'' otherwise.

By \refl{disjoint-paths-3}, the randomized algorithm solves {\sc Edge-Disjoint -Paths}
with probability .  
Since an -path  of length  can be found in time ~\cite{fomin2014efficient} as mentioned earlier in the introduction, 
the two tasks in Step 3 takes  time and thus 
the randomized algorithm runs in the same time. 
Let  be the number of nearby-edges and .
We can use -universal sets to derandomize our algorithm, and obtain
a deterministic FPT algorithm for {\sc Edge-Disjoint -Paths} running in time


For {\sc Edge-Disjoint -Paths}, Step 3 needs to find
an -path  of length  which takes  time.
Therefore our deterministic FPT algorithm for the problem takes time


\thm{one-long-path}{
	Both {\sc Edge-Disjoint -Paths} and {\sc Edge-Disjoint -Paths} 
	can be solved in time .
}

\section{Incompressibility}

Having obtained FPT algorithms, we are impelled to investigate the existence of polynomial
kernels for {\sc Edge-Disjoint -Paths}.
Our findings are negative as we will show that, unless ,
the problem admits no polynomial kernel for all nine different cases of length
constraints .

We start with relaxed-composition algorithms defined
by Cai and Cai~\cite{cai2014incompressibility},
which is a relaxation of composition algorithms introduced by
Bodlaender et al.~\cite{bodlaender2009problems} in their pioneer work on
the nonexistence of polynomial kernels.

\ddfn{relaxed-composition}{relaxed-composition~\cite{cai2014incompressibility}}{
	A relaxed-composition algorithm for a parameterized problem 
	takes  instances  as input and,
	in time polynomial in , outputs an instance
	 such that
	\begin{enumerate}
		\item  is a yes-instance of  iff some 
		is a yes-instance of , and
		\item  is polynomial in .
	\end{enumerate}
}

Note that relaxed-composition algorithms relax the requirement in
composition algorithms~\cite{bodlaender2009problems} for parameter  from
polynomial in  to polynomial in .
As observed by Cai and Cai~\cite{cai2014incompressibility},
the following important result is implicitly established in
Bodlaender et al.~\cite{bodlaender2009problems}.

\tthm{relaxed-composition}{\cite{bodlaender2009problems,fortnow2011infeasibility,bodlaender2014kernelization}}{
	If an NP-complete parameterized problem admits a relaxed-composition algorithm,
	then it has no polynomial kernel, unless .
}


We also need the following polynomial parameter transformation (ppt-reduction in
short). 
\ddfn{ppt-reduction}{ppt-reduction~\cite{bodlaender2011kernel,cai2014incompressibility}}{
	A ppt-reduction from a parameterized problem  to another
	parameterized problem  is an algorithm that, for input , takes time polynomial
	in  and outputs an instance  such that
	\begin{enumerate}
		\item  is a yes-instance of  iff  is a yes-instance of , and 
		
		\item parameter  is bounded above by a polynomial of .
	\end{enumerate}
}

\tthm{ppt-reduction}{\cite{bodlaender2011kernel}}{
	If there is a ppt-reduction from a parameterized problem  to another parameterized problem , 
	then  admits no polynomial kernel whenever  admits no polynomial kernel.
}

Now we show the nonexistence of polynomial kernels for seven easy cases. 
We first use relaxed-composition to show the nonexistence of polynomial kernels of 
{\sc --Path} (resp., {\sc Long -Path}) 
that are NP-complete problems of finding an -path of length  (resp., ). 
Then we present ppt-reductions from these two problems to {\sc Edge-Disjoint -Paths} problems. 

\lem{incompressibility-1}{
	Both {\sc --Path} and {\sc Long -Path} admit no polynomial kernel unless .
}
\proof{
	Given  instances of {\sc --Path} with  and  
	being the two terminal vertices of the -th instance for , 
	we can relaxed-composite these  instances into one instance 
	by identifying  (resp.,  ) as one vertex for all . 
	Then, by \reft{relaxed-composition}, 
	{\sc --Path} admits no polynomial kernel unless . 
	By the same relaxed-composition, we can deduce that {\sc Long -Path} admits no polynomial kernel unless . 
} 

\thm{incompressibility-1}{
	{\sc Edge-Disjoint -Paths} for
	 being , , 
	, , ,  or ,  admits no polynomial kernel unless . 
}
\proof{
	Given an instance of {\sc --Path}, 
	we construct an instance of {\sc Edge-Disjoint -Paths} as following:
	\begin{enumerate}
		\item Set  and , and , 
		
		\item add new vertices  and , and edge .  
	\end{enumerate}
	
	The above reduction is clearly a ppt-reduction, 
	and thus {\sc Edge-Disjoint -Paths} 
	admits no polynomial kernel unless . 
	For the other six cases, similar ppt-reductions  
	from {\sc --Path} or {\sc Long -Path} will work. 
	\qed} \\

Now we consider the remaining two cases of length constraints 
and .
Following our argument for the other cases, we can easily construct ppt-reductions from
the problem of determining whether  contains an -path of length at most .
Unfortunately, this short path problem is solvable in polynomial time and
thus admits a polynomial kernel, which makes such ppt-reductions meaningless
for the purpose of proving the nonexistence of polynomial kernels.
In fact, these two cases are difficult to deal with, and we will design
delicate relaxed-composition algorithms to establish the nonexistence of
their polynomial kernels.

\thm{incompressibility-2}{
	Both {\sc Edge-Disjoint -Paths} and {\sc Edge-Disjoint -Paths} admit no polynomial kernel unless . 
}
\proof{
	Let  be  instances of {\sc Edge-Disjoint -Paths}, and . 
	Let  and  be the two pairs of vertices of the -th instance for . 
	Assume that  is a power of two, say . 
	Otherwise we can add some redundant no-instances to make  a power of two. 
	
	We first show how to composite two instances into one instance, 
	which is the crucial step of our relaxed-composition. 
	Given the -th instance and -th instance, 
	we construct a new instance  as following 
	(See \reff{incompressibility} for an illustration.): 
	\begin{enumerate}
		\item Create two pairs of vertices  and , 
		and four vertices  and . 
		
		\item Connect these new vertices with graph  and  as showed in \reff{incompressibility}, 
		where each dashed/dotted edge is a {\em short-path} of length one, and 
		each normal edge is a {\em long-path} of length . 
		
		\item Denote by  the new graph and set , . 
	\end{enumerate}
	
	\begin{figure}[hbt]
		\centerline{
			\includegraphics[scale=1.00]{incompressibility.eps}
		}
		\caption{The relaxed-composition for two instances. Here a dashed/dotted edge is a short-path of length one, and a normal edge is a long-path of length .}
		\label{fig:incompressibility}
	\end{figure}
	
	We claim that  is a yes-instance iff one of these two instances is a yes-instance.
	
	Suppose that one of these two instances has a solution. 
	Without loss of generality, assume that  has a solution . 
	Let  be the -path concatenated by  and the four dashed short-paths, 
	and  be the -path going through  and , 
	whose -section is . 
	By the edge-disjointness between  and ,  and  are edge-disjoint. 
	Furthermore, we have  and  as  and . 
	Then  is a solution of . 
	
	Conversely, suppose that  is a solution of . 
	Since  has length at most , and each long-path has length , 
	 contains either all dotted short-paths or dashed short-paths. 
	Assume that  contains all dotted short-paths. 
	(The argument is similar when  contains all dashed short-paths.)
	Then the -section  of  is an -path in  of length at most . 
	Moreover,  must be an -path going through 
	the -long-path  and the -long-path .
	Since , the -section  of  has length at most 
	
	Then  is a solution of . 
	

	Now we are ready to present our relaxed-composition that contains  iterations.
	In the -th iteration, there are  instances and 
	we group these instances into  pairs for . 
	For each pair, we composite them into one instance as presented above. 
	Finally, there remains only one instance which completes the relaxed-composition.
	Let  be the length constraints after the -th iteration for . 
	Note that  and . 
	The recursion relation for  and  is 
	 
	as short-path and long-path respectively have length 1 and  in the -th iteration. 
	We have  and  for . 
	
	Let  be the final instance, 
	where  and . 
	By above proof for the composition of two instances, 
	we can deduce that  has a solution 
	iff one of these  instances has a solution. 
	Both  and  are polynomially bounded in  as . 
	This composition is a valid relaxed-composition. 
	Since {\sc Edge-Disjoint -Paths} is NP-complete, by \reft{relaxed-composition}, 
	it admits no polynomial kernel unless . 
	
	The relaxed-composition also holds if we discard the length constraint for the second path, i.e. 
	discard the length constraints  and , 
	which yields that {\sc Edge-Disjoint -Paths} 
	admits no polynomial kernel unless .
	\qed} 

\section{Concluding Remarks}

We have obtained FPT algorithms to solve {\sc Edge-Disjoint -Paths}
for seven of the nine different cases of length constraints ,
and also established the nonexistence of polynomial kernels for all nine cases,
assuming .
However parameterized complexities of the remaining two cases are open. 

\pro{edge-disjoint}{
	Determine the parameterized complexities of {\sc Edge-Disjoint -Paths} and
	{\sc Edge-Disjoint -Paths}. 
}

It is interesting to note that an FPT algorithm for
{\sc Edge-Disjoint -Paths} will actually yield a new polynomial-time
algorithm to solve {\sc Edge-Disjoint Paths} for two pairs of terminal vertices (i.e.,
{\sc Edge-Disjoint -Paths}).

We can consider vertex-disjoint paths, instead of edge-disjoint paths,
and form {\sc Vertex-Disjoint -Paths} problems for nine different length
constraints .
We note that it is straightforward to obtain FPT algorithms by color-coding
or random partition for the three cases of 
being  or ,
but the remaining six cases seem much harder than their corresponding edge-disjoint
counterparts.

\pro{vertex-disjoint}{
	Determine the parameterized complexity of {\sc Vertex-Disjoint -Paths}
	for the following six cases of : \\
	 and .
}

We note that structural properties similar to \refl{disjoint-paths-2} and
\refl{disjoint-paths-3}
seem not hold for vertex-disjoint paths with length constraints.
On the other hand, our proofs for the nonexistence of polynomial kernels for 
{\sc Edge-Disjoint -Paths} also work for {\sc Vertex-Disjoint -Paths}, and hence
{\sc Vertex-Disjoint -Paths} admits no polynomial kernel
unless  for all nine different cases of length constraints
.

Finally, we can consider both edge-disjoint and vertex-disjoint paths with length constraints
for digraphs, which appear to be much harder than these problems on undirected graphs.

\pro{digraph}{
	For digraphs, determine the parameterized complexity of
	{\sc Edge-Disjoint -Paths} and {\sc Vertex-Disjoint -Paths}
	for various length constraints .
}

\bibliographystyle{../splncs03}
\bibliography{disjoint-paths}

\end{document}