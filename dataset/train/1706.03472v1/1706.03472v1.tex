\documentclass{article}

\usepackage[dvipdfmx]{graphicx}
\usepackage{amsmath,amssymb,amscd,amsthm,bm}
\usepackage{color}
\usepackage[all]{xy}

\newtheorem{thm}{thm}[section]
\newtheorem{lem}[thm]{Lemma}
\newtheorem{prop}[thm]{Proposition}
\newtheorem{cor}[thm]{Corollary}
\newtheorem{exam}[thm]{Example}

\setlength{\evensidemargin}{0mm}
\setlength{\oddsidemargin}{0mm}
\setlength{\textwidth}{160mm}
\setlength{\topmargin}{-10mm}
\setlength{\textheight}{215mm}

\setlength{\evensidemargin}{-2mm}
\setlength{\oddsidemargin}{-2mm}
\setlength{\textwidth}{170mm}
\setlength{\topmargin}{-15mm}
\setlength{\textheight}{235mm}
\pagestyle{empty}

\makeatletter \@tempcnta\z@
\loop\ifnum\@tempcnta<26
\advance\@tempcnta\@ne
\expandafter\edef\csname f\@Alph\@tempcnta\endcsname{\noexpand\mathfrak{\@Alph\@tempcnta}}
\expandafter\edef\csname l\@Alph\@tempcnta\endcsname{\noexpand\mathbb{\@Alph\@tempcnta}}
\expandafter\edef\csname c\@Alph\@tempcnta\endcsname{\noexpand\mathcal{\@Alph\@tempcnta}}
\expandafter\edef\csname b\@Alph\@tempcnta\endcsname{\noexpand\mathbf{\@Alph\@tempcnta}}
\repeat

\newcommand{\dataset}{{\cal D}}
\newcommand{\fracpartial}[2]{\frac{\partial #1}{\partial  #2}}

\newcommand{\dd}{{\delta}}
\newcommand{\DD}{{\Delta}}
\newcommand{\ee}{{\varepsilon}}
\newcommand{\pa}{{\partial}}
\newcommand{\one}{\mbox{1}\hspace{-0.25em}\mbox{q}}

\newcommand{\ra}{{\rightarrow}}
\newcommand{\la}{{\leftarrow}}
\newcommand{\Ra}{{\Rightarrow}}
\newcommand{\La}{{\Leftarrow}}
\newcommand{\Lra}{{\Leftrightarrow}}
\newcommand{\rra}{{\rightrightarrows}}
\newcommand{\hra}{{\hookrightarrow}}
\newcommand{\lra}{{\leftrightarrow}}
\newcommand{\llra}{{\longleftrightarrow}}

\newcommand{\tr}{{\rm tr}}
\newcommand{\cech}{{\rm \check{C}ech}}
\newcommand{\Ce}{{\rm \check{\mathcal{C} }}}
\newcommand{\Cech}{{\rm \check{\lC}ech}}
\newcommand{\Rips}{{\rm {\lR}ips}}
\newcommand{\Sub}{{\rm {\lS}ub}}
\newcommand{\Filt}{{\rm {\lF}ilt}}
\newcommand{\dist}{{\rm dist}}
\newcommand{\Pers}{{\rm Pers}}
\newcommand{\pers}{{\rm pers}}
\newcommand{\dis}{{\rm dis}}
\newcommand{\Hom}{{\rm Hom}}
\newcommand{\Lip}{{\rm Lip}}
\newcommand{\Amp}{{\rm Amp}}
\newcommand{\diam}{{\rm diam}}
\newcommand{\mesh}{{\rm mesh}}
\newcommand{\Del}{{\rm Del}}
\newcommand{\supp}{{\rm supp}}
\newcommand{\median}{{\rm median}}
\newcommand{\KFDR}{{\rm KFDR}}
\newcommand{\id}{{\rm id}}


\newcommand{\lmid}{ \ \middle| \ }

\providecommand{\abs}[1]{\left\lvert#1\right\rvert}
\providecommand{\norm}[1]{\left\lVert#1\right\rVert}
\providecommand{\pare}[1]{\left( #1 \right)}
\providecommand{\rl}[1]{\left\{ #1 \right\}}
\providecommand{\card}[1]{{\rm card}\pare{#1}}
\providecommand{\recip}[1]{\frac{1}{#1}}
\providecommand{\inn}[2]{\langle #1, #2 \rangle}
\providecommand{\Ek}[3]{E_{#1}(\mu^{#2}_{#3})}
\providecommand{\dk}[4]{\norm{E_{#1}(\mu^{#2}_{#3}) - E_{#1}(\mu^{#2}_{#4})}_{\cH_{#1}}}

\title{Kernel method for persistence diagrams via kernel embedding and weight factor}
\author{Genki Kusano \thanks{Tohoku University, genki.kusano.r5@dc.tohoku.ac.jp}
\and Kenji Fukumizu \thanks{The Institute of Statistical Mathematics, fukumizu@ism.ac.jp}
\and Yasuaki Hiraoka \thanks{Tohoku University, hiraoka@tohoku.ac.jp}}
\date{}
\begin{document}
\maketitle




\begin{abstract}

Topological data analysis is an emerging mathematical concept for characterizing shapes in multi-scale data. In this field, persistence diagrams are widely used as a descriptor of the input data, and can distinguish robust and noisy topological properties.  
Nowadays, it is highly desired to develop a statistical framework on persistence diagrams to deal with practical data. 
This paper proposes a kernel method on persistence diagrams. 
A theoretical contribution of our method is that the proposed kernel allows one to control the effect of persistence, and, if necessary, noisy topological properties can be discounted in data analysis. 
Furthermore, the method provides a fast approximation technique. The method is applied into several problems including practical data in physics, and the results show the advantage compared to the existing kernel method on persistence diagrams.
\end{abstract}

\section{Introduction}
\label{sec:intro}

Recent years have witnessed an increasing interest in utilizing methods of algebraic topology for statistical data analysis.
In terms of algebraic topology, conventional clustering methods are regarded as charactering -dimensional topological features which mean connected components of data.
Furthermore, higher dimensional topological features also represent informative shape of data, such as rings (-dimension) and cavities (-dimension).
The research analyzing these topological features in data is called {\em topological data analysis} (TDA) \cite{Ca09}, which has been successfully applied to various areas including information science \cite{CIdSZ08,dSG07}, biology \cite{KZPSGP07,XW14}, brain science \cite{LCKKL11,PETCNHV14,SMISCR08}, biochemistry \cite{GHIKMN13}, material science \cite{HNHEMN16, NHHEN15, STRFH17}, and so on.
In many of these applications, data have complicated geometric structures, and thus it is important to extract informative topological features from the data.

A {\em persistent homology} \cite{ELZ02}, which is a key mathematical tool in TDA, extracts robust topological information from data, and it has a compact expression called a {\em persistence diagram}.
While it is applied to various problems such as the ones listed above, statistical or machine learning methods for analysis on persistence diagrams are still limited.
In TDA, analysts often elaborate only single persistence diagram and, in particular, methods for handling many persistence diagrams, which can contain randomness from the data, are at the beginning stage (see the end of this section for related works).
Hence, developing a framework of statistical data analysis on persistence diagrams is a significant issue for further success of TDA and, to this goal, this paper discusses kernel methods for persistence diagrams.

\subsection{Topological descriptor}
\label{subsec:persistent_homology}

In order to provide some intuitions for the persistent homology, let us consider a typical way of constructing persistent homology from data points in a Euclidean space, assuming that the point set lies on a submanifold.
The aim is to make inference on the topology of the underlying manifold from finite data points.
We consider the -balls (balls with radius ) to recover the topology of the manifold, as popularly employed in constructing an -neighbor graph in many manifold learning algorithms.
While it is expected that, with an appropriate choice of , the -ball model can represent the underlying topological structures of the manifold, it is also known that the result is sensitive to the choice of .
If  is too small (resp. large), the union of -balls consists simply of the disjoint -balls (resp. a contractible space).
Then, by considering not one specific  but all , the persistent homology gives robust topological features of the point set.
\begin{figure}[htbp]
\begin{center}
\includegraphics[width=0.9\hsize]{filtration.pdf}
\caption{Unions of -balls at data points (left) and its -st persistence diagram (right). The point  in the persistence diagram represents the ring , which appears at  and disappears at . The noisy rings are plotted as the points close to the diagonal.}
\label{fig:filtration}
\end{center}
\end{figure}

As a useful representation of persistent homology, a persistence diagram is often used in topological data analysis.
The persistence diagram is given in the form of a multiset  (Figure \ref{fig:filtration}).
Every point , called a {\em generator} of the persistent homology, represents a topological property (e.g., connected components, rings, cavities, etc.) which appears at  and disappears at  in the ball model.
Then, the {\em persistence}  of the generator shows the robustness of the topological property under the radius parameter.
A generator with large persistence can be regarded as a reliable structure, while that with small persistence (points close to the diagonal) is likely to be a structure caused by noise.
In this way, persistence diagrams encode topological and geometric information of data points.
See Section \ref{sec:background} and Appendix \ref{sec:topology} for more information.

\subsection{Contribution}
\label{subsec:contribution}

Since a persistence diagram is a point set of variable size, it is not straightforward to apply standard methods of statistical data analysis, which typically assume vectorial data.
To vectorize persistence diagrams, we employ the framework of kernel embedding of (probability and more general) measures into reproducing kernel Hilbert spaces (RKHS).
This framework has recently been developed and leading various new methods for nonparametric inference \cite{MFSS17,SGSS07,SFG13}.
It is known \cite{SFL11} that, with an appropriate choice of kernels, a signed Radon measure can be uniquely represented by the Bochner integral of the feature vectors with respect to the measure.
Since a persistence diagram can be regarded as a sum of Dirac delta measures, it can be embedded into an RKHS by the Bochner integral.
Once such a vector representation is obtained, we can introduce any kernel methods for persistence diagrams systematically (see Figure \ref{fig:overview}). 
\begin{figure}[htbp]
\begin{center}
\includegraphics[width=0.95\hsize]{overview.pdf}
\caption{
(1) A data set  is transformed into a persistence diagram  (Section \ref{subsec:persistence_diagram}).
(2) The persistence diagram  is mapped to an RKHS vector , where  is a positive definite kernel and  is a weight function controlling the effect of persistence (Section \ref{subsec:vectorization}).
(3) Statistical methods are applied to those vector representations of  persistence diagrams (Section \ref{sec:experiment}).}
\label{fig:overview}
\end{center}
\end{figure}

Furthermore, since each generator in a persistence diagram is equipped with a persistence which indicates a robustness of the topological features, we will utilize it as a weight on the generator.
For embedding persistence diagrams in an RKHS, we propose a useful class of positive definite kernels, called {\em persistence weighted Gaussian kernel} (PWGK).
The advantages of this kernel are as follows:
(i) We can explicitly control the effect of persistence by a weight function, and hence discount the noisy generators appropriately in statistical analysis.
(ii) As a theoretical contribution, the distance defined by the RKHS norm for the PWGK satisfies the stability property, which ensures the continuity from data to the vector representation of the persistence diagram.
(iii) The PWGK allows efficient computation by using the random Fourier features \cite{RR07}, and thus it is applicable to persistence diagrams with a large number of generators.

We demonstrate the performance of the proposed kernel method with synthesized and real-world data, including granular systems (taken by X-ray Computed Tomography on granular experiments), oxide glasses (taken by molecular dynamics simulations) and protein datasets (taken by NMR and X-ray crystallography experiments).
We remark that these real-world problems have physical and biochemical significance in their own right, as detailed in Section \ref{sec:experiment}.

\subsection{Related works}
\label{subsec:related_work}

There are already some relevant works on statistical approaches to persistence diagrams.
Some studies discuss how to transform a persistence diagram to a vector \cite{AEKNPSCHMZ17, Bu15, CMWOXW15, COO15, RHBK15, RT16}.
In these methods, a transformed vector is typically expressed in a Euclidean space  or a function space , and simple and ad-hoc summary statistics like means and variances are used for data analysis such as principal component analysis (PCA) and support vector machines (SVMs).
In this paper, we will compare the performance among the PWGK, the persistence scale-space kernel \cite{RHBK15}, the persistence landscape \cite{Bu15}, the persistence image \cite{AEKNPSCHMZ17}, and the molecular topological fingerprint \cite{CMWOXW15} in several machine learning tasks.
Furthermore, we show that our vectorization is a generalization of the persistence scale-space kernel and the persistence image although the constructions are different.
We also remark that there are some works discussing statistical properties of persistence diagrams for random data points:
\cite{CGLM15} show convergence rates of persistence diagram estimation, and \cite{FLRWBS14} discuss confidence sets in a persistence diagram.
These works consider a different but important direction to the statistical methods for persistence diagrams.

The remaining of this paper is organized as follows:
In Section \ref{sec:background}, we review some basics on persistence diagrams and kernel embedding methods.
In Section \ref{sec:pdkernel},  the PWGK is proposed, and some theoretical and computational issues are discussed.
Section \ref{sec:experiment} shows experimental results and compares the proposed kernel method with other methods.

This paper is an extended version of our ICML paper \cite{KFH16}.
The difference from this conference version is as follows:
(i) Comparisons with other relevant methods, in particular, persistence landscapes and persistence images, have been added to this version.
(ii) New experimental results in comparison with other relevant methods.
(iii) Detailed proofs of the stability theorem has been added.

\section{Backgrounds}
\label{sec:background}

We review the concepts of persistence diagrams and kernel methods.  For readers who are not familiar with algebraic topology, we give a brief summary in Appendix \ref{sec:topology}.  See also \cite{Ha02} as an accessible introduction to algebraic topology.

\subsection{Persistence diagram}
\label{subsec:persistence_diagram}

In order to define a persistence diagram, we transform a data set  into a filtration  and compute its persistent homology .
In this section, we will first introduce this mathematical framework of persistence diagrams. Then, by using a ball model filtration, we will intuitively explain geometrical meanings of persistence diagrams.
The ball model filtrations can be generalized toward two constructions using  complexes and sub-level sets.
The former construction is useful for computations of persistence diagrams and the later is useful to discuss theoretical properties.

\subsubsection{Mathematical framework of persistence diagrams}

Let  be a coefficient field of homology\footnote{In this setting, all homology are -vector spaces.
You may simply consider the case , but the theory is built with an arbitrary field.}.
Let  be a (right continuous) {\em filtration} of simplicial complexes (resp. topological spaces), i.e.,  is a subcomplex (resp. subspace) of  for  and .
For , the -linear map induced from the inclusion  is denoted by , where  is the -th homology of .
The -th {\em persistent homology}  of  is defined by the family of homology  and the induced linear maps . 

A {\em homological critical value} of  is the number  such that the linear map  is not isomorphic for any sufficiently small .
The persistent homology  is called {\em tame} if  for any  and the number of homological critical values is finite. A tame persistent homology  has a nice decomposition property:
\begin{thm}[\cite{ZC05}]\label{thm:decomposition}
A tame persistent homology can be uniquely expressed by

where  consists of a family of -vector spaces

and  for .
\end{thm}

Each summand  means a topological feature in  that appears at  and disappears at .
The birth-death pair  is called a {\em generator} of the persistent homology, and  a {\em persistence} of .
We note that, when  for any  (resp. for any ), the decomposition \eqref{eq:decom} should be understood in the sense that some  takes the value  (resp. ), where  are the elements in the extended real .
From the decomposition \eqref{eq:decom}, we define the {\em persistence diagram} of  as the multi-set\footnote{A {\em multi-set} is a set with multiplicity of each point.
We regard a persistence diagram as a multi-set, since several generators can have the same birth-death pairs.}


In this paper, we assume that all persistence diagrams have finite cardinality because a tame persistent homology defines a finite persistence diagram.
Moreover, we also assume that all birth-death pairs are bounded\footnote{This assumption will be justified in Section \ref{subsec:geometrical}.}, that is, all elements in a persistence diagram take neither  nor .
Here, we define the (abstract) persistence diagram  by a finite multi-set above the diagonal .

\subsubsection{Ball model filtrations}
\label{subsec:geometrical}

The example used in Figure \ref{fig:filtration} can be expressed as follows.
Let  be a finite subset in a metric space  and  be a union of balls  with radius .
For convenience, let .
Since  is a right-continuous filtration of topological spaces and  is a finite set,  is tame and the persistence diagram  is well-defined.
For notational simplicity, the persistence diagram of this ball model filtration is denoted by .

We remark that, in this model, there is only one generator in  that does not disappear in the filtration; its lifetime is .
From now on, we deal with  by removing this infinite lifetime generator\footnote{This is called the {\em reduced persistence diagram}.}.
Let  be the diameter of  defined by .
Then, all generators appear after  and disappear before  because  becomes a contractible space.
Thus, for any dimension , all birth-death pairs of  have finite values.


\subsubsection{Geometric complexes}

We review some standard methods of constructing a filtration from finite sets in a metric space. See also \cite{CdSO14} for more details.

Let  be a metric space and  be a finite subset in .
For a fixed , we form a -simplex  as a subset  of  whenever there exists  such that  for all , or equivalently, .
The set of these simplices forms a simplicial complex, called the {\em ech complex} of  with parameter , denoted by .
For , we define  as an empty set.
Since there is a natural inclusion  whenever ,  is a filtration.
When  is a subspace of , from the nerve lemma \cite{Ha02}, it is known that the topology of  is the same\footnote{Precisely, they are {\em homotopy equivalent}.} as  (Figure \ref{fig:cech}), and hence .

The Rips complex (or Vietoris-Rips complex) is also often used in TDA and it gives different topology from the  complex.
For a fixed , we form a -simplex  as a subset  of  that satisfies  for all .
The set of these simplices forms a simplicial complex, called the {\em Rips complex} of  with parameter , denoted by .
Similarly, the Rips complex also forms a filtration .
In general,  is not the same as  (see Figure \ref{fig:cech}).

\begin{figure}[htbp]
\begin{center}
\includegraphics[width=0.7\hsize]{cech.pdf}
\end{center}
\caption{A point set , the union of balls , the  complex  and the Rips complex . There are two rings in  and . However,  has only one ring because there is a -simplex.}
\label{fig:cech}
\end{figure}


\subsubsection{sub-level sets}

Let  be a topological space and  be a continuous map.
Then, we define a {\em sub-level set} by  for  and its filtration by .
Here,  is said to be {\em tame} if  is tame.

For a finite set  in a metric space , we define the distance function  by 

Then, we can see  and .
This means that the ball model is a special case of the sub-level set, and the  complex and the sub-level set with the distance function  give the same persistence diagram.
 
\subsection{Stability of persistence diagrams}
\label{sec:bottleneck_stability}

When we consider data analysis based on persistence diagrams, it is useful to introduce a distance measure among persistence diagrams for describing their relations.
In introducing a distance measure, it is desirable that, as a representation of data, the mapping from data to a persistence diagram is continuous with respect to the distance.
In many cases, data involve noise or stochasticity, and thus the persistence diagrams should be stable under perturbation of data.

The {\em bottleneck distance}  between two persistence diagrams  and  is defined by

where  is the diagonal set with infinite multiplicity and  ranges over all multi-bijections\footnote{A {\em multi-bijection} is a bijective map between two multi-sets counted with their multiplicity.} from  to .
Here, for ,  denotes . 
We note that there always exists such a multi-bijection  because the cardinalities of  and  are equal by considering the diagonal set  with infinite multiplicity.
For sets  and  in a metric space , let us recall the {\em Hausdorff distance}  given by

Then, the bottleneck distance satisfies the following stability property.
\begin{prop}[\cite{CdSO14,CEH07}]
\label{prop:point_stability}
Let  and  be finite subsets in a metric space .
Then the persistence diagrams satisfy

\end{prop}

Proposition \ref{prop:point_stability} provides a geometric intuition of the stability of persistence diagrams.
Assume that two point sets  and  are close to each other with . If there is a generator , then we can find at least one generator in  which is born in  and dies in  (see Figure \ref{fig:stability}).
Thus, the stability guarantees the similarity of two persistence diagrams, and hence we can infer the true topological features from the persistence diagrams given by noisy observation (see also \cite{FLRWBS14}).
\begin{figure}[htbp]
\begin{center}
\includegraphics[width=0.8\hsize]{stability.pdf}
\vspace{-3mm}
\caption{Two point sets  and  (left) and their persistence diagrams (right).
The green region is an -neighborhood of  and all generators in  are in the -neighborhood.}
\label{fig:stability}
\end{center}
\end{figure}

For , the {\em -Wasserstein distance} , which is also used as a distance between two persistence diagrams  and , is defined by

where  ranges over all multi-bijections from  to .
The -Wasserstein distance  is defined by the bottleneck distance . Here, we define the {\em degree- total persistence} of  by  for .

\begin{prop}[\cite{CEHM10}]
\label{prop:wasserstein_stability}
Let , and  and  be persistence diagrams whose degree- total persistences are bounded from above.
Then, 

\end{prop}

For a persistence diagram , its degree- total persistence is bounded from above by , where  denotes the number of generators in .
However, this bound may be weak because, in general,  cannot be bounded from above.
In particular, if data set has noise, the persistence diagram often has many generators close to the diagonal.
Thus, it is desirable that the total persistence is bounded from above independently of .
In the case of persistence diagrams obtained from a ball model filtration, we have the following upper bound (see Appendix \ref{sec:total} for the proof):
\begin{lem}
\label{lem:point_total}
Let  be a triangulable compact subspace in ,  be a finite subset of , and . 
Then, 

where  is a constant depending only on .
\end{lem}

\begin{cor}
\label{cor:point_wasserstein}
Let  be a triangulable compact subspace in ,  be finite subsets of , and . Then 

where  is a constant depending only on .
\end{cor}

\subsection{Kernel methods for representing signed measures}
\label{subsec:universal}

As a preliminary to our proposal of vector representation for persistence diagrams, we briefly summarize a method for embedding signed measures with a positive definite kernel.

Let  be a set and  be a {\em positive definite kernel} on , i.e.,  is symmetric, and for any number of points  in , the Gram matrix  is nonnegative definite.
A popular example of positive definite kernel on  is the Gaussian kernel , where  is the Euclidean norm in .
From Moore-Aronszajn theorem, it is also known that every positive definite kernel  on  is uniquely associated with a reproducing kernel Hilbert space  (RKHS).

We use a positive definite kernel to represent persistence diagrams by following the idea of the kernel mean embedding of distributions \cite{MFSS17, SGSS07,SFL11}.
Let  be a locally compact Hausdorff space,  be the space of all finite signed Radon measures\footnote{A {\em Radon measure}  on  is a Borel measure on  satisfying
(i)  for any compact subset , and 
(ii)  for any  in the Borel -algebra of .} on , and  be a bounded measurable kernel on .
Since  is finite, the integral  is well-defined as the Bochner integral \cite{DU77}.
Here, we define a mapping from  to  by


For a locally compact Hausdorff space , let  denote the space of continuous functions vanishing at infinity\footnote{A function  is said to {\em vanish at infinity} if for any  there is a compact set  such that .}.
A kernel  on  is said to be -kernel if  for any .
If  is -kernel, the associated RKHS  is a subspace of .
A -kernel  is called {\em -universal} if  is dense in .
It is known that the Gaussian kernel  is -universal on  \cite{SFL11}.
When  is -universal, the vector  in the RKHS uniquely determines the finite signed measure , and thus serves as a representation of . We summarize the property as follows:
\begin{prop}[\cite{SFL11}]
\label{prop:C0_distance}
Let  be a locally compact Hausdorff space.
If  is -universal on , the mapping  is injective. Thus,

defines a distance on .
\end{prop}



\section{Kernel methods for persistence diagrams}
\label{sec:pdkernel}
We propose a positive definite kernel for persistence diagrams, called the persistence weighted Gaussian kernel (PWGK), to embed the persistence diagrams into an RKHS.
This vectorization of persistence diagrams enables us to apply any kernel methods to persistence diagrams and explicitly control the effect of persistence.
We show the stability theorem with respect to the distance defined by the embedding and discuss the efficient and precise approximate computation of the PWGK.   


\subsection{Vectorization of persistence diagrams}
\label{subsec:vectorization}

We propose a method for vectorizing persistence diagrams using the kernel embedding \eqref{eq:E_k} by regarding a persistence diagram as a discrete measure.
In vectorizing persistence diagrams, it is desirable to have flexibility to discount the effect of generators close to the diagonal, since they often tend to be caused by noise.
To this goal, we explain slightly different two ways of embeddings, which turn out to give the same inner product for two persistence diagrams. 

First, for a persistence diagram , we introduce a measure  with a weight  for each generator  (Figure \ref{fig:weighted}), where  is the Dirac delta measure at .
By appropriately choosing , the measure  can discount the effect of generators close to the diagonal.
A concrete choice of  will be discussed later.
\begin{figure}[htbp]
\begin{center}
\includegraphics[width=0.8\hsize]{weighted_delta.pdf}
\caption{Unweighted (left) and weighted (right) measures.}
\vspace{-3mm}
\label{fig:weighted}
\end{center}
\end{figure}

As discussed in Section \ref{subsec:universal}, given a -universal kernel  above the diagonal , the measure  can be embedded as an element of the RKHS  via 

From the injectivity in Proposition \ref{prop:C0_distance}, this mapping identifies a persistence diagram; in other words, it does not lose any information about persistence diagrams.
Hence,  serves as a vector representation of the persistence diagram.

As the second construction, let

be the weighted kernel with the same weight function as above.
Then the mapping 

also defines a vectorization of persistence diagrams.
The first construction may be more intuitive by directly weighting a measure, while the second one is also practically useful since all the parameter tuning is reduced to kernel choice.
We note that the inner products introduced by two RKHS vectors \eqref{E_k:embed} and \eqref{E_k^w:embed} are the same:

In addition, these two RKHS vectors \eqref{E_k:embed} and \eqref{E_k^w:embed} are essentially equivalent, as seen from the next proposition:

\begin{prop}\label{prop:iso}
Let  be -universal on  and  be a positive function on .
Then the following mapping

defines an isomorphism between the RKHSs.  Under this isomorphism,  and  are identified. 
\end{prop}

\begin{proof}
Let  and define its inner product by

Then, it is easy to see that  is a Hilbert space and the mapping  gives an isomorphism between  and  of the Hilbert spaces.
In fact, we can show that  is the same as .
To see this, it is sufficient to check that  is a reproducing kernel of  from the uniqueness property of a reproducing kernel for an RKHS.
The reproducing property is proven from 

The second assertion is obvious from Equations \eqref{E_k:embed} and \eqref{E_k^w:embed}.
\end{proof}

 

\subsection{Stability with respect to the kernel embedding}
\label{subsec:stability}

Given a data set , we compute the persistence diagram  and vectorize it as an element  of the RKHS.
Then, for practical applications, this map  should be stable with respect to perturbations to the data as discussed in Section \ref{sec:bottleneck_stability}. 

Let  and  be persistence diagrams and  be any multi-bijection. Here, we partition  (resp. ) into  and  (resp.  and ) such as 

Then  and  are bijective under .
Now, let a weight function  be zero on the diagonal .
Then, the norm of the difference between RKHS vectors is calculated as follows:


In the sequel, we consider the Gaussian kernel  for a -universal kernel.
Since  (Lemma \ref{lemm:Lip_k} in Appendix \ref{sec:stability}) and  for any , we have 


In this paper, we propose to use a weight function 

This is a bounded and increasing function of .
The corresponding positive definite kernel is

We call it {\em persistence weighted Gaussian kernel} (PWGK).
This function  gives a small (resp. large) weight on a noisy (resp. essential) generator. 
In addition, by appropriately adjusting the parameters  and  in , we can control the effect of the persistence.  
Furthermore, we show that the PWGK has the following property: 

\begin{prop}
\label{prop:general_stability}
Let , and  and  be finite persistence diagrams whose degree- total persistence are bounded from above.
Then,

where  is a constant bounded from above by

\end{prop}

\begin{proof}
Let  and  be a multi-bijection achieving the bottleneck distance, i.e., . We have already observed 

in Equation \eqref{eq:dk_halfway}. 
From Lemma \ref{lemm:w_continuous} in Appendix \ref{sec:stability}, the right-hand side of the above inequality is bounded from above by

We have used the fact  in \eqref{eq:warc_bound} and  in \eqref{eq:total_last}.
Thus, if both degree- total persistences of  and  are bounded from above, since degree- total persistence of  is also bounded from above from Proposition \ref{prop:persistence_inequality}, the coefficient of  appearing in \eqref{eq:total_last} is bounded from above.
\end{proof}

The constant  is dependent on  and , and hence we cannot say that the map  is continuous.
In the case of persistence diagrams obtained from ball model filtrations, from Lemma \ref{lem:point_total}, the PWGK satisfies the following stability property.
Recall that  denotes the persistence diagram to the ball model for :

\begin{thm}
\label{thm:kernel_stability}
Let  be a triangulable compact subspace in ,  be finite subsets, and .
Then,

where  is a constant depending on .
\end{thm}

\begin{proof}
For any finite set , from Lemma \ref{lem:point_total}, there exists a constant  such that

By replacing  and  with  and  in \eqref{eq:total_last}, respectively, we have 

Then,  is a constant independent of  and .
\end{proof}

Let  be the set of finite subsets in a triangulable compact subspace .
Since the constant  is independent of  and , Proposition \ref{prop:point_stability} and Theorem \ref{thm:kernel_stability} conclude that the map

is Lipschitz continuous.
Note again that this implies a desirable stability property of the PWGK with the ball model: small perturbation of data points in terms of the Hausdorff distance causes only small perturbation of the persistence diagrams in terms of the RKHS distance with the PWGK.

As the most relevant work to the PWGK, the persistence scale-space kernel (PSSK, \cite{RHBK15})\footnote{See Section \ref{subsubsec:pssk}.} provides another kernel method for vectorization of persistence diagrams and its stability result is shown with respect to -Wasserstein distance.
However, to the best of our knowledge, -Wasserstein stability with respect to the Hausdorff distance is not shown, that is, for point sets  and ,  is not estimated by  such as Proposition \ref{prop:point_stability} or Corollary \ref{cor:point_wasserstein}.
Furthermore, it is shown \cite{RHBK15} that the PSSK does not satisfy the stability with respect to -Wasserstein distance for , including the bottleneck distance , and hence it is not ensured that results obtained from the PSSK are stable under perturbation of data points in terms of the Hausdorff distance.
On the other hand, since the PWGK has the desirable stability (Theorem \ref{thm:kernel_stability}), it is one of the advantages of our method over the previous research\footnote{In fact, if we apply Theorem 3 in \cite{RHBK15} to the PWGK directly, it concludes that the PWGK also does not satisfy the bottleneck stability. However, by using Proposition \ref{prop:general_stability}, we can avoid this difficulty, and Theorem  \ref{thm:kernel_stability} holds. For more details, see Appendix \ref{sec:additive}.}.
In addition, by the similar way in \cite{RHBK15}, we show the stability with respect to -Wasserstein distance for our kernel vectorization. 

\begin{prop}
\label{prop:pwgk_wasserstein}
Let  and  be persistence diagrams.
If a weight function  is zero on the diagonal and there exist constants  such that 

for any , then

\end{prop}

\begin{proof}
From Equation \eqref{eq:dk_halfway}, we have

Since this inequality holds for any multi-bijection ,  we obtain the -Wasserstein stability.
\end{proof}

The weight function  is bounded from above by , and for , from Lemma \ref{lemm:w_continuous}, we have

Therefore, from Proposition \ref{prop:pwgk_wasserstein}, the PWGK also have -Wasserstein stability:
\begin{cor}
Let , and  and  be persistence diagrams.
Then

\end{cor}

For , we have

from Lemma \ref{lemm:w_continuous} and, hence, from Equation \eqref{eq:pwgk_inequality}, we have

Although the above inequality does not directly imply the Lipschitz continuity of the PWGK for  with respect to -Wasserstein distance, combining with Lemma \ref{lem:point_total}, we have the following -Wasserstein stability:
\begin{cor}
Let  be a triangulable compact subspace in ,  be finite subsets, and .
Then,

for some constant .
\end{cor}



\subsection{Kernel methods on RKHS}

Once persistence diagrams are represented as RKHS vectors, we can apply any kernel methods to those vectors by defining a kernel over the vector representation.
In a similar way to the standard vectors, the simplest choice is to consider the inner product as a linear kernel

on the RKHS and we call it the {\em -linear kernel}.  

If  is a -universal kernel and  is strictly positive on , from Proposition \ref{prop:C0_distance},  defines a distance on the persistence diagrams and it is computed as 

Then, we can also consider a nonlinear kernel

on the RKHS and we call it the {\em -Gaussian kernel}.

In this paper, if there is no confusion, we also refer to the -Gaussian kernel as the PWGK.
\cite{MFDS12} observed better performance with nonlinear kernels for some complex tasks and this is one of the reasons that we will use the Gaussian kernel on the RKHS.


\subsection{Computation of Gram matrix}
\label{subsec:calculation}

Let  be a collection of persistence diagrams.
In many practical applications, the number of generators in a persistence diagram can be large, while  is often relatively small;
in Section \ref{subsec:glass}, for example, the number of generators is about 30000, while .

If the persistence diagrams contain at most  points, each element of the Gram matrix  involves  evaluations of , resulting the complexity  for obtaining the Gram matrix.
Hence, reducing computational cost with respect to  is an important issue.

We solve this computational issue by using the random Fourier features \cite{RR07}.
To be more precise, let  be random variables from the -dimensional normal distribution  where  is the identity matrix.  This method approximates  by , where  denotes the complex conjugate.
Then,   is approximated by , where .
As a result, the computational complexity of the approximated Gram matrix is , which is linear to .

We note that the approximation by the random Fourier features can be sensitive to the choice of .
If  is much smaller than , the relative error can be large.  
For example, in the case of  and ,  is about  while we observed the approximated value can be about  with .
As a whole, these  errors may cause a critical error to the statistical analysis.
Moreover, if  is largely deviated from the ensemble  for , then most values  become close to  or .

In order to obtain a good approximation and extract meaningful values, the choice of parameters is important.  For supervised learning such as SVM, we use the cross-validation (CV) approach.  For unsupervised case, we follow the heuristics proposed in \cite{GFTSSS07} and set 

so that  takes close values to many .
For the parameter , we also set 

Similarly, the parameter  in the -Gaussian kernel is defined by 




\section{Experiments}
\label{sec:experiment}

In this section, we apply the kernel method of the PWGK to synthesized and real data, and compare the performance between the PWGK and other statistical methods of persistence diagrams. 
All persistence diagrams are obtained from the ball model filtrations and computed by CGAL \cite{DLY15} and PHAT \cite{BKRW14}. 
With respect to the dimension of persistence diagrams, we use -dimensional persistence diagrams in Section \ref{subsec:granular} and -dimensional ones in other parts.

\subsection{Comparison to previous works}
\label{subsec:comparison}

\subsubsection{Persistence scale-space kernel}
\label{subsubsec:pssk}
The most relevant work to our method is the one proposed by \cite{RHBK15}.
Inspired by the heat equation, they propose a positive definite kernel called {\em persistence scale-space kernel} (PSSK)  on the persistence diagrams:

where  and  for .
We note that  also takes zero on the diagonal by subtracting the Gaussian kernels for  and .  

In fact, we can verify that the -linear kernel contains the PSSK.
Let  where .
Then,  can also be expressed as 

which is equal to .
Furthermore, the inner product in  is

By scaling the variance parameter  in the Gaussian kernel  and multiplying by an appropriate scalar, Equation \eqref{eq:pssk} is the same as Equation \eqref{eq:pssk_embedding}.
Thus, the PSSK can also be approximated by the random Fourier features. When we apply the random Fourier features for the PSSK, we set  as before and .

While both methods discount noisy generators, the PWGK has the following advantages over the PSSK.
(i) The PWGK can control the effect of the persistence by  and  in  independently of the bandwidth parameter  in the Gaussian factor, while in the PSSK only one parameter  cannot adjust the global bandwidth and the effect of persistence simultaneously.
(ii) The PSSK does not satisfy the stability with respect to the bottleneck distance (see also remarks after Theorem \ref{thm:kernel_stability}).

\subsubsection{Persistence landscape}
\label{subsubsec:pl}
The {\em persistence landscape} \cite{Bu15} is a well-known approach in TDA for vectorization of persistence diagrams.
For a persistence diagram , the persistence landscape  is defined by

where  denotes , and it is a vector in the Hilbert space .
Here, we define a positive definite kernel of persistence landscapes as a linear kernel on :

Since a persistence landscape does not have any parameters, we do not need to consider the parameter tuning.
However, the integral computation is required and it causes much computational time.
Let  be a collection of persistence diagrams which contain at most  points.
Since  for any , calculating , which needs sorting, is in  (see also \cite{BD17}).
For a fixed , we can calculate  in , and the Gram matrix  in , where  is the number of partitions in the integral interval.
Theoretically speaking, this implies that it takes more time to calculate the Gram matrix of  than the PWGK and the PSSK by the random Fourier features.


\subsubsection{Persistence image}
\label{subsubsec:pi}
As a finite dimensional vector representation of a persistence diagram, a {\em persistence image} is proposed in \cite{AEKNPSCHMZ17}.
First, we prepare a differentiable probability density function  with mean  and a weight function . For a persistence diagram , the {\em corresponding persistence surface} is defined by

Then, for fixed points , the {\em persistence image}  is defined by an  matrix whose -element is assigned to the integral of  over the pixel , i.e., 

Since the persistence image can be regarded as an -dimensional vector, we define a vector  by

and, in this paper, call it the persistence image vector.

In \cite{AEKNPSCHMZ17}, they use the -dimensional Gaussian distribution  as  and a piecewise linear weighting function  defined by

where  is a parameter.
In this paper, for a collection of persistence diagrams , we set  as 

For points  of a pixel , we set  and  for \footnote{Here, we set  because all generators in the ball model filtrations are born after .}.

Here, by choosing  and  in the proposed way, we define a positive definite kernel of persistence image vector as a linear kernel on :


If we choose  as a (normalized) positive definite kernel , the corresponding persistence surface \eqref{eq:pi} is the same as the RKHS vector \footnote{\cite{AEKNPSCHMZ17} use a persistence diagram in birth-persistence coordinates. That is, by a linear transformation , a persistence diagram  is transformed into . In this paper, in order to compare with the persistence image and the PWGK, we use birth-death coordinates.}.
Thus, it may be expected that the persistence image and the PWGK show similar performance for data analysis.
However, there are several differences between the persistence image and the PWGK.
(i) The mapping from a persistence diagram to the persistence image is not injective due to the discretization by the integral, on the other hand, the injectivity of the RKHS vector  is ensured in Proposition \ref{prop:C0_distance}.
(ii) It is also shown that the persistence image has a stability result with respect to -Wasserstein distance, but it does not satisfy the bottleneck stability (Remark 1 in \cite{AEKNPSCHMZ17}) or the Haussdorff stability as noted after Theorem \ref{thm:kernel_stability}.
(iii) The computational complexity of a persistence image does not depend on the number of generators in a persistence diagram, but instead, it depends on the number of pixels. We can reduce the computational time of the persistence image by choosing a small mesh size . However, as data in Section \ref{subsec:Synthesized}, some situations need a fine mesh (i.e., a large mesh size). Thus, we have to be careful with the choice of mesh size.


\subsection{Classification with synthesized data}
\label{subsec:Synthesized}
We compare the performance among the PWGK, the PSSK, the persistence landscape, and the persistence image for a simple binary classification task with SVMs.

\subsubsection{Synthesized data}  

In this experiment, we design data sets so that important generators close to the diagonal must be taken into account to solve the classification task.

Let 
 be a set composed of  points sampled with equal distance from a circle in -dimensional Euclidean space with radius  centered at .
When we compute the persistence diagram of  for , there always exists a generator whose birth time is approximately  (here we use  for small ) and death time is  (Figure \ref{fig:birth-death}).
\begin{figure}[htbp]
\begin{center}
\includegraphics[width=0.6\hsize]{birth-death.pdf}
\end{center}
\caption{Birth and death of the generator for .}
\label{fig:birth-death}
\end{figure}

In order to add randomness on , we extend it into  and change  to  and  as follows:

where \footnote{ is the -dimensional normal distribution with mean  and variance .},  and  is the smallest integer greater than or equal to .
Then, we add  to  with probability  and use it as the synthesized data.

In this paper, we choose parameters by

and set  as  (Figure \ref{fig:synthesized}).
\begin{figure}[htbp]
\begin{center}
\includegraphics[width=0.6\hsize]{combi_example.pdf}
\end{center}
\caption{Examples of synthesized data.  Left:  exits. Right:  does not exist. }
\label{fig:synthesized}
\end{figure}

For the binary classification, we introduce the following labels:

The class label of the data set is then given by .
By this construction, identifying  requires relatively smooth function in the area of long lifetimes, while classifying the existing of  needs delicate control of the resolution around the diagonal.

\subsubsection{SVM results} 

SVMs are trained from persistence diagrams given by 100 data sets, and evaluated with 100 independent test data sets.  
As a positive definite kernel , we choose the Gaussian kernel  and the linear kernel .
For a weight function , we use the proposed function , the piecewise linear weighting function  defined in Section \ref{subsubsec:pi}, and an unweighted function .
The hyper-parameters  in the PWGK and  in the PSSK are chosen by the 10-fold cross-validation, and the degree  in  is set as .
For  and , while they originally consider only the inner product, we also apply the Gaussian kernels on RKHS following Equation \eqref{eq:gauss_rkhs}.
Since  can be seen as a discretization of the -linear kernel, we also construct another kernel of persistence image by replacing  with , which is considered as a discretization of the PWGK.
In order to check whether the persistence image with  is an appropriate discretization of the PWGK, we try several mesh size .

\begin{table}[htbp]
\caption{Results of SVMs with the -linear/Gaussian kernel, the PSSK, the persistence landscape, and the persistence image. Average classification rates () and standard deviations for 100 test data sets are shown.}\label{table:Synth_results}
\centering
\begin{tabular}{ c  c | c | c }
\hline
\multicolumn{2}{c|}{} & Linear  & Gaussian \\ \hline
\multicolumn{2}{c|}{\textbf{PWGK}} &  	&    \\  
\textbf{kernel} & \textbf{weight} &  &  \\
			&  	&  75.7  2.31   		& 85.8  5.19 (PWGK) \\
			&  	&  75.8  2.47 ()		& 85.6  5.01 (PWGK, ) \\
   &  &  76.0  2.39		& 86.0  4.98 (PWGK) \\
			&  		&  49.3  2.72  		& 52.3  6.60  \\
			&  		&  53.8  4.76 		& 55.1  8.42 \\ \hline
			&  	&  49.3  6.92		& 51.8  3.52 \\
	&  		&  51.0  6.84		& 55.7  8.68 \\
			&  		&  50.5  6.90		& 53.0  4.89 \\ \hline
\multicolumn{2}{c|}{\textbf{PWGK with Persistence image}} &  	&    \\  
		& 	&  48.8  3.75 ()		& 52.0  5.65 () \\
		&  	&  49.2  5.77 ()		& 51.8  7.23 () \\
		&  	&  75.0  2.20 ()		& 85.8  4.15 () \\  \hline
\multicolumn{2}{c|}{\textbf{PSSK} }				&  50.5  5.60 () 	& 53.6  6.69 \\ \hline
\multicolumn{2}{c|}{\textbf{Persistence landscape}}   &   50.6  5.92 () 	& 48.8  4.25   \\  \hline
\multicolumn{2}{c|}{\textbf{Persistence image}} &  	&    \\  
		& 	 	&  51.1  4.38  )	& 51.7  6.86  \\
		& 	 	&  49.0  6.14  )	& 52.3  7.21  \\
		& 	 	&  54.5  8.76  )	& 52.1  6.70  \\ \hline
\end{tabular}
\end{table}
In Table \ref{table:Synth_results}, we can see that the PWGK  and the Gaussian kernel on the persistence image with  and large mesh size  show higher classification rates ( accuracy) than the other methods (, , and ).
Although the -Gaussian kernel and the persistence image with the original weight  discount noisy generators, the classification rates are close the chance level.
These unfavorable results must be caused by the difficulty in handling the local and global locations of generators simultaneously.
While the result of the persistence image with a large mesh size is similar to that of the PWGK (e.g.,  and ), a small mesh size gives bad approximation results (e.g.,  and ).
The reason is because a small mesh size makes rough pixels, and  itself and noisy generators are treated in some rough pixel.
On the other hand, we remark that a large mesh size  needs much computational time since the computational complexity of the persistence image depends on .  

We observe that the classification accuracies are not sensitive to . 
Thus, in the rest of this paper, we set  because the assumption  in Theorem \ref{thm:kernel_stability} ensures the continuity in the kernel embedding of persistence diagrams and all data points are obtained from .


\subsection{Analysis of granular system}
\label{subsec:granular} 

We apply the PWGK, the PSSK, the persistence landscape, and the persistence image to persistence diagrams obtained by experimental data in a granular packing system \cite{FSCS11}.
In this example, a partially crystallized packing with  monosized beads (diameter mm, polydispersity mm) in a container is obtained by experiments, where the configuration of the beads is imaged by means of X-ray Computed Tomography.
One of the fundamental interests in the study of granular packings is to understand the transition from random packings to crystallized packings.
In particular, the maximum packing density  that random packings can attain is still a controversial issue (e.g., see \cite{TTD00}).
Here, we apply the change point analysis to detect .  

In oder to observe configurations of various densities, we divide the original full system into  cubical subsets containing approximately  beads.
The data are provided by the authors of the paper \cite{FSCS11}.
The packing densities of the subsets range from  to .
\cite{STRFH17} computed a persistence diagram for each subset by taking the beads configuration as a finite subset in , and found that the persistence diagrams characterize different configurations in random packings (small ) and crystallized packings (large ).
Hence, it is expected that the change point analysis applied to these persistence diagrams can detect the maximum packing density  as a transition from the random to crystallized packings. 

Our strategy is to regard the maximum packing density as the change point and detect it from a collection  of persistence diagrams made by beads configurations of granular systems, where  is the index of the packing densities listed in the increasing order.
As a statistical quantity for the change point detection, we use the kernel Fisher discriminant ratio \cite{HMB09} defined by

where the empirical mean element  and the empirical covariance operator  with data  through  are given by

respectively, and  is a regularization parameter (in this paper we set ).
The index  achieving the maximum of  corresponds to the estimated change point. 
In Figure \ref{fig:packing_KFDR}, all the four methods detect  as the sharp maximizer of the KFDR. This result indicates that the maximum packing density  exists in the interval  and supports the traditional observation  \cite{An72}.

\begin{figure}[htbp]
\begin{center}
\includegraphics[width=0.9\hsize]{granular_kfdr.pdf}
\end{center}
\vspace{-3mm}
\caption{The  graphs of the PWGK, the PSSK, the persistence landscape, and the persistence image.}
\label{fig:packing_KFDR}
\end{figure}

We also apply kernel principal component analysis (KPCA) to the same collection of the 35 persistence diagrams. 
Figure \ref{fig:packing_kpca} shows the -dimensional KPCA plots where each green triangle (resp. red circle) indicates the persistence diagram of random packing (resp. crystallized packing).
We can see clear two-cluster structure corresponding to two physical states.

\begin{figure}[htbp]
\begin{center}
\includegraphics[width=0.9\hsize]{granular_kpca.pdf}
\end{center}
\vspace{-3mm}
\caption{The KPCA plots of the PWGK (contribution rate: 92.9\%), the PSSK (99.7\%), the persistence landscape (83.8\%), and the persistence image (98.7\%).}
\label{fig:packing_kpca}
\end{figure}





\subsection{Analysis of }
\label{subsec:glass}

When we rapidly cool down the liquid state of , it avoids the usual crystallization and changes into a glass state. 
Understanding the liquid-glass transition is an important issue for the current physics and industrial applications \cite{GS07}.
Glass is an amorphous solid, which does not have a clear structure in the configuration of molecules, but it is also known that the medium distance structure such as rings have important influence on the physical properties of the material.
It is thus promising to apply the persistent homology to express the topological and geometrical structure of the glass configuration.
For estimating the glass transition temperature by simulations, a traditional physical method is to prepare atomic configurations of  for a certain range of temperatures by molecular dynamics simulations, and then draw the temperature-enthalpy graph. 
The graph consists of two lines in high and low temperatures with slightly different slopes which correspond to the liquid and the glass states, respectively, and the glass transition temperature is conventionally estimated as an interval of the transient region combining these two lines
 (e.g., see \cite{El90}).
However, since the slopes of two lines are close to each other, determining the interval is a subtle problem.
Usually only the rough estimate of the interval is available. Hence, we apply our framework of topological data analysis with kernels to detect the glass transition temperature. 

Let  be a collection of the persistence diagrams made by atomic configurations of  and sorted by the decreasing order of the temperature. The same data was used in the previous works by \cite{HNHEMN16,NHHEN15}. The interval of the glass transition temperature  estimated by the conventional method explained above is , which corresponds to . 

\begin{figure}[htbp]
\begin{center}
\includegraphics[width=0.9\hsize]{SiO2_kfdr.pdf}
\end{center}
\vspace{-3mm}
\caption{The  graphs of the PWGK (left), the PSSK (center) and the persistence image (right).}
\label{fig:glass_KFDR}
\end{figure}
In Figure \ref{fig:glass_KFDR}, the KFDR plots show that the change point is estimated as  by the PWGK,  by the PSSK, and    by the persistence image. For the persistence landscape, we cannot obtain the KFDR or the KPCA results with reasonable computational time.

\begin{figure}[htbp]
\begin{center}
\includegraphics[width=0.9\hsize]{SiO2_kpca.pdf}
\end{center}
\vspace{-3mm}
\caption{The -dimensional and -dimensional KPCA plots of the PWGK (contribution rates for -dimension: 81.7\%, -dimension: 92.1\%), the PSSK (97.2\%, 99.3\%) and the persistence image (99.9\%, 99.9\%).}
\label{fig:glass_kpca}
\end{figure}
As we see from the -dimensional plots given by KPCA (Figure \ref{fig:glass_kpca}), the PWGK presents the clear phase change between before (green triangle) and after (red circle) the change point determined by the KFDR.
This strongly suggests that the glass transition occurs at the detected change point.
On the other hand, we cannot observe clear two-cluster structure in the KPCA plots of the PSSK and the persistence image.  
We also remark that the detailed cluster structure is observed in the -dimensional KPCA plots of the PWGK.

\subsection{Protein classification}
\label{subsec:Protein}

We apply the PWGK to two classification tasks studied in \cite{CMWOXW15}.
They introduced the molecular topological fingerprint (MTF) as a feature vector constructed from the persistent homology, and used it for the input to the SVM.
The MTF is given by the -dimensional vector whose elements consist of the persistences of some specific generators\footnote{The MTF method is not a general method for persistence diagrams because some elements of the MTF vector are specialized for protein data, e.g., the ninth element of the MTF vector is defined by the number of Betti  bars that locate at \AA, divided by the number of atoms. For the details, see \cite{CMWOXW15}.} in persistence diagrams.
We compare the performance between the PWGK and the MTF method under the same setting of the SVM reported in \cite{CMWOXW15}.

The first task is a protein-drug binding problem, where the binding and non-binding of drug to the M2 channel protein of the influenza A virus is to be classified.
For each of the two forms, 15 data were obtained by NMR experiments, and 10 data are used for training and the remaining for testing.
We randomly generate 100 ways of partitions and calculate the average classification rates.

In the second problem, the taut and relaxed forms of hemoglobin are to be classified.
For each form, 9 data were collected by the X-ray crystallography.
We select one data from each class for testing and use the remaining  for training.
All the 81 combinations are performed to calculate the CV classification rates.

The results of the two problems are shown in Table \ref{table:Protein_results}.
We can see that the PWGK achieves better performance than the MTF in both problems.
\begin{table}[ttt]
\caption{CV classification rates () of SVMs with the PWGK and the MTF (cited from \cite{CMWOXW15}).}
\label{table:Protein_results}
\centering
\begin{tabular}{c|c c}
\hline
& Protein-Drug  &  Hemoglobin \\ \hline
PWGK  &  100  & 88.90  \\
MTF  &  (nbd) 93.91 / (bd) 98.31 & 84.50 \\
\hline
\end{tabular}
\vspace{-3mm}
\end{table}


\section{Conclusion and Discussions}

One of the contributions of this paper is to introduce a kernel framework to topological data analysis with persistence diagrams.
We applied the kernel embedding approach to vectorize the persistence diagrams, which enables us to utilize any standard kernel methods for data analysis.
Another contribution is to propose a kernel specific to persistence diagrams, that is called persistence weighted Gaussian kernel (PWGK).
As a significant advantage, our kernel enables one to control the effect of persistence in data analysis.
We have also proven the stability property with respect to the distance in the Hilbert space.
Furthermore, we have analyzed the synthesized and real data by using the proposed kernel.
The change point detection, the principal component analysis, and the support vector machine derived meaningful results for the tasks. From the viewpoint of computations, our kernel can utilize an efficient approximation to compute the Gram matrix.

One of the main theoretical results of this paper is the stability of the PWGK (Theorem \ref{thm:kernel_stability}).
It is obtained as a corollary of Proposition \ref{prop:general_stability} by restricting the class of persistence diagrams to that obtained from ball model filtrations.
The reason of this restriction is because the total persistence can be bounded from above independent of the persistence diagram.
Thus, one direction to extend this work is to examine the boundedness condition about the total persistence of other persistence diagrams, for example obtained from sub-level sets or Rips complexes.

Another direction to extend this work is to generalize the class of weight functions.
The reason of the choice of  is mainly for the stability property, but in principle, we can apply any weight function to data analysis.
Then, the question is what types of weight functions have a stability property with respect to the bottleneck or -Wasserstein distance.
Even if we do not concern about stability properties, which weight function is practically good for data analysis?
Suppose generators close to the diagonal are sometimes seen as important features.
Then, our statistical framework can treat such small generators as significant ones by a weight function which has large weight close to the diagonal, while other statistical methods for persistence diagrams always see small generators as noisy ones.



\section*{Acknowledgement}
We thank Ulrich Bauer for giving us useful comments in Section \ref{subsubsec:pssk}, and Mohammad Saadatfar and Takenobu Nakamura for providing experimental and simulation data used in Section \ref{subsec:granular} and \ref{subsec:glass}.
This work is partially supported by JST CREST Mathematics (15656429), JSPS KAKENHI Grant Number 26540016, Structural Materials for Innovation Strategic Innovation Promotion Program D72, Materials research by Information Integration” Initiative (MII) project of the Support Program for Starting Up, Innovation Hub from JST, and JSPS Research Fellow (17J02401).

\newpage

\appendix

\section{Topological tools}
\label{sec:topology}
This section summarizes some topological tools used in the paper.
To study topological properties algebraically, simplicial complexes are often considered as basic objects.
We start with a brief explanation of simplicial complexes, and gradually increase the generality from simplicial homology to singular and persistent homology.
For more details, see \cite{Ha02}.

\subsection{Simplicial complex}\label{sec:sc}
We first introduce a combinatorial geometric model called simplicial complex to define homology.
Let  be a finite set (not necessarily points in a metric space).
A {\em simplicial complex} with the vertex set  is defined by a collection  of subsets in  satisfying the following properties:
\begin{enumerate}
\item  for , and
\item if  and , then .
\end{enumerate}

Each subset  with  vertices is called a -simplex.
We denote the set of -simplices by .
A subcollection  which also becomes a simplicial complex (with possibly less vertices) is called a subcomplex of .

We can visually deal with a simplicial complex  as a polyhedron by pasting simplices in  into a Euclidean space.
The simplicial complex obtained in this way is called a geometric realization, and its polyhedron is denoted by . 
In this context, the simplices with small  correspond to points (), edges (), triangles (), and tetrahedra (). 
\begin{exam}\label{exam:sc}
{\rm 
Figure \ref{fig:sc} shows two polyhedra of simplicial complexes


\begin{figure}[htbp]
\begin{center}
\includegraphics[width=0.3\hsize]{sc.pdf}
\end{center}
\caption{The polyhedra of the simplicial complexes  (left) and  (right).}
\label{fig:sc}
\end{figure}
}\end{exam}



\subsection{Homology}\label{sec:homology}
\subsubsection{Simplicial homology}\label{sec:simplicial_homology}
The procedure to define homology is summarized as follows:
\begin{enumerate}
\item Given a simplicial complex , build a chain complex  . This is an algebraization of  characterizing the boundary. 
\item Define homology by quotienting out  certain subspaces in  characterized by the boundary. 
\end{enumerate}

We begin with the procedure 1 by assigning orderings on simplices. 
When we deal with a -simplex  as an ordered set, there are  orderings on .
For , we define an equivalence relation  on two orderings of  such that they are mapped to each other by even permutations. 
By definition, two equivalence classes exist, and each of them is called an oriented simplex. 
An oriented simplex is denoted by , and its opposite orientation is expressed by adding the minus .
We write  for the equivalence class including . For , we suppose that we have only one orientation for each vertex. 

Let  be a field. We construct a -vector space  as 

for  and  for .
Here,  for a set  is a vector space over  such that the elements of  formally form a basis of the vector space.
Furthermore, we define a linear map called the {\em boundary map}  by the linear extension of 

where  means the removal of the vertex . We can regard the linear map  as algebraically capturing the -dimensional boundary of a -dimensional object. 

For example, the image of the -simplex  is given by , which is the boundary of  (see Figure \ref{fig:sc}). 

In practice, by arranging some orderings of the oriented - and - simplices, we can represent the boundary map as a matrix  with the entry  given by the coefficient in \eqref{eq:boundary}.
For the simplicial complex  in Example \ref{exam:sc}, the matrix representations  and  of the boundary maps are  given by 

Here, the -simplices (resp. -simplices) are ordered by  (resp. , , ).


We call a sequence of the vector spaces and linear maps

the {\em chain complex} of . As an easy exercise, we can show .
Hence, the subspaces  and  satisfy . Then, the -th (simplicial) {\em homology} is defined by taking the quotient space

Intuitively, the dimension of  counts the number of -dimensional holes in  and each generator of the vector space  corresponds to these holes.
We remark that the homology as a vector space is independent of the orientations of simplices. 

For a subcomplex  of , the inclusion map  naturally induces a linear map in homology . Namely, an element  is mapped to , where the equivalence class  is taken in each vector space. 

For example, the simplicial complex  in Example \ref{exam:sc} has 

 from (\ref{eq:matrix}). Hence , meaning that there are no -dimensional hole (ring) in .
On the other hand, since  and , we have , meaning that  consists of one ring.
Hence, the induced linear map  means that the ring in  disappears in  under .

A topological space  is called {\em triangulable} if there exists a geometric realization of a simplicial complex  whose polyhedron is homeomorphic\footnote{A continuous map  is said to be {\em homeomorphic} if  is bijective and the inverse  is also continuous.} to .
For such a triangulable topological space, the homology is defined by .
This is well-defined, since a different geometric realization provides an isomorphic homology. 

\subsubsection{Singular homology}\label{sec:singular_homology}
We here extend the homology to general topological spaces.
Let  be the standard basis of  (i.e., , 1 at  -th position, and 0 otherwise), and set

We also denote the inclusion by . 

For a topological space , a continuous map  is called a singular -simplex, and let  be the set of -simplices. 
We construct a -vector space  as 

The boundary map  is defined by the linear extension of


Even in this setting, we can show that , and hence the subspaces  and  satisfy . Then, the -th (singular) {\em homology} is similarly defined by 

It is known that, for a triangulable topological space, the homology of this definition is isomorphic to that defined in 
\ref{sec:simplicial_homology}. From this reason, we hereafter identify simplicial and singular homology. 

The induced linear map in homology for an inclusion pair of topological space  is similarly defined as in \ref{sec:simplicial_homology}. 




\section{Total persistence}
\label{sec:total}

Let  be a triangulable compact metric space.
For a Lipschitz function , we define the degree- total persistence over  by 

for , where  is the amplitude of .
Let  be a triangulated simplicial complex of  by a homeomorphism .
The diameter of a simplex  and the mesh of the triangulation  are defined by  and , respectively.
Furthermore, let us set .
Then, the degree- total persistence over  is bounded from above as follows:
\begin{lem}[\cite{CEHM10}]
Let  be a triangulable compact metric space and  be a tame Lipschitz function.
Then,  is bounded from above by

where  is the Lipschitz constant of .
\end{lem}

For a compact triangulable subspace  in , the number of -cubes with length  covering  is bounded from above by , and hence there exists some constant  depending only on  such that .

For , we can find the upper bounds for the both terms as follows:

and

Then, the upper bound of the total persistence  is given as follows:
\begin{lem}
\label{lem:total}
Let  be a triangulable compact subspace in  and .
For any Lipschitz function , 

where  is a constant depending only on .
\end{lem}

In the case of a finite subset , there always exists an -ball  containing  for some , which is a triangulable compact subspace in .
Moreover, by estimating , we show Lemma \ref{lem:point_total} as a corollary of Lemma \ref{lem:total}:

\begin{proof}[Lemma \ref{lem:point_total}]
The Lipschitz constant of  is , because, for any ,

Moreover, 

because  and .
Thus, for some constant  depending only on ,  we have

\end{proof}

For a persistence diagram , we construct a -dimensional vector

Then, the degree- total persistence is represented as 

where  denotes the -norm of .
Since , we have


\begin{prop}
\label{prop:persistence_inequality}
If  and  is bounded from above,  is also bounded from above.
\end{prop}



\section{Lemmata for Proposition \ref{prop:general_stability}}
\label{sec:stability}

\begin{lem}
\label{lemm:Lip_k}
For any , .
\end{lem}

\begin{proof}

We have used the fact  in \eqref{eq:eq1} and  in \eqref{eq:eq2}.
\end{proof}

\begin{lem}
\label{lemm:persistence}
For any , the difference of persistences  is less than or equal to .
\end{lem}

\begin{proof}
For , we have

\end{proof}

\begin{lem}
\label{lemm:w_continuous}
For any , we have 

\end{lem}

\begin{proof}

We have used the fact that the Lipschitz constant of  is  in \eqref{eq:arctan},

for any  in \eqref{eq:p}, and Lemma \ref{lemm:persistence} in \eqref{eq:infty_norm}.
\end{proof}

\section{Remark on the bottleneck stability of the PWGK}
\label{sec:additive}
Let  be a positive definite kernel on persistence diagrams.
Then, 

defines a semi-metric on persistence diagrams.
A positive definite kernel  is said to be {\em additive} if  and {\em trivial} if  for any persistence diagrams . 
It is shown that a non-trivial additive kernel does not satisfy the  stability for  by giving a counterexample.

\begin{prop}[\cite{RHBK15}]
\label{prop:additive}
Let  be a non-trivial additive positive definite kernel  on persistence diagrams such that  for the diagonal set .
Then, for any , there exists no constant  such that

\end{prop}

\begin{proof}
Since  is non-trivial, there exists a persistence diagram  such that .
Then, for any , we compute both distance between  and the diagonal set :

Hence,  cannot be bounded by  with a constant .
\end{proof}

Actually, since the kernel  defined by  is non-trivial, additive, and , it seems that the PWGK would not satisfy the bottleneck stability and it contradicts Theorem \ref{thm:kernel_stability}.
However, when this counterexample is applied to Proposition \ref{prop:general_stability}, 
because  and , we obtain

In other words, Proposition \ref{prop:general_stability} is not affected by  in  and Theorem \ref{thm:kernel_stability} does not contradict with Proposition \ref{prop:additive}.

\newpage

\bibliographystyle{alpha}
\bibliography{reference}

\end{document}
