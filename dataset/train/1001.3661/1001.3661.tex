\documentclass[12pt]{article} 

\usepackage{epic,eepic,framed,epsf,latexsym,amsmath,amssymb,amscd,mathrsfs,slashbox }
\usepackage{graphicx}
\newcommand{\narrowmarg}{
\setlength{\oddsidemargin}{0in}
\setlength{\evensidemargin}{0in}
\setlength{\textwidth}{6.5in}
\setlength{\textheight}{9in}
\setlength{\topmargin}{0in}
\setlength{\headheight}{0in}
\setlength{\headsep}{0in}
}

\tolerance=10000
\newcommand{\grat}{{\rm GR}}
\newcommand{\reggrat}{{\rm GR}}
\newcommand{\snbt}{{\rm SNBT}}
\newcommand{\EH}[1]{{\rm E}_{\cal H}\left[#1\right] }
\newcommand{\weight}{{\rm weight}}
\newcommand{\oneto}[1]{{\{1,\ldots,#1\}}}
\newcommand{\GM}{{\rm GM}}

\newtheorem{theorem}{Theorem}[section]
\newtheorem{lemma}[theorem]{Lemma}
\newtheorem{Remark}[theorem]{Remark}
\newtheorem{note}[theorem]{Note}
\newtheorem{proposition}[theorem]{Proposition}
\newtheorem{open question}[theorem]{Open question}
\newtheorem{corollary}[theorem]{Corollary}
\newtheorem{conjecture}[theorem]{Conjecture}
\newtheorem{definition}[theorem]{Definition}
\newtheorem{sublemma}[theorem]{Sublemma}
\newtheorem{numberless}{Theorem}
\newtheorem{claim}{Claim}
\renewcommand{\thenumberless}{\Alph{numberless}}

\newcommand{\mydot}{{\bullet}}
\newcommand{\comp}{{\rm comp}}
\newcommand{\kernel}{{\rm ker}}
\newcommand{\image}{{\rm im}}

\newcommand{\restrict}[2]{{{#1}[{#2}]}} 
\newcommand{\vs}[1]{\myfield\left( {#1} \right)}
\newcommand{\LR}{{{\rm L}\to{\rm R}}}
\newcommand{\RL}{{{\rm R}\to{\rm L}}}

\newcommand{\trace}{{\rm Tr}}
\newcommand{\gr}{{\cal G}}  \newcommand{\gross}[1]{{\gr\left( {#1} \right)}}
\newcommand{\myfield}{{\mathbb Q}} 
\newcommand{\objects}[1]{{{\rm Ob}\left( {#1} \right)}} \newcommand{\morphisms}[1]{{{\rm Fl}\left( {#1} \right)}} \newcommand{\fleches}[2]{{{\rm Fl}^{#1}\left( {#2} \right)}} \newcommand{\underfleches}[2]{{{\underline {\rm Fl}}^{#1}\left( {#2} \right)}} 
\newcommand{\twoleftarrows}{\;
  \mbox{\vbox{\hbox{}\vskip-.35truecm\hbox{}
  \vskip-.05truecm}}\;}
\newcommand{\threeleftarrows}{\;
  \mbox{\vbox{\hbox{}\vskip-.35truecm\hbox{}
  \vskip-.35truecm\hbox{}\vskip-.15truecm}}\;}

\newcommand{\tworightarrows}{\;
  \mbox{\vbox{\hbox{}\vskip-.35truecm\hbox{}
  \vskip-.05truecm}}\;}
\newcommand{\threerightarrows}{\;
  \mbox{\vbox{\hbox{}\vskip-.35truecm\hbox{}
  \vskip-.35truecm\hbox{}\vskip-.15truecm}}\;}

\newcommand{\kbig}{{K_{\rm big}}}
\newcommand{\ksmall}{{K_{\rm small}}}
\newcommand{\zbig}{{Z_{\rm big}}}
\newcommand{\zsmall}{{Z_{\rm small}}}

\newcommand{\isom}{\simeq} \newcommand{\scl}[1]{{\rm sc}\left({#1}\right)} \newcommand{\tcl}[1]{{\rm tc}\left({#1}\right)} 
\newcommand{\shriek}{{ATTENTION!!!! ATTENTION!!!! ATTENTION!!!! }}
\newcommand{\from}{\colon}
\newcommand{\ignore}[1]{}

\newcommand{\espace}{{\em espace \'etal\'e}}
\newcommand{\espaces}{{\em espaces \'etal\'es}}

\newcommand{\floor}[1]{\left\lfloor #1\right\rfloor}

\newcommand{\Hom}{{\rm Hom}}
\newcommand{\simexp}[2]{{\rm SHom}\left({#1},{#2}\right)}
\newcommand{\rder}{{\underline{\underline{ R}}}}
\newcommand{\lder}{{\underline{\underline{ L}}}}
\newcommand{\cat}[1]{{\Delta_{#1}}}
\newcommand{\dercat}[1]{{\cdb(\myfield({#1}))}}

\newcommand{\cohcomp}{{\rm cc}}      
\newcommand{\co}{{\cal O}}
\newcommand{\ca}{{\cal A}}
\newcommand{\cb}{{\cal B}}
\newcommand{\cd}{{\cal D}}
\newcommand{\cdb}{{\cal D}^{\rm b}}
\newcommand{\cc}{{\cal C}}
\newcommand{\ck}{{\cal K}}
\newcommand{\cq}{{\cal Q}}
\newcommand{\ce}{{\cal E}}
\newcommand{\ct}{{\cal T}}
\newcommand{\cg}{{\cal G}}
\newcommand{\ch}{{\cal H}}
\newcommand{\cm}{{\cal M}}
\newcommand{\ci}{{\cal I}}
\newcommand{\cj}{{\cal J}}
\newcommand{\cw}{{\cal W}}
\newcommand{\cl}{{\cal L}}
\newcommand{\cf}{{\cal F}}
\newcommand{\cv}{{\cal V}}
\newcommand{\cp}{{\cal P}}
\newcommand{\cu}{{\cal U}}
\newcommand{\cx}{{\cal X}}
\newcommand{\cy}{{\cal Y}}
\newcommand{\cz}{{\cal Z}}
\newcommand{\cs}{{\cal S}}
\newcommand{\cn}{{\cal N}}

\newcommand{\id}{{\rm Id}}

\renewcommand{\complement}[1]{#1^{\rm c}}

\newcommand{\lin}{{\rm Lin}}
\newcommand{\fdvs}{{\rm FDVS}}

\newcommand{\mono}{{\mathbb M}}
\newcommand{\bool}{{\mathbb B}}
\newcommand{\reals}{{\mathbb R}}
\newcommand{\integers}{{\mathbb Z}}
\newcommand{\complex}{{\mathbb C}}
\newcommand{\csphere}{\complex\cup\{\infty\}}
\newcommand{\one}{{\vec 1}}
\newcommand{\zero}{{\vec 0}}


\newcommand{\E}[1]{\mbox{E}\left[#1\right] }
\newcommand{\prob}[1]{{\rm Prob}\left\{ #1 \right\} }

\newcommand{\proof}{{\par\noindent {\bf Proof}\space\space}}
\newcommand{\proofbox}{\begin{flushright}\end{flushright}}



\newcommand{\commentjf}[1]{{\bf [#1]}}

\title{Tight products and Graph Expansion}

\author{
  Amit Daniely\thanks{Department of Mathematics,
    The Hebrew University, Jerusalem, Israel,
    {\tt amit.daniely@mail.huji.ac.il}}
\and
  Nathan Linial\thanks{Department of Computer Science,
    The Hebrew University, Jerusalem, Israel,
    {\tt nati@cs.huji.ac.il}}
}

\date{January 20, 2010}


\begin{document} 
\maketitle
\begin{abstract}
In this paper we study a new product of graphs called
{\em tight product}. A graph  is said to be a tight product of
two (undirected multi) graphs  and ,
if  and both projection maps 
and  are covering maps. It is not a priori
clear when two given graphs have a tight product (in fact, it
is -hard to decide). We investigate the conditions under which this
is possible. This perspective yields a new
characterization of class-1 -regular graphs.
We also obtain a new model of random -regular graphs whose
second eigenvalue is almost surely at most .
This construction resembles random graph lifts,
but requires fewer random bits.
\end{abstract}


\section{Introduction and Background}
\subsection{Notations and conventions}
All the graphs in this paper are undirected, possibly with
multiple edges and
self-loops. We denote the group of permutations on a set 
by . A {\em -factor} in a graph  is a spanning subgraph
that is the disjoint union of cycles. A {\em -factorization}
is a partitioning of  into -factors. It is well known
\cite{Diestel} that every -regular graph  has a -factorization.
In other words, every -regular graph  on vertex set  can be
constructed as follows: There are  permutations ,
such that .
We denote such a graph by . In calculating vertex degrees, multiple
edges are counted by multiplicity and by convention, the
loop corresponding to  is also counted twice. The neighbor set of
vertex  in a graph  is denoted by  (or just ). Note that  is also a multiset.

By  we denote the set of {\em ordered} pairs  such that .
\subsection{Expanders and Ramanujan graphs}
Let  be an -vertex -regular graph. We denote the eigenvalues of its adjacency matrix by 
. We say that  is
an {\em ) graph} if  for every .
We recall some basic facts about expander graphs and refer the reader to~\cite{HLW06}
for a recent survey on expander graphs and the rich theory around them.
The Alon-Boppana~\cite{alon} bound states that
. If
 we say that  is a {\em Ramanujan} graph.
It is known (\cite{LPS88, Mar88, Mor94}) that if  is a prime power, then there
exist infinitely many -regular Ramanujan graphs (with explicit constructions).
For other values of  it is still unknown whether arbitrarily large -regular Ramanujan graphs exist.
A major result due to Friedman~\cite{Fri} is that for every 
and every  almost every -regular graph satisfies .

\subsection{Lifts of graphs}\label{sub-se:lifts}
\begin{definition}
We say that a graph  is a {\em lift} of {\em a base graph}  (or that  {\em covers} ) if there
is a map  (a {\em covering map}) such that for every ,
 maps  one to one and onto .
For every  the set  is called the {\em fiber} of . 
Similarly, for , we say that  is the fiber of .
(We remind the reader that multiple edges and loops are allowed and the definitions here
should be modified accordingly).
\end{definition}
We recall some basic facts on lifts of graphs and refer the reader to~\cite{AL02,AL06,ALM02,LR05}
for a more thorough account and some recent work on the subject.
\begin{proposition}\label{lifts basic properties}
Let  be a covering map between two finite graphs.
\begin{enumerate}\label{pro:basic properties lifts}
\item For every , . In particular, if  is -regular then so is .
\item If  in an eigenfunction of  with eigenvalue ,
then  is an eigenfunction of  with eigenvalue .
Such an eigenfunction-eigenvalue pair of  is considered {\em old}. 
The other eigenfunctions and eigenvalues of  are considered {\em new}.
\item If  is disconnected then so is .
\item .
\item If  is connected then all the fibers of vertices in  have the same cardinality,
which we call the {\em covering number} of the lift.
\end{enumerate}
\end{proposition}

Let  be fixed connected graph. Denote by 
the collection of all lifts of  with covering number .
It is not hard to see that every
member  has the following description.
It has vertex set  where the projection on the first
coordinate is the covering map from  to . To define the edges in ,
fix an arbitrary orientation on the edges of  and associate a permutation  to every edge .
The edge set of  is . This definition
lends itself naturally to randomization. In particular,  is
a random -lift if the 's are chosen uniformly at random from .
It was shown in ~\cite{LP09} that for every -regular  it holds with probability
 that all new eigenvalues of a random  are bounded by .

\subsection{Tight products}
If  is a lift of both  and , we say that  is a {\em common lift} of these graphs.
This notion has been studied by Leighton~\cite{Lei82} who showed
that two finite graphs  and  have a common finite lift iff they share the same {\em universal cover}.
Thus, in particular, every two -regular graphs have a common finite cover (as observed already by~\cite{AG81}).

In this paper we study a special kind of common lift.
\begin{definition}
A graph  is called a {\em tight product} of graphs  and  if  and 
both projection maps  and  are covering maps .
\end{definition}
This definition extends in the obvious way to tight products of more than two graphs.

In section \ref{se:existence and basic properties} we study some basic properties of tight products.
Specifically, we find conditions for its existence.
It turns out that  and  can have a tight product only if they are regular graphs 
of the same regularity. We also give sufficient conditions for the existence of a tight product.
E.g. for  even every two -regular graphs have a tight product.
On the other hand, some complication is to be expected here, because when  is odd
it is -hard to determine whether a given pair of -regular graphs has a tight product.

In section \ref{se:random models} we present some random models for regular graphs based on tight products.
We start with a -regular graph  as defined above,
where  are permutations on .
We choose permutations  uniformly at random and define .
Namely,  and . 
Note that  is a tight product of  and the {\em random graph} . 
We use the trace method to show that all the new eigenvalues of  are bounded (in absolute value) by .
An adaptation of the methods of~\cite{LP09} might improve this upper bound to .

An interesting feature of this model is that compared with the standard model of random lifts,
it offers a reduction in the necessary number of random bits. Whereas a random lift uses a
random permutation for each edge of the base graph, this model uses only  random permutations. 
In addition, the generated graph has a concise representation. We discuss those aspects in the last 
subsection of section \ref{se:random models} and suggest some questions for future research.


\section{Existence and basic properties}\label{se:existence and basic properties}

 
\subsection{Basic properties}\label{se:basic properties}
In this section  is always a tight product of  and . Here are some fairly obvious
consequences of proposition \ref{lifts basic properties} and the definition of tight product:

\begin{proposition}
\
\begin{enumerate}\label{pro:basic properties}
\item Every eigenvalue of  or  is also an eigenvalue of 
\item If  or  is bipartite then so is .
\item If  or  is disconnected then so is .
\item If both  and  are bipartite then  is disconnected. 
\item If  and  have a tight product , then  and  are -regular with 
the same .
\end{enumerate}
\end{proposition}
The only fact that needs some elaboration is 4. To see it,
let  and   be bipartitions of  respectively. There is no edge
between  and .\\

We now turn to prove that every pair of -regular graphs
has a tight product. This is a simple but useful observation.

\begin{proposition} \label{existence-basic}
\begin{enumerate}
\item Every two -regular graphs have a tight product. \label{2d-existence}
\item If both  and  are -regular and have a perfect matching,
then  have a tight product. \label{2d+1-existence}
\end{enumerate}
\end{proposition}
\proof
Since ,  are -regular graphs, they can be expressed as

for some permutations . We note that the graph

is a tight product of , . Indeed the neighbor set of the vertex
 is .
It follows that the projection maps  map  
one to one and onto  and  respectively. 
The first claim follows.

The second claim can be proved in a similar manner, since a -regular
graph that contains a perfect matching is the union of  permutations,
one of which is an involution with
no fixed points (and corresponds to the perfect matching). The proof follows along the same lines, but the edges that
correspond to the perfect matching are counted only once and not twice.
\proofbox

The above construction suggests a method for generating random
tight products. Start with a fixed -regular graph  and 
pick a random -regular graph . Now let  be a tight product of , . 
To simplify matters, we can choose a fixed -factorization of  and compose  from  random permutations.
Alternatively, both  and  can be selected at random. In Section~\ref{se:random models} we investigate
the expansion properties of such graphs.

The rest of this section concerns
the problem of finding a tight product for a given pair of graphs.

\subsection{Class classifier and The computational hardness of finding a tight product}\label{hardness}
Proposition \ref{existence-basic} may suggest that every two -regular graphs have a tight product. 
This is, however, not true as we observe below.\\
Recall Vizing's Theorem: If  is the
largest vertex degree in , then 's edge-chromatic number, , is either  or .
Accordingly,  is said to be of {\em class 1} or {\em class 2}.
It is known (\cite{Hol81}) that it is NP-Hard to determine the edge chromatic number, even if we restrict ourselves to cubic graphs. We prove that:
\begin{theorem}\label{theorem-class-classifier}
For every positive integer , there is a -regular graph, , with the property that every -regular graph  is of class-1 if and only if it has a tight product with . 
\end{theorem}
Consequently:
\begin{theorem}
The following decision problem TIGHTPRODUCT is NP-complete.\\
Input: Two finite graphs .\\
Output: Do  have a tight product?
\end{theorem}
Before we turn to prove theorem \ref{theorem-class-classifier} we discuss two notions - {\em neighborly permutations} and {\em semi-coloring}.

{\em Neighborly permutations:} Suppose that  is a tight product of  and . As in every lift, every edge  defines a bijection  from the fiber of  to the fiber of  and we denote . Since the lift  is a tight product,  is a permutation of  that maps every vertex to one of its neighbors. A permutation  on the vertex set of a graph  with this property is called {\em neighborly permutation}. Note that neighborly permutations correspond to oriented spanning subgraphs of  that are the union of vertex-disjoint cycles (where we permit a single edge to be considered as a cycle as well).\\
Neighborly permutations are useful in characterizing tight products:

\begin{proposition}\label{pro:tight-product-def-via-neighborly-permutation}
Let  and  be -regular graphs. Suppose that  is a tight product of  and  with . Then:
\begin{enumerate}
\item For every , .
\item For , the mapping  from  to  is one to one and onto.\label{condition-2}
\end{enumerate}
Conversely, consider a collection  of neighborly permutations of , that satisfies the above conditions. There is a unique tight product  of  and  with .
\end{proposition}
\proof
Suppose that  is a tight product of . As in every lift, each  satisfies . For condition \ref{condition-2}, suppose that  for . Then, . As a covering map, the projection , maps  one to one and onto , so we have . We showed that the mapping is one to one and since  =  it is also onto.

Suppose now that  satisfies the above conditions. We define a tight product  of  and  by setting  and . It is clear that the projection  is a covering map and that . By condition \ref{condition-2}, for every , the projection  maps  one to one and onto , so  is a covering map and  is indeed a tight product. The uniqueness of  is clear.
\proofbox
\begin{note} Every regular graph  has a neighborly permutation. To see that, consider the standard -lift,  of  (The vertex set is  and  is adjacent to  iff  and ). The regular bipartite graph  contains a perfect matching  where the corresponding edges  form a collection of cycles (some of which may be single edges viewed as a cycle of length 2). We orient those cycles arbitrarily to obtain a neighborly permutation.
\end{note}

{\em Semi-coloring:} Let  be a graph and let  be the largest vertex degree in . A {\em semi-coloring} is a coloring of  with color set , i.e. each color is either an element of  or an
unordered pair of such elements. In the latter case we view the edge as being colored ``half  and half ''. The coloring must satisfy, for every :
\begin{enumerate}
\item For every , the total weight of  on the edges incident with  is at most .
\item For every , there are either  or  edges colored  incident with .
\end{enumerate}
Note that, for -regular graphs, the total weight of  on the edges incident with some vertex  is exactly . Also note that if  is semi-colored then the collection of edges colored by a specific pair is a union of vertex-disjoint cycles.\\
This seemingly strange concept is related to tight products via the following proposition:
\begin{proposition}\label{pro:semi-coloring}
Let , be -regular graphs such that  is semi-colored and  is of class-1. Then  and  have a tight product.
\end{proposition}
\proof
To prove the existence of a tight product of  and , we construct a collection  of neighborly permutations of  that satisfies the conditions of proposition \ref{pro:tight-product-def-via-neighborly-permutation}.

Since  is of class-1, there is 1-factorization . We define neighborly permutations on  as follows: For ,  is a union of vertex-disjoint cycles. Fix some orientation on those cycles and define  to be the corresponding permutation. Note that  is an involution.\\
We now define:
\begin{enumerate}
\item If  is colored by , we define .
\item The set of edges in  colored by the pair  is the union of vertex-disjoint cycles and we arbitrarily orient those cycles. If  is colored by  and the orientation is , we define .
\end{enumerate}
It is clear that  satisfies the first condition in proposition \ref{pro:tight-product-def-via-neighborly-permutation}.
To see that condition 2 holds, we note that if  with  then 
 for every . 
For example, if  is colored by , and  is colored by , then 
 is the vertex that is matched to  by the matching  and 
 is the vertex that is matched to 
 by the matching . Since  and  are disjoint, those vertices are different. 
Consequently, the mapping  is one to one, 
and since both graphs are -regular, it is also onto and condition 2 holds.  
\proofbox

\proof (of Theorem \ref{theorem-class-classifier}).
We postpone the construction of  to the end of the proof, and mention two properties it has on which we rely:
\begin{enumerate}
\item There is a vertex,  that does not belong
to any proper cycle, i.e. all edges incident with  are bridges.
\item  has a semi-coloring.
\end{enumerate}
By proposition \ref{pro:semi-coloring}, every -regular graph of class-1 has a tight product with . Conversely let  be a -regular graph, and suppose that  is a tight product of  and . We must show that  is of class-1.

Denote .
First, we claim that  for every . To this end,
express the permutation  as a product of disjoint cycles. By the defining property of neighborly permutation, 
for all indices ,  is a
(graph theoretic) simple cycle in . But the only simple cycles in  that contain the vertex  are of length 2 (i.e. a single edge), hence,  as required.

By the last discussion, we can define a -edge coloring  of  as follows: . To see that this yields a proper edge coloring of , consider two incident edges . By proposition \ref{pro:tight-product-def-via-neighborly-permutation}, .

{\em The construction of }. We take  copies of  and remove one edge from each copy. We add new vertex called the {\em secondary pivot} and connect it to every vertex that belonged to one of the removed edges. The graph we obtained is called {\em cluster}. To construct , we start with  clusters and add a new vertex called the {\em main pivot} and connect it to each of the secondary pivots. We enumerate the pivots by  where  is the main pivot and  are the secondary pivots. A picture is worth more than thousand words.

\begin{figure}[ht]
\centering
\setlength{\unitlength}{0.00033333in}
\begingroup\makeatletter\ifx\SetFigFont\undefined \gdef\SetFigFont#1#2#3#4#5{\reset@font\fontsize{#1}{#2pt}\fontfamily{#3}\fontseries{#4}\fontshape{#5}\selectfont}\fi\endgroup {\renewcommand{\dashlinestretch}{30}
\begin{picture}(8132,8455)(0,-10)
\put(4228,4223){\circle{166}}
\put(4055,8277){\circle{88}}
\put(4211,7790){\circle{88}}
\put(3525,8389){\circle{88}}
\put(3859,7394){\circle{88}}
\put(3183,8003){\circle{88}}
\put(3347,7511){\circle{88}}
\put(3070,7692){\circle{88}}
\put(2778,7864){\circle{88}}
\put(2948,7372){\circle{88}}
\put(2258,7975){\circle{88}}
\put(2588,6986){\circle{88}}
\put(2069,7089){\circle{88}}
\put(1899,7591){\circle{88}}
\put(324,5332){\circle{88}}
\put(836,5332){\circle{88}}
\put(52,4864){\circle{88}}
\put(1103,4874){\circle{88}}
\put(315,4419){\circle{88}}
\put(834,4423){\circle{88}}
\put(577,4215){\circle{88}}
\put(322,3990){\circle{88}}
\put(844,4000){\circle{88}}
\put(56,3531){\circle{88}}
\put(1100,3538){\circle{88}}
\put(840,3076){\circle{88}}
\put(312,3070){\circle{88}}
\put(8034,5632){\circle{88}}
\put(7619,5332){\circle{88}}
\put(7979,6171){\circle{88}}
\put(7134,5545){\circle{88}}
\put(7504,6376){\circle{88}}
\put(7086,6068){\circle{88}}
\put(7172,6388){\circle{88}}
\put(7248,6718){\circle{88}}
\put(6831,6404){\circle{88}}
\put(7192,7248){\circle{88}}
\put(6352,6628){\circle{88}}
\put(6291,7152){\circle{88}}
\put(6714,7470){\circle{88}}
\put(6740,1021){\circle{88}}
\put(6326,1323){\circle{88}}
\put(7236,1241){\circle{88}}
\put(6379,1851){\circle{88}}
\put(7284,1755){\circle{88}}
\put(6862,2058){\circle{88}}
\put(7192,2074){\circle{88}}
\put(7530,2105){\circle{88}}
\put(7102,2405){\circle{88}}
\put(8016,2321){\circle{88}}
\put(7168,2929){\circle{88}}
\put(7647,3149){\circle{88}}
\put(8080,2844){\circle{88}}
\put(1973,862){\circle{88}}
\put(2132,1349){\circle{88}}
\put(2335,459){\circle{88}}
\put(2650,1462){\circle{88}}
\put(2840,571){\circle{88}}
\put(2996,1066){\circle{88}}
\put(3114,757){\circle{88}}
\put(3248,445){\circle{88}}
\put(3401,945){\circle{88}}
\put(3604,51){\circle{88}}
\put(3920,1045){\circle{88}}
\put(4277,657){\circle{88}}
\put(4120,150){\circle{88}}
\drawline(2770,7864)(3074,7680)(3341,7505)
\drawline(3181,7995)(3072,7697)(2947,7375)
\drawline(4050,8280)(4214,7775)
\drawline(3530,8389)(4051,8279)
\drawline(3861,7388)(4216,7782)
\drawline(3172,7998)(3529,8395)
\drawline(3857,7393)(3336,7504)
\drawline(4049,8279)(3342,7505)
\drawline(3172,7999)(4214,7779)(3343,7503)
	(3527,8385)(3861,7390)(4049,8275)
	(3181,8000)(3175,7998)
\drawline(3526,8388)(4214,7779)
\drawline(3178,7999)(3863,7391)
\drawline(2067,7088)(1902,7594)
\drawline(2585,6980)(2066,7091)
\drawline(2255,7981)(1899,7585)
\drawline(2943,7370)(2588,6975)
\drawline(2258,7975)(2780,7865)
\drawline(2067,7091)(2774,7863)
\drawline(2944,7369)(1904,7590)(2774,7865)
	(2589,6984)(2255,7980)(2068,7093)
	(2935,7367)(2941,7370)
\drawline(2590,6982)(1904,7590)
\drawline(2938,7369)(2252,7979)
\texture{55888888 88555555 5522a222 a2555555 55888888 88555555 552a2a2a 2a555555 
	55888888 88555555 55a222a2 22555555 55888888 88555555 552a2a2a 2a555555 
	55888888 88555555 5522a222 a2555555 55888888 88555555 552a2a2a 2a555555 
	55888888 88555555 55a222a2 22555555 55888888 88555555 552a2a2a 2a555555 }
\drawline(3080,7705)(4212,4223)
\drawline(3080,7705)(4212,4223)
\drawline(319,3983)(590,4216)(837,4415)
\drawline(322,4415)(572,4218)(840,4001)
\drawline(319,5329)(851,5329)
\drawline(55,4869)(321,5329)
\drawline(1110,4874)(844,5334)
\drawline(316,4407)(49,4869)
\drawline(1104,4872)(837,4411)
\drawline(321,5327)(838,4416)
\drawline(316,4407)(846,5330)(840,4416)
	(58,4864)(1109,4875)(323,5326)
	(317,4416)(317,4410)
\drawline(55,4864)(846,5330)
\drawline(317,4413)(1109,4876)
\drawline(841,3075)(310,3075)
\drawline(1105,3535)(838,3075)
\drawline(49,3529)(316,3069)
\drawline(844,3996)(1110,3535)
\drawline(56,3532)(322,3993)
\drawline(838,3076)(321,3987)
\drawline(844,3996)(313,3075)(319,3987)
	(1101,3539)(50,3529)(837,3077)
	(844,3987)(844,3993)
\drawline(1105,3539)(313,3075)
\drawline(844,3990)(50,3527)
\drawline(567,4229)(4229,4229)
\drawline(567,4229)(4229,4229)
\drawline(7246,6726)(7162,6379)(7079,6072)
\drawline(7496,6376)(7178,6388)(6834,6406)
\drawline(8037,5638)(7606,5326)
\drawline(7979,6165)(8034,5637)
\drawline(7128,5541)(7615,5326)
\drawline(7497,6384)(7983,6168)
\drawline(7132,5546)(7077,6075)
\drawline(8034,5638)(7079,6071)
\drawline(7497,6385)(7611,5326)(7078,6069)
	(7974,6167)(7130,5541)(8031,5637)
	(7502,6377)(7498,6382)
\drawline(7976,6169)(7611,5326)
\drawline(7500,6380)(7131,5539)
\drawline(6289,7155)(6718,7467)
\drawline(6346,6628)(6291,7157)
\drawline(7196,7251)(6710,7468)
\drawline(6828,6408)(6342,6624)
\drawline(7193,7247)(7249,6717)
\drawline(6291,7155)(7246,6722)
\drawline(6827,6407)(6715,7465)(7247,6723)
	(6351,6625)(7195,7251)(6294,7155)
	(6823,6415)(6827,6410)
\drawline(6349,6624)(6715,7465)
\drawline(6825,6413)(7194,7254)
\drawline(7189,6382)(4226,4229)
\drawline(7189,6382)(4226,4229)
\drawline(7537,2110)(7180,2081)(6864,2066)
\drawline(7281,1762)(7194,2069)(7105,2402)
\drawline(6747,1021)(6316,1334)
\drawline(7231,1238)(6744,1022)
\drawline(6373,1854)(6319,1325)
\drawline(7290,1765)(7234,1235)
\drawline(6379,1852)(6867,2069)
\drawline(6745,1023)(6862,2065)
\drawline(7291,1765)(6318,1329)(6860,2066)
	(7231,1244)(6375,1853)(6743,1026)
	(7285,1758)(7287,1763)
\drawline(7232,1242)(6318,1329)
\drawline(7286,1761)(6372,1851)
\drawline(7649,3152)(8080,2839)
\drawline(7166,2935)(7651,3150)
\drawline(8021,2319)(8077,2848)
\drawline(7105,2408)(7160,2938)
\drawline(8015,2321)(7530,2104)
\drawline(7649,3149)(7533,2108)
\drawline(7104,2409)(8076,2841)(7534,2106)
	(7165,2930)(8021,2320)(7650,3147)
	(7112,2415)(7107,2410)
\drawline(7164,2931)(8076,2841)
\drawline(7109,2412)(8023,2321)
\drawline(7193,2057)(4229,4209)
\drawline(7193,2057)(4229,4209)
\drawline(3254,441)(3118,770)(3004,1067)
\drawline(2846,576)(3109,754)(3399,941)
\drawline(1975,855)(2139,1362)
\drawline(2331,462)(1976,858)
\drawline(2652,1468)(2131,1356)
\drawline(2850,569)(2329,458)
\drawline(2653,1462)(3008,1065)
\drawline(1977,857)(3003,1068)
\drawline(2851,568)(2136,1358)(3003,1069)
	(2336,464)(2652,1466)(1978,859)
	(2842,572)(2848,571)
\drawline(2335,462)(2136,1358)
\drawline(2846,571)(2649,1468)
\drawline(4280,656)(4116,150)
\drawline(3924,1048)(4279,653)
\drawline(3603,44)(4123,155)
\drawline(3406,943)(3925,1055)
\drawline(3603,51)(3246,446)
\drawline(4278,655)(3252,443)
\drawline(3406,944)(4117,154)(3251,442)
	(3919,1048)(3604,44)(4277,653)
	(3414,939)(3407,942)
\drawline(3921,1048)(4117,154)
\drawline(3410,940)(3606,43)
\drawline(3098,752)(4229,4235)
\drawline(3098,752)(4229,4235)
\end{picture}
}
 \caption{}
\label{fig:bla}
\end{figure}

It is clear that no cycle goes through the vertex , so it only remains to find a semi-coloring of .\\
For , consider the subgraph  of  that is the cluster whose secondary pivot is  together with the edge . We can choose a perfect matching  such that . The graph that is obtained from  upon removal of the edges in  and the vertex  is -regular. Therefore, it has -factorization . We decompose  into   pairs . For each , if  we color  by , and if  we color  by .
It easy to check that this coloring is semi-coloring. 


\proofbox



\subsection{Existence of tight products for cubic graphs and Vizing's theorem}\label{odd regularity}
This section is devoted to the following claim:
\begin{theorem}\label{theorem:3-clorable-graph-has-a-tight-product-with-every-cubic-graph}
Every graph , with  has a semi-coloring. Consequently, a class-1 cubic graph has a tight product with {\em every} cubic graph.
\end{theorem}
As an aside, we obtain a new proof to Vizing's theorem for the case of cubic graphs. 
\proof
It is convenient to introduce some terminology for the available colors - we denote Blue=1, Red=2, Green=3, Bright Blue=, Bright Red=, Bright Green=.\\
Let us assume first that  is bridgeless, in particular, all vertices have degree  or .\\
If  is cubic, then by Petersen's theorem, it has a perfect matching . We color all the edges in  blue and the rest of the edges bright blue. This is a semi-coloring.\\
Suppose now that  is
the set of degree 2 vertices in  and . We can construct a -regular graph , with at most one bridge, that contains a copy of  as an induced subgraph of  (E.g. take two copies of  and connect each corresponding pair of
degree 2 vertices). By Petersen's theorem,  has a perfect matching and like in the previous discussion,  has a semi-coloring  (note that we used a version of Petersen's theorem claiming that every cubic graph with {\em at most two} bridges have a perfect matching). Let  be the restriction of  to . Clearly, the conditions for semi-coloring  hold for  at every vertex  of degree 3. It might happen, however, that a degree-2 vertex  has exactly one brightly colored edge, say bright blue. However, in this case  is the end vertex of a bright blue path . We recolor  alternately red and green instead, thus eliminating the problem without creating any new problematic vertices. Repeating this procedure, if necessary, concludes with a semi-coloring.

If  contains a bridge  we remove it and deal separately with the two components using induction. By renaming the colors, if necessary, at one of the two components, we can combine them and color  as well to semi-color the whole .

Although this proof is formulated in the language of simple graphs, it carries through easily also when  may include parallel edges or
loops. The only thing worth mentioning is that it is easy to observe that Petersen's Theorem remains valid in this case.
\proofbox

Our approach sheds some new light on Vizing's classical theorem.

\begin{theorem}\label{theorem:vizing}
[Vizing's theorem for cubic graphs] Every cubic graph  can be 4-edge-colored.
\end{theorem}
\proof 
We start with a semi-coloring of . Let  be a cycle colored brightly, say bright-red. If  has even length we recolor its edges green/blue alternately. If  has odd length, we do likewise, except that the last edge is given our fourth color.
\proofbox
 
Finally we derive a necessary condition for two cubic graphs to have a tight product.
\begin{proposition}
Let  be a tight product of the graphs . If  has a bridge then  has a perfect matching.
\end{proposition}
\proof
Suppose  is a bridge. Denote  and define . Since  is a bridge,  is well defined - i.e.  (As in the proof of Theorem \ref{theorem-class-classifier}).

By proposition \ref{pro:tight-product-def-via-neighborly-permutation}, for every  there is exactly one  such that , so  is a perfect matching.
\proofbox
\begin{corollary}
If two cubic graphs have a tight product, at least one of them has a perfect matching.
\end{corollary}
\proof
Suppose that  are cubic graphs having a tight product. If  is bridgeless, it has a perfect matching by Petersen Theorem. Otherwise, the above proposition implies that  has a perfect matching.
\proofbox
\subsection{Conclusion and open questions}\label{se:open question}
Let  be -regular graphs with  odd. Table \ref{table} summarizes our knowledge and open questions regarding the existence of a tight product of  and .
\\\begin{figure}[ht]
\centering

\begin{tabular}{ |p{3cm}|p{3cm}|p{3cm}|p{3cm} }
\hline
\backslashbox{ is}{ is}  & class-1 & class-2 with a perfect matching & \multicolumn{1}{|p{3cm}|}{class-2 without a perfect matching} \\
\hline
class-1 & always exists \\
\cline{1-3}
class-2 with a perfect matching& always exists& \multicolumn{1}{|c|}{always exists}\\
\cline{1-4}
class-2 without a perfect matching& always exists for  (Open question for \mbox{})  & May not exist (We do not know a case where it exists)& \multicolumn{1}{|p{3cm}|}{Does not exist for  (Open question for \mbox{})}\\
\cline{1-4}
\end{tabular}
\caption{When does a tight product exist for two -regular graphs ( odd)?}
\label{table}
\end{figure}


Completion of the bottom left box in the table might be achieved by answering the following question: 
\begin{open question}
Does every graph have a semi-coloring?
\end{open question}
In this context, we note that the following families of graphs can be semi-colored:
\begin{enumerate}
\item Class-1 graphs (The -edge-coloring will do).
\item -regular graphs (Find a 2-factorization and color the -th factor half  half ).
\item -regular graphs containing a perfect matching (Use one color for the perfect matching. The remaining graph is -regular and can be handled as above).
\item Graphs with maximum degree  (Theorem \ref{theorem:3-clorable-graph-has-a-tight-product-with-every-cubic-graph}).
\end{enumerate}
It is of interest as well to seek tight products with additional properties, e.g.,:
\begin{open question}
Assume that  have a tight product. When do they have a connected tight product?
What can be said about the possible chromatic number of their tight products. Likewise for
other graph parameters.
\end{open question}
We also wonder whether Vizing's theorem can be proved in full along the same lines of theorem \ref{theorem:vizing}.

\section{Random models}\label{se:random models}

The spectrum of graph lifts (and more specifically random lifts of
of graphs) has attracted considerable interest. We now consider some basic questions in this vein
in the context of tight products.

\subsection{Background: Word maps}\label{Background:-Word-maps}
Let  be the alphabet consisting of the letters . We denote by  the set of all the words  with 
of length  with letters from . We associate a {\em word map} 
with every word  as follows:
For every  we define . With the uniform probability measure on ,  is a random (not necessarily uniform) permutation. We are interested in fixed points of such permutations and define .

A word  is
called {\em reduced} if it does not contain two inverse consecutive letters. If  is not reduced, we can drop a pair of consecutive inverse letters. This can be repeated until a reduced word is attained.  We denote the resulting word by . It is not hard to see that  does not depend on the order at which reduction steps are performed. It is clear that , so . We now define the {\em order} of , denoted by , to be the largest integer , such that  can be written as  with nonempty  (and  when  is empty). If  we say that  is {\em primitive}.

Bounds on  are the backbone of the analysis we'll present in the next subsection as well as in many theorems regarding expansion of random graphs (e.g. \cite{Fri03}). We now state two such bounds (proofs can be found in \cite{HLW06}). The first lemma says that
for a primitive  the behavior of  resembles
that of a random permutation. The second lemma bounds the number of imprimitive words.
\begin{lemma}
Let  be a primitive word. Then .
\end{lemma}
\begin{lemma}
The number of imprimitive words in  is  
\end{lemma}
A refined analysis of word maps can be found in \cite{LP09}. The (more involved) method of that paper might yield better bounds than what's shown below.



\subsection{First model - fixed base graph}
Fix a positive integer  and a -regular graph  expressed as

where  are permutation on .
Choose permutations  uniformly and independently at random and define:


 is called the {\em random product} of
the {\em random graph}  with the {\em base graph} .
Note that  is indeed a tight product of  and  (see proposition \ref{existence-basic}).
By proposition \ref{pro:basic properties}, all the eigenvalues of  (as well as of ) are also eigenvalues of . We use the trace method to bound  - the absolute value of the largest new eigenvalue of .\\
In \cite{Fri03} Friedman proved that the largest new eigenvalue in a (standard) lift of -regular graph is bounded in absolute value by  a.a.s. The proof we present is an adaptation of his proof.

\begin{theorem}

Consequently,  a.a.s. as .
\end{theorem}
\proof
Denote by  the adjacency matrices of  and . By Jensen's inequality,

Where the sum in the third expression is over
all new eigenvalues. But  has a combinatorial interpretation - it counts the {\em closed paths} of length  in .

Denote by  the set of all paths of length  in . We view  as the set  in the following manner: given a pair , the corresponding path is  where . It is clear that this correspondence is a bijection between  and . In the same manner, we denote the paths of length  in  and in  by  ,  respectively (and associate them with ). We denote by  the set of closed paths in .

Given , denote by  the indicator function of the event that  is a closed path in  and observe that . We define , similarly. It is clear that a path  in  is closed iff its projections on  and  are both closed. Consequently, \mbox{.}
With these notations and the lemmas from the previous subsection we obtain:

Following \cite{Fri03}, we split the sum according to whether  is primitive or not. For non-primitive , we overestimate  by . We then use the lemmas from subsection (\ref{Background:-Word-maps}).


By (\ref{mu bound}) we obtain:

For  (as we actually assume below),

The last inequality is justified since every entry on the diagonal of  is bounded by .
To finish, we choose . Then, from (\ref{trace bound 3}) we obtain:

Therefore,

\proofbox

\subsection{Conclusions and an open problem}
How large can  be? Again, following \cite{Fri03} we raise:
\begin{conjecture}
Let  be a random tight product as defined in the beginning of subsection \ref{Background:-Word-maps}. Then, for every ,  a.a.s.
\end{conjecture}
The potential advantage of this conjecture over Friedman's, is that it may allow to construct graphs with a near optimal spectral gap, using very limited randomness. More generally it is of interest to
find a distribution  on , with {\em small entropy}, such that if we choose  independently at random from the distribution , then  has small second eigenvalue w.h.p.
In this context we should mention
\cite{BG08}, where it is shown that there is  such that if 
 are chosen uniformly and independently at random, then the spectral radius of the Cayley graph of 
with generates  is a.a.s. (with respect to ) bounded by .

Tight products suggest another approach to this problem, as follows:
Consider  as a subset of . Here we allow  to grow with ,
so that  is much smaller than  and we indeed save in entropy.
Does this yield an expander family? Can we do this even with  that grows with ?

\section*{Acknowledgements}
We would like to thank Alex Lubotzky and Baoyindureng Wu for insightful comments.

\begin{thebibliography}{FMT06}

\bibitem[1]{AG81}
D. Angluin and A. Gardiner.
\newblock Finite Common Covering of Pairs of Regular Graphs.
\newblock {\em Journal of Combinatorial Theory}, Ser. B 30, pp. 184- 187, 1981.

\bibitem[2]{AL02}
N. Linial and A. Amit.
\newblock Random Graph Covering I: General Theory and Connectivity.
\newblock {\em Combinatorica}, 22(2002) 1- 18.

\bibitem[3]{AL06}
N. Linial and A. Amit.
\newblock Random lifts of graphs II: Edge expansion.
\newblock {\em Combinatorics Probability and Computing}, 15(2006) 317-332. 

\bibitem[4]{ALM02}
A. Amit, N. Linial and J. Matousek.
\newblock Random Lifts of Graphs III: Independence and Chromatic Number.
\newblock {\em Random Structures and Algorithms}, 20(2002) 1-22.

\bibitem[5]{alon}
N. Alon.
\newblock Eigenvalues and expanders.
\newblock {\em Combinatorica}, 6(2):83-96, 1986. Theory of Computing (Singer Island, FL, 1984). MR0875835 (88e:05077)

\bibitem[6]{BG08}
J. Bourgain and A. Gamburd.
\newblock Uniform expansion bounds for Cayely graphs of .
\newblock {\em Annals of Mathematics}, 167(2):625-642, 2008. 

\bibitem[7]{Diestel}
R. Diestel.
\newblock {\em Graph Theory.} 3rd ed.
\newblock Springer-Verlag Heidelberg, New York, 2005.


\bibitem[8]{Fri03}
J. Friedman.
\newblock Relative expanders or weakly relatively Ramanujan graphs.
\newblock {\em Duke Math.
J.}, 118(1):19-35, 2003. MR1978881 (2004m:05165)

\bibitem[9]{Fri}
J. Friedman.
\newblock A proof of Alon�s second eigenvalue conjecture.
\newblock {\em Memoirs of the A.M.S.}, to appear.

\bibitem[10]{Hol81}
I. Holyer.
\newblock The NP-Completeness of Edge-Coloring
\newblock {\em SIAM J. COMPUT} Vol. 10, No. 4, November 1981 (pp. 718-720)

\bibitem[11]{HLW06}
S. Hoory, N. Linial and A. Widgerson.
\newblock Expander Graphs and their Applications.
\newblock {\em Bulletin of the American Mathematical
Society}, Volume 43, Number 4, October 2006, pp. 439-531.

\bibitem[12]{Lei82}
F. T. Leighton.
\newblock Finite Common Covering of Graphs.
\newblock {\em Journal of Combinatorial Theory}, Ser. B 33, pp. 231- 238, 1981.

\bibitem[13]{LP09}
N. Linial and D. Puder.
\newblock Word Maps and Spectra of Random Graph Lifts.
\newblock {\em Random Structures and Algorithms}, to appear

\bibitem[14]{LPS88}
A. Lubotzky, R. Phillips, and P. Sarnak.
\newblock Ramanujan graphs.
\newblock {\em Combinatorica}, 8(3):261-277, 1988. MR0963118 (89m:05099)

\bibitem[15]{LR05}
N. Linial and E. Rozenman.
\newblock Random lifts of graphs: perfect matchings.
\newblock {\em Combinatorica}, 25(4):407-424, 2005. MR2143247 (2006c:05110)

\bibitem[16]{Mar88}
G. A. Margulis.
\newblock Explicit group-theoretic constructions of combinatorial schemes and their applications in the construction of expanders and concentrators.
\newblock {\em Problems of
Information Transmission}, 24(1):39-46, 1988. MR0939574 (89f:68054)

\bibitem[17]{Mor94}
M. Morgenstern.
\newblock Existence and explicit constructions of q + 1 regular Ramanujan graphs for every prime power q.
\newblock {\em J. Combin. Theory Ser. B}, 62(1):44-62, 1994. MR1290630 (95h:05089)

\bibitem[18]{Nil91}
A. Nilli.
\newblock On the Second Eigenvalue of a Graph.
\newblock {\em Discrete Math.}, 91(2):207-210, 1991. Mr1124768 (92j:05124)

\end{thebibliography}
\end{document} =
