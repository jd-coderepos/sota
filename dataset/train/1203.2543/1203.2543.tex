\documentclass{article}
\pdfminorversion=5

\usepackage[utf8]{inputenc}
\usepackage[numbers,sectionbib,sort&compress,comma,square]{natbib}
\usepackage[pdftex]{hyperref}
\usepackage{multirow}
\usepackage{verbatim}
\usepackage[ruled,vlined,linesnumbered]{algorithm2e}
\usepackage[table]{xcolor}
\usepackage{amssymb}
\usepackage{amsthm}
\usepackage{graphicx}
\usepackage[caption=false]{subfig}
\usepackage{authblk}

\newtheorem{probaux}{Problem}[section]
\newtheorem{theorem}{Theorem}
\newtheorem{corollary}[theorem]{Corollary}
\newtheorem{lemma}[theorem]{Lemma}

\newenvironment{prob}[3]{\bigskip\noindent\framebox{\parbox{0.97\textwidth}{\begin{probaux}{\sc
#1}\\{\hspace*{1cm} \bf \sf Instance:} #2\\{\hspace*{1cm} \bf \sf Question:}
#3\end{probaux}}}\bigskip}{}

\begin{document}

\title{Biclique-colouring verification complexity and 
biclique-colouring power graphs\thanks{An extended abstract published in: Proceedings
of Cologne Twente Workshop (CTW) 2012, pp. 134--138.
Research partially supported by FAPERJ--Cientistas do Nosso Estado,
and by CNPq-Universal.}}



\author[1]{H\'elio B. Mac\^edo Filho}
\author[2]{Simone Dantas}
\author[3]{\\ Raphael C. S. Machado}
\author[1]{Celina M. H. Figueiredo}
\affil[1]{COPPE, Universidade Federal do Rio de Janeiro}
\affil[2]{IME, Universidade Federal Fluminense}
\affil[3]{Inmetro --- Instituto Nacional de Metrologia, Qualidade e Tecnologia.}

\date{}

\maketitle

\let\thefootnote\relax\footnotetext{
\hfill\today
}

\begin{abstract}
Biclique-colouring is a colouring of the vertices of a graph in such a way that
no maximal complete bipartite subgraph with at least one edge is monochromatic. 
We show that it is co-complete to check whether a given 
function that associates a colour to each vertex is a biclique-colouring, 
a result that justifies the search for structured classes where the 
biclique-colouring problem could be efficiently solved. We consider 
biclique-colouring restricted to powers of paths and powers of cycles.    
We determine the biclique-chromatic number of
powers of paths and powers of cycles. The biclique-chromatic number of a power
of a path  is  if  and exactly
 otherwise. The biclique-chromatic number of a power of a cycle 
is at most~3 if  and exactly  otherwise; we additionally determine the powers
of cycles that are 2-biclique-colourable. All proofs are algorithmic and
provide polynomial-time biclique-colouring algorithms for graphs in the
investigated classes.
\end{abstract}


\section{Introduction}
\label{s:introduction}

Let  be a simple graph with order  vertices and
 edges.
A \emph{clique} of  is a maximal set of vertices of size at least~2 that
induces a complete subgraph of .
A \emph{biclique} of  is a maximal set of vertices that induces a complete
bipartite subgraph of  with at least one edge.
A \emph{clique-colouring} of~ is a function  that associates a colour to
each vertex such that no clique is monochromatic. If the function uses at
most~ colours we say that  is a \emph{-clique-colouring}.
A \emph{biclique-colouring} of  is a function  that associates a colour
to each vertex such that no biclique is monochromatic. If the function 
uses at most~ colours we say that  is a \emph{-biclique-colouring}.
The \emph{clique-chromatic number} of , denoted by , is the least
 for which  has a -clique-colouring. The \emph{biclique-chromatic
number} of , denoted by , is the least  for which  has a
-biclique-colouring.

Both clique-colouring and biclique-colouring have a ``hypergraph colouring
version.'' Recall that a hypergraph  is an ordered
pair where  is a set of vertices and  is a set of hyperedges,
each of which is a set of vertices. A colouring of hypergraph
 is a function that associates a colour to each
vertex such that no hyperedge is monochromatic. Let  be a graph and let
 and 
be the hypergraphs in which hyperedges are, respectively,  and  --- hypergraphs  and
 are called, resp., the \emph{clique-hypergraph} and the
\emph{biclique-hypergraph} of~. A clique-colouring of~ is a colouring of
its clique-hypergraph ; a biclique-colouring of~ is a
colouring of its biclique-hypergraph .


Clique-colouring and biclique-colouring are analogous problems in the sense that
they refer to the colouring of hypergraphs arising from graphs.
In particular, the hyperedges are subsets of vertices that are clique (resp. biclique).
The clique is a classical important
structure in graphs, hence it is natural that the clique-colouring problem has
been studied for a long time ---
see~\cite{Bacso,Defossez,Kratochvil,DanielMarx}. 
Only recently the biclique-colouring problem started to be
investigated~\cite{1210.7269}. 

Many other problems, initially stated for cliques, have
their version for bicliques~\cite{MR0409253,MR0065617}, such as
\emph{Ramsey number} and \emph{Tur\'an's theorem}.
The combinatorial game called on-line Ramsey
number also has a version for bicliques~\cite{MR2594965}. Although complexity
results for complete bipartite subgraph problems are mentioned
in~\cite{GareyJohnson} and the (maximum) biclique problem is shown to be
-hard in~\cite{Yannakakis}, only in the last decade the (maximal)
bicliques were rediscovered in the context of counting
problems~\cite{Gaspers,Prisner}, enumeration problems~\cite{Dias1,Nourine1}, and
intersection graphs~\cite{MarinaJayme}. 

Clique-colouring and biclique-colouring
have similarities with usual vertex-colouring. A proper vertex-colouring is also
a clique-colouring and a biclique-colouring --- in other words, both the
clique-chromatic number and the biclique-chromatic number are bounded above by
the vertex-chromatic number. Optimal vertex-colourings and clique-colourings
coincide in the case of -free graphs, while optimal vertex-colourings and
biclique-colourings coincide in the (much more restricted) case of
-free graphs --- notice that the triangle  is the minimal
complete graph that includes the graph induced by one edge (), while the
 is the minimal complete bipartite graph that includes the graph
induced by one edge (). But there are also essential differences. Most
remarkably, it is possible that a graph has a clique-colouring (resp.
biclique-colouring), which is not a clique-colouring (resp.
biclique-colouring) when restricted to one of its subgraphs.
Subgraphs may even have a larger clique-chromatic number (resp.
biclique-chromatic number) than the original graph. 

Clique-colouring and biclique-colouring also have similarities on complexity issues. 
It is known~\cite{Bacso} that it is co-complete
to check whether a given function that associates a colour to each vertex is a clique-colouring
by a reduction from . Later, an alternative -completeness
proof was obtained by a reduction from a variation of , in order to
construct the complement of a bipartite graph~\cite{Defossez}. Based on this, we open this
paper providing a corresponding result regarding the biclique-colouring problem:
it is co-complete to check whether a given function that associates 
a colour to each vertex is a biclique-colouring. The
co-completeness holds even when the input is a -free
graph.

We select two structured
classes for which we provide linear-time biclique-colouring algorithms:
powers of paths and powers of cycles. The choice of those classes has also
a strong motivation since they have been recently investigated in the context of
well studied variations of colouring problems. 
For instance, for a power of a path~, its -chromatic number is , if ; , if ; or , if ; whereas, for a power of a cycle
, its -chromatic number is , if ; , if ; at least , if ; , if ; or , if ~\cite{MR1979111}. Moreover, other well
studied variations of colouring problems when restricted to powers of cycles have been
investigated: chromatic number~\cite{MR1974376}, chromatic
index~\cite{meidanis}, total chromatic number~\cite{MR2303972}, choice
number~\cite{MR1974376}, and clique-chromatic number~\cite{MR2570638}.
It is known, for a power of a cycle , that
the chromatic number and the choice number are both , where  with ,  and , that the chromatic index is the maximum degree of  if, and only if,
 is even, that the total chromatic number is at most the maximum degree of
 plus 2, when  is even and , and that the
clique-chromatic number is~, when , and is at most~3, when .
Particularly, in the latter case, the clique-chromatic number is~3, when  is
odd and ; otherwise, it is~2. Note that total colouring is an open
and difficult problem and remains unsolved for powers of cycles~\cite{MR2303972}. Other
significant works have been done in power
graphs~\cite{MR1454439,MR2423405} and, in particular, in powers of
paths and powers of
cycles~\cite{MR1018529,MR1172679,MR2083449,MR2774114,MR1633075,MR2255625}.

\section{Complexity of biclique-colouring}

The biclique-colouring problem is a variation of the clique-colouring problem. Hence,
it is natural to investigate the complexity of biclique-colouring based on the
tools that were developed to determine the complexity of clique-colouring. 
We show that, similarly to the case of clique-colouring, it is
co-complete to check whether a given function that associates a
colour to each vertex of a graph is a biclique-colouring.
To achieve a result in this direction, we prove the -completeness
of the following problem: of deciding whether there exists a biclique of a
graph  contained in a given subset of vertices of~. Indeed, a function
that associates a colour to each vertex of a given graph  is a
biclique-colouring if, and only if, there is {\bf no} biclique of~
contained in a subset of the vertices of~ associated with the same colour.


We call {\sc Biclique Containment} the problem that decides whether there exists a
biclique of a graph  contained in a given subset of vertices of~.

\begin{prob}
	{
		Biclique Containment
	}
	{
		Graph~ and 
	}
	{
		Does there exist a biclique  of  such that ? 
	}
\label{prob:confinamentobicliquemaximal}
\end{prob}

In order to show that {\sc Biclique Containment} is -complete, 
we use in Theorem~\ref{thm:bicliquecoloracaoinvalidanpcompleto} a reduction
from {\sc 3SAT} problem.

\begin{theorem}
\label{thm:bicliquecoloracaoinvalidanpcompleto}
	The {\sc Biclique Containment} problem is -complete, even if
	the input graph is -free.
\end{theorem}

\begin{proof}
Deciding whether a graph has a biclique in a given
subset of vertices is in : a biclique is a
certificate and verifying this certificate is trivially polynomial.

We prove that {\sc Biclique Containment} problem is -hard by
reducing {\sc 3SAT} to it. The proof is outlined as follows. For every formula
, a graph  is constructed with a subset of vertices denoted by
, such that  is satisfiable if, and only if, there exists a
biclique  of  such that .

	Let  (resp. ) be the number of variables (resp. clauses) in formula
	. We define the graph  as follows.
	
	\begin{itemize}
		\item For each variable , , there exist two adjacent
		vertices  and~. Let .
		\item For each clause , , there exists a vertex 
		. Moreover, each , , is adjacent to a vertex 
		if, and only if, the literal corresponding to  is in the clause
		corresponding to vertex~. Let .
		\item There exists a universal vertex  adjacent to all ,
		, , and to all , .
		
	\end{itemize}
	
	We define the subset of vertices  as . Refer to 
	Figure~\ref{fig:cbmnpcompleto} for an example of such construction given a
	formula .

\begin{figure}[h]
\center
	\includegraphics[width=\textwidth]{cbmnpcompleto.pdf}
	\caption{Example for  }
	\label{fig:cbmnpcompleto}
\end{figure}
	
	We claim that formula  is satisfiable if, and only if, there exists a
	biclique of  that is also a biclique of~.
	
	Each biclique  of  containing vertex  corresponds to a
	choice of precisely one vertex of , for each , and so  corresponds to a truth assignment  that gives
	true value to variable  if, and only if, the corresponding vertex .
	
	Notice that we may assume three properties on the {\sc 3SAT} instance.
	
	\begin{itemize}
	  \item A variable and its negation do not appear in the same clause. Else,
	  any assignment of values (true or false) to such a variable satisfies the
	  clause.
	  \item A variable appears in at least one clause. Else,
	  any assignment of values (true or false) to such a variable is
	  indifferent to formula .
	  \item Two distinct clauses have at most one literal in common. 
	  Else, we can modify the instance as follows. For each clause 
	  , we replace it by clauses , 
	  , ,
	  and  with variables ,
	  , and . Clearly, the number of variables and clauses created is upper
	  bounded by 7 times the number of clauses in the original instance. 
	  Moreover, the original formula is satisfiable if, and only if, the new
	  formula is satisfiable. 
	\end{itemize}    
 
	We consider the bicliques of  according to two cases.
	\begin{enumerate}
		\item Biclique  does not contain vertex . Then, the biclique is
		precisely formed by a pair of vertices, say  and ,
		where . Now, our assumption says that there exists a 
		adjacent to one precise vertex in  which implies
		that  is not a biclique of .
		\item Biclique  contains vertex . Then, the biclique is
		precisely formed by vertex  and one vertex of , for each .  is a biclique of  if, and only if, for each
		, there exists a vertex  such that  is
		adjacent to , which in turn occurs if, and only if, the truth assignment
		 satisfies . Therefore,  is a biclique of  if, and only if,
		 satisfies .
	\end{enumerate}
	
	Now, we still have to prove that  is -free. 
	
	For the sake of contradiction, suppose that there exists a  in , say
	. There are no two distinct vertices of  in , since  is an independent set. There
	are no three distinct vertices of  in , since there is a non-edge
	between two of these three vertices. Hence,  precisely contains vertex ,
	one vertex of , and two vertices of . Since  is a complete set, the
	two vertices in  are adjacent and the vertex of  is adjacent
	to both vertices of . This contradicts our assumption that a
	variable and its negation do not appear in the same clause. 
	
	For the sake of contradiction, suppose
	there exists a  in , say~. The universal vertex 
	cannot belong to . Since  is an independent set,  contains at most
	two vertices of . Now, if  contains two vertices of , then the other
	two vertices of  must be two literals, which contradicts our assumption that
	two distinct clauses have at most one literal in common. Since~
	induces a matching,  is not contained in . Therefore,  contains
	one vertex of  and three vertices of , which by the construction of
	 gives the final contradiction.	
	\end{proof}

\begin{corollary}
\label{cor:checkbicliquecolouring}
Let  be a -free graph. It is co-complete to check
if a colouring of the vertices of  is a biclique-colouring.
\end{corollary}


\section{Powers of paths, powers of cycles, and their bicliques}
\label{sec:powerofcyclesandbicliques}
A \emph{power of a path} , for , is a simple
graph with  and  if,
and only if, . Note that  is the induced path  on
 vertices and , , is the complete graph  on 
vertices. 
In a power of a path , the \emph{reach} of an edge  is . A \emph{power of a cycle} , for , is a
simple graph with  and  if,
and only if, . Note that 
is the induced cycle  on  vertices and , ,
is the complete graph  on  vertices. 
In a power of a cycle , we take  to be a
\emph{cyclic order} on the vertex set of  and we always perform arithmetic modulo~ on
vertex indices. The \emph{reach} of an edge  is . The definition of reach is extended to an induced path to be
the sum of the reach of its edges. A \emph{block} is a maximal set of
consecutive vertices. The \emph{size} of a block is the number of vertices in
the block.

All power graphs considered in the present work contain a polynomial
number of bicliques, a sufficient condition for the {\sc Biclique Containment}
problem to be polynomial. In what follows, we explicitly identify the bicliques
of a power of a path and the bicliques of a power of a cycle. 
We say that a biclique of size~2 is a  biclique and that a biclique of size~3
is a  biclique. Notice that, for each value of  in the considered
range, every biclique in
Lemmas~\ref{lem:powerofpathsbicliques}~and~\ref{lem:powerofcyclesbicliques} always exists.
We refer to Figure~\ref{fig:l4} to illustrate the distinct biclique structures
for each considered case of non-complete powers of cycles.

\begin{figure}[t]
\centering
	\subfloat
		[Power of a cycle  \newline ()] {
			\includegraphics[scale=0.17]{c114-l4.pdf}
			\label{fig:c114-l4}
		}
	\qquad
	\subfloat
		[Power of a cycle  \newline ()] {
			\includegraphics[scale=0.17]{c113.pdf}
			\label{fig:c113-l4}
		}
	\qquad
	\subfloat
		[Power of a cycle  \newline ()] {
			\includegraphics[scale=0.17]{c112.pdf}
			\label{fig:c112-l4}
		}
	\caption{For each case of non-complete powers of
	cycles according to Lemma~\ref{lem:powerofcyclesbicliques}, we highlight in
	bold the distinct biclique structures.}
	\label{fig:l4}
\end{figure}
	
\begin{lemma}
\label{lem:powerofpathsbicliques}
 The bicliques of a power of a path  are precisely: 
  bicliques, if ;
  bicliques and  bicliques, if ; and
  bicliques if .
\end{lemma}
\begin{proof}
A power of a path is -free and -free. Thus, the bicliques of a
power of a path are possibly  or  bicliques. 

Let  be a power of a path with . Since , every pair of vertices is a  biclique.

Let  be a power of a path with . Since  and , the edge  exists
and both vertices  and  are adjacent to every other vertex of
. This implies that they define a  biclique. Clearly, vertices
, , and  are distinct and define a  biclique.

Now, let  be a power of a path with . We claim that always exists only  biclique. Let  and
 be two adjacent vertices in , such that . If ,  induce a , since  is not adjacent
to . Otherwise  and 
induce a , since  is not adjacent to . We
conclude that every  is contained in a , and so every biclique in
 is a  biclique.
\end{proof}

\begin{lemma}
\label{lem:powerofcyclesbicliques}
 The bicliques of a power of a cycle  are precisely:
   bicliques, if ;
   bicliques, if ;
   bicliques and  bicliques, if ; and
   bicliques, if .
\end{lemma}
\begin{proof}
A power of a cycle is -free. Thus, the bicliques of a power of a cycle
are possibly ,  or  bicliques. Let  be a
power of a cycle with . Since , every pair of
vertices is a  biclique. Otherwise, , and every  is
properly contained in a , as we explain next. Let  and  be two
adjacent vertices in  such that  (indices are taken modulo ). Let  be
the last consecutive vertex after  adjacent to  along the cyclic
order. It follows that  is not adjacent to  but  is adjacent to  and that vertices , , and  define a
. Thus, in what follows, each biclique is possibly  or 
biclique.

Let  be a power of a cycle  with . 
Since , the subset of vertices  is a  biclique. Hence,  has a 
biclique.

Let  be a power of a cycle  with . 
Suppose  is a~. If the missing edge is
, then, by symmetry, we may assume .
Since , vertices  and  have no common
neighbor with index at most  and at least . Hence,  does not
have a  biclique.

Let  be a power of a cycle  with .
Suppose  is a .
If the missing edge is , then, by symmetry, we may
assume . Since ,
vertices  and  have a common neighbor with index at most
 and at least  which is not a neighbor of
. We conclude that every  is contained in a  and 
therefore  contains only  biclique.

Now, let  be a power of a cycle  with . Consider
the  induced by vertices , , and . Since , vertices  and  have no common neighbor with index at least
. Hence,  has a  biclique.
\end{proof}

\section{Determining the biclique-chromatic number of~}
\label{sec:kappabpowerofpath}

The extreme cases are easy to compute: the densest case occurs when
, which implies that a power of a path  is the complete
graph  whose biclique-chromatic number is its order , whereas for the non-complete case, the
sparsest case  occurs when , which implies that a power of a path
 is the chordless path  whose biclique-chromatic number is 2.
According to Lemma~\ref{lem:powerofpathsbicliques}, we consider other two cases:
the less dense case , and the sparse case . The proof of
Theorem~\ref{thm:kappabpowerofpathfirstinterval} (resp.
Theorem~\ref{thm:kappabpowerofpathsecondinterval})
additionally yields an efficient -biclique-colouring (resp.
2-biclique-colouring) algorithm for the less dense case (resp. for the sparse
case).


\begin{theorem}
\label{thm:kappabpowerofpathfirstinterval}
 A power of a path , when , has
 biclique-chromatic number~.
\end{theorem}

\begin{proof}
Let  be a power of a path  with .
Each of the vertices  is universal and any pair of
vertices in  induces a  biclique in the
graph. Hence, we are forced to give distinct colours to each of the vertices
 and we have .
 
We define  by giving
(arbitrarily) distinct colours  to vertices . Now, use colour  in the uncoloured vertices before  and colour  in the uncoloured vertices after . 
Every monochromatic edge contains either both end vertices before  or
both end vertices after . By symmetry, consider  a
monochromatic edge such that . Now, vertices  induce a  biclique. Since any choice of three vertices either before
 or after  defines a triangle,  is a
biclique-colouring of .

We refer to Figure~\ref{fig:path1} to illustrate the
given -biclique-colouring.
\end{proof}

\begin{theorem}
\label{thm:kappabpowerofpathsecondinterval}
 A power of a path , when , has
 biclique-chromatic number~2.
\end{theorem}

\begin{proof}
Let  be a power of a path  with . 
Let , with . We define
 as follows. A number of 
monochromatic-blocks of size  switching colours \emph{red} and \emph{blue}
alternately, followed by a monochromatic-block of size  with \emph{red}
colour if  is even or \emph{blue} colour if  is odd.
We refer to Figure~\ref{fig:path2} to illustrate the given
2-biclique-colouring.

Lemma~\ref{lem:powerofpathsbicliques} says that every biclique of  is a
. Thus, every biclique is polychromatic, since it contains vertices from
two consecutive monochromatic-blocks (with distinct colours by the
given colouring).
\end{proof}

\begin{figure}[t]
\centering
	\subfloat
		[-biclique-colouring, when .] {
			\includegraphics[scale=0.34]{path1.pdf}
			\label{fig:path1}
		}
	\qquad
	\subfloat
		[-biclique-colouring, when  and .] {
			\includegraphics[scale=0.34]{path2.pdf}
			\label{fig:path2}
		}
	\caption{Biclique-colouring of
	powers of paths}
	\label{fig:pathbicliquecolouring}
\end{figure}


\section{Determining the biclique-chromatic number of~}
\label{sec:bicliqueupperboundpowerofcycle}

The extreme cases are easy to compute: the densest case occurs when
, which implies that a power of a cycle  is the complete
graph  whose biclique-chromatic number is its order , whereas for the
non-complete case, the sparsest case  occurs when , which implies
that a power of a cycle  is the chordless cycle  whose
biclique-chromatic number is 2.
According to Lemma~\ref{lem:powerofcyclesbicliques}, we consider other two
cases:
the less dense case , whose biclique-chromatic number is
2, and the sparse case .

The division algorithm says that any
natural number  can be expressed using the equation , with a
requirement that . We shall use the following version where  is
even and .


\begin{theorem}[Division algorithm]
\label{thm:division}
Given two natural numbers  and ,
with , there exist unique natural numbers  and  such that
,  is even, and .
\end{theorem}


Given a non-complete power of a cycle, Lemma~\ref{lem:3colouringnomonoP3} shows
that there exists a 3-colouring of its vertices such that no  is
monochromatic. Since every biclique contains a , 
Lemma~\ref{lem:3colouringnomonoP3} provides an upper bound of~3 for the 
biclique-chromatic number of a power of a cycle --- the proof of 
Lemma~\ref{lem:3colouringnomonoP3} additionally yields an efficient 
3-biclique-colouring algorithm using the version of the division 
algorithm stated in Theorem~\ref{thm:division}. 
Moreover, this upper bound of~3 to the biclique-chromatic number 
is tight. Please refer to Figure~\ref{fig:c113} for an example of a graph 
not 2-biclique-colourable.

\begin{lemma}
\label{lem:3colouringnomonoP3}
Let  be a power of a cycle , where . Then,  admits a
3-colouring of its vertices such that  has \textbf{no} monochromatic .
\end{lemma}
\begin{proof}
Let  be a power of a cycle  with . 
Theorem~\ref{thm:division} says that 
for natural numbers  and ,  is even, and . If , we define  as follows. An even number
 of monochromatic-blocks of size  switching colours \emph{red} and
\emph{blue} alternately, followed by a monochromatic-block of size  with
colour \emph{green}. Otherwise, i.e. , we define
 as follows. An odd number  of
monochromatic-blocks of size  switching colours \emph{red} and \emph{blue}
alternately, followed by a monochromatic-block of size  with colour
\emph{green}, a monochromatic-block of size  with colour \emph{blue}, and a
monochromatic-block of size  with colour \emph{green}.  We refer to
Figure~\ref{fig:restinho} to illustrate the former 3-biclique-colouring and to
Figure~\ref{fig:restao} to illustrate the latter 3-biclique-colouring.

Consider any three vertices ,  and  with the
same colour. Then, either they are in the same monochromatic-block --- and
induce a triangle --- or two of them are not in consecutive
monochromatic-blocks -- and induce a disconnected graph. In both cases, ,
 and  do not induce a . 
\end{proof}

\begin{theorem}
\label{thm:bicliqueupperboundpowerofcycle}
A power of a cycle , when ,
has biclique-chromatic number at most~3.
\end{theorem}


\begin{figure}[t]
\centering
	\includegraphics[scale=0.2]{c113.pdf}
	\caption{Power of a cycle  with biclique-chromatic number~3.
	We highlight in bold a  biclique of reach  and a  biclique.}
	\label{fig:c113}
\end{figure}

\begin{figure}[t]
\centering
	\subfloat
		[3-biclique-colouring, when  and .] {
			\includegraphics[scale=0.2]{restinho.pdf}
			\label{fig:restinho}
		}
	\qquad
	\subfloat
		[3-biclique-colouring, when  and .] {
			\includegraphics[scale=0.2]{restao.pdf}
			\label{fig:restao}
		}
	\qquad
		\subfloat
		[2-biclique-colouring
		when ] 
		{
			\includegraphics[scale=0.2]{excessao.pdf}
			\label{fig:excessao}
		}
	\qquad
		\subfloat
		[2-biclique-colouring of a 2-biclique-colourable graph,
		when ] 
		{
			\includegraphics[scale=0.2]{semresto.pdf}
			\label{fig:semresto}
		}
	\caption{Biclique-colouring of
	powers of cycles}
	\label{fig:2nd3bicliquecolouring}
\end{figure}

As a consequence of Theorem~\ref{thm:bicliqueupperboundpowerofcycle}, every
non-complete power of a cycle has biclique-chromatic number~2 or~3, and it is a
natural question how to decide between the two values.
We first settle this question in the less dense case .
In fact, we show that all powers of cycles in the less dense case  are 2-biclique-colourable --- the proof of
Theorem~\ref{thm:kappabpowerofcyclesecondinterval} additionally yields an
efficient 2-biclique-colouring algorithm.

\begin{theorem}
\label{thm:kappabpowerofcyclesecondinterval}
 A power of a cycle , when , has
 biclique-chromatic number~2.
\end{theorem}

\begin{proof}
Let  be a power of a cycle  with . We
define  as follows. A monochromatic-block of
size  with colour \emph{red} followed by a monochromatic-block of size 
with colour \emph{blue}. We refer to Figure~\ref{fig:excessao} to illustrate the
given 2-biclique-colouring.

Recall that every biclique of  is a  biclique. For the sake of
contradiction, suppose that there exists a monochromatic set  of four
vertices. If  is contained in the block of size , then  induces a
 and cannot be a . Otherwise,  is contained in the block of size
 and there exists a subset of  which induces a triangle,
so that  cannot be a  biclique.
\end{proof}


The sparse case  is more tricky. Let  be a power of a cycle
 with . Following
Lemma~\ref{lem:powerofcyclesbicliques}, there always exists a  biclique in
. Clearly, a biclique-colouring of  has every  biclique
polychromatic, but we may think that there exists some monochromatic  (not
biclique). Nevertheless, we prove that  has biclique-chromatic number~2 if,
and only if, there exists a 2-colouring of  such that \textbf{no}  is
monochromatic, which happens exactly when there exists a 2-colouring of 
where every monochromatic-block has size  or .

\begin{figure}[t]
\centering
	\subfloat
		[vertices , , and  induce a monochromatic
		 with reach ] {
			\includegraphics[scale=0.3]{catarina1.pdf}
			\label{fig:catarina1v2}
		}
	\qquad
	\subfloat
		[vertices , , and  induce a monochromatic 
		with reach ]
		{
			\includegraphics[scale=0.3]{catarina2.pdf}
			\label{fig:catarina2v2}
		}
	\qquad
	\subfloat
		[vertices , , and  induce a monochromatic
		 with reach ]
		{
			\includegraphics[scale=0.3]{catarina3.pdf}
			\label{fig:catarina3v2}
		}
	\caption{A monochromatic-block of size  in a power of a cycle
	, with , implies a monochromatic  with reach
	 or .}
	\label{fig:catarinav2}
\end{figure}

\begin{lemma}
\label{l:iff}
Let  be a power of a cycle , where , and consider a
2-colouring of its vertices. If every monochromatic-block has size  or
, then  has \textbf{no} monochromatic . Otherwise, if
\textbf{not} every monochromatic-block has size  or , then  has
a monochromatic  with reach  or ; in particular, when
,  has a monochromatic~ with reach  or
 has a monochromatic .
\end{lemma}

\begin{proof}
Let  be a power of a cycle  with .
Consider a 2-colouring  of the vertices of  such that every
monochromatic-block has size  or . 

Consider any three vertices ,  and  with the
same colour. Then, either they are in the same monochromatic-block --- and
induce a triangle --- or two of them have indices that differ by at least 
with respect to the third vertex --- and the three vertices induce a
disconnected graph. In both cases, ,  and  do not induce a
. Hence, no  is monochromatic.

Now, consider a 2-colouring  of the vertices of  such that there exists
a monochromatic-block of size .
Consider a monochromatic-block of size  with vertices
, , , , , , and
.
Notice that vertices , , and  induce a~. So, we
may assume that there exists a monochromatic-block with vertices
, , , , , , ,
where . By symmetry, consider that  has blue colour. Notice that
vertices  and  are adjacent and with red colour. Please
refer to Figure~\ref{fig:catarinav2}. Suppose that vertex  has
red colour. Then, vertices , , and  induce a monochromatic
 with reach  (see Figure~\ref{fig:catarina1v2}).
Now, consider vertex  has blue colour. Suppose that vertex 
has blue colour, then vertices , , and  induce a 
monochromatic  with reach  (see Figure~\ref{fig:catarina2v2}). Now,
consider vertex  has red colour and vertices ,
, and  induce a monochromatic  with reach  (see
Figure~\ref{fig:catarina3v2}).

Now, consider the case . We know that  has a monochromatic 
of reach  or . In the first case, we are done, so we assume that 
has a monochromatic  , , and  of red colour.
Moreover, vertex  (resp. vertex ) has blue colour, otherwise
vertices , , and  (resp. vertices ,
, and ) would induce a monochromatic  with reach
. Vertices , , , and  induce the
unique  that includes vertices , , and .
Please refer to Figure~\ref{fig:catarinav2.2}.
Suppose vertex  has red colour, then vertices ,
, , and  induce a monochromatic  (see
Figure~\ref{fig:catarina4v2}).
Now, consider vertex  has blue colour. Suppose that vertex
 (resp.) has blue colour, then vertices ,
, and  (resp. , , and
) induce a monochromatic  with reach  (see
Figure~\ref{fig:catarina5v2}). Now, consider vertices  and
 have red colour. Vertices , , and
 induce a monochromatic  with reach  (see
Figure~\ref{fig:catarina6v2}).
\end{proof}

\begin{figure}[t]
\centering
	\subfloat
		[vertices , , , and  induce a
		monochromatic ] {
			\includegraphics[scale=0.3]{catarina4.pdf}
			\label{fig:catarina4v2}
		}
	\qquad
	\subfloat
		[vertices , , and  (resp. ,
		, and ) induce a monochromatic  with reach
		] {
			\includegraphics[scale=0.3]{catarina5.pdf}
			\label{fig:catarina5v2}
		}
	\qquad
	\subfloat
		[vertices , , and  induce a monochromatic
		 with reach ]
		{
			\includegraphics[scale=0.3]{catarina6.pdf}
			\label{fig:catarina6v2}
		}
	\caption{A monochromatic-block of size  in a power of a cycle
	, with , implies a monochromatic  with reach
	 or a monochromatic .}
	\label{fig:catarinav2.2}
\end{figure}


\begin{theorem}
\label{thm:kappabpowerofcyclethirdinterval}
A power of a cycle , when , has biclique-chromatic
number~2 if, and only if, there exist natural numbers  and , such that  and  is even.
\end{theorem}

\begin{proof}
Let  be a power of a cycle  with .

First, consider natural numbers  and , such that  and  is even. Then, there exists a 2-colouring  such that every
monochromatic-block has size  or . Lemma~\ref{l:iff} says that  has 
\textbf{no} monochromatic  and therefore  is a 2-biclique-colouring.
We refer to Figure~\ref{fig:semresto} to illustrate such 2-biclique-colouring.

For the converse, suppose that there are no such  and , which implies
that any 2-colouring  of the vertices of  is such that there
exists a monochromatic-block of size . Consider . Lemma~\ref{l:iff} says that such 2-colouring of the vertices of  has
a monochromatic~ with reach  or a monochromatic~. Every
 with reach  is a biclique and every  is a biclique, which
implies that  is not a 2-biclique-colouring, which is a contradiction.
Now, consider . Lemma~\ref{l:iff} says that such 2-colouring of
the vertices of  has a monochromatic~ with reach  or . Every
 with reach  or  is a  biclique, which implies that
 is not a 2-biclique-colouring, which is a contradiction.
\end{proof}

 There exists an efficient algorithm that verifies if the system of equations of
 Theorem~\ref{thm:kappabpowerofcyclethirdinterval} has a solution. If so,
 it also computes values of  and  -- the proof of
 Theorem~\ref{thm:algoritmopowerofcycle} yields
 Algorithm~\ref{alg:numerobicliquecromaticopowerofcycle} to determine if the
 biclique-chromatic number is 2 or 3 and also computes values of  and .
 When the biclique-chromatic number is 2, we define a 2-biclique-colouring
  as follows. A number  of monochromatic-blocks
 of size  plus a number  of monochromatic-blocks of size  switching
 colours \emph{red} and \emph{blue} alternately. We refer to
 Figure~\ref{fig:semresto} to illustrate the given 2-biclique-colouring. 


\begin{theorem}
\label{thm:algoritmopowerofcycle}
 There exists an algorithm that computes the biclique-chromatic
 number of a power of a cycle , when .
\end{theorem}

\begin{proof}

Theorem~\ref{thm:bicliqueupperboundpowerofcycle} states that the
biclique-chromatic number of a power of a cycle  is at most~3 and
Theorem~\ref{thm:kappabpowerofcyclethirdinterval} states that a power of a cycle
 with  has biclique-chromatic number~2 if, and only if,
there exist natural numbers  and , such that  and  is even.

Let . We show that there exist natural numbers  and , such that
, , and  is even if, and only if, natural numbers  and  have the
following properties:  is even and ; or 
natural numbers  and  have the following properties:  is even and . 

Clearly,  and  (resp.  and ) are natural numbers such that
 (resp. ),  (resp. ),  (resp. ) is even, and  (resp. ) since .

For the converse, suppose that there exist natural numbers  and , such
that  and  is even. Let  and . While , do  and .
Clearly, in the end of the loop, we have  even, , and
. Moreover, we consider two cases.

\begin{itemize}
  \item  in the end of the loop. Then,  and .
  \item  in the end of the loop. Then,  and .
\end{itemize}
\end{proof}

As a remark, in Theorem~\ref{thm:algoritmopowerofcycle},
we let  and rewrite the equation  as ,
very similar to the Division Algorithm formula. Nevertheless, there is a rather
subtle difference: in the Division Algorithm formula, the choice for the
value of the remainder is bounded by the value of the divisor, while in the
equation , the choice for the value of the remainder is bounded by
the choice for the value of the quotient (recall ). This subtle
difference may change drastically the behavior of the equation. More precisely,
given two natural numbers  and , with , it is not
necessarily true that there exist natural numbers  and  such that ,  is even, and . For instance, there do not exist
natural numbers  and  such that ,  is even, and .


	\begin{algorithm}[t]
	\label{alg:numerobicliquecromaticopowerofcycle}
	\SetKwInOut{Input}{input}\SetKwInOut{Output}{output}
	\Input{, a power of a cycle with }
	\Output{, the biclique-chromatic number of .}
	\caption{To compute the biclique-chromatic number of a power of a cycle 
	with }
	\BlankLine
	\Begin
	{	
		\;
		\;
		\eIf{ and }
		{
				\Return{\;}
		}
		{
			\;
			\;
			\eIf{ and }
			{
					\Return{\;}
			}
			{
				\Return{\;}
			}
		}
	}
	\end{algorithm}

\section{Final considerations}
\label{sec:final}

The reader should notice the structure differences between the two
considered classes of power graphs and observe the similarities on giving lower
and upper bounds on the biclique-chromatic number. For instance,
the lower bound on the biclique-chromatic number in both cases when  
is a consequence of the existence of a set of  bicliques whose union
induces a complete graph --- in the case of powers of cycles, this can happen
only when such union is the whole vertex set, but in the case of powers of
paths such union can be the whole vertex set (when ) or a
vertex subset of size  (when ). When , monochromatic-blocks are the key step to construct optimal colourings. 
Nevertheless, in the given colourings, for powers of
paths, vertices  and  may have the same colour, which is not the
case for powers of cycles.

Table~\ref{t:tabela} highlights the exact values for the biclique-chromatic
number of the power graphs settled in this work. In
Figures~\ref{fig:kappaboscilapath}~and~\ref{fig:kappaboscilacycle}, we illustrate
the biclique-chromatic number for a fixed value of  and an increasing  of
powers of paths and powers of cycles, respectively.

\begin{figure}[t]
\centering
	\includegraphics[width=7cm]{kappaboscilapath.pdf}
	\caption{The biclique-chromatic number of a non-complete power of a path for a fixed value of
	 and an increasing }
	\label{fig:kappaboscilapath}
\end{figure}

\begin{figure}[t]
\centering
	\includegraphics[width=10cm]{kappaboscila.pdf}
	\caption{The biclique-chromatic number of a non-complete power of a cycle for a fixed value
	of  and an increasing }
	\label{fig:kappaboscilacycle}
\end{figure}

As a corollary of Theorem~\ref{thm:kappabpowerofcyclethirdinterval}, every non-complete
power of a cycle  with  has biclique-chromatic number~2.
Thus, the biclique-chromatic number of a power of a cycle , for a fixed
value of  and an increasing , does not oscillate forever.

\begin{corollary}
A non-complete power of a cycle  with  has biclique-chromatic
number~2. 
\end{corollary}
\begin{proof}
Theorem~\ref{thm:division} says that  for natural numbers
 and ,  is even, and . If we can rewrite  with natural numbers  and , such that  is
even, then Theorem~\ref{thm:kappabpowerofcyclethirdinterval} says that a power of a cycle
 with  has biclique-chromatic number~2. Since , , and  is an even natural number, we have 



Let  and . Clearly,  and  are natural numbers. 
Moreover,  is even.
\end{proof}

Groshaus, Soulignac, and Terlisky have recently proposed a related hypergraph
colouring, called \emph{star-colouring}~\cite{1210.7269}, defined as
follows. A \emph{star} is a maximal set of vertices that induces a
complete bipartite graph with a universal vertex and at least one edge. 
The definition of star-colouring follows the same line as clique-colouring and
biclique-colouring: a \emph{star-colouring} of a graph  is a function that
associates a colour to each vertex such that no star is monochromatic. The 
\emph{star-chromatic number} of a graph , denoted by , is the 
least number of colours  for which  has a star-colouring with at most~ 
colours. Many of the results of biclique-colouring
achieved in the present work are naturally extended to star-colouring. Since the
constructed graph of Corollary~\ref{cor:checkbicliquecolouring} is -free
and the bicliques in a -free graph are precisely the stars of the graph,
we can restate Corollary~\ref{cor:checkbicliquecolouring} as follows below.

\begin{corollary}
Let  be a -free graph. It is co-complete to
check if a colouring of the vertices of  is a star-colouring.
\end{corollary}

About star-colouring and the investigated classes of power graphs,
we also have some few remarks. On one hand, the bicliques of a power of a path  are
the stars of the graph and, consequently, all results obtained for
biclique-colouring powers of paths hold to star-colouring powers of paths.
On the other hand, a power of a cycle  is not necessarily -free, and
there are examples of powers of cycles with  stars that are not bicliques
due to the fact that such  stars are contained in  bicliques of the
graph. This happens for instance in the case  and one such
example is graph  exhibited in Figure~\ref{fig:c114}. Notice that the highlighted
vertices form a monochromatic  star, so that the colouring is not a
2-star-colouring. The three highlighted vertices together with vertex , on
the other hand, form a polychromatic  biclique --- indeed, the
exhibited colouring is a 2-biclique-colouring. 
We summarize the results about star-colouring powers of paths and powers of
cycles in the following theorems and also in Table~\ref{t:tabela}. Please refer
to the line of the table where we consider a power of a cycle with  to check the difference between the biclique-chromatic number
(which is always 2) and the star-chromatic number (which depends on  and
).

\begin{figure}[t]
\centering
	\includegraphics[scale=0.2]{c114.pdf}
	\caption{Power of a cycle  with a
	2-biclique-colouring which is not a 2-star-colouring. 
	Notice that there exists a monochromatic  star
	highlighted in bold.}
	\label{fig:c114}
\end{figure}

\begin{theorem}
For any power of a path, the star-chromatic number is equal to the
biclique-chromatic number.
\end{theorem}


\begin{theorem}
A power of a cycle , when  or , has
star-chromatic number equal to the biclique-chromatic number.
If , then  has star-chromatic number~2 if, and
only if, there exist natural numbers  and , such that  and
 is even. If there does not exist such natural numbers, it has
star-chromatic number~3.
\end{theorem}

\begin{table}[h]
\begin{center}
\begin{tabular}{|c||l|p{3cm}|p{3cm}|}
\hline
  Graph  & Range of  &  &  \\ \hline\hline
  \multirow{3}{*}{} &  &  &  \\
  \cline{2-4}
   &  &  & \\
  \cline{2-4}
   &  &  &  \\
  \hline\hline
  \multirow{5}{*}{} &  &  &  \\
  \cline{2-4}
   &  &  &\\
  \cline{2-3}
   & \multirow{4}{*}{} & \multicolumn{2}{l|}{, if there
   exist natural numbers  and ,} \\
   & & \multicolumn{2}{l|}{such that } \\
   & & \multicolumn{2}{l|}{and  is
   even;} \\
  & & \multicolumn{2}{l|}{, otherwise.} \\
  \cline{2-4}
   &  &  & \\
  \hline
\end{tabular}
\caption{Biclique- and star-chromatic numbers of powers of paths and powers of cycles}
\label{t:tabela}
\end{center}
\end{table}

A \emph{distance graph}  is a simple graph with
 and , such
that  if, and only if, it has reach -- in the context
of a power of a path -- . Notice that a distance graph
 is a power of a path if , , and
. A \emph{circulant graph}  has the same
definition as the distance graph, except by the reach, which, in turn, is in
the context of a power of a cycle. Notice that a circulant
graph~ is a power of a cycle if , ,
and .
Circulant graphs have been proposed for various practical
applications~\cite{circulantgraphapplication}.
We suggest, as a future work, to biclique colour the classes of distance
graphs and circulant graphs, since colouring problems for distance graphs and
for circulant graphs have been extensively
investigated~\cite{MR2567972,MR1900685,MR1632015}.
Moreover, some results of intractability have been obtained, e.g. determining
the chromatic number of circulant graphs in general is an -hard
problem~\cite{MR1653503}.


\section*{Acknowledgments}
The authors would like to thank Renan Henrique Finder for the discussions on
the algorithm to compute the biclique-chromatic number of a power of a cycle
, when ; and to thank Vin{\'i}cius Gusm{\~a}o Pereira de
S{\'a} and Guilherme Dias da Fonseca for discussions on the complexity of
numerical problems. At last, but not least, we thank Vanessa Cavalcante for
the careful proofreading of this paper.


\bibliographystyle{plainnat}

\bibliography{ctw-2012-full}

\end{document}