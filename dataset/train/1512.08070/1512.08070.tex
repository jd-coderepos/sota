\documentclass[oneeqnum,final]{siamltex1213}

\usepackage{subcaption}
\usepackage[format=hang]{caption}
\usepackage{amsmath}
\usepackage{amsfonts}

\usepackage[utf8]{inputenc} 

\newcommand{\Real}{{\mathbb R}}

\newcommand{\TwoEC}{\emph{2EC}}
\newcommand{\TwoECLP}{\ensuremath{\text{\TwoEC}^{\text{LP}}}}
\newcommand{\alphaTwoEC}{\ensuremath{\alpha\text{\TwoEC}}}

\newcommand{\Kn}{\ensuremath{K_{n}}}
\newcommand{\Gx}{\ensuremath{G_{x^{*}}}}

\newtheorem{conjecture}{Conjecture}
\newtheorem{case}{Case}
\newtheorem{subcase}{Case}


\title{Toward a 6/5 Bound for the Minimum Cost 2-Edge Connected Spanning Subgraph Problem \thanks{This research was partially supported by grants from the Natural Sciences and \mbox{Engineering} Research Council of Canada}}

\author{Sylvia Boyd \thanks{Email: \href{mailto:sylvia@site.uottawa.ca}   
   {\texttt{sylvia@site.uottawa.ca}}}
\and Philippe Legault \thanks{Email: \href{mailto:philippe@legault.cc}  
   {\texttt{philippe@legault.cc}}}}

\begin{document}

\maketitle

\begin{abstract}
Given a complete graph  with non-negative edge costs , the problem \TwoEC{} is that of finding a 2-edge connected spanning multi-subgraph of \Kn{} of minimum cost. The integrality gap \alphaTwoEC{} of the linear programming relaxation \TwoECLP{} for \TwoEC{} has been conjectured to be , although currently we only know that . In this paper, we explore the idea of using the structure of solutions for \TwoECLP{} and the concept of convex combination to obtain improved bounds for \alphaTwoEC. We focus our efforts on a family  of half-integer solutions that appear to give the largest integrality gap for \TwoECLP. We successfully show that the conjecture  is true for any cost functions optimized by some .
\end{abstract}

\begin{keywords}
minimum cost 2-edge connected subgraph problem, approximation algorithm, integrality gap.
\end{keywords}

\section{Introduction}
The \emph{2-edge connected subgraph problem} (\TwoEC{}) is that of finding a minimum cost 2-edge connected spanning multi-subgraph of the complete graph \sloppy{} with costs . This problem has many important applications in network design. It is known to be NP-hard even for very special cases\,\cite{csaba}. Currently, a -approximation algorithm is known for \TwoEC{}. This follows from the fact that for any instance of \TwoEC{}, we can assume, WLOG, that the costs are metric and the solutions do not include multi-edges\,\cite{alexander}, in which case we can apply the -approximation due to Frederickson and Ja'Ja'\,\cite{frederickson}. For \TwoEC{} where multi-edges are not allowed, a 2-approximation is known \cite{jain}.

For , letting  represent the number of copies of  in the \TwoEC{} solution, \TwoEC{} can be formulated as an integer linear program (ILP) as follows, i.e.:
\noindent The linear programming (LP) relaxation of \TwoEC{}, denoted by \TwoECLP, is obtained by relaxing the integer requirement in (\ref{ilpTwoEC}). We use OPT(\TwoEC{}) (resp. OPT(\TwoECLP)) to denote the optimal value of \TwoEC{} (resp. \TwoECLP). Also, given any feasible solution  for \TwoECLP, its \emph{support graph} \Gx{} is defined to be the subgraph of \Kn{} obtained by taking all edges  for which .

We are interested in the \emph{integrality gap} \alphaTwoEC{} for \TwoECLP, which is the worst case ratio between OPT(\TwoEC) and OPT(\TwoECLP), i.e.  This gives a measure of the quality of the lower bound provided by \TwoECLP. Moreover, a polynomial-time constructive proof of  would provide a -approximation algorithm for \alphaTwoEC.

Even though \TwoEC{} has been intensively studied, little is known about \alphaTwoEC, except that \,\cite{alexander} in general, and  for the unweighted form of the problem in which one is given a graph and all edge costs are 1 (see\,\cite{boyd} and \cite{sebo}). In\,\cite{carr}, Carr and Ravi study \alphaTwoEC{}, and conjecture that , however no examples are known for which the integrality gap ratio comes close to . In\,\cite{alexander}, Alexander, Boyd and Elliott-Magwood also study \alphaTwoEC{} and make the following stronger conjecture based on their findings:

\begin{conjecture}\cite{alexander}\label{conjecture}
The integrality gap \alphaTwoEC{} for \TwoECLP{} is .
\end{conjecture}

To investigate \alphaTwoEC{} further, a natural next step is to study  for some interesting class of cost functions. We investigate \alphaTwoEC{} for the set of cost functions optimized at a particular family of feasible solutions for \TwoECLP. A feasible solution  for \TwoECLP{} is called a \emph{half-integer solution} if  for all , and it is called \emph{degree-tight} if  for all . Finally, a degree-tight half-integer solution is called a \emph{half-triangle solution} if the edges in the support graph \Gx{} corresponding to  (called \emph{half-edges}) form disjoint 3-cycles (called \emph{half-triangles}) joined by paths of edges of value 1 (called \emph{\mbox{1-paths}}).

The half-triangle solutions are of interest for studies of \alphaTwoEC{} as there is \mbox{evidence} that  is greatest for cost functions optimized at such solutions (see\,\cite{alexander},\,\cite{carr}). For example, the largest such ratio known is asymptotically \,\cite{alexander}, and comes from the infinite family of \TwoEC{} problems shown in Figure \ref{65WorseCaseExample}, where the numbers shown are the edge costs, edges  not shown have cost equal to the minimum cost  path, and the ``gadget'' pattern is repeated  times. This family is optimized for \TwoECLP{} by the half-triangle solution  shown in Figure \ref{65WorseCaseExample2}. Also, in a computational study which found \alphaTwoEC{} exactly for all \Kn{} up to  and all half-integer solutions up to , \alphaTwoEC{} was given by a half-triangle solution for all values of \,\cite{alexander}.

\begin{figure}
	\centering
	\begin{subfigure}[t]{0.45\textwidth}
		\vskip 0pt
		\centering
		\includegraphics{65Worse.ps}
		\caption{The edge costs.}\label{65WorseCaseExample}
	\end{subfigure}
	\hfill
	\begin{subfigure}[t]{0.45\textwidth}
		\vskip 0pt
		\centering
		\includegraphics{65Worse2.ps}
\caption{The half-triangle optimal solution .}\label{65WorseCaseExample2}
	\end{subfigure}
	\caption{An example for which \cite{alexander}.}
\end{figure}

The main result of this paper is to show that Conjecture\,\ref{conjecture} is true for any cost function optimized at half-triangles solutions. More specifically, we show that for any half-triangle solution  and any cost function , there exists a solution of \TwoEC{} of cost at most , which implies that  for any cost function optimized at half-triangle solutions. Note that previously,  was known, as Carr and Ravi\,\cite{carr} showed that for any degree-tight half-integer solution  and any cost function , there exists a solution of \TwoEC{} of cost at most .

A key idea used in our methods is that of convex combination. In the context of this paper, given a graph , we say that a vector  is a \emph{convex combination} if there exist 2-edge connected spanning multi-subgraphs  with multipliers  such that  and . Here  is the \emph{incidence vector} of subgraph  (i.e.  is the number of copies of edge  in ). Our method is essentially an averaging argument, and can be described as follows: let  be any feasible solution of \TwoECLP, and suppose we can show that  is greater than or equal to a convex combination for some value  (in particular ). Then for any non-negative cost vector  we have . This implies that for at least one of the , . If  is optimized at  for \TwoECLP, we then have , , and thus  for .

\section{Main Result}

Given a graph , we sometimes use  to denote , and  to denote . A graph  is called \emph{cubic} if every vertex of  has degree three. A \emph{cut} in  is a set of edges whose removal disconnects  into two components, sometimes referred to as the \emph{shores} of the cut. We call a cut \emph{proper} if both shores have cardinality at least two. Given a vector  that is a convex combination, the \emph{occurrence} of an edge  in that convex combination is . We sometimes refer to the occurrence of a pattern \emph{A} of edges in a convex combination, in which case we mean , and we use the notation  to denote it.

In this section, we prove our main result which is that  can be expressed as a convex combination for any half-triangle solution . We do this by first considering the cubic graph we get by shrinking all half-triangles to pseudo-vertices and replacing all 1-paths by singles edges. We obtain a convex combination result for this cubic graph, then show how we can use this result and certain patterns for the half-triangle edges to obtain the result that  is a convex combination.

\begin{definition}
 Given a cubic 3-edge connected graph , the vector  defined by , for all , is a convex combination in which none of the 2-edge connected spanning subgraphs use more than one copy of any edge in .
\end{definition}

\begin{lemma}\label{thmThreeSeventh}
 holds for all cubic 3-edge connected graphs  with .
\end{lemma}

\begin{proof}
Suppose the contrary, and let  be the smallest counter-example for which  does not hold. Since  can be shown to be true directly for the unique graph  with  (see Figure \ref{BaseCase}, where bold lines indicate edges in  and dotted lines indicate edges omitted), we can assume .

\begin{figure}
\begin{center}
\includegraphics{BaseCase.ps}
\end{center}
\caption{Proof of Lemma \ref{thmThreeSeventh} for , when .}
\label{BaseCase}
\end{figure}
\vspace{1em}
\begin{case} \label{LemmaBaseCase}
 has no proper 3-edge cut.
\end{case}
For any edge , let the unlabeled adjacent vertices at  be  and , and the unlabeled adjacent vertices at  be  and . Since  is 3-edge connected, has no proper 3-edge cut and , it follows that , ,  and  are all distinct. This situation is illustrated on the left of Figure\,\ref{NoNonTrivial3EC}, where some incident edges are not shown for vertices , ,  and . Removing  and  and their incident edges, and adding edges  and  yield a new cubic 3-edge connected graph  with fewer vertices than . Therefore,  holds, so there exists a set of \mbox{2-edge} connected spanning subgraphs  with multipliers ,  such that  and  occur  times overall in the convex combination. There are four patterns possible depending on the absence of  and  in . These are indicated as patterns \emph{A}, \emph{B}, \emph{C} and \emph{D} in Figure\,\ref{PatternsToMissingEdge}, where an edge marked in bold indicates an edge which is in , and a dotted edge indicates an edge which is not in . For each pattern , we let  represent the total occurrence of pattern  over all  in the convex combination, i.e. .


\begin{figure}
\begin{center}
\includegraphics{NoNT3EC.ps}
\end{center}
\caption{Inductive step for Case \ref{LemmaBaseCase}.}
\label{NoNonTrivial3EC}
\end{figure}

Using the fact that  occurs exactly  of the time,  occurs exactly  of the time and , it follows that


To create a convex combination of subgraphs for , we create one or two \mbox{2-edge} connected spanning subgraphs for each subgraph  in the convex combination for , as shown in Figure \ref{PatternsToMissingEdge}. In the case we use two, we use multiplier  for each, otherwise we use multiplier . In Figure \ref{PatternsToMissingEdge} the resulting occurrences of the corresponding patterns in  are indicated. Moreover, using (\ref{lambdasEquiv}) we have the occurrence of edges  and  is , the occurrence of  and  is , and the occurrence of edge  is , and all the other edges occur  of the time (illustrated on the right of Figure \ref{PatternsToMissingEdge}). For simplicity, we will always work with exact fractions: should the occurrence of  be less than  of the time, we will add it back to arbitrary subgraphs so that it appears exactly  overall.

\begin{figure}
\begin{center}
\includegraphics{MissingE.ps}
\end{center}
\caption{Patterns for  and , and their transformations.}
\label{PatternsToMissingEdge}
\end{figure}

Applying the same technique for all edges  taken as edge  means that we have  convex combinations, which we will refer to as  for each . Note that for any edge ,  occurs  in ,  occurs  in  for each of the four edges  adjacent to , and  occurs  in the rest of the convex combinations . We now take a convex combination of the  convex combinations , , by multiplying every multiplier  used in these convex combinations by . Summing the occurrence of every edge in this new convex combination gives Therefore, we have a convex combination for  for  and  holds true, contradiction.

\begin{case} \label{caseTwo}
 has a proper 3-edge cut.
\end{case}
Notice that the ends of the three edges must be distinct since  is 3-edge connected. In this case we contract each shore of the cut to a single vertex, to obtain graphs  with pseudo-vertex  and  with pseudo-vertex  (as shown in Figure\,\ref{3ECSplitting}). Both  and  are smaller than ,  and , so  and  hold. Moreover the patterns formed by the occurrence of the edges incident to  and  are unique and identical in the subgraphs in the corresponding convex combinations. For instance, exactly  of the time, one of the incident edges will not be in the subgraph, on both sides of the cut, and this is true for each of the three incident edges. The remaining subgraphs contain all three incident edges. These constant patterns allow us to ``glue'' (reconnect the edges as there were before the inductive step) the subgraphs for  and  together, in such a way that identical patterns at  and  are matched. This results in a convex combination for  that shows  holds, which gives a contradiction.
\end{proof}

\begin{figure}
\begin{center}
\includegraphics{3ECSplit.ps}
\end{center}
\caption{Contracting both sides of a proper 3-edge cut of .}
\label{3ECSplitting}
\end{figure}

We now use Lemma \ref{thmThreeSeventh} to obtain our main result below. We call a graph \mbox{} a \emph{half-triangle graph} if  is the support graph of a half-triangle solution . If all 1-paths in  consist of a single edge, we call  \emph{simple}.

\begin{definition}\label{defThm}
 Given a simple half-triangle graph  and a specified 1-edge , the vector  defined by
is a convex combination in which none of the 2-edge connected spanning multi-subgraphs use more than one copy of a half-edge or the edge , and all of them use either one or two copies of a 1-edge.
\end{definition}

\begin{theorem}\label{mainThm}
 holds for all simple half-triangle graphs  and any 1-edge  not in a 2-edge cut in .
\end{theorem}

\begin{proof}
\setcounter {case} {0}

\begin{case}\label{case1}
 has no 2-edge cut.
\end{case}
\begin{figure}
\begin{center}
\includegraphics{Special.ps}
\end{center}
\caption{Convex combination for  when  has only two triangles.}
\label{SpecialCase}
\end{figure}
If  has only two half-triangles, then  can be shown directly, using the  and  shown in Figure \ref{SpecialCase}, where edges represented by dotted lines are omitted, and  and  contain a multi-edge. Otherwise, let \mbox{} be the graph obtained from  by shrinking each half-triangle to a pseudo-vertex. Graph  is cubic and 3-edge connected and has , therefore by Lemma \ref{thmThreeSeventh},  holds, and yields a convex combination for  with an edge occurrence of  for all edges. Let the subgraphs in this convex combination be  with multipliers .

For each subgraph  in the convex combination for , the half-triangles (previously contracted to pseudo-vertices) will now be expanded to conclude the proof. We will add 1-edges and half-edges to each expanded  in such a way that we create a convex combination for the original half-triangle graph  that gives the required occurrence for each edge for the theorem. To accomplish this, for each triangle  in each subgraph, we will add half-edges in patterns, where each pattern is used a fraction of the time (either , or ). To facilitate this, we simply assume that we start with a new convex combination of  which contains six copies of each subgraph , where each copy has a coefficient of . 

Now consider any triangle  in  and let its incident edges be ,  and . In the convex combination created for , we have all three of these edges, or just two of these edges occur in each subgraph . Let 

Note that , and each of the 1-edges ,  and  occur  of the time, thus each is missing exactly  of the time. Thus


First we consider any expanded triangle  which is not incident with edge . For each subgraph  in which all three edges ,  and  occur, we include two of the three edges in   of the time. These patterns and their corresponding occurrences are illustrated in Figure\,\ref{ExpandingPseudoVerticesNoEdgeMissing}, and result in an occurrence of  for each edge of  overall, by (\ref{newLambdas}). Note that using each pattern one third of the time can be accomplished by using the patterns of Figure\,\ref{ExpandingPseudoVerticesNoEdgeMissing} for  for two of the six copies of each  where ,  and  occur. Then for each subgraph  in which  is omitted and  and  occur, we consider both triangle  and  the other triangle  incident with . In this case we include the edges in  incident with   of the time, and the other edge in   of the time, and do the opposite in triangle . In all cases we also include two copies of edge . The patterns are illustrated in Figure\,\ref{ExpandingPseudoVerticesOneEdgeMissing} and result in an occurrence of  for each edge in . Note that using each pattern half of the time can be accomplished by using each of the two patterns shown in Figure\,\ref{ExpandingPseudoVerticesOneEdgeMissing} for  (and ) in three of the six copies of each  in which  is omitted. We do the same for the cases where  or  are omitted in . The total occurrence of each half-edge in  is 

which by (\ref{newLambdas}) is . We can arbitrarily add back half-edges in the convex combinations to obtain an occurrence of exactly  for these edges (for a complete illustration of the operations and the pattern occurrences, see Figure\,\ref{ExpandingPseudoVertices}).

\begin{figure}
\begin{center}
\includegraphics{NoEMiss.ps}
\end{center}
\caption{Patterns used for triangle expansion for subgraphs containing ,  and .}
\label{ExpandingPseudoVerticesNoEdgeMissing}
\end{figure}

\begin{figure}
\begin{center}
\includegraphics{OneEMiss.ps}
\end{center}
\caption{Patterns used for triangle expansion for an omitted edge .}
\label{ExpandingPseudoVerticesOneEdgeMissing}
\end{figure}

Note that each 1-edge which is not  is now doubled whenever it was previously omitted, and thus occurs  of the time. Also note that all patterns used in the expansion of the half-triangles ensure that the new multi-subgraphs created from the subgraphs  for  are also 2-edge connected and spanning in , as required. 

\begin{figure}
\begin{center}
\includegraphics{Compact.ps}
\end{center}
\caption{Examples of edge selection upon expanding the pseudo-vertices of .}
\label{ExpandingPseudoVertices}
\end{figure}

Next we consider any expanded triangle  which is incident with edge , and WLOG let . For each subgraph  in which all three edges ,  and   occur, we include two of the three edges in   of the time, using the two patterns illustrated in Figure\,\ref{ExpandingPseudoVerticesPNotMissing} (so we use each pattern in three of the six copies of ). Then for each subgraph  in which  is omitted and  and  occur we include the two edges of  incident with  in any of these operations. Recall this occurs  of the time, by (\ref{newLambdas}). Note that we do not double edge . The total occurrence of each edge of  is exactly , and  occurs exactly  of the time (see Figure \ref{ExpandingPseudoVertices} for a complete illustration of these operations and the pattern occurrences).

\begin{figure}
\begin{center}
\includegraphics{PNotMiss.ps}
\end{center}
\caption{Patterns used for triangle expansion for subgraphs containing ,  and .}
\label{ExpandingPseudoVerticesPNotMissing}
\end{figure}

We now have, over all cases, the half-edge occurrence is ,  occurs  of the time, and the occurrence of the other 1-edges is . Furthermore, none of the 2-edge connected spanning multi-subgraphs use more than one copy of a half-edge or the edge , and all of them use either one or two copies of a 1-edge. Thus  holds.

\begin{case}
 has a 2-edge cut .
\end{case}
Suppose the contrary, and let  be the smallest counter-example for which  does not hold. Let ,  be the two sides of the cut  in , with  and  in  and  and  in , and WLOG choose  such that  is 3-edge connected and does not contain . By smaller example and Case\,\ref{case1},  and  hold. We now ``glue'' together in the obvious way, the subgraphs in the convex combination for  where  is omitted with the subgraphs in the convex combination for  which have  doubled (both patterns occur\, of the time) by removing the double edge  and adding two copies of edges  and . Similarly, we glue the subgraphs for  and  where  and  occur as single edges in the subgraphs (both patterns occur  of the time) by removing  and  and adding edges  and . We obtain , contradiction.
\end{proof}

By replacing 1-edges by 1-paths in the convex combinations for , and doubling the path for  wherever  was omitted, we can obtain  as a convex combination for any half-triangle solution , i.e. there exist \mbox{2-edge} connected spanning multi-subgraphs  with multipliers ,  such that  and 

Now consider any non-negative cost vector  which is optimized at  for \TwoECLP{}, i.e. . By multiplying both sides of (\ref{costEqn}) by , we obtain  and thus, for at least one subgraph  in the convex combination,

Since  and , it follows that  for such cost functions. As there exists a family of half-triangle solutions which show  asymptotically\,\cite{alexander}, we obtain the following corollary to Theorem \ref{mainThm}.

\begin{corollary}
The integrality gap  when restricted to cost functions optimized at half-triangle solutions.
\end{corollary}

\begin{thebibliography}{10}\label{bibliography}

\bibitem{alexander} Alexander, A., Boyd, S., Elliott-Magwood, P., On the Integrality Gap of the 2-Edge Connected Subgraph Problem,  \emph{Technical Report TR-2006-04}, SITE, University of Ottawa, Ottawa, Canada.

\bibitem{boyd} Boyd, S., Fu, Y., Sun, Y., A 5/4-approximation for subcubic 2EC using circulations and obligated edges, \emph{Discrete Applied Mathematics}, doi:\,1016/j.dam.2015.10.014 (in press).
a
\bibitem{carr} Carr R., Ravi R., A new bound for the 2-edge connected subgraph problem, \emph{Proceedings of Integer Programming and Combinatorial Optimization (IPCO)}, Lecture Notes in Computer Science, Springer, 112-125, 1998.

\bibitem{csaba} Csaba, B., Karpinski, M., Krysta, P., Approximability of dense and sparse instances of minimum 2-connectivity, tsp and path problems, \emph{D. Eppstein (Ed.),
SODA}, ACM/SIAM, 74-83, 2002.

\bibitem{frederickson} Frederickson, G. N., Ja'Ja', J. On the relationship between the biconnectivity augmentation and travelling salesman problems, \emph{Theoretical Computer Science}, 19, 189-201, 1982.

\bibitem{jain} Jain, K., A factor 2 approximation algorithm for the Generalized Steiner Network Problem, \emph{Combinatorica 21 No 1}, 39-60, 2001.

\bibitem{sebo} Sebő, A., Vygen, J., Shorter tours by nicer ears: 7/5 approximation for the graph-tsp, 3/2 for the path version, and 4/3 for two-edge-connected subgraphs, \emph{Combinatorica 34}, 597-629, 2014.

\end{thebibliography}
\end{document}