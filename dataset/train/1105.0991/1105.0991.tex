

\documentclass[preprint,number,12pt]{elsarticle}








\usepackage{amssymb}










\journal{Theory of computing systems}

\begin{document}
\newtheorem{thm}{Theorem}
\newtheorem{lem}[thm]{Lemma}
\newtheorem{cor}{Corollary }
\newdefinition{defn}{Definition}
\begin{frontmatter}





\title{Extra connectivity measures of 3-ary -cubes\tnoteref{support}}
\tnotetext[support]{This work is supported by the Postdoc Research Fund of China and the National Natural Science Foundation of China under Grant Nos.60603098\  \& \ 60804021\  \& \ 60703118 }


\author[xidian]{Qiang Zhu\corref{cor1}}
\ead{qiangzhu@ustc.edu}
\author[xidian]{Xin-Ke Wang}
\ead{wxk1383@126.com}
\author[lgd]{Juanjuan Ren}
\ead{renjj8011@163.com}

\cortext[cor1]{Corresponding author}


\address[xidian]{Department of Mathematics, Xidian University, Xi'an, Shanxi 710071, China}
\address[lgd]{School of Management Science, Xian University of Technology, Xi'an Shanxi 710023, China}



\begin{abstract}
The -extra connectivity is an important parameter to measure the reliability and fault tolerance ability of large interconnection networks. The -ary -cube is an important interconnection network of parallel computing systems. The -restricted connectivity of -ary -cubes has been obtained by Chen et al. for  in \cite{Chen2007-p1848-1855}. Nevertheless, the -extra connectivity of 3-ary -cubes has not been obtained yet. In this paper we prove that  the 1-extra connectivity  of 3-ary -cube is  for  and the 2-extra connectivity  of 3-ary -cube is  for .
\end{abstract}

\begin{keyword}
-ary -cubes \sep extra connectivity \sep super connectivity \sep interconnection networks




\end{keyword}

\end{frontmatter}




\section{Introduction}
Concurrent systems  incorporating a large number of processors
have got much development in recent years. Communication network is the critical component of a concurrent supercomputer. Many algorithms are communication rather than processing limited. In some cases,  the
reliability of multiprocessor systems is quite important. Connectivity and edge connectivity are traditional measures of the reliability of a communication network. Later on, restricted connectivity, restricted edge connectivity, super connectivity and super edge connectivity are proposed as generalizations of connectivity and edge connectivity to better measure the reliability of communication networks \cite{Boesch1986-p240-246, Esfahanian1989-p1586-1591, Esfahanian1988-p195-199, Wang2004-p199-205, Wang2008-p587-596, Chen2007-p1848-1855}. The -extra connectivity and -extra edge connectivity of interconnection networks are introduced by F\'abrega and Fiol \cite{F`abrega1994-p163-170} in 1994. The 1-extra connectivity and The 1-extra edge connectivity are super connectivity and super edge connectivity respectively.
Thus, the -extra connectivity and the -extra edge connectivity are generalizations of super connectivity and super edge connectivity. So they can be used to better measure the fault tolerance ability of interconnection networks.
Xu et al.\cite{Xu2007-p222-226} and Zhu et al.\cite{Zhu2006-p111-121} studied the extra connectivity and extra edge connectivity measures of some interconnection networks.

The -ary -cube  has been used in the design of several concurrent computers\cite{Esfahanian1989-p1586-1591, Roth1993-p35-35}.
It has many good topological properties, for example, low degree and diameter, efficient distributed routing algorithms, low cost embedding of other topologies and so on. Some properties of the -ary -cube network have been investigated, such as resource placement, routing \cite{Yoshinaga2004-p49-58}, etc. In this paper, the ( = 1, 2)-extra connectivity of the 3-ary -cube  will be determined.

The rest of this paper is organized as follows. Definition and preliminaries are given in section II. Section III and Section IV discuss the 1-extra connectivity and the 2-extra connectivity of 3-ary -cubes respectively. Finally,  we present our conclusions in Section V.

\section{Preliminaries} \label{S2:Preliminaties}
For all the terminologies and notations not defined here, we
follow \cite{1097029}. For a graph  and 
or , we use   to denote the set of
neighboring vertices  of  in , when it is easy to know
from the context what  denotes, it is usually simplified with
. We use  to denote the union of  and .
And similarly   can be simplified with .
That is, . Given two vertex disjoint graphs  with the same order,
let  is a perfect matching between  and ,
the graph  is defined as follows:  and .

In \cite{F`abrega1994-p163-170}, J.Fbrega and M.A. Fiol introduced the extra connectivity of interconnection networks as below.

\begin{defn} \cite{F`abrega1994-p163-170}
An vertex-set  is called a -extra vertex-cut if  is disconnected and every component of  has more than  vertices. The -extra connectivity of , denoted by , is defined as the cardinality of a minimum -extra vertex-cut.
\end{defn}

In particular, the 1-extra vertex-cut is called as the extra vertex-cut and the 1-extra connectivity is called as the extra connectivity. Clearly,  for any graph G if G is not a complete graph.

In this paper, Lee weight and Lee distance will be used to explore the properties of -ary -cubes. And node labels will be written as  rather than -tuples .

\begin{defn}Lee weight \cite{Bose1995-p1021-1030}:
Let , where  for all , be an n-digit radix  vector. The Lee weight of  is defined as , where .
\end{defn}

\begin{defn}Lee distance \cite{Bose1995-p1021-1030}:
For any two an n-digit radix  vectors , the Lee distance between them, denoted by , is the Lee weight of their bit-wise difference (mod ). That is, .
\end{defn}

For example, for , , , , .

In a -ary -cube, denoted by , where  is referred to as the radix and  as the dimension, each node can be identified by an -digits radix  address , where  for .
Two nodes with addresses  and  in a -ary -cube have
an edge if and only if . Therefore, each node is connected to two neighboring nodes
in each dimension. So the -ary -cube is an -regular graph.
The -ary 1-cube is the well-known ring, while the -ary 2-cube and -ary 3-cube are best known as the torus; a variation of the mesh with wraparound connections. Note that when  the network collapses to the well-known hypercube topology. Fig. 1 shows some examples of -ary -cubes.

\begin{figure}\centering
\scalebox{1.0}{\includegraphics{Q13.pdf}}
\scalebox{1.0}{\includegraphics{Q23.pdf}}
\caption{Examples of 3-ary -cubes:  and }
\end{figure}

Given two vertices  and  in a -ary -cube, the Lee distance  is also the distance between them. So Lee distance can be used to explore the properties of -ary -cubes.

\begin{lem}\label{dl}
For any two vertices ,  iff  have exactly  different digits.
\end{lem}
\emph{Proof:}  Suppose , .
For any ,

Since ,  the result follows. \hfill\rule{1mm}{2mm}

The -th position from the left of the -digit radix 3 string  is called as the -th dimension of .
Given an (), let , where . Then the spanning subgraph of  is isomorphic to a -ary -cube. Thus, a -ary -cube can be divided into  vertex-disjoint 3-ary -cubes, denoted by .
Given such a decomposition, by the definition of , there is a perfect matching between
 and , where 
and .

Given a decomposition of , we define the following symbols to simplify some expressions.
For any vertex , we call its two neighbors not in   to be 's pair vertices. By the definition of ,  and its two pair vertices form a triangle in .  An edge  is called an -edge if  differs in their -th dimension. In a decomposition of  along the -th dimension, all -edges are between different 3-ary -subcubes, we call these edges as matching edges.

\section{1-extra connectivity}\label{S3:SSG}

\begin{lem}\label{l1}
1) Any two adjacent vertices in  have exactly one common neighbor for ; \ \ \  2) If any two nonadjacent vertices in  have common neighbors, they have exactly two common neighbors for .
\end{lem}
\emph{Proof:}
1) Suppose that  and  are adjacent vertices in . Let , then  with  (mod 3). Without loss of generality, assume that . Let , where  (mod 3)  (mod 3). Then it is clear that . Since , . Hence, . Suppose  and  have another common neighbor . Let  with . However,  and this is a contradiction.

2) Suppose that two nonadjacent vertices  and  in  have a common neighbor . Then the Lee distance between  and  is 2.  Let , then  has exactly two digits different from  by Lemma \ref{dl}. We suppose   with . Obviously,  and  have exactly two common neighbors  and  for . 



\begin{lem}\label{l3} \cite{Bose1995-p1021-1030}
.
\end{lem}

\begin{lem}\label{l4}
Let  be an arbitrary vertex in , then  is connected for .
\end{lem}
\emph{Proof:}
Given a decomposition of : . Without loss of generality, suppose .  Let  and  be 's pair vertices in   and  respectively. Since  for , both   and  are connected.
By the definition of ,
any vertex in both  and  has a pair vertex in .
Hence,  is connected. 

\begin{lem}\label{l5}
Let  be an arbitrary edge in  and . Then  and  is an 1-extra vertex-cut for .
\end{lem}
\emph{Proof:}
It is clear that  is disconnected. By Lemma \ref{l1},  and  have exactly one common neighbor, then .
In the following we will prove that  is connected.

Given a decomposition of :  such that
 are not in the same 3-ary -subcube. Without loss of generality, suppose
 and .
By Lemma \ref{l4}, both  and  are connected. Let
 be the common neighbor of  in .
Then  is connected since  for .
By the definition of ,
any vertex in both  and 
has a pair vertex in . Thus  is connected.
Obviously,  for .
Hence,  is an 1-extra vertex-cut for . 


\begin{thm}\label{t1}
 for .
\end{thm}
\emph{Proof:}
We first prove . For this purpose, We only need to show that for
any  with ,
if there is  no isolated vertex in , then  is connected.

Given a decomposition of : .
Let  with .
Then at least one of ,  and  is strictly less than  since  and  for . Without loss of generality, suppose . Then  is connected by Lemma \ref{l3}.
Let  and .
In the following we will prove that any vertex  in  can be 
connected to a vertex of the connected subgraph  in .

Let  be 's pair vertex in . If , then we are done.
So we assume that . Since there is no isolated vertex in , there exists
a vertex  adjacent to  in . According to  is 's pair vertex or not,
we consider the following two case.

\textbf{Case 1 }  is not 's pair vertex.
Then  or .
Without loss of generality, suppose .
Let  be 's pair vertex in . If ,
then we are done. So assume that .
By Lemma \ref{l1}, .
Since  for , and there exists a perfect matching between
 and , there exists a vertex  such that
 and its pair vertex  in  all don't belong to .
This implies that  can be connected to a vertex of the connected subgraph  in 
via a path passing through .

\textbf{Case 2 }  is 's pair vertex.
Then  and ,
or  and .
Without loss of generality, suppose  and .
By Lemma \ref{l1}, .
However, .
Thus there is a vertex  such that . 
Then  adjacent to  or  in . Without loss of generality, 
suppose  is adjacent to . Then .
Let  be 's pair vertex in . If , then we are done.
So we assume that .
By Lemma \ref{l1}, . Since
 for , and there exists a perfect matching between
 and , there
exists a vertex  in  such that  and its pair vertex  in 
all don't belong to .
This implies that  can be connected to a vertex of the connected subgraph  in
 via a path passing through .

So any vertex in  can be connected to a vertex of the connected subgraph  in . Therefore,  is connected.
Thus  for .
However, by Lemma \ref{l5},  for .

Hence,  for .






\section{2-extra connectivity}
\begin{lem}\label{l6}
 for .
\end{lem}
\emph{Proof:}
Suppose  is a path of length two in  and .
Then  is disconnected.
By Lemma \ref{l1},  and  (resp.  and ) have exactly one common neighbor  (resp. ), and  and  have exactly two common neighbors . By the definition of ,  is different from
 and  (see Figure 2 (a) for illustration).
Thus, .
In the following we will prove that  is connected.

\begin{figure}\centering
\scalebox{1.0}{\includegraphics{extra1.pdf}}
\scalebox{0.8}{\includegraphics{extra2.pdf}}
\caption{Illustration for Lemma \ref{l6}}
\end{figure}


Given a decomposition of :  such that
, and  is 's pair vertex in .
By Lemma \ref{l1},  has a pair vertex  in ,
and  and  have a common neighbor  in .
Clearly,  is connected for . By the definition of ,
any vertex in both  and 
has a neighbor in  (see Figure 2 (b) for illustration). Thus  is connected (see Fig. 3).
Obviously,  for .
So  is an 2-extra vertex-cut for .

Hence,  for .


\begin{thm}\label{t2}
 for .
\end{thm}
\emph{Proof:}
For this purpose, we only need to prove that for any  with , if
there is no isolated vertex or isolated edge in , then  is connected.

Given a decomposition of : .
Let  with .
Then at least one of ,  and  is not more than  since  and . Without loss of generality, suppose that . Then  is connected by Lemma \ref{l3}.
Let  and .
In the following we will prove that any vertex  in  can be connected to a vertex of the connected subgraph  in .

Let  be 's pair vertex in . If , then we are done. So assume that . Since there is no isolated vertex in , there exists a vertex  adjacent to  in . According to  is 's pair vertex or not, we consider the following two cases.

\textbf{Case 1 }  is not 's pair vertex.
Then  or .
Without loss of generality, suppose .
Let  be 's pair vertex in . If ,
then we are done. So assume that .
By Lemma \ref{l1}, .
Since  for , and there exists a perfect matching between
 and , there exists a vertex  such that
 and its pair vertex  in  all don't belong to .
This implies that  can be connected to a vertex of the connected subgraph  in 
via a path passing through .

\textbf{Case 2 }  is 's pair vertex. Then  and ,
or  and .
Without loss of generality, suppose  and .
Since there is no isolated edge in , there exists a vertex  adjacent to  or  in . Without loss of generality, suppose  is adjacent to . Then 
Let  be 's pair vertex in . If , then we are done.
So we assume that .
By Lemma \ref{l1}, . Since
 for , and there exists a perfect matching between
 and , there
exists a vertex  in  such that  and its pair vertex  in 
all don't belong to .
This implies that  can be connected to a vertex of the connected subgraph  in 
 via a path passing through .

So any vertex in  can be connected to a vertex of the connected subgraph  in . Therefore,  is connected. 

Hence,  for .



By Lemma \ref{l6} and Theorem \ref{t2}, we obtain the following corollary.
\begin{cor}
 for .
\end{cor}

\section{Conclusions and future research}
In this paper, we have obtained, for the first time, the 1-extra connectivity (the super connectivity) and the 2-extra connectivity of 3-ary -cubes.
The main results of this paper are as follows: 
The tools we have used to determine the 1-extra connectivity and the 2-extra connectivity of 3-ary -cubes is Lee distance.  Using Lee distance we have showed that 3-ary -cubes can be decomposed into 3 vertex disjoint -subcubes along any dimension. We believe this property will be useful in the study of other properties of 3-ary -cubes, for example,   the diagnosability and conditional diagnosability of 3-ary -cubes both  under the PMC model and Comparison model. In the future, we will explore the conditional diagnosability of -ary -cubes both under the PMC model and Comparison model.









\section*{References}
\bibliographystyle{elsarticle-num}
\bibliography{parallelcomputing}










\end{document}
