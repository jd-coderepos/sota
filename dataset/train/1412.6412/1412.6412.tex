\documentclass[letterpaper,compsoc,twoside]{IEEEtran}
\usepackage{fixltx2e} \usepackage{cmap} \usepackage{ifthen}
\usepackage[T1]{fontenc}
\usepackage[utf8]{inputenc}
\usepackage{amsmath}

\usepackage[font={small,it},labelfont=bf]{caption}
\usepackage{float}

\setcounter{secnumdepth}{3}

\pdfoutput=1
\usepackage{scipy}
\makeatletter
\def\PY@reset{\let\PY@it=\relax \let\PY@bf=\relax \let\PY@ul=\relax \let\PY@tc=\relax \let\PY@bc=\relax \let\PY@ff=\relax}
\def\PY@tok#1{\csname PY@tok@#1\endcsname}
\def\PY@toks#1+{\ifx\relax#1\empty\else \PY@tok{#1}\expandafter\PY@toks\fi}
\def\PY@do#1{\PY@bc{\PY@tc{\PY@ul{\PY@it{\PY@bf{\PY@ff{#1}}}}}}}
\def\PY#1#2{\PY@reset\PY@toks#1+\relax+\PY@do{#2}}

\expandafter\def\csname PY@tok@gd\endcsname{\def\PY@tc##1{\textcolor[rgb]{0.63,0.00,0.00}{##1}}}
\expandafter\def\csname PY@tok@gu\endcsname{\let\PY@bf=\textbf\def\PY@tc##1{\textcolor[rgb]{0.50,0.00,0.50}{##1}}}
\expandafter\def\csname PY@tok@gt\endcsname{\def\PY@tc##1{\textcolor[rgb]{0.00,0.27,0.87}{##1}}}
\expandafter\def\csname PY@tok@gs\endcsname{\let\PY@bf=\textbf}
\expandafter\def\csname PY@tok@gr\endcsname{\def\PY@tc##1{\textcolor[rgb]{1.00,0.00,0.00}{##1}}}
\expandafter\def\csname PY@tok@cm\endcsname{\let\PY@it=\textit\def\PY@tc##1{\textcolor[rgb]{0.25,0.50,0.56}{##1}}}
\expandafter\def\csname PY@tok@vg\endcsname{\def\PY@tc##1{\textcolor[rgb]{0.73,0.38,0.84}{##1}}}
\expandafter\def\csname PY@tok@m\endcsname{\def\PY@tc##1{\textcolor[rgb]{0.13,0.50,0.31}{##1}}}
\expandafter\def\csname PY@tok@mh\endcsname{\def\PY@tc##1{\textcolor[rgb]{0.13,0.50,0.31}{##1}}}
\expandafter\def\csname PY@tok@cs\endcsname{\def\PY@tc##1{\textcolor[rgb]{0.25,0.50,0.56}{##1}}\def\PY@bc##1{\setlength{\fboxsep}{0pt}\colorbox[rgb]{1.00,0.94,0.94}{\strut ##1}}}
\expandafter\def\csname PY@tok@ge\endcsname{\let\PY@it=\textit}
\expandafter\def\csname PY@tok@vc\endcsname{\def\PY@tc##1{\textcolor[rgb]{0.73,0.38,0.84}{##1}}}
\expandafter\def\csname PY@tok@il\endcsname{\def\PY@tc##1{\textcolor[rgb]{0.13,0.50,0.31}{##1}}}
\expandafter\def\csname PY@tok@go\endcsname{\def\PY@tc##1{\textcolor[rgb]{0.20,0.20,0.20}{##1}}}
\expandafter\def\csname PY@tok@cp\endcsname{\def\PY@tc##1{\textcolor[rgb]{0.00,0.44,0.13}{##1}}}
\expandafter\def\csname PY@tok@gi\endcsname{\def\PY@tc##1{\textcolor[rgb]{0.00,0.63,0.00}{##1}}}
\expandafter\def\csname PY@tok@gh\endcsname{\let\PY@bf=\textbf\def\PY@tc##1{\textcolor[rgb]{0.00,0.00,0.50}{##1}}}
\expandafter\def\csname PY@tok@ni\endcsname{\let\PY@bf=\textbf\def\PY@tc##1{\textcolor[rgb]{0.84,0.33,0.22}{##1}}}
\expandafter\def\csname PY@tok@nl\endcsname{\let\PY@bf=\textbf\def\PY@tc##1{\textcolor[rgb]{0.00,0.13,0.44}{##1}}}
\expandafter\def\csname PY@tok@nn\endcsname{\let\PY@bf=\textbf\def\PY@tc##1{\textcolor[rgb]{0.05,0.52,0.71}{##1}}}
\expandafter\def\csname PY@tok@no\endcsname{\def\PY@tc##1{\textcolor[rgb]{0.38,0.68,0.84}{##1}}}
\expandafter\def\csname PY@tok@na\endcsname{\def\PY@tc##1{\textcolor[rgb]{0.25,0.44,0.63}{##1}}}
\expandafter\def\csname PY@tok@nb\endcsname{\def\PY@tc##1{\textcolor[rgb]{0.00,0.44,0.13}{##1}}}
\expandafter\def\csname PY@tok@nc\endcsname{\let\PY@bf=\textbf\def\PY@tc##1{\textcolor[rgb]{0.05,0.52,0.71}{##1}}}
\expandafter\def\csname PY@tok@nd\endcsname{\let\PY@bf=\textbf\def\PY@tc##1{\textcolor[rgb]{0.33,0.33,0.33}{##1}}}
\expandafter\def\csname PY@tok@ne\endcsname{\def\PY@tc##1{\textcolor[rgb]{0.00,0.44,0.13}{##1}}}
\expandafter\def\csname PY@tok@nf\endcsname{\def\PY@tc##1{\textcolor[rgb]{0.02,0.16,0.49}{##1}}}
\expandafter\def\csname PY@tok@si\endcsname{\let\PY@it=\textit\def\PY@tc##1{\textcolor[rgb]{0.44,0.63,0.82}{##1}}}
\expandafter\def\csname PY@tok@s2\endcsname{\def\PY@tc##1{\textcolor[rgb]{0.25,0.44,0.63}{##1}}}
\expandafter\def\csname PY@tok@vi\endcsname{\def\PY@tc##1{\textcolor[rgb]{0.73,0.38,0.84}{##1}}}
\expandafter\def\csname PY@tok@nt\endcsname{\let\PY@bf=\textbf\def\PY@tc##1{\textcolor[rgb]{0.02,0.16,0.45}{##1}}}
\expandafter\def\csname PY@tok@nv\endcsname{\def\PY@tc##1{\textcolor[rgb]{0.73,0.38,0.84}{##1}}}
\expandafter\def\csname PY@tok@s1\endcsname{\def\PY@tc##1{\textcolor[rgb]{0.25,0.44,0.63}{##1}}}
\expandafter\def\csname PY@tok@gp\endcsname{\let\PY@bf=\textbf\def\PY@tc##1{\textcolor[rgb]{0.78,0.36,0.04}{##1}}}
\expandafter\def\csname PY@tok@sh\endcsname{\def\PY@tc##1{\textcolor[rgb]{0.25,0.44,0.63}{##1}}}
\expandafter\def\csname PY@tok@ow\endcsname{\let\PY@bf=\textbf\def\PY@tc##1{\textcolor[rgb]{0.00,0.44,0.13}{##1}}}
\expandafter\def\csname PY@tok@sx\endcsname{\def\PY@tc##1{\textcolor[rgb]{0.78,0.36,0.04}{##1}}}
\expandafter\def\csname PY@tok@bp\endcsname{\def\PY@tc##1{\textcolor[rgb]{0.00,0.44,0.13}{##1}}}
\expandafter\def\csname PY@tok@c1\endcsname{\let\PY@it=\textit\def\PY@tc##1{\textcolor[rgb]{0.25,0.50,0.56}{##1}}}
\expandafter\def\csname PY@tok@kc\endcsname{\let\PY@bf=\textbf\def\PY@tc##1{\textcolor[rgb]{0.00,0.44,0.13}{##1}}}
\expandafter\def\csname PY@tok@c\endcsname{\let\PY@it=\textit\def\PY@tc##1{\textcolor[rgb]{0.25,0.50,0.56}{##1}}}
\expandafter\def\csname PY@tok@mf\endcsname{\def\PY@tc##1{\textcolor[rgb]{0.13,0.50,0.31}{##1}}}
\expandafter\def\csname PY@tok@err\endcsname{\def\PY@bc##1{\setlength{\fboxsep}{0pt}\fcolorbox[rgb]{1.00,0.00,0.00}{1,1,1}{\strut ##1}}}
\expandafter\def\csname PY@tok@kd\endcsname{\let\PY@bf=\textbf\def\PY@tc##1{\textcolor[rgb]{0.00,0.44,0.13}{##1}}}
\expandafter\def\csname PY@tok@ss\endcsname{\def\PY@tc##1{\textcolor[rgb]{0.32,0.47,0.09}{##1}}}
\expandafter\def\csname PY@tok@sr\endcsname{\def\PY@tc##1{\textcolor[rgb]{0.14,0.33,0.53}{##1}}}
\expandafter\def\csname PY@tok@mo\endcsname{\def\PY@tc##1{\textcolor[rgb]{0.13,0.50,0.31}{##1}}}
\expandafter\def\csname PY@tok@mi\endcsname{\def\PY@tc##1{\textcolor[rgb]{0.13,0.50,0.31}{##1}}}
\expandafter\def\csname PY@tok@kn\endcsname{\let\PY@bf=\textbf\def\PY@tc##1{\textcolor[rgb]{0.00,0.44,0.13}{##1}}}
\expandafter\def\csname PY@tok@o\endcsname{\def\PY@tc##1{\textcolor[rgb]{0.40,0.40,0.40}{##1}}}
\expandafter\def\csname PY@tok@kr\endcsname{\let\PY@bf=\textbf\def\PY@tc##1{\textcolor[rgb]{0.00,0.44,0.13}{##1}}}
\expandafter\def\csname PY@tok@s\endcsname{\def\PY@tc##1{\textcolor[rgb]{0.25,0.44,0.63}{##1}}}
\expandafter\def\csname PY@tok@kp\endcsname{\def\PY@tc##1{\textcolor[rgb]{0.00,0.44,0.13}{##1}}}
\expandafter\def\csname PY@tok@w\endcsname{\def\PY@tc##1{\textcolor[rgb]{0.73,0.73,0.73}{##1}}}
\expandafter\def\csname PY@tok@kt\endcsname{\def\PY@tc##1{\textcolor[rgb]{0.56,0.13,0.00}{##1}}}
\expandafter\def\csname PY@tok@sc\endcsname{\def\PY@tc##1{\textcolor[rgb]{0.25,0.44,0.63}{##1}}}
\expandafter\def\csname PY@tok@sb\endcsname{\def\PY@tc##1{\textcolor[rgb]{0.25,0.44,0.63}{##1}}}
\expandafter\def\csname PY@tok@k\endcsname{\let\PY@bf=\textbf\def\PY@tc##1{\textcolor[rgb]{0.00,0.44,0.13}{##1}}}
\expandafter\def\csname PY@tok@se\endcsname{\let\PY@bf=\textbf\def\PY@tc##1{\textcolor[rgb]{0.25,0.44,0.63}{##1}}}
\expandafter\def\csname PY@tok@sd\endcsname{\let\PY@it=\textit\def\PY@tc##1{\textcolor[rgb]{0.25,0.44,0.63}{##1}}}

\def\PYZbs{\char`\\}
\def\PYZus{\char`\_}
\def\PYZob{\char`\{}
\def\PYZcb{\char`\}}
\def\PYZca{\char`\^}
\def\PYZam{\char`\&}
\def\PYZlt{\char`\<}
\def\PYZgt{\char`\>}
\def\PYZsh{\char`\#}
\def\PYZpc{\char`\%}
\def\PYZdl{\char`\^{\setcounter{footnotecounter}{1}\fnsymbol{footnotecounter}\setcounter{footnotecounter}{2}\fnsymbol{footnotecounter}}^{\setcounter{footnotecounter}{2}\fnsymbol{footnotecounter}}^{\setcounter{footnotecounter}{2}\fnsymbol{footnotecounter}}^{\setcounter{footnotecounter}{2}\fnsymbol{footnotecounter}}^{\setcounter{footnotecounter}{2}\fnsymbol{footnotecounter}}^{\setcounter{footnotecounter}{2}\fnsymbol{footnotecounter}}E(A)AR(A)B(A)\lambda \geq 0R(A)B(A)i = 1\dots N\mathbf{K}^iiG_j^iijG_j^i(p^i - p^j)ijG_j^iG_j^iG_i^jG_1^2G_2^3G_2^1G_3^2\neq 0G_i^j =
0p_i \in V^iq_i \in V^i_0i = 1,\dots,NV^iV^i_0CSC$ in the transverse
section. \DUrole{label}{simulation2}}
\end{figure}

\section{Conclusion\label{conclusion}}


Using Python with standard modules for scientific computing and image
processing together with the Python based finite element solver SfePy
and a collection of developed supporting applications, we are able to
produce a simplified patient specific liver model and numerically
simulate hepatic blood perfusion.

CT data are processed by the semi-automatic segmentation algorithms
generating 1D structures representing the vascular trees and a 3D
volumetric model of the liver tissue. In case of incomplete or
unreliable results of the vascular trees reconstruction, we fabricate
artificial trees using constructive optimization approach. The
reconstructed or fabricated 1D trees and the volumetric liver model
are employed in numerical simulations of liver blood perfusion using
the finite element method. The model of contrast fluid propagation
provides time-dependent concentration of the tracer, that can be
compared with the standard medical measurements. It will allow us to
solve the inverse problem in order to identify some of the perfusion
parameters of our models. This is a crucial point for further
development.

Despite the fact that there is still a wide gap between our current
research and clinical practice, the LISA application was successfully
tested by radiologists and surgeons for volumetric analyses of livers
prior to surgeries and is now actively used.

\subsection{Acknowledgment\label{acknowledgment}}


This research is partially supported by the Ministry of Health of the
Czech Republic, project NT 13326, and by the European Regional
Development Fund (ERDF), project \textquotedbl{}NTIS - New Technologies for the
Information Society\textquotedbl{}, European Centre of Excellence,
CZ.1.05/1.1.00/02.0090.
\begin{thebibliography}{Cim14b}
\bibitem[Boy01]{Boy01}{

Y. Boykov, O. Veksler, R. Zabih. \emph{Fast approximate energy
minimization via graph cuts.} In Pattern Analysis and
Machine Intelligence, 23(11):1222-1239, 2001.}
\bibitem[Boy06]{Boy06}{

Y. Boykov, G. Funka-Lea. \emph{Graph Cuts and Efficient N-D Image Segmentation.}
In International Journal of Computer Vision, 70:109–131, 2006.}
\bibitem[Cim14]{Cim14}{

R. Cimrman, et al. \emph{SfePy, finite element code and applications.}
Home page: \url{http://sfepy.org} {[}Accessed 2014-08-20{]}.}
\bibitem[Cim14b]{Cim14b}{

R. Cimrman. \emph{SfePy - Write Your Own \{FE\} Application.}
In Proceedings of the 6th European Conference on Python in
Science (EuroSciPy 2013), pages 65-70, 2014. \url{http://arxiv.org/abs/1404.6391}.}
\bibitem[Coo12]{Coo12}{

A. N. Cookson, J. Lee, C. Michler, R. Chabiniok, E. Hyde,
D. A. Nordsletten, M. Sinclair, M. Siebes, N. P. Smith.
\emph{A novel porous mechanical framework for modelling the
interaction between coronary perfusion and myocardial
mechanics.} In Journal of Biomechanics, 45(5):850-855, 2012.}
\bibitem[Geo10]{Geo10}{

M. Georg, T. Preusser, H. K. Hahn. \emph{Global Constructive
Optimization of Vascular Systems.} Technical Report:
Washington University in
St. Louis. \url{http://cse.wustl.edu/Research/Lists/TechnicalReports/Attachments/910/idealvessel_1.pdf}.}
\bibitem[Hei09]{Hei09}{

Heimann et al. \emph{Comparison and evaluation of methods for
liver segmentation from CT datasets.} In IEEE Transactions
on Medical Imaging, 28(8):1251-1265, 2009.}
\bibitem[Hom07]{Hom07}{

H. Homann. \emph{Implementation of a 3D thinning algorithm.} In
Insight Journal, July - December, 2007.}
\bibitem[Jir14]{Jir14}{

M. Jiřík. \emph{LISA - LIver Surgery Analyser.} Home page:
\url{https://github.com/mjirik/lisa} {[}Accessed 2014-08-20{]}.}
\bibitem[Joa14]{Joa14}{

A. Jonášová, E. Rohan, V. Lukeš, O. Bublík. \emph{Complex
hierarchical modeling of the dynamic perfusion test:
application to liver.} In Proceedings of 11th World Congres
of Computational Mechanics, 2014.}
\bibitem[Joa14b]{Joa14b}{

A. Jonášová, O. Bublík, E. Rohan, J. Vimmr. \emph{Simulation of
contrast medium propagation based on 1D and 3D portal
hemodynamics.} In: Proc. of the 20th International
Conference Engineering Mechanics, Svratka, Czech
Republic, 2014.}
\bibitem[Jon14]{Jon14}{

E. Jones, T. E. Oliphant, P. Peterson, et al. \emph{SciPy: Open
source scientific tools for Python.} Home page:
\url{http://www.scipy.org} {[}Accessed 2014-08-20{]}.}
\bibitem[Kol14]{Kol14}{

V. Kolmogorov. \emph{Max-flow/min-cut.} Home page:
\url{http://vision.csd.uwo.ca/code/} {[}Accessed 2014-08-20{]}.}
\bibitem[Lor87]{Lor87}{

W. E. Lorensen, H. E. Cline. \emph{Marching Cubes: A high
resolution 3D surface construction algorithm.} Computer
Graphics, Vol. 21, Nr. 4, 1987.}
\bibitem[Luk14]{Luk14}{

V. Lukeš. \emph{DICOM2FEM - application for semi-automatic
generation of finite element meshes.} Home page:
\url{http://sfepy.org/dicom2fem} {[}Accessed 2014-08-20{]}.}
\bibitem[Mas14]{Mas14}{

D. Mason. \emph{pydicom}, available at
\url{https://code.google.com/p/pydicom/} {[}Accessed 2014-08-20{]}.}
\bibitem[Mha12]{Mha12}{

A. M. Mharib, A. R. Ramli, S. Mashohor, R. B. Mahmood. \emph{Survey
on liver CT image segmentation methods.} In Artificial
Intelligence Review, 37(2):83-95, 2012.}
\bibitem[Mic13]{Mic13}{

C. Michler, A. Cookson, R. Chabiniok, E. Hyde, J. Lee,
M. Sinclair, T. Sochi, A. Goyal, G. Vigueras, D. Nordsletten, N. Smith.
\emph{A computationally efficient framework for the simulation
of cardiac perfusion using a multi-compartment Darcy
porous-media flow model.} Int. Journal for Numerical
Methods in Biomedical Engineering, 29(2):217-32, 2013.}
\bibitem[Mül14]{Mül14}{

A. Müller. \emph{Python wrappers for GCO alpha-expansion and
alpha-beta-swaps.} Home page:
\url{https://github.com/amueller/gco_python} {[}Accessed 2014-08-20{]}.}
\bibitem[Oli07]{Oli07}{

T. E. Oliphant. \emph{Python for scientific computing.} In
Computing in Science \& Engineering,
9(3):10-20, 2007. Home page: \url{http://www.numpy.org}.}
\bibitem[Tau95]{Tau95}{

G. Taubin. \emph{A signal processing approach to fair surface
design.} In Siggraph'95 Conference Proceedings, pages
351–358, 1995.}
\bibitem[Tau00]{Tau00}{

G. Taubin. \emph{Geometric Signal Processing on Polygonal
Meshes.}, In EUROGRAPHICS 2000, 2000.}
\bibitem[Roh12]{Roh12}{

E. Rohan, V. Lukeš, A. Jonášová, O. Bublík. \emph{Towards
microstructure based tissue perfusion reconstruction from
CT using multiscale modeling.} In Proc. of the 10th World
Congress on Computational Mechanics, Sao Paulo, Brasil, 2012.}
\bibitem[Roh12b]{Roh12b}{

E. Rohan, V. Lukeš. \emph{Modeling tissue perfusion using a
homogenized model with layer-wise decomposition.}
In Preprints MATHMOD 2012, Vienna University of Technology,
Austria, (2012).}
\bibitem[Sel02]{Sel02}{

D. Selle, B. Preim, A. Schenk, H. O. Peitgen. \emph{Analysis of
vasculature for liver surgical planning.} In IEEE
Transactions on Medical Imaging, 21(11):1344-1357, 2002.}
\end{thebibliography}

\end{document}
