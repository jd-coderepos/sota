




\begin{filecontents*}{example.eps}
gsave
newpath
  20 20 moveto
  20 220 lineto
  220 220 lineto
  220 20 lineto
closepath
2 setlinewidth
gsave
  .4 setgray fill
grestore
stroke
grestore
\end{filecontents*}
\RequirePackage{fix-cm}
\documentclass[twocolumn]{svjour3}          



\smartqed  \usepackage{graphicx}
\usepackage{times}
\usepackage{epsfig}
\usepackage{graphicx}
\usepackage{amsmath}
\usepackage{amssymb}
\usepackage{booktabs}
\usepackage{bm}
\usepackage{hyperref}
\usepackage{multirow}
\usepackage{multicol}
\usepackage{float}
\usepackage{stfloats}
\usepackage{subfiles}
\newcommand{\etal}{\textit{et al}. }
\newcommand{\ie}{\textit{i}.\textit{e}. }
\newcommand{\eg}{\textit{e}.\textit{g}.}
\hypersetup{
    colorlinks=true,
    linkcolor=blue,
    filecolor=magenta,      
    urlcolor=magenta,
    citecolor=blue,
}
\usepackage{natbib} 
\urlstyle{same}



\begin{document}


\title{FairMOT: On the Fairness of Detection and Re-Identification in Multiple Object Tracking }




\author{Yifu Zhang \and 
Chunyu Wang \and 
Xinggang Wang \and 
Wenjun Zeng \and 
Wenyu Liu 
}

\authorrunning{Yifu Zhang, Chunyu Wang, Xinggang Wang, Wenjun Zeng, Wenyu Liu} 

\institute{Yifu Zhang \at \email{yifuzhang@hust.edu.cn}          
          \and
           Chunyu Wang \at \email{chnuwa@microsoft.com}
           \and
           Xinggang Wang \at \email{xgwang@hust.edu.cn}
           \and
           Wenjun Zeng \at  \email{wezeng@microsoft.com}
           \and
           Wenyu Liu \at \email{liuwy@hust.edu.cn}
           \and
           \;\; Huazhong\ University\ of\ Science\ and\ Technology,\ Wuhan,\ China \\
           \;\; Microsoft\ Research\ Asia,\ Beijing,\ China \\
           \;\; Corresponding\ Author \\
           \;\; Yifu Zhang and Chunyu Wang have contributed equally. \\
}

\date{Received: date / Accepted: date}



\maketitle




\begin{abstract}

Multi-object tracking (MOT) is an important problem in computer vision which has a wide range of applications. Formulating MOT as multi-task learning of object detection and re-ID in a single network is appealing since it allows joint optimization of the two tasks and enjoys high computation efficiency. However, we find that the two tasks tend to compete with each other which need to be carefully addressed. In particular, previous works usually treat re-ID as a secondary task whose accuracy is heavily affected by the primary detection task. As a result, the network is biased to the primary detection task which is not \emph{fair} to the re-ID task.  To solve the problem, we present a simple yet effective approach termed as \emph{FairMOT} based on the anchor-free object detection architecture CenterNet. Note that it is not a naive combination of CenterNet and re-ID. Instead, we present a bunch of detailed designs which are critical to achieve good tracking results by thorough empirical studies. The resulting approach achieves high accuracy for both detection and tracking. The approach outperforms the state-of-the-art methods by a large margin on several public datasets. The source code and pre-trained models are released at \url{https://github.com/ifzhang/FairMOT}.

\keywords{FairMOT \and Multi-Object Tracking \and One-Shot \and Anchor-Free \and Real-Time Inference}
\end{abstract}

\section{Introduction}
\subfile{sub_files/1introduction}


\section{Related Work}
\subfile{sub_files/2relatedwork}

\section{Unfairness Issues in One-shot Trackers}
\subfile{sub_files/3unfairness}

\section{FairMOT}
\subfile{sub_files/4fairmot}

\section{Experiments}
\subfile{sub_files/5experiments}

\section{Summary and Future Work}
\subfile{sub_files/6conclusion}

\section*{Acknowledgements}
This work was in part supported by NSFC (No. 61733007 and No. 61876212) and MSRA Collaborative Research Fund. We thank all the anonymous reviewers for their valuable suggestions.









\bibliographystyle{spbasic}      \bibliography{ref.bib}   \end{document}
