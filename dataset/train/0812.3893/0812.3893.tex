\pdfoutput=1
\newif\ifFull
\Fulltrue
\ifFull
\documentclass[11pt]{article}
\else
\documentclass{llncs}
\fi

\usepackage{graphicx}
\usepackage{amsmath}
\ifFull
\usepackage{amsthm}
\usepackage{fullpage}
\usepackage{hyperref}
\fi
\usepackage{amssymb}
\usepackage{url}
\ifFull
\setlength{\pdfpagewidth}{8.5in}
\setlength{\pdfpageheight}{11in}
\fi



\ifFull
\newtheorem{theorem}{Theorem}
\newtheorem{lemma}[theorem]{Lemma}
\newtheorem{corollary}[theorem]{Corollary}
\newtheorem{definition}[theorem]{Definition}
\fi

\newcommand{\R}{{\bf R}}
\newcommand{\Z}{{\bf Z}}
\newcommand{\cycle}{\mathrm{cycle}}
\newcommand{\level}{\mathrm{level}}
\newcommand{\parent}{\mathrm{parent}}
\newcommand{\child}{\mathrm{child}}
\newcommand{\depth}{\mathrm{depth}}
\newcommand{\descendants}{\mathrm{descendants}}
\newcommand{\head}{\mathrm{head}}
\newcommand{\tail}{\mathrm{tail}}
\newcommand{\relativeDepth}{\mathrm{relativeDepth}}
\newcommand{\boundary}{\mathrm{boundary}}
\newcommand{\radialOrder}{\mathrm{radialOrder}}
\newcommand{\lefttext}{\mathrm{left}}
\newcommand{\righttext}{\mathrm{right}}
\newcommand{\acrosstext}{\mathrm{across}}
\newcommand{\superlevel}{\mathrm{superlevel}}


\ifFull
\newcommand{\leaveout}[1]{#1}  \newcommand{\Leaveout}[2]{#1}  \else
\newcommand{\leaveout}[1]{ }
\newcommand{\Leaveout}[2]{#2}
\let\oldendproof\endproof
\def\endproof{\qed\oldendproof}
\renewcommand{\subsection}[1]{\paragraph{#1.}}
\fi

\title{Succinct Greedy Geometric Routing \\
in the Euclidean Plane\thanks{This work was supported by
NSF grants 0724806, 0713046, 0830403, and ONR grant
N00014-08-1-1015.}\a) & (b)
    \end{tabular}
    \end{center}
\caption{ Arrows indicate valid greedy hops. 
    (a) Descendants of  can be reached by a simple path of 
        up and right hops, up and left hops, or
        a combination of the two.
    (b) If  is not a descendant of , then we route down and (left or right)
        in the direction of  until we reach an ancestor of .
    }
    \label{fig:subcactus_at_s}
    \label{fig:cactus_lca}
\end{figure}

Routing from start vertex  to a terminal vertex  in a Christmas
cactus graph embedding can be broken down into two cases: (1)  is 
a descendant of , and (2)  is not a descendant of .
\begin{enumerate}
    \item As shown in Fig.~\ref{fig:subcactus_at_s}(a), if  is a descendant 
          of , then we can route to  by a simple path of up and  
          right hops, up and left hops, or a 
          combination of the two. 

    \item As shown in Fig.~\ref{fig:cactus_lca}(b), if  is not a descendant
          of , then we route to the least common (cycle) ancestor of 
           and . Suppose, without loss of generality, that  is to 
          the left of , then we can reach this cycle by a sequence of 
          down and left hops. Once on the cycle, we can 
          move left until we reach an ancestor of . Now we are back in 
          case 1. 
\end{enumerate}
\ifFull
\subsection{A Succinct Compression Scheme}
Using the Christmas cactus graph embedding discussed above, we can assign 
succinct integer values to each vertex, allowing us perform greedy 
routing according to the Euclidean  metric. Our embedding 
 produces a triple of the following integers: 
: the number of vertices to the right of ;
: the number of semi-circles between the vertex 
and the origin, excluding the semi-circle that  is embedded on; and
: the smallest  value of all vertices that are 
descendants of . The Leighton-Moitra embedding has the 
property that all descendants of  fall between  and the 
vertex embedded immediately to the right of  on the same level as . 
Since each element of the triple can take on values in the range ,
the triple can be stored using  bits.

We can implement each step of the routing scheme using only the 
triples of , the neighbors of , and . Queries of the form 
\emph{ is left/right of }, involve a straightforward 
comparison of the  element of the triple. Likewise for \emph{ is 
above/below }, using . The same comparisons can be used to determine 
which neighbors of  are a left, right, down, or 
up move away. Finally, queries of the form \emph{ is a descendant
of } are true if and only if  and . 

To extend this routing scheme to graphs that have a spanning Christmas cactus 
subgraph, we need to ensure that the routing scheme
does not fail by following edges that are not in the Christmas cactus subgraph. 
Since the Christmas cactus graph has bounded degree , for a node , we can 
store the triples of neighbors of  in the Christmas cactus graph, in addition
to storing the triple for , and only allow our greedy routing scheme to 
choose vertices that are neighbors in the Christmas cactus subgraph. Storing
these extra triples in the coordinate does not increase its asymptotic 
bit-complexity.

This routing scheme is greedy according to the Euclidean coordinates of the 
vertices, using the Euclidean  metric. Unfortunately, if we only 
have access to the integer triples then it is not obvious that there 
is any metric that we can define that will satisfy the definition 
for greedy routing using just these integer values. Therefore, we must 
concede that, while this routing scheme fulfills the spirit of greedy 
routing, it is not greedy routing in the strictest sense. This is an 
example of a compression scheme and not a coordinate system.
\else
This routing scheme immediately gives rise to a simple 
succinct compression scheme for 3-connected planar graphs, which we discuss in
the full version of this paper. 
\fi

\section{Toward a Succinct Greedy Embedding} 
Given a 3-connected planar graph, we can find a spanning Christmas cactus
subgraph in polynomial time~\cite{lm-srgem-08}. Therefore, we restrict our
attention to Christmas cactus graphs. Our results apply to 
3-connected planar graphs with little or no modification. In this section, we construct a novel 
greedy embedding scheme for any Christmas cactus graph in . We 
then build a coordinate system from
our embedding and show that the coordinates can be represented using 
 bits. In the next section, we show how to achieve an
optimal -bit representation.



\subsection{Heavy Path Decompositions}

We begin by applying the Sleator and Tarjan~\cite{SleTar-JCSS-83} 
\emph{heavy path decomposition} to the depth tree  for .
\begin{definition} Let  be a rooted tree. For each node  in , 
let  denote the number of descendants of  in , including . 
For each edge  in , label  as a heavy edge 
if . Otherwise, label  as a light edge. 
Connected components of heavy edges form paths, called heavy paths. 
Vertices that are incident only to light edges are considered to be 
zero-length heavy paths. We call this the 
{\bfseries heavy path decomposition} of .
\end{definition}
For ease of discussion, we again apply the terminology from nodes in  to
cycles in . A cycle in  is on a heavy path  if its dual node in 
is on . Let  be a heavy path in . We say that  is the 
cycle in  that has minimum depth, we define  similarly. 
Let  and  be two cycles such that  and let 
.
If  and  are on the same heavy path then we call  a \emph{turnpike}.
If  and  are on different heavy paths (where  and 
) then we call  an \emph{off-ramp} for  and the vertices 
 \emph{on-ramps} for . 

\subsection{An Overview of Our Embedding Strategy}
Like Leighton and Moitra~\cite{lm-srgem-08}, we lay the cycles 
from our Christmas cactus graph on concentric semi-circles of radius 
; however, our embedding has the 
following distinct differences: we have  semi-circles
instead of  semi-circles, on-ramps to heavy paths are embedded on special 
semi-circles which we call \emph{super levels},
turnpikes are placed in a predefined position when cycles are embedded, 
and the radii of semi-circles can be computed without knowing the topology 
of the particular Christmas cactus graph being embedded. 
\ifFull
Since the path from the root to any leaf in the depth tree 
contains  heavy paths, our embedding has  
of super levels. Between super levels we lay out the non-trivial 
heavy paths on \emph{baby levels}. \fi

To make our embedding scheme amenable to a proof by induction, we
modify the input Christmas cactus graph. After constructing a greedy embedding of 
this modified graph, we use it to prove that we have a greedy 
embedding for the original graph.

\subsection{Modifying the Input Christmas Cactus Graph}
Given a Christmas cactus graph  on  vertices, we choose a depth 
tree  of , and compute the heavy path decomposition of . 
For a cycle  on a heavy path , we define  to be 
. For each ,  forming 
a light edge in , let . Split  into two 
vertices  and  each on their own cycle, and connect  to  
with a path of  edges. The new graph  
is also a Christmas cactus graph, and our new depth tree  looks 
like  stretched out so that heads of heavy paths (from ) are 
at depths that are multiples of . (See Fig.~\ref{fig:heavy}.) 
\leaveout{
\begin{figure}[!t]
\begin{center}
\begin{tabular}[b]{cc}
    \includegraphics[scale=0.7]{depth_tree.pdf} &
    \includegraphics[scale=0.7,clip=true,viewport=6cm 1.5cm 12.5cm 11cm]{mod_depth_tree.pdf} \\
    (a) & (b)
\end{tabular}
    \caption{(a) A depth tree  with positive-length heavy paths highlighted, and 
             (b) the new depth tree  after the modification procedure.}
    \label{fig:heavy}
\end{center}
\end{figure}
}
We continue to call the paths copied from  heavy paths (though they do 
not form a heavy path decomposition of ), and the newly inserted edges 
are \emph{dummy} edges.

\subsection{Embedding the Modified Christmas Cactus Graph in }
Given a Christmas cactus graph  on  vertices, run the modification 
procedure described above and get  and . We embed  in 
phases, and prove by induction that at the end of each phase
we have a greedy embedding of an induced subgraph of .

\begin{lemma}[Leighton and Moitra~\cite{lm-srgem-08}]
\label{lemma-geom}
\ifFull
If the coordinates 

are subject to the constraints

then .
\else If points
, , and 

are subject to the constraints
, , 
,
, and
 then .
\fi
\end{lemma}

We begin by embedding the root cycle, , of . 
We trace out a semi-circle of radius  centered at the origin and 
divide the perimeter of this semi-circle into  equal arcs. We allow
vertices to be placed at the leftmost point of each arc, numbering these 
positions  to . We place vertices  clockwise into 
any  distinct positions, reserving position  for 's turnpike. If  
does not have a turnpike, as is the case if  is a dummy edge or the tail 
of a heavy path, then position  remains empty. The embedding of  is 
greedy\Leaveout{}{\ (proof omitted here)}.

\leaveout{


\begin{proof} If  is a 2-cycle, then 
the embedding of  is greedy regardless of where the vertices are 
embedded. Otherwise, consider each segment . The 
perpendicular bisector to  does not intersect any of our
embedded vertices.  is the neighbor of  that is closer to every vertex
on the  side of the perpendicular bisector. Since all such segments have this
property, the embedding of  is greedy.
\end{proof}
}

\emph{Inductive Step:}
Suppose we have a greedy embedding all cycles in  up to depth ,
call this induced subgraph . We show that the embedding can
be extended to a greedy embedding of . Our proof relies on 
two values derived from the embedding of . 

\begin{definition}
Let ,  be any two distinct vertices in  and 
fix  to be a neighbor of  such that . 
We define .
\end{definition}

We refer to the difference  as the \emph{delta value} 
for distance-decreasing paths from  to  through .

\begin{definition}
Let  to be the minimum (non-zero) angle 
that any two vertices in the embedding of  form with the origin.
\end{definition}
\begin{figure}[!]
\vspace*{-28pt}
\begin{center}
\includegraphics[scale=0.7]{base.pdf}
\end{center}
\vspace*{-16pt}
\caption{,  and  form a lower bound for .}
\vspace*{-16pt}
\label{fig:delta_0}
\end{figure}
Since we do not specify exact placement of all vertices, we cannot compute 
 and  exactly. We instead compute positive 
underestimates,  and , by considering hypothetical
vertex placements, and by invoking the following lemma.

\begin{lemma}
\label{lemma-delta}
Let  and  be two neighboring vertices embedded in the 
plane. If there exists a vertex  that is simultaneously 
closest to the perpendicular bisector of  (on the  side), 
and farthest from the line , then the delta value for  to 
through  is the smallest for any choice of . 
\end{lemma}

Applying the above lemma to all hypothetical , , and  placements for the
embedding of  leads to the underestimate 
 where , , and  are shown in Fig.~\ref{fig:delta_0}. Trivially, .

We now show how to obtain a greedy embedding of , given a greedy embedding of  and values  and .

Let .
Trace out a semi-circle of radius  centered at 
the origin. Each cycle at depth  of  has the form  
where , the primary node of , has already been embedded on the th 
semi-circle. We embed vertices  to  in two subphases: \\

\noindent\emph{Subphase 1} We first embed vertex  from each . 
Choose an orientation for  so that  is not a turnpike.\footnote{For 
the case where  is a 2-cycle and  is a turnpike we insert a
temporary placeholder vertex  into  with edges to  and , and treat
 as the new . We can later remove this placeholder by transitivity.} We place  where the ray beginning at the origin 
and passing through  meets semi-circle . We now show that distance 
decreasing paths exist between all pairs of vertices embedded thus far. 

Distance decreasing paths between vertices in  are preserved by the 
induction hypothesis. For  placed during this subphase:  has a neighbor  
embedded on semi-circle . If  then 's neighbor  is strictly closer to . 
Otherwise if  then since  is within distance  of , 
then 's neighbor  that is closer to  is also closer to . 
\ifFull
By definition of , .

Since  is in the -ball around , 
, and .

Then,

Therefore, 's neighbor  that is closer to the primary node  is also closer to .
\fi
If  was placed during this subphase then  is within distance
 from its neighbor , and
the perpendicular bisector of  contains  on one side and 
every other vertex placed on the other side. Therefore 's neighbor 
 is closer to .

The next subphase requires new underestimates, which we call  
and . By construction, . 
No -- paths within  decrease the delta value.
Paths from  to  placed in this subphase have delta value at 
least  by design.
\ifFull This follows directly from the proof of greediness of this 
subphase.\fi
For paths from  placed in this subphase, 's neighbor  is 
the closest vertex to the perpendicular bisector 
of  on the  side. If we translate  along the perpendicular bisector of  
to a distance of  from , this hypothetical point allows us to invoke 
Lemma~\ref{lemma-delta} to get an underestimate for the delta value of all 
paths beginning with . Therefore, our new underestimate is: 
.\\ 

\noindent\emph{Subphase 2} We now finish embedding each cycle 
. Let the value  , 
s.t. .  
Trace out an arc of length  from the embedding of , clockwise
along semi-circle . We evenly divide this arc into  
positions, numbered  to . Position  is already filled by . 
We embed vertices in clockwise order around the arc in  
distinct positions; reserving position  for 's turnpike. If 
there is no such node, position  remains empty.

This completes the embedding of . We show that the embedding of 
 is greedy. We only need to consider distance decreasing paths 
that involve a vertex placed during this subphase. For  placed during
this subphase,  is within distance  from an , 
therefore, all previously placed  have a neighbor  that is closer 
to . If  the 's neighbor closer to  is . 
Finally, for  placed during this subphase, let the cycle that 
 is on be . For ,
since , the interior of the sector formed by 
,  and the origin is empty, therefore
 is either on the  side of the 
perpendicular bisector to  or on the 
side of the perpendicular bisector to .
If  If  is embedded to the left , the closer neighbor 
is . Otherwise, applying Lemma~\ref{lemma-geom},
our choice of 
forces the perpendicular bisector to  to have  on one side, and
all nodes to the right of  on the other side. All cases are considered,
so the embedding of  is greedy. 

To complete the inductive proof, we must compute  and . 
Trivially, .
Distance decreasing paths between vertices placed before this subphase
will not update the delta value. Therefore, we only evaluate paths 
with  or  embedded during this subphase.
By design, paths from  previously placed to  placed during 
this subphase have a delta value . Distance-decreasing paths 
from  placed in this subphase to  take two 
different directions. If 's neighbor  which is closer to 
is on semi-circle  then points that are closest to the 
perpendicular bisector to  are along the perimeter of the sector formed
by , , and the origin. The point closest to the perpendicular bisector 
is where the first semi-circle intersects the sector. We translate this 
point down  units along the perpendicular
bisector, and we have an underestimate for the delta value for
any path beginning with a left/right edge. If 's 
neighbor that is closer to  is on the th semi-circle, then
a down edge is followed. To finish, we evaluate down edges 
added during the second subphase. The closest vertex to the 
perpendicular bisector to  on the  side is either , or the
vertex placed in the next clockwise position the th semi-circle. 
Translating this point  units away from  along 
the perpendicular bisector gives us the an underestimate for
paths beginning with . 

This completes the proof for the greedy embedding of . 
 We call the levels where the on-ramps to heavy paths are embedded 
 \emph{super levels}, 
and all other levels are \emph{baby levels}. There are  
baby levels between consecutive super levels and, since any path from root to leaf
in a depth tree travels through  different heavy paths, there are  super 
levels. 
\ifFull \begin{figure}[!t]
\begin{center}
\begin{tabular}{ccc}
\includegraphics[scale=0.8,viewport=2.0cm 2.0cm 9.5cm 7.5cm,clip=true]{removal1.pdf}
&
\hspace{10pt}
&
\includegraphics[scale=0.8,viewport=2.0cm 2.0cm 9.5cm 7.5cm,clip=true]{removal2.pdf} \\
(a) && (b)
\end{tabular}
\caption{(a) Before removal of dummy nodes and (b) after removal.}
\label{fig:removal}
\end{center}
\end{figure}

\subsection{Obtaining a Greedy Embedding of }

Let  be a modified Christmas cactus graph greedily embedded using the 
procedure discussed above. We now show that collapsing the dummy edges 
leaves us with a graph  and a greedy embedding of . 

Let ,  be any two cycles with a path of dummy edges between 
them. We show that collapsing this path down to a single vertex gives 
us new graph that is also greedily embedded.

\begin{proof}
Assume, without loss of generality, that  is a descendant of .
Let  be the path of dummy edges between  and . Let  be the vertex 
that cycle  shares with , let  be the vertex that cycle  
shares with . 

Collapse the path  down to the vertex , call this new graph . 
We assign vertices in  the same coordinates in  that they are 
assigned in the embedding of . (See Fig.~\ref{fig:removal}.) 
We show that distance-decreasing paths exist between all pairs of vertices 
in the embedding of , using the greediness of the embedding of .

Consider any two vertices  and  in . There are four cases:

\begin{enumerate}

\item If a distance-decreasing path from  to  in  involves both 
 and , then there is a distance-decreasing path in  
by transitivity.

\item If a distance-decreasing path from  to  in  involves  
and not , then the same distance-decreasing path exists in 
since no vertices or edges on this path were modified.

\item If a distance-decreasing path from  to  in  involves  and 
not , then either  or  is not in . Therefore, this case 
is irrelevant.

\item If a distance-decreasing path from  to  in  involves neither 
 nor , then the same distance-decreasing path exists in 
since no vertices or edges on this path were modified.
\end{enumerate}
Therefore, since there are distance decreasing paths between all  and  in
our embedding of , there are distance-decreasing paths between all  and 
in the embedding of our new graph as well.
\end{proof}

Furthermore, every distance-decreasing path in  looks like the same
path from , but with vertices in  removed.

We apply the above modification algorithm to  repeatedly, until all
dummy edges are removed. After removing all of the dummy edges in this
way, we have our original graph  and a greedy embedding of .

\else
To obtain a greedy embedding for , we repeatedly collapse dummy edges
in  until we get . When we collapse an edge , 
where  is the primary node for the 2-cycle, we collapse 
the edge to vertex  in the embedding. Collapsing in the other
direction may break distance decreasing paths through  and neighbors of  
embedded on the same semi-circle as . After collapsing all such 
dummy edges, we have a greedy embedding of .

\fi

\subsection{Our Coordinate System}
Let  be a vertex in . We define  to be the number 
of baby levels between  and the previous super level 
(zero if  is on a super level) and  to be the 
position,  to , where  is placed when its 
cycle is embedded. These values can be assigned to vertices
without performing the embedding procedure.

Let  be 's ancestor on the first super level. 
The path from  to  passes through  heavy paths, entering
each heavy path at an on-ramp, and leaving at an off-ramp. 
We define 's coordinate to be a -tuple
consisting of the collection of  pairs for each 
off-ramp where a change in heavy paths occurs on the path from  to , and the pair
, which is either an off-ramp or a turnpike. 
Using the coordinate for  and the parameter , we can compute the
Euclidean coordinates for all the turnpikes and off-ramps where a change
in heavy path occurs on the path from  to , including the 
coordinate for . Thus, we have defined a coordinate system for 
the Euclidean plane.

Using a straightforward encoding scheme, each level-cycle pair is encoded 
using  bits. Since a coordinate contains  of 
these pairs, we encode each coordinate using  bits. 

\subsection{Greedy Routing with Coordinate Representations}
\ifFull
Although contrived, it is possible to perform greedy geometric
routing by converting our coordinates to Euclidean points and using
the Euclidean  metric whenever we need to make a comparison 
along the greedy route. Alternatively, we can define a comparison rule,
which can be used for greedy routing
in our coordinate system, and which evaluates consistently with the 
 metric for all vertices on the path from start to goal.
\fi

By design, the routing scheme discussed in Sect.~\ref{section:greedy-routing}
is greedy for our embedding. We develop a comparison rule 
using the potential number of edges that may be traversed on
a specific path from  to .

Let  be the vertex between super levels  and , whose 
- pair is in position  of 's coordinate.
We define  similarly. Let  be the position that 
contains the level-cycle pair for  itself.
Let  be the smallest integer such that  and  differ.
Using the level-cycle pairs for  and , we
can compute the level-cycle pair for the off-ramps on the least
common ancestor  that diverge toward  and , which we 
call  and . That is, if 
 then  and
. Otherwise, assume without loss of generality that
, then 's pair is 
and  is a turnpike with the pair . 

We define , , ,  be the potential number of left, 
right, down, and up edges that may be traversed from  to . Values 
 and  are simply the number of semi-circles passed through by 
down and up hops, respectively. That is,

{\small


}

If , then we count the maximum number left 
edges on the path from  to , and the maximum number of right 
edges from  to . That is,

{\small



}

If , then we count the maximum number of right 
edges on the path from  to , and the maximum number of right edges 
from  to . That is,

{\small




}

Our comparison rule is:


Following the routing scheme from Sect.~\ref{section:greedy-routing}, any 
move we make toward the goal will decrease , and all other
moves will will increase  or leave it unchanged. 
Therefore, we can use this comparison rule to perform greedy routing 
in our embedding efficiently, and comparisons made along the greedy route
will evaluate consistently with the corresponding Euclidean coordinates 
under the  metric.

\section{An Optimal Succinct Greedy Embedding}
Conceptually, the  and  values used
in the previous section are 
encoded as integers whose binary representation corresponds to
a path from root to a leaf in a full binary tree with  leaves. Instead 
of encoding with a static  bits per integer, we will 
modify our embedding procedure so we can further exploit the heavy path 
decomposition of the dual tree , using 
\emph{weight-balanced binary trees}~\ifFull\cite{GilMoo-BSTJ-59,Knu-AI-71}.\else \cite{Knu-AI-71}.\fi

\begin{definition} A {\bfseries weight-balanced binary tree} is a binary tree
which stores weighted items from a total order in its leaves. If item 
has weight , and all items have a combined weight of  then item 
 is stored at depth . An inorder listing of the leaves outputs
the items in order.
\end{definition}

\ifFull By using appropriate weight functions with our weight-balanced 
binary trees, we will be able to get telescoping sums
for the lengths of the codes for the  
and  values, 
giving us  bits per coordinate, which is optimal. \fi


\subsection{Encoding the Level Values}
\ifFull
As in the  embedding, we will lay the heavy paths between 
super levels. However, we no longer require the on-ramps of heavy paths to be
embedded on super levels, nor do we require adjacent cycles on the same 
heavy path to be embedded on consecutive levels; instead, cycles will be
assigned to baby levels by an encoding derived from a 
weight-balanced binary tree. \fi

\ifFull
We will have a different weight-balanced binary tree for each heavy path
in our depth tree. 
The items that we store in the tree are the cycles on the heavy path. 
The path in the weight-balanced binary tree from the root to 
the leaf containing a cycle gives us an encoding for the  that the 
cycle should be embedded on between super levels. 
\fi

Suppose we have a depth tree  for , and a heavy path 
decomposition of . Let  be a simple cycle in  on some heavy path
 and let  be the next cycle on the heavy path ,
if it exists. Let  be the number of vertex descendants of 
 in . We define a weight function  on the cycles 
in  as follows:
{\small

}
\ifFull \noindent That is,  is the number of descendants of cycle  in  
excluding the descendants of the next cycle on the heavy path with . \fi

For each heavy path , create a weight-balanced binary tree  containing
each cycle  in  as an item with weight , and impose a total order 
so that cycles are in their path order from  to . 

Let  be a vertex whose coordinate we wish to encode, and suppose  is located
between super levels  and . Let  be the vertex whose -
pair is in position  of 's coordinate. Let  be contained in cycle  
(such that  is not 's primary node) on heavy path . 
\ifFull
Then the coordinate for  will contain the collection of 
 values for each 
off-ramp  on the path to , and the  value for 
itself. Let  be the cycle containing vertex , such 
that  is not the primary node for . \fi 
The code for  is a bit-string representing the path 
from root to the leaf for  in the weight-balanced binary tree 
. Let  be the combined weight of the items in . 
Since  is at a depth of , this is 
length of the code.  
Thus, the level values in 's coordinate are encoded with
 bits total. \ifFull
We now show that this is a telescoping sum, giving us  bits total.
\ifFull
All descendants counted in  are counted in , 
therefore, we have that . By subtracting off  
descendants that are further along the heavy path, we ensure 
that . Thus, . \else
By design, this sum telescopes to  bits.
\fi

\subsection{Encoding the Cycle Values}
For a node  in  we define a weight function  to be the number
of descendants of  in .

Let  be a cycle in , where  is 
the primary node of . Let  be the turnpike that connects 
 to the next cycle on the heavy path, if it exists. 
Let  have weight  and impose a total order so 
 if . For each cycle , we create a weight-balanced binary tree  
containing nodes  to  as follows. We first create 
two weight-balanced binary trees  and  
where  contains  for  and  contains
  for . If no such  exists, then choose an 
integer  and insert items  for  into 
 and insert the remaining items into . We form our single 
weight-balanced binary tree  in two steps: (1) create a tree  with 
 as a left subtree and a node for  as a right subtree, and (2)
form  with  as a left subtree and  as a right subtree.
We build  in this way to ensure that every turnpike is given the same
path within its tree, and hence the same cycle code and value.

The code for  is a bit-string representing the path from root to the 
leaf for  in the weight-balanced binary tree . Let  be the
combined weight of the items in . Since 
is at a depth of , this is length of the code.  
Thus, the  values in 's coordinate are encoded
with  bits total. 
\ifFull We now show that this is a telescoping sum, giving us  
bits total.

Every descendant counted in  is also counted in , 
thus . By design, . Hence 
.
\else
By design, this sum telescopes to  bits.
\fi

\subsection{Interpreting the Codes}

Let  be the smallest integer constant such that item  stored 
in the weight-balanced binary
tree is at depth . We can treat the position of  
in the weight-balanced binary tree as a position in a full binary tree of 
height . We interpret this code to be the number of tree nodes 
preceding  in an in-order traversal of the full binary tree. Using our
codes as described, we require  baby levels between each super 
level and  cycle positions.

\subsection{An Overview of the Optimal Embedding}
Let  be the depth tree for our Christmas cactus graph . 
We create weight-balanced binary trees on the heavy paths in  and
on each of the cycles in , giving us the  and  codes for
every vertex. We adjust the graph modification procedure 
so that adjacent cycles on heavy paths are spaced out according to the
level codes. That is, adjacent cycles on the same heavy path have
heavy dummy edges (dummy edges that are considered to be on 
the heavy path) inserted between them so that they are placed on 
the appropriate baby levels. For cycles on different heavy paths, 
we insert dummy edges to pad out to the next superlevel, 
and heavy dummy edges to pad out to the appropriate baby level. 

We embed the modified graph analogously to our  embedding, except
that the cycle codes dictate vertex placements.
We augment our coordinate system to store the  value 
for elements on the root cycle, otherwise it is not possible to compute the 
corresponding Euclidean point from our succinct representation.
The same comparison rule applies to our new coordinate system, with little
change to account for the new range of  and  values. Using this 
embedding scheme and coordinate system we achieve optimal 
 bits per coordinate.

\ifFull
\section{Conclusion}
We have provided a succinct coordinate-based representation for the
vertices in 3-connected planar graphs so as to support greedy routing in
.
Our method uses  bits per vertex and allows greedy routing 
to proceed using only our representation, in a way that is consistent
with the Euclidean metric.
For future work, it would be interesting to design an efficient
distributed algorithm to perform such
embeddings.
\fi

\bibliographystyle{abbrv}
\bibliography{geom,greedy,roads}

\end{document}
