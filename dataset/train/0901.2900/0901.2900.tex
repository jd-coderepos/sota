\pdfoutput=1
\documentclass[a4paper,12pt]{article}
\usepackage[colorlinks=true, urlcolor=blue]{hyperref}
\usepackage[usenames,dvipsnames]{color}
\input xy
\xyoption{all}
\hypersetup{
    bookmarks=true,         unicode=false,          pdftoolbar=true,        pdfmenubar=true,        pdffitwindow=true,      pdftitle={Fully Dynamic Algorithm for Maximum matching},    pdfauthor={Ankit Sharma, Manoj Gupta, Surender Baswana},     pdfsubject={Subject},   pdfcreator={Creator},   pdfproducer={Producer}, pdfkeywords={keywords}, pdfnewwindow=true,      colorlinks=true,       linkcolor=blue,          citecolor=green,        filecolor=magenta,      urlcolor=blue           }
\usepackage{color}
\definecolor{light-gray}{gray}{0.85}
\usepackage{fancyhdr}
\usepackage{colortbl}
\usepackage[cm]{fullpage}
\pagestyle{fancy}
\renewcommand{\headrulewidth}{0pt}
\renewcommand{\footrulewidth}{0pt}
\title{An O(log(n)) Fully Dynamic Algorithm for Maximum matching in a tree}
\author{Manoj Gupta\footnote{gmanoj@iitk.ac.in}, Ankit Sharma\footnote{ankitsh@iitk.ac.in}\\Indian Institute of Technology Kanpur, India}
\bibliographystyle{unsrt}
\begin{document}
\maketitle
\begin{abstract}
In this paper, we have developed a fully-dynamic algorithm for maintaining cardinality of maximum-matching in a tree using the construction of top-trees. The time complexities are as follows:
\begin{enumerate}
\item Initialization Time:  to build the Top-tree.
\item Update Time: 
\item Query Time:  to query the cardinality of maximum-matching and  to find if a particular edge is matched.
\end{enumerate}
\end{abstract}
\section{Introduction}
Dynamic graph algorithms aim to maintain certain properties in a graph under insertion and/or deletion of edges or vertices from the graph. The motivation behind dynamic algorithms is to maintain the graph property without the need to compute from scratch after each update (insertion or deletion). A dynamic algorithm is incremental and decremental if it can handle the cases of insertion and deletion respectively. A dynamic algorithm is fully-dynamic if it is both incremental and decremental. Dynamic graph algorithms for maintaining connectivity and mimimum spanning tree in a graph have been well studied.

This paper presents a dynamic algorithm for maintaining maximum matching in a dynamic tree. Given a graph , maximum matching gives the subset of  of maximum cardinality such that no two edges in the subset share a vertex. The best static algorithm for finding maximum matching in a general graph has  time complexity \cite{maxmatch-static}. The static algorithm for finding maximum-matching in a tree has  time complexity. The algorithm involves randomly choosing a leaf node and making the edge, incident on the leaf, matched. The matched edge is then deleted from the tree along with all the edges which are incident on the node to which the leaf node was attached. The process is recursively followed on the new tree till no nodes are left. The time complexity of the algorithm is ,  being the number of nodes in the tree, since in each iteration atleast one node is removed and number of edges in a tree are . We omit the proof of correctness of this algorithm.

Among dynamic algorithm for maximum-matching in a graph, the best time complexity is of a randomized algorithm given by Piotr Sankowski which has time complexity of  \cite{maxmatch-dynamic}. In this paper, we give an  time algorithm to maintain maximum matching in a dynamic tree under insertion and deletion of edges,  being the number of nodes in the tree. We make use of top-tree, introduced in section \ref{sec:top-tree}, to represent the dynamic tree. The algorithm, the proof of correctness and time-complexity analysis are presented in subsequent sections.

\section{\label{sec:top-tree}Top Tree}
Top-tree\cite{werneck} partitions a tree into clusters, a cluster being defined as a connected subtree of the given tree. In figure \ref{fig:partition-example}, ,  and  are clusters of the tree. A top-tree has several levels and each level induces a partitioning of the tree into clusters. A higher level in a top-tree partitions the tree into smaller number of larger clusters compared to a lower level. To construct level  from , the clusters at level  are combined according to prescribed rules to form the clusters at the next level in the top tree. At the lowest level, each edge of the tree is a cluster. At the highest level, the whole tree is seen as a single cluster. The number of levels in a top-tree is of  where  denotes the number of nodes in the tree. The reason behind  levels is that the cluster combination operations and the fact that we are dealing with a tree ensure that each successive level has atmost a constant ratio of number of clusters of the previous level.

\begin{figure}
\begin{tabular}{lllll}
\xymatrix{
&&&&g\ar@{.}@/^/[dd] \ar@{.}@/_/[llld]&&J \ar@{.}@/_1pc/[lldd] \ar@{.}@/^1pc/[lldd]|{}="r"\\
&b\ar@{-}[dr] \ar@{.}@/_/[dddr]&&f\ar@{-}[ur] \ar@{-}[dr]&&i\ar@{-}[ur]\\
a\ar@{-}[ur] \ar@{.}@/_/[ur]|{}="p"  \ar@{.}@/^/[ur]&&c\ar@{-}[ur] \ar@{-}[dr]&&h\ar@{-}[ur]&&R\ar@{->}"r"\\
&P\ar@{->};"p"&&d\ar@{-}[dl]\\
&&e\ar@{.}@/_1pc/[rruu]|{}="q"&&Q\ar@{->}"q"&&\\
}&&&&
\xymatrix{
&b\ar@{-}[dr]&&j\\
a\ar@{-}[ur]&&h\ar@{-}[ur]&&\\
}
\end{tabular}
\caption{The figure on the left shows partition of the tree into three clusters ,  and , having boundary nodes ,  and  respectively. For instance, cluster  is the subtree spanned between  and . Clusters  and  are leaf-clusters. The figure on the right shows the equivalent top-tree representation of the tree partition wherein each cluster of the tree is denoted by an edge with end points of the edge being the boundary nodes of the corresponding cluster. For example, cluster  is represented by an edge having end points  and .}
\label{fig:partition-example}
\end{figure}

The cluster formation rules ensure that each cluster is connected to the rest of the tree at atmost two nodes. These two nodes are referred to as `boundary nodes' of the given cluster. Those clusters which share only one node with the rest of the tree are called `leaf-clusters'. Although a leaf cluster has only one node shared with the rest of the tree and therefore in principle should have only one boundary node, yet for the sake of uniformity across all clusters, we assign two nodes of a leaf-cluster as boundary nodes. One of the boundary nodes is the one through which the leaf-cluster is connected to the rest of the tree. The way the other boundary node is decided will be clear from discussion in the following paragraphs. At the lowest level of the top-tree, each edge is a cluster. The leaf edges of the tree are leaf-clusters. The boundary nodes for all clusters are the end-points of the edge forming the cluster.

As mentioned earlier, clusters, at level , are combined to form clusters at level  in a top-tree. Each cluster at level  is either a combination of two clusters of level  or is same as a cluster of level . To form level  clusters, we combine clusters of level  in pairs to form level  clusters till we cannot combine any more clusters. Each cluster of level  can participate in atmost one combine operation with another cluster of the same level to form a cluster at level .

At any level, more than one cluster may share a common boundary node. In this case, we have an ordering of the clusters around the boundary node which is in counterclockwise orientation around the node. In the example below, clusters ,  and  share the boundary node ;  is the successor of  and predecessor of . We call a cluster `incident' on a node if the node is a boundary node of the cluster and the node is a boundary node of atleast one more cluster.

\xymatrix{
&A\\
&B \ar@{}[u]|{P}^{}="u" \ar@{.}@/_/[u] \ar@{.}@/^/[u] \ar@{}[ld]|{Q}^{}="l" \ar@{}[rd]|{R}^{}="r" \ar@/_/"u";"l" \ar@/_/"l";"r" \ar@/_/ "r";"u"   \ar@{.}@/_/[ld] \ar@{.}@/^/[ld] \ar@{.}@/_/[rd] \ar@{.}@/^/[rd]&\\
C&&D
}
Two clusters can be combined only if they have a successor-predecessor relationship. Further, two clusters are combined either by a `rake' operation or a `compress' operation.
\begin{figure}
\begin{tabular}{ccc}
\xymatrix{
&B \ar@{.}@/_/[ld] \ar@{..}@/^/[ld] \ar@{.}@/_/[rd] \ar@{.}@/^/[rd] \ar@{}[ld]|{P}="p" \ar@{}[rd]|Q="q"\\
A&&C \ar@/_/"p";"q"\\
}&
\xymatrix{
\ar@{=>}[rr]^{\textrm{rake}}&&
}&
\xymatrix{
&B \ar@{.}@/^/[rd] \ar@{.}@/_3pc/[rd] \ar@{}[rd]|{R}\\
&&C
}
\end{tabular}
\caption{Example of Rake operation: Here cluster  is raked onto cluster  to form cluster .}
\label{fig:rake-example}
\end{figure}
\begin{figure}
\begin{tabular}{ccc}
\xymatrix{
A \ar@{}[rr]|{P}="p" \ar @/^1pc/ @{.} [rr] \ar@/_1pc/@{.}[rr]& & B \ar@{}[rrr]|{Q}="q" \ar@{<->}@/_1pc/"p";"q"\ar @/^1pc/ @{.} [rrr] \ar@/_1pc/@{.}[rrr]& & & C\\
}&
\xymatrix{
\ar@{=>}[rr]^{\textrm{compress}}&&
}&
\xymatrix{
A\ar@{}[rrr]|{R} \ar @/^1pc/ @{.} [rrr] \ar@/_1pc/@{.}[rrr]&  & & C\\
}
\end{tabular}
\caption{Example of compress operation: Here clusters  and  are compressed to form cluster .}
\label{fig:compress-example}
\end{figure}

\begin{enumerate}
\item Rake operation: One of the two clusters is necessarily a leaf cluster and the leaf cluster should be the predecessor of the other cluster. As shown in diagram, cluster  has boundary nodes  and  and cluster  has boundary nodes  and . Cluster  is a leaf cluster and is connected to the rest of the tree through . Cluster  may or may not be a leaf cluster and is connected to rest of the tree through  and . Cluster  is `raked onto' cluster  to form cluster  with boundary nodes  and . Cluster  contains all the edges and nodes which are part of clusters  and .\\


\item Compress operation: Here the two clusters, sharing a common boundary node, should be the only clusters incident on the shared boundary node. As shown in diagram, cluster  has boundary nodes  and  and cluster  has boundary nodes  and . Further, the shared boundary node  has no cluster incident on it other than  and . Clusters  and  are `compressed' to form cluster  with boundary nodes  and . Cluster  contains all the edges and nodes which are part of clusters  and .\\
\end{enumerate}


Top-tree represents each cluster, at a given level, by an edge with end-points of the edge being the boundary nodes of the cluster. For the clusters which share a boundary node, the edges, corresponding to the clusters, likewise share the common boundary node. Hence, at each level the cluster partition is represented by a tree or a forest formed of the edges representing the clusters. The top tree for a tree is shown in figure \ref{top-tree-example}. The tree is not shown explicitly in the figure and is same as the tree shown at level 0 of the top-tree. Let  denote the edge at level . At level 0, we have the original tree. The edge  is raked onto  around  to give . The edge  and  is compressed at  to give . These are the only operations at this level. Hence,  is the parent of  and  and  is the parent of  and . Cluster-combination operations are performed recursively to give successive levels and these levels form the top-tree for the given tree. 

\begin{figure}
\begin{tabular}{ll}
\xymatrix{
&l=0 &&& l=1 &&& l=2 \\
a \ar@{-}[dr] ^{}="a1" && b \ar@{-}[dl] ^{}="b1" \ar@/_/ "b1";"a1"|{r} && a \ar@{-}[d] ^{}="a2" &&& a \ar@{-}[d]  \\
& c \ar@{-}[d]^{}="c1" &&& c \ar@{-}[d]^{}="c2" \ar@{<->}@/_/|{c} "c2";"a2" &&& e\\
& d \ar@{-}[d]^{}="d1" &&& e\\
& e \ar@{<->}@/_/|{c} "d1";"c1"}&Top tree\\
\xymatrix{
\\
a\ar@{-}[dr] \ar@{.}@/_/[dr] \ar@{.}@/^/[dr] &&b\ar@{-}[dl] \ar@{.}@/_/[dl] \ar@{.}@/^/[dl] &&a\ar@{-}[dr] \ar@{.}@/_/[dr] &&b\ar@{-}[dl] \ar@{.}@/_/[ll] &a\ar@{-}[dr] \ar@{.}@/_/[dddr] &&b\ar@{-}[dl] \ar@{.}@/_/[ll] \\
&c\ar@{-}[d] \ar@{.}@/_/[d] \ar@{.}@/^/[d] &&&&c\ar@{-}[d] \ar@{.}@/_/[ur]  \ar@{.}@/_/[dd] \ar@{.}@/^/[dd] &&&c\ar@{-}[d]\\
&d\ar@{-}[d] \ar@{.}@/_/[d] \ar@{.}@/^/[d] &&&&d\ar@{-}[d]&&&d\ar@{-}[d]\\
&e&&&&e&&&e\ar@{.}@/_/[uuur] 
}&\begin{tabular}{l}Underlying partitioning\\of the tree into clusters\end{tabular}
\end{tabular}
\caption{This figure shows top-tree for the tree given at l=0 level. In the upper half of the figure, the different levels of the top-tree are shown. The lower half shows the underlying partition of the tree into clusters at each level. The arrows shown in the construction of top-tree are labeled by  and  for rake and compress operation respectively.}
\label{top-tree-example}
\end{figure}

\subsection{Updation in a Top Tree}
Updation in a top-tree under insertion or deletion of edges can be performed in  time. Addition or deletion of an edge makes some of the cluster-combination operations performed in the original top-tree invalid in the modified tree. Further, certain new combination operations might arise. We undo the invalid operations of the original top-tree and reconstruct the new top-tree level-by-level.

We explain the updation operation in a top-tree with the help of an example. For a more detailed description and the proof for its time complexity, we refer the reader to the Renato F. Werneck's thesis\cite{werneck}. Let us take the tree constructed in the earlier section and look at what happens when an edge  is added at the lowest level. The new top tree is given in Figure \ref{fig:top-tree-updation}. Let us see how we can transform the original top tree above to top tree below.

\begin{figure}
\xymatrix{
&l=0 &&& l=1 && l=2 && l=3\\
a\ar@{-}[dr] ^{}="a1" && b \ar@{-}[dl] ^{}="b1"  \ar@/_/|{r} "b1";"a1"   && a\ar@{-}[d] ^{}="a2" && a\ar@{-}[d] ^{}="a3" && a\ar@{-}[d]\\
& c \ar@{-}[d]^{}="c1" &&& c \ar@{-}[d] ^{}="c2"   \ar@{<->}@/_/|{c} "c2";"a2"  && d \ar@{-}[d] ^{}="d3" && f\\
& d \ar@{-}[dl]^{}="d1" \ar@{-}[dr]^{}="e1" &&& d \ar@{-}[d] ^{}="d2" && f \ar@{<->}@/_/|{c} "d3";"a3" \\
e && f \ar@/_/|{r}"d1";"e1"  && f}
\caption{This figure shows the construction of the new top-tree on addition of the edge . The arrows shown in the construction of top-tree are labeled by  and  for rake and compress operation respectively.}
\label{fig:top-tree-updation}
\end{figure}

We have to add cluster  at the lowest level. This makes the compress operation of  and , made in the original top-tree, invalid as the common shared boundary node  now has and additional cluster  incident on it. As the original compress operation is invalid, hence at level ,  is deleted. Further, a new rake operation of  onto  is performed to give  at level . Edge  is also added at level  as  could not participate in any combine operation at level . At level , edge  is deleted since its child  has been deleted. Also, edges  and  are added. Similarly, we build rest of the top-tree.

The construction of the top tree ensures that there are only a constant number of addition and deletion operations at each level. Hence, the time to update the top tree is . The intuitive reason behind the constant number of operations at each level is that each cluster is connected to the rest of the tree at atmost 2 points which means that any change to this cluster would affect only a constant number of clusters at that level namely the clusters which are the predecessor or successor of the cluster at its two boundary nodes. This completes our discussion on top-trees.

\section{Algorithm}
\subsection{Observations\label{observations}}
We now present an  update time fully dynamic algorithm for maintaing maximum-matching in a tree. We first present a few observations which aid in the development of the algorithm and then the algorithm itself.

{\bf Observation 1:} If we are able to maintain certain information with each cluster in a top-tree such that the information for a parent cluster can be derived from the information stored in its child clusters, then any change in information due to insertion or deletion of an edge at the bottom-most level of the top-tree can be propagated up the top-tree in as much time as the update operation takes for a top-tree. This observation motivates us to store some information with each cluster such that it follows the property mentioned above and that it helps in maintaining the maximum-matching.

Before we present the information which we store with each cluster, let us introduce a few notations to aid us in this task. In a tree , let  denote the set of edges contained in . We call a node `matched' if any of the edges incident on it are matched. If none of the edges incident on a node is matched, we call the node `unmatched'. If  be a subtree of tree  and , a particular matching of , let  denote the status - matched or unmatched - of node  under matching  when restricted to the subtree  i.e. we consider only those edges which are in  and are incident on  and in case, any of these edges under consideration are matched under M, we put  as matched and unmatched otherwise.



In the diagram below, a particular matching  is shown for tree  wherein the dashed edges are matched and the solid edges are unmatched.  is the subtree spanning between  and .  is unmatched since the only edge under consideration A--q1 is unmatched. Further,  is matched since out of the two edges, A---t1 and A---q1, under consideration, A---t1 is matched. Again,  is matched since B---q3, the only edge under consideration, is matched and  is also matched since out of the two edges B---q3 and B---t2, under consideration, B---q3 is matched.
\begin{tabular}{c}
\xymatrix{
\\
& A\ar @{-}[dr] \ar @{.}@/^3pc/ [rrrr]^{\textrm{Subtree Q}} \ar @{.}@/_5pc/ [rrrr]& & q2 \ar @{-}[dr]& & B\ar @{-}[dr] & \\
t1\ar @{--}[ur]& & q1 \ar @{--}[ur]& & q3 \ar @{--}[ur] & &t2 \\
\\
}
\end{tabular}


If a node is matched, we call it black () and if it is unmatched, we call it white (). In the previous example, , ,  and .


As mentioned earlier, each cluster in a top-tree is a sub-tree of the tree  such that it shares atmost two nodes with the rest of the tree, which we call the boundary nodes. Let us take a particular cluster, say , having boundary nodes  and . Let us consider the set of matchings in , which make node  matched () and  unmatched () and consider the matching  which has the maximum cardinality in this set. We denote the cardinality of this particular matching  by  where  and  denote that  is matched () and  is unmatched (). ,  and  are similarly defined.

We now return to our orginal question of what information should be maintained with each cluster in a top-tree such that it helps in maintaining maximum-matching and follows the property that the information of the parent can be deduced from that of its children. With each cluster , we maintain the following information: , ,  and . 

The reason behind this choice of information is presented below.  Essentially, this information fulfills the two critera mentioned above. First, the information is sufficient to calculate maximum-matching as the cardinality of maximum-matching of , denoted by , is the maximum number among , ,  and  since these four cases exhaust the set of possibilities of matchings cases of  and .

Secondly, this information obeys the property that the information of the parent cluster can be calculated using the information stored in the child-clusters. To see how, we present the following observation.


{\bf Observation 2:} Let us consider two trees  and  sharing a common node  as shown in figure. The internal structure of the subtrees is not shown in the figure. 

\xymatrix{
\\
\\
A\ar @/^2pc/ @{.} [rrr]^{\textrm{Subtree P}} \ar@/_2pc/@{.}[rrr]& & & B \ar @/^2pc/ @{.} [rrrr]^{\textrm{Subtree Q}} \ar@/_2pc/@{..}[rrrr]& & & & C
\\
\\
}

Let tree  denote the union of subtrees  and . Say we take a matching  of  and a matching  of  and produce a matching  of  as follows -- under matching , an edge  follows matching imposed by  if  and follows matching imposed by  if . Clearly, we cannot combine any two arbitrary matchings  and , since then we can have two matched edges incident on node , one in  and the other in . Hence, we can combine only those matchings of  and  which lead to atmost one matched edge incident on , in other words, atmost one of  and  is black.

Let us suppose we have , ,  and  for  and , ,  and . Can we generate  from this available information? We can see that by imposing the constraint that we can combine only those matchings of  and  which yield atmost one matched edge on  in , we get


In the above equation, we can understand that LHS  RHS, as each possibility in RHS is a valid combination of matchings of  and  inferred from the discussion above. The reason for LHS = RHS is that the options in RHS exhaust the set of possibilities. Similarly, we can obtain ,  and . Again,  would be the maximum of the four numbers.

Hence, we can see that if we know cardinality of certain constrained maximum-matchings for sub-trees  and , we can generate the cardinality of constrained maximum-matchings for . This fact is used in the algorithm. With each cluster, we maintain the {\em information} of cardinality of the four constrained maximum-matchings. The information of a parent cluster can be calculated using the information of the child-clusters. To obtain the cardinality of maximum-matching for the whole tree, we find the maximum cardinality among the four constrained maximum-matchings for the cluster at the highest level of the top-tree. Further, whenever we add or delete an edge, it is an O(1) time operation to update the information for each modified cluster in the top-tree as the information of the children clusters is enough to calculate information of the parent cluster.

\subsection{Algorithm}
We now describe the algorithm for maintaing maximum-matching in a tree using a top-tree. With each cluster  with boundary nodes, say,  and  we maintain the four values , ,  and .

For the base case, where each cluster, say , consists of an edge , we have  and  are .  and  are invalid cases since the only edge  can either be matched or unmatched meaning either both  and  are either matched or unmatched. For the invalid case, we assign a special symbol  i.e.  and . Any addition operation with a  value yields a  value and a maximum operation over a set consisting of  and non- values should yield maximum value over non- values if there are any and  otherwise.

Let us now consider how we maintain the values  when we have a rake or a compress operation. We have the following tables. The notation of clusters ,  and  are same as shown in Figures \ref{fig:rake-example} and \ref{fig:compress-example}.

\begin{itemize}
\item Compress
\begin{enumerate}\item 
\item 
\item 
\item 
\end{enumerate}


\item Rake
\begin{enumerate}
\item \}
\item \}
\item \}
\item \}
\end{enumerate}
\end{itemize}

\subsection{Illustration}
Let us take an example to illustrate the algorithm (figure \ref{fig:top-tree-aug}). We augment the top-tree of the earlier example with the information stored at each cluster. At each cluster ,  and  being the boundary nodes of , we store the tuple . For the purpose of illustration, we choose that boundary nodes as  such as  is situated vertically below  in the figure. For example in , we choose  and  as  occurs above  in the illustration. Further, we have represented  value by \,\,\)}="a2" && a \ar@{-}[d]^{(1,2,2,2)}  \\
& c \ar@{-}[d]^{(0,\,1)}="c1" &&& c \ar@{-}[d]^{(0,1,1,\,\(d,f),\,\,1)}="a2" && a \ar@{-}[d]^{(1,1,1,\,\,\,1,1)}&&f\\
& d \ar@{-}[dl]_{(0,\,1)}="d1" \ar@{-}[dr]^{(0,\,1)}="	d2" &&& d\ar@{-}[d]^{(0,\(d,f)O(log(n))O(1)ii+1i+1i+1O(log(n))O(log(n))O(log(n))kkkkO(k*log(n))M_{Q}^{b_{A}b_{B}}=w(A,B)Q=(A,B)w(A,B)(A,B)O(log(n))$ update time fully-dynamic algorithm for maintaining maximum-matching in a tree. 
Future work on developing dynamic algorithms for maximum-matching for general graphs in particular bipartite graphs and planar graphs.

\section{Acknowledgment}
We would like to acknowledge the effort and energy put in by Dr. Surender Baswana in this research.

\bibliography{dynamictreematching}
\end{document}
