
This section proves that the SyGuS problem for uninterpreted functions with equality (EUF) is undecidable. Our proof is based on a reduction from the Post correspondence problem (PCP)~\cite{post1946variant}. 
The reduction relies on a tree grammar that produces only applications of unary function symbols, showing that for EUF, even basic instances of SyGuS are undecidable. 



Recall that the Post correspondence problem is to decide, given pairs of finite strings , the existence of a non-empty sequence of integers  ( for all ) such that . 
PCP is undecidable even when the strings only contain two symbols, so we will focus on PCP problems with the alphabet . 

As an example consider the Post correspondence problem for the pairs of strings , , and  . 
The sequence of indices 1 2 2 3 is a solution: . 


\begin{theorem}
\label{thm:euf_und}
SyGuS for EUF is undecidable for the class of EUF-compatible tree grammars. 
\end{theorem}

\begin{proof}
Let  be an instance of PCP and let , , , , and  be (distinct) unary function symbols. 
From the PCP instance, we will construct an EUF-compatible tree grammar  and a correctness constraint . 



The terminals of the grammar are ``'', ``'', ``'', ``'', ``'', ``'', ``'', and ``''.
Given a pair  with  and , we add the following production rules to :


where  is the -th symbol in the string .
(The production rules occur twice in similar form, once for the start symbol  and once for , to guarantee that at least one pair of symbols  is produced and so the corresponding PCP solution is non-empty.)
To allow the termination of a production sequence we further add the production .

Intuitively, for any sequence of integers  such that , we would like  to be satisfied by the string constructed by following the productions .
(Since there is at most one non-terminal on the right-hand side of each production rule, this description yields a unique expression.)
We now construct the correctness constraint  such that chains of productions are valid iff the corresponding sequence of indices is a solution to the PCP instance.

Let  be the conjunction of the following constraints:


That is, we require that applications of the functions  commute with applications of the functions  (but not necessarily  with  or  with~). 
Further, applications of  and  and likewise of  and  are required to cancel each other when they occur directly inside an application of . 
We then define the correctness constraint  to be . 
Note that  is constructed independently of the PCP instance being studied. 



The easy direction is to show that every sequence of indices  such that  gives us a sequence of productions  that produces an expression that is valid under every model .
Given any model  that satisfies the antecedent , we can commute the outermost applications of a primed and an unprimed function (of , , , and ) to the ``top''. 
By repeatedly pulling matching applications to the top and eliminating them inside the application of , we arrive at the term , which satisfies the correctness specification.






We now turn to the other direction. 
Let  be a sequence of indices that is not a solution to the PCP problem. 
We show that for the expression obtained by the productions  there is a model that invalidates the correctness specification . 
We construct  such that the antecedent  holds for all , but  does not model  for any . 
We define  and define , , , and  to be functions from   to . 
We define these functions as follows, for  and :


That is,  and  add elements to the first list,  and  add elements to the second list, and  removes the common prefix. 
It is easy to verify that these function definitions satisfy the antecedent .

Assume the variable  in the correctness specifications is instantiated with the tuple  of empty strings. 
If the sequence of letters of  and  produced by the sequence of indices is different in length (say the first is shorter), then we can apply the function definitions above such that the correctness constraint reduces to  with  and is thus violated. 
If the sequences of letters  and  are of the same length but have a position in which they are different, we can apply the rules above to eliminate the common prefix and again arrive at a tuple that is different to .

We conclude that for any given sequence of indices that is not a solution to , the corresponding production does not yield an expression that is valid for every model. 
Since any sequence of productions of the grammar corresponds also to a sequence of indices this concludes the proof. 
\qed
\end{proof}





\subsection{Implications for Arrays}


\begin{corollary}
     SyGuS for AR is undecidable with the class of AR-compatible tree grammars, even when arrays cannot be nested. 
    \label{cor:arr_und}
\end{corollary}
\begin{proof}
We can recast the functions in the proof of Theorem \ref{thm:euf_und} as arrays to create a SyGuS problem in AR, which can only be solved if there exists a solution to the original SyGuS problem in EUF.
Take any Post-correspondence problem  and grammar  as defined above.
Let  be the EUF constraint formula defined above. 
We can replace any function call  in  or  with the statement  to get a SyGuS problem in AR.
It is easy to check that this problem can be solved if, and only if, the original SyGuS problem in EUF is solvable.
\qed
\end{proof}

\subsection{Extending the Undecidability Result to Regular Grammars}

In the proof of Theorem~\ref{thm:euf_und}, we assumed that the provided grammar was a tree grammar with at most one non-terminal on the right side of any production rule. 
We may wonder whether an even more restricted grammar could allow the SyGuS problem to become decidable. Consider the right-regular grammars, which generate the regular languages. 
These grammars cannot even produce the language of well-matched parenthesis, so we cannot expect them to only yield well-formed strings, that is strings that are expressions in the theory. 
However, given a right-regular grammar  and the grammar of well-formed strings , we can construct a context-free grammar  such that . 
Unfortunately, even in this restricted form, we can still specify the grammars used in the above undecidability proofs of EUF and arrays.

\begin{corollary}
SyGuS for EUF with right-regular grammars is undecidable.
\end{corollary}
\begin{proof}
Let  be a PCP instance. 
We construct  with the following productions:


For each  in , with  and , we add the following two production rules to :




We can see that the set of well-formed expressions from  precisely matches those of grammar  as defined in Theorem~\ref{thm:euf_und}. \qed
\end{proof}




\begin{corollary}
    SyGuS for AR is undecidable with regular languages, even when arrays cannot be nested.
\end{corollary}
\begin{proof}
Let  be a PCP instance. We create a grammar  and add the following productions:


\renewcommand{\read}[0]{\mathit{read}}

For each  in , with  and , we add the following transitions to : 



The well-formed expressions of  precisely match the expressions from Corollary~\ref{cor:arr_und}. 
\qed
\end{proof}




\subsection{On Well-Formed and Ill-Formed Expressions}

When we allow the grammar in a SyGuS problem to produce ill-formed expressions and consider the intersection with the expressions of the background theory, as in the previous subsection, we can derive a very strong proof of undecidability. 
The proof only relies on the well-formedness of expressions, and holds independently of the interpretation of the function symbols.
That is, the proof also holds for theories other than EUF. 



\begin{theorem}
SyGuS for CFGs is undecidable for any theory  with domain  containing 
two function symbols  and  of different non-zero arities  and . 
\end{theorem}

\begin{proof}
This proof is by reduction from the Post-correspondence problem.
Let  and  be function symbols in  of arity  and , respectively, where  and .
Let  be a PCP instance with alphabet . 
We construct a grammar  so that sequences of indices that are a solution to the PCP problem correspond to derivations in the grammar that produce a well-formed expression, and where sequences of indices that are not a solution to the PCP problem correspond to derivations in the grammar that produce an ill-formed expression. 

For each pair  in  we construct two terminals  and  in . 
For each position  in the sequence  we define  if  and  if . 
Correspondingly, we define for each position  in  that 


where  is the length of the word  and  is some variable.
That is, the terminals  are lists of lists of arguments of length  and  that are supposed to match the function calls of  and , respectively. 
We are only providing  and  arguments, respectively, as the first argument will (except for the innermost level) be the next function call.
Within the terminals  the argument lists occur in \emph{reverse order} compared to the strings~. 

We add two nonterminals  and  and for each pair of strings  in  we add one production rule for both nonterminals:

That is, the production rules produce a string over opening parts of function calls () matching, in reverse order, a string of argument lists () which each are reversed internally. 
Finally, we add one production rule , which provides the ``missing'' argument for the innermost function call. 
(The need for two nonterminals comes from the fact that we have to disallow the derivation yielding only .)

We can then define the correctness constraint  as , where  must be replaced by some expression in . 
For this correctness constraint, each syntactically correct replacement of  is valid in every theory. 
Thus, this SyGuS problem has a solution, if and only if the PCP instance  has a solution. 
\qed
\end{proof}

\begin{example}
Let , and . Then  is a solution to the Post-correspondence problem. Let  and  be functions in the background theory  with arities 1 and 2, respectively. We can associate  with  and  with . 
The grammar  can produce the expression , which corresponds to the solution \emph{bba}. 
\end{example}



