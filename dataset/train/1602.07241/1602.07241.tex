We study the complexity of one of the most basic problems in pattern matching, that of approximating the Hamming distance.  Given a pattern $P$ of length $n$  the task is to output a $(1+\eps)$-approximation of the Hamming distance between $P$ and every $n$-length substring of a longer text. We provide the first efficient one-way randomised communication protocols as well as a new, fast and space efficient streaming algorithm for this problem.  

The general task of efficiently computing the Hamming distances offline between a pattern and a text has been studied for many years.  When the input is binary and the text has length proportional to that of the pattern, then all outputs can be computed exactly in $\Oh (n\log{n})$ time by repeated application of the fast Fourier transform~\cite{FP:1974}. For larger alphabets, $\Oh (n\sqrt{n\log{n}})$ time solutions were first discovered in the 1980s~\cite{Abrahamson:1987,Kosaraju:1987}. The fastest randomised algorithm for $(1+\eps)$-approximate Hamming distance computation for large alphabets was due for many years to Karloff from 1993~\cite{Karloff:1993} running in $\Oh (\eps^{-2} n\log^2{n})$ time overall.  In a breakthrough paper in 2015 a new algorithm was given improving the time complexity to $\Oh (\eps^{-1} n\log^3{n}\log{\eps^{-1}} )$~\cite{KP:2015}.  These fast methods all require linear space and up until this point no sublinear space solutions have been known.

The first basic question that arises is whether it is in fact possible to give a $(1+\eps)$-approximation to the Hamming distance in a streaming setting while using only sublinear space. In order to explore this question we start our study by considering two natural communication complexity problems which may also be of independent interest. Any lower bound for these communication problems will give a lower bound for the space usage of a corresponding streaming algorithm. This follows from a standard reduction where a space efficient streaming algorithm is converted into a communication protocol by taking a snapshot of memory after some symbol of the input has been read in and then sending that this snapshot to the other player. On the other hand, the upper bounds we provide will set targets for space bounds in the streaming setting.

Any streaming pattern matching algorithm using a pattern of length $n$ can be reduced to repeated application of a streaming algorithm that runs on texts of length $2n$. This is done by splitting the stream into substreams of length $2n$ which overlap by $n$ symbols.  As a result we consider communication problems with these parameter settings for pattern and text length.

\begin{problem}\label{prob:pattern-known}
Consider a text $T$ of length $2n$ and a pattern $P$ of length~$n$. Let Alice hold the information about the first half of the text and the whole of the pattern, and let Bob hold the information about the second half of the text and the pattern. Bob must output $(1+\eps)$-approximations of the Hamming distance for each alignment of  $P$ and $T$.
\end{problem}

A lower bound for the communication complexity of this problem follows from a combination of the lower bound for the communication complexity of a windowed counting problem introduced by Datar et al. in 2002~\cite{DGIM:02} and the one-way communication complexity lower bound for approximating the Hamming distance between two $n$-bit strings from~\cite{JW:2013}. 

For the first part consider the following communication problem. Assume that there is a bit vector $B$ of length $2n$.  Let Alice hold the information about the first half of $B$, and let Bob hold the information about the second half of $B$. Bob must output $(1+\eps)$-approximations of the number of set bits in each window of length $n$. Datar et al. showed that 
Alice will have to send to Bob $\Omega (\eps^{-1} \log^2 \eps^{-1} n)$ bits of information. There is a straightforward reduction from this basic counting problem to Problem~\ref{prob:pattern-known} which then gives us the same lower bound. We set $T = B$ and $P = 00 \ldots 0$ and then a $(1+\eps)$-approximation of the Hamming distance at an alignment $i$ of  $P$ and $T$ gives  a $(1+\eps)$-approximation of the number of set bits in the window $T[i, i+ n-1]$. For the second part we use Theorem 4.1 from~\cite{JW:2013} which states that the one-way communication complexity of $(1+\eps)$-approximate Hamming distance for two strings of length $n$ is $\Omega(\eps^{-2} \log{n})$ for constant error probability. 
Combining these two lower bounds together we get a lower bound of $\Omega(\eps^{-2}\log{n} + \eps^{-1}\log^2{\eps^{-1} n})$ for the communication complexity of Problem~\ref{prob:pattern-known}.  

Our first result is an efficient one-way communication protocol for Problem~\ref{prob:pattern-known} whose complexity is only slightly higher than this lower bound.  In our protocol Alice uses the fact that Bob knows the pattern as well to give an efficient encoding for parts of her half of the text which are at small Hamming distance from the pattern.

\begin{theorem}\label{th:pattern-known}
Problem~\ref{prob:pattern-known} has one-way randomised communication complexity $\Oh (\eps^{-4}\log^2 n)$.
\end{theorem}

As a model for streaming pattern matching, this communication upper bound requires that a copy of the pattern is available at all times.  Our main interest is however in algorithms whose total space complexity is sublinear in the pattern size. In order to model this situation more accurately we now consider a stronger three party communication problem.  

\begin{problem}\label{prob:pattern-unknown}
Assume that there is a text $T$ of length $2n$ and a pattern $P$ of length~$n$. Let Alice hold the information about the pattern, let Bob hold the information about the first half of the text, and let Charlie hold the information about the second half of the text. Alice will send one message to Bob who will then send one message to Charlie. Charlie must output $(1+\eps)$-approximations of the Hamming distance for each alignment of $P$ and $T$.
\end{problem}

Somewhat surprisingly, we are still able to obtain a sublinear protocol for this new problem although the bound is higher than for the simpler Problem~\ref{prob:pattern-known}.  The main technical elements of this communication protocol combine the newly introduced idea of approximate periods with succinct run-length encoded representions of the input.



\begin{theorem}\label{th:pattern-unknown}
Problem~\ref{prob:pattern-unknown} has one-way randomised communication complexity $\Oh (\eps^{-2}\sqrt{n} \log n)$.
\end{theorem}

Having investigated the communication complexity of $(1+\eps)$-approximate Hamming distance we can now define the streaming $(1+\eps)$-approximate Hamming distance problem.

\begin{problem}\label{prob:streaming}
Consider a pattern $P$ of length $n$ and a stream arriving one symbol at a time. We must output a $(1+\eps)$-approximation to the Hamming distance between $P$ and the latest $n$-length suffix of the stream as soon as a new symbol arrives. In this problem we cannot, for example, store a copy of the pattern or stream without accounting for it.
\end{problem}

The upper bounds for the communication complexity of Problem~\ref{prob:pattern-unknown} suggest space upper bounds we shall aim for in order to develop an optimal algorithm for the $(1+\eps)$-approximate Hamming distance in the streaming setting. We make the first step towards this direction and show two randomised sublinear-space algorithms for the problem. We start by giving a solution for the case when both the pattern and the text are binary strings.  

\begin{theorem}\label{thm:streaming}
When both $P$ and $T$ are binary, there is an algorithm for Problem~\ref{prob:streaming} which uses $\Oh(\eps^{-3} \sqrt{n} \log^{2} n)$ bits of space and runs in $\Oh(\eps^{-2} \log{n})$ time per arriving symbol.
\end{theorem}

The same bounds hold for alphabets of constant size $\sigma$ as we can map each occurrence of the $i^{th}$ symbol of the alphabet in the pattern or in the text to a binary string $0^{i-1} 1 0^{\sigma-i}$, which will result in doubling the Hamming distance between the pattern and the text at each particular alignment. 

For polynomial size alphabets our bounds are higher by a factor $\eps^{-2} \log^2 n$ and our approach is based on the mapping idea of Karloff~\cite{Karloff:1993}. In that paper he showed that there exists $\Theta(\eps^{-2} \log^2 n)$ mappings $map_j$ of the alphabet onto $\{0,1\}$ such that an $(1+\eps/3)$-approximation of the Hamming distance between $P$ and $T$ at a particular alignment can be given by a normalised average of the Hamming distances between $map_j(P)$ and $map_j(T)$ at this alignment. Moreover, Karloff showed that the mappings can be generated in $\Oh (\eps^{-2} \log^3 n)$ space and $\Oh (\log n)$ time per symbol. For each pattern-text pair mapped on to a binary alphabet we then run the algorithm of Theorem~\ref{thm:streaming} to find $(1+\eps/3)$-approximations and finally obtain: 

\begin{theorem}
There is an algorithm for Problem~\ref{prob:streaming} which uses $\Oh(\eps^{-5} \sqrt{n} \log^{4} n)$ bits of space and runs in $\Oh(\eps^{-4} \log^3 {n})$ time per arriving symbol.
\end{theorem}

Our solution has guaranteed worst case complexity per arriving symbol and uses roughly the square root of the space required by the known offline $(1+\eps)$-approximate algorithms. A key technical innovation for our space reduction is the notion we introduce of a super-sketch. This a compact and efficiently updateable representation of consecutive text substrings which we require to be able to achieve sublinear space.  For simplicity we will make the natural assumption throughout that  $\eps < 1/2$.

\subsection{Related work and lower bounds}
The one-way communication complexity of a number of variants of Hamming distance computation has been studied over the years. These includes $(1+\eps)$-approximation~\cite{JW:2013}, the so called gap Hamming problem~\cite{CR:2012} and a bounded version known as $k$-mismatch~\cite{HSZZ:06}. However all this previous work has assumed that both Alice and Bob have strings of the same length and so need only give a single output. There has also been great interest in efficient streaming algorithms over the last 20 years, following the seminal work of~\cite{AMS:1996}. In relation specifically to pattern matching problems,  where space is not limited but where an output must be computed after every new symbol of the text arrives, the Hamming distance between the pattern and the latest suffix of the stream can be computed online in $O(\sqrt{n\log{n}})$ worst-case time per arriving symbol or $O(\sqrt{k}\log{k} + \log{n})$ time for the $k$-mismatch version~\cite{CS:2010}.  Both these methods however require $\Theta(n)$ space.  Using the same approach, a number of other approximate pattern matching algorithms have also been transformed into efficient linear space online algorithms including~\cite{AAKLP:2008,AABLLPSV:2009,AALP:2006,AFM:1994, AEE:2006,ACHP:2003,LV:1988kdiff}. In 2013 a small space streaming pattern matching algorithm was shown for parameterised matching~\cite{JPS:2013} and in 2016 for the $k$-mismatch problem~\cite{CFPSS:2016}. The latter $k$-mismatch paper is of particular relevance to our work. In ~\cite{CFPSS:2016} as a part of a space-efficient streaming algorithm for the $k$-mismatch problem, the authors presented a $(1+\eps)$-approximate algorithm with space $\Oh(\eps^{-2} k^2 \log^7 n)$ and running time $\Oh(\eps^{-2} \log^5 n)$ per arriving symbol that returns a $(1+\eps)$-approximation for all alignments of the pattern and text where the Hamming distance is at most $k$. The algorithm we give in this current paper can be seen a generalisation of this work, both removing the requirement for a prespecified threshold $k$ and also using less space when $k \gtrsim n^{1/4}$.
 
 
