\documentclass{jac}
\usepackage[utf8]{inputenc}

\usepackage{stmaryrd,amsmath,amsthm,amsfonts}
\usepackage{color,tikz}
\usepackage[pdftex,bookmarks=true,bookmarksnumbered=true,pdfpagelayout=SinglePage,colorlinks=false]{hyperref}

\let\dfn\emph
\let\impl\Rightarrow
\newcommand{\resp}[1]{\ (resp. #1)}
\newcommand{\TODO}[2][]{ \textcolor{red}{À FAIRE #1 }#2\textcolor{red}{} }
\newcommand{\BIB}[1]{\fbox{\small BIB #1}}
\newcommand{\Z}{\mathbb Z}
\newcommand{\Q}{\mathbb Q}
\newcommand{\N}{\mathbb N}
\newcommand{\M}{\mathbb M}
\newcommand{\Ns}{\N_+}
\newcommand{\az}{{A^\Z}}
\newcommand{\an}{{A^\N}}
\newcommand{\sett}[2]{\left\{\left.#1\vphantom{#2}\right|#2\right\}}
\newcommand{\set}[3]{\sett{#1\in#2}{#3}}
\newcommand{\oo}[2]{\left\rrbracket #1,#2\right\llbracket}
\newcommand{\cc}[2]{\left\llbracket #1,#2\right\rrbracket}
\newcommand{\scc}[2]{_{\cc{#1}{#2}}}\newcommand{\soo}[2]{_{\oo{#1}{#2}}}\newcommand{\co}[2]{\left\llbracket #1,#2\right\llbracket}\newcommand{\sco}[2]{_{\co{#1}{#2}}}\newcommand{\kaprx}[1]{\mathcal A_{#1}}

\newcommand{\dinf}[1]{\vphantom{#1}^\infty{#1}^\infty}
\newcommand{\lang}{\mathcal L}
\newcommand{\ie}{\textit{i.e.}\ }

\newcommand{\restr}[1]{_{\left|#1\right.}}
\newcommand{\soit}[1]{\left|\everymath{\displaystyle\everymath{}}\begin{array}{ll}#1\end{array}\right.}
\newcommand{\appl}[5]{\begin{array}{rrcl}#1:&#2&\to&#3\\
&#4&\mapsto&\displaystyle#5\end{array}}

\theoremstyle{definition}
\newtheorem{exmp}[thm]{Example}

\begin{document}

\title{Clandestine Simulations in Cellular Automata}

\author[lab1]{P. Guillon}{P. Guillon}
\address[lab1]{Department of Mathematics, University of Turku,\newline 20014 Turku, Finland}
\email[P. Guillon]{piegui@utu.fi}

\author[lab2]{P.-E. Meunier}{P.-E. Meunier}
\address[lab2]{LAMA (CNRS, Universit\'e de Savoie),\newline Campus Scientifique 73376 Le Bourget-du-Lac Cedex, France}
\email[P.-E. Meunier]{pierreetienne.meunier@univ-savoie.fr}
\email[G. Theyssier]{guillaume.theyssier@univ-savoie.fr} 

\author[lab2]{G. Theyssier}{G. Theyssier}

\thanks{Research partially supported by projects CONICYT-ANILLO ACT88 (Chile), FONDECYT 1090156 (Chile), ANR EMC NT09 555297 (France) and Academy of Finland project 131558.}

\keywords{Cellular automata, Simulation, Limit sets, Trace, Universality}
\subjclass{F.1.1; F.4.3}

\begin{abstract}
  \noindent This paper studies two kinds of simulation between cellular
  automata: simulations based on factor and simulations based on
  sub-automaton. We show that these two kinds of simulation behave in
  two opposite ways with respect to the complexity of attractors and
  factor subshifts. On the one hand, the factor simulation preserves the
  complexity of limits sets or column factors (the simulator CA must
  have a higher complexity than the simulated CA). On the other hand,
  we show that any CA is the sub-automaton of some CA with a simple
  limit set (NL-recognizable) and the sub-automaton of some CA with
  a simple column factor (finite type). As a corollary, we get
  intrinsically universal CA with simple limit sets or simple column
  factors. Hence we are able to 'hide' the simulation power of any CA
  under simple dynamical indicators.
\end{abstract}

\maketitle

\section*{Introduction}

Since the introduction of the model in the 40s, cellular automata have
been studied both as dynamical systems and as a computational model. In both aspects, they can show very complex behaviors, be it through their topological dynamics \cite{kurka} or through their ability to compute \cite{vneumann2,univind}. As such, they constitute a good model to tackle one of the major question of natural computing: what kind of dynamical behavior allows to support computation, and reciprocally, what does the ability to compute imply on the dynamical behavior of a system?

In this paper, we focus on asymptotic dynamics of cellular automata (notion of limit set \cite{cpyu}), and on their unprecise observation (notion of column factors \cite{classif}). Intuitively, the limit set is the set of configuration that can appear arbitrarily far in the evolution of the system, and the column factor is the set of sequences of states that a cell (or group of cells) can take in a valid orbit of the system. These notions have been intensively studied in the literature as indicators of the dynamical complexity of cellular automata \cite{langlim2,rice,oexp,utrace}. 

Concerning limit sets of cellular automata, the initial (wrong) intuition was that a universal CA should necessarily have a non-recursive limit set (such a statement appears in \cite{cpyu}). Later, a Turing-complete CA with a simple limit set was constructed in \cite{limuniv}. This result gives a first hint concerning the absence of correlation between the complexity of the limit set and the ability to handle computations. However, the construction goes through a slow simulation of two-register machines where registers are encoded in unary. Therefore, this results is \textit{extrinsic} to the model of cellular automata and shows only that any behavior of register machines can be embedded into a cellular automaton with a simple limit set.

Here, we aim at exploring \textit{intrinsic} versions of the same question: what can be the limit set complexity of a CA able to simulate any other CA? More precisely, we consider two flavors of simulations: one using factor (continuous projection) of a simulator CA onto the simulated CA, and the other using the local uniform injection (sub-automaton) of a simulated CA into a simulator CA. Such simulation relations were extensively studied in \cite{bulk2}, and the second flavor is the one giving rise to the notion of intrinsic universality \cite{ollinger,bulk2}. The main point of the present paper is that factor simulations preserve limit set complexity whereas sub-automaton simulations do not. We also show that the same phenomenon appears for the complexity of column factors. Our main results are two embedding theorems showing that there exist an intrinsically universal CA with column factors of finite type and an intrinsically universal CA with an NL-recognizable limit set (that is, its limit language can be recognized by a non-deterministic Turing machine in logarithmic space). The second result solves an open problem of \cite{theyssier}.


\section{Definitions}

Let us note . If , we note  the interval of integers  such that , ,  and so on.
Consider a fixed finite \dfn{alphabet} .
If  and , then we note  the \dfn{pattern} corresponding to the sequence of letters  (and similarly for other kinds of intervals).

If  for some  and , the \dfn{cylinder}  will denote the set of configurations  such that . We also note .
If , let  be the configuration  such that  for any .

A \dfn{dynamical system} is a pair  where  is a compact space and  is a continuous self-map of . We can then study iterations  for any \dfn{time step} .

Let  stand either for  or for . The \dfn{shift} map  is defined for any  and any  by .
A \dfn{subshift} is a dynamical system , or simply , that is a subset of  which is closed under the usual Tychonoff topology and invariant by the shift map.
Equivalently, it is the set  of infinite words that avoid the finite patterns from some given family .
If this family can be taken finite, then  is called a subshift of \dfn{finite type} (SFT). If it can be taken among words of length , it has \dfn{order} .
The \dfn{language}  of a subshift  is the set  of patterns appearing in the infinite words of .
If this language is regular, then we say that  is a \dfn{sofic} subshift. Equivalently thanks to the Weiss theorem \cite{weiss}, it is obtained from an SFT by a letter-to-letter projection.

A \dfn{cellular automaton} (CA) is a dynamical system  which commutes with the shift, \ie .
Equivalently, thanks to the Curtis-Hedlund-Lyndon theorem, it is obtained by some \dfn{local map}  for some \dfn{radius} , \ie for any  and , .
 admits a \dfn{spreading} state  if it admits a local rule  with  and  whenever there exists  such that .

An SFT  is \dfn{irreducible} if for any two words , there exists a word  such that .
CA can actually be applied in a very natural way to these subshifts: a \dfn{partial CA} will be a dynamical system over some irreducible SFT that commutes with the shift. It is known from \cite{edensof} that any injective partial CA is bijective and reversible (the inverse is also a partial CA).



\subsection{Simulations}

The central notion studied in this paper is that of simulation between dynamical
systems and more specifically cellular automata. We will distinguish two
families of simulation relations based on the following notions.

\begin{defi}[Factors and sub-systems]\leavevmode \begin{itemize}
  \item A \dfn{factor map} between two dynamical systems  and  is a
    continuous onto map  such that .
    In that case, we say that  is a \dfn{factor} of . 
  \item A \dfn{sub-system} of a dynamical system  is a dynamical system of
    the form  where  ( is closed) and .
\end{itemize}
\end{defi}

Note that a factor map  between two subshifts  and  respect the Curtis-Hedlund-Lyndon theorem: there exist a \dfn{radius}  and a \dfn{local rule}  such that  for any  and any . We say that the factor map between two subshifts is a \dfn{coloring} if the radius can be taken .

If  and  are cellular automata, we say that \emph{ simulates  by factor} if there is a factor map  from  onto  which is also a factor map from  onto .
Besides, when  is a cellular automaton and  is a full-shift included in , then  is a \dfn{sub-automaton} of . 

These two relations (factor and sub-system) are restrictive since they don't allow entropy to increase: a factor or a sub-system of a given system has always a lower entropy. As a consequence, they don't support universality (existence of a system able to simulate any other) since it is not difficult to build systems of arbitrarily large entropy. Hence, in the literature, other ingredients were introduced to obtain richer notions of simulation: for instance, in cellular automata, operations of space and time rescaling added to the notion of sub-automaton lead to a notion of simulation supporting universality \cite{bulk1,bulk2}. 
 
In this paper, such kind of spatio-temporal transformations are not considered explicitly for two reasons:
\begin{itemize}
\item our results about factor simulation (Section~\ref{s:facsim}) involve properties
(complexity of the subshifts) which are invariant by space and time
rescaling, hence the results still hold when considering rescaling as part of the simulation relation;
\item our results about sub-system simulation (Section~\ref{s:subsim}) are of the form
"any CA is the sub-automaton of a CA with some given property", which is
actually the most general we can get, and remain true when replacing
"sub-automaton" by more general notions of sub-system (such as
sub-system, or simulated system in the sense of \cite{bulk2}).
\end{itemize}





\subsection{Complexity of limit sets and column factors}

Many different points of view have been adopted to study the complexity of CA. We here use symbolic dynamics, and consider the complexity, as subshifts, of the limit set on the one hand, or the column factors on the other hand, as representing the actual complexity of the CA.

The \dfn{limit set} of the dynamical system is the nonempty closed subset . Its \dfn{limit system} is the maximal surjective subsystem, . It basically represent the asymptotic dynamics of the system.

With respect to CA limit sets, it is not difficult to see that the corresponding language is always corecursively enumerable (it is an \emph{effective} subshift). However, there are known examples of non-recursive ones \cite{langlim2}. Moreover, simple additional remarks \cite{revlim,kurka,nasu} give the following hierarchy:\\
 injective   injective   is an SFT   is stable   is sofic  \ldots\

The \dfn{column factor} of a dynamical system  upon interval  is the set , where . It is a factor subshift since . It can represent an observation of the system made by a measuring device with a finite precision (that cannot see cells which are far away). It can be seen that any factor subshift is essentially a factor of some column factor \cite{classif}.

In the case of CA, shift-invariance allows to consider only the central column factors  for radius .
It is known \cite{notes} that these CA column factors always have a context-sensitive language; they may actually be strictly context-sensitive. In \cite{classif}, strong links with topological notions are stated: finite column factors is equivalent to equicontinuity, SFT column factors imply the shadowing property which in turn imply sofic column factors.


\section{Factor simulations}\label{s:facsim}

The two hierarchies that we have just defined, based on subshift classifications, are then very robust to factor simulations, as we can see in this section.
Taking the vocabulary of order theory, we say that a class  of systems is \emph{an ideal for factor simulation} if, whenever  is a factor of , we have: .

\begin{prop}\label{p:limsim}
If  is a factor map from  onto , then .
\end{prop}
\begin{proof}
For any , .
First note that  is included in .
Conversely, let  and, for , . Note that  is closed (since  and  are continuous, and  is compact) and nonempty (since  is onto). By compactness, the intersection  is not empty, \ie .
We have proven that .
\end{proof}

Moreover, we can note that , since if , then .
In the following we will no longer mention time and space rescaling, which does not essentially alter the results.

\begin{cor}
 The class of stable systems is an ideal for factor simulation.
\end{cor}
\begin{proof}
If  is a factor map from  onto , and  for some , then .
\end{proof}

We can also derive the two following results.
\begin{prop}The class of CA whose limit set\resp{factor subshifts} is sofic\resp{context-free, context-sensitive,
recursive} is an ideal for factor simulation.
\end{prop}
\begin{proof}
It is known that each of the corresponding classes of subshifts is preserved by factor maps.
Proposition~\ref{p:limsim} states that the limit set of the simulated CA is a factor of the limit set of the simulating CA.\\
Moreover, note, thanks to the transitivity of the notion of factor, that a factor subshift of the simulated CA is also a factor subshift of the simulating CA.
\end{proof}



The following slightly generalizes a result in \cite{theyssier}.
\begin{prop}\label{p:injsim}
The class of reversible partial CA is an ideal for coloring.
\end{prop}
\begin{proof}
Let  be a factor map based on some radius- local rule , from a partial CA  onto another one , where  and  and there exists some partial CA  such that  is the identity.
By surjectivity of , for any letter , there exists a letter, denoted abusively  such that . If  represents the parallel application of  to , we obtain that  is the identity of .
We can define the map , in such a way that  is the identity of .  is an injective partial CA, hence it is reversible, and it is the actual inverse map of .
\end{proof}
As far as we know though, it is unknown whether the class of injective CA is still an ideal for factor simulation.
\begin{cor}
The class of CA which are reversible over the limit set is an ideal for coloring.
\end{cor}
\begin{proof}Consider such a CA.
From \cite{revlim}, it is stable and its limit set is an irreducible SFT.
Now if it is linked to some other CA by some coloring, then by Proposition~\ref{p:limsim}, so are the two limit systems (that are partial CA), which then respect the hypotheses of Proposition~\ref{p:injsim}.
\end{proof}

From the previous, we can see that any universal CA for factor simulation (up to rescaling) must have  a non-recursive limit set and strictly context-sensitive factor subshifts (and of course must not be injective), but the existence of such a CA is still open.

\section{Sub-system simulations}\label{s:subsim}
\label{sec:injsimu}

We will see in this section that, contrary to factor simulations, sub-system simulations allow to hide the complexity of the simulated CA into the simulator CA.

\subsection{Hiding the column factors}

If  is a CA over alphabet  of radius , local rule , then let us define the CA  over alphabet  with the same radius  as  and  local rule:
0,g(c_{-r}\ldots c_r))&\textrm{otherwise}.}}\kaprx2(\Sigma)=\set{z=(z^t)_{t\in\N}}{(\tilde C^k)^\N}{\forall t\in\N,\exists x^t\in[z^t]\cap\tilde G^{-1}([z^{t+1}])}.
x_i=\soit{x^0_i&\textrm{if }0\le i<k\1,c)&\textrm{if }i\ge k\textrm{ and }x^{i-k}_k=(\varepsilon,c)~.}

\tilde G^t(x)_i=\soit{x^t_i&\textrm{if }0\le i<k\1,c)&\textrm{if }i\ge k\textrm{ and }x^{t+i-k}_k=(\varepsilon,c)~.}
X_S = \set y{Q^\Z}{\exists (y^t)_{t\in\N}\in(Q^\Z)^\N, y^0=y\text{ and }\forall t\in\Ns,y^t\in Q'\text{ and }S(y^t)=y^{t-1}}~.
\delta_{F,S}:c\mapsto
    \begin{cases}
      f(a_{-r_F}\ldots a_{r_F})&\text{ if }c=a_{-r}\ldots a_r\in A^{2r+1}~;\hfill \text{ (1)}\\
      a_0&\text{ if }c=(a_{-r},\gamma)\ldots(a_r,\gamma)\in(A\times\{\gamma\})^{2r+1}~;\hfill \text{ (2)}\\
      (a_0,s(b_{-r_S}\ldots b_{r_S}))&\text{ if }c=(a_{-r},b_{-r})\ldots(a_r,b_r)\in(A\times Q')^{2r+1}~;\hfill \text{ (3)}\\
      0&\text{ otherwise}.\hfill \text{ (4)}
    \end{cases}

      L_1+R_1 &\rightarrow l_1 + r_1\\
      l_1 + r_2 &\rightarrow l_1 + r_2\\
      r_1 + l_2 &\rightarrow r_1 + l_2
    
      l_2+r_2 &\rightarrow \#\\
      l_1+\#'+r_1 &\rightarrow \#\\
      \# &\rightarrow L_1+l_2+\#'+r_2+R_1
    ^\omega\{L_1,l_1,B\}\{l_2,B\}^*A\{r_2,B\}^*\{R_1,r_1,B\}^\omega
            d_1+d_2&=&2d_2+d'-d_1\label{eqn1}\\
            2d_1&=&d_2+d'\nonumber
          
            d_2+d_1&=&2d_1+d'-d_2\label{eqn2}\\
            2d_2&=&d_1+d'\nonumber
          \#'B^{x_1}A_1\cdots B^{x_i}A_i\{R_1,r_1,B\}^\omega\{r_2,B\}^*l_1B^xl_2B^yL_1B^zl_2B^z\{\#',\#\}
        B^zr_2B^zR_1B^yr_2B^xr_1\{L_1,l_1,B\}^\omega\{R_1,r_1,B\}^\omega L_1B^zl_2B^z\{\#',\#\}
        B^zr_2B^zR_1\{L_1,l_1,B\}^\omega~,^\omega\{R_1,r_1,B\}\{r_2,B\}^*\#'\{l_2,B\}^*\{L_1,l_1,B\}^\omega^\omega\{l_1,r_2,B\}^\omega^\omega\{r_1,l_2,B\}^\omega^\omega\{r_1,R_1,B\}\{r_2,B\}^*\{l_2,B\}^*\{l_1,L_1,B\}^\omega
  \end{proof}

\end{document}
