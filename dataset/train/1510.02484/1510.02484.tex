\documentclass{llncs}


\usepackage{verbatim}
\usepackage{citesort}
\usepackage{graphicx}
\usepackage{amsfonts}
\usepackage{times}
\usepackage{wrapfig}


\pagestyle{plain}



\newtheorem{obs}[lemma]{Observation}



\renewenvironment{proof}{\medskip\noindent{\bf Proof:}}{\mbox{}\hfill\qed\par}
\newenvironment{proofof}[1]{\medskip\noindent{\bf Proof of #1:}}{\mbox{}\hfill\qed\par}
\newenvironment{sketch}{\medskip\noindent{\bf Proof Sketch:}}{\mbox{}\hfill\qed\par}



\title{On Contact Graphs with Cubes and Proportional Boxes}
\author
{
	Md.~Jawaherul Alam\inst{1} \and
	Michael Kaufmann\inst{2} \and
	Stephen G.~Kobourov\inst{1}
}

\institute{
	Department of Computer Science, University of Arizona, USA
\and
	Wilhelm-Schickhard-Institut f\"ur Informatik, Universit\"at T\"ubingen, Germany
}


\begin{document}
\date{}
\maketitle



\begin{abstract}
We study two variants of the problem of contact representation of planar graphs with
 axis-aligned boxes. In a \textit{cube-contact representation} we realize each
 vertex with a cube, while in a \textit{proportional box-contact representation} each
 vertex is an axis-aligned box with a prespecified volume. We present
 algorithms for constructing cube-contact representation and proportional box-contact
 representation for several classes of planar graphs.

\end{abstract}








\section{Introduction}

We study \textit{contact representations} of planar graphs in 3D, where vertices are
 represented by interior-disjoint axis-aligned boxes and edges are represented by shared
 boundaries between the corresponding boxes. A contact representation of a planar graph
  is \textit{proper} if for each edge  of , the boxes for  and  have a
 shared boundary with non-zero area. Such a contact between two boxes is also called a
 \textit{proper contact}. \textit{Cubes} are axis-aligned boxes where all sides have the same
 length. A contact representation of a planar graph with boxes is called a \textit{cube-contact}
 representation when all the boxes are cubes. In a weighted variant of
 the problem a \textit{proportional
 box-contact} representation is one where each vertex  is represented with
 a box of volume , for any function
 , assigning weights to the vertices .
 Note that this ``value-by-volume'' representation is a natural
 generalization of the ``value-by-area'' cartograms in 2D.





\smallskip\noindent{\bf Related Work:} The history of representing planar graphs as contact graphs dates back at least to
 Koebe's 1930 theorem~\cite{Koebe36} for representing planar graphs by touching disks
 in 2D. Proper contact representation with rectangles in 2D is the well-known {\em rectangular
 dual} problem, for which several characterizations exist~\cite{KK85,Ungar53}.
 Representations with other axis-aligned and non-axis-aligned
 polygons~\cite{FM94,GHKK10,YS93} have been studied. Related graph-theoretic, combinatorial
 and geometric problems continue to be of
 interest~\cite{BGPV08,Fusy09,EMSV12}. The weighted variant
 of the problem has been considered in the context of 
 rectangular, rectilinear, and unrestricted cartograms~\cite{BR11,EFK+13,KS07}.

Contact representations have been also considered in 3D. Thomassen~\cite{Thom88} shows that any planar graph has a proper
 contact representation with touching boxes, while Felsner and Francis~\cite{FF11} find
 a (not necessarily proper) contact representation of any planar graph with touching cubes.
 Recently, Bremner \textit{et al.}~\cite{BEF+12} asked whether any planar graph
 can be represented by proper contacts of cubes. They answered the question positively
 for the case of partial planar 3-trees and some planar grids, but the problem remains
 open for general planar graphs. The weighted variant of the problem
 in 3D is much less studied, although recently Alam \textit{et al.}~\cite{AKLPV14} have
 presented algorithms for proportional representation of several
 classes of graphs (e.g., outerplanar, planar bipartite, planar,
 complete), using 3D L-shapes.







 \smallskip\noindent{\bf Our Contribution:} Here we expand the class of planar graph representable by proper contact of cubes. We also
 show that several classes of planar graphs admit proportional box-contact representations.
Specifically, we show how to compute a proportional box-contact representation for plane 3-trees, while a cube-contact
 representation for the same graph class follows
 from~\cite{BEF+12}. We also show how to compute a proportional
 box-contact representation and a cube-contact representation for {\em nested maximal outerplanar
   graphs}, which are defined as follows.
A \textit{nested outerplanar graph} is either an outerplanar graph or a planar graph 
 where each component induced by the internal vertices is another nested outerplanar
 graph with exactly three neighbors in the outerface of . A \textit{nested maximal
 outerplanar graph} is a subclass of nested outerplanar graphs
that is either a maximal outerplanar graph or a
 maximal planar graph in which the vertices on the outerface induce a maximal outerplanar
 graph and each component induced by internal vertices is another nested maximal outerplanar
 graph. 













\section{Preliminaries}





A \textit{3-tree} is either a 3-cycle or a graph  with a vertex  of degree three in 
 such that  is a 3-tree and the neighbors of  form a triangle. If  is planar, then
 it is called a \textit{planar 3-tree}. A \textit{plane 3-tree} is a planar 3-tree along with a fixed
 planar embedding. Starting with a 3-cycle, any planar 3-tree can be
 formed by recursively inserting a vertex inside a face and adding an edge between the newly
 added vertex and each of the three vertices on the face~\cite{BE09,MNRA10}.
 Using this simple construction, we can create in linear time a {\em representative tree} for
 ~\cite{MNRA10}, which is an ordered rooted ternary tree  spanning all the internal
 vertices of . The root of  is the first vertex we have to insert into the face of the three
 outer vertices. Adding a new vertex  in  will introduce three new faces belonging to
 . The first vertex  we add in each of these faces will be a child of  in . The
 correct order of  can be obtained by adding new vertices according to the
 counterclockwise order of the introduced faces. 




An \textit{outerplanar graph} is one that has a planar embedding with all vertices
 on the same face (outerface). An outerplanar graph is \textit{maximal} if no edge can be added without violating its outerplanarity. Thus in a maximal outerplanar graph
 all the faces except for the outerface are triangles.
For , a -outerplanar
 graph  is an embedded graph such that deleting the outer-vertices from  yields a graph
 where each component is at most a -outerplanar graph; a -outerplanar graph is just an outerplanar graph.  Note that any planar graph is a
 -outerplanar graph for some integer . 


Let  be a planar graph. We define the \textit{pieces} of  as follows. If  is
 outerplanar, it has only one piece, the graph itself. Otherwise, let , , ,
  be the components of the graph obtained by deleting the outer vertices (and their
 incident edges) from . Then the pieces of  are all the pieces of  for each
 , as well as the subgraph of  induced by the outer-vertices
 of . Note that each piece of  is an outerplanar graph. Since  is an
 embedded graph, for each piece  of , we can define the \textit{interior} of 
 as the region bounded by the outer cycle of . Then we can define a rooted tree
  where the pieces of  are the vertices of  and the
 parent-child relationship in  is determined as follows: for each piece
  of , its children are all the pieces of  that are in the interior of  but
 not in the interior of any other pieces of . A piece of  has \textit{level} 
 if it is on the -th level of . All the vertices of a piece
 at level  are also \textit{-level} vertices. A planar graph is a \textit{nested outerplanar
 graph} if each of its pieces at level  has exactly three vertices of level  as
 a neighbor of some of its vertices. On the other hand a \textit{nested maximal outerplanar
 graph} is a maximal planar graph where all the pieces are maximal outerplanar graphs.


 








\section{Representations for Planar 3-trees}


Here we prove that planar 3-trees have proportional box-representations in two different ways.
The first one is a more intuitive proof; the second one includes a direct computation of the coordinates
for the representation.


\begin{theorem}
\label{th:p3t-box} Let  be a plane 3-tree with a weight function . Then a
 proportional box-contact representation of  can be computed in linear time.
\end{theorem}




\begin{figure}[t]
\centering
\includegraphics[width=0.8\textwidth]{prop-p3t-2alt.pdf}
\caption{Illustration for the proof of Theorem~\ref{th:p3t-box}.}
\label{fig:prop-p3t-2}
\end{figure}



\noindent
\textbf{First Proof:}
Let , ,  be the outer vertices of . We construct a representation  for
  where  occupies the bottom side of ,  occupies the back of
  and  occupies the right side of ; see
 Fig.~\ref{fig:prop-p3t-2}. 
 Here for a set of vertices ,  denotes the representation obtained from  by deleting
 the boxes representing the vertices in . The claim is trivial when
  is a triangle, so assume that  has at least one internal vertex. Let 
 be the root of the representation tree  of . Then  is adjacent to ,  and 
 and thus defines three regions ,  and  inside the triangles ,
  and , respectively (including the vertices of these triangles).
 By induction hypothesis ,  has a proportional box-contact representation
  where the boxes for the three vertices in  occupy the bottom, back
 and right sides of . Define . We now construct
 the desired representation for . First take a box for  with volume  and place it in
 a corner created by the intersection of three pairwise-touching boxes; see
 Fig.~\ref{fig:prop-p3t-2}.  For each , , there is a corner  formed by the
 intersection of the three boxes for . We now place  (after possible
 scaling) in the corner  so that it touches the boxes for the vertices in  by
 three planes. Note that this is always possible since we can choose the surface areas for ,  and  to
 be arbitrarily large and still realize their corresponding weights
by appropriately changing the thickness in the third dimension.
This construction requires only linear time, by keeping the scaling factor for each
 region in the representative tree  at the vertex representing
 that region. 
Then the exact coordinates can be computed with a top-down traversal of . \qed




\noindent
\textbf{Second Proof:}
 Assume (after possible factoring) that for each vertex  of , the weight  is at least
 1. 
Let  be the representative tree of . For
 any vertex  of , we denote by , the set of the descendants of  in 
 including . The \textit{predecessors} of  are the neighbors of  in  that are not
 in . Clearly each vertex of  has exactly three predecessors.
We now define a parameter 
 for each vertex  of . Let ,  and  be the three children of  in
  (where zero or more of these three children may be empty). Then  is defined as
 , where  is taken as zero when  is empty. We
 can compute the value of  for each vertex  of  by a linear-time bottom-up
 traversal of . Once we have computed these values, we proceed on constructing the
 box-contact representation as follows.


\begin{figure}[htbp]
\centering
\includegraphics[width=.9\textwidth]{prop-p3t}
\caption{Illustration for the second proof of Theorem~\ref{th:p3t-box}.}
\label{fig:prop-p3t}
\end{figure}


Let , ,  be the three outer vertices of  in the clockwise order and let  be the
 root of . We start by computing three boxes for ,  and  with the correct volume
 as illustrated in Fig.~\ref{fig:prop-p3t}(a), so that the volume of the dotted box  is .
 We will now construct a box representation of  inside  so that all the vertices in 
 adjacent to an outer vertex is represented by a box with a face co-planar on the face of 
 adjacent to box representing that outer vertex. We do this recursively by a top-down
 computation on . Let  be a vertex of  with the three predecessors , 
 and . Let  be a box with volume  and let , ,  be three faces
 of it with a common point. While traversing , we compute a proportional box-contact
 representation of  inside  where the vertices in  adjacent to  for some
  is represented by a box with a face co-planar with . Let ,  and
  are the three children of  in  (where zero or more of these children may be
 empty). Also assume that , ,  are the length, width and height of ,
 respectively and  is the common point of ,  and . Then first compute a
 box  of volume  for  with a corner at  where ,  and 
 are the length, width and height of , such that
 .
 These choices of 's also creates three boxes  with volume at least ,
 , as illustrated in Fig.~\ref{fig:prop-p3t}(b). Finally we recursively compute
 the box representations for  inside  for  to complete
 the construction. \qed






\begin{theorem}~\cite{BEF+12}
Let  be a plane 3-tree. Then a cube-contact representation of  can be computed in linear
 time.
\end{theorem}

The proof of this claim also relies on the recursive decomposition of
planar 3-trees. 











\section{Cube-Contacts for Nested Maximal Outerplanar Graphs}


We prove the following main theorem in this section:



\begin{theorem}
Any nested maximal outerplanar graph has a proper contact representation with cubes.
\label{th:nested}
\end{theorem}



We prove Theorem~\ref{th:nested} by construction, starting with a representation
 for each piece of , and combining the pieces to complete the
 representation for .



Let  be a nested maximal outerplanar graph. We first augment the graph  by adding
 three mutually adjacent dummy vertices  on the outerface
 and then triangulating the graph by adding dummy edges from these three
 vertices to the outer vertices of  such that the graph remains planar; see
 Fig.~\ref{fig:cube-merge1}(a). Call this graph the \textit{extended graph} of .
 For consistency, let the three dummy vertices have level .
 The observation below follows from the definition of nested maximal outerplanar graphs.


\begin{obs}
	\label{obs:extended} Let  be a nested-maximal planar graph and let  be the
	extended graph of . Then for each piece  of  at level , there is a triangle
	of -level vertices adjacent to the vertices of  and no other -level
	vertices with  are adjacent to any vertex of .
\end{obs}



Given this observation, we use the following strategy to obtain a contact representation of 
 with cubes. For each piece  of  at level , let ,  and  be the three
 -level vertices adjacent to 's vertices. Let  be
 the subgraph of 
 induced vertices of  as well as ,  and
 ; call  the {\em extended
 piece} of  for . We obtain a contact representation of  with cubes and delete
 the three cubes for ,  and  to obtain the contact representation of  with cubes.
 Finally, we combine the representations for the pieces to complete the desired
 representation of .


Before we give more details on this algorithm, we have the following lemma, that we use in this section.
 Furthermore this result is also interesting by itself, since for any outerplanar graph , where each
 face has at least one outer edge, Lemma~\ref{lem:square} gives a contact representation of  on
 the plane with squares such that the outer boundary of the representation is a rectangle.



\begin{lemma}
	\label{lem:square} Let  be planar graph with outerface  and at least one
	internal vertex, such that  is a maximal outerplanar graph. If there is no
	chord between any two neighbors of  and no chord between any two neighbors of
	, then  has a contact representation  in 2D where each inner vertex
	is represented by a square, the union of these squares forms a rectangle,
	and the four sides of these rectangles represent , ,  and , respectively.
\end{lemma}
\begin{proof}
 We prove this lemma by induction on the number of vertices in . Denote the
 maximal outerplanar graph ; see Fig.~\ref{fig:square}(a). If  contains
 only one internal vertex , then we compute  by
 representing  by a square  of arbitrary size and representing , , 
 and  by the left, bottom, right and top sides of .


\begin{figure}[t]
\centering
	\includegraphics[width=0.8\textwidth]{slicing-cube-alt.pdf}
	\caption{Illustration for the proof of Lemma~\ref{lem:square}.}
	\label{fig:square}
\end{figure}


 We thus assume that  has at least two internal vertices. Let  be the unique common
 neighbor of  in . If  is a neighbor of , then  is a maximal
 outerplanar graph. By induction hypothesis,  has a contact representation
  where each internal vertex of  is represented by a square and the left,
 bottom, right and top sides of  represent , ,  and . Then we
 compute  from  by adding a square  to represent  such
 that  spans the entire width of  and is placed on top of ; see
 Fig.~\ref{fig:square}(b). A similar construction can be used if  is a neighbor of ; see
Fig.~\ref{fig:square}(c).
We thus compute a contact representation for ; see Fig.~\ref{fig:square}(d).
\end{proof}





\subsection{Cube-Contact Representation for Extended Pieces}






\begin{lemma}
	\label{lem:piece} Let  be a piece of  at level  and  be the
	extended piece for  with -level vertices , , . Then  has a
	cube-contact representation.
\end{lemma}
\begin{proof} Let  be a common neighbor of  and ;  a common neighbor of 
 and ;  a common neighbor of  and . It is easy to find a contact
 representation of  if ,  and  are the only vertices of
 , so let
  have at least four vertices. The outer cycle of  can be
 partitioned into three paths:  is the path from  to ,  is the path from  to
  and  is the path from  to . Note that all vertices on the path  (,
 ) are adjacent to  (, ). A chord  is a \textit{short chord} if it is between
 two vertices on the same path from the set . (Note that a chord between two
 vertices from the set  is also a short chord.) We have the following two cases.

\smallskip\noindent
\textbf{Case A: There is no short chord in .} In this case all the chords of  are between two
 different paths. We consider the following two subcases.




\begin{figure}[tb]
\centering
	\includegraphics[width=0.7\textwidth]{cube-merge1}\\
	(a)\hspace{0.38\textwidth}(b)\hspace{0.05\textwidth}
	\caption{Illustration for \textbf{Case A1} in the proof of Lemma~\ref{lem:piece}.}
	\label{fig:cube-merge1}
\end{figure}


\textbf{Case A1: There is no chord with one end-point in .} In this case, due to
 maximal-planarity there exist three vertices ,  and , adjacent to , , and ,
 respectively such that (i)  is the chord between vertices of  and  farthest
 away from , (ii)  is the chord between vertices of  and  farthest away
 from , and (iii)  is the chord between vertices of  and  farthest away
 from ; see Fig.~\ref{fig:cube-merge1}(a). We can then find three interior-disjoint
 subgraphs of  defined by three
cycles of :  is the one induced by
 all vertices on or inside ;  is induced by all vertices on or inside ;
 and  is induced by all vertices on or inside . Each of these subgraphs
 has the common property that if we delete two vertices from the outerface (two vertices from
 the set  in each subgraph), we get an outerplanar graph.
From the representation with squares from the proof of Lemma~\ref{lem:square}, we find a contact
 representation of ,  where each internal vertex of  is represented by a cube
 and the union of all these cubes forms a rectangular box whose four sides realize the outer vertices.
 We use such a representation to obtain a contact representation of  with cubes as follows.





We draw pairwise adjacent cubes (of arbitrary size) for , ,
. We need to place the cubes for all the vertices of  in the a
corner defined by three faces of the cubes for , , . 
Then we place three mutually touching cubes for
 ,  and , which touch the walls for ,  and , respectively; see
 Fig.~\ref{fig:cube-merge1}(b). We also compute a contact representation of the internal
 vertices for each of the
 three graphs ,  and  with cubes using Lemma~\ref{lem:square}, so that
 the outer boundary for each of these representation forms a rectangular pipe. We adjust
 the sizes of the three cubes for ,  and  in such a way that the three highlighted
 rectangular pipes precisely fit these three representations (after some possible scaling).
 Note that this construction works even if one or more of the subgraphs ,  and
  are empty. This completes the analysis of \textbf{Case A1}.




\begin{figure}[tb]
\centering
	\includegraphics[width=0.7\textwidth]{cube-merge2}\\
	(a)\hspace{0.38\textwidth}(b)\hspace{0.05\textwidth}
	\caption{Illustration for \textbf{Case A2} in the proof of Lemma~\ref{lem:piece}.}
	\label{fig:cube-merge2}
\end{figure}




\textbf{Case A2: There is at least one chord with one end-point in .} Due to planarity
 all such chords will have the same end-point in . Suppose  is this common end
 point for these chords; see Fig.~\ref{fig:cube-merge2}(a).
 Let  and  be the first and last endpoints in the clockwise order of these chords
 around . Then we can find two subgraphs  and 
 induced by the vertices on or inside two separating cycles  and . We
 find contact representations for the internal vertices of these two graphs  and 
 using Lemma~\ref{lem:square} so that the outer-boundaries of these representation form
 rectangular pipes. We then obtain the desired contact representation for , starting with the
 three mutually touching walls for ,  and  at right angles
 from each other, placing
 the cubes for  and , ,  as illustrated in
 Fig.~\ref{fig:cube-merge2}(b), and fitting the representations for  and  (after some
 possible scaling) in the highlighted regions.






\smallskip\noindent
\textbf{Case B: there are some shord chords in .}
 In this case, we find at most four subgraphs from  as follows. At each path in
 , we find the \textit{outermost chord},
 i.e., one that is not contained inside any other chords on the same path.
 Suppose these chords are ,
  and , on the three paths , , , respectively.
 Then three of these subgraphs ,  and  are  induced by the vertices
 on or inside the three
triangles ,  and .
 The fourth subgraph  is obtained from  by deleting all the inner
 vertices of the three graphs ,  and ; see Fig.~\ref{fig:short-chord}.

\begin{figure}[t]
\centering
	\includegraphics[width=0.9\textwidth]{short-chord.pdf}
	\caption{Removing chords with end-vertices in the same neighborhood.}
	\label{fig:short-chord}
\end{figure}


 A cube representation of  can be found by the algorithm in \textbf{Case A}, as 
 fits the condition that there is no chord between any two neighbors of the same vertex in
 . Note that by moving the cubes in the representation by an arbitrarily
 small amount, we can make sure that for each triangle  in , the three cubes for ,
  and  form a corner surrounded by three mutually touching walls at right angles to
 each other. Now observe that each of the three graphs ,  and  is a planar
 3-tree; thus using the algorithm of either~\cite{BEF+12} or~\cite{FF11}, we can
 place the internal vertices of these three graphs in their corresponding corners, thereby
 completing the representation.
\end{proof}





\subsection{Cube-Contact Representation for a Nested Maximal Outerplanar Graph}


\noindent
\textbf{Proof of Theorem~\ref{th:nested}:} Let  be a nested maximal outerplanar graph.
 We build the contact representation of  by a top-down traversal of the rooted tree
  of the pieces of .
 We start by creating a corner surrounded by three mutually touching walls at right angle
 to each other. Then whenever we traverse any vertex of , we realize the
 corresponding piece  at level  by obtaining a representation using
 Lemma~\ref{lem:piece} and placing  this in the corner created by the three already-placed
 cubes for the three -level vertices adjacent to  (after possible scaling). \qed










\section{Proportional Box-Contacts for Nested Outerplanar Graphs}


In this section we prove the following main theorem.



\begin{theorem}
\label{th:nested-prop}
	Let  be a nested outerplanar graph and let 
	 be a weight function defining weights for the vertices of . Then  has a
	 proportional contact representation with axis-aligned boxes with respect to .
\end{theorem}

We construct a proportional representation for  using a similar strategy as in the previous
 section: we traverse the construction tree  of  and deal with each piece
 of  separately. Each piece  of  is an outerplanar graph and hence one can easily
 construct a proportional box-contact representation for  as follows. Any outerplanar
 graph  has a contact representation with rectangles in the plane. In fact in~\cite{ourAlg13},
 it was shown that  has a contact representation with rectangles
 on the plane where the rectangles realize prespecified weights by
 their areas. Thus by giving unit heights to all rectangles we can
 obtain a proportional box-contact representation of  for any given
 weight function. 
However if we construct proportional box-contact representation for each piece of  in this way,
it is not clear that we can combine them all to find a proportional contact representation of the
 whole graph . 
Instead, we use this construction idea in Lemmas~\ref{lem:stair}~and~\ref{lem:double-stair} to
build two different proportional rectangle-contact representations for
outerplanar graphs
and we use them in the proof of Theorem~\ref{th:nested-prop}. 

Suppose  is an outerplanar graph and  is a contact representation of  with rectangles
 in the plane. We say that a corner of a rectangle in  is \textit{exposed} if it is
 on the outer-boundary of  and is not shared with any other rectangles.





\begin{lemma}
\label{lem:stair}
	Let  be a maximal outerplanar graph with a weight function . Let
	, ,  be the clockwise order of the vertices around the
	outer-cycle. Then a proportional rectangle-contact representation  of
	 for  can be computed so that rectangle 
	for  is leftmost in , rectangle  for  is
        bottommost in , and the top-right corner for each rectangle is exposed in .
\end{lemma}
\begin{comment}
	\begin{figure}[t]
\centering
		\includegraphics[width=0.4\textwidth]{stair.pdf}
		\caption{Illustration for the proof of Lemma~\ref{lem:stair}.}
	\label{fig:stair}
	\end{figure}
\begin{sketch}
Constructing  is easy when  is a single edge , so
let  contain at least 3 vertices. Let  be the unique vertex adjacent to . Denote by
  the graph induced by all vertices between  and  and by  the graph
 induced by the vertices between  and . Recursively draw  and  and
 remove the rectangles for , ,  to find the drawings  and ,
 respectively. Finally draw the rectangles ,  and  for ,  and ,
 with the required areas and place  (after possible scaling) between
 ,  and  (after possible scaling)  between ,  to complete
 the drawing; see Fig.~\ref{fig:stair}.
\end{sketch}
\end{comment}


\begin{proof}
We give an algorithm that recursively computes . Constructing  is easy
 when  is a single edge . We thus assume that  has at least 3 vertices. Let 
 be the (unique) third vertex on the inner face that is adjacent to . Then graph  can
 be split into two graphs at vertex  and edge :  consists of the graph induced
 by all vertices between  and  in clockwise order around the outer-cycle; while 
 consists of the graph induced by the vertices between  and .





Recursively draw  and remove the rectangles for  and  from it; call the result
 . Again recursively draw  and remove  and  from it; call the result
 . Now draw a rectangle  for  with area . Let  and  be
 the width and height of , respectively. Then draw the rectangles  and  for
  and  touching the left and the bottom sides of , respectively with necessary areas.
 Select the widths and heights of these two rectangles such that the area 
 can contain  while the area  can contain , where
  and  denote the width and height of , respectively for . Finally
 place  (after possible scaling) touching the right side of  and the top side
 of  and place  (after possible scaling) touching the right side of  and
 the top side of  to complete the drawing; see Fig.~\ref{fig:stair}.
\end{proof}



\begin{figure}[tb]
\hfill
\begin{minipage}[b]{.4\textwidth}
	\centering
	\includegraphics[width=\textwidth]{stair-wrap.pdf}
	\caption{Illustration for the proof of Lemma~\ref{lem:stair}.}
	\label{fig:stair}
\end{minipage}
\hfill
\begin{minipage}[b]{.4\textwidth}
	\centering
	\includegraphics[width=\textwidth]{double-stair-alt.pdf}
	\caption{Illustration for the proof of Lemma~\ref{lem:double-stair}.}
\label{fig:double-stair}
\end{minipage}
\hfill
\end{figure}



 Note that in the layout obtained above
the top right corners of the rectangles for vertices 
 have increasing -coordinates and decreasing
 -coordinates. Thus we refer to them as \textbf{Staircase} layouts
 and to the algorithm as the \textbf{Staircase
 Algorithm}. 


\begin{lemma}
\label{lem:double-stair}
	Let  be a maximal outerplanar graph with a weight function . Let
	, ,  be the clockwise order of the vertices
        around the outer-cycle.
Then a proportional rectangle-contact representation   of 
for  can be computed so that rectangle  for  is
leftmost in , rectangle  for  is bottommost in
, and the top-right corners of all rectangles
	for vertices  and the bottom-right corners of all rectangles for vertices
	 are exposed in .
\end{lemma}
\begin{comment}
\begin{figure}[t]
\centering
	\includegraphics[width=0.9\textwidth]{double-stair}\\
	(a)\hspace{0.2\textwidth}(b)\hspace{0.22\textwidth}(c)
	\hspace{0.2\textwidth}(d)\hspace{0.05\textwidth}
	\caption{Illustration for the proof of Lemma~\ref{lem:double-stair}.}
\label{fig:double-stair}
\end{figure}

\begin{sketch}
Computing  is easy when  is a single edge , so let
  have at least 3 vertices and  be the unique vertex adjacent to . Define the two
 graphs  and  as in the proof of Lemma~\ref{lem:stair}; see Fig.~\ref{fig:double-stair}(a).
If , then recursively draw  and remove the rectangles for  and  from it;
 call the result . Draw  using the \textbf{Staircase Algorithm} and remove
  and  to find . Now draw three mutually touching rectangles , 
 and  for ,  and , with the necessary areas and place 
 (after possible scaling)  between ,  and  (after  clockwise
 rotation and possible scaling) between ,  to complete the drawing; see
 Fig.~\ref{fig:double-stair}(b). The cases when  and  follow similar constructions;
 see Fig.~\ref{fig:double-stair}(c)--(d).
\end{sketch}
\end{comment}




\begin{proof}
We again compute  recursively. Constructing  is easy when  is a single
 edge . We thus assume that  has at least 3 vertices. Let  be the (unique)
 third vertex on the inner face that is adjacent to . Define the two graphs 
 and  as in the proof of Lemma~\ref{lem:stair}; see also Fig.~\ref{fig:double-stair}(a).






If , then recursively draw  and remove the rectangles for  and  from it;
 call the result . Draw  using the \textbf{Staircase Algorithm}. Now draw
 three mutually touching rectangles ,  and  for ,  and , respectively
 with necessary areas such that the right side of  touches both  and  and
 the right side of  has greater -coordinate than the right side of ; see
 Fig.~\ref{fig:double-stair}(b). Finally place  (after possible scaling) touching
 the right side of  and the top side of  such that the right side of the rectangle
 for  extends past . Also place  (after  clockwise rotation
 and possible scaling) touching the bottom side of  and the right side of  to
 complete the drawing (the width of  can be chosen long enough so that 
 can be contained between the bottom side of  and the right side of ).
	



	
 On the other hand if  we follow almost the same procedure as in the previous paragraph.
	However, instead of the drawing of  recursively, we compute it by the \textbf{Staircase
	Algorithm} and then delete from it  and  to obtain . We compute 
	as in the previous section. We also draw ,  and  in the same way. Then we
	place  (after possible scaling) touching the right side of  and the top
	side of  (again the width of  is chosen long enough so that this can be done).
	We also place  as the same manner as before to complete the drawing; see
	Fig.~\ref{fig:double-stair}(c).
	
	
	Finally if , then we draw  by the \textbf{Staircase Algorithm} and delete
	from it  and  to obtain . However, to compute , we recursively
	draw  and delete  and  from it. We now draw ,  and  as
	before but this time the right side of  should extend past . We now place
	 (after possible scaling) touching the right side of  and the top side
	of  (this is again possible for suitable choice of the height of ). Finally
	we complete the drawing by placing  (after possible scaling) touching the
	right side of  and the top side of  so that the right side of the rectangle
	for  extends past ; see Fig.~\ref{fig:double-stair}(d).
\end{proof}


Note that in the layout obtained above the top-right corners for vertices  and the bottom-right corners for vertices  form two staircases.
Thus we refer to this as a \textbf{Double-Staircase} layout,  to the algorithm as the \textbf{Double-Staircase Algorithm}, and to vertex  as the \textit{pivot vertex}.




Let  be a maximal outerplanar graph and let  be either a \textbf{Staircase} or
 a \textbf{Double-Staircase} layout. Then any triangle  in  is represented by three rectangles and the shared boundaries of these rectangles define a \textit{T-shape}. The vertex whose two shared boundaries are  collinear in the T-shape is called the \textit{pole} of the triangle .







\smallskip\noindent
\textbf{Proof of Theorem~\ref{th:nested-prop}.}
	Let  be the construction tree for . We compute a
	representation for  by a top-down traversal of , constructing the
	representation for each piece as we traverse it. Let  be a piece of  at the
	-th level. If  is the root of , then we use the
	\textbf{Staircase Algorithm} to find a contact representation of  with rectangles
	in the plane and then we give necessary heights to these rectangles to obtain a proportional
	contact representation of  with boxes.
Otherwise, the vertices of  are adjacent to exactly three -level vertices
	, ,  that form a triangle in the parent piece of .
	Since , ,  belong to the parent piece of , their boxes have already been
	drawn when we start to draw . To find a correct
	representation of , we need that the boxes for the vertices in  have correct
	adjacencies with the boxes for
	, , and ; hence we assume a fixed structure for such a triangle.
We maintain the following invariant:
	
	
	\textit{Let  be three vertices in a piece  of  forming a triangle.
	 Then in the proportional contact representation of , the boxes for , , 
	 are drawn in such a way that (i) the projection of the mutually shared boundaries for
	 these boxes in the -plane forms a T-shape, (ii) the highest faces (faces
	 with largest -coordinate) of the three rectangles have different  coordinates
	and the highest face of the pole-vertex of the triangle has the smallest
	 -coordinate.}

	
	Note that by choosing the areas of the rectangles in the \textbf{Staircase} layout,
	we can maintain this invariant for the parent piece by appropriately adjusting the heights of the boxes(e.g., incrementally increasing heights for the vertices in the
	recursive \textbf{Staircase Algorithm}).
	
	
	We now describe the construction of a proportional box-contact representation of 
	with the correct adjacencies for ,  and .
	By the invariant the projection of the shared boundaries for  forms a T-shape
	in the -plane. 
Without loss of generality assume that  is the pole of the triangle and the highest faces
	of ,  and  are in this order according to decreasing -coordinates.
Also assume that  is a maximal outerplanar graph; we later argue
	that this assumption is not necessary.
	
Let  be a common neighbor of  and ;  a common neighbor of  and ;
	 a common neighbor of  and . Then the outer cycle of  can be partitioned
	into three paths:  is the path from  to ,  is the path from  to
	 and  is the path from  to . All the vertices on the path
	 (, , respectively) are adjacent to  (, , respectively). We
	first assume that there is no chord in  between  and a vertex on path .
	We consider the following two cases.
	
	


	
	\smallskip\noindent
	\textbf{Case 1: No vertex of  is adjacent to all of .} We label the vertices
	of  in the clockwise order starting from  and ending at , where  is the
	number of vertices in . Let  and  be the indices of vertices  and ,
	respectively. Let  be the index of the vertex that is the (unique) third vertex of
	the inner face of  containing the edge . Define the two graphs  and
	 as in the proof of Lemma~\ref{lem:stair}. We first find a proportional contact
	representation of  for  restricted to the vertices of 
	using rectangles in the plane, then we give necessary heights to this rectangles.
	Draw  using the \textbf{Staircase Algorithm} and delete the rectangles for  and
	 to obtain . Draw rectangles  and  for  and ,
	respectively, so that the bottom side of  touches the top side of , the left
	sides for both the rectangles have the same -coordinate and and the right side of
	 extends past . Now place  (after possible scaling) touching the
	right side of  and the top side of  (this is possible since we can make the
	width of  sufficiently long); see Fig.~\ref{fig:prop}. Place the rectangle 
	for  touching the left sides of  and  such that its bottom side is aligned
	with  and its top side is aligned with the top side of the rectangle for .
	To complete the rest of the drawing, we have the following two subcases:
	
	\textbf{Case 1a: .} We draw  using the \textbf{Staircase Algorithm} and
	delete from it the rectangles for  and  to obtain . We finally place
	 (after  counterclockwise rotation and possible scaling) touching
	the top side of  and left side of  (this is possible by choosing a sufficiently
	large height for ); see Fig.~\ref{fig:prop}(a).
	
	
	\textbf{Case 1b: .} We draw  using the \textbf{Double-Staircase Algorithm}
	where  is the pivot vertex. From this drawing, we delete the rectangles for 
and  to obtain . Finally place  (after  counterclockwise
	rotation and possible scaling) touching the top side of  and left side of  such
	that the topside of the rectangle for  goes past the top side of ; see Fig.~\ref{fig:prop}(b).
	

\begin{figure}[htbp]
\vspace{-1cm}
	\centering
	\includegraphics[width=0.88\textwidth]{prop-box.pdf}
	\includegraphics[width=0.88\textwidth]{prop-box_2.pdf}
	\caption{Illustration for Theorem~\ref{th:nested-prop}.
	Construction of the representation for a piece  of ,
when (a)--(b) no vertex of  is adjacent to all of , and
	(c)--(e) a vertex of  is adjacent to all of .}
\label{fig:prop}
\end{figure}



	
So far we used the function  to assign areas for the rectangles and obtained proportional box-contact representation of  from the rectangles by assigning unit heights. However, by changing the areas for the
	rectangles, we can obtain different heights for the boxes. We will use this property to maintain adjacencies with , as well as to maintain the invariant. Specifically, once we get the box
	representation of , we scale it 
by increasing the heights for the boxes, so that when we place it at the corner created by
	the T-shape for  it will not intersect the representation for
	any of its sibling pieces in . Consider the point  which is the intersection
	of the lines containing the right side of the rectangle for  and the top side of the
	rectangle for . We place  such that the point  superimposes
	on the corner for the T-shape in the projection on the -plane. Since the highest faces
	of ,  and  are in this order according to -coordinate,
	the adjacencies of the vertices in  with  are correct. By appropriately choosing the areas
	for the rectangles, we ensure that all the boxes for the vertices of 
	have their highest faces above that of  and that the invariant is maintained.
	






	


	\noindent
	\textbf{Case 2: A vertex of  is adjacent to all of .} In this case at
	least one of  has only one neighbor in . Assume first that a vertex 
	() of  is adjacent to all of  and this is the only neighbor of ;
	see Fig.~\ref{fig:prop}(c).
 Then we follow the steps for \textit{Case 1a}
	with  (and some vertex between  and  as ). But when we finally
	place this representation of  on the corner for the T-shape of  we find
	the point  to superimpose on this corner as follows. The point  is on the line
	containing the top side of the rectangle for  and has -coordinate between the
	right sides of the rectangles for  and .

	If a vertex  () is adjacent to all of  and is
	the only neighbor of  in , then we follow the steps of \textit{Case 1b} with 
	and ; see Fig.~\ref{fig:prop}(d).
 We find the point  to superimpose on
	the corner for the T-shape of  as follows. The point  is on the line
	containing the top side of the rectangle for  and has -coordinate between
	the left sides of the rectangles for  and .

	If a vertex  () is adjacent to all of  and is
	the only neighbor of  in , then we number the vertices of  in the clockwise
	order starting from the clockwise neighbor of  and ending at ;
	see Fig.~\ref{fig:prop}(e).
	 We use the
	\textbf{Staircase Algorithm} to find a representation of  with
	rectangles and give necessary heights to obtain a representation
	with boxes. On the corner for the T-shape of , we superimpose the
	intersection point for the lines containing the top side of the rectangle of  and the
	right side of the rectangle for .

	Finally, we consider the case when there is a chord between  and another vertex on the
	path . Take the innermost such chord and let its other end-vertex be . Then consider the two subgraphs 
	and  induced by all the vertices outside the chord and inside the chord (along with
	the two vertices  and ).  does not contain any chord from ;
	thus we use the algorithm above to obtain a
	representation of ; denote this by . In this
	representation  and  will play the roles of  and , respectively.
	Each vertex of  is adjacent to  and we find a proportional contact
	representation of  and attach it with  as follows. We use the \textbf{Staircase
	Algorithm} to find a proportional contact representation of  with rectangles in
	the plane and delete the rectangles for  and  from it to obtain .
	In , we change the height of the rectangle  for  to increase
	its area so that its bottom side extends past the bottom side of the rectangle 
	for . Then we place  (after reflecting with respect to the -axis
	and possible scaling) touching the right side of  and the bottom side of .
	Since the \textbf{Staircase Algorithm} can accommodate any given area for the layout,
	we can change the heights of the boxes for the vertices in  to maintain the invariant.
	
	
	Thus with the top-down traversal of , we obtain a proportional contact
	representation for . We assumed that each piece of  is maximal outerplanar.
	However in the contact representation, for each edge , either
	a face of the box  for  is adjacent to the box  for  and no other box;
	or a face of  is adjacent to  and no other box. In both cases the adjacency
	between these two coxes can be removed without affecting any other adjacency. Thus this
	algorithm holds for any nested outerplanar graph . \qed













\section{Conclusions and Future Work}

 
We proved that nested maximal outerplanar graphs have cube-contact
 representations and nested outerplanar graphs have proportional box-contact representations.
 These classes of graphs are special cases of -outerplanar graphs, and the set of
 -outerplanar graphs for all  is equivalent to the class of all planar graphs. 
Even though our approach might generalize to large classes, 
cube-contact representations and proportional box-contact representations are still open for general planar graphs.




\medskip
\noindent{\bf Acknowledgments:} We thank 
 Therese Biedl, Steve Chaplick, Stefan Felsner, and Torsten Ueckerdt for discussions about this problem.





{
\begin{small}
\begin{thebibliography}{10}
\providecommand{\url}[1]{\texttt{#1}}
\providecommand{\urlprefix}{URL }

\bibitem{AKLPV14}
Alam, M.J., Kobourov, S.G., Liotta, G., Pupyrev, S., Veeramoni, S.: {3D}
  proportional contact representations of graphs. In: Bourbakis, N.G.,
  Tsihrintzis, G.A., Virvou, M. (eds.) Information, Intelligence, Systems and
  Applications. pp. 27--32. {IEEE} (2014)

\bibitem{ourAlg13}
Alam, M.J., Biedl, T.C., Felsner, S., Gerasch, A., Kaufmann, M., Kobourov,
  S.G.: Linear-time algorithms for hole-free rectilinear proportional contact
  graph representations. Algorithmica  67(1),  3--22 (2013)

\bibitem{BE09}
Biedl, T.C., Vel{\'{a}}zquez, L.E.R.: Drawing planar 3-trees with given face
  areas. Computational Geometry: Theory and Applications  46(3),  276--285
  (2013)

\bibitem{BR11}
Biedl, T.C., Vel{\'{a}}zquez, L.E.R.: Orthogonal cartograms with at most 12
  corners per face. Computational Geometry: Theory and Applications  47(2),
  282--294 (2014)

\bibitem{BEF+12}
Bremner, D., Evans, W.S., Frati, F., Heyer, L.J., Kobourov, S.G., Lenhart,
  W.J., Liotta, G., Rappaport, D., Whitesides, S.: On representing graphs by
  touching cuboids. In: Didimo, W., Patrignani, M. (eds.) Graph Drawing. LNCS,
  vol. 7704, pp. 187--198. Springer (2012)

\bibitem{BGPV08}
Buchsbaum, A.L., Gansner, E.R., Procopiuc, C.M., Venkatasubramanian, S.:
  Rectangular layouts and contact graphs. ACM Transactions on Algorithms  4(1)
  (2008)

\bibitem{GHKK10}
Duncan, C.A., Gansner, E.R., Hu, Y.F., Kaufmann, M., Kobourov, S.G.: Optimal
  polygonal representation of planar graphs. Algorithmica  63(3),  672--691
  (2012)

\bibitem{EMSV12}
Eppstein, D., Mumford, E., Speckmann, B., Verbeek, K.: Area-universal and
  constrained rectangular layouts. SIAM Journal on Computing  41(3),  537--564
  (2012)

\bibitem{EFK+13}
Evans, W., Felsner, S., Kaufmann, M., Kobourov, S., Mondal, D., Nishat, R.,
  Verbeek, K.: Table cartograms. In: Bodlaender, H.L., Italiano, G.F. (eds.)
  European Symposium on Algorithms. LNCS, vol. 8125, pp. 421--432. Springer
  (2013)

\bibitem{FF11}
Felsner, S., Francis, M.C.: Contact representations of planar graphs with
  cubes. In: Hurtado, F., van Kreveld, M.J. (eds.) Symposium on Computational
  Geometry. pp. 315--320 (2011)

\bibitem{FM94}
de~Fraysseix, H., de~Mendez, P.O.: Regular orientations, arboricity, and
  augmentation. In: Graph Drawing. LNCS, vol. 894, pp. 111--118. Springer
  (1995)

\bibitem{Fusy09}
Fusy, {\'E}.: Transversal structures on triangulations: A combinatorial study
  and straight-line drawings. Discrete Mathematics  309(7),  1870--1894 (2009)

\bibitem{Koebe36}
Koebe, P.: {K}ontaktprobleme der konformen {A}bbildung. Berichte {\"u}ber die
  Verhandlungen der S{\"a}chsischen Akad. der Wissenschaften zu Leipzig.
  Math.-Phys. Klasse  88,  141--164 (1936)

\bibitem{KK85}
Kozminski, K., Kinnen, E.: Rectangular duals of planar graphs. Networks  15(2),
   145--157 (1985)

\bibitem{KS07}
van Kreveld, M.J., Speckmann, B.: On rectangular cartograms. Computational
  Geometry  37(3),  175--187 (2007)

\bibitem{MNRA10}
Mondal, D., Nishat, R.I., Rahman, M.S., Alam, M.J.: Minimum-area drawings of
  plane 3-trees. Journal of Graph Algorithms and Applications  15(2),  177--204
  (2011)

\bibitem{Thom88}
Thomassen, C.: Interval representations of planar graphs. Journal of
  Combinatorial Theory, Series B  40(1),  9--20 (1988)

\bibitem{Ungar53}
Ungar, P.: On diagrams representing maps. Journal of the London Mathematical
  Society  28,  336--342 (1953)

\bibitem{YS93}
Yeap, K.H., Sarrafzadeh, M.: Floor-planning by graph dualization: 2-concave
  rectilinear modules. SIAM Journal on Computing  22,  500--526 (1993)

\end{thebibliography}
 \end{small}
}




\end{document}
