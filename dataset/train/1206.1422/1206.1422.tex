\documentclass[11pt]{article}

\usepackage{amssymb,amsmath,amsthm}
\usepackage{graphicx}

 \addtolength{\oddsidemargin}{-1.5cm}
 \addtolength{\textwidth}{3cm}
 \addtolength{\topmargin}{-1.5cm}
 \addtolength{\textheight}{3cm}

	\newtheorem{thm}{Theorem}             \newtheorem{lem}[thm]{Lemma}
  \DeclareMathOperator{\vis}{\mathrm{visible}}
  \DeclareMathOperator{\xang}{\mathrm{xang}}
  \DeclareMathOperator{\conv}{\mathrm{conv}}
  \DeclareMathOperator{\per}{\mathrm{per}}
  \newcommand{\eps}{\varepsilon}
  \newcommand{\R}{\mathbb{R}}  

\begin{document}

\title{The Visible Perimeter of an Arrangement of Disks\footnote{A preliminary version of this paper appeared in \emph{Graph Drawing 2012} (LNCS 7704, pp. 364--375, 2013).}}

\author{Gabriel Nivasch\footnote{Ariel University, Ariel, Israel. Work done when the author was at EPFL, Lausanne, Switzerland.}\\\texttt{gabrieln@ariel.ac.il} \and J\'anos Pach\footnote{EPFL, Lausanne, Switzerland and R\'enyi Institute, Budapest, Hungary. Work supported by Hungarian Science Foundation EuroGIGA Grant OTKA NN 102029, by Swiss National Science Foundation Grants 200020-144531 and 20021-137574, and by NSF Grant CCF-08-30272.}\\\texttt{pach@cims.nyu.edu}  \and G\'abor Tardos\footnote{R\'enyi Institute, Budapest, Hungary. Work supported by an NSERC grant and by OTKA grants T-046234, AT048826 and NK-62321.}\\\texttt{tardos@renyi.hu}}

\maketitle


\begin{abstract}
Given a collection of  opaque unit disks in the plane, we want to find a stacking order for them that maximizes their {\em visible perimeter}, the total length of
all pieces of their boundaries visible from above.  We prove that if the centers of the disks form a {\em dense} point set,
{\em i.e.}, the ratio of their maximum to their minimum distance
is , then there is a stacking order for which the visible
perimeter is . We also show that
this bound cannot be improved in the case of a sufficiently small  uniform grid. On the other hand, if the
set of centers is dense and the maximum distance between them is
small, then the visible perimeter is  with respect to any stacking order. This latter bound cannot be improved either.

Finally, we address the case where no more than  disks can have a point in common.

These results partially answer some questions of Cabello, Haverkort, van Kreveld, and Speckmann.

Keywords: Visible perimeter, disk, unit disk, dense set.
\end{abstract}


\section{Introduction}

In cartography and data visualization, one often has to place similar copies of a symbol, typically an opaque disk, on a map or a figure at given locations \cite{De99}, \cite{Gr90}. The size of the symbol is sometimes proportional to the quantitative data associated with the location. On a cluttered map, it is difficult to identify the symbols. Therefore, it has been investigated in several studies how to minimize the amount of overlap \cite{GrC78}, \cite{SlM03}.

\begin{figure}
\centerline{\includegraphics{fig_stacking_order.pdf}}
\caption{\label{fig_stacking_order}Left: A collection of unit disks in the plane. Right: A stacking order for them.}
\end{figure}


In the present note, we follow the approach of Cabello, Haverkort, van
Kreveld, and Speckmann~\cite{CaH10}. We assume that the symbols used are
opaque circular disks of the same size. Given a collection  of 
distinct {\em unit} disks in the -plane, a {\em stacking order} is a one-to-one assignment . We consider the integer  to be the -coordinate of the disk . The {\em map} corresponding to this stacking order is the 2-dimensional view of this arrangement from the point at negative infinity of the -axis (for notational convenience, we look at the arrangement from below rather than from above.) In particular, for the lowest disk , we have , and this disk, including its full perimeter, is visible from below. The total length of the
boundary pieces of the disks visible from below is the {\em visible perimeter}
of  with respect to the stacking order , denoted by . We are interested in finding a stacking order for which the visible
perimeter of  is as large as possible. See
Figure~\ref{fig_stacking_order}.

There are other situations in which this setting is relevant. Sometimes the vertices of a graph are not represented as points but as circles of a given radius. It may happen that
some vertices overlap in the visualization (especially if they have
further constraints on their geometric position), and then it becomes important to choose a
convenient stacking order that maximizes the visible perimeter.

Given an integer , we define

where the maximum is taken over all stacking orders .
We would like to describe the asymptotic behavior of , as  tends to infinity.

Cabello {\em et al.}\ have already noted that ; in other
words, every set  of  disks of unit radii admits
a stacking order with respect to which its visible perimeter is
. Indeed, by a well-known result or Erd\H os and
Szekeres~\cite{ErSz35}, we can select a sequence of 
disks  such that their
centers form a monotone sequence. More precisely, letting  and 
denote the coordinates of the center of , we have  and either  or . Then, in any stacking order  such that  for every
, a full quarter of the perimeter of each  is visible from below. Therefore, the visible perimeter of  with respect to  satisfies 

At the problem session of {\em EuroCG'11} (Morschach, Switzerland), Cabello, Haverkort, van Kreveld, and Speckmann asked whether ; in other words, does there exist a positive constant  such that every set of  unit disks in the plane admits a stacking order, with respect to which its visible perimeter is at least ?  We answer this question in the negative; {\em cf.}\ Theorems~\ref{theorem2} and~\ref{theorem5} below.

Given a set of points  in the plane, let  denote the
collection of disks of radius  centered at the elements of . For any
positive real , let  stand for a similar copy of , scaled by
a factor of . For a stacking order  of  we will study
the quantity . (Note the slight abuse of notation:
We denote the stacking order of  and the corresponding
stacking order of  by the same symbol . The two orders are also identified in Lemmas~\ref{lemma1} and~\ref{lemma2.1} and in Theorems~\ref{theorem2}, \ref{theorem3}, and~\ref{theorem5}.) It is not hard to verify that, as  gets smaller, the function  decreases. To see this, it is enough to observe, as was also done by
Cabello {\em et al.}\ (unpublished), that as we contract the set of centers,
the part of the boundary of each unit disk visible from below shrinks. As we
will see in Lemma~\ref{lemma2.1}, the limit in the following lemma has a simple alternative geometric interpretation.

\begin{lem}\label{lemma1}  For every point set  in the plane and for
  every stacking order  of the collection of disks , we have

\end{lem}

As in \cite{AlKP89}, \cite{Va92}, and \cite{Va96}, we consider {\em -dense}
-element point sets , {\em i.e}., point sets in which the ratio of the maximum distance between two points to the minimum distance satisfies

(The above ratio is sometimes called the \emph{spread} of  \cite{Er03}; thus, we consider point sets with spread at most .)

\begin{thm}\label{theorem2} For any -dense -element point set  in the plane and for any stacking order , we have

where  is a constant depending only on .
\end{thm}

The order of magnitude of the upper bound in Theorem~\ref{theorem2} cannot be
improved:

\begin{thm}\label{theorem3} For every positive integer , there exists a -dense -element point set  in the plane and a stacking order  such that

\end{thm}

In the general case, where  is an arbitrary -element point set in the plane, we have been unable to improve on the easy lower bound

sketched above. However, under special assumptions on , we can do better.

\begin{thm}\label{theorem4} Every -dense -element point set  in the plane admits a stacking order  with

where  depends only on .
\end{thm}

In particular, Theorem~\ref{theorem4} provides an  lower bound for the
visible perimeter of a collection of  unit disks centered at the points of
an  uniform grid, under
a suitable stacking order. If the side length of the grid is very small, this is
better than the line-by-line ``lexicographic'' stacking order, for which the
visible perimeter is only . It turns out that
in this case there is no stacking order for which the order of the magnitude
of the visible perimeter would exceed .

\begin{thm}\label{theorem5}
Let  be a perfect square and let  denote an  by  uniform grid in the plane. For any stacking order , we have

\end{thm}

Consequently, we have .

Lemma~\ref{lemma1} implies that the worst collections of disks are those whose centers are very close to each other, so all disks have a point in common. This is, of course, not a
realistic assumption in the labeling problem in cartography that has motivated
our investigations. In practical applications, only a bounded number of unit
disks share a point. For such a case, we have the following result:

\begin{thm}\label{thm_bounded_overlap}
Let  be a collection of  unit disks in which at most  disks have a point in common. Then there exists a stacking order  for which

where  is given in (\ref{eq_def_v}). This bound is worst-case asymptotically tight.
\end{thm}

In Section 2, we establish Theorems~\ref{theorem2} and~\ref{theorem3}. The proof of Theorem~\ref{theorem4} is
presented in Section 3. In Section~4, we consider the square grid and present a
much simpler proof of this special case of Theorem~\ref{theorem4}
based on Jarnik's theorem~\cite{Ja25}; we then prove Theorem~\ref{theorem5}, which states that the
bound of Theorem~\ref{theorem4} is tight in this case. In Section~5, we prove Theorem~\ref{thm_bounded_overlap}. The last section contains concluding
remarks and open problems.

\section{Dense Sets with Largest Visible Perimeter}

In this section, we prove Theorems~\ref{theorem2} and~\ref{theorem3}.

First, we express the limit of visible perimeters in a simpler form. Given a set of points  in the plane, let 
stand for its convex hull. Let  denote the unit disk centered at
 and let  stand for the set .

Fix an orthogonal system of coordinates in the plane. For any point  and for any , let  denote the point with coordinates .


\begin{lem}\label{lemma2.1} Let  be a set of
  points in the plane, let , and let  be the stacking order of  given by  for .

We have

where , and for all other indices,  if  belongs to , and  is equal to the external angle of the convex polygon  at vertex , otherwise.
\end{lem}

\begin{figure}
\centerline{\includegraphics{fig_xang.pdf}}
\caption{\label{fig_xang}If  lies outside the convex hull of the preceding points, then  is defined as the external angle of the polygon  at vertex .}
\end{figure}

See Figure~\ref{fig_xang}.

\begin{proof}[Proof of Lemma~\ref{lemma2.1}] We prove that the contribution of
 to the visible perimeter tends to  as
 for each .

Since  is the lowest disk in , its whole
boundary is visible from below. Therefore, its contribution is . Let
.
If  belongs to the interior of , then there is a threshold  such that

for every . In this case, no portion of the boundary of  is visible from below, provided that  is sufficiently small. If
 lies on the boundary of , then it is
in between some points  and  with  and although  will not be entirely covered by earlier disks for any ,
the part of its boundary outside  tends to zero as .

Finally, if  lies outside , then it is a
vertex of . Consider the external unit normal vectors
to the two sides of  that meet at  (or in
case the convex hull is a single segment, the two unit normal vectors
for this segment). Drawing these vectors from , the arc on the
boundary of  between them is of length  and it is not covered by . Thus, it is visible from below, and, as , the total contribution of the remaining part of the boundary of  to the visible perimeter tends to , concluding the proof.
\end{proof}

{
\renewcommand{\thethm}{2}
\begin{thm}For any -dense -element point set  in the plane and for any stacking order , we have

where  is a constant depending only on .
\end{thm}
\addtocounter{thm}{-1}
}

\begin{proof}
Consider a -dense point set  in the
plane and let  be a stacking order for . Using Lemma~\ref{lemma2.1}, it
is enough to prove  for the angles  defined
in the lemma. As  whenever  is contained in
, we can assume this is never the case.

Since the quantity  is independent of scale, we can assume without
loss of generality that the minimum distance between points is
; thus, the maximum distance (diameter) is at most . We write
 with .

\begin{figure}
\centerline{\includegraphics{fig_perimeter.pdf}}
\caption{\label{fig_perimeter}The triangle  lies entirely outside the convex hull of .}
\end{figure}

For every , let  denote the perimeter of
. We define the perimeter of a segment to be
twice its length and the perimeter of a point to be . Let
, consider the two sides of the polygon  meeting at , and denote by  and  the points on these sides
at unit distance from . Since no point of  is closer to  than
, the triangle  does not contain any element of . (See Figure~\ref{fig_perimeter}.) Hence,  is contained in the
convex region obtained from  by cutting off the
triangle . (In the degenerate case when  is a
segment, we have ,
and the empty ``triangle'' becomes just a unit segment.) This observation implies that the perimeter of  satisfies

Here we used that the external angle of the triangle  at vertex  is .

Thus, we have 
for all . Adding up these inequalities, we obtain

Since  is at most  times the diameter of , that is, , we have

Applying the relationship between the arithmetic and quadratic means, we can conclude that


Taking into account that , the theorem follows by Lemma~\ref{lemma2.1}.
\end{proof}

{
\renewcommand{\thethm}{3}
\begin{thm}For every positive integer , there exists a -dense -element point set  in the plane and a stacking order  such that

\end{thm}
\addtocounter{thm}{-1}
}

\begin{figure}
\centerline{\includegraphics{fig_concentric.pdf}}
\caption{\label{fig_concentric}A dense point set that has a good stacking order.}
\end{figure}

\begin{proof}
Suppose for simplicity that  for some integer . Our point set  consists of the points having polar coordinates
 for  and . See Figure~\ref{fig_concentric}. The smallest
distance between two points in  is , and the largest distance
is less than ; thus,  is -dense, as required.

Our stacking order  takes the points by increasing , and for each  by increasing~.

We apply Lemma~\ref{lemma2.1} and calculate the sum of
the external angles determined by . Denote by  the circle of radius  centered at the origin.
Consider a point  on . Let  be the ray leaving  towards the right tangent to , and let  be the ray leaving  towards the left tangent to . Let  be the point of tangency between  and . Then all the points of  that precede  in the order  lie below  and . Thus, the external angle  contributed by  is at
least the supplement  of the angle between  and . We have 
.
The theorem follows.
\end{proof}

\section{All Dense Sets Have Good Stacking Orders}

We now turn to Theorem~\ref{theorem4}.

{
\renewcommand{\thethm}{4}
\begin{thm} Every -dense -element point set  in the plane admits a stacking order  with

where  depends only on .
\end{thm}
\addtocounter{thm}{-1}
}

Throughout this section, let  be a -dense -point set
in the plane. We will define a stacking order  for  for
which the
external angles  defined in Lemma~\ref{lemma2.1} satisfy
, for some constant
 depending only on . Then the theorem follows from
Lemma~\ref{lemma2.1}.

Assume without loss of generality that the minimum  distance in
 is . Then, since  is -dense, there exists a disk
of radius  that contains all of . Let  be such
a disk, and let  be a circle of radius 
concentric with .

\begin{figure}
\centerline{\includegraphics{fig_annular_sectors.pdf}}
\caption{\label{fig_annular_sectors}Left: Partition of  into annular sectors
centered at a point . Top right: The sector containing  is contained in
the rectangle  centered at . Bottom right: Point  contributes
external angle at least .}
\end{figure}

Given a point , we define a family   of
annular sectors that disjointly cover the plane, as follows:
For each positive integer , let  be a circle
centered at  with radius ; then divide each
annulus between two consecutive circles into sectors of angular
length  for a large enough constant
 (as will be specified below). See
Figure~\ref{fig_annular_sectors} (left).

Note that each annular sector that intersects  has area
 (since the radius of such a sector is ).
The number of annular sectors that intersect  is .
Call a sector \emph{occupied} if it contains at
least one point of .

\begin{lem}\label{lemma3.1} There exists a point 
for which  sectors of  are occupied.
\end{lem}

\begin{proof} Choose  uniformly at random on  and construct the
sectors using  and dividing the annuli into the correct-length sectors in
an arbitrary way.
For each point , define the random variable 
to be the number of points of  contained in the sector of
 that contains . We claim that the expected value
 of  satisfies

for some constant .

Indeed, let  be a rectangle centered at ,
with dimensions , and with
short sides parallel to the line , for an appropriate
constant . If  is large enough (but constant with
respect to ), then  completely contains the sector of
 that contains . See Figure~\ref{fig_annular_sectors} (top right).
Thus, it suffices to bound the
expected number of points of  in . Note that, as 
rotates around ,  rotates around its center together with
.

Partition the plane into annuli centered at  by tracing
circles around  of radii . The annulus
with inner radius  and outer radius  contains at most
 points of , for some constant . Each such
point has probability at most  of falling
in  (over the choice of ), for another constant ;
therefore, the expected contribution of this annulus to
 is at most . Summing up for all
annuli with inner radius , we obtain that
 for some constant , as claimed.

Now, call point  \emph{isolated} if . By
Markov's inequality, each point  has probability at least
 of being isolated. Therefore, the expected number of
isolated points is at least . There must exist a  that
achieves this expectation, and for it we obtain at least
 occupied sectors, proving the lemma.
\end{proof}

\begin{proof}[Proof of Theorem~\ref{theorem4}] Fix a point  for which
 has  occupied sectors. Color the sectors with
four colors, using colors  and  alternatingly on the
odd-numbered annuli and colors  and  alternatingly on the
even-numbered annuli.

There must be a color for which  sectors are
occupied. Consider only the occupied sectors with this color.
Let these sectors be , listed by
increasing distance from , and for each fixed distance, in
clockwise order around . Select one point 
from each of these sectors. Let the stacking order  start with these
points, that is,  for . The order of the
remaining points in  is arbitrary.

We claim that each selected point  contributes an external
angle of , which implies that , as desired.

Indeed, consider the -th selected point . Suppose
without loss of generality that  lies directly below .
Let  and  be the inner and outer circles bounding
the annulus that contains . Trace rays  and 
from  tangent to , touching  at points
 and . See Figure~\ref{fig_annular_sectors} (bottom right).

Every point , , that is \emph{not} contained in the
same annulus as  lies below these rays. Moreover, the
angle  that these rays make with the horizontal is
: Consider, for example, the ray . The
triangle  is right-angled, with angle . We have  and . It follows that , and so .

Now suppose that  lies in the same annulus as .
If the constant  in the definition of  is chosen
large enough, then  must have a smaller -coordinate
than . (In the worst case,  lies near the bottom-left
corner of its sector and  lies near the top-right
corner of its sector.)

Thus,  contributes external angle , as claimed.
\end{proof}

\section{The ``Worst'' Dense Set: the Grid}

In this section, we assume that  is a square number and  denotes an  by  integer grid. Note that  is a -dense set consisting of  points.

As we mentioned in the Introduction, in the special case where , Theorem~\ref{theorem4} has a simple proof. For , one can produce a stacking order with large visible perimeter using the following greedy algorithm (which can also be applied to any other point set ): Set , and select a vertex of  whose external angle is maximum. Let this vertex be , the last element in the desired order . Repeat the same step for the set , and continue in this fashion until the first element  gets defined.

By Jarnik's theorem~\cite{Ja25}, every convex polygon has  vertices in . Therefore, at each step, the greedy algorithm selects a point  that makes an external angle . Hence,  for the order . Lemma~\ref{lemma2.1} completes the proof.

Now we turn to Theorem~\ref{theorem5}.

{
\renewcommand{\thethm}{5}
\begin{thm}
Let  be a perfect square and let  denote an  by  uniform grid in the plane. For any stacking order , we have

\end{thm}
\addtocounter{thm}{-1}
}

Our proof is an improved version of the
proof of Theorem~\ref{theorem2}. There we were concerned with how the {\em perimeter} of the
convex hull grows as we add the points of our set one by one as prescribed by
the stacking order. As is well known, the perimeter of a convex set in the plane is the
integral of its width in all directions (this is known as Cauchy's theorem; see {\it e.g.} \cite{PaA95}, Theorem 16.15).
The proof of Theorem~\ref{theorem5} is very
similar, but we deal with the widths in different directions
in a non-uniform way. The width in a direction close to the
direction of a short grid vector is more important in the analysis than
widths in other directions.

\begin{proof}[Proof of Theorem~\ref{theorem5}.]
Let  be an enumeration of the points of  according to a given stacking order, and let  denote the corresponding external angles, as defined in Lemma~\ref{lemma2.1}. According to the lemma, we need to prove that . Let us partition this sum into several parts, and bound the contribution of each part separately.

Let . We start with the small angles. Let
 Clearly, we have 

\begin{figure}
\centerline{\includegraphics{fig_v_beta.pdf}}
\caption{\label{fig_v_beta}The triangle  is the largest isosceles triangle at point  that does not intersect the interior of .}
\end{figure}

As in the proof of Theorem~\ref{theorem2}, let  and denote the perimeter of  by . Since  is an  integer grid, we have . Consider only those indices  that do not belong to . For these indices, we have , so that  must be a vertex of . For each such point , let  denote the smallest number satisfying the following condition: the segment connecting the points  and  that lie on the boundary of  at distance  from , intersects . (In the case where  is a segment, we have .) Note that the segment  contains a point  with .
See Figure~\ref{fig_v_beta}.

In the proof of Theorem~\ref{theorem2}, we argued that . Now the same
argument gives that . Let
 For , we have . Since  is
monotone in , we conclude that 

Let  To bound the angles 
for indices , we need a charging scheme and we need to consider the
growth of the width of  in some specific directions. The {\em width} of a
planar set in a given direction is the diameter of the orthogonal projection
of the set to a line in this direction. Let us associate the directions in the
plane with the points of the unit circle . We identify opposite points of
this circle as the widths of the same set in opposite
directions are the same. This makes the total length of  become . We
define a set of arcs along  as follows.
For any non-zero grid vector  from the integer grid and for any integer
, let  denote the arc of length  symmetric
around the direction of . For any direction , let
 denote the width of  in the direction {\em orthogonal} to
 ({\em i.e.}, where the corresponding projection is parallel to ).

The perimeter  is equal to the integral of  along the circle  (note that after the identification of opposite points the length of  became ). We have , unless the direction  is tangent to  at the vertex . Let  denote the arc of directions where such a tangency occurs. Clearly, the length of  is , and for any arc  that contains , we have


For each index , choose a grid point  on the segment . (Recall that the points  and  are at distance  from , and that there is always a grid point between them.) We {\em charge} the index  to the pair , where  is the grid vector pointing from  to  and  is the largest integer such that  contains . Notice that . Denote by  the set of indices  that are charged to the pair .

Note that  is symmetric around the direction of the segment
. For the angle  between this direction and the
direction of  we have
 (refer again to Figure~\ref{fig_v_beta}). This implies
, and hence
. Finally, we also have


Let . The integral 
is monotone in  and grows by at least  at every
. We have  and
, so that the final integral satisfies
. Therefore, , which implies that .

Consider the set of all pairs  such that there is an index  charged to them. We have ,  and , which implies that . We proved that .
On the other hand, we also have . Thus,
for any given grid vector , there are at most 
possible values of , where  denotes the binary logarithm.

Hence, 
To evaluate this sum, we note that the number of grid vectors  satisfying  is . Thus,

In conclusion, we have 
completing the proof of the theorem.
\end{proof}

\section{Collections of disks with bounded overlap}

In this section, we prove Theorem~\ref{thm_bounded_overlap}.

{
\renewcommand{\thethm}{6}
\begin{thm}
Let  be a collection of  unit disks in which at most  disks have a point in common. Then there exists a stacking order  for which

where  is given in (\ref{eq_def_v}). This bound is worst-case asymptotically tight.
\end{thm}
\addtocounter{thm}{-1}
}

Note that Lemma~\ref{lemma2.1} is not relevant in this case, since we cannot contract the set of centers of .

\begin{proof}
Partition the plane into an infinite grid of axis-parallel square cells of side-length , where the position of the grid is chosen uniformly at random. For each unit disk, the probability that it belongs entirely to a single cell is . Thus, we
can fix the grid in such a way that at least  disks lie entirely in a cell. Let  be the number of disks entirely contained in cell . By area considerations, we have .

For each cell , we independently select a stacking order that achieves visible perimeter at least ; then we place all the remaining disks behind them. Thus, our stacking order achieves visible perimeter at least .

For any -element point set , we can take an -element point set
 as the union of  pairwise disjoint translates of . We
clearly have . This implies that
. Let , and we have
, thus


Since , the claimed bound follows.

To show that this bound is worst-case asymptotically tight, take the union of  worst-case sets of  disks far from each other.
\end{proof}

\section{Concluding remarks}

\noindent{\bf A.} The greedy algorithm described at the beginning of
Section~4 was first considered by Cabello {\em et al.}\ (unpublished) in the context
of maximizing the \emph{minimum} visible perimeter of a single disk.
They showed that the order  is always optimal
for this purpose. Unfortunately, this stacking order is \emph{not}
always optimal with respect to the total visible perimeter. Indeed, let  be
a perfect square and consider the set of points , where
the polar coordinates of  are  with 
sufficiently small. This point set is obtained as the intersection of
 equally-spaced rays
emanating from the origin, and  ``rounds'' of a very tight
logarithmic spiral centered at the origin. The greedy algorithm produces the
stacking order indicated by the indices, so it takes the
points of  outwards along the spiral. The contribution  is equal for
every point  with  and tends to  as 
goes to zero, making
 if  is small enough. However, taking the points ray by ray in a cyclic
order, going outwards along each ray, the contribution  is a constant
for the first half of the points, making .

\medskip

\noindent{\bf B.}
Theorem 4 can be generalized to point sets satisfying weaker density conditions. Indeed, let  be a set of  points in the plane with diameter  and minimum distance . A randomized construction, similar to the one used in the proof Theorem~4, guarantees the existence of a stacking order  such that . This beats the  bound
mentioned in the Introduction as long as .

\medskip

\subsubsection*{Acknowledgements} The authors express their gratitude to Radoslav Fulek and Andres Ruiz Vargas (EPFL), for many insightful discussions on the subject, as well as to the anonymous referees for their useful comments.

	\begin{thebibliography}{}

    \bibitem[AlKP89]{AlKP89}
    N. Alon, M. Katchalski, and W. R. Pulleyblank: The maximum size of a convex polygon in a restricted set of points in the plane, {\em Discrete Comput. Geom.} {\bf 4} (1989), 245--251.

	\bibitem[CaH10]{CaH10}
	S. Cabello, H. Haverkort, M. van Kreveld, and B. Speckmann:
    Algorithmic aspects of proportional symbol maps, {\em Algorithmica} {\bf 58} (2010), 543--565.

    \bibitem[De99]{De99}
    B. Dent: {\em Cartography. Thematic Map Design, 5th edn}, McGraw-Hill, New York, 1999.

    \bibitem[ErSz35]{ErSz35}
    P. Erd\H os and G. Szekeres: A combinatorial problem in geometry, {\em Compositio Math.} {\bf 2} (1935), 463--470.
    
    \bibitem[Er03]{Er03} J. Erickson: Nice point sets can have nasty Delaunay triangulations, {\em Discrete Comput. Geom.} {\bf 30} (2003), 109--132.

    \bibitem[Gr90]{Gr90}
    T. Griffin: The importance of visual contrast for graduated circles, {\em Cartography} {\bf 19} (1990), 21--30.

    \bibitem[GrC78]{GrC78}
    R. E. Groop and D. Cole: Overlapping graduated circles: Magnitude estimation and method of portrayal, {\em Can. Cartogr.} {\bf 15} (1978), 114--122.

    \bibitem[Ja25]{Ja25}
    V. Jarn\'{\i}k: \"Uber die Gitterpunkte auf konvexen Kurven, {\em Mathematische Zeitschrift} {\bf 24} (1926), 500--518.

    \bibitem[PaA95]{PaA95}
    J. Pach and P. K. Agarwal: {\em Combinatorial Geometry}, Wiley, New York, 1995.

    \bibitem[SlM03]{SlM03}
    T. A. Slocum, R. B. McMaster, F. C. Kessler, and H. H. Howard: {\em Thematic Cartography and Geographic Visualization, 2nd edn}, Prentice Hall, New York, 2003.

    \bibitem[Va92]{Va92}
    P. Valtr: Convex independent sets and 7-holes in restricted planar point sets, {\em Discrete Comput. Geom.} {\bf 7} (1992), 135--152.

    \bibitem[Va96]{Va96}
    P. Valtr: Lines, line-point incidences and crossing families in dense sets, {\em Combinatorica} {\bf 16} (1996), 269--294.

\end{thebibliography}
\end{document}
