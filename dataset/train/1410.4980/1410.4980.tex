\documentclass{LMCS}

\def\dOi{10(4:10)2014}
\lmcsheading {\dOi}
{1--41}
{}
{}
{Dec.~30, 2013}
{Dec.~16, 2014}
{}

\ACMCCS{[{\bf Theory of computation}]: Semantics and
  reasoning---Program semantics}

\subjclass{F.3.2 Semantics of Programming Languages}

\usepackage[utf8]{inputenc}
\usepackage{amssymb}
\usepackage{stmaryrd}
\usepackage{graphicx}
\usepackage{bussproofs}
\usepackage{hyperref}
\usepackage{xspace}

\mathcode`\<="4268      \mathcode`\>="5269      \mathchardef\gt="313E \mathchardef\lt="313C 

\theoremstyle{definition}
\newtheorem{definition}[thm]{Definition}
\newtheorem{example}[thm]{Example}
\theoremstyle{plain}
\newtheorem{lemma}[thm]{Lemma}
\newtheorem{proposition}[thm]{Proposition}
\newtheorem{theorem}[thm]{Theorem}
\newtheorem{remark}[thm]{Remark}

\newcommand{\red}{\longrightarrow}

\newcommand{\lollipop}{\to}
\newcommand{\lolli}[3]{#1\cdot #2 \to #3}
\newcommand{\lollid}[2]{#1\to #2}
\newcommand{\kw}[1]{\mathsf{#1}}

\newcommand{\NN}{\mathbb{N}}
\newcommand{\unit}{\mathtt{unit}}
\newcommand{\VN}{\mathtt{nat}}
\newcommand{\true}{\mathit{true}}
\newcommand{\false}{\mathit{false}}
\newcommand{\length}[1]{\mathit{length}(#1)}

\newcommand{\Capply}{\textit{apply}}

\newcommand{\Mid}{\ \mathrel{\big|} \ }
\newcommand{\I}[2]{#1 \colon #2}
\newcommand{\J}[3]{#1 \colon #2 \cdot #3}
\newcommand{\R}[1]{\textsc{#1}}
\newcommand{\Seq}[2]{#1 \vdash #2}
\newcommand{\SeqTm}[3]{#1 \vdash #2 \colon #3}
\newcommand{\SeqCps}[3]{#1 \vdash_{\kw{cps}} #2 \colon #3}
\newcommand{\SeqE}[3]{#1 \vdash #2 \colon #3}
\newcommand{\SeqP}[4]{#1\ \mid \ #3 \ \colon\ #2 \to #4}
\newcommand{\SeqPc}[3]{#2 \ \colon\ #1 \to #3}
\newcommand{\SeqDf}[4]{#1 \vdash #2 \colon #3} \newcommand{\SeqI}[5]{#1 \vdash #2 \colon #3;#4 \Downarrow #5}
\newcommand{\SeqW}[3]{#1 \,\vdash\, #2\,\, {:}\,\, #3}
\newcommand{\SeqU}[3]{#1 \,\vdash\, #2\,\, {:}\,\, #3}
\newcommand{\Defun}[3]{#1 \ \Downarrow\ #2\, ;\, #3}
\newcommand{\LL}{\mathcal{L}}
\newcommand{\LT}{\mathcal{L}_T}
\newcommand{\LeftLabelSc}[1]{\LeftLabel{\textsc{#1}}}
\newcommand{\sleq}{\triangleleft}

\newcommand{\xto}{\xrightarrow}
\newcommand{\id}{\mathit{id}}
\newcommand{\inl}{\kw{inl}}
\newcommand{\inr}{\kw{inr}}

\newcommand{\tlam}[2]{\lambda {#1}.\, {#2}}
\newcommand{\tlaml}[4]{\lambda^{#3} {#1}{:}{#2}.\, {#4}}
\newcommand{\tlamlu}[4]{\lambda^{#3} {#1}.\, {#4}}
\newcommand{\tlami}[3]{\lambda {#1}{:}{#2}.\, {#3}}
\newcommand{\tappl}[3]{{#1} {\,@_{#2}\,} {#3}}
\newcommand{\tlet}[3]{\kw{let}\ #2=#1\ \kw{in}\ #3}
\newcommand{\tif}[3]{\kw{if0}\ #1\ \kw{then}\ #2\ \kw{else}\ #3}
\newcommand{\tcase}[5]{\kw{case}\ #1\ \kw{of}\ \kw{inl}(#2) \Rightarrow #3;\, \kw{inr}(#4) \Rightarrow #5}
\newcommand{\tmcase}[5]{\!\begin{aligned}[t]
  \kw{case}\ #1\ \kw{of}\ 
     \kw{inl}(#2) &\Rightarrow #3 \
\Capply(f, a) =
\begin{aligned}[t]
  \kw{case}\ f\ \kw{of}\
   &l_1\Rightarrow \tlet a {<x,k>} {\Capply(l_2, l_3(x,k))}\\
   \mid\ &l_2 \Rightarrow \Capply(a, 1)\\
   \mid\ &l_3(x,k) \Rightarrow \Capply(x, l_4(k,a))\\
   \mid\ &l_4(k,u) \Rightarrow \Capply(k, u + a)
\end{aligned}

\Capply(l, a) =
\begin{aligned}[t]
\kw{case}\ l\ \kw{of}\ 
   &\dots \\
   \mid\ &l_5 \Rightarrow \Capply(a, 42)\\
   \mid\ &l_6 \Rightarrow \texttt{print\_int}(n)
\end{aligned}

\begin{aligned}
  \label{eq:defun}
  \Capply_{l_1}() &= \Capply_{l_2}() &
  \Capply_{l_2}() &= \Capply_{l_3}(1) \\ 
  \Capply_{l_3}(u) &= \Capply_{l_5}(u) &
  \hspace{1.9cm}\Capply_{l_4}(u, n) &= \Capply_{l_6}(u+n) \\
\end{aligned}

  \Capply_{l_5}(u) &= \Capply_{l_4}(u,42) 
  & \Capply_{l_6}(n) &= \texttt{print\_int}(n)

\textsc{core}
\ \subseteq\
\textsc{lin}
\ \subseteq\
\textsc{stl}
\ \subseteq\
\textsc{source}

\begin{aligned}
  {\textit{const}}(x) &= {\textit{const\_ret}}(23) \\
  {\textit{pow}}(<x,y>) &= {\textit{pow\_loop}}(<x,y>) \\
  {\textit{pow\_loop}}(<x,y>) &= \tmcase {\kw{iszero}(x)} {z} {{\textit{pow\_ret}}(y)} {z} {{\textit{pow\_loop}}(<x-1,y*y>)}
\end{aligned}

  \text{Types: } && 
   A, B &::= \alpha \Mid \unit \Mid \VN \Mid A \times B \Mid A + B \Mid \mu \alpha.\, A\\
  \text{Expressions: } && e,e_1,e_2 &::= 
    \begin{aligned}[t]
      x &\Mid <>
      \Mid n \Mid e_1 + e_2 \Mid \kw{iszero}(e) \\
      &
      \Mid <e_1, e_2> \Mid \tlet {e_1} {<x, y>} {e_2}\\
      &
      \Mid \kw{inl}(e) \Mid \kw{inr}(e) 
      \Mid \tcase {e} x {e_1} y {e_2}\\
      &
      \Mid \kw{fold}_A(e) \Mid \kw{unfold}_A(e)
    \end{aligned}
  
  \beta\ \kw{list} = \kw{datatype}\ \kw{nil}\ \kw{of}\ \kw{unit} \mid
  \kw{cons}\ \kw{of}\ \beta \times (\beta\ \kw{list})

   \text{Values:}&&
   v, w &::= <> 
   \Mid n
   \Mid <v, w>
   \Mid \kw{inl}(v) \Mid \kw{inr}(v) 
   \Mid \kw{fold}_A(v)
   \\
   \text{Evaluation Contexts:}&&
   C &\mathrel{::=} [] \Mid C + e
   \Mid v + C
   \Mid \kw{iszero}(C)
   \\
 &&&\mathrel{\phantom{::=}} \phantom{[]} \Mid <C, e> \Mid <v, C>
   \Mid \tlet {C} {<x, y>} {e}
   \\
   &&&\mathrel{\phantom{::=}} \phantom{[]} 
   \Mid \kw{inl}(C) \Mid \kw{inr}(C)
   \Mid \tcase {C} x {e_1} y {e_2}

  n_1 + n_2 &\red n_3 \text{ if~ is the sum of~ and~}\\
  \kw{iszero}(0) &\red \kw{inl}(<>)
  \\
  \kw{iszero}(n) &\red \kw{inr}(<>) \text{ if~ is non-zero}
  \\
  \tlet{<v_1,v_2>}{<x,y>}{e} &\red e[v_1/x, v_2/y]
  \\
  \tcase{\kw{inl}(v)} x {e_1} y {e_2} &\red e_1[v/x] 
  \\
  \tcase{\kw{inr}(v)} x {e_1} y {e_2} &\red e_2[v/x] 
  \\
  \kw{unfold}(\kw{fold}(v)) & \red v
  \\
  C[e_1] &\red C[e_2] \text{ if }

    f(x) &= g(e),
    &
    f(x) &= \tcase e y {g(e_1)} z {h(e_2)},

  \text{Types: } && 
  X, Y &\ ::=\  1  \Mid X \to Y \Mid \NN  \\
  \text{Terms: } && s, t  &\ ::=\  
    * \Mid \tlami x X t \Mid s\ t 
     \Mid n \Mid s+t 
     \Mid \tif s {t_1} {t_2}
     \Mid \kw{fix}_X

  \cps{1} &= \neg 1 
  & \cps{\NN} &= \neg \NN
  & \cps{X \to Y} &= \neg \cps X \times \cps Y
1em]
  \AxiomC{\phantom{X}}
  \UnaryInfC{}
  \DisplayProof
  &\  &
  \AxiomC{\phantom{X}}
  \UnaryInfC{}
  \DisplayProof
\1em]
    \AxiomC{}
    \UnaryInfC{}
  \DisplayProof
  &\  &
    \AxiomC{}
    \UnaryInfC{}
  \DisplayProof
\1.5em]
  \AxiomC{}
  \AxiomC{}
  \BinaryInfC{}
  \DisplayProof
  &\  &
  \AxiomC{}
  \AxiomC{}
  \BinaryInfC{}
  \DisplayProof
\1em]
  \AxiomC{}
  \AxiomC{}
  \BinaryInfC{}
  \DisplayProof
  &\ \ &
  \AxiomC{}
  \AxiomC{}
  \BinaryInfC{}
  \DisplayProof
\2.5em]
  \AxiomC{}
  \UnaryInfC{}
  \DisplayProof
  &\  &
  \AxiomC{}
  \UnaryInfC{}
  \DisplayProof
  \
  \eta(t, X) &= t \text{ if  is a base type (,  or )}\\
  \eta(t, X\times Y) &= \tlet t {<x,y>} <\eta(x,X), \eta(y, Y)>\\
  \eta(t, X\to Y) &= \tlam x {\eta(t\ \eta(x, X), Y)} \text{ where~ is fresh.}

  x^* &= x\\
  n^* &= n\\
  (\tif s {t} {u})^* &= \tcase {\kw{iszero}(s^*)} {\_} {t^*} {\_} {u^*} \\
  <s,t>^* &= <s^*, t^*>\\
  (\tlet t {<x,y>} s)^* &= \tlet{t^*}{<x,y>}{s^*}\\
  (\tappl s l t)^* &= \textit{apply}_l(s^*, t^*) \\  
  (\tlaml x A l t)^* &= {<x_1,\dots, x_n>} \text{ where }

  D(x) &= \emptyset\\
  D(n) &= \emptyset\\
  D(\tif s {t} {u}) &= D(s)\cup D(t)\cup D(u) \\
  D(<s,t>) &= D(s) \cup D(u) \\
  D(\tlet t {<x,y>} s) &= D(t) \cup D(s) \\
  D(\tappl s l t) &= D(s) \cup D(t) \\  
  D(\tlaml x A l t) &= D(t) \cup \{\textit{apply}_l(<x_1,\dots, x_n>, x)=t^*\} 

  \lambda^{l_1} z.\, \tlet {z} {<x,k>} {
\tappl {(\lambda^{l_2}k'.\, \tappl {k'} {l_3} 1)} {l_2} {(\lambda^{l_3} u.\, \tappl x {l_5} {(\lambda^{l_4} n.\, \tappl k {l_6} {(u+n)}))}}}.

  \Capply_{l_1}(<>, <x,k>) &=\  \Capply_{l_2}(<>,<x,k>) ,
  &
  \Capply_{l_2}(<>, k') &=\  \Capply_{l_3}(k', 1) ,
  \\
  \Capply_{l_3}(<x,k>, u) &=\  \Capply_{l_5}(x, <k,u>) ,
  &
  \Capply_{l_4}(<k,u>, n) &=\ \Capply_{l_6}(k, u+n) .

  \Capply_{l_5}(<>, k) &=\  \Capply_{l_4}(k, 42) ,
  &
  \Capply_{l_6}(<>, n) &=\ \texttt{print\_int}(n),

 \text{Types: } && 
 X, Y &\ ::=\ 1  \Mid X \to Y  \\
 \text{Terms: } && s, t &\ ::=\  * 
  \Mid \tlami x X t \Mid s\ t 

  1^- &= \unit
  &
  (X \lollipop Y)^- &= Y^- X^+
  \\
  1^+ &= \unit
  &
  (X \lollipop Y)^+ &= Y^+ X^-

  \sem{\SeqTm{\Gamma}{t}{X}}\colon X^-\Gamma^+ \to X^+\Gamma^-

  \kw{Erase}(i, E, o) := 
   (i, \{f() = g() \mid f(x) = g(e)  \in E\}, o)
 
    D=\{\Capply_{x_1^-(i)}() = \Capply_{x_2^-(i)}(),
    \Capply_{x_2^+(i)}() = \Capply_{x_1^+(i)}() \mid i=1,\dots,n \},
  
   \kw{Erase}(D(\tlamlu {<x,k>} {\neg\cps X  \times \cps Y} q {\tappl{\cps t}{q_t}{k}}))
   =
   \{\Capply_q() = \Capply_{q_t}()\} \cup \kw{Erase}(D(\cps t))
  
    (qz\delta^+\gamma^+, 
    \{\Capply_q() = \Capply_{q_s}()\} \cup D(\cps s) \cup D(\cps t),
    y^+\delta^-\gamma^-).
  
  \text{Types: } && 
  X, Y &\ ::=\  1  \Mid X \to Y \Mid \NN  \\
  \text{Terms: } && s, t  &\ ::=\  * 
  \Mid \tlami x X t \Mid s\ t 
  \Mid n \Mid s+t 
  \Mid \tif s {t_1} {t_2}

  1^- &= \unit
  &
  \NN^- &= \unit
  &
  (X \lollipop Y)^- &= Y^- X^+
  \\
  1^+ &= \unit
  &
  \NN^+ &= \VN
  &
  (X \lollipop Y)^+ &= Y^+ X^-

  X, Y \ ::=\  1  \Mid \lolli A X Y \Mid \NN  

  (\lolli A X Y)^- &= Y^- (A\times X^+) 
  &
  \Gamma^- &= A_n\times X^-_n\dots A_1\times X^-_1
  \\
  (\lolli A X Y)^+ &= Y^+ (A\times X^-)
  &
  \Gamma^+ &= A_n\times X^+_n\dots A_1\times X^+_1,

D(\cps{s}) \cup D(\cps{t}) \cup
\{
  \begin{aligned}[t]
    &\Capply_q(<>, k) =\Capply_{q_s}(\cps s^*, <k>), \\
    &\Capply_{a_s}(<k>, x) =\Capply_{q_t}(\cps t^*, <k ,x>),\\
    &\Capply_{a_t}(<k,x>, y) =\Capply_{a}(k, x+y) \} .
\end{aligned}

  D(\cps{s}) \cup D(\cps{t_1}) \cup D(\cps{t_2}) \cup 
\{
  \begin{aligned}[t]
    &\Capply_q(<>, k) =\Capply_{q_s}(\cps s^*, <k>), \\
    &\Capply_{a_s}(<k>, x) = 
      \tmcase {\kw{ifzero}(x)}
      {\_} {\Capply_{q_1}(\cps{t_1}^*, <k>)}
      {\_} {\Capply_{q_2}(\cps{t_2}^*, <k>)}
      \\
      &\Capply_{a_1}(<k>, y) =\Capply_{a}(k, y) ,\\
      &\Capply_{a_2}(<k>, y) =\Capply_{a}(k, y) \},
\end{aligned}

        I(s\ t):= I(s) \cup I(A\cdot t) \cup\{ \textit{apply}_q() = \textit{apply}_{q_s}() \}.
     
       I(s+t) := 
         I(s) \cup I(\VN\cdot t) \cup
       \{
       \begin{aligned}[t]
          &\textit{apply}_q() = \textit{apply}_{q_s}(), \\
          &\textit{apply}_{a_s}(x) = \textit{apply}_{q_t}(x,<>), \\
          &\textit{apply}_{a_t}(x,y) = \textit{apply}_{a}(x + y)\}
       \end{aligned}
     
       I_{\cps s} \cup I_{\cps {t_1}} \cup I_{\cps {t_2}} \cup
       \{\begin{aligned}[t]
           &\textit{apply}_q() = \textit{apply}_{q_s}(), \\
           &\textit{apply}_{a_s}(x) = \tmcase {\kw{iszero}(x)} y {\textit{apply}_{q_1}(y)} z {\Capply_{q_2}(z),} \\
           &\textit{apply}_{a_1}(x) = \textit{apply}_{a}(x),\\
           &\textit{apply}_{a_2}(x) = \textit{apply}_{a}(x)\}.
           \rlap{\hbox to 178 pt{\hfill\qEd}}
       \end{aligned}
    
  \lambda^{l_0} k.\, \tappl{\cps t} {l_1} {<\lambda^{l_5} k''.\, \tappl {k''}
    {l_4} {42}, k>}
  \colon \neg_{l_0} \neg_{l_6} \NN \enspace,

  \Capply_{l_0}(<>, k) &=\  \Capply_{l_1}(<>,<<>,k>),
  &
  \Capply_{l_1}(<>, <x,k>) &=\  \Capply_{l_2}(<>,<x,k>) ,
  \\
  \Capply_{l_2}(<>, k') &=\  \Capply_{l_3}(k', 1) ,
  &
  \Capply_{l_3}(<x,k>, u) &=\  \Capply_{l_5}(x, <k,u>) ,
  \\
  \Capply_{l_4}(<k,u>, n) &=\ \Capply_{l_6}(k, u+n) ,
  &
  \Capply_{l_5}(<>, k'') &=\  \Capply_{l_4}(k'',42).

  \Capply_{l_0}() &=\  \Capply_{l_1}(),
  &
  \Capply_{l_1}() &=\  \Capply_{l_2}() ,
  \\
  \Capply_{l_2}() &=\  \Capply_{l_3}(1) ,
  &
  \Capply_{l_3}(m) &=\  \Capply_{l_5}(m) ,
  \\
  \Capply_{l_4}(m, n) &=\ \Capply_{l_6}(m+n),
  &
  \Capply_{l_5}(m) &=\  \Capply_{l_4}(m,42) .

&\Capply_{l_0}(<>,<>)\  
\Capply_{l_1}(<>,<<>,<>>) \ 
\Capply_{l_2}(<>,<<>,<>>)\ 
\Capply_{l_3}(<<>,<>>, 1)\ 
\Capply_{l_5}(<>, <<>, 1>)\ 
\\
&
\Capply_{l_4}(<<>,1>, 42)\ 
\Capply_{l_6}(<>, 43)
\intertext{The call trace of the second program is:}
&\Capply_{l_0}()\ 
\Capply_{l_1}() \
\Capply_{l_2}()\
\Capply_{l_3}(1)\
\Capply_{l_5}(1)\
\Capply_{l_4}(1, 42)\
\Capply_{l_6}(43)

    c_i = 
    \begin{cases}
      \Capply_{\rho(l_i)}(v_i) & \text{if ,} \\
      \varepsilon & \text{otherwise.}
    \end{cases}
  
    c_i = 
    \begin{cases}
      \Capply_{\rho(l_i)}(w_i) & \text{if ,} \\
      \varepsilon & \text{otherwise.}
    \end{cases}
  
      \SeqTm{}
      {\lambda^q k.\, \tappl {\cps{s_1}} {q_1} 
      {(\lambda^{a_1} x.\, \tappl {\cps{s_2}} {q_2} {(\lambda^{a_2} y.\, \tappl k a {(x+y)})})}}
      {\overline \NN[q,a]},
      
        D(\cps{s_1}) \cup D(\cps{s_2}) \cup \{
          \begin{aligned}[t]
            &\Capply_q(<>, k)=\Capply_{q_1}(<>, <k>),\\
            &\Capply_{a_1}(<k>, x)=\Capply_{q_2}(<>, <k,x>),\\
            &\Capply_{a_2}(<k,x>, y)=\Capply_{a}(k, x+y)\}.
          \end{aligned}
      
        \semc{s_1} \cup \semc{\VN\cdot s_2} \cup \{
          \begin{aligned}[t]
           &\Capply_q()=\Capply_{q_1}(),\\
           &\Capply_{a_1}(x,<>)=\Capply_{q_2}(x),\\
           &\Capply_{a_2}(x,y)=\Capply_{a}(x+y)\}.
          \end{aligned}
      
      \Capply_q(<>, <>)\ \tau_1\ \tau_2\ \Capply_a(<>, x+y),
      
      \tau_1 &=\Capply_{q_1}(<>, <<>>)\dots \Capply_{a_1}(<>,x)\\
      \tau_2 &= \Capply_{q_2}(<>,<<>,x>) \dots \Capply_{a_2}(<<>,x>, y)
      
  \text{Types: } && 
  X, Y &\ ::=\  1  \Mid X \to Y \Mid \NN  \\
  \text{Terms: } && s, t  &\ ::=\   * 
  \Mid \tlami x X t \Mid s\ t 
  \Mid n \Mid s+t 
  \Mid \tif s {t_1} {t_2}

(\tappl s {\{l_1,\dots,l_n\}} t)^* = \fcase {s^*} 
{l_1(\vec x)} {\Capply_{l_1}(l_1(\vec x), t^*) ;  \dots} 
{l_n(\vec y)} {\Capply_{l_n}(l_n(\vec y), t^*) }.

  L_1, L_2 \ ::=\  l \Mid L_1 + L_2

  \text{Types: } && 
  X, Y &\ ::=\  1  \Mid X \xto L Y \Mid X\times Y\Mid \NN \Mid \bot \\
  \text{Terms: } && s, t  &\ ::=\  
  \begin{aligned}[t]
  * 
  &\Mid \tlaml x X l t \Mid \tappl s L t 
  \Mid <s, t> \Mid \tlet {s} {<x, y>} {t} 
  \\
  &\Mid n \Mid s+t 
  \Mid \tif s {t_1} {t_2} 
  \\
  &\Mid \Coercl{L}{t} \Mid \Coercr{L}{t}
\end{aligned}

  (\tappl s L t)^* &= \Capply_L(s^*, t^*)\\
  (\tlaml x X l t)^* &= l(x_1,\dots, x_n) \text{ where } \\
  (\Coercl{L}{t})^* &= \kw{inl}(t^*) \\
  (\Coercr{L}{t})^* &= \kw{inr}(t^*) \\
\intertext{Definitions:}
  D(\tappl s L t) &= D(s) \cup D(t) \cup D(L)\\
  D(\tlaml x X l t) &= D(t) \cup \{ \Capply_l(l(x_1,\dots, x_n), x) = t^* \} \\
  D(\Coercl{L}{t}) &= D(t) \cup D(L)\\
  D(\Coercr{L}{t}) &= D(t) \cup D(L)

  D(l) &= \emptyset \\
  D(L_1 + L_2) &= D(L_1) \cup D(L_2) \cup 
   \{\Capply_{L_1+L_2}(f,x) =
   \tmcase f {f_1} {\Capply_{L_1}(f_1, x)} {f_2} {\Capply_{L_2}(f_2, x)\}}

  \tau_{l} = \kw{datatype}\ l\ \kw{of}\ A

    \lambda^{l_1} <x,k>.\, 
      \tappl x {l_4} 
         {\coercl{l_2+l_3}{\lambda^{l_2} m.\, \tappl x {l_4} {\coercr{l_2+l_3}{\lambda^{l_3} n.\, \tappl k {l_5}{(m+n)}}}}}
  
    \Capply_{l_1}(l_1(), <x,k>) &= \Capply_{l_4}(x, \kw{inl}({l_2}(x,k)))
    \\
    \Capply_{l_2}(l_2(x,k), m) &= \Capply_{l_4}(x, \kw{inr}(l_3(m,k)))
    \\
    \Capply_{l_3}(l_3(m,k),n) &= \Capply_{l_5}(k, m+n) 
    \\
    \Capply_{l_4}(l_4(), k) 
    &=  \Capply_{l_2 + l_3}(k, 42)\\
    \Capply_{l_2+l_3}(k, n)  
    &=
    \tmcase k {f_1} {\Capply_{l_2}(f_1, n)} 
    {f_2} {\Capply_{l_3}(f_2, n)} 
    \\
    \Capply_{l_5}(l_5(), n) &= \texttt{print\_int}(n)
  
  \tau_{l_1} &= \kw{datatype}\ l_1\ \kw{of}\ \unit 
  &
  \tau_{l_2} &= \kw{datatype}\ l_2\ \kw{of}\ \tau_{l_4} \times \tau_{l_5} \\
  \tau_{l_3} &= \kw{datatype}\ l_2\ \kw{of}\ \VN \times \tau_{l_5} 
  &
  \tau_{l_4} &= \kw{datatype}\ l_4\ \kw{of}\ \unit \\
  \tau_{l_5} &= \kw{datatype}\ l_5\ \kw{of}\ \unit 

\SeqTm{\I {y} {\neg_{q_1} \neg_{a_1} \NN},\, \I {z} {\neg_{q_2} \neg_{a_2}
    \NN}}{\cps t}{\overline Y[y^-,y^+]}.

\SeqTm{\I x {\neg_q \neg_{a'_1+a'_2} \NN}}
{
  \lambda^{q_1} k.\, 
  \tappl x {q} 
  {\coercl {} {\lambda^{a_1'} y.\, \tappl k {a_1} y}}
}{\neg_{q_1} \neg_{a_1} \NN }
\\
\SeqTm{\I x {\neg_q \neg_{a'_1+a'_2} \NN}}
{
\lambda^{q_2} k.\, \tappl x {q} {\coercr {} {\lambda^{a_2'} y.\, \tappl k {a_2} y}}
}{\neg_{q_2} \neg_{a_2} \NN }

    \SeqTm{\I x {\overline X[q, a'_1 + a'_2]}}{t_1}{\overline X[q_1, a_1]}
      \qquad\text{and}\qquad
      \SeqTm{\I x {\overline X[q, a'_1 + a'_2]}}{t_2}{\overline X[q_2, a_2]}
  
        \SeqTm{\overline \Gamma[\gamma^-,\gamma^+],\,
        \I {x} {\overline X[x^-, a_1' + a_2']}}
        {\cps{t[x/x_1,x/x_2]}}{\overline Y[y^-, y^+]}
      
  \label{eq:1}
  \lambda^{l_1} <x,k>.\, \tappl {t_1} {q'} {(\lambda^{l_2} m.\, \tappl {t_2} {q''} {(\lambda^{l_3} n.\, \tappl k {l_5} {(m+n)})})}

  \Capply_{l_1}(<>, <x,k>) &= \Capply_{q'}(q'(x), {l_2}(x,k))
  \\
  \Capply_{l_2}(l_2(x,k), m) &= \Capply_{q''}(q''(x), l_3(m,k))
  \\
  \Capply_{l_3}(l_3(m,k),n) &= \Capply_{l_5}(k, m+n),

\begin{aligned}
  \label{eq:eta}
  \Capply_{q'}(q'(x), k) &= \Capply_{l_4}(x, \kw{inl}(a'(k)))
  &
  \Capply_{a'}(a'(k), n) &= \Capply_{l_2}(k, n)
  \\
  \Capply_{q''}(q''(x), k) &= \Capply_{l_4}(x, \kw{inr}(a''(k)))
  &
  \Capply_{a''}(a''(k), n) &= \Capply_{l_3}(k, n)
  \\
  \Capply_{a'+a''}(k, n)  
  &=
  \tmcase k {f_1} {\Capply_{a'}(f_1, n)} 
            {f_2} {\Capply_{a''}(f_2, n)} 
\end{aligned}

  \Capply_{l_4}(<>, k) 
  &=  \Capply_{a' + a''}(k, 42)
  \\
  \Capply_{l_5}(<>, n) &= \texttt{print\_int}(n)

    \Capply_{q_1(i)}(-, -) &= \Capply_{q(i)}(-, \kw{inl}(-))\\
    \Capply_{q_2(i)}(-, -) &= \Capply_{q(i)}(-, \kw{inr}(-))\\
    \Capply_{(a_1' + a_2')(i)}(x, -) &= \tmcase x y {\Capply_{a_1(i)}(y, -)} z
   {\Capply_{a_2(i)}(z, -)}\\
   \Capply_{a_1'(i)}(-, -) &= \Capply_{a_1(i)}(-, -)
  
    \cps t &= \lambda^{l_1}<g,k>.\, 
      \tappl g {l_4} 
      {<\coercl{}{\lambda^{l_2}<x,k_1>.\, \tappl g {l_4} {<\coercr{}{\lambda^{l_3}<y, k_3>.\, \tappl x {l_5+l_6} {k_3}}, k_1>}}, k>}
  \\
  \cps s &= \lambda^{l_4} <f,k>.\, 
     \tappl f {l_2+l_3} 
     { <\coercl{}{\lambda^{l_5} k_2 .\, \tappl f {l_2+l_3} {<\coercr{}{\lambda^{l_6} <x, k_1>.\, \tappl x {l_8} {k_1}}, k_2>}}, 
           k> }           
  \\
  \cps{t\ s} &= \lambda^{l_7} k.\, \tappl{\cps t} {l_1} {<\cps s, k>}

  \tau_{l_1} &= \kw{datatype}\ l_1\ \kw{of}\ \unit 
  &
  \tau_{l_2} &= \kw{datatype}\ l_2\ \kw{of}\ \tau_{l_4} \\
  \tau_{l_3} &= \kw{datatype}\ l_3\ \kw{of}\ (\tau_{l_5} + \tau_{l_6}) 
  &
  \tau_{l_4} &= \kw{datatype}\ l_4\ \kw{of}\ \unit \\
  \tau_{l_5} &= \kw{datatype}\ l_5\ \kw{of}\ (\tau_{l_2} + \tau_{l_3})  
  &
  \tau_{l_6} &= \kw{datatype}\ l_6\ \kw{of}\ \unit

  \lambda^{q_1} <f,k>.\, 
  \kw{fix}_{\overline X}\ (\lambda^{q_4} g.\, \lambda k_1.\,f\ <\lambda^{q_3} k_2.\,g\ k_2,\, \lambda^{q_2} x.\, k_1\ x>)\ k
  \enspace.

  \Capply(q_1(), <f,k>) &= \Capply(q_4(f), k) 
  &
  \Capply(q_2(k_1), x) &= apply(k_1, x)
  \\
  \Capply(q_3(g), k_2) &= apply(g, k_2)
  &
  \Capply(q_4(f), k_1) &= \Capply(f, <q_3(q_4(f)), q_2(k_1)>)

Informally, the first three definitions to correspond to the inputs of
the Int-interpretation above. A call to  starts
the recursion, a call to  corresponds to
the step function (that the fixed point is taken of) returning a result,
and a call to  corresponds to the step function
requesting its argument. The final equation does not contribute to the
external interface of the program and is used to implement the fixed point.
The call stack, which above is encoded using lists, appears more
implicitly in the continuations here.

\section{Conclusion}

We have observed that the non-standard compilation methods based on
computation by interaction are closely related to CPS-translation and
defunctionalization.  The interpretation in an interactive model may
be regarded as a simple direct description of the combination of
CPS-translation, defunctionalization and a final optimisation of
arguments.  It may be seen as a 
simple nameless formulation of a combined CPS-translation and
defunctionalization and it provides an alternative way of encoding 
continuations.

We have seen in this paper that working out the
technical details of defunctionalization with explicit labels can
become quite technical. The interactive interpretation admits 
a high-level description that abstracts from implementation details.
Interactive model constructions may perhaps be
useful in simplifying uses of defunctionalization.
In the other direction,
being aware that interactive models are related to standard
compilation methods may help to improve non-standard compilation
schemes based on interactive methods, such as~\cite{Ghica07,intml}. We
may hope that some of the many existing techniques for compiler
implementation can be adapted usefully to the non-standard schemes.

Types with subexponential annotations, in this form originally introduced in IntML~\cite{intml}, 
provide a logical account for the issues of managing value
environments that are inherent to defunctionalization.
With subexponential annotations, the type of a higher-order term fully 
specifies the interface of the target program obtained from it.
The type system contains enough information in order to give a fully
compositional definition of the translation to the target language.

The subexponential type system makes explicit issues 
that appear with defunctionalization and separate compilation. 
For example, in order to suppose we want to compile a function 
separately from the rest of the program. Then one may compile
the main program  
and  separately. A linker can combine the
resulting two programs knowing only their types.
Of course, the problems associated with defunctionalization and separate
compilation do not just disappear. 
Suppose, for example, we only know the term~, but not the
program~ in which it will be used.
Suppose further that  has the form . Then 
the choice of the subexponential annotation~ will limit 
which arguments~ can be applied to;  can only be applied to arguments
of type  with . Choosing~ without
knowledge of~ is possible, for example, if 
the target language has a type  
with  for any type~ (anything can be stored on the
heap). Then one may simply choose  to be . 
This is not the only possible choice; for performance reasons
one may consider performing a more precise analysis in a linker or similar.
The point is that the subexponential type system allows us to express
such issues at a high level and to apply different possible solutions.
Another example for this point is the explanation of the appearance
of recursive target types in Section~\ref{sect:stl},
which was given in terms of subexponential type annotations.

Subexponentials refine the exponentials in AJM
games~\cite{AbramskyJM00}, where  is implemented using
, where~ is a type of unbounded natural numbers.
If we had used full exponentials in the Int-interpretation 
above, then we would have obtained a compilation method that encodes
function values as values in~, which is akin to storing closures on the
heap. Subexponentials give us more control to avoid such encodings where
unnecessary.
The subexponential type system in this paper has
its origin in Bounded Linear Logic~\cite{girardscedrovscott,Schopp07}.
It is also similar
to the type system for Syntactic Control of Concurrency (SCC)~\cite{DBLP:journals/tcs/GhicaMO06}.
A main
difference appears to be that while SCC controls
the number of program threads,
subexponentials account for both
the threads and their local data.

The observation that there is a connection between 
game models and continuations is not new. It appears, for example,
in Levy's work on a jump-with-argument calculus~\cite{cbpv}
and in Melli{\`e}s work on tensorial logic~\cite{mellies12}.
Connections of game models to compilation have also been made, e.g.~\cite{mellies}.
Furthermore, it is well-known that continuation passing is related to 
message passing, see e.g.~\cite{thieleckePhd}.
However, we 
are not aware of work that makes explicit a connection to defunctionalization.

We believe that the connection between game models and machine languages 
deserves to be better known and studied further.
The call traces in this paper, for example, should 
have the same status as plays in Game Semantics.
This suggests that techniques from
Game Semantics help to analyse the possible traces 
of compiled machine code.
For Java-like languages, there is recent work that connects 
fully abstract trace semantics~\cite{DBLP:conf/esop/JeffreyR05} and
Game Semantic models~\cite{DBLP:conf/popl/MurawskiT14}.
We hope that similar connections can be identified for the traces of 
machine code generated by compilers.

Work on concurrency and process calculus emphasises the interactive nature of 
computation. 
Milner's translation of the -calculus in the 
-calculus \cite{DBLP:journals/mscs/Milner92} can be understood
as a CPS-translation~\cite{DBLP:journals/mscs/Sangiorgi99}. 
This connection was made more than once: see
\cite[\S10]{DBLP:journals/mscs/Sangiorgi99} for historical notes.
The interactive model that we have studied in this paper can be seen
as a very static form of communicating processes,
without process mobility or channel reconfiguration.
In Milner's translation, channel names identify continuations and these
are passed around explicitly. We have seen in this paper that 
by taking into account control flow information, it is possible
to avoid passing continuations names, as they can be determined statically.
It is an interesting direction for further work to find out if
Milner's translation relates to CPS-translation and defunctionalization
without control flow information. 
In this direction, it may perhaps be possible to connect to 
the interesting results of Berger~et~al.~\cite{BergerHY01}.

In further work, we should like to understand possible connections 
to Danvy's work on defunctionalized interpreters~\cite{danvy},
or more generally to work on continuations and abstract machines, 
e.g.~\cite{DanosHerbelinRegnier,DBLP:journals/jfp/StreicherR98}. 
A relation is not obvious: Danvy considers the 
defunctionalization of particular implementations of interpreters,
while here we show that the whole compilation itself may be
described extensionally by the Int construction. 

In another direction, an interactive view of CPS-translation and
defunctionalization may also help in identifying mathematical structure
of efficient compilation methods. In particular, capturing call-by-value 
defunctionalizing compilation, perhaps similar to~\cite{DBLP:conf/esop/CejtinJW00},
should be interesting. Other interesting issues are 
efficient separate compilation and polymorphism: 
the interpretation in  is compositional
and polymorphism can also be accounted for~\cite{Schopp07}.

Finally, this paper clarifies the definition of IntML~\cite{intml}, which
was introduced to capture \textsc{logspace}. In~\cite{intml} we 
observed that IntML supports control operators, such as~,
but their status remained somewhat unclear. It now turns out that
the  combinator of~\cite{intml} 
may be understood as the defunctionalization of a standard
CPS implementation of . 

\paragraph{Acknowledgements.} 
I would like to thank the anonymous referees for their constructive
feedback and suggestions, which helped to improve the presentation of the results.

\bibliographystyle{plain}
\bibliography{defun}
\end{document}
