main\documentclass[10pt,twocolumn,letterpaper]{article}

\usepackage[rebuttal]{cvpr}


\usepackage{times}
\usepackage{epsfig}
\usepackage{graphicx}
\usepackage{amsmath}
\usepackage{amssymb}
\usepackage{datetime2}
\usepackage{bm}
\usepackage{color,soul}          \usepackage{microtype}      \usepackage{dcolumn}        \usepackage{url}            \usepackage{booktabs}       \usepackage{amsfonts}       \usepackage{nicefrac}       \usepackage{enumitem}
\usepackage{multirow}
\usepackage{float}
\usepackage{algorithm}
\usepackage[noend]{algpseudocode}
\usepackage[numbers,sort,compress]{natbib}
\usepackage{amsthm}
\usepackage{subfiles}
\usepackage{adjustbox}
\usepackage{subcaption}
\usepackage{makecell}
\usepackage[normalem]{ulem}
\usepackage{caption}
\usepackage{nopageno}
\usepackage[flushmargin,para]{footmisc}
\usepackage[pagebackref,breaklinks,colorlinks]{hyperref} \usepackage{color, colortbl}
\usepackage[usenames,dvipsnames]{xcolor}
\usepackage{pifont}\newcommand{\cmark}{\ding{51}}\newcommand{\xmark}{\ding{55}}

\usepackage[capitalize]{cleveref}
\crefname{section}{Sec.}{Secs.}
\Crefname{section}{Section}{Sections}
\Crefname{table}{Table}{Tables}
\crefname{table}{Tab.}{Tabs.}
\newcommand{\fig}[1]{Fig.~\ref{#1}} 
\newcommand{\tbl}[1]{Table~\ref{#1}} 
\newcommand{\algo}[1]{Algorithm~\ref{#1}} 
\newcommand{\sect}[1]{Sec.~\ref{#1}} 

\newcommand{\rawal}[1]{\textcolor{red}{Rawal: #1}}
\newcommand{\brandon}[1]{\textcolor{blue}{Kris: [#1]}}
\newcommand{\sha}[1]{\textcolor{green}{Shashank: [#1]}}

\definecolor{lightgray}{gray}{0.97}
\definecolor{lightblue}{rgb}{0.93,0.95,1.0}

\def\cvprPaperID{276} \def\confName{CVPR}
\def\confYear{2022}

\definecolor{cadmiumgreen}{rgb}{0.1, 0.32, 0.1}
\newcommand{\Rone}{\textcolor{Red}{R1}}
\newcommand{\Rtwo}{\textcolor{Green}{R2}}
\newcommand{\Rthree}{\textcolor{Blue}{R3}}

\begin{document}

\title{Occluded Human Mesh Recovery}


\maketitle
\thispagestyle{empty}


\vspace*{-0.35in}
\noindent
We sincerely thank all the reviewers (\textbf{\Rone, \Rtwo, \Rthree}) for their constructive feedback. \textbf{\Rone} noted that our method is novel and outperforms prior art significantly on multiple datasets. \textbf{\Rtwo} said that the paper is well-motivated and well-written. \textbf{\Rthree} has admired our open-source intentions. All reviewers have appreciated the comprehensiveness of our experiments. 

\vspace{1.0mm}\noindent
\textbf{[\Rtwo, \Rthree] Technical novelty.} Our paper is the \textbf{\textit{first}} to propose a top-down method for multi-human occlusion -- prior works in this setting are primarily bottom-up. As noted by \textbf{\Rone}, our approach combining local+global centermaps with context normalization is a simple but powerful change. Further, {\small OCHMR} outperforms baselines by great margins. A simple method that works and can be easily applied to any top-down model is an important technical contribution to share with the community and can inspire new research. We believe these properties are at the core of a good {\small CVPR} paper.


\vspace{1.0mm}\noindent
\textbf{[\Rone, \Rthree] Occlusion analysis.} We discuss limitations under different occlusions in L730--734. Inspired by the feedback, following PARE, we report performance on 3DPW-OCC and 3DOH datasets in Tab.~\ref{table:rebuttal:object_occlusion}. OCHMR improves performance under person occlusion but suffers from object occlusion as the body centermaps do not encode person-object context. We will note this as potential future work in the paper.






\begin{table}[b]
\vspace*{-0.2in}
\captionsetup{font=scriptsize}
\centering
\small
\resizebox{3.35in}{!}{
    \renewcommand{\arraystretch}{1.0}
\begin{tabular}{@{}l|c c|c c@{}}
        \Xhline{3\arrayrulewidth}
        \multirow{2}{*}{\textbf{Method}} & \multicolumn{2}{c|}{\textbf{Person Occlusion}} & \multicolumn{2}{c}{\textbf{Object Occlusion}}  \\
         & \texttt{3DPW-PC $\downarrow$} & \texttt{OCHuman $\uparrow$} & \texttt{3DPW-OCC $\downarrow$} & \texttt{3DOH $\downarrow$}\\
        \hline
        SPIN & 128.4 & 16.9 & 95.6 & 104.3 \\
        ROMP & 119.7 & 15.6 & - & - \\
        HMR-EFT (pose) & 116.4 & 20.8  & 94.4 & 75.2 \\
        PARE & 122.5 & 18.0  & \textbf{90.5} & \textbf{63.3} \\
        OCHMR (Ours) & \textbf{112.2} & \textbf{37.7} & 93.5  & 98.6 \\
        \Xhline{3\arrayrulewidth}
    \end{tabular}
}
\vspace*{-0.1in}
\caption{We report MPJPE/AP of methods under person and object occlusion. Evaluations are done using ground-truth boxes. HMR-EFT uses AlphaPose for 2D pose.}
\label{table:rebuttal:object_occlusion}
\vspace*{-0.5in}
\end{table}
 
\vspace{1.0mm}\noindent
\textbf{[\Rtwo] Comparison to PARE, HMR-EFT.} Unlike OCHMR, PARE primarily addresses object occlusion using part segmentation maps. Being a top-down method, PARE inherits all their limitations under multi-human occlusion. Further, HMR-EFT, unlike OCHMR, is a fitting-based approach and assumes a 2D pose as input. As requested, we compare PARE and HMR-EFT with OCHMR in Tab.~\ref{table:rebuttal:object_occlusion}. Note, OCHMR outperforms all baselines under person occlusion.


\vspace{1.0mm}\noindent
\textbf{[\Rone] SPIN as choice.} We picked SPIN due to the code's reliability and ease of use. Most recent methods borrow heavily from it. We report results using PyMaf and I2L-MeshNet in Tab.~\ref{table:rebuttal:other_topdown}. Better top-down methods improve OCHMR's results further, especially under severe occlusion on OCHuman.

\vspace{1.0mm}\noindent
\textbf{[\Rtwo] Clarifications.} L376--377: The baseline is a top-down method (not bottom-up ROMP). L394--397: The ambiguity is resolved using the input bbox and centermaps as context. No multi-person losses: OCHMR with just $\mathcal{L}_{\text{single}}$ still outperforms ROMP under occlusion as shown in Tab.~\ref{table:rebuttal:other_topdown}. Extreme overlap: We report results using the OCHMR + CAR loss for training $F$ with $\gamma = 0.2$ in Tab.~\ref{table:rebuttal:other_topdown}. Interestingly, we observe similar performance to OCHMR. CAR loss helps ROMP as their centermap (512x512) and grid-cell resolution is limiting, whereas OCHMR centermaps are calculated on higher-res. crops. We show severe occlusion results in supp. 


\begin{table}[t]
\captionsetup{font=scriptsize}
\begin{center}
\vspace*{-0.1in}
\resizebox{3.30in}{!}{
    \small
\renewcommand{\arraystretch}{1.0}
    \begin{tabular}{@{}l|l|l|l @{}}
    \Xhline{3\arrayrulewidth}
         \small{\textbf{Method}} &\small{\texttt{3DPW-PC} $\downarrow$} &\small{\texttt{OCHuman} $\uparrow$}   & \small{\texttt{CrowdPose} $\uparrow$}  \\
    \hline
   ROMP \small{(bottom-up)}  & 119.7  & 15.6  & 18.9 \\
   \hline
   SPIN  & 128.4  & 16.9  & 17.2 \\
   OCHMR \small{(\textit{early-fusion})}  & 115.8 & 30.6  & 20.9  \\
   OCHMR ($\mathcal{L}_\text{single}$)  & 116.9  & 32.8  & 20.4 \\
   OCHMR + CAR  & 113.8   & 36.9  & 23.0 \\
   OCHMR (Ours)   & \textbf{112.2 (-16.2)}  & \textbf{37.7 (+20.8)}  & \textbf{23.6 (+6.4)} \\
   \hline
   I2LMeshNet  & 125.7  & 18.1  & 19.0 \\
   \small{OCHMR-I2LMeshNet}  & \textbf{110.0 (-15.7)}  & \textbf{38.0 (+19.9)}  & \textbf{23.7 (+4.7)} \\
   \hline
   PyMaf  & 123.6  & 19.3  & 20.8 \\
   \small{OCHMR-PyMaf} & \textbf{105.8 (-17.8)}  & \textbf{39.1 (+19.8)}  & \textbf{26.3 (+5.5)} \\
    \Xhline{3\arrayrulewidth}
    \end{tabular}
}
\vspace*{-0.1in}
\caption{OCHMR with top-down methods and losses. Metrics are MPJPE/AP.}
 \vspace*{-0.4in}
\label{table:rebuttal:other_topdown}
\end{center}
\end{table}
 
\vspace{1.0mm}\noindent
\textbf{[\Rthree] Use of centermaps. Similarity to BatchNorm.} We agree with \textbf{\Rthree} that centermaps are used before in 2D tasks (L134-138). However, this is the first work to use for top-down mesh recovery. In contrast to BatchNorm, the CoNorm block is learnable and can model explicit affine transformation for each instance. BatchNorm would mix contextual information in the batch -- which is undesirable.


\vspace{1.2mm}\noindent
\textbf{[\Rtwo] Keypoint metrics. Panoptic evaluation.} Our evaluations consist of both 3D and 2D evaluations under occlusion. As mentioned (L236--260), 2D datasets like OCHuman and CrowdPose depict much more severe multi-human occlusion than 3D datasets -- this is the underlying reason for reporting keypoint metrics. Further, as requested, we report MPJPE on the Panoptic dataset in Tab.~\ref{table:rebuttal:panoptic}. OCHMR outperforms bottom-up ROMP on activities with high bbox overlap. 

\begin{table}[b]
\captionsetup{font=scriptsize}
\begin{center}
\vspace*{-0.2in}
\resizebox{2.9in}{!}{
    \small
\renewcommand{\arraystretch}{1.0}
    \begin{tabular}{@{}l|llll|l @{}}
    \Xhline{3\arrayrulewidth}
         \small{\textbf{Method}} &\small{\textbf{Haggling}} &\small{\textbf{Mafia}}   & \small{\textbf{Ultim.}} & \small{\textbf{Pizza}} & \small{\textbf{Mean}}  \\
    \hline
   CRMH & 129.6 & 133.5 & 153.0 & 156.7 & 143.2 \\
   ROMP \small{(bottom-up)} & \textbf{111.8} & 129.0 & 148.5 & \textbf{149.1} & 134.6 \\
   SPIN & 124.3 & 132.4 & 150.4 & 153.5 & 140.2 \\
   OCHMR \small{(Ours)} & 115.5 & \textbf{123.7} & \textbf{142.6} & 150.6 & \textbf{133.1} \\
    \Xhline{3\arrayrulewidth}
    \end{tabular}
}
\vspace*{-0.1in}
\caption{We report MPJPE on the CMU-Panoptic benchmark. All methods are trained using same data for a fair comparison.}
 \vspace*{-0.6in}
\label{table:rebuttal:panoptic}
\end{center}
\end{table}
 
\vspace{1.0mm}\noindent
\textbf{[\Rone] Comparison to SPADE.} Firstly, we apologize for this oversight and will gladly correct it in the main paper. Importantly, CoNorm is the first work to use feature normalization for injecting contextual information for mesh recovery. Although SPADE is similar in design to CoNorm, it is used in image synthesis, requires a prior BatchNorm operation, and is limited to a single channel input. From this perspective, SPADE is also similar to normalization methods (\eg AdaIN {\scriptsize ICCV'17}, FiLM {\scriptsize AAAI '18}) which we will also cite and compare.

\vspace{1.0mm}\noindent
\textbf{[\Rthree] Source of gains. Comparison with concatenation. Other schemes.} Our gains majorly come from using the centermaps as shown in Tab.~\ref{table:rebuttal:other_topdown}. OCHMR-early-fusion (L620-622, Tab. 4 main paper) with simple concatenation outperforms ROMP. The CoNorm blocks further improve performance by incorporating context at multiple depths. Lastly, unlike OCHMR, other schemes do not consider using context in a top-down fashion for occlusion reasoning.


\vspace{1.0mm}\noindent
\textbf{[\Rtwo] Additional details.} 3DPW-PC: The subset is defined by 3DOH and used by ROMP. Context define: global + local. Societal impact: Thank you for the excellent points. We will refine our response. L643: No, tested on 3DPW. L507-509: $\mathcal{L}_{\text{depth}}$. Visualization: We used public vis. code for SPIN and ROMP, as suggested we will use our vis. for all methods. 

\end{document}
