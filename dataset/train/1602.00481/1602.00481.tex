In this section we show that we can effectively check the inclusion
test~ of priced zones. For the rest of this section, we~fix two
priced zones  and , and a
function~. To improve readability, we~write~ and~ in place
of~ and~.
\subsection{Formulation of the optimization problem}
\label{subsec:opt}

We~first express the inclusion of the two priced zones as an
optimization problem.

\begin{restatable}{lemma}{optimisation}
  \label{lemma:inclusion_sup}
  .
\end{restatable}


Note that  already requires some
relation between zones  and~: indeed, for the above inclusion
to hold, it should be the case that for every , there exists
some  such that .  Interestingly, this
corresponds to the test on (unpriced) zones developed in~\cite{HSW12}
(with~); this can be done efficiently (in time quadratic in the
number of clocks) as a preliminary test~\cite[Theorem 34]{HSW12}.

\begin{remark}
  The constraint  is not convex, and we have a bi-level
  optimization problem to solve.
  Hence common techniques for convex optimization, such as
  dualization~\cite{BV04}, do~not directly apply to the above
  problem. Still, it~is possible to transform it into finitely many
  so-called \emph{generalized semi-infinite optimization problems}
  (GSIPs)~\cite{RS01} (using~'s as defined later in this
  section).  As~far as we know, such problems do not have dedicated
  efficient algorithmic solutions. We~thus propose a more direct
  solution, that benefits from the specific structure of our problem
  (see~for instance~\Cref{sec:righty}); it~provides a feasible way to
  solve our optimization problems, hence to decide  on
  priced zones.
\end{remark}

In order to compute the above optima, we transform our problem into a
finite number of optimization problems that are easier to solve.
Let .  A~zone~ is -bounded on~ if, for every
, .  We note  the
restriction of~ to its -bounded-on- component:
.  Note that  may be empty, and that the
family~ forms a partition of~. We also define
 as the priced zone . We~define the
natural projection ,
which associates with  the valuation  that coincides with~ on~.



\begin{figure}[tb]
\colorlet{vert}{green!80!blue}
\colorlet{rouge}{red}
\begin{minipage}[t]{.48\linewidth}
\centering
\begin{tikzpicture}[scale=.4]
\draw[latex'-latex'] (11,0) node[right] {} -| (0,9) node[above] {};
\draw[dashed] (8,0) node[below] {} -- +(0,9);
\draw[dashed] (0,4) node[left] {} -- +(11,0);
\draw[draw=vert,fill=vert!50!white,opacity=.5,line width=1pt] (2,0) -- (2,2) --
  (9,9) -- (9,3) -- (6,0) -- (2,0);
\draw[draw=rouge,fill=rouge!50!white,opacity=.2,line width=1pt] (0,1) -- (6,7) --
  (10,7) -- (10,2) -- (8,0) -- (0,0) -- (0,1);
\draw (6,2.5) node {};
\draw (.7,.7) node {};
\begin{scope}
\path[clip] (3,4) -- (6,7) -- (8,7) -- (8,4) -- (3,4);
\foreach \x in {3,3.4,...,11} {\draw[rouge,opacity=.4] (\x,4) -- +(-4,4);}
\end{scope}
\begin{scope}
\path[clip] (4,4) -- (8,8) -- (8,4) -- (4,4);
\foreach \x in {3.2,3.6,...,11} {\draw[vert,opacity=.4] (\x,4) -- +(0,4);}
\end{scope}
\path[use as bounding box] (0,0);
\draw[line width=1mm,rouge,opacity=.6,cap=round] (3,.04) -- (8,.04);
\draw[loosely dotted] (3,.04) -- (3,4);
\draw[line width=1mm,vert,opacity=.4,cap=round] (4,-.04) -- (8,-.04);
\draw[loosely dotted] (4,.04) -- (4,4);
\path (5.2,1.4) node (a) {};
\draw[-latex'] (a.-150) .. controls +(-120:3mm) and +(70:3mm) .. (3.8,.2);
\path (5,-1.4) node (b) {};
\draw[-latex'] (b) -- (5.5,-.2);
\draw[dotted] (2.3,3.3) -| (8.7,8.7) -| (2.3,3.3);
\path (3.3,6) node (c) {};
\draw[-latex'] (c) .. controls +(-70:6mm) and +(140:6mm) .. (4.1,4.6);
\path (5,7.5) node (d) {};
\draw[-latex'] (d) .. controls +(-70:12mm) and +(140:12mm) .. (6.5,5);
\end{tikzpicture}
\caption{Two-dimensional zones~ and~, and sub-zones 
  and  for .}
\label{fig-ZY}
\end{minipage}\hfill
\begin{minipage}[t]{.42\linewidth}
\centering
\begin{tikzpicture}[scale=.72]
\draw[dotted] (2.3,3.3) -| (8.7,8.7) -| (2.3,3.3);
\begin{scope}
\path[clip] (2.2,3.2) -| (8.8,8.8) -| (2.2,3.2);
\draw[dashed] (8,0) node[below] {} -- +(0,9);
\draw[dashed] (0,4) node[left] {} -- +(11,0);
\draw[draw=vert,fill=vert!50!white,opacity=.5,line width=1pt] (2,0) -- (2,2) --
  (9,9) -- (9,3) -- (6,0) -- (2,0);
\draw[draw=rouge,fill=rouge!50!white,opacity=.2,line width=1pt] (0,1) -- (6,7) --
  (10,7) -- (10,2) -- (8,0) -- (0,0) -- (0,1);
\begin{scope}
\path[clip] (3,4) -- (6,7) -- (8,7) -- (8,4) -- (3,4);
\foreach \x in {3,3.4,...,11} {\draw[rouge,opacity=.4] (\x,4) -- +(-4,4);}
\end{scope}
\begin{scope}
\path[clip] (4,4) -- (8,8) -- (8,4) -- (4,4);
\foreach \x in {3.2,3.6,...,11} {\draw[vert,opacity=.4] (\x,4) -- +(0,4);}
\end{scope}
\end{scope}
\path[use as bounding box] (5,2.2);
\draw[dotted,-latex',line width=1pt,opacity=.7] (-1.5,7) .. controls +(20:1cm)
  and +(160:1cm) .. (2.1,7); 
\draw[rouge,line width=1mm,opacity=.5,cap=round] (3.1,4.1) -- (5.95,6.95);
\draw[rouge,line width=1mm,opacity=.5,cap=round] (6.1,7) -- (8,7);
\path (4,7) node[text width=2.2cm,align=center] (f) {upper facets of  w.r.t.~};
\draw[-latex'] (f) .. controls +(-90:5mm) and +(130:5mm) .. (4,5.1); 
\draw[-latex'] (f.10) .. controls +(20:2mm) and +(160:2mm) .. (6.4,7.1); 
\draw[vert,line width=1mm,opacity=.5,cap=round] (4.05,4.05) -- (8,8);
\path (5.3,8.3) node (g) {upper facet of  w.r.t.~};
\draw[-latex'] (g.-20) .. controls +(-80:2mm) and +(170:2mm) .. (7.5,7.6); 
\draw[rouge,line width=1mm,opacity=.5,cap=round] (3,3.96) -- (8,3.96);
\draw[vert,line width=1mm,opacity=.5,cap=round] (4.04,4.04) -- (8,4.04);
\path (5.5,2.4) node (h) {lower facets of  and~ w.r.t.~};
\draw[-latex'] (h.160) .. controls +(120:5mm) and +(-90:5mm) .. (3.5,3.8);
\draw[-latex'] (h.50) .. controls +(60:5mm) and +(-90:5mm) .. (6.5,3.8);
\end{tikzpicture}
\caption{Simple facets of  and  w.r.t. clock~.}
\label{fig-facets}
\end{minipage}
\end{figure}
 
\begin{restatable}{lemma}{timedinclusion}
  \label{lemma:timed_inclusion}
  The following two properties are equivalent:
  \begin{enumerate}[(i)]
  \item for every , there is  such that 
  \item for every , .
  \end{enumerate}
\end{restatable}

This allows to transform the initial optimization problem into
finitely many optimization problems.

\begin{restatable}{lemma}{decomposition}
  \label{lemma:decompos}
  \quad .
\end{restatable}

\begin{corollary}
  \label{coro:decompos}
   iff for every , 
\end{corollary}

In the sequel, we~write 

\Cref{lemma:inclusion_sup} and~\Cref{coro:decompos} suggest an
algorithm for deciding whether : enumerate
the subsets~ of , and prove that . 
We~now show how to solve the latter optimization problem (for~a
fixed~), and then show how we can drive the choice of~ so that
not all subsets of~ have to be analyzed.

\subsection{Computing }

We show the following main result to compute ,
which produces a simpler optimization problem, allowing to decide the
inclusion of two priced zones, on parts where cost functions are
lower-bounded.

\begin{restatable}{theorem}{projectionfacettes}
  \label{theo:projections-facets}
  Let  and  be two
  non-empty priced zones, and let  be such that
   and  and  are
  lower-bounded on  and  respectively.
  Then we can compute finite sets  and 
  of zones over , and affine functions  and 
  for every  and  s.t.:
  
\end{restatable}

The details of the proof of this theorem is given in~\Cref{app:proof}.
The idea behind this result is to first rewrite 
into:

which decouples the dependency of  on .
The algorithm then uses the notion of facets (introduced
in~\cite{LBB+01}), which corresponds to the boundary of the zone
w.r.t. a clock (if~ is the zone, a~facet of~ w.r.t.~ is
 or  whenever  or  is a constraint defining~). Given a
clock , we~consider the facets of~
w.r.t.~ that minimize, for any ,
the function  when .
The~restriction of~ on such a facet is a new affine function,
which we can compute. We~then iterate the process for all clocks in . We~do the same for~. This~yields the result
claimed above: sets~ and~ are sets of
projections of facets over~.

Facets are zones, and so are their projections on~ and
intersections thereof. Additionally, all functions  and
 are affine; hence the supremum in Eq.~\eqref{eq-thmS} is
reached at some vertex~ of zone , for some facets~
and~.
By~construction of~ and~, we~get


In particular,  has integral coordinates. We end up with the
following result, which will be useful for proving the termination
of~\Cref{algo:optreach}:
\begin{corollary}
  \label{theo:vertices}\label{coro:vertices}
  Let  and  be two
  non-empty priced zones, and let  be such that
   and  and  are
  lower-bounded on  and  respectively. Then the following
  holds:
  
\end{corollary}

The requirement for lower-bounded priced zones
in~\Cref{theo:projections-facets} is crucial in the proof. 
But the case when this requirement is not met can easily be handled separately, so
that  can always be effectively decided:

\begin{restatable}{lemma}{relaxation}
  \label{lemma:relaxation}
  Let  and  be two
  non-empty priced zones.
  \begin{itemize}
  \item If  is not lower-bounded on  but  is
    lower-bounded on , then .
  \item Let  such that . If  is not lower-bounded on
    , then .
  \end{itemize}
\end{restatable}

\begin{corollary}
  Let  and  be two
  priced zones. Then we can effectively decide whether .
\end{corollary}

\subsection{Finding the right }
\label{sec:righty}

Applying \Cref{lemma:decompos}, the main obstacle to efficiently
decide~ is to find the appropriate  in which the sought
supremum is reached. Unless good arguments can be found to guide the search
towards the best choice for~, an exhaustive enumeration of all the 's
will be required.

\begin{example}
  We consider the zone  defined by the constraints , ,  and .  We fix  and .  We then consider . The zone~ is equipped with a constant cost
  function~.
In~\Cref{fig:examples:middle},  is attached , and
  the  
  expression of the function 
 
  is given in each~, for~.
  It is then easy to see that the supremum of~ is reached at the
  point 
  , in the middle of the zone.  In~\Cref{fig:examples:border},
  we~take , and the expression of the function
   is given in
  each~.  The supremum of~ is then reached at the point 
  , on the border, but not at a corner of the zone.  The latter
  example also shows that  is not continuous on the whole zone~.
\end{example}

\begin{figure}[tb]
\colorlet{vert}{green!70!blue}
\centering
\subfigure[The sought supremum is reached in the middle of the
zone.\label{fig:examples:middle}]{
\begin{tikzpicture}[scale=.9]
\draw[latex'-latex'] (4,0) -| (0,5);
\path[use as bounding box] (-1,-.4);
\draw[-latex',line width=1pt] (.6,3.5) -- (1.3,4.2) node[midway,above left]
  {};
\begin{scope}[opacity=1]
\fill[shading=axis,lower left=vert!0!white,upper left=vert!40!white,lower
right=vert!20!white,upper right=vert!100!white] (2,2) -- (1,1) -| (0,2) -- (1,3) -| (2,2);
\fill[shading=axis,lower left=vert!60!white,lower right=vert!60!white,
  upper left=vert,upper right=vert]  (2,2) -- (3,3) -| (2,2);
\fill[shading=axis,lower left=vert!60!white,lower right=vert,
  upper left=vert!60!white,upper right=vert]  (1,3) -- (2,4) |- (1,3);
\fill[vert] (2,4) |- (3,3) -- (4,4) |- (3,5) -- (2,4);
\end{scope}
\draw[vert!60!black,line width=1pt] (4,4) -- (1,1) -| (0,2) -- (3,5);
\draw[dotted] (0,3) node[left] {} -- +(4,0);
\draw[dotted] (2,0) node[below] {} -- +(0,5);
\draw[-latex'] (1,2) -- +(.5,.5) node[midway,left] {};
\draw[-latex'] (1.4,3.1) -- +(.5,0) node[midway,above] {};
\draw[-latex'] (2.1,2.4) -- +(0,.5) node[midway,right] {};
\draw (3,4) node {};
\fill (2,3) circle (.5mm) node[above right] {};
\draw (1,.3) node {};
\draw (3,.3) node {};
\draw (1,5.3) node {};
\draw (3,5.3) node {};
\end{tikzpicture}
}
\hfill
\subfigure[The sought supremum is reached on the border of the
zone.\label{fig:examples:border}]{
\begin{tikzpicture}[scale=.9]
\draw[latex'-latex'] (4,0) -| (0,5);
\path[use as bounding box] (-1,-.4);
\draw[-latex',line width=1pt] (.3,4.2) -- (1.3,3.7) node[midway,above right]
  {};
\begin{scope}
\fill[shading=axis,shading angle=-71,lower left=vert!0!white,
  upper left=vert!60!white,lower right=vert!60!white,upper right=vert]  (2,2) -- (1,1) -| (0,2) -- (1,3) -| (2,2);
\fill[shading=axis,lower left=vert!85!white,lower right=vert!85!white,
  upper left=vert!75!white, upper right=vert!75!white]  (2,2) -- (3,3) -| (2,2);
\fill[shading=axis,lower left=vert!40!white,lower right=vert!60!white,upper right=vert!60!white]  (1,3) -- (2,4) |- (1,3);
\fill[vert!60!white] (2,4) |- (3,3) -- (4,4) |- (3,5) -- (2,4);
\end{scope}
\draw[vert!60!black,line width=1pt] (4,4) -- (1,1) -| (0,2) -- (3,5);
\draw[dotted] (0,3) node[left] {} -- +(4,0);
\draw[dotted] (2,0) node[below] {} -- +(0,5);
\draw[-latex'] (.7,2.2) -- +(.8,-.4) node[midway,above right] {};
\draw[-latex'] (1.4,3.1) -- +(.5,0) node[midway,above] {};
\draw[-latex'] (2.1,2.9) -- +(0,-.5) node[midway,right] {};
\draw (3,4) node {};
\fill (2,2) circle (.5mm) node[below right] {};
\draw (1,.3) node {};
\draw (3,.3) node {};
\draw (1,5.3) node {};
\draw (3,5.3) node {};
\end{tikzpicture}
}
\caption{The supremum may lie in the middle of zones or facets}
\label{fig:examples}
\end{figure}

Nevertheless, in many cases, we will be able to guide the search of the 
where the sought optimal is to be found.
The~following development
focuses on the zone, not on the cost function. Given a zone~, we define a
preorder~ on the clocks, such that if , then
~is downward-closed for~. In~other words, whenever ,
 and , then . The knowledge of 
can be a precious help to guide the enumeration of non-empty~'s.
Indeed, if , ~is downward-closed for~, and
candidates for~ are thus found by enumerating the antichains of~.
In particular, if  is total, then there are at most  sets 
such that .

To be concrete, 
let  and~ be the (disjoint) sets of clocks~ such that  and , respectively.
We~define the relation~ as the
least
relation satisfying the following conditions:
\begin{itemize}
\item for each , for each , ;
\item for each , for each , ;
\item for each , for each , ;
\item for all ,  implies .
\end{itemize}
It is not difficult to show that  is a preorder such that:
 and  implies , and
 and  implies .

\begin{restatable}{lemma}{Ydownwardclosed}
  Let  such that . Then  is
  downward-closed for .
\end{restatable}

The preorder  can be computed in polynomial time, since it
only requires to check emptiness of zones, which can be done in time
polynomial in  (cubic in  with DBMs for instance).

We recall that, if  is a zone generated in a timed automaton where
only resets of clocks to  are allowed, for any pair of clocks
, it~cannot be the case that  crosses the diagonal hyperplane of
equation . 

\begin{proposition}
  \label{prop:total_samebound}
  If  is generated by a timed automaton, and all clocks have the
  same bound , then  is total.
\end{proposition}

\begin{proof}
  Let  and  be two clocks. Since  is generated by a timed automaton,
  it~is contained either in the half-space of equation , or in the
  one of equation . By definition of , and since , the former entails , and the latter .
  Any two clocks are thus always comparable, and  is therefore
  total. \qed
\end{proof}

Under the assumptions of~\Cref{prop:total_samebound}, there are
polynomially many subsets  to~try.
  Note that these assumptions are easily realized by taking
   as the unique maximal constant for
  all the clocks.
  Formally, ~is~an under-approximation of
  the exact version of .  This approximation does not
  hinder correctness, and illustrate the trade-off between the
  complexity of the inclusion procedure and the number of priced zones
  that will be explored.
