In this section we show that we can effectively check the inclusion
test~$\sqsubseteq_M$ of priced zones. For the rest of this section, we~fix two
priced zones $\mathcal{Z} = (Z,\zeta)$ and $\mathcal{Z}' = (Z',\zeta')$, and a
function~$M$. To improve readability, we~write~$\equiv$ and~$\sqsubseteq$ in place
of~$\equiv_M$ and~$\sqsubseteq_M$.
\subsection{Formulation of the optimization problem}
\label{subsec:opt}

We~first express the inclusion of the two priced zones as an
optimization problem.

\begin{restatable}{lemma}{optimisation}
  \label{lemma:inclusion_sup}
  $\mathcal{Z} \sqsubseteq \mathcal{Z}' \iff \sup_{v \in Z}
  \inf_{\substack{v' \in Z'\\v' \equiv v}} \zeta'(v') - \zeta(v) \leq
  0$.
\end{restatable}


Note that $\mathcal{Z} \sqsubseteq \mathcal{Z}'$ already requires some
relation between zones $Z$ and~$Z'$: indeed, for the above inclusion
to hold, it should be the case that for every $v \in Z$, there exists
some $v' \in Z'$ such that $v \equiv v'$.  Interestingly, this
corresponds to the test on (unpriced) zones developed in~\cite{HSW12}
(with~$L=U=M$); this can be done efficiently (in time quadratic in the
number of clocks) as a preliminary test~\cite[Theorem 34]{HSW12}.

\begin{remark}
  The constraint $v \equiv v'$ is not convex, and we have a bi-level
  optimization problem to solve.
  Hence common techniques for convex optimization, such as
  dualization~\cite{BV04}, do~not directly apply to the above
  problem. Still, it~is possible to transform it into finitely many
  so-called \emph{generalized semi-infinite optimization problems}
  (GSIPs)~\cite{RS01} (using~$Z_Y$'s as defined later in this
  section).  As~far as we know, such problems do not have dedicated
  efficient algorithmic solutions. We~thus propose a more direct
  solution, that benefits from the specific structure of our problem
  (see~for instance~\Cref{sec:righty}); it~provides a feasible way to
  solve our optimization problems, hence to decide $\sqsubseteq$ on
  priced zones.
\end{remark}

In order to compute the above optima, we transform our problem into a
finite number of optimization problems that are easier to solve.
Let $Y \subseteq X$.  A~zone~$Z$ is $M$-bounded on~$Y$ if, for every
$v \in Z$, $\{ x \mid v(x) \leq M(x) \} = Y$.  We note $Z_Y$ the
restriction of~$Z$ to its $M$-bounded-on-$Y$ component:
$Z_Y = Z \cap \bigcap_{x \in Y} (x \leq M(x)) \cap \bigcap_{x \notin
  Y} (x > M(x))$.  Note that $Z_Y$ may be empty, and that the
family~$(Z_Y)_{Y\subseteq X}$ forms a partition of~$Z$. We also define
$\mathcal{Z}_Y$ as the priced zone $(Z_Y,\zeta)$. We~define the
natural projection $\pi_Y \colon \IR_{\ge 0}^X \to \IR_{\ge 0}^Y$,
which associates with $v \in \IR_{\ge 0}^X$ the valuation $v' \in
\IR_{\ge 0}^Y$ that coincides with~$v$ on~$Y$.



\begin{figure}[tb]
\colorlet{vert}{green!80!blue}
\colorlet{rouge}{red}
\begin{minipage}[t]{.48\linewidth}
\centering
\begin{tikzpicture}[scale=.4]
\draw[latex'-latex'] (11,0) node[right] {$x$} -| (0,9) node[above] {$y$};
\draw[dashed] (8,0) node[below] {$M(x)$} -- +(0,9);
\draw[dashed] (0,4) node[left] {$M(y)$} -- +(11,0);
\draw[draw=vert,fill=vert!50!white,opacity=.5,line width=1pt] (2,0) -- (2,2) --
  (9,9) -- (9,3) -- (6,0) -- (2,0);
\draw[draw=rouge,fill=rouge!50!white,opacity=.2,line width=1pt] (0,1) -- (6,7) --
  (10,7) -- (10,2) -- (8,0) -- (0,0) -- (0,1);
\draw (6,2.5) node {$Z$};
\draw (.7,.7) node {$Z'$};
\begin{scope}
\path[clip] (3,4) -- (6,7) -- (8,7) -- (8,4) -- (3,4);
\foreach \x in {3,3.4,...,11} {\draw[rouge,opacity=.4] (\x,4) -- +(-4,4);}
\end{scope}
\begin{scope}
\path[clip] (4,4) -- (8,8) -- (8,4) -- (4,4);
\foreach \x in {3.2,3.6,...,11} {\draw[vert,opacity=.4] (\x,4) -- +(0,4);}
\end{scope}
\path[use as bounding box] (0,0);
\draw[line width=1mm,rouge,opacity=.6,cap=round] (3,.04) -- (8,.04);
\draw[loosely dotted] (3,.04) -- (3,4);
\draw[line width=1mm,vert,opacity=.4,cap=round] (4,-.04) -- (8,-.04);
\draw[loosely dotted] (4,.04) -- (4,4);
\path (5.2,1.4) node (a) {$\scriptstyle\pi_{Y}(Z'_{Y})$};
\draw[-latex'] (a.-150) .. controls +(-120:3mm) and +(70:3mm) .. (3.8,.2);
\path (5,-1.4) node (b) {$\scriptstyle\pi_{Y}(Z_{Y})$};
\draw[-latex'] (b) -- (5.5,-.2);
\draw[dotted] (2.3,3.3) -| (8.7,8.7) -| (2.3,3.3);
\path (3.3,6) node (c) {$Z'_Y$};
\draw[-latex'] (c) .. controls +(-70:6mm) and +(140:6mm) .. (4.1,4.6);
\path (5,7.5) node (d) {$Z_Y$};
\draw[-latex'] (d) .. controls +(-70:12mm) and +(140:12mm) .. (6.5,5);
\end{tikzpicture}
\caption{Two-dimensional zones~$Z$ and~$Z'$, and sub-zones $Z_{Y}$
  and $Z'_{Y}$ for $Y=\{x\}$.}
\label{fig-ZY}
\end{minipage}\hfill
\begin{minipage}[t]{.42\linewidth}
\centering
\begin{tikzpicture}[scale=.72]
\draw[dotted] (2.3,3.3) -| (8.7,8.7) -| (2.3,3.3);
\begin{scope}
\path[clip] (2.2,3.2) -| (8.8,8.8) -| (2.2,3.2);
\draw[dashed] (8,0) node[below] {$M(x)$} -- +(0,9);
\draw[dashed] (0,4) node[left] {$M(y)$} -- +(11,0);
\draw[draw=vert,fill=vert!50!white,opacity=.5,line width=1pt] (2,0) -- (2,2) --
  (9,9) -- (9,3) -- (6,0) -- (2,0);
\draw[draw=rouge,fill=rouge!50!white,opacity=.2,line width=1pt] (0,1) -- (6,7) --
  (10,7) -- (10,2) -- (8,0) -- (0,0) -- (0,1);
\begin{scope}
\path[clip] (3,4) -- (6,7) -- (8,7) -- (8,4) -- (3,4);
\foreach \x in {3,3.4,...,11} {\draw[rouge,opacity=.4] (\x,4) -- +(-4,4);}
\end{scope}
\begin{scope}
\path[clip] (4,4) -- (8,8) -- (8,4) -- (4,4);
\foreach \x in {3.2,3.6,...,11} {\draw[vert,opacity=.4] (\x,4) -- +(0,4);}
\end{scope}
\end{scope}
\path[use as bounding box] (5,2.2);
\draw[dotted,-latex',line width=1pt,opacity=.7] (-1.5,7) .. controls +(20:1cm)
  and +(160:1cm) .. (2.1,7); 
\draw[rouge,line width=1mm,opacity=.5,cap=round] (3.1,4.1) -- (5.95,6.95);
\draw[rouge,line width=1mm,opacity=.5,cap=round] (6.1,7) -- (8,7);
\path (4,7) node[text width=2.2cm,align=center] (f) {upper facets of $Z'_Y$ w.r.t.~$y$};
\draw[-latex'] (f) .. controls +(-90:5mm) and +(130:5mm) .. (4,5.1); 
\draw[-latex'] (f.10) .. controls +(20:2mm) and +(160:2mm) .. (6.4,7.1); 
\draw[vert,line width=1mm,opacity=.5,cap=round] (4.05,4.05) -- (8,8);
\path (5.3,8.3) node (g) {upper facet of $Z_Y$ w.r.t.~$y$};
\draw[-latex'] (g.-20) .. controls +(-80:2mm) and +(170:2mm) .. (7.5,7.6); 
\draw[rouge,line width=1mm,opacity=.5,cap=round] (3,3.96) -- (8,3.96);
\draw[vert,line width=1mm,opacity=.5,cap=round] (4.04,4.04) -- (8,4.04);
\path (5.5,2.4) node (h) {lower facets of $Z_Y$ and~$Z'_Y$ w.r.t.~$y$};
\draw[-latex'] (h.160) .. controls +(120:5mm) and +(-90:5mm) .. (3.5,3.8);
\draw[-latex'] (h.50) .. controls +(60:5mm) and +(-90:5mm) .. (6.5,3.8);
\end{tikzpicture}
\caption{Simple facets of $Z_Y$ and $Z'_Y$ w.r.t. clock~$y$.}
\label{fig-facets}
\end{minipage}
\end{figure}
 
\begin{restatable}{lemma}{timedinclusion}
  \label{lemma:timed_inclusion}
  The following two properties are equivalent:
  \begin{enumerate}[(i)]
  \item for every $v \in Z$, there is $v' \in Z'$ such that $v'
    \equiv v$
  \item for every $Y \subseteq X$, $\pi_Y(Z_Y) \subseteq \pi_Y(Z'_Y)$.
  \end{enumerate}
\end{restatable}

This allows to transform the initial optimization problem into
finitely many optimization problems.

\begin{restatable}{lemma}{decomposition}
  \label{lemma:decompos}
  \quad \(
  \displaystyle
  \sup_{v \in Z} \pinf_{\substack{v' \in Z'\\v' \equiv v}} \zeta'(v')
  - \zeta(v) = \pmax_{Y \subseteq X} \sup_{v \in Z_Y}
  \pinf_{\substack{v' \in Z'_Y\\v' \equiv v}} \zeta'(v') - \zeta(v)
  \).
\end{restatable}

\begin{corollary}
  \label{coro:decompos}
  $\mathcal{Z} \sqsubseteq \mathcal{Z}'$ iff for every $Y \subseteq
  X$, $\mathcal{Z}_Y \sqsubseteq \mathcal{Z}'_Y$
\end{corollary}

In the sequel, we~write 
\[
S(\calZ,\calZ',Y)=\sup_{v \in Z_Y} \inf_{\substack{v' \in Z'_Y \\ v' \equiv v}}
\zeta'(v') - \zeta(v)
\]
\Cref{lemma:inclusion_sup} and~\Cref{coro:decompos} suggest an
algorithm for deciding whether $\calZ \sqsubseteq \calZ'$: enumerate
the subsets~$Y$ of $X$, and prove that $S(\calZ,\calZ',Y) \le
0$. 
We~now show how to solve the latter optimization problem (for~a
fixed~$Y$), and then show how we can drive the choice of~$Y$ so that
not all subsets of~$X$ have to be analyzed.

\subsection{Computing $S(\calZ,\calZ',Y)$}

We show the following main result to compute $S(\calZ,\calZ',Y)$,
which produces a simpler optimization problem, allowing to decide the
inclusion of two priced zones, on parts where cost functions are
lower-bounded.

\begin{restatable}{theorem}{projectionfacettes}
  \label{theo:projections-facets}
  Let $\mathcal{Z}=(Z,\zeta)$ and $\mathcal{Z}'=(Z',\zeta')$ be two
  non-empty priced zones, and let $Y \subseteq X$ be such that
  $\pi_Y(Z_Y) \subseteq \pi_Y(Z'_Y)$ and $\zeta$ and $\zeta'$ are
  lower-bounded on $Z_Y$ and $Z_Y'$ respectively.
  Then we can compute finite sets $\mathcal{K}_Y$ and $\mathcal{K}'_Y$
  of zones over $Y$, and affine functions $\zeta_F$ and $\zeta'_{F'}$
  for every $F \in \mathcal{K}_Y$ and $F' \in \mathcal{K}'_Y$ s.t.:
  \begin{equation}
  S(\calZ,\calZ',Y) = \pmax_{F \in \mathcal{K}_Y} \pmax_{F' \in
    \mathcal{K}'_Y} \sup_{u \in F \cap F'} \zeta'_{F'} (u) - \zeta_F(u).
  \label{eq-thmS}
  \end{equation}
\end{restatable}

The details of the proof of this theorem is given in~\Cref{app:proof}.
The idea behind this result is to first rewrite $S(\calZ,\calZ',Y)$
into:
\[
S(\calZ,\calZ',Y) = \sup_{u \in \pi_Y(Z_Y)}
\Bigl[\Bigl(\inf_{\substack{v' \in Z_Y'\\\pi_Y(v') = u}}
\zeta'(v')\Bigr) - \Bigl(\inf_{\substack{v \in Z_Y\\\pi_Y(v) = u}}
\zeta(v)\Bigr)\Bigr]
\]
which decouples the dependency of $v'$ on $v$.
The algorithm then uses the notion of facets (introduced
in~\cite{LBB+01}), which corresponds to the boundary of the zone
w.r.t. a clock (if~$W$ is the zone, a~facet of~$W$ w.r.t.~$x$ is
$\adherence{W} \cap (x=n)$ or $\adherence{W} \cap (x-y=m)$ whenever $x
\bowtie n$ or $x-y \bowtie m$ is a constraint defining~$W$). Given a
clock $x \in X \setminus Y$, we~consider the facets of~$Z_Y$
w.r.t.~$x$ that minimize, for any $w \in \pi_{X \setminus \{x\}}(Z_Y)$,
the function $v \mapsto \zeta(v)$ when $\pi_{X \setminus \{x\}}(v) = w$.
The~restriction of~$\zeta$ on such a facet is a new affine function,
which we can compute. We~then iterate the process for all clocks in $X
\setminus Y$. We~do the same for~$\zeta'$. This~yields the result
claimed above: sets~$\mathcal{K}_Y$ and~$\mathcal{K'}_Y$ are sets of
projections of facets over~$Y$.

Facets are zones, and so are their projections on~$Y$ and
intersections thereof. Additionally, all functions $\zeta_F$ and
$\zeta'_{F'}$ are affine; hence the supremum in Eq.~\eqref{eq-thmS} is
reached at some vertex~$u_0$ of zone $F \cap F'$, for some facets~$F$
and~$F'$.
By~construction of~$\zeta_F$ and~$\zeta'_{F'}$, we~get
\[
S(\calZ,\calZ',Y) = \inf_{\substack{v' \in Z'_Y \\ \pi(v') = u_0}}
\zeta'(v') - \inf_{\substack{v \in Z_Y\vphantom{Z'_Y} \\ \pi(v) =
    u_0}} \zeta(v)
\]

In particular, $u_0$ has integral coordinates. We end up with the
following result, which will be useful for proving the termination
of~\Cref{algo:optreach}:
\begin{corollary}
  \label{theo:vertices}\label{coro:vertices}
  Let $\mathcal{Z}=(Z,\zeta)$ and $\mathcal{Z}'=(Z',\zeta')$ be two
  non-empty priced zones, and let $Y \subseteq X$ be such that
  $\pi_Y(Z_Y) \subseteq \pi_Y(Z'_Y)$ and $\zeta$ and $\zeta'$ are
  lower-bounded on $Z_Y$ and $Z_Y'$ respectively. Then the following
  holds:
  \[
S(\calZ,\calZ',Y) = \max_{\substack{u_0 \in \pi_Y(\adherence{Z_Y})
      \\u_0\in\bbN^Y}} \Bigl[
  \min_{\substack{\vphantom{\adherence{Z_Y}}v' \in Z'_Y \\
      v'\equiv u_0}} \zeta'(v') -
  \min_{\substack{\vphantom{\adherence{Z_Y}Z'_Y}v \in Z_Y\\ \vphantom{v'}v\equiv u_0}}
  \zeta(v)\Bigr]
  \]
\end{corollary}

The requirement for lower-bounded priced zones
in~\Cref{theo:projections-facets} is crucial in the proof. 
But the case when this requirement is not met can easily be handled separately, so
that $\sqsubseteq$ can always be effectively decided:

\begin{restatable}{lemma}{relaxation}
  \label{lemma:relaxation}
  Let $\mathcal{Z}=(Z,\zeta)$ and $\mathcal{Z}'=(Z',\zeta')$ be two
  non-empty priced zones.
  \begin{itemize}
  \item If $\zeta$ is not lower-bounded on $Z$ but $\zeta'$ is
    lower-bounded on $Z'$, then $\mathcal{Z} \not\sqsubseteq
    \mathcal{Z}'$.
  \item Let $Y \subseteq X$ such that $\pi_Y(Z_Y) \subseteq \pi_Y(Z_Y')$. If $\zeta'$ is not lower-bounded on
    $Z'_Y$, then $\mathcal{Z}_Y \sqsubseteq \mathcal{Z}'_Y$.
  \end{itemize}
\end{restatable}

\begin{corollary}
  Let $\mathcal{Z}=(Z,\zeta)$ and $\mathcal{Z}'=(Z',\zeta')$ be two
  priced zones. Then we can effectively decide whether $\mathcal{Z}
  \sqsubseteq \mathcal{Z}'$.
\end{corollary}

\subsection{Finding the right $Y$}
\label{sec:righty}

Applying \Cref{lemma:decompos}, the main obstacle to efficiently
decide~$\sqsubseteq_M$ is to find the appropriate $Z_Y$ in which the sought
supremum is reached. Unless good arguments can be found to guide the search
towards the best choice for~$Y$, an exhaustive enumeration of all the $Y$'s
will be required.

\begin{example}
  We consider the zone $Z$ defined by the constraints $x \geq 0$, $y
  \geq 1$, $x \leq y$ and $y \leq x+2$.  We fix $M(x) = 2$ and $M(y) =
  3$.  We then consider $Z' = Z$. The zone~$Z$ is equipped with a constant cost
  function~$\zeta$.
In~\Cref{fig:examples:middle}, $Z'$ is attached $\zeta'(x,y) = x + y$, and
  the  
  expression of the function 
$f(v) = \inf_{v' \in Z',\ v' \equiv_M v} \zeta'(v')$ 
  is given in each~$Z_Y$, for~$Y\subseteq X$.
  It is then easy to see that the supremum of~$f$ is reached at the
  point 
  $(2,3)$, in the middle of the zone.  In~\Cref{fig:examples:border},
  we~take $\zeta'(x,y) = 2x - y$, and the expression of the function
  $f(v) = \inf_{v' \in Z'.\ v' \equiv_M v} \zeta'(v')$ is given in
  each~$Z_Y$.  The supremum of~$f$ is then reached at the point 
  $(2,2)$, on the border, but not at a corner of the zone.  The latter
  example also shows that $f$ is not continuous on the whole zone~$Z$.
\end{example}

\begin{figure}[tb]
\colorlet{vert}{green!70!blue}
\centering
\subfigure[The sought supremum is reached in the middle of the
zone.\label{fig:examples:middle}]{
\begin{tikzpicture}[scale=.9]
\draw[latex'-latex'] (4,0) -| (0,5);
\path[use as bounding box] (-1,-.4);
\draw[-latex',line width=1pt] (.6,3.5) -- (1.3,4.2) node[midway,above left]
  {$\zeta'$};
\begin{scope}[opacity=1]
\fill[shading=axis,lower left=vert!0!white,upper left=vert!40!white,lower
right=vert!20!white,upper right=vert!100!white] (2,2) -- (1,1) -| (0,2) -- (1,3) -| (2,2);
\fill[shading=axis,lower left=vert!60!white,lower right=vert!60!white,
  upper left=vert,upper right=vert]  (2,2) -- (3,3) -| (2,2);
\fill[shading=axis,lower left=vert!60!white,lower right=vert,
  upper left=vert!60!white,upper right=vert]  (1,3) -- (2,4) |- (1,3);
\fill[vert] (2,4) |- (3,3) -- (4,4) |- (3,5) -- (2,4);
\end{scope}
\draw[vert!60!black,line width=1pt] (4,4) -- (1,1) -| (0,2) -- (3,5);
\draw[dotted] (0,3) node[left] {$M(y)=3$} -- +(4,0);
\draw[dotted] (2,0) node[below] {$M(x)=2$} -- +(0,5);
\draw[-latex'] (1,2) -- +(.5,.5) node[midway,left] {$\scriptstyle x+y$};
\draw[-latex'] (1.4,3.1) -- +(.5,0) node[midway,above] {$\scriptstyle x+3$};
\draw[-latex'] (2.1,2.4) -- +(0,.5) node[midway,right] {$\scriptstyle y+2$};
\draw (3,4) node {$\scriptstyle 5$};
\fill (2,3) circle (.5mm) node[above right] {$\scriptstyle 5$};
\draw (1,.3) node {$Y=\emptyset$};
\draw (3,.3) node {$Y=\{x\}$};
\draw (1,5.3) node {$Y=\{y\}$};
\draw (3,5.3) node {$Y=\{x,y\}$};
\end{tikzpicture}
}
\hfill
\subfigure[The sought supremum is reached on the border of the
zone.\label{fig:examples:border}]{
\begin{tikzpicture}[scale=.9]
\draw[latex'-latex'] (4,0) -| (0,5);
\path[use as bounding box] (-1,-.4);
\draw[-latex',line width=1pt] (.3,4.2) -- (1.3,3.7) node[midway,above right]
  {$\zeta'$};
\begin{scope}
\fill[shading=axis,shading angle=-71,lower left=vert!0!white,
  upper left=vert!60!white,lower right=vert!60!white,upper right=vert]  (2,2) -- (1,1) -| (0,2) -- (1,3) -| (2,2);
\fill[shading=axis,lower left=vert!85!white,lower right=vert!85!white,
  upper left=vert!75!white, upper right=vert!75!white]  (2,2) -- (3,3) -| (2,2);
\fill[shading=axis,lower left=vert!40!white,lower right=vert!60!white,upper right=vert!60!white]  (1,3) -- (2,4) |- (1,3);
\fill[vert!60!white] (2,4) |- (3,3) -- (4,4) |- (3,5) -- (2,4);
\end{scope}
\draw[vert!60!black,line width=1pt] (4,4) -- (1,1) -| (0,2) -- (3,5);
\draw[dotted] (0,3) node[left] {$M(y)=3$} -- +(4,0);
\draw[dotted] (2,0) node[below] {$M(x)=2$} -- +(0,5);
\draw[-latex'] (.7,2.2) -- +(.8,-.4) node[midway,above right] {$\scriptstyle 2x-y$};
\draw[-latex'] (1.4,3.1) -- +(.5,0) node[midway,above] {$\scriptstyle x-2$};
\draw[-latex'] (2.1,2.9) -- +(0,-.5) node[midway,right] {$\scriptstyle 4-y$};
\draw (3,4) node {$\scriptstyle 0$};
\fill (2,2) circle (.5mm) node[below right] {$\scriptstyle 2$};
\draw (1,.3) node {$Y=\emptyset$};
\draw (3,.3) node {$Y=\{x\}$};
\draw (1,5.3) node {$Y=\{y\}$};
\draw (3,5.3) node {$Y=\{x,y\}$};
\end{tikzpicture}
}
\caption{The supremum may lie in the middle of zones or facets}
\label{fig:examples}
\end{figure}

Nevertheless, in many cases, we will be able to guide the search of the $Z_Y$
where the sought optimal is to be found.
The~following development
focuses on the zone, not on the cost function. Given a zone~$Z$, we define a
preorder~$\preceq$ on the clocks, such that if $Z_Y \neq \emptyset$, then
$Y$~is downward-closed for~$\preceq$. In~other words, whenever $x \preceq y$,
$y \in Y$ and $Z_Y \neq \emptyset$, then $x \in Y$. The knowledge of $\preceq$
can be a precious help to guide the enumeration of non-empty~$Z_Y$'s.
Indeed, if $Z_Y \neq \emptyset$, $Y$~is downward-closed for~$\preceq$, and
candidates for~$Y$ are thus found by enumerating the antichains of~$\preceq$.
In particular, if $\preceq$ is total, then there are at most $|X|+1$ sets $Y$
such that $Z_Y \neq \emptyset$.

To be concrete, 
let $X_{\le M}$ and~$X_{>M}$ be the (disjoint) sets of clocks~$x$ such that $Z
\subseteq (x \le M(x))$ and ${Z \subseteq (x>M(x))}$, respectively.
We~define the relation~$\preceq_Z$ as the
least
relation satisfying the following conditions:
\begin{itemize}
\item for each $x \in X_{\le M}$, for each $y \in X$, $x \preceq_Z y$;
\item for each $y \in X_{>M}$, for each $x \in X$, $x \preceq_Z y$;
\item for each $y \in X_{>M}$, for each $x \in X \setminus X_{>M}$, $y
  \not\preceq_Z x$;
\item for all $x,y \in X \setminus X_{>M}$, $Z \subseteq (x-y \le
  M(x)-M(y))$ implies $x \preceq_Z y$.
\end{itemize}
It is not difficult to show that $\preceq_Z$ is a preorder such that:
$y \in X_{\le M}$ and $x \preceq_Z y$ implies $x \in X_{\le M}$, and
$x \in X_{>M}$ and $x \preceq_Z y$ implies $y \in X_{>M}$.

\begin{restatable}{lemma}{Ydownwardclosed}
  Let $Y \subseteq X$ such that $Z_Y \ne \emptyset$. Then $Y$ is
  downward-closed for $\preceq_Z$.
\end{restatable}

The preorder $\preceq_Z$ can be computed in polynomial time, since it
only requires to check emptiness of zones, which can be done in time
polynomial in $|X|$ (cubic in $|X|$ with DBMs for instance).

We recall that, if $Z$ is a zone generated in a timed automaton where
only resets of clocks to $0$ are allowed, for any pair of clocks
$x,y$, it~cannot be the case that $Z$ crosses the diagonal hyperplane of
equation $x = y$. 

\begin{proposition}
  \label{prop:total_samebound}
  If $Z$ is generated by a timed automaton, and all clocks have the
  same bound $M$, then $\preceq_Z$ is total.
\end{proposition}

\begin{proof}
  Let $x$ and $y$ be two clocks. Since $Z$ is generated by a timed automaton,
  it~is contained either in the half-space of equation $[x \leq y]$, or in the
  one of equation $[x \geq y]$. By definition of $\preceq_Z$, and since $M(x)
  = M(y)$, the former entails $x \preceq_Z y$, and the latter $y \preceq_Z x$.
  Any two clocks are thus always comparable, and $\preceq_Z$ is therefore
  total. \qed
\end{proof}

Under the assumptions of~\Cref{prop:total_samebound}, there are
polynomially many subsets $Y\subseteq X$ to~try.
  Note that these assumptions are easily realized by taking
  $\widetilde{M} = \max_{x \in X} M(x)$ as the unique maximal constant for
  all the clocks.
  Formally, $\sqsubseteq_{\widetilde{M}}$~is~an under-approximation of
  the exact version of $\sqsubseteq_M$.  This approximation does not
  hinder correctness, and illustrate the trade-off between the
  complexity of the inclusion procedure and the number of priced zones
  that will be explored.
