\documentclass[preprint,authoryear,12pt]{elsarticle}




\usepackage{amssymb}
\usepackage{graphicx}
\usepackage{epsfig}
\usepackage{array}
\usepackage{mathrsfs}
\usepackage{amsfonts}
\usepackage{amsmath}
\usepackage{amssymb}
\usepackage{mathrsfs}
\input xy
\xyoption{all}


\usepackage{tipa}
\usepackage{pifont}
\usepackage{cases}
\usepackage{txfonts}
\usepackage{graphicx}
\usepackage{array}
\usepackage{mathrsfs}
\usepackage{amsfonts}
\usepackage{setspace}
\usepackage{subfigure}

\newtheorem{Theorem}{Theorem}
\newtheorem{Lemma}{Lemma}
\newtheorem{Proposition}{Proposition}
\newdefinition{Definition}{Definition}
\newtheorem{Corollary}{Corollary}
\newtheorem{Notation}{Notation}
\newtheorem{Remark}{Remark}
\newtheorem{Example}{Example}
\newtheorem{Problem}{Problem}
\newtheorem{Algorithm}{Algorithm}
\newproof{proof}{Proof}
\newproof{pot}{Proof of Theorem \ref{thm2}}







\journal{Automatica}

\begin{document}

\begin{frontmatter}
\thispagestyle{empty}



\title{Bisimilarity Enforcing Supervisory Control for Deterministic
Specifications \tnoteref{label1}}





\author[Singapore]{Yajuan Sun}\ead{sunyajuan@nus.edu.sg},    \author[USA]{Hai Lin}\ead{hlin1@nd.edu},               \author[Singapore]{Ben M. Chen}\ead{bmchen@nus.edu.sg}  
\address[Singapore]{Dept. of Electrical and Computer Engineering, National University of Singapore, Singapore}  \address[USA]{Dept. of Electrical Engineering, University of
Notre Dame, USA}







\begin{abstract}

This paper investigates the supervisory control of
nondeterministic discrete event systems to enforce bisimilarity
with respect to deterministic specifications. A notion of
synchronous simulation-based controllability is introduced as a
necessary and sufficient condition for the existence of a
bisimilarity enforcing supervisor, and a polynomial algorithm is
developed to verify such a condition. When the existence condition
holds, a supervisor achieving bisimulation equivalence is
constructed. Furthermore, when the existence condition does not
hold, two different methods are provided for synthesizing maximal
permissive sub-specifications.
\end{abstract}


\newpage
\setcounter{page}{1}

\begin{keyword}
Supervisory control \sep bisimulation \sep discrete event systems

\end{keyword}

\end{frontmatter}


\label{}




\section{INTRODUCTION}
The notion of bisimulation introduced by
\cite{milner1989communication} has been successfully used as a
behavior equivalence in model checking \citep{clarke1997model},
software verification \citep{chaki2004abstraction} and formal
analysis of continuous \citep{tabuada2004bisimilar}, hybrid
\citep{tabuada2004compositional} and discrete event systems
(DESs). What makes bisimulation appealing is its capability in
complexity mitigation and branching behavior preservation,
specially when we deal with large scale distributed and concurrent
systems such as multi-robot cooperative tasking, networked
embedded systems, and traffic management.




Therefore, recent years have seen increasing research activities
in employing bisimulation to DESs. References
\citep{barrett1998bisimulation}, \citep{komenda2005control} and
\citep{sumodel} used bisimulation for the control of deterministic
systems subject to language equivalence.
\cite{madhusudan2002branching} investigated the control for
bisimulation equivalence with respect to a partial specification,
in which the plant is taken to be deterministic and all events are
treated to be controllable. \cite{tabuada2008controller} solved
the controller synthesis problem for bisimulation equivalence in a
wide variety of scenarios including continuous system, hybrid
system and DESs, in which the bisimilarity controller is given as
a morphism in the framework of category theory.
\cite{zhou2006control} investigated the bisimilarity control for
nondeterministic plants and nondeterministic specifications. A
small model theorem was provided to show that a supervisor
enforcing the bisimulation equivalence between the supervised
system and the specification exists if and only if a state
controllable automaton exists over the Cartesian product of the
system and specification state spaces. This small model theorem
was also extended for partial observation in
\citep{zhou2007small}. In both these works, the existence of a
bisimilarity supervisor depends on the existence of a state
controllable automaton, which is hard to calculate in a systematic
way, and the complexity of checking the existence condition is
doubly exponential. To reduce the computational complexity,
\cite{zhoubisimilarity2011} specialized to deterministic
supervisors. The existence condition for a deterministic
bisimilarity supervisor considering nondeterministic plants and
nondeterministic specifications was identified. Moreover, the
synthesis of deterministic supervisors, feasible supspecifications
and infimal subspecifications were developed as well.
\cite{liu2011bisimilarity} introduced a simulation-based framework
upon which the bisimilarity control for nondeterministic plants
and nondeterministic specifications was studied. In particular, a
new scheme based on the simulation relation was proposed for
synchronization which is different from those commonly used
synchronization operators such as parallel composition and product
in the supervisory control literature.








This paper studies the supervisory control of nondeterministic
plants for bisimulation equivalence with respect to deterministic
specifications. Compared to the existing literature, the
contributions of this paper mainly lie on the following aspects.
First, a novel notion of synchronous simulation-based
controllability is introduced as a necessary and sufficient
condition for the existence of a bisimilarity enforcing
supervisor. Although it is equivalent to the conditions in
\citep{zhoubisimilarity2011} specialized to deterministic
specifications, it provides a great insight into what characters
should a deterministic specification possesses for bisimilarity
control. Second, a test algorithm is proposed to verify the
existence condition, which is shown to be polynomial complexity
(less than the complexity of the conditions in
\citep{zhoubisimilarity2011}). When the existence condition holds,
we further present a systematic way to construct bisimilarity
enforcing supervisors. Third, since a given specification does
always guarantee the existence of a bisimilarity enforcing
supervisor, a key question arises is how to find a maximal
permissive specification which enables the synthesis of
bisimilarity enforcing supervisors. To answer this question, we
investigate the calculation of supremal synchronously
simulation-based controllable sub-specifications by using two
different methods. One is based on a recursive algorithm and the
other directly computes such a sub-specification based on
formulas.








The rest of this paper is organized as follows. Section 2 gives
the preliminary and problem formulation. Section 3 presents the
synthesis of bisimilarity enforcing supervisors. Section 4
investigates the test algorithm for the existence of a
bisimilarity enforcing supervisor. Section 5 explores the
calculation of maximal permissive sub-specifications. This paper
concludes with section 6.


\section{Preliminary and Problem Formulation}

\subsection{Preliminary Results}
A DES is modeled as a nondeterministic automaton , where  is the set of states,
 is the set of events,  is the transition function,  is the initial
state and  is the set of marked states. The event
set  can be partitioned into  = , where  is the set of uncontrollable
events and  is the set of controllable events. Let
 be the set of all finite strings over 
including the empty string . The transition function
 can be extended from events to traces, , which is defined
inductively as: for any , ; for
any  and , . If the transition function
is a partial map , 
is said to be a deterministic automaton. For ,
the notation  means  is
restricted from a smaller domain  to
. Given , the subautomaton of  with
respect to , denoted by , is defined as: , where  and  = . The active event set at state  is defined as
 is defined\}.
Given a string , the length of the string ,
denoted as , is the total numbers of events, and  is
the - event of this string, where .
Given , a projection
:  is used to filter a string of events from 
to , and it is defined inductively as follows:
; for any
 and ,
 if , otherwise,
. The language generated by  is defined as
 is defined, and
the marked language generated by  is defined as \}.
Consider three languages . The
Kleene closure of , denoted as , is the language
, where  and
for any , . The prefix closure of ,
denoted as , is the language . The quotient
of  with respect to , denoted as , is the
language . For two languages  with , let  be a
deterministic automaton such that  and
. For a nondeterministic , let
 be a minimal deterministic automaton such that
 and .

To model the interaction between automata, we introduce parallel
composition as below \citep{cassandras2008introduction}.













\begin{Definition}\label{parallel}
Given  and , the parallel composition
of  and  is an automaton

where for any ,  and , the transition function is defined as:


\end{Definition}

When , parallel composition can be understood
as a form of control, where a supervisor is designed to restrict
the behavior of the plant.

Next we present the synchronized state map, which is used to find
the synchronized state pairs of two automata
\citep{zhou2006control}.

\begin{Definition}
Given  and , the synchronized state
map :  from  to 
is defined as

\end{Definition}








Most literature on supervisory control aims to achieve language
equivalence between the supervised system and the specification.
The necessary and sufficient condition for the existence of a
language enforcing supervisor is captured by the notion of
language controllability as below \citep{ramadge1987supervisory}.


\begin{Definition}\label{langc}
Given , a language  is said to be language controllable with respect to 
and  if

\end{Definition}






As a stronger behavior equivalence than language equivalence,
bisimulation is stated as follows \citep{milner1989communication}.
It is known that bisimulation implies language equivalence and
marked language equivalence, but the converse does not hold.


\begin{Definition}
Given  and , a simulation relation
 is a binary relation  such
that  implies:
\begin{enumerate}
\item[(1)]  such that ;

\item[(2)] .
\end{enumerate}
\end{Definition}

If there is a simulation relation    such that ,  is said to be
simulated by , denoted by . For
, if ,  and  is symmetric, 
is called a bisimulation relation between  and ,
denoted by . We sometimes omit the
subscript  from  or  when it is
clear from the context. Then we present a motivating example of
this paper.



\subsection{A Motivating Example}

\begin{figure}[!htb]
\begin{center}
\includegraphics*[scale=.5]{plant2.eps}
\caption{ multi-robot system (MRS) (Left),  (Middle) and
(Right)} \label{plantdet}
\end{center}
\end{figure}
Consider a cooperative multi-robot system (MRS) configured in Fig.
\ref{plantdet} (Left). The MRS consists of two robots  and
. Both of them have the same communication, position,
pushing, scent-sensing and frequency-sensing capabilities.
Furthermore,  has color-sensing capabilities, while  has
shape-sensing capability.  and  can cooperatively search
and clear a dangerous object (the white cube) in the workspace.
Initially,  and  are positioned outside the workspace.
Let . When the work request announces (event ), 
is required to enter the workspace. Due to actuator limitations,
it nondeterministically goes along one of two pre-defined paths
(event ). In the first path,  activates color-sensing
(event ) and scent-sensing (event ) capabilities to detect
the dangerous object; whereas in the second path, besides
color-sensing and scent-sensing capabilities,  also activates
frequency-sensing (event ) for detection. Similarly, 
activates shape-sensing (event ), scent-sensing and
frequency-sensing capabilities in the first path, while in the
second path it activates shape-sensing and scent-sensing
capabilities. After detecting the dangerous object,  pushes
the dangerous object outward the workspace (event ), and then
returns to the initial position (event ) for the next
implementation.

\begin{figure}[!htb]
\begin{center}
\includegraphics*[scale=.5]{moti.eps}
\caption{ (First Left),  (Second Left),  (Second
Right) and  (First Right)} \label{spec}
\end{center}
\end{figure}



The automaton model  of  with alphabet  is
shown in Fig. \ref{plantdet}, where  and . Since  can
not disable the host computer to broadcast the work announcement,
the event  is deemed uncontrollable, that is . The rest events are controllable. The cooperative
behavior of  and  can be represented as  (Fig.
\ref{spec} (First Left)). The specification , configured in
Fig. \ref{spec}, is given in order to restrict the cooperative
behavior . According to the specification, after both
 and  receive the work command and go to the workspace,
two possible states may be reached by the MRS
nondeterministically. In the first state, the color sensor, the
shape sensor and the scent sensors can be adopted to confirm an
objective is dangerous. However, to save the energy, in the second
state only the color sensor and the shape sensor can be adopted
for dangerous object detection. After the detection, the dangerous
object is cleared from the workspace.





\begin{figure}[!htb]
\begin{center}
\includegraphics*[scale=.5]{motispec1.eps}
\caption{ (Left),  (Middle)
and  (Right)} \label{proj}
\end{center}
\end{figure}


For such a MRS, if we use language equivalence as behavior
equivalence, the control target is to design supervisors  and
 such that .
According to the results in \citep{willner1991supervisory}, this
problem can be solved by designing  such that
. Since  is language controllable with respect to
 and , we can construct  as shown in
Fig. \ref{spec}. So the supervised system  (Fig. \ref{proj} (Left)) is language equivalent to
. However, it can be seen that 
enables all the color sensor, the shape sensor and the scent
sensors for dangerous object detection, which violates the energy
saving requirement in the specification. Hence langauge
equivalence is not adequate for this case, which calls for the use
of bisimulation as behavior equivalence. That is, we need design
supervisor  such that . For such a bisimilarity control problem, a promising method
\citep{karimadini2011guaranteed} is to decompose the global
specification  into sub-specifications  with alphabet
 for  (Fig. \ref{proj}) such that  . If we can design  such that , then . In
particular,  is deterministic, which motivates us to
consider the bisimilarity control for deterministic specifications
in this paper.





\subsection{Problem Formulation}
In the rest of paper, unless otherwise stated we will use , 
and  to denote the
nondeterministic plant, the deterministic specification and the
supervisor (possibly nondeterministic) respectively. Next we
formalize the notion of bisimilarity enforcing supervisor, which
always enables all uncontrollable events and enforces bisimilarity
between the supervised system and the specification.



\begin{Definition}\label{bissup}
Given a plant  and a specification , a supervisor  is
said to be a bisimilarity enforcing supervisor for  and  if:

(1) There is a bisimulation relation  such that ;

(2) .

\end{Definition}


This paper aims to solve the following problems.

{\bf Problem 1:} Given a nondeterministic plant  and a
deterministic specification , what condition guarantees the
existence of a bisimilarity enforcing supervisor  for  and
?

{\bf Problem 2:} How to check this condition effectively?

{\bf Problem 3:} If the condition is satisfied, how to construct a
bisimilarity enforcing supervisor ?

{\bf Problem 4:} If the condition is not satisfied, how to obtain
a maximal permissive sub-specification which enables the synthesis
of bisimilarity enforcing supervisors?






\section{ Supervisory Control for Bisimilarity}





This section investigates Problem 1 and Problem 3, also called the
bisimilarity enforcing supervisor synthesis problem. We begin with
the existence condition of a bisimilarity enforcing supervisor.
For sufficiency, since we need design a bisimilarity enforcing
supervisor, the following concept is introduced.

\begin{Definition}
Given , the
uncontrollable augment automaton  of  is defined as:

where for any  and :

\end{Definition}

We can see that an uncontrollable augment automaton can be
employed in the construction of bisimilarity enforcing supervisors
because it naturally satisfies the condition (2) required for a
bisimilarity enforcing supervisor (Definition \ref{bissup}).


On the other side, for necessity we have , which
implies . Hence  is a necessary
condition to guarantee the existence of a bisimilarity enforcing
supervisor. Moreover,  implies , thus
language controllability of the specification is also a necessary
condition for the existence of a bisimilarity enforcing
supervisor. To satisfy those necessary conditions, we will
introduce synchronous simulation-based controllability as a
property of the specification. Before that, we need the following
concept.


\begin{Definition}
Given ,  and a simulation
relation  such that ,  is
called a synchronous simulation relation from  to  if
 for any  and .
\end{Definition}

If there exists a synchronous simulation relation  from
 to ,  is said to be synchronously simulated by
, denoted as . For a deterministic
specification , if  is synchronously simulated by , then
 possesses the branches which are bisimilar to  and the
branches which are outside . Hence it turns out that . If  is further language controllable with respect to
 and , then , implying that
 is a candidate of bisimilarity enforcing supervisor. Base
on this observation, we provide the following concept.





\begin{Definition}\label{syncdef}
Given  and ,  is said to be
synchronously simulation-based controllable with respect to 
and  if it satisfies:

(1) There is a synchronous simulation relation  such that
;

(2)  is language controllable with respect to  and
.

\end{Definition}

It is immediate to see that when  is synchronously
simulation-based controllable with respect to  and
, it not only satisfies the necessary conditions ( and language controllability of ) for the existence
of a bisimilarity enforcing supervisor but also enables the
development of  as a bisimilarity enforcing supervisor to
accomplish the sufficiency of the existence condition.








Then we present a necessary and sufficient condition for the
existence of a bisimilarity enforcing supervisor.

\begin{Theorem}\label{t}
Given a plant  and a deterministic specification , there
exists a bisimilarity enforcing supervisor  for  and  if
and only if  is synchronously simulation-based controllable
with respect to  and .
\end{Theorem}

\begin{proof}
For sufficiency, we choose  as the supervisor. Let
 .
Consider a relation . We show that  is a
bisimulation relation from  to . First note that . Pick  and , where .
By the definition of parallel composition, we have , which implies .
When , then . On the other
side, pick  and . Since  and there is a synchronous
simulation relation  such that , we
have . Then there is 
such that , and if , then . It follows that 
and  when . That is, . Hence . Moreover from determinism and language
controllability of  and the fact that  adds every state
a transition to  through undefined uncontrollable events does
not change the result of parallel composition, we have
. It implies that .

For necessity, suppose there is a bisimilarity enforcing
supervisor  for  and . Then, there is a bisimulation
relation  such that  and . Let . Consider a relation
. We show that  is a synchronous simulation
relation from  to . By the definition of parallel
composition,  is a simulation relation from  to .
Assume there is  and  such that . Hence there exists  such
that  and . Since , for , there is  such that , which implies  and in turn implies
. Because , for , there is  such that . Since  is deterministic, we have .
Therefore, , which implies . It introduces a contradiction. Then the assumption is not
correct. That is, for any  and , . So . Next we show language
controllability of . Since a bisimilarity enforcing
supervisor  enables all uncontrollable events at each state,
 is language controllable with respect to  and
, further,  implies . It
follows that  is language controllable w.r.t.  and
. So  is synchronously simulation-based
controllable w.r.t.  and .
\end{proof}


\begin{Remark}
Theorem \ref{t} shows that if a deterministic  is synchronously
simulation-based controllable with respect to  and
,  is a bisimilarity enforcing supervisor for
 and . Here synchronous simulation-based controllability of
 is equivalent to the conditions ( and
language controllability of ) specialized to deterministic
specifications \citep{zhoubisimilarity2011} to ensure the
existence of a deterministic bisimilarity supervisor. However, the
notion of synchronous simulation-based controllability offers
computation advantages compared to the conditions in
\citep{zhoubisimilarity2011} (See section 4). Moreover, it enables
the calculation of maximal permissive sub-specification when the
existence condition for a bisimilarity enforcing supervisor does
not hold (See section 5).
\end{Remark}






\begin{figure}[!htb]
\begin{center}
\includegraphics*[scale=.5]{clsall.eps}
\caption{  (First Left),  (Second Left), 
(Second Right) and  (First Right)} \label{moticls}
\end{center}
\end{figure}


Now we revisit the motivating example.

\begin{Example}\label{lcbis}
Let . We need design supervisor  such that
. Since  is deterministic and
synchronously simulation-based controllable with respect to 
and , from Theorem \ref{t} we can design
 to be  (Fig. \ref{moticls} (Second Left)).
The supervised system  is shown in Fig. \ref{moticls}
(First Right) and it can be seen that , where  . In addition,
 for  can be designed as shown in Fig. \ref{moticls}
(First Left) according to our results in
\citep{sun2012bisimilarityacc}. Then 
(Fig. \ref{moticls} (Second Right)). As a result, .
\end{Example}






\section{A Test Algorithm for the Existence of a Bisimilarity
Enforcing Supervisor}

To solve Problem 2, an algorithm is proposed in this section to
test the existence of a bisimilarity enforcing supervisor. We
start by introducing synchronously simulation-based controllable
product, which will be used in the test algorithm.




\begin{Definition}
Given  and , the synchronously
simulation-based controllable product of  and  is an
automaton

where for any  and , the transition function is defined as:


\end{Definition}

Since synchronous simulation-based controllability is a necessary
and sufficient condition for the existence of a bisimilarity
enforcing supervisor, the following algorithm for testing
synchronous simulation-based controllability of  also verifies
the existence of a bisimilarity enforcing supervisor for  and
.



\begin{Algorithm}\label{algsync}
Given a plant  and a deterministic specification , the
algorithm for testing synchronous simulation-based controllability
of  with respect to  and  is described as
below.

Step 1: Obtain ;

Step 2:  is synchronously simulated-based controllable with
respect to  and  if and only if  and 
are not reachable in  and  for any
reachable state  in  with .
\end{Algorithm}



\begin{Theorem}\label{alg1c}
Algorithm 1 is correct.
\end{Theorem}
\begin{proof}
From the definition of synchronously simulation-based controllable
product, it is obvious that any  satisfying  is a state reachable in , and any  satisfies that . For synchronous simulation-based controllability to
hold, condition (1) and condition (2) of Definition \ref{syncdef}
should be satisfied. On the other hand, if condition (1) is
violated, there are two cases. Case 1: there exist  and
 such that  and . So . Case 2: there is  such that 
and  when . If condition (2) is
violated, i.e. there exist  and 
such that  and . So . It follows that  and  are reachable in
 or  for any reachable state  in  with  iff  is not
synchronously simulated-based controllable w.r.t.  and
.
\end{proof}


\begin{Remark}
Algorithm 1 can be terminated because the state sets and the event
sets of  and  are finite. Since  is nondeterministic and
 is deterministic, their numbers of transitions are  and  respectively. Then the complexity
of constructing  is . In
addition, the complexity of checking the reachability of  and
 in  is 
\citep{jones1975space}. So the complexity of Algorithm 1 is
. That is, the algorithm for testing
the existence of a bisimilarity enforcing supervisor has
polynomial complexity. \cite{zhoubisimilarity2011} used the
conditions such as  and  is language
controllable with respect to  and  to guarantee
the existence of a deterministic supervisor that achieves
bisimulation equivalence. The complexity of verifying those
conditions with respect to deterministic specifications is
 (Remark 2 in
\citep{zhoubisimilarity2011}). Hence, we argue that Algorithm 1 is
more effective.
\end{Remark}



We provide the following example to illustrate the algorithm for
checking synchronous simulation-based controllability.

\begin{figure}[!htb]
\begin{center}
\includegraphics*[scale=.5]{chk11.eps}
\caption{ Plant  (Left), Specification  (Middle) and
 (Right) of Example 2} \label{chk1}
\end{center}
\end{figure}


\begin{Example}\label{checksync}
Consider a plant  and a specification  with
 configured in Fig. \ref{chk1}. We can see
that  is not synchronously simulation-based controllable with
respect to  and  because for 
and , , and  is
defined at  but not .

Next we use Algorithm \ref{algsync} to test synchronously
simulation-based controllability of . The synchronously
simulation-based controllable product  is shown in
Fig. \ref{chk1} (Right). It can be seen that  and  are
reachable in . Hence  is not synchronously
simulation-based controllable with respect to  and
.
\end{Example}










\section{Supremal Synchronously Simulation-Based Controllable Sub-specifications}
This section studies Problem 4, i.e., the synthesis of supremal
synchronously simulation-based controllable sub-specifications,
because a synchronous simulation-based controllable
sub-specification ensures the existence of a bisimilarity
enforcing supervisor. First we introduce the notion of supremal.


Given  and , where  is a transitive and reflexive relation over ,  is said to be a supremal of , denoted by , if it
satisfies:

(1) : ;

(2) .

When we define the supremal of , a set  should be
given with respect to the element of . If the elements of 
are languages, the set  should be
applied because  includes all languages over
alphabet  and language inclusion fully captures the
comparison between two languages. However, if the elements of 
are automata, the set  should be applied, where  is
a full set of automata with alphabet  and  is the simulation relation, since  includes all
automata over alphabet  and the simulation relation is
adequate for automata (possibly nondeterministic) comparison.


We consider the class of sub-specifications that satisfies
synchronous simulation-based controllability as below.


It can be seen that the supremal of  with respect to  is a supremal synchronously simulation-based controllable
sub-specification. However, it is difficult to directly calculate
the supremal of  because  is not closed under the upper
bound (join) operator with respect to 
\citep{zhoubisimilarity2011}. To encounter this problem, we would
like to convert the automaton set  into equivalently
expressed language sets which are closed under the upper bound
(set union) operator with respect to 
\citep{cassandras2008introduction}. Next we do this conversion
item by item. First, for two deterministic automata  and ,
the condition  is equivalent to the language condition
 and . Second,
language controllability required in synchronous simulation-based
controllability is naturally a language description. It remains to
convert synchronous simulation relation required in synchronous
simulation-based controllability to an equivalent language
condition. To complete the conversion, we need the following
concept.



\begin{Definition}
Given , the synchronous state
merger operator on  is defined as an automaton

where , , and for any  and , the
transition function is defined as:



\end{Definition}





By using , the synchronous simulation relation from a
deterministic automaton  to a plant  is equivalent to
language conditions  and
, which is illustrated by
the following proposition.

\begin{Proposition}\label{gsyn}
Given a plant  and a deterministic automaton , there is a
synchronous simulation relation  such that  iff  and
.
\end{Proposition}
\begin{proof}
Let ,
 and
.
For sufficiency, consider a relation . We show that  is a
synchronous simulation relation from  to . First note
that . Pick  and , where . Since , there is  such that
 and . Hence
, moreover, . It follows that .
Therefore there exist  and . By the definition of , we have
 and , which
implies there is  such that , i.e. . Next we show that
 implies . Because ,
we have , in addition, . It follows , that is
, implying . So
.

For necessity, the induction method is used to prove  for any , that is . (1) , then . It is obvious that
. (2) Assume when , we have  for any . (3) . Let
, where . Because  and  is deterministic, for any , we have .
Since , for any , we have . It follows that . In addition,
 implies , which in turn implies
there is  such that .
Hence , that is, . Therefore for
any , we have , i.e. . Next we show  by proving  for any
. Since , there is  such that . Because
 implies  for any
, we have . Definition
of  implies , i.e. .
\end{proof}

Hence the automaton set  can be converted into the following
langauge sets:


The computation of supremal synchronously simulation-based
controllable sub-specification, i.e., , with respect to
, can be achieved through the computation of the
supremal languages of  and  with respect to
 as shown in the following theorem.



\begin{Theorem}\label{supeq}
Given a plant  and a deterministic specification , if
, then  .
\end{Theorem}
\begin{proof}
Let  and . First we show that
. Since , we have , which implies  is language controllable w.r.t.
 and  and . In
addition, definition of  implies . From Proposition \ref{gsyn}, it follows that
 is synchronously simulation-based controllable
w.r.t.  and . Since  also implies
 and  and  and
 are deterministic, we have . Therefore, . Next we show
that  for any . Suppose
there is  such that . Since , it implies ,
moreover,  and  are deterministic. It follows that  and . In
addition,  also implies synchronous
simulation-based controllability of . Hence  is
language controllable with respect to  and  and
there is a synchronous simulation relation  such that  implying 
and  according to
Proposition \ref{gsyn}. Hence . Moreover,
. By the definition of supremal,
we have  and , further,  and 
are deterministic. It follows that ,
which introduces a contradiction. Hence, the assumption is not
correct. That is, we have  for any
. So .

\end{proof}


Next we present a recursive algorithm for computing the supremal
synchronously simulation-based controllable sub-specification.

\begin{Algorithm}\label{alg2}
Given a plant  and a deterministic specification , the
algorithm for computing the supremal synchronously
simulation-based controllable sub-specification with respect to
 and  is described as follows:

Step 1: Obtain ,  and  ;

Step 2: ;

Step 3: , ;

Step 4: If , then the subautomaton
 of  is a supremal synchronously
simulation-based controllable sub-specification with respect to
 and .
\end{Algorithm}




\begin{Theorem}\label{calz}
Algorithm \ref{alg2} is correct.
\end{Theorem}
\begin{proof}
Consider , where  with  . First we show that . Definition of
 implies  is language controllable w.r.t. 
and , and the fact that  implies
  and . It follows that
. Next we show that  for
any . Suppose there is  such that
, that is, there is 
such that . Since , there exists  such that
, where ,
 and .
Hence there is  such that  and  with
, which implies . Moreover,
 and . It follows that . If , then ,
which implies . If , then
 because  is
language controllable w.r.t.  and ,  and . It in
turn follows that ,
, , . Hence . So there is a
contradiction, which implies the assumption is not correct. Then
 for any . As a result,
. It remains to show that .
By the definition of  and the fact that , we have 
. It follows that  is
a deterministic automaton such that  and
 . By Theorem \ref{supeq}, we have .

\end{proof}



\begin{Remark}
Algorithm \ref{alg2} can be terminated because the state set 
is finite. Because the state numbers of  and 
are both . Therefore, the complexity of Algorithm
\ref{alg2} is .
\end{Remark}









Furthermore, the supremal synchronously simulation-based
controllable sub-specification can be calculated by formulas
without applying the recursive algorithm.


\begin{Theorem}\label{calsupsub}
Given a plant  and a deterministic specification , if
, then
 is a supremal synchronously simulation-based
controllable sub-specification with respect to  and
, where .
\end{Theorem}
\begin{proof}
According to Theorem 1 and Theorem 2 in
\citep{brandt1990formulas}, we obtain . It follows that
. From Theorem \ref{supeq},  is a supremal
synchronously simulation-based controllable sub-specification
w.r.t.  and .
\end{proof}



Now we revisit Example \ref{checksync}.

\begin{figure}[!htb]
\begin{center}
\includegraphics*[scale=.5]{chk2.eps}
\caption{  (Left) and  (Right)} \label{chk2}
\end{center}
\end{figure}




\begin{Example}\label{sub2}
Example \ref{checksync} indicates that  is not synchronously
simulation-based controllable with respect to  and
. Thus, we would like to calculate the supremal
synchronously simulation-based controllable sub-specification with
respect to  and  by the proposed methods.

(1) Recursive Method: From Algorithm \ref{alg2}, we establish
 and , shown in Fig. \ref{chk2}. Then
 is achieved in (Fig.
\ref{chksup} (Left)). We obtain ,
 \\ and . Therefore, the supremal synchronously
simulation-based controllable sub-specification
 is obtained in Fig. \ref{chksup}.


\begin{figure}[!htb]
\begin{center}
\includegraphics*[scale=.5]{chksup1.eps}
\caption{  (Left) and
 (Right)} \label{chksup}
\end{center}
\end{figure}

(2) Formula-based Method: First we construct , which
can be seen in Fig. \ref{chk2} (Left). Hence . Thus,
=\\-
--
= and =. The
supremal synchronously simulation-based controllable
sub-specification  is achieved in
Fig. \ref{chksup} (Right).
\end{Example}

\section{Conclusion}
In this paper, we investigated the bisimilarity enforcing
supervisory control of nondeterministic plants for deterministic
specifications. A necessary and sufficient condition for the
existence of a bisimilarity enforcing supervisor was deduced from
synchronous simulation-based controllability of the specification,
which can be verified by a polynomial algorithm. For those
specifications fulling the existence condition, a bisimilarity
enforcing supervisor has been constructed. Contrarily, when the
existence condition does not hold, a recursive method and a
formula-based method have been developed to calculate the maximal
permissive sub-specifications.










\bibliographystyle{model5-names}
\bibliography{reference}
\end{document}
