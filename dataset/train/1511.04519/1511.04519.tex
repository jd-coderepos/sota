

\begin{comment}
Circuit simulation is an important and 
{\color{red} heavily used process}
during the cycle of integrated circuit (IC) designs.
The increasing integration between functional modules and interconnects
parasitic effects of modern VLSI design aggravates such simulation tasks,
rendering the transient simulation of the full system very inefficient.
Therefore, accurate yet fast circuit simulation methods have always been 
desired in the IC industry. 
\end{comment}
Theoretically sound and scalable algorithms
for linear circuit transient simulation have been always favored in the IC industry. 
It is the building block of all circuit simulators, 
which not only provides the foundation for fast general-purpose circuit anaylsis 
but also sharpen the edge dealing with modern computationally demanding power grid anaysis.
Nowadays, the emerging multi-core, many-core platforms bring powerful 
computing resource and opportunities for parallel computing. 
Even more, modern big data challenges drive cloud computing techniques \cite{cloud}, 
making distributed systems scaling to tens of thousands computing nodes
\cite{mesos, mapreduce, spark10, dryad}.
Such distributed systems will be also promising computing resources in EDA and IC industry.
However, building scalable and efficient distributed algorithmic framework for 
transient linear circuit simulation framework is still a challenge to leverage these powerful computing tools.
Previous works 
\cite{Dong08, Dong09, Li11, Weng12_ICCAD, Weng12_TCAD, Ye08, Wang13}
have been made in order to improve circuit simulation by novel algorithmics, 
parallel processing and distributed computing.

Traditional method solves differential algebra equations (DAEs) 
numerically in explicit ways, e.g., forward Euler (FE), or implicit ways, 
e.g., backward Euler (BE), Trapezoidal (TR), which are based on low order polynomial approximations for DAE.
Due to the stiffness of systems, which results from a wide range of time constants of a circuit,
the explicit methods require small time step sizes to ensure the stability.
However, implicit methods can conquer this problem by their larger stability
region. These implicit methods have to solve a linear system during each time step.
Linear systems resulted from circuit simulation is sparse and often ill conditioned.
Therefore, direct sparse matrix solver\cite{Davis06} are favored by 
linear circuit transient simulators and used as a state of art in 
current TAU power grid contest. 
There, only one factorization (LU or Cholesky factorization) is required
at the beginning of simulation.  
By reusing the factorization matrices, 
the transient computation performs only backward and forward substitutions 
and achieves better efficiency over adaptive stepping methods \cite{Xiong12, Yu12, Li12_Tau}.

Beyond the traditional ones, 
a new class of methods called exponential
time differencing (ETD) has been embraced by 
Weng \emph{et al.} \cite{Weng12_TCAD} 
which is called as MEXP.
The major computation of MEXP resides in calculating the matrix exponential. 
MEXP utilizes standard Krylov subspace method based on \cite{Moler03} to approximate matrix exponentials.  
These works solve the DAEs with higher 
polynomial expansion approximations \cite{Saad92} than traditional ones.
Nevertheless, there are still constraints for above methods. 
\begin{itemize}
\item
The solutions (state of system) at input transitions are required to solved and capture the 
system behavior, which force the adaptive stepping method to stop
and solve the linear systems or MEXP to regenerate Krylov subspace basis.
\item
The small resolution of transition distance constrains the step size of fixed time step method, 
while resorting to adaptive stepping alternatively brings matrix re-factorization overhead.
\item
Large dimension of Krylov subspace basis or small time step is needed to preserve the accuracy of
matrix exponential approximation when simulating stiff circuits, 
posing a memory bottleneck and computational burden for MEXP.
\end{itemize}


\emph{Our Contributions}. 
We devise a distributed algorithmic framework, called as \emph{MATEX}, 
of which the integral is still solved by matrix exponentials 
with variant Krylov subspace-based methods. 
\begin{itemize}
\item 
MATEX is a distributed framework leveraging the superposition property of linear system.
Through decomposing the input transition points, MATEX puts subsets of 
input transition points to different computing nodes and  process it parallelly.
\begin{itemize}
\item
In each computing node, it locally reduces the chance of Krylov subspace basis generations,
and provides more temporal interval (before the next transition point) 
so that matrix exponential can take advantage of large adaptive time steps only by scaling latest Krylov subspace basis.
\item
Meanwhile, traditional methods cannot benefit from this decomposition, since they still need to derive
solution step by step, either performing backward and forward substitutions 
or adaptive stepping with solving the linear system using direct or iterative methods.
\end{itemize}
\item 
Incorporating inverted and rational Krylov subspace bases into MATEX, we surmount the stiffness of MEXP and then form an efficient circuit solver within MATEX.
\begin{itemize}
\item 
This efficiently reduces the dimension of MEXP basis, 
remarkably decrease the memory consumptions 
and consequently boost computational performance. 
\item 
After unleashing from the constraint of stiffness, MATEX uses larger time size, expediting the whole time domain simulation.
\item
When dealing with singular capacitance/inductance matrix in DAE, these methods removes 
regularization process \cite{Chen12_TCAD, Weng12_TCAD}, which is required 
in MEXP\cite{Weng12_TCAD} and bring extra computational overhead.
\end{itemize}
\end{itemize}
\emph{Paper Organization}.
Section \ref{sec:method} introduces the background of
linear circuit simulation and matrix exponential formulations.
Section \ref{sec:matex} presents overall framework of MATEX.  
Section \ref{sec:advanced} illustrates the techniques to accelerate MATEX's circuit solver by
utilizing inverted and rational Krylov subspace methods. 
Section \ref{sec:complexity} presents complexity analysis.
Section \ref{sec:exp_results} shows numerical results 
and Section \ref{sec:conclusion} concludes this paper.


\begin{comment}
and its adaptive time controlling into MATEX. 
During the transient simulation, 
the input changes cause the updates of matrices. 
However, the updated region resides within only columns of the matrix.
By precomputing block LU calculation beforehand and 
updating that region step by step, MATEX does not require LU factorizations on-the-fly, 
which is very different from that of TR-like methods.


We also combine MATEX with \emph{inverted Krylov subspace} as an alternative technique,
which only requires decomposing the conductance matrix. 
Therefore,
it is advantageous the avoidance of inverting the 
matrix of capacitance and inductance, which may 
include many parasitic energy elements 
with large magnitude variations
after circuit extraction.


Additionally, the parallelism of MATEX
is based on the superposition property of linear circuit.
Within our theoritically sound skeleton,
we split the input sources, use all of divided subset of input sources 
to stimuli the system, seperately, making the whole simulation task divided
into serveral sub-tasks. 
These sub-tasks can be solved parallel, without sacrificing any accuracy. 
This circumvents the limitation (3).
Besides, the traditional methods (FE, BE, and TR) cannot benefit from this
splitting, due to their limitations not from the side of input sources
but from LTE and stability.
With the advent of inexpensive computer clusters, and multi/many-core technology,
this distributable algorithmic framework is a nice one for future implementations.


\begin{comment}


The major reason is that mainstream circuit simulators inherit implicit numerical
integration, e.g. backward Euler, Trapezoidal methods, 
for the sake of stability, where the explicit methods suffer. 
However, implict methods trouble the perspective of scalability
due to the requirement 
of solving large linear systems of differential equations.
On the contrary, the explict methods are more scalable and parallelizable 
for current multi-core and manycore architecture.
The merit of explict methods resides in simple time propogation scheme via 
matrix-vector multiplication who costs relative lower
computation and memory complexity about $O(n)$.
Even though the existence of stability issue, 
explicit methods still become a revived research 
area and verifies their parallelism and scalability 
within modern parallel computer architecture
~\cite{Dong08}\cite{Dong09}\cite{Weng12_ICCAD} during these years. 

One of promising and distinct method for circuit simulation
is matrix exponetial method (MATEX) proposed in 
\cite{Weng12_TCAD}. Apart from the above numerical methods, 
the DAE/ODE system (in one discritized region of certain time step) can  
be solved in analytical way via matrix expoential operator $e^{\mbf A}, 
\mbf A \in \mbf R_{n \times n}$. 
By the analytical property, this distinct class 
of numerical approach removes the local truncation  
and it is A-\emph{stable} for passive circuits. 
Therefore, it is not constraint by the stablity problem. 
{\color{red}{This property matters!}}
However, the kernel computation of MATEX lies in computing matrix expoential is very high,  
e.g., direct computation consumes around $O(n^3)$ complexity. 
Weng \emph{et al.} \cite{Weng12_TCAD}\cite{Weng12_ICCAD} introduced a \emph{standard Krylov subspace}
method-based MATEX for circuit simulation to solve this problem. 
In addition, unlike the implicit methods (Trapezoidal and Back Euler) 
they both require fixed step size so that they reduce the LU online for linear circuit
transient simulation,
this MATEX can utlize the scaling invariant property of Krylov subspace to do the adaptive time-steping 
without compromising any accuracy.
(1) MATEX can jump further than BE and Trap, because the exponenital 
denomator in the exponential approximation, it is much better
than the Taylor expansion.
(2) Under the (PWL) assumptions, MATEX can also carry 
the input information during the stepping.  
(3) With the help of Krylov subspace scaling invariant, it can re-evaluate
the inter time point without rebuilding Krylov subspace.

Nonetheless, the step size is actually limitted by the stiffness of matrix when utilizing standard Krylov subspace.
It is because the large spectrum of $\mbf A$ and the Arnoldi process focus on capturing well-distributed eigenvalues, so it ofter misses the dominant factors for the circuit system/for
matrix epxonential operator, which are the ones with smallest magnitudes.
Therefore, the Arnoldi process requires large number of order to preserve the 
accurate of matrix exponential calculation. 
Also, the standard Krylov-Based MATEX requires the process of regularization \cite{Chen12_TCAD} 
to deal with singular matrix $\mbf C$, i.e., nodes without grounded capacitance or 
inductance/voltage branches. Both drag the overall performance of MATEX for circuit simulation.  



In this paper, we integrated our previous works on MATEX for linear circuits 
using \emph{rational Krylov subspace}~\cite{Zhuang13_ASICON} and \emph{inverted Krylov subspace}.
They use 
$(\mbf I-\gamma \mbf A)^{-1}$ and $\mbf A^{-1}$, respectively, 
as the basis instead of $\mbf A$ (\emph{standard Krylov subspace}) in \cite{Weng12_TCAD}. 
The rational basis limits the spectrum of a circuit matrix into a unit circle,
and the inverted operation puts the important eigenvalues with smallest magnitudes from former spectrum
into a large magnitude ones. By using Arnoldi algorithm, the Hessenberg matrix can resemble the former matrix
$\mbf A$ projected onto the new invariant subspace better and contain the more accurate important eigenvalues
for matrix exponential operator.
This method provides a good convergent rate in Arnoldi process within MATEX framework.
The inverted Krylov subspace's convergent rate in Arnoldi process may not be as good as rational basis. However,
it only requires the LU factorization of conductance matrix $\mbf G$. It provides
a strategy dealing with $\mbf C$ is very complicated from post-layout extraction, but
$\mbf G$ is much more simple then. Similarly as rational basis, by the invertion $\mbf A$, the small magnitude eigenvalues 
become the large ones in the transformed spectrum, so that the exponential of a matrix can be accurately approximated.
A very typical and important application for circuit simulation is power grid analysis, which plays an 
important role in IC deisgn methodology. Given current stimulus and power grid structure, designers should verify
and pedict the worst-case voltage noise through the simulation before signing off their design. 
The model order reduction provides good way to reduce matrix of circuit system, e.g. 
PRIMA \cite{PRIMA98} and SPRIM \cite{SPRIM04}. However, with the large number of 
input sources, this kind of strategy cannot benefit from 
the reduction models any more. 
The majority works still use Trapezoidal method with fixed step size
for the power grid simulation.
Till now, accourding to Tau power grid simulation contest 2012 \cite{Li12_Tau}\cite{Yu12}, 
the trapezoidal method with fixed step size using one LU and 
many backward/forward substitutions lead to the best performance. 
The small and fixed time steps bring benefits for the transient power grid analysis 
when dealing the voltage variation 
with high frequency, i.e. the input is changing rapidly or constantly. 

However,
when package has an important impact on accurarte 
on-die power grid analysis~\cite{Pant06}\cite{Pant07}, 
and middle and low frequency phenomena is of interests to the designers, 
trapezoidal method with fixed time step becomes very inefficient. 
It is 
because such mehtod takes long time to represent the steady state of power grid system.
In the architecture analysis, power distribution networks (PDN) 
involves with pakcage and board 
level simulations \cite{PengLi12_ICCAD}\cite{Pant06}\cite{Pant07},
which requires longer timespan of simulation,
e.g. up to few $\mu s$, the simulation process are 
often time-consuming with the limitation of fixed step size to avoid online LU.
Our method can exploit large time step, ``steady region'' of waveform
during the simulation, meanwhile, it only requires one LU factoraization
as trapezoidal method, with backward/froward substitutions.
With equivalent error buget, the krylov subspace MATEX can jump larger time step/fewer 
number of time steps
than trapezoidal
method. Therefore, even though such fully implicit scheme can be used here,
when a sparse LU factorization is affordable,
the simulation burden is higher than our Krylov subspaces MATEX because of 
more time steps need.

The aim of this paper is to present and organize the Krylov subspace based 
MATEX solver in transient simulation of linear circuit.
We list our main contributions of this paper are as follows.
\begin{itemize}
\item Accelerate MATEX via rational and inverted Krylov 
subspace.
\item Using adaptive time 
stepping strategy for circuit simulation and overcome the 
problem to achieve low-frequency pheonomena of circuit system.
\item Eliminating the regularization process 
for singular matrix $\mbf C$ for explicit method 
of circuit simulation.
\item By input grouping and superposition property of linear circuit 
to further accelerate circuit simulation.
\item Establish the relation of MATEX with previous works on Model Order Reduction. 
\end{itemize}

By such contributions, we are able to achieve hihgly efficient trasient 
simulation. Compared to the works~\cite{Weng12_TCAD}.
It is very nice property from linear system, which they can be 
superposition. 
By this idea, the same system with the parts of inputs send to different
machines, cores, or threads and add together without communication 
during the process, only require on precomputed time stamp/anchors.
It is a nice parallelism framework, we provide such for the further 
extension.
It is noted that Trapezodal cannot achieve this benefit, because
it is not analytical solution, and the inputs are not the 
hurdles during their simulation but the stability and local trancation
error are.

The paper is organized as follows. Section II is the fundemantals 
of circuit simulation and matrix exponential inteperation.
Section III describes the Krylov methods for fast matrix exponential evaluations. 
They are three algorithms for such computation, Standard Krylov subspace, inverted 
Krylov subspace, Rational krylov subspace.
Section IV presents the computational results of simple linear circuit cases, also contains industrial power grid designs and the famous IBM benchmark suite.

\begin{comment}

\begin{itemize}
\item Accelerate MATEX via rational and inverted Krylov subspace and its coresponding adaptive time stepping strategy for circuit simulation.
\item Eliminating the regularization process for singular matrix $\mbf C$ for explicit method of circuit simulation.
\item Speedups linear circuit simulation, by adaptive time stepping.\end{itemize}
For solving ordinary differential equations $x'=Ax+f(t)$,
it has an anlytical solution $x(t+h)=e^{Ah}x(t)+ \int_{0}^{h}e^{A(h-\tau)}f(t+\tau)d\tau$. 
When $A$ is large, it is very expensive to cacluate the exponential operator directly.  
The explict and implict methods are strategies to approximate the operation of $e^{Ah}$.  
Explicit methods required matrix-by-vector multiplications, however, implicit methods need to solve 
a whole linear system.  
It is known that explicit methods are constraint on the size of the time steps required to make sure stability. 
Compared to explicit methods, Implicit methods have more difficulties on utilizing current 
parallel and distributed machines.  
Recent development of Krylov subspace accelerates the exponential operations.  
The  
\begin{itemize}
\item without regularization, 
    \item Accelerate and enpower MATEX framework via rational Krylov subspace algorithm.
    \item Enabling adaptive time stepping for circuit simulation.
    \item Establish the relation of MATEX with previous works on Model Order Reduction. As a explicit circuit solver, which is also enjoy the model order reduction framework.
    \item ?? Generalize the explict method for future parallelism circuit simulation (Stablity).
\end{itemize}
The MATEX has a good characteristic that the exponential operator bares the propagation in the perspective of
time derivitate,
which the method at Forward Euler (FE), Backward Euler (BE) and Trapezoidal (TRAP) method use time step $h$ to discretize the
dynamical system with respect to time. 
They lead to explicit and implicit methods,
which bring the stability and limitted time step due to the system, e.g. 
stiff systems.
Explicit methods are relatively cheap computation; Implicit methods are usually time consuming because they require
to solve the large linear systems. For this regard, in power grid simulation, it precomputes the matrix 
($\mbf C/h+ \mbf G$) in BE, or ($\mbf C/h + \mbf G/2$) in TRAP. This involes the time step $h$. 

Also, in linear circuit, the the mexponential operator is a explicit method, of which the matrix-vector multiplication
take the major part of computation.



Power grid simulation plays an important role in IC design methodology. Given current stimulus and the power grid structure, 
designers could verify and predict the worst-case voltage noise through the simulation before signing off their design.  
However, with the huge size of modern design, power grid simulation is a time-consuming process. Moreover, manifesting effects 
from the package and the board would require longer simulation time, e.g., up to few $\mu s$, which worsens the performance 
of the power grid simulation. Therefore, an efficient power grid simulation is always a demand from industry. 

Conventionally, the power grid simulation is based on the trapezoidal method where the major computation is to solve a linear 
system by either iterative approaches\cite{Saad03} or direct methods\cite{Davis06}. Many previous 
works~\cite{Chen01_Efficient}\cite{Feng08}\cite{Kozhaya02}\cite{Nassif00_Fast}\cite{Su03} utilize iterative approaches to tackle 
the scalability issue and enable adaptive stepping in the power grid simulation. However, the iterative methods usually suffer
from the convergence problem because of the ill-conditioned matrix from the power grid design. On the other hand, the direct 
methods, i.e., Cholesky or LU factorizations, are more general for solving a linear system. Despite the huge memory 
demanding and computational effort, with a carefully chosen step size, the power grid simulation could perform only one 
factorization at the beginning while the rest of operations are just backward/forward substitutions. Since a power grid design 
usually includes board and package models, a long simulation time is required to manifest the low-frequency response. Hence, 
the cost of expensive factorization can be amortized by many faster backward/forward substitutions. Such general factorization 
and fixed step size strategy~\cite{Xiong12}\cite{Yang12}\cite{Yu12}\cite{Zhao02} is widely adopted in industry.  

Beyond the traditional trapezoidal method, Weng \emph{et al.}~\cite{Weng12_TCAD, Weng12_ICCAD} introduced 
the matrix exponential method (MATEX) for the circuit simulation recently. MEXP has better accuracy and adaptivity because of the 
analytical formulation and the scaling invariant Krylov subspace approximation. Unlike the fixed step size strategy, MATEX could 
dynamically adjust the step size to exploit the low-frequency response of the power grid without expensive computation. However, 
the step size in MATEX is limited by the stiffness of circuit. Furthermore, MEXP has to regularize~\cite{Chen12_TCAD} the 
singularity in power grid, i.e., nodes without grounded capacitance or inductance/voltage branches. Both constraints would drag 
the overall performance of MATEX for the power grid simulation. 

In this work, we tailor MATEX using \emph{rational Krylov subspace} for the power grid simulation with adaptive time stepping. 
The rational Krylov subspace uses $(\mbf{I}-\gamma \mbf{A})^{-1}$ as the basis instead of $\mbf{A}$ used in the conventional Krylov 
subspace. The rational basis limits the spectrum of a circuit so that the exponential of a matrix can be accurately approximated 
even under a large step size. Moreover, the rational avoids $\mbf{C}^{-1}$ and thus eliminates the regularization process. 
As a result, MATEX with rational Krylov subspace is free from the stiffness constraint and the regularization process. The 
simulation can purely enjoy benefits of the adaptivity and the accuracy of MATEX. Even though the rational Krylov subspace still needs 
to solve a linear system as the trapezoidal method does, MATEX can factorize the matrix only once and then constructs the rest of 
rational Krylov subspaces by backward/forward substitutions. Therefore, MATEX can utilize its capability of adaptivity to accelerate 
the simulation with the same kernel operations as the fixed step size strategy. Overall, our MATEX has the following advantages: 
\begin{itemize}
    \item enabling adaptive time stepping for the power grid simulation with only one lu factorization.
    \item eliminating the regularization process for the singular power grid design.  
    \item allowing scaling large step size without compromising the accuracy. 
\end{itemize}
The experimental results demonstrate the effectiveness of MATEX with adaptive step size. The power grid simulation for IBM benchmark 
suite~\cite{Nassif08_Power} and industrial power grid designs can be accelerated 6X and 15X on average compared to the trapezoidal method. 

The rest of paper is organized as follows. Section~\ref{sec:preliminary} presents the background of the power grid simulation and MATEX. 
Sections~\ref{sec:rational_krylov} and~\ref{sec:adaptive_step_control} show the theoretical foundation of the rational Krylov subspace 
and our adaptive step scheme for the power grid simulation, respectively.  Section~\ref{sec:exp_results} presents experimental results 
of the IBM benchmark suite~\cite{Nassif08_Power} and several industrial power grid designs. Finally, Section~\ref{sec:conclusion} concludes 
this paper. 
\end{comment}

