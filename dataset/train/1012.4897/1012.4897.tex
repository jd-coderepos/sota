\documentclass[copyright]{eptcs}
\providecommand{\event}{PAR-2010 Edinburgh, UK} \usepackage {bsymb,b2latex,amsmath,proof,listings, graphicx}
\graphicspath{{./Figures/}}   \newtheorem{theorem}{Theorem}[section]
\newtheorem{lemma}[theorem]{Lemma}
\newtheorem{proposition}[theorem]{Proposition}
\newtheorem{corollary}[theorem]{Corollary}
\newtheorem{definition}[theorem]{Definition}
\title{Rewriting and Well-Definedness within a Proof System\thanks{This research is carried out as part of DEPLOY (an European Commission FP7 Project Grant 214158). The first author is partially supported by the Algerian Ministry of Higher Education (MESRS).}}
\author{Issam Maamria and Michael Butler
\email{\{im06r,mjb\}@ecs.soton.ac.uk}
\institute{Electronics and Computer Science\\}
\institute{University of Southampton\\ Southampton, UK}
}
\def\titlerunning{Rewriting and Well-Definedness within a Proof System}
\def\authorrunning{Issam Maamria and Michael Butler}
\begin{document}
\maketitle
\begin{abstract}
Term rewriting has a significant presence in various areas, not least in automated theorem proving where it is used as a proof technique. Many theorem provers employ specialised proof tactics for rewriting. This results in an interleaving between deduction and computation (i.e., rewriting) steps. If the logic of reasoning supports partial functions, it is necessary that rewriting copes with potentially ill-defined terms. In this paper, we provide a basis for integrating rewriting with a deductive proof system that deals with well-definedness. The definitions and theorems presented in this paper are the theoretical foundations for an extensible rewriting-based prover that has been implemented for the set theoretical formalism Event-B.
\end{abstract}

\section{Introduction}
Term rewriting has an important presence in many areas including abstract data type specifications and automated reasoning. In this regard, many automated theorem provers employ rewriting as a proof technique where it may interleave with deduction. PVS~\cite{886667} and Isabelle/HOL~\cite{513715} are higher-order theorem provers that include specialised tactics for rewriting.
\par
The interleaving between rewriting steps and deduction steps poses several difficulties. The termination of rewriting becomes an issue of paramount importance. Many techniques, such as term orderings~\cite{280474}, have been explored to provide good practical solutions to termination problems. We argue that, in the presence of potentially ill-defined terms, rewriting has to be further constrained.
\par
Ill-defined terms arise in the presence of partial functions. They result from the application of functions to terms outside their domain. If ill-definedness is a concern, the adopted reasoning framework has to cope with it. Different approaches exist to reason in the presence of partial functions. Each of these approaches has its own specialised proof calculus. In~\cite{icfemMehta08}, it is shown that it is possible to reason about partiality without abandoning the well-understood domain of two-valued predicate logic. In that approach, the reasoning is achieved by extending the standard calculus with derived proof rules that preserve well-definedness across proofs. We argue that, in order to integrate rewriting as a proof step in such a calculus, it is necessary that rewriting preserves well-definedness.
\par
In this paper, we present a treatment of term rewriting where term well-definedness is an issue. Our treatment unifies the notions of well-definedness (WD) and rewriting, and provides a basis to integrate rewriting as a proof step within the proof system presented in~\cite{icfemMehta08}. Central to our contribution is the concept of WD-preserving rewriting where rewrite rules preserve well-definedness in the same direction in which they are applied. We establish the necessary conditions under which rewriting preserves well-definedness. We, finally, show how a rewrite step can be interleaved with deduction steps in a valid fashion.

\subsection{Practical Setting}\label{pracSetting}
Event-B~\cite{1365974} is a formalism for discrete systems modelling based on Action Systems~\cite{91938}. It can be used to model and reason about complex systems such as concurrent and reactive systems. The semantics of a model developed in Event-B is given by means of its proof obligations. These obligations have to be discharged to show consistency of the model with respect to some behavioural semantics.
\par
Modelling in Event-B is conducted by defining contexts and machines. Contexts describe static properties of a model by specifying carrier sets and constants. Machines, as their name suggests, define the dynamics of a model by means of variables (state) and events (transitions). Variables are constrained by invariants. A machine can be refined by another machine, and can see (import) contexts. Proof obligations arise to verify the consistency of a model. For instance, there are proof obligations to establish the refinement relationship between two machines, and to establish invariant preservation by the events (transitions). The logic used in Event-B is typed set theory built on first-order predicate logic, and allows the definition of partial functions. As such, it is necessary that the used proof system handles ill-definedness. Indeed, the proof calculus outlined in~\cite{icfemMehta08} is the one used to reason in Event-B. Figure \ref{simpleModel} illustrates a simple Event-B model for a door entry system.
\begin{figure}[tbp]
\includegraphics[width=4.5in, height=5in]{m0.png}
\caption{A Simple Model for A Door Entry System}\label{simpleModel}
\end{figure}
\par
The Rodin platform~\cite{Abrial-etal06} is an open extensible tool for Event-B based on Eclipse\footnote{http://www.eclipse.org/}. It offers support
for specification and proof, and it can be easily extended with other useful tools e.g., there is a plug-in for model checking called Pro-B~\cite{LeuschelButler:FME03}.
\subsection{Motivation}
The Rodin platform has a proving infrastructure which is extensible with new proof rules. External provers can also be used; Atelier-B~\cite{abrial_03_clickn} provers ML and PP have been incorporated into Rodin. Adding new proof rules requires the use of the Java programming language, knowledge of Eclipse as well as an understanding of the internal architecture of Rodin. A complication of such approach is that newly implemented rules could compromise the soundness of the prover. This work has been carried out as part of an effort to address this limitation of Rodin from the viewpoint of prover extensibility. This paper discusses some theoretical results in the context of rewriting and well-definedness. The ideas presented in this paper have resulted in providing proof support for the set theoretical formalism Event-B~\cite{1365974}. An extensible rewriting-based prover~\cite{issam1984} has been implemented and integrated into Rodin. 
\par
\textit{Outline.} In Section \ref{pre}, we recall some preliminary concepts of term rewriting systems. Section \ref{wdpres} describes the necessary conditions under which rewriting preserves well-definedness. Section \ref{pstep} shows how a WD-preserving rewrite rule can be used in proofs. The application of the previous ideas in the context of Event-B~\cite{1365974} is shown in Section \ref{eapp}. We conclude in Section \ref{future} by stating what we have achieved and its impact on the Event-B toolset~\cite{3056425}.
\subsection{Related Work}
The interleaving between deduction and rewriting steps has gathered much interest given its importance to automated reasoning. In this work, we identify the necessary conditions under which rewriting can interleave with deduction in the proof calculus defined in~\cite{icfemMehta08}. In other works, this interleaving is studied from different perspectives.
\par
Theorem proving modulo is an approach that removes computational steps from proofs by reasoning modulo a congruence on propositions~\cite{949773}. The advantage of this technique is that it separates computation steps (i.e., rewriting) from deduction steps in a clean way. In~\cite{949773}, a proof-theoretic account of the combination between computations and deductions is presented in the shape of a sequent calculus modulo. The congruence on propositions, on the other hand, is defined by rewrite rules and equational axioms.
\par
The combination of rewriting and deduction makes properties of rewrite systems of practical interest. Termination and confluence properties of term rewriting systems are important, and have been studied extensively~\cite{280474,303448}. When rewriting is interleaved with deduction, it is critical that computation steps terminate. Term orderings, in which any term that is syntactically simpler than another is smaller than the other, provides a practical technique to assess the termination of rewrite systems.
\par
In our work, we aim to unify the notions of well-definedness and rewrite systems. Our objective is to characterise the interaction between deduction and rewriting when well-definedness is taken into consideration. This is achieved by identifying the necessary conditions under which computations can interleave with the deduction steps (i.e., proof rules) in~\cite{icfemMehta08}.

\section{Preliminaries}\label{pre}
In this section, we lay the groundwork for the rest of the paper. We briefly introduce the proof calculus defined in~\cite{icfemMehta08}. We also shed some light on basic concepts of term rewriting systems. For the rest of this paper, we use the language signature  defined by a set  of variable symbols, a set  of function symbols and a set  of predicate symbols. In the next two definitions, we introduce the syntax of the first-order predicate calculus with equality that will be used in the subsequent sections.

\begin{definition}[Term]
, the set of -terms is inductively defined by:\\
~~ each variable of  is a term;\\
~~ if ,  and each of  is a term, then  is a term.
\end{definition}
\begin{definition}[Formula]
, the set of -formulas is inductively defined by:\\
~~  is a formula;\\
~~  is a formula provided ,  and each of  is a term;\\
~~  is a formula provided  and  are terms;\\
~~  is a formula if  and  are formulas;\\
~~  is a formula if  is a formula;\\
~~  is a formula if  and  is a formula.
\end{definition}
Note that other logical operators (e.g., ) can be defined (as in~\cite{farhad}) by means of the operators in the previous definition. For the rest of the paper, we assume a syntactic operator  such that  is the set of variables occurring free in .

\subsection{The Well-Definedness Operator}
The well-definedness operator '' encodes what is meant by well-definedness.  is a syntactic operator that maps terms and formulae to their well-definedness predicates. We interpret the formula  as being valid if and only if  is well-defined.
For a detailed treatment of the  operator, we refer to~\cite{723108}.
\par
The well-definedness (WD) of terms is defined recursively as follows:

where  effectively defines the domain of the function . For this study, we assume that predicate symbols are total. As a result, ill-definedness can only be introduced by terms. Therefore, we have the following:

For the well-definedness of other formulae, we use the following expansions~\cite{723108}:

The well-definedness of formulae built using derived logical operators can be straightforwardly derived, see~\cite{farhad}. An important property of well-definedness conditions is that they are themselves well-defined~\cite{icfemMehta08}; i.e.,


\subsection{The WD-preserving Sequent Calculus}
We assume the signature  is equipped with a proof theory in the shape of a WD-preserving first-order sequent calculus similar to the one appearing in~\cite{icfemMehta08}. A judgement in the aforementioned calculus is called a well-defined sequent, and is of the form  defined as follows:

That is, the well-definedness of  and  is assumed when proving . Generally speaking, when proving a sequent , the approach suggests proving its validity as well as its well-definedness:

where  is defined as  such that  are the free variables of  and .
\par
A proof rule is said to preserve well-definedness (WD) iff its consequent and antecedents only contain well-defined sequents (i.e.,  sequents). Figure \ref{fopced} introduces the theory \textit{\textbf{FoPCe}} (a collection of WD-preserving inference rules) as developed in~\cite{icfemMehta08}. Note that we use  to denote the non-freeness condition of  in .  We also use  to denote the syntactic replacement of all free occurrences of the variable  in  by the term . The boxed sequents in Figure \ref{fopced} correspond to the additional sequents that has to be discharged compared to the classical version of the rule in order to preserve well-definedness.
\begin{figure}[h]
\fbox{
\begin{minipage}{15.5cm}
\begin{center}

\end{center}
\begin{center}

\end{center}
\begin{center}

\end{center}
\begin{center}

\end{center}
\begin{center}

\end{center}
\begin{center}

\end{center}
\end{minipage}
}
\caption{Inference Rules of \textbf{\textit{FoPCe}}}
\label{fopced}
\end{figure}
\newline Proof rules for derived logical operator (i.e., , ,  and ) can be derived directly from the rules of \textbf{\textit{FoPCe}}. The following two proof rules can be derived with a detour through  sequents (classical reasoning):

and

\par
In Section \ref{wdpres} and \ref{pstep}, we show how rewriting can be interleaved with the inference rules of \textbf{\textit{FoPCe}}. For the rest of the paper, we assume that the reader is familiar with the basic notions of rewriting as found, for instance, in~\cite{280474}. We define the domain and range of a substitution  (both finite), denoted  and  respectively, as follows:

Note that the application of a substitution  to a term  \textit{simultaneously} replaces occurrences of variables by their respective -images. For the rest of this work, we restrict substitutions according to the following definition:
\begin{definition}[Non-conflicting Substitution]
A substitution  is said to be non-conflicting iff

\end{definition}
Intuitively, a non-conflicting substitution can be simulated by a syntactic replacement as follows:

such that  are the free variables in , and  for all  and  where  and . In this case, we have the following important property:

which can proved by induction on the structure of terms.
\par
One of the main concepts of term rewriting is that of positions in terms and formulae where  denotes the root position. Positions within a formula (or a term) describe paths to its subterms and subformulae. When  is a position in a formula , we write  for the term or formula at position  in formula . We write  for the formula that results from replacing  with  in . 
\section{WD-Preserving Rewriting}\label{wdpres}
In this section, we show how rewriting \textit{preserves} equality of terms, validity of formulae and well-definedness of both terms and formulae. The next definitions describe what is meant by a conditional rewrite rule.

\begin{definition}[Conditional Identity]
A -conditional identity (simply conditional identity) is a triplet . In this case,  is called the left hand side,  the right hand side, and  the condition of the identity.
\end{definition}

\begin{definition}[Valid Conditional Identity]\label{validCI}
A conditional identity  is valid iff the following sequent is provable

\end{definition}

A conditional identity can be turned into a rewrite rule if it satisfies the syntactic restrictions presented in the following definition:

\begin{definition}[Conditional Term Rewrite Rule]
A conditional term rewrite rule is a conditional identity  such that:
\begin{enumerate}
\item  is not a variable,
\item ,
\item .
\end{enumerate}
In this case, we use the notation  instead of .
\end{definition}

In the derivations of Figure \ref{gRewDeriv} and Figure \ref{hRewDeriv}, we single out the necessary conditions under which rewriting can be performed. Figure \ref{hRewDeriv} concerns the rewriting of an hypothesis that has an occurrence of a rewrite rule left hand side . Note the presence of the condition . We assume that the free variables of  also occur free in ; this ensures that  denotes the same substitution in both  and . Figure \ref{gRewDeriv}, on the other hand, concerns the rewriting of a goal which has an occurrence of a rewrite rule left hand side .
\begin{figure}[h]
\centering
\fbox{
\begin{footnotesize}

\end{footnotesize}
}
\caption{Hypothesis Rewriting}
\label{hRewDeriv}
\end{figure}

\begin{figure}[h]
\centering
\fbox{
\begin{footnotesize}

\end{footnotesize}
}
\caption{Goal Rewriting}
\label{gRewDeriv}
\end{figure}

The boxed sequents correspond to the conditions under which a formula (an hypothesis or the goal) can be rewritten. In summary, a conditional term rewrite rule  can be applied to a formula  (the goal or one of the hypothesises) iff the following sequents are provable:


In the rest of this section, we examine the sufficient restrictions on conditional term rewrite rules to ensure that sequents \ref{exCon2} and \ref{exCon1} are provable for a given formula , a position  and a substitution .

\begin{definition}\label{wdRewPres}
A conditional rewrite rule  is said to be WD-preserving iff the following sequent is provable:

\end{definition}

We turn our attention to rewrite rule application. Consider applying rule  to formulae  where  is a term as is . The left hand side  is matched against  by finding a substitution  such that  (one-way matching). Provided  holds,   can be rewritten to . 
\par
The following theorem states that the application of a valid and well-definedness preserving conditional term rewrite rule preserves equality (\ref{termVal}) and well-definedness (\ref{termWD}) of terms.
\begin{theorem}\label{thm1}
Let  be a conditional term rewrite rule,  be a term,  be a position within , and  be a non-conflicting substitution such that

If  is valid and WD-preserving, then the following two sequents are provable:

\end{theorem}
\noindent \textbf{\textit{Proof}. } The following lemma is needed to prove Theorem \ref{thm1}:
\begin{lemma}\label{lemma1}
Let  be a conditional term rewrite rule, and  be a non-conflicting substitution such that

\begin{enumerate}
\item If  is valid, then the following sequent is provable:

\item If  is WD-preserving, then the following sequents are provable:

\end{enumerate}
\end{lemma}
\noindent \textbf{\textit{Proof}. }
We observe that the sequent

( are the free variables of ) is provable if the sequent

is also provable. We also observe that the sequent

is provable. Since the substitution  can be simulated as a sequence of syntactic replacements, instantiating  in (\ref{lemma1seq1}) with the appropriate terms in  is the main idea of the proof of the first claim. The proof of the second claim follows a similar approach.

\begin{enumerate}
\item \textit{Proof of sequent} (\ref{termVal}): We proceed by induction on the structure of the term .
\begin{enumerate}
\item \textbf{Base Case:}  is a variable, . In this case (\ref{termVal}) becomes

since variables have only one position ( the root position). This simplifies to 

which is a provable sequent according to Lemma \ref{lemma1}.
\item \textbf{Inductive Case:}  is a function, . We distinguish the cases  and  for  and some position .
\begin{enumerate}
\item Case : this case is similar to the base case.
\item Case : we assume the following inductive hypothesis (in this case a provable sequent)

and we show that
\begin{small}

\end{small}
is a provable sequent where .
\end{enumerate}
\end{enumerate}
\item \textit{Proof of sequent} (\ref{termWD}): We proceed by induction on the structure of the term .
\begin{enumerate}
\item \textbf{Base Case:}  is a variable, . In this case (\ref{termWD}) becomes

since variables only have the root position . This simplifies to 

which is a provable sequent according to Lemma \ref{lemma1}.
\item \textbf{Inductive Case:}  is a function, . We distinguish the cases  and  for  and some position .
\begin{enumerate}
\item Case : this case is similar to the base case.
\item Case : We assume the following inductive hypothesis

and we show that
\begin{small}

\end{small}
is a provable sequent where .
\end{enumerate}
\end{enumerate}
\end{enumerate}

The following theorem asserts that Definition \ref{validCI} and \ref{wdRewPres} are adequate for a conditional term rewrite rule to preserve validity and well-definedness when applied to a formula.
\begin{theorem}\label{thm2}
Let  be a conditional term rewrite rule,  be a formula,  be a position within  such that  is a term, and  be a non-conflicting substitution such that

If  is valid and WD-preserving, then the following two sequents are provable:

\end{theorem}
\noindent \textbf{\textit{Proof}. }
\begin{enumerate}
\item \textit{Proof of sequent} (\ref{formulaVal}): We proceed by induction on the structure of the formula . We show a sketch of the proof, and only cover three interesting cases.
\begin{enumerate}
\item \textbf{Base Case:}  is of the shape  such that  and  are terms. In this case, position  can only be of the form  for some position  and  since the root position is of a formula.
Therefore, (\ref{formulaVal}) becomes
\begin{small}

\end{small}
where  for some position  and . This can be rewritten to
\begin{small}

\end{small}
This amounts to proving the following two sequents:
\begin{small}

\end{small}
Using Theorem \ref{thm1}, both sequents can be shown to be provable.
\item \textbf{Inductive Case:}  is of the shape  such that  and  are formulae. In this case, (\ref{formulaVal}) becomes
\begin{small}

\end{small}
Position  can only be of the form  or  for some position . We distinguish the two cases:
\begin{enumerate}
\item : In this case, sequent (\ref{seq3.5}) becomes
\begin{small}

\end{small}
To proceed, we assume the following inductive hypothesis
\begin{small}

\end{small}
and we show that sequent (\ref{seq3.6}) is provable.
\item : analogous to the previous case.
\end{enumerate}
\item \textbf{Inductive Case:}  is of the shape  such that  is a formula. In this case, (\ref{formulaVal}) becomes
\begin{small}

\end{small}
Position  can only be of the form  for some position . Sequent (\ref{seq3.8}) simplifies to
\begin{small}

\end{small}
To proceed, we assume that the following sequent is provable:
\begin{small}

\end{small}
and we show that sequent (\ref{seq3.9}) is provable.
\end{enumerate}
\item \textit{Proof of sequent} (\ref{formulaWD}):
is similar to the proof of sequent (\ref{formulaVal}). We only show one inductive case.
\begin{enumerate}
\item \textbf{Inductive Case:}  is of the shape  such that  and  are formulae. In this case, (\ref{formulaWD}) becomes
\begin{small}

\end{small}
Position  can only be of the form  or  for some position . We distinguish the two cases:
\begin{enumerate}
\item : In this case, sequent (\ref{seq3.11}) becomes
\begin{small}

\end{small}
To proceed, we assume that the following sequent is provable:
\begin{small}

\end{small}
and we show that sequent (\ref{seq3.12}) is provable.
\item : analogous to the previous case.
\end{enumerate}
\end{enumerate}
\end{enumerate}

\textbf{Summary.} In this section, we have defined the criteria for the validity and well-definedness preservation of term rewrite rules when rewriting interleaves with the rule of the proof system developed in~\cite{icfemMehta08}. In the next section, we show how rewriting can be systematically used as a proof step.
\section{Rewriting as a Proof Step}\label{pstep}
Rewriting can be used in proofs alongside the WD-preserving sequent calculus. Conditional term rewrite rules which have the same left hand side are grouped together. For this purpose, we use a more convenient notation. Given a valid and WD-preserving (grouped) conditional term rewrite rule

we can add the following proof step to our calculus

under the proviso that all free variables of  (for all  such that ) occur free in . This proof step allows the hypothesis  to be rewritten to several cases according to the rewrite rule.
Under the proviso that all free variables of  (for all  such that ) occur free in , the following proof step can be added for goal rewriting

Proof steps (\ref{hyp_app}) and (\ref{goal_app}) can be derived using the cut rule, followed by a disjunction elimination after which rewriting can be applied. We now examine some special cases that can be used to facilitate proofs.
\subsection{Unconditional Term Rewrite Rules}
A term rewrite rule  is called \textit{unconditional} iff . In this case, steps (\ref{hyp_app}) and (\ref{goal_app}) can be simplified as follows:


\subsection{Case-complete Grouped Term Rewrite Rules}
A grouped term rewrite rule 

is called \textit{case-complete} iff the following sequent is provable:

In this case, steps (\ref{hyp_app}) and (\ref{goal_app}) can be simplified as follows:


\subsection{Top-level Occurrence}

\begin{definition}[Top-Level Occurrence]
Let  be a term,  be a formula,  be a position within . We say that  has a top-level occurrence in  if  is either of the form 
\begin{enumerate}
\item  where  and ,...,  are terms, or; 
\item  where  and  are terms.  
\end{enumerate}
If  has a top-level occurrence in , then it also has a top-level occurrence in .
\end{definition}
We have the following interesting property:
\begin{proposition}\label{toplevel_prop}
If the term  has a top-level occurrence in formula , then the following holds

\end{proposition}
If we further constrain grouped conditional term rewrite rules such that we have

Proposition \ref{toplevel_prop} can be used to simplify proofs. Let  be a formula such that  occurs at the top-level. Since the grouped term rewrite rule is valid and WD-preserving, and using the previous proposition, we have the following

and, consequently:

under the proviso that all free variables of  (for all  such that ) occur free in . In this particular case, the sequents

in (\ref{hyp_app}) and (\ref{goal_app}) respectively, are guaranteed to be provable. As such, they could be removed from the list of sub-goals that the modeller sees.
\section{Applications to Event-B}\label{eapp}
As mentioned in \ref{pracSetting}, Event-B modelling is carried out using two constructs: contexts and machines. A third construct, called \textit{theory}, has been implemented to bring a degree of meta-reasoning to the Rodin platform~\cite{3056425}. The theory construct has the following shape:
\begin{figure}[ht]
\begin{center}
\fbox{
\begin{minipage}{7cm}
\vspace{0.2in}
\textbf{Theory}~~theory\_name\\\\
\textbf{Sets}~~\\\\
\textbf{Metavariables}~~\\\\
\textbf{Rewrite Rules}~~\\\\
\textbf{End}\\
\end{minipage}
}
\end{center}
\caption{The Theory Construct}\label{thy}
\end{figure}
\begin{enumerate}
\item \textit{Sets}. A theory can define a number of given sets which define the types on which the theory is parametrised.
\item \textit{Metavariables}. A theory can define a number of metavariables that can be used to specify rewrite rules. Each metavariable is associated with a type; this can be constructed using the given sets of the theory as well as the built-in types (e.g., ) using type constructors. For example, if a given set  is defined within a theory, then  can be used as a type for a metavariable.
\item \textit{Rewrite Rules}. Rewrite rules are one-directional equations that can be used to rewrite formulae to equivalent forms. As part of specifying a rewrite rule, the \textit{theory developer} decides whether the rule can be applied automatically without user intervention or interactively following a user request.
\end{enumerate}
The theory construct can be extended to enable the specification of inference rules. In brief, it facilitates the following:
\begin{itemize}
\item specification of proof rules within the same platform providing a degree of meta-reasoning within Rodin,
\item validation of specified proof rules to ensure that the soundness of the prover is not compromised.
\end{itemize}
The validation of rewrite rules is achieved by means of proof obligations. Definition \ref{validCI} and \ref{wdRewPres} defined the criteria for validity and WD-preservation of rewrite rules.
\par
The theory construct has been developed as part of a \textit{rule-based prover}~\cite{issam1984} which, in brief, offers the following capabilities:
\begin{enumerate}
\item Users can develop theories in the same way as contexts and machines. At the moment, theory development includes specification of rewrite rules including definition of sets and metavariables. Metavariables must be defined with their types which can be constructed from the theory sets and any built-in types (e.g., ) using type constructors (e.g., ).
\item Users can validate rewrite rules through generated proof obligations. The proof obligations generated for rules are to establish soundness, well-definedness preservation and case-completeness.
\item Users can deploy theories to a specific directory where they become available to the interactive and automatic provers of Rodin. Theory deployment adds soundness information to all deployed rules.
\item Users can use rewrite rules defined within the deployed theories as a part of the proving activity. A pattern matching mechanism is implemented to calculate applicable rewrite rules to any given sequent. 
\end{enumerate}
\vspace{2mm}
\textit{Examples.} The following two rules are valid and WD-preserving:

where  and  are integers,  denotes an integer range, and  denotes the cardinality operator. The following rules are not WD-preserving:

where  is an integer,  a relation, , and  are of arbitrary types. Moreover, `' denotes relational override. Rule \ref{unsRelO} is not WD-preserving since there could be a case where  is a function but  is not. For instance, consider , then 
. In this case,  is well-defined, but  is not.
\section{Future Work \& Conclusions}\label{future}
In this paper, we provided a treatment of well-definedness and rewriting. We singled out the necessary conditions under which rewriting preserves well-definedness. These conditions are necessary for the valid interleaving between rewriting steps and deduction in the WD-preserving proof calculus presented in~\cite{icfemMehta08}. In our study, we used the language signature  whereby terms are only defined using other terms. In general, however, terms can also be constructed using formulae e.g., set comprehension . This changes the well-definedness conditions of terms, and it is interesting to establish whether the conditions outlined in Definition \ref{validCI} and \ref{wdRewPres} are indeed sufficient.
\par
We have presented a study unifying the notions of term rewriting and well-definedness in the context of the interleaving between deduction and rewriting. The results of this paper provided the theoretical foundations of an extensible rewriting-based prover (also called rule-based prover) that has been implemented for Event-B.


\bibliographystyle{plain}
\bibliography{par}
\end{document}
