\documentclass{stacs_proc}
\usepackage{graphicx}
\usepackage{pstricks}
\usepackage{pst-node}
\usepackage{pst-coil}
\usepackage{amsmath,amsthm}
\usepackage{amsmath}
\usepackage{amssymb}
\usepackage{amsfonts}
\usepackage{epsfig}
\usepackage{verbatim}

\theoremstyle{plain}\newtheorem{satz}[thm]{Satz}
\def\eg{{\em e.g.}}
\def\cf{{\em cf.}}

\psset{unit=1pt}

\renewcommand{\labelenumi}{\arabic{enumi}.}
\renewcommand{\labelenumii}{\arabic{enumi}.\arabic{enumii}.}
\renewcommand{\labelenumiii}{\arabic{enumi}.\arabic{enumii}.\arabic{enumiii}.}
\renewcommand{\labelenumiv}{\arabic{enumi}.\arabic{enumii}.\arabic{enumiii}.\arabic{enumiv}.}
\newlength{\alginputwidth}
\newlength{\algboxwidth}
\newcommand{\alginput}[1]{\makebox[1.5cm][l]{ {\sc Input:}} \parbox[t]{\alginputwidth}{{\it #1}}}
\newcommand{\algoutput}[1]{\makebox[1.5cm][l]{ {\sc Output:}} \parbox[t]{\alginputwidth}{{\it #1}}}
\newcommand{\algtitle}[1]{\underline{Algorithm \ {\bf #1}} \vspace*{1mm}\\}

\newsavebox{\algbox}
\newsavebox{\captionbox}
\newenvironment{algorthm}[2]{
	\setlength{\algboxwidth}{\columnwidth}
	\addtolength{\algboxwidth}{-\columnsep}
	\addtolength{\algboxwidth}{-1mm}
	\setlength{\alginputwidth}{\algboxwidth}
	\addtolength{\alginputwidth}{-1.7cm}
	\begin{figure}[tb]
	    \vspace*{2mm}
	    \centering
	    \begin{lrbox}{\captionbox}
		\begin{minipage}[b]{\algboxwidth}
		    \centering
		    \caption{#1}
		    \label{#2}
		\end{minipage}
	    \end{lrbox}
	    \begin{lrbox}{\algbox}
		\begin{minipage}[b]{\algboxwidth}
		    \footnotesize
		    \vspace*{2mm}
    } {
		    \vspace*{0.2mm}
	       \end{minipage}
	    \end{lrbox}
	    \fbox{\usebox{\algbox}\hspace*{1mm}}
	    \usebox{\captionbox}
	    \vspace*{-4mm}
	\end{figure}
    }
\newsavebox{\algcodebox}
\newenvironment{codeblock}{
	\begin{enumerate}
	    \setlength{\itemsep}{2pt}
	    \setlength{\parsep}{0pt}
	    \setlength{\topsep}{0pt}
	    \setlength{\parskip}{0pt}
	    \setlength{\partopsep}{0pt}
    } {\end{enumerate}}
\newcommand{\step}{\item}


\begin{document}

\title{On Geometric Spanners of Euclidean and Unit Disk Graphs}
\author[]{I. Kanj}{Iyad A. Kanj}
\address[]{DePaul University, Chicago, IL 60604, USA.}
\thanks{The work of the first author was supported in part by a DePaul
University Competitive Research Grant.}

\author[]{L. Perkovi\'{c}}{Ljubomir Perkovi\'{c}}
\email{{ikanj,lperkovic}@cs.depaul.edu}

\keywords{Geometric spanner, euclidean graph, unit disk graph, wireless ad-hoc
networks}
\subjclass{ C.1.4, F.2.2, G.2.2}

\begin{abstract}
  \noindent We consider the problem of constructing bounded-degree planar
geometric spanners of Euclidean and unit-disk graphs. It is well known that
the Delaunay subgraph is a planar geometric spanner with stretch
factor ; however, its degree may not be
bounded. Our first result is a very simple linear time
algorithm for constructing a subgraph of the Delaunay graph with
stretch factor  and degree
bounded by , for any integer parameter . This result
immediately implies an algorithm for constructing a planar geometric
spanner of a Euclidean graph with stretch factor  and degree bounded by , for any integer parameter . Moreover, the resulting spanner contains a Euclidean Minimum Spanning Tree
(EMST) as a subgraph. Our second contribution lies in developing the structural
results necessary to transfer our analysis and algorithm from
Euclidean graphs to unit disk graphs, the usual model for wireless ad-hoc
networks. We obtain a very simple distributed, {\em strictly-localized}
algorithm that, given a unit disk graph embedded in the plane, constructs a 
geometric spanner with the above stretch factor and degree bound, and also
containing an EMST as a subgraph.  The obtained results
dramatically improve the previous results in all aspects, as shown
in the paper. 
\end{abstract}

\maketitle

\stacsheading{2008}{409-420}{Bordeaux}
\firstpageno{409}

\section*{Introduction} \label{intro}
Given a set of points  in the plane, the Euclidean graph  on
 is defined to be the complete graph whose vertex-set is .
Each edge  connecting points  and  is assumed to be
embedded in the plane as the straight line segment ; we define
its cost to be the Euclidean distance . We
define the unit disk graph  to be the subgraph of  consisting
of all edges  with .

Let  be a subgraph of . The cost of a simple path  in  is  Among
all paths between  and  in , a path with the smallest cost
is defined to be a {\em smallest cost path} and we denote its cost
as . A spanning subgraph  of  is said to be a {\em geometric
spanner} of  if there is a constant  such that for every
two points  we have: .
The constant  is called the {\em stretch factor} of  (with
respect to the underlying graph ).

The problem of constructing geometric spanners of Euclidean graphs has
recently received a lot of attention due to its applications in
computational geometry, wireless computing, and computer graphics
(see, for example, the recent book~\cite{spannerbook} for a survey on
geometric spanners and their applications in networks). Dobkin et
al.~\cite{dobkin} showed that the Delaunay graph is a planar
geometric spanner of the Euclidean graph with stretch factor
. This ratio was improved by
Keil et al~\cite{keil} to , which currently stands as the best upper bound on the stretch
factor of the Delaunay graph. Many researchers believe, however, that
the lower bound of  shown in~\cite{chew} is also an upper
bound on the stretch factor of the Delaunay graph. While
Delaunay graphs are good planar geometric spanners of Euclidean
graphs, they may have unbounded degree. 

Other geometric (sparse) spanners were also proposed in the literature
including the Yao graphs~\cite{yao}, the -graphs~\cite{keil},
and many others (see~\cite{spannerbook}). However, most of these proposed
spanners either do not guarantee planarity, or do not guarantee bounded
degree.

Bose et al.~\cite{boseesa,bosealgorithmica} were the first to show
how to extract a subgraph of the Delaunay graph that is a planar
geometric spanner of the Euclidean graph with
stretch factor  and degree bounded by . In the
context of unit disk graphs, Li et
al.~\cite{iitunbounded1,iitunbounded} gave a distributed algorithm
that constructs a planar geometric spanner of a unit disk graph with
stretch factor ; however, the spanner constructed can have
unbounded degree. Wang and Li~\cite{iitbounded1,iitbounded} then
showed how to construct a bounded-degree planar spanner of a unit
disk graph with stretch factor  and degree bounded by , where  is a parameter. Very
recently, Bose et. al~\cite{bose1} improved the earlier result in
~\cite{boseesa,bosealgorithmica} and showed how to construct a
subgraph of the Delaunay graph that is a geometric spanner of the
Euclidean graph with stretch factor:  if  and  when , and whose degree is bounded by . Bose et al. then applied their construction to obtain
a planar geometric spanner of a unit disk graph with stretch factor
 and degree
bounded by , for any . This
was the best bound on the stretch factor and the degree.

We have two new results in this paper. We develop structural results about
Delaunay graphs that allow us to present a very simple linear-time algorithm
that, given a Delaunay graph, constructs a subgraph of the Delaunay graph
with stretch factor  (with respect
to the Delaunay graph) and degree at most , for any integer
parameter . This result immediately implies an 
algorithm for constructing a planar geometric spanner of a Euclidean graph
with stretch factor of  and degree at most , for any integer parameter  ( is the number of vertices in the graph). We then translate our
work to unit disk graphs and present our second result: a
very simple and {\em strictly-localized} distributed algorithm that, given a
unit-disk graph embedded in the plane, constructs a planar geometric spanner
of the unit disk graph with stretch factor  and degree bounded by , for
any integer parameter . This efficient distributed algorithm
exchanges no more than  messages in total, and runs in  local time at a node of degree . We show that both
spanners include a Euclidean Minimum Spanning Tree as a subgraph.

Both algorithms significantly improve previous results (described above)
in terms of the stretch factor and the degree bound. To show this, we compare
our results with previous results in more detail. For a degree bound ,
our result on Euclidean graphs imply a bound of at most  on the stretch
factor. As the degree bound  approaches , our bound on the
stretch factor approaches . The very recent
results of Bose et al.~\cite{bose1} achieve a lowest degree bound of
, and that corresponds to a bound on the stretch factor of at
least . If Bose et al.~\cite{bose1} allow the degree bound to be
arbitrarily large (i.e., to approach ), their bound on the
stretch factor approaches . Our stretch factor
and degree bounds for unit disk graphs are the same as our results
for Euclidean graphs. The smallest
degree bound derived by Bose et al.~\cite{bose1} is 20, and that
corresponds to a stretch factor of at least 6.19. If Bose et
al.~\cite{bose1} allow the degree bound to be arbitrarily large,
then their bound on the stretch factor approaches . On the other hand, the smallest degree bound derived in
Wang et al.~\cite{iitbounded1,iitbounded} is 25, and that
corresponds to a bound of 6.19 on the stretch factor. If Wang et
al.~\cite{iitbounded1,iitbounded} allow the degree bound to be
arbitrarily large, then their bound on the stretch factor approaches
. Therefore, even the worst bound of
at most 3.54 on the stretch factor corresponding to our lowest bound
on the degree , beats the best bound on the stretch factor of
at least 3.75 corresponding to arbitrarily large degree in both Bose
et al.~\cite{bose1} and Wang et al.~\cite{iitbounded1,iitbounded}!


\section{Definitions and Background}
\label{background}

We start with the following well known observation:
\begin{observation} \label{optimal} A subgraph  of graph  has
stretch factor  if and only if for every edge : the
length of a shortest path in  from  to  is at most .
\end{observation}
For three non-collinear points , ,  in the plane we denote
by  the circumscribed circle of triangle
. A {\em Delaunay triangulation} of a set of points  in
the plane is a triangulation of  in which the circumscribed circle of every
triangle contains no point of  in its interior. It is well known
that if the points in  are {\em in general position} (i.e., no
four points in  are cocircular) then the Delaunay triangulation
of  is unique~\cite{book}. In this paper---as in most papers in
the literature---we shall assume that the points in  are in
general position; otherwise, the input can be slightly perturbed so
that this condition is satisfied. The {\em Delaunay graph} of  is
defined as the plane graph whose point-set is  and whose edges
are the edges of the Delaunay triangulation of . An alternative definition
that we end up using is:
\begin{definition}
\label{DelaunayDef}
An edge  is in the Delaunay graph of  if and only if there exists a
circle through points  and  whose interior contains no point in .
\end{definition}
It is well known
that the Delaunay graph of a set of points  is a spanning
subgraph of the Euclidean graph defined on  (i.e., the complete
graph on point-set ) whose stretch factor is bounded by ~\cite{keil}.

Given integer parameter , the {\em Yao subgraph}~\cite{yao}
of a plane graph  is constructed by performing the
following {\em Yao step} at every point  of : place  equally-separated
rays out of  (arbitrarily defined), thus creating  closed cones of
size  each, and choose the shortest edge in  out of  (if
any) in each cone. The Yao subraph consists of edges in  chosen by
{\em either} endpoint. Note that the degree of a point in the Yao subgraph
of  may be unbounded.

Two edges ,  incident on a point  in a graph  are said
to be {\em consecutive} if one of the angular sectors determined by
 and  contains no neighbors of .

\section{Bounded Degree Spanners of Delaunay Graphs}
\label{euclidean} 

Let  be a set of points in the plane and let 
be the complete, Euclidean graph defined on point-set . Let  be the
Delaunay graph of . This section is devoted to proving the following
theorem:

\begin{theorem}
\label{maintheorem} For every integer , there exists a
subgraph  of  such that  has maximum degree  and
stretch factor .
\end{theorem}

A linear time algorithm that computes  from  is the key
component of our proof. This very simple algorithm essentially
performs a {\em modified Yao step} (see
Section~\ref{modifiedyaoalgo}) and selects up to  edges out of
every point of .  is simply the spanning subgraph of  consisting
of edges chosen by {\em both} endpoints.

In order to describe the modified Yao step, we must first develop a
better understanding of the structure of the Delaunay graph . Let
 and  be edges incident on point  in  such that
 and  is the shortest edge within the
angular sector . We will show how the above theorem easily
follows if, for every such pair of edges  and :
\begin{enumerate}

\item we show that there exists a path  from  to  in  of length ,
such that:\\    , and

\item we modify the standard Yao step to include the edges of this
path in , in addition to including the edges picked by the
standard Yao step but without increasing the number of edges chosen at each
point beyond .
\end{enumerate}
This will ensure that: for any edge  that is not included in 
by the modified Yao step, there is a path from  to  in , whose
edges are all included in  by the modified Yao step, and whose cost is
at most . In the lemma below, we prove
the existence of this path and show some properties satisfied by edges of this
path; we will then modify the standard Yao step to include edges satisfying
these properties.
\begin{lemma}
\label{canpath} Let  be an integer, and let  and 
be edges in  such that  and  is the
shortest edge in the angular sector . There exists a
path ~ in  such that:

\begin{itemize}
\item[(i)] .
\item[(ii)] There is no edge in  between any pair  and  lying in the closed region delimited by
,  and the edges of , for any  and 
satisfying .
\item[(iii)] , for .
\item[(iv)] .
\end{itemize}
\end{lemma}
We break down the proof of the above lemma into two cases: when
 contains no point of  in its interior, and when there
 are points of  inside . We define some additional notation
and terminology first. We define the circle  with center
, and set . Note that . We will use  to denote the arc of 
determined by points  and  and facing . We will make use
of the following easily verified Delaunay graph property:
\begin{proposition}
\label{Euclidproperty} If  and  are edges of  then the region inside 
subtended by chord  and away from  and the region inside  subtended
by chord  and away from  contain no points.
\end{proposition}

\subsection{The Outward Path}
\label{outpath} We consider first the case when no points of  are
inside . Since both  and  are edges in  and by
Proposition~\ref{Euclidproperty}, the region of  subtended by chord 
closer to  has no points of  in its interior. Keil and Gutwin~\cite{keil}
showed that, in this case, there exists a path between  and  in 
inside the region of  subtended by chord  away from , whose
length is bounded by the length of 
(see Lemma~1 in~\cite{keil}). We find it convenient to use a recursive
definition of their path (for more details, we refer the reader
to~\cite{keil}):
\begin{enumerate}
\item {\bf Base case:} If , the path consists of edge .

\item {\bf Recursive step:} Otherwise, a point must reside in the
region of  subtended by chord  and away from .
Let  be such a point with the property that the region of
 subtended by chord  closer to  is empty.
We call  an {\em intermediate point} with respect to the pair of
points . Let  be the circle passing through  and
 whose center  lies on segment  and let  be the
circle passing through  and  whose center  lies on
segment . Then both  and  lie inside , and
 and  are both less than . Moreover, the region of  subtended by chord
 that contains  is empty, and the region of 
subtended by chord  and containing  is empty. Therefore, we
can recursively construct a path from  to  and a path from 
to , and then concatenate them to obtain a path from  to .
\end{enumerate}

\begin{definition}\rm
\label{outwardpathDEF} We call the path constructed above the {\em
outward path} between  and .
\end{definition}
Keil and Gutwin~\cite{keil}, from this point on, use
a purely geometric argument (with no use of Delaunay graph
properties) to show that the length of the obtained path  (where each point , for , is
an intermediate point with respect to a pair , where ) is smaller than the length of
. Figure~\ref{outwardpath} illustrates an
outward path between  and .

\begin{figure}[htbp]

\begin{center} \small

\begin{pspicture}(-10,-10)(220,100)



\psarc[linecolor=darkgray, linewidth=0.3pt](100,-52){113}{20}{160}
\psarc[linewidth=0.5pt](0,0){25}{0}{35}

\cnode*(0,0){2pt}{c}\nput{-160}{c}{}
\cnode*(200,0){2pt}{b}\nput{-20}{b}{}
\cnode*(90,61){2pt}{a}\nput{90}{a}{}

\cnode*(145,45){2pt}{m1}\nput[labelsep=8pt]{35}{m1}{}
\cnode*(180,24){2pt}{m2}\nput{20}{m2}{}

\ncline{c}{b} \ncline{b}{m2} \ncline{m2}{m1} \ncline{m1}{a}
\ncline{a}{c} \ncline{c}{m1} \ncline{c}{m2}

\put(0,40){\scriptsize }\pnode(10,40){a1}
\pnode(15,10){a2} \ncline[nodesepA=5,nodesepB=5]{->}{a1}{a2}


\end{pspicture} \caption{Illustration of an outward path.}\label{outwardpath}
\end{center}
\end{figure}

\begin{proposition}
\label{nocross} In every recursive step of the outward path
construction described above, if  is an intermediate point with
respect to a pair of points , then:
\begin{itemize}

\item[(a)] there is a circle passing through  and  that contains
no point of , and

\item[(b)] circles  and  contain no
points of  except, possibly, in the region subtended by chords 
and , respectively, away from .
\end{itemize}
\begin{proof}
We assume, by induction, that there are circles  and
 passing through  and , and  and ,
respectively, containing no points of , and that the circle
 contains no point of  in the interior of
the region  subtended by chord  closer to .
(This is certainly true in the base case because , by
Proposition \ref{Euclidproperty}, and by our initial assumptions).

Since  is not an edge in , the point  chosen in the
construction is the point with the property that the region  of
 subtended by chord  away from , contains
no point of . Then the circle passing
through  and  and tangent to  at 
is completely inside , and
therefore devoid of points of . This proves part (a).

The region of  subtended by chord
 and containing  is inside ,
and therefore contains no point of  in its interior. The same is
true for the region of  subtended by chord
 and containing , and part (b) holds as well.
\end{proof}
\end{proposition}

We are now ready to prove Lemma~\ref{canpath} in the case when no
point of  lies inside . In this case we define
the path in Lemma~\ref{canpath} to be the outward path between 
and .
\begin{proof}[Proof of Lemma~\ref{canpath} for the case of outward
path.]
\begin{itemize}
\item[]
\item[] With , we have  and . We note that
 is largest
when , i.e. when  and  are symmetrical with
respect to the diameter of  passing through ; this
follows from the fact that the perimeter of a
convex body is not smaller than the perimeter of a convex body
containing it (see page 42 in~\cite{bookconvexity}). If ,
. Using elementary
trigonometry, it follows from the above facts and from
 that:

The last inequality follows from  and .

\item[] If  was an edge in  then, for every  between 
and , the circle  would not contain . This,
however, contradicts part (a) of Proposition~\ref{nocross}.

\item[] If the outward path contains a single intermediate point ,
then since  lies inside ,  (note that
), as desired. Now the statement
follows by induction on the number of steps taken to construct the
outward path between  and , using the fact (proved
in~\cite{keil}) that each angle  at the
center of the circle  defining the intermediate point ,
is bounded by .
\item[] This follows from the fact that . The last inequality is true
because  and  in
.
\end{itemize}
\end{proof}

\subsection{The Inward Path}
\label{inwardpatheuclidean} We consider now the case when the
interior of  contains points of . Let  be the
set of points consisting of points  and  plus all the points
interior to  (note that ). Let 
be the points on the convex hull of . Then  consists of
points  and , and points  of 
interior to . We have the following proposition:


\begin{proposition}
\label{inwardproperties} For every 
\begin{itemize}
\item[(a)] ,
\item[(b)] , and
\item[(c)] , where  is the angle facing point
.
\end{itemize}
\begin{proof}
These follow follow from the following facts:  and  are edges in ,
 is the shortest edge in its cone, and hence , for , and points  are on  in the listed order.
\end{proof}
\end{proposition}

Since  and no point of  lies inside
 ( and  are on ),
 is the shortest edge in the angular sector
. Since , by Lemma~\ref{canpath} there exists an
outward path  between  and , for every , satisfying all the properties of Lemma~\ref{canpath}.
Let  be the concatenation of the paths
, for .

\begin{definition}\rm
\label{inwardpathDEF} We call the path 
constructed above the {\em inward path} between  and .
\end{definition}
Figure~\ref{inwardpath} illustrates an inward path between  and
.

\begin{figure}[htbp]
\begin{center} \small
\begin{pspicture}(-10,-10)(220,80)

\psarc[linecolor=lightgray, linewidth=0.5pt](100,-75){125}{30}{150}
\psarc[linewidth=0.5pt](0,0){25}{0}{28}

\cnode*(0,0){2pt}{c}\nput{-160}{c}{}
\cnode*(200,0){2pt}{b}\nput{-20}{b}{}
\cnode*(95,50){2pt}{a}\nput{90}{a}{}

\cnode*(110,35){2pt}{m1}\cnode*(116,24){2pt}{m2}\nput[labelsep=1pt]{45}{m2}{\scriptsize
}
\cnode*(128,21){2pt}{m3}\cnode*(141,16){2pt}{m4}\cnode*(158,7){2pt}{m5}\nput[labelsep=1pt]{45}{m5}{\scriptsize
}



\ncline{c}{b} \ncline{c}{m1} \ncline{c}{m2} \ncline{c}{m3}
\ncline{c}{m4} \ncline{c}{m5} \ncline{b}{m5} \ncline{m5}{m4}
\ncline{m4}{m3} \ncline{m3}{m2} \ncline{m2}{m1} \ncline{m1}{a}
\ncline{a}{c}

\put(0,40){\scriptsize }\pnode(10,40){a1}
\pnode(15,10){a2} \ncline[nodesepA=5,nodesepB=5]{->}{a1}{a2}
\end{pspicture}
\caption{Illustration of an inward path.}\label{inwardpath}
\end{center}
\end{figure}

We now prove Lemma~\ref{canpath} in the case when there are points
of  interior to . In this case we define the path
in Lemma~\ref{canpath} to be the inward path between  and .

\begin{proof}[Proof of Lemma~\ref{canpath} for the case of inward
path.]
\begin{itemize}
\item[]
\item[]


Define  to be a point on the half-line  such that , and let . Denote by  the
length of the arc of  subtended by chord  and
facing . For every , we define
arc  to be the arc of  subtended by
chord  and facing . For every , we define  to be the point on the half-line
 such that ,  to be the circle
, and  to be the arc of 
subtended by chord  and facing .
Finally, for every , we define  to be the
point of intersection of the half-line  and circle ,
and  to be the arc of  subtended by chord
 and facing . As shown in
section~\ref{outpath}, the length of the outward path  between
 and  is bounded by the length of . Since
the convex body  delimited by ,  and 
is contained inside the convex body  delimited by ,
 and , by~\cite{bookconvexity}, the perimeter
of  is not larger than that of . Denoting by  the
length of path , we get:

Since  and  are concentric circles (of center ,
and the radius of  is not larger than that of , we
have , for . It follows
from Inequality~(\ref{eq1}) that:

Using Inequalities~(\ref{eq1}) and (\ref{eq2}) we get:

Noting that , that
, and using the same argument as in
part () of Lemma~\ref{canpath}) completes the proof.

\item[] Since  for  by part  of
Proposition~\ref{inwardproperties}, by planarity of , if such an
edge between two points  and  exists, then  and 
must belong to an outward path between two points  and
 of . But this contradicts part  of
Lemma~\ref{canpath} for the case of the outward path applied to
 and .

\item[] For each , either ,
or  is an intermediate point on the outward path between two
points  and  in . In the former case
 for  ( and  are points
before and after  on ), by part (c) of
Proposition~\ref{inwardproperties}. In the latter case
 by the proof of part
 of Lemma~\ref{canpath} applied to the outward path between
 and .

\item[] This follows from  and
, in triangle
.
\end{itemize}
\end{proof}

\subsection{The Modified Yao Step}
\label{modifiedyaoalgo}

We now augment the {\em Yao step} so edges forming the paths described in
Lemma~\ref{canpath} are included in , in addition to the edges chosen
in the standard Yao step. Lemma~\ref{canpath} says that consecutive edges
on such paths form moderately large angles. The modified Yao step will
ensure that consecutive edges forming large angles are included in .
The algorithm is described in Figure~\ref{yaostep}.

\begin{algorthm}{The modified Yao Step.}{yaostep} \algtitle{Modified Yao step}
\alginput{A Delaunay graph ; integer } \algoutput{A
subgraph  of  of maximum degree }
\begin{codeblock}

\step define  disjoint cones of size  around every point
 in ;

\step  in every non-empty cone, select the shortest edge incident on
 in this cone;

\step {\bf for} every maximal sequence of  consecutive
empty cones:

\hspace*{3mm} 3.1. {\bf if}  {\bf then} select the first
 unselected  \\
\hspace*{7.9mm}   incident edges on  clockwise from the sequence
\\ \hspace*{7.9mm} of empty cones and the first  unselected  \\ \hspace*{7.9mm} edges
incident on  counterclockwise from the \\
\hspace*{7.9mm} sequence  of empty cones; \\
\hspace*{3mm} 3.2. {\bf else} (i.e., ) let  and 
be the incident  \\ \hspace*{7.9mm} edges on  clockwise and
counterclockwise,  \\ \hspace*{7.9mm} respectively, from the empty
cone; {\bf if} either  \\ \hspace*{7.9mm} or  is selected
{\bf then}  select the other edge  \\
\hspace*{7.9mm} (in case it has not been selected); {\bf otherwise} select\\
\hspace*{7.9mm}  the shorter edge between  and
  breaking \\ \hspace*{7.9mm} ties arbitrarily;

\step  is the spanning subgraph of  consisting of edges selected by
both endpoints.
\end{codeblock}
\end{algorthm}

Since the algorithm selects at most  edges incident on any point
 and since only edges chosen by both endpoints are included in ,
each point has degree at most  in .

Before we complete the proof of Theorem~\ref{maintheorem}, we show that the
running time of the algorithm is linear. Note first that all edges incident
on point  of degree  can be mapped to the  cones around  in
linear time in . Then, the shortest edge in every cone can be found in
time  (step 2. in the algorithm). Since  is a constant, selecting
the  edges clockwise (or counterclockwise) from a sequence of 
empty cones around  (step 3.1.) can be done in  time. Noting that
the total number of edges in  is linear in the number of vertices completes
the analysis.

To complete the proof of Theorem~\ref{maintheorem}, all we need to do is show:
\begin{lemma}
\label{edgeselected}
If edge  is not selected by the algorithm, let  be the shortest edge in
the cone out of  to which  belongs. Then the edges of the path
described in Lemma~\ref{canpath} are included in   by the algorithm.
\begin{proof}
For brevity, instead of saying that the algorithm {\bf Modified Yao
Step} selects an edge  out of a point , we will say that 
selects edge . To get started, it is obvious that  will select edge
.

By part  of Lemma~\ref{canpath}, the angle  for . Therefore, at least
two empty cones must fall within the sector 
determined by the two consecutive edges  and , and edges
 and  will both be selected by . Since edge  is
also selected by point , edge .

By part  of Lemma~\ref{canpath}, for every , the angle  for , and hence at least four cones fall within
the angular sector . Since by part {\em
(ii)} of Lemma~\ref{canpath}  is the only possible edge inside
the angular sector , it is easy to see
that regardless of the position of these four cones with respect to
edge ,  ends up selecting all edges ,
 and  in steps 2 and/or 3 of the algorithm. Since
we showed above that  selects edge , this shows that all
edges , for , are selected by both
their endpoints, and hence must be in . Moreover, edge
 is selected by point .

We now argue that edge  will be selected by . First, observe
that . 
Let  be the other consecutive edge to  in  (other than
). Because  does not select , it follows that
. Otherwise, since  and
 are in the same cone, two empty cones would fall within the
sector  and  would select . Since  is an edge
in , by the characterization of Delaunay edges~\cite{book},
. By considering the
quadrilateral , we have . This, together with the fact that
, imply that , for . Therefore,
 contains at least three cones of size 
out of . If one of these cones falls within the angular sector
 then, since ,  must
have been selected out of .

Suppose now that  contains no cone inside and
hence . If one of these three cones
within sector  contains edge , then the
remaining two cones must fall within  and 
will get selected out of  when considering the sequence of at
least two empty cones contained within . Suppose now
that all three empty cones fall within . Then we have
.

If , then since  and 
belong to the same cone, the sector  must contain an
empty cone. Because  is exterior to ,
, and ,
it follows that . Therefore, by considering the triangle
, we note that . But then edge 
would have been selected by  in step 3 since the sector
 contains an empty cone, a contradiction.

It follows that , and therefore
 for .
This means that at least four cones are contained inside sector
. It is easy to check now that regardless of the
placement of the edge  with respect to these cones, edge
 is always selected out of  by the algorithm. This
completes the proof.
\end{proof}
\end{lemma}

\begin{corollary}
A Euclidean Minimum Spanning Tree (EMST) on  is a subgraph of . 
\begin{proof}
It is well known that a Delaunay graph () contains a EMST. If an edge 
is not in , then, by Lemma~\ref{edgeselected}, a path from  to  is
included in . All edges on this path are no longer than , so there is
a EMST not including . 
\end{proof}
\end{corollary}
Since a Delaunay graph of a Euclidean graph of  points can be
computed in time ~\cite{book} and has stretch factor
, we have the following theorem.
\begin{theorem}
\label{spannereuclidean} There exists an algorithm that, given a
set  of  points in the plane, computes a plane geometric spanner
of the Euclidean graph on  that contains a EMST, has maximum degree ,
and has stretch factor
, where 
is an integer. Moreover, the algorithm runs in time .
\end{theorem}


\section{Geometric Spanners of Unit Disk Graphs}
  \label{structresults}  In this section we generalize our planar geometric
spanner algorithm to unit disk graphs. Unit disk graphs model wireless ad-hoc
and sensor networks and, for packet routing and other applications, a
bounded-degree planar geometric spanner of the wireless network is often
desired. Due to the limited computational power of the network devices and
the requirement that the network be robust with respect to device joining
and leaving the network, the construction/algorithm should ideally be
{\em strictly-localized}: the computation performed at a point depends solely
on the information available at the point and its -hop neighbors, for some
constant  (in our case ). In particular, no global propagation of
information should take place in the network.

The results in the previous section do not carry over to unit disk graphs
because not all Delaunay graph edges on a point-set  are
unit disk edges. However, if  is the unit disk graph on points in  and
 is the subgraph of the Delaunay graph on  obtained by deleting
edges of length greater than one unit, then  is a connected,
planar, spanning subgraph of  with stretch factor bounded by 
(see~\cite{iitunbounded1, BoseMNSZ04}). Therefore, if we apply the results
from the previous section to
 and observe that all edges on the path defined in
Lemma~\ref{canpath} must be unit disk edges (given that edges 
and  are), it is easy to see that Theorem~\ref{maintheorem} and
Theorem~\ref{spannereuclidean} carry over to unit disk graphs. The
only problem, however, is that the construction of  cannot
be done in a strictly-localized manner.

To solve this problem, Wang et al.~\cite{iitunbounded1,iitunbounded}
introduced a subgraph of  denoted . It was shown
in~\cite{iitunbounded1,iitunbounded} that  is a
planar supergraph of , and hence also has stretch factor bounded
by . Moreover, the results in~\cite{gruia,iitbounded} show
how  can be computed with a strictly-localized
distributed algorithm exchanging no more than  messages in
total ( is the number of points in ), and having a local
processing time of  at a point of
degree . In a style similar to Definition~\ref{DelaunayDef},
 can be defined as follows:
\begin{definition}
\label{ldel2}
An edge  of  is in  if and only if there exists a
circle through points  and  whose interior contains no point of 
that is a 2-hop neighbor of  or .
\end{definition}
We will use  as the underlying subgraph
of  to replace the Delaunay graph  used in the previous
section. We note that  is planar, is a
supergraph of , and hence has stretch factor . To
translate our results to unit disk graphs, we need to show that the
inward and outward paths are still well defined in .
In particular, we need to show that Lemma~\ref{canpath} holds for
. We outline the general approach and omit the
details for lack of space.

The following is equivalent to Proposition~\ref{Euclidproperty}:
\begin{lemma}
\label{UnitDiskproperty} If  and  are edges of  then the region
of  subtended by chord  and away from  and
the region of  subtended by chord  and away from  contain no
points that are two hop neighbors of ,  and .
\begin{proof}
By symmetry it is enough to prove the lemma for  the region of 
subtended by chord  and away from . By Definition~\ref{ldel2},
there is a circle  passing through  and  whose interior
is empty of any point within two hops of  or . The region of 
subtended by chord  and away from  is inside this circle, so we only
need to argue that it doesn't contain two hop neighbors of  either. If it
did, say point , then any neighbor of  and  would have to be a
neighbor of  or  as well, a contradiction.
\end{proof}
\end{lemma}
With this lemma in hand, the recursive construction of the outward path given
in Subsection~\ref{outpath} can be applied to the graph .
The following proposition for  corresponds to
Proposition~\ref{nocross} for Delaunay graphs and is proven in an equivalent
manner:
\begin{proposition}
\label{nocross2} In every recursive step of the outward path
construction, if  is an intermediate point with
respect to a pair of points , then:
\begin{itemize}

\item[(a)] there is a circle passing through  and  that contains
no point of  that is a two-hop neighbor of  or , and

\item[(b)] circles  and  contain no
points of  that are two-hop neighbors of ,  and  and ,
, and , respectively, except, possibly, in the region subtended by
chords  and , respectively, away from .
\end{itemize}
\end{proposition}
With this proposition, we can show that Lemma~\ref{canpath} holds
true for  for outward paths. It holds for inward paths
as well, using the same argument as in Section~\ref{inwardpatheuclidean}.  
Finally, it is obvious how the {\bf Modified Yao Step} algorithm in
Section~\ref{modifiedyaoalgo} can be easily described as a
strictly-localized algorithm. We can show, therefore, the following theorem:
\begin{theorem}
\label{spannerunit} There exists a distributed strictly-localized
algorithm that, given a set  of  points in the plane, computes a plane
geometric spanner of the unit disk graph on  that contains a EMST, has
maximum degree , and has stretch factor
,
for any integer . Moreover, the algorithm exchanges no
more than  messages in total, and has a local processing time
of  at a point of degree .
\end{theorem}
Due to the strictly-localized nature of the algorithm, the algorithm
is very robust to topological changes (such as wireless devices moving or
joining or leaving the network), an essential property for
the application of the algorithm in a wireless ad-hoc environment.
\bibliographystyle{plain}
\begin{thebibliography}{10}

\bibitem{bookconvexity}
R.~Benson.
\newblock {\em Euclidean Geometry and Convexity}.
\newblock Mc-Graw Hill, New York, 1966.

\bibitem{boseesa}
P.~Bose, J.~Gudmundsson, and M.~Smid.
\newblock Constructing plane spanners of bounded degree and low weight.
\newblock In {\em Proceedings of the 10th Annual European Symposium on
  Algorithms}, volume 2461 of {\em Lecture Notes in Computer Science}, pages
  234--246. Springer, 2002.

\bibitem{bosealgorithmica}
P.~Bose, J.~Gudmundsson, and M.~Smid.
\newblock Constructing plane spanners of bounded degree and low weight.
\newblock {\em Algorithmica}, 42(3-4):249--264, 2005.

\bibitem{BoseMNSZ04}
P.~Bose, A.~Maheshwari, G.~Narasimhan, M.~Smid, and N.~Zeh.
\newblock Approximating geometric bottleneck shortest paths.
\newblock {\em Computational Geometry: Theory and Applications}, 29:233--249,
  2004.

\bibitem{bose1}
P.~Bose, M.~Smid, and D.~Xu.
\newblock Diamond triangulations contain spanners of bounded degree.
\newblock {\em To appear in Algorithmica}, 2007.

\bibitem{gruia}
G.~Calinescu.
\newblock Computing 2-hop neighborhoods in {Ad Hoc} wireless networks.
\newblock In {\em Proceedingsof the 2nd International Conference on Ad-Hoc,
  Mobile, and Wireless Networks}, volume 2865 of {\em Lecture Notes in Computer
  Science}, pages 175--186. Springer, 2003.

\bibitem{chew}
P.~Chew.
\newblock There are planar graphs almost as good as the complete graph.
\newblock {\em Journal of Computers and System Sciences.}, 39(2):205--219,
  1989.

\bibitem{book}
M.~de~Berg, M.~van Kreveld, M.~Overmars, and O.~Schwarzkopf.
\newblock {\em Computational Geometry: Algorithms and Applications}.
\newblock Springer-Verlag, second edition, 2000.

\bibitem{dobkin}
D.~Dobkin, S.~Friedman, and K.~Supowit.
\newblock Delaunay graphs are almost as good as complete graphs.
\newblock {\em Discrete Computational Geometry}, 5(4):399--407, 1990.

\bibitem{keil}
J.~Keil and C.~Gutwin.
\newblock Classes of graphs which approximate the complete {Euclidean} graph.
\newblock {\em Discrete {\&} Computational Geometry}, 7:13--28, 1992.

\bibitem{iitunbounded1}
X.-Y. Li, G.~Calinescu, and P.-J. Wan.
\newblock Distributed construction of planar spanner and routing for ad hoc
  wireless networks.
\newblock In {\em Proceedings of the {IEEE} INFOCOM}, 2002.

\bibitem{iitunbounded}
X.-Y. Li, G.~Calinescu, P.-J. Wan, and Y.~Wang.
\newblock Localized delaunay triangulation with application in {Ad Hoc}
  wireless networks.
\newblock {\em IEEE Transactions on Parallel and Distributed Systems.},
  14(10):1035--1047, 2003.

\bibitem{spannerbook}
G.~Narasimhan and M.~Smid.
\newblock {\em Geometric Spanner Networks}.
\newblock Cambridge University Press, 2007.

\bibitem{iitbounded1}
Y.~Wang and X.-Y. Li.
\newblock Localized construction of bounded degree and planar spanner for
  wireless ad hoc networks.
\newblock In {\em Proceedings of the {DIALM-POMC} Joint Workshop on Foundations
  of Mobile Computing}, pages 59--68. ACM, 2003.

\bibitem{iitbounded}
Y.~Wang and X.-Y. Li.
\newblock Localized construction of bounded degree and planar spanner for
  wireless ad hoc networks.
\newblock {\em MONET}, 11(2):161--175, 2006.

\bibitem{yao}
A.~C.-C. Yao.
\newblock On constructing minimum spanning trees in k-dimensional spaces and
  related problems.
\newblock {\em SIAM Journal on Computing}, 11(4):721--736, 1982.

\end{thebibliography}

\end{document}
