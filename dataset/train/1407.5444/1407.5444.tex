\subsection{Frame of a tagged process}
\label{subset: frame flagged process}

In this subsection, we will state and prove the lemmas regarding
frames and static equivalence.
Let  be a frame such that:

Let  be a recipe, \emph{i.e.} a term such that  and , we define the measure  as follows:
 
where  is the maximal indice  such that
, and  denotes the size of the term~, \emph{i.e.} the number of
symbols that occur in~.

We have that  when
either ;
or  and .

Once again, we denote by  and  the assignment variables of the extended processes that we are considering. 









\begin{definition}
 Let  be an extended process,  be a total order on  and  be a mapping from  to . We say that   is a \emph{derived well-tagged extended process} w.r.t.~ and  if for every  (resp. ), there exists  such that one of the following condition is
satisfied:
\begin{enumerate}
\item there exist  and  such that
  , , and for all ,  and either  or there exists  such that ; or
\item there exists  such that ,  and
    .
\end{enumerate}
where  (resp. ). 
\end{definition}

{In the case of variables instantiated through an output, and or an internal communication, it will be the first item that needs to hold; while in the case of variables intantiated through inputs on public channels it is the second item that needs to hold.} Intuitively, the order  on  corresponds to the order in which the variables in  have been introduced along the execution. In particular, we have that  where . In the following, we sometimes simply say that  is a derived well-tagged extended process.







\begin{lemma}
  \label{lem:Flawedandframeelement}
  Let  be a derived well-tagged extended process w.r.t  and . Let  (resp. ) and  (resp. ). We have that there exists  such that ,  and .
\end{lemma}

\begin{proof}
We prove this result by induction on 
with the order .

\smallskip{}

\noindent \emph{Base case  or  with  for any .} Assume .
By definition of a derived well-tagged extended process w.r.t  and , one of the
following condition is satisfied:
\begin{enumerate}
\item There exist  and  such that , , and  for any .  Since  and , we can apply Lemma~\ref{lem: flawed color and tag term}
 to  and~. Thus, we have that there exists  such that . However, since  is mimimal w.r.t. ,
we know that .   Hence, we obtain a
contradiction. This case is impossible.
\item There exists  such that , , and . Thus, we have that , and we
  have that .
\end{enumerate}

\medskip{}

\noindent \emph{Inductive case  or .} Assume .
By definition of a derived well-tagged extended process w.r.t  and , one of the
following condition is satisfied:
\begin{enumerate}
\item There exist  and  such that , , and  for any .  Since  and , we can apply Lemma~\ref{lem: flawed color and tag term}
 to  and~. Thus, we have that there exists  such that , and we have that .
Hence, we conclude by applying our induction hypothesis.
\item There exists  such that , , and . Thus, we have that , and we
  have that .
\end{enumerate}
This allows us to conclude.
\end{proof}



\begin{lemma}
  \label{lem:FlawedColor and frame element direct element}
  Let  be a derived well-tagged extended process w.r.t  and . Let . Let  (resp. ) such that . Let  (resp. ). Let . We have that 
  \begin{itemize}
  \item either there exists  such that ,  and ; 
  \item otherwise there exists  such that  and .
  \end{itemize}
\end{lemma}

\begin{proof}
We prove this result by induction on 
with the order .

\smallskip{}

\noindent \emph{Base case  or  with  for any .} Let  and  with .
By definition of a derived well-tagged extended process w.r.t  and , one of the
following condition is satisfied:
\begin{enumerate}
\item There exist  and  such that , , and  for any . Since  is minimal by  then . Hence . Thus we deduce that . Hence there is a contradiction with  and so this condition cannot be satisfied.
\item There exists  such that , , and . Thus, we have that  and so the result holds.
  \end{enumerate}
  
  \medskip{}
  
\noindent \emph{Inductive case  or .} Assume  and .
By definition of a derived well-tagged extended process w.r.t  and , one of the
following condition is satisfied:  
\begin{enumerate}
\item There exist  and  such that
  , , and for all ,  and either  or there exists  such that .
  Since  and , we can apply Lemma~\ref{lem: flawed color and tag term 2}
 to  and~. Thus we have that . In such a case, it means that there exists  with  such that  and one of the two conditions is satisfied:
 \begin{itemize}
 \item : In such a case, we can apply our inductive hypothesis on  and  and so the result holds.
 \item there exists  such that : Otherwise, we know by hypothesis that  or . Since , we deduce that  and so . But this implies that . Hence the result holds.
 \end{itemize}
\item There exists  such that , , and . Thus, we have that , and we
  have that .
\end{enumerate}
This allows us to conclude.
\end{proof}



\begin{lemma}
  \label{lem:flawed,smallerrecipe}
  Let  be a derived well-tagged extended process. Let  be a term such that  and . Let .  There exists  such that , , , and , for all .
\end{lemma}


\begin{proof}
 We prove this result by induction on .

\smallskip{}


 \noindent\emph{Base case :} In this case, either we
 have that  or . If , then we have
  and . Thus the result holds. If 
then, by Lemma~\ref{lem:Flawedandframeelement},
 implies that 
there exists  such that:
\begin{itemize}
\item  , 
\item , and 
\item .
\end{itemize} 
Since , thanks to
 our inductive hypothesis, we  deduce that there exist 
 such that for each , we have that:
, ,
 , and 
.

 \medskip

 \noindent \emph{Inductive step :} In such a case, we have that . Let . We do a case analysis on .

\smallskip{}

 \emph{Case  for some :} In such a  case, . By definition, we know that for all , we have that . Thus, thanks to Lemma~\ref{lem:app_CRicalp05}, we 
 deduce that 
 
Since  for any , thanks to our inductive hypothesis, we know that there exists  such that , ,  and , for
 . Hence the result holds.

\smallskip{}

 \emph{Case :} In such a case, . Moreover, we have
 that 
. Since ,  and
 , we
 conclude by applying our inductive hypothesis on  (or ).

\smallskip{}

 \emph{Case :} In this case,  and we have
 that . Hence the result trivially holds.

\smallskip{}
 
 \emph{Case :} In such a case, we
 have that
 . We need
 to distinguish whether  for
 some  or not.

 If  for some ,  then there exists  such that . Hence, we have that . We have also that:

 We deduce that  or . Since  and ,
 we conclude by applying our inductive hypothesis on  or .

 Otherwise . In such a
 case, . If , we have that ,  and , . Thus the result
 holds. If , we conclude by applying our inductive
 hypothesis on  or .


\smallskip{}

 \emph{Case :} This case is analogous to the previous one and can
 be handled similarly.


\smallskip{}

 \emph{Case :} In such a case,
 we have to distinguish two cases depending on whether  is reduced in
 , or not.

 If  is not reduced, \emph{i.e.} , then we have that
  
Thus if , we have that ,  and , . Thus the result
 holds. Otherwise, we have that  or . Since , , we
 can conclude by applying our inductive hypothesis on~ or~.

 If  is reduced, then we have that  with  and . If  for some , then
 we have that there exists  such that ,
  and . Thus, we have that 
.
Otherwise, if , then we
have that 

 and
 . Thus, 
 . In
 both cases, we have that  and since , we can conclude by
 applying our inductive hypothesis on~.
\end{proof}


\newcommand{\fctpair}{\fct_{\langle\ \rangle}}

In the following lemma, we will use the factors of the signature only composed of , denoted . Typically, for all terms , for all context built only on , for all terms , if  and for all ,  then .




\begin{lemma}
  \label{lem:FlawedColor and frame element direct element 2}
  Let  be a derived well-tagged extended process w.r.t  and . Assume that for all assignment variables , . Let  such that , . For all , for all , if  and for all assignment variable , for all ,  and  implies  then there exists  such that ,  and .
\end{lemma}

\begin{proof}
We do a proof by induction on :

\medskip

\noindent\emph{Base case :} In this case, we have that  which means that  and . Thus the result holds. 

\medskip

\noindent\emph{Base case :} In this case, we have . Let . Let  such that . We do a case analysis on :

\emph{Case :} In this case, since for all assignment variable , for all ,  and  implies , than we can deduce that for all assignment variables , . Thus by Lemma~\ref{lem:FlawedColor and frame element direct element}, we obtain that there exists  such that ,  and .  implies that  with  and so . If  then the result holds. Otherwise, we can apply our inductive hypothesis on  and  and so the result holds.
 
 \emph{Case  :} Since , we deduce that there exists  s.t.  and . Note that  otherwise it would contradict the fact that . But . Moreover,  implies that  is deducible in . Thus we deduce that for all assignment variables , . By applying the same proof as case , we deduce that there exists  such that ,  and . But ,  and  implies that  and . Hence we can apply our inductive hypothesis on  and  which allows us to conclude.

\medskip

 \noindent \emph{Inductive step :} In such a case, we have that . Let  such that . We do a case analysis on .

\smallskip{}

 \emph{Case  for some :} In such a  case, . By definition, we know that for all , we have that . Thus, thanks to Lemma~\ref{lem: flawed color and tag term 2}, we 
 deduce that there exists 
 
Thus there exists  such that . If  then the result holds, else we apply our inductive hypothesis on  and  and so the result also holds.

\smallskip{}

 \emph{Case  for some :} In such a  case, . We assumed that  hence there exists  s.t.  and . But it also implies that . Hence, by applying Lemma~\ref{lem:app_CRicalp05}, we deduce that there exists  such that . Moreover, it also implies that . 
 
 If  then we deduce that  and so, by Lemma~\ref{lem:app_CRicalp05}, . Since we had , then we also have  and so we conclude by applying our inductive hypothesis on  and .
 
 if  then we can apply our inductive hypothesis on  and . Indeed, since , then  is deducible in  and so we deduce that for all assignment variable , . Hence we obtain that there exists  such that ,  and . But  and . Hence we deduce that  and . We conclude by applying once again our inductive hypothesis but on  and .
 
 \smallskip{}

 \emph{Case :} In such a case, . Moreover, we have
 that 
. Since ,  and
 , we
 conclude by applying our inductive hypothesis on  and  (or ).

\smallskip{}

 \emph{Case :} In this case,  and we have
 that . Hence the result trivially holds.

\smallskip{}
 
 \emph{Case :} In such a case, we
 have that
 . We need
 to distinguish whether  for
 some  or not.

 If  for some ,  then there exists  such that . Assume first that . In such a case  and . Hence it contradicts the fact that . We can thus deduce that . But in such a case, we have that  and:

 We deduce that  or . Since  and ,
 we conclude by applying our inductive hypothesis on  or .

 Otherwise . In such a
 case,  and . But we assume that  hence this case is impossible.

\smallskip{}

 \emph{Case :} This case is analogous to the previous one and can
 be handled similarly.


\smallskip{}

 \emph{Case :} In such a case,
 we have to distinguish two cases depending on whether  is reduced in
 , or not.

 If  is not reduced, \emph{i.e.} , then we have that
  
Once again this is in contradiction with our hypothesis that .

We now focus on the case where  is reduced: we have that  with  and . We have to do a case analysis on :
 \begin{itemize}
 \item if  for some . In such a case, there exists  such that ,  and . Thus we deduce that . We can conclude thanks to our inductive hypothesis on  and . 
\item if  for some . In such a case,  which contradicts the hypothesis .
\item otherwise, , then we
have that . By Lemma~\ref{lem:flawed,smallerrecipe}, we deduce that there exists  such that ,  and . Since  and  then we can apply our inductive hypothesis on  and  and so the result holds.
\end{itemize}
\end{proof}



\begin{lemma}
  \label{lem : deductibily of fct_C}
 {Let  be a derived well-tagged process, and let  be compatible with .} Let  be a ground term in normal form that do not use names in . We have that there exists a context  (possibly a hole) built only using , and terms  such that , and for all ,
  \begin{itemize}
  \item either ;
  \item or  and ,
  \item or  for some  and ,
  \item or .
  \end{itemize}
\end{lemma}

\begin{proof}
 Let  a ground term in normal form and let . Thus there exists a context  (possibly a hole) built
 on  such that . We now prove the result by
 induction on .

 \medskip

 \noindent \emph{Base case :} 
We show that the result holds and in such a case the context  is
reduced to a hole.
Since , we know that  and so
 either  with  or
 . 
If , then the
result trivially holds. Otherwise, we have 
that  by definition of
 and . Hence the result holds.


 \medskip

 \noindent \emph{Inductive step :} There exists , and  such that . We do a case analysis on .

\smallskip{}

 \emph{Case :} In such a case, there exist
 two contexts   (possibly holes) built on  such that:
\begin{itemize}
\item   with , 
\item  and ,
\item  and 
\end{itemize}
By applying our inductive hypothesis on  and , we know that
there exist two contexts  and .
Since 
\begin{itemize}
\item , and 
\item  , 
\end{itemize}
we conclude
   that  satisfies all the conditions
   stated in the lemma.
  

\smallskip{}

 \emph{Case  and  for some :}
 The result trivially hold by choosing the context  to be a hole.

\smallskip{}

Otherwise, we have that 
 
Since , we can choose  to be the context reduced
to a hole. 
The result trivially holds.
\end{proof}




\begin{lemma}
\label{lem:samerecipesymmetric}
Let  be a derived well-tagged extended process, and let  be compatible with .
Let   be a term such that  and . We assume that 
, 
 , and
one of the two following conditions
is satisfied: 
\begin{enumerate}
\item   for any ; or
\item   for any {.} 
\end{enumerate}
with t. 
We have that  with .
\end{lemma}

\begin{proof}
Let .
We prove this result by induction on :

\medskip

\noindent\emph{Base case :} There exists no term  such that , thus the result holds.

\medskip

\noindent\emph{Inductive step :} We first prove there exists  such that  and then we show that
.

\medskip{}

Assume first that , \emph{i.e.} either  or there exists  such that . 

\noindent \emph{Case .} In such a case, we have that
, and . Hence, we have that 
 and also that . {In case condition  is
satisfied, we easily deduce that . Otherwise, we know
that the condition  is satisfied, and thus . Again, we want to conclude
that . Assume that this is not the case, \emph{i.e.} . This means that  is a name in  (or
). Hence, we  have that , and {}. Hence, we
deduce that , and this leads to a
contradiction, since in such a case, by hypothesis  can not be deducible from .} 
Thus, in any case, we have that  , and
thus .
Hence, we have that 
 for any .

\noindent \emph{Case  for some .} 
We know that  is colored with . Hence, we have
that  . Since  is in normal form, then by
Lemma~\ref{lem:deltaandnormalform}, 
we know that  is also in
normal form. Thus, we have that .

\medskip{}

Otherwise, if , then there exists a symbol  and  such that . We do a case analysis on .

\smallskip{}

\emph{Case  with .} Consider  such that . In such a case, let . Since  (resp. ), then there exists a
context  built upon  (resp. ) such that  and
 are factor of  in normal form. By
Lemma~\ref{lem:app_CRicalp05}, we know that there exists a context  (possibly
a hole) over  (resp. ) such that  with  and . But
thanks to Lemma~\ref{lem:change alien}, ~\ref{lem:transfoV} and~\ref{lem:deltaandnormalform}, we also that . But  and  are both built on  (resp. ), thus by definition of , we have that
 and . Hence, the
equality, ,
holds. But  which means that
.
We have that:
 
Since , \ldots, , we can apply our inductive hypothesis
on . This gives us . Thus we can conclude that
.

\smallskip{}

\emph{Case :} In
this case, we have that . By applying our inductive hypothesis on , we have that 
\begin{center}
, for all . 
\end{center}
Thus we
have that  with .
Applying our inductive hypothesis on , we
deduce that 



\smallskip{}

\emph{Case :} If we first assume that
the root occurence  is not reduced in  then the proof
is similar to the previous case. Thus, we focus on the case where the root
occurence of  is reduced, and we consider the case where . The other cases can be done in a similar way.
In such a situation, we know that there exist  such that
,  and
 . According to the definition of , we
 know that there exists  such that
 . For such , we have that
 . But by applying our inductive hypothesis on
  and , we obtain .

\bigskip{}

It remains to prove that . We have shown that there exists 
such that . Thanks to Lemma~\ref{lem : deductibily of fct_C},
we know that there exists a context  built over , and
 terms such that  and
for all :
\begin{itemize}
\item either 
\item or  and .
\item or  for some  and ,
\item or .
\end{itemize}

Note that  being built upon  means that
 is deducible in  for all . Furthermore,
since  is in normal form,

But we have shown that
, thus
 is deducible from , for all . Now, we distinguish several cases depending on
which condition is fullfilled by .
\smallskip{}

\emph{Case :} There exists  terms and a function symbol  such that . By Lemma~\ref{lem:flawed,smallerrecipe}, there exists  such that for all , 
and . Hence, by applying inductive hypothesis on
, we obtain that , for all . Thus,
thanks to  being in normal form, we can conclude that .

\smallskip{}

\emph{Case :} In
  such a case, we have that 
. Hence, we easily conclude.

\smallskip{}

\emph{Case  for some  and :} {By
hypothesis, we know that either ,
for all ; or , for
all .
Since we have shown that  is deducible from 
and  is deducible from , both
hypotheses imply that , and so .}

\smallskip{}

\emph{Case :} 
{By hypothesis, we know that either ,
for all ; or , for
all .
Since we have shown that  is deducible from 
and  is deducible from , both
hypotheses imply that  and lead us to a contradiction.}
\end{proof}



\begin{corollary}
\label{cor:deltaandkeyhidden}
Let  be a derived well-tagged extended
process and let  be compatible with , such that , and
.  The two following
conditions are equivalent:
\begin{enumerate}
\item   for any ; or
\item   for any {.} 
\end{enumerate}
with 
t. 
\end{corollary}

\smallskip{}

\begin{proof}
We prove the two implications separately.

\noindent : Let  such that . In such a case, there exists  such that
, , and
. We assume w.l.o.g. that t. Let . 
By Lemma~\ref{lem:transfoV}, we
have that {.}
Thanks to Lemma~\ref{lem:samerecipesymmetric}, we have that
{}, and by
Definition of {}, we have that
{}. Thus, we deduce that there
exists {} such that .

\smallskip{}

\noindent :{Let  with }, and  be a term such that , , and . { implies the existence of  such that , and thus such that . Thanks to Lemma~\ref{lem:samerecipesymmetric}, we have that . Now, if  there must exist  such that either , or , or . In any case, because  is in normal form, we know that  and thus that . Hence . But, then according to Lemma~\ref{lem:deltaandnormalform}, . Finally, thanks to Lemma~\ref{lem:transfoV} we can derive that .} This implies that , and thus there is a term in  that is deducible from .
\end{proof}



\begin{corollary}
\label{cor:framestatequiv}
Let  be a derived extended process and let  be compatible with 
such
that , 
 , 
and
 for 
any .
We have that
.
\end{corollary}

\begin{proof}
 The proof directly follows from Lemmas~\ref{lem:transfoV}
 and~\ref{lem:samerecipesymmetric}. 
Indeed,  is
 equivalent to  (thanks to Lemma~\ref{lem:transfoV}), which
 is equivalent to  (thanks to Lemma~\ref{lem:samerecipesymmetric}).
\end{proof}

