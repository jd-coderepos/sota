\subsection{Frame of a tagged process}
\label{subset: frame flagged process}

In this subsection, we will state and prove the lemmas regarding
frames and static equivalence.
Let $\nu \Ec. \Phi$ be a frame such that:
\[
\Phi = \{ w_1
\refer u_1, \ldots, w_n \refer u_n\}.
\]
Let $M$ be a recipe, \emph{i.e.} a term such that $\fv(M) \subseteq \dom(\Phi)$ and $\fn(M) \cap
\Ec = \emptyset$, we define the measure $\M$ as follows:
 \[
\M(M) = (i_\mathsf{max}, |M|)
\]
where $i_\mathsf{max} \in \{1,\ldots,n\}$ is the maximal indice $i$ such that
$w_ i \in \fv(M)$, and $|M|$ denotes the size of the term~$M$, \emph{i.e.} the number of
symbols that occur in~$M$.

We have that $\M(M_1) \stackrel{\defi}{=} (i_1, s_1) < \M(M_2)
\stackrel{\defi}{=} (i_2,s_2)$ when
either $i_1 < i_2$;
or $i_1 = i_2$ and $s_1 < s_2$.

Once again, we denote by $z^\alpha_1, \ldots, z^\alpha_k$ and $z^\beta_1, \ldots, z^\beta_\ell$ the assignment variables of the extended processes that we are considering. 









\begin{definition}
 Let $\quadruple{\Ec}{\p}{\Phi}{\sigma}$ be an extended process, $\prec$ be a total order on $\dom(\Phi) \cup \dom(\sigma)$ and $\vcol$ be a mapping from $\dom(\Phi) \cup \dom(\sigma)$ to $\{ 1, \ldots, p\}$. We say that  $\quadruple{\Ec}{\p}{\Phi}{\sigma}$ is a \emph{derived well-tagged extended process} w.r.t.~$\prec$ and $\vcol$ if for every $x \in \dom(\Phi)$ (resp. $x \in \dom(\sigma)$), there exists $\{\gamma,\gamma'\} = \{\alpha,\beta\}$ such that one of the following condition is
satisfied:
\begin{enumerate}
\item there exist $v$ and $i = \vcol(x) \in \gamma$ such that
  $u = \TAG{v}{i}\sigma$, $\sigma  \vDash
  \TestTag{\TAG{v}{i}}{i}$, and for all $z \in \fv(v)$, $z \prec x$ and either $\vcol(z) \in \gamma$ or there exists $j$ such that $z = z^{\gamma'}_j$; or
\item there exists $M$ such that $\fv(M)
    \subseteq \dom(\Phi) \cap \{z ~|~ z \prec x\}$, $\fn(M) \cap \Ec = \emptyset$ and
    $M\Phi = u$.
\end{enumerate}
where $u = x\Phi$ (resp. $u = x\sigma$). 
\end{definition}

{In the case of variables instantiated through an output, and or an internal communication, it will be the first item that needs to hold; while in the case of variables intantiated through inputs on public channels it is the second item that needs to hold.} Intuitively, the order $\prec$ on $\dom(\Phi) \cup \dom(\sigma)$ corresponds to the order in which the variables in $\dom(\Phi) \cup \dom(\sigma)$ have been introduced along the execution. In particular, we have that $w_1 \prec w_2 \prec \ldots \prec w_n$ where $\dom(\Phi) = \{w_1,\ldots, w_n\}$. In the following, we sometimes simply say that $\quadruple{\Ec}{\p}{\Phi}{\sigma}$ is a derived well-tagged extended process.







\begin{lemma}
  \label{lem:Flawedandframeelement}
  Let $\quadruple{\Ec}{\p}{\Phi}{\sigma}$ be a derived well-tagged extended process w.r.t $\prec$ and $\vcol$. Let $x \in \dom(\Phi)$ (resp. $x \in \dom(\sigma)$) and $t \in   \Flawed{x\Phi\mydownarrow}$ (resp. $t \in \Flawed{x\sigma\mydownarrow}$). We have that there exists $M$ such that $\fv(M) \subseteq \dom(\Phi) \cap \{z ~|~ z \prec x\}$, $\fn(M) \cap \Ec = \emptyset$ and $t \in \Flawed{M\Phi\mydownarrow}$.
\end{lemma}

\begin{proof}
We prove this result by induction on $\dom(\Phi) \cup \dom(\sigma)$
with the order $\prec$.

\smallskip{}

\noindent \emph{Base case $u = x\sigma$ or $u = x\Phi$ with $x \prec
  z$ for any $z \in \dom(\Phi) \cup \dom(\sigma)$.} Assume $t \in \Flawed{u\mydownarrow}$.
By definition of a derived well-tagged extended process w.r.t $\prec$ and $\vcol$, one of the
following condition is satisfied:
\begin{enumerate}
\item There exist $v$ and $i = \vcol(x)$ such that $u
=\TAG{v}{i}\sigma$, $\sigma \vDash \TestTag{\TAG{v}{i}}{i}$, and $z
\prec x$ for any $z \in \fv(v)$.  Since $u = \TAG{v}{i}\sigma$ and $\sigma \vDash
 \TestTag{\TAG{v}{i}}{i}$, we can apply Lemma~\ref{lem: flawed color and tag term}
 to $v$ and~$\sigma\mydownarrow$. Thus, we have that there exists $z \in
 \fv(\TAG{v}{i})$ such that $t \in
 \Flawed{z\sigma\mydownarrow}$. However, since $x$ is mimimal w.r.t. $\prec$,
we know that $\fv(v) = \emptyset$.   Hence, we obtain a
contradiction. This case is impossible.
\item There exists $M$ such that $\fv(M) \subseteq \dom(\Phi) \cap \{z
  ~|~ z \prec x\}$, $\fn(M) \cap \Ec = \emptyset$, and $M\Phi =
  u$. Thus, we have that $M\Phi\mydownarrow = u\mydownarrow$, and we
  have that $t \in \Flawed{M\Phi\mydownarrow}$.
\end{enumerate}

\medskip{}

\noindent \emph{Inductive case $u = x\sigma$ or $u = x\Phi$.} Assume $t \in \Flawed{u\mydownarrow}$.
By definition of a derived well-tagged extended process w.r.t $\prec$ and $\vcol$, one of the
following condition is satisfied:
\begin{enumerate}
\item There exist $v$ and $i = \vcol(x)$ such that $u
=\TAG{v}{i}\sigma$, $\sigma \vDash \TestTag{\TAG{v}{i}}{i}$, and $z
\prec x$ for any $z \in \fv(v)$.  Since $u = \TAG{v}{i}\sigma$ and $\sigma \vDash
 \TestTag{\TAG{v}{i}}{i}$, we can apply Lemma~\ref{lem: flawed color and tag term}
 to $v$ and~$\sigma\mydownarrow$. Thus, we have that there exists $z \in
 \fv(\TAG{v}{i})$ such that $t \in
 \Flawed{z\sigma\mydownarrow}$, and we have that $z \prec x$.
Hence, we conclude by applying our induction hypothesis.
\item There exists $M$ such that $\fv(M) \subseteq \dom(\Phi) \cap \{z
  ~|~ z \prec x\}$, $\fn(M) \cap \Ec = \emptyset$, and $M\Phi =
  u$. Thus, we have that $M\Phi\mydownarrow = u\mydownarrow$, and we
  have that $t \in \Flawed{M\Phi\mydownarrow}$.
\end{enumerate}
This allows us to conclude.
\end{proof}



\begin{lemma}
  \label{lem:FlawedColor and frame element direct element}
  Let $\quadruple{\Ec}{\p}{\Phi}{\sigma}$ be a derived well-tagged extended process w.r.t $\prec$ and $\vcol$. Let $\{\gamma,\gamma'\} = \{\alpha,\beta\}$. Let $x \in \dom(\Phi)$ (resp. $x \in \dom(\sigma)$) such that $\vcol(x) \in \gamma$. Let $u = x\Phi$ (resp. $u = x\sigma$). Let $t \in \fct_\gamma(u\mydownarrow)$. We have that 
  \begin{itemize}
  \item either there exists $M$ such that $\fv(M) \subseteq \dom(\Phi) \cap \{z ~|~ z \prec x\}$, $\fn(M) \cap \Ec = \emptyset$ and $t \in \fct_\gamma(M\Phi\mydownarrow)$; 
  \item otherwise there exists $j$ such that $z^{\gamma'}_j \prec x$ and $z^{\gamma'}_j\sigma\mydownarrow = t$.
  \end{itemize}
\end{lemma}

\begin{proof}
We prove this result by induction on $\dom(\Phi) \cup \dom(\sigma)$
with the order $\prec$.

\smallskip{}

\noindent \emph{Base case $u = x\sigma$ or $u = x\Phi$ with $x \prec
  z$ for any $z \in \dom(\Phi) \cup \dom(\sigma)$.} Let $t \in \fct_\gamma(u\mydownarrow)$ and $\vcol(x) \in \gamma$ with $\gamma \in \{\alpha,\beta\}$.
By definition of a derived well-tagged extended process w.r.t $\prec$ and $\vcol$, one of the
following condition is satisfied:
\begin{enumerate}
\item There exist $v$ and $i = \vcol(x)$ such that $u
=\TAG{v}{i}\sigma$, $\sigma \vDash \TestTag{\TAG{v}{i}}{i}$, and $z
\prec x$ for any $z \in \fv(v)$. Since $x$ is minimal by $\prec$ then $\fv(v)  = \emptyset$. Hence $u = \TAG{v}{i}$. Thus we deduce that $\fct_\gamma(u\mydownarrow) = \emptyset$. Hence there is a contradiction with $t \in \fct_\gamma(u\mydownarrow)$ and so this condition cannot be satisfied.
\item There exists $M$ such that $\fv(M) \subseteq \dom(\Phi) \cap \{z
  ~|~ z \prec x\}$, $\fn(M) \cap \Ec = \emptyset$, and $M\Phi =
  u$. Thus, we have that $M\Phi\mydownarrow = u\mydownarrow$ and so the result holds.
  \end{enumerate}
  
  \medskip{}
  
\noindent \emph{Inductive case $u = x\sigma$ or $u = x\Phi$.} Assume $t \in \fct_\gamma(u\mydownarrow)$ and $\vcol(x) \in \gamma$.
By definition of a derived well-tagged extended process w.r.t $\prec$ and $\vcol$, one of the
following condition is satisfied:  
\begin{enumerate}
\item There exist $v$ and $i = \vcol(x) \in \gamma$ such that
  $u = \TAG{v}{i}\sigma$, $\sigma  \vDash
  \TestTag{\TAG{v}{i}}{i}$, and for all $z \in \fv(v)$, $z \prec x$ and either $\vcol(z) \in \gamma$ or there exists $j$ such that $z = z^{\gamma'}_j$.
  Since $u = \TAG{v}{i}\sigma$ and $\sigma \vDash
 \TestTag{\TAG{v}{i}}{i}$, we can apply Lemma~\ref{lem: flawed color and tag term 2}
 to $v$ and~$\sigma\mydownarrow$. Thus we have that $t \in \fct_\gamma(\TAG{v}{i}(\sigma\mydownarrow)$. In such a case, it means that there exists $z \in \fv(v)$ with $z \prec x$ such that $t \in \fct_\gamma(z\sigma\mydownarrow)$ and one of the two conditions is satisfied:
 \begin{itemize}
 \item $\vcol(z) \in \gamma$: In such a case, we can apply our inductive hypothesis on $t$ and $z$ and so the result holds.
 \item there exists $j$ such that $z = z^{\gamma'}_j$: Otherwise, we know by hypothesis that $z^{\gamma'}\sigma\mydownarrow \in \N$ or $\fct_\gamma(z^{\gamma'}\sigma\mydownarrow) = \{ z^{\gamma'}\sigma\mydownarrow\}$. Since $t \in \fct_\gamma(z\sigma\mydownarrow)$, we deduce that $z^{\gamma'}\sigma\mydownarrow \not\in \N$ and so $\fct_\gamma(z^{\gamma'}\sigma\mydownarrow) = \{ z^{\gamma'}\sigma\mydownarrow\}$. But this implies that $t = z\sigma\mydownarrow$. Hence the result holds.
 \end{itemize}
\item There exists $M$ such that $\fv(M) \subseteq \dom(\Phi) \cap \{z
  ~|~ z \prec x\}$, $\fn(M) \cap \Ec = \emptyset$, and $M\Phi =
  u$. Thus, we have that $M\Phi\mydownarrow = u\mydownarrow$, and we
  have that $t \in \fct_\gamma(M\Phi\mydownarrow)$.
\end{enumerate}
This allows us to conclude.
\end{proof}



\begin{lemma}
  \label{lem:flawed,smallerrecipe}
  Let $\quadruple{\Ec}{\p}{\Phi}{\sigma}$ be a derived well-tagged extended process. Let $M$ be a term such that $\fn(M) \cap \Ec = \emptyset$ and $\fv(M) \subseteq \dom(\Phi)$. Let $\ffun(t_1, \ldots, t_m) \in \Flawed{M\Phi\mydownarrow}$.  There exists $M_1, \ldots, M_m$ such that $\fv(M_k) \subseteq \dom(\Phi)$, $\fn(M_k) \cap \Ec = \emptyset$, $M_k\Phi\mydownarrow = t_k$, and $\M(M_k) < \M(M)$, for all $k \in \{1, \ldots, m\}$.
\end{lemma}


\begin{proof}
 We prove this result by induction on $\M(M)$.

\smallskip{}


 \noindent\emph{Base case $\M(M) = (j, 1)$:} In this case, either we
 have that $M \in \N$ or $M = w_j$. If $M \in \N$, then we have
 $M\Phi\mydownarrow = M \in \N$ and $\Flawed{M\Phi\mydownarrow} =
 \emptyset$. Thus the result holds. If $M = w_j$
then, by Lemma~\ref{lem:Flawedandframeelement},
$\ffun(t_1, \ldots, t_m) \in
 \Flawed{w_j\Phi\mydownarrow}$ implies that 
there exists $M'$ such that:
\begin{itemize}
\item  $\fv(M') \subseteq
 \{w_1, \ldots, w_{j-1}\}$, 
\item $\fn(M) \cap \Ec = \emptyset$, and 
\item $\ffun(t_1,
 \ldots, t_m) \in \Flawed{M'\Phi\mydownarrow}$.
\end{itemize} 
Since $\M(M') < \M(M)$, thanks to
 our inductive hypothesis, we  deduce that there exist $M_1, \ldots, M_m$
 such that for each $k \in \{1,
 \ldots, m\}$, we have that:
$\fv(M_k) \subseteq \dom(\Phi)$, $\fn(M_k) \cap \Ec =
  \emptyset$,
 $M_k\Phi\mydownarrow = t_k$, and 
$\M(M_k) < \M(M') < \M(M)$.

 \medskip

 \noindent \emph{Inductive step $\M(M) > (j,1)$:} In such a case, we have that $M = \ffun(M_1,
 \ldots, M_n)$. Let $t = \gfun(t_1, \ldots, t_m) \in
 \Flawed{M\Phi\mydownarrow}$. We do a case analysis on $\ffun$.

\smallskip{}

 \emph{Case $\ffun \in \Sigma_i \cup \Sigma_{\Tag_i}$ for some $i \in
   \{1,\ldots,p\}$:} In such a  case, $M\Phi\mydownarrow = \ffun(M_1\Phi\mydownarrow, \ldots,
 M_n\Phi\mydownarrow)\mydownarrow$. By definition, we know that for all $t \in
 \Flawed{M\Phi\mydownarrow}$, we have that $\racine(t) \not\in \Sigma_i \cup
 \Sigma_{\Tag_i}$. Thus, thanks to Lemma~\ref{lem:app_CRicalp05}, we 
 deduce that 
\[
\Flawed{M\Phi\mydownarrow} \subseteq  \Flawed{M_1\Phi\mydownarrow}
\cup \ldots \cup \Flawed{M_n\Phi\mydownarrow}.
\] 
Since $\M(M_k)
 < \M(M)$ for any $k \in \{1,\ldots, n\}$, thanks to our inductive hypothesis, we know that there exists $M'_1,
 \ldots, M'_m$ such that $\fv(M'_j) \subseteq \dom(\Phi)$, $\fn(M'_j) \cap \Ec =
 \emptyset$, $M'_j\Phi\mydownarrow = t_i$ and $\M(M'_j) < \M(M_k) < \M(M)$, for
 $j \in \{1, \ldots, m\}$. Hence the result holds.

\smallskip{}

 \emph{Case $\ffun = \langle\;\rangle$:} In such a case, $M\Phi\mydownarrow =
 \ffun(M_1\Phi\mydownarrow, M_2\Phi\mydownarrow)$. Moreover, we have
 that 
$\Flawed{M\Phi\mydownarrow} = \Flawed{M_1\Phi\mydownarrow} \cup
 \Flawed{M_2\Phi\mydownarrow}$. Since $\M(M_1) < \M(M)$, $\M(M_2) < \M(M)$ and
 $t \in \Flawed{M_1\Phi\mydownarrow} \cup \Flawed{M_2\Phi\mydownarrow}$, we
 conclude by applying our inductive hypothesis on $M_1$ (or $M_2$).

\smallskip{}

 \emph{Case $\ffun \in \{\pk, \vk\}$:} In this case, $M\Phi\mydownarrow =
 \ffun(M_1\Phi\mydownarrow)$ and we have
 that $\Flawed{M\Phi\mydownarrow} = \emptyset$. Hence the result trivially holds.

\smallskip{}
 
 \emph{Case $\ffun \in \{\senc, \aenc, \sign\}$:} In such a case, we
 have that
 $M\Phi\mydownarrow = \ffun(M_1\Phi\mydownarrow, M_2\Phi\mydownarrow)$. We need
 to distinguish whether $\racine(M_1\Phi\mydownarrow) = \Tag_i$ for
 some $i \in \{1,\ldots,p\}$ or not.

 If $\racine(M_1\Phi\mydownarrow) = \Tag_i$ for some $i \in
 \{1,\ldots, p\}$,  then there exists $u_1$ such that $M_1\Phi\mydownarrow =
 \Tag_i(u_1)$. Hence, we have that $\Flawed{M_1\Phi\mydownarrow} =
 \Flawed{u_1}$. We have also that:
\[
 \Flawed{M\Phi\mydownarrow} = \Flawed{u_1} \cup
 \Flawed{M_2\Phi\mydownarrow}.
\]
 We deduce that $t \in \Flawed{M_1\Phi\mydownarrow}$ or $t \in
 \Flawed{M_2\Phi\mydownarrow}$. Since $\M(M_1) < \M(M)$ and $\M(M_2) < \M(M)$,
 we conclude by applying our inductive hypothesis on $M_1$ or $M_2$.

 Otherwise $\racine(M_1\Phi\mydownarrow) \not\in \{\Tag_1, \ldots, \Tag_p\}$. In such a
 case, $\Flawed{M\Phi\mydownarrow} = \Flawed{M_1\Phi\mydownarrow} \cup
 \Flawed{M_2\Phi\mydownarrow} \cup \{M\Phi\mydownarrow\}$. If $t =
 M\Phi\mydownarrow$, we have that $t_1 = M_1\Phi\mydownarrow$, $t_2 =
 M_2\Phi\mydownarrow$ and $\M(M_1) < \M(M)$, $\M(M_2) < \M(M)$. Thus the result
 holds. If $t \in \Flawed{M_1\Phi\mydownarrow} \cup
 \Flawed{M_2\Phi\mydownarrow}$, we conclude by applying our inductive
 hypothesis on $M_1$ or $M_2$.


\smallskip{}

 \emph{Case $\ffun = \h$:} This case is analogous to the previous one and can
 be handled similarly.


\smallskip{}

 \emph{Case $\ffun \in \{\sdec, \adec, \checksign\}$:} In such a case,
 we have to distinguish two cases depending on whether $\ffun$ is reduced in
 $M\Phi\mydownarrow$, or not.

 If $\ffun$ is not reduced, \emph{i.e.} $M\Phi\mydownarrow =
 \ffun(M_1\Phi\mydownarrow, M_2\Phi\mydownarrow)$, then we have that
 \[
\Flawed{M\Phi\mydownarrow} = \{M\Phi\mydownarrow\} \cup
 \Flawed{M_1\Phi\mydownarrow} \cup \Flawed{M_2\Phi\mydownarrow}.
\] 
Thus if $t =
 M\Phi\mydownarrow$, we have that $t_1 = M_1\Phi\mydownarrow$, $t_2 =
 M_2\Phi\mydownarrow$ and $\M(M_1) < \M(M)$, $\M(M_2) < \M(M)$. Thus the result
 holds. Otherwise, we have that $t \in \Flawed{M_1\Phi\mydownarrow}$ or $t \in
 \Flawed{M_2\Phi\mydownarrow}$. Since $\M(M_1) < \M(M)$, $\M(M_2) < \M(M)$, we
 can conclude by applying our inductive hypothesis on~$M_1$ or~$M_2$.

 If $\ffun$ is reduced, then we have that $M_1\Phi\mydownarrow =
 \ffun'(u_1,u_2)$ with $M\Phi\mydownarrow = u_1$ and $\ffun' \in \{\senc,
 \aenc, \sign\}$. If $\racine(u_1) = \Tag_i$ for some $i \in \{1,\ldots,p\}$, then
 we have that there exists $u'_1$ such that $u_1 = \Tag_i(u'_1)$,
 $\Flawed{M\Phi\mydownarrow} = \Flawed{u'_1}$ and $\Flawed{M_1\Phi\mydownarrow}
 = \Flawed{u'_1} \cup \Flawed{u_2}$. Thus, we have that 
$\Flawed{M\Phi\mydownarrow} \subseteq
 \Flawed{M_1\Phi\mydownarrow}$.
Otherwise, if $\racine(u_1) \not\in \{\Tag_1,\ldots,\Tag_p\}$, then we
have that 
\[
\Flawed{M_1\Phi\mydownarrow} =
 \{M_1\Phi\mydownarrow\} \cup \Flawed{u_1} \cup \Flawed{u_2}
\]
 and
 $\Flawed{M\Phi\mydownarrow} = \Flawed{u_1}$. Thus, 
 $\Flawed{M\Phi\mydownarrow} \subseteq \Flawed{M_1\Phi\mydownarrow}$. In
 both cases, we have that $\Flawed{M\Phi\mydownarrow} \subseteq
 \Flawed{M_1\Phi\mydownarrow}$ and since $\M(M_1) < \M(M)$, we can conclude by
 applying our inductive hypothesis on~$M_1$.
\end{proof}


\newcommand{\fctpair}{\fct_{\langle\ \rangle}}

In the following lemma, we will use the factors of the signature only composed of $\langle\ \rangle$, denoted $\fctpair$. Typically, for all terms $u$, for all context built only on $\langle\ \rangle$, for all terms $u_1, \ldots, u_n$, if $u = C[u_1, \ldots, u_n]$ and for all $k \in \{1, \ldots, n\}$, $\racine(u_i) \neq \langle\ \rangle$ then $\fctpair(u) = \{ u_1, \ldots, u_n\}$.




\begin{lemma}
  \label{lem:FlawedColor and frame element direct element 2}
  Let $\quadruple{\Ec}{\p}{\Phi}{\sigma}$ be a derived well-tagged extended process w.r.t $\prec$ and $\vcol$. Assume that for all assignment variables $z$, $\new \Ec. \Phi \not\vdash z\sigma\mydownarrow$. Let $M$ such that $\fv(M) \subseteq \dom(\Phi)$, $\fn(M) \cap \Ec = \emptyset$. For all $\{\gamma,\gamma'\} = \{\alpha,\beta\}$, for all $t \in \fct_\gamma(M\Phi\mydownarrow)$, if $t \not\in \fctpair(M\Phi\mydownarrow)$ and for all assignment variable $z$, for all $w \in \dom(\Phi)$, $z \prec w$ and $\M(w) \leq \M(M)$ implies $z\sigma\mydownarrow \neq t$ then there exists $M'$ such that $\M(M') < \M(M)$, $\fn(M) \cap \Ec = \emptyset$ and $t \in \fctpair(M'\Phi\mydownarrow)$.
\end{lemma}

\begin{proof}
We do a proof by induction on $\M(M)$:

\medskip

\noindent\emph{Base case $\M(M) = (0, 1)$:} In this case, we have that $M \in \N$ which means that $M\Phi\mydownarrow = M \in \N$ and $\fct_\gamma(M\Phi\mydownarrow) = \emptyset$. Thus the result holds. 

\medskip

\noindent\emph{Base case $\M(M) = (j,1)$:} In this case, we have $M = w_j$. Let $\{\gamma,\gamma'\} = \{\alpha,\beta\}$. Let $t \in \fct_\gamma(M\Phi\mydownarrow)$ such that $t \not\in \fctpair(M\Phi\mydownarrow)$. We do a case analysis on $\vcol(w_j)$:

\emph{Case $\vcol(w_j) \in \gamma$:} In this case, since for all assignment variable $z$, for all $w \in \dom(\Phi)$, $z \prec w$ and $\M(w) \leq \M(M)$ implies $z\sigma\mydownarrow \neq t$, than we can deduce that for all assignment variables $z \prec w_j$, $z\sigma\mydownarrow \neq t$. Thus by Lemma~\ref{lem:FlawedColor and frame element direct element}, we obtain that there exists $M'$ such that $\fv(M') \subseteq \dom(\Phi) \cap \{z ~|~ z \prec x\}$, $\fn(M') \cap \Ec = \emptyset$ and $t \in \fct_\gamma(M'\Phi\mydownarrow)$. $\fv(M') \subseteq \dom(\Phi) \cap \{z ~|~ z \prec x\}$ implies that $\M(M') = (k,k')$ with $k < j$ and so $\M(M') < \M(M)$. If $t \in \fctpair(M'\Phi\mydownarrow)$ then the result holds. Otherwise, we can apply our inductive hypothesis on $t$ and $M'$ and so the result holds.
 
 \emph{Case $\vcol(w_j) \in \gamma'$ :} Since $t \not\in \fctpair(M\Phi\mydownarrow)$, we deduce that there exists $u \in \fctpair(M\Phi\mydownarrow)$ s.t. $\racinebis(u) = \gamma$ and $t \in \fct_\gamma(u)$. Note that $\racinebis(u) \not\in \gamma' \cup \{0\}$ otherwise it would contradict the fact that $t \in \fct_\gamma(M\Phi\mydownarrow)$. But $u \in \fct_{\gamma'}(M\Phi\mydownarrow)$. Moreover, $u \in \fctpair(M\Phi\mydownarrow)$ implies that $u$ is deducible in $\new\ \Ec. \Phi$. Thus we deduce that for all assignment variables $z$, $z\sigma\mydownarrow \neq u$. By applying the same proof as case $\vcol(w_j) \in \gamma$, we deduce that there exists $M'$ such that $\fn(M') \cap \Ec = \emptyset$, $\M(M') < \M(M)$ and $u \in \fctpair(M'\Phi\mydownarrow)$. But $t \in \fct_\gamma(u)$, $\racinebis(u) = \gamma$ and $u \in \fctpair(M'\Phi\mydownarrow)$ implies that $t \in \fct_\gamma(M'\Phi\mydownarrow)$ and $t \not\in \fctpair(M'\Phi\mydownarrow)$. Hence we can apply our inductive hypothesis on $M'$ and $t$ which allows us to conclude.

\medskip

 \noindent \emph{Inductive step $\M(M) > (j,1)$:} In such a case, we have that $M = \ffun(M_1,\ldots, M_n)$. Let $t \in \fct_\gamma(M\Phi\mydownarrow)$ such that $t \not\in \fctpair(M\Phi\mydownarrow)$. We do a case analysis on $\ffun$.

\smallskip{}

 \emph{Case $\ffun \in \Sigma_i \cup \Sigma_{\Tag_i}$ for some $i \in \gamma$:} In such a  case, $M\Phi\mydownarrow = \ffun(M_1\Phi\mydownarrow, \ldots,
 M_n\Phi\mydownarrow)\mydownarrow$. By definition, we know that for all $t \in
 \fct_\gamma(M\Phi\mydownarrow)$, we have that $\racine(t) \not\in \Sigma_i \cup
 \Sigma_{\Tag_i}$. Thus, thanks to Lemma~\ref{lem: flawed color and tag term 2}, we 
 deduce that there exists 
\[
\fct_\gamma(M\Phi\mydownarrow) \subseteq  \fct_\gamma(M_1\Phi\mydownarrow)
\cup \ldots \cup \fct_\gamma(M_n\Phi\mydownarrow).
\] 
Thus there exists $k \in \{1, \ldots, n\}$ such that $t \in \fct_\gamma(M_k\Phi\mydownarrow)$. If $t \in \fctpair(M_k\Phi\mydownarrow)$ then the result holds, else we apply our inductive hypothesis on $t$ and $M_k$ and so the result also holds.

\smallskip{}

 \emph{Case $\ffun \in \Sigma_i \cup \Sigma_{\Tag_i}$ for some $i \not\in \gamma$:} In such a  case, $M\Phi\mydownarrow = \ffun(M_1\Phi\mydownarrow, \ldots, M_n\Phi\mydownarrow)\mydownarrow$. We assumed that $t \not\in \fctpair(M\Phi\mydownarrow)$ hence there exists $u \in \fctpair(M\Phi\mydownarrow)$ s.t. $\racinebis(u) = \gamma$ and $t \in \fct_\gamma(u)$. But it also implies that $\racinebis(M\Phi\mydownarrow) \in \gamma \cup \{ 0 \}$. Hence, by applying Lemma~\ref{lem:app_CRicalp05}, we deduce that there exists $k \in \{1, \ldots, n\}$ such that $M\Phi\mydownarrow \in \st(M_k\Phi\mydownarrow)$. Moreover, it also implies that $u \in \fct_\gamma'(M_k\Phi\mydownarrow)$. 
 
 If $u \in \fctpair(M_k\Phi\mydownarrow)$ then we deduce that $\racine(M_k\Phi\mydownarrow) \not\in \gamma'$ and so, by Lemma~\ref{lem:app_CRicalp05}, $M_k\Phi\mydownarrow = M\Phi\mydownarrow$. Since we had $t \not\in \fctpair(M\Phi\mydownarrow)$, then we also have $t \not\in \fctpair(M_k\Phi\mydownarrow)$ and so we conclude by applying our inductive hypothesis on $t$ and $M_k$.
 
 if $u \not\in \fctpair(M_k\Phi\mydownarrow)$ then we can apply our inductive hypothesis on $u, \gamma'$ and $M_k$. Indeed, since $u \in \fctpair(M\Phi\mydownarrow)$, then $u$ is deducible in $\new\ \Ec. \Phi$ and so we deduce that for all assignment variable $z$, $z\sigma\mydownarrow \neq u$. Hence we obtain that there exists $M'$ such that $\M(M') < \M(M_k)$, $\fn(M) \cap \Ec = \emptyset$ and $u \in \fctpair(M'\Phi\mydownarrow)$. But $t \in \fct_\gamma(u)$ and $u \in \fctpair(M'\Phi\mydownarrow)$. Hence we deduce that $t \in \fct_\gamma(M'\Phi\mydownarrow)$ and $t \not\in \fctpair(M'\Phi\mydownarrow)$. We conclude by applying once again our inductive hypothesis but on $t, \gamma$ and $M'$.
 
 \smallskip{}

 \emph{Case $\ffun = \langle\;\rangle$:} In such a case, $M\Phi\mydownarrow =
 \ffun(M_1\Phi\mydownarrow, M_2\Phi\mydownarrow)$. Moreover, we have
 that 
$\fct_\gamma(M\Phi\mydownarrow) = \fct_\gamma(M_1\Phi\mydownarrow) \cup
 \fct_\gamma(M_2\Phi\mydownarrow)$. Since $\M(M_1) < \M(M)$, $\M(M_2) < \M(M)$ and
 $t \in \fct_\gamma(M_1\Phi\mydownarrow) \cup \fct_\gamma(M_2\Phi\mydownarrow)$, we
 conclude by applying our inductive hypothesis on $t$ and $M_1$ (or $M_2$).

\smallskip{}

 \emph{Case $\ffun \in \{\pk, \vk\}$:} In this case, $M\Phi\mydownarrow =
 \ffun(M_1\Phi\mydownarrow)$ and we have
 that $\fct_\gamma{M\Phi\mydownarrow} = \emptyset$. Hence the result trivially holds.

\smallskip{}
 
 \emph{Case $\ffun \in \{\senc, \aenc, \sign\}$:} In such a case, we
 have that
 $M\Phi\mydownarrow = \ffun(M_1\Phi\mydownarrow, M_2\Phi\mydownarrow)$. We need
 to distinguish whether $\racine(M_1\Phi\mydownarrow) = \Tag_i$ for
 some $i \in \{1,\ldots,p\}$ or not.

 If $\racine(M_1\Phi\mydownarrow) = \Tag_i$ for some $i \in
 \{1,\ldots, p\}$,  then there exists $u_1$ such that $M_1\Phi\mydownarrow =
 \Tag_i(u_1)$. Assume first that $i \in \gamma'$. In such a case $\fct_\gamma(M\Phi\mydownarrow) = \{ \fct_\gamma(M\Phi\mydownarrow) \}$ and $\fctpair(M\Phi\mydownarrow) = \{ \fctpair(M\Phi\mydownarrow) \}$. Hence it contradicts the fact that $t \not\in \fctpair(M\Phi\mydownarrow)$. We can thus deduce that $i \in \gamma$. But in such a case, we have that $\fct_\gamma(M_1\Phi\mydownarrow) = \fct_\gamma(u_1)$ and:
\[
 \fct_\gamma(M\Phi\mydownarrow) = \fct_\gamma(u_1) \cup
 \fct_\gamma(M_2\Phi\mydownarrow).
\]
 We deduce that $t \in \fct_\gamma(M_1\Phi\mydownarrow)$ or $t \in
 \fct_\gamma(M_2\Phi\mydownarrow)$. Since $\M(M_1) < \M(M)$ and $\M(M_2) < \M(M)$,
 we conclude by applying our inductive hypothesis on $M_1$ or $M_2$.

 Otherwise $\racine(M_1\Phi\mydownarrow) \not\in \{\Tag_1, \ldots, \Tag_p\}$. In such a
 case, $\fct_\gamma(M\Phi\mydownarrow) = \{M\Phi\mydownarrow\}$ and $\fctpair(M\Phi\mydownarrow) = \{ M\Phi\mydownarrow\}$. But we assume that $t \not\in \fctpair(M\Phi\mydownarrow)$ hence this case is impossible.

\smallskip{}

 \emph{Case $\ffun = \h$:} This case is analogous to the previous one and can
 be handled similarly.


\smallskip{}

 \emph{Case $\ffun \in \{\sdec, \adec, \checksign\}$:} In such a case,
 we have to distinguish two cases depending on whether $\ffun$ is reduced in
 $M\Phi\mydownarrow$, or not.

 If $\ffun$ is not reduced, \emph{i.e.} $M\Phi\mydownarrow =
 \ffun(M_1\Phi\mydownarrow, M_2\Phi\mydownarrow)$, then we have that
 \[
\fct_\gamma(M\Phi\mydownarrow) = \{M\Phi\mydownarrow\}.
\] 
Once again this is in contradiction with our hypothesis that $t \not\in \fctpair(M\Phi\mydownarrow)$.

We now focus on the case where $\ffun$ is reduced: we have that $M_1\Phi\mydownarrow =
 \ffun'(u_1,u_2)$ with $M\Phi\mydownarrow = u_1$ and $\ffun' \in \{\senc,
 \aenc, \sign\}$. We have to do a case analysis on $\racine(u_1)$:
 \begin{itemize}
 \item if $\racine(u_1) = \Tag_i$ for some $i \in \gamma$. In such a case, there exists $u'_1$ such that $u_1 = \Tag_i(u'_1)$, $\fct_\gamma(M\Phi\mydownarrow) = \fct_\gamma(u'_1)$ and $\fct_\gamma(M_1\Phi\mydownarrow) = \fct_\gamma(u'_1) \cup \fct_\gamma(u_2)$. Thus we deduce that $\fct_\gamma(M\Phi\mydownarrow) \subseteq \fct_\gamma(M_1\Phi\mydownarrow)$. We can conclude thanks to our inductive hypothesis on $t$ and $M_1$. 
\item if $\racine(u_1) = \Tag_i$ for some $i \not\in \gamma$. In such a case, $\fct_\gamma(M\Phi\mydownarrow) = \{ M\Phi\mydownarrow)$ which contradicts the hypothesis $t \not\in \fctpair(M\Phi\mydownarrow)$.
\item otherwise, $\racine(u_1) \not\in \{\Tag_1,\ldots,\Tag_p\}$, then we
have that $\ffun'(u_1,u_2) \in \Flawed{M_1\Phi\mydownarrow}$. By Lemma~\ref{lem:flawed,smallerrecipe}, we deduce that there exists $M'$ such that $\M(M') < \M(M_1)$, $\fn(M') \cap \Ec = \emptyset$ and $M'\Phi\mydownarrow = u_1$. Since $u_1 = M\Phi\mydownarrow$ and $\M(M') < \M(M)$ then we can apply our inductive hypothesis on $t, \alpha$ and $M'$ and so the result holds.
\end{itemize}
\end{proof}



\begin{lemma}
  \label{lem : deductibily of fct_C}
 {Let $A = \quadruple{\Ec}{\p}{\Phi}{\sigma}$ be a derived well-tagged process, and let $(\rho_\alpha, \rho_\beta)$ be compatible with $A$.} Let $u$ be a ground term in normal form that do not use names in $\Ec_\alpha \uplus \Ec_\beta$. We have that there exists a context $C$ (possibly a hole) built only using $\langle\; \rangle$, and terms $u_1, \ldots, u_m$ such that $u = C[u_1, \ldots, u_m]$, and for all $i \in \{1, \ldots, m\}$,
  \begin{itemize}
  \item either $u_i \in \Flawed{u}$;
  \item or $u_i \in \fct_{\Sigmazero}(u)$ and $\delta_\alpha(u_i) =
   \delta_\beta(u_i)$,
  \item or $u_i = \ffun(n)$ for some $\ffun \in \{\pk, \vk\}$ and $n \in \N$,
  \item or $u_i \in \dom(\rho^+_\alpha) \cup \dom(\rho^+_\beta)$.
  \end{itemize}
\end{lemma}

\begin{proof}
 Let $u$ a ground term in normal form and let $\{v_1, \ldots, v_n\} =
 \fct_{\Sigmazero}(u)$. Thus there exists a context $D$ (possibly a hole) built
 on $\Sigmazero$ such that $u = D[v_1, \ldots, v_n]$. We now prove the result by
 induction on $|D|$.

 \medskip

 \noindent \emph{Base case $|D| = 0$:} 
We show that the result holds and in such a case the context $C$ is
reduced to a hole.
Since $|D| = 0$, we know that $\fct_{\Sigmazero}(u) = u$ and so
 either $\racinebis(u) = i$ with $i \in \{1,\ldots,p\}$ or
 $\racinebis(u) = \bot$. 
If $u \in \dom(\rho^+_\alpha) \cup \dom(\rho^+_\beta)$, then the
result trivially holds. Otherwise, we have 
that $\delta_\alpha(u) = \delta_\beta(u)$ by definition of
$\delta_\alpha$ and $\delta_\beta$. Hence the result holds.


 \medskip

 \noindent \emph{Inductive step $|D| > 0$:} There exists $\ffun \in
 \Sigmazero$, and $v_1, \ldots, v_k$ such that $u = \ffun(u_1, \ldots,
 u_k)$. We do a case analysis on $\ffun$.

\smallskip{}

 \emph{Case $\ffun = \langle\;\rangle$:} In such a case, there exist
 two contexts $D_1,
 D_2$  (possibly holes) built on $\Sigmazero$ such that:
\begin{itemize}
\item  $D =
 \pair{D_1}{D_2}$ with $|D_1|, |D_2| < |D|$, 
\item $u_1 = D_1[v^1_1, \ldots, v^1_{n_1}]$ and $\{v^1_1, \ldots,
 v^1_{n_1}\} = \fct_{\Sigmazero}(u_1)$,
\item $u_2 = D_1[v^2_1, \ldots, v^2_{n_1}]$ and $\{v^2_1, \ldots,
 v^2_{n_2}\} = \fct_{\Sigmazero}(u_2)$
\end{itemize}
By applying our inductive hypothesis on $u_1$ and $u_2$, we know that
there exist two contexts $C_1$ and $C_2$.
Since 
\begin{itemize}
\item $\Flawed{u} =
   \Flawed{u_1} \cup \Flawed{u_2}$, and 
\item  $\fct_{\Sigmazero}(u) =
   \fct_{\Sigmazero}(u_1) \uplus \fct_{\Sigmazero}(u_2)$, 
\end{itemize}
we conclude
   that $C = \langle C_1,C_2 \rangle$ satisfies all the conditions
   stated in the lemma.
  

\smallskip{}

 \emph{Case $\ffun \in \{\pk, \vk\}$ and $u = \ffun(n)$ for some $n \in N$:}
 The result trivially hold by choosing the context $C$ to be a hole.

\smallskip{}

Otherwise, we have that 
\[
\Flawed{u} = \{u\} \cup \Flawed{u_1} \cup \ldots \cup 
\Flawed{u_k}.
\] 
Since $u \in \Flawed{u}$, we can choose $C$ to be the context reduced
to a hole. 
The result trivially holds.
\end{proof}




\begin{lemma}
\label{lem:samerecipesymmetric}
Let $A = \quadruple{\Ec}{\p}{\Phi}{\sigma}$ be a derived well-tagged extended process, and let $(\rho_\alpha, \rho_\beta)$ be compatible with $A$.
Let  $M$ be a term such that $\fv(M) \subseteq \dom(\Phi)$ and $\fn(M) \cap \Ec =
\emptyset$. We assume that 
$\Ec =  \Ec_0 \uplus \Ec_\alpha
\uplus \Ec_\beta$, 
 $\fn(\Phi) \cap (\Ec_\alpha \uplus \Ec_\beta) = \emptyset$, and
one of the two following conditions
is satisfied: 
\begin{enumerate}
\item  $\new\ \Ec. \Phi \not\vdash k$ for any $k \in K_S$; or
\item  $\new\ \Ec. \delta(\Phi\mydownarrow) \not\vdash k$ for any {$k \in \delta_\alpha(K_S) \cup \delta_\beta(K_S)$.} 
\end{enumerate}
with $K_S = \{t, \pk(t),\vk(t) ~|~  \mbox{$t$ ground}, t \in \dom(\rho^+_\alpha) \cup
\dom(\rho^+_\beta)\}$. 
We have that $\delta_\gamma(M\Phi\mydownarrow) =
M\delta(\Phi\mydownarrow)\mydownarrow$ with $\gamma \in \{\alpha,\beta\}$.
\end{lemma}

\begin{proof}
Let $\Phi\mydownarrow =\{w_1 \refer u_1,\ldots, w_n \refer u_n\}$.
We prove this result by induction on $\M(M)$:

\medskip

\noindent\emph{Base case $\M(M) = (0,0)$:} There exists no term $M$ such that $|M| = 0$, thus the result holds.

\medskip

\noindent\emph{Inductive step $\M(M) > (0,0)$:} We first prove there exists $\gamma
\in \{\alpha, \beta\}$ such that $\delta_{\gamma}(M\Phi\mydownarrow) =
M\delta(\Phi\mydownarrow)\mydownarrow$ and then we show that
$\delta_\alpha(M\Phi\mydownarrow) = \delta_\beta(M\Phi\mydownarrow)$.

\medskip{}

Assume first that $|M| = 1$, \emph{i.e.} either $M\in\N$ or there exists $j \in \{1,
\ldots, n\}$ such that $M = w_j$. 

\noindent \emph{Case $M\in \N$.} In such a case, we have that
$M\Phi\mydownarrow = M$, and $M \not\in \Ec$. Hence, we have that 
$\new \ \Ec. \Phi \vdash M$ and also that $\new\
\Ec. \delta(\Phi\mydownarrow) \vdash M$. {In case condition $1$ is
satisfied, we easily deduce that $M \not\in K_S$. Otherwise, we know
that the condition $2$ is satisfied, and thus $M \not\in
\delta_\alpha(K_S) \cup \delta_\beta(K_S)$. Again, we want to conclude
that $M \not\in K_S$. Assume that this is not the case, \emph{i.e.} $M \in
K_S$. This means that $M$ is a name in $\dom(\rho^+_\alpha)$ (or
$\dom(\rho^+_\beta)$). Hence, we  have that $\delta_\beta(M) \in
\delta_\beta(K_S)$, and {$\delta_\beta(M) = M$}. Hence, we
deduce that $M \in \delta_\beta(K_S)$, and this leads to a
contradiction, since in such a case, by hypothesis $M$ can not be deducible from $\new\
\Ec. \delta(\Phi\mydownarrow)$.} 
Thus, in any case, we have that  $M \not\in K_S$, and
thus $M \not\in \dom(\rho^+_\alpha) \cup \dom(\rho^+_\beta)$.
Hence, we have that 
$\delta_\gamma(M\Phi\mydownarrow) =
\delta_\gamma(M) = M = M\delta(\Phi\mydownarrow)\mydownarrow$ for any $\gamma\in\{\alpha, \beta\}$.

\noindent \emph{Case $M = w_j$ for some $j \in \{1,\ldots,n\}$.} 
We know that $w_j$ is colored with $\gamma \in \{\alpha,\beta\}$. Hence, we have
that  $w_j\delta(\Phi\mydownarrow) =
\delta_\gamma(w_j\Phi\mydownarrow)$. Since $u_j$ is in normal form, then by
Lemma~\ref{lem:deltaandnormalform}, 
we know that $\delta_\gamma(w_j\Phi)$ is also in
normal form. Thus, we have that $\delta_\gamma(M\Phi\mydownarrow) =
M\delta(\Phi\mydownarrow)\mydownarrow$.

\medskip{}

Otherwise, if $|M| > 1$, then there exists a symbol $\ffun$ and $M_1, \ldots,
M_n$ such that $M = \ffun(M_1, \ldots, M_n)$. We do a case analysis on $\ffun$.

\smallskip{}

\emph{Case $\ffun \in \Sigma_i\cup \Sigma_{\Tag_i}$ with $i \in \{ 1,
  \ldots, p\}$.} Consider $\gamma \in \{\alpha,\beta\}$ such that $i \in \gamma$. In such a case, let $t = \ffun(M_1\Phi\mydownarrow, \ldots,
M_n\Phi\mydownarrow)$. Since $\ffun \in \Sigma_i$ (resp. $\Sigma_{\Tag_i}$), then there exists a
context $C$ built upon $\Sigma_i$ (resp. $\Sigma_{\Tag_i}$) such that $t = C[u_1, \ldots, u_m]$ and
$u_1, \ldots, u_m$ are factor of $t$ in normal form. By
Lemma~\ref{lem:app_CRicalp05}, we know that there exists a context $D$ (possibly
a hole) over $\Sigma_i$ (resp. $\Sigma_{\Tag_i}$) such that $t\mydownarrow = D[u_{i_1}, \ldots,
u_{i_k}]$ with $i_1, \ldots, i_k \in \{0, \ldots, m\}$ and $u_0 = n_{min}$. But
thanks to Lemma~\ref{lem:change alien}, ~\ref{lem:transfoV} and~\ref{lem:deltaandnormalform}, we also that $C[\delta_\gamma(u_1), \ldots,
\delta_\gamma(u_m)]\mydownarrow = D[\delta_\gamma(u_{i_1}), \ldots,
\delta_\gamma(u_{i_k})]$. But $C$ and $D$ are both built on $\Sigma_i$ (resp. $\Sigma_{\Tag_i}$), thus by definition of $\delta_\gamma$, we have that
$\delta_\gamma(t)\mydownarrow = C[\delta_\gamma(u_1), \ldots,
\delta_\gamma(u_m)]\mydownarrow$ and $\delta_\gamma(t\mydownarrow) =
D[\delta_\gamma(u_{i_1}), \ldots, \delta_\gamma(u_{i_k})]$. Hence, the
equality, $\delta_\gamma(t\mydownarrow) = \delta_\gamma(t)\mydownarrow$,
holds. But $t\mydownarrow = M\Phi\mydownarrow$ which means that
$\delta_\gamma(M\Phi\mydownarrow) = \delta_\gamma(t)\mydownarrow$.
We have that:
 \[
\begin{array}{rcl}
\delta_\gamma(t)\mydownarrow &=&
\delta_\gamma(\ffun(M_1\Phi\mydownarrow, \ldots,
M_n\Phi\mydownarrow))\mydownarrow \\ &=&
\ffun(\delta_\gamma(M_1\Phi\mydownarrow), \ldots,
\delta_\gamma(M_n\Phi\mydownarrow))\mydownarrow
\end{array}
\]
Since $\M(M_1) <
\M(M)$, \ldots, $\M(M_n) < \M(M)$, we can apply our inductive hypothesis
on $M_1, \ldots, M_n$. This gives us $\delta_\gamma(t)\mydownarrow =
\ffun(M_1\delta(\Phi\mydownarrow)\mydownarrow, \ldots,
M_n\delta(\Phi\mydownarrow)\mydownarrow)\mydownarrow =  \ffun(M_1,\ldots,
M_n)\delta(\Phi\mydownarrow)\mydownarrow$. Thus we can conclude that
$\delta_\gamma(M\Phi\mydownarrow) = \delta_\gamma(t)\mydownarrow = M\delta(\Phi\mydownarrow)\mydownarrow$.

\smallskip{}

\emph{Case $\ffun \in \Sigmazero \smallsetminus \{\sdec, \adec, \checksign\}$:} In
this case, we have that $M\Phi\mydownarrow = \ffun(M_1\Phi\mydownarrow,\allowbreak \ldots,
M_n\Phi\mydownarrow)$. By applying our inductive hypothesis on $M_1, \ldots,
M_n$, we have that 
\begin{center}
$\delta_\alpha(M_k\Phi\mydownarrow) =
\delta_\beta(M_k\Phi\mydownarrow)$, for all $k \in \{1, \ldots,
n\}$. 
\end{center}
Thus we
have that $\delta_\gamma(M\Phi\mydownarrow) =
\ffun(\delta_{\gamma'}(M_1\Phi\mydownarrow), \ldots,
\delta_{\gamma'}(M_n\Phi\mydownarrow))$ with $\gamma,
\gamma'\in\{\alpha, \beta\}$.
Applying our inductive hypothesis on $M_1, \ldots, M_n$, we
deduce that 
\[\delta_\gamma(M\Phi\mydownarrow) =
\ffun(M_1\delta(\Phi\mydownarrow)\mydownarrow, \ldots,
M_n\delta(\Phi\mydownarrow)\mydownarrow) = M\delta(\Phi\mydownarrow)\mydownarrow.
\]


\smallskip{}

\emph{Case $\ffun \in \{\sdec, \adec, \checksign\}$:} If we first assume that
the root occurence $\ffun$ is not reduced in $M\Phi\mydownarrow$ then the proof
is similar to the previous case. Thus, we focus on the case where the root
occurence of $\ffun$ is reduced, and we consider the case where $\ffun
= \sdec$. The other cases can be done in a similar way.
In such a situation, we know that there exist $v_1, v_2$ such that
$M_1\Phi\mydownarrow = \senc(v_1,v_2)$, $M_2\Phi\mydownarrow = v_2$ and
 $M\Phi\mydownarrow = v_1$. According to the definition of $\delta_\gamma$, we
 know that there exists $\gamma \in \{\alpha,\beta\}$ such that
 $\delta_{\gamma}(\senc(v_1,v_2)) = \senc(\delta_{\gamma}(v_1),
 \delta_{\gamma}(v_2))$. For such $\gamma$, we have that
 $\sdec(\delta_{\gamma}(M_1\Phi\mydownarrow),
 \delta_\gamma(M_2\Phi\mydownarrow))\mydownarrow =
 \delta_\gamma(M\Phi\mydownarrow)$. But by applying our inductive hypothesis on
 $M_1$ and $M_2$, we obtain $\delta_\gamma(M\Phi\mydownarrow) =
 \sdec(M_1\delta(\Phi\mydownarrow)\mydownarrow,
 M_2\delta(\Phi\mydownarrow)\mydownarrow)\mydownarrow =
 M\delta(\Phi\mydownarrow)\mydownarrow$.

\bigskip{}

It remains to prove that $\delta_\alpha(M\Phi\mydownarrow) =
\delta_\beta(M\Phi\mydownarrow)$. We have shown that there exists $\gamma_0 \in \{\alpha,\beta\}$
such that $\delta_{\gamma_0}(M\Phi\mydownarrow) =
M\delta(\Phi\mydownarrow)\mydownarrow$. Thanks to Lemma~\ref{lem : deductibily of fct_C},
we know that there exists a context $C$ built over $\{ \langle\rangle\}$, and
$v_1, \ldots, v_m$ terms such that $M\Phi\mydownarrow = C[v_1, \ldots, v_m]$ and
for all $i \in \{1, \ldots, m\}$:
\begin{itemize}
\item either $v_i \in \Flawed{M\Phi\mydownarrow}$
\item or $v_i \in \fct_{\Sigmazero}(M\Phi\mydownarrow)$ and $\delta_\alpha(v_i) = \delta_\beta(v_i)$.
\item or $v_i = \ffun(n)$ for some $\ffun \in \{\pk, \vk\}$ and $n \in \N$,
\item or $v_i \in \dom(\rho^+_\alpha) \cup \dom(\rho^+_\beta)$.
\end{itemize}

Note that $C$ being built upon $\{ \langle \rangle \}$ means that
$v_i$ is deducible in $\new\ \Ec. \Phi$ for all $i \in \{1, \ldots, m\}$. Furthermore,
since $C[v_1, \ldots, v_m]$ is in normal form,
\[
\delta_{\gamma_0}(M\Phi\mydownarrow) = C[\delta_{\gamma_0}(v_1),\ldots,
\delta_{\gamma_0}(v_m)].
\]
But we have shown that
$\delta_{\gamma_0}(M\Phi\mydownarrow) = M\delta(\Phi\mydownarrow)\mydownarrow$, thus
$\delta_{\gamma_0}(v_i)$ is deducible from $\delta(\Phi\mydownarrow)$, for all $i \in
\{1, \ldots, m\}$. Now, we distinguish several cases depending on
which condition is fullfilled by $v_i$.
\smallskip{}

\emph{Case $v_i \in \Flawed{M\Phi\mydownarrow}$:} There exists $w_1, \ldots,
w_\ell$ terms and a function symbol $\ffun$ such that $v_i = \ffun(w_1, \ldots,
w_\ell)$. By Lemma~\ref{lem:flawed,smallerrecipe}, there exists $N_1,
\ldots, N_\ell$ such that for all $k \in \{1, \ldots, \ell\}$, $\M(N_k) < \M(M)$
and $N_k\Phi\mydownarrow = w_k$. Hence, by applying inductive hypothesis on
$N_1, \ldots, N_\ell$, we obtain that $\delta_\alpha(N_k\Phi\mydownarrow) =
\delta_\beta(N_k\Phi\mydownarrow)$, for all $k \in \{1, \ldots, \ell\}$. Thus,
thanks to $v_i$ being in normal form, we can conclude that $\delta_\alpha(v_i) =
\delta_\beta(v_i)$.

\smallskip{}

\emph{Case $v_i \in \fct_{\Sigmazero}(M\Phi\mydownarrow)$:} In
  such a case, we have that 
$\delta_\alpha(v_i) = \delta_\beta(v_i)$. Hence, we easily conclude.

\smallskip{}

\emph{Case $v_i = \ffun(n)$ for some $\ffun \in \{\pk, \vk\}$ and $n \in \N$:} {By
hypothesis, we know that either $\new\ \Ec. \Phi \not\vdash k$,
for all $k \in K_S$; or $\new\ \Ec. \delta(\Phi\mydownarrow) \not\vdash k$, for
all $k \in \delta_\alpha(K_S) \cup \delta_\beta(K_S)$.
Since we have shown that $v_i$ is deducible from $\new\ \Ec. \Phi$
and $\delta_{\gamma_0}(v_i)$ is deducible from $\new\ \Ec. \delta(\Phi\mydownarrow)$, both
hypotheses imply that $n \not\in \dom(\rho^+_\alpha) \cup
\dom(\rho^+_\beta)$, and so $\delta_\alpha(v_i) =
\delta_\beta(v_i)$.}

\smallskip{}

\emph{Case $v_i \in \dom(\rho^+_\alpha) \cup \dom(\rho^+_\beta)$:} 
{By hypothesis, we know that either $\new\ \Ec. \Phi \not\vdash k$,
for all $k \in K_S$; or $\new\ \Ec. \delta(\Phi\mydownarrow) \not\vdash k$, for
all $k \in \delta_\alpha(K_S) \cup \delta_\beta(K_S)$.
Since we have shown that $v_i$ is deducible from $\new\ \Ec. \Phi$
and $\delta_{\gamma_0}(v_i)$ is deducible from $\new\ \Ec. \delta(\Phi\mydownarrow)$, both
hypotheses imply that $v_i \not\in \dom(\rho^+_\alpha) \cup
\dom(\rho^+_\beta)$ and lead us to a contradiction.}
\end{proof}



\begin{corollary}
\label{cor:deltaandkeyhidden}
Let $A = \quadruple{\Ec}{\p}{\Phi}{\sigma}$ be a derived well-tagged extended
process and let $(\rho_\alpha, \rho_\beta)$ be compatible with $A$, such that $\Ec = \Ec_0 \uplus \Ec_\alpha \uplus \Ec_\beta$, and
$\fn(\Phi) \cap (\Ec_\alpha \uplus \Ec_\beta) = \emptyset$.  The two following
conditions are equivalent:
\begin{enumerate}
\item  $\new\ \Ec. \Phi \not\vdash k$ for any $k \in K_S$; or
\item  $\new\ \Ec. \delta(\Phi\mydownarrow) \not\vdash k$ for any {$k \in \delta_\alpha(K_S) \cup \delta_\beta(K_S)$.} 
\end{enumerate}
with 
$K_S = \{t, \pk(t),\vk(t) ~|~ t \in \dom(\rho^+_\alpha) \cup
\dom(\rho^+_\beta),  \mbox{ $t$ ground}\}$. 
\end{corollary}

\smallskip{}

\begin{proof}
We prove the two implications separately.

\noindent $(2) \Rightarrow (1)$: Let $k \in K_S$ such that $\new\
\Ec. \Phi \vdash k$. In such a case, there exists $M$ such that
$\fv(M) \subseteq \dom(\Phi)$, $\fn(M) \cap \Ec = \emptyset$, and
$M\Phi\mydownarrow = k\mydownarrow$. We assume w.l.o.g. that $k \in
\{t, \pk(t),\vk(t) ~|~ t \in \dom(\rho^+_\alpha) \mbox{ and $t$
  ground}\}$. Let $\gamma \in \{\alpha,\beta\}$. 
By Lemma~\ref{lem:transfoV}, we
have that {$\delta_\gamma(M\Phi\mydownarrow) =
\delta_\gamma(k\mydownarrow)$.}
Thanks to Lemma~\ref{lem:samerecipesymmetric}, we have that
{$\delta_\gamma(M\Phi\mydownarrow) = M\delta(\Phi\mydownarrow)\mydownarrow$}, and by
Definition of {$\delta_\gamma$}, we have that
{$\delta_\gamma(k\mydownarrow) \in \delta_\gamma(K_S)$}. Thus, we deduce that there
exists {$k' \in \delta_\gamma(K_S)$} such that $\new\ \Ec. \delta(\Phi\mydownarrow) \vdash
k'$.

\smallskip{}

\noindent $(1) \Rightarrow (2)$:{Let $k \in \delta_\gamma(K_S)$ with $\gamma \in \{\alpha,\beta\}$}, and $M$ be a term such that $\fv(M) \subseteq \dom(\Phi)$, $\fn(M) \cap \Ec = \emptyset$, and $M\delta(\Phi\mydownarrow)\mydownarrow = k\mydownarrow$. {$k \in \delta_\gamma(K_S)$ implies the existence of $k' \in K_S$ such that $k =\delta_\gamma(k')$, and thus such that $M\delta(\Phi\mydownarrow)\mydownarrow = \delta_\gamma(k')\mydownarrow$. Thanks to Lemma~\ref{lem:samerecipesymmetric}, we have that $\delta_\gamma(M\Phi\mydownarrow)\mydownarrow = \delta_\gamma(k')\mydownarrow$. Now, if $k'\in K_S$ there must exist $k''\in dom(\rho_{\gamma'})$ such that either $k' = k''$, or $k' = \pk(k'')$, or $k'=\vk(k'')$. In any case, because $\rho_{\gamma'}$ is in normal form, we know that $k''\mydownarrow = k''$ and thus that $k'\mydownarrow = k'$. Hence $M\delta(\Phi\mydownarrow)\mydownarrow = \delta_\gamma(k'\mydownarrow)\mydownarrow$. But, then according to Lemma~\ref{lem:deltaandnormalform}, $\delta_\gamma(M\Phi\mydownarrow) = \delta_\gamma(M\Phi\mydownarrow)\mydownarrow = \delta_\gamma(k'\mydownarrow)\mydownarrow = \delta_\gamma(k'\mydownarrow)$. Finally, thanks to Lemma~\ref{lem:transfoV} we can derive that $M\Phi\mydownarrow = k'\mydownarrow = k'$.} This implies that $M\Phi\mydownarrow \in K_S$, and thus there is a term in $K_S$ that is deducible from $\new\, \Ec. \Phi$.
\end{proof}



\begin{corollary}
\label{cor:framestatequiv}
Let $A = \quadruple{\Ec}{\p}{\Phi}{\sigma}$ be a derived extended process and let $(\rho_\alpha, \rho_\beta)$ be compatible with $A$
such
that $\Ec =  \Ec_0 \uplus \Ec_\alpha
\uplus \Ec_\beta$, 
 $\fn(\Phi) \cap (\Ec_\alpha \uplus \Ec_\beta) = \emptyset$, 
and
$\new\ \Ec. \Phi \not\vdash k$ for 
any $k \in K_S$.
We have that
$\new\; \Ec. \Phi \sim
  \new\; \Ec. \delta(\Phi\mydownarrow)$.
\end{corollary}

\begin{proof}
 The proof directly follows from Lemmas~\ref{lem:transfoV}
 and~\ref{lem:samerecipesymmetric}. 
Indeed, $M\Phi\mydownarrow = N\Phi\mydownarrow$ is
 equivalent to $\delta_\gamma(M\Phi\mydownarrow) =
 \delta_\gamma(N\Phi\mydownarrow)$ (thanks to Lemma~\ref{lem:transfoV}), which
 is equivalent to $M\delta(\Phi\mydownarrow)\mydownarrow =
 N\delta(\Phi\mydownarrow)\mydownarrow$ (thanks to Lemma~\ref{lem:samerecipesymmetric}).
\end{proof}

