
\documentclass[a4paper,USenglish,numberwithinsect]{lipics}

\usepackage{microtype}


\usepackage{amssymb}

\usepackage{graphicx}

\usepackage{amsmath}

\usepackage{color}





\newcommand{\A}{\mathcal{A}}
\newcommand{\C}{\mathcal{C}}
\newcommand{\R}{\mathcal{R}}
\newcommand{\Oe}{\mathcal{O}}
\newcommand{\G}{\mathcal{G}}
\newcommand{\M}{\mathcal{M}}
\newcommand{\Pe}{\mathcal{P}}
\theoremstyle{plain}
\newtheorem{proposition}[theorem]{Proposition}


\title{Almost Tight Bounds for  Conflict-Free Chromatic Guarding of Orthogonal Art Galleries}
\titlerunning{Bounds for Conflict-Free Guarding}
\author[1]{Frank Hoffmann}
\author[1]{Klaus Kriegel}
\author[1]{Max Willert}
\affil[1]{Freie Universit\"at Berlin, Institut f\"ur Informatik, 14195 Berlin, Germany\\
  \texttt{\{hoffmann,kriegel,willerma\}@mi.fu-berlin.de}}
\authorrunning{F.\ Hoffmann, K.\ Kriegel, and M.\ Willert}
\Copyright{Frank Hoffmann, Klaus Kriegel, and Max Willert}
\subjclass{F.2.2 Nonnumerical Algorithms, G.2.2 Graph Theory}
\keywords{Orthogonal polygons, art gallery problem, 
hypergraph coloring}

\begin{document}

\maketitle

\begin{abstract}
\ \ \ \ We address recently proposed chromatic versions of the
classic Art Gallery Problem. Assume a 
simple  polygon  is guarded by a finite set of point guards and each
guard is assigned one of  colors. Such a chromatic guarding is said to be
conflict-free if each point  sees at least one guard with a
unique color among all guards visible from . The goal is to establish
bounds on the function  of the number of colors sufficient
to guarantee the existence of a conflict-free chromatic guarding for any
-vertex polygon.

 B\"artschi and Suri showed   (Algorithmica,
2014) for simple orthogonal polygons and the same bound applies to
general simple polygons (B\"artschi et al., SoCG 2014).\\
In this paper, we assume the r-visibility model
instead of standard line visibility. Points  and  in an orthogonal polygon  are r-visible to
each other if the rectangle spanned by the points is contained in .
For this model we show   and
.\\
Most interestingly, we can show that the lower bound proof extends to guards with line visibility. 
To this end we introduce and utilize a novel discrete combinatorial structure called multicolor tableau. This is the first non-trivial lower bound for this problem setting.

Furthermore, for the strong chromatic version of the problem, where all
guards r-visible from a point must have distinct colors, we prove a
-bound.
Our results can be interpreted as coloring results for special
geometric hypergraphs.

\end{abstract}

\section{Introduction}

\ \ \ \ The classic Art Gallery Problem (AGP) posed by Klee in 1973 asks for the
minimum number of guards sufficient to watch an art gallery modelled by
an -sided simple polygon . A  guard sees a point in  if the
connecting line segment is contained in . Therefore, a guard
watches a star polygon contained in  and the question is to cover 
by a collection of stars with smallest possible cardinality. The answer
is  as shown by Chv\'atal, \cite{Ch}. This result was
the starting point for a rich body of research about algorithms,
complexity and combinatorial aspects for many variants of the original
question. Surveys can be found in the seminal monograph by O'Rourke
\cite{ORourke}, in Shermer \cite{Sh} or Urrutia \cite{Ur}. 

Graph  coloring arguments have been frequently
used  for proving worst case combinatorial bounds for art
gallery type questions starting with Fisk's proof \cite{Fisk}. Somehow
surprisingly, chromatic versions of the AGP have been
proposed and studied only recently. There are two chromatic variants: strong
 and conflict-free chromatic guarding  of a polygon . In  both
versions we look for a guard set  and give each guard one of 
colors. The chromatic guarding is said to be strong if for each point
 all guards  that see  have pairwise different colors
\cite{ELV}.  It is conflict-free if in  each   there is at least
one guard with a unique color, see \cite{BS}. The goal is to determine guard
sets such that the   number of colors sufficient  for these purposes is
minimal.  Observe, in both versions minimizing the number of guards is not
part of the objective function.  Figure \ref{cf_st_example} shows a simple orthogonal polygon with
both conflict-free and strong chromatic guardings in the r-visibility model.
\begin{figure}
\centering
\includegraphics[scale=0.3]{figures/CF_Example.pdf}
\ \ \ \ \ \ \includegraphics[scale=.3]{figures/Strong_Example.pdf}

\caption{Example of conflict-free (left) and strong chromatic (right) r-guarding}
\label{cf_st_example}
\end{figure}
To grasp the nature of the problem, observe that it has two conflicting aspects. 
We have to guard the polygon but at the same time we want the guards to hide from each other, 
since then we can give them the same color. For example, in the strong version we want a guard set 
that can be partitioned into a minimal number of subsets and in each subset the pairwise link distance is at least 3.
Moreover, we will see a strong dependence of the results on the underlying visibility model, l-visibility vs. r-visibility. We use superscripts
 and  in the bounds to indicate the model.

Let  and   denote the minimal number of colors sufficient for any simple polygon on  vertices in the strong chromatic and in the conflict-free version if based on line visibility.

Here is a short summary of known bounds. For simple orthogonal polygons on  vertices
, as shown in  \cite{BS}. The same bound
applies to simple general polygons, see \cite{B_etal}.
Both proofs are based on subdividing the polygon into weak visibility
subpolygons that are in a certain sense independent with respect to
cf-chromatic guarding.
\\
For the strong chromatic version we have 
for simple polygons and  even for
the monotone orthogonal case,
 see \cite{ELV}. NP-hardness is dicussed in \cite{FFH}. In \cite{ELV}, simple  upper bounds are shown
for special polygon classes like spiral polygons and orthogonal
staircase polygons combined with line visibility. 

Next we state our main  contributions for simple orthogonal polygons:\vspace*{0.2cm}
\begin{enumerate}
\item We show    and 
.
\item The lower bound holds for line visibility, too: . This is the first super-constant
lower bound for this problem.
\item  For the strong chromatic version  we have .
\end{enumerate}
\vspace*{0.2cm}
\ \ \ \ The chromatic AGP versions  can be easily interpreted as  hypergraph coloring 
questions. Smorodinsky \cite{Smo} gives a nice survey of both
practical and theoretical aspects of hypergraph coloring.
A special role play hypergraphs that arise in geometry. For example,
given a set of points  in the plane and a set of regions 
like rectangles, disks etc. we can define the hypergraph . The discrete interval
hypergraph   is a concrete example of such a hypergraph:
We take  points on a line and all possible intervals as regions. It
is not difficult to see that .
As to our AGP versions, we can associate with a given polygon and a guard set a geometric hypergraph.
Its vertices are the guards and a hyperedge is defined by a set of guards that have a nonempty common intersection of their visibility regions and in the intersection there is
a point that sees exactly these guards.  Then one wants to color this graph in a conflict-free or in a strong manner. \\
Another  example is the following  rectangle hypergraph.
Vertex set is a finite set of  axis-aligned rectangles and each
maximal subset of rectangles with a common intersection forms a
hyperedge. Here
the order for the cf-chromatic number is  and
 as shown in \cite{Smo,PaTo}.

Looking at our results,  it is  not a big surprise that the 
combination of orthogonal polygons with r-visibility yields the
strongest bounds. This is simply due to additional structural properties and this  phenomenon has already been observed for the original AGP. For example, the  tight
worst case bound for covering simple orthogonal polygons with general
stars can also be proven for r-stars (see \cite{ORourke}) and it
holds even for orthogonal polygons with holes, see (\cite{Hoff}).
Further, while minimizing the number of guards is NP-hard both for
simple general and orthogonal polygons if based on line visibility, it
becomes polynomially solvable for r-visibility in the simple orthogonal
case, see \cite{MRS,Wor}. The latter result is based on the solution of
the strong perfect graph conjecture.

The paper is organized as follows. We give neccesary basic definitions in the next section. Then we prove upper bounds in Section 3 using techniques developed 
in \cite{BS, B_etal}. Our main contribution are the lower bound proofs in Section 4. Especially, we introduce a novel combinatorial structure called multicolor tableau.
 This structure enables us to extend the lower bound proof for r-visibility
to the line visibility model.

Omitted proofs can be found in the Appendix.

\section{Preliminaries}
\subsection{Orthogonal polygons, r-visibility and general position assumption}
\ \ \ \ We study simple orthogonal polygons, i.e., polygons   consisting of
alternating  vertical and horizontal edges only that do not have holes.
By  we denote the number of vertices, by   the boundary
and by  the interior of the polygon.
Vertices can be reflex or convex. A reflex vertex has an interior angle 
 while convex vertices have an interior angle of . To
simplify the presentation we make the following very weak assumption
about general position of orthogonal polygons: If two reflex vertices
 are connected in  by a horizontal/vertical chord
then the four rays emanating from  and  towards the interior along
the incident edges represent only 3 of the 4 main compass directions.
That is, two rays are opposite to each other and the other two point in
the same direction. 

Points  are line visible (or l-visible for short) to each
other if the line segment  is containd in . Observe that the
segment  is allowed
to contain parts of boundary edges. The points  are r-visible to
each other if the closed axis-parallel rectangle  spanned by the points is contained
in . 
For  we denote by  and
 the set of all points l-visible
from   and r-visibility, respectively. This is also called the
visibility polygon of a point . If it is clear from the context
which polygon is meant we omit the index.  A polygon that is fully
visible from one of its points is called a star and, again, we have to
distinguish between l-stars and r-stars. Most notably, for a point 
in an orthogonal polygon the visibility polygon 
is itself orthogonal while  usually is not. We can generalize
this by defining
for a subpolygon  its visibility polygon by
. The windows of a subpolygon  in 
are those parts of  
 that do not belong to .

For an orthogonal polygon   we define its induced  r-visibility line
arragement .
 Two points  are equivalent with respect to r-visibility if
.
This is an equivalence relation. What are the equivalence classes? 
First of all, there is a simple geometric construction to find .
For each reflex vertex of  we extend both incident boundary edges
into   until they meet the boundary again, therefore
defining a subdivision of the polygon. The faces of this line
arrangement are rectangles, line segments, and intersection points.
Clearly, two points from the interior of the same rectangle define the
same r-star. What about line segments in the arrangement? We extend a
line segment  into both directions until we hit a convex vertex or
the interior of a boundary edge. Let's call this extension . By our
general position assumption we know that on one side
 ({\it inner} side) of  there is only polygon interior.  Consider a
point  in the interior of a line segment that is incident with two
rectangular faces. It is not difficult to see, that     inherits the
r-visibility from the incident rectangle on its inner side and the same
rule applies to intersection points which can have up to four incident
rectangles.

Finally, we define special classes of orthogonal polygons. A weak
r-visibility polygon (also known as histogram) has a boundary edge 
(called base edge) connecting two convex vertices such that
. This is therefore a monotone polygon with respect to  the
orientation of . A weak r-visibility polygon that is an r-star is
called a {\it pyramid}.   


\subsection{Conflict-free and strong chromatic guarding}
\ \ \ \ A set  of points is an r-guard set for an orthogonal polygon  if their r-visibility polygons
jointly  cover the whole polygon. That is: , analogously for l-visibility. If in addition  each guard
 is assigned one color  from a fixed finite set of
colors  we have a chromatic guarding .
Next we give the central
definition of this paper.

\begin{definition} A chromatic r-guard set  for  is  strong
 if for any two  guards  we have
 implies .\\
 A chromatic r-guard set   is  conflict-free   if for any
point  in the guard set  there is at least one
guard with a unique color. 
\end{definition}
\ \ \ \ We denote by  the minimal  such that there is
conflict-free chromatic guarding set for  using  colors. Maximizing this value
over  all polygons with  vertices from  a specified polygon class is
denoted by .\\
Consequently, we  denote by  the minimal  such that there is
strong chromatic guarding set using  colors. Maximizing this value
for all polygons with  vertices from  a specified polygon class defines
the value  .\\
The notions for line visibility are completely analogous and use superscript .


\section{Upper Bounds }
\ \ \ \ We show  upper bounds for both strong and conflict-free r-guarding of
simple orthogonal polygons of size :  and 
. These bounds are even  realized by r-guards
placed in the interior of visibility cells. This restriction will
simplify the arguments.  The proof (see also \cite{Wi}) follows closely ideas
developed in \cite{B_etal, BS} for conflict-free
l-guarding of simple  polygons. Therefore we only recall the
general ideas, omit some proof details and emphasize the differences
stemming from the underlying r-visibility. 
 







\subsection{Partition into independent weak visibility polygons}
\ \ \ \ First of all, we reuse the central concept of {\it independence} introduced
in \cite{BS, B_etal} for line
visibility. Independence means that one can use the same color sets for
coloring guards in independent subpolygons.
 The following definition  suffices for our purposes.

\begin{definition}
Let  be a simple orthogonal polygon and   and 
subpolygons of .
 We call  and   independent if  .
\end{definition}

Next, we are going to subdivide hierarchically an orthogonal polygon 
into  weak visibility subpolygons by a standard window partitioning
process as  described in \cite{BS}. \\
Remark: In the following we use the term subdivision not in the strong set-theoretic sense. A subdivision of   into closed subpolygons  means that 
 and for all  we have . \\
The subdivision is
represented by a tree
 with the weak
 visibility polygons as node set. Let   be a highest horizontal edge
of , the ``starting'' window.   is a weak
visibility polygon and is the root vertex  of . Now  splits
 into parts and defines a finite set (possibly empty if ) of
vertical windows . Each window  corresponds to a left or
right turn of a shortest orthogonal path from  to the subpolygon
lying entirely behind the window. Then we recurse, see Figure \ref{partitionFigures}.\\
By the partitioning process we can obtain a linear number of subpolygons only. There are  reflex vertices in a simple orthogonal polygon
 with  vertices. Each window uses at least one reflex vertex. Therefore,
we get at most  weak visibility polygons. This bound is realized for example by spiral polygons.


\begin{figure}
\centering
\includegraphics[scale=.3]{figures/bild1.pdf}\quad\quad
\includegraphics[scale=.3]{figures/bild2.pdf}\quad\quad
\includegraphics[scale=.4]{figures/SchematicTree.pdf}
\caption{The partitioning process and the corresponding schematic tree.}
\label{partitionFigures}
\end{figure}
Let  be the family of all weak visibility polygons
corresponding to nodes of depth congruent  in . We
partition   into  consisting of  and all those
subpolygons which are left
 children and, on the other side,   consisting of the remaining ``right'' parts.

\begin{lemma}
\label{indep}
Let  be a polygon and   the family of subpolygons corresponding to  left nodes in
 with depth congruent . Then the interior of  subpolygons in 
 have  pairwise link distance at least three, analogously for .
\end{lemma}
\begin{proof}
Suppose there are two different  subpolygons  and  in 
. If  they have different depth then for arbitrary points
 and  any orthogonal path connecting
these points has
length
 at least 3. Otherwise they have the same depth. In this  case they
could be sibling nodes with parent node . To walk orthogonally 
from  to  it needs  two parallel edges to cross the windows
plus one more edge in
 . If the lowest common ancestor  is more than 1 level above
then a shortest orthogonal path from  to  has to visit the
parent node of , the parent node of  and then descend to 
which takes at least three edges.
\end{proof}

Observe that distinguishing left and right nodes is essential. It can be
possible to walk with one step from a left node to a right sibling.

\begin{corollary} Let  be subpolygons computed in the subdivision process for .
Then  and  are independent and there exists a strong chromatic r-guarding for  in which guards in  and  use the same
color set. The same is true for conflict-free chromatic guarding.
\end{corollary}

\begin{proof} 
Assume we have  r-guards  in  ,  with  containing a point . Then there exist points  and a
connecting orthogonal path
 of length 2. But  this contradicts the previous lemma.
Therefore   and  are independent and strong chromatic r-guardings for  and  do
not interfere with each other,
the same holds for conflict-free r-guardings.
\end{proof}
Remark: We will restrict the guards to sit in the interior of visibility cells. However, this does not effect the asymptotic upper bounds on the number of colors used.
\subsection{Guarding  a weak visibility polygon}

\ \ \ \ Consider  a weak visibility polygon  with a horizontal base edge .
An edge of  opposite to  is an -edge if it connects two reflex vertices, it is a
-edge if it has two convex vertices. Among the horizontal edges opposite
to  there is at least one -edge and  a chain connecting two consecutive 
-edges contains exactly one  edge. Recall that a pyramid 
contains exactly one -edge . We guard a pyramid with one r-guard
stationed opposite to  
 in the visibility cell just below base edge . Next we describe (see
\cite{BS}) a simple truncation process that decomposes a weak
visibility problem into pyramids.

{\bf The truncation process:}
Let  be a weak visibility polygon with  vertices, a  horizontal  base
edge  on top of the polygon and  the set of -edges opposite to . If there is only
one such -edge we stop and return . Otherwise, for each -edge  we
sweep  from  
towards  until the sweep line reaches the first neighboring  -edge.  
We truncate  by cutting of the pyramid below. After processing all
edges in  we have again a weak visibility polygon  with base
edge . Observe that  does not depend on the order in which
we process the edges in 
 and, moreover, the pyramids associated with  are independent.\\
Then we iterate with  and get  and so on. Eventually, we have indeed
partitioned  completely into pyramids. These pyramids have an
important structural property. By construction, the unique -edge in
each pyramid contains a non-empty segment of the original boundary
, we call them ``solid'' segments. As guard position for
such a pyramid we choose an interior point  just below the base edge of
the pyramid opposite to an interior point of a solid segment.  
 
In Figure \ref{covGuards} we see an example of a weak visibility
polygon, its decomposition into pyramids and the chosen guard positions.
Again, there is a canonical schematic tree representing the
decomposition and the guard positons.



\begin{figure}
\centering
\includegraphics[scale=.3]{figures/WeakDecisionLines.pdf}\quad\quad
\includegraphics[scale=.3]{figures/WeakGuardTree.pdf}
\caption{The truncation  process of a  weak visibility polygon and its
schematic tree with guard positions}
\label{covGuards}
\end{figure}

Clearly, the height of  is in . In the worst case, this is best possible as shown by the spike
polygons  in Section 4  we use for our lower
bound proofs.


\begin{figure}
\centering
\includegraphics[scale=.3]{figures/WeakLemma.pdf}
\caption{Single points are seen by connected  chains of r-guards}
\label{guardTree}
\end{figure}


The following lemma states the main structural property for this  tree
of guards.
\begin{lemma}
\label{chain}
Let  be  a weak visibility polygon and  the
guard-tree computed in the truncation process. Then for each 
all guards in   form  a connected subpath of a  root-to-leaf
path in .
\end{lemma}

\begin{proof}
Assume two nodes representing pyramids  and  are not on a root
to leaf path in . Consider the lowest common anchestor node, say it
is pyramid .  has a solid segment in its c-edge. Therefore
 and  are independent and r-guards from both pyramids cannot
see the same point . Now we know that all r-guards watching a common
point  are indeed on a common root-to-leaf path. Let  be the
deepest and  the highest guard among them with . We
have to show, that all guards in between see point , too. Where can
point  be? It has to be in the vertical strip above the base line of
the pyramid with guard  and below the base line of the pyramid
corresponding to , since the parent node of  does not see 
by assumption. This region is a rectangle . For any guard  between 
and  the vertical strip above the corresponding base line contains
 and  sees .  
\end{proof}
In Figure \ref{guardTree} the paths formed by r-guards watching point  and for point 
are indicated.

\begin{theorem}
\label{stTheorem}
Let  be an orthogonal  polygon with . We have
.


\end{theorem}

\begin{proof} We decompose  into pairwise independent weak visibility
polygons. Each weak visibility polygon can be further decomposed into 
pyramids and the corresponding guard trees have height 
. We color each guard by its depth in the tree.
 This is a strong chromatic guarding since for each  by Lemma \ref{chain} all  of its guards have pairwise different colors.
\end{proof}

We use the same r-guard positions but a different coloring scheme to get a
conflict-free coloring. Consider the color alphabet  and the following recursively defined set of words. Let 
and . The following is straightforward
and has been used before for conflict-free coloring the discrete
interval hypergraph.
\begin{lemma}
\label{sequence}
A prefix of  with length  has no more than
 different colors and each connected subword contains a unique color.
\end{lemma}



\begin{theorem}
\label{cfTheorem}
Let  be an orthogonal polygon with . Then 
 .
\end{theorem}

\begin{proof}
The only difference in comparison with the proof above is the coloring
scheme. Each r-guard tree gets colored top-down with the sequence  of
length at most height of the tree, that is . By Lemma
\ref{sequence} the color alphabet needs to be of size
 and the coloring is conflict-free by Lemma \ref{chain}.
\end{proof}

We illustrate the construction in Figure \ref{coveringGuards}.
\begin{figure}
\centering
\includegraphics[scale=.3]{figures/WeakStrongColoring.pdf}\quad\quad
\includegraphics[scale=.3]{figures/WeakCFColoring.pdf}
\caption{Strong (left)  and conflict-free guarding (right) of a  weak visibility
polygon}
\label{coveringGuards}
\end{figure}

\section{Lower Bounds}


\subsection{Spike polygons}
\ \ \ \ All lower bounds established in this paper are based on a simple,
recursively defined family of so called
spike polygons , where  is a simple square and  is
formed by two copies of
 separated by a vertical spike, but joined by an additional
horizontal layer.
The left side of Figure \ref{spikePoly}  illustrates this construction
together
with the subdivision of  into visibility cells. Observe that the
height sequence of the
spikes in  is nothing else but the word  used in Theorem \ref{cfTheorem} above.

Columns of  are numbered  left to right by indices ,
and  cells in column  top down by an addditional index  where
 is the depth of column  in . Formally, we have
  where  is the multiplicity of  factor
 in the prime decomposition of .  
Obviously, a column has maximal depth  iff its index is odd.

We introduce the notions of the left and right wing of  column  in
order to distinguish
guard positions: The left wing  is the set of all points
strictly on the left side of the midline of column   and the
right wing is  the complement .

\begin{figure}
\centering
\includegraphics[scale=.35]{figures/SpikeSmall.pdf} \qquad
\includegraphics[scale=.35]{figures/SpikeBig.pdf}       \qquad
\includegraphics[scale=.35]{figures/Spike_B_k.pdf}
\caption{Spike polygons  and  (left), left wing and right wing
of column  in  (middle), blocks and
  subblocks (right)}
\label{spikePoly}
\end{figure}




We will prove three lower bound results for guarding spike polygons. 
The easiest version refers to strong chromatic
r-guardings.

\noindent
\begin{theorem}
We have .
\end{theorem}
\begin{proof} The proof is by induction. The induction base for  is straightforward.
Next we show the induction step by contradiction. Assume that the claim is true for some    and suppose that there is  a
strong chromatic r-guarding of  with  colors only.
There must be a unique color  for the top cell in the middle
column. Since the
corresponding guard  sees all cells in the first row, it is
the only one of color  in 
(any other -guard  would produce a conflict in at least one cell in
the first row).
The deletion of the  top row splits the remaining part of  
into two copies of .
Depending on the position of  in , at least in one  copy  no cell is
-visible from .
Thus, we have  a strong -chromatic r-guarding  of this copy.
But this contradicts
 the induction hypothesis.\end{proof}

The other two lower bound proofs  are much more involved, but they follow the same scheme. They are by induction and the induction step is shown by contradiction. 
But now the  induction step  
can require a sequence of   steps cutting out from  the original 
 smaller units 
until arriving at a contradiction. We start with the proof  for
cf-guardings
with respect to -visibility. In its quintessence  
 it  relies on purely combinatorial properties of a  discrete structure which we call multicolor tableau.
We will then rediscover a slightly  weaker version of this structure having similar  properties when discussing lower bounds for conflict-free guardings
based on  l-visibility
for appropriately vertically stretched spike polygons.

 
 
\subsection{Blocks and multicolor tableaux}
\ \ \ \ Consider the spike polygon . It has  columns.
We define the block  of column  as the interval
of all neighbouring columns of depth at least , see Figure \ref{spikePoly}: \\
 . Geometrically, a block is nothing but a smaller spike polygon. Deleting its central column a block splits into a left and a right subblock: \\

For odd  we have  and .
Later  it will be necessary to subdivide a left
or right subblock again into
its left and right subblocks. These ``quarter''-subblocks can be described
making use of the definition above
together with the central column  in block
 and column
 in  block :\\



Let  be a finite set of -guards covering  and

a cf-coloring of  . By   we denote the  multiset of
all colors of guards that see the th
visibility cell  in column , and let  denote the
multiplicity of color  in  .
Then the  combinatorial scheme

will be called a conflict-free multicolor tableau.
\\
We
formally define   the set of unique colors of a cell by  and the  standard inclusion relation  for multisets  by:
              .

The following simple fact  about r-visibility in spike polygons
makes  the crucial difference between
the simpler lower bound  proof for
r-visibility and the more involved proof for l-visibility.


\begin{lemma}
\label{rVis}
Let  be an r-guard in  that sees a cell ,
then  is  in a cell of depth 
and it sees all cells  with
 and .
\end{lemma}
\begin{proof}
The first assertion is straightforward because otherwise the spike in
column  would block
the visibility between  and . For the second claim
consider the minimal rectangle  enclosing the cell of  and 
. Since the lower
side of  has depth   and all columns  have
depth  
one can extend  within  horizontally to the whole width of the
block 
and upwards to the top edge of .
\end{proof}

\begin{lemma}
\label{uniqueInterval}
Let  be an r-guard set covering  and  
a cf-coloring of . Then for any color  and for any column in
the multicolor tableau  the following holds: The
multiplicity 
is a monotonically decreasing function with respect to  row index .
In particular, if  is a unique color somewhere in column  then
row indices of cells
with  form an interval  and  .
\end{lemma}

\begin{proposition}
\label{r-visAdmiss}
For  a conflict-free r-guarding   of 
the multicolor
tableau 
has  three combinatorial properties:
\begin{enumerate}
\item cf-Property: .
\item Monotonicity: .
\item Left-right rule:   If  is a unique color in the top cell
 of column  then for
all  or for all  the following three
conditions hold
\begin{enumerate}
\item 
\item If  then  .
\item If  then  for all .
\end{enumerate}
\end{enumerate}
\end{proposition}
\begin{proof}
There is nothing to prove for the cf-property and the monotonicity
follows from
Lemma \ref{uniqueInterval}.
It remains to establish the left-right rule.
Assume   and consider  the corresponding
guard .
Depending on whether  is in  or in  we 
prove that the
three properties hold in the opposite block  or in , respectively. Again,
it suffices to
discuss the first case. By Lemma \ref{rVis}  sees all cells 
with .
Condition (a) holds because . \\
We prove  condition (b) by contradiction assuming  that 
and 
for some . Since  is in the right wing
of , it can't see any cell of depth  in the left wing. Thus
 implies the existence of another c-colored guard
, that sees
. But again by Lemma \ref{rVis}   sees also
 what contradicts
the uniqueness of  for this cell. \\
Finally, if  is not unique for   then there are at least two
guards with color  that watch
. Both of them watch all cells  with  what proves
condition (c). \end{proof}


\begin{theorem}
\label{r-cf-lowerbound} For simple orthogonal polygons
.
\end{theorem}
\begin{proof}(Sketch)
Let   be defined by  and  for
.

\noindent
{\em Claim:}
.


\noindent
It is easy to  deduce the theorem from the claim. A simple inductive
argument shows 
 and thus  for some  ,
because this  is an upper bound on the vertex number of   .
This inequality is equivalent to  what implies
 and, finally,
.  


We prove the claim by induction on . For the base case 
we must show that it is impossible to guard  conflict free with one
color.
Suppose the opposite and consider the corresponding multicolor tableau
. The only way to
fulfill
the uniqueness condition is to set  for all pairs
 and color . This already  contradicts  condition
(b) of the
left-right rule applied to the situation .


Next, we illustrate the induction step in detail for the step from
  to  with  and . We prove it by contradiction.
Suppose there  is an r-guard set  and a
coloring  that is a conflict-free guarding of 
and let  be the corresponding  multicolor tableau.
A contradiction will be derived by  a sequence of at most two
cutting stages with the goal to identify a subpolygon  in  that has a
conflict-free r-guarding with only one  color.\\
We start with a unique color  of the top cell  in the central column
 of . W.l.o.g.~the corresponding guard  is located in the right wing
 and the three conditions of the left-right rule apply for all . \\ 
The  subblocks  and   cover  with the exception
of the separating column .  Considering the two central top cells  and 
(the green cells in  Figure \ref{r-vis-indStep1})
 we distinguish  two cases:\\
  (1): \\
  (2): \\
\begin{figure}
\centering
    \includegraphics[width=0.6\columnwidth]{figures/SpikeProof1.pdf}
\caption{The first subdivision stage in : Case 1 holds if   is
unique for both green cells,  and .
Then  would have a conflict-free r-guarding with only one color, a
contradiction.}
\label{r-vis-indStep1}
\end{figure}
Whenever Case (1) occurs  this is a stopping
rule, because  one can directly identify a subpolygon  with the
shape of  together a conflict-free guarding that uses  one color only, a contradiction.
To that end we construct the subpolygon  consisting of all cells
 with  and , the  grey shaded region 
in Figure \ref{r-vis-indStep1}.
One can make two basic observations about :
\\
(i) The shapes of  and   are the same in the sense that 
 is a stretched version of  and their
decompositions  into r-visibility cells are isomorphic. 
\\
(ii) Let  be the set of all guards from  that  are positioned in
. We extend it to a
set  by pulling down
all guards in cells above  onto the top edge of . In Figure \ref{r-vis-indStep1}
 this is illustrated by small downarrows. Then  
with the original coloring  is a cf-guarding of
 with one color only because color  does not occur. \\The last
assertion is
straightforward because the presence of any -colored guard in
 would contradict
the uniqueness of  for    or  (the assumption of case 1).
It is also clear that  covers  because any original guard
for a cell 
with   remains in  and it will cover all 
with  as well. 
Finally the cf-condition also extends from a cell  to all
 with  because all columns in  and 
 are truncated from
below at level . The  argumentation applies to cells 
with . 
\\
Observations (i) and (ii) together give a contradiction to the
inductive assumption.\\
In contrast, the ocurrence of case (2)  invokes a second  stage.
Choose one index  such that  , set 
and  repeat the former procedure in the block . Remark, the left-right rule
implies  for all .  
Now the second color  must be unique for cell  and the position
of the corresponding guard  implies that the three conditions of the left-right rule apply 
for all  or for all .
\begin{figure}
\centering
    \includegraphics[width=0.6\columnwidth]{figures/SpikeProof2}
\caption{Case 2 ocurred in the first subdivision stage because of ,  and . Then
the next subdivision stage applies to block .}
\label{r-vis-indStep2}
\end{figure}
Figure \ref{r-vis-indStep2} illustrates the situation for   and  .
Note, guard  could sit also outside of block .
The three conditions of the left-right rule apply for all . 
Again, there are two subblocks   and  (now single columns) that cover  
with exception of the separating column . The next case distinction 
refers to their  top cells : \\
  (1): \\
  (2): \\
In Case (1) one can cut out the subpolygon  consisting of all cells 
with  and , see Figure \ref{r-vis-indStep2}, and construct 
a guard set  as in case 1 before. This would result in a cf-guarding of  without ,
a contradiction. 
\\
In  Case (2) there is a   with  , but moreover 
 because Case (2) occured in the first stage. This implies ,
a contradiction again.
\\
Now we present the  general induction step from  to , again shown by contradiction. \\
Assume  that  there is
no conflict-free r-guarding of  with
 colors for , but there  is an r-guard set  and a
coloring  that is a conflict-free guarding of 
for . Again we make use of the corresponding  multicolor tableau
. \\
A contradiction will be derived by  a sequence of at most 
cutting stages. \\
Stage  ()  always starts with the precondition that 
there is a column  of depth ,
the block  of  with  
columns and a set  of  colors such that 
 for all  and for  all .
The first stage starts with  (the central column of ),
,  and an empty precondition.
Each stage results in a case distinction where the ocurrence of the first case
would finish the proof by a contradiction with the inductive assumption,
whereas the second case implies  the precondition of the next stage.
Since the precondition of  stage  states a contradiction of the form  
(because  is the set of all colors), it  won't be necessary to execute that stage. 
\\
Now, suppose that the precondition of a stage  is fulfilled in a block 
with the color set . Choose some color
 ( by the  precondition)
and consider  the corresponding guard  in the left or 
right wing of column . By symmetry it is sufficient to discuss the 
first case  where the three conditions of the left-right rule
hold for all . Let  be the 
set of all columns of depth  in  . Note that this condition implies
for all  that  and thus .
The new case inspection applies to the top cells of the rows : 
\\
  (1): \\
  (2): \\
If Case (2) occurs with  for a  then  for all 
by condition (c) of the left-right rule. This immediately implies the precondition for the next
stage with  and . \\
If Case (1) occurs, we consider the polygon  formed by the union of all cells
 with  and . 
As discussed above the cell decomposition of the polygon  is isomorphic to that
of . Moreover extending the set  of original guards in 
by pulling down all guards that sit direcly above   onto the top edge of 
, we obtain a cf-guarding of  with   colors, because 
  can't occur as a color in the extended guard set .
This contradicts the inductive assumption and finishes the proof.   
\end{proof}


Any attempt to adapt  this proof to cf-guardings of 
with respect to line
visibility encounters the following  problems. \\
{\bf  Problem 1}: It is impossible to subdivide the polygon into a finite set of
visibility cells such that any two points
in a cell would have the same visibility polygon.\\
Solution: The guard set watching a given cell  is replaced by
the guard set watching a  single special
point in the cell.
We always choose the midpoint  of the lower side of .\\
{\bf  Problem 2}: Guards from the left wing of a column  can possibly see points
in the right wing that are much deeper than
.\\
Solution:
 The heights of  rows in  will be stretched in an
appropriate way such that no
guard from the left wing of a column  can see a special  point  with
 and .\\
{\bf  Problem 3}: A guard that sits deeper than  in   can possibly watch
points in column  and even points in .\\
Solution: One can't avoid this, but the stretching of  rows will assure that it won't see
points in .
It turns out that the left-right rule must be relaxed in such a way that
conditions (a), (b) and (c)
do not hold in whole opposite half block, but they hold (in slightly
modified form) at least in a quarter
subblock. Formally we will refer to this fact by a quantified formula of
the type
\\
{\bf  Problem 4}: Pulling a guard to another position like in the construction of
the guard set  for the subpolygon 
changes the visibility range of the guards and might result in a guard
set that does not cover the same subpolygon it
covered before.\\
Solution: Any conflict-free guarding of  will be translated into  purely
combinatorial properties of
the corresponding multicolor tableau, such that
concrete  guard positions don't play any role in the subsequent lower bound proof.






\subsection{Stretched spike polygons and {\bf }-conform tableaux}
\ \ \ \ For the purpose of forcing similar properties for l-visibility as we
used for r-visibility
we introduce a vertically stretched version  of   with the
following geometric properties:
\begin{itemize}
\item The width of each column is  and hence the total width of  
 is
.   
\item We  distinguish between combinatorial and geometric depth of a
column:
While   is still used for the combinatorial depth,  we
want  the  geometric depth to  be  .
Therefore the height of the first row is  and the height of
the -th row  .
\end{itemize}
Consider the decomposition of  into r-visibility cells
 and let
 be the midpoint at the bottom side of  .
If  for guard set   is a  conflict-free l-guarding of
, then
let  be the multiset of all colors of guards that see
 and 
the corresponding multicolor tableau.

The following two observations establish  similarities between 
l-visibility in   
and r-visibility in  and substitute Lemmata \ref{rVis} and \ref{uniqueInterval}.
\begin{lemma}
\label{lVis1}
Let  be a guard  in ,  a column
of this polygon with
combinatorial depth  and geometric depth
.
If  ()
then  can't see any point  at depth  in the left (right) block
of , especially  can't see
any point  with  () and .
\end{lemma}
\begin{proof}
By symmetry it is sufficient to study the first case with ,  and
 a point in the subpolygon  .
Let  be the left vertex of the horizontal polygon edge in column  and
consider the  slopes  and  of the lines  and
.
Since the width of  is  and
  we get

Since  is in the right wing of  it is at least one half unit right
of  and it is at most
 higher than 

Thus,  what shows that the corner at  blocks the
l-visibility between  and .
\end{proof}
\begin{lemma}
\label{lVis2}
Let  be an l-guard watching a point .
Then for all  and for all  or for all 
 sees also   .
\end{lemma}
\begin{proof}
Let   be the geometric
depth of   in .
\\
{\em Case 1:} If  is even an r-guard for  the claim follows
for all  
by Lemma \ref{rVis}. Otherwise there are two more cases, namely that
 is strictly smaller or strictly larger than
 (see Figure \ref{three_guard_pos}).
\\
{\em Case 2:} , i.e.,  sees  
from above. If
  then  can see all   with  beause
the line segments  and   are contained in

and they form a chain that is convex from above.
If  then  can see all   with  beause
the line segments  and   are contained in

and they form a chain that is convex from above. Moreover it is clear
that in 
any guard that sees a point   will see also all points directly
above, especially
the points  with .
\\
{\em Case 3:} , i.e.  sees  
from below.  
Since   would imply case 1,
we can additionally
assume  , i.e. 
is in a cell
 with  and  or .
Now we can apply Lemma \ref{lVis1} with  in the role of the
guard 
and  in the role of a point  watched by . It turns out  that
, i.e.,  lies in row  of
.
It follows that depending whether  lies in  or 
it sees all  with  or  (and all
points directly above as well).
\end{proof}

A  tableau  is in 
standard form if it has   rows and 
columns. But by various constructions, for example restricting it to a single block, one creates  
a tableau having  rows and  columns for some .
The following definition of -conformity specifies some necessary, but not sufficient conditions a multicolor tableau has if it stems 
from a conflict-free -coloring  of a stretched spike polygon. The advantage is that -conformity is preserved when acting on the tableau with 
operations defined below.

\noindent
\begin{definition}
\label{t-admis}
Let   be natural numbers and .
A scheme of  multisets over the set  of the form

is called a  
-conform -multicolor tableau
if the following properties hold:
\begin{enumerate}
\item .
\item .
\item  \\
where the predicate   is the conjunction of 
three conditions:
\begin{enumerate}
\item 
\item    
\item .
\end{enumerate}
\end{enumerate}
\end{definition}

Note the first two properties are identical with
those 
in Proposition \ref{r-visAdmiss}.
The third one, however,  is a proper relaxation of the left-right rule  there.
Thus, any tableau  resulting from a conflict-free
r-guarding
of  with  colors is also -conform.
\begin{proposition} \label{l-visAdmiss}
The multicolor tableau   for a
conflict-free
l-guarding of the polygon  with  colors is
-conform.
\end{proposition}
\begin{proof}
There is nothing to prove for the uniqueness condition and for the
monotonicity.
Now let us assume  with a
corresponding guard .
By symmetry we may suppose . Like in  Lemma
\ref{lVis2}
there are three cases to distinguish (see Figure \ref{three_guard_pos}):
\begin{enumerate}
\item  is r-visible from .
\item  is not r-visible from  and  is deeper than .
\item   is not r-visible from  and  is deeper than .
\end{enumerate}
\begin{figure}
\centering
\includegraphics[scale=.5]{figures/Spike_RL1.pdf}\quad\quad
\includegraphics[scale=.5]{figures/Spike_RL2.pdf}
\includegraphics[scale=.5]{figures/Spike_RL3.pdf}
\caption{Possible guard positions  with respect to the
point . Note that it is impossible to
  display the exponential growth of the row heights in the drawing.}
\label{three_guard_pos}
\end{figure}
In Case 1 and Case 2 we choose  (but  would also work -
the gray points).
In Case 3 the choice depends on the position of  relative to the central
column  of the block :

It remains to establish the three conditions of  for all
. Condition (a) is obvious in case 1 and case 2. In
case 3 it follows from
the fact that  can't be deeper than  (see
Case 3 in the proof of
Lemma \ref{lVis2}).
\\
For condition (b) suppose that  . This implies that 
is the only guard with
color  that sees . However in all three
cases  is in the wing opposite  to  block  and then 
can't see any point of
combinatorial depth  in  by Lemma \ref{lVis1}.
It's worth observing that
depth   would not suffice in case 3.
However, any other guard with color  watching  would also watch 
 and
contradicts the uniqueness of . Thus .
\\
Finally, let us suppose , then there is a second
guard 
for  . Now  we can conclude from  Lemma \ref{lVis2} that 
watches
all  points  for    or for all .
This proves condition (c).
\end{proof}

\begin{proposition} \label{TabConstr}
If 
is a -conform  -multicolor tableau with  for some . Then the following three constructions
yield new   -conform tableaux   :
\begin{enumerate}
\item  is the restriction of  to a block
;
\item   results from  deleting  the top  rows
of  ;
\item  results from  selecting   columns
for some 
with respect to the following rules:
\begin{itemize}
\item For all even  choose column 
of  as
column  of .
\item For all odd  choose any column     of
 with
, delete from that
column all entries
of depth  and use this truncated column as column  of
.
\end{itemize}
\end{enumerate}
\end{proposition}
\begin{proof}
Recall, the width of  is   where  
.
So the only thing that has to do  for  is  shifting  the column
numbering
from the interval  to
. Then   is t-conform.
\\
For the second construction it is sufficient to shift down the indices
of all undeleted
rows by . Then  is an  tableau. Note
that an old row index
 becomes . Having that in mind, it
is also trivial
that   is -conform.
\\
The construction of  already contains the renumbering of
indices.  Again, it
is not hard to conclude the -conformity because the construction
inherits the relations
of being a column in the left (or right) subblock of another column.
\end{proof}

\begin{theorem}
\label{l-cf-lowerbound}
.
\end{theorem}
\begin{proof}
Despite similarities to the proof of Theorem
\ref{r-cf-lowerbound}
some essential modifications have to be implemented.
The function   is now
defined by  and
 for
 .

\noindent
{\em Claim:} An -tableau cannot be -conform.
 

\noindent
The inequality  is no longer valid in general,
but, it still holds for all . In fact, 
and the  induction
step works for any  as follows:
\\

\\
Hence using Proposition \ref{l-visAdmiss} one can then deduce the theorem
from the claim like before.

\noindent
In the  proof of the claim  by induction on   the induction base for
 works
with similar arguments as before. Any -conform  tableau
requires to set  for all  and all .
However, applying property 3 to the situation  
yields a contradiction with condition (c).

The induction step is proved by contradiction again. Assume 
that  there are no -conform
-tableaux with  and , but there  is
a -conform
-tableau  for  and .
The proof consists of  stages. The precondition of stage 
is the existence of a -conform -tableau
where  and the additional property that there is a
set 
consisting of  colors, such that for all  and for all
 holds . The precondition for the first
stage is given by  
 with  and , but  will
change after every stage.
The postcondition  of the -th stage
is either a contradiction obtained by constructing a   -conform
-tableau (the stop condition, case 1) or the creation of
the precondition for the next step
(case 2). Note that if the stop condition did not occur after the -th
stage, then
the new precondition gives also a contradiction because
 and , i.e.,~it would result in  a -conform
-tableau (a single column) such that no color can be  unique
in .
\\
Now suppose that an  -tableau  with a
color set   fulfills   the
precondition for stage   with .
Let  be the central column of  and
.
Note that the precondition implies . Then by 
property 3 of -conform
tableaux there is some  such that predicate
 is true
for all . Again we subdivide the block 
into  subblocks of equal width. These subblocks can be
defined by their central columns
 where . Note that their width  just fits to the
precondition of the next stage because  has width
 and
consequently all    have width:
  
Due to the weaker conditions encoded in predicate  we have to modify the case inspection:\\
(1) \\
(2) \\
In Case 1 we can immediately derive a contradiction  using the
constructions of Proposition
\ref{TabConstr}: First we restrict (the current) 
to the block , then we use the column selection with 
where
the even columns (numbered  for ) of the new tableau are
the ones that separate the subblocks
 and  from each other and the odd columns 
are chosen from 
with respect to the property . Supposing that  
is not unique in the top set of
an even column would contradict condition (c) of
predicate .
Thus  is unique everywhere in the first row of the new tableau and
with respect to condition (b)
it does not occur at all in third row or deeper. Each column of this
new  tableau 
has depth  because all columns of  had been
selected from a quarter
subblock . Now we apply construction 2 (deletion of top rows)
to  to obtain an
-tableau .  This way at least the two top
rows of 
are deleted and thus color  doesn't occur anymore in .
Finally, we can  replace color  by  color  to obtain a
-conform
-tableau.
\\
Case 2 is now the easier one because replacing  by a block
 such that
 (construction 1) yields
the precondition for the next
stage with .
\end{proof}







\section{Conclusions}
\ \ \ \ We have shown almost tight bounds for the chromatic AGP for orthogonal 
simple polygons if based on r-visibility. While the upper bound proofs use known techniques, we consider the multicolor tableau method
 for the lower bounds to be the main technical contribution of our paper. This method seems to be unnecessarily complicated for the lower
 bound on . But it shows its strength when applied to the line visibility case. It is this discrete structure which enables
 one to apply induction. Otherwise we would not know how to show a lower bound for a continuum of possible  guard positions with strange dependencies 
plus all possible colorings.\\
We conjecture that indeed   using spike polygons and this should also yield a  lower bound for the line visibility case via the stretched version.
But one cannot hope for more,  is also an upper bound for cf-guarding of stretched spike polygons using line visibility. To improve this lower bound one has to look for other polygons. 


















\bibliographystyle{splncs}

\begin{thebibliography}{999}


\bibitem {BS} A. B\"artschi and S. Suri. Conflict-free Chromatic Art Gallery Coverage. Algorithmica 68(1): 265--283, 2014.

\bibitem {B_etal} A. B\"artschi, S.K. Ghosh, M. Mihalak, T. Tschager, and P. Widmayer. 
Improved bounds for the conflict-free chromatic art gallery problem. In Proc. of 30th Symposium on Computational Geometry, pages 144--153, 2014.

\bibitem {Ch} V. Chv\'atal. A combinatorial theorem in plane geometry. Journal of Combinatorial Theory, Series B, 18(1):39--41, 1975.

\bibitem {ELV} L. H. Erickson und S. M. LaValle. An Art Gallery Approach to Ensuring that Landmarks are Distinguishable. In Proc. Robotics: Science and Systems VII, Los Angeles, pages 81--88, 2011.



\bibitem{Fisk} S. Fisk. A short proof of Chv\'atal's Watchman Theorem. Journal of Combinatorial Theory, Series B, 24(3):374-374, 1978.
\bibitem{FFH} S. P. Fekete, S. Friedrichs, and M. Hemmer. Complexity of the General Chromatic Art Gallery Problem. arXiv 1403.2972[cs.CG], 2014.
\bibitem {Hoff}F. Hoffmann, On the Rectilinear Art Gallery Problem, Proc. 17th ICALP, Springer LNCS 443,  717-728, 1990. 
\bibitem {MRS} R. Motwani, A. Raghunathan and H. Saran. Covering orthogonal polygons with star polygons: The perfect graph approach. Comput. Syst. Sci. 40 (1990) 19-48.
\bibitem {ORourke} J. O'Rourke. Art Gallery Theorems and Algorithms. Oxford University Press, New York, NY, 1987.
\bibitem {PaTo} J. Pach G. Tardos. Coloring axis-parallel rectangles. J. Comb. Theory Ser. A, 117(6):776--782, Aug 2010.
\bibitem {Sh} T. Shermer. Recent results in art galleries (geometry). Proceedings of the IEEE, 80(9): 1383--1399, September 1992.
\bibitem {Smo} S. Smorodinski. Conflict-Free Coloring and its Applications. In Geometry - Intuitive, Discrete, and Convex, volume 24 of Bolyai Society Mathematical Studies, Springer Verlag, Berlin 2014.
\bibitem {Ur} J. Urrutia. Art gallery and illumination problems, in J.-R. Sack and J. Urrutia, editors, Handbook on Computational Geometry, pages 973--1026, Elsevier Sc. Publishers, 2000.
\bibitem{Wi} M. Willert. Schranken für eine orthogonale Variante des chromatischen Art Gallery Problems, Bachelor Thesis, FU Berlin, November 2014.
\bibitem{Wor} C. Worman, M. Keil. Polygon Decomposition and the Orthogonal Art Gallery Problem, Int. J. Comput. Geometry Appl. 17(2):105--138, 2007.
\end{thebibliography}
\end{document}
