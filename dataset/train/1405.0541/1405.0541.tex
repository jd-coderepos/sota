\documentclass[11pt]{llncs}

\usepackage{etex}
\usepackage{graphicx}
\usepackage[disable,colorinlistoftodos]{todonotes}
\usepackage{authblk}
\usepackage{tabularx}
\usepackage{bussproofs}
\usepackage{proof}
\usepackage{xypic}
\usepackage{fancyhdr}
\usepackage{lscape}

\usepackage[english]{babel}
\usepackage{url}
\usepackage{algorithm}
\usepackage{algorithmic}
\usepackage{amsmath} 
\usepackage{amssymb}
\usepackage{color}

\newtheorem{fact}{Fact}


\newcommand{\Log}[1]{\ensuremath{\mathcal{#1}}}
\newcommand{\NatDed}[1]{\ensuremath{\mathcal{N}_{\mathcal{#1}}}}
\newcommand{\abrev}[1]{\textsf{#1}}
\newcommand{\struct}[1]{\ensuremath{\langle #1 \rangle}}
\pagestyle{fancy}
\rfoot{2014}

\newtheorem{axiom}{Axiom}


\newcommand{\imply}{\ensuremath{\rightarrow}}
\newcommand{\mil}{\ensuremath{\mathbf{M}_{\rightarrow}}}
\newcommand{\alc}{\ensuremath{\mathcal{ALC}}}
\newcommand{\ialc}{\ensuremath{i\mathcal{ALC}}}
\newcommand{\system}[2]{\ensuremath{\textsf{#1}_{#2}}}
\newcommand{\nominal}[2]{\ensuremath{#1\colon{}\mkern-4mu #2}}

\newcommand{\slf}[2]{{\vphantom{#2}}^{#1}{#2}}
\newcommand{\negslf}[2]{\slf{\neg#1}{\neg#2}}
\newcommand{\Islf}[2]{\ensuremath{\sigma(\slf{#1}{#2})^{\mathcal{I}}}}

\newcommand{\reskolen}[2]{\ensuremath{\mathcal{F}_{#2}(#1)}}

\newcommand{\Tlf}[3]{\ensuremath{\T{\lf{#1}{#2}{#3}}}}
\newcommand{\Tilf}[3]{\ensuremath{\Ti{\lf{#1}{#2}{#3}}}}

\newcommand{\SEQ}{\ensuremath{\Rightarrow}}
\newcommand{\Seq}[2]{\ensuremath{#1 \SEQ #2}}

\newcommand{\dland}{\sqcap}        \newcommand{\dlor}{\sqcup}         \newcommand{\subs}{\sqsubseteq}    \newcommand{\conc}[1]{\ensuremath{\mathtt{#1}}} \newcommand{\role}[1]{\ensuremath{\mathtt{#1}}} 

\newcommand{\length}[1]{\ensuremath{\left| #1 \right|}}
\newcommand{\dl}{Description Logics}

\newcommand{\rl}[1]{\RightLabel{\scriptsize #1}}

\newenvironment{seqproof} 
{\begin{prooftree}\def\fCenter{\SEQ}}
{\end{prooftree}}
\def\fCenter{ \SEQ\ }

\newcommand{\nota}[1]{
   \marginpar{\em\bf \tiny
      \begin{flushleft} #1 \end{flushleft}
   }
}

\newenvironment{dedsystem}
{\renewcommand{\arraystretch}{1.2}
 \newcolumntype{Y}{>{\centering\arraybackslash}X}
 \setlength{\extrarowheight}{2.5ex}
 \tabularx{\textwidth}{YcY}}
{\endtabularx}

 \newcommand{\bm}[1]{\ensuremath{\mathbf{#1}}}

\title{How many times do we need an assumption ?}

\author{Edward Hermann Haeusler}

 
\institute{Dep. Inform\'atica \\ PUC-Rio \\ \texttt{hermann@inf.puc-rio.br}}

\date{}

\renewcommand\Authands{ and }

\begin{document}

\maketitle

\begin{abstract}
In this article we present a class of formulas , , that need at least  assumptions to be proved in a normal proof in Natural Deduction for purely implicational minimal propositional logic. In purely implicational classical propositional logic, with Peirce's rule, each  is proved with only one assumption in Natural Deduction in a normal proof. Hence, the formulas  have exponentially sized proofs in cut-free Sequent Calculus and  Tableaux. In fact  is the lower-bound for normal proofs in ND, cut-free Sequent proofs and Tableaux. We discuss the consequences of the existence of this class of formulas for designing automatic proof-procedures based on these deductive systems.  
\end{abstract} 



\section{Introduction}\label{intro}

Providing proofs for propositional tautologies seems to be a hard task.  Huge proofs are such that their size is  super-polynomial with regard to  the size of their conclusions.  Knowing that there is a classical propositional logic tautology having only huge proofs is related to know whether  or not (see \cite{Cook}). Intuitionistic logic is PSPACE-complete (\cite{Ladner}) and Richard Statman (see \cite{Statman}) showed that purely implicational minimal logic (\mil) is PSPACE-complete too. 
We showed in \cite{Haeusler} that, if a propositional logic has a Natural Deduction (ND) with the sub-formula property then it is PSPACE-Hard. This follows from the fact that \mil polynomially encodes any propositional logic that has such ND system.  Thus, 
the existence of huge proofs for a more general class of propositional logics is related to the existence of huge proofs in \mil that amounts to know whether  or not. The relations between these computational complexity classes and the existence of huge proofs involve arbitrary proof systems, indeed. For example,  is the case, if and only if, for any \mil tautology there is a proof system that produces a polynomially sized proof of this tautology. 

Dealing with arbitrary and general proof systems is quite hard, and is out of scope of this article. However, studying particular proof systems for key logics, like \mil or classical logic, can shed some light on practical aspects of implementing propositional theorem provers from the efficiency and economy of storage point of view.    

\mil carries almost all the proof-theoretical and logical information to produce polynomially bounded proofs in well-behaved\footnote{With sub-formula property} propositional logics. Thus we can conclude that focusing investigations on \mil is worth of noticing. 

There are many proof systems for \mil. The most well-known are structural/analytic  proof systems. Well-known systems are the Sequent Calculus (\cite{Gentzen}, Natural Deduction (\cite{Gentzen} and \cite{Prawitz}) and Tableaux (\cite{Beth,Smullyan}) based. These systems, mainly the first and the third kind, are quite good in providing means to produce proofs automatically. The backward chaining procedure, for example, if applied to a Sequent Calculus based proof system provides an automatic way to produce proofs. The problem with these proof procedures is when a decision on which rule to apply has to be made and how to deal with non-provable formulas when it is the case. With respect to this feature of dealing with invalid formulas, the  literature on both systems, Sequent Calculus and Tableaux, provides methods that either produce a proof or a counter-model. 

In \mil,  to provide a counter-model is so hard as to provide a proof, since it is a PSPACE-complete problem and the complexity class is deterministic. We know that \mil has finite model property and that the size of the counter-model is super-polynomially  upper bounded with respect to  the formula. It is interesting to investigate how this is related to the size of proofs in \mil, or at least to have a concrete evidence that huge proofs may be the case. Our intention is not only show huge proofs in \mil. To do that, we can use the polynomial translations reported in \cite{Statman} or \cite{Haeusler} to generate a formula from the Pigeon-Hole principle into \mil. We know, from \cite{Haken}, that this formula has only super-polynomially sized proofs in Resolution, and hence in cut-free Sequent Calculus and the same happens to the translation to \mil in cut-free Sequent Calculus and Natural Deduction. It is quite hard to detect from this approach to obtain huge proofs in \mil any precise reason for the super-size of the resulting proof. We believe that directly focusing on \mil is a more promising path, since \mil has less combinatorial alternatives, less logical constants, less alternative deductive system. The genesis of huge proofs in \mil is interesting and may shed some new light in propositional logic complexity. 

This article, motivated by the relationship between counter-model and proof construction in an integrated process,  improves a bit the knowledge on this subject by showing  a class of formulas in \mil that have proofs with at least exponentially-many assumptions, and hence are super-polynomially sized. The relation of these formulas with counter-model generation is discussed in section~\ref{counter-model}. In fact what is reported here was generated by the need of providing counter-models in a naive proof-procedure  for \mil based in Sequent Calculus as it is briefly discussed in section~\ref{counter-model}. In section~\ref{sec:formula} we introduce the class of formulas and in section~\ref{limits} we show  that  
they have exponentially sized normal proofs in the usual Natural Deduction for \mil. In the same section we also show that this is a lower bound in \mil. In classical propositional logic, these formulas have linear-sized proofs as it is shown in section~\ref{sec:formula}.

 All the formal propositional proofs/derivations in this article are presented in Prawitz-style Natural Deduction. The size of these normal proofs/derivations is polynomially simulated by cut-free Sequent Calculus and/or Tableaux. Thus, the lower bound shown here also applies to them.   



\section{The purely implicational minimal logic}

The (purely) implicational minimal logic \mil is the fragment of minimal logic containing only the logical constant . Its semantics is the intuitionistic Kripke semantics restricted to  only. Given propositional language , a \mil model is a structure , where  is a non-empty set (worlds),  is a partial order relation on  and  is a function from  into the power set of , such that if  and  then . Given a model, the satisfaction relationship  between worlds, in the model,  and formulas is defined as:
\begin{itemize}
\item , , iff, 
\item , iff, for every , such that , if  then .
\end{itemize}

Obs: In (full) minimal logic,  has no special meaning, so there is no item declaring that . We remind that \mil does not have the  in its language.  

As usual a formula  is valid in a model , namely , if and only if, it is satisfiable in every world  of the model, namely . A formula is a \mil tautology, if and only if, it is valid in every model. A formula is satisfiable in \mil if it is valid in a model  of \mil.   
The problem of knowing whether a formula is satisfiable or not is trivial in \mil. Every formula is satisfiable in the model 
, where  is the only world, and , for every . Thus,  is not an interesting problem in \mil. The same cannot be told about knowing whether a formula is a \mil tautology or not. 

It is known that Prawitz Natural Deduction system for minimal logic with only the -rules (-Elim and -Intro below)  is sound and complete for the \mil Krikpe semantics. As a consequence of this, Gentzen's  system (see \cite{Takeuti}) containing only right and left -rules is also sound and complete. As it is well-known one of these rules is not double-sounded  and not invertible. A naive proof-procedure for \mil based only on this usual Gentzen sequent calculus is not possible. 
\begin{prooftree} 
\AxiomC{}
\noLine
\UnaryInfC{}
\noLine
\UnaryInfC{}
\RightLabel{-Intro}
\UnaryInfC{}
\AxiomC{}
\AxiomC{}
\RightLabel{-Elim}
\BinaryInfC{}
\noLine
\BinaryInfC{}
\end{prooftree}

In section~\ref{counter-model} we discuss the consequences of the existence of the class of formulas presented in this article to obtain a naive, complete and sound proof-procedure for \mil.  

\section{Needing exponentially many assumptions}\label{sec:formula}

In \cite{GillesJiang} we can find a  discussion on the fact that when proving theorems in a logic weaker than classical logic, the need of using an assumption more than once has a strong influence on how complex is the proof procedure and consequently the decision procedure for this logic. There, we can find the formula . Considering the proof systems of ND and CS  mentioned in the previous section, this formula needs to use the assumption  at least twice in order to be proved in \mil. Inspired by this example, we can define a class of formulas with no bounds on the use of assumptions. This shows that limiting the use of assumptions in an automatic proof-procedure for \mil is not an alternative that ensures completeness. In the sequel we define the class of formulas. Below you find a normal proof of 
. Note that it cannot be proved with less than 2 assumptions of .




The following formula combines two instances of the formula mentioned above in order to have a formula that needs 4 times an assumption. 

, where 
.

 
In figure~\ref{prova} we show a normal derivation of this formula~\ref{formula} above. We can see that it has 4 assumptions of 
. We can see how to use this pattern such that if it is repeated n-times we define a formula , such that, any normal proof of  has to use an assumption at least  times, see section ~\ref{limits}. Before we proceed with  definition, we have to show that the need for repeating assumptions is not the case for classical propositional logic. 

Consider now that the logic is the purely implicational {\em classical} logic instead of the purely implicational minimal logic. That is, we consider the  introduction and elimination rules, plus the classical absurdity rule, or the Peirce's rule: from  then infer . Taking into account the version with Peirce's rule, we provide the proof of the formula~\ref{formula} with only use of assumption, as shown in figure~\ref{outraprova}. This comes from the fact that  in an instance of the implicational form of Peirce's rule, so it is provable. From this proof and  we prove .  itself is provable by means of a proof of the Peirce's formula  and the  discharged to proof the desired formula. The purely implicational classical logic is not the focus of this article, in \cite{Studia} and \cite{LeoGordeev} we can find a detailed presentation of the purely implicational classical logic with some  proof-theoretic results. Our discussion on the classical setting has the purpose of showing how the use of classical logic can, in some cases, turns proofs smaller.

\begin{landscape}
\begin{figure}[h]
{\tiny
\begin{prooftree}
\def\defaultHypSeparation{\hskip 0mm}
\insertBetweenHyps{\hskip -3cm}
\AxiomC{}
\UnaryInfC{}
\AxiomC{}
\UnaryInfC{}
\AxiomC{}
\BinaryInfC{}
\RightLabel{1}
\UnaryInfC{}
\AxiomC{}
\BinaryInfC{}
\RightLabel{2}
\UnaryInfC{}
\AxiomC{}
\BinaryInfC{}
\BinaryInfC{}
\RightLabel{}
\UnaryInfC{}
\AxiomC{}
\BinaryInfC{}
\RightLabel{4}
\UnaryInfC{}
\AxiomC{}
\noLine
\UnaryInfC{}
\noLine
\UnaryInfC{}
\BinaryInfC{}
\RightLabel{5}
\UnaryInfC{}
\noLine
\UnaryInfC{}
\end{prooftree}
\begin{prooftree}
\AxiomC{}
\noLine
\UnaryInfC{}
\noLine
\UnaryInfC{}
\AxiomC{is}
\AxiomC{}
\UnaryInfC{}
\AxiomC{}
\BinaryInfC{}
\RightLabel{1}
\UnaryInfC{}
\AxiomC{}
\BinaryInfC{}
\RightLabel{2}
\UnaryInfC{}
\AxiomC{}
\BinaryInfC{}
\noLine
\TrinaryInfC{}
\end{prooftree}
}
\caption{Proof of the formula in purely implicational minimal logic}\label{prova}
\end{figure}
\end{landscape}

\begin{figure}[h]
{\tiny
\begin{prooftree}
\AxiomC{}
\noLine
\UnaryInfC{}
\AxiomC{}
\noLine
\UnaryInfC{}
\AxiomC{}
\BinaryInfC{}
\BinaryInfC{}
\RightLabel{1}
\UnaryInfC{}
\end{prooftree}
}
\caption{Proof of the formula in purely implicational {\em classical} logic}\label{outraprova}
\end{figure}

\section{No bounds for occurrence  assumptions in \mil}
\label{limits}

In this section we prove that for each  there is a formula , such that, any normal proof of  has at least  occurrence assumptions of the same formula, that are all of them discharged in only one introduction rule. 
The following proposition~\ref{upper} shows that  is an upper bound by showing the normal proof that uses  assumptions for proving . Theorem~\ref{lower} shows that there is no normal proof for any of the , in \mil, with less than  assumptions discharged. In the sequel we define . As it was already said in section~\ref{sec:formula},  raises from an iteration process derived from the previous examples. 

\begin{definition}
Let . Using  we define recursively a family of formulas. Consider the propositional letters , , . Let , , be the formula recursively defined as: 

Using this family of formulas we define the formula , , such that, for any :

\end{definition}

We can observe that  can be proved by using proof , replacing  for  and  for , and applying an -introduction as the last rule. The obtained proof has 2 occurrence assumptions of the formula . The proof of  is the proof shown in figure~\ref{prova}, replacing  by ,  by  and  by  , resulting in the proof showed in figure~\ref{provaphi2}. Note that the discharged formula  is broke in two lines for reasons of space economy. 
The following lemma will be used in the proof of proposition~\ref{upper}.

\begin{lemma}\label{subs}
In the formula , , if we proceed in a simultaneous substitution, replacing  by , and for each , replacing  by , the resulting formula is . 
\end{lemma}

\begin{proof}
This lemma is proved by induction on . For  we observe that replacing  by  and  by  in , the resulting formula is . Assuming that for , replacing of  by  and, for each , simultaneously replacing   by  in ,  yields . Observing that  and by inductive hypothesis, simultaneous replacing of  by  and  by  in , , yields . As  does not occur in , finally replacing  by  in  yields . This proves the inductive step. 
\end{proof}

Another observation is that substitutions as the above shown in the lemma, if applied in a derivation  in \mil, do imply that the resulting tree is a valid derivation too. This fact is justified by observing that the replacements are always on atomic formulas and the rules of \mil do not have provisos to be unsatisfied as consequence of these replacements. Thus,we have the following fact.

\begin{fact}\label{fato}
If  is a derivation of  from  and a substitution  (of atomic formulas only) is applied to  then  is a derivation of  from . Besides that, if  is normal then  is normal too. 
\end{fact} 

\begin{landscape}
\begin{figure}[h]
{\tiny
\begin{prooftree}
\def\defaultHypSeparation{\hskip 0mm}
\insertBetweenHyps{\hskip -4cm}
\AxiomC{}
\UnaryInfC{}
\AxiomC{}
\UnaryInfC{}
\AxiomC{}
\noLine
\UnaryInfC{}
\BinaryInfC{}
\RightLabel{1}
\UnaryInfC{}
\AxiomC{}
\BinaryInfC{}
\RightLabel{2}
\UnaryInfC{}
\AxiomC{}
\noLine
\UnaryInfC{}
\BinaryInfC{}
\BinaryInfC{}
\RightLabel{}
\UnaryInfC{}
\AxiomC{}
\BinaryInfC{}
\RightLabel{4}
\UnaryInfC{}
\AxiomC{}
\noLine
\UnaryInfC{}
\noLine
\UnaryInfC{}
\noLine
\UnaryInfC{}
\BinaryInfC{}
\RightLabel{5}
\UnaryInfC{}
\noLine
\UnaryInfC{}
\end{prooftree}
\begin{prooftree}
\AxiomC{}
\noLine
\UnaryInfC{}
\noLine
\UnaryInfC{}
\AxiomC{is}
\AxiomC{}
\UnaryInfC{}
\AxiomC{}
\BinaryInfC{}
\RightLabel{1}
\UnaryInfC{}
\AxiomC{}
\BinaryInfC{}
\RightLabel{2}
\UnaryInfC{}
\AxiomC{}
\noLine
\UnaryInfC{}
\BinaryInfC{}
\noLine
\TrinaryInfC{}
\end{prooftree}
}
\caption{Proof of  in \mil}\label{provaphi2}
\end{figure}
\end{landscape}

As  has two () occurrences of the same assumption and  has four () occurrences of the same assumptions, we have the following result.

\begin{proposition}\label{upper}
For any , there is a normal proof of  having  occurrences of the same assumptions, that are  discharged by the last rule of the proof.  
\end{proposition}

\begin{proof}
The proof proceeds by induction. The basis  is the proof shown inside figure~\ref{provaphi2}. Assuming that ,  has a normal proof  having  occurrences of  discharged by its last inference rule. Thus, we have a normal derivation  of C from  occurrences of , remembering that . We argue that if we simultaneously replace  by , and for each , replace  by , we will have, by lemma~\ref{subs} and fact~\ref{fato}, a normal derivation of  from  occurrences of . Let us call this derivation . The following derivation (see figure~\ref{ind-hyp}) is a derivation of  from  
, i.e., it is a derivation of  from 
, and hence, by an -introduction of we have a normal derivation of  discharging  assumptions of the formula  
\begin{figure}[h]
{\tiny
\begin{prooftree}
\def\defaultHypSeparation{\hskip 0mm}
\insertBetweenHyps{\hskip -1cm}
\AxiomC{}
\UnaryInfC{}
\AxiomC{}
\noLine
\UnaryInfC{}
\noLine
\UnaryInfC{}
\BinaryInfC{}
\RightLabel{}
\UnaryInfC{}
\AxiomC{}
\BinaryInfC{}
\RightLabel{}
\UnaryInfC{}
\AxiomC{}
\noLine
\UnaryInfC{}
\noLine
\UnaryInfC{}
\BinaryInfC{}
\RightLabel{5}
\UnaryInfC{}
\end{prooftree}
}
\caption{Proof of  in \mil with  discharged assumptions of }\label{ind-hyp}
\end{figure}
\end{proof}
\begin{center}
Q.E.D.
\end{center}

The following proposition provides  as the lower bound for number of assumption occurrences of a sole formula in proving  by means of normal proofs in \mil.

\begin{theorem}\label{lower}
Any normal proof of  in \mil has at least  assumption occurrences of .
\end{theorem}

\begin{proof}
We prove that for any , there is no normal proof of  with less than  assumption occurrences of .
We first observe that , i.e.,   is not provable with 
only one occurrence of . If this was the case we would have
that  is provable in \mil, and this cannot be since this formula is only classically valid.
A Kripke model with two worlds such that in the first world neither  nor  holds and in second  holds but not  falsifies .

Consider that there are normal proofs of  with less than  assumption occurrences of . So there is the least  (), such that,  has a normal proof with less than  assumption occurrences of . Let  be such proof. Since , this proof is as follows. We remember that  is the only open assumption in . 
\begin{prooftree}
\AxiomC{}
\noLine
\UnaryInfC{}
\noLine
\UnaryInfC{}
\RightLabel{}
\UnaryInfC{}
\end{prooftree}
Since , it has to be major premise of an -elim rule. If this is not the case then  is minor premise of a -elim rule having a major premise of the form . This formula on its turn has to be sub-formula of the open assumption of this branch, for the derivation is normal and  can be only conclusion of an application of an -elim rule. Since the only open assumption in  is  itself, the case of  as minor premise is not possible. Thus, as  is major premise,  is of the following form, remembering how is , showed  in the first line of this paragraph.
\begin{prooftree}
\AxiomC{}
\noLine
\UnaryInfC{}
\AxiomC{}
\BinaryInfC{}
\noLine
\UnaryInfC{}
\noLine
\UnaryInfC{}
\RightLabel{}
\UnaryInfC{}
\end{prooftree}
Note that  is a sub-derivation of  and it may have  as open assumption too, but this is not necessary. If we remove every sub-derivation like  from  we end up with a proof as following:
\begin{prooftree}
\AxiomC{}
\noLine
\UnaryInfC{}
\noLine
\UnaryInfC{}
\RightLabel{}
\UnaryInfC{}
\end{prooftree}
The proof above is a proof of  with less than  assumption occurrences of  discharged by the last rule. This contradicts the fact that  is the least number holding this property. 
\end{proof}
\begin{center}
Q.E.D.
\end{center}

\section{Counter-model construction in Sequent Calculus for \mil}
\label{counter-model}
In this section we show how it is easy, from the computational aspect, to produce \mil counter-models from the following (incomplete) sequent calculus for \mil.
\begin{prooftree}
\AxiomC{}
\RightLabel{Axiom}
\UnaryInfC{\Seq{\Xi,p}{p,[\Delta]}}
\end{prooftree}



\begin{prooftree}
\AxiomC{\Seq{\Xi,\gamma_1}{\gamma_2,[\Delta]}}
\RightLabel{-right}
\UnaryInfC{\Seq{\Xi}{\gamma_1\imply\gamma_2,[\Delta]}}
\end{prooftree}

\begin{prooftree}
\AxiomC{\Seq{\Xi}{\alpha,[\gamma,\Delta]}}
\AxiomC{\Seq{\Xi,\beta}{\gamma}}
\RightLabel{-left}
\BinaryInfC{\Seq{\Xi,\alpha\imply\beta}{\gamma,[\Delta]}}
\end{prooftree}

The formulas in the right-hand side of the sequent and between the brackets are used only for counter-model construction. The main idea is that a sequent of the form  having all members of  as propositional letters and  is falsified in a Kripke model with only one world. From this case and using the reversible (if any premise is not valid the conclusion is too) rules of the system it is possible to build a polynomially sized Kripke model for the conclusion of the tree. Remember that in this case we do not have a proof. As already said, this system is incomplete, for it is unable to prove any of the formulas belonging to the class we presented here. The mentioned formulas are only provable if the -left rule used is the following, instead of the above shown. 

\begin{prooftree}
\AxiomC{\Seq{\Xi,\alpha\imply\beta}{\alpha,[\gamma,\Delta]}}
\AxiomC{\Seq{\Xi,\alpha\imply\beta,\beta}{\gamma}}
\RightLabel{-left}
\BinaryInfC{\Seq{\Xi,\alpha\imply\beta}{\gamma,[\Delta]}}
\end{prooftree}

In this case a counter-model generation is not so obvious, in fact we did not obtained one. If there were a bound on the use of repeated formulas, we could have used both versions of the -rule for a counter-model generation. Of course, for every formula there is a bound, for example the formula  the bound is 2. What we have shown  is that there is no fixed bound for every formula. In fact, if such a fixed bound existed we would have that every \mil formula would have a polynomially sized search-space to find either proofs or counter-models.  
  
\section{Conclusion}

Our contribution is in the context that \mil is the hardest and most representative propositional logic to define efficient proof-procedures. We show an example alerting for the fact that allowing unlimited use of assumptions is worth for any complete proof-procedure. This example runs in \mil. We are not aware of a similar example for classical logic. In this case classical propositional logic would be more efficient than \mil if such example does not existed. Propositional logic complexity has a lot of conjectures, starting with the relations between the main complexity classes. This article has the sole purpose of providing an example where the exponential grow of proofs has nothing to do with disjunction and combinatorial principles like the Pigeon-Hole\footnote{The pigeon-hole principle was used to provide a super-polynomial lower bound for Robinson's (propositional) Resolution}. We provided such example. 

Developers of theorem provers have to be aware of many aspects of the logic in order to design a efficient system. A system that saves memory and it is fast. Of course, dealing with PSPACE-complete problems is not a so easy task. Any information that can guide the designer is of help. Knowing that the number of copies of a formulas in a proof cab be a ``bottleneck''  for saving memory, an obvious solution would be the use of references instead of copies when representing proofs. The number of references is exponential, but references to formulas are smaller than formulas in most of the cases. This approach points out to the use of graphs (digraphs in fact) for representing proofs. There are a lot of developments done in this direction reported in the literature. Most of them are more semantically than implementation driven. Proof-nets (see \cite{Girard}) represents an approach that defends the use of graphs as the most adequate representation for proofs. We agree with that, but we add a practical motivation for considering digraphs instead of trees for representing proofs (see \cite{EPTCS}) 

\section{Acknowledgments} 

The author would like to thank prof. Gilles Dowek for hearing and reading the initial ideas presented here and providing very good suggestions. We want to thank Jefferson dos Santos for reading, pointing unclear explanations and helping improving the text. 

\bibliographystyle{plain}
\bibliography{example-more-twice}



\end{document}
