
\documentclass{llncs}
\usepackage{multicol}

\usepackage{amsmath}
\usepackage{amsfonts}
\usepackage{amssymb}
\usepackage{verbatim}

\usepackage{graphicx}
\usepackage{epic}
\usepackage{eepic}
\usepackage{epsfig,float}
\usepackage{bm}

\usepackage{multicol}


\pagestyle{plain}
\DeclareGraphicsRule{.tif}{png}{.png}{`convert #1 `dirname #1`/`basename #1 .tif`.png}

\renewcommand{\le}{\leqslant}
\renewcommand{\ge}{\geqslant}

\newcommand{\co}{companion}
\newcommand{\ol}{\overline}
\newcommand{\eps}{\varepsilon}
\newcommand{\emp}{\emptyset}
\newcommand{\rhoR}{R}
\newcommand{\Sig}{\Sigma}
\newcommand{\sig}{\sigma}
\newcommand{\noin}{\noindent}
\newcommand{\pf}{prefix-focused}
\newcommand{\ur}{uniquely reachable}
\newcommand{\bi}{\begin{itemize}}
\newcommand{\ei}{\end{itemize}}
\newcommand{\be}{\begin{enumerate}}
\newcommand{\ee}{\end{enumerate}}
\newcommand{\bd}{\begin{description}}
\newcommand{\ed}{\end{description}}
\newcommand{\txt}[1]{\mbox{ #1 }}

\newcommand{\etc}{\mbox{\it etc.}}
\newcommand{\ie}{\mbox{\it i.e.}}
\newcommand{\eg}{\mbox{\it e.g.}}
\newcommand{\FigureDirectory}{FIGS}

\newcommand{\inv}[1]{\mbox{}}

\newcommand{\stress}[1]{{\fontfamily{cmtt}\selectfont #1}}

\def\shu{\mathbin{\mathchoice
{\rule{.3pt}{1ex}\rule{.3em}{.3pt}\rule{.3pt}{1ex}
\rule{.3em}{.3pt}\rule{.3pt}{1ex}}
{\rule{.3pt}{1ex}\rule{.3em}{.3pt}\rule{.3pt}{1ex}
\rule{.3em}{.3pt}\rule{.3pt}{1ex}}
{\rule{.2pt}{.7ex}\rule{.2em}{.2pt}\rule{.2pt}{.7ex}
\rule{.2em}{.2pt}\rule{.2pt}{.7ex}}
{\rule{.3pt}{1ex}\rule{.3em}{.3pt}\rule{.3pt}{1ex}
\rule{.3em}{.3pt}\rule{.3pt}{1ex}}\mkern2mu}}

\newcommand{\bA}{{\mathbf A}}
\newcommand{\ba}{{\mathbf a}}
\newcommand{\bB}{{\mathbf B}}
\newcommand{\bC}{{\mathbf C}}
\newcommand{\bD}{{\mathbf D}}
\newcommand{\bE}{{\mathbf E}}
\newcommand{\bF}{{\mathbf F}}
\newcommand{\bG}{{\mathbf G}}
\newcommand{\bH}{{\mathbf H}}
\newcommand{\bI}{{\mathbf I}}
\newcommand{\bJ}{{\mathbf J}}
\newcommand{\bK}{{\mathbf K}}
\newcommand{\bQ}{{\mathbf Q}}
\newcommand{\bq}{{\mathbf q}}
\newcommand{\bR}{{\mathbf R}}
\newcommand{\bS}{{\mathbf S}}

\newcommand{\bmA}{\bm{A}}
\newcommand{\bmB}{\bm{B}}
\newcommand{\bmF}{\bm{F}}
\newcommand{\bmI}{\bm{I}}
\newcommand{\bmK}{\bm{K}}
\newcommand{\bmS}{\bm{S}}

\newcommand{\cA}{{\mathcal A}}
\newcommand{\cB}{{\mathcal B}}
\newcommand{\cC}{{\mathcal C}}
\newcommand{\cD}{{\mathcal D}}
\newcommand{\cE}{{\mathcal E}}
\newcommand{\cF}{{\mathcal F}}
\newcommand{\cI}{{\mathcal I}}
\newcommand{\cK}{{\mathcal K}}
\newcommand{\cL}{{\mathcal L}}
\newcommand{\cM}{{\mathcal M}}
\newcommand{\cN}{{\mathcal N}}
\newcommand{\cP}{{\mathcal P}}
\newcommand{\cQ}{{\mathcal Q}}
\newcommand{\cR}{{\mathcal R}}
\newcommand{\cS}{{\mathcal S}}
\newcommand{\cT}{{\mathcal T}}
\newcommand{\cU}{{\mathcal U}}
\newcommand{\cZ}{{\mathcal Z}}

\newcommand{\bbA}{{\mathbb A}}
\newcommand{\bbB}{{\mathbb B}}
\newcommand{\bbC}{{\mathbb C}}
\newcommand{\bbD}{{\mathbb D}}
\newcommand{\bbE}{{\mathbb E}}
\newcommand{\bbI}{{\mathbb I}}
\newcommand{\bbK}{{\mathbb K}}
\newcommand{\bbL}{{\mathbb L}}
\newcommand{\bbM}{{\mathbb M}}
\newcommand{\bbN}{{\mathbb N}}
\newcommand{\bbP}{{\mathbb P}}
\newcommand{\bbR}{{\mathbb R}}
\newcommand{\bbS}{{\mathbb S}}
\newcommand{\bbT}{{\mathbb T}}
\newcommand{\bbU}{{\mathbb U}}
\newcommand{\bbZ}{{\mathbb Z}}

\newcommand{\fA}{{\mathfrak A}}
\newcommand{\fB}{{\mathfrak B}}
\newcommand{\fC}{{\mathfrak C}}
\newcommand{\fD}{{\mathfrak D}}
\newcommand{\fK}{{\mathfrak K}}
\newcommand{\fL}{{\mathfrak L}}
\newcommand{\fN}{{\mathfrak N}}
\newcommand{\fM}{{\mathfrak M}}
\newcommand{\fQ}{{\mathfrak Q}}
\newcommand{\fR}{{\mathfrak R}}

\newcommand{\one}{{\mathbf 1}}

\newcommand{\Lra}{{\hspace{.1cm}\Leftrightarrow\hspace{.1cm}}}
\newcommand{\lra}{{\hspace{.1cm}\leftrightarrow\hspace{.1cm}}}
\newcommand{\la}{{\hspace{.1cm}\leftarrow\hspace{.1cm}}}
\newcommand{\raL}{{\hspace{.1cm}{\rightarrow_L} \hspace{.1cm}}}
\newcommand{\lraL}{{\hspace{.1cm}{\leftrightarrow_L} \hspace{.1cm}}}

\newcommand{\sn}{{semiautomaton}}
\newcommand{\sa}{{semiautomata}}
\newcommand{\Sn}{{Semiautomaton}}
\newcommand{\Sa}{{Semiautomata}}
\newcommand{\se}{{settable}}
\newcommand{\Se}{{Settable}}
\newcommand{\pc}{{prefix-continuous}}

\newcommand{\cover}{{version}}

\newcommand{\com}{\mathbb{C}}
\newcommand{\rev}{\mathbb{R}}
\newcommand{\deter}{\mathbb{D}}
\newcommand{\mini}{\mathbb{M}}
\newcommand{\trim}{\mathbb{T}}

\newcommand{\qedb}{\hfill} 

\newtheorem{open}[theorem]{Open problem}

\title{Minimal Nondeterministic Finite Automata and Atoms of Regular Languages
\thanks{This work was supported 
by the Natural Sciences and Engineering Research Council of Canada under grant No.~OGP0000871, 
the ERDF funded Estonian Center of Excellence in Computer Science, EXCS, 
and the Estonian Ministry of Education and Research target-financed 
research theme no. 0140007s12.}}
\author{Janusz~Brzozowski\inst{1} \and Hellis~Tamm\inst{2}}

\authorrunning{Brzozowski, Tamm}   

\institute{David R. Cheriton School of Computer Science, University of Waterloo, \\
Waterloo, ON, Canada N2L 3G1\\
\{{\tt brzozo@uwaterloo.ca}\}
\and
Institute of Cybernetics, Tallinn University of Technology,\\
Akadeemia tee 21, 12618 Tallinn, Estonia\\
\{{\tt hellis@cs.ioc.ee}\} 
}
\begin{document}
\maketitle

\begin{abstract}
We examine the NFA minimization problem in terms of atomic NFA's, that is, NFA's in which the right language  of every state is a union of atoms, where the atoms of a regular language  are non-empty intersections of complemented and uncomplemented left quotients of the language. We characterize all reduced atomic NFA's of a given language,  that is, those NFA's that have no equivalent states. Using atomic NFA's, we formalize Sengoku's approach to NFA minimization  and prove that his method fails to find all minimal NFA's. We also formulate the Kameda-Weiner NFA minimization in terms of quotients and atoms. 
\medskip

\noin
{\bf Keywords:}
regular language,
quotient,
atom,
atomic NFA,
minimal NFA
\end{abstract}

\section{Introduction}

Nondeterministic finite automata (NFA's) have played a major role in the theory of finite automata and regular expressions  and their applications ever since their introduction in 1959 by Rabin and Scott~\cite{RaSc59}.
In particular, the intriguing problem of finding NFA's with the minimal number of states has received much attention.
The problem was first stated by Ott and Feinstein~\cite{OtFe61} in 1961.
Various approaches have then been used over the years in attempts to answer this question; we mention a few examples here.
In 1970, Kameda and Weiner~\cite{KaWe70} studied this problem using matrices related to the states of the minimal deterministic finite automata (DFA's) for a given language and its reverse. 
In 1992, Arnold, Dicky, and Nivat~\cite{ADN92} used a ``canonical'' NFA. 
In the same year, Sengoku~\cite{Sen92} used ``normal'' NFA's and ``standard formed" NFA's.
In 1995, Matz and Potthoff~\cite{MaPo95} returned to the ``canonical'' automaton and introduced the ``fundamental'' automaton.
In 2003, Ilie and Yu~\cite{IlYu03} applied equivalence relations.
In 2005, Pol\'ak~\cite{Pol05} used the ``universal'' automaton.

Our approach is to use the recently introduced atoms and atomic 
languages~\cite{BrTa11} for this question;  
we briefly state some of their basic properties here.

The \emph{(left) quotient} of a regular language  over an 
alphabet  by a word  is the language 
.
It is well known that the number of states in the complete minimal 
deterministic finite automaton recognizing  is precisely 
the number of distinct quotients of . 
Also,  is its own quotient by the empty word , that is .
A \emph{quotient DFA} is a DFA uniquely determined by a regular language; its states correspond to left quotients. The quotient DFA is isomorphic to the minimal DFA.

An \emph{atom}\footnote{The definition in \cite{BrTa11} does not consider 
the intersection of all the complemented quotients to be an atom. 
Our new definition in~\cite{BrTa12} adds symmetry to the theory.}
of a regular language  with quotients  is any 
non-empty language of the form 
, 
where  is either  or , and  is the complement of  with respect to .   
If the intersection with all quotients complemented is non-empty, then it constitutes the \emph{negative} atom;  
all the other atoms are \emph{positive}.
Let the number of atoms be , and let the number of positive atoms be . 
Thus, if the negative atom is present, ; otherwise, .

So atoms of  are regular languages uniquely determined by . 
They are pairwise disjoint and define a partition of . 
Every quotient of  (including  itself)  
is a union of atoms, and  every quotient of an atom is a union of atoms.
Thus the atoms of a regular language are its basic building blocks. 
Also,  defines the same atoms as   . 
The \emph{\'atomaton} is an NFA uniquely determined by a regular language; its states correspond to atoms. 
An NFA is \emph{atomic} if the right language of every state is a union of atoms.
\smallskip

Our contributions are as follows: 
\be
\item 
We characterize all trim reduced atomic NFA's of a given language, where 
an NFA is reduced if it has no equivalent states. 
\item
We show that, if  is the minimal number of states of any NFA of a language,  then the language may have trim reduced atomic NFA's 
with as few as  states, and as many as  states.
\item
We demonstrate that the number of atomic minimal NFA's  can be 
as low as 1, or very high. For example, the language  with 3 
quotients has  atomic minimal NFA's, and additional non-atomic ones.
\item
We formalize the work of Sengoku~\cite{Sen92} in our framework. He had no concept of atoms, but used an NFA equivalent to the \'atomaton 
and NFA's equivalent to atomic NFA's.
Our use of atoms  significantly clarifies Sengoku's method.
\item
We prove that Sengoku's claim that an NFA can be made atomic by adding transitions and without changing the number of states is false.
We show that there exist languages for which the minimal NFA's are all non-atomic. 
So  Sengoku's claim that his method can always find a minimal NFA is also incorrect.
\item
We formulate the Kameda-Weiner NFA minimization method~\cite{KaWe70} in terms of 
quotients and atoms.
\ee 

In Section~\ref{sec:aut} we recall some properties of automata and \'atomata.
Atomic NFA's are then presented in Section~\ref{sec:atomic}.
Sengoku's method is studied in Section~\ref{sec:Sengoku}, and
the Kameda-Weiner method, in Section~\ref{sec:KW}.
Section~\ref{sec:conc} concludes the paper.


\section{Automata and \'Atomata of Regular Languages}
\label{sec:aut}

A~\emph{nondeterministic finite automaton (NFA)} is a quintuple 
, where 
 is a finite, non-empty set of \emph{states}, 
 is a finite non-empty \emph{alphabet}, 
 is the  \emph{transition function},
 is the set of  \emph{initial states},
and  is the set of \emph{final states}.
As usual, we extend the transition function to functions 
, and 
, but
use 
 for all three.

The \emph{language accepted} by an NFA  is 
.
Two NFA's are \emph{equivalent} if they accept the same language. 
The \emph{right language} of a state   is
.
The \emph{right language} of a set  of states of  is
; so
.
A~state is \emph{empty} if its right language is empty.
Two states   are \emph{equivalent} if their right 
languages are equal. 
An NFA is \emph{reduced} if it has no equivalent states.
The \emph{left language} of a state   is
.
A state is \emph{unreachable} if its left language is empty.
An NFA is \emph{trim} if it has no empty or unreachable states.
An NFA is \emph{minimal} if it has the minimal number of states among all
the equivalent NFA's.

A \emph{deterministic finite automaton (DFA)} is a quintuple 
, where
, , and  are as in an NFA, 
 is the transition function, 
and  is the initial state. 

We use the following operations on automata: \\
\hglue10pt 1.
The \emph{determinization} operation  
applied to an NFA  yields a DFA  obtained by 
the subset construction, where only subsets reachable 
from the initial subset of  are used, and the empty subset, 
if present, is included. \\
\hglue10pt 2.
The \emph{reversal} operation  applied to NFA  yields 
an NFA , where the sets of initial and final states are 
interchanged and all transitions are reversed. \\
\hglue10pt 3.
The \emph{trimming} operation  applied to an NFA deletes all unreachable and empty states.
\smallskip



The following theorem is from~\cite{Brz63}, and was also discussed in~\cite{BrTa11}: 

\begin{theorem}[Determinization]
\label{thm:Brz}
If  is a DFA accepting a language , then  is 
a minimal DFA for .
\end{theorem}

Let  be any non-empty regular language, and let
its set of quotients be . 
One of the quotients of  is  itself;
this is called the \emph{initial} quotient and is denoted by .
A quotient is  \emph{final} if it contains the empty word .
The set of final quotients is  .

In the following definition we use a 1-1 correspondence 
 between quotients  of 
a language  and the states  of the \emph{quotient DFA} 
 defined below.
We refer to the  as \emph{quotient symbols}.
\begin{definition}
\label{def:quotientDFA}
\vskip-0.1cm
The \emph{quotient DFA} of  is 
, where
,
 corresponds to ,
, and 
 if and only if 
, for all  and .
\end{definition}

In a quotient DFA
the right language of  is , and 
its left language 
is .
The  language   is the right language of , and hence 
.
DFA  is minimal, since all quotients in  are distinct.


It follows from the definition of an atom, that a regular language   has at most  atoms. 
An atom is \emph{initial} if it has  (rather than ) as a term;
it is \emph{final} if it contains~.
Since  is non-empty, it has at least one quotient containing~. 
Hence it has exactly one final atom, the atom 
, where 
 if , and  otherwise.
Let   be the set of atoms of .
By convention,  is the set of initial atoms,   is the final atom and the negative atom, if present, is .
The negative atom  is not reachable from  and can never be final, since there must be at least one final quotient in its intersection.

As above, we use a 1-1 correspondence 
 between atoms  of a language  and 
the states  of the NFA  defined below.
We refer to the  as \emph{atom symbols.}

\begin{definition}
\label{def:atomaton}
The \emph{\'atomaton} of 
 is the NFA 
 where ,
 , 
  corresponds to ,
 and  if and only if 
, for all  and .
\end{definition}

In the \'atomaton, the right language of any state  is the atom .
\smallskip

The results from~\cite{BrTa11} and  our definition of 
atoms in~\cite{BrTa12} imply that  is a minimal DFA 
that accepts . It follows from Theorem~\ref{thm:Brz} that  
 is isomorphic to .
The following result from~\cite{BrTa12} makes this isomorphism precise:

\begin{theorem}[Isomorphism]
\label{thm:isomorphism}
Let  be the collection of all subsets of the set  of quotient symbols.
Let  be the mapping assigning to state 
, corresponding to
 of , the set 
.
Then  is a DFA isomorphism between  and 
. 
\end{theorem}

\begin{corollary}
\label{cor:isomorphism}
The mapping  is an NFA isomorphism between 
 and .
\end{corollary}



\section{Atomic NFA's}
\label{sec:atomic}


A new class of NFA's was defined in~\cite{BrTa11} as follows:   

\begin{definition}
\label{def:atomic}
An NFA  is \emph{atomic} if for every  
, the right language  of  is a union of some positive  atoms of . 
\end{definition}

The following theorem, slightly restated, was proved in~\cite{BrTa11}:
\newpage

\begin{theorem}[Atomicity]
\label{thm:atomic}
A trim NFA  is atomic if and only if 
is minimal.
\end{theorem}

This theorem allows us to test whether an NFA  accepting a language  is atomic. To do this, reverse  and apply the subset construction. Then  is atomic if and only if 
is isomorphic to the minimal DFA of .

All three possibilities for the atomic nature of  and  exist:
NFA  of Table~\ref{tab:Na} and its reverse are not atomic.
NFA  of Table~\ref{tab:Nb} is atomic, but its reverse is not.
NFA  of Table~\ref{tab:Nc} and its reverse are both atomic.
Note that all three of these NFA's are equivalent, and they accept .
\begin{table}[t]
\begin{minipage}[b]{0.3\linewidth}
\caption{.}
\label{tab:Na}
\begin{center}

\end{center}
\end{minipage}
\hspace{0.3cm}
\begin{minipage}[b]{0.3\linewidth}
\caption{.}
\label{tab:Nb}
\begin{center}

\end{center}
\end{minipage}
\hspace{0.3cm}
\begin{minipage}[b]{0.3\linewidth}
\caption{.}
\label{tab:Nc}
\begin{center}

\end{center}
\end{minipage}

\end{table}


If we allow equivalent states, there is an infinite number of atomic NFA's, 
but their behaviours are not distinct; hence we consider only reduced NFA's.
Suppose  is any trim reduced atomic NFA accepting .
Since  is atomic, the right language of any state in  is a union of positive atoms
of ; hence the states of  can be represented by sets of positive atom symbols.
Because  is trim, it does not have a state with the empty set of atom symbols.
Since  is reduced, no set of atom symbols appears twice.
Thus the state set  is a collection of non-empty sets of positive atom symbols.


\begin{theorem}[Legality]
\label{thm:unions}
Suppose  is a regular language,  its \'atomaton is
, and
 is a trim NFA, where 
 is a collection of sets of positive atom symbols and 
.
If , define 
 to be the set of atom symbols 
appearing in the sets   of . 
Then  is a reduced atomic NFA of  if and only if it satisfies the following
conditions:
\be
\item
\label{cond:in}
.
\item
\label{cond:trans}
For all , .
\item
\label{cond:out}
For all , we have  if and only if 
.
\ee
\end{theorem}

Before proving the theorem, we require the following lemma:

\begin{lemma}
\label{lem:beta}
If  satisfies Condition~\ref{cond:trans} of Theorem~\ref{thm:unions}, then
 
for every  and .
\end{lemma}
\begin{proof}
For , we have , and 
; so the claim holds for this case.

Assume that 
for all  and all  with length less than or 
equal to . 
We prove that  for every .
Let  for
some . 
Since , we have
.
By   Condition~\ref{cond:trans}, the latter is equal 
to .
By the inductive assumption, we get
, which proves our claim.
\qed
\end{proof}
\noin
{\bf Proof of Theorem~\ref{thm:unions}}
\begin{proof}
First we prove that any NFA  satisfying
Conditions~\ref{cond:in}--\ref{cond:out} is an atomic NFA of .
Let  be a state of . 
If , then by 
Condition~\ref{cond:out}, there exists 
 such that , and 
we have . 
By Lemma~\ref{lem:beta}, we get , 
implying that there is some  such that 
. 
Conversely, if  and , then 
. 
Hence there exists  such that . 
Consequently, every word accepted in  from state  is 
in some atom  such that , and 
every word in an atom  such that , is also in 
.
Therefore the right language of  in  is equal to 
the union of atoms  such that .
In particular,  is the union of atoms whose atom symbols
appear in the initial collection of  which, by Condition~\ref{cond:in},  
is the same as the union of atoms whose atom symbols are initial in .
But that last union is precisely .
Since any two sets  and  are different, and  
atoms are disjoint,  is reduced.
Hence  is a reduced atomic NFA of .

Conversely, we show that if  is a reduced atomic NFA of , 
then it must satisfy Conditions~\ref{cond:in}--\ref{cond:out}.
So in the following we assume that  is atomic, that is, 
for every state  of , the right language of  
is equal to the union of atoms  such that .


First, we show that Condition~\ref{cond:in} holds.
Let . Then there is a state  such that 
. So for any , . 
Since , we have  for all .
Thus .
Conversely, if , then for all , . 
Since  is atomic, there is an initial state  such that 
. Hence . 

Next, we prove Condition~\ref{cond:trans}. 
If , then  must contain . 
So there must exist some  such that .
Thus .
Conversely, if , then there is an atom 
 such that , implying .
Since ,  must contain .
Hence .

To show that Condition~\ref{cond:out} holds, we first suppose that
. Then  is in the right language of .
Since  is atomic,  must be in one of the atoms of .
However, the only atom containing  is , so .
Conversely, if , then  is in the right language of  
, and  is a final state by definition of an NFA.
\qed
\end{proof}

\begin{example}
\label{ex:atomconstr}
Consider the trim \'atomaton  of Table~\ref{tab:deflegal1} and 
the atomic NFA  of Table~\ref{tab:deflegal2}.
Here , where ,
, and 
.
The initial collection is , and the final
collection is 
.
One verifies that all the conditions of Theorem~\ref{thm:unions} hold,
and  NFA's  and  are equivalent.
\begin{table}[t]
\begin{minipage}[b]{0.37\linewidth}
\caption{\'Atomaton .}
\label{tab:deflegal1}

\end{minipage}
\hspace{.5cm}
\begin{minipage}[b]{0.55\linewidth}
\caption{Atomic NFA .}
\label{tab:deflegal2}

\end{minipage}
\end{table}
\qedb
\end{example}


The number of trim reduced atomic NFA's can be very large. 
There can be such NFA's with as many as  non-empty states, 
since there are that many non-empty sets of positive atoms. 
However, in a general case, not all sets of positive atom symbols can be states of 
an atomic NFA. 
The largest reduced atomic NFA is characterized in the following theorem.

\begin{theorem}[Maximal atomic NFA]
\label{thm:subsets}
If  is the collection of all sets  such that 
 is a non-empty subset of the set of positive atom symbols 
 of any quotient  of , 
then there exists a trim reduced atomic NFA of  with  state set .
\end{theorem}
\begin{proof}
Let  be an NFA in which the state set 
 is the collection of all sets  such that 
 is a non-empty subset of the set of atom symbols 
 of any quotient  of , 
where ,

for every  and ,
 if and only if  is a subset of the set of atom symbols 
of the initial quotient , and 
 if and only if . 
We claim that  is a trim reduced atomic NFA of .

First, we show that  is trim. 
Let us consider any state  of .
Let  be a quotient such that  is a subset of the set of atom symbols
of , and let  be the set of atom symbols corresponding to .
Let  be the set of atom symbols corresponding to the initial
quotient  of . Note that .
Since every set of atom symbols corresponding to some quotient is reachable
from the initial set of atom symbols in the \'atomaton , 
there must be a word , such that  is reachable 
from  by  in . 
We show that  is reachable from some initial state of  by .
If , then , and since , it follows 
that  is an initial state of  reachable from itself 
by .
If  for some  and , then there is a state
 of , reachable from  by , such that 
 corresponds to the quotient  of  and 
. 
Since  and , 
by the definition of  we have 
. 
Thus,  is reachable from  in  by .

We also have to show that there is a word , such that some 
final state of  is reachable from  by .
If  is final, then it is reachable from itself by .
If  is not final, then let us consider any . 
Since the right language of the state  in the \'atomaton  
is not empty, and  cannot be the final state of , there must be 
some state  of  and some , such that 
.
Now we know that there is some  such that  and
. Since  is the collection
of all non-empty subsets of , it follows that 
.
Since the final state  of  is reachable from  by 
any word , we get 
 by the definition of . 
So a final state  of  is reachable from  by .   
Thus,  is trim.

To see that  is a reduced atomic NFA, one verifies that 
Conditions~\ref{cond:in}--\ref{cond:out} of Theorem~\ref{thm:unions} hold.
Thus by Theorem~\ref{thm:unions},  is a trim reduced atomic NFA of .
\qed
\end{proof} 


\begin{theorem}[NFA with  states]
\label{thm:maximal}
A regular language  has a trim reduced atomic NFA with  states if and 
only if for some quotient  of , .
\end{theorem}

\begin{proof}
Let  be a trim reduced atomic NFA of  
with  states. Then there must be a state  of  such that
. Since the right language of any state 
of a trim NFA is a subset of some quotient, we have
 for some
quotient  of .
On the other hand,  must be a union of some positive atoms, 
so we get .

Conversely, let  be a quotient of 
which includes all the positive atoms of . Then 
by Theorem~\ref{thm:subsets}, there is a trim reduced atomic NFA of 
in which the state set is the collection of all non-empty subsets of the set 
of positive atom symbols. This NFA has  states. 
\qed
\end{proof} 




The construction of reduced atomic NFA's is illustrated in the following example.
To simplify the notation, we do not use atom symbols in examples.

\begin{example}
\label{ex:reducedatomic}
Consider the minimal DFA  taken from~\cite{KaWe70} and  shown in 
Table~\ref{tab:dkw}.
It accepts the language , and its quotients are
, 
, and
.
NFA  and the isomorphic trim \'atomaton  with states renamed  are shown in Tables~\ref{tab:drdrkw} and~\ref{tab:akw}.
The positive atoms are
,  and , and
, 
,
and .


\begin{table}[b]
\begin{minipage}[b]{0.25\linewidth}
\caption{.}
\label{tab:dkw}
\begin{center}

\end{center}
\end{minipage}
\hspace{0.05cm}
\begin{minipage}[b]{0.25\linewidth}
\caption{.}
\label{tab:drdrkw}
\begin{center}

\end{center}
\end{minipage}
\hspace{1.5cm}
\begin{minipage}[b]{0.25\linewidth}
\caption{.}
\label{tab:akw}
\begin{center}

\end{center}
\end{minipage}
\end{table}

\begin{table}[hbt]
\begin{minipage}[b]{0.45\linewidth}
\caption{NFA .}
\label{tab:b1}
\begin{center}

\end{center}
\end{minipage}
\hspace{1cm}
\begin{minipage}[b]{0.45\linewidth}
\caption{Atomic NFA .}
\label{tab:b2}
\begin{center}

\end{center}
\end{minipage}
\end{table}




\begin{table}[t]
\begin{minipage}[b]{0.45\linewidth}
\caption{A 5-state NFA.}
\label{tab:b3}
\begin{center}

\end{center}
\end{minipage}
\hspace{.4cm}
\begin{minipage}[b]{0.45\linewidth}
\caption{A 7-state NFA.}
\label{tab:b4}
\begin{center}

\end{center}
\end{minipage}
\end{table}

Since the set  of initial atoms does not contain all positive atoms, no 1-state NFA exists.
\be
\item 
For the initial state we could pick one state  with two atoms.  From there, the \'atomaton reaches 
 under , and   under . 
        \be
        \item
If we pick 
as the second state,  we can cover  by  and 
, as  in Table~\ref{tab:b1}.
Here the minimal atomic NFA is unique.
        \item
        We can also use  as a state. Then we need 
        for the transition under . This gives an NFA  isomorphic to the DFA of Table~\ref{tab:dkw}.
        \item
        We can use state 
        as shown in Table~\ref{tab:b2}.
        \ee
\item
We can pick two initial states,  and . 
        \be
        \item
        If we add , this leads to the  \'atomaton of Table~\ref{tab:akw}.
        \item
        A 5-state solution is shown in Table~\ref{tab:b3}.
        \ee
\item
We can use three initial states, ,  and . 
        A 7-state NFA is shown in  Table~\ref{tab:b4}. This 
           is a largest possible reduced solution.\qedb
\ee

\end{example}


The number of minimal atomic NFA's can also be very large. 
\begin{example}
\label{ex:atomicminimal}
Let  and consider the language .
The quotients of  are ,  and .
The quotient DFA of  is shown in Table~\ref{tab:d}, and its \'atomaton, in Tables~\ref{tab:a} and~\ref{tab:a_relabel} (where the atoms have been relabelled). 
The atoms  are ,  and , and there is no negative atom.
Thus the quotients are , , and .

We find all the minimal atomic NFA's of .
Obviously, there is no 1-state solution.
The states of any atomic NFA are sets of atoms, and 
there are seven non-empty sets of atoms to choose from. 
Since there is only one initial atom, there is no choice: we must take .
For the transition , we can add  or . 
If there are only two states, atom  cannot be reached. So there is no  2-state atomic NFA.
The results for 3-state atomic NFA's  are summarized in Proposition~\ref{prop:281}. 


\begin{table}[hbt]
\begin{minipage}[b]{0.3\linewidth}
\caption{DFA .}
\label{tab:d}
\begin{center}

\end{center}
\end{minipage}
\hspace{0.1cm}
\begin{minipage}[b]{0.3\linewidth}
\caption{\'Atomaton .}
\label{tab:a}
\begin{center}

\end{center}
\end{minipage}
\hspace{0.3cm}
\begin{minipage}[b]{0.3\linewidth}
\caption{ relabelled.}
\label{tab:a_relabel}
\begin{center}

\end{center}
\end{minipage}
\end{table}

\begin{table}[hbt]
\begin{minipage}[b]{0.45\linewidth}
\caption{NFA .}
\label{tab:fn1}
\begin{center}

\end{center}
\end{minipage}
\hspace{0.2cm}
\begin{minipage}[b]{0.45\linewidth}
\caption{NFA .}
\label{tab:fn9}
\begin{center}

\end{center}
\end{minipage}
\end{table}

\begin{proposition} 
\label{prop:281}
The language  has 281 minimal atomic NFA's. 
\end{proposition}
\begin{proof}
We concentrate on 3-state solutions. 
We drop the curly brackets and commas and represent sets of atoms by words. Thus  stands for 
.

State  is the only initial state and so it must be included.
To implement the transition
 from ,
either  or  must be chosen. 
\be
\item
If  is chosen, then there must be a set containing  but not ; otherwise 
the transition 
  cannot be realized.
        \be
        \item
        If  is taken, then  must be taken, and this would make four states.
        \item
         Hence  must be chosen, giving states , , and .
         This yields the \'atomaton .
         \ee
\item
If  is chosen, then we could choose ,  or , since  would also require
 . Thus there are three cases:
        \be
        \item
         yields  of Table~\ref{tab:fn1}, if the minimal number of 
        transitions is used. 
        The following transitions can also be added: , , .
        Since these can be added independently, we have eight more NFA's. 
        Using the maximal number of transitions, we get  of Table~\ref{tab:fn9}.
        \item
         results in  with the minimal number of transitions, and 
         with the maximal one.
        \item
         results in  (the quotient DFA) with the minimal number 
of transitions, and   with the maximal one.
        \ee
\ee


\begin{table}[hbt]
\begin{minipage}[b]{0.45\linewidth}
\caption{NFA .}
\label{tab:fn10}
\begin{center}

\end{center}
\end{minipage}
\hspace{0.2cm}
\begin{minipage}[b]{0.45\linewidth}
\caption{NFA .}
\label{tab:fn25}
\begin{center}

\end{center}
\end{minipage}
\end{table}

\begin{table}[h]
\begin{minipage}[b]{0.45\linewidth}
\caption{NFA .}
\label{tab:fn26}
\begin{center}

\end{center}
\end{minipage}
\hspace{0.2cm}
\begin{minipage}[b]{0.45\linewidth}
\caption{NFA .}
\label{tab:fn281}
\begin{center}

\end{center}
\end{minipage}
\end{table}
\begin{table}[htb]
\caption{NFA .}
\label{tab:na282}
\begin{center}

\end{center}
\end{table}

As well,  has 3-state non-atomic NFA's.
The determinized version of NFA  of Table~\ref{tab:fn10} is not minimal.
By Theorem~\ref{thm:atomic},  is not atomic. But ;
hence we obtain a non-atomic 3-state NFA for  by reversing  and interchanging  and .
That NFA with renamed states is shown in Table~\ref{tab:na282}.

The right languages of the states of  are:
, 
, and
, which is not a union of atoms.
Six more non-atomic NFA's can be derived from NFA's between  and  .
\qed
\end{proof}


This is a rather large number of NFA's for a language with  3 quotients. 
\qedb
\end{example}

One can verify that there is no NFA with fewer than 3 states which
accepts the language .
This implies that every minimal atomic NFA of  is also 
a minimal NFA of .
However, this is not the case with all regular languages, as we will see 
in the next section. 


\section{Sengoku's NFA Minimization Method}
\label{sec:Sengoku}

Sengoku had no concept of atom, but he came very close to discovering it.
For a language accepted by a minimal DFA , the \emph{normal} 
NFA~\cite{Sen92}(p.~18) is isomorphic to , and hence to 
the trim \'atomaton, by our Corollary~\ref{cor:isomorphism}. 
Moreover, he defines an NFA  to be in \emph{standard form}~\cite{Sen92}(p.~19) 
if  is minimal.
By our Theorem~\ref{thm:atomic}, such an  is atomic.
Sengoku makes the following claim~\cite{Sen92}(p.~20):
\begin{quote}
\vskip-0.1cm
\emph{We can transform the nondeterministic automaton into its standard form 
by adding some extra transitions to the automaton. Therefore the number of 
states is unchangeable.}
\end{quote}
\vskip-0.1cm
This claim amounts to stating that any NFA can be transformed to an equivalent  
atomic NFA by adding some transitions. Unfortunately, the claim is false:
\begin{theorem}
\label{thm:Sengoku}
There exists a language for which no minimal NFA is atomic.
\end{theorem}
\begin{proof}
\vskip-0.1cm
This example is from~\cite{MaPo95}. 
A quotient DFA , the NFA , and its isomorphic \'atomaton 
 with relabelled states are  in Tables~\ref{tab:d_mp}--\ref{tab:a_mp}, 
respectively (there is no negative atom). We  now drop the curly brackets and commas in tables, and 
represent sets of atoms by words.
A minimal NFA  of this language, having four states, is shown in 
Table~\ref{tab:n_mp}; it is not atomic and it is not unique. 
We try to construct a 4-state atomic NFA  equivalent to . 

\vskip-0.5cm
\begin{table}[hbt]
\begin{minipage}[b]{0.25\linewidth}
\caption{.}
\label{tab:d_mp}
\begin{center}

\end{center}
\end{minipage}
\hspace{0.05cm}
\begin{minipage}[b]{0.35\linewidth}
\caption{.}
\label{tab:drdr_mp}
\begin{center}

\end{center}
\end{minipage}
\hspace{0.8cm}
\begin{minipage}[b]{0.3\linewidth}
\caption{ .}
\label{tab:a_mp}
\begin{center}

\end{center}
\end{minipage}
\end{table}
\vskip-0.4cm
First, we note that quotients corresponding to the states of  can be expressed 
as sets of atoms as follows:
, , , ,
, , , , and
. One can verify that these are the states of the determinized 
version of the \'atomaton, which is isomorphic to the original DFA . 
Now, every state of  must be a subset of a set of atoms of some quotient, 
and all these sets of atoms of quotients must be covered by the states of .
We note that quotients , , and 
do not contain any other quotients as subsets, while all the other quotients do.
It is easy to see that there is no combination of three or fewer sets of atoms, 
other than these three sets, that can cover these quotients. 
So we have to use these sets as states of . 
We also need at least one set containing the atom . 
If we use only one set of atoms with , that set has to be
a subset of every quotient having . So it must be
a subset of . If we use  as a state, then by the transition 
table of the \'atomaton, there must be at least one more state to cover 
. Similarly, if we use , then we must have another state to cover 
. If we use , then we must have a state to cover 
. And if we use , then we must have a state to cover 
. We conclude that a smallest atomic NFA has at least five states.
There is a five-state atomic NFA, as 
shown in Table~\ref{tab:n5_mp}. It is not unique. 

Since there does not exist a four-state atomic NFA equivalent to the DFA ,
it is not possible to convert the non-atomic 
minimal NFA  to an atomic NFA by adding transitions.
\qed
\end{proof}

\begin{table}[t]
\vskip-0.5cm
\begin{minipage}[b]{0.3\linewidth}
\caption{NFA .}
\label{tab:n_mp}
\begin{center}

\end{center}
\end{minipage}
\hspace{0.5cm}
\begin{minipage}[b]{0.45\linewidth}
\caption{.}
\label{tab:n5_mp}
\begin{center}

\end{center}
\end{minipage}
\vskip-0.3cm
\end{table}

\vskip-0.1cm
In summary, Sengoku's method cannot find the minimal NFA's in all cases. 
However, it is able to find all atomic minimal NFA's.
His minimization algorithm proceeds by 
``merging some states of the normal nondeterministic automaton.''
This is similar to our search for subsets of atoms that satisfy 
Theorem~\ref{thm:unions}.

\section{The Kameda-Weiner Minimization Method}
\label{sec:KW}

We present a short and modified outline of the properties of the Kameda-Weiner 
NFA minimization method~\cite{KaWe70} using mostly our terminology and notation. 
They consider a trim minimal DFA  with  of 
cardinality , and its reversed  determinized and trim version ; 
the set of states of  is a subset  of cardinality  of 
. 
They then 
form an  matrix  where the rows correspond to non-empty states  of , 
which is the trim minimal DFA of a language , 
and columns, to states  of , 
which is the trim minimal DFA of the language  by Theorem~\ref{thm:Brz}.
The entry  of the matrix  is 1 if , and 0 otherwise.

We use , the trim \'atomaton,  instead of , 
since the state sets of these two automata are identical.  
Interpret the rows of the matrix as non-empty quotients of  and columns, 
as positive atoms of . Then  if and only if quotient  contains 
atom , and it is clear that every regular language defines a 
unique such matrix, which we will refer to as the \emph{quotient-atom matrix}.

The ordered pair  with  and  is a \emph{point} 
of  if .
A \emph{grid}  of  is the direct product  of a set  of quotients with a set  of atoms.
If  and  are two grids  of , 
then  if and only if  and .
Thus  is a partial order on the set of all grids of , 
and a grid is \emph{maximal} if it is not contained in any other grid.
A \emph{cover}  of  is a set  of grids,  
such that every point  belongs to some grid  in .
A \emph{minimal cover}  has the minimal number
of grids.

Let  be the function that assigns to quotient 
 the set of grids  such that .
The NFA constructed by the Kameda-Weiner method is , 
where  is a cover consisting of maximal grids, 
 is the set of grids corresponding to the initial quotient
, and  is defined by  if and only if  
implies that  is a final quotient.
For every grid  and , we can compute 
 by the formula .


It may be the case that  is not equivalent to DFA .
A cover  is called \emph{legal} if .
To find a minimal NFA of a language ,
the method in~\cite{KaWe70} 
tests the covers of the quotient-atom matrix of  in the order of 
increasing size to see if they are legal. 
The first legal NFA is a minimal one.

When we apply the Kameda-Weiner method~\cite{KaWe70} to the  example in 
Theorem~\ref{thm:Sengoku}, we get the NFA of Table~\ref{tab:n_mp}.


We apply the Kameda-Weiner method~\cite{KaWe70} to the  example in Theorem~\ref{thm:Sengoku}.
The quotients in the example are referred to as the integers 0--8, as in Table~\ref{tab:d_mp}.
The atoms are those in Table~\ref{tab:drdr_mp} relabelled as in Table~\ref{tab:a_mp}. 
The quotient-atom matrix is shown in Table~\ref{tab:c_mp}, where the non-blank entries are to be interpreted as 1's and the blank entries as 0's. 
Table~\ref{tab:c_mp} also shows a minimal cover  and  for each quotient  of .



\begin{table}[hbt]
\caption{Cover  for quotient-atom matrix of .}
\label{tab:c_mp}
\begin{center}

\end{center}
\end{table}


The construction of the NFA  is shown in Table~\ref{tab:con_mp}.
For each grid , we show its set of quotients , with
 replaced by .
For each input , we give , and then the intersection 
of the  for . 
For example, the set  for  is expressed as ,
the set of quotients  of the set  by  is , and 
.
Table~\ref{tab:n_mp} shows the constructed NFA , 
where 's are replaced by 's.
Since  is equivalent to ,   is a legal cover.
However,  is not atomic, since the right language of 
state  is not a union of atoms, although it includes atoms  and  as
its subsets. The right languages of the other states of  are sets of atoms:
, 
, and   
.


\begin{table}
\caption{Construction of NFA .}
\label{tab:con_mp}
\begin{center}

\end{center}
\end{table}



We believe that NFA's defined by grids are a topic for future research.


\section{Conclusions}
\label{sec:conc}
We have studied the properties of atomic NFA's.
We have shown that atoms play an important role in NFA minimization and
proved that it is not enough to search for atomic NFA's only.




\begin{thebibliography}{10}
\providecommand{\url}[1]{\texttt{#1}}
\providecommand{\urlprefix}{URL }

\bibitem{ADN92}
Arnold, A., Dicky, A., Nivat, M.: A note about minimal non-deterministic
  automata. Bull. EATCS  47,  166--169 (1992)

\bibitem{Brz63}
Brzozowski, J.: Canonical regular expressions and minimal state graphs for
  definite events. In: Proc. Symp. on Mathematical Theory of Automata. MRI
  Symposia Series, vol.~12, pp. 529--561. Polytechnic Institute of Brooklyn,
  N.Y. (1963)

\bibitem{BrTa11}
Brzozowski, J., Tamm, H.: Theory of \'atomata. In: Mauri, G., Leporati, A.
  (eds.) DLT 2011. LNCS, vol. 6795, pp. 105--116. Springer (2011)

\bibitem{BrTa12}
Brzozowski, J., Tamm, H.: Quotient complexities of atoms of regular languages.
  In: Yen, H.C., Ibarra, O. (eds.) DLT 2012. LNCS, vol. 7410, pp. 50--61.
  Springer (2012)

\bibitem{IlYu03}
Ilie, L., Yu, S.: Reducing {NFA}s by invariant equivalences. Theoret. Comput.
  Sci.  306,  373--390 (2003)

\bibitem{KaWe70}
Kameda, T., Weiner, P.: On the state minimization of nondeterministic automata.
  IEEE Trans. Comput.  C-19(7),  617--627 (1970)

\bibitem{MaPo95}
Matz, O., Potthoff, A.: Computing small finite nondeterministic automata. In:
  Engberg, U.H., Larsen, K.G., Skou, A. (eds.) Proc. Workshop on Tools and
  Algorithms for Construction and Analysis of Systems. pp. 74--88. BRICS,
  Aarhus, Denmark (1995)

\bibitem{OtFe61}
Ott, G., Feinstein, N.: Design of sequential machines from their regular
  expressions. J. ACM  8,  585--600 (1961)

\bibitem{Pol05}
Pol\'ak, L.: Minimalizations of {NFA} using the universal automaton. Internat.
  J. Found. Comput. Sci.  16(5),  999--1010 (2005)

\bibitem{RaSc59}
Rabin, M., Scott, D.: Finite automata and their decision problems. IBM J. Res.\
  and Dev.  3,  114--129 (1959)

\bibitem{Sen92}
Sengoku, H.: Minimization of nondeterministic finite automata. Master's thesis,
  Kyoto University, Department of Information Science, Kyoto, Japan (1992)

\end{thebibliography}





\end{document}
