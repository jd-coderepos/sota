

\documentclass{mytlp}

\usepackage{graphics}
\usepackage[all]{xy}
\usepackage{times}
\usepackage{url}
\def\UrlFont{\tt}
\usepackage{amssymb}
\renewcommand{\tt}{\usefont{OT1}{cmtt}{m}{n}\selectfont}



\usepackage{wrapfig}

\def\boxit#1{\raisebox{-3.5pt}{\vbox{\hrule\hbox{\vrule\kern3pt
  \vbox{\kern3pt\hbox{#1}\kern3pt}\kern3pt\vrule}\hrule}}}



\newcommand{\inparen}[1]{\mbox{\rm (}#1\mbox{\rm )}}
\newcommand{\modulo}{\inparen{modulo a renaming of nodes}}

\newcommand{\equprogram}[1]{\def\separator{1ex}
\frenchspacing
\refstepcounter{equation}\par\vspace\separator\hspace{2.5em}\kern-35pt\llap{(\theequation)}
\par\vspace\separator\noindent\kern-.0em}
\newcommand{\code}[1]{\mbox{\tt #1}}   \newcommand{\bs}{\char92} \newcommand{\us}{\char95} 



\newcommand{\Roots}{\ensuremath{\mathcal{R}\mathrm{oots}}}
\newcommand{\nodes}{\ensuremath{\mathcal{N}}}
\newcommand{\exprs}{\ensuremath{\mathcal{E}}}
\newcommand{\finger}{\ensuremath{\mathcal{F}}}
\newcommand{\xlabel}{\ensuremath{\mathcal{L}}}
\newcommand{\xsucc}{\ensuremath{\mathcal{S}}}
\newcommand{\Path}{\ensuremath{\mathcal{P}}}

\newcommand{\trs}{\ensuremath{\mathcal{R}}}
\newcommand{\vars}{\ensuremath{\mathcal{X}}}
\newcommand{\constrs}{\ensuremath{\mathcal{C}}}
\newcommand{\opers}{\ensuremath{\mathcal{D}}}
\newcommand{\signat}{\ensuremath{\Sigma}}
\newcommand{\tree}{\ensuremath{\mathcal{T}}}
\newcommand{\strat}{{\mathcal{S}}}


\newcommand{\COMMENT}[1]{}
\usepackage{color}
\newcommand{\todo}[1]{{\color{red}\sc [{#1}]}}

\usepackage{theorem}

\newtheorem{principle}{Principle}
\newtheorem{definition}{Definition}
\newtheorem{lemma}{Lemma}
\newtheorem{example}{Example}
\newtheorem{theorem}{Theorem}
\newtheorem{corollary}{Corollary}
\newcommand{\myproof}[1]{
  \ifx\showproof\undefined
\else
\mbox{\it Proof\ \ } #1 ~~ \2ex]
\emph{Copying} (or \emph{cloning}) is an approach that
fixes the inherent incompleteness
of backtracking.  Evaluating a choice in some context, say
, consists in evaluating simultaneously (e.g., by
interleaving steps) and independently both  and .  In
typical interpreters, if and when the computation of either
completes, the result is consumed, e.g., printed, and the user is
given the option to either terminate the execution or continue with the
computation of the other.  Copying is
simpler than backtracking and it is used in some experimental
implementations of functional logic languages
\cite{AntoyHanusLiuTolmach04IFL,TolmachAntoyNita04icfp}.
The major objection to copying is the significant investment
of time and memory made when a non-deterministic step is executed.
If an alternative of a choice eventually fails, cloning the context
may have been largely useless.  For a contrived example, notice
that in \code{1+(2+(+( `div` coin)))}
an arbitrarily large context is cloned when the choice is
evaluated, but for one alternative this context
is almost immediately discarded.
\-1ex]
\item Otherwise, if  \inparen{an ordinary rewrite}
  and  is labeled by the choice symbol,
  then , for an arbitrary 
  provided that 
  for all  and all  and 
  for  \inparen{i.e.,  is fresh}. \1ex]
\emph{Soundness:} 
if \cal S is a
computation of  in which each step is according to 
and  is a value (constructor normal form), then
.  \1ex]
In the definitions of soundness and completeness proposed
above, the same expression is evaluated both
according to  and by rewriting.
This is adequate with some conventions.
Rewriting is not concerned with choice identifiers.
This decoration can be simply ignored in rewriting computations.
In particular, in rewriting (as opposed to rewriting and pull-tabbing)
a computation is always consistent.
In graph rewriting, \emph{equality of graphs} is modulo a renaming of nodes.
A precise definition of this concept
is in \cite[Sect.~2.5]{EchahedJanodet97IMAG}.

Typically, the proof of soundness is trivial for strategies that
execute only rewrite steps, but our strategy executes also
pull-tab steps, hence it creates expressions that cannot be
produced by rewriting.  Indeed, some of these expressions
will have to be discarded to ensure the soundness.
The proof of correctness of pull-tabbing is
non-trivial and relies on two additional concepts,
\emph{representation} and \emph{invariance},
which are presented in following sections.

\subsection{Parallel Moves}

\begin{wrapfigure}[12]{r}{1.8in}
  \vspace*{-4ex}
  \begin{center}
    \boxit{
      \xymatrix@!C=3ex@!R=3ex{
        & e \ar@{->}[dl]^(.75){n_1,r_1}  \ar@{->}[dr]_(.75){n_2,r_2} \\
        e_1 \ar@{->}[dr]^(.65){=}_(.75){n_2,r_2} 
        & & e_2 \ar@{->}[dl]_(.65){=}^(.75){n_1,r_1}  \\
        & e'
      }
    }
  \end{center}
    \begin{minipage}{1.7in}
  \caption{\it The Parallel Moves Lemma
    for LOIS graph rewriting systems under appropriate conditions
    on nodes and rules.}
  \end{minipage}
  \label{fig:parallel-moves}
\end{wrapfigure}

Proofs of properties of a computation are often accomplished
by ``rearranging'' the computation's steps in some convenient order.
A fundamental result in rewriting, known as
the Parallel Moves Lemma \cite{HuetLevy91},
shows that in orthogonal systems the steps of a computation
can be rearranged at will.
A slightly weaker form of this result carries over
to \emph{LOIS} systems.
A pictorial representation of this result
is provided in Fig.~4. The symbol ``'' denotes the reflexive closure
of the rewrite relation.
The notation ``'', where  is a node and 
is a rule, denotes either equality or a rewrite step at
node  with rule .

\def\lemmapmoves{
\begin{lemma}[\emph{LOIS} parallel moves]
\label{parallel-moves}
Let ,  and  be expressions such that
,
where for ,  is a node and  is a rule.
If  or both  and ,
then there exists an expression  such that 
\modulo{}
.
\end{lemma}
\myproof{
By cases on the assumption's condition.
When both  and ,
the two steps are the same, hence .
When : the claim is a restriction
of \cite[Lemma 20]{Antoy97ALP} to rewriting in \emph{LOIS} systems.
}
}

\lemmapmoves


\subsection{Representation}

A characteristic of
pull-tabbing, similar to bubbling and copying,
is that the completeness of computations is obtained
by avoiding or delaying a commitment to
either alternative of a choice.
In pull-tabbing, similar to bubbling, \emph{both}
the alternatives of a choice are
kept or ``represented'' in a \emph{single} expression
throughout a good part of a computation.
The proof of the correctness of pull-tabbing
is obtained by reasoning about this concept,
which we formalize below.

\begin{definition}[Representation]
\label{representation}
We define a mapping  that takes an expression 
and returns a set  called the
\emph{represented set of } as follows.
Let  be an expression.
An expression  is in  iff
there exists a consistent computation 
\modulo{}
that makes all and only the choice steps of .
\end{definition}
In other words, we select either alternative for every
choice of an expression.  For choices with the same identifier,
we select the same alternative.
Since distinct choice steps occur at distinct nodes, by Lemma
\ref{parallel-moves} the order in which the choice steps are
executed to produce any member of the represented set 
is irrelevant.
Therefore, the notion of represented set is \emph{well defined}.
The notion of represented set of  is a simple syntactic
abstraction not to be confused with the notion of set of values of
an expression ~\cite{AntoyHanus09PPDP}, which is a semantic
abstraction fairly more complicated.

\COMMENT{
\todo{NOTES:
If a choice has been pulled up, its two alternatives have
a non-empty fingerprint.
\\
If a choice is pulled above a node that has a non-empty
fingerprint, incompatibility can be detected locally.
}
}

\subsection{Invariance}

The proof of correctness of pull-tabbing
is based on two results that informally speaking 
establish that the notion of represented set
is invariant both by pull-tab steps and by non-choice steps.

\COMMENT{
\todo{Define  and  are \emph{para-equal} iff
\emph{(1)}~ for any expression ,
there exists an expression 
such that  \modulo{}, and
\emph{(2)}~ for any expression , 
there exists an expression  
such that  \modulo{}.
}
}

\def\lemmainvarpulltab{
\begin{lemma}[Invariance by pull-tab]
\label{invariance-by-pull-tab}
If  is a pull-tab step, then
\emph{(1)}~ for any expression ,
there exists an expression 
such that  \modulo{}, and
\emph{(2)}~ for any expression , 
there exists an expression  
such that  \modulo{}.
\end{lemma}
\myproof{
We set up the notation for both claims.
If  is a pull-tab step,
then by Def.~\ref{def:pull-tab}:
.75ex]
    g'&=&C[n' \col ?(n_{f_1} \col f (s_1,\ldots n_1 \col x,\ldots s_k),
                     n_{f_2} \col f(s_1,\ldots n_2 \col y,\ldots s_k))]
  \end{array}
1ex]
Claim (1):
let  be a computation
witnessing that , i.e., a consistent
computation making all and only the choice steps of .
From this computation we construct a computation  of 
that produces an expression  satisfying the claim.
Without loss of generality, since the notion of
represented set is well defined, we assume that the first 
step of  is , for some , 
where  is either  or  of (\ref{binary-choice-rules}).
Let  and  be expressions defined by the following
computation: .
Node  may or may not be in .
In particular,  is in  iff  has more
than one predecessor in .  
If , then ;
otherwise, the step  does not
replace any subexpression of  and .
We explicitly construct a homomorphism 
that shows that  modulo a renaming of nodes.
We show that (a) for each node ,
with , , and vice versa,
(b) for each node , with ,
.
To prove (a), let  be a node of .
Since  is a choice step,
 is either in the context  of 
or in the subexpression rooted by .
These portions of  are preserved by the steps
that produce ,  and .
Thus, .
The proof of (b) is analogous.
Therefore, we define 
and , if .
By (a) and (b) and by construction,  is a bijection.
By construction,  preserves root,
label, and successors of every node.
Thus the computation  starts with ,
followed by  if .
Then, for any step of  starting with expression  at node
 with rule  there is a step of  starting with expression 
at node  with rule .
These computations start at equal expressions
(modulo a renaming of nodes)
and make the same steps, hence
they end at equal expressions (modulo a renaming of nodes).
Since  is consistent, so is .
Let  be the last expression of .
This proves that .
\
    \xymatrix@C=12pt@R=16pt{
      \llap{} g_0 \ar[r] \ar[d] & g_1 \ar[r] \ar[d]^(.65)= 
          & \ldots & g_n \rlap{}  \ar[d]^(.65)= \\
      \llap{} g'_0 \ar[r]^(.65)= & g'_1 \ar[r]^(.65)= 
          & \ldots & g'_n \rlap{}
    }
1ex]
Claim (2):
let  be a computation
witnessing that , i.e., a consistent
computation making all and only the choice steps of .
From this computation we construct a computation of 
that produces an expression  satisfying the claim.
Let , i.e., be the set of nodes
both in  and .
Suppose that the cardinality of  is , for some .
We reorder the steps of , which is possible by
Lemma~\ref{parallel-moves}, so that any step at some node of 
occurs before any step at some node not in .
Let  
be one such computation.
From , we construct a computation  that produces
an expression  satisfying the claim.
Consider the following diagram, where the top row is 
and the bottom row is :

By induction on , for , we both
define  and prove that the diagram commutes.
To support the induction, we strengthen the statement to
include the definition of the step  and
the condition that this step is not a choice step.
Base case, : 
Let  be the root of the redex of .
By assumption, . Hence .
We let .
Thus we have the steps 
where by assumption the first is not a choice step and
by construction the second is a choice step.
Since these steps are at distinct nodes, by
Lemma \ref{parallel-moves} there exists some  such that
.
Therefore,  and the diagram commutes.
The step  is not a choice step because either 
 or it is at the same node as .
Ind.~case, :
Let  be the root of the redex of 
.
By assumption, , hence in .
Thus, node  in  is not created
by the step .
Consequently,  is a node of  too.
We let .
Thus we have the steps 
where by assumption the first is not a choice step and
by construction the second is a choice step.
Since these steps are at distinct nodes, by
Lemma \ref{parallel-moves} there exists some  such that
.
Therefore,  and the diagram commutes.
The step  is not a choice step because either
 or it is at the same node as .
Since  is consistent,  up to  is consistent as well,
since corresponding steps use the same rule.
We reduce any remaining choice of  consistently with
the preceding steps of , say ,
to produce an expression .
We show that .
If , then there exists
a choice step  in .
By construction, this step is at some node  which is not in 
and hence is not in .
This means that the step  erases node .
Thus, we have the steps ,
for some rule  , where by construction the first is not a choice step
and by assumption the second is a choice step. 
Since these steps are at distinct nodes, by
Lemma \ref{parallel-moves} there exists some  such that
.
Since , 
and .
The same above reasoning proves that for any expression  of
 in , .
In particular, .
Thus,  witnesses the claim.
}
}

\lemmainvarnonchoice
We combine the previous lemmas
into computations of any length.

\def\corollsequence{
\begin{corollary}
\label{pre-correctness}
If  with no choice steps, then 
\emph{(1)}~ for any expression ,
there exists an expression 
such that  \modulo{}, and
\emph{(2)}~ for any expression , 
there exists an expression  
such that  \modulo{}.
\end{corollary}
\myproof{
Both claims are proved by a trivial
induction on the number of steps of 
using Lemmas~\ref{invariance-by-pull-tab} and
\ref{invariance-by-non-choice}.
}
}

\corollsequence

\def\theoremcorrect{
\begin{theorem}[Correctness]
\label{correctness}
If  with no choice steps, then 
\emph{(1)}~ for any value  such that 
is a consistent computation,
there exists a value  such that 
is a consistent computation,
and  \modulo{}, and
\emph{(2)}~ for any value   such that 
is a consistent computation,
there exists a value  such that 
is a consistent computation,
and  \modulo{}.
\end{theorem}
\myproof{
Claim (1):
let  a consistent computation of 
into a value .
By Lemma~\ref{parallel-moves}, without loss of generality
we assume that ,
where the segment  consists of all
the choice steps of .
Since  is consistent, .
By Corollary~\ref{pre-correctness}, there exists
a consistent computation  such that 
 (modulo a renaming of nodes).
Since  (modulo a renaming of nodes) and ,
there exists a computation 
such that  (modulo a renaming of nodes).
\1ex]
Clam (2): the proof is analogous to that of claim (1).
}
}

\theoremcorrect
Theorem \ref{correctness} suggests to apply both non-choice and
pull-tab steps to an expression. Choices pulled up to the root are
reduced consistently and without context cloning. Of course, by
the time a choice is reduced, all its spines have been
cloned---similar to bubbling and copying.
A better option, available to pull-tabbing only,
is discussed in the next section.

\section{Application}
\label{Application}

The pull-tab transformation is meant to be used in conjunction
with some evaluation strategy.  
We showed that pull-tabbing is not tied to any particular strategy.
However, the strategy should be pull-tab-aware in that:
(1) a choice should be evaluated (to a head normal form)
only when it is \emph{needed} \cite{Antoy97ALP},
(2) a choice in a root position is reduced (consistently),
whereas in a non-root position is pulled, and
(3) before pulling a choice,
one of the choice's alternatives should be a head-normal form.
The formalization of such a strategy
would take us well beyond the scope of this paper.

In well-designed, non-deterministic programs,
either or both alternatives of most (but not all) choices
should fail \cite{Antoy10JSC}.
Under the assumption that a choice is evaluated to a head normal form
only when it is \emph{needed} \cite{Antoy97ALP},
if an alternative of the choice fails,
the choice is no longer non-deterministic---the
failing alternative cannot produce a value.
Thus, the choice can be reduced to the other alternative
without loss of completeness and without context cloning.
This is where pull-tabbing is advantageous over
copying and bubbling---any portion of a choice's context not yet cloned
when an alternative fails no longer needs to be cloned.
Of course, the implementation must still identify the choice,
and choice's single remaining strand as either left or right, 
to ensure consistency.

\section{Related Work}

We investigated pull-tabbing, an approach to
non-deterministic computations in functional logic programming.
Section \ref{Approaches} recalls copying and bubbling,
the competitors of pull-tabbing.
Here, we briefly highlight the key differences between these approaches.
Pull-tabbing ensures the completeness of computations
in the sense that no alternative of a choice is left behind
until all the results of some other alternative have been produced.
Similar to every approach with this property, it must clone
portions of the context of a choice.
In contrast to copying and bubbling, it clones
the context of a choice in minimal increments
with the intent and the possibility of stopping cloning
the context as soon as an alternative of the choice fails.

The idea of identifying choices to avoid combining in some
expression the left and right alternatives of the same choice
appears in \cite{BrasselHuchAPLAS07}. The idea is developed
in the framework of a natural semantics for the translation
of (flat) Curry programs into Haskell.
A proof of the correctness of this idea will appear in
\cite{Brassel2011PhD} which also addresses the similarities
between the natural semantics and graph rewriting.  
This discussion, although informal, is enlightening.

\section{Conclusion}

We formally defined the pull-tab transformation,
characterized the class of programs for which the transformation
is intended, extended the computations in these programs
to include the transformation, proved the correctness
of these extended computations, and described the condition
that reduces context cloning.
In contrast to its competitors, in pull-tabbing any step is a
simple and localized graph transformation.
This fact should ease executing the steps in parallel.
Future work, aims at defining a pull-tab-aware parallel strategy
and implementing it to measure the effectiveness of pull-tabbing.


\newpage
\bibliographystyle{acmtrans}


\begin{thebibliography}{}

\bibitem[\protect\citeauthoryear{Alqaddoumi, Antoy, Fischer, and
  Reck}{Alqaddoumi et~al\mbox{.}}{2010}]{Alqaddoumi10GCM}
{\sc Alqaddoumi, A.}, {\sc Antoy, S.}, {\sc Fischer, S.}, {\sc and} {\sc Reck,
  F.} 2010.
\newblock The pull-tab transformation.
\newblock In {\em Proceedinsg of the Third International Workshop on Graph
  Computation Models}. Enschede, The Netherlands, 127--133.
\newblock Available at
  \url{http://gcm-events.org/gcm2010/pages/gcm2010-preproceedings.pdf}.

\bibitem[\protect\citeauthoryear{Antoy}{Antoy}{1992}]{Antoy92ALP}
{\sc Antoy, S.} 1992.
\newblock Definitional trees.
\newblock In {\em Proceedings of the Third International Conference on
  Algebraic and Logic Programming}, {H.~Kirchner} {and} {G.~Levi}, Eds.
  Springer LNCS 632, Volterra, Italy, 143--157.

\bibitem[\protect\citeauthoryear{Antoy}{Antoy}{1997}]{Antoy97ALP}
{\sc Antoy, S.} 1997.
\newblock Optimal non-deterministic functional logic computations.
\newblock In {\em Proceedings of the Sixth International Conference on
  Algebraic and Logic Programming (ALP'97)}. Springer LNCS 1298, Southampton,
  UK, 16--30.

\bibitem[\protect\citeauthoryear{Antoy}{Antoy}{2001}]{Antoy01PPDP}
{\sc Antoy, S.} 2001.
\newblock Constructor-based conditional narrowing.
\newblock In {\em Proc.\ of the 3rd International Conference on Principles and
  Practice of Declarative Programming (PPDP'01)}. ACM, Florence, Italy,
  199--206.

\bibitem[\protect\citeauthoryear{Antoy}{Antoy}{2005}]{Antoy05JSC}
{\sc Antoy, S.} 2005.
\newblock Evaluation strategies for functional logic programming.
\newblock {\em Journal of Symbolic Computation\/}~{\em 40,\/}~1, 875--903.

\bibitem[\protect\citeauthoryear{Antoy}{Antoy}{2010}]{Antoy10JSC}
{\sc Antoy, S.} 2010.
\newblock Programming with narrowing.
\newblock {\em Journal of Symbolic Computation\/}~{\em 45,\/}~5 (May),
  501--522.

\bibitem[\protect\citeauthoryear{Antoy, Brown, and Chiang}{Antoy
  et~al\mbox{.}}{2006}]{AntoyBrownChiang06Termgraph}
{\sc Antoy, S.}, {\sc Brown, D.}, {\sc and} {\sc Chiang, S.} 2006.
\newblock Lazy context cloning for non-deterministic graph rewriting.
\newblock In {\em Proc.~of the 3rd International Workshop on Term Graph
  Rewriting, Termgraph'06}. Vienna, Austria, 61--70.

\bibitem[\protect\citeauthoryear{Antoy and Hanus}{Antoy and
  Hanus}{2002}]{AntoyHanus02FLOPS}
{\sc Antoy, S.} {\sc and} {\sc Hanus, M.} 2002.
\newblock Functional logic design patterns.
\newblock In {\em Proceedings of the Sixth International Symposium on
  Functional and Logic Programming (FLOPS'02)}. Springer LNCS 2441, Aizu,
  Japan, 67--87.

\bibitem[\protect\citeauthoryear{Antoy and Hanus}{Antoy and
  Hanus}{2005}]{AntoyHanus05LOPSTR}
{\sc Antoy, S.} {\sc and} {\sc Hanus, M.} 2005.
\newblock Declarative programming with function patterns.
\newblock In {\em 15th Int'nl Symp. on Logic-based Program Synthesis and
  Transformation (LOPSTR 2005)}. Springer LNCS 3901, London, UK, 6--22.

\bibitem[\protect\citeauthoryear{Antoy and Hanus}{Antoy and
  Hanus}{2006}]{AntoyHanus06ICLP}
{\sc Antoy, S.} {\sc and} {\sc Hanus, M.} 2006.
\newblock Overlapping rules and logic variables in functional logic programs.
\newblock In {\em Twenty Second International Conference on Logic Programming}.
  Springer LNCS 4079, Seattle, WA, 87--101.

\bibitem[\protect\citeauthoryear{Antoy and Hanus}{Antoy and
  Hanus}{2009}]{AntoyHanus09PPDP}
{\sc Antoy, S.} {\sc and} {\sc Hanus, M.} 2009.
\newblock Set functions for functional logic programming.
\newblock In {\em Proceedings of the 11th ACM SIGPLAN International Conference
  on Principles and Practice of Declarative Programming (PPDP 2009)}. Lisbon,
  Portugal, 73--82.

\bibitem[\protect\citeauthoryear{Antoy and Hanus}{Antoy and
  Hanus}{2010}]{AntoyHanus10CACM}
{\sc Antoy, S.} {\sc and} {\sc Hanus, M.} 2010.
\newblock Functional logic programming.
\newblock {\em Comm. of the ACM\/}~{\em 53,\/}~4 (April), 74--85.

\bibitem[\protect\citeauthoryear{Antoy, Hanus, Liu, and Tolmach}{Antoy
  et~al\mbox{.}}{2005}]{AntoyHanusLiuTolmach04IFL}
{\sc Antoy, S.}, {\sc Hanus, M.}, {\sc Liu, J.}, {\sc and} {\sc Tolmach, A.}
  2005.
\newblock A virtual machine for functional logic computations.
\newblock In {\em Proc.\ of the 16th International Workshop on Implementation
  and Application of Functional Languages (IFL 2004)}. Springer LNCS 3474,
  Lubeck, Germany, 108--125.

\bibitem[\protect\citeauthoryear{Baader and Nipkow}{Baader and
  Nipkow}{1998}]{BaaderNipkow98}
{\sc Baader, F.} {\sc and} {\sc Nipkow, T.} 1998.
\newblock {\em Term Rewriting and All That}.
\newblock Cambridge University Press.

\bibitem[\protect\citeauthoryear{Bezem, Klop, and de~Vrijer~(eds.)}{Bezem
  et~al\mbox{.}}{2003}]{Terese03}
{\sc Bezem, M.}, {\sc Klop, J.~W.}, {\sc and} {\sc de~Vrijer~(eds.), R.} 2003.
\newblock {\em Term Rewriting Systems}.
\newblock Cambridge University Press.

\bibitem[\protect\citeauthoryear{Brassel}{Brassel}{2011}]{Brassel2011PhD}
{\sc Brassel, B.} 2011.
\newblock Implementing functional logic programs by translation into purely
  functional programs.
\newblock Ph.D. thesis, Brandeis University.
\newblock (To appear).

\bibitem[\protect\citeauthoryear{Brassel and Huch}{Brassel and
  Huch}{2007}]{BrasselHuchAPLAS07}
{\sc Brassel, B.} {\sc and} {\sc Huch, F.} 2007.
\newblock On a tighter integration of functional and logic programming.
\newblock In {\em APLAS'07: Proceedings of the 5th Asian conference on
  Programming languages and systems}. Springer-Verlag, Berlin, Heidelberg,
  122--138.

\bibitem[\protect\citeauthoryear{Caballero and S\'{a}nchez}{Caballero and
  S\'{a}nchez}{2007}]{ToyHomepage}
{\sc Caballero, R.} {\sc and} {\sc S\'{a}nchez, J.}, Eds. 2007.
\newblock {\em TOY: A Multiparadigm Declarative Language (version 2.3.1)}.
\newblock Available at \url{http://toy.sourceforge.net}.

\bibitem[\protect\citeauthoryear{Dershowitz and Jouannaud}{Dershowitz and
  Jouannaud}{1990}]{DershowitzJouannaud90handbook}
{\sc Dershowitz, N.} {\sc and} {\sc Jouannaud, J.} 1990.
\newblock Rewrite systems.
\newblock In {\em Handbook of Theoretical Computer Science B: Formal Methods
  and Semantics}, {J.~{van Leeuwen}}, Ed. North Holland, Amsterdam, Chapter~6,
  243--320.

\bibitem[\protect\citeauthoryear{Echahed and Janodet}{Echahed and
  Janodet}{1997}]{EchahedJanodet97IMAG}
{\sc Echahed, R.} {\sc and} {\sc Janodet, J.~C.} 1997.
\newblock On constructor-based graph rewriting systems.
\newblock Tech. Rep. 985-I, IMAG.
\newblock Available at
  \url{ftp://ftp.imag.fr/pub/labo-LEIBNIZ/OLD-archives/PMP/c-graph-rewriting.ps.gz}.

\bibitem[\protect\citeauthoryear{Echahed and Janodet}{Echahed and
  Janodet}{1998}]{EchahedJanodet98JICSLP}
{\sc Echahed, R.} {\sc and} {\sc Janodet, J.~C.} 1998.
\newblock Admissible graph rewriting and narrowing.
\newblock In {\em Proceedings of the Joint International Conference and
  Symposium on Logic Programming}. MIT Press, Manchester, 325 -- 340.

\bibitem[\protect\citeauthoryear{Hanus}{Hanus}{2006}]{Hanus06Curry}
{\sc Hanus, M.}, Ed. 2006.
\newblock {\em Curry: An Integrated Functional Logic Language (Vers.\ 0.8.2)}.
\newblock Available at \url{http://www.informatik.uni-kiel.de/~curry}.

\bibitem[\protect\citeauthoryear{Hanus}{Hanus}{2008}]{Hanus08PAKCS}
{\sc Hanus, M.}, Ed. 2008.
\newblock {\em {PAKCS} 1.9.1: {The Portland Aachen Kiel Curry System}}.
\newblock Available at \url{http://www.informatik.uni-kiel.de/~pakcs}.

\bibitem[\protect\citeauthoryear{Huet and L\'evy}{Huet and
  L\'evy}{1991}]{HuetLevy91}
{\sc Huet, G.} {\sc and} {\sc L\'evy, J.-J.} 1991.
\newblock Computations in orthogonal term rewriting systems.
\newblock In {\em Computational logic: essays in honour of Alan Robinson},
  {J.-L. Lassez} {and} {G.~Plotkin}, Eds. MIT Press, Cambridge, MA.

\bibitem[\protect\citeauthoryear{Hussmann}{Hussmann}{1992}]{Hussmann92JLP}
{\sc Hussmann, H.} 1992.
\newblock Nondeterministic algebraic specifications and nonconfluent rewriting.
\newblock {\em Journal of Logic Programming\/}~{\em 12}, 237--255.

\bibitem[\protect\citeauthoryear{{ISO}}{{ISO}}{1995}]{IsoProlog}
{\sc {ISO}}. 1995.
\newblock {Information technology - Programming languages - Prolog - Part 1}.
\newblock General Core. ISO/IEC 13211-1, 1995.

\bibitem[\protect\citeauthoryear{Klop}{Klop}{1992}]{Klop92Handbook}
{\sc Klop, J.~W.} 1992.
\newblock {Term Rewriting Systems}.
\newblock In {\em Handbook of Logic in Computer Science, Vol.\ II},
  {S.~Abramsky}, {D.~Gabbay}, {and} {T.~Maibaum}, Eds. Oxford University Press,
  1--112.

\bibitem[\protect\citeauthoryear{L\'{o}pez-Fraguas, Rodr\'{\i}guez-Hortal\'{a},
  and S\'{a}nchez-Hern\'{a}ndez}{L\'{o}pez-Fraguas
  et~al\mbox{.}}{2007}]{Lopez-FraguasRodriguez-HortalaSanchez-Hernandez07PPDP}
{\sc L\'{o}pez-Fraguas, F.~J.}, {\sc Rodr\'{\i}guez-Hortal\'{a}, J.}, {\sc and}
  {\sc S\'{a}nchez-Hern\'{a}ndez, J.} 2007.
\newblock A simple rewrite notion for call-time choice semantics.
\newblock In {\em PPDP '07: Proceedings of the 9th ACM SIGPLAN international
  conference on Principles and practice of declarative programming}. ACM, New
  York, NY, USA, 197--208.

\bibitem[\protect\citeauthoryear{L\'opez-Fraguas, Rodr\'iguez-Hortal\'a, and
  S\'anchez-Hern\'andez}{L\'opez-Fraguas
  et~al\mbox{.}}{2008}]{Lopez-Fraguas08FLOPS}
{\sc L\'opez-Fraguas, F.~J.}, {\sc Rodr\'iguez-Hortal\'a, J.}, {\sc and} {\sc
  S\'anchez-Hern\'andez, J.} 2008.
\newblock Rewriting and call-time choice: The {HO} case.
\newblock In {\em Proc. of the 9th International Symposium on Functional and
  Logic Programming (FLOPS 2008)}. Springer LNCS 4989, 147--162.

\bibitem[\protect\citeauthoryear{O'Donnell}{O'Donnell}{1985}]{Odonnell85Book}
{\sc O'Donnell, M.~J.} 1985.
\newblock {\em Equational Logic as a Programming Language}.
\newblock {MIT} Press.

\bibitem[\protect\citeauthoryear{Plump}{Plump}{1999}]{Plump99Handbook}
{\sc Plump, D.} 1999.
\newblock Term graph rewriting.
\newblock In {\em Handbook of Graph Grammars}, {H.-J.~K. H.~Ehrig, G.~Engels}
  {and} {G.~Rozenberg}, Eds. Vol.~2. World Scientific, 3--61.

\bibitem[\protect\citeauthoryear{Tolmach, Antoy, and Nita}{Tolmach
  et~al\mbox{.}}{2004}]{TolmachAntoyNita04icfp}
{\sc Tolmach, A.}, {\sc Antoy, S.}, {\sc and} {\sc Nita, M.} 2004.
\newblock Implementing functional logic languages using multiple threads and
  stores.
\newblock In {\em Proc.\ of the 2004 International Conference on Functional
  Programming (ICFP)}. ACM, Snowbird, Utah, USA, 90--102.

\end{thebibliography}

\end{document}
