\documentclass[12pt, draftclsnofoot, journal, letterpaper, onecolumn]{IEEEtran}
\usepackage[cmex10]{amsmath}
\usepackage{algorithmic}
\usepackage{array}
\usepackage[caption=false,font=footnotesize]{subfig}
\usepackage{fixltx2e}
\usepackage{stfloats}
\usepackage{mathrsfs}
\usepackage{amsmath}
\usepackage{amssymb}
\usepackage{bbm}
\usepackage{url}
\usepackage{xcolor}
\usepackage{cleveref}
\usepackage{graphicx}
\usepackage{epstopdf}
\usepackage{cite}
\newtheorem{lemma}{Lemma}[section]
\newtheorem{corollary}{Corollary}[section]
\newtheorem{theorem}{Theorem}[section]
\newtheorem{proposition}{Propositon}[section]

\begin{document}

\title{Power Efficiency for Device-to-Device Communications}

\author{\IEEEauthorblockN{Yanbo~Ma\IEEEauthorrefmark{1}, Yuan~Liu\IEEEauthorrefmark{1}, and~Meixia~Tao\IEEEauthorrefmark{1}}\\\IEEEauthorblockA{\IEEEauthorrefmark{1}Department of Electronic Engineering, Shanghai Jiao Tong University, Shanghai, P. R. China}\\
Email: myb800@sjtu.edu.cn, \{eeyuanliu, mxtao\}@ieee.org
}


\maketitle

\begin{abstract}
The concept of device-to-Device (D2D) communication as an underlay coexistence with cellular networks gains many advantages of improving system performance.
In this paper, we model such a two-layer heterogenous network based on stochastic geometry approach. We aim at minimizing the expected power consumption of the D2D layer while satisfying the outage performance of both D2D layer and cellular layer. We consider two kinds of power control schemes. The first one is referred as to \emph{independent power control} where the transmit powers are statistically independent of the networks and all channel conditions. The second is named as \emph{dependent power control} where the transmit power of each user is dependent on its own channel condition. A closed-form expression of optimal independent power control is derived, and we point out that the optimal power control for this case is fixed and not relevant to the randomness of the network. For the dependent power control case, we propose an efficient way to find the close-to-optimal solution for the power-efficiency optimization problem.
Numerical results show that dependent power control scheme saves about half of power that the independent power control scheme demands.
\end{abstract}

\begin{IEEEkeywords}
Device-to-Device (D2D), stochastic geometry, power efficiency.
\end{IEEEkeywords}



\IEEEpeerreviewmaketitle



\section{Introduction}

The exponentially increasing data traffic and requirements of user experiences call for dramatic expansion of energy consumption. To meet the high demands of resource saving and environment protection for wireless networks today, it is quite necessary and urgent to save energy consumption from network nodes, such as base stations, access points, and mobile devices.

Device-to-Device (D2D) communication is a promising technology and allowed as an underlay coexistence with cellular networks. It enables a pair of devices in proximity of each other to establish a direct local link and is not through a base station or access point. Such a heterogenous network infrastructure has attracted much attention due to its potential of improving system performance, such as throughput enhancing, coverage extension, and data offloading \cite{D2DLTEA,DAD2D,PCD2D}.


Stochastic geometry \cite{baccelli2009stochastic,baccelli2009stochastic2,interference} is a powerful tool that provides tractable analysis for large wireless networks \cite{overlaid,SGRG,RPC2,FPC,OP}. For instance, \cite{overlaid} investigates feasibility region about node density. The work in \cite{SGRG} summarizes fundamental limits of wireless cellular network based on stochastic geometry, like connectivity, capacity, outage, etc.
The authors in \cite{RPC2} consider power control in random networks with Poisson distributed nodes using a game theoretic approach. Moreover, \cite{RPC2} adopts ALOHA-type random on-off power control policies to maximize expected local performance of each link. The authors in \cite{OP} present the channel inversion based power control for an ad hoc network. The study of \cite{FPC} propose the fractional power control for a single homogeneous network, where the power control is the fractional exponent of the channel.

In this paper, we consider a large scale network where D2D communication reuse the uplink spectrum of the cellular communication. Using stochastic geometry, we model the network nodes as Poisson Point Processes (PPP). Our goal is to study the feasibility of such a two-layer network with power optimization first. Based on the feasibility conditions, we then minimize the expected power consumption of the D2D layer while maintaining the outage performance of both cellular layer and D2D layer. The motivation of minimizing the power consumption of the D2D layer is to ensuring quality-of-service or protecting the cellular layer from harmful interference caused by the D2D layer. Two power control scheme, upon transmit power is \emph{independent} or \emph{dependent} of channel conditions, are taken into account. We derive the feasibility conditions for the power-efficiency optimization problem for both independent and dependent power control cases. For the independent power control, we prove that fixed power control is optimal and we derive its closed-form. For the dependent power control case, we first show that the expected power consumption can be minimized if the power allocations are the deterministic functions of channel conditions. Then we propose an efficient method to find the close-to-optimal solution based on some approximations.

The rest of the paper are structured as follows. Section II describes the system model and problem formulation. Sections III and IV present our main results of independent and dependent power control schemes, respectively. Comprehensive numerical results are provided in Section V. Finally, Section VI concludes this paper.

\section{System Model}
\subsection{Network Model}

We consider the cellular network where the D2D users and cellular users coexist by the spectrum-sharing manner. Here we assume that the transmission mode, i.e., direct mode via D2D link and cellular mode via BS, is predetermined for each user. The locations of the D2D transmitters and cellular users are modeled as independent stationary Poisson Point Processes (PPP). Denote  () and  () as the locations and density of cellular (D2D) users, respectively. For simplicity, the distance between transmitter and receiver of each D2D (cellular) user is assumed to be fixed and denoted as  ().
In this paper, we consider that the D2D users reuse the uplink spectrum of the cellular users. The network is assumed to be interference limited, so that the background noise is neglected \cite{overlaid,SGLD}.

\subsection{Channel Model}

Denote   as the distance between users  and , .
The signal-to-interference ratio (SIR) at the receiver of D2D user  is given by

where  is the path-loss exponent,  is the transmit power of user  and  denote the i.i.d Rayleigh fading coefficients with , where  is the expectation operator.
If  is a cellular user, we just swap the subscripts  by .

The transmit powers are assumed to be independent random variables for each user. It is also assumed that the users in the same network have identical transmit power distribution, but the distribution may vary from one network to another.
We consider two scenarios. The first one is that the transmit power of each user is independent of channel conditions. We refer to this scenario as \emph{independent power control}. The second one is that the transmit power of each user is dependent on the channel condition between its transmitter and its receiver. This is referred to as \emph{dependent power control}.

According to properties of Palm distribution in \cite{SG}, the SIR distributions of all users in the same network are identical if the networks are stationary PPP and independent. Therefore, without loss of generality, in the following, our attention will focus on two \emph{typical} users, one for cellular user and the other for D2D user. The concept of typical is commonly used in the literature \cite{TC,TC2,OP}.

We define the feasibility region  as the set of density pair  such that make the outage probability is below specified thresholds  and  for the typical cellular and D2D users, respectively. Mathematically,

where  and  represent the typical cellular and D2D users, respectively;  and  is the SIR requirements for the typical cellular and D2D users, respectively.
Then the feasibility region can be expressed as:


\subsection{Problem Formulation}


Based on the Campbell's formula \cite{SG}, we know that if  is any region with area 1, . That is to say  is the average power consumption of D2D users per unit area. Hence, the goal of this paper
is to minimize the averaged power consumption of the typical D2D user while maintaining the
outage probabilities of the typical D2D and cellular users are below specified thresholds. The
problem can be formulated as



\section{Independent Case}

In this section, we analyze the independent case where the transmit power of the typical D2D
user is independent of channel conditions.

\subsection{Feasibility Region}

Due to the outage probability constraints, the problem \textbf{P1} is not always feasible. To make sure that \textbf{P1} will not have an empty set of solution, we should firstly analyze the feasibility region that depends on the system parameters.

\begin{lemma}\label{OPLA}
For the independent power control,

where  and , .
\end{lemma}

\begin{IEEEproof}
  Please see Appendix \ref{COPIPC}.
\end{IEEEproof}

Note that \eqref{pr6} implies that the interference is infinity if . This means that the independent power control cannot be used in the networks where the path-loss exponent . Thus we only consider the case  in this section.


With the distribution of SIRs derived in \eqref{pr6}, we provide the feasibility region in the following theorem.




\begin{theorem}\label{PFR1t}
Assume that , for given , the feasibility region with respect to the independent power control is given by

where ,  is given by lemma \ref{qd}.
\end{theorem}

\begin{IEEEproof}
  Please see Appendix \ref{PFR1proof}.
\end{IEEEproof}
The assumption of  with  is reasonable since the target outage probabilities are not designed too large in practice.

The following corollary yields the way to achieve the feasibility region.
\begin{corollary}\label{PFR1tC}
Let . For any pair  that is in the feasibility region given by Theorem \ref{PFR1t}, the constraint \eqref{constrainc} holds if

\end{corollary}
The proof of Corollary \ref{PFR1tC} can be obtained from the proof of Theorem \ref{PFR1t}.




\subsection{Minimizing Averaged Power Consumption}

In this subsection, we study the optimization problem \textbf{P1}. It is assumed that the density pair  is given and feasible. , , and . Thus, according to Lemma \ref{qd}, there exists a  and , such that the problem becomes\footnote{Without loss of generality, it is also assumed that .}




\begin{theorem}\label{lowestpower1}
The optimal solution for the problem \textbf{P2} is

\end{theorem}
where ``a.s." means with probability 1.

\begin{IEEEproof}
  Please see Appendix \ref{lowestpower1proof}.
\end{IEEEproof}

Note that the optimal power allocation in \eqref{pdo} is a constant, which means that the optimal power allocation is fixed and without any randomness of the networks.

It is remarkable that we assume  in above derivations. If without the assumption, our fixed power control will be relaxed to random power control as in \cite{RPC} and the feasibility region may be enlarged. However,  is not practical for outage threshold design.



\section{Dependent Case}


In this section, we analyze the dependent case where the transmit power of the typical D2D
user is dependent on the channel condition between its transmitter and receiver.

\subsection{Feasibility Region}
Like the previous section, we first study the feasibility region of the dependent power control case. However, the dependent case is more complex than the independent case. To make our analysis tractable, we adopt a lower bound of the expression of outage probability.
\begin{lemma}\label{OPLB}
For the dependent power control, a lower bound of outage probability is

where ,  and  is the channel gain between the transmitter and receiver, .
\end{lemma}

\begin{IEEEproof}
  Please see Appendix \ref{COPDPC}.
\end{IEEEproof}

The tightness of this lower bound can be proved using the similar method in \cite{TC2} and the details are omitted here.
It is observed that if  is small,  is also small. According to Jensen's inequality and the fact that function  is close to be a linear function, we further introduce an approximation for \eqref{lb} to ease the analysis.  
where . Notice that the approximation is also similarly used in \cite{FPC} where a specific power control scheme in a single network.

With the approximation \eqref{approx}, the outage probability constraint becomes

and the feasibility region can be characterized by the following theorem.
\begin{theorem}\label{PFR2t}
The feasibility region with the dependent power control is given by

\end{theorem}

\begin{IEEEproof}
  Please see Appendix \ref{PFR2tproof}.
\end{IEEEproof}


\begin{corollary}\label{PFR2tC}
For any density  that is in the feasibility region given by Theorem \ref{PFR2t}, \eqref{approxconstrain} holds if

\end{corollary}

Corollary \ref{PFR2tC} can be regarded as fractional power control with the exponent . The properties of fractional power control has been discussed in \cite{FPC},  where the authors  prove that the fractional power control with exponent  can minimize the approximation \eqref{approx} among all the fractional power control policies. Our result in Corollary \ref{PFR2tC} demonstrates another optimality: the fractional power control with exponent  also leads to the largest feasibility region among all the dependent power control policies.


\subsection{Minimizing Averaged Power Consumption}
In this subsection, we consider the optimization problem \textbf{P1} under dependent power control. Like \cite{FPC}, we drop the peak power constrain. We also assume that  is given and feasible. The problem can be rewritten as


\begin{lemma}\label{function}
If the optimal power allocation  satisfies \eqref{constrainc5}-\eqref{constraine5}, then there exists  that is a deterministic function, such that \eqref{constrainc5}-\eqref{constraine5} hold. Meanwhile, .
\end{lemma}

\begin{IEEEproof}
  Please see Appendix \ref{functionproof}.
\end{IEEEproof}

The above Lemma reveals that we should only focus on the deterministic functions, because the additional randomness will not improve performance.

Due to extremely sophisticated structure of the function spaces, finding the extremal point for functionals is very nontrivial. We then propose an efficient method to solve the power control problem in a suboptimal way by conducting three ordinary relaxations. Firstly, instead of considering the function  in , we consider it in  where  is large. This is possible since channel gain can not be infinity in practice. For the points outside this interval, the function is defined to be . This relaxation is plausible because when the  is large, we can cover almost all the status of channel gain. The second relaxation is to consider the function to be piecewise constant with each interval of constant very small. Based on these two relaxation, the function is converted to be:

where . In this case the choices of  decide the accuracy of the relaxation.
Hence, \textbf{P3} becomes:

where , . This is a geometry programming problem and can be solved efficiently since geometry programming problems can be transformed into convex problems.
\section{Numerical Results}
In this section, the implications of independent and dependent power control are illustrated through plots and figures. As default parameters, the simulation assumes:

\subsection{Feasibility Region under Different Schemes}

If  is given, Theorems \ref{PFR1t} and \ref{PFR2t} describe the feasibility region under independent and dependent control respectively.  Fig. \ref{PFR1} reveals the feasibility versus peak power constraint for the independent control. It can be observed that a) the peak power constraint of  can significantly reduce the supported ; b) when the peak power constraint exceeds a certain level, it will not have any effect on the feasibility region.

On the other hand, Fig. \ref{PFR2} illustrates the case of dependent power control when different 's are given. Recall Theorem \ref{PFR2} that if then the feasibility region is calculated for a specific , any scalar of  will not change this region. Thus, we omit the coefficient of . We study the feasibility region when , i.e.,  is using fractional power control \cite{FPC}. We can observe from the figure: a) channel inverse and constant power control have the same feasibility region, b)  outperforms others that was considered.



\begin{figure}
\center
\includegraphics[scale=0.65]{PFR1.eps}
\center\caption{Feasibility region when  is given.}\label{PFR1}
\end{figure}
\begin{figure}
\center
\includegraphics[scale=0.65]{PFR2.eps}
\center\caption{Feasibility region when  is given.}\label{PFR2}
\end{figure}

\subsection{Numerical Results of Minimal Averaged Power}
In this subsection, we consider two problems.

\subsubsection{Convergence Problem}\label{convergence}
First, we consider the minimal averaged power consumption versus . Note that  changes really fast when  is close to zero. Thus, we design  as follows:

we will show that if  is sufficient large, the minimal averaged power consumption will finally converge. The result is given in Fig. \ref{convergencefig}.  is set and we simulate for different 's. From the figure, it is observed that when  is greater than , the minimal average power consumption will converge. This shows the effectiveness of our proposed method in Section IV-B.

\begin{figure}
\center
\includegraphics[scale=0.65]{converge.eps}
\center\caption{Convergence with respect to .}\label{convergencefig}
\end{figure}

\begin{figure}
\center
\includegraphics[scale=0.65]{schemes.eps}
\center\caption{Independent power control and dependent power control for .}\label{pcschemes}
\end{figure}

\begin{figure}
\center
\includegraphics[scale=0.65]{shemes_smallh.eps}
\center\caption{Independent power control and dependent power control for , when channel gain is small.}\label{pcschemessmallh}
\end{figure}

Fig. \ref{pcschemes} shows the performance of power control schemes when  and . It can be observed that for most of the channel status, the power under independent power control is consumed more than the dependent case. However, when the channel gain is very close to zero, the power under dependent power control increases very fast. Fig. \ref{pcschemessmallh} demonstrates this increasing more accurately.

\subsubsection{Comparison between Independent and Dependent Power Control}

In this part, we compare the performance of the independent and dependent power control policies. Intuitively, dependent power control should have better performance. We investigate the relationship between  and the minimal power consumption. This comparison is conducted under the condition that  is a constant. Here  is set to be . The result is shown in Fig. \ref{powercompare}. From the figure, it is clear that dependent power control saves about 50\% power than independent power control. However, there is no free lunch. In order to save this energy, the D2D users should know the channel status at the transmitter side.

\begin{figure}
\center
\includegraphics[scale=0.65]{minimal_power.eps}
\center\caption{Feasibility region when  is designed in priori.}\label{powercompare}
\end{figure}

\section{Conclusion}

In this paper, we considered the power minimization problem of D2D users to guarantee outage probabilities of both D2D and cellular users. For the random networks, two power control schemes, namely independent and dependent power control, were proposed based on stochastic geometry. For these two schemes, we first analyzed the feasibility of the power-efficiency problem. Then optimal and close-to-optimal solutions for the two schemes were proposed respectively. Numerical results showed that the dependent power control saves about 50\% power than the independent power control.








\appendices
\section{Calculation of Outage Probability under Independent Power Control}\label{COPIPC}
\begin{IEEEproof}
If  is a typical D2D user, then

where ,
(a) from the fact that power is independent of channel,
(b) from the fact that power, ,  are independent,
(c) from the definition of Laplace transform,
(d) from \cite{RPC2} Lemma 1. The other part of this lemma can be deduced similarly.
\end{IEEEproof}
\section{Calculation of Outage Probability under Dependent Power Control}\label{COPDPC}
\begin{IEEEproof}
We extend the proof in \cite{OP}, where only one network is considered.
If  is the typical D2D user,
then

Set

Therefore,

where (a) from  are independent, (b) follows from (103) in \cite{OP}, (c) follows from (104),(105) in \cite{OP}, (d) from the fact that power is a random variable which is independent of the channel that is not between the transmitter receiver pair and . If  is a typical D2D cellular, the proof is identical.
\end{IEEEproof}
\section{Proof of Theorem \ref{PFR1t}}\label{PFR1proof}
Before proving theorem \ref{PFR1t}, we need some lemmas.
\begin{lemma}\label{continuity}
The function,

is a continuous function of  for an arbitrary random variable  with .
\end{lemma}
\begin{IEEEproof}
Note that , then according to dominated convergence theorem \cite{probability}, we have, for , .
\end{IEEEproof}
\begin{lemma}\label{qd}
If  has been designed in priori and is independent of the networks as well as channel fading, , then there exists a constant  such that  if and only if:

\end{lemma}
\begin{IEEEproof}
Set , . According to Lemma \ref{continuity},  is a continuous function with respect to . We then consider the inequality .

Since  is an open set, then continuity implies that  is an open set. On the other hand, dominated convergence theorem ensures that . So there exists a positive number  such that


That is to say, if , then .
Meanwhile, it is easy to show that  is non-increasing with . Thus for all  we have


Therefore, the  is the  we are looking for and this lemma is proved.
\end{IEEEproof}
\emph{Remark:} In Lemma \ref{qd}, if  is changed into , then  is changed into  and the proof is similar.
As for , because of monotonicity, the  can be found through numerical method such as bisection method \cite{bisection}.

\begin{lemma}\label{max}
If  is a random variable,  and  are constants, consider this optimization problem:

The solution of this optimization problem follows:
If  is the cumulative distribution function(CDF) of the optimal  then,\\

\end{lemma}
\begin{IEEEproof}
Set  is a random variable with probability density function(PDF) , and  is independent of .
Then we have:

On the other hand,

Then the result comes directly from Theorem 1 in \cite{RPC3}
\end{IEEEproof}

Then we can start to proof theorem \ref{PFR1t}
\begin{IEEEproof}
Set , according to \eqref{feasibilityregion} we have:




We calculate . Set

Then calculating  is equivalent to solving the following optimization problem:


Set the CDF of  as , according to Lemma \ref{max}, we have the solution:

The  is given as following:

However, noting that,

That is to say, under ,  is achieved when  is a constant.
Then, we have

where (a) from the fact indicated by \eqref{CPC} that the region can be reached when  is constant.
\end{IEEEproof}

\section{Proof of Theorem \ref{lowestpower1}}\label{lowestpower1proof}
\begin{lemma}\label{eq}
For the two problems \eqref{max3}-\eqref{constraine2} and \eqref{max4}-\eqref{constraine3}, they have the same solution.\\

\end{lemma}
\begin{IEEEproof}
It suffices to prove that when \eqref{max3} is minimized, the equation \eqref{constraind2} or \eqref{constrainc3} holds. Let  minimizes \eqref{max3}. We assume  or . Recall Lemma \ref{continuity} and notice that

where .
There exists a  such that

Set , then  satisfies the constrains (\eqref{constrainc2}, \eqref{constraind2}, \eqref{constraine2}) and . This contradicts with the assumption that  is the solution. Thus we complete the proof.
\end{IEEEproof}
\begin{lemma}
Set , then one solution to the optimization problem:

is that 
\end{lemma}
\begin{IEEEproof}
By simple calculation, it is verified that  satisfies \eqref{constrainf}.
Then according to H\"{o}lder's inequality\cite{probability} and , we have  When , the equation holds and  is minimized.
\end{IEEEproof}
then we can prove the theorem.
\begin{IEEEproof}
Note that if , then,  Thus,
The optimum to problem

is no less than . So, it suffices to prove that when , there is no  that satisfies \eqref{constrainc3}, \eqref{constraind3} and \eqref{constraine3}.\\
We consider the problem:

It is equivalent to find the :

Then,

where (a) is from Lemma \ref{max} and .
Thus, when , there is no  that satisfies \eqref{constrainc3}-\eqref{constraine3}.
\end{IEEEproof}

\section{Proof of Theorem \ref{PFR2t}}\label{PFR2tproof}
Before proof the theorem, we need a lemma.
\begin{lemma}\label{min}
If  is a random variable,  is a constant and  is an exponentially distributed random variable with PDF  , consider this optimization problem:

The solution is given by:

\end{lemma}
\begin{IEEEproof}
According to Cauchy-Schwartz inequality \cite{probability}, we have:

Then,

with equation holding if and only if .
\end{IEEEproof}

Then we can prove the theorem.
\begin{IEEEproof}
Set .
We first calculate . According to Lemma \ref{min}, we have


\end{IEEEproof}


\section{Proof of Lemma \ref{function}}\label{functionproof}


\begin{IEEEproof}
If , set   \footnote{Strictly, it should be written as , to simplify the notation we use }, so given  is a deterministic function \cite{probabilityross}. According to Jensen's inequality for conditional expectation \cite{probability},

Then \eqref{constrainc5}, \eqref{constraind5} and \eqref{constraine5} are satisfied.
Meanwhile,

If , set , so given  is a deterministic function \cite{probabilityross}. According to Jensen's inequality for conditional expectation \cite{probability},

Then \eqref{constrainc5}, \eqref{constraind5} and \eqref{constraine5} are satisfied.
Meanwhile,

\end{IEEEproof}







\bibliographystyle{IEEEtran}
\bibliography{IEEEabrv,Reference}
\end{document}
