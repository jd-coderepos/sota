



In this section we consider a generic constraint theory CT  which allows the built-in predicate  and we assume that the signature contains at least a constant and a function (of arity  0) symbol. We have seen that in this case both CHR and CHR are Turing powerful, which means that in principle one can always encode CHR into CHR. The question is how difficult and how acceptable such an encoding is and this question can have important practical consequences: for example, a distributed algorithm can be implemented in one language in a reasonably simple way and cannot be implemented in another (Turing powerful) language, unless one introduces rather complicated data structures or loses some compositionality properties (see \cite{VPP07}).


We prove now that, when considering \emph{acceptable encodings} and generic goals (whose components can share variables) CHR cannot be embedded into CHR  while preserving data sufficient answers. As a corollary we obtain that also qualified answers cannot be preserved. This general result is obtained by proving two more specific results.

First we have to formally define what an acceptable encoding is. We do this by giving a generic definition, which will be used also in the next section, which considers separately program and goal encodings. Hence in the following we denote by CHR some CHR (sub)language and assume that  is the set of all the CHR programs while  is the set of possible CHR goals. Usually the sub-language is defined by suitable syntactic restrictions, as in the case of CHR, however in some cases we will use also a semantic characterization, that is, by a slight abuse of notation, we will identify a sub-language with a set of programs having some specific semantic property.
A  \emph{program encoding} of CHR into CHR is then defined as any  function . To simplify the treatment we assume that both the source and the target language of the program encoding use the same built-in constraints semantically described by a theory CT. Note that we do not impose any other restriction on the program translation (which, in particular, could also be non compositional).

Next we have to define how the initial goal of the source language has to be translated into the target language. Analogously to the case of programs, the goal encoding is a function , however here we require that such a function is compositional (actually, an homomorphism) with respect to the conjunction of atoms, as mentioned in the introduction. Moreover, since both the source and the target language share the same constraint theory, we assume that the built-ins present in the goal are left unchanged. These assumptions  essentially mean that our encoding respects the structure of the original goal and does not introduce new relations among the variables which appear in the goal.
Note that we differentiate the goals  of the source language from those   of the target one because, in principle, a CHR program could use some user defined predicates which are not allowed in the goals of the original program -- this means that the signatures of (language of) the original and the translated program could be different.
Note also that the following definitions are parametric  with respect to a class  of goals: clearly considering different classes of goals could affect our encodability results. Such a parameter will be instantiated when the notion of acceptable encoding will be used.


Finally, as mentioned before, we are interested in preserving data sufficient or qualified answers.  Hence we have the following definition.


\begin{definition}[Acceptable encoding ]\label{def:resenc}
Let  be  a class of CHR goals and let CHR and CHR be two CHR (sub)languages. An {\em acceptable encoding} of CHR into CHR, for the class of goals , is a pair of mappings  and  which satisfy the following conditions:
\begin{enumerate}
 \item  and   share the same CT;
 \item for any goal ,  =  holds. We also assume that the built-ins present in the goal are left unchanged;
 \item Data sufficient answers are preserved for the set of programs  and the class of goals , that is, for all  and ,
 .
\end{enumerate}
Moreover we define an  {\em acceptable encoding for qualified answers} of CHR into CHR, for the class of goals , exactly as an acceptable encoding, with the exception that the third condition above is replaced by the following:
\begin{description}
 \item[(3')] Qualified answers are preserved for the set of programs  and the class of goals , that is, for all  and  ,
 .
\end{description}
\end{definition}







Obviously the notion of acceptable encoding for qualified answers is stronger than that one of acceptable encoding, since   holds.
Note also that, since we consider goals as multisets, with the second condition in the above definition we are not requiring that the order of atoms in the goals is preserved by the translation: We are only requiring that
the translation of  is the conjunction of the translation of  and of , i.e. the encoding is homomorphic.
Weakening this condition by requiring that the translation of  is some form of composition of the translation of  and of  does not seem reasonable, as conjunction is the only form for goal composition available in these languages.
Moreover, homomorphic encoding are a quite common assumption in the papers studying expressiveness of concurrent languages, see for example \cite{Pal03}.

We are now ready to prove our separation results. Next section considers only data sufficient answers.







\subsection{Separating CHR and CHR by considering data sufficient answers}

In order to prove our first separation result we need the following lemma which states two key properties of CHR computations.
The first one says that if the conjunctive  with input constraint  produces a data sufficient answer , then when considering one component, say , with the input constraint 
we obtain the same data sufficient answer.  The second one states that when considering the subgoals  and  there exists at least one of them which allows to obtain the same data sufficient answer  also starting with an input constraint  weaker than .





\begin{lemma}\label{lem:preservarisposte}
 Let  be a CHR program  and let  be a goal, where  is a built-in constraint,  and  are multisets of CHR constraints. Let  and assume that  in  has the data sufficient answer .
Then the following holds:
\begin{itemize}
\item
Both the goals   and
 have the same data sufficient answer .
\item If  then there exists a built-in constraint  such that ,  and
at least one of the two goals   and  has the data sufficient answer .
\end{itemize}

\end{lemma}


\begin{proof}
The proof of the first statement is straightforward (since we consider single headed programs). In fact, since the goal  has the data sufficient answer  in , the goal  can either answer  or can produce a configuration where the user defined constraints are waiting for some guards to be satisfied in order to apply a rule , but since the goal contains all the built-in constraints in the answer all the guards are satisfied letting the program to answer .

We prove the second statement.
Let   be the derivation producing the data sufficient answer  for the goal .

By definition of derivation and since by hypothesis ,
 must be of the form
 
where for ,  and   are built-in constraints such that
 and
.
We choose .
By definition of derivation and since  is a CHR program, the transition
 must be a rule application of a single headed rule , which must match with a constraint  that was derived (in the obvious sense) by  or . Without loss of generality, we can assume that  was derived from .
By construction  suffices to satisfy the guards needed to reproduce , which can then fire the rule , after which all the rules needed to let the constraints of  disappear can fire.
Therefore we have that  where  and then the thesis follows.
\end{proof}









 Note that Lemma \ref{lem:preservarisposte} is not true anymore if we consider (multiple headed) CHR programs.
 Indeed if we consider the program  consisting of the single rule  then the goal  has the data sufficient answer  in , but for each constraint  the goal   has no data sufficient answer in .
With the help of the previous lemma we can now prove our main separation result.
The idea of the proof is that any possible encoding of the  rule  into CHR would either produce more answers for the goal  (or ), or would not be able to provide the answer  for
the goal .
Using the notation introduced in Definition \ref{def:dialects} and considering  as multiset inclusion, we have then the following.


\begin{theorem}\label{th:general}
Let  be a class of goals such that if    is a head of a rule then  for any . Then, for n, there exists no acceptable encoding of
CHR in CHR for the class .
\end{theorem}
\begin{proof}
The proof is by contradiction. Assume that there exists an acceptable encoding
 and 
 of CHR into CHR for the class of goals  and let  be the program consisting of the single rule
 
Assume also, that  (restricted to the variables in ) is not the weakest constraint, i.e. assume that there exists    such that  where . Note that this assumption does not imply any loss of generality, since, as mentioned at the beginning of this section, we assume that the constraint theory allows the built-in predicate  and the signature contains at least a constant and a function (of arity  0) symbol.

Since  the goal    has the data sufficient answer   in the program  and since the encoding preserves data sufficient answers,  the goal  has the data sufficient answer   also in the program .  From the compositionality of the translation of goals and the previous Lemma \ref{lem:preservarisposte} it follows that there exists a constraint  such that ,
 and
at least one of the two goals 
 and  has the data sufficient answer  in the encoded program .

However neither  
 nor  has any  data sufficient answer in the original program . This contradicts the fact that
 there exists an acceptable encoding
 of CHR into CHR for the class of goals , thus concluding the proof.
\end{proof}





Obviously, previous theorem implies that (under the same hypothesis) no acceptable encod\-ing for qualified answers of CHR into CHR exists,
since  . The hypothesis we made on the class of goals  is rather weak, as typically heads of rules have to be used as goals.
From  Theorem~\ref{th:general} we have immediately the
following.

\begin{corollary}\label{ch:general}
Let  be a class of goals such that if    is a head of a rule then  for any . Then, for n, there exists no acceptable encoding (for qualified answers) of
CHR in CHR for the class .
\end{corollary}


As an example of the application of the previous theorem consider the program (from  \cite{Fruhwirth98}) contained
in Figure \ref{tab:lessequal} which allows one to define the user-defined constraint \textit{Lessequal} (to be interpreted as  ) in terms of the only  built-in constraint = (to be interpreted as syntactic equality).
\begin{figure}[t]
\begin{center}
\fbox{
\begin{minipage}{12cm}

      \end{minipage}
}
\end{center}
\caption{A program for defining    in CHR}\label{tab:lessequal}
\end{figure}
For instance, given the goal  
 after a few computational steps the program will answer .
 Now, for obtaining this behaviour, it is essential to use multiple heads, as already claimed in \cite{Fruhwirth98} and formally proved by the previous theorem.
In fact, following the lines of the proof of Theorem  \ref{th:general}, one can show that
if a single headed program  is  any translation of the program in Figure \ref{tab:lessequal} which produces the correct answer for the goal above, then there exists a subgoal which has an answer in  but not in the original program.

\subsection{Separating CHR and CHR by considering qualified answers}

Theorem \ref{th:general} assumes that programs have non trivial data sufficient answers. Nevertheless, since qualified answers are the most interesting ones for CHR programs, one could wonder what happens when considering the CHR language (see Definition \ref{def:dialects}).

Here we prove that also CHR  cannot be encoded into CHR.  The proof of this result is somehow easier to obtain since the multiplicity of atomic formulae here is important. In fact,  if  is a user-defined constraint,
the meaning of ,  does not necessarily coincide with that one of . This is well known also in the case of logic programs (see any article on the S-semantics of logic programs): consider, for example, the  program:

which is essentially a pure logic program written with the CHR syntax. Notice that when considering an abstract operational semantics, as the one that we consider here, the presence of commit-choice does not affect the possible results. For example, in the previous program when reducing the goal  one can always choose (non deterministically) either the first or the second rule.



Now the goal  in such a program has the (data sufficient) answer   while this is not the case for the goal  which has the answer   and the answer  (of course, using guards one can make more significant examples). Thus, when considering user-defined predicates, it is acceptable to distinguish   from , i.e. to take into account the multiplicity. This is not the case for  ``pure'' built-in constraints, since the meaning of a (pure) built-in is defined by a first order theory CT in terms of logical consequences,  and from this point of view  is equivalent to .

In order to prove our result we need first the following result which states that, when considering single headed rules, if the goal is replicated then there exists a computation where at every step a rule is applied twice. Hence it is easy to observe that if the computation will terminate producing a qualified answer which contains an atomic user-defined constraint, then such a constraint is replicated. More precisely we have the following Lemma whose proof is immediate.

\begin{lemma}\label{lem:single}
 Let  be a CHR program. If  is a goal whose evaluation in  produces a qualified answer  containing the atomic user-defined constraint , then the goal  has a qualified answer containing .
\end{lemma}

Hence we can prove the following separation result.


\begin{theorem}
Let  be a class of goals such that if    is a head of a rule then  for any . Then, for n, there exists no acceptable encoding for qualified answers of CHR into CHR for the class .
\end{theorem}

\begin{proof}
The proof will proceed by contradiction. Assume that there exists an acceptable encoding for qualified answers
 and 
 of CHR into CHR for the class of goals  and let  be the program consisting of the single rule:
 where  is an atomic user-defined constraint. The goal  in  has a qualified answer  (note that for each goal ,  has no trivial data sufficient answers different from ).

Therefore, by definition of acceptable encoding for qualified answers, the goal
   in  has a qualified answer  (with the built-in constraint ).
Since the compositionality hypothesis implies that   =
, from Lem\-ma \ref{lem:single} it follows that  in program  has also a qualified answer , but this answer cannot be obtained in the program with multiple heads thus contradicting one of the hypothesis on the acceptable encoding for qualified answers.
Therefore such an encoding cannot exist.
\end{proof}

From previous theorem and Theorem \ref{th:general} follows that, in general, no acceptable encoding for qualified answers of CHR in CHR exists.

\begin{corollary}\label{ch:general1}
Let  be a class of goals such that if    is a head of a rule then  for any . Then there exists no acceptable encoding (for qualified answers) of
CHR in CHR for the class .
\end{corollary}


