
\documentclass[10pt, conference, compsocconf]{IEEEtran}
\usepackage{graphicx}
\usepackage{array}
\usepackage{multirow}
\usepackage{amsmath}
\usepackage{algorithm}
\usepackage{algorithmic}
\usepackage{subfigure}
\usepackage{float}
\begin{document}

\title{Minimizing Electricity Theft\\ using Smart Meters in AMI}

\author{M. Anas, N. Javaid, A. Mahmood, S. M. Raza, U. Qasim, Z. A. Khan\\

        University of Alberta, Alberta, Canada\\
        Department of Electrical Engineering, COMSATS\\ Institute of
        Information Technology, Islamabad, Pakistan. \\
        Faculty of Engineering, Dalhousie University, Halifax, Canada.
        }

\maketitle


\begin{abstract}
Global energy crises are increasing every moment. Every one has the attention towards more and more energy production and also trying to save it. Electricity can be produced through many ways which is then synchronized on a main grid for usage. The main issue for which we have written this survey paper is losses in electrical system. Weather these losses are technical or non-technical. Technical losses can be calculated easily, as we discussed in section of mathematical modeling that how to calculate technical losses. Where as non-technical losses can be evaluated if technical losses are known. Theft in electricity produce non-technical losses. To reduce or control theft one can save his economic resources. Smart meter can be the best option to minimize electricity theft, because of its high security, best efficiency, and excellent resistance towards many of theft ideas in electromechanical meters. So in this paper we have mostly concentrated on theft issues.
\end{abstract}

\begin{IEEEkeywords}
Electricity theft, Smart meter, Non-Technical losses, Advanced Metering Infrastructure.
\end{IEEEkeywords}


\section{Introduction}
\PARstart{E}{lectricity}, generated through many ways, is synchronized on a single bus bar of the grid for transmission. Before utilization of electricity, it passes from certain phases. It is first generated, step upped in transformer deck, passed from switch yard for transmission through power lines. After transmission it is distributed for utilization to the customers. This energy needs to be billed as well. Usually two types of devices are mainly used for billing procedure.

\begin{enumerate}
\item
Electromechanical KWh meters.
\item
Smart meters.
\end{enumerate}

Our energy is strained to the utmost now a day, so using energy efficiently is one of the issues which need urgent attention. That is why electricity is to be dealt with great care. As for as knowledge is concerned there is no such password which can not be cracked but best password is the one which is being cracked in a larger period of time. This is one basic reason that whole world is shifting from analog devices to digital devices. That is why analog electromechanical meters are being substituted by smart meters.
Digital devices provide better security and controlling options. The better detection and controlling of losses is one of the reasons for substitution of smart meters.

Every thing occurs for a reason, so the reason for this substitution is losses in electrical systems. There are mainly two types of losses.

\begin{enumerate}
\item
Technical losses.
\item
Non-Technical/Commercial losses.
\end{enumerate}

In developing countries electricity theft is a common practice specially in remote areas, as they do not pay utility bills to a government company in case of electricity and gas as well. To solve this problem governments must think of an idea to provide help in terms of subsidy to manage this issue.

In section-II related work and motivation is explained. In section-III of this paper losses are discussed which are caused due to electricity theft. In section-IV ways of communication are discussed, to send data from end user to the grid. In section-V causes and effects of electricity theft is explained. Some mathematical techniques are discussed in section VI. In section VII we have concluded this paper.

\section{Related Work and Motivation}
In [1,3] authors explained theft control very well in a sense that they proposed a model. In this model they calculated NTL in external control section, and if NTL \textgreater\ 5\%, legal customers are disconnected for some interval. Harmonic generator is operated in this time period, which destroys the electrical equipment of all the illegal consumers. Reconnect normal supply for genuine customers. Although this is a good model that electricity theft is an issue that one can make equipments of an illegal users starts malfunctioning. However this model can be improved to stop functioning of the equipment of an illegal users, weather using smart meters or any other technique.

S. McLaughlin \textit{et al.} explained some of the energy theft in Advanced Metering Infrastructure (AMI), proposed an idea of a communication architecture from smart meter to grid using meter to meter communication. For boosting the data signals using collectors and receptors. It defines this procedure in a network known as Backhaul network, used to transport data to utility. However energy theft in smart meters can be a technical person, if he removes the -controller from his meter. It will not be able to measure readings and send it to utility for further process.

\begin{table*}[t]
\begin{center}
\begin{tabular}{|m{2cm}|c|c|r|r|r|r|r|r|r|r|}
\hline
\multicolumn{ 3}{|c|}{{\bf Months }} & {\bf July} & {\bf Aug} & {\bf Sept} & {\bf Oct} & {\bf Nov} & {\bf Dec} & {\bf Jan} & {\bf Feb} \\
\hline
\multicolumn{ 1}{|c|}{{\bf 2010-2011}} & \multicolumn{ 1}{|c|}{{\bf Energy (MKWH)}} & {\bf Received} &  1764.81 &  1777.29 &  1518.89 &  1461.89 &  1136.25 &  1179.97 &  1169.85 &  1058.03 \\ \cline{3-11}

\multicolumn{ 1}{|c|}{{\bf }} & \multicolumn{ 1}{|c|}{{\bf }} & {\bf Sold} &  1508.41 &  1513.76 &  1311.82 &  1282.98 &  1047.91 &  1060.11 &  1057.74 &  1009.38 \\
\cline{2-11}
\multicolumn{ 1}{|c|}{{\bf }} & \multicolumn{ 2}{|c|}{{\bf Percentage Losses}} &    14.53 &    14.83 &    13.63 &    12.24 &     7.77 &    10.16 &     9.58 &     4.60 \\
\hline
\multicolumn{ 1}{|c|}{{\bf 2011-2012}} & \multicolumn{ 1}{|c|}{{\bf Energy (MKWH)}} & {\bf Received} &  1693.09 &  1768.82 &  1570.68 &  1509.01 &  1199.71 &  1179.12 &  1127.43 &  1140.52 \\ \cline{3-11}

\multicolumn{ 1}{|c|}{{\bf }} & \multicolumn{ 1}{|c|}{{\bf }} & {\bf Sold} &  1449.12 &  1510.51 &  1365.99 &  1329.63 &  1106.89 &  1115.94 &  1024.03 &  1085.20 \\
\cline{2-11}
\multicolumn{ 1}{|c|}{{\bf }} & \multicolumn{ 2}{|c|}{{\bf Percentage Losses}} &    14.41 &    14.60 &    13.03 &    11.89 &     7.74 &     5.36 &     9.17 &     4.85 \\
\hline
\multicolumn{ 3}{|c|}{{\bf Decrease}} &     0.12 &     0.22 &     0.60 &     0.35 &     0.04 &     4.80 &     0.41 &    -0.25 \\
\hline
                                  \multicolumn{ 11}{c}{Table 1. Energy Losses in a populated city in year 2010 till 2012} \\
\end{tabular}
\end{center}
\end{table*}

\begin{figure*}[t]
\centering
\includegraphics[height=10cm, width=14cm]{matlab.eps}
\caption{Month wise Graphical Representation of losses in a populated city in year 2010-2012.}
\end{figure*}

In [4,5] authors elaborated ways of communication, in how many ways we can transmit the data of smart meter to utility. S. S. S. R Depuru and \textit{et. al.} shown how an electromechanical meter works and why is smart meter better than electromechanical meter.

In [6] central observer meter is placed, which is cost effective because a smart meter is placed at secondary side of transformer. It used matrix based approach in excel to show electricity theft case and normal case. Where as if large amount of data has to be managed than larger matrices will be required. Memory requirements will increase, time consumption to solve large matrices will increase.

[8,9] have some mathematical modeling techniques which helps to detect and control electricity theft using some classifiers. [8] discussed a Graphical User Interface (GUI) based software implemented in Malayesia.

\section{Losses Due To Electricity Theft}

Electricity theft is basically an illegal way of getting the energy for different uses, resulting in loss for utility companies. Losses consist of  technical and non technical losses. There are about \^1\mu2^{32}2^{128}$
Solving this equation we can find losses which are normally called as technical losses, that is losses on generation side.

Using Lagrange for SVM we can change parameters in this formula for desired situation, further in this method we can get a matrix form, graph of standard form and collect data to identify theft.

Their is one other technique ANN-MLP which is based on modeling techniques, obeys some of non-linear statistical modeling or tree diagram. Other part of it is MLP which is a type of linear classifier and selects better output among outputs from its input.

Ramos, C. C. O \textit {et al.} proposed OPF based technique [8]. It is an approach in which better output is replaced for the previous value selected to reach to identify theft. It needs no parameters to be assumed. Its training phase operation is very fast, an overview is tested by [8] and showed that an OPF has a higher hit rate of theft and having more accuracy than SVM-LINEAR, SVM-RBF, and ANN-MLP [8].

\section{Conclusion}
Electricity thefts are of many types. They are summarized in this paper. Theft can be possible in smart meter as well, which can also be controlled by spreading awareness in peoples on media etc. Some mathematical models are also helpful to detect and control electricity theft.

\begin{thebibliography}{1}

\bibitem{IEEEhowto:kopka}
S. S. S. R Depuru, L. Wang, V. Devabhaktuni. ``Electricity theft: Overview, issues, prevention and a smart meter based approach to control theft."
\hskip 1em plus 0.5em minus 0.4em\relax Energy Policy 39 (2011) 1007-1015.

\bibitem{}
S. McLaughlin, D. Podkuiko, and P. McDaniel. ``Energy theft in Advanced Metering Infrastructure" \hskip 1em plus 0.5em minus 0.4em\relax Pennsylvania State University, University Park.

\bibitem{}
S. S. S. R Depuru, L. Wang, V. Devabhaktuni and N. Gudi. ``Measures and setbacks for controlling electricity theft."

\bibitem{}
S. S. S. R Depuru, L. Wang, V. Devabhaktuni. ``Smart meters for power grid: Challenges, issues, advantages and status." \hskip 1em plus 0.5em minus 0.4em\relax Renewable and sustainable energy reviews 15 (2011) 2736-2742.

\bibitem{}
S. S. S. R Depuru, L. Wang, V. Devabhaktuni and N. Gudi.``Smart meters for power grid: Challenges, issues, advantages and status". \hskip 1em plus 0.5em minus 0.4em\relax IEEE 2011.

\bibitem{}
C. J. Bandim, J. E. R. Alves Jr., A. V. Pinto Jr, F. C. Souza, M. R. B. Loureiro, C. A.Mangalhaes and F. Galvez-Durand. ``Identification of energy theft and tampered meters using a central observer meter: A mathematical approach" \hskip 1em plus 0.5em minus 0.4em\relax IEEE 2003.

\bibitem{}
www.lesco.gov.pk  (12-04-2012)

\bibitem{}
J. Nagi, A. M. Mohammad, K. S. Yap, S. K. Tiog, S. K. Ahmed. ``Non-Technical Loss for Detection of Electricity Theft using Support Vector Machines"
\hskip 1em plus 0.5em minus 0.4em\relax 2nd IEEE international conference on power and energy.

\bibitem{}
C. C. O Ramos, A. N. Souza, J. P. Papa, A. X. Falcao. ``Fast Non-Technical Lasses Identification through Optimum-Path Forest."

\end{thebibliography}


\end{document}