\documentclass[final,3p,times,twocolumn]{elsarticle}
\usepackage{amssymb,amsmath,amsthm,bm}
\usepackage{geometry}
\usepackage{algorithm}
\usepackage{algpseudocode}



\usepackage[retainorgcmds]{IEEEtrantools}

\usepackage{graphicx}
\graphicspath{{figures-pdf/}}

\usepackage[normalem]{ulem}
\usepackage{url}
\usepackage{lineno}
\usepackage{bm}
\usepackage{graphicx,color,overpic}



\newlength\imagewidth
\setlength\imagewidth{0.9\columnwidth}
\newlength\figwidth
\setlength\figwidth{0.45\columnwidth}
\newlength\sfigwidth
\setlength\sfigwidth{0.3\columnwidth}
\newlength\vfigskip
\setlength\vfigskip{0.5em}
\journal{Elsevier}
\geometry{left=1.8cm,right=1.8cm}

\usepackage{color}
\definecolor{dgreen}{rgb}{0,.6,0}
\newcommand\comments[1]{\textcolor{blue}{#1}}   \newcommand\commented[1]{\textcolor{red}{#1}}   \newcommand\modified[1]{\textcolor{dgreen}{#1}} 

\newtheorem{theorem}{Theorem}
\newtheorem{Proposition}{Proposition}
\newtheorem{Property}{Property}
\newtheorem{property}{Property}
\newtheorem{corollary}{Corollary}
\newtheorem{Corollary}{Corollary}
\newtheorem{Fact}{Fact}
\newtheorem{definition}{Definition}
\newcommand{\diag}{\mathop{\mathrm{diag}}}



\usepackage{hyperref}
\hypersetup{
    colorlinks=true,
    linkcolor=blue,
    filecolor=magenta,
    urlcolor=cyan,
}

\begin{document}

\begin{frontmatter}

\title{Cracking a hierarchical chaotic image encryption algorithm based on permutation}


\author{Chengqing Li\corref{corr}}
\ead{DrChengqingLi@gmail.com}


\cortext[corr]{Corresponding author.}

\address{College of Information Engineering, Xiangtan University, Xiangtan 411105, Hunan, China}


\begin{abstract}
In year 2000, an efficient hierarchical chaotic image encryption (HCIE) algorithm was proposed, which divides a plain-image
of size  with  possible value levels into  blocks of the same size and then operates position permutation on two levels: intra-block and inter-block. As a typical position permutation-only encryption algorithm, it has received intensive attention. The present paper analyzes specific security performance of HCIE against ciphertext-only attack and known/chosen-plaintext attack. It is found that only  known/chosen plain-images are sufficient to achieve a good performance, and
the computational complexity is , which effectively demonstrates that hierarchical permutation-only image
encryption algorithms are less secure than normal (i.e., non-hierarchical) ones. Detailed experiment results are given to verify the feasibility of the known-plaintext attack. In addition, it is pointed out that the security of HCIE against ciphertext-only attack was much overestimated.
\end{abstract}
\begin{keyword}
Chosen-plaintext attack\sep chaotic cryptanalysis \sep known-plaintext attack\sep permutation.
\end{keyword}
\end{frontmatter}

\section{Introduction}

With the increasing transmission speeds of wired/wireless networks and popularization of image capturing devices and cloud storage
services, image data are transmitted over open networks more and more frequently. This makes security of image data become
more and more important. The public concern of it becomes serious as news about the illegal online leak of personal photos of some celebrities was released.
As a chaotic system owns some similar properties as that of modern encryption schemes, it has been intensively studied as an alternative approach for designing
secure and efficient encryption schemes \cite{Li:Dissertation2003,YaobinMao:CSF2004,AlvarezLi:Rules:IJBC2006}. The main idea and principle of applying chaos theory to protecting images can be traced back to 1986 \cite{Crutchfield:Chaos:SA86}, which demonstrates the stretching effect of a chaotic map on a painting of Henri Poincar\'{e}, a founder of modern chaos theory.

The simplest and most efficient method for protecting multimedia data is permuting the positions of their spatial pixels \cite{Matias&Shamir:CurveImageEncryption:Crypto87} or frequency coefficients \cite{Zeng:VideoScrambling:IEEETMM2003}. In the literature, some synonyms of permutation, transposition, shuffle, scramble \cite{Zeng:VideoScrambling:IEEETMM2003}, swap and shift,
are used. Security scrutiny on some specific permutation-only encryption algorithms against known/chosen-plaintext attacks were previously developed \cite{Jan-Tseng:BreakingSCAN:IPL96,Chang-Yu:BreakingSCAN:PRL2002}. In \cite{WHLi:shuffle:ACMM2012}, a ciphertext-only attack on
a specially simple permutation-only encryption algorithm was proposed utilizing correlation redundancy remaining in the cipher-image. No matter how the permutation relationships are generated and what the permutation object is,
any permutation-only encryption algorithm can always be represented by a \textit{permutation relationship matrix}, whose entry
stores the corresponding permuted location in the cipher-text \cite{Li:Permutation:SPIC2008}. The security of permutation-only encryption algorithm
relies on its \textit{real permutation domain}, in which any element in the permutation object can be permuted independently. As for a permutation domain of size  with  possible value levels, it is estimated that the required number of known/chosen-plaintexts for an efficient plaintext attack is , where  denotes the ceiling function. An upper bound of the attack complexity is also derived therein to be
, where  is the number of known/chosen plain-images \cite{Li:Permutation:SPIC2008}. In \cite{Lcq:Optimal:SP11}, the computational complexity of the attack is
further reduced to  by replacing the set intersection operations of quadratic complexity with linear element access operations.
Even so, all kinds of permutation operations are still being used in multimedia protection today \cite{Li:scramble:ITCSVT2008,Sohn:scrambling:TCSVT2011,Zhouyc:perturbation:SP14,SMYU:ARM:IJBC2014}.

In \cite{Yen-Guo:HCIE:IEEPVISP2000}, a typical example of permutation-only image encryption algorithms, called HCIE (hierarchical chaotic image encryption),
was proposed. Although security performance of general permutation-only image encryption algorithms against plaintext attack has been
quantitatively analyzed, specific security performance of HCIE is still not evaluated. The core of HCIE is a
permutation function composed of rotation operations of four directions, originates from an intellectual toy, Rubik's Cube \cite{Korf:Cube:97}.
In \cite{Yen-Guo:HCIE:IEEPVISP2000}, the authors claimed about the security property of HCIE as follows: ``By way of collecting some original images and their
encryption results or collecting some specified images and their corresponding encryption results, it is still difficult for
the cryptanalysts to decrypt an encrypted image correctly because the permutation relationship is different for each
image." In this paper, we will demonstrate that the claim on the robustness of HCIE against known/chosen-plaintext attack is groundless. Further more, we find that the hierarchical encryption structure suggested in HCIE does not provide any higher security against known/chosen-plaintext attack, but actually make
the overall security even weaker. In addition, we find the capability of HCIE against ciphertext-only attack was much over-estimated.

The rest of this paper is organized as follows. The algorithm HCIE is briefly introduced in Sec.~\ref{sec:HCIE}.
Detailed cryptanalysis on HCIE is provided in Sec.~\ref{sec:cryptanalysisHCIE2}, with some experimental results.
The last section concludes the paper.

\section{The hierarchical chaotic image encryption algorithm (HCIE)}
\label{sec:HCIE}

HCIE is a two-level hierarchical permutation-only image encryption algorithm, in which all involved permutation relationships are defined by pseudo-random
combinations of four rotation mappings with pseudo-random
parameters. For an image, , the four
mapping operations are described as follows, where 
holds for each mapping.

\begin{definition}
The mapping   is defined to
rotate the -th row of , in the left (when ) or right
(when ) direction, by  pixels.
\end{definition}

\begin{definition}
The mapping   is defined to
rotate the -th column of , in the up (when ) or down
(when ) direction, by  pixels.
\end{definition}

\begin{definition}
The mapping   is
defined to rotate all pixels satisfying , in the lower-left
(when ) or upper-right (when ) direction, by  pixels.
\end{definition}

\begin{definition}
The mapping   is
defined to rotate all pixels satisfying , in the upper-left
(when ) or lower-right (when ) direction, by  pixels.
\end{definition}

Given a pseudo-random bit sequence  starting from ,
the \texttt{Sub\_HCIE} function in Algorithm~1 is used to permute an
 image  to become another 
image , where  are control
parameters.
\begin{algorithm}
\caption{The \texttt{Sub\_HCIE} function}
\begin{algorithmic}[1]
\Function {\texttt{Sub\_HCIE}}{, , , , }
\For {}
  \State 
  \State 
  \For {}
  \State 
  \EndFor
  \For { }
  \State 
  \EndFor
  \For {}
  \State 
  \EndFor
  \For {}
  \State 
  \EndFor
\EndFor
\State 
\State \textbf{return} 
\EndFunction
\end{algorithmic}
\end{algorithm}
One can see that the \texttt{Sub\_HCIE} function actually
defines an  permutation relationship matrix pseudo-randomly
controlled by  bits in the bit sequence
 from . Based on this function, for an 
image , the four basic parts of HCIE
can be briefly described as follows.
\begin{itemize}
\item \textit{The secret key} is the initial condition  and
the control parameter  of the chaotic Logistic map,  \cite{Devaney:Chaos:2003}, which is realized in -bit finite precision.

\item \textit{Some public parameters}: , , ,
,  and , where ,
, , and .

Note: {\it Although () can all be
included in the secret key, they are not suitable for such a use
due to the following reasons: 1)  are related to ;
2)  are related to  (and then
related to , too); 3)  can be easily guessed from
the mosaic effect of the cipher-image; 4) iteration number  cannot be too large
to achieve an acceptable encryption speed.}

\item \textit{The initialization procedure} of generating the bit
sequence used in the \texttt{Sub\_HCIE} function: run the Logistic
map starting from  to generate a chaotic sequence,
, and then extract 8
bits following the decimal point of each chaotic state  to
yield a bit sequence , where
; finally, set  to let the \texttt{Sub\_HCIE} function
run starting from .

\item \textit{The two-level hierarchical encryption procedure}:

\textit{1) The high-level encryption -- permuting image
blocks}: divide the plain-image  into blocks of size , which compose an 
block-image

where  is the block of size  at the
position . Then, permute the positions of all blocks with
the \texttt{Sub\_HCIE} function in the following way:
a) create a pseudo-image 
containing  non-zero
indices of all image blocks in  and  zero-elements, and
permute  with the \texttt{Sub\_HCIE} function to get a
shuffled pseudo-image ; b) generate a permuted block-image  from  (i.e.,
permute  blockwise) using the shuffled indices contained in
.
The above high-level encryption procedure can be considered as a
permutation of the block-image:
, where
 actually corresponds to an
 permutation relationship matrix.

\textit{2) The low-level encryption -- permuting pixels in every
image block one by one}: for  and
, call the \texttt{Sub\_HCIE}
function to permute each block  so as to get the
corresponding block of the cipher-image :

\end{itemize}

As normalized in \cite{Li:Permutation:SPIC2008}, any permutation-only encryption algorithm exerting on
an object of size  can be represented with a \textit{permutation relationship matrix} of size
, denoted by

where ,
, and
 for any .

In HCIE, a total of
 permutation relationship
matrices are involved: 1) one high-level permutation relationship matrix of
size ; 2)
 low-level
permutation relationship matrices of size . With the
above-mentioned representation of a permutation-only image encryption algorithm,
the secret key  of HCIE is equivalent to the
 permutation relationship matrices
for plain-images of the same size. To facilitate the following discussions, we use
 to denote
the high-level permutation relationship matrix, and use

to denote the 
low-level permutation relationship matrices, where
.
Apparently, the 
permutation relationship matrices can be easily transformed to an equivalent
permutation relationship matrix of size , .

When  and  (or ), the two hierarchical
encryption levels merge into a single one; the
 permutation relationship matrices become one permutation relationship matrix of size ,
a typical permutation-only image encryption algorithm in which each pixel can be independently
permuted to any other position in the whole image by a single
 permutation relationship matrix .

\section{Cryptanalysis of HCIE}
\label{sec:cryptanalysisHCIE2}

\subsection{Ciphertext-only attack}

In \cite{Yen-Guo:HCIE:IEEPVISP2000}, it was claimed that the complexity of
brute-force attacks to HCIE is , since
there are
 secret chaotic bits in  that are
unknown to the attackers. However, this statement is not true due to
the following fact: the  bits are uniquely determined by the
secret key, i.e., the initial condition  and the control
parameter , which have only  secret bits. This means that
there are only  different chaotic bit sequences.

Now, let
us study the real complexity of brute-force attacks. For each pair
of guessed values of  and , the following operations
are needed:
\begin{itemize}
\item generating the chaotic bit sequence: there are  chaotic
iterations;

\item creating the pseudo-image : the complexity is ;

\item shuffling the pseudo-image : running the
\texttt{Sub\_HCIE} function once;

\item generating : the complexity is ;

\item shuffling the partition image : running the
\texttt{Sub\_HCIE} function for
 times.
\end{itemize}
Assume that the computing complexity of the \texttt{Sub\_HCIE}
function is . Then, the total complexity of
brute-force attacks to HCIE can be estimated to be
, which is much smaller than
 when  is not too small.
Additionally, considering the fact that the Logistic map can
exhibit a sufficiently strong chaotic behavior only when  is
close to 4 \cite{Li:logistic:ND2014}, the above complexity should
be even smaller. This analysis shows that the security of
HCIE was much over-estimated by the authors in
\cite{Yen-Guo:HCIE:ISC99,Yen-Guo:HCIE:IEEPVISP2000}, for brute-force attacks.

Observing the hierarchical permutation structure of HCIE, one can see that
the histogram of each  block in the plain-image will keep unchanged
during the whole permutation process of HCIE. Due to the strong correlation between neighbouring pixels
(and even blocks) of natural images (See \cite[Fig.~5]{Li:RCES:JSS2008}), there exists some correlation between histograms of neighbouring blocks.
So, one may determine the relative locations of some blocks in cipher-image by comparing similarity degrees of
histograms for every pair of cipher-blocks \cite{HSLI:Banknote:TM14,Ling:histogramcomparison:PAMI07}.

\subsection{The known-plaintext attack}

Since HCIE is a permutation-only image encryption algorithm, given  known
plain-images  of size  and the
corresponding cipher-images , one can simply call
the \texttt{Get\_Permutation\_Matrix} function defined in \cite[Sec.~3.1]{Li:Permutation:SPIC2008} or its enhanced version in
\cite[Sec.~4]{Lcq:Optimal:SP11} with the input parameter  to estimate an
 permutation relationship matrix , which is equivalent to
the  smaller
permutation relationship matrices. However, if the hierarchical structure of
HCIE is considered, the known-plaintext attack may be quicker and
the estimation will be more effective. Thus, the following hierarchical procedure of
known-plaintext attacks to HCIE is suggested\footnote{For HCIE,
the permutation relationship matrices also depend on the values of the public
parameters. To simplify the following description, without loss of
generality, it is assumed that all public parameters are fixed for
all known plain-images.}:
\begin{itemize}
\item \textit{Reconstruct the high-level permutation relationship matrix
}: \textit{1)  for  and
}: calculate the mean values of the  blocks ,
 and denote them by
 and
;

\textit{2) generate  images  and  of size  as
follows}: ,

and

and call the \texttt{Get\_Permutation\_Matrix} function with the
input parameters

to get an estimated permutation relationship matrix

and its inverse
.
\end{itemize}

\textit{3) \textit{Reconstruct the
 low-level
permutation relationship matrices
}}:
\begin{itemize}
\item for  and
, call the
\texttt{Get\_Permutation\_Matrix} function with the input
parameters , where , to
determine an estimated permutation relationship matrix
 and its inverse
.
\end{itemize}

With the  inverse
matrices  and
,
one can decrypt a new cipher-image  with \texttt{Dermutation} function
given in Algorithm~2 to get an estimated plain-image :
\begin{algorithm}
\caption{The function \texttt{Dermutation}}
\begin{algorithmic}[1]
\Function{\texttt{Dermutation}}{, , }
\For{}
  \For {}
     \State 
        \For{}
           \For{  }
              \State 
              \State 
            \EndFor
      \EndFor
  \EndFor
\EndFor
\State \textbf{return} 
\EndFunction
\end{algorithmic}
\end{algorithm}

In fact, in the above procedure, any measure keeping invariant in
the block permutations can be used instead of the mean value. A
typical measure is the histogram of each  block.
Although the mean value is less precise than the histogram, it
works well in most cases and is effective to reduce the time
complexity. When  and  are both very small, the
efficiency of the mean value will become low, in this case the histogram or
the array of all pixel values can be used as a replacement.
As for an image of size  and  possible value levels,
the number of different histogram is

As  is a huge number for a non-tiny image and histogram is sensitive to the change of pixel value, it is easier to get the high-level
permutation relationship matrix  than the low-level permutation relationship matrices.

Finally, let us see whether the hierarchical structure used in
HCIE is helpful for enhancing the security against the
known-plaintext attack to the common permutation image ciphers. As
discussed in \cite[Sec.~3.1]{Li:Permutation:SPIC2008},  known
plain-images are needed to achieve an acceptable breaking
performance. Since the hierarchical structure makes it possible
for an attacker to work on permutation relationship matrices of size  or  (both smaller than
), it is obvious that for HCIE the number of required
known plain-images will be smaller than
. As the permutation relationship matrix 
can be recovered in a very high probability with one or two known plain-images, one
can conclude as follows: the smaller the  is, the smaller the  is. Also, the attack complexity will
become lower, since the number of used plaintexts is reduced. In this sense, hierarchical permutation-only image
ciphers are less secure than non-hierarchical ones, which discourages the use of HCIE.

\subsection{Experimental results on known-plaintext attack}
\label{section:Experiments}

To verify the decryption performance of the above-discussed
known-plaintext attack to HCIE, some
experiments are performed using the six  test
images with 256 gray scales shown in Fig.~\ref{figure:6TestImages}. Assume that the first  test
images are known to an attacker, the cipher-image of the last test
image is decrypted with the estimated permutation relationship matrices, to evaluate
the breaking performance.

\begin{figure}[!htb]
\centering
\begin{minipage}{\sfigwidth}
\centering
\includegraphics[width=\textwidth]{P0}\\
Image \#1
\end{minipage}
\begin{minipage}{\sfigwidth}
\centering
\includegraphics[width=\textwidth]{P1}\\
Image \#2
\end{minipage}
\begin{minipage}{\sfigwidth}
\centering
\includegraphics[width=\textwidth]{P2}
Image \#3
\end{minipage}\\vfigskip]
\begin{minipage}{\sfigwidth}
\centering
\includegraphics[width=\textwidth]{C23}
Cipher-image \#4
\end{minipage}
\begin{minipage}{\sfigwidth}
\centering
\includegraphics[width=\textwidth]{C24}
Cipher-image \#5
\end{minipage}
\begin{minipage}{\sfigwidth}
\centering
\includegraphics[width=\textwidth]{C25}
Cipher-image \#6
\end{minipage}
\caption{The cipher-images of the six  test images,
when .} \label{figure:Experiment32}
\end{figure}

\begin{figure}[!htb]
\centering
\begin{minipage}{\sfigwidth}
\centering
\includegraphics[width=\textwidth]{D21}\\

\end{minipage}
\begin{minipage}{\sfigwidth}
\centering
\includegraphics[width=\textwidth]{D22}\\

\end{minipage}
\begin{minipage}{\sfigwidth}
\centering
\includegraphics[width=\textwidth]{D23}

\end{minipage}\\vfigskip]
\begin{minipage}{\sfigwidth}
\centering
\includegraphics[width=\textwidth]{C33}
Cipher-image \#4
\end{minipage}
\begin{minipage}{\sfigwidth}
\centering
\includegraphics[width=\textwidth]{C34}
Cipher-image \#5
\end{minipage}
\begin{minipage}{\sfigwidth}
\centering
\includegraphics[width=\textwidth]{C35}
Cipher-image \#6
\end{minipage}
\caption{The cipher-images of the six  test images,
when .} \label{figure:Experiment16}
\end{figure}

\begin{figure}[!htb]
\centering
\begin{minipage}{\sfigwidth}
\centering
\includegraphics[width=\textwidth]{D31}\\

\end{minipage}
\begin{minipage}{\sfigwidth}
\centering
\includegraphics[width=\textwidth]{D32}\\

\end{minipage}
\begin{minipage}{\sfigwidth}
\centering
\includegraphics[width=\textwidth]{D33}

\end{minipage}\1em]
a) decryption error ratio\\
\begin{overpic}[width=2\figwidth]{cardinality}
    \put(50,-3){}
\end{overpic}\
n=\lceil \lceil\log_2(M\cdot N)\rceil/\lceil\log_2(T)\rceil \rceil.

n= \lceil\log_2(M\cdot N)/\log_2(T)\rceil =\lceil \log_T(M\cdot N)\rceil
\label{eq:numofchosenplaintext}

n & = & \max\left(\left\lceil\log_T(S_M\cdot
S_N)\right\rceil,\left\lceil\log_T\left(K\right)\right\rceil\right)\nonumber\\
& \leq &
\left\lceil\log_T(M\cdot N)\right\rceil,\label{equation:HCIE_plaintext_number}

where .
The above equality holds if and only if the hierarchical
encryption structure is disabled, i.e., when .

As the  permutation relationship matrices of HCIE
are uniquely determined by the bit sequence  and the public parameters, one may recover reversely some consecutive bits of  \cite[Sec.~3.3.6]{Lcq:MCS:JSS10}. Furthermore, one can derive
the secret key  and  following the approach given in \cite[Sec.3.3.2]{Li:RCES:JSS2008}.

\section{Conclusion}

Specific security performance of a typical permutation-only encryption algorithm, called HCIE, against ciphertext-only attack and known/chosen-plaintext attacks has been studied in detail. It is found that the capability of HCIE against the former attack was over-estimated much and hierarchical permutation-only image encryption algorithms such as HCIE are less secure than normal permutation-only ones without using hierarchical encryption structures. This work effectively demonstrates that the size of the real permutation domain of a permutation algorithm should be as large as possible in order to reach the best performance. As permutation operation alone cannot provide high level of security, it should be combined with other value substitution functions.

\section*{Acknowledgement}

This research was supported by the Distinguished Young Scholar Program of the Hunan Provincial Natural Science Foundation of China (No.~2015JJ1013).
Some parts of Sec.~3 were completed with the help of \href{www.hooklee.com}{Dr.~Shujun Li}, from Surrey University, UK.


\bibliographystyle{elsarticle-num}
\bibliography{HCIE2}
\end{document} 