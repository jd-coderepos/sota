\newif\ifabstract
\abstracttrue
\abstractfalse \newif\iffull
\ifabstract \fullfalse \else \fulltrue \fi


\ifabstract
\documentclass[twoside,leqno,twocolumn]{article}  
\usepackage{ltexpprt} 
\fi
\iffull
\documentclass{article}  
\usepackage{fullpage}
\fi



\usepackage{amsfonts}
\usepackage{amssymb}
\usepackage{amstext}
\usepackage{amsmath}
\usepackage{xspace}
\usepackage{graphicx} \pdfminorversion=5
\usepackage{url}
\usepackage{subfigure}

\newcommand{\connect}{\leadsto}

\newenvironment{proofof}[1]{\noindent{\bf Proof of #1.}}{\hspace*{\fill}\par\vspace{4mm}}



\newcommand{\MS}{{\sf Machine Scheduling}\xspace}
\newcommand{\RMS}{{\sf Restricted Machine Scheduling}\xspace}
\newcommand{\CM}{{\sf Congestion Minimization}\xspace}
\newcommand{\ST}{{\sf Steiner Tree}\xspace}
\newcommand{\CD}{{\sf Cost-Distance}\xspace}
\newcommand{\PST}{{\sf Priority Steiner Tree}\xspace}
\newcommand{\FCNF}{{\sf FCNF}\xspace}
\newcommand{\AC}{{\sf Asymmetric -Center}\xspace}
\newcommand{\kC}{{\sf -Center}\xspace}
\newcommand{\dH}{{\sf -Hypergraph-Cover}\xspace}
\newcommand{\kH}{{\sf -Hypergraph-Cover}\xspace}
\newcommand{\GS}{{\sf Group-Steiner-Tree}\xspace}
\newcommand{\SC}{{\sf Set-Cover}\xspace}
\newcommand{\GAPSAT}{{\sf Gap-3SAT(5)}\xspace}
\newcommand{\clique}{{\sf Max-Clique}\xspace}
\newcommand{\ML}{{\sf Metric Labeling}\xspace}
\newcommand{\ZE}{{\sf 0-extension}\xspace}
\newcommand{\ZIE}{{\sf -extension}\xspace}
\newcommand{\sat}{{\sf SAT}\xspace}
\newcommand{\mdc}{{\sf MDC}\xspace}
\newcommand{\csp}{{\sf HA-CSP}\xspace}



\newcommand{\size}[1]{\ensuremath{\left|#1\right|}}
\newcommand{\ceil}[1]{\ensuremath{\left\lceil#1\right\rceil}}
\newcommand{\floor}[1]{\ensuremath{\left\lfloor#1\right\rfloor}}
\newcommand{\Dexp}[1]{\dexp\{#1\}}
\newcommand{\Tower}[2]{\operatorname{tower}^{(#1)}\{#2\}}
\newcommand{\logi}[2]{\operatorname{log}^{(#1)}{#2}}
\newcommand{\norm}[1]{\lVert #1\rVert}
\newcommand{\abs}[1]{\lvert #1\rvert}
\newcommand{\paren}[1]{\left ( #1 \right ) }
\newcommand{\union}{\cup}
\newcommand{\band}{\wedge}
\newcommand{\bor}{\vee}

\newcommand{\psd}{\succeq}

\newcommand{\event}{{\cal{E}}}




\renewcommand{\P}{\mbox{\sf P}}
\newcommand{\NP}{\mbox{\sf NP}}
\newcommand{\PCP}{\mbox{\sf PCP}}
\newcommand{\ZPP}{\mbox{\sf ZPP}}
\newcommand{\polylog}[1]{\mathrm{polylog}(#1)}
\newcommand{\DTIME}{\mbox{\sf DTIME}}

\newcommand{\opt}{\mbox{\sf OPT}}


\newcommand{\set}[1]{\left\{ #1 \right\}}
\newcommand{\sse}{\subseteq}

\newcommand{\B}{{\mathcal{B}}}
\newcommand{\tset}{{\mathcal T}}
\newcommand{\iset}{{\mathcal{I}}}
\newcommand{\pset}{{\mathcal{P}}}
\newcommand{\qset}{{\mathcal{Q}}}
\newcommand{\lset}{{\mathcal{L}}}
\newcommand{\bset}{{\mathcal{B}}}
\newcommand{\aset}{{\mathcal{A}}}
\newcommand{\cset}{{\mathcal{C}}}
\newcommand{\fset}{{\mathcal{F}}}
\newcommand{\mset}{{\mathcal M}}
\newcommand{\jset}{{\mathcal{J}}}
\newcommand{\xset}{{\mathcal{X}}}
\newcommand{\yset}{{\mathcal{Y}}}
\newcommand{\rset}{{\mathcal{R}}}
\newcommand{\zset}{{\mathcal{Z}}}

\newcommand{\I}{{\mathcal I}}
\newcommand{\hset}{{\mathcal{H}}}
\newcommand{\sset}{{\mathcal{S}}}
\newcommand{\gset}{{\mathcal{G}}}
\newcommand{\notu}{\overline U}
\newcommand{\nots}{\overline S}
\newcommand{\noth}{\overline H}


\newcommand{\be}{\begin{enumerate}}
\newcommand{\ee}{\end{enumerate}}
\newcommand{\bd}{\begin{description}}
\newcommand{\ed}{\end{description}}
\newcommand{\bi}{\begin{itemize}}
\newcommand{\ei}{\end{itemize}}


\ifabstract

\newtheorem{claim}{Claim}[section] 
\newtheorem{observation}{Observation}[section]
\newtheorem{remark}{Remark}[section]
\fi

\iffull
\newtheorem{remark}{Remark}
\newtheorem{lemma}{Lemma}
\newtheorem{theorem}{Theorem}
\newtheorem{observation}{Observation}
\newtheorem{corollary}{Corollary}
\newtheorem{claim}{Claim}
\newenvironment{Definition}{{\bf Definition}: }{}
\newtheorem{proposition}{Proposition}
\newenvironment{proof}{\par \smallskip{\bf Proof:}}{\hfill\stopproof}
\def\stopproof{\square}
\def\square{\vbox{\hrule height.2pt\hbox{\vrule width.2pt height5pt \kern5pt
\vrule width.2pt} \hrule height.2pt}}
\fi

\newcommand{\qed}{\hfill\vbox{\hrule height.2pt\hbox{\vrule width.2pt height5pt \kern5pt
\vrule width.2pt} \hrule height.2pt}}

\newcommand{\scale}[2]{\scalebox{#1}{#2}}

\newcommand{\fig}[1]{
\begin{figure}[h]
\rotatebox{0}{\includegraphics{#1}}
\end{figure}}

\newcommand{\figcap}[2]
{
\begin{figure}[h]
\rotatebox{270}{\includegraphics{#1}} \caption{#2}
\end{figure}
}

\newcommand{\scalefig}[2]{
\begin{figure}[h]
\scalebox{#1}{\includegraphics{#2}}
\end{figure}}

\newcommand{\scalefigcap}[3]{
\begin{figure}[h]
\scalebox{#1}{\rotatebox{0}{\includegraphics{#2}}} \caption{#3}
\end{figure}}

\newcommand{\scalefigcaplabel}[4]{
\begin{figure}[h]
 \scalebox{#1}{\includegraphics{#2}}\caption{#3}\label{#4}
\end{figure}}



\newcommand{\program}[2]{\fbox{\vspace{2mm}\begin{prog}{#1} #2 \end{prog}\vspace{2mm}}}


\renewcommand{\phi}{\varphi}
\newcommand{\eps}{\epsilon}
\newcommand{\Sum}{\displaystyle\sum}
\newcommand{\half}{\ensuremath{\frac{1}{2}}}

\newcommand{\poly}{\operatorname{poly}}
\newcommand{\dist}{\mbox{\sf dist}}

\newcommand{\reals}{{\mathbb R}}

\newcommand{\rn}{\reals^n}
\newcommand{\rk}{\reals^k}
\newcommand{\newsigma}{\Sigma^{d^{\ceil{t/2}}}}
\newcommand{\R}{\ensuremath{\mathbb R}}
\newcommand{\Z}{\ensuremath{\mathbb Z}}
\newcommand{\N}{\ensuremath{\mathbb N}}
\newcommand{\F}{\ensuremath{\mathbb F}}
\newcommand{\fm}{\enshuremath{\mathbb F}^m}
\newcommand{\func}{:{\mathbb F}^m\rightarrow {\mathbb F}}

\newcommand{\expect}[2][]{\text{\bf E}_{#1}\left [#2\right]}
\newcommand{\prob}[2][]{\text{\bf Pr}_{#1}\left [#2\right]}

\newcommand{\four}[1]{\widehat {#1}}
\newcommand{\fourf}{\widehat {f}}
\newcommand{\fourg}{\widehat {g}}
\newcommand{\chisx}{\chi_S(x)}
\newcommand{\chitx}{\chi_T(x)}
\newcommand{\chisy}{\chi_S(y)}
\newcommand{\chity}{\chi_T(y)}
\newcommand{\chiwx}{\chi_W(x)}
\newcommand{\chiwy}{\chi_W(y)}
\newcommand{\chis}{\chi_S}
\newcommand{\chit}{\chi_T}
\newcommand{\chiw}{\chi_W}
\newcommand{\chisminust}{\chi_{S\triangle T}}

\newenvironment{properties}[2][0]
{\renewcommand{\theenumi}{#2\arabic{enumi}}
\begin{enumerate} \setcounter{enumi}{#1}}{\end{enumerate}\renewcommand{\theenumi}{\arabic{enumi}}}



\setlength{\parskip}{2mm} \setlength{\parindent}{0mm}


\newcommand{\notsat}[1]{\overline{\text{SAT}(#1)}}
\newcommand{\notsats}[2]{\overline{\text{SAT}_{#1}(#2)}}
\newcommand{\MP}{\mbox{\sf Minimum Planarization}\xspace}
\newcommand{\MCN}{\mbox{\sf Minimum Crossing Number}\xspace}
\newcommand{\sndp}{\mbox{\sf SNDP}}
\newcommand{\ecsndp}{\mbox{\sf EC-SNDP}}
\newcommand{\vcsndp}{\mbox{\sf VC-SNDP}}
\newcommand{\kec}{k\mbox{\sf -edge connectivity}}
\newcommand{\kvc}{k\mbox{\sf -vertex connectivity}}
\newcommand{\sskec}{\mbox{\sf single-source}~k\mbox{\sf -edge connectivity}}
\newcommand{\sskvc}{\mbox{\sf single-source}~k\mbox{\sf -vertex connectivity}}
\newcommand{\subkvc}{\mbox{\sf subset}~k\mbox{\sf -vertex connectivity}}
\newcommand{\oneec}{1\mbox{\sf -edge connectivity}}
\newcommand{\onevc}{1\mbox{\sf -vertex connectivity}}
\newcommand{\kvcssp}{k\mbox{\sf -vertex-connected spanning subgraph problem}}

\newcommand{\spn}{\mathsf{span}}
\newcommand{\core}{\mathsf{core}}

\newcommand{\optmp}[1]{\mathsf{OPT}_{\mathsf{MP}}(#1)}
\newcommand{\optcro}[1]{\mathsf{OPT}_{\mathsf{cr}}(#1)}
\newcommand{\optcrosq}[1]{\mathsf{OPT}^{2}_{\mathsf{cr}}(#1)}

\newcommand{\pcro}{\mathsf{pcr}}

\newcommand{\cro}{\mathsf{cr}}
\newcommand{\irreg}{\mathsf{IRG}}
\newcommand{\dmax}{d_{\mbox{\textup{\footnotesize{max}}}}}

\newcommand{\G}{{\mathbf{G}}}
\renewcommand{\H}{{\mathbf{H}}}
\newcommand{\bphi}{{\boldsymbol{\varphi}}}
\newcommand{\bpsi}{{\boldsymbol{\psi}}}
\newcommand{\ndew}{\mathsf{ndew}}


\newcommand{\mynote}[1]{{\sc\bf{[#1]}}}
\newcommand{\fillin}{{\sc \bf{[Fill in]}}}
\renewcommand{\check}{{\sc \bf{[Check!]}}}
\newcommand{\explain}{{\sc \bf{[Explain]}}}


\iffull
\pagestyle{plain}
\fi

\begin{document}

\title{\Large On Graph Crossing Number and Edge Planarization\iffull\footnote{Extended abstract to appear in SODA 2011}\fi}
\author{Julia Chuzhoy\thanks{Toyota Technological Institute, Chicago, IL
60637. Email: {\tt cjulia@ttic.edu}. Supported in part by NSF CAREER award CCF-0844872.}
 \and Yury Makarychev\thanks{Toyota Technological Institute, Chicago, IL
60637. Email: {\tt yury@ttic.edu}}\and Anastasios Sidiropoulos\thanks{Toyota Technological Institute, Chicago, IL
60637. Email: {\tt tasos@ttic.edu}}}
\date{}
\maketitle

\begin{abstract}
\ifabstract
\small\baselineskip=9pt\fi
Given an -vertex graph , a \emph{drawing} of  in the plane is a mapping of its vertices into points of the plane, and its edges into continuous curves, connecting the images of their endpoints. 
A \emph{crossing} in such a drawing is a point where two such curves intersect. In the \MCN problem, the goal is to find a drawing of  with minimum number of crossings. The value of the optimal solution, denoted by , is called the graph's \emph{crossing number}. This is a very basic problem in topological graph theory, that has received a significant amount of attention, but is still poorly understood algorithmically. The best currently known efficient algorithm produces drawings with   crossings on bounded-degree graphs, while only a constant factor hardness of approximation is known. A closely related problem is \MP, in which the goal is to remove a minimum-cardinality subset of edges from , such that the remaining graph is planar.

\ifabstract
\small\baselineskip=9pt\fi
Our main technical result establishes the following connection between the two problems: if we are given a solution of cost  to the \MP problem on graph , then we can efficiently find a drawing of  with at most  crossings, where  is the maximum degree in . This result implies an -approximation for \MCN, as well as improved algorithms for
special cases of the problem, such as, for example, -apex and bounded-genus graphs.
\end{abstract}

\section{Introduction}

A \emph{drawing} of a graph  in the plane is a mapping, in which every vertex is mapped into a point of the plane, and every edge into a continuous curve connecting the images of its endpoints.
We assume that no three curves meet at the same point (except at their endpoints), and that no curve contains an image of any vertex other than its endpoints.
A \emph{crossing} in such a drawing is a point where the drawings of two edges intersect, and the \emph{crossing number} of a graph , denoted by , is the smallest integer , such that  admits a drawing with  crossings.
In the \MCN problem, given an -vertex graph ,  the goal is to find a drawing of  in the plane that minimizes the number of crossings.
A closely related problem is \MP, in which the goal is to find a minimum-cardinality subset  of edges, such that the graph  is planar. The optimal solution cost of the \MP problem on graph  is denoted by , and it is easy to see that .


The problem of computing the crossing number of a graph was first considered by Tur\'{a}n \cite{turan_first}, who posed the question of estimating the crossing number of the complete bipartite graph.
Since then, the problem has been a subject of intensive study.
We refer the interested reader to the expositions by Richter and Salazar \cite{richter_survey}, Pach and T\'{o}th \cite{pach_survey}, and Matou\v{s}ek \cite{matousek_book}, and the extensive bibliography maintained by Vrt'o \cite{vrto_biblio}. Despite the enormous interest in the problem, and several breakthroughs over the last four decades, there is still very little understanding of even some of the most basic questions. For example, to the time of this writing, the crossing number of  remains unknown.


Perhaps even more surprisingly, the \MCN problem remains poorly understood algorithmically.
In their seminal paper, Leighton and Rao \cite{LR}, combining their algorithm for balanced separators with the framework of Bhatt and Leighton \cite{bhatt84}, gave the first non-trivial algorithm for the problem. Their algorithm computes a drawing with at most  crossings, when the degree of the input graph is bounded.
This algorithm was later improved to  by Even et al.~\cite{EvenGS02}, and the new approximation algorithm for the Balanced Cut problem by Arora, Rao and Vazirani~\cite{ARV} improves it further to , thus implying an -approximation for \MCN on bounded-degree graphs. Their result can also be shown to give an -approximation for general graphs with maximum degree . We remark that in the worst case, the crossing number of a graph can be as large as , e.g.~for the complete graph.



\iffalse
Moreover, Ajtai et al.~\cite{ajtai82}, and independently Leighton \cite{leighton_book}, settling a conjecture of Erd\"{o}s and Guy \cite{erdos_guy73}, proved that every graph with  edges has crossing number , where  is the number of graph's edges. Therefore, graphs whose average degree is higher than 8 have crossing number .
\fi

On the negative side, computing the crossing number of a graph was shown to be NP-complete by Garey and Johnson \cite{crossing_np_complete}, and it remains NP-complete even on cubic graphs~\cite{Hlineny06a}.
Combining the reduction of \cite{crossing_np_complete} with the inapproximability result for Minimum Linear Arrangement~\cite{Ambuhl07}, we get that there is no PTAS for the \MCN problem unless problems in NP have randomized subexponential time algorithms.
Interestingly, even for the very restricted special case, where there is an edge  in , such that  is planar, the \MCN problem still remains NP-hard~\cite{cabello_edge}. However, an -approximation algorithm is known for this special case, where  is the maximum degree in ~\cite{HlinenyS06}.
Therefore, while the current techniques cannot exclude the existence of a constant factor approximation for \MCN, the state of the art gives just an -approximation algorithm. 
\iffalse
While the approximability of the problem is still widely open in general, arguably the most interesting regime is when the crossing number of the graph is small. The following is therefore a central open problem in topological graph theory:
\begin{quote}
\emph{
What is the best possible efficient algorithm for the \MCN problem, when the optimum is small?
}
\end{quote}

This problem has been explicitly stated by Sz\'{e}kely \cite{szekely_survey}, and emphasized by Kolman and Matou\v{s}ek \cite{KolmanM04}.
\fi

In this paper, we provide new technical tools that we hope will lead to a better understanding of the \MCN problem. We also obtain improved approximation algorithms for special cases where the optimal solution for the \MP problem is small or can be approximated efficiently.


\iffalse
Our main technical result establishes the following connection between \MCN and \MP: Suppose we are given an -vertex graph  with maximum degree , and a subset  of edges, where , such that  is planar. Then we can find a drawing of  with at most  crossings. This result immediately gives improved algorithms for the \MCN problem:  using the planar separator theorem, it is easy to show that any graph  has such a subset  of size , and this set can be found efficiently. This gives an algorithm for drawing graph  with  crossings, and implies an  -approximation for general graphs, an -approximation for bounded degree graphs. The above framework also allows us to obtain simple approximation algorithms for several special families of graphs that have been studied before, such as apex graphs, bounded genus graphs, and so on. We state the results formally below.
\fi

\subsection{Our Results}
Our main technical result establishes the following connection between the \MCN and the \MP problems:

\begin{theorem}\label{thm:main}
Let  be any -vertex graph with maximum degree , and suppose we are given a subset  of edges, , such that  is planar. Then we can efficiently find a drawing of  with at most  crossings.
\end{theorem}
\begin{remark}
Note that there always exists a subset  of edges of size , such that  is planar.
However, in Theorem~\ref{thm:main}, we do not assume that  is the optimal solution to the \MP problem on , and we allow
 to be greater than .
\end{remark}


A direct consequence of Theorem~\ref{thm:main} is that an -approximation algorithm for \MP would immediately give an algorithm for drawing any graph  with  crossings. We note that while this connection between \MP and \MCN looks natural, it is possible that in the optimal solution  to the \MCN problem on , the induced drawing of the planar subgraph  is not planar, that is, the edges of  may have to cross each other (see Figure~\ref{fig: example} for an example).

Theorem~\ref{thm:main} immediately implies a slightly improved algorithm for \MCN. In particular, while we are not aware of any approximation algorithms for the \MP problem, the following is an easy consequence of the Planar Separator theorem of Lipton and Tarjan~\cite{planar-separator}:
\begin{theorem}\label{thm:sqrt n}
There is an efficient -approximation algorithm for {\sf Minimum Planarization}.
\end{theorem}
The next corollary then follows from combining Theorems~\ref{thm:main} and \ref{thm:sqrt n}, and using the algorithm of~\cite{EvenGS02}.
\begin{corollary}\label{corollary:result for general graphs}
There is an efficient algorithm, that, given any -vertex graph  with maximum degree , finds a drawing of  with at most  crossings. Moreover,  there is an efficient -approximation algorithm for \MCN.
\end{corollary} 

\begin{figure}[h]
\begin{center}
\scalebox{0.25}{\rotatebox{0}{\includegraphics{example-planarization-crossing-cut.pdf}}} \caption{(a) Graph . Red edges belong to , blue edges to the planar sub-graph . Any drawing of  in which the edges of  do not cross each other has at least  crossings. (b) An optimal drawing of , with  crossings.} \label{fig: example}
\end{center}
\end{figure}


Theorem~\ref{thm:main} also implies improved algorithms for several special cases of the problem, that are discussed below.

\noindent{\bf Nearly-Planar and Apex Graphs.} We say that a graph  is \emph{-nearly planar}, if it can be decomposed into a planar graph , and a collection of at most  additional edges. For the cases where the decomposition is given, or where  is constant, Theorem~\ref{thm:main} immediately gives an efficient -approximation algorithm for \MCN.
It is worth noting that although this graph family might seem restricted, there has been a significant amount of work on the crossing number of -nearly planar graphs.
Cabello and Mohar \cite{cabello_edge} proved that computing the crossing number remains NP-hard even for this special case, while
Hlin\v{e}n\'{y} and Salazar \cite{HlinenyS06} gave an -approximation.
Riskin \cite{riskin_edge} gave a simple efficient procedure for computing the crossing number when the planar sub-graph  is 3-connected, and
Mohar \cite{Mohar_almost} showed that Riskin's technique cannot be extended to arbitrary 3-connected planar graphs.
Gutwenger et al.~\cite{GutwengerMW05} gave a linear-time algorithm for the case where every crossing is required to be between  and an edge of .


A graph  is a -apex graph iff there are  vertices , whose removal makes it planar. Chimani et al.~\cite{crossing_apex} obtained an -approximation for \MCN on -apex graphs. Theorem~\ref{thm:main} immediately implies an -approximation for -apex graphs, where either  is constant, or  the  apices are explicitly given.



\noindent{\bf Bounded Genus Graphs.}
Recall that the genus of a graph  is the minimum integer  such that  can drawn on an orientable surface of genus  with no crossings.


B\"{o}r\"{o}zky et al.~\cite{BorozkyPT06} proved that 
the crossing number of a bounded-degree graph of bounded genus is .
Djidjev and Venkatesan~\cite{genus-planarization} show that  for any genus- graph. Moreover, if the embedding of  into a genus- surface is given, a planarizing set of this size can be found in time .
If no such embedding is given, they show how to efficiently compute a planarizing set of size .

Hlin\v{e}n\'{y} and Chimani \cite{crossing_genus}, building on the work of Gitler et al.~\cite{crossing_projective} and Hlin\v{e}n\'{y} and Salazar~\cite{crossing_torus}  gave an algorithm for approximating \MCN on graphs that can be drawn ``densely enough\footnote{More precisely, the density requirement is that the nonseparating dual edge-width of the drawing is .}'' in an orientable surface of genus ,
with an approximation guarantee of . \iffull
Despite the rather technical conditions on the input, this is the largest family of graphs for which a constant-factor approximation for the crossing number is known.\fi
We prove the following easy consequence of Theorem~\ref{thm:main} and the result of~\cite{crossing_genus}:


\begin{theorem}\label{thm: bounded genus}
Let  be any graph embedded in an orientable surface of genus .
Then we can efficiently find a drawing of  into the plane, with at most  crossings.
Moreover, for any , there is an efficient -approximation for \MCN on bounded degree graphs embedded into a genus- surface.
\end{theorem}

We notice that when  is a constant, a drawing of a genus- graph on a genus- surface can be found in linear time \cite{Mohar99, KawarabayashiMR08}.


\subsection{Our Techniques}
We now provide an informal overview of the proof of Theorem~\ref{thm:main}. We will use the words ``drawing'' and ``embedding'' interchangeably.
We say that a drawing  of the planar graph  is \emph{planar} iff  contains no crossings. Let  be the optimal drawing of , and let  be the induced drawing of .
For simplicity, let us first assume that the graph  is -vertex connected. 
Then we can efficiently find a planar drawing  of , which by Whitney's Theorem~\cite{Whitney} is guaranteed to be unique.
Notice however that the two drawings  and  of  are not necessarily identical, and in particular  may be non-planar. 

We now add the edges  to the drawing  of . The algorithm for adding the edges is very simple. 
For each edge , we  choose the drawing  that minimizes the number of crossings between  and the images of the edges of  in . This task reduces to finding the shortest path in the graph dual to .
We can ensure that the drawings of any pair  of edges in  cross at most once, by performing an un-crossing step, which does not increase the number of other crossings. Let  denote this new drawing of the whole graph. The total number of crossings between pairs of edges that both belong to  is then bounded by , and it only remains to  bound the number of crossings between the edges of  and the edges of . In order to complete the analysis,
it is enough, therefore, to show, that for every edge , there is a drawing of  in ,  
that has at most  crossings with the edges of .
Since our algorithm finds the best possible drawing for each edge , the bound on the total number of crossings will follow.

One of our main ideas is the notion of \emph{routing edges along paths}. Consider the optimal drawing  of , and let  be some
edge in , that is mapped into some curve  in . We show that we can find a path  in the graph , whose endpoints are  and , such that, instead of drawing the edge  along , we can draw it along a different curve , that ``follows'' the drawing of the path . That is, we draw  very close to the drawing of , in parallel to it. Moreover, we show that this re-routing of the edge  along  does not increase the number of crossings in which it participates by much. Consider now the drawing of  in the planar embedding  of the graph . We can again draw the edge  along the embedding of the same path  in . Let  be the resulting curve. Since the embeddings  and  are different, it is possible that  participates in more crossings than . However, we show that the number of such new crossings can be bounded by the number of vertices and edges in , whose local embeddings are different in  and . We then bound this number, in turn, by .

We now explain the notion of local embeddings in more detail. Given two drawings  and  of the graph , we say that a vertex  is \emph{irregular} iff the ordering of its adjacent edges, as their images enter , is different in the two drawings. In other words, the local drawing around the vertex  is different in  and  (see Figure~\ref{fig: irregular vertices}). We say that an edge  is \emph{irregular} iff both of its endpoints are not irregular, but their orientations are different. That is, the orderings of the edges adjacent to each one of the two endpoints are the same in both  and , but say, for vertex , both orderings are clock-wise, while for vertex , one is clock-wise and the other is counter-clock-wise (see Figure~\ref{fig: irregular edges}). In a way, the number of irregular edges and vertices measures the difference between the two drawings. We show that, on the one hand, if  is -vertex connected, and  is a planar embedding of , then the number of irregular vertices and edges is bounded by roughly the number of crossings in , which is in turn bounded by . On the other hand, we show that for each edge , the number of new crossings incurred by the curve  is bounded by the total number of irregular edges and vertices on the path , thus obtaining the desired bound.

Assume now that  is not -vertex connected. In this case, it is easy to see that the number of irregular vertices and edges cannot be bounded by the number of crossings in  anymore. In fact, it is possible that both  and  are planar drawings of , so the number of crossings in  is , while the number of irregular vertices may be large (see Figure~\ref{fig: example2} for an example). However, if the original graph  was -vertex connected, then for any -vertex cut  in , there is an edge  connecting the resulting two components of . We use this fact to find a specific planar drawing  of , that is ``close'' to , in the sense that, if we define the irregular edges and vertices with respect to the embeddings  of , then we can bound their number  by the number of crossings in .

Finally, if  is not -vertex  connected, then we first decompose it into -vertex connected components, and then apply the above algorithm to each one of the components separately. In the end, we put all the resulting drawings together, while only losing a small additional factor in the number of crossings.


\begin{figure}[h]
\centering
\subfigure[Vertex  is irregular.]{
	\scalebox{0.3}{\includegraphics{bad-vertex-cut}}
	\label{fig: irregular vertices}
}
\hspace{2cm}
\subfigure[Edge  is irregular.]{
	\scalebox{0.3}{\includegraphics{bad-edge-cut}}
	\label{fig: irregular edges}
}
\caption{Irregular vertices and edges.}
\end{figure}


\begin{figure}[h]
\begin{center}
\scalebox{0.25}{\rotatebox{0}{\includegraphics{example-planarization-crossing2-cut.pdf}}} \caption{Example of planar drawings  and  of graph . Irregular vertices are shown in red.}\label{fig: example2}
\end{center}
\end{figure}



\subsection{Other Related work}
Although it is impossible to summarize here the vast body of work on \MCN, we give a brief overview of some of the highlights, and related results.

\textbf{Exact algorithms.}
Grohe \cite{Grohe04}, answering a question of Downey and Fellows \cite{fpt}, proved that the crossing number is fixed-parameter tractable.
In particular, for any fixed number of crossings his algorithm computes an optimal drawing in  time.
Building upon the breakthrough result of Mohar \cite{Mohar99} for embedding graphs into a surface of bounded genus, Kawarabayashi and Reed \cite{KawarabayashiR07} gave an improved fixed-parameter algorithm with running time .

\textbf{Bounds on the crossing number of special graphs.}
Ajtai et al.~\cite{ajtai82}, and independently Leighton \cite{leighton_book}, settling a conjecture of Erd\"{o}s and Guy \cite{erdos_guy73}, proved that every graph with  edges has crossing number .
B\"{o}r\"{o}zky et al.~\cite{BorozkyPT06} proved that 
the crossing number of a bounded-degree graph of bounded genus is .
This bound has been extended to all families of bounded-degree graphs that exclude a fixed minor by Wood and Telle \cite{WoodT06}.
Spencer and T\'{o}th \cite{spencer_random} gave bounds on the expected value of the crossing number of a random graph.







\iffalse


\textbf{Approximability.}
In their seminal paper, Leighton and Rao \cite{LR}, combining their separator theorem with the framework of Bhatt and Leighton \cite{bhatt84}, gave the first non-trivial approximation algorithm for crossing number.
More precisely, for a given bounded-degree graph  their algorithm computes a drawing with at most  crossings.
This was later improved to  by Even et al.~\cite{EvenGS02}.
Note that in general, these algorithms imply only a -approximation.

Computing the crossing number of a graph was shown to be NP-complete by Garey and Johnson \cite{crossing_np_complete}.
Hlin\v{e}n\'{y} \cite{Hlineny06a} showed that the problem remains NP-complete even on cubic graphs.
Combining the reduction from \cite{crossing_np_complete} with the inapproximability result for Minimum-Linear-Arrangement of Ambuhl et al.~\cite{Ambuhl07}, it follows that there is no PTAS for crossing number unless NP-hard problems can be solved in randomized subexponential time.
Using a reduction from Min-Uncut, and assuming the Unique Games Conjecture, one can obtain that the crossing number is hard to approximate within any constant factor.


\textbf{Nearly-planar graphs.}
There has been a significant amount of work on the crossing number of -nearly planar graphs, i.e.~graphs that can be decomposed into , where  is a planar graph, and  is just one extra edge.
Cabello and Mohar \cite{cabello_edge} proved that computing the crossing number remains NP-hard even on this apparently restricted class of graphs.
Hlin\v{e}n\'{y} and Salazar \cite{HlinenyS06} gave a constant-factor approximation when  has bounded-degree.
Riskin \cite{riskin_edge} gave a simple efficient procedure for computing the crossing number when the planar piece  is 3-connected.
Mohar \cite{Mohar_almost} showed that Riskin's technique cannot be extended to arbitrary 3-connected planar graphs.
Gutwenger et al.~\cite{GutwengerMW05} gave a linear-time algorithm for the case where every crossing is required to be between  and an edge of .




Gitler et al.~\cite{crossing_projective} gave a -approximation for projective graphs of bounded degree, and Hlin\v{e}n\'{y} and Salazar \cite{crossing_torus} a
-approximation for toroidal graphs of bounded degree.

This later algorithm has been extended by Hlin\v{e}n\'{y} and Chimani \cite{crossing_genus} to graphs that can be drawn ``densely enough\footnote{More precisely, the density requirement is that the nonseparating dual edge-width of the drawing is .}'' in an orientable surface of genus ,
with an approximation guarantee of .
Despite the rather technical conditions on the input, this is the largest family of graphs for which a constant-factor approximation for the crossing number is known.
\fi

\noindent{\bf Organization} Most of this paper is dedicated to proving Theorem~\ref{thm:main}. We start in Section~\ref{sec:prelims} with preliminaries, where we introduce some notation and basic tools. We then prove Theorem~\ref{thm:main} in Section~\ref{sec:alg}. We prove Theorem~\ref{thm:sqrt n} and Corollary~\ref{corollary:result for general graphs} in Section~\ref{sec:planarization for general graph}. 
\ifabstract
The proof of Theorem~\ref{thm: bounded genus} appears in the full version of the paper, available from the authors' web pages.
\fi
\iffull 
The proof of Theorem~\ref{thm: bounded genus} appears in Section~\ref{sec:genus} of the Appendix.
\fi

\section{Preliminaries}\label{sec:prelims}
In this section we provide some basic definitions and tools used in the proof of Theorem~\ref{thm:main}. In order to avoid confusion, throughout the paper, we denote the input graph by , with , and maximum degree . We also denote , the planar sub-graph of  (where  is the set of edges from the statement of Theorem~\ref{thm:main}), and by  the optimal drawing of  with  crossings. When stating definitions or results for general arbitrary graphs, we will be denoting them by  and , to distinguish them from the specific graphs  and . 

We use the words ``drawing'' and ``embedding'' interchangeably. 
Given any graph , a drawing  of , and any sub-graph  of , we denote by  the drawing of  induced by , and by  the number of crossings in the drawing  of . For any pair  of subsets of edges, we denote by  the number of crossings in  in which images of edges of  and edges of  intersect, and by  the number of crossings in  between pairs of edges that both belong to . Finally, for any curve , we denote by  the number of crossings between  and the images of the edges of , and  denotes . We will omit the subscript  when clear from context.
If  is a planar graph, and  is a drawing of  that contains no crossings, then we say that  is a \emph{planar} drawing of .


For the sake of brevity, we write  to denote that a path  connects vertices  and . Similarly, if we have a drawing of a graph, we write  to denote that a curve  connects the images of vertices  and  (curve  may not be a part of the current drawing). 
In order to avoid confusion, when a curve  is a part of a drawing  of some graph , we write . We denote by  the set of all curves that can be added to the drawing  of . In other words, these are all curves that do not contain images of vertices of  (except as their endpoints), and do not contain any crossing points of .
Finally, for a graph , and subsets ,  of its vertices and edges respectively, we denote by ,  the sub-graphs of  induced by , and , respectively.

\begin{Definition} For any graph , a subset  of vertices is called a -separator, iff , and the graph  is not connected. We say that  is -connected iff it does not contain any -separators, for any .
 \end{Definition}
 
We will be using the following two well-known results:

\begin{theorem} (Whitney~\cite{Whitney})\label{thm:Whitney} Every 3-connected planar graph has a unique planar embedding.
\end{theorem}

\begin{theorem}(Hopcroft-Tarjan~\cite{planar-drawing})\label{thm:planar drawing}
For any graph , there is an efficient algorithm to determine whether  is planar, and if so, to find a planar drawing of .
\end{theorem}

\paragraph{Irregular Vertices and Edges}
Given any pair  of drawings of a graph , we measure the distance between them in terms of \textit{irregular edges} and \textit{irregular vertices}:

\begin{Definition}
We say that a vertex  of  is irregular iff its degree is greater than , 
and the circular ordering of the edges incident on it, as their images enter , is different in  and  (ignoring the orientation). Otherwise we say that  is regular.
We denote the set of irregular vertices by . (See Figure~\ref{fig: irregular vertices}).
\end{Definition}

\begin{Definition}
For any pair  of vertices in ,
we say that a path  in  is irregular iff  and  have degree at least , all other
vertices on  have degree  in , vertices  and  are regular, but their
orientations differ in  and . That is, the orderings of the edges adjacent to  and to  are identical in both drawings, but the pairwise orientations are different: for one of the two vertices, the orientations are identical in both drawings (say clock-wise), while for the other vertex, the orientations are opposite (one is clock-wise, and the other is counter-clock-wise). An edge  is an irregular edge iff it is the first or the last edge on an irregular path. In particular, if the irregular path only consists of edge , then  is an irregular edge (see Figure~\ref{fig: irregular edges}). If an edge is not irregular, then we say that it is regular.
We denote the set of irregular edges by .
\end{Definition}










\paragraph{Routing along Paths.} One of the central concepts in our proof is  that of routing along paths. Let  be any graph, and  any drawing of . Let  be any edge of , and let  be any path connecting  to  in . It is possible that the image of  crosses itself in . We will first define a very thin strip  around the image of  in . We then say that the edge  is routed along the path , iff its drawing follows the drawing of the path  inside the strip , possibly crossing .

In order to formally define the strip , we first consider the graph , obtained from , by replacing every edge of  with a -path containing  inner vertices. The drawing  of  then induces a drawing  of , such that, if  is the path corresponding to   in , then every edge of  crosses the image of  at most once; every edge of  has at most one endpoint that belongs to ; and if an image of  crosses , then no endpoint of  belongs to . Let  denote the subset of edges of  whose images cross the image of , let  denote the subset of edges of  whose images cross the images of other edges in , and let  denote the set of edges in  that have exactly one endpoint belonging to .

We now define a thin strip  around the drawing of path  in , by adding two curves,  and , immediately to the left and to the right of the image of  respectively, that follow the drawing of . Each edge in  is crossed exactly once by , and once by . Each edge in  is crossed exactly once by either  or . For each pair  of edges in  whose images cross,  and  will both cross each one of the edges  and  exactly once. Curves  and  do not have any other crossings with the edges of . The region of the plane between the drawings of  and , which contains the drawing of , defines the strip . We let  denote the same strip, only when added to the drawing  of . Let  and  denote the two curves that form the boundary of , and let . Then the crossings between  and the edges of  can be partitioned into four sets,  (see Figure~\ref{crossings-c1-c2-c3}), where: (1) There is a  mapping between  and the crossings between the edges of  and the edges of ; (2) For each edge  that has exactly one endpoint in , there is at most one crossing between  and  in , and there are no other crossings in ; (3) For each edge  that has exactly two endpoints in , there are at most two crossings of  and  in , and there are no other crossings in ; and (4) for each crossing between a pair  of edges, there is one crossing between  and , and one crossing between  and . Additionally, if  crosses itself  times, then  also crosses itself  times.

\begin{Definition}
We say that the edge  is \emph{routed along the path }, 
iff its drawing follows the drawing of path  inside the strip , in parallel to the drawing of , 
except that it is allowed to cross the path . 
\end{Definition}


\begin{figure}
\begin{center}
\scalebox{0.8}{\includegraphics{Routing-C1-C2-C3-C4-b.pdf}}
\caption{Strip  and the four types of crossing between  and edges of . Crossings in , ,  and  are labeled with ``1'', ``2'', ``3'' and ``4'' respectively. Path  is shown in solid line, dotted lines correspond to other edges of .}
\label{crossings-c1-c2-c3}
\end{center}
\end{figure}




\section{Proof of Theorem~\ref{thm:main}}\label{sec:alg}
The proof consists of two steps. We first assume that the input graph  is -vertex connected, and prove a slightly stronger version of Theorem~\ref{thm:main} for this case. Next, we show how to reduce the problem on general graphs to the -vertex connected case, while only losing a small additional factor in the number of crossings.

\subsection{Handling -connected Graphs}
In this section we assume that the input graph  is -vertex connected, and we prove a slightly stronger version of Theorem~\ref{thm:main} for this special case, that is summarized below. 

\begin{theorem}\label{thm:main2}
Let  and  be as in Theorem~\ref{thm:main}, and assume that  is -connected and has no parallel edges.
 Then we can efficiently find a drawing of  with at most  crossings.
\end{theorem}



Notice that we can assume w.l.o.g. that graph  is connected. Otherwise, we can choose an edge  whose endpoints belong to two distinct connected components of , remove  from  and add it to . It is easy to see that this operation preserves the planarity of , and we can repeat it until  becomes connected. We therefore assume from now on that  is connected.

Recall that  denotes the optimal drawing of , and  is the drawing of  induced by .
Since the graph  is planar, we can efficiently find its planar drawing, using Theorem~\ref{thm:planar drawing}. However, since  is not necessarily -connected, there could be a number of such drawings, and we need to find one that is ``close'' to . 
We use the following theorem, whose proof appears in Appendix.
\begin{theorem}\label{thm: good planar drawing}
We can efficiently find a planar drawing  of , such that

\end{theorem}

We are now ready to describe the algorithm for finding a drawing of . We start with the planar embedding  of , guaranteed by Theorem~\ref{thm: good planar drawing}.
For every edge , we add an embedding of  to the drawing  of , via a curve , , that crosses the minimum possible number of edges of . Such a curve can be computed as follows.
Let  be the dual graph of the drawing  of . 
Every curve  , ,  
defines a path in .
The length of the path, measured in the number of edges of  it contains, is exactly the number of edges of  that  crosses.
Similarly, every path in  corresponds to a curve in .
Let  be the set of all faces of  (equivalently, vertices of ) whose boundaries contain , and
let  be the set of all faces whose boundaries contain . 
We find the shortest path  between sets  and  in ,
and the corresponding curve  in .
Clearly, the number of crossings between  and the edges of  is 
minimal among all curves connecting  and  in . 
By slightly perturbing the lengths of edges in , we may
assume that for every pair of vertices in , there is exactly one shortest path connecting them. In particular, any pair of such shortest paths may share at most one consecutive segment.
Consequently, for any pair  of edges, the drawings  that we have obtained cross at most once.

Let  denote the union of  with the drawings  of edges  that we have computed. It now only remains to bound the number of crossings in . Clearly, . In order to bound , we use the following theorem, whose proof appears in the next section.

 
\begin{theorem}\label{thm:main-along-path}
Let  and  be two drawings of any planar connected graph , whose maximum degree is , where  is a planar drawing.
Then for every curve , 
there is a  curve , ,
that participates in at most  crossings.
\end{theorem}

In other words, the number of additional crossings incurred by  is roughly bounded by the total number of crossings in , and the difference between the two drawings, that is, the number of irregular vertices and edges.

Since the optimal embedding  of  contains an embedding of every edge , Theorem~\ref{thm:main-along-path} guarantees that for every edge , there is a curve  in , that participates in at most  crossings. Combining this with Theorem~\ref{thm: good planar drawing}, the number of crossings between  and  is bounded by . 
Since for each edge , our algorithm chooses the optimal curve , we are guaranteed that  participates in at most  crossings with edges of .
Summing up over all edges , we obtain that , as required. In order to complete the proof of Theorem~\ref{thm:main2}, it now only remains to prove Theorem~\ref{thm:main-along-path}.


\subsection{Proof of Theorem~\ref{thm:main-along-path}: Routing along Paths}\label{sec:routing along paths}
The proof consists of two steps. In the first step, we focus on the drawing  of , and we show that for any curve  in , there is a path  in , and another curve  in  routed along  in , such that the number of crossings in which  is involved is small. In the second step, we consider the planar drawing  of , and show how to route a curve  along the same path  in , so that the number of crossings is suitably bounded. The next proposition handles the first step of the proof.

\begin{proposition}\label{thm:rerouting-along-path} 
Let   be any curve in , where  is a drawing of . Then there is a path  in , and a curve  in  routed along , such that 
 
Moreover,  does not cross the images of the edges of . Path  is not necessarily simple, but an edge may appear at most twice on .
\end{proposition}
\begin{proof}
Consider the drawing  of , together with the curve .
Let  be the subset of edges whose images cross the images of other edges of , and
let  be the subset of edges whose images cross  and that are not in . 
Let . Note that  is a planar drawing of , and  does not cross any edges of . Therefore, vertices  and  lie on the boundary of one face, denoted by , of . Without loss of generality, we may assume that 
 is the outer face of . The boundary of 
consists of one or several connected components. 
Let  be the boundary walks of the face  (where  is the number of connected components):
each  is the (not necessarily simple) cycle obtained by walking around the boundary of the th connected component,
if the component contains at least 2 vertices; and it is a single vertex otherwise.


Consider two cases. First, assume that  and  are connected in , and so they both belong to the same component .
We then let  be one of the two segments of  that connect  and . Notice that while  is not necessarily simple, each edge appears at most twice on it. We let  be a curve drawn along the path  inside the face . 
Notice that the only edges that  crosses in the drawing  of , are the edges of  that have at least one endpoint on . Each such edge is crossed at most twice by  (once for each endpoint that belongs to ). Therefore, 
Assume now that  and  are not connected in , and assume w.l.o.g. that  and .
Let  be a minimal set of edges of , such that  and  are connected in .
Each edge  connects two distinct components  and  (as otherwise we could remove
 without affecting the connectivity of ). In particular, the drawings of all edges of  in  lie inside 
the face . 
Consider the following graph : each vertex of  corresponds to a component , for , and the edges of  are the edges of  connecting these components. Since  and  are connected in , vertices representing  and  belong to the same connected component  of . Moreover, because of the minimality of , this connected component is a simple path  connecting the vertices representing  and  in .


\begin{figure}
\begin{center}
\scalebox{0.45}{\includegraphics{edge-routing.pdf}}
\caption{Routing the curve .\label{fig:curve}}
\end{center}
\end{figure}

Denote the edges of this path by ;
denote its vertices by ; each edge , for , corresponds to an edge , that connects a pair  of vertices, where ,  , in . We denote . Notice that since  is simple, each component ,  appears at most once on the path.
We now define the path  and the curve . The path  is defined as follows: , where for each ,  is obtained by traversing the boundary  in the clock-wise direction from  to . The curve  is simply routed along  on the inside of the face . That is, curve  never crosses the images of the edges of  (see Figure~\ref{fig:curve}).

We now bound the number of edges of , whose images in  are being crossed by . We partition the crossings in which  participates into four sets , like in the definition of routing along paths in Section~\ref{sec:prelims}. The only edges incident on vertices of path  that  crosses are the edges in , and each such edge contributes at most two crossings to ,  while the number of crossings in  is bounded by . 
\iffalse
There are two possible types of crossings:
\begin{enumerate}
\item When  switches from one edge to another at vertex , it can cross an edge from
 incident to . Since every edge has two endpoints, the total number of such crossings is at most
. Note that  and .
Therefore, the number of crossings of the first type is .

\item When  goes along an edge  it crosses another edge  if and only if  crosses .
Note that each crossing in  involves two edges and 
 visits each of these edges at most twice (at most once when  goes on one
side of the edge and at most once when  goes on the other side). 
Therefore, there are at most  crossings of the second type.  
\end{enumerate}
\fi
Therefore, .
\ifabstract \qed \fi \end{proof}

We now focus on the other embedding,  of , and show how to obtain the final curve , , that participates in a small number of crossings.

\begin{proposition}\label{thm:embedding-along-path}
There is a curve  in , that has no self-crossings, and participates in at most  crossings with the edges of .
\end{proposition}
\begin{proof}
We will route  along the path  in . Since an edge
may appear at most twice on , path  may visit a vertex at most  times. We will assume however that  visits every irregular vertex
at most once, by changing  as follows: whenever an irregular vertex  appears more than once on , we create a shortcut, by removing the segment of  that lies between the two consecutive appearances of  on .  As a result, in the final path , each edge appears at most twice, and each irregular vertex at most once.

We will route the curve  along , but we will allow it to cross the image of the path . Therefore, we only need to specify, for each edge , whether  crosses it, and if not, on which side of  it is routed. Since  is planar, the edges of  do not cross each other.

We partition the path  into consecutive segments , where for each ,  contains regular edges only, and all its vertices are regular, except perhaps the first and the last. For each , either  contains one or several consecutive  irregular edges connecting the last vertex of  and the first vertex of ; or it contains a single irregular vertex, which serves as the last vertex of  and the first vertex of . 

Consider some such segment , and a thin strip  around this segment. Then the parts of the drawings of the edges incident on the vertices of , that fall inside  are identical in both  and  (except possibly for the edges incident on the first and the last vertex of ). We can therefore route  along the same side of  along which  is routed. If necessary, we may need to cross the path  once for each consecutive pair of segments, if the routings are performed on different sides of . Let  and  denote the segments of  and , respectively, that are routed along , and include crossings with all edges incident on .
It is easy to see that the difference  is bounded by : we pay at most  for crossing the edges incident on each endpoint of , which may be an irregular vertex. We may additionally pay  crossing for each irregular edge on . Since each irregular vertex appears at most once on , and each irregular edge at most twice, . Finally, if  crosses itself, we can simply short-cut it by removing all resulting loops. 
\ifabstract \qed \fi \end{proof}

\iffalse 
Consider some regular vertex  of degree at least 3. The circular order  of edges incident to 
is the same in  and , and it defines a correspondence between sides of edges incident to .
If an edge  is regular, then the orders  and  define the same correspondence 
of sides. 
That allows us to define a universal correspondence between sides of regular edges in embeddings  and 
(if two vertices of degree at least 3 are connected by an irregular path we use the correspondence
defined by one of the endpoints of the path for all regular edges on the path).

We route  along the same side of every regular edge 
along which  goes in the embedding . When we switch
from one edge to another at a regular vertex , we route 
so that it does not cross any edges incident to  except those that  crosses in .
We route  along an irregular edge  so that  and  go 
along the same side of  w.r.t.~ near the vertex , they go
along the same side of  w.r.t.~ near the vertex . We allow 
to cross cross  once. We also allow  to cross edges incident on a vertex 
when we route  near an irregular vertex .



The crossings of curve  in  can be categorized as follows:
\begin{itemize}
\item If  is an edge that is incident to some vertex , and  crosses  in , 
then  may cross  in  as well (those are ``old'' crossings).
\item If  is an irregular edge, then this can add one crossing between  and . 
\item If  is an irregular vertex, then this can add at most  crossings between edges incident on  and .
\end{itemize}
Therefore, the total number of crossings is at most . 

If the curve  that we constructed has self crossings, we eliminate them by making shortcuts.
This clearly can only decrease the number of crossings with edges of .
\fi


Combining Propositions~\ref{thm:rerouting-along-path} and~\ref{thm:embedding-along-path}, we get that , and .


\subsection{Non 3-Connected Graphs}
\vspace{-3mm}

We briefly explain how to reduce the general case to the -connected case. We decompose the graph into a collection of sub-graphs. For each sub-graph, we find a drawing separately, and then combine them together to obtain the final solution. Each one of the sub-graphs is either a -connected graph, for which we can find a drawing using Theorem~\ref{thm:main2}, or it can be decomposed into a planar graph plus one additional edge. In the latter case, we employ the algorithm of Hlineny and Salazar~\cite{HlinenyS06} to find an -approximate drawing. 
\ifabstract
The detailed proof of this part appears in the full version of this paper.
\fi\iffull
The detailed proof of this part is presented in Section~\ref{sec:reduce-three-connected} in the Appendix.
\fi

\section{Improved Algorithm for General Graphs}\label{sec:planarization for general graph}
In this section we prove Theorem~\ref{thm:sqrt n} and Corollary~\ref{corollary:result for general graphs}.
We will rely on the Planar Separator Theorem of Lipton and Tarjan~\cite{planar-separator}, and on the approximation algorithm for the Balanced Cut problem of Arora, Rao and Vazirani~\cite{ARV}, that we state below.
\begin{theorem}[Planar Separator Theorem~\cite{planar-separator}]\label{thm: planar separator}
Let  be any -vertex planar graph. Then there is an efficient algorithm to partition the vertices of  into three sets , such that , , and there are no edges in  connecting the vertices of  to the vertices of .
\end{theorem}
\begin{theorem}[Balanced Cut~\cite{ARV}]\label{thm: ARV}
Let  be any -vertex graph, and suppose there is a partition of vertices of  into two sets,  and , with , and 
. Then there is an efficient algorithm to find a partition  of vertices of , such that  for some constant , and .
\end{theorem}
Combining the two theorems together, we get the following corollary\ifabstract, whose proof appears in the full version of the paper.\fi\iffull:\fi
\begin{corollary}\label{corollary: cut}
Let  be any -vertex graph with maximum degree . Then there is an efficient algorithm to partition the vertices of  into two sets , with  for some constant , such that .
\end{corollary}
\iffull
\begin{proof}
Let  be an optimal solution for the \MP problem on , , and let . Since  is a planar graph, there is a partition  of its vertices as in Theorem~\ref{thm: planar separator}. Assume w.l.o.g. that , and consider the partition . Then , and . We can now apply Theorem~\ref{thm: ARV} to obtain the desired partition of .
\ifabstract \qed \fi \end{proof}
\fi

We are now ready to describe the algorithm from Theorem~\ref{thm:sqrt n}. The algorithm consists of  iterations, and in each iteration , we are given a collection  of disjoint sub-graphs of , with . The number of vertices in each such sub-graph is bounded by , where  is the constant from Corollary~\ref{corollary: cut}. In the input to the first iteration, , and .
Iteration , for  is performed as follows. Consider some graph , for . We apply Corollary~\ref{corollary: cut} to this graph, and denote by  the two sub-graphs of  induced by  and , respectively. The number of vertices in each one of the subgraphs is at most . We denote by  the corresponding set of edges , and let . Since for all , , and , we get that , as . Finally, consider the collection  of the new graphs, and let  contain the non-planar graphs. Then , and the graphs in  become the input to the next iteration. 
Since we can efficiently check whether a graph is planar, the set  can be computed efficiently.

The algorithm stops, when all remaining sub-graphs contain at most  edges. We then add the edges of all remaining sub-graphs to set , where  is the last iteration. Our final solution is , and its cost is bounded by , since the values  form a decreasing geometric series for . This finishes the proof of Theorem~\ref{thm:sqrt n}.
We now show how to obtain Corollary~\ref{corollary:result for general graphs}. Combining Theorems~\ref{thm:main} and \ref{thm:sqrt n}, we immediately obtain an efficient algorithm for drawing any graph  with at most  crossings. In order to get the approximation guarantee of , we use an extension of the result of Even et al.~\cite{EvenGS02} to arbitrary graphs, that we formulate in the next theorem, \iffull whose proof appears in Appendix.\fi \ifabstract whose proof appears in the full version of the paper.\fi

\begin{theorem}[Extension of~\cite{EvenGS02}]\label{thm: Even: extension}
There is an efficient algorithm that, given any -vertex graph  with maximum degree , outputs a drawing of  with  crossings.
\end{theorem}

We run our algorithm, and the algorithm given by Theorem~\ref{thm: Even: extension} on the input graph , and output the better of the two solutions. If , then the algorithm of Even et al. is an -approximation. Otherwise, our algorithm gives an -approximation.







\bibliography{soda}
\bibliographystyle{siam}

\iffull
\newpage
\fi

\appendix
\section{Block Decompositions}\label{sec: blocks}
In this section we introduce the notion of \textit{blocks}, and present a theorem for computing block decompositions of graphs, that we will
later use to handle graphs that are not -connected. 


\begin{Definition}
Let  be a -connected graph. 
A subgraph  of  is called a \emph{block} iff: 

\begin{itemize}
\item  and ;
\item There are two special vertices , called \emph{block end-points} and denoted by , such that there are no edges connecting vertices in  to  in , that is, . All other vertices of  are called \emph{inner vertices}; 

\item   is the subgraph of  induced by , except that it {\bf does not} contain the edge  even if it is present in .
\end{itemize}
\end{Definition}

Notice that every -separator  of  defines at least two internally disjoint blocks  with .

\begin{Definition}
Let  be a laminar family of sub-graphs of , and let  be the decomposition tree associated with . We say that  is a \emph{block decomposition} of , iff:

\begin{itemize}
\item The root of the tree  is , and all other vertices of  are blocks. For consistency, we will call the root vertex ``block'' as well.

\item For each block , let  be the graph obtained by replacing each child  of 
with an artificial edge connecting its endpoints. Let  be the graph obtained from  by adding an artificial edge connecting the endpoints of  (for the root vertex , ). Then  is -connected.\item If a block  has exactly one child  then .
\end{itemize} 
\end{Definition}



\iffull
The next theorem states that we can always find a good block decomposition for any 2-connected graph. 
\fi
\ifabstract{The proof of the next theorem appears in the full version of the paper.}\fi

\begin{theorem}\label{thm: block decomposition}
Given a -connected graph  with , we can efficiently find a laminar block decomposition  of , such that for
every vertex  that participates in any -separator  of , one of the following holds:
\iffull
\begin{itemize}
\item\fi Either  is an endpoint of a block ;
\iffull
\item\fi or   has exactly two neighbors in , and there is an edge , such that  is an endpoint of a block .
\iffull
\end{itemize}
\fi
\end{theorem}




\iffull



We will use the notion of \textit{SPQR-trees} in the proof. 
Recall that an SQPR tree  for the graph  defines a recursive decomposition
of , as follows. Each node  of the tree  is associated with a graph , where , and  consists of edges of  (called  \textit{actual} edges), and some additional edges, called \textit{artificial} or \textit{virtual} edges.
Graph  is allowed to contain parallel edges. Additionally, we are given a bijection  between the artificial edges of , and the edges adjacent to the vertex  in the tree .

Each actual edge of graph  belongs to exactly one of the graphs , for . The edges of the tree , together with the artificial edges of the graphs , show how to compose the graphs  together to obtain the original graph . More specifically, if  is an edge in the tree , then there is a unique artificial edge  in  associated with it, and a unique artificial edge  in  associated with it (that is, ). The endpoints of both these artificial edges are copies of the same two vertices, that is, , for . 
The graphs  and  share no vertices other than  and .
The removal of the edge  from  decomposes the tree into two connected components,  and . Let  denote the set of all vertices appearing in the graphs , for , and similarly, let  denote the vertices appearing in the graphs  where . Then , and . 

The vertices of the tree  belong to one of the four types: , , , and .
\begin{itemize}
\item If  is an \emph{-node}, then the graph  is a cycle, with no parallel edges.
\item If  is a \emph{-node}, then  consists of a pair of vertices connected by at least 
three parallel edges, and at most one of these edges is an actual edge.
\item If  is an \emph{-node}, then  is a 3-connected graph, with more than 3 vertices and
no parallel edges.
\item If  is a \emph{-node}, then  is just a single edge. The tree  has
-nodes only if  itself is an edge, and then  has no other nodes.
\end{itemize}
No two -nodes and no two -nodes are adjacent in .

For every 2-connected graph , there is a unique SPQR tree. Moreover,
this tree can be found in linear time, as was shown by Gutwenger and Mutzel~\cite{SPQR-trees}.  
The SPQR tree of  describes the set of all 2-separators of , as follows: a pair of 
vertices  and  of  is a 2-separator if and only if either (1) there is an artificial edge  in some graph , for , or (2)  and  are non-adjacent vertices in some graph 
, where  is an -node.


We now describe how to construct the laminar block decomposition , and the associated decomposition tree , for . Since  and  is -connected, the tree  does not contain any -nodes.
We assume first that  is not a cycle, and we treat the case of the cycle graph 
separately. We start by computing the SPQR tree  of . We then choose an arbitrary 
 or -node  of  to serve as the root of the tree . For each node , we denote the subtree rooted at
 by . If  is the parent of  in the tree , we call the artificial edge 
of  of  that is mapped to  a \emph{child edge};
we call the edge of  that is mapped to  a \emph{parent edge}. For each ,   has exactly one parent edge. If  is the parent edge of  and graph  contains an actual edge , we can assume w.l.o.g that  does not contain the actual edge , as we can remove it from  and add it to . This operation may introduce parallel edges in graphs  corresponding to -nodes or -nodes. 

We now proceed in two steps. First, for each node , we define a block , that is added to . This will define a valid laminar block decomposition , except that if  is an -node, then  is not necessarily -connected. For each such -node , we then add additional blocks to  in the second step, to avoid this problem.

\paragraph{Step 1:}
Consider a node  of . If , denote the parent edge of  by 
 (if , we do not define  and ). Let  be the union
of all graphs  associated with the nodes  in , with all artificial 
edges removed. 

Since  does not contain any artificial edges, it is a subgraph of .
Clearly,  since every edge of  appears in some graph , for .
We now verify that for every node ,  is indeed a block. 
Let , and let  be the parent edge of . 
Since the graph does not contain any -nodes,  for all , and since for every adjacent pair  of vertices on ,  and  only share two vertices, .
We now show that  is an induced subgraph of  (except that edge  is not in ,  even if it is present in ).
Let  be the edge of , connecting  to its parent. Recall that  decomposes the tree  into two connected components,  and , where one of the components is . Assume it is . We have also defined  to be the union of  for , and similarly  is the union of  for . Recall that , and by definition of , .
Consider some edge  of  that connects two vertices of . Then since  , vertices  do not both belong to . 
By the definition of SPQR trees, the edge  belongs to some graph , for . Therefore  must hold, and . Also, as we have observed before, there are no edges connecting  to  in . Therefore,  is indeed a block.
Note that if  is a -node with only one child then  consists of 2 edges: one parent artificial edge, and one child artificial edge (in this case, originally  contained a third, actual edge, that we have moved to the graph of its father). Therefore for the child node  of ,  contains all vertices and edges that  contains. We add all blocks , for all , to  except if  is a -node with only one child. Notice that under this definition of , for each block , . Clearly, if  is not an -node, then  is -connected.


\paragraph{Step 2} In this step we take care of the -nodes. Consider an -node , and assume that  is a cycle
, where  is the parent artificial edge of . We define a nested set  of blocks, where , that will be added to , in addition to . In order to define the blocks , we define a collection  paths, as follows. Path  is obtained from  by removing the artificial parent edge , so . Path , for , is obtained from  as follows: if  is even, remove the first edge of , and if it is odd, remove the last edge of . Therefore,  is the portion of  between  and . Let  be the set of all child nodes of  in , corresponding to the artificial edges of .
We are now ready to define , for : it contains all actual edges of  and
all blocks  for . 
Graph  has two types of vertices: inner vertices of blocks , for , and
the vertices of . 
We now show that  is a block. Since each , for , is a block, no edge connects the interior of 
to  (and therefore to ). So in order to prove that
 is a block with endpoints  and ,
it remains to show that there is no edge connecting an internal vertex  of  and 
a vertex in . But this is clearly true since we have already proved that  is a block.
We add all blocks , for  to .
For convenience, we denote  and .

This completes the description of the family of blocks .
It is clear from the construction, that  is a laminar family of blocks.
Now consider a block , and the corresponding graph . We need to prove that  is -connected. First,
if , where  is an  or -node then , and therefore it is 3-connected. 
If , for , for an -node , where  is a cycle on  vertices, then  is obtained from  by replacing  with an artificial edge, and connecting the endpoints of  with another artificial edge. Therefore,  is the triangle graph, which is -connected.

We now prove that every vertex  that belongs to some 2-separator of  
is an endpoint of a block in , or it is a degree- vertex, and it has a neighbor that serves as an endpoint of a block in . By the properties of the SPQR trees, every such vertex 
 is either an endpoint of some artificial edge , lying in some graph , for , or it belongs to some graph 
associated with an -node . In the former case, let   be the graph for which the artificial edge , containing , is the parent edge. Then  belongs to , and  is one of its endpoints. The only exception is when  is a -node, with a unique child . But then  cannot be a -node, and  is one of its endpoints. In the latter case, we consider the graph
, containing , where  is an -node. As before, we denote the vertices of  by .
Assume that . If , then  is an endpoint of some path
, , and thus, it is an endpoint of the block . Assume now that . Then the vertex 
 is connected to  and  by edges  and , respectively, in . If either of these edges is an artificial edge, then there is a block , such that , or . Otherwise, if both edges are actual edges, then the degree of  is , and  is a neighbor of  that serves as an endpoint of a block in . 


Finally, we show that if a block  has only one child  then . Observe that if  for a  or -node  then  is the set of endpoints of the parent artificial edge, and  is the set of endpoints of the only child artificial edge. Therefore, if , then  has two parallel aritificial edges, and thus  must be a -node. However, we add a block  associated with a -node to  only if it has more than one child. Now let  be either  or  for some -node . Since all paths  (defined for ) have distinct pairs of endpoints, and they differ from the endpoints of edges, it is straightforward that .

\iffalse
Finally, we prove that every vertex  is an endpoint of at most  blocks in . Let  be the set of nodes , such that . It is easy to see that the sub-graph of  induced by  is a connected tree, that we denote by .
Let  be the set of nodes  for which  contains an actual edge incident to .
Since there are at most  edges incident to , and every actual edge belongs to only one  graph ,
. Recall that every for all , every vertex in  has degree at least . Therefore, if  is a leaf of , then , as the degree of  in  is at least , and at most one of these adjacent edges may be artificial (the parent edge). We now bound . Notice that if  is
 an  or a  node, then every vertex of  has degree at least 3 (since every vertex of a 3-connected graph
with more than 3 vertices has degree at least 3). In particular, if  is an  or a  node, then  contains at least three artificial edges incident on , and therefore  has at least two children in . Therefore, the number of  and -nodes in  is bounded by . Finally, since no two -nodes are adjacent in , the parent of every -node is either a  or an -node, and so the number of -nodes in  is bounded by . Recall that  is an endpoint of a block , for , only if  is an endpoint of the 
parent artificial edge in . Therefore, the number of blocks , for , with  is
at most . Also for each -node  in , if , then there are at most  blocks  for which  serves as an endpoint. We conclude that there are at most  blocks in , for which  is an endpoint.
\fi

In the proof, we did not consider the case where  is the cycle graph. We now briefly address this case. 
Denote the vertices of the cycle by .  
We create blocks  defined by  .
Additionally, we create a block  if .
It is straightforward to verify that this family of blocks together with  satisfies the conditions 
of the theorem.














\fi



\section{Proof of Theorem~\ref{thm: good planar drawing}}\label{sec:finding a planar drawing}




We subdivide the sets of irregular vertices and edges into several subsets, that are then bounded separately. 
We start by defining the following sets of vertices and edges.

Let  be the set of all -connected components of .
For every 2-connected component , we define
\ifabstract

\fi
\iffull

\fi
Let  and . 
We start by showing that the number of vertices and edges in sets  and , respectively, is small, in the next lemma, whose proof appears in Section~\ref{subsec:lemma1}.

\begin{lemma}[Irregular 1-separators]\label{lem:irregular1} We can bound the sizes of sets   and  as follows:
 and .
Moreover, .
\end{lemma}

Next, we show that for {\bf any} planar drawing  of , the number of irregular vertices and edges that do not belong to sets , and , respectively is small, in the next lemma, whose proof appears in Section~\ref{subsec:lemma2}.
Given any drawing  of any graph ,
we denote by  the number of pairs of crossing edges in the drawing  of . Clearly,  for any drawing  of .


\begin{lemma}\label{lem:irregular3}
Let  be any planar graph, and let the sets  of vertices and the sets  of edges be defined as above for . Let  be an arbitrary drawing of  and  be a planar drawing of .
Then
\ifabstract

\fi\iffull

\fi
\end{lemma}
Finally, we need to bound the number of irregular vertices in  and irregular edges in . The bound does  not necessarily hold for {\bf every} drawing . However, we show how to efficiently find a planar drawing, for which we can bound this number, in the next lemma.


\begin{lemma}[Irregular 2-separators]\label{lem:irregular2}
Let , ,  and  be as in Theorem \ref{thm: good planar drawing}.
Given ,  and  (but not ), 
we can efficiently compute a planar drawing  of , such that

and

\end{lemma}

Theorem \ref{thm: good planar drawing} then immediately follows from Lemmas \ref{lem:irregular1}, \ref{lem:irregular3}, and \ref{lem:irregular2}, where we apply Lemma~\ref{lem:irregular3} to the drawings  and  of the graph .
In the following subsections, we present the proofs of Lemmas~\ref{lem:irregular1}, \ref{lem:irregular3} and \ref{lem:irregular2}.

\subsection{Proof of Lemma~\ref{lem:irregular1}}\label{subsec:lemma1}
Consider the following tree : the vertices of  are , and there is an edge between  and  iff .
We partition the set  into three subsets: set  contains the leaf vertices of , set  contains vertices whose degree in  is , and set  contains all remaining vertices.
Since  is 3-connected, for every component  with , there is an edge  with 
one end-point in . We \textit{charge} edge  for . 
Clearly, we charge each edge at most twice (at most once for each of its endpoints), and
therefore, . Since the number of vertices of degree greater than  is bounded by the number of leaves in any tree, we get that , and so .  Since the parent of every vertex  in the tree is a vertex of the form  for , this implies that , and .

We now bound the sum . The sum equals the number of pairs , where  and . The number of such pairs in the tree  is bounded by the number of edges in the tree, which in turn is bounded by the number of vertices, .
\iffalse
Denote the number of two connected
components in  by . Then  has  vertices, and, therefore,
 edges. On the other hand, there are  components
that have degree at least  in . Thus  has at least 
edges. Since every 1-separator separates at least two connected components,
the number of edges is at least .
We get that 
 and
. Therefore, 
 and . So .

Since every edge in  is incident to a vertex from  in ,
. The sum  equals
the number of pairs , where  is a connected component of  and 
, which in turn equals the number of edges in , which is less than .
\fi
This finishes the proof of Lemma~\ref{lem:irregular1}.


\subsection{Proof of Lemma~\ref{lem:irregular3}}\label{subsec:lemma2}
\label{sec:irregular-vertices-edges}


In this section we bound on the number of irregular vertices and irregular edges that do not belong to 
 and , respectively. Lemma~\ref{lem:irregular-vertices} bounds the number of irregular vertices and Lemma~\ref{lem:irregular-edges} the number of irregular edges.

\begin{lemma}
\label{lem:irregular-vertices}
Let  be an arbitrary drawing of  and 
let  be a planar drawing of . Let . Then

\end{lemma}
\begin{proof}
Note first that we may assume that no two adjacent edges cross each other
in the drawing . Indeed, if the images of two edges incident to a vertex  cross, we can uncross their drawings, possibly
changing the cyclic order of edges adjacent to , and preserving the cyclic order for all other
vertices. The right-hand side of (\ref{eq:bound-conflicts})  will then decrease by
12, and the left hand side by at most 1, so we only strengthen the inequality.
We can also assume w.l.o.g. that the graph  is -connected: otherwise, if  is the set of all -connected components of , then,  
since , it is enough to prove the inequality~(\ref{eq:bound-conflicts}) for each component  separately. So we assume below that  is 2-connected.


Consider some vertex . Let  be the face of  
that contains the image of  in the drawing . Note that graph   is 2-connected: otherwise,
if  is a vertex separator of  then  is a -separator for , contradicting
the fact that . Therefore, the boundary of  is a simple cycle, that we denote by . Let  be the neighbors of  in the order induced by . Vertices  partition  into  paths , where path  connects vertices  and  (we identify indices  and ). Let  be the face of the planar drawing , that is
bounded by ,  and . Note that since for all , the two paths  and  do not share
any internal vertices, the total number of vertices that the boundaries of  and  for  share is at most , with the only possibilities being ,  and  (the endpoints of ).

Consider the graph  formed by , , and edges , for . This graph is homeomorphic to the wheel 
graph on  vertices. In any planar embedding of , the ordering of the vertices  is . So if the drawing 
 of  is planar, then the circular ordering of the edges adjacent to  in  is
 -- the same as in , up to orientation. Therefore, if
 then either there is a pair  of paths, with , whose images cross in , or an image of an edge  crosses a path  (recall that we have assumed that no two 
edges  and  cross each other; all self-intersections of paths  can be removed without changing the rest of the embedding).
We say that this crossing point pays for .
Thus every irregular vertex is paid for by a crossing in the drawing .
It only remains to show that every pair of crossing edges pays for at most 12 vertices.

Suppose that  is paid for by a crossing of edges  and . 
For each edge , there is a face  (in the embedding of )
such that  and  lie on the boundary of : if  lies on path 
 then ; if  then  is either 
 or . Since in the latter case we have two choices
for , we can choose distinct faces  and .
Therefore, if a crossing of edges  and  pays for a vertex ,
then there are two distinct faces  and  in , incident to  and 
 respectively, such that  lies on the intersection of the boundaries of  and
.
We say that the pair of faces  and  is the witness 
for the irregular vertex . Since the boundaries of  and  may share at most  vertices that do not belong to , the pair  is a witness
for at most 3 irregular vertices. Since each edge  is incident to at most two faces in ,
there are at most  ways to choose  and , and for each such choice  is a witness for at most 3 irregular vertices. We conclude that each pair of edges that cross in  pays for at most  irregular vertices.
\ifabstract \qed \fi \end{proof}

\begin{lemma}\label{lem:irregular-edges}
Let  be an arbitrary drawing of  and  be its planar drawing. 
Let . Then

\end{lemma}
\begin{proof}
We can assume w.l.o.g. that there are no vertices of degree  in , by iteratively removing such vertices , and replacing the two edges incident on  with a single edge. This operation may decrease the number of irregular edges by at most factor , and can only decrease the number of pairs of crossing edges.
Similarly to the proof of Lemma~\ref{lem:irregular-vertices}, we assume that the graph  is -connected:
otherwise, we can apply the argument below separately to each 2-connected component.



We say that the orientation of a regular vertex  is positive,
if the ordering of the edges incident to  is the same  in  and , including the flip.
If the flips in  and  are opposite, we say that the orientation is negative.
For every irregular edge , the orientation of one of its endpoints is positive, and of the other is negative.

Consider an irregular edge , and assume w.l.o.g. that the orientation of  is positive and the orientation of  is negative.
Let  and  be the two faces incident to  in the embedding . 
Since  is 2-connected, the boundaries of  and  are simple cycles. Denote them by  and . 
Let  be the sub-path of  that connects  to . We now prove that  and  do not share any
vertices except for  and . 
Indeed, assume for contradiction that a vertex  lies
on both  and . Since , either  or  (or both) are not in . Assume w.l.o.g. that .
 We draw two curves, connecting  to the middle
of the edge  inside the planar drawing  of ; one of the two curves lies inside  and the other lies inside .
The union of the two curves defines a cycle that separates  into two pieces, with  belonging to one piece and 
 to the other. Denote these pieces by  and , respectively. (We assume that , ). We will now show that  is a -separator for , leading to a contradiction. Observe first that  since the degrees of  and  are at least , and the separating cycle only crosses one edge of  (the edge ), both  and  contain at least  vertices each. Since every path from 
 to  must cross the separating cycle,  each such path either contains
the vertex  or the edge . Therefore,  is a -separator for , contradicting our assumption that .

We say that the pair of faces  is the witness for the irregular edge . From the above discussion,
each pair of faces is a witness for at most one irregular edge.

\begin{figure}
\begin{center}
\scalebox{0.65}{\includegraphics{bad-edge.pdf}}
\caption{Graph , irregular edge , paths  and .\label{fig:irregular-edge}}
\end{center}
\end{figure}
Let  be the orientation --- either clockwise or counterclockwise ---
in which paths paths , , and  leave  in the embedding 
(where  is either  or ). If the orientation is clockwise
; otherwise . Similarly, we define .
Note that in any embedding  in which paths ,  and 
do not cross each other, . 
In particular, since  is a planar embedding, . But since the orientation of  is positive, and the orientation of  is negative, 
.
Therefore, there is a pair  of crossing edges in , where either , ;
or , ; or  and . 
We say that the crossing of  and  pays for the irregular edge .
The edges  and  lie on the boundaries of  and  respectively.
Similarly to the previous lemma, given two crossing edges  and , there are at most  ways to choose the 
faces  incident to them, and each such pair of faces is a witness for at most one edge. 
Therefore, each pair of crossing edges pays
for at most 4 irregular edges. We conclude that the number of irregular edges is bounded by . Replacing the edges back by the original -paths increases the number of irregular edges by at most factor , as each irregular -path contains two irregular edges.
\ifabstract \qed \fi \end{proof}





\subsection{Proof of Lemma \ref{lem:irregular2}} 
We start with a high level overview of the proof. Assume first that the graph  is 2-connected. We can then use Theorem~\ref{thm: block decomposition} to find a laminar block decomposition  of . Moreover, each vertex  is either an endpoint of a block in , or it is a neighbor of an endpoint of a block in . Therefore,  is roughly bounded by . On the other hand, since the graph  is 3-connected, each block  must contain an endpoint of an edge from  as an inner vertex, that can be charged for the block , for its endpoints, and for the neighbors of its endpoints. This approach would work if we could show that every edge  is only charged for a small number of blocks. This unfortunately is not necessarily true, and an edge  may be charged for many blocks in . However, this may only happen if there is a large number of nested blocks, all of which contain the same endpoint of the edge . We call such set of blocks a ``tunnel''. We then proceed in two steps. First, we bound the number of blocks of  that do not participate in such tunnels, by charging them to the edges of , as above. Next, we perform some local changes in the embeddings of the tunnels (by suitably flipping the embedding of each block of the tunnel), so that we can charge the number of irregular vertices that serve as endpoints of blocks participating in the tunnels to the crossings in .


We now proceed with the formal proof.
We start with an arbitrary planar drawing  of . Let  be the set of all -connected components of . We consider each component  separately.
For a component ,
let  
denote the number of crossings in  in which edges of  participate, and let  denote the subset of edges of  that have at least one endpoint in .
We will modify  locally on each 2-connected component  and obtain a planar
drawing  of  such that
\ifabstract

\fi \iffull

\fi Summing up over all , and using Lemma~\ref{lem:irregular1} gives the desired bound.
Since we  guarantee that the modifications of  are restricted to ,
we can modify the 2-connected components  independently to obtain 
the final desired drawing.

Fix a 2-connected component . If  is 3-connected then 
 and there is nothing to prove. So we assume below that  is not
3-connected. We compute the laminar block decomposition  and the corresponding decomposition tree  for , given by Theorem \ref{thm: block decomposition}. For convenience, we use  to denote the set of all blocks in , excluding the whole component .
We now proceed in three steps. Our first step is to explore some structural properties of the blocks . We will use these properties, on the one hand, to bound the number of blocks that do not participate in tunnels, and on the other hand, to find the layout of the tunnels.
In the second step, we define the subsets of blocks that we can charge to the edges in . We then charge some of the vertices in  and edges in  to these blocks. In the last step, we define tunnels, to which all remaining blocks belong, and we show how to take care of them.

\subsubsection*{Step 1: Structural properties of blocks}
Consider some block , with endpoints  and . Since  is 2-connected, 
there is a path  in .
Moreover, if  is the parent of  in , whose endpoints are  and , we can ensure that , as follows. Consider the graph  obtained from  after we remove all inner vertices of  from it. Since  is -connected, so is . Therefore, there are  vertex disjoint paths in , connecting the vertices in  to the vertices in . We assume w.l.o.g. that these paths are  and . We can then set  (see Figure~\ref{fig: paths pout}). Therefore, from now on we assume that if  is the parent of , then .



\begin{figure}[h]
\begin{center}
\scalebox{0.3}{\rotatebox{0}{\includegraphics{paths-pout-cut.pdf}}} \caption{Paths , .} \label{fig: paths pout}
\end{center}
\end{figure}



Since we have assumed that  is 3-vertex connected, for every block , there is also a path  in , connecting
an {\bf inner} vertex of the block , with an {\bf inner} vertex of the path . 
Let  be the last vertex on  that belongs to  and  
be the first vertex on  that belongs to  (notice that , since  does not contain  or ). We denote the segment of  
between  and  by , and we call the vertex  \textit{the connector vertex} for 
the block . 

Note that if  is a child block of  and  is an inner vertex of  as well, then since , we can choose
 to be the connector vertex of  as well, and use .  So we assume that each connector 
vertex  appears contiguously in the tree . That is, if  is a descendant
of  and an ancestor of  and , then . We also assume that in this case .
We denote the segment of  between  and  by  and the segment
between  and  by . 

Since  is 2-connected, there are 
two vertex disjoint paths between  and  in . One of them must pass through 
and the other through . We denote the segment between  and  of the former path
by  and the segment between  and  of the latter path by .
Let  be the concatenation of , .
Note that the paths , , ,  and 
do not intersect, except at endpoints (see Figure~\ref{fig: paths pout2}). We emphasize that  is an {\bf inner} vertex of , and  is an {\bf inner} vertex on path  --- a fact that we use later.


\begin{figure}[h]
\begin{center}
\scalebox{0.3}{\rotatebox{0}{\includegraphics{paths-pout2-cut.pdf}}} \caption{Paths , , ,  and . Vertex  is an inner vertex of , and vertex  is an inner vertex of . All five paths are non-empty and completely disjoint except for their endpoints.} \label{fig: paths pout2}
\end{center}
\end{figure}

For each component , let  be the union of 
(i) the set  and (ii) the set of vertices of  incident to edges of . 
 Using Lemma~\ref{lem:irregular1},



We now show that for each block , the connector vertex .
Indeed, consider the first edge  of the path
. If , then , as by the definition
of the block, no edges of  connect inner vertices of  to . Otherwise, if , then  must be a -separator, so . 



Finally, we study structural properties of chains of nested blocks. We also introduce a notion of a \textit{simple} block, and show that all non-simple blocks contain a certain useful structure.

\begin{Definition}
Let  be any block, whose endpoints are denoted by  and . We say that  is a \emph{simple block} iff it contains exactly three vertices, , and , and has exactly one child in , denoted by  (assume w.l.o.g. that the endpoints of  are ). Moreover,  is obtained by adding exactly one edge, , to  (see Figure~\ref{fig:irregular-block}). If  has exactly one child in , but it is not a simple block, then we say that it is \emph{complex}.
\end{Definition}

\begin{figure}
\begin{center}
\scalebox{0.6}{\includegraphics{irregular-block.pdf}}
\caption{A simple block .
\label{fig:irregular-block}}
\end{center}
\end{figure}


\begin{figure}
\begin{center}
\scalebox{0.6}{\includegraphics{regular-block-n.pdf}}
\caption{A complex block. Paths  are pairwise vertex disjoint, except for containing  as a common endpoint.}
\label{regular-block}
\end{center}
\end{figure}

We need the following two claims.

\begin{claim}
\label{cor:regular-block-in-a-chain}
Consider a chain of 5 nested blocks: , , ,  and , where  is the only child of  (for ).
Assume that no vertices in  have degree 2 in . Then one of the blocks ,,, or  is complex.
\end{claim}
\begin{proof}
Notice that from the definition of simple blocks, if all blocks  are simple, at least one vertex  must have degree  in  (where  and  are endpoints of ), contradicting the fact that  cannot contain such vertices.
\ifabstract \qed \fi \end{proof}


\begin{claim}\label{claim: irregular}
Suppose that a non-simple block  has exactly one child  in . Denote the endpoints of  by  and , and the endpoints of  by  and . 
Then for every vertex , 
there are three paths , , and , with , and all three paths are contained in . Moreover,  and  do not share any vertices, except for the vertex  that serves as their endpoint. (See Figure~\ref{regular-block} for an illustration.)
\end{claim}

\begin{proof}
Since  has only one child,  or  
(or both). Let us assume w.l.o.g. that .
 In particular,  contains at least  vertices.
 
We consider two cases. Assume first that  contains exactly  vertices. Then these vertices must be  and . The only valid choice for the vertex  is . From the definition of blocks,  cannot contain the edge . But since it is connected, it must contain the edge . Therefore, the only way for  not to be simple (since we have assumed that  contains no parallel edges) is if  contains the edge . But in this case, we get the following three paths: , and .




Assume now that  contains at least  vertices.  From Theorem~\ref{thm: block decomposition}, the graph  is 3-connected.  Let  be an arbitrary inner vertex of . Assume first that .
Recall that the Fan Lemma states that for every -connected graph
, a vertex  in  and a set of 
vertices , there exist 
paths that connect  to vertices of  that have no
common vertices other than .
We apply the Fan Lemma in graph  to  and .
Let  be the resulting path between  and ,
 the path between  and , and
 the path between  and . 
Note that paths  and  do not contain 
the artificial edge , as otherwise they would
contain . 
Notice also that none of the three paths contains the artificial edge , as this would violate their disjointness. 
Finally, let  be equal to either , if  does not visit ,
or the segment of  between  and , if it does (the latter can only happen if ).
We have thus constructed the required paths ,  and .
Assume now that .
Since  is 3-vertex connected (and ),
the graph  is 2-vertex connected. We again apply the Fan Lemma
to  and  in this graph and find the desired paths  and .
We let  to be the trivial path of length .
\ifabstract \qed \fi \end{proof}


\subsubsection*{Step 2: Blocks we can pay for}
Fix a -connected component .
In this step, we define three subsets  of , and bound the number of blocks contained in them. We also define a subset  of vertices and a subset  of edges, that can be charged to these blocks. The remaining blocks of  will be partitioned into structures called tunnels, and we take care of them in the next step.


\paragraph{Set :} Let  denote the set of blocks , 
such that  is either the root of , or it is one of its leaves, 
or it has a degree greater than  in ,
or it contains a vertex from  that does not belong to any of its child blocks.
We also add five immediate ancestors of every such block to .


\begin{claim} .
\end{claim}
\begin{proof}
Denote the number of leaves in  by .
For each leaf block , we charge the connector vertex  for .
For each non-leaf block , such that  contains a vertex  that does not belong to any of its children, we charge  for  (even if ).
Since  is a laminar family, it is easy to see that each vertex  is charged at most once.
The number of vertices of degree at least  in  is at most . 
By adding five ancestors of each block, we increase the size of 
by at most a factor of . 
Therefore, .
\ifabstract \qed \fi \end{proof}


\paragraph{Set :} 
Consider a vertex .
Notice that the set of blocks  with  must be a nested set.
We add the smallest such block and its five immediate ancestors to .


\begin{claim} .
\end{claim}
\begin{proof}
For each block , we charge the connector vertex  for . 
By the definition of , each connector vertex pays for at most  blocks.
Therefore, .
\ifabstract \qed \fi \end{proof}

\paragraph{Set :} 
Note that the blocks of  that do not belong to  
all have degree exactly 2 in , and therefore the sub-graph of  induced by such blocks is simply a collection of disjoint paths. 
Consider some block .
It has exactly one child in , that we denote by .
Let  and  be the endpoints of , and let  and  be the endpoints of .
Consider the graph  obtained from  by first replacing  with 
an artificial edge  and then by adding a new artificial edge .
By Theorem \ref{thm: block decomposition}, the graph  is -vertex connected.
Therefore, it has a unique planar drawing .
We add  to  iff the four vertices  {\bf do not lie} 
on the boundary of the same face in this drawing.

\begin{lemma}\label{lemma:r3}
 
\end{lemma}
\begin{proof}
Consider some block . Denote , and for ,
let  be the child of  in . For each , let  denote the endpoints of the block .
Since when we added a block to  or , we also added 
five its immediate ancestors to  or , respectively, each of the blocks , for ,
has a unique child, and moreover, for ,  and .
Let  denote the edges of  that do not belong to , that is, . We will show that for each , there is at least one crossing in , in which the edges of  participate. Since every edge may belong to at most  such sets , it will follow that , and .
Therefore, it now only remains to show that for each block , the edges of  participate in at least one crossing in . Assume for contradiction that this is not true, and let  be the violating block. We will show that we can find a planar drawing of , in which the vertices  all lie on the boundary of the same face, contradicting the fact that .

We denote by  the graph obtained from  after we remove all inner vertices of  and their adjacent edges from it. Notice that all edges of  belong to . We also denote  and . Recall that for all ,  and .
Recall that by definition,  is an {\bf inner} vertex on  for all , and  is an {\bf inner} vertex on .

We start with a high-level intuition for the proof. Let , and assume for now that  only contains the edges of  (this is not necessarily true in general). Observe that  contains no edges of . Therefore, the sets  and  of edges are completely disjoint. 
Consider the drawing  of , and erase from it all edges and vertices, except those participating in ,  and . Let  be the resulting drawing. For convenience, we call the edges of  \emph{blue edges}, and the remaining edges \emph{red edges}. By our assumption, the blue edges do not participate in any crossings. Since we have assumed that  only consists of blue edges, all crossings in  are between the edges of ,  and . All these three paths share a common endpoint, , and they are completely disjoint otherwise. Therefore, we can uncross their drawings in , and obtain a planar drawing  of . Erase the drawing of  from , and replace the drawings of paths  and  by drawings of edges , , respectively, to obtain a planar drawing  of . 
Note that in , the drawings of edges  and  (and therefore
their endpoints) lie on the boundary of one face, since the drawing of the path  in  connects {\bf internal} points of edges
 and  and does not cross the images of any edges.  Therefore, we have found a planar drawing of , in which the vertices  lie on the boundary of the same face, contradicting the fact that . The only problem with this approach is that  does not necessarily only consist of edges of . We overcome this by finding a new path  that only contains edges of  but no edges of , and another path  connecting an inner vertex  of  to the vertex . 
If we ensure that (1)  only contains edges of  but no edges of ; (2) path  connects an inner vertex  of  to  and contains no edges of ; and (3) The paths  and  are completely disjoint, except for possibly sharing endpoints, then we can again apply the above argument, while replacing the path  with , and path  with . We now provide the formal proof. 





\iffalse
----------------------------------------------------------------------------------------------
----------------------------------------------------------------------------------------------
----------------------------------------------------------------------------------------------
----------------------------------------------------------------------------------------------

We now show that we paid for all blocks in . Assume to the contrary that we did not
pay for a block . That is, edges of  do not participate in any
crossings in the drawing  of .
We will show how to construct a planar drawing of ,
in which vertices , ,  and  lie on the boundary of one face,
contradicting our assumption the . 

Consider the drawing of  in . By our assumption,
this drawing is planar (since  is a subgraph of ).
We want to add drawings of edges  and  and obtain
a drawing  of  so that  is planar and vertices , ,  and  
lie on the boundary of one face.
Assume for a moment that there is a path  in  that connects
 to , and a path  outside of 
that connects some internal vertex  of the path  to  and, moreover,
paths , , and  share no vertices except that
 belongs to both paths  and  and  lies on both
 and . 
 
We consider the drawings of paths  ,  and  in . 
By our assumption, they do not cross the drawing of , and, in particular, 
of . Also the drawings of  and  do not cross the 
drawing of . So the only crossings in
the drawing of 
are the crossings between paths  and .
Recall that the vertex  splits the path 
into two pieces  and .
Three paths ,  and  start at one vertex, .
Therefore, we can uncross their drawings and obtain a planar drawing 
of .
Finally, we erase the drawing of  and contract all internal vertices 
on paths  and . We get a drawing  of .


It remains to show that we can always find paths  and 
as above.
----------------------------------------------------------------------------------------------
----------------------------------------------------------------------------------------------
----------------------------------------------------------------------------------------------
----------------------------------------------------------------------------------------------
\fi


We first note that at least one of the four blocks  is complex. 
Indeed, by Claim~\ref{cor:regular-block-in-a-chain} it suffices to show that  does not contain a vertex  whose degree is  in . Note that if  and the degree of  in  is , then . This is since  is -connected, and so all degree- vertices in  must either be incident on an edge of , or belong to . Therefore, one of the blocks  must have been added to , together with its five immediate ancestors. 

We finally show that since one of the blocks , for , is complex, we can find the planar drawing of  in which  lie on the same face, thus leading to contradiction.


\begin{claim}
If at least one of the blocks , for  is complex, then there is a planar drawing of , in which  all lie on the boundary of the same face.
\end{claim}

\begin{proof} 
Let  be the first complex block among , ,  and . Notice that since  has only one child in , it must contain at least one inner vertex.
Choose an arbitrary inner vertex  of . Since  is complex, 
there are three paths , ,
and , as in Claim~\ref{claim: irregular}. We assume w.l.o.g, that .
We extend paths  and  to paths  and , connecting  to vertices  and , as follows.
Since  is 2-connected, there are two vertex disjoint paths connecting  to  in . We assume w.l.o.g. that these paths are  and .
We append these paths to  and , obtaining the desired paths  and . 
Finally, we define paths  and , as follows. 
Let  be the union of paths  and .
Let  be the union of paths ,
 and  (see Figure~\ref{fig:b3-paths}).
Observe that  is indeed an inner vertex of , so  connects an inner vertex of  to an inner vertex of , as required.

\begin{figure}
\begin{center}
\scalebox{0.7}{\includegraphics{b3-paths.pdf}}
\caption{Paths ,  (and their extensions  and ), and . Recall that path , and path .
\label{fig:b3-paths}}
\end{center}
\end{figure}


We now verify that paths  and  satisfy other required conditions.
First,  only contains edges of  but no edges of , since all paths , , , 
lie in  but do not contain edges of . 
Next, path   does not contain edges of , since it
is the concatenation of the path ,
the path  and the path , that does not contain edges of . 
It is straightforward to verify that paths , , and  
share no vertices except for  and . Therefore, the sets  and  of edges are completely disjoint, as required. 

We now consider the drawing  obtained from , after we remove all edges and vertices, except those participating in ,  and . We call the edges of  blue, and the remaining edges red. Then  only consists of blue edges, but it does not contain edges of . Since in the resulting drawing, , no blue edges participate in crossings, the only crossings involve paths  and . As before, we can uncross them and obtain a planar drawing , which gives a planar drawing  of , in which the vertices  all lie on the same face.
\ifabstract \qed \fi \end{proof}

\iffalse
Finally, we prove that at least one of the four blocks  is regular. We first observe that  cannot contain a vertex  whose degree is  in . Indeed, if  and the degree of  in  is , then . This is since  is -connected, and so all degree- vertices in  must either be incident on an edge of , or belong to . Therefore, one of the blocks  must have been added to , together with its five immediate ancestors. Finally, the next claim will finish the proof:
\fi


\iffalse

\noindent\textbf{Case 2.} Now suppose that all blocks , , , and  
are irregular.
We need the following simple lemma that shows that essentially there is only one 
type of irregular blocks.
\fi






\iffalse
\begin{lemma}\label{lem:three-paths}
Suppose that a block  has only one child  in .
Then one of the following must be true.
\begin{itemize}
\item  is a regular block.  
\item ,  is adjacent to , and  
(possibly after we interchange the names of  and ), or
\item ,  is adjacent to , and 
(possibly after we interchange the names of  and ).
\end{itemize}
\end{lemma}

We apply this lemma to blocks , ,  and .
Since all of them are irregular, each  is equal to 
plus an edge, either  or .
Therefore, the sum of the lengths of paths from  to 
and from  to  is equal to 3. One of these two paths must 
be of length at least 2. Let us say it is the path from  to .
Let  be an interior vertex on this path. The vertex  has 
degree 2 in , therefore, it belongs to . Thus 
one of the blocks  (if it contains ) or  (otherwise)
meets the criterion for inclusion in (X): it contains 
a vertex from  that none of its children contains.
Then 5 immediate ancestors of  or  must be also in .
That is, , which contradicts to our assumption 
that .   
We finished the proof that .
\fi

\iffalse
\begin{figure}
\begin{center}
\scalebox{0.85}{\includegraphics{R3}}
\caption{Construction in the proof of Claim \ref{claim:r1r2r3}.\label{fig:R3}}
\end{center}
\end{figure}
\fi

\ifabstract \qed \fi \end{proof}


Let , and let
 be the set of all blocks , whose parent belongs to .
Since all leaves of tree  belong to , it is easy to see that .
Therefore, we get the following corollary:

\begin{corollary}\label{corollary: sizes of R's}

\end{corollary}

By Theorem~\ref{thm: block decomposition}, every vertex in  is an endpoint of 
a block in , or it has degree 2 in . 
Let  denote the set of vertices of  that either have degree  in , or serve as endpoints of blocks in , and let . Additionally, let , and . Since, as we already observed, vertices that have degree  in  belong to , we have that: 
\ifabstract

\fi\iffull

\fi We let  denote the edges of  that have at least one endpoint in , and . Additionally, let , and . Clearly,


It now only remains to bound the number of irregular vertices in set , and the number of irregular edges in set . From our definitions,  for each , for each , there is a block , such that  is an endpoint of . Moreover, for each , both endpoints of  belong to .


\iffalse
In particular every \textit{irregular vertex} in  is an endpoint
of a block in  or has degree 2 in ;
and every \textit{irregular edge} in  is incident to an endpoint of a block or to 
a vertex of degree 2.

Each vertex of degree 2 in  belongs to , so the
total number of such vertices is .
For each , let  denote the vertices that serve as endpoints of blocks in , or that belong to , and let .
By Corollary~\ref{corollary: sizes of R's}, . It now therefore remains to bound the number of irregular vertices in . Each vertex in  is an endpoint of a block in 
 
Similarly, let  denote the set of edges incident to endpoints of blocks
in , their child blocks, and
edges incident to vertices of degree 2 in , and let . Then .
Every edge in  
must belong to a block , where it is adjacent to one of its endpoints.
It now only remains to bound the number of irregular vertices in  and irregular edges in , over all .
\fi

\subsubsection*{Step 3: Taking care of tunnels} 
We now consider blocks of .
The degree of each such block in  is .
A \emph{tunnel}  is a maximal path in  containing blocks in 
.
Let  denote the set of all such tunnels in , and let . Notice that each pair of tunnels is completely 
disjoint in the tree  (but their blocks may share vertices: if the first block of one of the tunnels is a descendant of the last block of another in , then the blocks are nested; also, the first blocks of two tunnels can share endpoints). 


The parent of the first block (closest to the root of ) in a tunnel 
belongs to . 
Therefore, by Corollary~\ref{corollary: sizes of R's}, the total number of tunnels is at most


Consider some tunnel . Denote the endpoints of the block  by , for .
Let  be the unique child of block  in , and denote its endpoints by . Since a tunnel consists of consecutive blocks in , none of which
are in , all blocks in the tunnel have the same connector vertex.
Denote , , , and recall that for all , , , and . Let
 and .
Note that  is an inner vertex of , and  is an inner vertex of .
All three paths ,  and  share no vertices except for  and .

We define two auxiliary graphs corresponding to the tunnel . First, we remove all inner vertices of  from , to obtain the graph . 
We then add paths , ,  to , contracting all degree- vertices
in the subgraph , to obtain the graph .
Therefore, the paths ,  and  are represented by  edges in  (see Figure \ref{fig:tunnel}).
We call these edges \textit{artificial edges}. 



Observe that  induces a planar drawing  of the graph . However, in this drawing, we are not guaranteed that the vertices  all lie on the boundary of the same face. Our next goal is to change the drawing  to ensure that all these vertices lie on the boundary of the same face. We can then extend this drawing to obtain a planar drawing of . Combining the final drawings  for all tunnels  will give the final drawing  of the whole graph.

We start with the drawing  of , induced by . We then perform  iterations. In iteration , we ensure that  all vertices in  lie on the boundary of the same face. We refer to this face as the outer face. For convenience, we denote  and  by  and , respectively.

Consider some iteration , and assume that we are given a current drawing  of , in which the vertices in 
 lie on the boundary  of the outer face  of the drawing. Let  be the drawing, induced by , of the graph . Let  be the drawing obtained from  after we replace  with a single edge. Notice that  both lie on , so we can view  as the drawing of the path . Recall that in the unique planar drawing  of , the four vertices  all lie on the boundary of the same face. In particular, there is a cycle , such that , and if  denotes the drawing of  given by , then all edges and vertices of  are drawn inside . Let ,  be the two segments connecting  to  in . Notice that both  and  must belong to the same segment, since otherwise, the ordering of the four vertices along  is  either , or , and the images of the artificial edges  and  would cross in . Assume w.l.o.g. that 
 We have three possibilities. The first possibility is that the vertices  belong to  -- in this case we do nothing. The second possibility is that the segment . In this case we can ``flip'' the drawing of , so that now  lies on the boundary of the outer face of the drawing of , thus ensuring that all vertices   lie on the boundary of the outer face.
The third possibility is that there is an edge  that belongs to . In this case, we ``flip'' the image of the edge  (possibly together with the image of ), so that  becomes the part of the boundary of the outer face (see Figure~\ref{fig: flipping one round}).

\begin{figure}
\begin{center}
\ifabstract
\scalebox{0.4}{\includegraphics{flip-one-round-cut.pdf}}
\fi\iffull
\scalebox{0.5}{\includegraphics{flip-one-round-cut.pdf}}
\fi
\caption{Iteration .\label{fig: flipping one round}}
\end{center}
\end{figure}

\iffalse
We now obtain a planar embedding of  from , as follows. Since none of the blocks , for , belongs to , 
in the unique planar drawing  of , the four vertices , ,  and  
lie on the boundary of same face of the embedding. 
Let  be the embedding of , induced by the initial planar embedding  of , and let  be the embedding of  induced by . Observe that since  is a planar embedding of , it also induces a planar embedding  of , as follows: start with , and erase all edges and vertices, except for those participating in ,  and . Now route the artificial edge  along the image of the path , and the artificial edge  along the image of the path . Since graph  has a unique planar embedding,  must be identical to . Therefore, the four vertices  lie on the boundary of the same face of . Moreover, the circular ordering of these vertices is either , or  (since otherwise, the images of the artificial edges  and  would cross in .
\fi


Let  be this new embedding of the graph . 
Since different tunnels are completely disjoint (except that it is possible that the last block of one tunnel contains the first block of another), we can perform this operation independently for each tunnel , for all  and the resulting planar 
embedding  is our final planar embedding of . 
Notice that for every tunnel , we can naturally extend  to a planar embedding  of , by adding a planar drawing of the  artificial edges of  inside the face on whose boundary the vertices  lie.



\begin{figure}
\begin{center}
\ifabstract
\scalebox{0.47}{\includegraphics{tunnel}}
\fi\iffull
\scalebox{0.6}{\includegraphics{tunnel}}
\fi
\caption{Graph . Bold lines are the artificial edges, representing the paths  and . The second figure shows the outcome of the flipping procedure, where all vertices  lie on the boundary of one face.\label{fig:tunnel}}
\end{center}
\end{figure}

It now only remains to bound the number of irregular vertices in , and the number of irregular edges in .


For every tunnel , let .
We need the following lemma, whose proof \iffull appears in the end of this section\fi \ifabstract appears in the full version of the paper\fi.

\begin{lemma} \label{lem:JZ-connected}
For every tunnel , .
\end{lemma}

We now show how to complete the proof of
Lemma~\ref{lem:irregular2}, using Lemma~\ref{lem:JZ-connected}.

Recall that  is the optimal embedding of . For each tunnel , we define the following drawing : first, erase from  all edges and vertices, except those participating in , ,  and  (that have been defined for ). Next, route the five artificial edges of  along the images of the paths ,  and . Finally, if any pair of artificial edges crosses more than once in the resulting embedding, perform uncrossing, that eliminates such multiple crossings, without increasing the number of other crossings in the drawing. Let  denote the number of crossings in the resulting drawing. Since the five artificial edges may have at most  crossings with each other, we have that:

and


\iffalse
Note that every irregular vertex in  w.r.t.~ and ,
except possibly , ,  and , is also an irregular vertex w.r.t.~ and ,
since drawings of all edges incident to it coincide in  and , and in
 and . Similarly, every irregular path (or segment of an irregular path) in   
w.r.t.~ and  is also
an irregular path w.r.t.~ and  unless it starts at , ,  or .
\fi


Fix some tunnel . Since the drawing  is planar, we can apply
Lemma~\ref{lem:irregular3} to the drawings  of , and get that:

and

Summing up over all tunnels , we get that:
\ifabstract

\fi\iffull

\fi
and


Finally, we observe that since the tunnels are disjoint, if , , and , then either , or  is an endpoint of the first block of the tunnel . Therefore,
\ifabstract

\fi\iffull

\fi
Each edge in  has both endpoints in , and therefore must be either completely contained in some tunnel, or be adjacent to an endpoint of the first block of a tunnel. So if , and , then either  for some tunnel , or it is adjacent to an endpoint of the first block of some tunnel . Therefore,
\ifabstract

\fi\iffull

\fi

\iffull
It now only remains to prove Lemma~\ref{lem:JZ-connected}.
\fi

\iffull
\subsubsection*{Proof of Lemma~\ref{lem:JZ-connected}}

Consider a tunnel ,  for some .
We will show that if we remove any pair of vertices, except for, possibly, pairs in the set
, the graph  remains connected.
For convenience, we denote  and , the endpoints of the unique child  of .

Observe that every vertex  of  has degree
at least  in : otherwise we would have a vertex  with . Then either  is incident on an edge in , or it belongs to , and therefore
and . Then the smallest  of  that contains 
would belong to . 

Consider now a vertex .  
Let  be the smallest block that contains . Since  contains an inner vertex ,  must be a complex block. Therefore,  from Claim~\ref{claim: irregular},  is connected to ,  and either 
or  by three vertex disjoint paths. It is obvious that each of the 
vertices  and  is connected to , , and
either  and  by three vertex disjoint paths. 

Let .
Observe that it is enough to show that for any pair  of vertices, all vertices in  remain connected in the graph . Indeed, assume that all vertices in set  remain connected in graph . Let  be any pair of vertices of . We show that  remain connected as well in the resulting graph. Indeed, if both , each one of these vertices has three paths disjoint paths connecting them to vertices of , and at least one of the three paths must survive even after the removal of  from . Similarly, if one of vertices  belongs to , they remain connected as long as all vertices of  remain connected.

We now show that for any pair  of vertices, , the vertices in  all remain connected in graph .

Consider the unique planar embedding of . If we remove the edge  from this embedding, we obtain a cycle  containing the vertices of : this cycle is simply the boundary of the face that contained the edge .
Denote the ordering of vertices on the cycle by , where  are endpoints of block  (but observe that some consecutive vertices in this ordering may coincide, e.g. it is possible that ).
Let , and .

Consider a vertex  with . Denote its neighbors on the cycle
by  and  (since some vertices  might coincide, the vertex
 is not necessarily equal to ).
Let  be a neighbor of  other than  and . Clearly,
, because of definition of block. 

Assume first that  either.
Consider the smallest block  that contains .
Again, since , block  has to be complex. Therefore, there is a path from  to  that 
does not visit any vertex in . By concatenating this path with the edge , 
we get a path  from  to   that does not visit any other vertex in . If , then the path  is simply the edge .
Similarly, there is a path  from every vertex  of degree at least  
to some , that does not visit any other vertex in . 

Assume for contradiction that there is some pair , such that in the graph , the set  is not connected. Since all vertices of  lie on the cycle , it is clear that  and  must belong to .
Moreover, both of them must lie on the same of the two arcs connecting  and  in .
Let us say that  belong to the arc on which the vertices  lie.
Then the other arc of  remains connected. But then for each vertex  of degree at least , the path  will connect it to the vertices in . Similarly, if  belong to the arc containing vertices of , each vertex  of degree at least , remains connected to the vertices of  via the path .
 
 Now if  has degree , then it must be either  or .
It is clear that we can disconnect  from  and 
only by removing both  and  (since  is -vertex connected). Similarly, we can disconnect
 from  and  only by removing both  and . If vertex  has degree , then it must be either , or , and so the only pairs of vertices that can disconnect it are  and .
\fi





\iffalse
\subsection{Laminar block partition: Proof of Lemma \ref{lem:laminar_family}}

We begin with some definitions.

\begin{Definition}
A block  is called \emph{elementary} iff it contains exactly one inner vertex.
\end{Definition}

\begin{Definition}
A block , with  is called \emph{basic} iff the following conditions hold:
\begin{itemize}
\item There is no vertex  that separates .

\item For every pair  of inner vertices of , there is a path  in , with .
\end{itemize}
\end{Definition}

Notice that elementary blocks are not basic.

\begin{center}
\scalebox{0.90}{\includegraphics{block-types}}
\end{center}

\begin{Definition}
A block is called \emph{minimal} iff it contains a minimum number of vertices among all blocks. 
\end{Definition}

\begin{observation}\label{observation: minimal block elementary or basic}
A minimal block is either elementary or basic.
\end{observation}
\begin{proof}
If  is not elementary or basic, then there must either exist a vertex  separating  from , or  can be decomposed into smaller blocks with the same end-points, and disjoint interiors.
In the later case,  is clearly not minimal.
In the former case, either , or  are the end-points of a strictly smaller block contained in , contradicting again its minimality.
\ifabstract \qed \fi \end{proof}

We start with , and perform a series of \emph{block contraction steps}.
In a block contraction step, we select a minimal 
block  and add it to our family . We then remove it from  and replace it with an artificial edge connecting its endpoints. We repeat until there are no basic blocks left.
The resulting family of blocks is clearly laminar.
Let  be the corresponding rooted tree with vertex set .

We remark that  may contain non-basic blocks, elementary or non-elementary. It is easy to see that over the course of the algorithm, a non-elementary non-basic block may become minimal and may then be added to .


We first argue that almost all vertices in  are near an end-point of some block in .

\begin{claim}\label{claim: vertex not in Q1 has block in F}
There is a set , with , such that for any , there exists a block , such that either , or , where  is a neighbor of .
\end{claim}
\begin{proof}
Let  be the set of degree-2 vertices in , let  be the set of their neighbors in , and set .
Since  is 3-connected, each vertex in  has an edge in  adjacent to it, so , and .

It suffices now to show that for every vertex , there exists a block satisfying the assertion.
Assume for contradiction that no such block exists.
Remove  from graph , to obtain a graph , and apply Procedure -separator to . Let  be the resulting set of components (some of which may consist of a single edge), and let  denote the resulting decomposition tree. A component  consisting of a single edge is called a \emph{simple} component. Let  be any two components that serve as leaves of . Since graph  does not have -separators,  must have neighbors in both  and . 
Notice that  cannot be a simple component, since in this case  would have a neighbor of degree , so  (if  consists of a single edge , and  is the parent of  in tree , then  must be connected to  with an edge, and the degree of  is ); similarly,  is not a simple component.

If  has exactly one neighbor in , then we let ; notice that  must be a basic block of  in this case. Otherwise, we let  be the block induced by . Again,  must be a basic block. We perform the same action with , obtaining a basic block . The two basic blocks are internally disjoint. From our observation on the life of a basic block, at least one of them must belong to .
\ifabstract \qed \fi \end{proof}



\begin{observation}\label{not internally disjoint blocks contain endpoints}
Let  be any pair of blocks, such that . Assume that some vertex  is an inner vertex for both  and . Then  must contain an endpoint of . Moreover, if  is basic, then it must contain the endpoint of  as an inner vertex.
\end{observation}

\begin{proof}
Since , there is some vertex . So there is a path , . If  does not contain an endpoint of , then we have found a path , connecting a vertex  to a vertex , and . This is a contradiction to  being a block.

If  is basic, then there is a path  , , such that all vertices of  (except possibly ) are inner vertices of . Since , it follows that an endpoint of  must be an inner vertex of .
\ifabstract \qed \fi \end{proof}

\begin{lemma}\label{lemma: minimal basic block does not contain endpoints of another}
Let  be a minimal block, and let  be any other block. Then  does not contain any endpoint  as its inner vertex, except when  is elementary. In the latter case, exactly one endpoint  must belong to  and it must be a -separator for , so  cannot be basic.
\end{lemma}

\begin{proof}
Let , and let  be any (possibly non-basic block), with . Assume for contradiction that  is an inner vertex of .

First, if  is an elementary block, then the only neighbors of  are  and . Therefore, either  or . It is impossible that both , because then  must be a -separator in . 
Assume w.l.o.g. that , . Then the only neighbor of  in  is , and so  must be a -separator in , and therefore  cannot be basic.

Assume now that  is not elementary, so by Observation~\ref{observation: minimal block elementary or basic} it must be basic.
We first claim that in this case  as well. Assume otherwise. Since  is a basic block, vertex  is not a -separator for it. So for every pair  of vertices of , there is a path , . Therefore, when  are removed from , all neighbors of  can still reach each other. In other words, the removal of  has no effect on connectivity. Since  is a separator in , it follows that  must be a -separator in , a contradiction. So we can assume that  (it may either be an endpoint or an inner vertex of ). 

Since  is a -separator in , there are at least two internally disjoint blocks  in  with . 
We next claim that either , or  must hold. This will contradict the minimality of . The next claim will then finish the proof.


\begin{claim}
Either , or .
\end{claim}

\begin{proof}
There are two cases to consider. Assume first that  is an inner vertex of . From Observation~\ref{observation: path from u to v}, there must be a path , . Since both  are inner vertices of , either path  is completely contained in one of the blocks, or it is completely disjoint from both of them. In either case, we can assume w.l.o.g. that . We now show that .
Assume otherwise. Then there is a vertex . Since , and  is connected, there is a path  in , connecting  to . But since  is completely contained in , and , we obtain a path connecting a vertex in  to a vertex not in  that does not contain a vertex of , a contradiction. Therefore, , and since , .

Assume now that , say . Only one of the blocks  may contain , so we can assume w.l.o.g. that . Notice that every inner vertex  has a path ,  (if no such path exists then again  is a -separator for ). We now claim that : otherwise, there is a vertex , and a path : , so . Since  is connecting a vertex of  to a vertex outside , this is a contradiction. Therefore,  .
\ifabstract \qed \fi \end{proof}
\ifabstract \qed \fi \end{proof}


\begin{claim}\label{claim: interaction of block with min basic block} Let  be a minimal block, and  any (possibly non-basic) block. Then one of the following must happen:
\begin{itemize}
\item Either .
\item Or both blocks are internally vertex-disjoint;
\item Or  is elementary, and its unique inner vertex  is an endpoint of .
\end{itemize}
\end{claim}

 Notice that in the latter case case,  has exactly two neighbors (the endpoints of ). Exactly one of these endpoints belongs to , and  cannot be basic (this is like in the beginning of the proof of Lemma~\ref{lemma: minimal basic block does not contain endpoints of another}).
\begin{proof}
Assume that none of the above happens. So , and yet there is a vertex  that is internal for both blocks. Recall that  is either basic or elementary. If it is basic, then from Observation~\ref{not internally disjoint blocks contain endpoints},  must contain an endpoint of  as its inner vertex. However, from Lemma~\ref{lemma: minimal basic block does not contain endpoints of another}, this can only happen when  is elementary. So  must be elementary, and its unique inner vertex  is an endpoint of .
\ifabstract \qed \fi \end{proof}


\paragraph{A Life of a Basic Block}
We will next fix some basic block  and follow it throughout the algorithm. Our first claim is that  remains basic after each contraction step.

\begin{claim}
Let  be a basic block in the current graph , and assume that in the graph , obtained from  by contracting some minimal block , . Then  remains a basic block in .
\end{claim}
\begin{proof}
From Claim~\ref{claim: interaction of block with min basic block}, either , or they are internally disjoint. If the two blocks are internally disjoint, then the contraction step does not affect  in any way, and as long as , block  will remain a basic block in the new graph.

If , then it is clear that  will remain a block in the new graph. The contraction step cannot create any new -separators in , and it does not affect the connectivity of vertices inside . So  must remain a basic block.
\ifabstract \qed \fi \end{proof}

Assume that  is a basic block, with , and . Then we can partition the algorithm into two phases. In the first phase, block  remains a basic block in . The first phase ends when . This means that in the last iteration, the contracted block  has , and moreover,  can be partitioned into a number of sets , with , such that for each , there is a block  with  and . 


\fi


\iffull
\section{Handling Non 3-Connected Graphs} \label{sec:reduce-three-connected}



In this section we prove Theorem~\ref{thm:main}, by describing a reduction from the general case --- when
the graph  is not necessarily 3-connected --- to the 3-connected case. 
Our algorithm consists of two parts. In the first part, we decompose the original graph  into a number of sub-graphs, and find a drawing for each one of the sub-graphs separately. In the second part, we combine these drawings together to obtain the final drawing.

\subsection{Part 1: Decomposition}
We first note that we can assume w.l.o.g. that the input graph  is -connected:
Otherwise, we can separately embed the 2-connected components of  and then combine their embeddings.
We also assume that the graph  is connected, since otherwise we can start removing edges from  and adding them to , until it becomes connected, as in Section~\ref{sec:alg}. Finally, we can assume w.l.o.g. that the input graph contains no parallel edges: otherwise, if there is a collection  of parallel edges, we can subdivide each edge  by adding a vertex  to it, and add edges connecting every consecutive pair  of vertices, for  (we identify  and ). It is easy to verify that this transformation does not increase the maximum vertex degree in , and does not increase the cost of the optimal solution.
We will use the following theorem of Hlin\v{e}n\'{y} and Salazar~\cite{HlinenyS06}.

\begin{theorem}[\cite{HlinenyS06}]\label{thm planar and edge}
Let  be any graph of maximum degree , , such that  is planar. Then we can efficiently find a drawing  of  with at most  crossings.
\end{theorem}

Our high-level idea is to decompose the graph  into blocks (see Section~\ref{sec: blocks} for the definition). For each such block , we will add an artificial edge connecting its endpoints, that will ``simulate'' the rest of the graph, . Similarly, we will add an artificial edge connecting the endpoints of  to the remaining graph, , that will simulate . In the course of such recursive decomposition, a block may end up containing a number of such artificial edges. We will then try to find drawings of each such augmented block separately. We need to argue that the total optimal solution cost in these new sub-problems does not increase by much. This is done as follows. We will have two types of blocks. The first type is blocks that have at most two artificial edges, and when one of these edges is removed from the block, we obtain a planar graph.  For such blocks, we will argue that their total solution cost is bounded by , and then use Theorem~\ref{thm planar and edge} to find their drawings. For the remaining blocks, we will show that we can augment the set  of edges, to set , with , such that for each such block ,  is planar. (Observe that now  may have to contain artificial edges). We then use Theorem~\ref{thm:main2} to find the drawing of each such block separately. We now describe the decomposition procedure in more detail.

We say that a path  in graph  is \emph{nice} if it does not contain any edge of . We will be repeatedly using the following two easy observations.

\begin{observation}\label{observation: remains planar}
Let  be any graph,  a subset of edges, such that  is planar. Let  be a block of , whose endpoints are , and assume that  contains a nice path connecting  to . Let  be the graph obtained by removing  from , and adding an artificial edge  to it. Then  is also planar.
\end{observation}

\begin{observation}\label{observation: path}
Let  be any -connected graph, and let  be any block of  with endpoints . Then there is a path  contained in .
\end{observation}

We say that a block  of  is nice, iff it does not contain any edge of . We start by iteratively removing {\bf maximal} (w.r.t. inclusion) nice blocks from , and adding each one of them to the set  of nice blocks we construct. 
Consider one such block  with endpoints . We remove block  from , and add it to the set  of nice blocks. We then add an artificial edge  both to the remaining graph , and to the block . Let  be the resulting set of these augmented nice blocks, and let  be the resulting remaining graph. From Observation~\ref{observation: remains planar}, graph  is planar. It is also easy to see that  does not contain any nice blocks, and it is -connected.
Consider now some block . Since the algorithm was repeatedly choosing maximal nice blocks,  contains a unique artificial edge, , connecting its endpoints . Moreover, from Observation~\ref{observation: path}, the graph  contains a path . Therefore, , and . The artificial edges that have been added to graph  at this step are called type-1 artificial edges, and all artificial edges that will be added throughout the rest of the algorithm are called type-2 artificial edges.
As our next step, we use Theorem~\ref{thm: block decomposition} to find a laminar block decomposition  for graph .
\iffalse
find a laminar decomposition  of the graph  into blocks, as follows. 

\begin{Definition}
A block is called \emph{minimal} iff it contains a minimum number of vertices among all blocks. 
\end{Definition}

We start with , and perform a series of \emph{block contraction steps}.
In a block contraction step, we select a minimal 
block  and add it to our family . We then remove it from  and replace it with an artificial edge connecting its endpoints. We repeat until there are no basic blocks left. Finally, we add the remaining graph  to . For consistency, we will also refer to this last component as a block, and it will serve as a root in the tree  representing this laminar family  of blocks.
For each block , we denote by  the ``un-contracted'' graph corresponding to , that is, the graph that does not contain any artificial edges, and includes all descendants of  in the tree . We denote by  the contracted graph, where each child of  is replaced by an artificial edge connecting its endpoints, and by  the set of all such graphs  for . Finally, we denote by  the graph obtained from , where we add an additional artificial edge, connecting the endpoints of , and by  the set of all such graphs . We need the following claim.

\begin{claim}\label{claim:decomposition1}
For each block , the graph  is -connected.
\end{claim}

Notice that we allow  to have parallel edges.
\begin{proof}
Assume otherwise. Let , and consider the step when  has been added to . Since  is not -connected, there is a -separator  in , that partitions  into at least two components. 
Assume first that . Then since  and  are connected by an edge in , they belong to the same component, denoted by . Let  be any of the other components. Then  is a block, that is smaller than . Therefore, it should have been added to  in the current step, instead of .

Assume now that , but . Then again  is a -separator for . Let  be the resulting component that contains , and let  be the remaining component. Then again  is a block that is smaller than , and should have been added to  in the current step.

Finally, if , then we can decompose  into two smaller blocks, and so we should have added one of them to  in the current iteration.
\ifabstract \qed \fi \end{proof}
\fi Recall that for each block , we denote by  the graph obtained from  by replacing its children with artificial edges, and  is obtained from  by adding an artificial edge connecting the endpoints of . For each , we are guaranteed that  is -connected. Intuitively, we would now like to solve each one of the blocks , for , separately. However, since we have added artificial edges to such blocks, the graph  may not be planar anymore. Of course, if we add all type- artificial edges to the set , this problem will be resolved, but then the resulting set  may become too large. We show below how to avoid this problem. We start by defining a set  of blocks, whose size will be bounded by . We show that the type-2 artificial edges belonging to all such blocks can be added to the set  without increasing its size by too much. We then show how to take care of remaining blocks.

\begin{itemize}
\item Let  be the set of blocks , for , such that  contains an edge of . Clearly, , and all the leaves of the tree  belong to  (since otherwise such a leaf would be a nice block).

\item Let  be the set of blocks , such that  has at least two children in the tree . Since all leaves of the tree  belong to , it is easy to see that .
\end{itemize}

Consider the decomposition tree , and remove all the vertices corresponding to the blocks in  and  from it. The resulting sub-graph of  is simply a collection  of disjoint paths. Moreover, . Consider some such path , and assume that , where . Let  be the largest index , such that  contains a nice path connecting its endpoints. We add 
 to . We show in the next claim that , so . 

\begin{claim}
.
\end{claim}
\begin{proof}
Assume otherwise. Then the laminar family  contains four blocks , such that for ,  is the father of  in , and   does not contain edges of . Moreover, each one of the four blocks has exactly one child, and none of these blocks contains a nice path connecting its endpoints.

Denote  and . Since  does not contain a nice path connecting its endpoints, and  do not contain edges of , all paths  in  must contain  and . Therefore, if we remove the vertices of  from , we will obtain two connected components,  and , where  and . Each one of the two components contains exactly one vertex from , and we assume w.l.o.g. that , .

We now claim that  does not contain any vertices outside of  and , that is,  is just a collection of parallel edges (or just a single edge): otherwise,  is a nice block with end-points at  and , and we have assumed that  does not contain any nice blocks. Similarly,  does not contain any vertices outside of  and . Therefore, the set of vertices of block  is . But then it is impossible that there are two additional blocks, , such that , as every pair of distinct blocks must differ in their vertices. 
\ifabstract \qed \fi \end{proof}


Finally, we let . We add to  two artificial edges: edge  connecting the endpoints of , and edge  connecting the endpoints of . For simplicity, we denote the new graph by , and the old graph by . We then add  to a new set  of sub-graphs of . We need the following claim:

\begin{claim}
Graph  is planar. Moreover, .
\end{claim}
\begin{proof}
 Since block  contained a nice path, denoted by , connecting  to , and since  did not contain edges of , from Observation~\ref{observation: remains planar},  is planar.

For the second part, from Observation~\ref{observation: path}, there is a path , connecting the endpoints of  in graph . 
Consider now the optimal embedding  of , and remove from it all edges and vertices except for those in . This drawing gives a drawing  of graph , where edge  is drawn along the image of , and edge  along the image of . The number of crossings in this drawing is at most . Finally, if the images of edges  cross multiple times in , we can un-cross them, without increasing the number of crossings between any other pair of edges. This will result in a drawing of  with at most  crossings.
\ifabstract \qed \fi \end{proof}

Since , we have that . 

Finally, let . From the above discussion, . Moreover, the total number of children of blocks in  in the tree  is also bounded by . 
Let  be any such block, and let  be its children. Recall that  is obtained from  by replacing each child  by an artificial edge , connecting the endpoints  of . For each such child , there is also a path , . Additionally, we have added an edge  connecting the endpoints  of block . We associate this edge with a path , , that is guaranteed by Observation~\ref{observation: path}. We now add the edges 
 to , and we denote by  be the resulting set of edges. We then have .

Clearly, for each ,  is planar. This is since  was planar, and all the edges that have been added to  in order to obtain , were also added to . Moreover, , since we have a collection  of edge disjoint paths associated with the edges , that are contained in , connecting the endpoints of their corresponding edges.

Let . Notice that since we have added artificial edges, it is possible that graphs  now contain parallel edges. For each such graph , we let  denote the corresponding graph with no parallel edges, that is, we replace every set of parallel edges with a single edge. Let  and  be the collections of these modified graphs, corresponding to the collections  and , respectively.

As we have already observed, each graph  can be decomposed into a planar graph plus one additional edge, and
 . Using Theorem~\ref{thm planar and edge}, we can efficiently find drawings  of graphs , with at most  crossings in total.
 We can also use Theorem~\ref{thm:main2} to find a drawing  for each graph , having at most  crossings in total. Overall, from the above discussion, for each graph , we can efficiently find a drawing  of , such that the total number of crossings in these drawings is bounded by .

\subsection{Part 2: Composition of Drawings}
In this section we show how to compose the drawings of the graphs in , to obtain the final drawing of .
We build a binary decomposition tree  corresponding to the
collection  of sub-graphs of , as follows.
The graph at the root of the tree is . The graphs at the leaves of 
are the graphs in . For every non-leaf node, the corresponding graph 
is the composition of its two child subgraphs 
and  along the unique artificial edge that belongs to both  and .
Notice that our original decomposition tree  can be turned into a binary tree whose leaves are graphs in , and we can add graphs in  to this tree one-by-one, as we merge them with the root of the tree, to obtain the final binary tree .


\begin{theorem}
Suppose that we are given the decomposition tree ,
and drawings  of graphs .
Then we can efficiently find a drawing of  with at most
 crossings.
\end{theorem}
\begin{proof}
We start by assigning weights to the edges of the graphs in the decomposition tree . Once the weights are assigned, for each graph  in the tree, the weighted degree of a vertex , denoted by , is the sum of 
the weights of the edges incident to  in . 
We assign the weights to the edges of the graphs from the top to the bottom of the tree . For the root graph , the weights of all its edges (which are non-artificial edges), are . Let  be the current graph, with its two children  and ,
that share an artificial edge .  The weights of all edges, other than the edge  remain in graphs  and  the same as in graph . The weight of the edge  is set in both graphs  and  to be:
 



 It is easy to see that if, for all vertices , , then for all vertices , for ,  as well. Therefore, the weighted degrees of all vertices in all graphs in the tree  are bounded by .
Finally, we assign weights to edges of graphs , as follows. Let  be the graph corresponding to . For each set  of parallel edges in , the weight of the corresponding edge in  is the sum of the weights of the edges  in . The weights of all other edges are identical in  and .

We now define the weighted cost of a drawing  of any edge-weighted graph , , as follows.
The weighted cost of a crossing of two edges of weights  and  is . 
The cost of the drawing is the sum of weighted costs of all crossings. 

Notice that for each graph , the drawing  of  induces a drawing  of the corresponding graph , such that the weighted cost of  is bounded by that of . 
Since the weighted degrees of vertices in all graphs in  are bounded by , we have

We now combine all drawings of graphs in  as follows. We proceed from the bottom 
to the top of the tree . At each node , we combine the
two drawings  and  of its two children  and 
into a drawing  of  so that 

Finally, we obtain a drawing  of  with


We now show how to combine the drawings  and  of graphs  and . Let  be the unique artificial edge shared by  and .
Without loss of generality, we assume that .

We note that we can assume that the following properties hold (for each ):
\begin{itemize} 
\item the vertex  lies on the external boundary of the drawing ;
\item there is a point  on the drawing of the edge 
 in , such that the segment of the drawing of  
between  and  lies on the external boundary of the drawing .
\end{itemize}

If these properties do not hold, we transform each drawing  as follows.
For convenience, assume that the drawing  is on the 2-sphere.
We take a point  on the curve corresponding to the edge  in the drawing , so that there
are no crossing points on the segment of 
between  and .
Then we take a point  that lies on the same face of   as  and .
Finally, we perform a stereographic projection from 
and obtain the desired drawing . 
Since  and  lie on the face bounding  in , it follows that they both lie on the outer face in .

\iffalse
If these properties do not hold, we transform each drawing  as follows.
We take a point  on the drawing  of  so that there
are no crossing points on the segment of 
between  and . Then we take a point  close enough
to  so that the line segment connecting  and  does not cross
the drawing  of . Finally, we inverse each drawing 
with respect to point  and obtain the desired drawing . 
Since the inversion sends point  to infinity, the map of the line segment 
is a ray from the image of  that does not cross the drawing of  in .
Therefore, the image of  lies on the external boundary of . The image of the
segment of the edge  connecting  and the image of 
does not cross the drawing of , and, therefore, also lies on the external boundary
of .
\fi

We superimpose drawings  and  so that drawings of  and 
do not overlap and points , , , and 
lie on the external boundary of the drawing. We then connect points
 and  with a curve  and points  and 
with a curve  so that curves  and 
do not cross each other and do not cross the drawings of  and  (see Figure~\ref{fig:composition}).
Now, we erase the drawings of segments of 
between points  and . Let  be the concatenation
of remaining pieces of  and 
and . The curve  connects 
and . Finally, we ``contract'' curves  and :
we move points  and  along the curves  and ,
until they reach  and . We route each edge  incident to 
(respectively ) in : first along the curve  (respectively )
and then along the original drawing  of . (If edges parallel to  belong to , we re-route them in the same way: first along , then along their original drawing in , and finally along ).
We obtain an embedding  of  (curves ,  and
the embeddings of the edge  are not parts of ).
Figure \ref{fig:composition} depicts an example of the above composition step.

\begin{figure}
\begin{center}
\scalebox{0.80}{\includegraphics{composition}}
\caption{Obtaining a drawing for  by composing the drawings for  and .\label{fig:composition}}
\end{center}
\end{figure}

Let us compute the cost of drawing .
Since  does not cross the drawings  and ,
we do not introduce any new crossings when we contract .
For every crossing of an edge  with  or ,
we introduce crossings between all edges incident to  in  and  .
The total weighted cost of these crossings is .
It is equal to ,
the cost of the crossing between  and . Therefore, the total weighted 
cost of the drawing does not increase, that is,

\ifabstract \qed \fi \end{proof}

\fi
\iffull
\section{Algorithms for bounded-genus graphs}\label{sec:genus}
In this section we present the proof of Theorem \ref{thm: bounded genus}.
\iffalse
\noindent\textbf{Theorem~\ref{thm: bounded genus}}\textit{
Let  be a graph embedded in an orientable surface of genus .
Then we can efficiently find a drawing of  into the plane, with at most  crossings.
Moreover, for any , there is an efficient -approximation for \MCN on bounded degree graphs embedded into a genus- surface.}
\fi

\begin{remark}
Note that for any \emph{fixed} , given a graph  of genus , we can find an embedding into a surface of genus  in linear time \cite{Mohar99,KawarabayashiMR08}.
However, in our algorithm we do not assume that the input graph is embedded into a surface of \emph{minimum} genus.
Therefore, if the input graph  has genus , but we are only given an embedding into a surface of genus , the approximation guarantee of the drawing produced by our algorithm will depend on .
This is in particular interesting when the genus of the graph  is super-constant, in which case computing an embedding of minimum genus becomes NP-hard \cite{genus-np-complete}.
\end{remark}


For any graph , and an embedding  of  into a surface of genus at least 1, the \emph{nonseparating dual edge-width} of , denoted by , is the length of the shortest surface-nonseparating cycle in the dual of , w.r.t. the embedding .
We also write  when  is clear form the context.
We will use the following algorithmic result by Hlin\v{e}n\'{y} and Chimani~\cite{crossing_genus}.

\begin{theorem}[Hlin\v{e}n\'{y} \& Chimani \cite{crossing_genus}]\label{thm:genus_soda}
Let  be a graph embedded in an orientable surface of genus , with .
Then there is an efficient algorithm that computes a drawing of  in the plane with at most  crossings.
\end{theorem}

Let  be a graph, and let  be an embedding of  into an orientable surface  of genus .
We begin by computing an integer , and a sequence of graphs .
For each graph  we also compute a drawing  of  into a surface of genus .
Initially, we set , and .
For each , if , then we set , and we terminate the sequence .
Otherwise, if , then 
we first compute a shortest surface-nonseparating cycle  in the dual of  w.r.t. the embedding . Such a cycle can be found in time  using the
algorithm of Cabello and Chambers \cite{CabelloC07}.
We construct  by removing from  all edges whose duals are in .
We also construct an embedding of  by cutting the surface into which  is embedded along the cycle .
This gives us an embedding  of .
As observed in \cite{crossing_genus}, the graph  is a spanning subgraph of , and the embedding  is into a surface of genus .

Let us define

We have


If , then the graph  is drawn into a surface of genus , and therefore  is planar.
Otherwise, if , then we have a drawing  of the graph  into a surface of genus , and with

This means in particular that we can run the algorithm from Theorem \ref{thm:genus_soda} to obtain a drawing  of  into the plane with at most  crossings.
Define the set  of edges to contain all edges participating in crossings in .
We have


Let .
Observe that the graph  is planar.
The assertion of Theorem~\ref{thm: bounded genus} follows trivially if the graph  is planar, so we may assume that .
By \eqref{eq:genus1} and \eqref{eq:genus2} we therefore have

Running the algorithm from Theorem \ref{thm:main} with the planarizing set , we obtain a drawing of  into the plane with at most  crossings.

To obtain an -approximation for bounded-degree graphs, 
run the above algorithm, and the algorithm of Even et al.~\cite{EvenGS02}, and output the drawing with fewer crossings.
\hfill \ensuremath{\Box}


\fi

\iffull
\section{Proof of Theorem~\ref{thm: Even: extension}}
Recall that the algorithm of Even et al.~\cite{EvenGS02} finds a drawing of a bounded degree graph with at most 
crossings. We first show that this algorithm can be extended to arbitrary graphs
to produce drawings with at most  crossings, where  is the maximum vertex degree in .
We then note that by using the approximation algorithm of Arora et al.~\cite{ARV} for {\sf Balanced Separator}
instead of the algorithm of Leighton and Rao~\cite{LR}, this guarantee can be improved to .

\begin{lemma}  
Suppose that there is a polynomial time algorithm , that, for any -vertex graph  with vertex degrees at most 3, finds a drawing of  with at most  crossings. 
Then there is a polynomial time algorithm that finds a drawing of any graph  with at most  crossings, where  is the maximum vertex degree in .
\end{lemma}
\begin{proof}
The algorithm first constructs an auxiliary graph  with maximum vertex degree . Informally,  is the graph obtained from  by replacing every vertex   with a path  of length  (e.g., if  is a -regular graph then  is the replacement product of  and the path of length ). Formally, the vertices of  are pairs , where ,  and  is incident on . The edges of  consist of two subsets. First, for every edge , we connect the vertices  and  of  with an edge . We call such edges ``type 1 edges''. Additionally, for each , if   is the list of all edges incident
on  (in an arbitrary order), then we connect every consecutive pair  of vertices, for , with a type-2 edge. Let  denote the resulting path formed by these edges. This completes the description of . Note that  has at most  vertices, and every vertex of  has degree at most 3. Also note that if we contract every path   in , for , into a vertex,  we obtain the graph .

We now bound the crossing number of . Observe that we can obtain a drawing  of  from any drawing  of  as follows. We put each vertex  on the drawing of the edge  very close to the drawing of . We draw each type-1 edge  of   along the segment of the drawing of  in , connecting the images of  and . We draw type-2 edges on the line segments connecting their endpoints. We now bound the number of crossings in this drawing. Notice that there are no crossings between the type-1 and the type-2 edges. The number of crossings between pairs of type-1 edges is bounded by the total number of crossings in . Finally, in order to bound the number of crossings between pairs of type-2 edges, we notice that if , then the edges of  and  do not cross. Any pair of edges on path  may cross at most once, since any pair of line segments crosses at most once. Therefore, 
there are at most  crossings among the edges of the path  for every vertex . Overall, 


and 



Our algorithm runs  on  and finds a drawing  of  with at most 
 crossings.
We now show how to transform the resulting drawing  of  into a drawing  of . Informally, this is done by
contracting the drawing of every path  into a point.
More precisely, we draw every vertex  at the point  (where  is the first edge in the incidence list for ).
For each path , for , we construct  auxiliary curves , \dots, , where for each , curve  connects the images of  and  in ,
and no pair of such curves cross (though curves that correspond to different vertices of  are allowed to cross).
This is done as follows. First, we draw each curve   along the image of the path  in ,
following the segment that connects the images of  and ; in a neighborhood of each vertex  of , we draw the curve   on the side of  opposite to the side where the edge  enters  (thus  does not cross the drawing of  near the point 
). We make sure that all curves  are drawn in general position. 
Next, if any of the resulting curves cross themselves, or cross each other, we perform uncrossing. Since all these curves start at the same point -- the image of  -- we can uncross them so that the final curves do not cross each other, and do not cross themselves.

We are now ready to describe the drawing of every edge . The drawing of  is a concatenation of three curves: 
, , and . The second segment of this drawing is called a type-1 segment, while the first and the third segments are called type-2 segments.
Note that it is possible that some pairs of edges have more than one crossing, adjacent edges cross each other and some edges have self crossings in this drawing; we will fix that later.
We now bound the number of crossings .
\begin{itemize}
\item The number of crossings between all pairs of type-1 segments is bounded by .
\item The number of crossings between any pair of type-2 segments  and  (where ), is bounded by the number of crossings between
paths  and . Since every crossing between the paths  and  may pay for crossings of at most  such pairs of curves,
the total number of such crossings is at most .
\item Similarly, the number of crossings between a type-2 curve  and a type-1 curve  (for an arbitrary edge  of )
is at most the number of crossings between  and  in . So the total number of such crossings is 
at most .
\end{itemize}
We conclude that the number of crossings is
.
Finally, the algorithm uncrosses drawings of edges that cross more than once, crossing pairs of adjacent edges, and edges that cross themselves.
During this step the number of crossings can only go down.
\ifabstract \qed \fi \end{proof}

The algorithm of Even et al.~\cite{EvenGS02} uses an algorithm for \textsf{Balanced Separator} as a subroutine.
We need a few definitions. Suppose we are given a graph  with non-negative vertex weights . For each subset  of vertices, let  denote the total weight of vertices in . We say that a cut  is -balanced w.r.t. the weights , iff .
The cost of the cut is .
Even et al. prove the following theorem.

\begin{theorem}\label{lem:ARV-corollary}
Suppose that there is some function , and an efficient algorithm for Balanced Separator, with the following property. Given any -vertex, vertex-weighted graph  and values , , such that every sub-graph of  has a -balanced separator of size at most , the algorithm returns a -balanced cut of , whose cost is  (the constant in the -notation may depend on ).   Then there is an efficient algorithm to find a drawing of any bounded degree graph  with  crossings.
\end{theorem}
Even et al. use the algorithm of Leighton and Rao~\cite{LR} that gives an algorithm for balanced separators with approximation factor . We note that the results of Arora, Rao and Vazirani~\cite{ARV} gives an improved algorithm, with , and thus we can efficiently find a drawing of a bounded degree graph with at most  crossings. 
\begin{theorem}[Arora et al. \cite{ARV}]\label{thm:ARV-alg}
For every constant  and some  (that depends on ), there is a bi-criteria approximation algorithm for the Balanced Cut Problem with the following approximation guarantee. Given a graph  and a set of vertex weights , the algorithm finds a -balanced cut w.r.t.  of cost at most , where  is the cost of the optimal -balanced cut w.r.t. . (The constant in the -notation depends on .)
\end{theorem}
We point out that the algorithm in Theorem~\ref{thm:ARV-alg} does not directly satisfy the requirements of Theorem~\ref{lem:ARV-corollary}, since it finds a -balanced cut only for \textit{some}  that depends on , while we are required to produce a -balanced cut.
For completeness, we show that the algorithm of~\cite{ARV} can still be used to obtain an algorithm for balanced separator as required in the statement of Theorem~\ref{lem:ARV-corollary}, for .

We iteratively apply the algorithm of Arora et al. We start with  and . We first find a -balanced cut in  
(where  is the constant guaranteed by Theorem~~\ref{thm:ARV-alg}). If the larger side of the cut contains at most  of the total weight,
then the cut is -balanced and we are done. Otherwise, we let  be the larger side of the cut (w.r.t. weights ),
and we add the smaller side of the cut to . Then we iteratively apply the ARV-algorithm to , update sets  and , and repeat.
We stop when  contains at most  of the total weight of . Note that after each iteration the weight of  decreases 
by at least a factor , since the algorithm of \cite{ARV} finds a -balanced cut. Therefore, the algorithm stops in at most 
 steps. Observe that after each iteration, , since before each iteration
 and we let  to be the larger of the two sides of the cut. Hence, 
when the algorithm terminates. That is, the cut  is -balanced. 

In every iteration, we cut at most  edges, and the total number of iterations is at most .
Thus the cost of the cut is .

\fi
\end{document}