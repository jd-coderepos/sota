


\pdfoutput=1  \documentclass[11pt]{article}
\usepackage{url}
\usepackage{epsf}
\usepackage{graphicx}
\usepackage{amsmath}
\usepackage{fullpage}




\newtheorem{lemma}{Lemma}
\newtheorem{theorem}{Theorem}

\newcommand{\qed}{\rule{0.5em}{1.5ex}}
\newcommand{\fqed}{{\hfill~\qed}}
\newenvironment{proof}{{\noindent \bf Proof.}}
                      {{\hfill \fqed} \vspace{1em}}






\newtheorem{proposition}{Proposition}
\newtheorem{property}[theorem]{Property}
\newtheorem{corollary}[theorem]{Corollary}


\newcommand{\e}{\varepsilon}
\newcommand{\R}{\mathcal R}
\newcommand{\Pa}{{\mathcal P}}
\newcommand{\squeezelist}{\setlength{\itemsep}{0pt}}

{\makeatletter
 \gdef\xxxmark{\expandafter\ifx\csname @mpargs\endcsname\relax \expandafter\ifx\csname @captype\endcsname\relax \marginpar{xxx}\else
       xxx \fi
   \else
     xxx \fi}
 \gdef\xxx{\@ifnextchar[\xxx@lab\xxx@nolab}
 \long\gdef\xxx@lab[#1]#2{{\bf [\xxxmark #2 ---{\sc #1}]}}
 \long\gdef\xxx@nolab#1{{\bf [\xxxmark #1]}}
\gdef\turnoffxxx{\long\gdef\xxx@lab[##1]##2{}\long\gdef\xxx@nolab##1{}}}

\title{-Angle Yao Graphs are Spanners}
\author{
Prosenjit Bose
    \thanks{School of Computer Science, Carleton University, Ottawa, Canada. \protect\url{jit@scs.carleton.ca}.
    Supported by NSERC.}
\and
Mirela Damian
    \thanks{Department of Computer Science, Villanova University, Villanova, USA.
    \protect\url{mirela.damian@villanova.edu}. Supported by NSF grant CCF-0728909.}
\and
Karim Dou\"ieb
    \thanks{School of Computer Science, Carleton University, Ottawa, Canada. \protect\url{kdouieb@ulb.ac.be}.
    Supported by NSERC.}
\and
Joseph O'Rourke
    \thanks{Department of Computer Science, Smith College, Northampton, USA. \protect\url{orourke@cs.smith.edu.}
    Supported by NSERC.}
\and
Ben Seamone
    \thanks{School of Mathematics and Statistics, Carleton University, Ottawa, Canada. \protect\url{bseamone@connect.carleton.ca}.}
\and
Michiel Smid
    \thanks{School of Computer Science, Carleton University, Ottawa, Canada. \protect\url{michiel@scs.carleton.ca.}
    Supported by NSERC.}
\and
Stefanie Wuhrer
    \thanks{NRC Institute for Information Technology, Ottawa, Canada. \protect\url{Stefanie.Wuhrer@nrc-cnrc.gc.ca}.}
}
\date{}

\begin{document}

\maketitle

\begin{abstract}
We show that the Yao graph  in the  metric is a spanner with
stretch factor . Enroute to this, we also show that
the Yao graph  in the  metric is a planar spanner
with stretch factor .
\end{abstract}

\section{Introduction}
Let  be a finite set of points in the plane and let  be
the complete Euclidean graph on . We will refer to the points in
 as \emph{nodes}, to distinguish them from other points in the
plane.
The \emph{Yao graph}~\cite{Yao82} with an integer parameter ,
denoted , is defined as follows. At each node , any 
equally-separated rays originating at  define  cones.
In each cone, pick a shortest edge , if there is one, and add to 
the directed edge . Ties are broken arbitrarily.
Most of the time we ignore the direction of an edge ; we refer
to the directed version  of  only when its
origin () is important and unclear from the context.
We will distinguish between , the Yao graph in the Euclidean 
metric, and , the Yao graph in the  metric. Unlike
 however, in constructing  ties are broken by always
selecting the most counterclockwise edge; the reason for this choice
will become clear in Section~\ref{sec:y4inf}.

For a given subgraph  and a fixed ,  is
called a -\emph{spanner} for  if, for any two nodes
, the shortest path in  from  to  is no longer than
 times the length of .
The value  is called the \emph{dilation} or the
\emph{stretch factor} of . If  is constant, then  is called a
\emph{length spanner}, or simply a \emph{spanner}.

The class of graphs  has been much studied. Bose et al.~\cite{bmnsz-agbsp-03}
showed that, for ,  is a spanner with stretch factor
. In the appendix, we improve the stretch factor and show that, in fact, 
 is a spanner for any .
Recently, Molla~\cite{M09} showed that  and  are 
not spanners, and that  is a spanner with stretch factor 
, for the special case when the nodes in  are 
in convex position (see also ~\cite{DMP09}). 
The authors conjectured that  is a spanner for arbitrary point 
sets.
In this paper, we settle their conjecture and prove that  is a
spanner with stretch factor .

The paper is organized as follows.
In Section~\ref{sec:y4inf}, we prove that the graph  is a
spanner with stretch factor .
In Section~\ref{secY4L2}, we prove, in a sequence of Lemmas, several
properties for the graph . Finally, in Section~\ref{sec:Yinf+Y4},
we use the properties of Section~\ref{secY4L2} to prove that for every
edge  in , there exists a path between  and  in
, whose length is not much more than the Euclidean distance between
 and . By combining this with the result of Section~\ref{sec:y4inf},
it follows that  is a spanner.


\section{: in the  Metric}
\label{sec:y4inf}
In this section we focus on , which has a nicer structure
compared to . First we prove that  is planar. Then we
use this property to show that  is an -spanner.
To be more precise, we prove that for any two nodes  and ,
the graph  contains a path between  and  whose
length (in the -metric) is at most .

We need a few definitions.
We say that two edges  and  \emph{properly cross} (or \emph{cross},
for short) if they share a point other than an endpoint (, ,  or );
we say that  and  \emph{intersect} if they share a point (either an
interior point or an endpoint).
\begin{figure}[htbp]
\centering
\includegraphics[width=0.6\linewidth]{y4infplanar}
\caption{(a) Definitions: ,  and . (b) Lemma~\ref{lem:y4infplanar}:  and  cannot cross.}
\label{fig:y4infplanar}
\end{figure}
Let , ,  and  be
the four quadrants at , as in Figure~\ref{fig:y4infplanar}a.
Let  be the path that starts at point  and follows the directed
Yao edges in quadrant .
Let  be the subpath of  that starts at  and ends at .
Let  be the  distance between  and .
Let  denote a shortest path in  between  and .
Let  denote the open square with corner  whose boundary
contains , and let  denote the boundary of .
These definitions are illustrated in Figure~\ref{fig:y4infplanar}a.
For a node , let  denote the -coordinate of  and
 denote the -coordinate of .

\begin{lemma}
 is planar.
\label{lem:y4infplanar}
\end{lemma}
\begin{proof}
The proof is by contradiction. Assume the opposite. Then there are
two edges 
that cross each other. Since ,
 must be empty of nodes in , and similarly for . Let
 be the intersection point between  and .
Then , meaning that  and  must
overlap. However, neither square may contain  or . It
follows that  and  coincide, meaning that  and 
lie on  (see Figure~\ref{fig:y4infplanar}b). Since 
intersects ,  and  must lie on opposite sides of .  Thus either  or
 lies counterclockwise from . Assume without loss of generality that
 lies counterclockwise from ; the other case is identical.
Because  coincides with , we have that
. In this case however,  would break
the tie between  and  by selecting the most counterclockwise edge,
which is . This contradicts the fact that
.
\end{proof}

\noindent
It can be easily shown that each face of  is either a triangle
or a quadrilateral (except for the outer face). We skip this proof however,
since we do not make use of this property in this paper.

\begin{theorem}
 is an -spanner.
\label{thm:y4infspanner}
\end{theorem}
\begin{proof}
We show that, for any pair of points , .
The proof is by induction on the pairwise distance between the points in .
Assume without loss of generality that , and .
Consider the case in which  is a closest pair of points in  (the base case
for our induction). If , then .
Otherwise, there must be , with
. But then 
(see Figure~\ref{fig:y4inf}a), a contradiction.

\begin{figure}[htp]
\centering
\includegraphics[width=\linewidth]{y4infcases}
\caption{(a) Base case. (b)  empty (c)  non-empty, 
(d)  non-empty, ,  above 
(e)  non-empty, ,  below .}
\label{fig:y4inf}
\end{figure}


Assume now that the inductive hypothesis holds for all pairs of points
closer than . If , then 
and the proof is finished. If , then the square  must be nonempty.

Let  be the rectangle  as in Figure~\ref{fig:y4inf}b, where  and
 are parallel to the diagonals of .
If  is nonempty, then we can use induction to prove that 
as follows. Pick  arbitrary. Then ,
and by the inductive hypothesis  is a path in  no
longer than ; here  represents the 
concatenation operator.
Assume now that  is empty.
Let  be at the intersection between the line supporting  and the
vertical line through  (see Figure~\ref{fig:y4inf}b).
We discuss two cases, depending on whether
 is empty of points or not.

\paragraph{Case 1: }  is empty of points. Let . We show that
 cannot contain an edge crossing . Assume the opposite, and let
 cross . Since  is empty,  must lie above
 and  below ,  therefore
,
contradicting the fact that . It follows that
 and  must meet in a point  (see Figure~\ref{fig:y4inf}b).
Now note that
.
Thus we have that

\paragraph{Case 2: }  is nonempty.
In this case, we seek a short path from  to  that does not
cross to the underside of . This is to avoid oscillating paths
that cross  arbitrarily many times. Let  be the rightmost point
that lies inside . Arguments similar to the ones used in Case~1 show that
 cannot cross  and therefore it must meet  in a point . Then
 is a path in  of length

The term  in the inequality above represents the fact that
.
Consider first the simpler situation in which  meets  in a
point  (see Figure~\ref{fig:y4inf}c).
Let  be the subpath of  extending between  and .
Then  is a path in  from  to ,
therefore


Consider now the case when  does not intersect . We argue that, in this
case,  may not be empty. Assume the opposite. Then
no edge  may cross . This is because,
for any such edge, , contradicting
. This implies that  intersects , again
a contradiction to our assumption.

We have established that  is nonempty. Let .
The fact that  does not intersect  implies that 
lies to the left of .
The fact that  is the rightmost point in  implies that
 lies outside  (see Figure~\ref{fig:y4inf}d). It also implies that
 shares no points with . This along with
arguments similar to the ones used in case 1 show that
 and  meet in a point .
Thus we have found a path

extending from  to  in . If , then
, and the path  has length

In the above, we used the fact that
.
Suppose now that

In this case, it is unclear whether the path  defined by~(\ref{eq:pabinf}) is short,
since  can be arbitrarily long compared to . Let  be the clockwise
neighbor of  along the path  ( and  may coincide).
Then  lies below , and either , or  (or both).
\begin{enumerate}
\item
If  lies above , or at the same level as  (i.e., , as in Figure~\ref{fig:y4inf}d), then

Since  and  is in the same quadrant of  as , we have
. This along with inequalities~(\ref{eq:rdlen})
and~(\ref{eq:yrd}) implies , which in turn
implies , and so
.
Then inequality~(\ref{eq:Pab}) applies here
as well, showing that .

\item
If  lies below  (as in Figure~\ref{fig:y4inf}e), then

Assume first that , or . In either case,

This along with inequality~(\ref{eq:rde}) shows that .
Substituting this upper bound in~(\ref{eq:pabinf}), we get

Assume now that , and . Then
 cannot go above  (otherwise ,
contradicting ). This along with the fact  implies
that  intersects  in a point . Redefine

Then  is a path in  from  to  of length


\end{enumerate}
We have established that .
This concludes the proof.
\end{proof}

\noindent
This theorem will be employed in Section~\ref{sec:Yinf+Y4}.

\section{: in the  Metric}
\label{secY4L2}
In this section we establish basic properties of . The ultimate goal of this section is
to show that, if two edges in  cross, there is a short path between their endpoints (Lemma~\ref{lem:recross}). We
begin with a few definitions.

Let  denote the infinite quadrant with origin at  that contains .
For a pair of nodes , define recursively a directed path
 from  to  in  as follows.
If , then .
If , there must exist  that
lies in . In this case, define

Recall that  represents the concatenation operator.
This definition is illustrated in Figure~\ref{fig:defs}a.
Fischer et al.~\cite{FLZ98} show that  is well defined and
lies entirely inside the square centered at  whose boundary contains .


\begin{figure}[htbp]
\centering
\includegraphics[width=0.6\linewidth]{defs}
\caption{Definitions. (a)  and . (b) .}
\label{fig:defs}
\end{figure}


For any node , let  denote the open disk centered at
 of radius , and let  denote the boundary of
. Let .
For any path  and any pair of nodes  and  on , let 
denote the subpath of  that starts at  and ends at .
Let  denote the closed rectangle with diagonal .

For a fixed pair of nodes , define a path
 as follows. Let  be the first node
along  that is not strictly interior to
. Then  is the subpath of 
that extends between  and . In other words,  is
the path that follows the  edges pointing towards ,  truncated as soon
as it leaves the rectangle with diagonal , or as it reaches . Formally,

This definition is illustrated in Figure~\ref{fig:defs}b.

Our proofs will make use of the following two propositions.

\begin{proposition}
The sum of the lengths of crossing diagonals of a nondegenerate
(necessarily convex) quadrilateral  is strictly greater than the sum of
the lengths of either pair of opposite sides:

\label{fact:quad}
\end{proposition}
This can be proved by partitioning
the diagonals into two pieces each at their intersection
point, and then applying the triangle inequality twice.

\begin{proposition}
For any triangle , the following inequalities hold:

\label{fact:tri}
\end{proposition}
This proposition follows immediately from the Law of Cosines applied
to triangle .

\begin{lemma}
For each pair of nodes ,

Furthermore, each edge of  is no longer than .
\label{lem:PR}
\end{lemma}
\begin{proof}
Let  be one of the two corners of , other than  and .
Let  be the last edge
on , which necessarily intersects
 (note that it is possible that ). Refer to
Figure~\ref{fig:defs}b.
Then , otherwise  could not be in .
Since  lies in the rectangle with diagonal , we have that
,
and similarly for each edge on .
This establishes the latter claim of the lemma. For the first claim of
the lemma, let

Since , we have that .
Since  lies entirely inside  and consists of edges pointing
towards , we have that  is an -monotone path. It follows that
. We now show that
, thus establishing the first
claim of the lemma.

Let  and . Then the inequality
 can be written as
, which is equivalent to
. This latter inequality obviously holds,
completing the proof of the lemma.
\end{proof}

\begin{lemma}
Let  be four disjoint nodes such that
, 
and , for some . Then  and 
cannot cross each other.
\label{lem:same.quadrant}
\end{lemma}
\begin{proof}
We may assume without loss of generality that  and  is to the
left of . The proof is by contradiction. Assume that  and  cross each
other. Let  be the intersection point between  and  (see Figure~\ref{fig:ratio}a).
Since ,
it follows that  and . Thus , because
otherwise,  cannot be in .
By
Proposition~\ref{fact:quad}
applied to the quadrilateral ,

This along with the fact that  implies that
, contradicting the fact that .
\end{proof}

The next four lemmas (\ref{lem:quad}--\ref{lem:recross})
each concern a pair of crossing  edges,
culminating (in Lemma~\ref{lem:recross}) in the conclusion
that there is a short path in  between a pair of endpoints
of those edges.

\begin{lemma}
Let , ,  and  be four disjoint nodes in  such that
, and
 crosses . Then the following are true: (i) the ratio between the
shortest side and the longer diagonal of the quadrilateral  is
no greater than , and (ii) the
shortest side of the quadrilateral  is strictly shorter
than either diagonal.
\label{lem:quad}
\end{lemma}
\begin{proof}
The first part of the lemma is a well-known fact that holds for
any quadrilateral (see~\cite{Quad51}, for instance).
For the second part of the lemma, let  be the shorter of the
diagonals of , and assume without loss of generality
that .
Imagine two disks  and ,
as in Figure~\ref{fig:ratio}b. If either  or  belongs to
,
then the lemma follows: a shortest quadrilateral edge is shorter
than .

\begin{figure}[htbp]
\centering
\includegraphics[width=0.85\linewidth]{ratio}
\caption{(a) Lemma~\ref{lem:same.quadrant}. (b) Lemma~\ref{lem:quad}:  (c) Lemma~\ref{lem:quad}: .}
\label{fig:ratio}
\end{figure}


So suppose that neither  nor  lies in . In
this case, we use the fact that  crosses  to show that
 cannot be an edge in .
Define the following regions (see Figure~\ref{fig:ratio}b):

If the node  is not inside any of the regions , for , then
the nodes  and  are in the same quadrant of  as . In this case, note that
either  or , which implies that either
 or  is strictly smaller than . These together show that
.

So assume that  is in  for some . In this situation,
the node  must lie in the region , with  (with the
understanding that ),
because otherwise, (i)  and  are in the same quadrant of  and
 or (ii)  and  are in the same quadrant of  and
. Either case contradicts the fact .
Consider now the case  and ; the other cases are treated similarly.
Let  and  be the intersection points between  and the
vertical line through . Similarly, let  and  be the
intersection points between  and the vertical line through 
(see Figure~\ref{fig:ratio}c).
Since  is a diameter of , we have that  and similarly
. Also note that ,
meaning that . Similarly, ,
meaning that . These along with the fact that at least one of  and
 is in the same quadrant for  as , imply that
. This completes the proof.
\end{proof}

\begin{lemma}
Let  be four distinct nodes in , with , such that
\begin{enumerate}
\item [(a)]  and
 are two edges in  that cross each other.
\item [(b)]  is a shortest side of the quadrilateral . \end{enumerate}
Then  and  have a nonempty intersection.
\label{lem:ec1}
\end{lemma}
\begin{proof}
The proof consists of two parts showing that the following claims hold:
(i)  and (ii)  does not cross .

Before we prove these two claims, let us argue that they are sufficient
to prove the lemma. Lemma~\ref{lem:same.quadrant} and (i) imply that
 cannot cross . As a result,
 intersects the left side of the rectangle .
Consider the last edge  of the path
. If this edge crosses the right side of ,
then (ii) implies that  is in the wedge bounded by  and the upwards
vertical ray starting at ; this implies that , contradicting
the fact that  is an edge in . Therefore,
 intersects the bottom side of , and the lemma
follows (see Figure~\ref{fig:PRcross}b).

To prove the first claim (i), we observe that the assumptions in the lemma imply
that . Therefore, it suffices to prove that 
is not in . Assume to the contrary that . Since
, it must be that ; otherwise, ,
which implies , contradicting the fact that
. Let  and  be the
intersection points between  and , where  is to
the left of .
Since , we have . This,
together with the fact that  and  are in the same quadrant ,
contradicts the assumption that  is an edge in .
This completes the proof of claim (i).

Next we prove claim (ii) by contradiction.
Thus, we assume that there is an edge  on the path
 that crosses . Then necessarily
 and .
If , then , meaning that
, a contradiction to the fact that .
Thus, it must be that , as in Figure~\ref{fig:PRcross}a.
This implies that , because .

\begin{figure}[htbp]
\centering
\includegraphics[width=0.65\linewidth]{PRcross}
\caption{(a) Lemma~\ref{lem:ec1}:  cannot cross .}
\label{fig:PRcross}
\end{figure}


The contradiction to our assumption that  crosses 
will be obtained by proving that . Indeed, this inequality
contradicts the fact that .

Let  be the distance from  to the horizontal line through .
Our intermediate goal is to show that

We claim that . Indeed, if this is not the case,
then , contradicting the fact that  is an
edge in . By a similar argument, and using the fact that
 is an edge in , we obtain the inequality
.
We now consider two cases, depending on the relative lengths of  and .

\begin{enumerate}

\item

Assume first that .
If , then , contradicting
the fact that  is an edge in  (recall that 
and  are in the same quadrant of ). Therefore, we have
.
Thus far we have established that three angles of the convex quadrilateral
 are acute. It follows that the fourth one () is obtuse.
Proposition~\ref{fact:tri}
applied to  tells us that

where the latter inequality follows from the assumption that  is a
shortest side of  (and, therefore, ).
Thus, we have that . This along with the fact that
 implies inequality~(\ref{eq:d}).

\item
Assume now that . Let  be the intersection
point between  and the horizontal line through 
(refer to Figure~\ref{fig:PRcross}a). Note that
 and 
(these two angles sum to ).
This along with
Proposition~\ref{fact:tri}
applied to triangle  shows that

Similarly,
Proposition~\ref{fact:tri}
applied to triangle  shows that

The two inequalities above along with our assumption that 
imply that , which in turn implies that ,
because . Since  is below  (otherwise, ,
contradicting the fact that  is an edge in ),
we have . It follows that . \end{enumerate}

Finally we derive a contradiction using the now established inequality~(\ref{eq:d}).
Let  be the orthogonal projection of  onto the vertical line through
 (thus ).
Note that , because .
By
Proposition~\ref{fact:tri}
applied to , we have

Since  and  are in the same quadrant of , and since
, we have that .
This along with the inequality above and~(\ref{eq:d}) implies that
. By
Proposition~\ref{fact:tri}
applied to ,
we have
. It
follows that , contradicting our assumption that
.
\end{proof}

\begin{lemma}
Let  be four distinct nodes in , with , such that
\begin{enumerate}
\item [(a)]  and  are two edges
in  that cross each other.
\item [(b)]  is a shortest side of the quadrilateral .
\end{enumerate}
Then  does not cross .
\label{lem:c1}
\end{lemma}
\begin{proof}
\begin{figure}[htbp]
\centering
\includegraphics[width=0.6\linewidth]{quadsola}
\caption{Lemma~\ref{lem:c1}: (a)~ does not cross .
(b)~If  is not the shortest side of , the lemma conclusion might not hold.}
\label{fig:c1}
\end{figure}
We first show that . Assume the opposite. Since 
and , we have that . This implies that
, which along with the fact that  contradict
the fact that . Also note that ,
since in that case  and  could not intersect. In the following we
discuss the case ; the case  is symmetric.

A first observation is that  must lie below ; otherwise 
(since ), which would contradict the fact that
. We now prove by contradiction that there
is no edge in  crossing .
Assume the contrary, and let 
be such an edge. Then necessarily  and
. Note that  cannot lie below ;
otherwise 
(since ), which would contradict the fact that
.
Also  must lie outside , otherwise
 could not be in . These together show that 
sits to the right of . See Figure~\ref{fig:c1}(a).
Then the following inequalities regarding the quadrilateral 
must hold:
\begin{itemize}
\item[(i)] , due to the fact that
.
\item[(ii)]  ( if  and  coincide).
If  and  are distinct, the inequality  follows from
the fact that  (since  is outside ),
and
Proposition~\ref{fact:quad}
applied to the quadrilateral :

\end{itemize}
Inequalities (i) and (ii) show that  and  are longer
than sides of the quadrilateral , and so they must be longer
than the shortest side of , which by assumption~(b) of the lemma
is :
 (this latter inequality follows
from the fact that ). Also note that ,
since  and  lies in the same quadrant
of  as . The fact that both diagonals of  are in 
enables us to apply Lemma~\ref{lem:quad}(ii) to conclude that
 is not a shortest side of the quadrilateral . Thus
 is a shortest side of the quadrilateral , and we can use
Lemma~\ref{lem:quad}(ii) to claim that

This contradicts our assumption that .
\end{proof}

\noindent
Figure~\ref{fig:c1}(b) shows that the claim of the lemma might be false without
assumption~(b).
The next lemma relies on all of Lemmas~\ref{lem:PR}--\ref{lem:c1}.

\begin{lemma}
Let  be four distinct nodes such that
 crosses
, and let
 be a shortest side of the quadrilateral .
Then there exist two paths  and  in , where
 has  as an endpoint and  has  as an endpoint,
with the following properties:
\begin{enumerate}
\item[(a)]  and  have a nonempty intersection.
\item[(b)] .
\item[(c)] Each edge on  is no longer than .
\end{enumerate}
\label{lem:pathcross}
\end{lemma}
\begin{proof}
Assume without loss of generality that .
We discuss the following exhaustive cases:
\begin{enumerate}
\item , and .
In this case,  and  cannot cross each other
(by Lemma~\ref{lem:same.quadrant}), so this case is finished.
\begin{figure}[htbp]
\centering
\includegraphics[width=0.8\linewidth]{pathcross}
\caption{Lemma~\ref{lem:pathcross}: (a) , and 
(b), and 
(c)  (d) .}
\label{fig:pathcross}
\end{figure}
\item , and , as in
Figure~\ref{fig:pathcross}a. Since  crosses , .
Since , .
Since , .
These along with Lemma~\ref{lem:quad} imply that  and  are the only
candidates for a shortest edge of .

Assume first that  is a shortest edge of .
By Lemma~\ref{lem:same.quadrant},
 does not cross .
It follows from
Lemma~\ref{lem:ec1}
that  and
 have a nonempty intersection.
Furthermore, by Lemma~\ref{lem:PR},
 and
, and no edge on these paths
is longer than , proving the lemma
true for this case.

Consider now the case when 
is a shortest edge of  (see Figure~\ref{fig:pathcross}a).
Note that  is below  (otherwise,  and
) and, therefore, ).
By Lemma~\ref{lem:same.quadrant}, 
does not cross . If 
does not cross , then  and  have a nonempty intersection,
proving the lemma true for this case.
Otherwise, there exists 
that crosses  (see Figure~\ref{fig:pathcross}a). Define

By Lemma~\ref{lem:same.quadrant},  does not cross .
Then  and  must have a nonempty intersection.
We now show that  and  satisfy conditions (b) and (c) of the lemma.
Proposition~\ref{fact:quad}
applied on the quadrilateral  tells us that

We also have that , since  and  is
in the same quadrant of  as . This along with the inequality above implies
. Because , by Lemma~\ref{lem:PR} we
have that , which along with the previous inequality shows that
.
This along with Lemma~\ref{lem:PR} shows that condition (c) of the lemma
is satisfied. Furthermore,
 and
.
It follows that .

\item , and , as in Figure~\ref{fig:pathcross}b.
Then , and by Lemma~\ref{lem:quad}  is not
a shortest edge of .
The case when  is a shortest edge of  is settled by
Lemmas~\ref{lem:same.quadrant} and~\ref{lem:PR}: Lemma~\ref{lem:same.quadrant}
tells us that  does not cross , and
 does not cross . It follows that  and
 have a nonempty intersection. Furthermore, Lemma ~\ref{lem:PR} guarantees
that  and  satisfy conditions (b) and (c) of the lemma.

Consider now the case when  is a shortest edge of ; the case
when  is shortest is symmetric.
By
Lemma~\ref{lem:c1},
 does not cross .
If  does not cross , then this case is settled:
 and  satisfy the three conditions
of the lemma. Otherwise, let 
be the edge crossing . Arguments similar to the ones used in case 1 above show that

are two paths that satisfy the conditions of the lemma.

\item , and , as in Figure~\ref{fig:pathcross}c.
Note that a horizontal reflection of Figure~\ref{fig:pathcross}c, followed
by a rotation of , depicts a case identical to case 1, which has
already been settled.

\item , as in Figure~\ref{fig:pathcross}d.
Note that Figure~\ref{fig:pathcross}d rotated by  depicts
a case identical to case 1, which has already been settled.

\item . Then it must be that , otherwise
 cannot cross . By Lemma~\ref{lem:same.quadrant} however,
 and  may not cross, unless one of them is not in .

\item , as in Figure~\ref{fig:pathcross}e.
Note that a vertical reflection of Figure~\ref{fig:pathcross}e depicts
a case identical to case 1, so this case is settled as well.
\end{enumerate}
Having exhausted all cases, we conclude that the lemma holds.
\end{proof}

\noindent
We are now ready to establish the main lemma of this section, showing
that there is a short path between the endpoints of two intersecting
edges in .

\begin{lemma}
Let  be four distinct nodes such that
 crosses
, and let
 be a shortest side of the quadrilateral .
Then  contains a path  connecting  and , of length

Furthermore, no edge on  is longer than .
\label{lem:recross}
\end{lemma}
\begin{proof}
Let  and  be the two paths whose existence in  is
guaranteed by Lemma~\ref{lem:pathcross}. By condition (c) of
Lemma~\ref{lem:pathcross}, no edge on  and  is longer
than . By condition (a) of Lemma~\ref{lem:pathcross},  and
 have a nonempty intersection. If  and 
share a node , then the path

is a path from  to  in  no longer than ;
the length restriction follows from guarantee (b) of Lemma~\ref{lem:pathcross}.
Otherwise, let  and
 be two edges crossing
each other. Let  be a shortest side of the quadrilateral
, with  and .
Lemma~\ref{lem:pathcross} tells us that  and .
These along with Lemma~\ref{lem:quad} imply that

This enables us to derive a recursive formula for computing a path
 as follows:

Next we use induction on the length of  to prove the claim of the lemma.
The base case corresponds to , case in which  degenerates to a point
and .
To prove the inductive step, pick a shortest side  of a quadrilateral , with
 crossing each other, and assume that
the lemma holds for all such sides shorter than .
Let  be the path determined recursively as in~(\ref{eq:pab}).
By the inductive hypothesis, we have that  contains no edges longer
than , and

This latter inequality follows from~(\ref{eq:rq}). This along with Lemma~\ref{lem:pathcross}
and formula~(\ref{eq:pab}) implies

This completes the proof.
\end{proof}

\section{ and }
\label{sec:Yinf+Y4}
We prove that every individual edge of  is spanned by a
short path in . This, along with the result of
Theorem~\ref{thm:y4infspanner}, establishes that  is a spanner.

Fix an edge . Define an edge or
a path as \emph{short} if its length is within a constant factor
of . In our proof that  is spanned by a short path in
, we will make use of the following three statements
(which will be proved in the appendix).
\begin{description}
\item[S1]
If  is short, then ,
and therefore its reverse, ,
are short by Lemma~\ref{lem:PR}.
\item[S2]
If  and  are short, and if
 intersects , Lemma~\ref{lem:recross} shows that
then there is a short path between any two of the endpoints
of these edges.
\item[S3]
If  and  are short paths that intersect,
then there is a short path  between any two of the endpoints
of these paths, by {\bf S2}.
\end{description}

\begin{lemma}
For any edge , there is a short path  of length

\label{lem:lool2short}
\end{lemma}
\begin{proof}
For the sake of clarity, we only prove here that there is a short path , and defer
the calculations of the actual stretch factor of  to the appendix.
Assume without loss of generality that , and
. If , then 
and the proof is finished. So assume the opposite, and let
 be the edge in ; since  is nonempty,
 exists.
Because  and  is in the same quadrant of 
as , we have that

Thus both  and  are short.
And this in turn implies that  is short by {\bf S1}.
We next focus on .
Let  be the other endpoint of .
We distinguish three cases.

\begin{figure}[hp]
\centering
\includegraphics[width=0.8\linewidth]{lool2}
\caption{Lemma~\ref{lem:lool2short}: (a) Case 1:  and  have a nonempty intersection.
(b) Case 2:  and  have an empty intersection.
(c) Case 3:  and  have a non-empty intersection.}
\label{fig:lool2}
\end{figure}


\noindent
{\bf Case 1:}  and  intersect.
Then by {\bf S3} there is a short path  between  and .

\medskip
\noindent
{\bf Case 2:}  and  do not intersect,
and  and  do not intersect (see Figure~\ref{fig:lool2}b).
Note that because  is
the endpoint of the short path , the triangle inequality on 
implies that  is short, and therefore
 is short. We consider two cases:
\begin{enumerate}
\item[(i)]  intersects .
Then by {\bf S3} there is a short path .
So

is short.

\item[(ii)]  does not intersect .
Then  must intersect
.
Next we establish that  is short. Let  be the last edge
of , and so incident to 
(note that  and  may coincide). Because  does not intersect
,  and  are in the same quadrant for . It follows that
 and .
These along with
Proposition~\ref{fact:tri}
for  imply that
 (this latter inequality
uses the fact that , which implies that ).
It follows that

Thus  is short, and by {\bf S1} we have that  is short.
Since  intersects the short path
,
there is by {\bf S3} a short path , and so

is short.

\end{enumerate}

\medskip
\noindent
{\bf Case 3:}  and  do not intersect,
and  intersects  (see Figure~\ref{fig:lool2}c).
If  intersects  at , then
 is short.
So assume otherwise, in which case
there is an edge  that
crosses .
Then , , and  and  are in
the same quadrant for . Note however that  cannot lie in
, since in that case , which would imply
, which in turn would imply .
So it must be that .

Next we show that  does not cross . Assume the opposite,
and let  cross . Then
, , and  and  are in
the same quadrant for . Arguments similar to the ones above show that
, so  must lie in . Let  be the 
distance from  to . Let  be the projection of  on the horizontal
line through . Then

Because  and  are in the same quadrant for , the inequality above
contradicts .

We have established that  does not cross . Then
 must intersect .
Note that  is short because it is in the short path .
Thus  is short, and so  and 
are short.
Thus we have two intersecting short paths, and so by {\bf S3}
there is a short path .
Then

is short. Calculations deferred to the appendix show that, in each of these cases, the
stretch factor for  does not exceed .
\end{proof}

\noindent
Our main result follows immediately from Theorem~\ref{thm:y4infspanner} and Lemma~\ref{lem:lool2short}:

\begin{theorem}
 is a -spanner, for .
\end{theorem}

\section{Conclusion}
Our results settle a long-standing open problem, asking whether 
is a spanner or not. We answer this question positively, and
establish a loose stretch factor of .
Experimental results, however, indicate a stretch factor of the
order , a factor of 200 smaller.
Finding tighter stretch factors for both  and 
remain interesting open problems. Establishing
whether  and  are spanners or not is also open. 

\noindent


\def\cprime{}
\begin{thebibliography}{1}

\bibitem{bmnsz-agbsp-03}
P.~Bose, A.~Maheshwari, G.~Narasimhan, M.~Smid, and N.~Zeh.
\newblock Approximating geometric bottleneck shortest paths.
\newblock {\em Computational Geometry: Theory and Applications}, 29:233--249,
  2004.

\bibitem{DMP09}
M.~Damian, N.~Molla, and V.~Pinciu.
\newblock Spanner properties of -angle Yao graphs.
\newblock In {\em Proc. of the 25th European Workshop on Computational
  Geometry}, pages 21--24, March 2009.

\bibitem{FLZ98}
M.~Fischer, T.~Lukovszki, and M.~Ziegler.
\newblock Geometric searching in walkthrough animations with weak spanners in
  real time.
\newblock In {\em {ESA} '98: Proc. of the 6th Annual European Symposium on
  Algorithms}, pages 163--174, 1998.

\bibitem{M09}
N.~Molla.
\newblock Yao spanners for wireless ad hoc networks.
\newblock M.S. Thesis, Department of Computer Science, Villanova University, December 2009.

\bibitem{Quad51}
J.W. Green.
\newblock A note on the chords of a convex curve.
\newblock {\em Portugaliae Mathematica}, 10(3):121--123, 1951.

\bibitem{Yao82}
A.C.-C. Yao.
\newblock On constructing minimum spanning trees in -dimensional spaces and
  related problems.
\newblock {\em SIAM Journal on Computing}, 11(4):721--736, 1982.

\end{thebibliography}

\section{Appendix}
\label{sec:appendix}
\subsection{Calculations for the stretch factor of  in Lemma~\ref{lem:lool2short}}
We start by computing the stretch factor of the short paths claimed by statements {\bf S2}
and {\bf S3}.

\begin{description}
\item[S2]
If  and  are short, and if
 intersects , then there is a short path  between any two of the endpoints
of these edges, of length 
This upper bound can be derived as follows. Let  be a shortest side of
the quadrilateral . By Lemma~\ref{lem:recross}, 
contains a path  no longer than .
By Lemma~\ref{lem:quad}, .
These together with the fact that
 yield inequality~(\ref{eq:s2}).
\item[S3]
If  and  are short paths that intersect,
then there is a short path  between any two of the endpoints
of these paths, of length 
This follows immediately from {\bf S2} and the fact that
no edge on  is longer than  (by Lemma~\ref{lem:recross}).
\end{description}

\medskip
\noindent
{\bf Case 1:}  and  intersect. Then by {\bf S3} we have



\medskip
\noindent
{\bf Case 2(i):}
 and  do not intersect;
 and  do not intersect;
and  intersects .
By {\bf S3}, there is a short path  of length

Next we establish an upper bound on . By the triangle inequality,

Substituting this inequality in~(\ref{eq:o3}) yields

Thus  is a path in  of length


\medskip
\noindent
{\bf Case 2(ii):}
 and  do not intersect;
 and  do not intersect;
and  does not intersect .
Then  must intersect
.
By {\bf S3} there is a short path  of length

Inequalities~(\ref{eq:o1})ii, ~(\ref{eq:o4a}) and~(\ref{eq:o4b}) imply that
. Substituting in the above,
we get

Thus  is a path in  from  to  of length


\medskip
\noindent
{\bf Case 3:}  and  do not intersect,
and  intersects .
If  intersects  at , then
 is clearly short and
does not exceed the spanning ratio of the lemma.
Otherwise, there is an edge 
that crosses , and
 intersects 
(as established in the proof of Lemma~\ref{lem:lool2short}).
By {\bf S3} there is a short path  of length

A loose upper bound on  can be obtained by employing
Proposition~\ref{fact:quad}
to the quadrilateral
: . Substituting the upper bound for
 from~(\ref{eq:o4b}) yields

By Lemma~\ref{lem:PR},  (since ), which
along with~(\ref{eq:o4b}) implies

Substituting~(\ref{eq:ae}) and~(\ref{eq:de}) in ~(\ref{eq:o9}) yields

Then

is a path from  to  of length


\subsection{ is a Spanner, for }
\begin{lemma}
Let  be a real number with , and let

Let , , and  be three distinct points in the plane such that
, let , and assume that
. Then

\label{lem:60degrees}
\end{lemma}
\begin{proof}
Refer to Figure~\ref{fig:60degrees}. By the Law of Cosines, we have

\begin{figure}[htbp]
\centering
\includegraphics[width=0.26\linewidth]{60degrees}
\caption{Lemma 1: If  and , then .}
\label{fig:60degrees}
\end{figure}
Since  and , the right-hand side in
(\ref{eqtoprove}) is positive, so (\ref{eqtoprove}) is equivalent
to

Thus, we have to show that

which simplifies to

Since  and ,
(\ref{eqtoshow}) holds if

which can be rewritten as

By our choice of , equality holds in (\ref{eqtoshow3}).
\end{proof}

\noindent
An immediate consequence of Lemma~\ref{lem:60degrees} is the
following result.
\begin{theorem}
For any  with , the Yao-graph with cones of angle
, is a -spanner for

\label{thm:Y6}
\end{theorem}
\begin{proof}
The proof of this claim is by induction on the distances defined by
the  pairs of nodes. Since , any closest
pair is connected by an edge in the Yao-graph; this proves the
basis of the induction. The induction step follows from
Lemma~\ref{lem:60degrees}.
\end{proof}

What happens to the value of  from Lemma~\ref{lem:60degrees},
if  gets close to :
Let , so that  is close to zero.
Then


\end{document}
