\documentclass{ws-ijbc}



\usepackage{graphicx}
\graphicspath{{figures-pdf/}}
\usepackage{multirow,bigstrut}
\usepackage{amssymb,amsmath,bm}

\usepackage[normalem]{ulem}
\usepackage[retainorgcmds]{IEEEtrantools}
\usepackage{xcolor}

\newlength\figwidth
\setlength\figwidth{0.3\textwidth}
\newlength\imagewidth
\setlength\imagewidth{0.3\columnwidth}

\usepackage{color}
\definecolor{dgreen}{rgb}{0,.6,0}
\newcommand\comments[1]{\textcolor{blue}{#1}}   \newcommand\commented[1]{\textcolor{red}{#1}}   \newcommand\modified[1]{\textcolor{dgreen}{#1}} 
\newtheorem{property}{Property}



\begin{document}

\catchline{}{}{}{}{} 
\markboth{C. Li et al.}{Breaking a chaotic image encryption algorithm based on modulo addition and XOR operation}

\title{Breaking a chaotic image encryption algorithm based on modulo addition and XOR operation}

\author{Chengqing Li\textsuperscript{1}\thanks{Corresponding author, chengqingg@gmail.com},
Yuansheng Liu\textsuperscript{1}, Leo Yu Zhang\textsuperscript{2}, and Michael Z. Q. Chen\textsuperscript{3}\\
\textsuperscript{1} College of Information Engineering,\\ Xiangtan University, Xiangtan 411105, Hunan, China\
y=(\alpha\dotplus x)\oplus (\beta \dotplus x)
\label{eq:essentialfunction}

x(k+1)=3.9\cdot x(k)\cdot(1-x(k)),\label{equation:Logistic}

J'(k)=
\begin{cases}
(J(k)\dotplus key_1)\oplus key_1 & \mbox{if }B(k)=3;\\
(J(k)\dotplus key_1)\odot  key_1 & \mbox{if }B(k)=2;\\
(J(k)\dotplus key_2)\oplus key_2 & \mbox{if }B(k)=1;\\
(J(k)\dotplus key_2)\odot  key_2 & \mbox{if }B(k)=0,
\end{cases}
\label{eq:Encrypt}

J(k)=
\begin{cases}
(J'(k)\oplus key_1)\dot{-}key_1            & \mbox{if }B(k)=3;\\
(J'(k)\oplus \overline{key_1})\dot{-}key_1 & \mbox{if }B(k)=2;\\
(J'(k)\oplus key_2)\dot{-}key_2            & \mbox{if }B(k)=1;\\
(J'(k)\oplus \overline{key_2})\dot{-}key_2 & \mbox{if }B(k)=0,
\end{cases}
\label{eq:decrypt}

J_1'(k)\oplus J_2'(k)=
\begin{cases}
(J_1(k)\dotplus key_1)\oplus (J_2(k)\dotplus key_1) & \mbox{if }B(k)\in\{2, 3\};\\
(J_1(k)\dotplus key_2)\oplus (J_2(k)\dotplus key_2) & \mbox{if }B(k)\in\{0, 1\}.
\end{cases}
\label{eq:DifferentialEquation}

\tilde{y}=y\oplus \alpha\oplus \beta=(\alpha\dotplus x)\oplus(\beta\dotplus x)\oplus \alpha\oplus \beta
\label{eq:essentialfunctionform}

\left\{
\begin{IEEEeqnarraybox}[][c]{rCl}
\tilde{y}_{i+1} & = & c_{i+1}\oplus \tilde{c}_{i+1},\\
c_{i+1}         & = & (x_i\cdot \alpha_i) \oplus (x_i\cdot c_i) \oplus (\alpha_i\cdot c_i),\\
\tilde{c}_{i+1} & = & (x_i\cdot \beta_i) \oplus (x_i\cdot \tilde{c}_i) \oplus (\beta_i\cdot \tilde{c}_i),
\end{IEEEeqnarraybox}\right.
\label{eq:bitdecomposition}

c_{i+1}=(x_i\cdot \alpha_i) \oplus (x_i\cdot c_i) \oplus (\alpha_i\cdot c_i)
\label{eq:getcarrybit}

\tilde{c}_{i+1}=(x_i\cdot \beta_i) \oplus (x_i\cdot \tilde{c}_i) \oplus (\beta_i\cdot \tilde{c}_i)

\tilde{y}_{i+1} & = & (\alpha_{i+1}\oplus c_{i+1}\oplus x_{i+1})\oplus (\beta_{i+1}\oplus\tilde{c}_{i+1}\oplus x_{i+1})\oplus \alpha_{i+1}\oplus \beta_{i+1}\\
                & = & c_{i+1}\oplus \tilde{c}_{i+1},

(x_i, c_i)\in
\begin{cases}
\{(0, 0), (1, 1)\} & \mbox{if } \tilde{y}_{i+1}=0;\\
\{(0, 1), (1, 0)\} & \mbox{if } \tilde{y}_{i+1}=1,
\end{cases}
\label{eq:cases2_4}

x_0=
\begin{cases}
0 & \mbox{if } \tilde{y}_{1}=0;\\
1 & \mbox{if } \tilde{y}_{1}=1,
\end{cases}
\label{eq:x0}

c_1=
\begin{cases}
0 & \mbox{if } \tilde{y}_{1}=0;\\
1 & \mbox{if } \alpha_0=1 \mbox{ and }\tilde{y}_{1}=1;\\
0 & \mbox{if } \alpha_0=0 \mbox{ and }\tilde{y}_{1}=1.
\end{cases}

\{c_i+\alpha_i\}\in \{0, 2\}\mbox{ is known.}
\label{eq:condition}

x_i=\alpha_i\oplus \tilde{y}_{i+1}
\label{eq:xi17}

x_i=
\begin{cases}
1-\tilde{y}_{i+1} & \mbox{if }  c_i=1;\\
\tilde{y}_{i+1}  & \mbox{if }  c_i=0,
\end{cases}
\label{eq:xi24}

Prob(c_i=1)=\frac{3}{4}Prob(c_{i-1}=1)+\frac{1}{4}Prob(c_{i-1}=0)

Prob(\tilde{y}_i=0)=\frac{2}{3}+\frac{1}{3\cdot 4^i}.
\label{eq:probyi}
-4pt]
    \botrule
\end{tabular}}
\label{table:numbersolution}
\end{table}

Referring to Proposition~\ref{propsition:lsb} and Eq.~(\ref{eq:Encrypt}), one can obtain the scope of ,

for .

\begin{proposition}
Assume that  and  are both -bit integers and ,
 has the same parity as  and  has opposite parity as .
\label{propsition:lsb}
\end{proposition}
\begin{proof}
Existence of four equations

is independent of , so the proposition is proved.
\end{proof}

\begin{proposition}
Assume that  and  are both -bit integers, , one has the following two equations

\label{propsition:msb}
\end{proposition}
\begin{proof}
See the proof of Proposition~1 in \cite{LCQ:MCKBA:IJBC11}.
\end{proof}

According to the pre-defined condition , there are only two possible combinations of  and .  Let  denote the searched version of . Proposition~\ref{propsition:msb} illustrates that the unknown most significant bit of  and  has no influence on decryption of MCKBA/HCKBA. Then, one can further obtain the approximate value of , , and the  least significant bits of
 and  by the following two different ways:
\begin{itemlist}
\item  \textit{W1)} For , one has

where , . Note that
Eq.~(\ref{eq:determinebits2}) makes only two conditions in Eq.~(\ref{eq:determineBk}) need being
verified. Obviously, one can assure that , .


\item \textit{W2)} When there exists  satisfying that , one can obtain

for . Then, the value of  can be obtained by setting  for
. Finally, one can conclude that , , and
 can work together as equivalent secret key of MCKBA/HCKBA due to that 
and  are equivalent for Eq.~(\ref{eq:decrypt}).
\end{itemlist}

Now, let's study the success probability of the above two methods. The success of the method \textit{W1) depends on existence of one of the eight
condition in Eq.~(\ref{eq:determineBk}). As  if and only if , only one of the two functions
need being studed. Obviously, }

So,  can not be confirmed by  Eq.~(\ref{eq:determineBk}) when
  and  exist at the same time. It is very hard to derive the probability for other cases theoretically.
Instead, we calculate the probability that  via simulation, where

, . Assume  distributes uniformly, the distribution of
probability  under different values of  and some values of  is shown in Fig.~\ref{fig:probability}.

\begin{figure}[!htb]
\centering
\begin{minipage}{2\figwidth}
\centering
\includegraphics[width=\textwidth]{prob}
\end{minipage}
\caption{The probability  under different value of , where  satisfy the constraint condition (\ref{eq:constraint}).}
\label{fig:probability}
\end{figure}

The success of the method \textit{W2) can be analyzed as follows. Assume  distributes uniformly, one can get the probability that at least one bit (exclude the most significant bit) of 
satisfy  and one condition of Eq.~(\ref{eq:determinebits1}) is}
\begin{IEEEeqnarray*}{rcl}
Prob[ \mbox{Eq.}~(\ref{eq:determinebits1}) \mbox{ holds} ] & =& 1-\sum_{i=0}^{n-1}\binom{n-1}{i}\left(Prob[x_i]\right)^i\left(1-Prob[x_i]\right)^{n-1-i}\left(\frac{1}{2}\right)^i\\
& \ge & 1-\sum_{i=0}^{n-1}\binom{n-1}{i}\left(\frac{1}{2}\right)^i\left(1-\frac{1}{2}\right)^{n-1-i}\left(\frac{1}{2}\right)^i\\
     & = & 1-\left(\frac{1}{2}\right)^{n-1}\sum_{i=0}^{n-1}\binom{n-1}{i}\left(\frac{1}{2}\right)^i\\
     & = & 1-\left(\frac{3}{4}\right)^{n-1}.
\end{IEEEeqnarray*}
From the above analysis, one can see that , , can be determined with a probability larger than  when
,  and  in Eq.~(\ref{eq:essentialfunction}) distributes uniformly. Although pixels of natural images follow Gaussian distribution, and
 and  in Eq.~(\ref{eq:DifferentialEquation}) do not distribute uniformly, we still can believe that the success probability of this method
is very high since only one bit satisfying the conditions in Eq.~(\ref{eq:determinebits1}) is needed, especially when  is relatively large.

To verify the real performance of the above analysis, a number of experiments are carried out on some plain-images of size  with the method \textit{W1)} when . When , , and . Two known plain-images ``Peppers" and ``Baboon", and the corresponding cipher-images are adopted.
Equivalent key ,  and  is used to decrypt another cipher-image
shown in Fig.~\ref{fig:DecryptedLenna}a) and the recovered result is shown in Fig.~\ref{fig:DecryptedLenna}b), which is identical with the original version. From the
experiment, we found that only a little pixels (no more than ten pixels for plain-image of size ) are not recovered correctly when . This agree with our expectation as the probability that none of condition of Eq.~(\ref{eq:determineBk}) and condition~(\ref{eq:SpecialCondition}) are satisfied become larger when  is smaller.

\begin{figure}[!htb]
\centering
\begin{minipage}{\figwidth}
\centering
\includegraphics[width=\textwidth]{lenna_c}
a)
\end{minipage}
\begin{minipage}{\figwidth}
\centering
\includegraphics[width=\textwidth]{lenna_c_d}
b)
\end{minipage}
\caption{The decryption result of another cipher-image encrypted with the same secret key: a) cipher-image; b)
decrypted plain-image.}
\label{fig:DecryptedLenna}
\end{figure}

\subsection{Chosen-plaintext attack}

Chosen-plaintext attack is an enhanced version of known-plaintext attack, where the plaintext can be chosen arbitrarily to
obtain the information about the secret key in a more efficient way. In this subsection, the chosen-plaintext attack on MCKBA/HCKBA is briefly
introduced due to the following two points: 1) the known-plaintext attack on MCKBA/HCKBA works well in a relatively high probability and the chosen-plaintext
version can improve its performance a little; 2) the underlying theorem supporting the attack proposed in \cite[Theorem 1]{LCQ:MCKBA:IJBC11} is not right and corrected in Proposition~3.

\begin{proposition}
Assume that  are all -bit integers, then a lower bound on the number of queries  to solve
Eq.~(\ref{eq:essentialfunction}) in terms of modulo  for any  is 1 if ; 2 if .
\end{proposition}
\begin{proof}
When , one can obtain  by choosing  .
When ,  may be equal to zero or one no matter what  is, which means that
it is impossible to satisfy the condition of Property~3 for any . So, we have to resort to another
query . Let  and  denote the counterparts of , and 
corresponding to . Given a set of  and , one can obtain
 and  from  and , respectively,
where  are non-negative integers. Let arrows of plain head and ``V-back" head denote  and , respectively,
Fig.~\ref{fig:relationship} illustrates the mapping relationship between  and 
for a given , where . Since ,
the dashed arrows in Fig.~\ref{fig:relationship} describe operations of Eq.~(\ref{eq:essentialfunctionform}) in the two least significant bit planes corresponding to two sets of . Note that the data in the third column is exactly the same as the first one. Therefore, Fig.~\ref{fig:relationship} demonstrates operations of Eq.~(\ref{eq:essentialfunctionform})
under all different bit levels if the variable  goes through , where  and . Referring to Fig.~\ref{fig:relationship}, it can be
easily verified that  is always satisfied, which means that  can be derived from Table~1.
\end{proof}
\begin{figure}[!htb]
\centering
\begin{minipage}{1.5\figwidth}
\centering
\includegraphics[width=\textwidth]{table_v4}
a)
\end{minipage}
\caption{Relationship between   and   for
a given  , where .}
\label{fig:relationship}
\end{figure}

Under scenario of chosen-plaintext attack, one may make the plaintext satisfy that at least one pair of elements in  whose -th bit plane satisfy the condition of Property~3. The same case exists for . The expected chosen-plaintext can be
obtained in a high probability by assigning  with one of the two sets of number given in Corollary~\ref{coro:msb} randomly. Compared with
the known-plaintext attack, the chosen-plaintext attack has the following two superior performances:
1) the set  can be reconstructed with much less complexity and much higher degree of accuracy;
2) the bits of  can be confirmed with a little higher probability.

\begin{corollary}
The  least significant bits of  in Eq.~(\ref{eq:essentialfunction}) can be determined easily
by setting  with the following two sets of numbers
\label{coro:msb}

and checking the corresponding .
\end{corollary}
\begin{proof}
The proof is straightforward and therefore omitted.
\end{proof}

\section{Conclusion}

In this paper, the security of the image encryption algorithm MCKBA/HCKBA has been restudied
in detail. Based on some properties of a composite function composed of the modulo addition
and the XOR operation, a known-plaintext attack and an improved chosen-plaintext attack were provided
to determine an equivalent secret key of MCKBA/HCKBA. The cryptanalysis provided in
this paper sheds some light on breaking other encryption schemes based on multiple combination
of the modulo addition and XOR operations.

\section*{Acknowledgement}

This research was supported by the National Natural Science Foundation of China (No.~61100216), Scientific Research Fund of Hunan Provincial Education Department (No.~11B124), and Research Fund of Xiangtan University (No.~2011XZX16).

\bibliographystyle{ws-ijbc}
\bibliography{EMCKBA}
\end{document} 