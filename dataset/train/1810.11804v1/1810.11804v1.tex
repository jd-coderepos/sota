\section{Introduction}
In research on early language we often find the claim that children's early productive vocabularies were dominated by nouns referring to
concrete objects such as foods or toys. This assumption appears to have been picked up and reinforced by work in robotics on symbol grounding
(cf. \cite{Stramandinoli2017}). As a consequence there are plenty of studies that focus on the acquisition of precisely these
types of words \cite{Hollich2000}.\\
The importance of mutual or joint reference between mothers and children to perceptible objects and events is emphasized by more recent, so called
usage-based theories of language development \cite{Tomasello2003}. Mother, child, and external referent make up a triadic joint-attentional
frame which are very much the focus of these later theories. Cognitively such triadic interactional constellations are more complex than a simple
dyadic interaction.

In those areas of developmental robotics concerned with language acquisition in artificial agents the linguistic units in focus are similarly those
that can be construed as referents for concrete physical objects, object properties, or perceptible events.
Central in this area of research is the notion of symbol grounding, the construction of links between abstract symbols and signals or constructs that are
based on the embodiment of the agent \cite{Harnad1990}. The construction of such links may be regarded as a form of sense making with respect to the 
linguistic entities under consideration. The linguistic units in question are typically words or simple grammatical constructions. The agent's embodiment 
often presents itself with respect to a stream of sensorimotor data. 

In stark contrast to this focus on concrete referents, and the words that label them, is the observation that amongst the very first words produced
by a toddler are many which do not fall into this category. We find words such as `no', `hi', or 'bye' \cite{Fenson1994} which are sometimes referred
to as \emph{social words}. Negation words are thus amongst the very first words in many toddlers' active vocabularies and are used by them to
reject things or to self-prohibit \cite{Gopnik1988,Volterra1979}, the latter being a function that is rarely seen with adult speakers.
The idea to link these social words with a robot's sensorimotor data appears strangely inappropriate. The question then is how these types
of words should be handled in an embodied language acquisition framework.

\section{Symbol Grounding}
Artificial symbol grounding or perceptual symbol systems \cite{Barsalou1999} attempt to solve or break out of the symbol grounding problem, the formulation of
which is frequently attributed to Harnad \cite{Harnad1990}: If symbols are recursively defined or explained merely by the concatenation of other symbols, as is the case 
in a dictionary, how can the agent make sense of such symbols given the often circular relationship between the explanandum and explanans (see \cite{Roy2005} for 
an example)? The principle method employed in solving this problem is to connect some or all symbols of the system to sensorimotor data that originates in the agent's own 
embodiment: its visual sensors, its haptic sensors, its auditory perception, or any kind of derivative constructs that are computed from data originating from such sensory
channels. The way existing symbol grounding systems differ from each other is mainly in the method how the link between symbolic and sensor-derived data is established. 
The methods for constructing and maintaining such links may involve neural networks \cite{Sugita2005,Cangelosi2010}, symbolic artificial intelligence approaches 
\cite{Siskind2001,Dominey2005,Steels2003a}, statistical learning methods \cite{Stepanova2018}, or methods inspired by an enactive approach \cite{Nehaniv2013,Lyon2016}. 
The latter are typically data-driven and, arguably, representation-free in the sense that no models of object or event categories are constructed. Technically this 
can be made possible through the use of lazy learning algorithms \cite{Aha1997} which compute retrieval or classification requests directly on the `remembered' data.

Most of the existing work focuses and is limited to the grounding of words that may be seen to either refer to concrete objects (`concrete nouns') or to temporally unfolding 
perceptible events and processes (`concrete verbs'). More recently Stramandinoli et al. \cite{Stramandinoli2017} provided an example for grounding more abstract verbs
such as `use' or `make' on the back of already grounded concrete verbs such as `cut' or `slide'. While this is certainly an improvement in terms of the ability to ground a 
more general class of words, it is not clear how this approach would help to ground social or socio-pragmatic words such as `no', `yes', or `hi', or emotion words such
as `sad', or `upset'.

\section{Negation and Affect in Language Acquisition}
\label{sec:negation_affect}
Authors such as Pea emphasize the significance of affect in the context of acquisition of early linguistic negation \cite{Pea1980}.
Less well understood are the concrete ramifications of this primacy of affect for a cognitive architecture in terms of required
components, learning mechanisms, or the dynamics of the learning process. Hence roboticists have so far been unable to create machines
that could acquire and engage in this aspect of human speech.

Spitz \cite{Spitz1957} hypothesized that infants' major source of negation words is rooted in parental prohibitive utterances. Under this hypothesis, the 
infant's frustration, brought about by adult prohibition, leads to negative affect on the child's part who subsequently associates negative affect 
with the negative utterance. Via role reversal the child is then thought to use these negative symbols for the purpose of rejection. 

\subsection{Rejection Experiment}
In the \emph{rejection experiment} \cite{Foerster2017} we tested whether the robot's display of affect would lead to measurable changes in participants' speech
when addressing the robot. The kind of change we envisioned is best described with the notion of \emph{negative intent interpretations}: Intent interpretations
are linguistic descriptions or ascriptions of the addressee's affective or volitional state. They have been hypothesized to play a vital role for infants
to learn how to express their intent \cite{Ryan1974,Pea1980}. We use the results of the \emph{rejection experiment} within the present article as point of
reference in order to assess the comparative efficacy of the prohibition task in terms of the acquisition of negation words.\\text{between\_pushes} \Leftrightarrow \text{after\_push}(i) \wedge \text{before\_push}(j)\text{ with }j > i\text{during\_several\_pushes} \Leftrightarrow \text{overlap\_after}(i) \wedge \text{overlap\_before}(j) \text{ with }j > i.
Based on observations of the experimental video recordings we imposed a time constraint of 4 seconds as maximum gap size between
linguistic and corporal action for any two instances of \emph{prohibition} and \emph{push} to be considered in any of the stated temporal relations.
Table \ref{tbl:prohibition_rel} shows the resulting counts for each participant and session in which the prohibitive task was active. As can be seen from
there, participants often did not restrain the robot's arm movement or restrained it only after uttering a \emph{prohibition} (``before push'', see also main
text). This is opposed to how we imagined them to act and leads to a violation of the simultaneity constraint in our learning architecture.

\subsubsection{How temporal relations of prohibitive action impact on the acquisition of negation}\hfill \break
In the following we give some examples in order for the reader to better understand why our participants' unexpected behavior, that is participants
either not restraining the robot's arm as advised or them restraining the robot's arm after uttering prohibitions, is detrimental for the acquisition of
negation.
Let us assume \emph{no} as default negative word and let us further assume that \emph{no} is the salient word of the -type utterances
in question.

Similar to other mechanisms, which establish an association between object labels and perceptual features in other symbol-grounding architectures we may regard our
memory-based learner as a roughly associative learner. Associations between labels and other sensorimotor-motivational (\emph{smm}) data come to be by virtue of there
being a majority of exemplars in the memory where such an association is established. For our purposes this means that, all other things, i.e. sensorimotor-states,
being equal, such an association is established as soon as the majority of \emph{no}'s are attached to sensorimotor-motivational data with a negative motivational
entry. This means that any temporal relation leading to a \emph{no} with negative motivational value attached being added to the lexicon is \textbf{beneficial}
for our learning target. By contrast any temporal relation which leads to a \emph{no} with positive motivational value being added is \textbf{detrimental} to
this purpose.

Our version of symbol grounding is implemented such that the salient word is associated with all variants of the sensorimotor-motivational (\emph{smm}) 
vector that co-occurred during the time when the utterance was produced. During short time frames of a few seconds, the typical length of an utterance, 
in most cases nothing in this vector changes: The robot's behaviour does not change, the presented object stays the same, and, importantly, 
the object is recognized by the object detection to be the same. If this is the case while participants produce an utterance, the outcome will be one 
additional exemplar or lexical entry that is added to the robot's embodied lexicon. Yet, if one \emph{smm} change occurs during this production, two 
lexical entries for the same word will be created, one for each variant of the \emph{smm} vector.
Changes  in the \emph{smm} data are caused through changes of the robot's behaviour, which for their part are caused by either timeouts in the body behavior
system or changes in the object recognition. Also changes in the object id itself are forms of sensorimotor changes and so is the change of the robot's
motivational state. We will in the following make the simplifying assumption that the robot's object recognition works perfectly.

The humanoids behaviour was implemented such that it would only grasp for objects that it likes, i.e. objects that cause its motivational state to be 
positive. Under perfect object recognition the robot's motivational state will be positive before the participant restrains its arm movement (push action).
Restraints of the robot's agency lead immediately to Deechee becoming `grumpy', i.e. a negative motivational state.
\begin{figure*}[h]
  \hspace*{-2ex}
  \includegraphics[scale=0.55]{pushes.jpg}
  \caption{\textbf{Basic temporal relations between corporal constraints and prohibition}: The depicted temporal relations between
    \emph{prohibition} and corporal constraints (``push'') were observed within the prohibition scenario. Additionally
    two complex relations were observed which can be decomposed in the depicted ones (see text).}
  \label{fig:push_rels}
\end{figure*}
For \emph{no push} relations the following holds: \emph{No} is uttered while the robot is and continues to be in a `positive mood', for its agency 
is not impeded. Instead of restraining the robot's arm, as they were taught to do, participants often just held the object out of the robot's reach, 
which has no impact on its motivational state. Such interaction will lead to at least one exemplar of \emph{no} in the robot's lexicon which is 
associated to a \emph{smm} vector which has a positive motivational entry. This is \textbf{detrimental} to the learning outcome.

In contrast, Deechee will already be `in a negative mood' in case of participants starting to restrain its arm before uttering a 
(\emph{during push}). In this case one embodied word will enter the lexicon: \emph{no} associated with a \emph{smm} vector containing a negative 
motivational value. This is how we imagined the interaction to unfold motivated by assumptions of simultaneity in ostensive theories of meaning. 
This is \textbf{beneficial}.

In case of a participant starting to produce an utterance followed by him or her constraining the robot's arm movement during that production (\emph{overlap before}), 
two lexical entries will be created: a \emph{no}, associated with a \emph{smm} vector with a positive motivational entry, and additionally a \emph{no}, which is 
associated with an otherwise identical \emph{smm} vector but with a negative motivational entry. This is \textbf{in-between}.

If the onset of utterance production happens during a \emph{push} but extends to after the end of the push (\emph{overlap after}), the result will be one
additional \emph{no} in Deechee's lexicon associated to a \emph{smm} vector with negative motivational entry \emph{as long as the utterance is not overly long}.
The robot's motivational system is implemented such, that its motivational state has a certain time lag. The only exception to this rule are restrictions of
Deechee's freedom of movement which will make it grumpy immediately. Therefore the presence of said \emph{overlap after} relation between the mentioned actions
is most probably \textbf{beneficial}.
\begin{table*}[h]
\caption{\textbf{Counts of temporal relationships between physical constraints and prohibitive utterances}. Given are the counts of observed
    temporal relationships. Both \emph{prohibitions} as well as \emph{disallowances} were taken into consideration in the given count. Counts 
    are given for all participants and sessions in the prohibition scenario in which participants were told to physically restrain the robot 
    in case of it approaching a forbidden object. Furthermore a total count per participants is given in the last section of the table. A missing
    relationship type in a session indicates that all counts were 0. Temporal relationships of the listed types set in bold are very likely to be
    detrimental for an association of the salient word with negative affect in our architecture. Relationships of a type set in italic are less likely to be
    detrimental for said association depending on the length of the gap between push(es) and utterance and the hypothesized duration of the 
    motivational state triggered by physical restraint.}
  \label{tbl:prohibition_rel}
\begin{tabular*}{\hsize}{@{\extracolsep{\fill}}llllllllllll}
    \toprule
    &  & P13 & P14 & P15 & P16 & P17 & P18 & P19 & P20 & P21 & P22\\
    \midrule
    \multirow{8}{*}{s1} & \textbf{no\_push} & 0 & 0 & 15 & 14 & 1 & 4 & 6 & 0 & 1 & 4\\
    & \textbf{before\_push} & 1 & 0 & 0 & 0 & 0 & 1 & 3 & 2 & 5 & 0\\
    & \textbf{overlap\_before} & 1 & 0 & 0 & 0 & 1 & 0 & 0 & 0 & 1 & 0\\
    & \textbf{overlap\_before\_and\_after} & 1 & 0 & 0 & 0 & 2 & 0 & 0 & 0 & 0 & 0\\
    & \textsl{after\_push} & 0 & 0 & 1 & 0 & 1 & 0 & 1 & 2 & 3 & 0\\
    & overlap\_after & 1 & 0 & 1 & 0 & 1 & 1 & 0 & 0 & 0 & 0\\
    & \textsl{between\_pushes} & 0 & 0 & 0 & 0 & 1 & 0 & 0 & 2 & 1 & 0\\
    & during\_push & 4 & 0 & 1 & 0 & 2 & 0 & 1 & 4 & 3 & 0\\
    \midrule
    \multirow{8}{*}{s2} & \textbf{no\_push} & 0 & 3 & 5 & 10 & 0 & 0 & 0 & 3 & 30 & 2\\
    & \textbf{before\_push} & 3 & 4 & 1 & 0 & 0 & 2 & 0 & 3 & 0 & 0\\
    & \textbf{overlap\_before} & 2 & 0 & 1 & 0 & 1 & 1 & 0 & 1 & 0 & 0\\
    & \textbf{overlap\_before\_and\_after} & 0 & 0 & 1 & 0 & 0 & 0 & 0 & 1 & 0 & 0\\
    & \textsl{after\_push} & 0 & 1 & 0 & 0 & 0 & 0 & 0 & 1 & 0 & 0\\
    & overlap\_after & 1 & 0 & 0 & 0 & 0 & 0 & 0 & 0 & 0 & 0\\
    & \textsl{between\_pushes} & 0 & 0 & 0 & 0 & 0 & 0 & 0 & 1 & 0 & 0\\
    & \textsl{during\_several\_pushes} & 0 & 0 & 1 & 0 & 0 & 0 & 0 & 2 & 0 & 0\\
    & during\_push & 5 & 0 & 2 & 0 & 4 & 2 & 0 & 2 & 0 & 0\\
    \midrule
    \multirow{8}{*}{s3} & \textbf{no\_push} & 0 & 2 & 1 & 3 & 0 & 1 & 0 & 3 & 30 & 5\\
    & \textbf{before\_push} & 0 & 3 & 2 & 0 & 0 & 1 & 3 & 2 & 0 & 0\\
    & \textbf{overlap\_before} & 1 & 1 & 6 & 3 & 3 & 0 & 1 & 2 & 0 & 0\\
    & \textbf{overlap\_before\_and\_after} & 0 & 1 & 0 & 0 & 0 & 1 & 0 & 1 & 0 & 0\\
    & \textsl{after\_push} & 0 & 1 & 0 & 1 & 1 & 1 & 0 & 0 & 0 & 0\\
    & overlap\_after & 0 & 0 & 2 & 1 & 1 & 0 & 0 & 1 & 0 & 0\\
    & \textsl{between\_pushes} & 0 & 0 & 5 & 1 & 4 & 0 & 0 & 0 & 0 & 0\\
    & \textsl{during\_several\_pushes} & 0 & 0 & 0 & 0 & 0 & 0 & 0 & 1 & 0 & 0\\
    & during\_push & 2 & 0 & 8 & 1 & 8 & 0 & 2 & 5 & 0 & 0\\
    \midrule
    \multirow{8}{*}{total} & \textbf{no\_push} & 0 & 5 & 21 & 27 & 1 & 5 & 6 & 6 & 61 & 11\\
    & \textbf{before\_push} & 4 & 7 & 3 & 0 & 0 & 4 & 6 & 7 & 5 & 0\\
    & \textbf{overlap\_before} & 4 & 1 & 7 & 3 & 5 & 1 & 1 & 3 & 1 & 0\\
    & \textbf{overlap\_before\_and\_after} & 1 & 1 & 1 & 0 & 2 & 1 & 0 & 2 & 0 & 0\\
    & \textsl{after\_push} & 0 & 2 & 1 & 1 & 2 & 1 & 1 & 3 & 3 & 0\\
    & overlap\_after & 2 & 0 & 3 & 1 & 2 & 1 & 0 & 1 & 0 & 0\\
    & \textsl{between\_pushes} & 0 & 0 & 5 & 1 & 5 & 0 & 0 & 3 & 1 & 0\\
    & \textsl{during\_several\_pushes} & 0 & 0 & 1 & 0 & 0 & 0 & 0 & 3 & 0 & 0\\
    & during\_push & 11 & 0 & 11 & 1 & 14 & 2 & 3 & 11 & 3 & 0\\
    \bottomrule
  \end{tabular*}
\end{table*}
\noindent Table \ref{tbl:prohibition_mot} shows the counts of the various temporal relations between corporal restraint and prohibition as well as the motivational
states in which the robot was in when the respective form of prohibition was performed.
Table \ref{tbl:neg_int_int_mot1} shows the motivational states of the robot during the other highly frequent negation types \emph{negative intent interpretations}
and \emph{negative motivational questions} in the rejection experiment. Table \ref{tbl:neg_int_int_mot2} shows the same for our participants from the prohibition
experiment.
\begin{table*}[h]
  \caption{\textbf{Motivational states during utterances of prohibition and disallowance}. 
    Given are the counts of the robot's motivational states for each temporal relationship between prohibition/disallowance and physical restraint.
    The counts are listed per participant within the prohibition experiment (see table \ref{tbl:prohibition_rel} for the frequencies of these relationships). 
    The counts are accumulated over the first three sessions, i.e. the sessions in which physical restraint could possibly occur.
    Note that one occurrence of such a temporal relationship can yield more than 1 to the count as the robot's motivational state can change while the respective
    utterance is being produced. The entries for \emph{P13} for \emph{overlap\_before\_and\_after}
    are so large due to a glitch in the motivational and/or behavioral system.
    Symbols used: \emph{-}: negative motivation, \emph{+}: positive motivation, \emph{O}: neutral motivation}
  \label{tbl:prohibition_mot}
  \begin{tabular*}{\hsize}{@{\extracolsep{\fill}}llllllllllllllll}
    \toprule
    &  & P13 &  &  & P14 &  &  & P15 &  &  & P16 &  &  & P17\\
    \cmidrule(lr){2-4}\cmidrule(lr){5-7}\cmidrule(lr){8-10}\cmidrule(lr){11-13}\cmidrule(lr){14-16}
    & - & O & + & - & O & + & - & O & + & - & O & + & - & O & +\\
    \cmidrule(lr){2-4}\cmidrule(lr){5-7}\cmidrule(lr){8-10}\cmidrule(lr){11-13}\cmidrule(lr){14-16}
    no\_push & 0 & 0 & 0 & 0 & 0 & 5 & 4 & 4 & 17 & 0 & 1 & 14 & 0 & 0 & 1\\
    before\_push & 0 & 1 & 4 & 0 & 0 & 7 & 0 & 0 & 3 & 0 & 0 & 0 & 0 & 0 & 0\\
    overlap\_before & 4 & 0 & 4 & 1 & 0 & 1 & 6 & 0 & 6 & 3 & 0 & 3 & 5 & 0 & 3\\
    overlap\_before\_and\_after & 11 & 12 & 0 & 1 & 0 & 1 & 1 & 0 & 1 & 0 & 0 & 0 & 2 & 0 & 2\\
    after\_push & 0 & 0 & 0 & 2 & 0 & 1 & 1 & 0 & 0 & 0 & 0 & 1 & 2 & 0 & 0\\
    overlap\_after & 2 & 0 & 0 & 0 & 0 & 0 & 3 & 1 & 0 & 1 & 0 & 0 & 2 & 0 & 0\\
    between\_pushes & 0 & 0 & 0 & 0 & 0 & 0 & 2 & 0 & 3 & 1 & 1 & 1 & 4 & 0 & 2\\
    during\_several\_pushes & 0 & 0 & 0 & 0 & 0 & 0 & 1 & 0 & 0 & 0 & 0 & 0 & 0 & 0 & 0\\
    during\_push & 11 & 0 & 0 & 0 & 0 & 0 & 11 & 0 & 0 & 1 & 0 & 0 & 14 & 0 & 0\\
\end{tabular*}
  \begin{tabular*}{\hsize}{@{\extracolsep{\fill}}llllllllllllllll}
    \toprule
    & & P18 &  &  & P19 &  &  & P20 &  &  & P21 &  &  & P22\\
    \cmidrule(lr){2-4}\cmidrule(lr){5-7}\cmidrule(lr){8-10}\cmidrule(lr){11-13}\cmidrule(lr){14-16}
    & - & O & + & - & O & + & - & O & + & - & O & + & - & O & +\\
    \cmidrule(lr){2-4}\cmidrule(lr){5-7}\cmidrule(lr){8-10}\cmidrule(lr){11-13}\cmidrule(lr){14-16}
    no\_push & 0 & 0 & 5 & 0 & 2 & 4 & 2 & 4 & 1 & 3 & 14 & 16 & 0 & 1 & 7\\
    before\_push & 0 & 0 & 4 & 0 & 0 & 6 & 0 & 0 & 7 & 0 & 0 & 5 & 0 & 0 & 0\\
    overlap\_before & 1 & 0 & 1 & 1 & 0 & 1 & 3 & 0 & 3 & 1 & 0 & 1 & 0 & 0 & 0\\
    overlap\_before\_and\_after & 1 & 0 & 1 & 0 & 0 & 0 & 2 & 0 & 2 & 0 & 0 & 0 & 0 & 0 & 0\\
    after\_push & 1 & 0 & 0 & 1 & 0 & 1 & 3 & 0 & 0 & 3 & 0 & 0 & 0 & 0 & 0\\
    overlap\_after & 1 & 0 & 0 & 0 & 0 & 0 & 1 & 0 & 0 & 0 & 0 & 0 & 0 & 0 & 0\\
    between\_pushes & 0 & 0 & 0 & 0 & 0 & 0 & 2 & 0 & 1 & 0 & 0 & 1 & 0 & 0 & 0\\
    during\_several\_pushes & 0 & 0 & 0 & 0 & 0 & 0 & 3 & 0 & 0 & 0 & 0 & 0 & 0 & 0 & 0\\
    during\_push & 2 & 0 & 0 & 3 & 0 & 0 & 11 & 0 & 0 & 3 & 0 & 0 & 0 & 0 & 0\\
    \bottomrule
  \end{tabular*}
\end{table*}
\begin{table*}[h]
  \setlength{\tabcolsep}{0.95ex}
  \caption{\textbf{Motivational states during negative intent interpretations and neg. mot. questions within prohibition experiment}. 
    Given are the counts/number of associations of the robot's motivational states per stated utterance type. These frequencies are
    listed per session and accumulated across sessions. Symbols used: \emph{\#}: number of occurrences of the stated utterance type,
    \emph{-}: frequency of negative motivational state, \emph{+}: frequency of positive motivational state, \emph{O}: frequency of
    neutral motivational state.}
  \label{tbl:neg_int_int_mot2}
  \begin{small}
    \begin{tabular*}{\hsize}{@{\extracolsep{\fill}}llllllllllllllllllllll}
      \toprule
      &  & \multicolumn{4}{c}{P13} & \multicolumn{4}{c}{P14} & \multicolumn{4}{c}{P15} & \multicolumn{4}{c}{P16} & \multicolumn{4}{c}{P17}\\
      \cmidrule(lr){3-6}\cmidrule(lr){7-10}\cmidrule(lr){11-14}\cmidrule(lr){15-18}\cmidrule(lr){19-22}
      &  & \# & - & O & + & \# & - & O & + & \# & - & O & + & \# & - & O & + & \# & - & O & +\\
      \cmidrule(lr){3-6}\cmidrule(lr){7-10}\cmidrule(lr){11-14}\cmidrule(lr){15-18}\cmidrule(lr){19-22}
      \multirow{2}{*}{s1} & neg. int. int. & 6 & 5 & 3 & 1 & 0 & 0 & 0 & 0 & 15 & 9 & 7 & 0 & 7 & 2 & 4 & 1 & 4 & 3 & 1 & 0\\
      & neg. mot. question & 4 & 2 & 1 & 1 & 0 & 0 & 0 & 0 & 11 & 5 & 6 & 1 & 3 & 2 & 2 & 0 & 1 & 1 & 0 & 0\\
      \multirow{2}{*}{s2} & neg. int. int. & 6 & 6 & 0 & 0 & 0 & 0 & 0 & 0 & 4 & 4 & 0 & 0 & 4 & 3 & 2 & 0 & 4 & 4 & 1 & 0\\
      & neg. mot. question & 1 & 0 & 0 & 1 & 0 & 0 & 0 & 0 & 8 & 2 & 4 & 3 & 3 & 1 & 2 & 0 & 1 & 1 & 0 & 0\\
      \multirow{2}{*}{s3} & neg. int. int. & 0 & 0 & 0 & 0 & 0 & 0 & 0 & 0 & 4 & 3 & 1 & 0 & 2 & 1 & 1 & 0 & 1 & 1 & 0 & 0\\
      & neg. mot. question & 1 & 1 & 0 & 0 & 0 & 0 & 0 & 0 & 6 & 4 & 3 & 1 & 1 & 0 & 1 & 0 & 1 & 0 & 1 & 0\\
      \multirow{2}{*}{s4} & neg. int. int. & 1 & 1 & 0 & 0 & 0 & 0 & 0 & 0 & 8 & 5 & 4 & 1 & 5 & 2 & 4 & 1 & 1 & 1 & 0 & 0\\
      & neg. mot. question & 4 & 4 & 0 & 0 & 0 & 0 & 0 & 0 & 9 & 4 & 4 & 2 & 4 & 1 & 3 & 1 & 0 & 0 & 0 & 0\\
      \multirow{2}{*}{s5} & neg. int. int. & 2 & 2 & 0 & 0 & 0 & 0 & 0 & 0 & 7 & 5 & 1 & 1 & 13 & 4 & 9 & 0 & 3 & 2 & 1 & 0\\
      & neg. mot. question & 2 & 2 & 0 & 0 & 0 & 0 & 0 & 0 & 18 & 9 & 8 & 2 & 3 & 3 & 0 & 0 & 4 & 2 & 2 & 0\\
      \multirow{2}{*}{total} & neg. int. int. & 15 & 14 & 3 & 1 & 0 & 0 & 0 & 0 & 38 & 26 & 13 & 2 & 31 & 12 & 20 & 2 & 13 & 11 & 3 & 0\\
      & neg. mot. question & 12 & 9 & 1 & 2 & 0 & 0 & 0 & 0 & 52 & 24 & 25 & 9 & 14 & 7 & 8 & 1 & 7 & 4 & 3 & 0\\
    \end{tabular*}
    \setlength{\tabcolsep}{1.1ex}
    \begin{tabular*}{\hsize}{@{\extracolsep{\fill}}llllllllllllllllllllll}
      \toprule
      &  & \multicolumn{4}{c}{P18} & \multicolumn{4}{c}{P19} & \multicolumn{4}{c}{P20} & \multicolumn{4}{c}{P21} & \multicolumn{4}{c}{P22}\\
      \cmidrule(lr){3-6}\cmidrule(lr){7-10}\cmidrule(lr){11-14}\cmidrule(lr){15-18}\cmidrule(lr){19-22}
      &  & \# & - & O & + & \# & - & O & + & \# & - & O & + & \# & - & O & + & \# & - & O & +\\
      \cmidrule(lr){3-6}\cmidrule(lr){7-10}\cmidrule(lr){11-14}\cmidrule(lr){15-18}\cmidrule(lr){19-22}
      \multirow{2}{*}{s1} & neg. int. int. & 15 & 9 & 4 & 3 & 4 & 3 & 1 & 0 & 8 & 4 & 5 & 0 & 1 & 1 & 0 & 0 & 2 & 0 & 1 & 1\\
      & neg. mot. question & 8 & 7 & 2 & 1 & 4 & 4 & 1 & 0 & 2 & 1 & 1 & 1 & 3 & 1 & 3 & 0 & 0 & 0 & 0 & 0\\
      \multirow{2}{*}{s2} & neg. int. int. & 4 & 4 & 0 & 0 & 4 & 4 & 0 & 0 & 1 & 1 & 0 & 0 & 0 & 0 & 0 & 0 & 0 & 0 & 0 & 0\\
      & neg. mot. question & 2 & 1 & 1 & 0 & 3 & 3 & 1 & 0 & 10 & 7 & 4 & 0 & 3 & 3 & 0 & 0 & 0 & 0 & 0 & 0\\
      \multirow{2}{*}{s3} & neg. int. int. & 6 & 4 & 3 & 0 & 5 & 5 & 0 & 0 & 5 & 4 & 2 & 1 & 0 & 0 & 0 & 0 & 0 & 0 & 0 & 0\\
      & neg. mot. question & 2 & 0 & 2 & 0 & 0 & 0 & 0 & 0 & 7 & 5 & 3 & 0 & 3 & 2 & 0 & 1 & 0 & 0 & 0 & 0\\
      \multirow{2}{*}{s4} & neg. int. int. & 2 & 2 & 0 & 0 & 1 & 1 & 0 & 0 & 6 & 4 & 2 & 0 & 0 & 0 & 0 & 0 & 0 & 0 & 0 & 0\\
      & neg. mot. question & 1 & 1 & 0 & 0 & 3 & 3 & 0 & 0 & 6 & 5 & 2 & 1 & 3 & 2 & 1 & 0 & 0 & 0 & 0 & 0\\
      \multirow{2}{*}{s5} & neg. int. int. & 3 & 2 & 0 & 1 & 4 & 4 & 1 & 0 & 2 & 2 & 1 & 0 & 3 & 3 & 0 & 0 & 0 & 0 & 0 & 0\\
      & neg. mot. question & 1 & 1 & 0 & 0 & 2 & 2 & 0 & 0 & 13 & 10 & 3 & 1 & 8 & 7 & 1 & 0 & 0 & 0 & 0 & 0\\
      \multirow{2}{*}{total} & neg. int. int. & 30 & 21 & 7 & 4 & 18 & 17 & 2 & 0 & 22 & 15 & 10 & 1 & 4 & 4 & 0 & 0 & 2 & 0 & 1 & 1\\
      & neg. mot. question & 14 & 10 & 5 & 1 & 12 & 12 & 2 & 0 & 38 & 28 & 13 & 3 & 20 & 15 & 5 & 1 & 0 & 0 & 0 & 0\\
      \bottomrule
    \end{tabular*}
  \end{small}
\end{table*}
\begin{table*}[h]
  \setlength{\tabcolsep}{0.93ex}
  \caption{\textbf{Motivational states during negative intent interpretations and neg. mot. questions within rejection experiment}. 
    Given are the counts/number of associations of the robot's motivational states per stated utterances type. These frequencies are
    listed per session and accumulated across sessions. Note that one utterance of these types can co-occur with more than one motivational state such that
    the sum of motivational states in the table  may be larger than the total number of utterances. Symbols used: \emph{\#}: number of occurrences of the
    stated utterance type, \emph{-}: frequency of negative motivational state, \emph{+}: frequency of positive motivational state, \emph{O}: frequency of
    neutral motivational state.}
  \label{tbl:neg_int_int_mot1}
  \begin{small}
    \begin{tabular*}{\hsize}{@{\extracolsep{\fill}}llllllllllllllllllllll}
      \toprule
      &  & \multicolumn{4}{c}{P01}  & \multicolumn{4}{c}{P04} & \multicolumn{4}{c}{P05} & \multicolumn{4}{c}{P06} & \multicolumn{4}{c}{P07}\\
      \cmidrule(lr){3-6}\cmidrule(lr){7-10}\cmidrule(lr){11-14}\cmidrule(lr){15-18}\cmidrule(lr){19-22}
      &  & \# & - & O & + & \# & - & O & + & \# & - & O & + & \# & - & O & + & \# & - & O & +\\
      \cmidrule(lr){3-6}\cmidrule(lr){7-10}\cmidrule(lr){11-14}\cmidrule(lr){15-18}\cmidrule(lr){19-22}
      \multirow{2}{*}{s1} & neg. int. int. & 1 & 1 & 0 & 0 & 6 & 6 & 0 & 0 & 17 & 11 & 0 & 7 & 5 & 4 & 1 & 1 & 12 & 10 & 2 & 0\\
      & neg. mot. question & 0 & 0 & 0 & 0 & 3 & 3 & 0 & 0 & 6 & 1 & 3 & 2 & 3 & 2 & 0 & 1 & 13 & 5 & 7 & 2\\
      \multirow{2}{*}{s2} & neg. int. int. & 1 & 1 & 0 & 0 & 6 & 6 & 1 & 0 & 7 & 2 & 3 & 2 & 8 & 4 & 3 & 1 & 16 & 8 & 8 & 0\\
      & neg. mot. question & 0 & 0 & 0 & 0 & 7 & 7 & 0 & 0 & 11 & 7 & 4 & 1 & 1 & 1 & 1 & 0 & 7 & 1 & 5 & 2\\
      \multirow{2}{*}{s3} & neg. int. int. & 0 & 0 & 0 & 0 & 3 & 3 & 0 & 0 & 6 & 4 & 3 & 0 & 12 & 12 & 2 & 0 & 7 & 5 & 2 & 0\\
      & neg. mot. question & 0 & 0 & 0 & 0 & 5 & 3 & 2 & 0 & 2 & 0 & 2 & 1 & 2 & 2 & 0 & 0 & 18 & 2 & 9 & 9\\
      \multirow{2}{*}{s4} & neg. int. int. & 0 & 0 & 0 & 0 & 0 & 0 & 0 & 0 & 3 & 2 & 0 & 1 & 13 & 12 & 3 & 1 & 7 & 6 & 0 & 2\\
      & neg. mot. question & 0 & 0 & 0 & 0 & 1 & 0 & 1 & 0 & 5 & 4 & 1 & 1 & 3 & 2 & 1 & 0 & 10 & 0 & 6 & 5\\
      \multirow{2}{*}{s5} & neg. int. int. & 0 & 0 & 0 & 0 & 8 & 7 & 0 & 1 & 3 & 1 & 2 & 0 & 11 & 8 & 5 & 0 & 7 & 6 & 2 & 0\\
      & neg. mot. question & 0 & 0 & 0 & 0 & 8 & 8 & 0 & 0 & 6 & 4 & 3 & 1 & 1 & 0 & 1 & 0 & 24 & 5 & 14 & 7\\
      \multirow{2}{*}{total} & neg. int. int. & 2 & 2 & 0 & 0 & 23 & 22 & 1 & 1 & 36 & 20 & 8 & 10 & 49 & 40 & 14 & 3 & 49 & 35 & 14 & 2\\
      & neg. mot. question & 0 & 0 & 0 & 0 & 24 & 21 & 3 & 0 & 30 & 16 & 13 & 6 & 10 & 7 & 3 & 1 & 72 & 13 & 41 & 25\\
    \end{tabular*}
    \setlength{\tabcolsep}{1.12ex}
    \begin{tabular*}{\hsize}{@{\extracolsep{\fill}}llllllllllllllllllllllll}
      \toprule
      &  & \multicolumn{4}{c}{P08} & \multicolumn{4}{c}{P09} & \multicolumn{4}{c}{P10} & \multicolumn{4}{c}{P11} & \multicolumn{4}{c}{P12}\\
      \cmidrule(lr){3-6}\cmidrule(lr){7-10}\cmidrule(lr){11-14}\cmidrule(lr){15-18}\cmidrule(lr){19-22}
      &  & \# & - & O & + & \# & - & O & + & \# & - & O & + & \# & - & O & + & \# & - & O & + &\\
      \cmidrule(lr){3-6}\cmidrule(lr){7-10}\cmidrule(lr){11-14}\cmidrule(lr){15-18}\cmidrule(lr){19-22}
      \multirow{2}{*}{s1} & neg. int. int. & 6 & 3 & 3 & 0 & 6 & 6 & 1 & 1 & 0 & 0 & 0 & 0 & 9 & 6 & 4 & 1 & 7 & 4 & 2 & 3\\
      & neg. mot. question & 10 & 1 & 9 & 1 & 4 & 3 & 2 & 0 & 0 & 0 & 0 & 0 & 3 & 2 & 0 & 1 & 2 & 0 & 2 & 0\\
      \multirow{2}{*}{s2} & neg. int. int. & 3 & 2 & 1 & 0 & 6 & 4 & 4 & 0 & 0 & 0 & 0 & 0 & 4 & 3 & 1 & 0 & 2 & 1 & 1 & 0\\
      & neg. mot. question & 7 & 1 & 6 & 0 & 5 & 2 & 1 & 2 & 2 & 2 & 0 & 0 & 0 & 0 & 0 & 0 & 2 & 1 & 1 & 0\\
      \multirow{2}{*}{s3} & neg. int. int. & 1 & 1 & 0 & 0 & 6 & 4 & 1 & 1 & 1 & 1 & 1 & 1 & 6 & 6 & 0 & 0 & 0 & 0 & 0 & 0\\
      & neg. mot. question & 10 & 6 & 1 & 3 & 7 & 3 & 6 & 1 & 1 & 1 & 0 & 0 & 2 & 2 & 1 & 0 & 0 & 0 & 0 & 0\\
      \multirow{2}{*}{s4} & neg. int. int. & 4 & 3 & 1 & 0 & 5 & 2 & 5 & 0 & 0 & 0 & 0 & 0 & 0 & 0 & 0 & 0 & 0 & 0 & 0 & 0\\
      & neg. mot. question & 10 & 6 & 4 & 0 & 0 & 0 & 0 & 0 & 3 & 3 & 3 & 2 & 2 & 1 & 0 & 1 & 0 & 0 & 0 & 0\\
      \multirow{2}{*}{s5} & neg. int. int. & 4 & 3 & 2 & 0 & 2 & 2 & 0 & 0 & 0 & 0 & 0 & 0 & 0 & 0 & 0 & 0 & 0 & 0 & 0 & 0\\
      & neg. mot. question & 6 & 3 & 2 & 1 & 4 & 1 & 1 & 3 & 3 & 4 & 2 & 0 & 0 & 0 & 0 & 0 & 0 & 0 & 0 & 0\\
      \multirow{2}{*}{total} & neg. int. int. & 18 & 12 & 7 & 0 & 25 & 18 & 11 & 2 & 1 & 1 & 1 & 1 & 19 & 15 & 5 & 1 & 9 & 5 & 3 & 3\\
      & neg. mot. question & 43 & 17 & 22 & 5 & 20 & 9 & 10 & 6 & 9 & 10 & 5 & 2 & 7 & 5 & 1 & 2 & 4 & 1 & 3 & 0\\
      \bottomrule
    \end{tabular*}
  \end{small}
\end{table*}


\end{screenonly}


\clearpage

\bibliographystyle{ACM-Reference-Format}
\bibliography{arXiv-prohibition}
