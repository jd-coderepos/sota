\documentclass{llncs}






\usepackage{amsmath}

\usepackage{verbatim}
\usepackage{floatrow}
\usepackage{amssymb}
\usepackage{amsmath}
\usepackage{color} 
\usepackage{fullpage}
\usepackage{float}
\floatstyle{boxed} 
\restylefloat{figure}
\usepackage{cite}
\usepackage{subfig}
\usepackage{tabularx}
\usepackage{nameref}

\usepackage{color} 


\usepackage{algorithm}
\usepackage{algpseudocode}
\usepackage[titletoc]{appendix}



\ifx\pdftexversion\undefined
  \usepackage[dvips]{graphicx}
\else
  \usepackage[pdftex]{graphicx}
  \DeclareGraphicsRule{*}{mps}{*}{}
\fi



\newcommand{\punt}[1]{}
\newcommand{\cmnt}[1]{}


\newcommand{\nosplit}{\linebreak}

\def\nohyphens{\hyphenpenalty=10000\exhyphenpenalty=10000}

\newcommand{\tilda}{\symbol{126}}



\newcommand{\ang}[1]{\langle #1 \rangle}
\newcommand{\Ang}[1]{\Big\langle #1 \Big\rangle}
\newcommand{\ceil}[1]{\lceil #1 \rceil}
\newcommand{\floor}[1]{\lfloor #1 \rfloor}


\newcommand{\myforall}[3]{\ang{\forall \: #1 : #2 : #3 }}
\newcommand{\myexists}[3]{\ang{\exists \: #1 : #2 : #3 }}
\newcommand{\mynexists}[3]{\ang{\nexists \: #1 : #2 : #3 }}
\newcommand{\Myforall}[3]{\Ang{\forall \: #1 : #2 : #3 }}
\newcommand{\Myexists}[3]{\Ang{\exists \: #1 : #2 : #3 }}
\newcommand{\Mynexists}[3]{\Ang{\nexists \: #1 : #2 : #3 }}


\newcommand{\defined}{\;\;\triangleq\;\;}




\renewcommand{\implies}{\Rightarrow}
\newcommand{\follows}{\Leftarrow}

\newcommand{\myspace}{\hspace*{0.0in}}
\newcommand{\halflinespacing}{\vspace*{0.5em}}
\newcommand{\greaterlinespacing}{\vspace*{0.75em}}





\newcommand{\myif}{\textbf{if}}
\newcommand{\mythen}{\textbf{then}}
\newcommand{\myelse}{\textbf{else}}
\newcommand{\myendif}{\textbf{endif}}
\newcommand{\myor}{\textbf{or}}
\newcommand{\myand}{\textbf{and}}
\newcommand{\mynot}{\textbf{not}}



\renewcommand{\thetheorem}{\arabic{theorem}}









\newtheorem{observation}[theorem]{Observation}




\newtheorem{assume}{Assumption}




\newtheorem{axiom}[theorem]{Proposition}


\newtheorem{invariant}[theorem]{Invariant}





\newtheorem{acknowledgement}[theorem]{Acknowledgement}
\newenvironment{proofsketch}[1][Proof sketch]{\noindent\textbf{#1.} }{\hfill \
\shist{\comm(H^{T_i})\cup\{\applfn{T_i}{H}\}}{H^{T_i}},

\label{eq:wr}
c_i <_{H2} r_j(x, v)


Now, we have two cases based on the ordering of  and  in :

\begin{itemize}
\item[] Case (1) : Since , from r-w conflict order we get that . This implies that  can not be 's \lastw{} in  which is a contradiction. 

\item[] Case (2) : From w-r conflict order equivalence of , we get that . Here, we again have two cases based on the ordering of  and  in , Case (2.1) : In this case, we have that  which implies that  is not 's \lastw{} in , a contradiction. Case (2.2) : From w-w conflict order equivalence of , we get that . Combining this with \eqnref{wr}, we have that . This implies that  can not be 's \lastw{} in  which is a contradiction.
\end{itemize}
Thus in all the cases, we get that  can not be 's
\lastw. This implies that  is legal. 
}
\end{proof}
From the above lemma we get the following interesting corollary.

\begin{corollary}
\label{cor:coop-legal}
Every \coop{} history  is \legal{} as well.
\end{corollary}
Based on the conflict graph construction,
we have the following graph characterization for \coop.

\begin{theorem}
\label{thm:ap-graph}
A \legal{} history  is \coop{} iff  is acyclic. 
\end{theorem}

\begin{proof}


\noindent
\textit{(Only if)} If  is \coop{} and legal, then
 is acyclic:
Since  is \coop{}, there exists a legal t-sequential
history  equivalent to  and  respects
 and . Thus from the conflict graph
construction we have that  is a sub
graph of . Since  is sequential, it can be inferred that
 is acyclic. Any sub graph of an acyclic graph is also
acyclic. Hence  is also acyclic. 

\vspace{1mm}
\noindent
\textit{(if)} If  is \legal{} and  is acyclic then  is
\coop: 
Suppose that  is acyclic. Thus we can
perform a topological sort on the vertices of the graph and obtain a
sequential order. Using this order, we can obtain a sequential
schedule  that is equivalent to .
Moreover, by construction,  respects  
and . 

Since every two events related by the conflict relation (w-w, r-w, or
w-r)in  are also related by , we obtain 
. 
Since  is legal,  is also legal. 
Combining this with \lemref{ap-eqv-legal}, we get that  is also
\legal. 
This satisfies all the conditions necessary for  to be \coop. \qed
\end{proof}



\subsection{Proofs of local correctness and non-interference}
\label{subsec:ap-local}

\begin{theorem}
\label{lem:ap-perm_ni}
For every local correctness property , 
.
\end{theorem}

\begin{proof}
We proceed by contradiction. Assume that  is in
 but not in . 
More precisely, let  be an aborted transaction in , 
 and  
, such that  .

On the other hand, since , we have  
.
Since  is local and , we have , 
 . 
Thus, for all transactions  that
 committed before the last event of , we have  
.    

Now we construct  as , except that the aborted
operation of  is replaced with the last operation of  in
.  Since  is in , and  is local, 
we have . 
For all transactions  that committed before the last event of  in
, we have  .
Hence, since  is local, we have .
But, by construction, ---a contradiction with
the assumption that .
\qed
\end{proof}



\subsection{Proof for Garbage Collection}
\label{subsec:ap-garbage}

\begin{lemma}
\label{lem:ap-trimmed}   
 is in {\clo} if and only if  and  are in {\clo}. 
\end{lemma}

\begin{proof}
(Only if)
Suppose that  is in {\clo}.
By \corref{coop-legal},  is legal.
Since  is a prefix of , it is also legal,
and its conflict
graph is a sub-graph of . By Theorem~\ref{thm:graph}, 
 is acyclic and, thus,  is
in {\clo}.

Further, let  be any read operation in
.
Since  is legal,   is also legal.
Note that since no read operation of {\obsolete} transactions in
 appears in ,  is not in
.  
Let  be  's \textit{\lastw{}} in . 
If  appears in  , then 
 is also  's \textit{\lastw{}} in
, and, thus,  is also legal.    
Now, suppose, by contradiction, that  does not appear in
, i.e.,  is not the last ({\obsolete})
transaction in  to commit a value on , i.e., there exists a
transaction  writing to  such that  appears after
  in .
Since  is  's \textit{\lastw{}} in , 
appears after  in . 
But  is {\obsolete}, and, thus, no read operation 
can appear before  in ---a contradiction.   
Thus,  is  's \textit{\lastw{}} in
, and, hence,  is legal.

Since   is a legal sub-sequence of ,
 is a sub-graph of  and, by Theorem~\ref{thm:graph}, 
 is acyclic and  is in {\clo}.

\noindent
(If) Suppose now that  and  are in {\clo}. 
By  \corref{coop-legal}, both histories are legal, and, by
Theorem~\ref{thm:graph}, produce acyclic conflict graphs.
Immediately, every read operation in  that also appears in
 is legal.
By the arguments above, the {\lastw} for every read operation in
 is also its {\lastw} in .
Thus,  is legal.

Recall that  can be represented as  with
read events of transactions in  inserted in accordance to its prefix  .
Thus,  can be represented as  
with several additional edges directed to and from transactions in
.

Suppose, by contradiction that  contains a cycle . 
Since  is acyclic,  must contain an edge
directed to or from a transaction in  that does not appear in .   

Thus, we can represent the cycle  as
, where  and for all
, . 
Since   is acyclic  must contain an edge
that does not appear in . 
Let  be the latest transaction in
 such that .

Note that . This is because 
must precede or be concurrent to  in . 
Since , by the construction of ,
 must have committed in .
But then ---a contradiction.

Now, since ,  cannot be complete in
.   
Again, by the construction of , no transaction
that is not complete in  can begin before
 completes. 
Hence,  precedes  in the real-time order and, since
 also appears in  , . 
Thus,  contains a cycle ---a contradiction. 

Thus,  is acyclic and, by Theorem~\ref{thm:graph}, 
is in {\clo}. \qed
\end{proof}





\end{document}
