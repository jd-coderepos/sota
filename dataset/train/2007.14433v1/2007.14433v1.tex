\documentclass{article}









\usepackage[nonatbib,preprint]{neurips_2020}
\usepackage{graphicx}
\usepackage[]{algorithm2e}

\usepackage{adjustbox}
\usepackage[utf8]{inputenc} \usepackage[T1]{fontenc}    \usepackage{hyperref}       \usepackage{url}            \usepackage{booktabs}       \usepackage{amsfonts}       \usepackage{nicefrac}       \usepackage{microtype}      \usepackage{color}
\usepackage{xcolor}
\usepackage{ulem}
\usepackage{amsmath}
\usepackage{mathtools}
\usepackage{graphicx,subfigure}

\newcommand{\norm}[1]{\left\lVert#1\right\rVert}
\newcommand{\MS}[1]{{\color{blue}  #1}}
\newcommand{\NR}[1]{{\color{orange}  NR: #1}}




\title{Cassandra: Detecting Trojaned Networks from Adversarial Perturbations}



\author{
    Xiaoyu Zhang\thanks{Center for Research in Computer Vision, University of Central Florida} \\
    \texttt{x.zhang@knights.ucf.edu} \\
    \And
    Ajmal Mian\thanks{Computer Science and Software Engineering, The University of Western Australia} \\
    \texttt{ajmal.mian@uwa.edu.au} \\
    \And
    Rohit Gupta\textsuperscript{*} \\
    \texttt{rohitg@knights.ucf.edu} \\
    \And
    Nazanin Rahnavard\thanks{Department of Electrical and Computer Engineering, University of Central Florida} \\
    \texttt{nazanin@eecs.ucf.edu} \\\
    \And
    Mubarak Shah\textsuperscript{*} \\
    \texttt{shah@crcv.ucf.edu} \\
}




\begin{document}

\maketitle

\begin{abstract}





\vspace{-3mm}
Deep neural networks are being widely deployed for many critical tasks due to their high classification accuracy. In many cases, pre-trained models are sourced from vendors who may have disrupted the training pipeline to insert Trojan behaviors into the models. These malicious behaviors can be triggered at the adversary's will and hence, cause a serious threat to the widespread deployment of deep models. We propose a method to verify if a pre-trained model is Trojaned or benign. Our method captures fingerprints of neural networks in the form of adversarial perturbations learned from the network gradients. Inserting backdoors into a network alters its decision boundaries which are effectively encoded in their adversarial perturbations. We train a two stream network for Trojan detection from its global ( and  bounded) perturbations and the localized region of high energy within each perturbation. The former encodes decision boundaries of the network and latter encodes the unknown trigger shape. We also propose an anomaly detection method to identify the target class in a Trojaned network. Our methods are invariant to the trigger type, trigger size, training data and network architecture. We evaluate our methods on MNIST, NIST-Round0 and NIST-Round1 datasets, with up to 1,000 pre-trained models making this the largest study to date on Trojaned network detection, and achieve over 92\% detection accuracy to set the new state-of-the-art.
\end{abstract}





\vspace{-5mm}
\section{Introduction}
\vspace{-3mm}
Deep neural networks (DNNs) are the main driving force behind the current success of Artificial Intelligence. However, training DNN models requires enormous amounts of data and computational resources. Hence, many users prefer to source and deploy pre-trained models in their, often security critical, applications such as drug discovery~\cite{chen2018rise,zhang2018seq3seq}, facial recognition~\cite{sun2015deepid3}, autonomous driving~\cite{geiger2012we}, surveillance~\cite{javed2002tracking}. It is well known that DNNs easily learn any bias that is present in the training data. Vendors of DNN models, with malicious intentions, can exploit this vulnerability of DNNs and intentionally inject Trojan behavior into the network during the training process. This is generally achieved by inserting a trigger into some of the  samples and then training the DNN to exhibit malicious behavior for data that contains the trigger and normal behavior for data without the trigger. With full control over the DNN training process, the adversary is able to choose any trigger shape. Triggers are chosen such that they do not appear suspicious to the human observer e.g. a yellow rectangular sticker on a stop sign can be used to trigger a DNN to classify it as a speed limit sign.  Since only the adversaries have knowledge of the trigger, they can initiate malicious behaviour at will and with no knowledge of the trigger, users of pre-trained models may not even suspect the presence of backdoors. This causes a serious threat to the widespread deployment of pre-trained models. Note that attacking Trojaned DNNs is much easier and different than adversarial attacks on clean DNNs, since the former has access to the DNN training process itself, while the latter only exploits intrinsic vulnerabilities of neural networks~\cite{akhtar2018threat}.   









Given that noise-based adversarial attacks are inherent to CNN models, it is no surprise that trigger-based Trojan attacks also exist~\cite{yuan2019adversarial,tramer2017ensemble}. Trojans are generally inserted into the deep model during training or transfer learning~\cite{liu2020survey,liu2017trojaning,evtimov2017robust,chen2017targeted}. A backdoor is typically inserted into a network~\cite{gu2017badnets} to make the CNNs mis-classify some specific class or classes. Instead of training a model with a dataset poisoned with triggers, another possible way the adversary can Trojan a network is by modifying the weights of selected neurons so that the model responds maliciously to a specific trigger \cite{liu2017trojaning}.  

Current challenges for Trojan (backdoor) detection in practice are: 1) Lack of a deep learning-based model trained on a large-scale dataset for Trojan detection; 2) Unavailability of trigger information for a suspected Trojan infected model; usually only limited training data of clean samples is available;  3) Very limited information which can be obtained from the query model predictions since the test accuracy for Trojaned DNNs is normal for clean inputs, and 4) The target class in the infected model is unknown, and it is computationally expensive to search all possible targeted attacks when the output labels are in the hundreds.



To address these challenges, we propose the first deep learning based Trojan Detection Network (TDN). Our method has two stages, the first one is a two stream neural network that outputs the probability of a model containing a Trojan, and the second stage predicts the target class in a Trojaned model. Our contributions are summarized as follows. First, we propose a deep neural network for Trojan detection from only a few clean samples. To the best of our knowledge, we are the first to use a DNN classifier, trained on a large scale dataset of benign and Trojaned models, for Trojan detection. Second, we propose a method for target class prediction in a Trojaned model. We introduce a new variable () that quantifies the difficulty of attacking a model. This variable is a critical indicator for the target class of a Trojan infected model.


{\bf Theoretical Justification:} Inserting Trojan behaviour into a network essentially puts an additional constraint on the model optimization during the training process. The model must learn to exhibit normal behavior and achieve an expected high classification accuracy on clean training/validation samples but exhibit the chosen malicious behaviour on samples containing a trigger, a localized pattern. This  has two important consequences. Firstly, the decision boundaries of the model must adjust to allow such a behavior. Secondly, the model must become more responsive to local patterns (the trigger). Our hypothesis is that if we can encode these two aspects, we will be able to detect Trojaned models accurately. For the former, we use universal adversarial perturbations \cite{moosavi2017universal} which, being image agnostic, reasonably capture a fingerprint of the decision boundaries. For the latter, we look for a localized region of high energy in the adversarial perturbation. Thirdly, we also hypothesize that Trojaned models are easier to fool with minimal universal perturbation energy compared to clean models. Our proposed method basically capitalizes on these three factors to detect Trojaned networks and the target class of such networks. 







\vspace{-4mm}
\section{Related Work} \vspace{-4mm}

Adversarial attacks on CNNs have focused on the phenomenon of noise-based adversarial examples~\cite{szegedy2013intriguing,akhtar2018defense}, which are visually almost indistinct from the original images, but can mislead DNN classifiers into making incorrect predictions. Even universal adversarial perturbations~\cite{moosavi2017universal} have been discovered that are image agnostic and when added to any image of any class, can cause the DNN to mis-classify them. By computing singular vectors of the Jacobian matrices of hidden layers, universal perturbations can be constructed with very few images~\cite{Khrulkov_2018_CVPR}. Adversarial attacks generally do not assume access to the training process of deep models. A comprehensive survey of such method is reported in \cite{akhtar2018threat}. In this paper, we focus on defending against Trojan attacks where the attacker disrupts the training pipeline of the DNN to insert a backdoor. 






The risk of Trojan models arises when the training process of a DNN is outsourced or a pre-trained model from an untrusted source is deployed. This security risk was first investigated in Badnets\cite{gu2017badnets}. It was shown that backdoors in networks infected with Trojans can remain a threat even after transfer learning. Chen et al.~\cite{chen2017targeted} proposed a backdoor attack algorithm that uses poisoned data to contaminate the CNN model. Trojaning attack~\cite{liu2017trojaning} introduced a way to generate triggers and maximize the activation of some specific neurons to insert a backdoor. The embedded backdoors are stealthy and the unexpected malicious behavior is activated only by triggers, making them extremely challenging to detect with only clean data samples. 

Defense methods were first developed to detect adversarial images ~\cite{Akhtar_2018_CVPR,xie2019feature,yuan2019adversarial,Liao_2018_CVPR,grosse2017statistical,hendrycks2016early}. Metzen et al.~\cite{metzen2017detecting} detect  adversarial perturbations with a target classification network. Feinman et al.~\cite{feinman2017detecting} also use a binary classifier to detect adversarial perturbations. Magnet~\cite{meng2017magnet} trains a classifier on manifolds of normal examples to discriminate adversarial perturbations without any prior knowledge of the attack. Safetynet~\cite{lusafetynet} is designed to detect adversarial-noise based attacks and exploits the different adversarial perturbations produced to train a SVM classifier.





Methods for detecting and defending against Trojan attacks have also been proposed. Liu et al. ~\cite{liu2018fine} proposed a pruning and fine-tuning procedure to suppress backdoor attacks.  Chen et al. \cite{chen2018detecting} proposed Activation Clustering methodology for detecting and removing backdoors from DNNs. SentiNet\cite{chou2018sentinet} uses the behavior of adversarial misclassification of poisoned networks to detect an attack. However, all these methods fail in the realistic settings where access to poisoned data is not available.
Neural Cleanse~\cite{wang2019neural} was the first  method to detect Trojan infected models with clean samples by reverse engineering the trigger. They employ the Median Absolute Deviation (MAD) technique to  compute the anomaly in the  norm of the reversed triggers to detect Trojaned models. However,  the trigger must be reverse engineered  for each class, which is not scalable in practice for DNNs with hundreds and thousands of classes.  DeepInspect~\cite{chen2019deepinspect} uses conditional GAN to reconstruct trigger patterns for Trojan detection.
NeuronInspect~\cite{huang2019neuroninspect} detects backdoor from the output features, such as sparsity, smoothness, and persistence of saliency maps obtained from back-propagation of the confidence scores. 
 Tabor~\cite{guo2019tabor} propose metrics to measure the quality of reversed triggers  and achieve improved performance than Neural Cleanse by introducing several regularization terms to refine the generated triggers. 


The above methods~\cite{wang2019neural,chen2019deepinspect,huang2019neuroninspect} 
are sub-optimal because they are not learning-based and employ the MAD technique and manually tuned anomaly thresholds to detect the outliers of reverse engineered triggers. More importantly, none of these techniques report results on large scale data of benign/Trojaned models and none of them can predict the target class of a Trojaned model. To address these challenges, we propose {\it Cassandra}, a Trojan detection method that exploits universal adversarial perturbations~\cite{Khrulkov_2018_CVPR} generated from a very limited number of clean samples. Given their image-agnostic nature, we compute universal adversarial perturbations for a batch of clean samples, where the batches could be as few as 5. Note that this holds even if the number of classes is in thousands, unlike prior work such as Neural Cleanse, where one perturbation per class is necessary. Our method also provides the target class of a Trojan infected model.

\vspace{-4mm}
\section{Detecting Trojan Infected Models}
\label{sec:blind}
\vspace{-3mm}

\begin{figure}
\centering
\includegraphics[width =\linewidth, trim=2cm 0cm 0cm 0cm, clip]{figs/mtv_2.pdf}
\vspace{-10mm}
\caption{(a) and (b) illustrate how decision boundaries can change after inserting a Trojan in a network. The Trojaned model (b) has a complicated decision boundary after being compromised.
Introducing triggers in the training data changes the decision boundary of model (b) to accommodate the poisoned samples. This makes it easier to perturb the class label of a sample from class A to B (shown by arrows) since the distance across decision boundary is smaller compared to in (a). (c) Shows universal adversarial perturbations computed using  (top row) and  norm (bottom row) for benign and Trojaned models. Models from left to right: Trojaned Inception-v3, Trojaned DenseNet-121, benign ResNet50, and benign DenseNet121.}
\label{fig:db}
\vspace{-4mm}
\end{figure}










During training, a neural network simultaneously learns feature representation and decision boundaries that partition the feature vector space into the respective classes. When an adversary inserts a backdoor into a network, the decision boundaries are altered. Our hypothesis is that Trojan infected networks exhibit decision boundaries that are different from typical, benign classification networks. Our approach exploits this fact by retrieving the fingerprints of the decision boundaries of a network, and subsequently trains a classifier on these fingerprints to classify a query network as benign or Trojan infected. We use adversarial perturbations to retrieve fingerprints of the decision boundaries of the query network. In contrast to image specific perturbations, universal perturbations~\cite{moosavi2017universal} are image agnostic,  such that the generated perturbations when added to any input image sends it across the decision boundary to change its label. The success of universal perturbations is measured by its fooling rate, the  proportion of images that are successfully mis-classified after the perturbation is added. Since universal adversarial perturbations capture the geometry of the decision boundaries~\cite{moosavi2017universal}, the perturbations for benign and Trojan models are expected to be significantly different in character. 





\vspace{-2mm}
\subsection{Fingerprinting Decision Boundaries with Adversarial Perturbations}
\vspace{-2mm}







We formulate Trojan detection as a classification problem. For a query neural network model, ,  we define a  Trojan detection classifier  as

\vspace{-3mm}
\label{eq:cf1}
\underset{x \sim \mu}{\mathbb{P}}\{f(x +  \triangle x) \neq y \} \ge 1- \delta \ \quad \text{and} \quad
\underset{x \sim \mu}{\mathbb{P}}\{\hat{f}(x +  \triangle \hat{x}) \neq \hat{y}\} \ge 1-\delta \ ,\label{eq:ef}
E = \norm{\triangle x}_1 = \norm{h(x)}_1,
\label{eq:defineAD}
\gamma = E / S,
-1em] \hline
 240  & I     & 60  & I     &  \\
        \hline
 240  & II    & 60  & II    &  \\
        \hline
  240  & I     & 60  & II    &  \\
        \hline
  240  & II    & 60  & I     &  \\
        \hline
  480  & I, II & 120 & I, II &  \\
  \hline
\hline
		\end{tabular}
\label{table:model-table7}
\end{table}

\begin{table}[b!]
\caption{\textbf{Trojan detection accuracy on Triggered MNIST, NIST-Round0 \& Round1 datasets}.}
\begin{center}
\label{tab:comaprison_f}
\begin{tabular}{|l|c|c|c|}
\hline
 & \multicolumn{3}{c|}{\textbf{Dataset}} \\
\hline
\textbf{Method} & \textbf{Triggered MNIST} & \textbf{NIST-Round0}  &\textbf{NIST-Round1} \\ \hline
\-1em] \hline
 without  &    &  &  \\  \hline
 with predicted  &   &  &  \\\hline
 with ground truth  &  &  &  \\ \hline

\end{tabular}
\end{center}
\end{table}

Table \ref{tab:tar_pred} shows our target class prediction results. The proposed two stage prediction algorithm based on the attack difficulty and predicted  improves the classification accuracy significantly over the baseline (without ) from 76.1\%, 72.5\% and 70.0\% to 90.0\%, 94.7\% and 88.1\% on Triggered MNIST, NIST-Round0 and NIST-Round1 datasets respectively. Using Ground truth P(Trojan) further improves classification accuracy which demonstrates attack difficulty is a critical indicator of target class.















 
\subsection{Ablation Study}
\vspace{-2mm}

{\bf Trojan Detector Network Modules:}
In Table~\ref{tab:tro_det}, we explore different network architectures and the functionality of individual modules of our method. Using only universal perturbations computed from the complete training data of NIST-Round0, we achieved 77.5\% classification accuracy. After dividing the training data into 10 batches (these are different from the training mini-batches), we generate 10 perturbations for each model. With these 10 perturbations, the accuracy improves to 85\%. Adding the attack difficulty further improves the classification accuracy in all cases. Finally, with multi-batch and two stream architecture we achieve 92.5\% classification accuracy. 

























\begin{table}[h!]
\caption{\textbf{Effects of using multiple perturbations and attack difficulty.} Trojan detection accuracy on the NIST-Round0 validation data improves significantly after using multiple perturbations (n=10) calculated from different batches of training data. Using  and  perturbations in a two stream architecture combined with attack difficulty () further improves the accuracy.}
\begin{center}
\label{tab:tro_det}
\begin{tabular}{|l|c|c|}
\hline
 & \multicolumn{2}{c|}{\textbf{Classification Accuracy}} \\ 
\hline
\textbf{Input for classifier} & \textbf{without }  & \textbf{with } \\
 \hline
 \hline 
 perturbation from all training data &   &  \\ \hline
multi-batch  perturbations                             &  &   \\ \hline
 multi-batch + two stream ( \& ) perturbations  &  &   \\ \hline

\end{tabular}
\end{center}
\end{table}

\begin{table}[h!]
\caption{\textbf{Effect of perturbation generator hyper-parameters on Trojan detection accuracy}
}
\begin{center}
\label{tab:comap_hyper}
\begin{tabular}{|c|c|c|c|c|c|}
\hline
 & \multicolumn{5}{c|}{\rule{0pt}{1.2em}\textbf{  Perturbation Magnitude ()}} \\ \hline
 & \multicolumn{5}{c|}{\rule{0pt}{1.2em} = 10, \#  iterations = 10} \\ \hline
\textbf{ of Iterations} &\textbf{0.1} & \textbf{0.2}  & \textbf{0.4} & \textbf{0.8}  & \textbf{1}  \\ \hline
\-1em] \hline
  5 &  &   &  &  &  \\ \hline
 10 &  &   &  &  &  \\ \hline
 15 &  &   &  &  &  \\ \hline
\end{tabular}
\end{center}
\end{table}





{\bf  Universal Perturbation Generator Hyper-parameters:} The choice of hyper-parameters may impact on the effectiveness of the generated universal adversarial perturbations for various tasks. However, our experiments show that the proposed method is robust to these parameters. We compare the mean classification accuracy when using different number of iterations and magnitudes for  and  bounded universal adversarial perturbations, and find that the Trojan detection accuracy varies only slightly as shown in Table~\ref{tab:comap_hyper}. 



























\vspace{-3mm}
\section{Conclusion}
\vspace{-3mm}

We proposed the first deep learning based method, that is trained on a large scale dataset of Trojaned and clean models, for detecting Trojan infected models. 
We exploit the universal adversarial perturbations to retrieve the fingerprints of Trojans in the DNNs and train our proposed TDN based on the features of the perturbations and attack difficulty to discriminate benign and Trojaned models. We also proposed simple variable, coined attack difficulty , to measure the energy needed to achieve an average unit fooling rate. Based on the attack difficulty, we proposed a two stage target class prediction method that can predict the target class of a Trojaned model in addition to the Trojan probability. This provides further information on the type of malicious behaviour embedded in a Trojan infected model e.g. which identity is being impersonated in a Trojaned face recognition model.












\section*{Acknowledgments}
This research  was supported in part under ARC Discovery Grant DP190102443. Xiaoyu Zhang and Rohit Gupta were supported by University of Central Floria ORC fellowships. 

\bibliographystyle{unsrt}
\bibliography{egbib}

\section*{Supplementary Material}
\vspace{-3mm}
\subsection*{TrojAI Leaderboard Results on NIST-Round0 Dataset}
\vspace{-3mm}
\label{sec:leaderboard}

Figure \ref{fig:TrojAI_LB} shows a snapshot of the TrojAI Leaderboard for NIST Round0 (see Section \ref{sec:nist_data} for dataset description). These results are compiled by the NIST server using a held-out test set not available publicly. The snapshot was taken at 21:30 hours on 10 June 2020 and adjusted to fit on this page without interfering with the results. We can see that Cassandra outperforms all other competitors by a significant margin. The best results in terms of Cross-Entropy Loss and ROC-AUC for each method are repeated in Table~\ref{tab:F1}.  Notice that we have the lowest loss and the highest ROC-AUC.

\begin{figure}[h!]
\centering
\includegraphics[width=0.9\columnwidth]{figs/leaderboard_snapshot.png}
\caption{A snapshot of the TrojAI Leaderboard shows our method on top.}
\label{fig:TrojAI_LB}
\end{figure}












\begin{table}[h!!!]
\caption{TrojAI Leaderboard results on NIST-Round0 dataset sorted by ROC (AUC). We have included the best results in terms of Loss and ROC for our competitors. Best results in each column are in bold. Note that we cannot access the F1 score, Precision and Recall for other methods.}
\begin{center}
\label{tab:F1}
\begin{tabular}{|l|c|c|c|c|c|}
\hline
\textbf{Team} & \textbf{Loss (Cross-Entropy)} & \textbf{ROC (AUC)} &\textbf{F1} &\textbf{Precision} & \textbf{Recall} \\ \hline\hline
Cassandra   & {\bf 0.1601} & {\bf 0.9833} & {\bf0.9746}&{\bf0.9897} &{\bf0.9600}\\ \hline
UCF-XR     & 0.3149 & 0.9575 &-&-&-  \\ \hline
UCF-XR     & 0.2882 & 0.9563 &-&-&- \\ \hline
IceTorch   & 0.2891 & 0.9320 &-&-&-\\ \hline
IceTorch   & 0.2917 & 0.9275 &-&-&-\\ \hline
\end{tabular}
\end{center}
\end{table}







\subsection*{ MNIST Model Generation}

\subsubsection*{Clean Model Generation}

 The data is split into training set : 60,000 images (6,000 images per class), and test set: 10,000 images (1,000 from each class). The clean data are used for training 300 benign/clean models with three architecture types (ModdedBadnet, Badnet and ModdedLenet5net), each with 100 models (see Table~\ref{table:model-table1}) .

\subsubsection*{Trojaned Model Generation} 

\textbf{Clean Data}: The MNIST dataset has 10 classes with  70,000 clean images (without triggers). 

\textbf{Triggered Data}: Two types of triggers, Type I and Type II (see Figure in main paper) were inserted into  images of MNIST dataset.
The Triggered MNIST data was combined with clean data to generate Trojaned models. The following three data splits were used in our experiments:\\
Data split 1: training: 60,000 (triggered data: 10\%),  testing:10,000 (triggered data: 10\%).\\
Data split 2: training: 60,000 (triggered data: 15\%),  testing:10,000 (triggered data: 15\%).\\
Data split 3: training: 60,000 (triggered data: 20\%),  testing:10,000 (triggered data: 20\%).
    
\textbf{Models}: 
In addition to the 300 benign models, another 600 Trojaned models of the same three architectures (ModdedBadnet, Badnet and ModdedLenet5net) were generated. 
 Trojaned models were trained by the Triggered MNIST data and clean data where the proportion of triggered data varied as 10\%, 15\% and 20\%. Table~\ref{table:model-table1} shows the details of both clean  and infected models trained for any-to-any Trojan attack. Any-to-one attack models were generated similar to any-to-any models.
300 Trojaned models were trained by any-to-any targeted attack, and another 300 were trained for any-to-one targeted attack.

Evaluations of ModdedBadnet, Badnet and ModdedLeNet5 models are shown in Table~\ref{tab:MNIST_model-A} for any-to-any attack and in Table~\ref{tab:MNIST_model-B} for any-to-one targeted attack. The clean models and Trojaned models both have high classification accuracy when the test data is clean. The clean  models also have high classification accuracy when the test data is triggered. Since there is no Trojan in the clean model, the triggered image samples are correctly classified. However, for the Trojaned models, the classification accuracy (100  Attack Success Rate) for triggered data is low since the triggered images are misclassified. The tables show Attack Success Rates only for the triggered data which is very high. These results imply that the Trojan (backdoor) was successfully inserted into the models.



\begin{table}[htbp]
	\centering
	\caption{Three models trained on Triggered MNIST dataset. Half the models are for any-to-any attack and half are for any-to-one attack. For the latter case each model only has one target class.}
	\footnotesize


		\begin{tabular}{|l|l|c|c|c|c|}
 		\hline
{\bf  Model name} & {\bf Model Architecture} & {\bf Trigger} & {\bf Triggered data } & {\bf \#} \\
		\hline
		\hline
   ModdedBadNet & 2 Conv + 1 Dense & Type I, II & 10\%, 15\% and 20\% & 100+100\\
		\hline
   BadNet       & 2 Conv + 2 Dense & Type I, II & 10\%, 15\% and 20\% & 100+100\\
		\hline
   ModdedLeNet5 & 3 Conv + 2 Dense & Type I, II & 10\%, 15\% and 20\% & 100+100 \\


 		\hline
		\end{tabular}
\label{table:model-table1}


\vspace{5mm}

\centering
        \caption{Attack success rate and classification accuracy for three types of trojaned models (any-to-any attack) for Triggered MNIST dataset. Success rate is the proportion of images for which predictions by the Trojaned model is changed to an incorrect label.}
        \begin{tabular}{|l|c|c|c|c|c|}
        \hline
&  \multicolumn{4}{c|}{\textbf{Trojaned Model}} & \textbf{Clean Model} \\
        \hline

         & \multicolumn{2}{c|}{\textbf{Attack Success Rate}} & \multicolumn{3}{c|}{\textbf{Classification Accuracy}}  \\
        \hline
        \textbf{Model Type}  & \textbf{Trigger I} & \textbf{Trigger II}  & \textbf{Trigger I} & \textbf{Trigger II} & \\
        \hline\hline
        BadNet          & 98.7 & 98.7 & 98.9 & 99.0 & 99.1 \\
        ModdedBadNet    & 97.3 & 97.6 & 97.2 & 96.5 & 98.8 \\
        ModdedLeNet5net & 97.8 & 98.6 & 98.0 & 97.3 & 98.7 \\
\hline
    \end{tabular}
        \label{tab:MNIST_model-A}\end{table}




 \begin{table}[h!]
        \centering
        \caption{Attack success rate and classification accuracy for three types of trojaned models (any-to-one targeted attack) on the Triggered MNIST dataset. Success rate is the proportion of images that changed label to the target class for Trojaned model.}
        \begin{tabular}{|l|c|c|c|c|}
        \hline
&  \multicolumn{4}{c|}{\textbf{Trojaned Model}}  \\
        \hline

         & \multicolumn{2}{c|}{\textbf{Attack Success Rate}} & \multicolumn{2}{c|}{\textbf{Classification Accuracy}} \\
        \hline
         \textbf{Model Type} & \textbf{Trigger I} & \textbf{Trigger II}  & \textbf{Trigger I} & \textbf{Trigger II} \\
        \hline\hline
        Badnet  &99.1 & 99.0  & 98.8 & 98.9 \\
        ModdedBadnet &98.5 & 98.3 & 97.6 & 97.4\\
        ModdedLenet5net &98.8 & 98.4 & 98.0 & 97.5 \\
\hline
    \end{tabular}
        \label{tab:MNIST_model-B}\end{table}








\hfill








\newpage

\subsubsection*{NIST Round0 and NIST Round1 Datasets}
\label{sec:nist_data}
The NIST datasets consist of CNN classification models for traffic sign signals. Half of the models are benign models and half are Trojaned models. The models have three architectures namely, Inception-v3, DenseNet-121, and ResNet50. The models were trained on synthetically created image data of artificial traffic signs superimposed on road background scenes. The Trojaned models have been poisoned with triggers of different color, size and shape. \textbf{Round0} dataset consists of 200 models, while \textbf{Round1} dataset has 1,000 models. NIST also holds a sequestered test dataset to evaluate models. For that, models must be uploaded to the TrojAI Leaderboard website. Section \ref{sec:leaderboard} and Table \ref{tab:F1} discuss our results on the TrojAI Leaderboard.







Table \ref{tab:NIST0_model} and Table \ref{tab:NIST1_model} show the model details and the performance of the three architecture types present in the NIST Round0 and Round1 datasets.  Notice that the Trojan infected models have accuracy at par with the clean models and yet they have a very high attack success rate on the triggered data.

\begin{table}[h]
        \centering
        \caption{Attack success rate (for Trojan trigger infused data) and top-1 classification accuracy (for clean data) for NIST-Round0 dataset. Success rate is the proportion of images for which the prediction changes to the target label in Trojaned models.}
        \begin{tabular}{|l|c|c|c|c|}
        \hline
& \multicolumn{2}{c|}{\textbf{Trojan Infected Model}} & \textbf{Clean Model} &  \\
        \hline
        \textbf{Model Type} & \textbf{Attack Success Rate} & \multicolumn{2}{c|}{\textbf{Classification Accuracy}} & \# \textbf{models} \\
        \hline \hline
        DenseNet-121 & 99.82 & 99.76 & 99.90 & 63 \\
        Inception-v3 & 99.87 & 99.69 & 99.75 & 69\\
        ResNet50    & 99.80 & 99.68 & 99.76 & 68 \\
        \hline
        \# \textbf{models} & \multicolumn{2}{c|}{100} & 100 & \textbf{200} \\
\hline
    \end{tabular}
        \label{tab:NIST0_model}\end{table}

\vspace{5mm}

\begin{table}[h]
\centering
        \caption{Attack success rate (for Trojan trigger infused data) and top-1 classification accuracy (for clean data) for three types of Trojaned and clean models from the NIST-Round1 dataset. Success rate is the proportion of images for which the prediction changes to the target label in Trojaned models.}
        \begin{tabular}{|l|c|c|c|c|}
        \hline
&  \multicolumn{2}{c|}{\textbf{Trojan Infected Model}} & \textbf{Clean Model} & \\
        \hline
        \textbf{Model Type}  & \textbf{Attack Success Rate} & \multicolumn{2}{c|}{\textbf{Classification Accuracy}} & \# \textbf{models}  \\
        \hline \hline
        DenseNet-121  & 99.88 & 99.81 & 99.88 & 313 \\
        Inception-v3  & 99.84 & 99.85 & 99.89 & 250 \\
        ResNet50     & 99.58 & 99.81 & 99.83 & 437 \\
\hline
        \# \textbf{models} & \multicolumn{2}{c|}{500} & 500 & \textbf{1000} \\
        \hline
    \end{tabular}
        \label{tab:NIST1_model}\end{table}



\subsection*{Target Class Detection Algorithm}
The procedure for target class prediction is given in Algorithm \ref{alg:algorithm1}.

\begin{algorithm}[H]
    \KwData{Query model}
    \KwResult{ and Target Class}
Stage One: Use Trojan Detection network to get ; \\
    \eIf{ >= 0.5}
    {\For{ \KwTo  }
        {
            use FGSM to calculate adversarial perturbations with  as the target class;\\
            compute attack difficulty () for perturbation;\\
        }
         = perform outlier detection over the attack difficulties s;\\
        output  and target class prediction;\\
    }
    {
        output  and target class(None);\\
    }
 
 \caption{Two-stage method to detect a Trojan infected model and predict its target class using only clean image samples.}
 \label{alg:algorithm1}
\end{algorithm}






















\end{document}
