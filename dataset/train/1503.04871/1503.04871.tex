\documentclass[11pt,a4paper]{article}

\usepackage{fullpage}
\usepackage[utf8x]{inputenc}
\usepackage{amsmath, amsthm}
\usepackage{wrapfig,graphicx,amssymb,textcomp,array,wasysym}
\usepackage{algpseudocode} 
\usepackage{enumerate}
\usepackage{multirow}
\algtext*{EndWhile}\algtext*{EndIf}\usepackage{algorithm}
\usepackage{color}

\setlength{\arraycolsep}{0in}

\newcommand{\MST}{\text{\em MST}}
\newcommand{\HSC}{\text{\em HamSandwichCut}}
\newcommand{\SMGG}{\text{\em Strong-matching}}
\newcommand{\RB}{\text{\em RB-matching}}

\newcommand{\require}{\textbf{Input: }}
\newcommand{\ensure}{\textbf{Output: }}


\newcommand{\dg}[1]{deg(#1)}
\newcommand{\nb}[1]{N(#1)}
\newcommand{\nnb}[1]{\overline{N}(#1)}
\newcommand{\dG}[2]{deg^+_{#1}(#2)}
\newcommand{\tr}[1]{t(#1)}
\newcommand{\tru}[1]{t'(#1)}
\newcommand{\trmin}{t}
\newcommand{\emin}{e}
\newcommand{\cmin}{D(u,v)}
\newcommand{\cone}[2]{C^{#1}_{#2}}
\newcommand{\hex}[2]{X(#1,#2)}
\newcommand{\tra}[2]{{#1}({#2})}
\newcommand{\G}[2]{G_{#1}({#2})}
\newcommand{\Inf}[1]{\text{Inf}(#1)}
\newcommand{\disc}{\Circle}
\newcommand{\discs}{\scriptsize \Circle}
\newcommand{\ddisc}{\ominus}
\newcommand{\ddiscs}{\scriptsize \ominus}
\newcommand{\sqr}{\Box}
\newcommand{\sqrs}{\scriptsize\Box}
\newcommand{\trid}{\bigtriangledown}
\newcommand{\trids}{\scriptsize\bigtriangledown}
\newcommand{\triu}{\bigtriangleup}
\newcommand{\trius}{\scriptsize\bigtriangleup}
\newcommand{\tri}{C}
\newcommand{\ftri}{{\bigtriangledown}}
\newcommand{\ftriu}{{\bigtriangleup}}
\newcommand{\ftriud}{{\bigtriangledown \hspace*{-9.6pt} \bigtriangleup}}
\newcommand{\GD}{G_{\bigtriangledown}}
\newcommand{\GU}{G_{\bigtriangleup}}
\newcommand{\GUD}{G_{\bigtriangledown \hspace*{-7.5pt} \bigtriangleup}}
\newcommand{\GUDFIG}{G_{\bigtriangledown \hspace*{-6.5pt} \bigtriangleup}}
\newcommand{\sq}{\square}
\newcommand{\GS}[1]{G_{\sq}(#1)}
\newcommand{\SP}[2]{S^{\text{\tiny +#1}}_\text{\tiny #2}}
\newcommand{\SM}[2]{S^{\text{-\tiny #1}}_\text{\tiny #2}}

\title{Strong Matching of Points with Geometric Shapes\thanks{Research supported by NSERC.}}



\author{
Ahmad Biniaz\thanks{School of Computer Science, Carleton University, Ottawa, Canada.}
\and 
Anil Maheshwari\footnotemark[2]
\and 
Michiel Smid\footnotemark[2]
}
\date{\today}
\newtheorem{lemma}{Lemma}
\newtheorem{corollary}{Corollary}
\newtheorem{conjecture}{Conjecture}
\newtheorem{theorem}{Theorem}
\newtheorem{observation}{Observation}
\newtheorem{definition}{Definition}
\begin{document}

\maketitle

\begin{abstract}
Let  be a set of  points in general position in the plane. Given a convex geometric shape , a geometric graph  on  is defined to have an edge between two points if and only if there exists an empty homothet of  having the two points on its boundary. A matching in  is said to be {\em strong}, if the homothests of  representing the edges of the matching, are pairwise disjoint, i.e., do not share any point in the plane. We consider the problem of computing a strong matching in , where  is a diametral-disk, an equilateral-triangle, or a square. We present an algorithm which computes a strong matching in ; if  is a diametral-disk, then it computes a strong matching of size at least , and if  is an equilateral-triangle, then it computes a strong matching of size at least . If  can be a downward or an upward equilateral-triangle, we compute a strong matching of size at least  in . When  is an axis-aligned square we compute a strong matching of size  in , which improves the previous lower bound of . 
\end{abstract}

\section{Introduction}
\label{intro}

Let  be a compact and convex set in the plane that contains the origin in its interior. A {\em homothet} of  is obtained by scaling  with respect to the origin by some factor , followed by a translation to a point  in the plane: .
For a point set  in the plane, we define  as the geometric graph on  which has an straight-line edge between two points  and  if and only if there exists a homothet of  having  and  on its boundary and whose interior does not contain any point of . If  is in ``general position'', i.e., no four points of  lie on the boundary of any homothet of , then  is plane~\cite{Bose2010}. Hereafter, we assume that  is a set of  points in the plane, which is in general position with respect to  (we will define the general position in Section~\ref{preliminaries}). 
If  is a disk  whose center is the origin, then  is the Delaunay triangulation of . If  is an equilateral triangle  whose barycenter is the origin, then  is the triangular-distance Delaunay graph of  which is introduced by Chew~\cite{Chew1989}.

A {\em matching} in a graph  is a set of edges which do not share any vertices. A {\em maximum matching} is a matching with maximum cardinality. A {\em perfect matching} is a matching which matches all the vertices of . Let  be a matching in .  is referred to as a {\em matching of points with shape }, e.g., a matching in  is a matching of points with with disks. Let  be a set of homothets of  representing the edges of .  is called a {\em strong matching} if there exists a set  whose elements are pairwise disjoint, i.e., the objects in  do not share any point in the plane. Otherwise,  is a {\em weak matching}. See Figure~\ref{strong-example}. To be consistent with the definition of the matching in the graph theory, we use the term ``matching'' to refer to a weak matching. Given a point set  in the plane and a shape , the {\em (strong) matching problem} is to compute a (strong) matching of maximum cardinality in .
In this paper we consider the strong matching problem of points in general position in the plane with respect to a given shape  (see Section~\ref{preliminaries} for the definition), where by  we mean the line segment between the two points on the boundary of the disk is a diameter of that disk.

\begin{figure}[htb]
  \centering
\setlength{\tabcolsep}{0in}
  
  \caption{Point set  and (a) a perfect weak matching in , (b) a perfect strong matching in , and (c) a perfect strong matching in .}
\label{strong-example}
\end{figure}

\subsection{Previous Work}
\label{previous-work}
The problem of computing a maximum matching in  is one of the fundamental problems in computational geometry and graph theory \cite{Abrego2004, Abrego2009, Babu2014, Bereg2009, Biniaz2014, Biniaz2015, Dillencourt1990}. 
Dillencourt~\cite{Dillencourt1990} and \'{A}brego et al. \cite{Abrego2004} considered the problem of matching points with disks. Let  be a closed disk  whose center is the origin, and let  be a set of  points in the plane which is in general position with respect to . Then,  is the graph which has an edge between two points  if there exists a homothet of  having  and  on its boundary and does not contain any point of .  is equal to the Delaunay triangulation on , . Dillencourt~\cite{Dillencourt1990} proved that  contains a perfect (weak) matching. \'{A}brego et al. \cite{Abrego2004} proved that  has a strong matching of size at least . They also showed that there exists a set  of  points in the plane with arbitrarily large , such that  does not contain a strong matching of size more than .

For two points  and , the disk which has the line segment  as its diameter is called the diametral-disk between  and . We denote a diametral-disk by . Let  be the graph which has an edge between two points  if the diametral-disk between  and  does not contain any point of .  is equal to the Gabriel graph on , . Biniaz et al.~\cite{Biniaz2014} proved that  has a matching of size at least , and this bound is tight.

The problem of matching of points with equilateral triangles has been considered by Babu et al.~\cite{Babu2014}.
Let  be a downward equilateral triangle  whose barycenter is the origin and one of its vertices is on the negative -axis. Let  be a set of  points in the plane which is in general position with respect to . Let  be the graph which has an edge between two points  if there exists a homothet of  having  and  on its boundary and does not contain any point of .  is equal to the triangular-distance Delaunay graph on , which was introduced by Chew~\cite{Chew1989}. Bonichon et al.~\cite{Bonichon2010} showed that  is equal to the half-theta six graph on , . Babu et al.~\cite{Babu2014} proved that  has a matching of size at least , and this bound is tight. If we consider an upward triangle , then  is defined similarly. Let  be the graph on  which is the union of  and . Bonichon et al.~\cite{Bonichon2010} showed that  is equal to the theta six graph on , . Since  is a subgraph of , the lower bound of  on the size of maximum matching in  holds for .  

The problem of strong matching of points with axis-aligned rectangles is trivial. An obvious algorithm is to repeatedly match the two leftmost points. The problem of matching points with axis-aligned squares was considered by \'{A}brego et al. \cite{Abrego2009}.
Let  be an axis-aligned square  whose center is the origin. Let  be a set of  points in the plane which is in general position with respect to . Let  be the graph which has an edge between two points  if there exists a homothet of  having  and  on its boundary and does not contain any point of .  is equal to the -Delaunay graph on . \'{A}brego et al. \cite{Abrego2004, Abrego2009} proved that  has a perfect (weak) matching and a strong matching of size at least . Further, they showed that there exists a set  of  points in the plane with arbitrarily large , such that  does not contain a strong matching of size more than . Table~\ref{table1} summarizes the results.

Bereg et al.~\cite{Bereg2009} concentrated on matching points of  with axis-aligned rectangles and squares, where  is not necessarily in general position. 
They proved that any set of  points in the plane has a strong rectangle matching of size at least , and such a matching can be computed in  time. As for squares, they presented a  time algorithm that decides whether a given matching has a weak square realization, 
and an  time algorithm for the strong square matching realization. They also proved that it is NP-hard to decide whether a given point set has a perfect strong square-matching. 

\begin{table}
\centering
\begin{minipage}{13cm}\centering
\caption{Lower bounds on the size of weak and strong matchings in .}
\label{table1}
    \begin{tabular}{|c|c|c||c|c|}
         \hline
              	& weak matching & reference& strong matching &reference \\  \hline  \hline
	      {}& 	&\cite{Dillencourt1990}& &\cite{Abrego2004}\\
	      {} &&\cite{Biniaz2014} &&Theorem~\ref{Gabriel-thr}\\  
             {} &&\cite{Babu2014} &&Theorem~\ref{half-theta-six-thr}\\  
	    { or } && \cite{Babu2014} &&Theorem~\ref{theta-six-thr}\\ \hline\hline
	    \multirow{2}{*}{}& \multirow{2}{*}{} &\multirow{2}{*}{\cite{Abrego2004, Abrego2009}}&
	     & \cite{Abrego2004, Abrego2009}       \\ 
		& &	& & Theorem~\ref{infty-Delaunay-thr} \\ \hline
    \end{tabular}
    \end{minipage}
\end{table}

\subsection{Our results}
\label{our-results}
In this paper we consider the problem of computing a strong matching in , where . In Section~\ref{preliminaries}, we provide some observations and prove necessary Lemmas. Given a point set  in which is in general position with respect to a given shape , in Section~\ref{algorithm-section}, we present an algorithm which computes a strong matching in . In Section~\ref{Gabriel-section}, we prove that if  is a diametral-disk, then the algorithm of Section~\ref{algorithm-section} computes a strong matching of size at least  in . In Section~\ref{half-theta-six-section}, we prove that if  is an equilateral triangle, then the algorithm of Section~\ref{algorithm-section} computes a strong matching of size at least  in . In Section~\ref{theta-six-section}, we compute a strong matching of size at least  in . In Section~\ref{infty-Delaunay-section}, we compute a strong matching of size at least  in ; this improves the previous lower bound of . A summary of the results is given in Table~\ref{table1}. In Section~\ref{conjecture-section} we discuss a possible way to further improve upon the result obtained for diametral-disks in Section~\ref{Gabriel-section}. Concluding remarks and open problems are given in Section~\ref{conclusion}.

\section{Preliminaries}
\label{preliminaries}

Let , and let  and  be two homothets of . We say that  is {\em smaller then}  if the area of  is smaller than the area of . For two points , let  be a smallest homothet of  having  and  on its boundary. If  is a diametral-disk, a downward equilateral-triangle, or a square, then we denote  by , , or , respectively. If  is a diametral-disk, then  is uniquely defined by  and . If  is an equilateral-triangle or a square, then  has the {\em shrinkability} property: if there exists a homothet  of  that contains two points  and , then there exists a homothet  of  such that , and  and  are on the boundary of . If  is an equilateral-triangle, then we can
shrink  further, such that each side of  contains either  or . If  is a square, then we can
shrink  further, such that  and  are on opposite sides of . Thus, we have the following observation:

\begin{observation}
\label{shrink-triangle-obs}
For two points ,
\begin{itemize}
 \item  is uniquely defined by  and , and it has the line segment  as a diameter.
\item  is uniquely defined by  and , and it has one of  and  on a corner and the other point is
on the side opposite to that corner.
\item  has  and  on opposite sides.
\end{itemize}
\end{observation}

\begin{figure}[htb]
  \centering
\setlength{\tabcolsep}{0in}
  
  \caption{Illustration of Observation~\ref{obs1}.}
\label{mst-in-GS-fig}
\end{figure}
Given a shape , we define an order on the homothets of . Let  and  be two homothets of . We say that  if the area of  is less than the area of . Similarly,  if the area of  is less than or equal to the area of . We denote the homothet with the larger area by . As illustrated in Figure~\ref{mst-in-GS-fig}, if  contains a point , then both  and  have smaller area than . Thus, we have the following observation:
 
\begin{observation}
\label{obs1}
 If  contains a point , then .
\end{observation}


\begin{definition}
 Given a point set  and a shape , we say that  is in ``general position'' with respect to  if
\begin{description}
 \item[:] no four points of  lie on the boundary of any diametral disk defined by any two points of .
 \item[:] the line passing through any two points of  does not make angles , , or  with the horizontal. This implies that no four points of  are on the boundary of any homothet of .
 \item[:] (i) no two points in  have the same -coordinate or the same -coordinate, and (ii) no four points of  lie on the boundary of any homothet of .
\end{description}
\end{definition}

Given a point set  which is in general position with respect to a given shape , let  be the complete edge-weighted geometric graph on . For each edge  in , we define  to be the shape , i.e., a smallest homothet of  having  and  on its boundary. We say that  {\em represents} , and vice versa. Furthermore, we assume that the weight  (resp. ) of  is equal to the area of . Thus,

Note that  is a subgraph of , and has an edge  iff  does not contain any point of . 


\begin{lemma}
\label{mst-in-GS}
Let  be a set of  points in the plane which is in general position with respect to a given shape . Then, any minimum spanning tree of  is a subgraph of . 
\end{lemma}
\begin{proof}
 The proof is by contradiction. Assume there exists an edge  in a minimum spanning tree  of  such that . Since  is not an edge in ,  contains a point  such that . By Observation~\ref{obs1}, . Thus,  and . By replacing the edge  in  with either  or , we obtain a spanning tree in  which is smaller than . This contradicts the minimality of .
\end{proof}

\begin{lemma}
\label{cycle-lemma}
Let  be an edge-weighted graph with edge set  and edge-weight function . For any cycle  in , if the maximum-weight edge in  is unique, then that edge is not in any minimum spanning tree of .
\end{lemma}

\begin{proof}
The proof is by contradiction. Let  be the unique maximum-weight edge in a cycle  in , such that  is in a minimum spanning tree  of . Let  and  be the two trees obtained by removing  from . Let  be an edge in  which connects a vertex  to a vertex . By assumption, . Thus, in , by replacing  with , we obtain a tree  in  such that . This contradicts the minimality of . 
\end{proof}

Recall that  is the smallest homothet of  which has  and  on its boundary. Similarly, let  denote the smallest upward equilateral-triangle  having  and  on its boundary. Note that  is uniquely defined by  and , and it has one of  and  on a corner and the other point is on the side opposite to that corner. In addition the area of  is equal to the area of . 

\begin{wrapfigure}{r}{0.35\textwidth}
\vspace{-20pt}
 \begin{center}
\includegraphics[width=.3\textwidth]{fig/cones.pdf}
  \end{center}
\vspace{-5pt}
  \caption{The construction of .}
\label{cones}
\vspace{-5pt}
\end{wrapfigure}

 is equal to the triangular-distance Delaunay graph , which is in turn equal to a half theta-six graph ~\cite{Bonichon2010}. 
A half theta-six graph on , and equivalently , can be constructed in the following way. For each point  in , let  be the horizontal line through . Define  as the line obtained by rotating  by -degrees in counter-clockwise direction around . Thus, . Consider three lines , , and  which partition the plane into six disjoint cones with apex . Let  be the cones in counter-clockwise order around  as shown in Figure~\ref{cones}.  will be referred to as {\em odd cones}, and  will be referred to as {\em even cones}. For each even cone , connect  to the ``nearest'' point  in . The {\em distance} between  and , is defined as the Euclidean distance between  and the orthogonal projection of  onto the bisector of . See Figure~\ref{cones}. In other words, the nearest point to  in  is a point  in  which minimizes the area of . The resulting graph is the half theta-six graph which is defined by even cones \cite{Bonichon2010}. Moreover, the resulting graph is  which is defined with respect to the homothets of . By considering the odd cones,  is obtained. By considering the odd cones and the even cones, \textemdash which is equal to \textemdash is obtained. Note that  is the union of  and . 

Let  be the regular hexagon centered at  which has  on its boundary, and its sides are parallel to , , and . Then, we have the following observation:
\begin{observation}
\label{obs2}
If  contains a point , then .
\end{observation}

\section{Strong Matching in }
\label{algorithm-section}

Given a point set  in the plane which is in general position with respect to a given shape , in this section we present an algorithm which computes a strong matching in . Recall that  is the complete edge-weighted graph on  with the weight of each edge  is equal to the area of , where  is a smallest homothet of  representing . Let  be a minimum spanning tree of . By Lemma~\ref{mst-in-GS},  is a subgraph of . For each edge  we denote by  the set of all edges in  whose weight is at least . Moreover, we define the {\em influence set} of , as the set of all edges in  whose representing shapes overlap with , i.e.,


Note that  is not empty, as . Consequently, we define the {\em influence number} of  to be the maximum size of a set among the influence sets of edges in , i.e.,


Algorithm~\ref{alg1} receives  as input and computes a strong matching in  as follows. The algorithm starts by computing a minimum spanning tree  of , where the weight of each edge is equal to the area of its representing shape. Then it initializes a forest  by , and a matching  by an empty set. Afterwards, as long as  is not empty, the algorithm adds to , the smallest edge  in , and removes the influence set of  from . Finally, it returns .
\begin{algorithm}                      \caption{\SMGG}          \label{alg1} 
\begin{algorithmic}[1]
\State 
      \State 
      \State 
      \While {}
	  \State  smallest edge in 
\State 
	  \State 
	  \EndWhile
    \State \Return 
\end{algorithmic}
\end{algorithm}
\begin{theorem}
\label{GS-thr}
Given a set  of  points in the plane and a shape , Algorithm~\ref{alg1} computes a strong matching of size at least  in , where  is a minimum spanning tree of . 
\end{theorem}
\begin{proof}
Let  be the matching returned by Algorithm~\ref{alg1}. First we show that  is a strong matching. If  contains one edge, then trivially,  is a strong matching. Consider any two edges  and  in . Without loss of generality assume that  is considered before  in the {\sf while} loop. At the time  is added to , the algorithm removes from , the edges in , i.e., all the edges whose representing shapes intersect . Since  remains in  after the removal of , . This implies that , and hence  is a strong matching.

In each iteration of the {\sf while} loop we select  as the smallest edge in , where  is a subgraph of . Then, all edges in  have weight at least . Thus, ; which implies that the set of edges in  whose representing shapes intersect  is a subset of . Therefore, in each iteration of the {\sf while} loop, out of at most -many edges of , we add one edge to . Since  and  has  edges, we conclude that .
\end{proof}

\paragraph{Remark}
Let  be the minimum spanning tree computed by Algorithm~\ref{alg1}. Let  be an edge in . Recall that  contains all the edges of  whose weight is at least . We define the {\em degree} of  as , where  and  are the number of edges incident on  and  in , respectively. Note that all the edges incident on  or  in  are in the influence set of . Thus, , and consequently .

\section{Strong Matching in }
\label{Gabriel-section}
In this section we consider the case where  is a diametral-disk . Recall that  is an edge-weighted geometric graph, where the weight of an edge  is equal to the area of .  is equal to the Gabriel graph, . We prove that , and consequently , has a strong diametral-disk matching of size at least . 

We run Algorithm~\ref{alg1} on  to compute a matching . By Theorem~\ref{GS-thr},  is a strong matching of size at least , where  is a minimum spanning tree in . By Lemma~\ref{mst-in-GS},  is a minimum spanning tree of the complete graph . Observe that  is a Euclidean minimum spanning tree for  as well. In order to prove the desired lower bound, we show that . Since  is the maximum size of a set among the
influence sets of edges in , it suffices to show that for every edge  in , the influence set of  contains at most 17 edges. 
\begin{lemma}
\label{disk-inf-lemma}
Let  be a minimum spanning tree of , and let  be any edge in . Then, .
\end{lemma}
We will prove this lemma in the rest of this section. Recall that, for each two points ,  is the closed diametral-disk with diameter . Let  denote the set of diametral-disks representing the edges in . Since  is a subgraph of , we have the following observation:

\begin{observation}
\label{no-point-in-circle-obs}
 Each disk in  does not contain any point of  in its interior.
\end{observation}

Recall that, for each two points ,  is the closed diametral-disk with diameter . Let  denote the set of diametral-disks representing the edges in . Since  is a subgraph of , we have the following observation:

\begin{observation}
\label{no-point-in-circle-obs}
 Each disk in  does not contain any point of  in its interior.
\end{observation}

\begin{lemma}
\label{center-in-lemma}
 For each pair  and  of disks in ,  (resp. ) does not contain the center of  (resp ).
\end{lemma}

\begin{proof}
 Let  and  respectively be the edges of  which correspond to  and . Let  and  be the circles representing the boundary of  and . W.l.o.g. assume that  is the bigger circle, i.e., . By contradiction, suppose that  contains the center  of . Let  and  denote the intersections of  and . Let  (resp. ) be the intersection of  (resp. ) with the line through  and  (resp. ). Similarly, let  (resp. ) be the intersection of  (resp. ) with the line through  and  (resp. ). 

\begin{figure}[htb]
  \centering
  \includegraphics[width=.6\columnwidth]{fig/center-in.pdf}
 \caption{Illustration of Lemma~\ref{center-in-lemma}:  and  intersect, and  contains the center of .}
  \label{center-in-fig}
\end{figure}

As illustrated in Figure~\ref{center-in-fig}, the arcs , , , and  are the potential positions for the points , , , and , respectively. First we will show that the line segment  passes through  and . The angles  and  are right angles, thus the line segment  goes through . Since  (resp. ), for any point  (resp. ). Therefore, 
Consider triangle  which is partitioned by segment  into  and . It is easy to see that  in  is equal to  in , and the segment  is shared by  and . Since  is inside  and , the angle . Thus,  in  is smaller than  (and hence smaller than  in ). That is,   in  is smaller than  in . Therefore,



By symmetry . Therefore . Therefore, the cycle  contradicts Lemma~\ref{cycle-lemma}.
\end{proof}

Let  be an edge in . Without loss of generality, we suppose that  has radius 1 and centered at the origin  such that  and . For any point  in the plane, let  denote the distance of  from . Let  be the disks in  representing the edges of . Recall that  contains the edges of  whose weight is at least , where  is equal to the area of . Since the area of any circle is directly related to its radius, we have the following observation:

\begin{observation}
 \label{radius-one}
The disks in  have radius at least .
\end{observation}

Let  (resp. ) be the circle (resp. closed disk) of radius  which is centered at a point  in the plane. 
Let  be the set of disks in  intersecting . We show that  contains at most sixteen disks, i.e., .

For , let  denote the center of the disk . 
In addition, let  be the intersection point between  and the ray with origin at  which passing through . Let the point  be , if , and , otherwise. See Figure~\ref{distance-fig}. Finally, let . 

\begin{observation}
\label{obs}
Let  be the center of a disk  in , where . Then, the disk  is contained in the disk . Moreover, the disk  is contained in the disk . See Figure~\ref{distance-fig}.
\end{observation}

\begin{figure}[htb]
  \centering
  \includegraphics[width=.5\columnwidth]{fig/cone3-3.pdf}
 \caption{Proof of Lemma~\ref{distance-lemma}; , , and .}
  \label{distance-fig}
\end{figure}

\begin{lemma}
\label{distance-lemma}
The distance between any pair of points in  is at least 1.
\end{lemma}
\begin{proof}
Let  and  be two points in . We are going to prove that . We distinguish between the following three cases. 
\begin{itemize}
 \item . In this case the claim is trivial.
\item . If , then  is on , and hence . If , then  is the center of a disk  in . By Observation~\ref{no-point-in-circle-obs},  does not contain  and , and by Lemma~\ref{center-in-lemma},  does not contain . Since  has radius at least 1, we conclude that .

\item . Without loss of generality assume  and , where . We differentiate between three cases:
\begin{itemize}
 \item  and . In this case  and  are the centers of  and , respectively. By Lemma~\ref{center-in-lemma} and Observation~\ref{radius-one}, we conclude that .

\item  and . By Observation~\ref{obs} the disk  is contained in the disk . By Lemma~\ref{center-in-lemma},  is not in the interior of , and consequently, it is not in the interior of . Therefore, .
\item  and . Recall that  and  are the centers of  and , such that  and . Without loss of generality assume . For the sake of contradiction assume that . Then, for the angle  we have . Then, . By the law of cosines in the triangle , we have

By Observation~\ref{obs} the disk  is contained in ; see Figure~\ref{distance-fig}. By Lemma~\ref{center-in-lemma},  is not in the interior of , and consequently, is not in the interior of . Thus, . In combination with Inequality~(\ref{ineq1}), this gives


In combination with the assumption that , Inequality~(\ref{ineq2}) gives


To satisfy this inequality, we should have , contradicting the fact
that . This completes the proof.
\end{itemize}
\end{itemize}
\end{proof}

By Lemma~\ref{distance-lemma}, the points in  has mutual distance 1. Moreover, the points in  lie in (including the boundary) .
Bateman and Erd\H{o}s~\cite{Bateman1951} proved that it is impossible to have 20 points in (including the boundary) a circle of radius 2 such that one of the points is at the center and all of the mutual distances are at least 1.
Therefore,  contains at most  points, including , , and . This implies that , and hence  contains at most sixteen edges. This completes the proof of Lemma~\ref{disk-inf-lemma}.

\begin{theorem}
 \label{Gabriel-thr}
Algorithm~\ref{alg1} computes a strong matching of size at least  in .
\end{theorem}


\section{Strong Matching in }
\label{half-theta-six-section}
In this section we consider the case where  is a downward equilateral triangle , whose barycenter is the origin and one of its vertices is on the negative -axis. In this section we assume that  is in general position, i.e., for each point , there is no point of  on , , and . In combination with Observation~\ref{shrink-triangle-obs}, this implies that for two points , no point of  are on the boundary of  (resp. ). Recall that  is the smallest homothet of  having of  and  on a corner and the other point on the side opposite to that corner. We prove that , and consequently , has a strong triangle matching of size at least . 

We run Algorithm~\ref{alg1} on  to compute a matching . Recall that  is an edge-weighted graph with the weight of each edge  is equal to the area of . By Theorem~\ref{GS-thr},  is a strong matching of size at least , where  is a minimum spanning tree in . In order to prove the desired lower bound, we show that . Since  is the maximum size of a set among the
influence sets of edges in , it suffices to show that for every edge  in , the influence set of  has at most nine edges. 
\begin{lemma}
\label{triangle-inf-lemma}
Let  be a minimum spanning tree of , and let  be any edge in . Then, .
\end{lemma}
  

 \begin{figure}[htb]
  \centering
\setlength{\tabcolsep}{0in}
  
  \caption{(a) Labeling the vertices and the sides of a downward triangle. (b) Labeling the vertices and the sides of an upward triangle. (c) Two intersecting triangles.}
  \label{triangle-fig}
\end{figure}

We will prove this lemma in the rest of this section. We label the vertices and the sides of a downward equilateral-triangle, , and an upward equilateral-triangle, , as depicted in Figures~\ref{triangle-fig}(a) and ~\ref{triangle-fig}(b). We refer to a vertex  and a side  of a triangle  by  and , respectively.

Recall that  is a subgraph of the minimum spanning tree  in . In each iteration of the {\sf while} loop in Algorithm~\ref{alg1}, let  denote the set of triangles representing the edges in . By Lemma~\ref{mst-in-GS} and the general position assumption we have

\begin{observation}
\label{no-point-in-triangle-obs}
Each triangle  in  does not contain any point of  in its interior or on its boundary.\vspace{-5pt}
\end{observation}

Consider two intersecting triangles  and  in . By Observation~\ref{shrink-triangle-obs}, each side of  contains either  or , and each side of  contains either  or . Thus, by Observation~\ref{no-point-in-triangle-obs}, we argue that no side of  is completely in the interior of , and vice versa. Therefore, either exactly one vertex (corner) of  is in the interior of , or exactly one vertex of  is in the interior of . Without loss of generality assume that a corner of  is in the interior of , as shown in Figure~\ref{triangle-fig}(c). In this case we say that  intersects  through the vertex , or symmetrically,  intersects  through the side . 


The following two lemmas have been proved by Biniaz et al.~\cite{Biniaz2015}:
\begin{lemma}[Biniaz et al.~\cite{Biniaz2015}]
\label{triangle3}
Let  be a downward triangle which intersects a downward triangle  through , and let a horizontal line  intersects both  and . Let  and  be two points on  and , respectively, which are above . Let  and  be two points on  and , respectively, which are above . Then, . See Figure~\ref{triangle-intersection-fig}(b).
\end{lemma}

\begin{lemma}[Biniaz et al.~\cite{Biniaz2015}]
\label{intersection-lemma}
For every four triangles , . 
\end{lemma}

As a consequence of Lemma~\ref{triangle3}, we have the following corollary:
\begin{corollary}
\label{biniaz-cor}
 Let  be three triangles in . Then , , and  cannot make a chain configuration, such that  intersects  through  and  intersects both  and  through  and . See Figure~\ref{triangle-intersection-fig}(b).
\end{corollary}


\begin{figure}[htb]
  \centering
\setlength{\tabcolsep}{0in}
  
  \caption{(a) Illustration of Lemma~\ref{triangle-intersection-lemma}. (b) Illustration of Lemma~\ref{triangle3}.}
  \label{triangle-intersection-fig}
\end{figure}

\begin{lemma}
 \label{triangle-intersection-lemma}
Let  be a downward triangle which intersects a downward triangle  through . Let  be a point on  and to the left of , and let  be a point on  and to the right of . Then, .
\end{lemma}
\begin{proof}
 Refer to Figure~\ref{triangle-intersection-fig}(a). Let  be the part of the line segment  which is to the left of , and let  be the part of the line segment  which is to the right of . Without loss of generality assume that  is larger than . Let  be an upward triangle having  as its left side. Then, , which implies that . Since  has both  and  on its boundary, the area of the downward triangle  is smaller than the area of . Therefore, ; which completes the proof.
\end{proof}

Because of the symmetry, the statement of Lemma~\ref{triangle-intersection-lemma} holds even if  is above  and  is on .
Consider the six cones with apex at , as shown in Figure~\ref{cones}.
\begin{lemma}
\label{deg-six-half}
Let  be a minimum spanning tree in . Then, in , every point  is adjacent to at most one point in each cone , where .
\end{lemma}
\begin{proof}
If  is even, then by the construction of , which is given in Section~\ref{preliminaries},  is adjacent to at most one point in . Assume  is odd. For the sake of contradiction, assume in , the point  is adjacent to two points  and  in a cone . Then,  has  on a corner, and  has  on a corner. Without loss of generality assume . Then, the hexagon  has  in its interior. Thus, . Then the cycle  contradicts Lemma~\ref{cycle-lemma}. Therefore,  is adjacent to at most one point in each of the six cones.
\end{proof}


In Algorithm~\ref{alg1}, in each iteration of the {\sf while} loop, let  be the triangles representing the edges of . Recall that  is the smallest edge in , and hence,  is a smallest triangle in .
Let  and let  be the set of triangles in  (excluding ) which intersect . We show that  contains at most eight triangles.
We partition the triangles in  into , such that every triangle  shares only  or  with , i.e., , and every triangle  intersects  either through a side or through corner which is not  nor .


\begin{wrapfigure}{r}{0.4\textwidth}
\vspace{-20pt}
 \begin{center}
\includegraphics[width=.35\textwidth]{fig/cones-pq.pdf}
  \end{center}
\vspace{-15pt}
  \caption{Illustration of the triangles in .}
\label{cones-pq}
\vspace{-8pt}
\end{wrapfigure}
By Observation~\ref{shrink-triangle-obs}, for each triangle , one of  and  is on a corner of  and the other one is on the side opposite to that corner. Without loss of generality assume that  is on the corner , and hence,  is on the side . See Figure~\ref{cones-pq}. Note that the other cases, where  is on  or on  are similar.
Since the intersection of  with any triangle  is either  or ,  has either  or  on its boundary. In combination with Observation~\ref{no-point-in-triangle-obs}, this implies that  represent an edge  in , and hence, either  or  is an endpoint of . As illustrated in Figure~\ref{cones-pq}, the other endpoint of  can be either in , , , or in , because otherwise . By Lemma~\ref{deg-six-half},  has at most one neighbor in each of , , , and  has at most one neighbor in . Therefore,  contains at most four triangles. We are going to show that  also contains at most four triangles. 

The point  divides  into two parts. Let  and  be the parts of  which are below and above , respectively; see Figure~\ref{cones-pq}. The triangles in  intersect  either through  or through ; which are shown by red and blue polylines in Figure~\ref{cones-pq}. We show that most two triangles in  intersect  through each of  or . Because of symmetry, we only prove for . When a triangle  intersects  through both  and  we say  intersects  through . In the next lemma, we prove that at most one triangle in  intersects  through each of , . Again, because of symmetry, we only prove for . 

\begin{figure}[htb]
  \centering
\setlength{\tabcolsep}{0in}
  
  \caption{Illustration of Lemma~\ref{side-intersection}: (a) . (b)  and .}
  \label{two-triangles-fig}
\end{figure}

\begin{lemma}
\label{side-intersection}
At most one triangle in  intersects  through .
\end{lemma}
\begin{proof}
The proof is by contradiction. Assume two triangles  and  in  intersect  through . Without loss of generality assume that  is on  and  is on  for . Recall that the area of  and the area of  are at least the area of . If  is in the interior of  (as shown in Figure~\ref{two-triangles-fig}(a)) or  is in the interior of , then we get a contradiction to Corollary~\ref{biniaz-cor}. Thus, assume that  and . 

Without loss of generality assume that  is above ; see Figure~\ref{two-triangles-fig}(b). By Lemma~\ref{triangle-intersection-lemma}, we have . If  is in , then by Observation~\ref{obs2}, . Then, the cycle  contradicts Lemma~\ref{cycle-lemma}. Thus, assume that . In this case  is to the left of , because otherwise  lies in  which contradicts Observation~\ref{no-point-in-triangle-obs}.
Since both  and  are larger than ,  intersects  through , and hence  is in the interior of . This implies that  is on . In addition,  is on the part of  which lies in the interior of . By Observation~\ref{obs2} and Lemma~\ref{triangle-intersection-lemma}, we have  and , respectively. Thus, the cycle  contradicts Lemma~\ref{cycle-lemma}. 
\end{proof}

\begin{lemma}
\label{vertex-intersection}
 At most two triangles in  intersect  through .
\end{lemma}
\begin{proof}
For the sake of contradiction assume three triangles  intersect  through . This implies that  belongs to four triangles , which contradicts Lemma~\ref{intersection-lemma}. \end{proof}

\begin{figure}[htb]
  \centering
\setlength{\tabcolsep}{0in}
  
  \caption{Illustration of Lemma~\ref{vertex-side-intersection-2}: (a)  is to the right of , (b) , (c) , and (d) .}
\label{three-triangle-fig2}
\end{figure}
\begin{lemma}
\label{vertex-side-intersection-2}
If two triangles in  intersect  through , then no other triangle in  intersects  through  or through . 
\end{lemma}

\begin{proof}
The proof is by contradiction. Assume two triangles  and  in  intersect  through , and a triangle  in  intersects  through  or . Let  be the point which lies on  for . By Lemma~\ref{vertex-intersection},  cannot intersect both  and . Thus,  intersects  either through  or through . We prove the former case; the proof for the latter case is similar. Assume that  intersects  through . By Lemma~\ref{triangle-intersection-lemma}, . See Figure~\ref{three-triangle-fig2}. In addition, both  and  are to the left of , because otherwise  lies in . If  we get a contradiction to Observation~\ref{no-point-in-triangle-obs}. If  then by Observation~\ref{obs2}, we have , and hence, the cycle  contradicts Lemma~\ref{cycle-lemma}.

Without loss of generality assume that  is above ; see Figure~\ref{three-triangle-fig2}. If  is in  or  is in , then we get a contradiction to Corollary~\ref{biniaz-cor}. Thus, assume that  and . This implies that either (i)  is to the right of  or (ii)  is to the left of . We show that both cases lead to a contradiction.

In case (i),  lies in the interior of , and then by Observation~\ref{obs2}, we have ; see Figure~\ref{three-triangle-fig2}(a). In addition, Lemma~\ref{triangle-intersection-lemma} implies that . Thus, the cycle  contradicts Lemma~\ref{cycle-lemma}.

Now consider case (ii) where  is above  and  is to the left of . If  is to the right of , then as in case (i), the cycle  contradicts Lemma~\ref{cycle-lemma}. Thus, assume that  is to the left of , as shown in Figure~\ref{three-triangle-fig2}(b). By Lemma~\ref{triangle-intersection-lemma}, we have . Each side of  contains either  or , while  is on the part of  which is to the left of , thus,  is on . Consider the six cones around ; see Figure~\ref{three-triangle-fig2}(b). We have three cases: (a) , (b)  or (c) . 

In case (a), which is shown in Figure~\ref{three-triangle-fig2}(b), by Lemma~\ref{triangle3}, we have . Thus, the cycle  contradicts Lemma~\ref{cycle-lemma}. In Case (b), which is shown in Figure~\ref{three-triangle-fig2}(c), we have , because if we map  to a downward triangle \textemdash of area equal to the area of \textemdash which has  on , then  contains both  and . Therefore, the cycle  contradicts Lemma~\ref{cycle-lemma}. In Case (c), which is shown in Figure~\ref{three-triangle-fig2}(d), by Observation~\ref{obs2}, , and then, the cycle  contradicts Lemma~\ref{cycle-lemma}.
\end{proof}

\begin{figure}[htb]
  \centering
\setlength{\tabcolsep}{0in}
  
  \caption{Illustration of Lemma~\ref{vertex-side-intersection-1}: (a) , and (b) .}
\label{three-triangle-fig}
\end{figure}

\begin{lemma}
\label{vertex-side-intersection-1}
If three triangles intersect  through  and . Then, at least one of the three triangles is not in . 
\end{lemma}
\begin{proof}
The proof is by contradiction. Assume that three triangles  in  intersect  through , respectively. Let  be the point which lies on  for . See Figure~\ref{three-triangle-fig}(a). By Lemma~\ref{triangle-intersection-lemma}, we have  and . If  is in the interior of , then by Observation~\ref{obs2}, , and hence, the cycle  contradicts Lemma~\ref{cycle-lemma}. If  is in , then by Observation~\ref{obs2}, , and hence, the cycle  contradicts Lemma~\ref{cycle-lemma}; see Figure~\ref{three-triangle-fig}(b). Thus, assume that  and . Let  and  be the parts of  which are to the right of  and to the left of , respectively. Consider the point  which lies on . 
If , then  and by Observation~\ref{obs2}, . In addition, Lemma ~\ref{triangle-intersection-lemma} implies that . Thus, the cycle  contradicts Lemma~\ref{cycle-lemma}; see Figure~\ref{three-triangle-fig}(a).
If , then  and by Observation~\ref{obs2}, . In addition, Lemma ~\ref{triangle-intersection-lemma} implies that . Thus, the cycle  contradicts Lemma~\ref{cycle-lemma}; see Figure~\ref{three-triangle-fig}(b).
\end{proof}



Putting Lemmas~\ref{side-intersection}, \ref{vertex-intersection}, \ref{vertex-side-intersection-2}, and \ref{vertex-side-intersection-1} together, implies that at most two triangles in  intersect  through ,  and consequently, at most two triangles in  intersect  through . Thus,  contains at most four triangles. Recall that  contains at most four triangles. Then,  has at most eight triangles. Therefore, the influence set of , contains at most 9 edges (including  itself). This completes the proof of Lemma~\ref{triangle-inf-lemma}. 

\begin{theorem}
\label{half-theta-six-thr}
Algorithm~\ref{alg1} computes a strong matching of size at least  in .
\end{theorem}

\begin{figure}[htb]
  \centering
\includegraphics[width=.4\columnwidth]{fig/five-triangles.pdf}
  \caption{Four triangles in  (in red) and four triangles in  (in blue) intersect with .}
\label{five-fig}
\end{figure}

The bound obtained by Lemma~\ref{triangle-inf-lemma} is tight. Figure~\ref{five-fig} shows a configuration of 10 points in general position such that the influence set of a minimal edge is 9. In Figure~\ref{five-fig},  represents a smallest edge of weight 1; the minimum spanning tree is shown in bold-green line segments. The weight of all edges\textemdash the area of the triangles representing these edges\textemdash is at least 1. The red triangles are in  and share either  or  with . The blue triangles are in  and intersect  through  or through ; as show in Figure~\ref{five-fig}, two of them share only the points  and .

\section{Strong Matching in }

In this section we consider the problem of computing a strong matching in . Recall that  is the union of  and , and is equal to the graph . We assume that  is in general position, i.e., for each point , there is no point of  on , , and . A matching  in  is a strong matching if for each edge  in  there is a homothet of  or a homothet of  representing , such that these homothets are pairwise disjoint. See Figure~\ref{strong-example}(b). Using a similar approach as in~\cite{Abrego2009}, we prove the following theorem:

\label{theta-six-section}
\begin{theorem}
\label{theta-six-thr}
Let  be a set of  points in general position in the plane. Let  be an upward or a downward equilateral-triangle that contains . Then, it is possible to find a strong matching of size at least  for  in .
\end{theorem}

\begin{proof}
The proof is by induction. Assume that any point set of size  in a triangle , has a strong matching of size  in . Without loss of generality, assume  is an upward equilateral-triangle. If  is  or , then there is no matching in , and if , then by shrinking , it is possible to find a strongly matched pair; the statement of the theorem holds. Suppose that , and , where . If , then 
, and by
induction we are done. Suppose that , for some . We prove that there are  disjoint equilateral-triangles (upward or downward) in ,
each of them matches a pair of points in . Partition  into four equal area equilateral triangles  containing  points, respectively; see Figure~\ref{Theta-six-fig}(a). Let , where . 
By induction, in , we have a strong matching of size at least


{\bf Claim 1:} .
\begin{proof}
By Equation~(\ref{A-eq}), we have

Since  is an integer, we argue that .
\end{proof}

If , then we are done. Assume that ; in fact, by the induction hypothesis we have an strong matching of size  for . In order to complete the proof, we have to get one more strongly matched pair. Let  be the multiset .


{\bf Claim 2:} {\em If , then either (i) one element in  is equal to  and the other elements are equal to , or (ii) two elements in  are equal to  and the other elements are equal to .}
\begin{proof}
Let , where . Then . Since , , for some . Thus, , where .

By induction, in , we get a matching of size at least . Hence, in , we get a matching of size at least



Since  and , we have 



Note that .
We go through some case analysis: (i) , (ii) , (iii) . In case (i), we have . In order to have  equal to 0 in Equation~(\ref{k-eq}), no element in  should be more than 1; this happens only if two elements in  are equal to 0 and the other two elements are equal to 1. In case (ii), we have . In order to have  equal to 1 in Equation~(\ref{k-eq}), at most one element in  should be greater than 1; this happens only if three elements in  are equal to 1 and the other element is equal to 3 (note that all elements in  are smaller than 4). In case (iii), we have . In order to have  equal to 2 in Equation~(\ref{k-eq}), at most two elements in  should be greater than 1; which is not possible.
\end{proof} 
We show how to find one more matched pair in each case of Claim 2.


We define x as the smallest upward equilateral-triangle contained in  and anchored at the top corner of , which contains all the points in  except  points. If  contains less than  points, then the area of x is zero. We also define x as the smallest upward equilateral-triangle that contains  and anchored at the top corner of , which has all the points in  plus  other points of . Similarly we define upward triangles x and x which are anchored at the left corner of . Moreover, we define upward triangles x and x which are anchored at the right corner of . We define downward triangles x, x, x which are anchored at the top-left corner, top-right corner, and bottom corner of , respectively. See Figure~\ref{Theta-six-fig}(a). 

{\bf Case 1:} {\em One element in  is equal to 3 and the other elements are equal to 1.}

In this case, we have . Because of the symmetry, we have two cases: (i) , (ii)  for some .

\begin{itemize}
 
\begin{figure}[h!]
  \centering
\setlength{\tabcolsep}{0in}
  
  \caption{(a) Split  into four equal area triangles. (b)  is larger than  and .}
\label{Theta-six-fig}
\end{figure}

 \item {.}

In this case . We differentiate between two cases, where all the elements of the multiset  are equal to zero, or some of them are greater than zero.

\begin{itemize}
 \item {\em All elements of  are equal zero.} In this case, we have . Consider the triangles  and . See Figure~\ref{Theta-six-fig}(a). Note that  and  are disjoint,  contains two points, and  contains  points. By induction, we get a matched pair in  and a matching of size at least  in . Thus, in total, we get a matching of size at least  in .

 \item {\em Some elements of  are greater than zero.} Consider the triangles , , and . Note that the area of some of these triangles\textemdash but not all\textemdash may be equal to zero. See Figure~\ref{Theta-six-fig}(b). By induction, we get matchings of size , , and  in , , and , respectively. Without loss of generality, assume , is larger than  and . Consider the half-lines  and  which are parallel to  and  axis, and have their endpoints on the top corner and right corner of , respectively. We define  as the downward equilateral-triangle which is bounded by , , and the right side of ; the dashed triangle in Figure~\ref{Theta-six-fig}(b). Note that  and  do not intersect  and . In addition, , , , and  are pairwise disjoint. If any point of  is to the right of , then consider  and . By induction, we get a matching of size  in , and hence a matching of size  in . If any point of  is above , then consider  and . By induction, we get a matching of size  in , and hence a matching of size  in . Otherwise,  contains  points. Thus, by induction, we get a matching of size  in , and hence a matching of size  in .
\end{itemize}


\item {\em , for some .}

Without loss of generality, assume that . Then, . Consider the triangles , , and . See Figure~\ref{Theta-six-fig2}(a). By induction, we get matchings of size , , and  in , , and , respectively. 
Now we consider the largest triangle among , , and . Because of the symmetry, we have two cases: (i)  is the largest, or (ii)  is the largest.
\begin{figure}[h!]
  \centering
\setlength{\tabcolsep}{0in}
  
  \caption{(a)  is larger than  and . (b)  is larger than  and .}
\label{Theta-six-fig2}
\end{figure}
\begin{itemize}
 \item {\em  is larger than  and .}
Define the half-lines , , and the triangle  as in the previous case. See Figure~\ref{Theta-six-fig2}(a). If any point of  is to the right of , then consider  and . By induction, we get a matching of size  in . If any point of  is above , then consider  and . By induction, we get a matching of size  in . Otherwise,  contains  points. Thus, by induction, we get a matching of size  in . As a result, in all cases we get a matching of size  in .

\item {\em  is larger than  and .}
Define the half-lines , , and the triangle  as in Figure~\ref{Theta-six-fig2}(b). If any point of  is above , then by induction, we get a matching of size  in . If at least three points of  are to the left of , then consider  and . Note that  contains  points. By induction, we get a matching of size  in . Otherwise,  contains at least  points. Thus, by induction, we get a matching of size  in . As a result, in all cases we get a matching of size  in .
\end{itemize}
\end{itemize}


{\bf Case 2:} {\em Two elements in  are equal to 0 and the other elements are equal to 1.}

In this case, we have . Again, because of the symmetry, we have two cases: (i) , (ii) .

\begin{itemize}
 \item 

Without loss of generality assume that  and . Thus, , , , and . If all elements of  are equal to zero, then we have , where . Consider the triangles  and , which are disjoint. By induction, we get a matched pair in  and a matching of size at least  in . Thus, in total, we get a matching of size at least  in . Assume some elements in  are greater than zero. Consider the triangles , , and . See Figure~\ref{Theta-six-fig3}(a). By induction, we get a matching of size , , and  in , , and , respectively. 
Now we consider the largest triangle among , , and . Because of the symmetry, we have two cases: (i)  is the largest, or (ii)  is the largest.

\begin{figure}[h!]
  \centering
\setlength{\tabcolsep}{0in}
  
  \caption{(a)  is larger than  and . (b)  is larger than  and .}
\label{Theta-six-fig3}
\end{figure}
\begin{itemize}
 \item {\em  is larger than  and .}
Define , ,  as in Figure~\ref{Theta-six-fig3}(a). If any point of  is to the right of , then by induction, we get a matching of size  in . If any point of  is above , then by induction, we get a matching of size  in . Otherwise,  contains  points. Thus, by induction, we get a matching of size  in . In all cases we get a matching of size  in .

\item {\em  is larger than  and .}
Define , ,  as in Figure~\ref{Theta-six-fig3}(b). If any point of  is above , then by induction, we get a matching of size  in . If at least two points of  are to the left of , then by induction, we get a matching of size  in . Otherwise,  contains at least  points. Thus, by induction, we get a matching of size  in . In all cases we get a matching of size  in .
\end{itemize}
  \item 

In this case , and without loss of generality, assume that ; which means . Thus, , , , and . If all elements of  are equal to zero, then we have , where . Consider the triangles  and , which are disjoint. By induction, we get a matched pair in  and a matching of size at least  in . Thus, in total, we get a matching of size at least  in . Assume some elements in  are greater than zero. Consider the triangles , , and . See Figure~\ref{Theta-six-fig4}(a). By induction, we get matchings of size , , and  in , , and , respectively. 
Now we consider the largest triangle among , , and . Because of symmetry, we have two cases: (i)  is the largest, or (ii)  is the largest.

\begin{figure}[h!]
  \centering
\setlength{\tabcolsep}{0in}
  
  \caption{(a)  is larger than  and . (b)  is larger than  and .}
\label{Theta-six-fig4}
\end{figure}
\begin{itemize}
 \item {\em  is larger than  and .}
Define , ,  as in Figure~\ref{Theta-six-fig4}(a). If at least two points of  are to the right of , then by induction, we get a matching of size  in . If at least two points of  are above , then by induction, we get a matching of size  in . Otherwise,  contains  points, and we get a matching of size  in . In all cases we get a matching of size  in .

\item {\em  is larger than  and .}
Define , ,  as in Figure~\ref{Theta-six-fig4}(b). If at least two points of  are above , then by induction, we get a matching of size  in . If any point of  is to the left of , then by induction, we get a matching of size  in . Otherwise,  contains at least  points, and we get a matching of size  in . In all cases we get a matching of size  in . 
\end{itemize}
\end{itemize}
\end{proof}
\section{Strong Matching in 
}

In this section we consider the problem of computing a strong matching in , where  is an axis-aligned square whose center is the origin. We assume that  is in general position, i.e., (i) no two points have the same -coordinate or the same -coordinate, and (ii) no four points are on the boundary of any homothet of . Recall that  is equal to the -Delaunay graph on . \'{A}brego et al. \cite{Abrego2004, Abrego2009} proved that  has a strong matching of size at least . Using a similar approach as in Section~\ref{theta-six-section}, we prove that  has a strong  matching of size at least .

\label{infty-Delaunay-section}
\begin{theorem}
\label{infty-Delaunay-thr}
Let  be a set of  points in general position in the plane. Let  be an axis-parallel square that contains . Then, it is possible to find a strong matching of size at least  for  in .
\end{theorem}

\begin{proof}
The proof is by induction. Assume that any point set of size  in an axis-parallel square , has a strong matching of size  in . If  is  or , then there is no matching in , and if , then by shrinking , it is possible to find a strongly matched pair. Suppose that , and , where . If , then 
, and by
induction we are done. Suppose that , for some . We prove that there are  disjoint squares in ,
each of them matches a pair of points in . Partition  into four equal area squares  which contain  points, respectively; see Figure~\ref{square-fig1}(a). Let  for , where . Let  be the multiset . 
By induction, in , we have a strong matching of size at least
 

In the proof of Theorem~\ref{theta-six-thr}, we have shown the following two claims:

{\bf Claim 1:} {.}

{\bf Claim 2:} {\em If , then either (i) one element in  is equal to  and the other elements are equal to , or (ii) two elements in  are equal to  and the other elements are equal to .}


If , then we are done. Assume that ; in fact, by the induction hypothesis we have an strong matching of size  in . 
We show how to find one more strongly matched pair in each case of Claim 2.

We define x as the smallest axis-parallel square contained in  and anchored at the top-left corner of , which contains all the points in  except  points. If  contains less than  points, then the area of x is zero. We also define x as the smallest axis-parallel square that contains  and anchored at the top-left corner of , which has all the points in  plus  other points of . See Figure~\ref{square-fig1}(a). Similarly we define the squares x, x, and x, x, and x, x which are anchored at the top-right corner of , and the bottom-left corner of , and the bottom-right corner of , respectively.

{\bf Case 1:} {\em One element in  is equal to 3 and the other elements are equal to 1.}

In this case, we have . Without loss of generality, assume that  and . Consider the squares , , , and . Note that the area of some of these squares\textemdash but not all\textemdash may be
equal to zero. See Figure~\ref{square-fig1}(b). By induction, we get matchings of size , , , and , in 
, , , and , respectively. Now consider the largest square among , , , and . Because of the symmetry, we have only three cases: (i)  is the largest, (ii)  is the largest, and (iii)  is the largest.
\begin{figure}[htb]
  \centering
\setlength{\tabcolsep}{0in}
  
  \caption{(a) Split  into four equal area squares. (b)  is larger than , , and . (c)  is larger than , , and .}
\label{square-fig1}
\end{figure}
\begin{itemize}
\item {\em  is the largest square.}
Consider the lines  and  which contain the bottom side and right side of , respectively; the dashed lines in Figure~\ref{square-fig1}(b). Note that  and  do not intersect any of , , and . If any point of  is to the right of , then by induction, we get a matching of size  in . Otherwise, by induction, we get a matching of size  in . In all cases we get a matching of size  in . 
\item {\em  is the largest square.}
Consider the lines  and  which contain the bottom side and left side of , respectively; the dashed lines in Figure~\ref{square-fig1}(c). Note that  and  do not intersect any of , , and . If any point of  is below , then by induction, we get a matching of size  in . Otherwise, by induction, we get a matching of size  in ; see Figure~\ref{square-fig1}(c). In all cases we get a matching of size  in . 

\item {\em  is the largest square.}
Consider the lines  and  which contain the top side and left side of , respectively. If any point of  is above , then by induction, we get a matching of size  in . Otherwise, by induction, we get a matching of size  in . In all cases we get a matching of size  in . 
\end{itemize}


{\bf Case 2:} {\em Two elements in  are equal to 0 and two elements are equal to 1.}

In this case, we have . Because of the symmetry, only two cases may arise: (i)  and , (ii)  and . 
\begin{itemize}
 \item {\em  and .}
Consider the squares , , , and . By induction, we get matchings of size , , , and , in , , , and , respectively. Now consider the largest square among , , , and . Because of the symmetry, we have only two cases: (a)  is the largest, (b)  is the largest. In case (a) we get one more matched pair either in  or in . In case (b) we get one more matched pair either in  or in .

 \item {\em  and .}
Consider the squares , , , and . By induction, we get matchings of size , , , and , in , , , and , respectively. Now consider the largest square among , , , and . Because of the symmetry, we have only two cases: (a)  is the largest, (b)  is the largest. In case (a) we get one more matched pair either in  or in . In case (b) we get one more matched pair either in  or in .
\end{itemize}
\end{proof}

\section{A Conjecture on Strong Matching in }
\label{conjecture-section}
In this section, we discuss a possible way to further improve upon Theorem~\ref{Gabriel-thr}, as well as
a construction leading to the conjecture that Algorithm~\ref{alg1} computes a strong matching of size at least ; unfortunately we are not able to prove this. 

In Section~\ref{Gabriel-section} we proved that  contains at most 16 edges. In order to achieve this upper bound we used the fact that the centers of the disks in  should be far apart. We did not consider the endpoints of the edges representing these disks. By Observation~\ref{no-point-in-circle-obs}, the disks representing the edges in  cannot contain any of the endpoints. We applied this observation only on  and . Unfortunately, our attempts to apply this observation on the endpoints of edges in  have been so far unsuccessful.

Recall that  is a Euclidean minimum spanning tree of , and for every edge  in ,  is the degree of  in , where  is the set of all edges of  with weight at least . Note that  is directly related to the Euclidean distance between  and . Observe that the discs representing the edges adjacent to  intersect . Thus, these edges are in . 
We call an edge  in  a {\em minimal edge} if  is not longer than any of its adjacent edges. We observed that the maximum degree of a minimal edge is an upper bound for . We conjecture that,

\begin{conjecture}
{\em \Inf{}} is at most the maximum degree of a minimal edge.
\end{conjecture}

Monma and Suri~\cite{Monma1992} showed that for every point set  there exists a Euclidean minimum spanning tree, , of maximum vertex degree five. Thus, the maximum edge degree in  is 9. We show that for every point set , there exists a Euclidean minimum spanning tree, , such that the degree of each node is at most five and the degree of each minimal edge is at most eight. This implies the conjecture that . That is, Algorithm~\ref{alg1} returns a strong matching of size at least .

\begin{figure}[ht]
  \centering
    \includegraphics[width=0.4\textwidth]{fig/empty-triangle-figure.pdf}
  \caption{In , the triangle  formed by two adjacent edges  and , is empty.}
\label{empty-triangle-figure}
\end{figure}
\begin{lemma}
\label{empty-triangle-lemma}
If  and  are two adjacent edges in , then the triangle  has no point of  in its interior or on its boundary.
\end{lemma}
\begin{proof}
If the angle between  and  is equal to , then there is no other point of  on  and . Assume that . Refer to Figure \ref{empty-triangle-figure}. Since  is a subgraph of the Gabriel graph, the circles  and  with diameters  and  are empty. Since ,  and  intersect each other at two points, say  and . Connect ,  and  to . Since  and  are the diameters of  and , .
This means that  is a straight line segment. Since  and  are empty and , it follows that .
\end{proof}

\begin{figure}[htb]
  \centering
  \includegraphics[width=.4\textwidth]{fig/convex-quadrilateral.pdf}
  \caption{Illustration of Lemma~\ref{convex-quadrilateral-lemma}: , , , and .}
\label{convex-quadrilateral-fig}
\end{figure}

\begin{lemma}
\label{convex-quadrilateral-lemma}
Follow Figure~\ref{convex-quadrilateral-fig}. For a convex-quadrilateral  with , if   and , then .
\end{lemma}
\begin{proof}
 Let , , , , , , , and ; see Figure~\ref{convex-quadrilateral-fig}. Since ,  Let  be a line passing through  which is parallel to . Since ,  intersects the line segment . This implies that . If , then , and hence  and we are done. Assume that . In this case, . Now consider the two triangles  and . Since  and , . Then we have 

Since , , where the equality holds only if , i.e.,  is a diamond. This completes the proof.
\end{proof}

\begin{figure}[htb]
  \centering
  \includegraphics[width=.7\textwidth]{fig/convex-quadrilateral2.pdf}
  \caption{Solid segments represent the edges of . Dashed segments represent the swapped edges. Dotted segments represent the edges which cannot exist.}
\label{degree8-fig}
\end{figure}

\begin{lemma} 
\label{edge-degree-lemma}
Every finite set of points  in the plane admits a minimum spanning tree whose node degree is at most five and whose minimal-edge degree is at most nine.
\end{lemma}
\begin{proof}
Consider a minimum spanning tree, , of maximum vertex degree 5. The maximum edge degree in  is 9. Consider any minimal edge, . If the degree of  is 8, then  satisfies the statement of the lemma. Assume that the degree of  is 9. Let  and  be the the neighbors of  and  in clockwise and counterclockwise orders, respectively. See Figure~\ref{degree8-fig}. In , the angles between two adjacent edges are at least . Since  and  for , either  or . Without loss of generality assume that  and . We prove that the spanning tree
obtained by swapping the edge  with  is also a minimum spanning tree, and it has one fewer minimal-edge of degree 9. By repeating this procedure at each minimal-edge of degree 9, we obtain a minimum spanning tree which satisfies the statement of the lemma. 
Let . By Lemma~\ref{empty-triangle-lemma},  is outside the triangle , and  is outside the triangle . In addition,  and  are on the same side of the line subtended from . Thus,  is a convex quadrilateral.
Without loss of generality assume that . By Lemma~\ref{convex-quadrilateral-lemma}, . If , we get a contradiction to Lemma~\ref{cycle-lemma}. Thus, assume that . As shown in the proof of Lemma~\ref{convex-quadrilateral-lemma}, this case happens only when  is a diamond. This implies that , and consequently . In addition,  and  for . To establish the validity of our edge-swap, observe that the nine edges incident to  and  are all equal in length. Therefore, swapping  with  does not change the cost of the spanning tree and, furthermore, the resulting tree is a valid spanning tree
since  is not an edge of the original spanning tree ; otherwise , and  would form a cycle. We have removed a minimal edge  of degree 9, but it remains to show that the degree of  and  does not increase to six and new minimal edge of degree 9 is not generated. Note that  and  are not the edges of , and hence,  and  are still less than six. In order to show that no new minimal edge is generated, we differentiate between two cases:

\begin{itemize}
 \item . Since  and ,  can be adjacent to at most two vertices other than  and , and hence ; similarly . Thus, , , , and  are of degree at most four, and hence no new minimal edge of degree 9 is generated.

  \item . W.l.o.g. assume that . This implies that . Since  and ,  is adjacent to at most three vertices other than  and . Let  be the neighbors of  in clockwise order. Note that  is not adjacent to ,  nor . But  can be connected to another vertex, say , which implies that . We prove that the spanning tree obtained by swapping the edge  with  is also a minimum spanning tree of node degree at most five, which has one fewer minimal edge of degree 9. The new tree is a legal minimum spanning tree for , because . In addition,  and . Since  and  are illegal edges, . Thus, , , , , and  are of degree at most four and no new minimal edge of degree 9 is generated. This completes the proof that our edge-swap reduces the number of minimal-edges of degree nine by one.
\end{itemize}
\end{proof}

\section{Conclusion}
\label{conclusion}

Given a set of  points in general position in the plane, we considered the problem of strong matching of points with convex geometric shapes. A matching is strong if the objects representing whose edges are pairwise disjoint. In this paper we presented algorithms which compute strong matchings of points with diametral-disks, equilateral-triangles, and squares. Specifically we showed that:
\begin{itemize}
 \item There exists a strong matching of points with diametral-disks of size at least .
 \item There exists a strong matching of points with downward equilateral-triangles of size at least .
\item There exists a strong matching of points with downward/upward equilateral-triangles of size at least .
\item There exists a strong matching of points with axis-parallel squares of size at least .
\end{itemize}
 
The existence of a downward/upward equilateral-triangle matching of size at least , implies the existence of either a downward equilateral-triangle matching of size at least  or an upward equilateral-triangle matching of size at least . This does not imply a lower bound better than  for downward equilateral-triangle matching (or any fixed oriented equilateral-triangle).

A natural open problem is to improve any of the provided lower bounds, or extend these results for other convex shapes. The specific open problem is to prove that Algorithm~\ref{alg1} computes a strong matching of points with diametral-disks of size at least  as discussed in Section~\ref{conjecture-section}.
\bibliographystyle{abbrv}
\bibliography{Strong-Matching.bib}
\end{document}
