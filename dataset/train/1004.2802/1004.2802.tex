\documentclass[11pt,reqno,a4paper]{amsart}
\pdfoutput=1
\usepackage{bold-extra}
\usepackage{microtype}
\usepackage{amsfonts,amstext,amsmath,amssymb,stmaryrd,calc,mathrsfs,booktabs}
\def\mathrlapinternal#1#2{\rlap{}}
\def\mathrlap{\mathpalette\mathrlapinternal}
\newcommand{\rua}[1]{\mathrel{\mathrlap{\:\xrightarrow{{\color{white} #1}}}{\xrightarrow{#1}\:}}}
\newcommand{\eqdef}{\stackrel{\raisebox{-.15ex}{\scalebox{.5}{\upshape\textrm{def}}}}{=}}
\def\vec#1{\mathchoice{\mbox{\boldmath}}
{\mbox{\boldmath}}
{\mbox{\boldmath}}
{\mbox{\boldmath}}}
\newcommand{\norm}[1]{\|#1\|}
\newcommand{\tup}[1]{\langle #1\rangle}
\newcommand{\fire}[1]{[#1\rangle}
\newcommand{\dc}{\mathop{\downarrow}\!}
\newcommand{\uc}{\mathop{\uparrow}\!}
\newcommand{\ra}{\mathrel{\rightarrow^\ast}}
\newcommand{\ru}[1]{\xrightarrow{#1}}
\newcommand{\dom}{\mathop{\mathsf{dom}}}
\newcommand{\remove}[1]{\overline{#1}}
\newcommand{\prefix}[1]{\mathsf{Pref}(#1)}
\newcommand{\nat}{\mathbb{N}}
\newcommand{\infwords}{\Sigma^\ast_\mathsf{acc}}
\newcommand{\lub}{\mathsf{lub}}
\newcommand{\Ltrans}{L_{\mathsf{acc}}}
\newcommand{\R}{\mathcal{R}}
\newcommand{\subword}{\preceq}
\newcommand{\tickYes}{Yes}\newcommand{\tickNo}{No}\usepackage{color,graphicx,tikz}
\usetikzlibrary{petri}
\usetikzlibrary{positioning}
\usetikzlibrary{arrows}
\usetikzlibrary{automata}
\usetikzlibrary{trees}
\usepackage{multirow,longtable}
\usepackage{listings}
\lstset{language=C++,flexiblecolumns=true,basicstyle=\footnotesize,numbers=left,numberstyle=\tiny}
\newcommand{\cpp}{\lstinline[language=C++]}
\usepackage[numbers]{natbib}\renewcommand{\cite}{\citep}
\newcommand{\natconfdetails}[2][]{, #1 #2}
\makeatletter
\def\NAT@spacechar{~}\makeatother
\renewcommand{\cite}{\citep}
\def\bibfont{\footnotesize}
\providecommand{\doi}[1]{doi:\href{http://dx.doi.org/#1}{\nolinkurl{#1}}}
\makeatletter
\newcommand{\citepay}[2][\@empty]{\citeauthor{#2}~(\ifx#1\@empty\relax\else#1
  \fi\citeyear{#2})}
\def\bibinfo@X@doi#1{#1}
\providecommand{\doi}[1]{doi:\href{http://dx.doi.org/#1}{\nolinkurl{#1}}}
\makeatother
\newcommand{\citeay}[1]{\citeauthor{#1}, \citeyear{#1}}
\usepackage{amsthm,thmtools}
\theoremstyle{plain}
\newtheorem{theorem}{Theorem}
\newtheorem{lemma}[theorem]{Lemma}
\newtheorem{proposition}[theorem]{Proposition}
\newtheorem{corollary}[theorem]{Corollary}
\newtheorem{fact}[theorem]{Fact}
\theoremstyle{definition}
\newtheorem{definition}[theorem]{Definition}
\newtheorem{example}[theorem]{Example}
\theoremstyle{remark}
\newtheorem{claim}{Claim}[theorem]
\providecommand{\qedhere}{\qed}
\renewcommand{\paragraph}{\subsubsection*}
\usepackage{url}
\usepackage{hyperref}
\def\UrlBreaks{\do\@\do\\\do\/\do\!\do\|\do\;\do\>\do
\dom u^\omega&\eqdef\{s\in\dom u\:\mid s\leq u(s)\}\\
u^\omega(s)&\eqdef\mathsf{lub}(\{u^n(s)\mid
n\in\mathbb{N}\})&\text{for all 
  in .}
\label{eq-bff}
    D = \bigcup_{s_f\in B_f,s_{f'}\in B_{f'}}(\uc s_f \cap\uc s_{f'})
  
  T(\mathcal{N}'(1,0,0,0))=a^\ast\cup\bigcup_{n\geq 0}a^nb\{c,d\}^{\leq n}\;.

  s_i&\ru{a}s_{i+1}\tag{single step}\\
  \intertext{or there exists  in  such that}
  s_i&\ru{u^\omega}s_{i+1}\;.\tag{accelerated step}

    w_n=v_{n+1}u_n^\omega v_n\cdots u_1^\omega v_1\in T_\mathsf{acc}(\mathcal{S})
  
    w'_n=v_{n+1}u_n^{k_n}v_n\cdots u_1^{k_1}v_{1}\in T(\mathcal{S})\;.
  
    s_0\rua{v_{n+1}u_n^\omega}s\rua{v_nu_{n-1}^\omega v_{n-1}\cdots
  u_1^\omega v_1}s_f\;,
  
    w'_{n-1}=v_nu_{n-1}^{k_{n-1}}v_{n-1}\cdots u_1^{k_1}v_1\in T(\mathcal{S}(s))\;.
  
  \forall n\in\mathbb{N},n\ru{a}&\;n+1,&
  \omega\ru{a}&\;\omega,&
  \omega\ru{b}&\;\omega\;.

    w \{au,bv\}^\ast\subseteq T_\mathsf{acc}(\mathcal{S})\;.
  
    w (bv)^Nau(bv)^Nau\cdots au(bv)^N\in T_\mathsf{acc}(\mathcal{S})
  
    w'(bv)^Nau_1(bv)^Nau_2\cdots au_{N-1}(bv)^N\in T(\mathcal{S})\;.
  
  q_0&\ru{w'(bv)^{N-k_1-k'_1}}q_1\ru{(bv)^{k_1}}q_1\ru{(bv)^{k'_1}au_1(bv)^{N-k_2-k'_2}}q_2,\\q_2&\ru{(bv)^{k_2}}q_2\ru{(bv)^{k'_2}au_2\cdots
    au_{N-1}(bv)^{N-k_N-k'_N}}q_N,\\q_N&\ru{(bv)^{k_N}}q_N\ru{(bv)^{k'_N}}q_f\in F
  
    \delta(q_i,(bv)^{k'_i}au_i\cdots au_{j-1}(bv)^{N-k_j-k'_j})=q_i\;.
  
    (bv)^{k_i+k'_i}au_i\cdots au_{j-1}(bv)^{N-k_j-k'_j}\neq
    (bv)^{k'_i}au_i\cdots au_{j-1}(bv)^{N-k_j-k'_j+k_i}
  
  s_{i} \ru{x} s && s \rua{azu_{i+1}^\omega
  v_{i+2}u_{i+2}^\omega\cdots v_ju_j^\omega x} s_a && s \ru{byx} s_b\;.\tag*{\qedhere}
  
    q:&\;\mathtt{if}\;c=0\;\mathtt{goto}\:q'\;\mathtt{else}\;c\text-\text-;\;\mathtt{goto}\:q''\\
    q:&\;c\text{++};\;\mathtt{goto}\:q'\\
    q:&\;\mathtt{halt}

    v_m=w_1^{j_1}\cdots w_n^{j_n}\;,
  \label{eq:alt}
    \alt(w_i^{j_i})\leq 2|w_i|\;.
  
    2m = \alt(w_1^{j_1}\cdots w_n^{j_n})\leq \sum_{i=1}^n\alt(w_i^{j_i})\leq
    2\sum_{i=1}^n|w_i|\:.\tag*{\qedhere}
  
    q':&\;\mathtt{if}\;c_3=0\;\mathtt{goto}\:q^\dagger\;\mathtt{else}\;c_3\text-\text-;\;\mathtt{goto}\:q''\\
    q'':&\;c_4\text{++};\;\mathtt{goto}\:q'\\
    q^\dagger:&\;\mathtt{if}\;c_4=0\;\mathtt{goto}\:q^\flat\;\mathtt{else}\;c_4\text-\text-;\;\mathtt{goto}\:q^\ddagger\\
    q^\ddagger:&\;c_3\text{++};\;\mathtt{goto}\:q^\dagger\\
    q^\flat:&\;c_3\text{++};\;\mathtt{goto}\:q
  
    q_0(q'_1q''_1)^0q'_1u_1q_1(q'_2q''_2)^1&q'_2u_2q_2(q'_3q''_3)^2q'_3u_3\\&\cdots
    q_i(q'_{i+1}q''_{i+1})^iq'_{i+1}u_{i+1}q_{i+1}\cdots
  
m'(p)\eqdef\begin{cases}f(t,p)&\text{if }(p,t)\in R\\m(p) - f(p,t) +
f(t,p)&\text{otherwise.}\end{cases}

(q,w)&\ru{!a}(q',w')& \text{ if }(q,!,a,q')\in\delta\text{ and }\exists
  w''\in M^\ast,\\&& w''\preceq w\text{ and }w'\preceq w''a\\
(q,w)&\ru{?a}(q',w')& \text{ if }(q,?,a,q')\in\delta\text{ and }\exists
  w''\in M^\ast,\\&& aw''\preceq w\text{ and }w'\preceq w''\;.

(q,w)&\ru{!a}(q',wa)&\text{if }(q,!,a,q')\in\delta\\
(q,uaw)&\ru{?a}(q',w)&\!\!\!\!\!\!\!\text{if }(q,?,a,q')\in\delta\text{ and }u\in(M\setminus\{a\})^\ast\!.

  &\{(q,?,a,q_!)\mid a\in M,q\in Q\}\\\cup\;&\{(q_!,!,a,q_?)\mid a\in\{c,d\}\}\\\cup\;&\{(q_?,?,a,q_!)\mid a\in M\}\;.

2^{F_\omega(n)}=2^{F_{n+1}(n)} \leq
F_2\!\left(F_{n+1}(n)\right)\leq F^2_{n+2}\!\left(n\right)\leq F_{n+3}\!\left(n+1\right)\,,

  A(n)&\eqdef A'_n(2) & A'_0(n) &\eqdef 2n+1 \\ A'_{m+1}(0)&\eqdef 1 & A'_{m+1}(n+1) &\eqdef A'_m(A'_{m+1}(n))\;.

  p=A'_2(A'_3(\dots (A'_n(1) - 1)\dots)-1)\;.

  J_0&\eqdef\{0,\dots,|v_1u_1|-1\}\text{ and}\\
  J_i&\eqdef\{|u_i|,\dots,|v_{i+1}u_{i+1}|-1\}\text{ for .}

    k_0&\eqdef 0,&k_{i+1}&\eqdef k_i+|v_{i+1}|+|u_{i+1}|\;,
  
    \norm{s_{i,j}}\leq g^{k_i + j}(n)\;.
  
    \ell_{0,\min J_0}&\eqdef 0,&\ell_{i,j+1}&\eqdef \ell_{i,j}+1,&\ell_{i+1,\min
      J_{i+1}}&\eqdef\ell_{i,\min J_i}+|J_i|\;.
  
    \norm{s_{i,j}}\leq (g^2)^{\ell_{i,j}}(n)\;,
  
    k_i+j \leq 2\cdot \ell_{i,j}\;.
  
    k_{i+1}+\min J_{i+1}
    &=k_{i+1}+|u_{i+1}|\tag{by def.\ of }\\
    &=k_i+2|u_{i+1}|+|v_{i+1}|\tag{by def.\ of }\\
    &=k_i+|u_i|+2|u_{i+1}|+|v_{i+1}|-|u_i|\\
    &=k_i+\min J_i+2|u_{i+1}|+|v_{i+1}|-|u_i|\tag{by def.\ of }\\
    &\leq 2\cdot\ell_{i,\min J_i}+2|u_{i+1}|+|v_{i+1}|-|u_i|\tag{by ind.\ hyp.}\\
    &\leq 2\cdot\ell_{i,\min
      J_i}+2|u_{i+1}|+2|v_{i+1}|-2|u_i|\tag{since }\\
    &=2\cdot\ell_{i+1,\min J_{i+1}}\;.\tag{by def.\ of }
  
    \norm{s_{i,j}}\leq g^{k_i+j}(n)\leq g^{2\cdot\ell_{i,j}}(n)\;.\qedhere
  
    s_{i,j}\rua{p'_{i,j}u_{i+1}^\omega x}s&&s\rua{azu_{i+2}^\omega\cdots
      v_{i'}u_{i'}^\omega
      p_{i',j'}}s_{i',j'}\rua{p'_{i,j}u_{i+1}^\omega x}s_a&&s\ru{byx}s_b\;.
  
 m_1&\eqdef\max_{f(\vec x)=\vec A\vec x+\vec b\in L,1\leq i,j\leq k}\vec A[i,j]
\intertext{and  as the maximal constant}
m_2&\eqdef\max_{f(\vec x)=\vec A\vec x+\vec b\in L,1\leq i\leq k}\vec b[i]\;.

\norm{f(\vec x)}&\leq k\cdot{}m_1\cdot\norm{x}+m_2\;,

  \vec A^n[i,j] &= \sum_{\psi\in \{i\} \times [1,k]^{n-1} \times \{j\}}
  \prod_{\ell\in [0,n-1]} \vec A[\psi_\ell,\psi_{\ell+1}]\\
  \vec d_n[j]&=\sum_{\psi\in\times [1,k]^{n} \times \{j\}} \left(\vec d_0[\psi_0]\cdot
  \prod_{\ell\in [0,n-1]} \vec A[\psi_\ell,\psi_{\ell+1}]\right)\,.
  
    \vec A &= \prod_{i=n}^1\vec A_i\qquad\qquad
    \vec b =\sum_{j=1}^{n}\left(\prod_{i=n}^{j+1} \vec
      A_i\right)\cdot \vec b_j\\
    u^k(\vec x)&= \vec A^k\cdot\vec x+\sum_{\ell=0}^{k-1}\vec
    A^\ell\cdot\vec b\\
    &=\left(\prod_{i=n}^1 \vec A_i\right)^{\!\!\!k}\cdot\vec
    x+\sum_{\ell=0}^{k-1}\sum_{j=1}^{n}\left(\prod_{i=n}^1 \vec
      A_i\right)^{\!\!\!\ell}\cdot\left(\prod_{i=n}^{j+1} \vec
      A_i\right)\cdot \vec b_j
    \intertext{thus}
    \norm{u^\omega(\vec x)[j]}
    &\leq\norm{\vec A^k\cdot\vec x}+\sum_{j=0}^{k-1}\norm{\vec A^j\cdot\vec b}\\
    &\leq(k\cdot m_1)^{n\cdot k}\cdot\norm{\vec x}+n\cdot k\cdot(k\cdot
    m_1)^{n\cdot k}\cdot m_2\\
    &=(k\cdot m_1)^{n\cdot k}\cdot(\norm{\vec x}+n\cdot k\cdot m_2)\qedhere
  
    p_1(\omega)&\eqdef 0&p_1(n)&\eqdef n&
    p_2(\omega)&\eqdef\omega&p_2(n)&\eqdef 1
  
  \Pi_M=\{(a+\varepsilon)\mid a\in M\}\cup\{A^\ast\mid
  A\subseteq M\}

  (q,a)&\mapsto e_ia &\text{if }q\in E_i\\
  (q,a)&\mapsto a    &\text{otherwise.}

    \varphi &::=
    \varphi\wedge\varphi\mid\varphi\vee\varphi\mid\mathsf{X}\varphi\mid\alpha\mathsf{U}\varphi\mid\mathsf{G}\alpha\tag{flat
    formul\ae}\\
    \alpha&::= \bigwedge_{p\in a}p\wedge\bigwedge_{p\not\in a}\neg
    p\;.\tag{alphabetic formul\ae}
  
      a^\ast\cup\bigcup_{0\leq 2m\leq n} a^nb(cd)^m(c^{\leq n-2m}\cup d^{\leq
        n-2m})\;,
    
  \clq(I)\eqdef\{C\subseteq\Sigma\mid\forall a,b\in C, (a,b)\in I\}\;.

  \nu(\{a_1,a_2,\dots,a_k\})&= a_1a_2\cdots a_k&\!\text{if
  }a_1<a_2<\cdots<a_k.

    \begin{cases}(C_1,C_2\cup\{b\},b)&\text{if
    }a<b,~\exists d\in C_1, (b,d)\in D,\\&\phantom{\text{if }}\text{and }\forall d\in C_2, (b,d)\in I,\\
    (C_2,\{b\},b)&\text{if }\exists d\in C_2,(b,d)\in D\;.\end{cases}
  \label{eq:liveabp}
  \mathsf{G}(\mathsf{snd}\Rightarrow\mathsf{X}(\neg\,\mathsf{snd}\mathrel{\mathsf{U}}\mathsf{rcv}))\;,\tag{}

  T_\omega(\mathcal{S})\cap L=\emptyset\text{  iff 
  }T_\omega(\mathcal{S}')\cap L=\emptyset\;.
  
  Obviously, if  is empty, then the same
  holds for .  For the converse, let
   be a word in .  Then, since
   is -closed,  also belongs to
   and to , and thus to
  .  And because  is -closed,
   further belongs to , hence to
  .
\end{proof}

Once our system is normalized against partial commutations, the only
remaining source of trace unboundedness is the main control loop.  By
bounding the number of sessions of the protocol, i.e.\ by unfolding
this main control loop a bounded number of times, we obtain a trace
bounded system.

This transformation would disrupt the verification of \eqref{eq:liveabp},
if it were not for the two following observations:
\begin{enumerate}
\item The full set of all reachable configurations is already explored
  after two traversals of the main control loop.  This is established
  automatically thanks to \autoref{cor:cover} on the 2-unfolding
  of the normalized ABP, which is a trace bounded cd-WSTS.  Thus any
  possible session, with any possible reachable initial configuration,
  can already be exhibited at the second traversal of the system.
\item Our property \eqref{eq:liveabp} is intra-session: it only requires
  to be tested against any possible session.
\end{enumerate}
The overall approach, thanks to the concept of trace boundedness modulo
partial commutations, thus succeeds in reducing the ABP to a trace bounded
system where our liveness property can be verified.
 
\section{Trace Boundedness is not a Weakness}\label{sec:appl}
\begin{table}[t]\caption{\label{tab:decsum}Some decidability results
    for selected classes of cd-WSTS---Petri nets (PN), affine counter
    systems (ACS), and functional lossy channel systems (LCS)---in the
    general and trace bounded cases (t.b.).}\centering{\small
  \begin{tabular}{lcccccc}
    \toprule
    & PN & t.b.\ PN& ACS & t.b.\ ACS & LCS & t.b.\ LCS\\
    \midrule
    Reachability& \tickYes & \tickYes & \tickNo & \tickNo & \tickYes &
    \tickYes\\
     inclusion& \tickNo & \tickYes & \tickNo &
    \tickNo & \tickNo & \tickYes\\
    Liveness & \tickYes & \tickYes & \tickNo & \tickYes & \tickNo &
    \tickYes\\
    \bottomrule
  \end{tabular}}
\end{table}
To paraphrase the title \emph{Flatness is not a
  Weakness}~\citep{flatltl}, trace boundedness is a powerful property for the
analysis of systems, as demonstrated with the termination of forward
analyses and the decidability of -regular
properties for trace bounded WSTS (see also \autoref{tab:decsum})---and is implied by
flatness.  More examples of its interest can be found in the recent
literature on the verification of multithreaded programs, where
trace boundedness of the context-free synchronization languages yields
decidable reachability~\citep{vineet,parikhb}. 

Most prominently, trace boundedness has the
considerable virtue of being decidable for a large class of systems,
the -effective complete deterministic WSTS.  There is
furthermore a range of unexplored possibilities beyond partial
commutations (starting with semi-commutations or contextual
commutations) that could help turn a system into a trace bounded one.

\subsection*{Acknowledgments}
We thank the anonymous reviewers for their careful reading, which
improved the paper.

\bibliographystyle{abbrvnat}
\bibliography{journalsabbr,bounded}
\end{document}
