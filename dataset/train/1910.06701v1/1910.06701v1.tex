

\documentclass[11pt,a4paper]{article}
\usepackage[hyperref]{emnlp-ijcnlp-2019}
\usepackage{times}
\usepackage{latexsym}
\usepackage{bm}
\usepackage{amsfonts}
\usepackage{amsmath}
\usepackage{xspace}
\usepackage{url}
\usepackage{multirow}
\usepackage{graphicx}

\usepackage{booktabs}


\aclfinalcopy 



\newcommand\BibTeX{B{\sc ib}\TeX}
\newcommand\confname{EMNLP-IJCNLP 2019}
\newcommand\conforg{SIGDAT}
\newcommand\OurModel{NumNet\xspace}



\title{Supplementary Material: NumNet: Machine Reading Comprehension with Numerical Reasoning}

\author{Qiu Ran$^1$\thanks{\ \ indicates equal contribution}, Yankai Lin$^1$\footnotemark[1], Peng Li$^1$, Jie Zhou$^1$, Zhiyuan Liu$^2$ \\
  $^1$Pattern Recognition Center, WeChat AI, Tencent Inc, China \\
  $^{2}$Department of Computer Science and Technology, Tsinghua University, Beijing, China\\
Institute for Artificial Intelligence, Tsinghua University, Beijing, China\\
State Key Lab on Intelligent Technology and Systems, Tsinghua University, Beijing, China \\
  \texttt{\{soulcaptran,yankailin,patrickpli,withtomzhou\}@tencent.com}\\
  \texttt{liuzy@tsinghua.edu.cn}
}
\date{}

\begin{document}
\maketitle

\appendix
\begin{table*}[h!]
  \centering
  \small
  
  \begin{tabular}{lcccccc}
    \toprule
    \multicolumn{1}{c}{\multirow{2}{*}{Method}}  & \multicolumn{2}{c}{Comparison}        &\multicolumn{2}{c}{Number} & \multicolumn{2}{c}{ALL} \\
    \cmidrule(r){2-3}  \cmidrule(r){4-5} \cmidrule(r){6-7}
                                           & EM    & F1    & EM    & F1    & EM    & F1\\
    \midrule
    NAQANet+                               & 69.11 & 75.62 & 66.92 & 66.94 & 61.11 & 64.54\\
    $\quad$ - real number                  & 66.87 & 73.25 & 45.82 & 45.85 & 47.82 & 51.22\\
    $\quad$ - richer arithmetic expression & 68.62 & 74.55 & 52.48 & 52.51 & 52.02 & 55.32\\
    
    $\quad$ - passage-preferred            & 64.06 & 72.34 & 66.46 & 66.47 & 59.64 & 63.34\\
    $\quad$ - data augmentation            & 65.28 & 71.81 & 67.05 & 67.07 & 61.21 & 64.60\\
    \bottomrule
  \end{tabular}
  \caption{Baseline enhancements ablation.}
  \label{tab:tricks}
\end{table*}

\section{Baseline Enhancements}
\label{sec:tricks}

The major enhancements leveraged by our implemented NAQANet+ model
include:

(1) ``real number'': Unlike NAQANet only considers integer numbers, we also consider real numbers.

(2) ``richer arithmetic expression'': We conceptually append an extra number ``100'' to the passage to support arithmetic expressions like ``100-25'', which is required for answering questions such as ``\emph{How many percent were not American?}''.





(3) ``passage-preferred'': If an answer is both a span of the question and the passage, we only propagate gradients through the output layer for processing ``Passage span'' type answers.

(4) ``data augmentation'': The original questions in the DROP dataset are generated by crowdsourced workers. For the comparing questions which contain answer candidates, we observe that the workers frequently only change the incorrect answer candidate to generate a new question. For example, ``\emph{How many from the census is bigger: \underline{Germans} or \underline{English}?}'' whose golden answer is ``Germans'' is modified to ``\emph{How many from the census is bigger: \underline{Germans} or \underline{Irish}?}''. This may introduce undesired inductive bias to the model. Therefore, we propose to augment the training dataset with new questions automatically generated by swapping the candidate answers, e.g., ``\emph{How many from the census is bigger: \underline{English} or \underline{Germans}?}'' is added to the training dataset.

We further conduct ablation studies on the enhancements. And the validation scores on the development set are shown in Table~\ref{tab:tricks}.
As can be seen from Table~\ref{tab:tricks}:

(1) The uses of real number and richer arithmetic expression are crucial for answering numerical questions: both EM and F1 drop drastically by up to $15-21$ points if they are removed. 

(2) The passage-preferred strategy and data augmentation are also necessary components that contribute significant improvements for those comparing questions.


\end{document}
