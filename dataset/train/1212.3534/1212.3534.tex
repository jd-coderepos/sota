\documentclass{article}
\usepackage[margin=1.2in]{geometry}
\usepackage{amsmath,amsthm,xspace}
\newtheorem{theorem}{Theorem}

\newcommand{\aest}{{\bf A}}
\newcommand{\best}{{\bf B}}
\newcommand{\cest}{{\bf C}}
\newcommand{\nexptime}{\textsc{NExpTime}\xspace}

\makeatletter
\newcommand{\manuallabel}[2]{\def\@currentlabel{#2}\label{#1}}
\makeatother




\begin{document}

\title{A note on the product homomorphism problem and CQ-definability\thanks{We are grateful to Ross Willard for discussions on the topic
  and 
  for comments on an earlier draft.}}

\author{Balder ten Cate and Victor Dalmau}

\maketitle
 
The \emph{product homomorphism problem} (PHP) takes as input a finite
collection of relational structures  and another
relational structure , all over the same schema, and asks whether
there is a homomorphism from the direct product  to . This problem is clearly solvable in
non-deterministic exponential time. It follows from results in
\cite{Willard10} that the problem is \nexptime-complete. The proof,
based on a reduction from an exponential tiling problem, uses
structures of bounded domain size but with 
relations of unbounded arity.  In this note, we provide a
self-contained proof of \nexptime-hardness of PHP,
and we show that it holds already for directed
graphs, as well as for structures of
bounded arity with a bounded domain size (but without a bound on the
number of relations). More precisely, we obtain:

\begin{theorem}
  \manuallabel{thm:binary}{\thetheorem.1}
  \manuallabel{thm:Willard}{\thetheorem.2}
  \manuallabel{thm:digraphs}{\thetheorem.3}
The PHP is \nexptime-complete~\cite{Willard10}. The lower bound
holds already for
\begin{enumerate}
\item structures with binary relations and a bounded domain size;
\item structures with a single relation and a bounded domain size;
\item structures with a single binary relation
\end{enumerate}
\end{theorem}
This completes the picture, since PHP is solvable in polynomial time when all three of the
above parameters (i.e., number of relations, arity, and domain size) are bounded, as follows from the fact that, in this case, there are only finitely
many different possible input structures up to isomorphism. 

 Theorem~\ref{thm:binary} is proved by an adaptations of the technique
used in \cite{Willard10}. Theorem~\ref{thm:Willard} is proved by a
reduction
from~\ref{thm:binary}. Theorem~\ref{thm:digraphs} is proved
by a reduction 
from 
Theorem~\ref{thm:Willard}.





We also present an application of the above result to the
CQ-definability problem (also known as the PP-definability problem).


\section{Proof of Theorem~\ref{thm:binary}}

\begin{proof}
    By reduction from the exponential tiling problem. We are assuming a fixed set of tile types with associated horizontal and vertical compatibility relations, and the input of the 
    tiling problem consists of an integer  (specified in unary) together with a 
  sequence of (not necessarily distinct) tile types . The problem is to
  decide whether the -by- grid has a valid tiling where 
   is a prefix of the sequence of tiles on the first row, starting at
   the origin.
  It is known that there is a fixed finite set of tile types for which
  this problem is \nexptime-hard.



  For ease of exposition, we will also make use of unary
  relations. These can easily be replaced by binary ones.

   The idea of the reduction is very simple.     We will define  structures, ,
   each having domain . In this way, each element of the product  is a bitstring of length , which we will interpret as a pair of bistrings of length , where the first bitstring is the binary encoding of a horizontal coordinate and the second bitstring is the binary encoding of a vertical coordinate. The structure  will have one element for each tile type, so that a map  can be viewed as a way of assigning a tile type to each position on the -by- grid. Furthermore, by endowing the structures involved with suitable relations, we will ensure that every homomorphism  corresponds to a valid tiling, and vice versa.

    Let  be the horizontal and vertical successor relations on coordinate
    pairs.
    In other words,  and .
 Let  be singleton sets denoting the coordinate pairs
    . In order to make our reduction work, we need to somehow make
sure that the relations  are ``available'' in the product structure , by choosing the factor structures  appropriately.

Let us say that an -ary relation  over the domain  is \emph{decomposable} if it can be represented as a product  where each  is an -ary relation over . Intuitively, this means that
if we include in each factor structure  the relation , then
the product structure  contains the relation . 
Each of the unary relations , being a singleton, is trivially decomposable.
Indeed, if we use the notation  to denote the value of the -th bit of the binary encoding of the number , then .
The binary relations  is \emph{not} decomposable. However, it turns out to be a union of decomposable relations, which will suffice for our purposes. First, observe that whenever  then 
   , and therefore  must have at least one bit that is set to zero. For each , let  be the subrelation of  containing all  for which it is the case that 
the -th bit of  is the least significant bit that is 0.
By definition, we have that . Then 
     decomposes: ,
    where \textsf{id} is the identity relation on  and \textsf{diff} is the difference relation on .
The exact same story holds for 
(where we have that ).

We are now ready to defined the structures  and .
The signature consists of the relations . 
    For , we define 
    
    
    

   The structure  is defined as follows:
   its domain is the set of all tile types. The unary predicate  
  denotes the singleton set  as specified in the instance of the tiling problem.
   The relations  and  contain all pairs of tile types that are horizontally,
    respectively vertically, compatible.

    It is now straightforward to verify that there is a homomorphism 
   if and only if there is a valid tiling.
\end{proof}




\section{Proof of Theorem~\ref{thm:Willard}}



\begin{proof}
The proof proceed by a reduction from Theorem~\ref{thm:binary},
 which states that PHP is in \nexptime even
  for structures of a bounded domain size.  The reduction goes in two steps.
  We first reduce to the case with two
  relations. Let  be any structure  with domain  and with multiple relations
   of respective arity  over .
  We denote by  the structure with domain 
  that has 
\begin{enumerate}
\item[(i)] a unary relation  denoting the set 
\item[(ii)] a relation  of arity
    consisting of 
   the all-zeroes tuple , and, 
   for every  (), the tuple whose first  coordinates
   are all , whose subsequent  coordinates are ,  and whose final  coordinates are 
   again.
\end{enumerate}
This transformation can be carried out in
  polynomial time, and it increases the domain of each structure with
  at most one element. Furthermore, we claim that
   if and
  only if . 
In one direction, suppose . By
  construction (and, more specifically, due to the presence of the
  unary relation ),  must map every element of  to
  an element of . It is then easy to see that  is in fact a
  homomorphism from  to . 
  Conversely, suppose . Let  be the map from 
  to  that extends  such that every element of 
  containing a 0 is sent to the element 0 of . 
  Then  is a homomorphism from 
   to . This follows from 
  the fact that (i) no element containing a 0 can 
  belong to the  relation in , 
  and (ii) if a tuple in the relation  of 
  includes an element containing a , then this
  tuple consists entirely of elements that contain a 0, 
  and hence  maps the tuple in question to 
  the all-zeroes tuple, which belongs again to  in .

  As a final step, we further reduce to the case with a single relation.
  This is done by replacing each structure with two relations,  and
  , by the structure with the same domain and with a single
  relation that is defined as the cartesian product of  and . 
  Again, this transformation can be carried out in polynomial time,
  it does not affect the domains of the structures involved, and
  it preserves the existence or non-existence of a homomorphism
  from  to .  
\end{proof}







\section{Proof of Theorem~\ref{thm:digraphs}}

\begin{proof}
  We shall give a reduction from the PHP with a single
  relation (Theorem~\ref{thm:Willard}). Let  be
  an instance of the PHP, using a single -ary relation . We may
  assume without loss of generality that, for each structure
    among  , the projection of
   to the first coordinate is the entire domain .
  This is because we can always replace the -ary relation  by
  the
  -ary relation  where 
  indicates here the cartesian product.
 This
  transformation can be carried out in polynomial time and it does not affect
  the existence or non-existence of a homomorphism from
   to .
Henceforth, we shall use  to denote the unique relation, and  to
denote its arity. 
If  is a -ary tuple and  we shall denote by  the th component of .

For every , we define  to be the following digraph:


The nodes of  include all elements of . Furthermore,
for every tuple ,  contains 
additional nodes, which we
denote by  with .
 These nodes are connected by
the following directed edges:
\begin{itemize}

\item  for every .

\item  for every .

\end{itemize}



We define  as  the digraph obtained from  in the
same way, except that we further add 
 additional elements  called {\em sink nodes},
connected by edges  for every , and 
an edge from every element of  to every sink node.

\begin{trivlist}
\item \textbf{Claim:} there is a homomorphism  if and only if
  there is a homomorphism .
\end{trivlist}

In the remainder, we prove this claim,
which immediately implies the theorem. 
We start with the more difficult direction: let  be a homomorphism from  to .  We shall define from  a homomorphism  from  to . Let  be a node of  . 



\begin{itemize}

\item If  for all  then we say that  is of ``type ''. In this case we define .

\item If, for all ,  where  is a tuple in (the relation of) 
and  then:

\begin{itemize}

\item If, in addition, there exists some  such that  for every  then we say that  is of ``type ''.  Note that  is a tuple in 
 and hence  (where  is applied component-wise) is a tuple of . In this case, define  to be .

\item Otherwise we say that  is of ``type '' and we set 
  to the sink node  where . Observe that, in this case, necessarily .

\end{itemize}

\item If  is not in any of the previous types then we say that is of ``type ''. In this case, we shall prove there exists a vertex  of type  such that for 
every vertex  of type  the following holds: 

In this case we set .
Let us show that such  exists. If there exists  such that  and  and  then clearly  does not have an outgoing edge to any vertex of type  and we can set  to be any arbitrary vertex of type . Same applies if there exist  such that . Consequently we are left with the case in which there exists some  such that for every ,  or  for some tuple  in . Define  to be  in the first case and  in the second and set .

Let  be a node of type . We shall prove that for every , if  is an edge of  then so
if . The claim is obvious whenever . 
Assume now that . Since  has only one outgoing edge (to ) in  it follows that 
. The claim follows from the fact that  and  contains edge . 
\end{itemize}




Let us prove that  is indeed a homomorphism. Let  be an edge in  and let  and . We shall prove 
that  belongs to  by means of a case analysis on the types of  and . Notice that  is necessarly of type  or  since nodes of type  or  do not have incoming edges.



\begin{itemize}

\item  is of type . If  is of type  the claim follows from the fact that  has an edge from every element in  to every sink vertex. Assume now that  is of type , that is, of the form .
Since  is an edge of  and  is of type  it follows that
 for every . Hence  and, since  defines a homomorphism, 
 is the th component of  ( is applied component-wise). It follows that  contains the edge from  to
.



\item  is of type . Then necessarily there exists  and  such that  and  and the claim follows directly from the definitions.

\item  is of type  then  is necessarily of type  as well. Furthermore, it follows that if  is  then necessarily .




\item  is of type . It follows directly from the definition of  and the fact that every vertex of type  is mapped by  to a sink node.
\end{itemize}
 
 

Conversely, let  be a homomorphism from  to
. Recall that each element of  is in
particular an element of .  We claim that the
restriction of  to  is a
homomorphism from  to . 

First, we show 
that, for each element  of ,  is an
element of .
Let  be any element of .  Recall
that we have assumed that the projection of  on the first
coordinate is the entire domain of . Hence, each  is the
first component of some tuple in . By construction of
, this implies that  has an outgoing path of length
 in , and hence,  has an outgoing path of length 
in .  It follows that  must map  to a node of
 that has an outgoing path of length . By construction
of , then,  must be an element of .

Next, we shall show that  is a
homomorphism. Let
, where each . Then we have that
 belongs to ,  for each .
Consequently,  satisfies the conjunctive
query 
It follows that  satisfies the same conjunctive
query in , and therefore, since conjunctive
queries are preserved by homomorphisms, 
 satisfies  in . It follows by
construction of  that .
\end{proof}













\section{Application: CQ-definability}



The \emph{CQ-definability problem} (also known under the name
PP-definability, and several other names), is the problem with input an instance  and a relation 
over the domain of , to decide whether there is a conjuctive query
 such that .  It has been long known that this problem is
decidable in co\nexptime (see discussion and references in
\cite{Willard10}). It was shown in \cite{Willard10} that the
CQ-definability problem is co\nexptime-complete, even for instances of
a bounded domain size. On the other hand, the proof used relations of
arbitrarily large arity. We show that  the same problem is
co\nexptime-complete for a fixed schema (but without a bound on the
size of the domains of the instances).



\begin{theorem}
 The CQ-definability problem is
  co\nexptime-hard already for unary queries over a fixed schema
  consisting of a single binary relation. 
\end{theorem}



\begin{proof}
 Reduction from PHP with a single binary relation 
  (Theorem~\ref{thm:digraphs}).  Let instances 
  and  be given. Inspection of the proof of
  Theorem~\ref{thm:digraphs} shows that we may assume that, in each
  of these stuctures, the maximum length of a directed path is precisely ,
  for some fixed natural number .  Let  be the instance consisting
  of the disjoint union of  and , extended with
  the facts  for all  and , and 
  for all , where  and  are fresh
  elements. Observe that each , and also , by construction, has an outgoing
  path of length , while no other elements have an outgoing
  path of length .
  Let .  Then we claim that  if and
  only if  is not definable inside  by a conjunctive query. In one direction,
  if  then clearly  is not
  definable by a conjunctive query, because, by homomorphism preservation, the same
  conjunctive query would have to select .  On the other hand, if 
  , then we can
  construct a query  defining  as follows: first we define 
  to be the canonical Boolean  conjunctice query of , and we define  to be the unary conjunctive query
  expressing that  holds in the submodel of  consisting of
  all elements reachable (in one step) from . By
  construction,   includes all of  and excludes . It
  is also easy to see that  contains no elements other than
   and . Therefore,  defines . 
\end{proof}




\bibliographystyle{plain}
\bibliography{prod}
\end{document}
