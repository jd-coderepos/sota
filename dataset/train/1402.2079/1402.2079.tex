\section{Evaluations}
How might one find that delicate balance in a software license that brings with it simplicity, freedom for the developer, freedom for the user and is most practical. Here we look at the GNU General Public License v2, the GNU General Public License v3, the MIT license and the 2-clause BSD license.

The ISC license, which is virtually identical to the MIT license, is included in the MIT license examination while the Apache v2 license is not discussed here as it delves into patents, which is for another paper. Here we only delve into expression of an idea as shown in the form of code that developers use to produce software.

\subsection{Summary of the Licenses}
\subsubsection{GPL v2}
The GNU General Public License Version 2 comes from the mind of Richard Stallman of the Free Software Foundation. Released in June 1991 By Stallman, it has the following:

\textbf{Advantages}
\begin{itemize}
\item Allows commercial use
\item You are free to modify the source
\item You are free to re-distribute the program, both modified and unmodified.
\item You can place a warranty on the software.
\end{itemize}
\textbf{Disadvantages} are:
\begin{itemize}
\item You must include a copy of the original sources, or provide a written offer for the user to be able to get the sources from you.
\item You must also provide any modified copies of the source code you used to build your program.
\end{itemize}
\textbf{Restrictions} include:
\begin{itemize}
\item You can not sub-license the code under any other further restricted license.
\item You can not sue the author(s) for damages.
\item You must provide a copy of the license with all distributions.
\end{itemize}
This license ensures the maximum freedom of the code to remain open (viewable) and free (as in speech), by placing restrictions upon users and their usage of the code.

\subsubsection{GPL v3}
The GNU General Public License Version 3 is a sequel to Version 2 previously discussed, and comes from a list of refinements that Richard Stallman felt would better address new technological changes and improvements since Version 2 was released. Written by Richard Stallman with assistance from Eben Moglen\cite{gpl3} and published on 29th June 2007\cite{gpl3published}, it features the following:

\textbf{Advantages}
\begin{itemize}
\item Allows commercial use
\item You are free to modify the source
\item You are free to re-distribute the program, both modified and unmodified.
\item You can place a warranty on the software.
\end{itemize}
\textbf{Disadvantages}
\begin{itemize}
\item You must include a copy of the original sources, or provide a written offer for the user to be able to get the sources from you.
\item You must also provide any modified copies of the source code you used to build your program.
\item You must explicitly state your changes made to the source code.
\end{itemize}
\textbf{Restrictions}
\begin{itemize}
\item You can not sub-license the code under any other further restricted license.
\item You can not sue the author(s) for damages.
\item You must provide a copy of the license with all distributions.
\item If you use this software in an embedded system which requires security keys verified in order to operate and work, you must provide the source to the verifier and/or key generator. You can not use this software with hardware Digital Rights Management, for example.
\end{itemize}
This license ensures the maximum freedom of the code to remain open (viewable) and free (as in speech), by placing restrictions upon users and their usage of the code, be it normal software, secured firmware and/or embedded.

\subsubsection{2-clause BSD}
The Simplified aka 2-Clause BSD license, created by the University at Berkeley and named after their operating system which was called the Berkeley Software Distribution (of Unix) is a basic license, practically identical in spirit and intent as the MIT license. It features:

\textbf{Advantages}
\begin{itemize}
\item Allows commercial use
\item You are free to modify the source
\item You are free to distribute the program, whether modified or not.
\item You can sub-license the code under any other license you choose, provided they have similar requirements in attribution.
\item You can provide a warranty on the end product.
\end{itemize}
\textbf{Disadvantages}
\begin{itemize}
\item You must include a copy of the license which acknowledges the original author(s) and has the disclaimer.
\end{itemize}
\textbf{Restrictions}
\begin{itemize}
\item You can not sue the author(s) for damages.
\end{itemize}

\subsubsection{MIT}
The MIT license, also famous for being associated with the X Window System (also created at MIT), the MIT license is a very basic license that is very easy to understand and follow. Written by MIT, features include:

\textbf{Advantages}:
\begin{itemize}
\item Allows commercial use
\item You are free to modify the source
\item You are free to distribute the program, whether modified or not.
\item You can sub-license the code under any other license you choose, provided they have similar requirements in attribution.
\end{itemize}
\textbf{Disadvantages}
\begin{itemize}
\item You must include a copy of the license which acknowledges the original author(s) and has the disclaimer.
\end{itemize}
\textbf{Restrictions}
\begin{itemize}
\item You can not sue the author(s) for damages.
\end{itemize}
The MIT license is functionally equivalent to the 2-clause BSD license, except with wording removed that was made redundant with the Berne Convention. Another license, the ISC license, is functionally identical to the MIT license as well.

\subsubsection{Others}
There are other "licenses" which are available which offer more and more freedoms, both for the developer, the user and the code. They include:
\begin{itemize}
\item Unlicense - Essentially Public Domain but with an implicit fall-back to an MIT-like license if the user is in a jurisdiction that does not recognise the voluntary abandonment of an author's rights.
\item Public Domain - This is the ultimate permission for source code: Do what you want, you already have every permission there is. You already own it. Does not apply to all jurisdictions and is said to be "tricky". Not applicable in Germany and most parts of Europe.
\item Basic Software License - A very simple license. Contains a copyright line, a statement saying "Do with this as you wish", and a disclaimer against liability. For those who are looking for the next best thing to public domain, but something shorter than the Unlicense.
\end{itemize}

\subsection{Evaluating the Philosophies}
\subsubsection{Source to Study}
At a minimum, one might find just having the right to be able to read the source to study it and ensure there is nothing malicious in it might be acceptable. Following on from that, a user would then rely upon copyright law to allow them to make their desired changes to the source for personal use. If the developer wished to share those changes with others, it would be necessary to make a list of these changes so others can re-produce them, or provide them in a "patch" file that other developers can apply to the source to re-create those changes automatically. This is the method Daniel J. Bernstein originally followed with his qmail software suite \cite{djb}.

\subsubsection{Source to Modify and Share}
One step beyond the above is to allow fellow developers to share their changes freely with others, without any restriction. If they add a new feature, modify an existing feature or just fix a bug, it is important that developers feel they can freely send those changes back to the original developer, or just re-distribute their modified version of the code. Either way, the improved program gets released and contributes to a universal pool of useful software.

\subsubsection{Free as in Price}
Capitalism states\cite{supplydemand} that where there is a demand, there is a market, and where there is a market, there is always a certain high price that people are willing to pay, and sellers should always try to charge the most they can get away with and that the market can bare.

This does not help developing and third-world countries, students and those on very low income who are just trying to get ahead and make their way in the world. So many people find it very useful to be able to just purchase computer hardware, and then to load it with free as in price software. If they want to learn how the software works then it helps if the software source code is open. In some cases, this works well for those who can not afford sometimes expensive guides and books to learn from.

Free as in price means everyone can get the software, either completely free or for just the cost of distribution.

\subsubsection{Free as in Freedom}
Like free speech, developers and users alike appreciate that when they receive a program, whether as source and binary or just source, that it allows them to be able to modify it to suit their needs, fix any bugs, add any new features and then to be able to share those changes with others. This is in stark contrast with proprietary software which is provided in binary form only, and you have to rely upon the original developer(s) to fix bugs and add new features.

\subsubsection{Freedom of...}
Freedom of the Source, Freedom of the Developer and Freedom of the User is where the ideals behind open source licenses really diverge. Freedom of the code is best emphasised in the GPL2 and GPL3 licenses, Freedom of the User is a mix between GPL2, GPL3 and MIT/BSD and Freedom of the Developer is best met with the BSD/MIT licenses.

Guaranteeing that the source code to a program can be studied, modified and that those changes will also be kept open so that others can use them as well is the basic concept behind the GNU General Public License. By forcing the developer to keep their changes public as well, this ensures that the source of the program, regardless of who has it and what they may or may not have done to it, is always available for people to use. Though an admirable intent, it does means that if a developer extends the software with some trade secrets, those secrets must now be known and revealed to whomever the software is distributed to.

Any software linked to that program must also have its source code open with the same restrictions. Sometimes, people don't make any changes at all beyond compiling it and making a larger system combined with other pieces to make a more complete environment. However, even if no changes are made at all, the distributor is still required to host the source code themselves. This leads to 1,000's of copies of the same software being hosted and can be considered redundant, especially if the original developer has reliable hosting for the code.

This is in contrast to the simpler MIT and BSD licenses which state that the receiving developer can use the code, modify it, re-write it, do whatever they like with it provided that the original developers name is left in tact within the code as acknowledgement of original authorship and that no-one tries to sue them. This provides a much easier way for anyone, individual or corporation, to be able to benefit from the code. Many developers find programming to be a fun and intellectual exercise which sets out to achieve a purpose, after which, they might release and would get a sense of satisfaction knowing that they have helped other developers have to do less in order to make a great product or system.

\section{Results and Discussion}
Can one gain a freedom from a restriction? Can having more restrictions on a license make it seem more free? Code is not a living being, so it has no rights granted to it. So it comes down to what the developer permits you to do with it. The MIT license is clearly a license, whereas several parts of the GPL make it seem more like a contract; a contract that is enacted upon your receiving it.

This contract states what you can and can't do with it, what you must do when making changes and what you are required to do if sharing it or re-distributing it to others. It does this via a long list of detailed conditions that many feel you need a law degree just to read and comprehend.

Whereas, the MIT license is a license which says to the receiving developer: use this code however you want, just leave my name in the code, do not sue me if something in here breaks and make sure a copy of this list of permissions and do-not-sue-me notice stays with the software.

Take the above into account, from the perspective of freedom to the user and developer to be able use and benefit from the code and it's capabilities, and it would appear that the MIT and BSD license (with the similarly worded ISC license) allow the most freedom to use and enjoy the code in whatever way the user sees fit.

No-one likes being forced to do something they do not particular want to do, and therefore anything they are forced to do may have a lower quality output from obligation. However, if giving back was optional and keeping in the spirit of how a developer meant the code to be used and shared, they are more likely to get less submissions back, but they would be of higher quality as there would most likely be more pride in the code and it's functionality. This is in contrast to the forced giving back of the GPL which would get a much higher level of submissions, but many of them would be of a lower quality and may only be of benefit to the developer who implemented those changes to suit their own environment and would be of little practical use to anyone else.
