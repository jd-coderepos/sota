\documentclass{article}[11pt]
\usepackage{amssymb,verbatim}


\usepackage[letterpaper,hmargin=1in,vmargin=1.25in]{geometry}
\usepackage{graphicx}

\newcommand{\av}{\mathbf{a}}
\newcommand{\bv}{\mathbf{b}}
\newcommand{\cv}{\mathbf{c}}
\newcommand{\kv}{\mathbf{k}}
\newcommand{\uv}{\mathbf{u}}
\newcommand{\vv}{\mathbf{v}}
\newcommand{\xv}{\mathbf{x}}
\newcommand{\yv}{\mathbf{y}}
\newcommand{\zv}{\mathbf{z}}

\newcommand{\am}{\mathbf{A}}
\newcommand{\gm}{\mathbf{G}}
\newcommand{\km}{\mathbf{K}}
\newcommand{\um}{\mathbf{U}}
\newcommand{\vm}{\mathbf{V}}
\newcommand{\xm}{\mathbf{X}}
\newcommand{\zm}{\mathbf{Z}}

\newcommand{\rk}{{\rm rank}}

\newtheorem{lemma}{Lemma}
\newtheorem{cor}{Corollary}
\newtheorem{theor}{Theorem}
\newtheorem{example}{Example}

\newenvironment{proof}{\noindent {\bf Proof. \ }}{\hfill \vrule height 2pt depth 4pt width 6pt\par\noindent}

\begin{document}


\title{
An Information-Theoretic Analysis \\
of the Security of Communication Systems \\
Employing the Encoding-Encryption Paradigm
}
\author
{Fr\'ed\'erique Oggier and Miodrag J. Mihaljevi\'c
\thanks{
Fr\'ed\'erique Oggier is with Division of Mathematical Sciences,
School of Physical and Mathematical Sciences, Nanyang
Technological University, Singapore. Miodrag Mihaljevi\'c is with
Mathematical Institute, Serbian Academy of Sciences and Arts,
Belgrade, Serbia, and with Research Center for Information
Security (RCIS), Institute of Advanced Industrial Science and
Technology (AIST), Tokyo, Japan.  Email: frederique@ntu.edu.sg,
miodragm@turing.mi.sanu.ac.rs. Part of this work already appeared
at IEEE ICT 2010. } } \maketitle

\begin{abstract}
This paper proposes a generic approach for providing enhanced
security to communication systems which encode their data for
reliability before encrypting it through a stream cipher for
security. We call this counter-intuitive technique the {\em
encoding-encryption} paradigm, and use as motivating example the
standard for mobile telephony GSM. The enhanced security is based
on a dedicated homophonic or wire-tap channel coding that
introduces pure randomness, combined with the randomness of the
noise occurring over the communication channel. Security
evaluation regarding recovery of the secret key employed in the
keystream generator is done through an information theoretical
approach.

We show that with the aid of a dedicated wire-tap encoder, the
amount of uncertainty that the adversary must face about the
secret key given all the information he could gather during
different passive or active attacks he can mount, is a decreasing
function of the sample available for cryptanalysis. This means
that the wire-tap encoder can indeed provide an information
theoretical security level over a period of time, but after a
large enough sample is collected the function tends to zero,
entering a regime in which a computational security analysis is
needed for estimation of the resistance against the secret key
recovery.
\\
{\em Keywords}: error-correction coding, security evaluation,
stream ciphers, randomness, wireless communications, homophonic
coding, wire-tap channel coding.
\end{abstract}





\section{Introduction}

Most communication systems take into account not only the reliability
but also the security of the data they transmit. This is particularly
true in wireless environment, where the data is inherently more sensible
to security threats. Consequently, the design of such systems need to
include both coding schemes for providing error-correction and
ciphering algorithms for encryption-decryption. It is common practice to first
encrypt the data to ensure its safety, and then to encode it for reliability.
In this paper, we consider the reverse scenario, namely systems which first
encode the data, and then encrypt it, which we call the
{\em encoding-encryption paradigm}.

Though counter-intuitive at first, there are actually many real
life applications where the encoding encryption paradigm is used.
A famous illustrative example is the most widespread standard for
mobile telephony GSM, standing for ``Global System for Mobile
Communications" (see \cite{GSM-coding} and \cite{GSM-encryption},
for the coding, respectively security details). In the GSM
protocol, the data is first encoded using an error-correction code
so as to withstand reception errors, which considerably increases
the size of the message to be transmitted. The encoded data is
then encrypted to provide privacy (secrecy of the communications)
for the users.

It is interesting to mention that block ciphers are not suitable
in the context of the encoding-encryption paradigm, since the
receiver needs to first decrypt the data despite the noise, before
performing the decoding. This leads to use of stream ciphers and
thus when we refer to the security of systems using the
encoding-encryption paradigm, we implicitly mean the security of
the keystream generator and the users' secret key.

From a security perspective, there are of course pros and cons to
the encoding-encryption paradigm. Since it implies encryption of
redundant data (introduced by error-correction), it could be an
origin for mounting attacks against the employed keystream
generator. Undesirability of redundant data from a cryptographic
security point of view has indeed been already pointed out in the
seminal work by Shannon \cite{shannon}, where cryptography as a
scientific topic has been established. On the other hand, the
encoding-encryption paradigm has the advantage to offer protection
in the case of a known plaintext attacking scenario, since an
adversary can only learn a noisy version of the keystream, which
makes the cryptanalyis of the employed keystream generator more
complex.

Security evaluation can be performed under two attacking scenarios, depending
on whether one considers an active or passive adversary.

A {\em passive adversary}'s ability is limited to monitoring (and recording)
communications between the legitimate parties, so as to use the recorded data
as input for mounting a {\em known plaintext attack} against the considered
system.

Stronger attacks come from {\em active adversaries}, which can possibly
include many attacking settings. In this paper, we consider active attacks
motivated by the class of so-called Hopper and Blum (HB) authentication
protocols \cite{hopper-ASIACRYPT2001},\cite{juels-CRYPTO2005},
\cite{katz-EUROCRYPT2006},\cite{gilbert-EUROCRYPT2008},\cite{gilbert-FC2008}.
Following the original work by \cite{hopper-ASIACRYPT2001}, HB authentication
protocols are challenge-response based, where the response could be considered
as the encoded and encrypted version of the challenge, which is deliberately
degraded by random noise. A simple active attack on the improved HB
authentication protocol \cite{katz-EUROCRYPT2006} was provided in
\cite{gilbert}, where it is assumed that an adversary can manipulate
challenges sent during the authentication exchange, and thus learn whether
such manipulations give an authentication failure. The attack consists of
choosing a constant vector and using it to perturb the challenges by computing
the XOR of the selected vector with each authentication challenge vector, and
that for each of the authentication rounds. To summarize, the active attacker
has the following abilities: (i) he can modify the data in the communication
channel between the legitimate parties; and (ii) he can can learn the effect
of the performed modification at the receiving side.
This is the model that will be adopted in this work.

To evaluate the security of systems using the encoding-encryption
paradigm under threats of both passive and active adversaries as
described above, both computational and information theoretical
analyses are valid. In this paper, we focus on the latter. We
propose a security enhanced approach which employs a dedicated
coding, following the frameworks of homophonic \cite{jendal,
massey-1994, ryabko} and wire-tap channel coding
\cite{wyner,thangaraj-IEEE-IT-2007}. The improved security is a
consequence of combining the pure randomness introduced by the
wire-tap coding and the random noise which is inherent in the
communication channel.

We measure the security increase with respect to the secret key in
terms of its equivocation, that is the amount of uncertainty that
the adversary has on the key, given all the information he can
collect. A preliminary study of the security enhancement has been
provided in \cite{misa&frederique} in the case of a passive
adversary. The enhancement is based on the constructions reported
in \cite{mihaljevic-Computimg2009,mihaljevic-IOSpress2009}, and
also motivated by the fact that in the computational complexity
evaluation scenarios, this approach provides resistance against
the generic time-memory trade-off based attacking approaches
\cite{hellman,mihaljevic-IEEE-CL-2007}, and particular powerful
techniques like the correlation attacks
\cite{fossorier-IEEE-IT-2007}.

\vspace*{0.25cm} \noindent{\em Motivation for the Work}. The aim
of this work is to propose and elaborate a model for the security
evaluation of communication systems which employ the
encoding-encryption paradigm together with a dedicated wire-tap
encoder for security enhancement. In a general security evaluation
scenario, both passive and active attacks should be treated, and
while the enhanced system should be resistant to these, it should
be with a slight/moderate increase of the implementation
complexity and the communications overhead. It may be worth
emphasizing that our target is to increase the security of
existing schemes, such as GSM, which is why we have a small margin
of freedom in designing the security scheme, since we cannot touch
most of the existing components of the system.

\vspace*{0.25cm} \noindent{\em Summary of the Results}. This paper
proposes and analyzes from the information-theoretic point of view
the security of communications systems based on the
encoding-encryption paradigm under passive and active attacks,
when equipped with an additional wire-tap encoder. We show that
with the aid of a dedicated wire-tap encoder, the amount of
uncertainty that the adversary must face about the secret key
given all the information he could gather during different passive
or active attacks he can mount, is a decreasing function of the
sample available for cryptanalysis. This means that the wire-tap
encoder can indeed provide an information theoretical security
level over a period of time, but after a large enough sample is
collected the function tends to zero, entering a regime in which a
computational security analysis is needed for estimation of the
resistance against the secret key recovery.


\vspace*{0.25cm} \noindent{\em Organization of the Paper}. In
Section \ref{sec:model}, we start by describing precisely the
system model together with its security enhanced version and we
dedicate Subsection \ref{subsec:wiretap} to the design of the
wire-tap encoder. The security analysis is done in two parts:
first the passive adversary is studied in Section
\ref{sec:passive}, while the  active one is investigated in
Section \ref{sec:active}. Practical implications of the given
security analysis and some guidelines for design of security
enhanced encoding-encryption based systems are pointed out in
Section \ref{sec:appli}. Concluding remarks including some
directions for future work are given in Section
\ref{sec:conclusion}.



\section{System Model and Wiretap Coding}
\label{sec:model}

We consider a class of communication systems which, to provide
both reliability and security, employs the encoding and then encryption
paradigm, namely: the message is first encoded, and then encrypted
using a stream ciphering.

\begin{figure}[htb]
\leavevmode
\begin{center}
\includegraphics[width=0.55\hsize]{Model.eps}
\caption{Communication system model.}
\label{fig:figure-1}
\end{center}
\end{figure}
The detailed model is shown in Figure \ref{fig:figure-1}.
The transmitter first encodes a binary message/plain text

using an error-correcting code 

that maps a -dimensional plain text to an -dimensional
encoded message, .
The encryption is done using a keystream generator, which takes
as input the secret key  of the transmitter, and outputs

yielding

as the message to be sent over the noisy channel, where
 denotes XOR or modulo 2 addition. We denote the noise
vector by

where each  is the realization of a random variable 
such that  and .
Upon reception of the corrupted encrypted binary sequence of ciphertext

the receiver who shares the secret key  with the transmitter can decrypt
first the message

and then decode  despite of the noise thanks to the error-correction code.
We remark that in practice a keystream generator can be considered as a finite
state machine whose initial state is determined by the secret key and some public
data. For simplicity, and because it does not affect our analysis, we can ignore
the existence of the known data, and focus on the secret key. In this setting the
output of the keystream generator is determined uniquely by the secret key, and
it is enough to assume that the transmitter and receiver only share the key.

Note further that the trick of reversing the order of encryption and error-correction
would not have been possible if a block cipher was used for encryption, since decryption
must be done before removing the channel noise.

We finally assume that there is a noiseless feedback link that connects the
receiver to the transmitter, so that the receiver can either acknowledge the
reception of the message, or inform of the decoding failure, so as to get
the missing message sent back.


\subsection{Enhanced model}

Origins for the construction given in this paper are the
approaches for stream ciphers design recently reported in
\cite{mihaljevic-Computimg2009,mihaljevic-IOSpress2009}, though
the focus of this paper is very different, since its goal is
enhancing the security of existing encryption schemes.
This difference has a number of implications regarding the
security issues and implementation complexity of the scheme.

The construction proposed in this paper employs the following main
underlying ideas for enhancing security:

\begin{itemize}

\item Involve pure randomness into the coding\&ciphering scheme
so that the decoding complexity without knowledge of the secret
key employed in the system approaches the complexity of the
exhaustive search for the secret key.

\item  Enhance security of the existing stream cipher via joint
employment of pure randomness and coding theory, and particularly
a dedicated encoding following the homophonic or wire-tap channel
encoding approaches.

\item Allow a suitable trade-off between the security and the
communications rate: Increase the security towards the limit
implied by the secret-key length at the expense of a low-moderate
decrease of the communications rate.

\end{itemize}

Regarding the homophonic and wire-tap channel coding, note the
following. The main goals of homophonic coding are to provide:
(i) multiple substitutions of a given source vector via
randomness so that the coded versions of the source vectors appear
as realizations of a random source; (ii) recoverability of the
source vector based on the given codeword without knowledge of the
randomization. The main goals of wire-tap channel coding are:
(i) amplification of the noise difference between the main and
wire-tap channel via randomness; (ii) a reliable
transmission in the main channel and at the same time to provide a
total confusion of the wire-tapper who observes the communication
in the main channel via a noisy channel (wire-tap channel).
Accordingly, homophonic coding schemes and wire-tap channel ones
have different goals and belong to different coding classes,
the source coding and the error-correction ones, but they employ
the same underlying ideas of using randomness and dedicated
coding for achieving the desired goals.

For enhancing the security we exploit the underlying approaches of
universal homophonic coding \cite{massey-1994} and generic
wire-tap coding when the main channel is error-free (see
\cite{wyner} and \cite{thangaraj-IEEE-IT-2007}, for example).
Accordingly, we may say either
``homophonic coding'' or ``wire-tap channel coding'' to address
the dedicated coding that enhances security. The
main feature of the dedicated coding is that the
encoding is based on randomness and that the
legitimate receiving party who shares a secret key with the
corresponding transmitting one can perform decoding without
knowledge of the randomness employed for the encoding. For
simplicity of the terminology we mainly (but not always) say
``wire-tap channel coding'' to describe the dedicated coding which
provides the enhanced security.

Let  denote a wiretap or homophonic code encoder. To
enhance the security of the system considered, it is added at the
transmitter end (see Figure \ref{fig:figure-2}) involving a vector
of pure randomness

that is, each  is the realization of a random variable  with
distribution .
Note that  is invertible.
The wiretap encoding is done prior to error-correcting encoding, thus
out of the  bits of data to be sent,  are replaced by random data,
letting actually only  bits

of plaintext, to get as in (\ref{eq:y})

as codeword to be sent.
\begin{figure}\leavevmode
\begin{center}
\includegraphics[width=0.55\hsize]{model-enhanced.eps}
\caption{Communication system model enhanced with a wire-tap encoder.}
\label{fig:figure-2}
\end{center}
\end{figure}
As before, the receiver obtains

and starts with the decryption

He then first decodes

If the decoding is successful, he computes  using  and
let the transmitter know he could decode. Otherwise he informs the transmitter
than retransmission is required.

Similarly to a linear error-correction code where  can be
represented by multiplying the data vector by the generator matrix of the code,
we can write , following the so-called coset encoding proposed by Wyner
\cite{wyner}, as follows:
\begin {equation}\label{eq:coset}
C_H({\bf a}|| {\bf u})=
[{\bf a}|| {\bf u}]
\left[ \begin{array}{l}
{\bf h}_1\\
{\bf h}_2\\
\vdots \\
{\bf h}_l\\
{\bf G}^C
\end{array} \right]
=
 [{\bf a}|| {\bf u}] {\bf G}_H,

{\bf a} \mapsto a_1 {\bf h}_1 \oplus a_2 {\bf h}_2 \oplus \ldots
\oplus a_l {\bf h}_l \oplus C.

{\bf c} = a_1 {\bf h}_1 \oplus a_2 {\bf h}_2 \oplus \ldots \oplus a_l
{\bf h}_l \oplus u_1{\bf g}^C_1 \oplus u_2 {\bf g}^C_2 \oplus \ldots
\oplus u_{m-l} {\bf g}^C_{m-l}

C_{ECC}(C_H({\bf a}||{\bf u})),

C_H({\bf a}||{\bf u})=[\av||\uv]{\bf G}_H,

C_{ECC}(C_H({\bf a}||{\bf u}))
&=& C_{ECC}([{\bf a}||{\bf u}] {\bf G}_H) \nonumber \\
&=&[{\bf a}||{\bf u}] {\bf G}_H{\bf G}_{ECC} \nonumber \\
&=& [{\bf a}||{\bf u}] {\bf G} \label{eq:zfinal}
\label{RECURRMATR}
{\bf G}_H =
\left[
\begin{array}{cc}
{\bf G}_H^{(1)} & {\bf G}_H^{(2)} \\
{\bf I}_{m-l} & {\bf G}_H^{(4)}
\end{array}
\right]

{\bf G}_H =
\left[
\begin{array}{cc}
{\bf 0}_{l\times(m-l)} & {\bf I}_l \\
{\bf I}_{m-l} & {\bf G}_H^{(4)}
\end{array}
\right].

[\av||\uv]
\left[
\begin{array}{cc}
{\bf 0}_{l\times(m-l)} & {\bf I}_l \\
{\bf I}_{m-l} & {\bf G}_H^{(4)}
\end{array}
\right]=[\uv,\av+\uv{\bf G}_H^{(4)}],

{\bf G}_H =
\left[
\begin{array}{cc}
{\bf G}_H^{(1)} & {\bf G}_H^{(2)} \\
{\bf I}_2 & {\bf G}_H^{(4)}
\end{array}
\right] =
\left[
\begin{array}{cccc}
0 & 0 & 1 & 0 \\
0 & 0 & 0 & 1 \\
1 & 0 & 1 & 0 \\
0 & 1 & 0 & 1
\end{array}
\right].

(\mbox{plaintext, noisy ciphertext}) = (\av,\zv= C_{ECC}(\av)\oplus\xv\oplus\vv),

C_{ECC}(\av)\oplus\zv=\xv\oplus\vv.

{\bf z} &=&  C_{ECC}(C_H({\bf a}||{\bf u}))\oplus {\bf x}\oplus {\bf v}\\
        &=&  [{\bf a}||{\bf u}] {\bf G}\oplus {\bf x}\oplus {\bf v}

z_i=((\bigoplus_{k=1}^{\ell} g_{k,i} a_k )\oplus
(\bigoplus_{k=1}^{m-\ell} g_{\ell+k,i} u_k ) \oplus x_i )\oplus v_i,~i=1,2,...,n,

Z_i = (( \bigoplus_{k=1}^{\ell} g_{k,i} A_k ) \oplus
(\bigoplus_{k=1}^{m-\ell} g_{\ell+k,i} U_k ) \oplus X_i )\oplus V_i,~ i=1,2,...,n.
\label{eq:Z}
{\bf Z}^n = C_{ECC}(C_H({\bf A}^l||{\bf U}^{m-l}))\oplus{\bf X}^n
\oplus{\bf V}^n.

C_H(\am^l||\um^{m-l})
&=&[\am^l,\um^{m-l}]{\bf G}_H\\
&=&[\am^l,\um^{m-l}]
\left[
\begin{array}{cc}
{\bf G}_H^{(1)} & {\bf G}_H^{(2)} \\
{\bf I}_{m-l} & {\bf G}_H^{(4)}
\end{array}
\right]\\
&=&
[\am^l{\bf G}_H^{(1)}, \am^l{\bf G}_H^{(2)}]+
[\um^{m-l},\um^{m-l}{\bf G}_H^{(4)}],

C_H(\am^l||\um^{m-l})=C_{H,a}(\am^l)\oplus C_{H,u}(\um^{m-l}),

C_{H,a}(\am^l)=[\am^l{\bf G}_H^{(1)}, \am^l{\bf G}_H^{(2)}],~
C_{H,u}(\um^{m-l})=[\um^{m-l},\um^{m-l}{\bf G}_H^{(4)}].
\label{eq:Z2}
{\bf Z}^n = C_{ECC}(C_{H,a}({\bf A}^l))\oplus C_{ECC}(C_{H,u}({\bf U}^{m-l}))
\oplus{\bf X}^n \oplus {\bf V}^n.

&& H(\xm^n|\am^l,\zm^n)\geq \\
& & \min\{H(\um^{m-l}), H(\xm^n)+H(\vm^n)\} +\\
& & \min\{H(\vm^n),H(\xm^n) \}- \delta(C_{ECC}),

\delta(C_{ECC}) & = & H(\epsilon)+\epsilon \log(2^{m-l}-1) \\
                &\rightarrow & 0

& & H(\am^l,\um^{m-l},\xm^n,\vm^n,\zm^n) \\
&=& H(\am^l) + H(\zm^n|\am^l) + H(\um^{m-l}|\am^l,\zm^n)+\\
& & H(\vm^n|\am^l,\um^{m-l},\zm^n) + H(\xm^n|\am^l,\um^{m-l},\vm^n,\zm^n)\\
&=& H(\am^l) + H(\zm^n|\am^l) + H(\um^{m-l}|\am^l,\zm^n)\\
& &  + H(\vm^n|\am^l,\um^{m-l},\zm^n),

& & H(\am^l,\um^{m-l},\xm^n,\vm^n,\zm^n) \\
&=& H(\am^l) + H(\zm^n|\am^l) + H(\xm^n|\am^l,\zm^n)+\\
& & H(\um^{m-l}|\am^l,\xm^n,\zm^n) + H(\vm^n|\am^l,\um^{m-l},\xm^n,\zm^n)\\
&=& H(\am^l) + H(\zm^n|\am^l) + H(\xm^n|\am^l,\zm^n)\\
& &  + H(\um^{m-l}|\am^l,\xm^n,\zm^n),

&& H(\xm^n|\am^l,\zm^n)\\
&=& H(\um^{m-l}|\am^l,\zm^n) + H(\vm^n|\am^l,\um^{m-l},\zm^n) \\
&&  - H(\um^{m-l}|\am^l,\xm^n,\zm^n).

H(\um^{m-l}|\am^l,\zm^n)=H({\bf X}^n \oplus {\bf V}^n ).

H(\um^{m-l}|\am^l,\zm^n)\leq H(\um^{m-l}),

H(\um^{m-l}|\am^l,\zm^n) &=& \min \{H(\um^{m-l}), H(\xm^n\oplus\vm^n) \}\\
                        &=& \min \{H(\um^{m-l}), H(\xm^n) + H(\vm^n) \}

H(\vm^n|\am^l,\um^{m-l},\zm^n)\leq H(\vm^n),

H(\vm^n|\am^l,\um^{m-l},\zm^n) = \min \{H(\vm^n), H(\xm^n) \},

H(\um^{m-l}|\am^l,\xm^n,\zm^n)
&\leq &  H(P_e) + P_e \log(2^{m-l}-1)\\
&\leq &  H(\epsilon)+\epsilon \log(2^{m-l}-1) \rightarrow 0

H(\xm^n|\am^l,\zm^n)\geq H(\um^{m-l})+H(\vm^n)-\delta(C_{ECC}).

H(\xm^n|\am^l,\zm^n)\geq H(\vm^n)

z_i= x_i\oplus (\bigoplus_{k=1}^{m-\ell} g_{\ell+k,i} u_k )\oplus v_i,
~i=1,2,...,n.

H(\xm^n|\am^l,\zm^n) \geq   \min\{H(\um^{m-l}), H(\xm^n)\}.

& & H(\xm^n|\am^l,\zm^n) \\
&\geq & \min\{H(\um^{m-l}), H(\xm^n)\} +\\
&     & \min\{0,H(\xm^n) \}- \delta(C_{ECC})\\
& = & \min\{H(\um^{m-l}), H(\xm^n)\}.

\xm^{(t)}=\xm^{(t)}(\km)=f^{(t)}(\km),~t=1,\ldots,\tau.

{\bf Z}^{(t)} = C_{ECC} (C_{H,a}({\bf A}^{(t)}) \oplus
C_{H,u}({\bf U}^{(t)})) \oplus f^{(t)}({\bf K}) \oplus {\bf
V}^{(t)}.

{\bf A}^{\tau l}& = &[{\bf A}^{(1)}|| {\bf A}^{(2)}||\ldots ||{\bf A}^{(\tau)}]\\
{\bf U}^{\tau (m-l)}&=&[{\bf U}^{(1)}||{\bf U}^{(2)}||\ldots||{\bf U}^{(\tau)}]\\
{\bf V}^{\tau n}& = &[{\bf V}^{(1)}|| {\bf V}^{(2)}||\ldots|| {\bf V}^{(\tau)}]\\
{\bf Z}^{\tau n}& = & [{\bf Z}^{(1)}|| {\bf Z}^{(2)}|| ... || {\bf Z}^{(\tau)}].

H(\km|\am^{\tau l},\zm^{\tau n}) \;\;
\left\{ \begin{tabular}{lll}
 & {\rm for} &  \\
   & {\rm for} & \\
\end{tabular}\right.

&&H(\km|\am^{\tau l},\zm^{\tau n})\nonumber \\
&=& H(\um^{\tau(m-l)}|\am^{\tau l},\zm^{\tau n}) +
   H(\vm^{\tau n}|\am^{\tau l},\um^{\tau(m-l)},\zm^{\tau n})\nonumber \\
&& -H(\um^{\tau(m-l)}|\am^{\tau l},\km,\zm^{\tau n}).
\label{eq:HKAZ}

H(\um^{\tau(m-l)}|\am^{\tau l},\zm^{\tau n})  <
H(\um^{\tau(m-l)}, \km |\am^{\tau l},\zm^{\tau n})

\leq H(P_e^*) + P_e^* \log(2^{\tau(m-\ell) + |{\bf K}|}-1) \;.

H(\vm^{\tau n}|\am^{\tau l}\!,\um^{\tau(m-l)}\!,\zm^{\tau n}) <
H(\vm^{\tau n}\!, \km |\am^{\tau l}\!,\um^{\tau(m-l)}\!,\zm^{\tau n}),

{\bf U}^{\tau (m-l)}=[{\bf U}^{(1)}||{\bf U}^{(2)}||\ldots||{\bf
U}^{(\tau)}],

H(\um^{\tau(m-l)}|\am^{\tau l},\km,\zm^{\tau n})
\!\!\!&\leq\!\!\! &H(P_e^\tau)+P_e^\tau\log(2^{\tau(m-l)}-1)\\
\!\!\!&\leq\!\!\! &H(\epsilon^\tau)+\epsilon^\tau\log(2^{\tau(m-1)}-1)
\label{eq:F}
H(\um^{\tau(m-l)}|\am^{\tau l},\km,\zm^{\tau n}) \rightarrow  0

H(\km |\am^{\tau l},\zm^{\tau n}) = 0 \;. \\

{\bf y} = C_{ECC}(C_H({\bf a}||{\bf u}))\oplus {\bf x}

\zv=\yv \oplus \vv.

{\bf z} & =&  {\bf y} \oplus {\bf v} \oplus {\bf v}^* \nonumber \\
        & =&  C_{ECC}(C_H({\bf a}||{\bf u}))\oplus {\bf x}\oplus
              {\bf v} \oplus {\bf v}^*. \label{eq:zactive}

\zv\oplus\xv = C_{ECC}(C_H({\bf a}||{\bf u}))\oplus {\bf v} \oplus{\bf v}^*

C_{ECC}^{-1}({\bf z} \oplus {\bf x})
&=& C_{ECC}^{-1}(C_{ECC}(C_H({\bf a}||{\bf u}))\oplus {\bf v} \oplus {\bf v}^* )\\
&=& C_H({\bf a}||{\bf u})

{\bf a} = C_H^{-1}(C_{ECC}^{-1}({\bf z} \oplus {\bf x})).
\label{eq:Z2act}
{\bf Z}^n = C_{ECC}(C_{H,a}({\bf A}^l))\oplus C_{ECC}(C_{H,u}({\bf
U}^{m-l})) \oplus{\bf X}^n \oplus {\bf V}^n\oplus{{\bf V}^*}^n.

&& H(\xm^n|\am^l,\zm^n, F_d)\geq \\
& & \min \{H(\um^{m-l}), H(\xm^n)+ H(\vm^n|F_d)\}+
\min \{H(\vm^n|F_d), H(\xm^n) \} - \delta(C_{ECC}),

\delta(C_{ECC})  =  H(P_e)+P_e \log(2^{m-l}-1)\rightarrow 0,

& & H(\am^l,\um^{m-l},\xm^n,{\vm'}^n,\zm^n, F_d) \\
&=& H(\am^l) + H(\zm^n|\am^l) + H(\um^{m-l}|\am^l,\zm^n)+\\
& & H(F_d|\am^l,\um^{m-l},\zm^n) + H({\vm'}^n|\am^l,\um^{m-l},\zm^n, F_d) + \\
& & H(\xm^n|\am^l,\um^{m-l},{\vm'}^n,\zm^n, F_d)\\
&=& H(\am^l) + H(\zm^n|\am^l) + H(\um^{m-l}|\am^l,\zm^n)\\
& & + H({\vm'}^n|\am^l,\um^{m-l},\zm^n, F_d),

& & H(\am^l,\um^{m-l},\xm^n,{\vm'}^n,\zm^n, F_d) \\
&=& H(\am^l) + H(\zm^n|\am^l) + H(\xm^n|\am^l,\zm^n)+\\
& & H(F_d|\am^l,\xm^n,\zm^n) + H(\um^{m-l}|\am^l,\xm^n,\zm^n, F_d) + \\
& & H({\vm'}^n|\am^l,\um^{m-l},\xm^n,\zm^n, F_d)\\
&=& H(\am^l) + H(\zm^n|\am^l) + H(\xm^n|\am^l,\zm^n)\\
& & + H(\um^{m-l}|\am^l,\xm^n,\zm^n, F_d),
\label{eq:hxaz}
H(\xm^n|\am^l,\zm^n)
= H(\um^{m-l}|\am^l,\zm^n,F_d) + H({\vm'}^n|\am^l,\um^{m-l},\zm^n, F_d)
 - H(\um^{m-l}|\am^l,\xm^n,\zm^n, F_d),

H(\um^{m-l}|\am^l,\zm^n, F_d) =  \min \{H(\um^{m-l}), H(\xm^n\oplus{\vm'}^n|F_d) \}

H(\um^{m-l}|\am^l,\zm^n, F_d) =H(\xm^n\oplus{\vm'}^n|F_d)

H(\um^{m-l}|\am^l,\zm^n, F_d)\leq H(\um^{m-1}).

H({\vm'}^n|\am^l,\um^{m-l},\zm^n, F_d)\leq H({\vm'}^n|F_d),

H({\vm'}^n|\am^l,\um^{m-l},\zm^n, F_d) =\min \{H({\vm'}^n|F_d), H(\xm^n) \}.

&&H(\xm^n|\am^l,\zm^n)\\
&=&\min \{H(\um^{m-l}), H(\xm^n\oplus{\vm'}^n|F_d) \}+
    \min \{H({\vm'}^n|F_d), H(\xm^n) \}\\
&&-H(\um^{m-l}|\am^l,\xm^n,\zm^n,F_d)

&&H(\xm^n|\am^l,\zm^n)\\
&=& \min \{H(\um^{m-l}), H(\xm^n)+ H(\vm^n|F_d)\}+
\min \{H(\vm^n|F_d), H(\xm^n) \}\\
&&-H(\um^{m-l}|\am^l,\xm^n,\zm^n,F_d)

H(\um^{m-l}|\am^l,\xm^n,\zm^n, F_d)
&=& H(\um^{m-l}|\am^l,\xm^n,\zm^n)\\
&\leq &  H(P_e) + P_e \log(2^{m-l}-1)

H(\xm^n|\am^l,\zm^n)&\geq & \min\{H(\um^{m-l}), H(\xm^n)+H(\vm^n)\}+ \min\{H(\vm^n),H(\xm^n) \}- \delta(C_{ECC}),\\
H(\xm^n|\am^l,\zm^n, F_d) &\geq & \min \{H(\um^{m-l}), H(\xm^n)+ H(\vm^n|F_d)\}+\min \{H(\vm^n|F_d), H(\xm^n) \}-\delta(C_{ECC}),

\delta(C_{ECC})  =  H(P_e)+P_e \log(2^{m-l}-1)\rightarrow 0,

H(\km|\am^{\tau l},\zm^{\tau n},F_d) \;\;
\left\{ \begin{tabular}{lll}
 & {\rm for} &  \\
   & {\rm for} & \\
\end{tabular}\right.

&&H(\km|\am^{\tau l},\zm^{\tau n},F_d)\nonumber \\
&=& H(\um^{\tau(m-l)}|\am^{\tau l},\zm^{\tau n},F_d) +
   H(\vm^{\tau n}|\am^{\tau l},\um^{\tau(m-l)},\zm^{\tau n},F_d)\nonumber \\
&& -H(\um^{\tau(m-l)}|\am^{\tau l},\km,\zm^{\tau n},F_d).

Now it is shown in the proof of Theorem \ref{theor:pass} that every term tends to zero,
using a decoding argument, which will hold similarly here, since the knowledge of 
cannot make the decoding more difficult.
\end{proof}



\section{Practical Implications and Applications Issues}
\label{sec:appli}

This section provides a generic discussion of the usefulness and
possible applications of the proposed approach.

\subsection{Implications of the security evaluation}

The analysis given in Sections \ref{sec:passive} and \ref{sec:active} shows
that in systems where the encoding-encryption paradigm is employed,
involvement of pure randomness via concatenation of dedicated wire-tap
and error-correction coding (instead of error-correction only)
provides an increased cryptographic security, by combining pseudo-randomness,
randomness and coding, which in a known-plaintext cryptanalytic scenario
implies an increased resistance against threats on the secret key.

The performed information-security evaluation more precisely points out
the following desirable security properties of the proposed approach:
(i) When the sample available for cryptanalysis is below a
certain size, the scheme provides uncertainty about the secret
key; (ii) Complexity of the secret key recovery appears as a
highly computationally complex problem even if the
available sample is such that the posterior uncertainty about the
secret key tends to zero. The main consequence of (i) is that even if
exhaustive search were to be employed for the secret key recovery,
a (large) number of candidates will appear. The statement
(ii) is an implication of the proofs of Theorems 1 and 2, where
the reduction to zero of the posterior uncertainty about the
secret key appears assuming employment of a decoding which has
complexity proportional to the exhaustive search over all possible
secret keys. Accordingly, the uncertainty tends to zero at the
expense of a decoding with exponential complexity. Actually, the
decrease of the uncertainty about the secret key with the increase
of the sample available for cryptanalysis appears as a consequence
of decoding capabilities of a low rate random binary block codes,
but at the expense of the decoding complexity which is exponential
in the secret key length.

The above features (i) and (ii) hold not only in a passive
attacking scenario where the attacker performs cryptanalysis based
on recording the ciphertext from a public communication channels,
but also in certain active attacking scenarios where the attacker
can modify the ciphertext and learn the effects of these
modifications.

\subsection{Framework for applications}

The encoding-encryption paradigm for secure and reliable
communications enjoys the following desirable properties: (i) When
the decryption is performed by bitwise XORing the keystream to the
ciphertext, an error in a bit before decryption causes an error in
the corresponding bit after decryption, without any
error-propagation, and (ii) Provides non-availability of the
error-free keystream when the communication channel is a noisy
one.

The proposed approach for enhancing the security of the
communications systems which follow encoding-encryption paradigm
could be employed in the design of these systems from scratch as
well as in upgrading of the existing ones.

In the case of upgrading the existing systems, the implementation
assumption is that the employed, already existing, binary linear
block error-correction code  which encodes  bits into a
codeword from , could be replaced with a binary block
code  with the same error correction capability but with
. Accordingly,  random bits can be concatenated with
 information bits and mapped into the new -bits via a
homophonic encoder. The obtained output from homophonic encoder is
the input for the error-correcting one. Taking into account the
notation from Section 2, the previous means that instead of
performing  which is a linear mapping
, the following should be
performed:  where  is a concatenation of an -dimensional vector and an
-dimensional one,  is a linear mapping
 and  is a
linear mapping . On the
receiving side, the decoding procedures after decryption are
straightforward (see Fig. 2): The error correction decoding
removes the random errors, and the message  is obtained
by truncating of the inverse linear mapping corresponding to the
homophonic decoding.

In the case of a design of the encoding-encryption system from
the scratch, the design should include a coding box which performs
the concatenation of homophonic and error-correction coding in a
manner which fits the rate of the concatenated code to the given
constraints.

Note that from an implementation point of view, replacement of a
linear block encoding by a concatenation of a block linear
homophonic and error correction encoding is a replacement of one
binary matrix with another binary matrix which is the product of
the matrices corresponding to the homophonic and error-correction
encoders. Accordingly, the implementation complexity of two
concatenated codes could be approximately the same as the
implementation complexity of an error correcting code only.



\section{Conclusion}
\label{sec:conclusion}

The problem addressed in this paper is the one of enhancing
security of certain communications systems which employ error
correction encoding of the messages and encryption of the obtained
codewords in order to provide both secrecy and reliability of
the transmission. This paper yields a proposal for providing the
enhanced security of the considered systems employing randomness
and dedicated coding and the information-theoretic security
evaluation of the proposed approach. The analysis given in this
paper implies that in the systems where the encoding-encryption
paradigm is employed, the cryptographic security can
be enhanced via involvement of a homophonic coding based on pure
randomness as follows: Instead of just error-correction encoding
before the encryption, this paper proposes employment of a
concatenation of linear block homophonic and error-correction
encoding. The proposal and its cryptographic security evaluation
are given in a generic manner and accordingly yield a generic
framework for particular applications.

Note that the information-theoretic consideration of the
cryptographic security yields a basic evaluation of the related
cryptographic features. On the other hand, the
information-theoretic security evaluation also provides
specification of the settings when it is possible to perform the
secret key recovery, but it does not specify and only indicates
the expected complexity of this problem. We show that with the aid
of a dedicated wire-tap encoder, the amount of uncertainty that
the adversary has about the key given all the information he could
gather during different passive or active attacks he can mount, is
a decreasing function of the sample available for cryptanalysis.
This means that the wire-tap encoder can indeed provide an
information theoretical security level over a period of time,
after which a large enough sample is collected and the function
tends to zero, entering a regime in which a computational security
analysis is needed.

An interesting issue for further work is the characterization of
the transition region in which the uncertainty drops from a
certain value to close to zero. Also, because after all, the
uncertainty tends to zero, the computational complexity based
evaluation of cryptographic security is a direction for a future
work.



\section*{Acknowledgments}

The research of F. Oggier is supported in part by the Singapore National
Research Foundation under Research Grant NRF-RF2009-07 and
NRF-CRP2-2007-03, and in part by the Nanyang Technological University
under Research Grant M58110049 and M58110070.
This work was done partly while M. Mihaljevi\'c was visiting the division of
mathematical sciences, Nanyang Technological University, Singapore, and
partly while F. Oggier was visiting the Research Center for Information
Security, Tokyo.




\begin{thebibliography}{99}
\bibitem{GSM-encryption}
GSM Technical Specifications: European Telecommunications
Standards Institute (ETSI), {\em Digital cellular
telecommunications system (Phase 2+); Physical layer on the radio
path; General description}, TS 100 573 (GSM 05.01),
http://www.etsi.org.
\bibitem{GSM-coding}
GSM Technical Specifications: European Telecommunications
Standards Institute (ETSI), {\em Digital cellular
telecommunications system (Phase 2+); Channel Coding}, TS 100 909
(GSM 05.03), http://www.etsi.org.
\bibitem{fossorier-IEEE-IT-2007} M. Fossorier, M.J. Mihaljevi\'{c} and H. Imai,
``Modeling Block Encoding Approaches for Fast Correlation
Attack'', {\em IEEE Transactions on Information Theory}, vol. 53,
no. 12, pp. 4728-4737, Dec. 2007.
\bibitem{gilbert}
H. Gilbert, M. Robshaw and H. Sibert, ``An Active Attack against
HB - a Provably Secure Lightweight Authentication Protocol'',
{\em IEE Electronics Letters}, vol. 41, no. 21, pp. 1169-1170,
2005.
\bibitem{gilbert-FC2008} H. Gilbert, M.J.B. Robshaw and Y Seurin,
``Good Variants of HB are Hard to Find'', Financial
Cryptography and Data Security 2008, {\em Lecture Notes in
Computer Science}, vol. 5143, pp. 156-170, 2008.
\bibitem{gilbert-EUROCRYPT2008}
H. Gilbert, M.J.B. Robshaw and Y. Seurin, "HB: Increasing
the Security and Efficiency of HB", EUROCRYPT2008, {\em
Lecture Notes in Computer Science}, vol. 4965, pp. 361-378, 2008.
\bibitem{hellman}
M.E. Hellman, ``A cryptanalytic time-memory trade-off'', {\em IEEE
Transactions on Information Theory}, vol. 26, pp. 401-406, July
1980.
\bibitem{hopper-ASIACRYPT2001}
N. Hopper and M. Blum, ``Secure Human Identification Protocols'',
ASIACRYPT 2001, {\em Lecture Notes in Computer Science}, vol.
2248, pp. 52-66, 2001.
\bibitem{juels-CRYPTO2005}
A. Juels and S. Weis, ``Authenticating Pervasive Devices with
Human Protocols'',  CRYPTO2005, {\em Lecture Notes in Computer
Science}, vol. 3621, pp. 293-308, 2005.
\bibitem{katz-EUROCRYPT2006}
J. Katz and J.S. Shin, ``Parallel and Concurrent Security of the
HB and HB Protocols'', EUROCRYPT2006, {\em Lecture Notes in
Computer Science}, vol. 4004, pp. 73-87, 2006.
\bibitem{jendal}
H.N. Jendal, Y.J.B. Kuhn, and J.L. Massey, ``An
information-theoretic treatment of homophonic substitution'',
EUROCRYPT'89,  {\em Lecture Notes in Computer Science}, vol. 434,
pp. 382-394, 1990.
\bibitem{massey-1994}
J. Massey, ``Some Applications of Source Coding in Cryptography'',
{\em European Transactions on Telecommunications}, vol. 5, pp.
421-429, July-August 1994.
\bibitem{mihaljevic-IEEE-CL-2007} M.J. Mihaljevi\'{c}, M. Fossorier and
H. Imai, ``Security Evaluation of Certain Broadcast Encryption Schemes
Employing a Generalized Time-Memory-Data Trade-Off'', {\em IEEE Communications
Letters}, vol. 11, no. 12, pp. 988-990, Dec. 2007.
\bibitem{mihaljevic-Computimg2009}
M.J. Mihaljevi\'{c} and H. Imai, ``An approach for stream ciphers
design based on joint computing over random and secret data'',
{\em Computing}, vol. 85, no. 1-2, pp. 153-168, June 2009. (DOI:
10.1007/s00607-009-0035-x)
\bibitem{mihaljevic-IOSpress2009}
M.J. Mihaljevi\'{c}, "A Framework for Stream Ciphers Based on
Pseudorandomness, Randomness and Error-Correcting Coding", in
{\em Enhancing Cryptographic Primitives with Techniques from Error
Correcting Codes}, B. Preneel, at al Eds.,  Vol. 23 in the Series
Information and Communication Security, pp. 117-139,
IOS Press, Amsterdam, The Netherlands, June 2009. DOI:
10.3233/978-1-60750-002-5-117 (ISSN: 1874-6268; ISBN:
978-1-60750-002-5)
\bibitem{misa&frederique}
M. Mihaljevi\'{c} and F. Oggier, "A Wire-tap Approach to Enhance
Security in Communication Systems using the Encoding-Encryption
Paradigm", {\em IEEE ICT 2010 - Int. Comm. Conf.}, Proceedings,
pp. 484-489, April 2010.
\bibitem{ryabko}
B. Ryabko and A. Fionov, ``Efficient Homophonic Coding'', {\em
IEEE Transactions on Information Theory}, vol. 45, no. 6, pp.
2083-2094, Sept. 1999.
\bibitem{shannon}
C.E. Shannon, ``Communication theory of secrecy systems'', {\em
Bell Systems Technical Journal}, vol. 28, pp. 656-715, Oct. 1949.
\bibitem{thangaraj-IEEE-IT-2007}
A. Thangaraj, S. Dihidar, A.R. Calderbank, S.W. McLaughlin, and
J.-M. Merolla, "Applications of LDPC Codes to the Wiretap
Channel", {\em IEEE Trans. Information Theory}, vol. 53, no. 8,
pp. 2933-2945, August 2007 .
\bibitem{wyner} A.D. Wyner, ``The wire-tap channel'',
{\em Bell Systems Technical Journal}, vol. 54, pp. 1355-1387, Oct.
1975.
\end{thebibliography}
\end{document}
