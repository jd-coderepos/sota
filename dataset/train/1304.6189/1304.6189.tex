\documentclass[a4paper,11pt]{article}
\usepackage[utf8x,utf8]{inputenc} 
\usepackage[T1]{fontenc}
\usepackage{lmodern}
\usepackage[vmargin=33mm,hmargin=37mm]{geometry}

\usepackage{url,graphicx,color,microtype,enumitem,hyperref}
\usepackage{amsfonts,amssymb,amsmath,euscript,amsthm}

\hypersetup{
	bookmarks,
	pdftex,
    pdfauthor={Fedor V. Fomin, Petr A. Golovach and Janne H. Korhonen},
    pdftitle={On the parameterized complexity of cutting a few vertices from a graph},
	colorlinks=true,
	linkcolor=black,
	citecolor=black,
	filecolor=black,
	urlcolor=black,
}

\setitemize{noitemsep}
\setenumerate{noitemsep,itemsep=1ex}


\newtheorem{theorem}{Theorem}
\newtheorem{lemma}[theorem]{Lemma}
\theoremstyle{definition}
\newtheorem{definition}[theorem]{Definition}
\theoremstyle{remark}
\newtheorem{remark}[theorem]{Remark}


\newcommand{\N}{\mathbb{N}}
\newcommand{\Z}{\mathbb{Z}}
\newcommand{\X}{\EuScript{X}}
\newcommand{\Zz}{\EuScript{Z}}
\newcommand{\Y}{\EuScript{Y}}

\newcommand{\card}[1]{\left\lvert {#1} \right\rvert}

\DeclareMathOperator{\operatorClassP}{P}
\newcommand{\classP}{\ensuremath{\operatorClassP}}
\DeclareMathOperator{\operatorClassNP}{NP}
\newcommand{\classNP}{\ensuremath{\operatorClassNP}}
\DeclareMathOperator{\operatorClassFPT}{FPT}
\newcommand{\classFPT}{\ensuremath{\operatorClassFPT}}
\DeclareMathOperator{\operatorClassW}{W}
\newcommand{\classW}[1]{\ensuremath{\operatorClassW[#1]}}
\DeclareMathOperator{\operatorClassXP}{XP}
\newcommand{\classXP}{\ensuremath{\operatorClassXP}}

\newcommand{\vbound}[1]{\card{N(#1)}}
\newcommand{\vboundG}[2]{\card{N_{#1}(#2)}}
\newcommand{\ebound}[1]{\card{\partial(#1)}}

\newcommand{\defparproblem}[5]{
  \vspace{1mm}
\noindent\fbox{
  \begin{minipage}{0.97\textwidth}
  \begin{tabular*}{\textwidth}{@{\extracolsep{\fill}}lr} #1  \\ \end{tabular*}
  {\bf{Input:}} #2  \\
    {\bf{Parameter 1:}} #3   \\
    {\bf{Parameter 2:}} #4\\
  {\bf{Question:}} #5
  \end{minipage}
  }
  \vspace{1mm}
}

\newenvironment{myabstract}
               {\list{}{\listparindent 1.5em\itemindent    \listparindent
                        \leftmargin    0pt
                        \rightmargin   0pt
                        \parsep        0pt}\item\relax}
               {\endlist}

\newenvironment{mycover}
               {\list{}{\listparindent 0pt
                        \itemindent    \listparindent
                        \leftmargin    0pt
                        \rightmargin   0pt
                        \parsep        0pt}\raggedright
                \item\relax}
               {\endlist}


\pagestyle{plain}

\begin{document}

\vspace*{2ex}
\begin{mycover}
{\LARGE \textbf{On the parameterized complexity of cutting a few vertices from a graph\footnote{This work is supported by the European Research Council (ERC) via grant Rigorous Theory of Preprocessing, reference 267959 (F.F., P.G.) and by the Helsinki Doctoral Programme in Computer Science -- Advanced Computing and Intelligent Systems (J.K.).}\footnote{This work has been published as a part of the proceedings of the 38th International Symposium on Mathematical Foundations of Computer Science ({MFCS} 2013) \cite{small-cuts-conf}. The final publication is available at \url{http://link.springer.com/chapter/10.1007\%2F978-3-642-40313-2_38}.}}\par}


\bigskip
\bigskip

\medskip
\textbf{Fedor V.\ Fomin}\\
{\small Department of Informatics, University of Bergen, Norway\\
\nolinkurl{fedor.fomin@ii.uib.no}\par}

\medskip
\textbf{Petr A.\ Golovach}\\
{\small Department of Informatics, University of Bergen, Norway\\
\nolinkurl{petr.golovach@ii.uib.no}\par}

\medskip
\textbf{Janne H.\ Korhonen}\\
{\small Helsinki Institute for Information Technology HIIT, \\
Department of Computer Science, University of Helsinki, Finland\\
\nolinkurl{janne.h.korhonen@cs.helsinki.fi}\par}
\end{mycover}

\bigskip
\bigskip

\begin{myabstract}
\noindent\textbf{Abstract.} We study the parameterized complexity of separating a small set of vertices from a graph by a small vertex-separator. That is, given a graph  and integers , , the task is to find a vertex set  with  and . We show that
\begin{itemize}
	\item the problem is fixed-parameter tractable (FPT) when parameterized by  but W[1]-hard when parameterized by , and
	\item a terminal variant of the problem, where  must contain a given vertex , is W[1]-hard when parameterized either by  or by  alone, but is FPT when parameterized by .  
\end{itemize}
We also show that if we consider edge cuts instead of vertex cuts, the terminal variant is NP-hard.
\end{myabstract}

\thispagestyle{empty}
\setcounter{page}{0}
\newpage








\section{Introduction}

We investigate two related problems that concern separating a small vertex set from a graph . Specifically, we consider finding a vertex set  of size at most  such that
\begin{enumerate}
	\item  is separated from the rest of  by a small cut (e.g. \emph{finding communities in a social network}, cf.~\cite{li2012}), or
	\item  is separated from the rest of  by a small cut and contains a specified terminal vertex  (e.g. \emph{isolating a dangerous node}, cf.~\cite{galbiati2011,hayrapetyan2005}).
\end{enumerate}
We focus on \emph{parameterized complexity} of the \emph{vertex-cut versions} of these problems.

\paragraph{Parameterized vertex cuts.} Our interest in the vertex-cut version stems from the following parameterized separation problem, studied by Marx \cite{marx2006parameterized}. Let  denote the vertex-neighborhood of .

\medskip 

\defparproblem{\textsc{Cutting  Vertices}}{Graph , 
integers , 
}{}{}{Is there a set  such that  and ?}

\medskip

In particular, Marx showed that \textsc{Cutting  Vertices} is \classW{1}-hard even when parameterized by both  and . We contrast this result by investigating the parameterized complexity of the two related separation problems with relaxed requirement on the size of the separated set .

\medskip

\defparproblem{\textsc{Cutting at Most   Vertices}}{Graph , 
integers , }{}{}{Is there a non-empty set  such that  and ?}

\medskip

\defparproblem{\textsc{Cutting at Most   Vertices with Terminal}}{Graph , terminal vertex , 
integers , 
}{}{}{Is there a non-empty set  such that ,  and ?}

\medskip

We show that these closely related problems exhibit quite different complexity behaviors. In particular, we show that \textsc{Cutting at Most   Vertices} is fixed-parameter tractable (\classFPT) when parameterized by the size of the separator , while we need both  and  as parameters to obtain an \classFPT{} algorithm for \textsc{Cutting at Most   Vertices with Terminal}. A full summary of the parameterized complexity of these problems and our results is given in Table \ref{tabl:compl}. 

\begin{table}[b]
\begin{center}
\begin{tabular}{c|c|c|c}
Parameter & \textsc{Cutting } & \textsc{Cutting  } & \textsc{Cutting     Vertices}  \\
 &  \textsc{Vertices}  & \textsc{Vertices} & \textsc{with Terminal} \\
\hline
 &  \classW{1}-hard,   \cite{marx2006parameterized} &  \classW{1}-hard,  Thm~\ref{thm:svc_is_hard}&    \classW{1}-hard, Thm~\ref{thm:svc_is_hard} \\
\hline
 & \classW{1}-hard,    \cite{marx2006parameterized} & \classFPT,  Thm~\ref{thm:svc_fpt}  &    \classW{1}-hard,  Thm~\ref{thm:svc_terminal_hard}  \\
\hline
  and  & \classW{1}-hard,    \cite{marx2006parameterized}  & \classFPT, Thm~\ref{thm:svc_fpt_kt}  &  \classFPT,     Thm~\ref{thm:svc_fpt_kt} \\
\end{tabular}
\vspace{2mm}
\caption{Parameterized complexity of  \textsc{Cutting  Vertices}, \textsc{Cutting at Most   Vertices},  and \textsc{Cutting at Most   Vertices with Terminal}.}\label{tabl:compl}
\end{center}
\end{table}

The main algorithmic contribution of our paper is the proof that  \textsc{Cutting at most   vertices}  is \classFPT{} when parameterized by  (Theorem \ref{thm:svc_fpt}). To obtain this result, we utilize the concept of \emph{important separators} introduced by Marx \cite{marx2006parameterized}. However, a direct application of important separators---guess a vertex contained in the separated set, and find a minimal set containing this vertex that can be separated from the remaining graph by at most  vertices---does not work. Indeed, pursuing this approach would bring us to essentially solving  \textsc{Cutting at most   vertices with terminal}, which is
\classW{1}-hard when parameterized by . 
Our \classFPT{} algorithm is based on new  structural results about unique important separators of minimum size separating pairs of vertices.   
We also observe that it is unlikely that \textsc{Cutting at most   vertices} has a polynomial kernel.

\paragraph{Edge cuts.} Although our main focus is on vertex cuts, we will also make some remarks on the edge-cut versions of the problems. In particular, the edge-cut versions again exhibit a different kind of complexity behavior. Let  denote the edge-boundary of .

\medskip

\defparproblem{\textsc{Cutting at Most   Vertices by Edge-Cut}}{Graph , 
integers , }{}{}{Is there a non-empty set  such that  and ?}
\medskip

\defparproblem{\textsc{Cutting  Vertices by Edge-Cut with Terminal}}
{Graph , terminal vertex , 
integers , }
{}{}
{Is there a set  such that ,  and ?}

\medskip

Results by Watanabe and Nakamura~\cite{WatanabeN87} imply that \textsc{Cutting at most   vertices by edge-cut} can be done in polynomial time even when  and  are part of the input; more recently, Armon and Zwick \cite{armon2006} have shown that this also holds in the edge-weighted case. Lokshtanov and Marx~\cite{Lokshtanov2013278} have proven that \textsc{Cutting at most  vertices by edge-cut with terminal} is fixed-parameter tractable when parameterized by  or by ; see also \cite{yixin}. We complete the picture by showing that \textsc{Cutting at most   vertices by edge-cut with terminal} is \classNP-hard (Theorem \ref{thm:sec_terminal_np}). The color-coding techniques we employ in Theorem~\ref{thm:svc_fpt_kt} also give a simple algorithm with running time .

Related edge-cut problems have received attention in the context of approximation algorithms. In contrast to \textsc{Cutting at most   vertices by edge-cut}, finding a minimum-weight edge-cut that separates exactly  vertices is NP-hard. Feige et al. \cite{Feige2003643} give a PTAS for  and an -approximation for ; Li and Zhang \cite{li2012} give an -approximation. Approximation algorithms have also been given for unbalanced --cuts, where  and  are specified terminal vertices and the task is to find an edge cut  with  and  such that (a)  and weight of the cut is minimized \cite{galbiati2011,li2012}, or (b) weight of the cut is at most  and  is minimized \cite{hayrapetyan2005}.





\section{Basic definitions and preliminaries}\label{sec:defs}

\paragraph{Graph theory.} We follow the conventions of Diestel~\cite{Diestel10} with graph-theoretic notations. 
We only consider finite, undirected graphs that do not contain loops or multiple
edges.
The vertex set of a graph  is denoted by  and  
the edge set is denoted by , or simply by  and , respectively.
Typically we use  to denote the number of vertices of  and  the number of edges.

For a set of vertices , we write
 for the subgraph of  induced by ,
and  for the graph obtained form  by the removal of all the vertices of , i.e., the subgraph of  induced by .
Similarly, for a set of edges , the graph obtained from  by the removal of all the edges in  is denoted by .

For a vertex , we denote by  its
(\emph{open}) \emph{neighborhood}, that is, the set of vertices which are
adjacent to . The \emph{degree} of a vertex  is .
For a set of vertices , we write  and
. 
We may omit subscripts in these notations if there is no danger of ambiguity.

\paragraph{Submodularity.} We will make use of the well-known fact that given a graph , the mapping  defined by  is \emph{submodular}. That is, for  we have


\paragraph{Important separators.} Let  be a graph. For disjoint sets , a vertex set  is a (\emph{vertex}) \emph{-separator} if there is no path from  to  in . An \emph{edge -separator}  is defined analogously. 
Note that we do not allow deletion of vertices in  and , and thus there are no vertex -separators if  and  are adjacent.
As our main focus is on the vertex-cut problems, all separators are henceforth vertex-separators unless otherwise specified.

We will make use of the concept of \emph{important separators}, introduced by Marx~\cite{marx2006parameterized}.
A vertex  is \emph{reachable} from a set  if  has a path that joins a vertex of  and . 
For any sets  and , we denote the set of vertices reachable from  in  by .
An -separator  is \emph{minimal} if no proper subset of  is an -separator. For -separators  and , we say that  \emph{dominates}  if  and  is a proper subset of .
For singleton sets, we will write  instead of  in the notations defined above.

\begin{definition}[\cite{marx2006parameterized}]
An -separator  is \emph{important} if it is minimal and there is no other -separator dominating .
\end{definition}

In particular, this definition implies that for any -separator  there exists an important -separator  with  and . If  is not important, then at least one of the aforementioned relations is proper.

The algorithmic usefulness of important separators follows from the fact that the number of important separators of size at most  is bounded by  alone, and furthermore, these separators can be listed efficiently. Moreover, minimum-size important separators are unique and can be found in polynomial time. That is, we will make use of the following lemmas.

\begin{lemma}[\cite{chen2009improved}]\label{lemma:computing_imp_seps}
For any disjoint sets , the number of important -separators of size at most  is at most , and all important -separators of size at most  can be listed in time .
\end{lemma}

\begin{lemma}[\cite{marx2006parameterized}]\label{lemma:unique_min_sep}
For any sets , if there exists an -separator, then there is exactly one important -separator of minimum size. This separator can be found in polynomial time.
\end{lemma}

\paragraph{Parameterized complexity.} We will briefly review the basic notions of parameterized complexity, though we refer to the books of Downey and Fellows~\cite{DowneyF99}, Flum and Grohe~\cite{FlumG06}, and   Niedermeier~\cite{Niedermeierbook06} for a detailed introduction. 
Parameterized complexity is a two-dimensional framework for studying the computational complexity of a problem; one dimension is the input size
 and another one is a parameter . A parameterized problem is \emph{fixed-parameter tractable} (or \classFPT) if it can be solved in time  for some function , and in the class \classXP{} if it can be solved in time  for some function .

Between \classFPT{} and \classXP{} lies the class \classW{1}. One of basic assumptions of the parameterized complexity theory is the conjecture that , and it is thus held to be unlikely that a \classW{1}-hard problem would be in \classFPT{}. For exact definition of \classW{1}, 
we refer to the books mentioned above.  We mention only that {\sc Indpendent Set} and {\sc Clique}
parameterized by solution size are two fundamental problems that are known to be \classW{1}-complete.

The basic way of showing that a parameterized problem is unlikely to be fixed-parameter tractable is to prove \classW{1}-hardness. To show that a problem is \classW{1}-hard, it is enough to give a \emph{parameterized reduction} from a known \classW{1}-hard problem.
That is, let  be parameterized problems. We say that  is (uniformly many-one) {\em \classFPT{}-reducible} to  if there exist functions
, a constant  and
 an algorithm  that transforms an instance  of  into an instance  of 
in time  so that  if and only if .


\paragraph{Cutting problems with parameters  and .} In the remainder of this section, we consider \textsc{Cutting at most   vertices}, \textsc{Cutting at most  vertices with terminal}, and \textsc{Cutting at most   vertices by edge-cut with terminal} with parameters  and . We first note that if there exists a solution for one of the problems, then
there is also a solution in which  is connected; 
indeed, it suffices to take any maximal connected . Furthermore, we note finding a connected set  with  and  is fixed-parameter tractable with parameters  and  due to a result by Marx~\cite[Theorem 13]{marx2006parameterized}, and thus 
\textsc{Cutting at most  vertices} is also fixed-parameter tractable with parameters  and .

We now give a simple color-coding algorithm~\cite{AlonYZ95,cai2006random} for the three problems with parameters  and , in particular improving upon the running time of the aforementioned algorithm for \textsc{Cutting at most   vertices}.

\begin{theorem}\label{thm:svc_fpt_kt}
\textsc{Cutting at most   vertices}, \textsc{Cutting at most  vertices with terminal}, and \textsc{Cutting at most   vertices by edge-cut with terminal} can be solved in time .
\end{theorem}

\begin{proof}
We first consider a -colored version of \textsc{Cutting at most  vertices}. That is, we are given a graph  where each vertex is either colored red or blue (this is not required to be a proper coloring), and the task is to find a connected red set  with  such that  is blue and . If such a set exists, it can be found in polynomial time by trying all maximal connected red sets.

Now let  be a graph. Assume that there is a set  with  and ; we may assume that  is connected. It suffices to find a coloring of  such that  is colored red and  is colored blue. This can be done by coloring each vertex  either red or blue independently and uniformly at random. Indeed, this gives a desired coloring with probability at least , which immediately yields a  time randomized algorithm for \textsc{Cutting at Most   Vertices}.

This algorithm can be derandomized in standard fashion using universal sets (compare with Cai et al.~\cite{cai2006random}). Recall that \emph{a -universal set} is a collection of binary vectors of length  such that for each index subset of size , each of the  possible combinations of values appears in some vector of the set. A construction of Naor et al.~\cite{naor1995splitters} gives a -universal set of size  that can be listed in linear time. It suffices to try all colorings induced by a -universal set obtained trough this construction. 

The given algorithm works for \textsc{Cutting at most   vertices with terminal}  with obvious modifications. That is, given a coloring, we simply check if the terminal  is red and its connected red component is a solution. This also works for \textsc{Cutting at most   vertices by edge-cut with terminal}, as we have .
\end{proof}







\section{Cutting at most   vertices parameterized by }

In this section we show that \textsc{Cutting at Most   Vertices} is fixed-parameter tractable when parameterized by the size of the separator  only. 
Specifically, we will prove the following theorem.

\begin{theorem}\label{thm:svc_fpt}
\textsc{Cutting at Most   Vertices}  can be solved in time .
\end{theorem}

The remainder of this section consists of the proof of Theorem \ref{thm:svc_fpt}. Note that we may assume . Indeed, if  for a fixed constant ,  then we can apply the algorithm of Theorem \ref{thm:svc_fpt_kt} to solve \textsc{Cutting at Most   Vertices} in time . On the other hand, if , then any vertex set  of size  is a solution, as .

We start by guessing a vertex  that belongs to a solution set  if one exists; specifically, we can try all choices of . We cannot expect to necessarily find a solution  that contains the chosen vertex , even if the guess is correct, as the terminal variant is \classW{1}-hard. We will nonetheless try; turns out that the only thing that can prevent us from finding a solution containing  is that we find a solution \emph{not} containing .

With  fixed, we compute for each  the unique minimum important -separator . This can be done in polynomial time by Lemma \ref{lemma:unique_min_sep}. Let  be set of those  with , and denote . Finally, let  be a set family consisting of those  for  that are inclusion-minimal,
i.e., if , then there is no  such that . 
Note that we can compute the sets ,  and  in polynomial time.

There are now three possible cases that may occur.
\begin{enumerate}
    \item If , we conclude that we have no solution containing .
    \item If there is  such that , then  gives a solution, and we stop and return a YES-answer.
    \item Otherwise,  is non-empty and for all sets  we have .
\end{enumerate}
We only have to consider the last case, as otherwise we are done. We will show that in that case, the sets  can be used to find a solution  containing  if one exists. For this, we need the following structural results about the sets .

\begin{lemma}\label{lemma:R_containment}
For any , if  then .
\end{lemma}

\begin{proof}
Let  and . Since  is a -separator of minimum size, we must have . By \eqref{eq:vsubm}, we have

Because , the set  is a -separator.
Thus, if , then  is a -separator that witnesses that  is not an important separator. But this is not possible by the definition of , so we have .
\end{proof}

\begin{lemma}\label{lemma:R_disjoint}
Any distinct  are disjoint.
\end{lemma}

\begin{proof}
Assume that  are distinct and intersect. Then there is .
Since , the set  is a -separator of size at most , and .
Recall that  contains inclusion-minimal sets  for . 
But by Lemma~\ref{lemma:R_containment},   is a proper subset of both  and , which is not possible by the definition of .
\end{proof}

Now assume that the input graph  has a solution for \textsc{Cutting at Most   Vertices} containing . In particular, then there is an inclusion-minimal set  with  satisfying  and . Let us fix one such set~.

\begin{lemma}\label{lemma:R_intersects_U}
For all , the set  is either contained in  or does not intersect it.
\end{lemma}

\begin{proof}
Suppose that there is a set  that intersects both  and its complement. Let .

Now let . By Lemma \ref{lemma:R_containment}, we have . 
If  then it follows from \eqref{eq:vsubm} that

However, this would imply that  is a -separator smaller than .

Thus, we have . But  is a proper subset of ; furthermore, any vertex of  that is not in  is also in , so we have . This is in contradiction with the minimality of .
\end{proof}

\begin{lemma}\label{lemma:finding_solution}
Let  be the union of all  that do not intersect . Then  and there is an important -separator  of size at most  such that .
\end{lemma}

\begin{proof}
Let . Consider an arbitrary ; such vertex exists, since . Since  separates  from , the set  is well-defined.

Suppose now that  is not contained in . Let . Since  is a minimum-size -separator we have . But  also separates  and , so we have . By \eqref{eq:vsubm}, we have

But since  is not contained , it follows that  is a proper subset of , which contradicts the minimality of .

Thus we have . It follows that  contains a set , which implies that  and . Furthermore, since  was chosen arbitrarily, we have that .

If  is an important -separator, we are done. Otherwise, there is an important -separator  with  and . But then we have , and , that is, .
\end{proof}

Recall now that we may assume  for all . Furthermore, we have  and the sets  are disjoint by Lemma \ref{lemma:R_disjoint}. Thus, at most two sets  fit inside  by Lemma \ref{lemma:R_intersects_U}. This means that if we let  be the union of all  that do not intersect , then as we have already computed , we can guess  by trying all  possible choices.

Assume now that  is a minimal solution containing  and our guess for  is correct. We enumerate all important -separators of size at most . We will find by Lemma \ref{lemma:finding_solution} an important -separator  such that  and . If , we have found a solution.
Otherwise, we delete a set  of  elements from  to obtain a solution . To see that this suffices, observe that . Therefore, . 
As all important -separators can be listed in time  by Lemma \ref{lemma:computing_imp_seps}, the proof of Theorem \ref{thm:svc_fpt} is complete.









\section{Hardness results}
We start this section by complementing Theorem~\ref{thm:svc_fpt}, as we show that \textsc{Cutting at Most   Vertices} is \classNP-complete and \classW{1}-hard when parameterized by . We also show that same holds for
\textsc{Cutting at Most   Vertices with Terminal}. Note that both of these problems are in \classXP{} when parameterized by , as they can be solved by checking all vertex subsets of size at most .

\begin{theorem}\label{thm:svc_is_hard}
\textsc{Cutting at Most   Vertices} and \textsc{Cutting at Most   Vertices with Terminal} are \classNP-complete and \classW{1}-hard with the parameter~.
\end{theorem}

\begin{proof}
We prove the \classW{1}-hardness claim for \textsc{Cutting at Most   Vertices} by a reduction from  \textsc{Clique}. 
Recall that this \classW{1}-complete (see~\cite{DowneyF99}) problem asks for a graph  and a positive integer  where  is a parameter, whether  contains a clique of size .
Let  be an instance of \textsc{Clique},  and ; we construct an instance  of \textsc{Cutting at Most   Vertices} as follows. Let  be a clique of size  and identify  vertices of  with the vertices of . Let  be a clique of size  and identify the vertices of  with the edges of . Finally, add an edge between vertex  of  and vertex  of  whenever  is incident to  in . Set  and . The construction is shown in Fig.~\ref{fig:W} a).

If  has a -clique , then for the set  that consists of the vertices  of  corresponding to edges of  we have  and . On the other hand, suppose that there is a set of vertices  of  such that  and .
 First, we note that  cannot contain any vertices of , as then  would be too large. Thus, the set  consists of vertices of . Furthermore, we have that . Indeed, assume that this is not the case. If , then, since  has at least one neighbor in , we have

and if , then  has at least  neighbors in , and thus

Thus, we have that  only consist of vertices of  and . But then the vertices of  that are in  form a -clique in .

The \classW{1}-hardness proof for \textsc{Cutting at Most   Vertices with Terminal} uses the same arguments. The only difference is that we add the terminal  in the clique  and let  (see Fig.~\ref{fig:W} b).

Because \textsc{Clique} is well known to be \classNP-complete~\cite{GareyJ79} and our parameterized reductions are polynomial in , it immediately follows that \textsc{Cutting at Most   Vertices} and \textsc{Cutting at Most   Vertices with Terminal} are \classNP-complete.
\end{proof}

While we have an \classFPT-algorithm for  \textsc{Cutting at Most   Vertices} when parameterized by  and  or by  only, it is unlikely that the problem has a polynomial kernel (we refer to~\cite{DowneyF99,FlumG06,Niedermeierbook06} for the formal definitions of kernels). Let  be a graph with  connected components , and let ,  be integers. Now  is a YES-instance of  \textsc{Cutting at Most   Vertices} if and only if  is a YES-instance for some , because 
it can always be assumed that a solution is connected.
By the results of Bodlaender et al.~\cite{BodlaenderDFH09}, this together with Theorem~\ref{thm:svc_is_hard} implies the following.

\begin{theorem}\label{prop:kernel}
 \textsc{Cutting at Most   Vertices} has no polynomial kernel when parameterized either by  and  or by  only, unless .
\end{theorem}

\begin{figure}[ht]
\centering\scalebox{0.6}{\begin{picture}(0,0)\includegraphics{Fig1.pdf}\end{picture}\setlength{\unitlength}{3947sp}\begingroup\makeatletter\ifx\SetFigFont\undefined \gdef\SetFigFont#1#2#3#4#5{\reset@font\fontsize{#1}{#2pt}\fontfamily{#3}\fontseries{#4}\fontshape{#5}\selectfont}\fi\endgroup \begin{picture}(8724,2635)(739,-2384)
\put(8251,-2011){\makebox(0,0)[lb]{\smash{{\SetFigFont{12}{14.4}{\rmdefault}{\mddefault}{\updefault}{\color[rgb]{0,0,0}}}}}}
\put(2040,-1444){\makebox(0,0)[lb]{\smash{{\SetFigFont{12}{14.4}{\rmdefault}{\mddefault}{\updefault}{\color[rgb]{0,0,0}}}}}}
\put(2026,-1786){\makebox(0,0)[lb]{\smash{{\SetFigFont{12}{14.4}{\rmdefault}{\mddefault}{\updefault}{\color[rgb]{0,0,0}}}}}}
\put(1951,-886){\makebox(0,0)[lb]{\smash{{\SetFigFont{12}{14.4}{\rmdefault}{\mddefault}{\updefault}{\color[rgb]{0,0,0}}}}}}
\put(2626,-886){\makebox(0,0)[lb]{\smash{{\SetFigFont{12}{14.4}{\rmdefault}{\mddefault}{\updefault}{\color[rgb]{0,0,0}}}}}}
\put(901, 14){\makebox(0,0)[lb]{\smash{{\SetFigFont{12}{14.4}{\rmdefault}{\mddefault}{\updefault}{\color[rgb]{0,0,0}}}}}}
\put(2401,-61){\makebox(0,0)[lb]{\smash{{\SetFigFont{12}{14.4}{\rmdefault}{\mddefault}{\updefault}{\color[rgb]{0,0,0}}}}}}
\put(5040,-1444){\makebox(0,0)[lb]{\smash{{\SetFigFont{12}{14.4}{\rmdefault}{\mddefault}{\updefault}{\color[rgb]{0,0,0}}}}}}
\put(5026,-1786){\makebox(0,0)[lb]{\smash{{\SetFigFont{12}{14.4}{\rmdefault}{\mddefault}{\updefault}{\color[rgb]{0,0,0}}}}}}
\put(4951,-886){\makebox(0,0)[lb]{\smash{{\SetFigFont{12}{14.4}{\rmdefault}{\mddefault}{\updefault}{\color[rgb]{0,0,0}}}}}}
\put(5626,-886){\makebox(0,0)[lb]{\smash{{\SetFigFont{12}{14.4}{\rmdefault}{\mddefault}{\updefault}{\color[rgb]{0,0,0}}}}}}
\put(3901, 14){\makebox(0,0)[lb]{\smash{{\SetFigFont{12}{14.4}{\rmdefault}{\mddefault}{\updefault}{\color[rgb]{0,0,0}}}}}}
\put(5401,-61){\makebox(0,0)[lb]{\smash{{\SetFigFont{12}{14.4}{\rmdefault}{\mddefault}{\updefault}{\color[rgb]{0,0,0}}}}}}
\put(4801, 14){\makebox(0,0)[lb]{\smash{{\SetFigFont{12}{14.4}{\rmdefault}{\mddefault}{\updefault}{\color[rgb]{0,0,0}}}}}}
\put(2026,-2311){\makebox(0,0)[lb]{\smash{{\SetFigFont{12}{14.4}{\rmdefault}{\mddefault}{\updefault}{\color[rgb]{0,0,0}a)}}}}}
\put(5026,-2311){\makebox(0,0)[lb]{\smash{{\SetFigFont{12}{14.4}{\rmdefault}{\mddefault}{\updefault}{\color[rgb]{0,0,0}b)}}}}}
\put(7951,-886){\makebox(0,0)[lb]{\smash{{\SetFigFont{12}{14.4}{\rmdefault}{\mddefault}{\updefault}{\color[rgb]{0,0,0}}}}}}
\put(8626,-886){\makebox(0,0)[lb]{\smash{{\SetFigFont{12}{14.4}{\rmdefault}{\mddefault}{\updefault}{\color[rgb]{0,0,0}}}}}}
\put(6901, 14){\makebox(0,0)[lb]{\smash{{\SetFigFont{12}{14.4}{\rmdefault}{\mddefault}{\updefault}{\color[rgb]{0,0,0}}}}}}
\put(8401,-61){\makebox(0,0)[lb]{\smash{{\SetFigFont{12}{14.4}{\rmdefault}{\mddefault}{\updefault}{\color[rgb]{0,0,0}}}}}}
\put(6901,-886){\makebox(0,0)[lb]{\smash{{\SetFigFont{12}{14.4}{\rmdefault}{\mddefault}{\updefault}{\color[rgb]{0,0,0}}}}}}
\put(8026,-2311){\makebox(0,0)[lb]{\smash{{\SetFigFont{12}{14.4}{\rmdefault}{\mddefault}{\updefault}{\color[rgb]{0,0,0}c)}}}}}
\end{picture} }
\caption{Constructions of  in the proofs of Theorems~\ref{thm:svc_is_hard} and \ref{thm:svc_terminal_hard}.
\label{fig:W}}
\end{figure}

We will next show that when we consider the size of the separator  as the sole parameter, adding a terminal makes the problem harder. Indeed, while \textsc{Cutting at Most   Vertices with Terminal} with parameter  is trivially in XP, we next show that it is also \classW{1}-hard, in contrast to Theorem~\ref{thm:svc_fpt}.

\begin{theorem}\label{thm:svc_terminal_hard}
\textsc{Cutting at Most   Vertices with Terminal} is \classW{1}-hard with parameter .
\end{theorem}

\begin{proof}
Again, we prove the claim by a reduction from \textsc{Clique}. Let  be a clique instance,  and ; we create an instance  of \textsc{Cutting at Most   Vertices with Terminal}. The graph  is constructed as follows. Create a new vertex  as the terminal. For each vertex and edge of , add a corresponding vertex to , and add an edge between vertices  and  in  when  is incident to  in . Finally, connect all vertices of  corresponding to vertices of  to the terminal , and set  and . The construction is shown in Fig.~\ref{fig:W} c).

If  has a -clique , then cutting away the  vertices of  corresponding to  leaves exactly  vertices in the connected component of  containing . Now suppose that  is a set with  such that  and , and let . Note that the elements of  that do not belong to  are exactly the elements  and the vertices corresponding to  such that  and ; denote this latter set of elements by . Since  is a solution, we have , and thus . But this is only possible if  is a -clique in .
\end{proof}

Finally, we show that \textsc{Cutting at most   vertices by edge-cut with terminal} is also \classNP-hard.

\begin{theorem}\label{thm:sec_terminal_np}
\textsc{Cutting at most   vertices by edge-cut with terminal} is \classNP-complete.
\end{theorem}

\begin{proof}
We give a reduction from the \textsc{Clique} problem. It is known that this problem is \classNP-complete for regular graphs~\cite{GareyJ79}. Let  be an instance of \textsc{Clique}, with  being a -regular -vertex graph. We create an instance  of \textsc{Cutting  Vertices by Edge-Cut with Terminal} as follows. The graph  is constructed by starting from a base clique of size . One vertex in this base clique is selected as the terminal , and we additionally distinguish  special vertices. For each , we add a new vertex to , and add an edge between this vertex and all of the  distinguished vertices of the base clique. For each edge  in , we also add a new vertex to , and add edges between this vertex and vertices corresponding to  and .
The construction is shown in Fig.~\ref{fig:NPc}.
We set  and .


\begin{figure}[ht]
\centering\scalebox{0.6}{\begin{picture}(0,0)\includegraphics{Fig2.pdf}\end{picture}\setlength{\unitlength}{3947sp}\begingroup\makeatletter\ifx\SetFigFont\undefined \gdef\SetFigFont#1#2#3#4#5{\reset@font\fontsize{#1}{#2pt}\fontfamily{#3}\fontseries{#4}\fontshape{#5}\selectfont}\fi\endgroup \begin{picture}(2802,2724)(289,-2773)
\put(3076,-436){\makebox(0,0)[lb]{\smash{{\SetFigFont{12}{14.4}{\rmdefault}{\mddefault}{\updefault}{\color[rgb]{0,0,0}}}}}}
\put(1576,-2011){\makebox(0,0)[lb]{\smash{{\SetFigFont{12}{14.4}{\rmdefault}{\mddefault}{\updefault}{\color[rgb]{0,0,0}}}}}}
\put(1801,-2386){\makebox(0,0)[lb]{\smash{{\SetFigFont{12}{14.4}{\rmdefault}{\mddefault}{\updefault}{\color[rgb]{0,0,0}}}}}}
\put(1426,-961){\makebox(0,0)[lb]{\smash{{\SetFigFont{12}{14.4}{\rmdefault}{\mddefault}{\updefault}{\color[rgb]{0,0,0}}}}}}
\put(2176,-961){\makebox(0,0)[lb]{\smash{{\SetFigFont{12}{14.4}{\rmdefault}{\mddefault}{\updefault}{\color[rgb]{0,0,0}}}}}}
\put(1951,-286){\makebox(0,0)[lb]{\smash{{\SetFigFont{12}{14.4}{\rmdefault}{\mddefault}{\updefault}{\color[rgb]{0,0,0}}}}}}
\put(1201,-2611){\makebox(0,0)[lb]{\smash{{\SetFigFont{12}{14.4}{\rmdefault}{\mddefault}{\updefault}{\color[rgb]{0,0,0}base clique}}}}}
\put(2851,-1036){\makebox(0,0)[lb]{\smash{{\SetFigFont{12}{14.4}{\rmdefault}{\mddefault}{\updefault}{\color[rgb]{0,0,0}}}}}}
\end{picture} }
\caption{Construction of  in the proof of Theorem~\ref{thm:sec_terminal_np}.
\label{fig:NPc}}
\end{figure}


If  has a -clique , then selecting as  the base clique and all vertices of  corresponding to vertices and edges of  gives a solution to , as we have  and 
. For the other direction, consider any solution  to instance . The set  must contain the whole base clique, as otherwise there are at least  edges inside the base clique that belong to . Let  and  be the subsets of  corresponding to vertices and edges of , respectively. If , then . Assume now that  is fixed, and consider how adding vertices to  changes  . For each edge , if neither of the endpoints of  is in , then adding  to  adds  to . If exactly one of the endpoints of  is in , then adding  to  does not change . Finally, if both of the endpoints of  are in , then adding  to  reduces  by . Thus, in order to have , we must have that  and the endpoints of all edges in  are in . But due to the requirement that , this is only possible if  induces a clique in .
\end{proof}

\paragraph{Acknowledgements.} We thank the anonymous reviewers for pointing out the related work of Lokshtanov and Marx.

\bibliographystyle{amsplain}
\bibliography{refs}

\end{document}
