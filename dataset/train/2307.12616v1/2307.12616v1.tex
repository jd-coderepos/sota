\def\cvprPaperID{5546}
\def\confName{ICCV}
\def\confYear{2023}

\def\paperTitle{CTVIS: Consistent Training for Online Video Instance Segmentation}

\def\authorBlock{
    Kaining Ying\tt kaining.ying.cv@gmail.com     \quad
    Qing Zhong     \quad
    Weian Mao  \quad
    Zhenhua Wang  \quad
    Hao Chen \quad
    \\ 
    Lin Yuanbo Wu \quad 
     Yifan Liu \quad
    Chengxiang Fan \quad
    Yunzhi Zhuge \quad
    Chunhua Shen
    
    \.2cm]
    \normalsize  Zhejiang University  \qquad
    \normalsize  College of Computer Science and Technology, Zhejiang University of Technology \\
    \normalsize  College of Information Engineering, Northwest A\&F University  \\
    \normalsize  The University of Adelaide, Australia \qquad
    \normalsize  Swansea University, UK \\
    \small \href{https://github.com/KainingYing/CTVIS}{https://github.com/KainingYing/CTVIS}


}

\newif\ifreview\newcommand{\review}{\reviewtrue}
\newif\ifarxiv\newcommand{\arxiv}{\arxivtrue}
\newif\ifcamera\newcommand{\cameraready}{\cameratrue}
\newif\ifrebuttal\newcommand{\rebuttal}{\rebuttaltrue}

\newcommand{\red}[1]{\textcolor{RubineRed}{#1}}
\newcommand{\blue}[1]{\textcolor{blue}{#1}}
\newcommand{\green}[1]{\textcolor{Green}{#1}}
\newcommand{\orange}[1]{\textcolor{Orange}{#1}}


\newcommand{\ykn}[1]{\textbf{\color[]{KY: #1}}}



 \rebuttal

\documentclass[10pt,twocolumn,letterpaper]{article}
\ifreview \usepackage[review]{cvpr} \fi
\ifarxiv \usepackage[pagenumbers]{cvpr} \fi
\ifrebuttal \usepackage[rebuttal]{cvpr} \fi
\ifcamera \usepackage{cvpr} \fi

\usepackage{graphicx}
\usepackage{amsmath}
\usepackage{amssymb}
\usepackage{booktabs}



\usepackage{times}
\usepackage{microtype}
\usepackage{epsfig}
\usepackage[table,xcdraw]{xcolor}
\usepackage{caption}
\usepackage{float}
\usepackage{placeins}
\usepackage{color, colortbl}
\usepackage{stfloats}
\usepackage{enumitem}
\usepackage{tabularx}
\usepackage{graphicx}
\usepackage{xstring}
\usepackage{multirow}
\usepackage{xspace}
\usepackage{url}
\usepackage{subcaption}
\usepackage{xcolor}
\usepackage{rotating}
\usepackage[hang,flushmargin]{footmisc}
\usepackage{bm}
\usepackage{amssymb}
\usepackage{pifont}


\ifcamera \usepackage[accsupp]{axessibility} \fi









\newcommand{\nbf}[1]{{\noindent \textbf{#1.}}}

\newcommand{\supp}{supplemental material\xspace}
\ifarxiv \renewcommand{\supp}{appendix\xspace} \fi

\newcommand{\todo}[1]{{\textcolor{red}{[TODO: #1]}}}

\newcommand{\R}[1]{{\textbf{\ifstrequal{#1}{1}{\textcolor{red}{R#1}}{\ifstrequal{#1}{2}{\textcolor{blue}{R#1}}{\ifstrequal{#1}{3}{\textcolor{green!70!black}{R#1}}{\ifstrequal{#1}{4}{\textcolor{teal}{R#1}}{\textcolor{cyan}{R#1}}}}}}}}

\newenvironment{packed_enum}{
\begin{enumerate}
  \setlength{\itemsep}{1pt}
  \setlength{\parskip}{2pt}
  \setlength{\parsep}{0pt}
}{\end{enumerate}}

\newenvironment{packed_item}{
\begin{itemize}
  \setlength{\itemsep}{1pt}
  \setlength{\parskip}{2pt}
  \setlength{\parsep}{0pt}
}{\end{itemize}}
 

\usepackage{xr-hyper}

\makeatletter
\newcommand*{\addFileDependency}[1]{
  \typeout{(#1)}
  \@addtofilelist{#1}
  \IfFileExists{#1}{}{\typeout{No file #1.}}
}

\makeatother
\newcommand*{\myexternaldocument}[1]{
    \externaldocument{#1}
    \addFileDependency{#1.tex}
    \addFileDependency{#1.aux}
}


\usepackage[pagebackref,breaklinks,colorlinks]{hyperref}
\usepackage[capitalize]{cleveref}
\crefname{section}{Sec.}{Secs.}
\crefname{table}{Table}{Tables}
\crefname{figure}{Fig.}{Figs.}

\frenchspacing
 \myexternaldocument{_main}
\begin{document}
\title{CTVIS: Consistent Training for Online Video Instance Segmentation\\
\big(Authors' Response to Review\big)}
\maketitle
\thispagestyle{empty}
\appendix


\noindent\textbf{Initial Paper Ratings}: B/L (\R1), W/A (\R2), W/A (\R3).



We thank all the reviewers for their comments and suggestions. Overall, they thought our CTVIS method is novel, well-motivated (\R2 and \R3), reasonable and easy to understand (\R1). They also found that CTVIS is very competitive in performance (\R1, \R2 and \R3). 
Below are our responses to concerns and suggestions raised in the review.

\section{Response to \R1}

\noindent\textbf{Weakness \#1:} \textbf{Inconsistent datasets and backbones in ablation study.} 
Thank you for this question. The ablation study (\ie Section~4.2) used ResNet-50 as the backbone and performed experiments on both YTVIS19 and OVIS, which are actually consistent. 
However, if you meant the experiments in Sections~4.3 and 4.4, where YTVIS21 is used, instead of YTVIS19, is because the \emph{hand} class in YTVIS19 conflicts with COCO's \emph{person} class (note that YTVIS21 removed the \emph{hand} class). 
Also, in Section~{4.3}, to allow a fair comparison with MinVIS (which only reports the results 
by training the model on
sampling frames using Swin-L), we employed Swin-L.  We will clarify these points in the new version. 

\noindent\textbf{Weakness \#2:} \textbf{Add a column about data augmentation strategy in Table~1.}
It is worth noting that \textbf{NO} extra augmentations (compared with SOTA models) were adopted to achieve our reported results. However, we think it would be helpful to add the suggested column if the space permits.

\noindent\textbf{Weakness \#3:} \textbf{Inconsistent results in Tables~1 and 5.}
As pointed out in L578, L741 and L785, the two versions are trained (YTVIS21 \vs COCO) and tested (YTVIS21 \vs YTVIS21) on different benchmarks. To be specific,  in Table~1 is obtained by training jointly on COCO and YTVIS21, under standard supervision, enabling fair comparison with other fully-supervised methods. The 49.7 in Table 5 is obtained on YTVIS21,  
where the model was \emph{trained on pseudo videos}. Please refer to Section~4.3 for more details. 


\noindent\textbf{Additional comments with respect to codes and models.} 
Yes, we will definitely release codes and models upon the acceptance of this work.

\section{Response to \R2}

\noindent\textbf{Weakness \#1:} \textbf{Implementation details of extending vanilla IDOL to Multiple Reference Frames.}
We extended vanilla IDOL to multi-reference IDOL in a straightforward way. Specifically, we randomly sample  reference frames surrounding the key frame, forming a training clip of -frames (note that CTVIS uses clips of -frames as well). Our new version will include the details for this extension. 


\noindent\textbf{Weakness \#2:} \textbf{Report the training GPU time and inference FPS.}
Vanilla IDOL is faster than CTVIS. For training (batch size is 1), the throughput of IDOL is 3.3 samples per second, while the throughput of CTVIS is 1.3. 
In terms of inference (using ResNet-50 as the backbone, tested on 1 piece of GeForce RTX 3090, and the batch size is 1), CTVIS runs at 13.9 frames per second, which is 
negligibly slower than IDOL's 14.3. However, CTVIS notably outperforms IDOL by 5\% in terms of AP. 
As suggested, the details and comparison to IDOL on GPU training time and inference FPS will be added in our next version.  

\noindent\textbf{Additional comments: Some minor typos.} 
Thank you for your meticulous review. We will definitely fix these typos.

\section{Response to \R3}

\noindent\textbf{Weakness \#1:} \textbf{Evaluate how incorporating noise helps reduce ID switch.} 
Thanks for your question and we agree that evaluating the occurrence of ID switches is necessary to understand the contribution of noise better.
However, at this stage, we have a tight budget for time and GPU resources, and would like to try this (need to train at least two models) in the future. We also agree that the results in Table~3 cannot directly verify that adding noise can reduce the ID switch, as the performance gain may also come from some other aspects. 
Intuitively, Figure 5 shows that the IDs of a fish (the left example) and a dog (the right example) switched when using VITA and IDOL, but CTVIS does not have this trouble (or can recover the tracking ID). 

\noindent\textbf{Weakness \#2:} \textbf{Missing performance without using COCO joint training.} 
We didn't mention the results due to the fact that all (nearly) SOTA methods (\eg VITA, SeqFromer and IDOL) use COCO joint training as default (as reported in Table~1).
Moreover, CTVIS (ResNet-50 as the backbone) trained without COCO data achieves 49.3 on AP, close to 50.1 with COCO joint training.

\noindent\textbf{Weakness \#3:} \textbf{The comparison of FPS.} 
Please refer to our reply to \R2's Weakness \#2.

\noindent\textbf{Additional comments:} \textbf{The limitation and failure cases of the proposed method are not discussed.} 
This question is inspiring. We admit that CTVIS has its own limitations. For example, it introduces extra computation in order to maintain the memory bank during training. Moreover, as CTVIS is a training strategy, its performance 
heavily depends on the performance of the base frame-wise segmentor (\eg Mask2Former). 
Hence, CTVIS is liable to go wrong if the base segmentor cannot provide decent segmentation results. We will add a discussion on this.

\end{document}