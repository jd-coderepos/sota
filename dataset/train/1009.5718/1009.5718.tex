\section{Experimental Results}\label{sec:results}

\subsection{Forest Signatures}

\begin{figure*}
\begin{center}
\mbox{
\subfigure[Frequency of detection for 25 species of terrestrial  birds and mammals on BCI. The horizontal dotted line at Y=1000 stands for the theoretical total maximum number of species at BCI. \label{fig:sigpolt1}]{\includegraphics[scale=0.4]{figs/signatureplot2}} \quad
\subfigure[The daily pattern of animal activity on the forest floor. Colors match the species names in Fig~\ref{fig:sigpolt1}.\label{fig:sigplot2}.]{\includegraphics[scale=0.45]{figs/signatureplot1}}
}
\caption{Forest signatures from camera trap deployment at  BCI}
\label{fig:fs-sign}
\end{center}
\end{figure*}    

Our year-long deployment of remote cameras at randomized locations has given us a unique and unbiased view of the overall activity of animals on the rainforest floor including the species present and their relative abundance (Figure~\ref{fig:fs-sign}). The deployment resulted in a total of 764 deployments with 17111 animal detections and 25 species detections. This measure of animal activity is simply the number of times a given species walked across a sample plot, and offers a direct metric of potential ecological impact. For example, as shown in Figure~\ref{fig:fs-sign}, on BCI agoutis, peccaries, and paca are the most frequently detected species, and thus the most likely to have an impact on local plant populations through seed predation or dispersal. If calibrated into density (animals/km$^2$)~\citep{Rowcliffe-08} these could also be used to derive estimates of biomass for each species or ecological group. 
	
The standardized measures of species diversity and abundance represented by these �signatures� (Figure~\ref{fig:fs-sign}) are exactly those needed to evaluate the effects of modern environmental change.  Effects of climate change and invasive species would be reflected in changes in species composition, while changes in abundance would reflect natural population fluctuations, as well as more dramatic crashes or explosions typical of human influenced dynamics.  




\subsection{Sample Size Optimization}
\begin{figure*}
\begin{center}
\mbox{
\subfigure[Estimation of species diversity with increased sample size. Each camera deployment represents 8 days of monitoring.  Curves are drawn using a rarefaction (Sobs) or Jackknife (Jack1) resampling of 200 camera deployments on BCI.\label{fig:power-diversity}]{\includegraphics[scale=0.65]{figs/power-diversity}} \quad
\subfigure[The variation in estimated detection rate  for agoutis with sampling effort. The mean rate (black line) changes little, but the variation (min/max are thin red lines, 95\% confidence intervals are thick red lines) in estimates decreases with increasing sample effort, leveling off after around 15-20 camera deployments. Each camera deployment is 8-days long. All estimates come from 1000 resamples of data from one study plot.\label{fig:power-trap}]{\includegraphics[scale=0.40]{figs/power-trap}}
}
\caption{Sample Size Optimization Study Results}
\label{fig:power-study}
\end{center}
\end{figure*}    

Our year-round survey is unique in offering a seasonal view on the animal community.  However, many basic objectives of estimating the diversity and abundance of the community can be met with less effort.  We used our year-round data set to evaluate the sample size needed to meet these objectives.   
Figure~\ref{fig:power-diversity} shows the relationship between estimated mammalian species diversity and sampling effort. Each deployment represents one camera in the field for 8 days, and levels off after 15-25 deployments. There are 19 large and medium-sized terrestrial mammal species theoretically possible on BCI, although 4 of these (jaguar, jaguarundi, margray, and grison) are very rarely recorded on BCI, probably only as sporadic visitors. 

We also evaluated the sampling intensity needed to obtain an accurate estimate of detection frequency, an index of animal abundance (Figure~\ref{fig:power-trap}). This shows that the variation in average agouti detection rate levels off after 15-20 camera deployments, suggesting this is an appropriate sample effort. This could be met, for example, with 15 8-day deployments of one camera, or 3 deployments of 5 cameras. This relationship varies across species, with accurate estimates for species that are rare, or variable in their activity, requiring more sample effort. 



\subsection{Camera Deployments Strategies}
Sensor deployment and placement strategies has received considerable attention from the research community~\citep{Dhillon-03, Gonz-01,
Tilak-02, Wu-07, Xu-07}. However, to the best of our knowledge, this is is first study that takes into account application-level metric in a year-long real-world deployment. To evaluate the effect of camera placement on animal detections we compared the detection rate for cameras placed right on hiking trails (n=76) with those places in random locations within the forest (n=905).  
We found that there was a significant difference between trail and random trap rates for three out of 14 species tested (Figure~\ref{fig:deployment-strategies}). Ocelots favour trails (6-fold higher trap rate on trails), while brocket deer and peccary avoid them (respectively 3.3 and 2.8-fold higher trap rates on random placements). Paca also show a non-significant tendency to avoid trails, while tamandua show a slight tendency to favor them, but none of the other nine species show any evidence for a difference in trap rates between random and trail placements. Thus, in our study area, trail side cameras appear to be giving a biased view for a minority of species, although the degree of bias where it exists can be very high. It may also be worth noting that serious bias occurs only in the larger species ($>$10 kg) in this set.


\begin{figure*}
\includegraphics [scale=0.8]{figs/deployment-strategies}
\caption{Comparison of trap rates (log transformed) between random (Rand, n=905) and trail (Trail, n=76) camera placements for 14 species. Error bars are standard errors. P-values for each species give the significance of the difference between trail and random using F-tests (allowing for overdispersion) on quasi-Poisson generalised linear models of species counts, controlled for deployment duration by including the log of this value as an offset.}
\label{fig:deployment-strategies}
\end{figure*}		








\subsection{Spatial Autocorrelation Of Detection Rates:}



\subfiglabelskip=0pt
\begin{figure*}
\centering
\subfigure [ ] [ ] {\label{fig:rat_close}
\includegraphics[scale=0.55]{data/autocorrel/rat_close}}
\hspace{8pt}\subfigure [ ] [ ] {\label{fig:agouti_close}
\includegraphics[scale=0.55]{data/autocorrel/agouti_close}}
\subfigure [ ] [ ] {\label{fig:coati_close}
\includegraphics[scale=0.55]{data/autocorrel/coati_close}}
\hspace{8pt}\subfigure [ ] [ ] {\label{fig:deer_close}
\includegraphics[scale=0.55]{data/autocorrel/deer_close}}
\subfigure [ ] [ ] {\label{fig:pec_close}
\includegraphics[scale=0.55]{data/autocorrel/pec_close}}
\hspace{8pt}\subfigure [ ] [ ] {\label{fig:peccary}
\includegraphics[scale=0.55]{data/autocorrel/peccary}}

\caption{Semivariograms of detection rates for five species of mammals recorded by camera traps showing the decline in spatial autocorrelation after 25m out to 300m \subref{fig:rat_close} Spiny Rat  \subref{fig:agouti_close} Agouti, \subref{fig:coati_close} Coati, \subref{fig:deer_close} Brocket Deer, \subref{fig:pec_close} Peccary. Test for autocorrelation at larger scales produced similar results for all species, as represented by one graph for our largest species \subref{fig:peccary}. Graphs show the mean (center square), standard error (box) and standard deviation (wisker) for all pairs of camera traps within a given distance class.
}
\label{fig:corr-mammals}
\end{figure*}    



A common concern for all camera trap surveys not using mark/recapture analyses is to determine how far apart to camera sites must be to be spatially independent.  Typically, studies take extreme caution in this regard, spacing camera traps far enough apart to minimize the potential that the same individual animal would be detected by two cameras.  The usual measure is to estimate the diameter of the home range of a target species and make this the minimum spacing for cameras, which is often many km~\citep{Gompper-06}.  However, no study has empirically evaluated the autocorrelation of camera trap data.  Our data presents an excellent opportunity to do this, with pairs of cameras within a plot offering small-scale comparisons, and comparisons across plots offering larger-scale comparisons.
	We used the Geostatistical analyst extension of ArcGIS9 (ESRI) function to evaluate spatial autocorrelation by comparing the detection rate for a given species across all pairs of cameras.  We analyzed data within 2-month time windows to take into account that spatial patterns of animal activity may vary seasonally. We analyzed data for the five most common species, which include a large range of body size and scale of movement across the landscape. 
 
	The main result was that we found very little spatial autocorrelation in animal detection rates for any of the $5$ mammal species considered.  Figure~\ref{fig:corr-mammals} shows this result in that the covariance (y axis) between all pairs of traps is not significantly different from 0 when the cameras are greater than 25m apart (x axis).  The results were similar across species, with a positive correlation between the detection rates for camera sites very close to each other ($ < 25$ m) but not at any other spatial scales. This result suggests that cameras can be placed much closer to each other than is typically done and still record statistically independent data; instead of the many km minimum distance, we suggest cameras be a minimum of 25 m apart. 
Cameras set more closely might still be useful, if spatial autocorrelation is irrelevant, or taken into account by analyses.
	
 




\subsection{Camera Performance In Real-world}

\begin{figure*}
\begin{center}
\mbox{
\subfigure[Impact of seasonality on camera performance.\label{fig:ct-perf}]{\includegraphics[scale=0.45]{figs/ct-perf}} \quad
\subfigure[Camera failures in our study.\label{fig:camera-failures}]{\includegraphics[scale=0.4]{figs/camera-failures-new}}
}
\caption{Study of Camera Performance in Real-World}
\label{fig:ct-rw}
\end{center}
\end{figure*}    

Due to challenging weather and environmental conditions camera traps are often more difficult to operate in rainy reasons. To minimize the impact of seasonality on camera performance we suggest  use of silica desiccant packets (2 if possible) to keep the insides dry. 







There was a strong effect of seasonality Figure~\ref{fig:ct-perf}, with detection distance (measured by walking in front of the camera when setting it out) shortening during the rainy season.  This is probably a combination of moisture on the sensor, in the air between the sensor and the target, and on the target itself.  Together, these would dampen the difference between the IR signature of an animal compared with the background, and thus reduce its ability to detect an animal.
Shrinking the effective area each camera surveys has obvious impacts on the number of animals it detects.  Thus it is important to document these effects, and take them into account for comparisons of animal activity across seasons or sites. We also advise keeping cameras in dry-closet whenever not in use. Based on our experience rotating cameras out of service every 2 months for preventative maintenance works well. 

\subsection{Relation Between Animal Size And Camera Parameters}

An inherent property of the triggers used on camera traps is that they are less likely to detect small animals than large, all else being equal. This can be seen in the distribution of positions of different species relative to the camera when first detected (Figure~\ref{fig:detect-dist-angle}), demonstrating much shorter average distances for the smaller species and, to a lesser extent, narrower angles. We are currently developing methods to model this phenomenon~\citep{Rowcliffe-sub10}, allowing us to quantify camera sensitivity for any given species, camera or environment (illustrated by detection zone sectors in (ref. Figure~\ref{fig:detect-dist-angle}). This approach will be important in enabling us to extract abundance signals from trap rates by controlling for camera sensitivity.



\begin{figure*}
\includegraphics [scale=0.6]{figs/detect-dist-angle-new}
\caption{Positions of animals on first detection (open points) relative to the camera (filled triangles) for three representative species of contrasting size: spiny rat (0.4 kg), paca (8 kg) and peccary (25 kg). The camera is at the origin, with axis values in metres. The open sectors illustrate effective detection zones, estimated by fitting detection models to the distance and angle data for locations.}
\label{fig:detect-dist-angle}
\end{figure*}


\subsection{Camera Failures In Real-world}

We observed that only 30\% of the deployed cameras never failed during the year (Figure~\ref{fig:camera-failures}). This shows that operating a camera trap based solution over extended periods of time does require monitoring and debugging. Approximately 40\% failures  were due camera lens being blurry. 
The manufacturer repaired all cameras, and used our experience to find that the problem was caused by humidity de-laminating a filter on the lens.  They have since improved the seal on the lens.  
The second major source of failures was (-20\%) caused by humidity affecting circuitry of the camera.  The manufacture has since developed a new coating for their circuit boards which should improve their performance in high humidity.  

