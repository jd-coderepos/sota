

\documentclass[copyright]{eptcs}
\providecommand{\event}{ACAC 2009} \providecommand{\volume}{??} \providecommand{\anno}{20??}
\providecommand{\firstpage}{1} \providecommand{\eid}{??}
\usepackage{breakurl}        \usepackage{latexsym} \usepackage{amssymb,amstext,amsmath}
\usepackage{color}

\def\Proof{{\noindent \bf Proof. }}
\def\QED{{\hfill  \medskip}}



\newtheorem{Def}{Definition}
\newtheorem{Lem}{Lemma}
\newtheorem{Prop}{Proposition}
\newtheorem{Theo}{Theorem}
\newtheorem{Cor}{Corollary}
\newtheorem{Ex}{Example}
\newtheorem{Pre}{Property}
\newtheorem{Rmk}{Remark}
\usepackage{algorithmic}
\usepackage{algorithm}
\usepackage{graphicx}
\def\NN{\hbox{\sf I\kern-.13em\hbox{N}}}

\renewcommand{\algorithmicfor}{\textbf{For}}
\renewcommand{\algorithmicendfor}{\textbf{EndFor}}
\renewcommand{\algorithmicensure}{\textbf{Return:}}
\newcommand{\mc}[1]{\mathcal{#1}}
\newcommand{\chg}[1]{{#1}}
\newcommand{\suppr}[1]{}



\title{Cartesian product of hypergraphs: properties and algorithms}
\author{Alain  Bretto \email{alain.bretto@info.unicaen.fr} \and Yannick Silvestre \email{yannick.silvestre@info.unicaen.fr}  \and Thierry Vall\'ee \email{vallee@pps.jussieu.fr}
\institute{Universit\'e de Caen, GREYC CNRS UMR-6072, Campus
II, Bd Marechal Juin BP 5186, 14032 Caen cedex, France.}}
\def\titlerunning{Cartesian product of hypergraphs: properties and algorithms}
\def\authorrunning{A. Bretto \& Y. Silvestre \& T. Vall\'ee}
\begin{document}
\nocite{*}
\maketitle

\begin{abstract}
Cartesian products of graphs have been studied extensively since the 1960s. They make it possible to decrease the algorithmic complexity of problems by using the factorization of the product. Hypergraphs were introduced as a generalization of graphs and the definition of Cartesian products extends naturally to them. In this paper, we give new properties and algorithms concerning coloring aspects of Cartesian products of hypergraphs. We also extend a classical prime factorization algorithm initially designed for graphs to connected conformal hypergraphs using 2-sections of hypergraphs.
\end{abstract}

\section{Introduction}
Cartesian products of graphs have been studied since the 1960s by Vizing and Sabidussi. In \cite{Vizing} and \cite{Sabi} they independently showed, among other things, that for every finite connected graph there is a unique (up to isomorphism) prime decomposition of the graph into factors. This fundamental theorem was the starting point for research concerning the relations between a Cartesian product and its factors \cite{Zerovnik,ImPiZe,AC,Lauri}.
Some of the questions raised are still open, as in the case of the Vizing's conjecture\footnote{This conjecture expressed by Vizing in 1968 states that the domination number of the Cartesian product of graphs is greater than the product of the domination numbers of its factors.}.
These relations are of particular interest as they allow us to break down problems by transferring algorithmic complexity from the product to the factors. In 2006, Imrich and Peterin \cite{ImPe} gave an algorithm able to compute the prime factorization of connected graphs in linear time and space, making the use of Cartesian product decomposition particularly attractive.

Most of networks used in the context of parallel and distributed computation are Cartesian products: the hypercube, grid graphs, etc. In this context, the problem of finding a ``Cartesian'' embedding of an interconnection
network into another is also of fundamental importance and thus has gained considerable attention. Cartesian products are also used in telecommunication \cite{Vesel}.\smallskip


Hypergraph theory has been introduced in the 1960s as a generalization of graph theory. A lot of applications of hypergraphs have been developed since (for a survey see \cite{Bretto}). Cartesian products of hypergraphs can be defined in a same way as graphs, and similarly it is easier to study the hypergraph factors than the product.

In this paper, we give some new properties and algorithms concerning coloring aspects of Cartesian products of hypergraphs.
We show that the algorithm of Imrich and Peterin \cite{ImPe} can be used to find the prime factorization of connected conformal hypergraphs by considering the \emph{2-section} and the \emph{labelled 2-section} of the hypergraph to be factorized.\\

\section{Preliminaries}
The general terminology concerning graphs and hypergraphs in this
article is similar to the one used in \cite{berge1,berge2}. The
cardinality of a set  is denoted by . For  a function,
we define .

A \emph{hypergraph}  on a set of vertices  is a pair 
where  is a set of non-empty subsets of  called
\emph{hyperedges} such that . This implies in
particular that every vertex is included in at least one
hyperedge. A hypergraph is \emph{simple} if no hyperedge is
contained in another. In the sequel, we suppose that hypergraphs
are simple and that no edge is a loop, that is, the cardinality of
a hyperedge is at least . The number of hyperedges of a
hypergraph  is denoted by . We sometime write 
to denote the set of vertices of a hypergraph and  for
its set of edges. Given a hyperedge , we sometime
write  to denote the pairs of vertices of . A
\emph{graph} is a particular case of simple hypergraph where every
(hyper)edge is of size 2.

\chg{Given a graph  and  a subset of , we
define  as a subgraph of  where . }


The \emph{ degree of}  (denoted by ) is  and
 is the maximal degree of a vertex in .
A \emph{k-coloring} of a hypergraph  is \chg{a map  such that  and  such that every hyperedge  has two vertices , with .} The
\emph{chromatic number}, denoted by , is the smallest
integer  for which  admits a -coloring. A \emph{strong
k-coloring} of  is \chg{{a map  such that  and such that every
hyperedge  verifies: , .}}

Let  be a hypergraph, the \emph{chromatic index} of  is the least number of colors necessary to color the hyperedges of  such that two intersecting hyperedges are always colored differently. This number is denoted by . It is easy to see that . A hypergraph  has the \emph{colored hyperedge property} if  .

For  the set  is the \emph{partial hypergraph} generated by . 

The \emph{2-section}  of a hypergraph  is the graph
whose vertices are the vertices of , and where two vertices are
adjacent iff they belong to a same hyperedge. Notice that every
hyperedge of  is a clique of . A hypergraph  is
\emph{conformal} if, for every ,  is a
maximal clique of  iff  is a hyperedge of .

Finally, an isomorphism from the hypergraph  to the
hypergraph  is a bijection from  to  such that,
for every ,  iff . Note that the isomorphism  induces a bijection  defined by , for
every .

\section{Cartesian Product of  Hypergraphs: definition and coloring properties.}

Let  and  
be hypergraphs. The \emph{Cartesian product} of  and  is the hypergraph  with set of vertices
 and set of edges:


Note that up to the isomorphism the Cartesian product is
commutative and associative. That allows us to denote simply by
 the vertices of .
\suppr{SE REPETE AVEC LA PREUVE DU LEMME Note also that a
hyperedge  of  and a hyperedge  of  have a non-empty intersection iff  and
. In that case .}


In the sequel  will always stand for the Cartesian product of two hypergraphs  and , unless explicitely stated. We use  to denote the vertices of  and  to denote the vertices of .

\begin{Lem}\label{Fond}
. Moreover,  for any  and any .
\end{Lem}





\begin{Prop}\label{Prop2Section}

The 2-section of  is the Cartesian product of the 2-section of  and the 2-section of 
\end{Prop}

\begin{Theo}\label{ThmConformalCartesianStable} If  then  is conformal if and only if   and   are conformal.
\end{Theo}



\noindent We give now two new results about coloring aspects of Cartesian products of hypergraphs.

\begin{Theo}

 If  and  \chg{have} both the colored hyperedge property then  has the colored hyperedge property.
\end{Theo}


\begin{Theo} Let  and  (respectively  and ) be the chromatic number (resp. the strong chromatic number) of  and . We have the following:
\begin{enumerate}
\item 
\item .
\end{enumerate}
\end{Theo}


This leads to a straightforward algorithm to compute a minimal coloring (Algorithm \ref{COL}) of a given hypergraph  thanks to the minimal colorings of its factors. In the case the hypergraph is prime, there is no other choice but to use classical coloring algorithms. As the problem of determining if the chromatic number of a given hypergraph is less than an integer  is yet NP-complete (for ), it is worthwhile to add this preliminary step at the beginning of the investigation. An algorithm specifying how to compute the factors for conformal hypergraphs will be given in the following pages.

\begin{algorithm}
        \caption{Coloring algorithm COL}
        \label{COL}
        \begin{algorithmic}[1]
        \REQUIRE A hypergraph .
        \ENSURE A -coloring .
    \STATE Find a factorization in prime hypergraphs .
    \STATE Compute for each prime factor  a minimal coloring
     with .
    \RETURN  such that 
    \end{algorithmic}
    \end{algorithm}


\section{Hypergraph factorization algorithm}
In the sequel, we use some of the results from Imrich and Peterin in \cite{ImPe}. In order to find prime factorizations of conformal hypergraphs, we extend the algorithm given in \cite{ImPe}. This algorithm is based on a coloring of the edges of the graph  to be factorized in such a way that the factors are proper subgraphs of  said \emph{layers}. It uses the fact that if  is a Cartesian product of graphs then, for all  and , there is a subgraph  of  such that the  projection  induces an isomorphism between  and . Indeed, we remark that  is an edge of  iff there exists some  such that  is an edge of  and  differ only on their  coordinates. The graph  is then defined as the graph whose vertices are the -tuples which differ from  at most on the -th coordinate, and where  is an edge of  iff  is an edge of .

Subgraphs of the form  are the layers of  and it can be easily shown that every edge of  is contained in exactly one layer. Moreover the edge sets of the layers partition the edge set of .

\chg{We recall the square lemma, given in \cite{ImPe}:}

\begin{Lem}[Square lemma]\label{LmSq}
 \chg{If two edges are adjacent edges which belong to non-isomorphic layers, then these edges lay in a unique induced square.}
\end{Lem}

 A \chg{straightforward} consequence of the Square Lemma given in \cite{ImPe} is that every triangle of  is necessarily contained in the same layer. From these facts we get easily the following result.

\begin{Lem}\label{LmCliqueInLayer} Let  be the Cartesian product of graphs. Then every clique of  is contained in the same layer. Moreover, if two cliques share an edge then they are both contained in the same layer.
\end{Lem}

The extension to hypergraphs of the algorithm of \cite{ImPe} uses L2-sections. We start by some general definitions and basic properties then we extend Cartesian products and isomorphisms to L2-sections.


\begin{Def}\label{DfL2Section} Let  be a hypergraph, we define the -section  of  as the triple  where   is the 2-section of  and  is defined by .\\
\end{Def}

Hence, the -section of a hypergraph is a labelled version of the 2-section where every edge  is labelled with the set of hyperedges containing  and . In that way, it is possible to keep track of all the hyperedges from which the edge  comes from. It is then possible to build back the hypergraph from its labelled 2-section as shown in the definition below.

\begin{Def}\label{DfL2SectionToHypergraph} Let  be a -section, we define the hypergraph  by (see definition of  in section
Preliminaries).
\end{Def}

\noindent Not surprisingly, from the two definitions above, we get easily:

\begin{Prop}\label{PrInjectivityL2} For all hypergraphs  and L2-sections  we have
 and .
\end{Prop}

We extend now the Cartesian product operation to L2-sections.


\begin{Def}\label{DfProductLabeledGraph} Let  and  be the -sections of  and . We define their Cartesian product  as the triple  where:
\begin{itemize}
 \item  is the Cartesian product of  and .
 \item  is the map from  to  defined by:

\end{itemize}
\end{Def}

Note that the definition of  above is
correct. Indeed, by definition of the L2-section, for every edge
 of  there exists an hyperedge
 such that . Now, by definition of , either
, where  and ,
or , where  and . In the first case we have then  and so 
(otherwise  would be a loop), and in the second
 and so . It is moreover easy to check that .

\begin{Lem}\label{LemSCommuteWithCartesian} For all hypergraphs  we have:
\begin{enumerate}
\item \2pt]
\item .
\end{enumerate}
\end{Lem}


\begin{Def}\label{DfSubsection} Let  be the -section of . A triple  is a \emph{subsection} of  if the following conditions are satisfied:
\begin{enumerate}
\item \chg{ is a subset of  and .}
\item  is the restriction of  to . \item
\chg{If , then .}
\end{enumerate}
\end{Def}



It is easy to check that if  is a subsection then it is
the L2-section of the hypergraph . It
is also easy to verify that  is a partial hypergraph of .
We have moreover the following.

\begin{Lem}\label{LmSubsectionConformal} Let  be the L2-section of the conformal hypergraph  and  be a subsection of . Then  is a conformal partial hypergraph of .
\end{Lem}



\begin{Def}\label{DfIsomorphismL2Section} An isomorphism between two -sections  and  is a bijection  from \chg{} to \chg{} such that:
\begin{itemize}
 \item  if and only if , for all .
 \item  if and only if , for all  and .
\end{itemize}
We write  to express that there exists an isomorphism between  and .
\end{Def}


\begin{Lem}\label{LmIsoEquivSection} Let  and  be two hypergraphs. The two first statements below are equivalent. If moreover  and  are conformal then they are equivalent to the third one.
\begin{enumerate}
 \item 
 \item 
 \item 
\end{enumerate}
\end{Lem}


By combining the first point of Lemma
\ref{LemSCommuteWithCartesian} and the two first points of Lemma
\ref{LmIsoEquivSection}, it is straightforward to check that, up
to isomorphism, the Cartesian product is commutative and
associative on L2-sections. That allows us to overlook parenthesis
for Cartesian products of L2-sections. We give now a last
essential lemma before we introduce the factorization algorithm
for conformal hypergraphs.

\begin{Lem}\label{LmPassageAuProduitEtiquetage} Let  be the L2-section of the conformal hypergraph  and let  be its 2-section. Suppose  where \chg{, } are \chg{layers in} . Define  , where  is the restriction of  to . Then  is a conformal partial hypergraph of . Moreover we have .
\end{Lem}


We introduce now an algorithm which gives the prime factorization of conformal hypergraphs. The idea is the following. From the connected hypergraph  it first builds the -section  of . Then it runs the algorithm of Imrich and Peterin which colors the edges of the unlabelled underlying graph  with color  all edges of all layers that belong to the same factor . When obtained the factorization  of , the labels of the edges of the 's are used to build hypergraphs  they come from.

    \begin{algorithm}[!h]
        \caption{Hypergraph-prime decomposition}
        \label{algo2}
        \begin{algorithmic}[1]
        \REQUIRE A conformal hypergraph .
        \ENSURE The prime factors of , that is  such that
        .
    \STATE Compute , the L2-section of .
    \STATE Run the algorithm of prime-decomposition on the underlying graph  and call  its prime factors such that .
    \STATE Define  as the restriction of  to , , and build , for every .
\RETURN 
\end{algorithmic}
    \end{algorithm}


\begin{Theo} Algorithm 2 is sound and complete for every conformal hypergraph .
\end{Theo}


As the algorithm of Imrich and Peterin is able to return the factors of connected graphs with respect to the Cartesian product in linear time and space, the overall complexity of the given algorithm for a hypergraph  is in , as 2-section operations demand up to  per hyperedge.

\chg{
\begin{Cor}[Corollary of lemma \ref{LmIsoEquivSection}]
 Given a conformal hypergraph , there is a unique decomposition in prime factors with respect to the Cartesian product (up to isomorphisms).
\end{Cor}
\Proof
 The uniqueness of the decomposition of the 2-sections, imply the uniqueness of the decompositionn of the conformal hypergraphs they come from. \QED}




\section{Conclusion: Cartesian product and algorithmic complexity of problems}
In the previous section we dealt with some properties which were Cartesian product stable. The question we are interested in here is whether, in order to solve a decision problem , it is possible to design an operator on the hypergraph space which fulfill the following conditions:
\begin{itemize}
 \item The operator  must connect a plain hypergraph to a Cartesian product.
 \item For any hypergraph ,  must introduce a relation between  and  about , and this relation must be polynomially evaluable from  (that is to say, if one knows  or , one has to be able to compute in polynomial time whether  is in  or not). In the ideal case, this relation is constant-time or linear-time evaluable.
 \item The operator  has to be designable in polynomial time (relatively to the size or the order of the hypergraph).
\end{itemize}
Designing such operators is an interesting issue. Once such operators are built, assuming that no polynomial algorithm is known to solve the problem , it is possible to design competitive algorithms running on the hypergraph factors rather than on the whole Cartesian product.\\



\bibliographystyle{alpha}
\bibliography{bib}

\end{document}
