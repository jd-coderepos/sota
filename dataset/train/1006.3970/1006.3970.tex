In this section we prove Theorem~\ref{thm:flow-main} for  (the proof for the general case appears in Section~\ref{sec:mainproof}). For Markov flow graph  and corresponding Markov process  as in the theorem, for any integers , and any vertices  we will let  be the probability of reaching , and similarly, we define . In particular, when  and  is an edge (transition) then  is also the capacity of this edge. Note that, by the symmetry of , we have .

\subsection{A potential function for Markov flow graphs}

\iffalse
In this section is a symmetric Markov flow graph, representing a Markov process  with  and , and  for . Each layer has at most  vertices. Each vertex  has probability  of being reached in the Markov flow starting from , and each transition  has probability  of being traversed (i.e.\ capacity). We extend this notation and write, for any vertices , where , .
\fi

We will define  a potential function on the layers of , which will allow us to rephrase Theorem~\ref{thm:main} in more convenient terms. First, for any every layer  and vertex , let us define 
This function satisfies the following stochastic property:

\begin{lemma}\label{lem:A-stochastic} For and  and  we have 
\end{lemma}
\ifprocs\else
\begin{proof} By the Markov property, we have

\end{proof}
\fi

Now, for every layer , let us define the following potential function:

The following lemma show that this potential function is monotone decreasing in , and relates the decrease directly to the transitions in the Markov process:

\begin{lemma}\label{lem:potential} For all , we have
\ifprocs 
\else 
\fi
\end{lemma}
\ifprocs\else
\begin{proof}
By Lemma~\ref{lem:A-stochastic}, we have 
Therefore, we have

\end{proof}
\fi

Recall that we want to bound the possible flow from  to  by .  We may assume that   (i.e.\ ), since otherwise the bound is trivial. Note that, by symmetry, we have  and . Since we have only two sources and two sinks, this implies


Therefore, to prove Theorem~\ref{thm:flow-main} it suffices to show
\begin{lemma}\label{lem:flow-main} For every  there is some constant  such that for any  as above, with , there is a cut in  separating  from  of capacity at most .
\end{lemma}

Symmetry implies that  is even, and the case of  is fairly trivial. We will first consider the simpler case of , while the general case is shown in Section~\ref{sec:mainproof}.

\subsection{Treewidth 2}
Let us start with the case of , or . This shows some of the main ideas in the analysis for larger , while still being relatively simple. It is also an non-trivial special case, as it covers series-parallel graphs. A more careful analysis would yield a smaller constant C (we did not optimize).

\begin{lemma}\label{lem:flow-k=4} Lemma~\ref{lem:flow-main} holds for  and .
\end{lemma}
\ifprocs\else
\begin{proof} We may assume that . Otherwise, since the capacity of  is , the cost of simply cutting the outgoing edges of  is  Let us denote . Let us start by cutting all edges  for which . By Lemma~\ref{lem:potential}, the total capacity of these edges is at most


Let us examine the rest of the graph. By symmetry, and since  is monotone decreasing, every layer  must contain some vertex  such that , and a corresponding vertex  with . Consider the inner two vertices  (with ). Since there are no direct edges from  to  (the edge would be longer than ), the flow must travel along paths using the vertices . Since we have cut long edges, the  values of these paths must pass through the interval . Let  be the first such interval for which there is flow from  to . By a similar argument, any flow from  to  must pass through the interval  (before layer ). Let us take the last such layer, say  (it can be checked that all flow to  and  must pass through ). To cut all flow to , it suffices to remove vertex , or equivalently, to cut all outgoing edges from . Note that for all vertices  we have , since , , and . Hence, by Lemma~\ref{lem:potential}, the cost of cutting vertex  is at most


It is easy to see that any more flow from  to  must start at  for some layer , and so we can repeat the above argument, cutting vertices with  value in , and paying   each time for non-overlapping intervals , until we have severed all flow. Combining this with the cost incurred in~\eqref{eq:k=4edges}, we can bound the total capacity of edges cut by 
\end{proof}
\fi

To summarize the above approach, our cutting technique follows a two phase process. First, we cut all ``long" edges, which helps us isolate individual paths in the flow. Then, we isolate portions of the graph where the individual paths have a large shift in  value (e.g. move from the interval  to the interval ), and cut such paths by removing a single vertex, charging to the difference in potential along that portion of the graph.

There are a number of technical difficulties involved in extending this argument to work for larger . First, we cannot isolate specific intervals through which flow must pass in an isolated path, as these depend on the  values of other vertices in nearby layers. Secondly, even after cutting a path at some node, we are not guaranteed that there is no other path which routes flow around the node we cut. Rather than decompose the graph into isolated paths, we cut in several (roughly ) phases using a cut-and-cluster approach. Namely, after cutting, we ``cluster" together all vertices (or clusters from the previous phase) in a single layer that are close together in  value, and ensure that the number of clusters per layer which can contain flow from  to  decreases after every phase.

Unfortunately, our threshold for clustering vertices increases by roughly a  factor after every phase, thus ultimately incurring a loss which is exponential in  (or doubly-exponential in ). We note that while this does not match our  lower-bound, the lower-bound at least shows that we can not expect to get any ``reasonable" dependence on  (say, ) with our rounding.

\iffalse

\subsection{Performance guarantee for larger treewidth}

Let us begin with a simple lemma which was implicit in the analysis of Lemma~\ref{lem:flow-k=4}.

\begin{lemma}\label{lem:cut-charge} Let   be a sequence of non-overlapping intervals for integers  and let  be sets of nodes in layer  such that for every node  and every node  we have  for some . Then the cost of removing all  for every  (i.e.\ cutting all outgoing edges from ) is at most .
\end{lemma}

As we will use this lemma repeatedly, for a cut (set of edges) , we will call the value  the \emph{relative cost} of , where  is the total capacity of the edges in . Thus our goal will be to find a cut of constant relative cost.

Let us introduce some terminology and notation:

\begin{definition} For any , an -\emph{cluster} is a set of nodes belonging to a single layer, such that when ordered by their respective -values, every two consecutive nodes are at distance at most  from each other. For any cluster , we will denote  and , and we will refer to the value  as the \emph{width} of .
\end{definition}

Note that the width of any -cluster is at most .

\begin{definition} In the context of this section, the \emph{distance} between two nodes  will always refer to the value , which we will also call the  of  when  is an edge. For two non-overlapping clusters, we define the distance between them to be the minimum distance between two vertices, one in each cluster.
\end{definition}

\begin{definition} In a Markov flow graph as above, with some edges already cut, we will call a cluster  \emph{viable} if there is any capacity-respecting flow in the remaining graph from  to  which goes through at least one node of . For any clustering of the graph, we will define the \emph{clustered capacity} to be the maximum number of viable clusters per layer, over all layers.
\end{definition}

Let us now prove Lemma~\ref{lem:flow-main}

\begin{proof}[Proof of Lemma~\ref{lem:flow-main}]
Let us assume that  (recall that we've assumed ). Otherwise, simply cutting the outgoing edges of  yields a cut of relative cost .

As discussed earlier, we will proceed in  phases. In each phase, we will reduce the clustered capacity of the graph. Each phase will consist of first cutting some clusters (i.e.\ removing all outgoing edges from the nodes in these clusters), and then increasing the size of certain other clusters (by increasing the threshold for clustering).

We begin by first cutting all edges of length at least  (for some  to be determined soon). By Lemma~\ref{lem:cut-charge}, the relative cost of this cut is at most . At the end of each phase , we will cluster the vertices with clustering threshold , while we will require that . Thus we set . As we shall see, the relative cost of the cut at phase  will be  at most . Thus, the total relative cost of our cut will be at most 

Before describing and analyzing the individual phases, let us note that at phase  every cluster has width at most . Since we have cut all edges of length at least , for two clusters  in consecutive layers, there can be flow from  to  only if  and . In particular, when there is such flow, we have


We now proceed by induction on the clustered capacity. If the clustered capacity is 1, then there is a single ``path" of clusters from  to . Let  be the current phase . By our choice of , and by \eqref{eq:cluster-step}, the value  of any cluster in the path can increase or decrease by at most  at each step. Since this value starts at , and ends at , at some point it must pass through the interval . Call this layer , and the corresponding cluster . Now , and since the width of this cluster is at most , we also have . Thus all vertices in the cluster are at distance at least  from  and . Thus, by Lemma~\ref{lem:cut-charge}, the relative cost of cutting all flow along the path by removing  is at most 36.

Now suppose the clustered capacity is  for some , and let  denote the current phase. For any given layer, if the number of viable clusters is strictly less than , then we are done with that particular layer. Otherwise, if there are  viable clusters in a layer and any two of them are at distance at most  from each other, then again we are done with that layer, since at the end of the phase the two clusters will be merged (possibly along with additional clusters) into a single -cluster. Thus, we only need to reduce the number of viable clusters in layers which contain  distinct viable clusters whose pairwise distances are all greater than .

Let us denote the first such layer by , and the viable clusters by  in increasing order of  values. Note that any viable cluster  must have a corresponding viable cluster  in the subsequent layer satisfying~\eqref{eq:cluster-step}. Moreover, for any two -clusters  such that  with flow into clusters , respectively, in the subsequent layer, there cannot be any flow from  to  or from  to . Indeed, the distance from, say,   to  is greater than . In particular, for any layer containing  viable clusters with pairwise distance greater than , each cluster must have flow into exactly one cluster in the subsequent layer (there cannot be more, since no layer contains more than  viable clusters in this phase).

Let us denote by  the first layer after  in which some pair of adjacent clusters  are at distance at most . By the above argument, all viable flow in the portion of the graph from  to  flows through  disjoint cluster-paths  (for ). Therefore, to reduce the number of viable clusters in all layers , it suffices to cut just one cluster  for some  and some . Consider the pair of clusters . Since these clusters are at distance at most  and the corresponding clusters  are at distance at least , either , or . Without loss of generality, suppose the former.

Now consider the open real interval . It has length at least , and contains at most  values in  (at least one vertex  in layer  has ). Therefore there is an open subinterval  of length at least  containing none of these values. By~\eqref{eq:cluster-step}, there must be some layer  (for some ) for which . Since  has width at most , it must also satisfy . In particular, all vertices in  are at distance at least  from all vertices in layer . We can now repeat this argument for layers , until we have exhausted all layers in the graph, and by Lemma~\ref{lem:cut-charge}, the relative cost of the cut will be at most , as required.
\end{proof}

\fi

