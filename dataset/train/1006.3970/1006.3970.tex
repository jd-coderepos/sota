In this section we prove Theorem~\ref{thm:flow-main} for $k=4$ (the proof for the general case appears in Section~\ref{sec:mainproof}). For Markov flow graph $G=(L_0,\ldots,L_N,E)$ and corresponding Markov process $X_0,\ldots,X_N$ as in the theorem, for any integers $0\leq l_1\leq l_2\leq N$, and any vertices $u\in L_{l_1},v\in L_{l_2}$ we will let $p(u)=\prob[X_{l_1}=u]$ be the probability of reaching $u$, and similarly, we define $p(u,v)=\prob[X_{l_1}=u\wedge X_{l_2}=v]$. In particular, when $l_2=l_1+1$ and $\overrightarrow{(u,v)}$ is an edge (transition) then $p(u,v)$ is also the capacity of this edge. Note that, by the symmetry of $G$, we have $p(s_0)=p(s_1)=\frac12$.

\subsection{A potential function for Markov flow graphs}

\iffalse
In this section$G=(L_0,\ldots,L_N,E)$ is a symmetric Markov flow graph, representing a Markov process $X_0,\ldots,X_N$ with $L_0=\{s_0,s_1\}$ and $L_N=\{t_0,t_1\}$, and $L_l=\{v^l_i\}$ for $l=1,\ldots,t-1$. Each layer has at most $k$ vertices. Each vertex $v$ has probability $p(v)$ of being reached in the Markov flow starting from $p(s_0)=p(s_1)=\frac12$, and each transition $\overrightarrow{(u,v)}$ has probability $p(u,v)$ of being traversed (i.e.\ capacity). We extend this notation and write, for any vertices $u\in L_{l_1}, v\in L_{l_2}$, where $l_1<l_2$, $p(u,v)=\prob[X_{l_1}=u\wedge X_{l_2}=v]$.
\fi

We will define  a potential function on the layers of $G$, which will allow us to rephrase Theorem~\ref{thm:main} in more convenient terms. First, for any every layer $l$ and vertex $v\in L_l$, let us define $$A(v)=\prob[X_0=s_0\mid X_l=v]-\textstyle\frac12.$$
This function satisfies the following stochastic property:

\begin{lemma}\label{lem:A-stochastic} For and $0<l_1<l_2$ and $v\in L_{l_2}$ we have $$A(v)=\frac{\sum_{u\in L_{l_1}}p(u,v)A(u)}{\sum_{u\in L_{l_1}}p(u,v)}.$$
\end{lemma}
\ifprocs\else
\begin{proof} By the Markov property, we have
\begin{align*}
A(v)+{\textstyle\frac12}=\prob[X_0=s_0\mid X_{l_2}=v]=p(s_0,v)/p(v)&={\textstyle\frac1{p(v)}}{\sum_{u\in L_{l_2}}(p(u,v)/p(u))p(s_0,u)}\\
&={\textstyle\frac1{p(v)}}{\sum_{u\in L_{l_2}}p(u,v)(A(u)+{\textstyle\frac12})}\\
&=\frac{\sum_{u\in L_{l_2}}p(u,v)(A(u)+{\textstyle\frac12})}{\sum_{u\in L_{l_2}}p(u,v)}.
\end{align*}
\end{proof}
\fi

Now, for every layer $l=0,\ldots, N$, let us define the following potential function:
$$\varphi(l)=\var[A(X_l)]={\textstyle\sum}_{v\in L_l}p(v)A(v)^2-\left({\textstyle\sum}_{v\in L_l}p(v)A(v)\right)^2.$$
The following lemma show that this potential function is monotone decreasing in $l$, and relates the decrease directly to the transitions in the Markov process:

\begin{lemma}\label{lem:potential} For all $0<l_1<l_2$, we have
\ifprocs $$\varphi(l_1)-\varphi(l_2)=\displaystyle\sum_{u\in L_{l_1},v\in L_{l_2}}p(u,v)(A(u)-A(v))^2.$$
\else $\varphi(l_1)-\varphi(l_2)=\displaystyle\sum_{u\in L_{l_1},v\in L_{l_2}}p(u,v)(A(u)-A(v))^2.$
\fi
\end{lemma}
\ifprocs\else
\begin{proof}
By Lemma~\ref{lem:A-stochastic}, we have $$\sum_{v\in L_{l_2}}p(v)A(v)=\sum_{v\in L_{l_2}}\sum_{u\in L_{l_1}}p(u,v)A(u)=\sum_{u\in L_{l_1}}(\sum_{v\in L_{l_2}}p(u,v))A(u)=\sum_{u\in L_{l_1}}p(u)A(u).$$
Therefore, we have
\begin{align*}
\varphi(l_1)-\varphi(l_2)&=\sum_{u\in L_{l_1}}p(u)A(u)^2-\sum_{v\in L_{l_2}}p(v)A(v)^2\\
&=\sum_{u}\sum_{v}p(u,v)A(u)^2-\sum_{v}p(v)A(v)^2&\text{since }p(u)=\sum_vp(u,v)\\
&=\sum_{u}\sum_{v}p(u,v)\left(A(u)^2+A(v)^2\right)-2\sum_{v}p(v)A(v)^2&\text{since }p(v)=\sum_up(u,v)\\
&=\sum_{v}\sum_{u}p(u,v)\left(A(u)^2+A(v)^2-2A(u)A(v))\right).&\text{by Lemma~\ref{lem:A-stochastic}}
\end{align*}
\end{proof}
\fi

Recall that we want to bound the possible flow from $s_0$ to $t_1$ by $O(p(s_0,t_1))$.  We may assume that  $p(s_0,t_1)<\frac14$ (i.e.\ $A(t_1)<0$), since otherwise the bound is trivial. Note that, by symmetry, we have $A(s_0)=-A(s_1)$ and $A(t_0)=-A(t_1)$. Since we have only two sources and two sinks, this implies
\begin{align*}\varphi(0)-\varphi(N)=A(s_0)^2-A(t_1)^2={\textstyle\frac14}-A(t_1)^2={\textstyle\frac14}-(2p(s_0,t_1)-{\textstyle\frac12})^2\leq2p(s_0,t_1).
\end{align*}

Therefore, to prove Theorem~\ref{thm:flow-main} it suffices to show
\begin{lemma}\label{lem:flow-main} For every $k>0$ there is some constant $C=C(k)$ such that for any $G$ as above, with $A(t_1)<0$, there is a cut in $G$ separating $s_0$ from $t_1$ of capacity at most $C\cdot(\varphi(0)-\varphi(N))$.
\end{lemma}

Symmetry implies that $k$ is even, and the case of $k=2$ is fairly trivial. We will first consider the simpler case of $k=4$, while the general case is shown in Section~\ref{sec:mainproof}.

\subsection{Treewidth 2}
Let us start with the case of $k=4$, or $r=2$. This shows some of the main ideas in the analysis for larger $k$, while still being relatively simple. It is also an non-trivial special case, as it covers series-parallel graphs. A more careful analysis would yield a smaller constant C (we did not optimize).

\begin{lemma}\label{lem:flow-k=4} Lemma~\ref{lem:flow-main} holds for $k=4$ and $C=100$.
\end{lemma}
\ifprocs\else
\begin{proof} We may assume that $\varphi(N)\geq\frac{49}{200}$. Otherwise, since the capacity of $s_0$ is $\frac12$, the cost of simply cutting the outgoing edges of $s_0$ is ${\textstyle\frac12}=C/200\leq C(\varphi(0)-\varphi(N)).$ Let us denote $A^*=A(t_0)=\sqrt{\varphi(N)}(\geq\frac7{10\sqrt2})$. Let us start by cutting all edges $(u,v)$ for which $|A(u)-A(v)|\geq\frac27A^*$. By Lemma~\ref{lem:potential}, the total capacity of these edges is at most
\begin{equation}\label{eq:k=4edges}
\begin{split}
\sum_{l=1}^{t}\sum_{\substack{u\in L_{l-1},v_{L_l}\\|A(u)-A(v)|\geq\frac27A^*}}p(u,v)
&\leq({\textstyle\frac{7}{2A^*}})^2\sum_{l=1}^{t}\sum_{\substack{u\in L_{l-1},v_{L_l}\\|A(u)-A(v)|\geq\frac27A^*}}p(u,v)(A(u)-A(v))^2\\
&\leq({\textstyle\frac{7}{2A^*}})^2\sum_{l=1}^{t}(\varphi(l-1)-\varphi(l))=({\textstyle\frac{7}{2A^*}})^2(\varphi(0)-\varphi(N)).
\end{split}
\end{equation}

Let us examine the rest of the graph. By symmetry, and since $\varphi$ is monotone decreasing, every layer $l$ must contain some vertex $v_l$ such that $A(v_l)>A(t_0)=A^*$, and a corresponding vertex $\tilde{v_l}$ with $A(\tilde{v_l})=-A(v_l)$. Consider the inner two vertices $u_l,\tilde u_l$ (with $A(v_l)\geq A(u_l)=-A(\tilde u_l)\geq0$). Since there are no direct edges from $v_{l-1}$ to $\tilde v_l$ (the edge would be longer than $\frac27A^*$), the flow must travel along paths using the vertices $\{u_l,\tilde u_l\}_l$. Since we have cut long edges, the $A(\cdot)$ values of these paths must pass through the interval $[-\frac17A^*,\frac17A^*]$. Let $l_1$ be the first such interval for which there is flow from $s_0$ to $u_{l_1}$. By a similar argument, any flow from $s_0$ to $u_{l_1}$ must pass through the interval $[\frac37A^*,\frac57A^*]$ (before layer $l_1$). Let us take the last such layer, say $l_0$ (it can be checked that all flow to $u_{l_1}$ and $\tilde u_{l_1}$ must pass through $u_{l_0}$). To cut all flow to $u_{l_1},\tilde u_{l_1}$, it suffices to remove vertex $u_{l_0}$, or equivalently, to cut all outgoing edges from $u_{l_0}$. Note that for all vertices $w\in L_{l_1}$ we have $|A(w)-A(u_{l_0})|\geq\frac27A^*$, since $A(u_{l_0})\in [\frac37A^*,\frac57A^*]$, $A(v_{l_1})>A^*$, and $A(\tilde v_{l_1})\leq A(\tilde u_{l_1})\leq A(u_{l_1})\leq \frac17A^*$. Hence, by Lemma~\ref{lem:potential}, the cost of cutting vertex $u_{l_0}$ is at most
\begin{equation}
\begin{split}
p(u_{l_0})=\sum_{w\in L_{l_1}}p(u_{l_0},w)
&\leq({\textstyle\frac{7}{2A^*}})^2\sum_{w\in L_{l_1}}p(u_{l_0},w)(A(u_{l_0})-A(w))^2\\
&\leq({\textstyle\frac{7}{2A^*}})^2(\varphi(l_0)-\varphi(l_1)).
\end{split}
\end{equation}

It is easy to see that any more flow from $s_0$ to $t_1$ must start at $v_{l_2}$ for some layer $l_2\geq l_1$, and so we can repeat the above argument, cutting vertices with $A(\cdot)$ value in $[\frac37A^*,\frac57]$, and paying $(\frac7{2A^*})^2(\varphi(l_i)-\varphi(l_{i+1}))$  each time for non-overlapping intervals $[l_0,l_1],[l_2,l_3],\ldots,[l_m,l_{m+1}]$, until we have severed all flow. Combining this with the cost incurred in~\eqref{eq:k=4edges}, we can bound the total capacity of edges cut by $2({\textstyle\frac7{2A^*}})^2(\varphi(0)-\varphi(N))\leq100(\varphi(0)-\varphi(N)).$
\end{proof}
\fi

To summarize the above approach, our cutting technique follows a two phase process. First, we cut all ``long" edges, which helps us isolate individual paths in the flow. Then, we isolate portions of the graph where the individual paths have a large shift in $A(\cdot)$ value (e.g. move from the interval $[\frac37A^*,\frac57A^*]$ to the interval $[-\frac17A^*,\frac17A^*]$), and cut such paths by removing a single vertex, charging to the difference in potential along that portion of the graph.

There are a number of technical difficulties involved in extending this argument to work for larger $k$. First, we cannot isolate specific intervals through which flow must pass in an isolated path, as these depend on the $A(\cdot)$ values of other vertices in nearby layers. Secondly, even after cutting a path at some node, we are not guaranteed that there is no other path which routes flow around the node we cut. Rather than decompose the graph into isolated paths, we cut in several (roughly $k$) phases using a cut-and-cluster approach. Namely, after cutting, we ``cluster" together all vertices (or clusters from the previous phase) in a single layer that are close together in $A(\cdot)$ value, and ensure that the number of clusters per layer which can contain flow from $s_0$ to $t_1$ decreases after every phase.

Unfortunately, our threshold for clustering vertices increases by roughly a $k^2$ factor after every phase, thus ultimately incurring a loss which is exponential in $k$ (or doubly-exponential in $r$). We note that while this does not match our $\Omega(k)(=\Omega(2^r))$ lower-bound, the lower-bound at least shows that we can not expect to get any ``reasonable" dependence on $r$ (say, $O(\log r)$) with our rounding.

\iffalse

\subsection{Performance guarantee for larger treewidth}

Let us begin with a simple lemma which was implicit in the analysis of Lemma~\ref{lem:flow-k=4}.

\begin{lemma}\label{lem:cut-charge} Let  $\{[l^0_j,l^1_j]\}$ be a sequence of non-overlapping intervals for integers $0<l^0_j<l^1_j<n$ and let $W_j\subseteq L_{l^0_j}$ be sets of nodes in layer $L_{l^0_j}$ such that for every node $w\in W_j$ and every node $x\in L_{l^1_j}$ we have $|A(w)-A(x)|\geq\rho$ for some $\rho>0$. Then the cost of removing all $w\in W_j$ for every $j$ (i.e.\ cutting all outgoing edges from $w$) is at most $\frac{1}{\rho^2}(\varphi(0)-\varphi(N))$.
\end{lemma}

As we will use this lemma repeatedly, for a cut (set of edges) $T$, we will call the value $p(T)/(\varphi(0)-\varphi(N))$ the \emph{relative cost} of $T$, where $p(T)$ is the total capacity of the edges in $T$. Thus our goal will be to find a cut of constant relative cost.

Let us introduce some terminology and notation:

\begin{definition} For any $\eps>0$, an $\eps$-\emph{cluster} is a set of nodes belonging to a single layer, such that when ordered by their respective $A(\cdot)$-values, every two consecutive nodes are at distance at most $\eps$ from each other. For any cluster $X$, we will denote $A^+(X)=\max_{v\in X}A(v)$ and $A^-(X)=\min_{v\in X}A(v)$, and we will refer to the value $A^+(X)-A^-(X)$ as the \emph{width} of $X$.
\end{definition}

Note that the width of any $\eps$-cluster is at most $(k-1)\eps$.

\begin{definition} In the context of this section, the \emph{distance} between two nodes $u,v$ will always refer to the value $|A(u)-A(v)|$, which we will also call the $\emph{length}$ of $(u,v)$ when $(u,v)$ is an edge. For two non-overlapping clusters, we define the distance between them to be the minimum distance between two vertices, one in each cluster.
\end{definition}

\begin{definition} In a Markov flow graph as above, with some edges already cut, we will call a cluster $X$ \emph{viable} if there is any capacity-respecting flow in the remaining graph from $s_0$ to $t_1$ which goes through at least one node of $X$. For any clustering of the graph, we will define the \emph{clustered capacity} to be the maximum number of viable clusters per layer, over all layers.
\end{definition}

Let us now prove Lemma~\ref{lem:flow-main}

\begin{proof}[Proof of Lemma~\ref{lem:flow-main}]
Let us assume that $A(t_1)<-\frac13$ (recall that we've assumed $A(t_1)\leq0$). Otherwise, simply cutting the outgoing edges of $s_0$ yields a cut of relative cost $\frac12/(\varphi(0)-\varphi(N))\geq\frac{18}5$.

As discussed earlier, we will proceed in $k$ phases. In each phase, we will reduce the clustered capacity of the graph. Each phase will consist of first cutting some clusters (i.e.\ removing all outgoing edges from the nodes in these clusters), and then increasing the size of certain other clusters (by increasing the threshold for clustering).

We begin by first cutting all edges of length at least $\eps_0$ (for some $\eps_0>0$ to be determined soon). By Lemma~\ref{lem:cut-charge}, the relative cost of this cut is at most $1/\eps_0^2$. At the end of each phase $j$, we will cluster the vertices with clustering threshold $\eps_j=(12k^2)^j\eps_0$, while we will require that $k\eps_{k-1}\leq\frac16$. Thus we set $\eps_0=1/(6k(12k^2)^{k-1})$. As we shall see, the relative cost of the cut at phase $j$ will be  at most $1/(k\eps_{j-1})^2$. Thus, the total relative cost of our cut will be at most $$\frac1{\eps_0^2}+\sum_{j>0}\frac1{k\eps_{j-1}^2}=\frac1{\eps_0^2}\left(1+\sum_{j>0}\frac1{144^{j-1}k^{4j-2}}\right)=O\left(\frac1{\eps_0^2}\right)=k^{O(k)}.$$

Before describing and analyzing the individual phases, let us note that at phase $j$ every cluster has width at most $(k-1)\eps_{j-1}$. Since we have cut all edges of length at least $\eps_0$, for two clusters $X_1,X_2$ in consecutive layers, there can be flow from $X_1$ to $X_2$ only if $A^-(X_2)-A^+(X_1)<\eps_0$ and $A^-(X_1)-A^+(X_2)<\eps_0$. In particular, when there is such flow, we have
\begin{equation}\label{eq:cluster-step}
\max\{|A^+(X_2)-A^+(X_1)|,|A^-(X_2)-A^-(X_1)|\}\leq (k-1)\eps_{j-1}+\eps_0\leq k\eps_{j-1}.
\end{equation}

We now proceed by induction on the clustered capacity. If the clustered capacity is 1, then there is a single ``path" of clusters from $s_0$ to $t_1$. Let $j$ be the current phase $(1\leq j\leq k-1)$. By our choice of $\eps_j$, and by \eqref{eq:cluster-step}, the value $|A^+(X)|$ of any cluster in the path can increase or decrease by at most $k\eps_{k-1}\leq\frac16$ at each step. Since this value starts at $A(s_0)=\frac12$, and ends at $A(t_1)<-\frac13$, at some point it must pass through the interval $[0,\frac16]$. Call this layer $l_1$, and the corresponding cluster $X_{l_1}$. Now $A^+(X_{l+1})\leq\frac16$, and since the width of this cluster is at most $\frac16$, we also have $A^-(X_{l+1})\geq-\frac16$. Thus all vertices in the cluster are at distance at least $\frac16$ from $t_0$ and $t_1$. Thus, by Lemma~\ref{lem:cut-charge}, the relative cost of cutting all flow along the path by removing $X_{l_1}$ is at most 36.

Now suppose the clustered capacity is $k'$ for some $1<k'\leq k$, and let $j$ denote the current phase. For any given layer, if the number of viable clusters is strictly less than $k'$, then we are done with that particular layer. Otherwise, if there are $k'$ viable clusters in a layer and any two of them are at distance at most $\eps_j$ from each other, then again we are done with that layer, since at the end of the phase the two clusters will be merged (possibly along with additional clusters) into a single $\eps_j$-cluster. Thus, we only need to reduce the number of viable clusters in layers which contain $k'$ distinct viable clusters whose pairwise distances are all greater than $\eps_j$.

Let us denote the first such layer by $l_1$, and the viable clusters by $X^{l_1}_1,\ldots,X^{l_1}_{k'}$ in increasing order of $A(\cdot)$ values. Note that any viable cluster $X_1$ must have a corresponding viable cluster $X'_1$ in the subsequent layer satisfying~\eqref{eq:cluster-step}. Moreover, for any two $\eps_{j-1}$-clusters $X_1,X_2$ such that $A^+(X_1)<A^-(X_2)-\frac{\eps_j}{3}$ with flow into clusters $X'_1,X'_2$, respectively, in the subsequent layer, there cannot be any flow from $X_1$ to $X'_2$ or from $X_2$ to $X'_1$. Indeed, the distance from, say,  $X_1$ to $X'_2$ is greater than $\frac{\eps_j}3-k\eps_{j-1}>\eps_{j-1}>\eps_0$. In particular, for any layer containing $k'$ viable clusters with pairwise distance greater than $\frac{\eps_{j-1}}3$, each cluster must have flow into exactly one cluster in the subsequent layer (there cannot be more, since no layer contains more than $k'$ viable clusters in this phase).

Let us denote by $l_2$ the first layer after $l_1$ in which some pair of adjacent clusters $X^{l_2}_{i'},X^{l_2}_{i'+1}$ are at distance at most $\frac{\eps_j}3$. By the above argument, all viable flow in the portion of the graph from $L_{l_1}$ to $L_{l_2}$ flows through $k'$ disjoint cluster-paths $X^{l_1}_i\rightarrow X^{l_1+1}_i\rightarrow\ldots\rightarrow X^{l_2+1}$ (for $i=1,\ldots,k'$). Therefore, to reduce the number of viable clusters in all layers $L_{l_1},\ldots,L_{l_2}$, it suffices to cut just one cluster $X^l_i$ for some $i\in\{1,\ldots,k'\}$ and some $l_1<l<l_2$. Consider the pair of clusters $X^{l_2}_{i'},X^{l_2}_{i'+1}$. Since these clusters are at distance at most $\frac{\eps_j}3$ and the corresponding clusters $X^{l_1}_{i'},X^{l_1}_{i'+1}$ are at distance at least $\eps_j$, either $A^+(X^{l_2}_{i'})-A^+(X^{l_1}_{i'})\geq\frac{\eps_j}3$, or $A^-(X^{l_1}_{i'+1})-A^+(X^{l_2}_{i'+1})\geq\frac{\eps_j}3$. Without loss of generality, suppose the former.

Now consider the open real interval $(A^+(X^{l_1}_{i'}),A^+(X^{l_2}_{i'}))$. It has length at least $\frac{\eps_j}3$, and contains at most $k-1$ values in $\{A(v)\mid v\in L_{l_2}\}$ (at least one vertex $v$ in layer $L_{l_2}$ has $A(v)=A^+(X^{l_2}_{i'})$). Therefore there is an open subinterval $(a_0,a_1)$ of length at least $\frac{\eps_j}{3k}=4k\eps_{j-1}$ containing none of these values. By~\eqref{eq:cluster-step}, there must be some layer $L_l$ (for some $l_1<l<l_2$) for which $A^+(X^l_{i'})\in(a_0+2k\eps_{j-1},a_0+3k\eps_{j-1})$. Since $X^l_{i'}$ has width at most $k\eps_{j-1}$, it must also satisfy $A^-(X^l_{i'})\in(a_0+k\eps_{j-1},a_0+3k\eps_{j-1})$. In particular, all vertices in $X^l_{i'}$ are at distance at least $k\eps_{j-1}$ from all vertices in layer $L_{l_2}$. We can now repeat this argument for layers $>l_2$, until we have exhausted all layers in the graph, and by Lemma~\ref{lem:cut-charge}, the relative cost of the cut will be at most $1/(k\eps_{j-1})^2$, as required.
\end{proof}

\fi

