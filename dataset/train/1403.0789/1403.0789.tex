\documentclass{llncs}



\newcommand{\knip}[1]{}
\pagestyle{plain}
\usepackage{url}

\usepackage{graphicx}
\usepackage{boxedminipage}

\newtheorem{observation}{Observation}

\newcommand{\ig}[2][scale=1.0]{\includegraphics[#1]{#2}}

\newcommand{\sP}{{\sf P}}
\newcommand{\NP}{{\sf NP}}
\newcommand{\W}{\ensuremath{\mathsf{W}}}
\newcommand{\FPT}{{\sf FPT}}
\newcommand{\XP}{{\sf XP}}

\newcommand{\etal}{et al.~}
\newcommand{\etaln}{et al.}
\newcommand{\ie}{i.e.~}
\newcommand{\eg}{e.g.~}
\newcommand{\wloge}{w.l.o.g.~}
\newcommand{\mc}{\mathcal}
\newcommand{\poly}{\mathrm{poly}}
\newcommand{\nca}{\mathrm{nca}}
\newcommand{\dist}{\mathrm{dist}}
\newcommand{\ds}[1]{\gamma(#1)}
\newcommand{\cds}{\gamma_{c}}
\newcommand{\vc}{\tau}
\newcommand{\eds}[1]{\rho(#1)}
\newcommand{\mm}{\nu}
\newcommand{\binom}[2]{{#1 \choose #2}}
\newcommand{\marge}[1]{}

\newcommand{\inc}{\textrm{inc}} 
\newcommand{\adj}{\textrm{adj}} 

\DeclareSymbolFont{AMSb}{U}{msb}{m}{n}
\DeclareSymbolFontAlphabet{\mathbb}{AMSb}

\newcommand{\problemIDP}{\textsc{Induced Disjoint Paths}}
\newcommand{\problemRIDP}{\textsc{Requirement Induced Disjoint Paths}}
\newcommand{\problemDP}{\textsc{Disjoint Paths}}
\newcommand{\problemMCIS}{\textsc{Multicolored Independent Set}}


\begin{document}

\title{Induced Disjoint Paths in\\ Circular-Arc Graphs in Linear Time
\thanks{This work is supported by EPSRC (EP/K025090/1) and Royal Society (JP100692).
The research leading to these results has also received funding from the European Research Council under the European Union's Seventh Framework Programme (FP/2007-2013) / ERC Grant Agreement n.\ 267959.}
}
\author{
Petr A.~Golovach\inst{1}
\and Dani\"el Paulusma\inst{2}
\and Erik Jan van Leeuwen\inst{3}
\institute{Department of informatics, University of Bergen, Norway\\
\email{petr.golovach@ii.uib.no}
\and
School of Engineering and Computer Science, Durham University, UK\\ 
\email{daniel.paulusma@durham.ac.uk}
\and 
Max-Planck Institut f\"{u}r Informatik, Saarbr\"{u}cken, Germany,
\email{erikjan@mpi-inf.mpg.de}
}}
\maketitle
\thispagestyle{plain}
\setcounter{footnote}{0}

\begin{abstract}
The \problemIDP{} problem is to test whether a graph  with  distinct pairs of vertices  contains paths  such that  connects  and  for , and  and  have neither common vertices nor adjacent vertices (except perhaps their ends) for .
We present a linear-time algorithm for \problemIDP{} on circular-arc graphs.
For interval graphs, we exhibit a linear-time algorithm for the generalization of \problemIDP{} where the pairs  are not necessarily distinct.
\end{abstract}

\section{Introduction}\label{s-intro}
A classic algorithmic problem on a graph  with  distinct pairs of vertices  is to find vertex-disjoint
\footnote{There is also a version of the problem in which the paths are required to be edge-disjoint. We do not consider that version in this paper.} 
paths  such that  connects  and . Known as the \problemDP{} problem, it is \NP-complete on general graphs~\cite{Ka75}, but can be solved in  time for any fixed integer ~\cite{RS95} (\ie it is fixed-parameter tractable). A generalization of this problem is \problemIDP{}: given  distinct pairs of vertices  in a graph , find paths  such that  connects  and  for  and the paths are \emph{mutually induced}, that is, no two paths  have common or adjacent vertices (except perhaps their end-vertices). The \problemIDP{} problem indeed generalizes the \problemDP{} problem, since the latter can be reduced to the former by subdividing every edge of the graph. This makes the problem much harder: \problemIDP{} is \NP-complete even for instances with ~\cite{Bi91,Fe89}, and thus in particular is not fixed-parameter tractable unless \sP=\NP.

The hardness of both \problemDP{} and \problemIDP{} on general graphs inspired research on their complexity on structured graph classes. On the negative side, \problemDP{} remains \NP-complete on line graphs~\cite{Ly75} and split graphs~\cite{HvtHSvL2014}, \problemIDP{} remains \NP-complete on claw-free graphs~\cite{FKLP12}, and both problems remain \NP-complete on planar graphs~\cite{KvL84,GPV12}. In these cases, however, fixed-parameter algorithms are known~\cite{GPV12b,HvtHSvL2014,KK12,RRSS93,RS95}. On the positive side, polynomial-time algorithms for \problemDP{} exist on graphs of bounded treewidth~\cite{Reed1997} and graphs of cliquewidth at most~~\cite{GurWan2006}, and for \problemIDP{} on AT-free graphs~\cite{GPV12} and chordal graphs~\cite{BGHHKP12}.

We focus on the complexity of \problemIDP{} on circular-arc graphs. Recall that a \emph{circular-arc graph}  has a \emph{representation} in which each vertex of  corresponds to an arc of a circle, and two vertices of  are adjacent if and only if their corresponding arcs intersect. Circular-arc graphs generalize \emph{interval graphs}, which have a representation in which each vertex corresponds to an interval of the line, and two vertices are adjacent if and only if their corresponding intervals intersect. The complexity of \problemDP{} is known: it is \NP-complete already on interval graphs~\cite{NS1996}. In contrast, for \problemIDP{}, the authors of the present work recently showed a polynomial-time algorithm on circular-arc graphs~\cite{GPV12b}, and a polynomial-time algorithm on interval graphs is implied by that work, as well as by the polynomial-time algorithms on AT-free graphs~\cite{GPV12} and chordal graphs~\cite{BGHHKP12}. These algorithms, however, do not fully settle the complexity of \problemIDP{} on circular-arc graphs (and interval graphs), because the question whether a linear-time algorithm exists has been left open.

In this paper, we exhibit a linear-time algorithm for \problemIDP{} on circular-arc graphs. This improves on the known algorithm on circular-arc graphs as well as the known algorithms for interval graphs. We also introduce a generalization of \problemIDP{} called \problemRIDP{}, which is to find  paths that connect  and  for , such that all paths are mutually induced. We present a linear-time algorithm for \problemRIDP{} on interval graphs. To solve these problems, our algorithms first preprocesses the instance. Some of the preprocessing rules build on our earlier work on \problemIDP{}~\cite{GPV12,GPV12b}, but special care is required to adapt them for \problemRIDP{} and to execute them in linear time. Most preprocessing rules, however, are novel. After the preprocessing stage, the algorithms identify a set of candidate paths for each pair . For each candidate path for a pair , we add an arc with color  that corresponds to the path to an auxiliary graph. Finally, we show that it suffices to find an independent set in this auxiliary graph that contains  arcs of each color. We show that the algorithms perform all stages in linear time.

\section{Preliminaries}\label{s-pre}
We only consider finite undirected graphs that have no loops and no multiple edges. 
We refer to the textbook of Diestel~\cite{Di05} for any standard graph terminology not defined here. Let  be a graph. 
For a set , the graph  denotes the subgraph of  {\it induced by} , that is, the graph with vertex set  and edge set .  We write . 
We denote the (open) neighborhood of a vertex  by  and its closed neighborhood by .
We denote the neighborhood of a set  by  and . 
We denote the degree of a vertex  by .

We denote an unordered pair of elements  by   (\ie ).

\paragraph{Problem Definition}
Let  be a path (we call such a path a {\em -path}).
The vertices  and   are the {\it ends} or {\it end-vertices} of , and the vertices  are the \emph{inner vertices} of .
We say that an edge , , is an \emph{inner chord} of  if  or  is an inner vertex of .
Distinct paths  in a graph   are {\it mutually induced} if:
\begin{itemize}
\item [(i)] each  has no inner chords;
\item [(ii)]  any distinct  may only share vertices that are ends of both paths;
\item [(iii)]  no inner vertex  of any  is adjacent to a vertex  of some  for , except when  is an end-vertex of both  and .
\end{itemize}
Notice that condition~(i) may be assumed without loss of generality. This definition is more general than the definition in Section~\ref{s-intro}, as it allows the end-vertices of distinct paths to be the same or adjacent. 
We can now formally state our decision problem (where a {\em terminal} is some specified vertex).
\begin{center}
\begin{boxedminipage}{1\textwidth}
\ \problemRIDP{}\3mm]
\begin{tabular}{r l }
\textit{Instance:} & a graph , an integer , and a function \\
\textit{Question:} & does  have an independent set  with ?  \\
\end{tabular}
\end{boxedminipage}
\end{center}
In Step~10, we essentially show that such an instance can be solved in polynomial time on interval graphs if for any two vertices  with  there is no vertex  with  and . However, on general interval graphs, this problem becomes NP-complete.

\begin{theorem}
\problemMCIS{} on interval graphs is NP-complete.
\end{theorem}
\begin{proof}
We show in fact that the problem is already NP-complete on disjoint unions of double stars (\ie graphs obtained from two disjoint stars by joining the central vertices), which form a subclass of interval graphs. We reduce from \textsc{3-SAT}. Consider an instance of \textsc{3-SAT} with  variables  and  clauses . We construct a graph  and a function  as follows. For each , we create two adjacent vertices  and  with . For each , we create three vertices and set  of these vertices to . We then make these three vertices adjacent to the corresponding literal vertices (for example, if  contains , then we join the first vertex with the vertex , the second with  and the third with ). This completes the construction. Note that it is indeed a disjoint union of double stars. The correctness can be seen as follows: we set  to true if and only if the vertex  is not in the independent set.
\qed\end{proof}
It is easy to show that \problemMCIS{} is fixed-parameter tractable on interval graphs: guess an ordering of the colors, and for each choice, run a procedure similar to the one described for Step~10. A faster algorithm can be obtained using dynamic programming.
\end{document}