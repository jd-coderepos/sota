\ourprotocolNSP~is a novel framework allowing clouds to provide privacy preserving photo sharing and searching services to the mobile devices users.
It can attract users who need both outsourced photo service and privacy protection.
Figure~\ref{fig:system_overview} illustrates the architecture and workflow of our framework\footnote[2]{We
  logically divide the cloud into \textit{sharing cloud} and
  \textit{search cloud} for explanation purpose, but revealing
  this structure does not breach users' privacy at
  all.}. With this framework design, \ourprotocolNSP~ can provide the following services:
  (1)privacy-preserving photo storage outsourcing;
  (2)fine-grained photo sharing with privacy protection enforcement;
  (3)light-weight photo searching for mobile devices.
\begin{figure}[h]
  \centering
  \subfigure[Photo Owner Side]{\label{fig:system_overview-a}
    \includegraphics[width=0.8\linewidth, clip,keepaspectratio]{ownerupload.eps}}
  \subfigure[Querier Side]{\label{fig:system_overview-b}
   \includegraphics[width=0.8\linewidth, clip,keepaspectratio]{search.eps}}\vspace{-5pt}
  \caption{\ourprotocol System Overview}\vspace{-10pt}
  \label{fig:system_overview}
\end{figure}













\vspace{-0.05in}
\subsection{Privacy Preserving Photo Storage}\label{sec:privacy_storage}
\vspace{-0.05in}

Before users upload their photos to cloud
servers for sharing, the photos need to be pre-processed.
Firstly, the region of privacy (ROP), which is a rectangle defined by two pixel-level
coordinates (top-left and bottom-right) on the photo, is either automatically or manually determined.
In the automatic manner, the user can define a category of objects as private content, \eg, faces.
Then all private objects will be detected by object recognition algorithm and set as ROPs, \eg, the face in Fig.~\ref{fig:mask}.
Otherwise, the owner can also manually define the ROP by selecting a rectangle region on the photo.
Then, the feature vectors of the ROP are extracted according to the
definition of the image descriptor (Section \ref{section:image_descriptor}).
Note that, hereafter we use the human faces as example ROPs of photos in this work,
 but other objects such as pedestrians and cars, can also be defined as ROPs with
corresponding recognition algorithms.
Moreover, except defining ROP, all the following operations are conducted automatically
by the system and transparent to users.



After the ROP is selected, it is separated into public part and secret
part, where the public part doesn't contain any sensitive information
and the secret part is encrypted such that only the authorized
users with keys can access to it and recover the original ROP. We review
the following three different methods for the separation:

\noindent 1. \textbf{Mask}: fills public part of ROP with solid black (all intensity values `0') and takes the original ROP as secret part.


\noindent 2. \textbf{P3\cite{ra2013p3}}: separates ROP based on a threshold in the DCT frequency domain; sets the higher frequency part as secret part and the remaining as public part.

\noindent 3. \textbf{Blur\cite{mcdonnell1981box}}: a normalization box filter is applied to ROP to generate the public part; subtracts the public part from ROP in a pixel-wise way to achieve the secret part.


\noindent
Then, the public part of the whole photo is produced by replacing its ROP with the public part of ROP (as shown in Fig.~\ref{fig:mask}).
Our experiment (Section~\ref{sec:eva}) shows that all three methods
are resistant to automatic detection algorithms, but the blur based
method outperforms others in the storage cost, hence we adopt the blur
as the default separation method in \ourprotocolNSP. After extracting the
private part from ROP, the owner encrypts the secret part as well as
its image descriptor as a \textit{private bag}, and uploads the private
bag to the search cloud. Then, he also uploads the public part of the
original photo as a \textit{public bag} to the sharing cloud (Fig.~\ref{fig:system_overview-a}).


\CUTTaeho{
\begin{figure}[h]
\begin{center}
\includegraphics[width=0.8\linewidth, clip,keepaspectratio]{checkbox.eps}\vspace{-5pt}
\caption{Hierarchical Automatic ROP Detection}\vspace{-10pt}
\label{fig:checklist}
\end{center}
\end{figure}
}


\vspace{-0.05in}
\subsection{Fine-grained Photo Sharing}
\vspace{-0.05in}

\ourprotocolNSP~allows fine-grained photo sharing among users. The photo owner uses an access control scheme (\eg, \cite{bethencourt2007ciphertext, chase2009improving}) to encrypt the \textit{search keys} so that only the authorized users with certain attributes can obtain search keys.
As the Step 5 in Fig.~\ref{fig:system_overview-a}, the owner encrypts the search keys under the \textit{access rule} that he defines,
 and the encrypted search keys are uploaded to the sharing cloud and made published.
Obtaining the search key, the authorized user can generate valid photo queries and decrypt the private part of ROP.
The completed original images can be recovered simply by merging the public parts of images and the private parts of ROPs.
Here, all these operations are also automatic and transparent to users, and the authorized user can browse the shared images as usual.


\vspace{-0.05in}
\subsection{Light-weight Photo Searching}
\vspace{-0.05in}
When a querier wants to search a photo among someone else's photos, he
pre-processes the querying photo to achieve the corresponding image
descriptor. Then, only if satisfying the owner's access rule, he can
retrieve the search keys to search on the owner's photos, but it is
the cloud who conducts the searching job and returns the result to the
querier obliviously, \ie without knowing contents of the owner's ROPs
or the contents of the query photo. After fetching the query result,
as mentioned above, system generate the original image for the querier transparently.
For the querier, the whole system appears like common image search systems.
Fig.~\ref{fig:system_overview-b} illustrates the search procedure.

\begin{figure}[t!]
\begin{center}
\includegraphics[width=0.95\linewidth, clip,keepaspectratio]{mask.eps}\vspace{-5pt}
\caption{Public/Private Part of ROP. Images in the upper row are public part of image and images in the lower row are private part of ROP.}\vspace{-20pt}
\label{fig:mask}
\end{center}
\end{figure}

\vspace{-0.05in}
\subsection{System Design Goals}
\vspace{-0.05in}




Our system is designed to achieve efficiency, privacy protection and accuracy goals.

$\bullet$ \textbf{Efficiency}:
To overcome the resource limitation of mobile client,
 operations at the user side should be light-weight,
 and most of the expensive computations should be outsourced to the cloud side.

$\bullet$ \textbf{Privacy Preservation}:
Users outsource not only the storage of photos but also the searching to the cloud side in \ourprotocolNSP.
Therefore, the framework is expected to protect users' privacy in various aspects:


 1. ROP Privacy: Unauthorized party should not learn secret part of ROP including cloud servers,

 2. Query Privacy: Cloud servers should not learn query photos,

 3. Result Privacy: Cloud servers should not learn search results,
\noindent which are all non-trivial challenges since cloud servers are the party who conducts the searching jobs on the photos stored at his side.


$\bullet$ \textbf{Accuracy}:
Introducing the privacy protection mechanism should not bring much accuracy loss. That is, the search result from \ourprotocolNSP~should be comparable with traditional image search technologies conducted on plain texts of photos.


\vspace{-0.05in}
\subsection{Threat Model}
\vspace{-0.05in}




\textit{W.l.o.g.}, we assume \textbf{curious-but-honest} cloud servers and \textbf{malicious} users in this work. Cloud servers will follow the protocol specification in general, but they will try their best to harvest any information about user's photos. This is a justifiable assumption because deviating from the protocol and not returning a correct search result will lead to bad user experience as well as potential revenue loss of the service provider. However, they might conduct extra work to illegally harvest useful information from the protocol communications in order to infer the \textbf{secret part of ROP} or the \textbf{contents of queries}, which is sensitive information to be protected. On the other hand, queriers may misbehave throughout the protocol to infer the \textbf{search keys} to forge a valid photo query, where the search keys are supposed to be kept secret as well.



\CUTTaeho{
Our system is designed to support different
 levels of privacy-preserving image search:
\begin{compactenum}
\item Image level search: a user inputs a required image,
 the system returns the image or a set of images most likely
 containing this sub-image.
This is the basic and most general image search.
In this level of image search,
 the system doesn't care the content of the required image.
So the search is based on the image similarity.
In our system, the similarity can be measured by
 visual descriptors or visual words.


\item Object level search: To search the images of a specific object,
 a user inputs an image of the required object,
 the system return images containing this object.
 In most applications,
 there are many images for a single object, \eg, a specific person, a logo, a landmark.
For the system implementation,
 there is a template descriptor for a known object.
Compared with the image level search,
 a modification is required for the data storage.


\item Conceptual search:User input the search keywords (text description of the target image),
 \eg, the person name or color name.
 and the related images are returned.
\end{compactenum}

Though the image search itself is a difficult
 issue, the privacy consideration greatly increases the challenges.
Facing these challenges,
 we designed a series of methods to achieve the following privacy requirements
 during the storage, sharing and search processes:
 \begin{compactenum}
 \item the private regions of images can only be accessed (viewed/detected) by the owner and valid visitors;
 \item protecting the owner's and requestor's visual and conceptual descriptors from untrusted parties;
 \item protecting the similarities between descriptors from untrusted parties.
 \end{compactenum}
Besides, our system also protects requestors' attributes
 and retrieve patterns, \ie the search process must be oblivious to the services provider and other parties.


For the system performance,
 we design our system to achieve comparable accuracy with exiting
   mechanisms without privacy consideration,
   meanwhile, not to greatly increase the storage, communication and computation cost.
Moreover, the ease of use, multiple image formats, the extensibility and scalability
 should also be taken into consideration during the system design.



\subsection{User Defined Privacy Protection Policy}
The image privacy protection policy is set by the image owner in the client side.
As shown in Fig.~\ref{fig:system_overview},
 two aspects of protection policy are determined:
 in Step {\large{\textcircled{\small{1}}}}, the private content is defined;
 and in Step {\large{\textcircled{\small{2}}}} the access rule is defined.

\subsubsection{Private Content}
In most cases,
 the private content (\eg faces and landmarks) in an image are subregions of the image.
For an image $I$,
 we define a private subregion as a \emph{Region Of Privacy} (ROP), denoted as $\rop=(tl, br)$,
 here $tl$ and $br$  are the pixel-level coordinates of its top-left and bottom-right in $I$.
Let $\rop(I)$ denote the sub-image extracted from $I$ according to $\rop$.
In our system,
 both automatic and manual ways are supported for an image owner to
 determine the ROP.
For the ease of use,
 our system provides a hierarchical check list for the owner to define
 private content (\eg, an object or a class of objects, recognized by
 the state-of-the-art computer vision technologies, or tagged by user).
For object classes with mature automatic detection methods,
 the owner can select which class is private, \eg, people faces.
For the known objects with tags, \eg people faces tagged with names,
  the owner can select the specific private objects by checking the name.
Specially, when the private object is a whole image,
 the owner just need to select the ID of this image.
Fig.~\ref{fig:checklist} shows an example check list of our system.
From the checked items,
 our system runs the object detection algorithm over the input images,
 and records the ROPs of each images using  bounding boxes.
as the red rectangle in Fig.~\ref{fig:checklist}.

So far, most privacy related objects,
 \eg faces and texts, can be automatically detected.
In case  there are still some sensitive objects or scenes
 cannot be automatically detected,
 a manual ROP definition mode is also supported.
In this mode,
 user defines the ROP $\rop$ by selecting its bounding box $(tl, br)$.

While the ROP $\rop_{I_x}$ of an image $I_x$ is determined,
 for the vision feature based search,
 the client of our system will generate a feature descriptor $\fx$ for
 this sub-image $\rop_{I_x}(I_x)$.
First, a set of interest points $p=\{p_1, p_2, \cdots, p_\alpha\}$
 are detected with a feature detector.
Then, a fix-length feature vector $\vx_i$ will be generated from
 the interest point $p_i$ of this sub-image using a describing method, \eg, SIFT \cite{lowe2004distinctive} or SURF.
These feature vectors compose the feature descriptor $\fx=\{\vx_1, \vx_2, \cdots, \vx_\alpha\}$.




\subsubsection{Access Rule}

To facilitate the cloud-based image sharing
 while protect the image privacy from unauthorized users,
 the owner makes access rules for the ROPs to determine how he wants to share his images.
In most social networking and image sharing systems, \eg, Facebook, Flickr,
 each user is assigned with a set of attributes, \eg, school, profession.
The access rule is defined by determining the
 attributes of the authorized visitors.
Specifically, let $\mathbb{A} = \{a_1, a_2,\cdots, a_m\}$ be the universal set of attributes.
The owner need  define an access structure, often expressed by a tree structure
 with AND and OR gates, \eg, \cite{bethencourt2007ciphertext}.
The owner-defined access structures over $\mathbf{A}$ will be used in
 Step {\large{\textcircled{\small{5}}}} of the privacy concealing component
 to enable access control of image privacy.

Before uploading to cloud,
 all data will be packed into two bags automatically by the client side of our system:
A \emph{public bag} containing the public part of the images without any sensitive information,
 which can be uploaded to any exiting sharing or storage cloud;
A \emph{private bag} containing the encrypted sensitive information
 for search, image recovery and access control.
In the following two subsections,
 we will present the mechanisms of privacy concealing and encryption
 of these two components.


\subsection{Public Image Privacy Concealing}
\label{sec:concealing}

In Step {\large{\textcircled{\small{3}}}}, for each image $I_x$,
 sensitive information of its ROP
 will be concealed from the original images to generate a public image $\public(I)$.
The public image  can prevent unauthorized users
 and automatic feature detection by machine.
As a result, the public image can be used for common image sharing systems.
The secret images are required when recovering the original image.
A secret image $\secret(\rop(I_x))$ is then generated for
  the sensitive information from the ROP.
Fig.~\ref{fig:mask} illustrates the public images and secret images
 generated by different concealing methods.



Given $\rop(I_x)$, which is the sub-image of $I_x$ extracted according to ROP definition,
 there are several optional methods to conceal
 the sensitive information to produce a public part $\public(\rop(I_x))$.
For each processing method,
 a corresponding secret part $\secret(\rop(I_X))$ is generated.
\begin{enumerate}
\item \emph{Mask} based method: the public part $\public(\rop(I_X))$ is filled with solid black (all intensity values equal `0'); the secret part is the same as the original sub-image,
\item \emph{Threshold} based method: a threshold is applied to $\rop(I_x)$ to
  separate it in a defined domain (\eg P3 separates image by threshold
  in the frequency domain, bit plane separates image by extracting
  specific bits from intensity values); $\public(\rop(I_x))$ contains
  the less significant information, while $\secret(\rop(I_x))$
  contains the more significant information;
\item \emph{Blur} based method: a normalization box filter is applied to
  $\rop(I_x)$ to generate $\public(\rop(I_x))$; the private part is
  generated by subtracting the blurred public part from original
  sub-image.
\end{enumerate}
Except P3 which requires a DCT processing,
 for the other methods the secret part can be generated by
 subtracting the public part from the original sub-image pixel-wise,
 \ie $\secret(\rop(I_x))=\rop(I_x)-\public(\rop(I_x))=\rop(I_x)$.
Then the public part of the  original image $I$
 is produced by simply replacing its ROP with the public part
 of ROP, \ie $\public(I_x) = I_x - \rop(I_x) + \public(\rop(I_x))$.
Fig.~\ref{fig:mask} shows an example of privacy concealed result of the public part
 of image and their corresponding secret parts of the ROP.
The three methods are all resistant to automatic detection and human recognition.
As we will present in our evaluation ,
 the blur based method performs best in both storage cost and aesthetic measure.
Hence, \ourprotocol uses blur as the default method
 for public part privacy concealing.



\subsection{Private Part Encryption}



For an image $I$, the private part consists of the secret part of ROP, \ie $\secret(\rop(I))$,
 and the feature descriptors $\fx$ of each ROP.
Both $\secret(\rop(I))$ and $\fx$ require protection.
The vision-based search are conducted with feature descriptors.
The great challenge comes from the
 conflicted requirements of privacy protection and efficient search.
In our work, we design a special encryption method for the feature descriptor,
 to provide efficient various privacy-preserving image searches.
In our design, PPS3  outsources as much computation as possible
 to the  cloud with as few interactions as possible with the
 requestor.
And we will present the detail of private part encryption in Section~\ref{sec:search}.




In Step {\large{\textcircled{\small{4}}}},
 after encrypting the feature vectors for an image $I$,
 the owner encrypts the secret part $\secret(\rop(I))$  by his
 public key $PK$.
In Step {\large{\textcircled{\small{5}}}} the private key $SK$ and search keys
 are encrypted by a CP-ABE algorithm \cite{bethencourt2007ciphertext, waters2011ciphertext}.
Let $c = \abe(m)$ be the ciphertext of a message $m$ with the public key $MPK$
 and the access structure $\mathbf{A}$ using CP-ABE algorithm.
}

\CUTXY{The CP-ABE mechanism mainly consists four components:
\begin{enumerate}
\item Setup: the authority outputs a public parameter $MPK$ and a master secret key $MSK$.
\item Encryption: the algorithm $c = \abe(m)$ encrypts a message $m$ with the public parameter $MPK$ and the access structure $\mathbf{A}$ and produce a ciphertext $c$ that only a user whose attributes satisfy $\mathbf{A}$ is able to decrypt it.
\item Key generation: the algorithm $ASK = \kg(MSK,A)$ takes as input the master secret key $MSK$ and a set $A$ of attributes and produces
 a private attributes-key $ASK$ for $A$.
\item Decryption: the algorithm $\dabe(c)$ decrypts the ciphertext $c$ with the public parameter $MPK$ and the private key $ASK$
 and produce the plain message if the private key's corresponding attributes satisfy the access structure $\mathbf{A}$.
\end{enumerate}
}


\CUTTaeho{
Specifically, our system leverage the CP-ABE mechanism in the following way:
 the owner uses the encryption algorithm to generate $\abe(SK)$, $\abe(\{K^0, K^1\})$
  using the access structure $\mathbf{A}$.
And the encrypted keys will be uploaded to the searching server.
A requestor who wants to search images in the cloud
 first goes to the authority to generate his private decryption key $ASK = \kg(MSK,A)$
 with his attributes $A$.
Then the requestor visits the searching cloud to obtain $\abe(SK)$ and $\abe(\{K^0, K^1\})$
 and tries to decrypt them with his key $ASK$.
If he is an valid user with authorized attributes,
 he can get the correct $SK$ and $\{K^0, K^1\}$
 to enable  further search.
Note that, the Paillier public key $PK$ of the owner is  encrypted by CP-ABE.
The search server will be allowed and only allowed to access the public key $PK$
 for the privacy-preserving cloud-based search.


After Step {\large{\textcircled{\small{5}}}},
 the privacy information in the image have been concealed,
 and all images are packed into two bags:
 the \emph{public bag} consisting of public parts $\public(I_x)$ of all images$I_x$;
 the \emph{private bag} consisting of Paillier encrypted secret parts of ROP $\en_{PK}(\secret(\rop(I_x)))$,
 garbled circuit encrypted feature descriptors $\garbled(\fx)$,
 and attribute based encrypted $\abe(SK)$ and $\abe(\{K^0, K^1\})$.
For image with metadata (\eg, location) or tags,
 each text entry is encoded into a text vector
 using some well-studied method, \eg hashing or cosine vector.
These text vector is also encrypted by the garbled circuit method
 and then included in the private bag to enable privacy-preserving text-based image search.
With the user defined access structure and
 and the CP-ABE encryption,
 \ourprotocol protect the content of private bag
 from \emph{unauthorized ROP access}.

\subsection{Cloud Storage}

The public bag can be uploaded to any common share or storage platform.
Each image has a unique ID and a URL pointing to its public part, here
 the URL is also encrypted by the owner using $PK$.
While the private bag will be uploaded to the cloud server supporting
 our search mechanism.
Fig.~\ref{fig:storage} sketches the storage structure of the private
data on the search cloud.

\begin{figure}[h]
\begin{center}
\includegraphics[width=0.9\linewidth, clip,keepaspectratio]{cloudstorage.eps}
\caption{The private data stored on the search server.}
\label{fig:storage}
\end{center}
\end{figure}


For image owner,
 the search server maintains the pair of Paillier keys and search keys
 encrypted by ABE (an owner could have multiple pairs of Paillier
 keys, each for a group of images that need individual attribute-based
 access control).
Each image could have more than one ROPs.
For each ROP, the server maintains its secret part encrypted by the
 public key $PK$
 and its feature vectors encrypted by the modified garbled circuit.




\subsection{Privacy Preserving Image Search}
When a requestor $P_1$ needs to search an image of a specific owner,
 he obtains the owner's attribute encrypted private key and search key from the server
 and tries to decrypt them with his attributes as described in the previous Subsection~\ref{sec:ppe}.
A requestor with authorized attributes set  can get correct $SK$, $K^0$
and $K^1$.
Then he encodes the vectors of his request feature descriptor $\fy$ using $K^0$ and $K^1$ bit by bit,
 \ie $\code(Y) := \{\code(\vy_1), \cdots, \code(\vy_n)\}$.
For each bit $\vy_j(k)$ of $\vy_j$, if $\vy_j(k)=0$, $\code(\vy_j(k))=K^0$, else $\code(\vy_j(k))=K^1$.
Requestor $P_1$ uploads $\code(\fy)$ to the search server $P_a$.
Note that the order of $K^0$ and $K^1$ is revealed to $P_1$,
 but unknown to the server $P_a$, since the server $P_a$ cannot pass the attribute authentication.
So the request descriptor $\fy$ is protected from $P_a$.
Based on our design for feature vector encryption in Subsection~\ref{sec:ppe},
  an efficient \emph{noninteractive} privacy-preserving feature vector distance protocol is presented as Protocol 1.
The first step leverages our garbled circuit encrypted feature vector to compute
 the bitwise encrypted XOR result between $\vx_i(k)$ and $\vy_j(k)$.
The second step leverages Paillier homomorphic addition to get the squared distance between $\vx_i$ and $\vy_j$.





\subsection{Oblivious Image Retrieve with Access Control}

Until now,
 the server and other potential adversaries learn nothing
 about the content of queries and ROPs.
But when users retrieve images,
 statistics about which images they retrieve could also reveal
 some information about images and queries, \eg which images contain
 the most popular person,
 or who may have common friends appearing in same images.
In our system, we use a simple oblivious retrieve method
 to maximize the privacy protection for both image owner and requestor:
 which meets the following requirements
 (1) only the user with authorized attributes can access the corresponding images;
 (2) the search/storage server doesn't learn the attributes of a user when he retrieve images;
the search/storage server doesn't learn exactly which image a user
 retrieve.

To achieve this requirement,
 an $k$-out-of-$n$ oblivious transfer (OT), \eg,\cite{camenisch2007simulatable}
 is usually conducted combined with some access control policy, \eg,
 \cite{camenisch2007simulatable, camenisch2009oblivious}.
But the existing OT schemes do not hide the privacy information from the database server,
 and the construction is usually expensive.
In our system,
 the server has $N$ image records $\mathbf{R}=\{R_1, \cdots R_N\}$.
Each record $R_i$ contains the URL to its public part
 and information of its private part.
The requestor has a secret  subset $\sigma \subseteq \{1,2 \cdots,N\}$
  which are the indices of his desired images computed using image
  matching method, and $|\sigma| = k$.
To achieve the requestor's $k$-out-of-$n$ selection oblivious to the server,
 the requestor can use $\sigma'$ with randomly added indices to blind the server,
 here $\sigma \subset \sigma'$ and $|\sigma'|=n$.
Note that $n$ is not equal $N$,
 instead, for large dataset with big $N \gg n$.

With the encoded request and the oblivious image retrieval strategy,
 \ourprotocol hides the search content from untrusted parties.
}

\CUTXY{Obtaining the records from the search server and decrypting
 the URLs and ROPs,
 the requestor retrieves the public parts by URLs.
A malicious search server may
 analyze the network flow of the public part storage server
 and attempt to correlate it with the private part retrievals.
Here we emphasize that,
 even knowing the content of the public part,
 the search server cannot learn its corresponding entry
 in the privacy database,since the URL is encrypted.
So it cannot decide which image entry is retrieved.
With an anonymous communication,
 a server cannot tell who is requesting the public parts,
 and a better protection of the requestor privacy can be achieved.
}

\CUTTaeho{\subsection{Image Recovery}
Fetching the public and private parts of an image to the client,
 the client need to recovery the original image from both parts.
For different mask mechanism, the recovery method varies.
Except the method P3, which gives a DTC based recovery method,
 for the other privacy concealing methods,
 the recovery is simply adding the decrypted secret part of ROP to the public part of image,
 \ie $I = \public(I) + \secret(\rop(I))$.
}


