\documentclass[10pt,twocolumn,letterpaper]{article}
\usepackage[dvipsnames,svgnames,x11names]{xcolor}
\definecolor{citecolor}{RGB}{119,185,0} 
\usepackage[pagebackref=false,breaklinks=true,letterpaper=true,colorlinks,citecolor=citecolor,bookmarks=false]{hyperref}
\usepackage{iccv}  
\usepackage{cite}
\usepackage{times}
\usepackage{epsfig}
\usepackage{graphicx}
\usepackage{amsmath}
\usepackage{amssymb}
\usepackage{algorithm}
\usepackage{algorithmic}
\usepackage{comment}
\usepackage{array}
\usepackage{url}
\usepackage{animate}
\usepackage{colortbl}
\usepackage{subfiles}



\usepackage{soul}
\usepackage[utf8]{inputenc}
\usepackage[small]{caption}
\usepackage{multirow}



\newcommand{\zznote}[1]{\textcolor{blue}{ZZ:#1}}
\newcommand{\jynote}[1]{\textcolor{orange}{Jiayin:#1}}
\newcommand{\tabincell}[2]{\begin{tabular}{@{}#1@{}}#2\end{tabular}}
\usepackage{pifont}
\newcommand{\cmark}{\ding{51}}\newcommand{\xmark}{\ding{55}}



\def\httilde{\mbox{\tt\raisebox{-.5ex}{\symbol{126}}}}
\def\eg{\emph{e.g.}} 
\def\ie{\emph{i.e.}} 
\def\etal{\emph{et~al.}} 

\newcommand{\resblock}[2]{\multirow{3}{*}{-.1em] \text{33, #1} \end{array}\right]\times^1^1^1^2^1^1^2^{\circ}\rightarrow\rightarrow^\dagger^*^*^*^\dagger\rightarrow\rightarrowI_iM_iE_cE_sE_{UV}E_A\bar{S}\phi\pi\hat{I_i}\hat{M_i}C_iI_iM_ii \in [1, N]NE_SE_CE_AE_{UV}\bar{S} \in \mathbb{R}^{|\bar{S}|\times3}6421280642\bar{S}I_iM_i\bar{S}\Delta S_i \in \mathbb{R}^{642\times3}S_iS_i \in \mathbb{R}^{642\times3}\Delta S_iI_iM_iE_S(I_i, M_i, \bar{S})\Delta S_i\bar{S}(I_i, M_i)\bar{S}C_iI_iM_iS_iUV_iC_iA_i\hat{I_i}\phiS_iUV_i\piC_iA_i\hat{M_i}\phi\pil_1\odot\mathbb{E}\hat{I_i}\hat{M_i}\mathcal{L}_{img}\oplusI_i\oplus M_iD\hat{I_i}I_ipS_i\delta_p = p - \sum_{k\in K(p)}\frac{k}{||K(p)||}K(p)p\mathcal{L}_{lpl} = \mathbb{E}[||\hat{\delta_p}-\delta_p||^2_2\hat{\delta_p}\delta_p\mathcal{L}_{flat} = \mathbb{E}[(cos(\Delta\theta_i)+1)^2]\Delta\theta_i180^{\circ}\mathcal{L}_{sym} = \mathbb{E}[||Z(p) + Z(\tilde{p})||_1Z\tilde{p}p\mathcal{L}_{deform} = \mathbb{E}[||\Delta S||_2 \lambda_{rec} = 2\lambda_{att} = 1\lambda_{adv} = 1\times10^{-5}\lambda_{reg} = 0.1\lambda_{lpl} = 0.1\lambda_{flat} = 0.01S_i, UV_i, C_i, A_i\bar{S}\Delta S_iE_cE_AE_{UV}C_iA_iUV_iS_i\bar{S}\Delta S_i\bar{S}\Delta S_i\bar{S} = \bar{S} + \mathbb{E}[\Delta S]17700160007001000128\times64512\times25612936751338675059945794\rightarrow\rightarrow_{recon}_{novel}_{90}\rightarrow\rightarrow\uparrow\rightarrow\rightarrow81.1\%81.8\%83.5\%_{novel}_{novel}\uparrow\uparrow_{novel}\downarrow_{recon}_{90}_{novel}_{90}_{recon}\downarrow_{novel}\downarrow_{90}\downarrow_{recon}\downarrow_{novel}\downarrow\uparrow\uparrow\uparrow\uparrow^\dagger^\daggerS_iUV_jS_iUV_j90^{\circ}17,70016,0007001,000128\times64512\times25612,9367513,3867505,9945,794\times\times128\times64160\times96128\times128256\times2565\times10^{-5}(\beta_1, \beta_2) = (0.95, 0.99)6000.05 \times E_CE_SE_{UV}E_AD4\times128\times64E_s1926 = 642\times3\Delta SE_c\_\_\times\timesazimuths_yazimuths_xazimuths_y / azimuths_xE_S\times\times\times\times\times\times\times\times\times\timesE_C\times\times\times\times\times\times\times\times\times\times\times\times\times\times\times\times\times\times\times\times\times\times\times\times\times\timesE_A\times\times\times\times\times\times\times\times\times\times\times\times\times\timesE_{UV}\times\times\times\times\times\times\times\times\times\times\times\times\times\times\times\times\times\times\times\times\times\times\times\times\times\times\times\times\times\times\times\times\times\times\times\times\times\times\times\times\times\times\times\times\times\times\times\times\times\times\times\timesD\times\times\times\times\times\times\times\times\times\times\times\times\times\times\times\times\times\times\times\times\times\times\times\times\times\times\times\times\times\times\times\times\times\times\times\times\times\times\times\times\times\times\times\times\times\times^{\circ}^{\circ}^{\circ}^{\circ}^{\circ}2\sim4^\circ^\circ^\circ\Delta S_{recon}\downarrow_{novel}\downarrow_{90}\downarrow_{recon}\downarrow_{novel}\downarrow\uparrow\uparrow\uparrow\uparrowC\Delta S\rightarrow\rightarrow$ MSMT.  After the re-id model ``sees'' our generated 3D-aware data,  the high-fidelity variants further facilitates the model scalability to diverse environment. Duke and MSMT are only for academic comparison and reference.
}
\label{table:baseline}\vspace{-.2in}
\begin{center}\footnotesize
\begin{tabular}{c|c|c|c|c|c}
\hline 
\multirow{2}{*}{Methods} &  \multirow{2}{*}{Training Set}  & \multicolumn{2}{c|}{Market2Duke} & \multicolumn{2}{c}{Market2MSMT} \\
 &   & R@1  &  mAP & R@1  &  mAP \\
\shline
\multirow{2}{*}{ResNet50-ibn~\cite{pan2018IBNNet}} & Original  &  &    & 	& \\
 & Original+3D  & 	&     & 	& \\
\hline
\multirow{2}{*}{HR18-Net~\cite{wang2020deep}} & Original  &  &  & 	&\\
 & Original+3D  &  &   & 	& \\
\hline
\end{tabular}
\end{center}\vspace{-.2in}
\end{table}
\end{comment}

 
\end{document}
