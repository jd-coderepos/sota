\documentclass[a4paper,10pt]{llncs}
\usepackage[utf8]{inputenc}
\usepackage[english]{babel}
\usepackage{amsmath,amssymb,amsfonts,graphics,graphicx}

\newtheorem{observation}{Observation}{\itshape}{\rmfamily}

\title{Polynomial-time approximability\\of the -{\sc Sink Location} problem}
\author{R\'emy Belmonte \and Yuya Higashikawa \and Naoki Katoh \and Yoshio Okamoto}

\author{
\mbox{
R\'emy Belmonte
\hspace{1cm}
Yuya Higashikawa
}
\mbox{
Naoki Katoh
\hspace{1cm}
Yoshio Okamoto
}
}
\institute{
Department of Architecture and Architectural Engineering,\\
C-2 cluster, 4 Katsura, Nishikyo-ku, Kyoto University, Japan\\
\texttt{remybelmonte@gmail.com, \{as.higashikawa,naoki\}@archi.kyoto-u.ac.jp}\\
\medskip
Department of Communication Engineering and Informatics,\\
Graduate School of Informatics and Engineering,\\
The University of Electro-Communications\\
\texttt{okamotoy@uec.ac.jp}
}



\begin{document}

\maketitle

\begin{abstract}
A dynamic network  where  is a graph, integers  and  represent, for each edge , the time required to traverse edge  and its nonnegative capacity, and the set  is a set of sources. In the -{\sc Sink Location} problem, one is given as input a dynamic network  where every source  is given a nonnegative supply value . The task is then to find a set of sinks  in  that minimizes the routing time of all supply to . Note that, in the case where  is an undirected graph, the optimal position of the sinks in  needs not be at vertices, and can be located along edges. Hoppe and Tardos\cite{HT00} showed that, given an instance of -{\sc Sink Location} and a set of  vertices , one can find an optimal routing scheme of all the supply in  to  in polynomial time, in the case where graph  is directed. Note that when  is directed, this suffices to obtain polynomial-time solvability of the -{\sc Sink Location} problem, since any optimal position will be located at vertices of .  However, the computational complexity of the -{\sc Sink Location} problem on general undirected graphs is still open. In this paper, we show that the -{\sc Sink Location} problem admits a fully polynomial-time approximation scheme (FPTAS) for every fixed , and that the problem is -hard when parameterized by .
\end{abstract}

\section{Introduction}

One of the most well-studied family of problems in algorithmic graph theory is that of so-called network flow problems. Such problems originate from the seemingly simple idea that one wishes to move from one location to another in a network, while minimizing some cost function. From this first idea, a wealth of natural problems were derived, where one usually seeks to transport a set of items from a prescribed set of sources to a non-necessarily predetermined set of target locations. In this paper, we study the problem known as -{\sc Sink Location} on dynamic networks, where one seeks to find a set of locations in a network which minimizes the amount of time required to evacuate all the supply located on each vertex of the graph to these locations. A dynamic network is graph with a prescribed set of sources, and nonnegative capacities and integral transit times for every edge. A similar problem, called {\sc Quickest Transshipment}, was studied by Hoppe and Tardos~\cite{HT00}. In this problem, one is given a set of sinks together with the sources and each sink has a demand value. The task is then to send the supply from the sources to the sinks as quickly as possible, in such a way that each sink receives exactly as much supply as its demand value.  They showed that the problem is polynomial-time solvable in the case where the dynamic network is directed, which can easily be seen to imply polynomial time solvability of the -{\sc Sink Location} problem on directed graphs, for every fixed . Note that polynomial-time solvability for the directed case does not readily imply solvability for the undirected case. This is due to the fact that while in a directed network an optimal solution can only be located at a vertex, an optimal solution in an undirected network may be located at any point along an edge.

{\em Related work.} For the 1-{\sc Sink Location} problem, Kamiyama, Katoh and Takizawa~\cite{KKT06,KKT09} gave efficient algorithms for several cases where the structure of the network satisfies requirements regarding the length of the edges and the structure of the given graph. Mamada et al.~\cite{MUMF06} gave an  algorithm for the case where the input graph is a tree. Most of the recent work on the {\sc Sink Location} problem has been considering computationally harder variants of the problem on restricted graph classes. In~\cite{HGK14}, Higashikawa, Golin and Katoh considered the generalized version of the problem where one seeks not only to find a single sink, but some given number , when the input graph is an undirected path. They showed that this problem can be solved in  time in the so-called minimax setting, and  in the minisum setting. In~\cite{HGK14a}, the same authors considered the minimax  regret version of the problem when the input graph is a tree and the edges have uniform capacities, and showed that the problem can be solved in  time. As an intermediate result, they provide an  algorithm to find a sink location that minimizes the evacuation time.

{\em Some definitions.} We now define formally the notion of dynamic network and the -{\sc Sink Location} problem. A {\em dynamic network}  consists of a graph  where each edge  is given a {\em capacity}  and a {\em transit time} , and a prescribed set of {\em sources} . The notion of dynamic network was first introduced by Ford and Fulkerson~\cite{FF62}. In the -{\sc Sink Location} problem, one is given a dynamic network  together with a supply value  for each vertex  of the set  of sources of . The task is then to find  positions  in the graph, called {\em sinks}, which minimizes the amount of time required to send the supply  from each vertex  to the positions . A position  is defined as a triple , where  is an edge of , and  represent the time required to travel from position  to , and . Observe that the number of positions for every edge  is exactly , and can therefore be exponentially large in the size of the input graph .
In the {\sc Quickest Transshipment} problem, we are also given a function . For every vertex , if  then  is called a {\em source}, otherwise it is called a {\em sink}. The question is then to send all the supply from sources to the sinks in a minimum amount of time, in such a way that each sink  receives exactly  units of supply. Note that this immediately implies . Hoppe and Tardos~\cite{HT00} proved that {\sc Quickest Transshipment} can be solved in polynomial time on directed graphs.
For additional terminology and notation, refer to the monograph by Diestel~\cite{Die05}.

{\em Our contribution.} We study the computational complexity of the -{\sc Sink Location} problem on general undirected graphs and prove the following result:

\begin{theorem}
\label{thm:FPTAS}
The -{\sc Sink Location} problem admits an FPTAS on undirected dynamic networks for every fixed . 
\end{theorem}

A parameterized problem is said to be FPT  by some parameter  if there is an algorithm that solves the problem in time . Intuitively, a -hard problem is a problem that is unlikely to admit an FPT algorithm. We refer the reader to~\cite{FG06,Nie06} for more information about parameterized complexity and algorithms.
We complement Theorem~\ref{thm:FPTAS} by showing that is unlikely to be significantly improved:

\begin{theorem}
\label{thm:W-hard}
The -{\sc Sink Location} problem is -hard when parameterized by . 
\end{theorem}

\section{Polynomial-time approximation scheme}

In this section, we prove our main result, namely that the -{\sc Sink Location} problem admits an FPTAS on general undirected graphs, for every fixed . To that end, we will first need to show that, given an instance  of the -{\sc Sink Location} problem and a set of  positions  in , one can compute the minimum amount of time required to send all the supply to . Note the following result of Hoppe and Tardos~\cite{HT00}:
\begin{theorem}[\cite{HT00}]
The {\sc Quickest Transshipment} problem can be solved in polynomial time on directed graphs.
\end{theorem}
Recall that in the {\sc Quickest Transshipment} problem, one is given the set of sinks , and each sink  is given a demand value that must be met exactly. We show that the -{\sc Sink Location} problem on undirected graphs can be reduced to the {\sc Quickest Transshipment} problem on directed graphs when the set of sinks is given. This will later allow us to use Hoppe and Tardos' algorithm to evaluate the time required to send all the supply to a given set of sinks.

\begin{lemma}
\label{lem:reduction}
There is an algorithm that, given an instance  of the -{\sc Sink Location} problem where  is undirected and a set of  positions  in  such that no two positions in  lie on the same edge, computes the minimum amount of time required to send all supply to  and runs in polynomial time.
\end{lemma}

\begin{proof}
We reduce our problem to {\sc Quickest Transshipment} on directed graphs. Since Hoppe and Tardos~\cite{HT00} proved that the problem is polynomial-time solvable in that case, this will immediately imply our lemma. Given an instance  of the -{\sc Sink Location} problem with , where  is undirected, and a set of positions  such that  in , we create an instance  of the {\sc Quickest Transshipment} problem with , where  is directed. Our construction is as follows:
\begin{itemize}
\item , with ;
\item ;
\item For every edge , we create~2 new opposite edges  and  such that  and ;
\item For every edge  such that there is a position  that lies on , we replace  with~4 edges  such that  and  and ;
\item We add edges  for every  and set  and ;
\item For every vertex ,  for every  and .
\end{itemize}
We now claim that the minimum amount of time required to send all supplies to  in  is equal to the minimum amount of time required in .
The fact that any solution in  corresponds to an equivalent solution in  follows from the fact that in , an edge is never traversed in both directions at a given time, and the length and capacity of every edge is the same as in . Similarly, for every routing scheme in , there exists an equivalent routing that can be completed in the same amount of time where, at any given time, at most one edge out of every pair of opposite edges has non-zero flow in the routing. This concludes the proof of the lemma.
\qed
\end{proof}

We are now ready to describe our FPTAS for the -{\sc Sink Location} problem on undirected graphs. Roughly speaking, we will ``guess'' near-optimal positions for the  sinks by performing sampling at regular interval over each edge of . One can then reduce the approximation ratio by increasing the amount of sampling points. In this section, given an instance of the -{\sc Sink Location} problem on undirected graphs, we denote by  the minimum amount of time required to send all supplies to the set of positions , and , for all , such that   and .

\begin{proof}[of Theorem~\ref{thm:FPTAS}]
We describe an algorithm that takes as input an instance  of the -{\sc Sink Location} problem and  and returns a set of  positions  in  such that . Let us define . 
We first define a set  of positions in the following way:  consists of all the vertices of , together with positions , with  . Observe that . Our algorithm then tries every possible set  of  positions in , computes  using Lemma~\ref{lem:reduction}, and returns . Our algorithm runs in time , where  is the running time of Hoppe and Tardos' algorithm. The running time of our algorithm is then , as desired.
To complete the proof, it only remains to show that there exists a set  of  vertices in  such that .
Consider a set of  positions  in  such that sending all the supply in  to positions in  takes time , i.e.,  is an optimal set of positions in . First, we show that we may assume, without loss of generality, that at most~2 positions in  lie on the same edge , and if exactly~2 positions of  lie on , then these positions are exactly  and . Indeed, observe first that we may safely assume that no edge  contains more than~2 positions of , since every unit of supply that is sent to a sink on  will have to pass through either the position  in  closest from , or the position  closest from . Note that  and  may happen. Therefore, we may remove all positions of  that lie on  other than  and . Additionally, since every unit of supply sent to  and  have to pass through either  or  in order to reach a sink, we may safely replace  and  with  and  in , without increasing the total amount of time required to send the supply to the sinks.
Consider now the set  such that for every position , we add  to  if , otherwise we add the position  in  closest from . Note that if , then , and therefore  lies on a edge, and lies between two positions of . In case  is equidistant from these two positions, we choose one of them arbitrarily. Moreover, since every edge  contains either at most~1 position of  or both  and , no two distinct positions of  correspond to the same position  in , and hence, each position  is associated with a position  in a bijective manner.
We now claim that  satisfies , as desired.
To prove this claim, we show that any routing to  that can be achieved in time  can be transformed into a new routing to  that can be achieved in time at most . This immediately follows from the fact that for every pair of positions  and , the supply sent to  either passes through , in which case it can simply stop there, or it can be sent to  as soon as it reaches , without violating the capacity constraint. Since  is chosen to be closest from  among the positions in , and the distance between two positions of  lying on an edge  is at most , we obtain that the supply reaches  in the new routing at most  units of time after it reaches . Note that, if we denote by  the time at which the last unit of supply reaches  in the original routing, and  the time at which the last unit of supply reaches  in the modified routing, we have for every pair of positions :

Where  is the edge that contains  and . Moreover, observe that if , then  and , and if  then .
Therefore, we have

Finally, observe that if , we way assume without loss of generality that , and hence there is at least~1 unit of supply reaching  in the original routing that passes through  and at least~1 unit that passes through . Hence, for every position  lying on edge :

Which in turn implies

Since this inequality holds for every pair of positions  and , we obtain the following inequality for the global solutions  and :

This concludes the proof of our main theorem.
\qed
\end{proof}

\section{Hardness of -{\sc Sink Location} parameterized by }

In this section, we provide a simple reduction from the well-known -complete problem -{\sc Hitting Set}. In this problem, one is given as input a ground set  and a family of sets , and the task is to find a subset  of  containing at most  elements, such that every set in  contains at least~1 element of .

\begin{theorem}
The -{\sc Sink Location} problem is -hard parameterized by  on undirected graphs, even when all the edges in the input graph  have unit length and capacity.
\end{theorem}

\begin{proof}
Given an instance  of {\sc Hitting Set}, we build a graph  such that , and two vertices  and  are made adjacent whenever  . We then set  for every edge  and add exactly~1 unit of supply to each vertex in . We now claim that  admits a hitting set of size  if and only if there exist  sinks in  to which all the supply can be sent in exactly~1 unit of time.

For the forward direction, observe that if  has a hitting set of size , then choosing those  elements of  as sinks will allow to send all the supply in a single unit of time.

For the converse direction, observe that since the vertices of  form an independent and every edge in  has length~1, every sink that lies on an edge  with , but not at , will only be able to receive supply from . Hence, we may assume that all the sinks lie on vertices of . It is then clear that the  sinks must be adjacent to every vertex of , and form therefore a hitting set of . 
\qed
\end{proof}

\section{Conclusion}

In this paper, we proved that the -{\sc Sink Location} problem admits an FPTAS on general undirected graphs for every fixed , but that, on the negative side, it is -hard when  is not fixed, but used as a parameter instead. Two natural questions immediately follow. The first of these two questions is whether the -{\sc Sink Location} problem can be solved in polynomial time for every fixed . The second is whether the problem admits an FPTAS when parameterized by , i.e., can be solved in time  with approximation ratio , for some function .

\begin{thebibliography}{10}

\bibitem{Die05}
Diestel, R.:
\newblock Graph Theory.
\newblock Springer-Verlag, Electronic Edition (2005)

\bibitem{FF62}
Ford, L. R., D. R. Fulkerson:
Flows in Networks.
Princeton University Press (1962)

\bibitem{FG06}
Flum, J., Grohe, M.:
Parameterized Complexity Theory.
Springer-Verlag (2006)

\bibitem{HGK14}
Higashikawa, Y., Golin, M.~J., Katoh, N.
Improved Algorithms for Multiple Sink Location Problems in Dynamic Path Networks.
CoRR abs/1405.5613 (2014)

\bibitem{HGK14a}
Higashikawa, Y., Golin, M.~J., Katoh, N.
Minimax Regret Sink Location Problem in Dynamic Tree Networks with Uniform Capacity.
In: WALCOM 2014: 125-137 (2014)

\bibitem{HT00}
Hoppe, B., Tardos, \'E.:
The Quickest Transshipment Problem.
Math. Oper. Res. 25(1): 36--62 (2000)

\bibitem{KKT06}
Kamiyama, N., Katoh, N., Takizawa, A.:
An Efficient Algorithm for Evacuation Problem in Dynamic Network Flows with Uniform Arc Capacity.
IEICE Transactions 89-D(8): 2372--2379 (2006)

\bibitem{KKT09}
Kamiyama, N., Katoh, N., Takizawa, A.:
An efficient algorithm for the evacuation problem in a certain class of networks with uniform path-lengths.
Discrete Applied Mathematics 157(17): 3665--3677 (2009)

\bibitem{Nie06}
Niedermeier, R.:
Invitation to Fixed-Parameter Algorithms.
Oxford University Press (2006)

\bibitem{MUMF06}
Mamada, S., Uno, T., Makino, K., Fujishige, S.:
An  algorithm for the optimal sink location problem in dynamic tree networks.
Discrete Applied Mathematics 154(16): 2387--2401 (2006)

\end{thebibliography}

\end{document}
