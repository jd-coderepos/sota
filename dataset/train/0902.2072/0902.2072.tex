\documentclass[proceedings]{stacs}
\stacsheading{2009}{673--684}{Freiburg}
\firstpageno{673}

\theoremstyle{definition}
\theoremstyle{plain}


\usepackage{times}
\usepackage{graphicx}
\usepackage{latexsym}
\usepackage{amsmath,amssymb,stmaryrd}


\newcommand{\mi}[1]{\mathit{#1}}



\newcommand{\ASP}{\mathit{ASP}}
\newcommand{\SP}{\mathit{SP}}
\newcommand{\UDD}{\mathit{UDD}}
\newcommand{\UT}{\mathit{T}}
\newcommand{\UN}{\mathit{U}}
\newcommand{\UDecl}{\mathit{Dcl}}
\newcommand{\UDefn}{\mathit{Dfn}}

\newcommand{\parspec}[3]{#1{\stackrel{#2}{\longrightarrow}}#3}

\newcommand{\archspec}[2]{\textbf{arch spec }#1\textbf{ result }#2}
\newcommand{\decl}[2]{#1\colon#2}
\newcommand{\dfn}[2]{#1=#2}
\newcommand{\amlg}[4]{#1\textbf{ with }#2\textbf{ and }#3\textbf{ with }#4}
\newcommand{\appl}[3]{#1\lbrack#2\textbf{ fit }#3\rbrack}
\newcommand{\gris}{\mathrel{::=}}


\newcommand{\Mset}{\mathcal{M}}
\newcommand{\Fset}{\mathcal{F}}

\newcommand{\Dgm}{\mathit{Diag}}
\newcommand{\Shape}[1]{\mathit{Shape}(#1)}
\newcommand{\Nodes}[1]{\mathit{Nodes}(#1)}
\newcommand{\Edges}[1]{\mathit{Edges}(#1)}


\newcommand{\fm}[2]{\langle#1\rangle_{#2}}

\newcommand{\incl}[2]{\iota_{{#1}\subseteq{#2}}}

\newcommand{\st}{\mathit{st}}
\newcommand{\stC}{\mathit{C}_{\st}}
\newcommand{\stB}{\mathit{B}_{\st}}
\newcommand{\stP}{\mathit{P}_{\st}}
\newcommand{\emptystC}{\stC^\emptyset}
\newcommand{\emptyUC}{\UC^\emptyset}
\newcommand{\UC}{\mathcal{C}}
\newcommand{\UEv}{\mathit{UEv}}
\newcommand{\Uset}{\mathcal{V}}
\newcommand{\UU}{\mathit{V}}

\newcommand{\EstC}{\mathcal{C}_{\st}}
\newcommand{\emptyEstC}{\EstC^\emptyset}
\newcommand{\EstB}{\mathcal{B}_{\st}}
\newcommand{\SD}{D}

\newcommand{\CstC}{\mathfrak{C}_{\st}}

\newcommand{\getstctx}{\mathit{ctx}}

\newcommand{\env}{\mathit{E}}

\newcommand{\lambdaexp}[2]{\lambda#1\cdot#2}
\newcommand{\reduct}[2]{#1|_{#2}}

\newlength{\croutw}
\newlength{\crouth}
\newcommand{\crossout}[1]{\settowidth{\croutw}{}\settoheight{\crouth}{}#1\hspace{-1.0\croutw}\raisebox{0.3\crouth}{\rule{\croutw}{0.1ex}}}

\newcommand{\commentout}[1]{\ignorespaces}
\newcommand{\underscore}{\rule{3mm}{0.005in}}

\newcommand{\extaticsem}[3]{#1 \vdash #2 \extsmb #3}
\newcommand{\extsmb}{\mathrel{\rhd\hspace{-0.5em}\rhd}}

\newcommand{\rulesection}[1]{}
\newcommand{\infrule}[2]{\frac{#1}{#2}}
\newcommand{\staticsem}[3]{#1 \vdash #2 \rhd #3}
\newcommand{\modelsem}[3]{#1 \vdash #2 \Rightarrow #3}
\newcommand{\staticsembreak}[3]{\begin{array}{r}#1 \vdash #2 \qquad\\ 
                                                \rhd #3 \end{array}}
\newcommand{\modelsembreak}[3]{\begin{array}{r}#1 \vdash #2 \qquad\\
                                                 \Rightarrow #3 \end{array}}










\newcommand{\ttsize}{\footnotesize }
\newcommand{\NT}[1]{{\ttsize\texttt{#1}}}




\newcommand{\cofidirectory}{}


\newcommand{\parrightarrow}{\p\rightarrow}
\newcommand{\p}[1]{\mathrel{\ooalign{\hfil\hfil\cr}}}
\newcommand{\totrightarrow}{\rightarrow}


\newcommand{\X}{CASL}
\newcommand{\LSX}{{W_{\Sigma}(X)}}

\newcommand{\Pow}{\mathcal{P}}
\newcommand{\Powfin}{\mathcal{P}_\omega}
\newcommand{\gen}[1]{\mid_{#1}}
\newcommand{\cogen}[1]{\mid^{#1}}
\newcommand{\induce}[2]{\mid^{#1}_{#2}}
\newcommand{\inducefrom}[1]{\!\mid_{#1}}
\newcommand{\induceto}[1]{\mid^{#1}}



\newcommand{\sen}{\mathbf{Sen}}



\newcommand{\PF}{\mathit{PF}}
\newcommand{\TF}{\mathit{TF}}

\newcommand{\SY}{\mathit SY}
\newcommand{\SYs}{\mathit SYs}
\newcommand{\RSY}{\mathit RSY}
\newcommand{\RSYs}{\mathit RSYs}
\newcommand{\Sym}{\mathit{Sym}}
\newcommand{\RawSym}{\mathit{RawSym}}
\newcommand{\RawSymMap}{\mathit{RawSymMap}}
\newcommand{\IDAsRawSym}{\mathit{IDAsRawSym}}
\newcommand{\SymAsRawSym}{\mathit{SymAsRawSym}}








\newcommand{\TSX}{{T_{\Sigma^\#}(X)}}

\newcommand{\bit}{\begin{itemize}}
\newcommand{\eit}{\end{itemize}}
\newcommand{\Equivalent}{\Leftrightarrow}
\newcommand{\logand}{\wedge}
\newcommand{\logor}{\vee}
\newcommand{\intersection}{\cap}
\newcommand{\comp}{\circ}
\newcommand{\suchthat}{\mid}
\newcommand{\congr}{\equiv}
\newcommand{\disjoint}{\uplus}
\newcommand{\buildset}[1]{\{#1\}}
\newcommand{\Si}{\Sigma}
\newcommand{\al}{\alpha}
\newcommand{\pp}[2]{#1_1\commadots #1_{#2}}
\newcommand{\mmap}[2]{#1\!\longrightarrow\!#2}
\newcommand{\anddots}{\logand\,\cdots\,\logand}
\newcommand{\lldots}{\,\ldots\,}
\newcommand{\ccdots}{\,\cdots\,}
\newcommand{\commadots}{,\lldots,}
\newcommand{\sigspec}[1]{\langle #1 \rangle}
\newcommand{\eq}{\stackrel{\mbox{\scriptsize e}}{=}}
\newcommand{\weq}{\stackrel{\mbox{\scriptsize w}}{=}}

\newcommand{\mBox}[1]{\, \mbox{#1} \,}

\newcommand{\engtwocase}[3]{
\left\{
\begin{array}{ll}
  #1,&\mBox{if }#2\\
  #3,&\mBox{otherwise}
\end{array}\right.}

\newcommand{\threecase}[5]{
\left\{
\begin{array}{ll}
  #1,&\mbox{if }#2\\
  #3,&\mbox{if }#4\\
  #5,&\mbox{otherwise}
\end{array}\right.}

\newcommand{\threefullcase}[6]{
\left\{
\begin{array}{ll}
  #1,&\mbox{if }#2\\
  #3,&\mbox{if }#4\\
  #5,&\mbox{if }#6
\end{array}\right.}

\newcommand{\delete}[1]{}
\newcommand{\new}[1]{{\it\tiny #1}}

\newcommand{\mbs}[1]{\mbox{}}




\newcommand{\ASL}[1]{[\![#1]\!]}

\newcommand{\sym}{\mathit{sym}}
\newcommand{\totqual}{\mathsf{t}}
\newcommand{\parqual}{\mathsf{p}}
\newcommand{\ws}{\mathit{ws}}
\newcommand{\FinSet}[1]{\mathit{FinSet}(#1)}
\newcommand{\RawSymOf}{\mathit{RawSymOf}}
\newcommand{\Gram}[1]{{\textup{\texttt{#1}}}}

\newcommand{\implicitqualkind}{\mathit{implicit}}
\newcommand{\sortqualkind}{\mathit{sort}}
\newcommand{\funqualkind}{\mathit{fun}}
\newcommand{\predqualkind}{\mathit{pred}}
\newcommand{\SymKind}{\mathit{SymKind}}

\newcommand{\QualFunName}{\mathit{QualFunName}}
\newcommand{\QualPredName}{\mathit{QualPredName}}






\hfuzz1pc
\sloppy





\newenvironment{rmenumerate}{\begin{enumerate}\renewcommand{\labelenumi}{(\roman{enumi})}}{\end{enumerate}}
\newenvironment{alenumerate}{\begin{enumerate}\renewcommand{\labelenumi}{(\alph{enumi})}}{\end{enumerate}}
\newenvironment{Alenumerate}{\begin{enumerate}\renewcommand{\labelenumi}{\Alph{enumi})}}{\end{enumerate}}
\newenvironment{numenumerate}{\begin{enumerate}\renewcommand{\labelenumi}{\arabic{enumi})}}{\end{enumerate}}






\newcommand{\Cat}{\mathbf}
\newcommand{\Cls}{\mathcal}
\newcommand{\Opname}{\mathrm}

\newcommand{\Ob}{{\Opname{Ob}\,}}
\newcommand{\Mor}{{\Opname{Mor}\,}}
\newcommand{\Epi}{{\Opname{Epi}\,}}
\newcommand{\Mono}{{\Opname{Mono}\,}}
\newcommand{\Iso}{{\Opname{Iso}\,}}
\newcommand{\Ident}{{\Opname{Ident}\,}}
\newcommand{\Op}{{op}}
\newcommand{\colim}{{\Opname{colim}\,}}

\newcommand{\lrule}[3]{(#1)\;\;\infrule{#2}{#3}}
\newcommand{\lrrule}[3]{(#1)\;\;\infrule{\underline{#2}}{#3}}
\newcommand{\laxiom}[2]{(#1)\;\;#2}

\newcommand{\BA}{{\Cat A}}
\newcommand{\BC}{{\Cat C}}
\newcommand{\BD}{{\Cat D}}
\newcommand{\BI}{{\Cat I}}
\newcommand{\BK}{{\Cat K}}
\newcommand{\BL}{{\Cat L}}
\newcommand{\BM}{{\Cat M}}
\newcommand{\BN}{{\Cat N}}
\newcommand{\BS}{{\Cat S}}
\newcommand{\BW}{{\Cat W}}
\newcommand{\Btwo}{{\Cat 2}}

\newcommand{\CC}{{\Cls C}}
\newcommand{\CE}{{\Cls E}}
\newcommand{\CM}{{\Cls M}}
\newcommand{\CO}{{\Cls O}}
\newcommand{\CP}{{\Cls P}}
\newcommand{\CS}{{\Cls S}}
\newcommand{\CT}{{\Cls T}}
\newcommand{\eps}{\varepsilon}

\newcommand{\integers}{{\mathbb Z}}
\newcommand{\tensor}{\otimes}

\newcommand{\mystrut}[1]{\rule[#1]{0cm}{0.1cm}}



\newcommand{\red}{\vdash}
\newcommand{\idred}{\red}
\newcommand{\step}{\red_c}
\newcommand{\downto}{\succeq}
\newcommand{\upto}{\preceq}
\newcommand{\moremonicthan}{\Rightarrow}
\newcommand{\into}{\hookrightarrow}
\newcommand{\impl}{\Rightarrow}
\newcommand{\id}{{id}}
\newcommand{\ev}{{ev}}
\newcommand{\Hom}{{hom}}
\newcommand{\Dom}{{dom\, }}
\newcommand{\Cod}{{cod}}
\newcommand{\adj}{\dashv}
\newcommand{\powerset}{{\mathcal P}}
\newcommand{\finpowerset}{{\powerset_\omega}}
\newcommand{\kVec}{\mbox{-}}
\newcommand{\termObj}{1}

\newcommand{\map}[2]{:#1\to #2}
\newcommand{\restr}[2]{{#1}|_{#2}}
\newcommand{\wordbrace}[1]{\underbrace{\hspace{1.5cm}}_{\displaystyle{#1}}}
\newcommand{\forget}[1]{|_{#1}}
\newcommand{\argument}{\_\!\_}

\newcommand{\CASLsign}{\Cat{CASLsign}}
\newcommand{\enrCASLsign}{\Cat{enrCASLsign}}
\newcommand{\refCASLsign}{\Cat{refCASLsign}}
\newcommand{\Mod}{\Cat{Mod}}
\newcommand{\EnrMod}{\Cat{Mod}_e}
\newcommand{\CAT}{\Cat{CAT}}
\newcommand{\Sen}{\Cat{Sen}}
\newcommand{\Sign}{\Cat{Sign}}
\newcommand{\SignVar}{\Cat{Var}}
\newcommand{\EnrSign}{\Cat{EnrSign}}
\newcommand{\Set}{\Cat{Set}}
\newcommand{\CLS}{\Cat{CLS}}

\newcommand{\SmallSpecName}[1]{\textsc{\small #1}}
\newcommand{\EnrInst}{EnrPCFOL}
\newcommand{\CASLInst}{SubPCFOL}
\newcommand{\PCFOLInst}{PCFOL}

\newcommand{\Word}{\mathbf}
\newcommand{\Ba}{{\Word{a}}}
\newcommand{\Bb}{{\Word{b}}}
\newcommand{\Bc}{{\Word{c}}}
\newcommand{\Bf}{{\Word{f}}}
\newcommand{\Bg}{{\Word{g}}}
\newcommand{\Bh}{{\Word{h}}}
\newcommand{\Bl}{{\Word{l}}}
\newcommand{\Br}{{\Word{r}}}





\newcommand{\HasCASL}{{\sc HasCasl}\xspace}
\newcommand{\ML}{{\textsf{ML}}\xspace}
\newcommand{\BaseTypes}{B}
\newcommand{\SimpleTypes}{T}
\newcommand{\TotalFunType}[2]{#1\to #2}
\newcommand{\sbullet}{\mbox{  }}
\newcommand{\lambdaTotal}[2]{\lambda\, #1\sbullet\!\! !\, #2}
\newcommand{\PartialFunType}[2]{#1\to ?#2}
\newcommand{\lambdaPartial}[2]{\lambda\, #1\sbullet #2}
\newcommand{\lambdaType}[2]{\lambda\, #1\sbullet #2}
\newcommand{\NonStrictFunType}[2]{\PartialFunType{\Maybe{#1}}{#2}}
\newcommand{\lambdaNonStrict}[2]{\lambda\, #1\sbullet #2}
\newcommand{\substTerm}[3]{{#1[#3/#2]}}
\newcommand{\PredType}[1]{{\textsf{pred}(#1)}}
\newcommand{\ProdType}[2]{#1 * #2}
\newcommand{\pMonType}[2]{#1 -\!\!\mu\!\!\to\hspace{-1pt}? #2}
\newcommand{\tMonType}[2]{#1 -\!\!\mu\!\!\to #2}
\newcommand{\pContType}{Pcont}
\newcommand{\tContType}{Tcont}
\newcommand{\projOp}[1]{{pr_#1}}
\newcommand{\fst}{{\textsf{fst}}}
\newcommand{\snd}{{\textsf{snd}}}
\newcommand{\UnitType}{{\textsf{unit}}}
\newcommand{\UnitOp}{{()}}
\newcommand{\Maybe}[1]{?#1}
\newcommand{\TypeSchemes}{{TS}}
\newcommand{\TypeContext}{\mathcal{C}}
\newcommand{\UniversalType}[2]{{\forall #1\sbullet #2}}
\newcommand{\applyPoly}[2]{{#1[#2]}}
\newcommand{\ConstrainedType}[2]{{#1 \Rightarrow #2}}
\newcommand{\Constraint}{C}
\newcommand{\TypeKind}{\textsf{type}}
\newcommand{\TypeVars}{{TV}}
\newcommand{\TypeOps}{{TO}}
\newcommand{\PseudoTypes}{{PT}}
\newcommand{\SubstType}[3]{{#1[#3/#2]}}
\newcommand{\varco}[1]{{v^+(#1)}}
\newcommand{\varcontra}[1]{{v^-(#1)}}
\newcommand{\Covar}{\textsf{covariant}}
\newcommand{\Contravar}{\textsf{contravariant}}
\newcommand{\sigSorts}{S}
\newcommand{\sigClasses}{C}
\newcommand{\sigConstrs}{T}
\newcommand{\sigAlias}{{A}}
\newcommand{\sigExpand}{{\varepsilon}}
\newcommand{\unalias}{{\bar\sigExpand}}
\newcommand{\sigSub}{\leq}
\newcommand{\extSub}{\leq_*}
\newcommand{\elPred}[2]{{#1\in #2}}
\newcommand{\downCast}[2]{{#1\textsf{ as }#2}}
\newcommand{\upCast}[2]{{#1:#2}}
\newcommand{\sigOps}{O}
\newcommand{\IndVars}{V}
\newcommand{\BT}{B}
\newcommand{\emb}{{emb}}

\newcommand{\Context}{\Gamma}
\newcommand{\ContextTerm}[3]{{#1 \rhd #2: #3}}
\newcommand{\ContextEq}[2]{{#1 \rhd #2}}
\newcommand{\ContextEntails}[3]{{#2 \entails_{#1} #3}}
\newcommand{\ContextImpl}[3]{{#2 \impl_{#1} #3}}
\newcommand{\exeq}{\stackrel{e}{=}}
\newcommand{\IsDef}{\operatorname{def}}

\newcommand{\isFormula}[1]{{#1\in\textsf{Prop}}}

\newcommand{\Model}{M}
\newcommand{\Sem}[1]{{[\![#1]\!]}}
\newcommand{\extendsTo}{\preceq}
\newcommand{\True}{\top}
\newcommand{\False}{\bot}
\newcommand{\uexists}{\exists !}
\newcommand{\iotaTerm}[2]{\iota #1\sbullet #2}
\renewcommand{\conj}{\wedge}
\newcommand{\disj}{\vee}
\newcommand{\modimpl}{\to}
\newcommand{\modiff}{\leftrightarrow}
\newcommand{\intEq}{eq}
\newcommand{\dOrder}[3]{#2 \le\![#1]\,#3}
\newcommand{\resTerm}[2]{#1\ res\ #2}
\newcommand{\mono}[2]{\operatorname{mono}_{#1}#2}
\newcommand{\entails}{\vdash}



\newcommand{\Class}{\textbf{class}}
\newcommand{\Instance}{\textbf{instance}}
\newcommand{\Program}{\textbf{program}}
\newcommand{\Internal}{\textbf{internal}}
\providecommand{\pfun}{\mathrel{\rightarrow?}}
\newcommand{\iimpl}{\Rightarrow}
\newcommand{\iconj}{\wedge}

\newcommand{\BBT}{\mathbb{T}}
\newcommand{\DO}{\operatorname{do}}
\newcommand{\Let}{\operatorname{let}}
\newcommand{\retOp}{\operatorname{ret}}
\newcommand{\retOpHC}{\mathit{ret}}
\newcommand{\ret}[1]{\retOp #1}
\newcommand{\letTerm}[2]{\DO\, #1; \ #2}
\newcommand{\leteq}{\leftarrow}
\newcommand{\se}[2]{\mathsf{se}(#1,#2)}
\newcommand{\Pfin}{\mathcal P_{\mathit{fin}}}

\newcommand{\nec}{\square\,}
\newcommand{\gbox}{\square\hspace{-6.5pt}\raisebox{2pt}{\tiny{G}}\;}
\newcommand{\pbox}[1]{[#1]\,}
\newcommand{\pdiamond}[1]{<\hspace{-3pt}#1\hspace{-3pt}>\!}
\newcommand{\lbox}{\Box}
\newcommand{\ldiamond}{\Diamond}
\newcommand{\pmodal}[2]{[#1]_G\,#2}
\newcommand{\Timpl}{\Rightarrow_T}
\newcommand{\Tconj}{\wedge}
\newcommand{\HTriple}[3]{\{#1\}~#2~\{#3\}}
\newcommand{\ContextHTriple}[4]{#1\rhd\HTriple{#2}{#3}{#4}}
\newcommand{\DummyCHT}[4]{\HTriple{#2}{#3}{#4}}
\newcommand{\sef}{\mathit{sef}}
\newcommand{\dsef}{\mathit{dsef}}

\newcommand{\extInst}[1]{\mathrm{Ext}(#1)}
\newcommand{\Poly}[1]{\mathrm{Poly}(#1)}



\newcommand{\ModFunctor}{\Cat{Mod}\,}


\newcommand{\la}{\langle}
\newcommand{\ra}{\rangle}


\newcommand{\si}{\sigma}

\newcommand{\ISig}{\mathit{Sig}}
\newcommand{\IMod}{\mathit{Mod}}
\newcommand{\IAx}{\mathit{Ax}}

\newenvironment{specsem}{\smallskip\par\noindent\qquad\par\noindent}


\newlength{\myboxwidth}
\setlength{\myboxwidth}{\textwidth}
\addtolength{\myboxwidth}{-20pt}
\newenvironment{myfigure}{\begin{figure}\begin{center}
	\setlength{\fboxsep}{10pt}}{\end{center}\end{figure}}

\newcommand{\Lang}{\mathcal{L}}	
\newcommand{\FLang}{\mathcal{F}}	
\newcommand{\pls}{\Lambda}
\newcommand{\pl}[3]{#1\in #2(#3)}
\newcommand{\polypl}[3]{#1\in #2 #3}
\newcommand{\negpl}[3]{#1\notin #2(#3)}
\newcommand{\plbox}[1]{[#1]}
\newcommand{\pldiamond}[1]{\langle#1\rangle}
\newcommand{\gldiamond}[1]{\Diamond_{#1}}
\newcommand{\glbox}[1]{\square_{#1}}

\newcommand{\HMLBox}[1]{\square\hspace{-5.5pt}\raisebox{2pt}{}\;}
\newcommand{\HMLDiamondb}[1]{\Diamond\hspace{-5pt}\raisebox{1.5pt}{}\;}
\newcommand{\HMLDiamonda}[1]{\Diamond\hspace{-5.3pt}\raisebox{1.9pt}{}\;}
\newcommand{\pldbox}[2]{[#1]_{#2}}
\newcommand{\plempty}{\plbox{\emptyset}}
\newcommand{\plcan}{\square}
\newcommand{\FC}{{\mathfrak C}}
\newcommand{\negcl}[1]{\mathit{cl}(#1)}
\newcommand{\Nat}{{\mathbb{N}}}
\newcommand{\Int}{{\mathbb{Z}}}
\newcommand{\Rat}{{\mathbb{Q}}}
\newcommand{\Real}{{\mathbb{R}}}
\newcommand{\List}{\mathsf{list}}
\newcommand{\PDist}{D_\omega}\newcommand{\SFun}{\mathcal{SP}}
\newcommand{\plcomp}{\circledast}
\newcommand{\contrapower}{2^{\argument}}
\newcommand{\trans}[1]{#1^\flat}
\newcommand{\boolcl}[2]{\mathrm{bcl}_{#1}(#2)}
\newcommand{\Bag}{\mathcal{B}}
\newcommand{\Baginfty}{\mathcal{B}_\infty}
\newcommand{\Prop}{\mathsf{Prop}}
\newcommand{\Up}{\mathsf{Up}}
\newcommand{\Lit}{\mathsf{Lit}}
\newcommand{\nneg}{\sim}
\newcommand{\satisfies}{\vDash}
\newcommand{\gsatisfies}{\satisfies_g}
\newcommand{\nsatisfies}{\nvDash}
\newcommand{\PSPACE}{\mathit{PSPACE}}
\newcommand{\Rules}{\mathcal{R}}
\newcommand{\RulesC}{\Rules_C}
\newcommand{\Frame}[1]{2^{2^{#1}}}
\newcommand{\UpP}{\mathsf{Up}\mathcal{P}}
\newcommand{\one}{\mathbb{1}}
\newcommand{\total}{\nu}
\newcommand{\sgn}{\mathit{sgn}}
\newcommand{\size}{\mathit{size}}
\newcommand{\arity}{\mathit{ar}}

\newcommand{\Sorts}{\mathcal{S}}
\newcommand{\MSorts}{\Sorts_0}
\newcommand{\ModOp}{L}
\newcommand{\contrapow}{\mathcal{Q}}
\newcommand{\profto}{\stackrel{\bullet}{\to}}
\newcommand{\Struct}{\mathcal{M}}
\newcommand{\Id}{\mathit{Id}}
\newcommand{\Det}{\mathsf{Det}}
\newcommand{\hDet}{\mathsf{hDet}}
\newcommand{\FF}{\mathfrak{F}}
\newcommand{\idOp}{\plbox{\iota}}
\newcommand{\Sig}{\mathit{Sig}}

\newcommand{\NEXP}{\mi{NEXPTIME}}
\newcommand{\EXP}{\mi{EXPTIME}}
\newcommand{\NP}{\mi{NP}}
\newcommand{\at}{\mi{at}}
\newcommand{\Max}{\mi{max}}
\newcommand{\muprod}{\textstyle\prod^\mu}
\newcommand{\tsum}{\textstyle\sum}
\DeclareMathOperator{\mge}{\ge}
\DeclareMathOperator{\meq}{=}
\DeclareMathOperator{\mle}{\le}
\DeclareMathOperator{\fixarrow}{\uparrow}
\newcommand{\Stable}{S}
\newcommand{\Zero}{Z}
\newcommand{\Regular}{U}
\newcommand{\last}{\lambda}
\newcommand{\fix}[2]{#1\!\fixarrow #2}
\newcommand{\msf}{\mathsf}

\newcommand{\CKCMi}{\CK\!+\!\CMi\xspace}
\newcommand{\CK}{\mathit{CK}}
\newcommand{\CMi}{\mathit{CMi}}
\newcommand{\Ax}{\mathcal{A}}

\newcommand{\PLentails}{\entails_{\mi{PL}}}
\newcommand{\ModSig}{\Lambda}

\newcommand{\CB}{\mathcal{B}}
 
\newenvironment{axarraycomment}{\begin{array}{@{\hspace{2em}}p{5em}p{20em}p{20em}}}{\end{array}}

\newcommand{\invlim}{\varprojlim}
\newcommand{\modelsOS}{\models^1}
\newcommand{\modelsPL}{\models^0}
\newcommand{\FA}{\mathfrak{A}}
\newcommand{\RBox}{\mathcal{R}}
\newcommand{\TBox}{\mathcal{T}}
\newcommand{\CI}{\mathcal{I}}
\newcommand{\ALCQ}{\mathcal{ALCQ}}
\newcommand{\Land}{\bigwedge}
\newcommand{\Lor}{\bigvee}
\newcommand{\CondArrow}{\Rightarrow}
\newcommand{\CF}{\mathit{Cf}}
\newcommand{\Sel}{\mathcal{S}}
\newcommand{\supp}{\mathrm{supp}}
\newcommand{\Dist}{\mathcal{D}}
\newcommand{\AtProp}{P}
\newcommand{\Form}{\FLang}
\newcommand{\op}{^\mathrm{op}}
\newcommand{\inv}{^{-1}}
\newcommand{\To}{\Rightarrow}
\newcommand{\lsem}{\llbracket}
\newcommand{\rsem}{\rrbracket}
\newcommand{\biimpl}{\leftrightarrow}
\newcommand{\GML}{\mathrm{GML}}
\newcommand{\Meas}{\mathcal{M}}
\newcommand{\CL}{\mathrm{CL}}
\newcommand{\PML}{\mathrm{PML}}


\newcounter{blubber}

\newenvironment{sparenumerate}
{\begin{list}
  {\arabic{blubber}.}
  {\usecounter{blubber}
   \setlength{\leftmargin}{0pt}
    \setlength{\parsep}{0pt}
    \setlength{\itemindent}{3ex}
    \setlength{\itemsep}{2pt}   
    \setlength{\listparindent}{3ex}
  }
}
{\end{list}}

\newenvironment{sparitemize}
{\begin{list}{}{
    \setlength{\leftmargin}{0pt}
    \setlength{\parsep}{0pt}
    \setlength{\itemindent}{4ex}
    \setlength{\itemsep}{0pt}
  }
}{\end{list}}



\begin{document}

\title{Strong Completeness of Coalgebraic Modal Logics}

\author[DFKIUHB]{L. Schr{\"o}der}{Lutz
  Schr{\"o}der}
\address[DFKIUHB]{DFKI Bremen and Department of Computer Science,  
Universit\"at  Bremen}
\email{Lutz.Schroeder@dfki.de}
\author[IC]{D. Pattinson}{Dirk Pattinson}
\address[IC]{Department of Computing, 
Imperial College London}

\email{dirk@doc.ic.ac.uk}

\thanks{Work of the first author performed as part of the DFG project
  \emph{Generic Algorithms and Complexity Bounds in Coalgebraic Modal
    Logic} (SCHR 1118/5-1). Work of the second author partially
  supported by EPSRC grant EP/F031173/1}

\keywords{Logic in computer science, semantics, deduction, modal
  logic, coalgebra} 

\subjclass{F.4.1 [Mathematical Logic and Formal Languages]:
  Mathematical Logic --- modal logic; I.2.4 [Artificial Intelligence]:
  Knowledge Representation Formalisms and Methods --- modal logic,
  representation languages}

\begin{abstract}
  Canonical models are of central importance in modal logic, in
  particular as they witness strong completeness and hence
  compactness. While the canonical model construction is well
  understood for Kripke semantics, non-normal modal logics often
  present subtle difficulties -- up to the point that canonical models
  may fail to exist, as is the case e.g.\ in most probabilistic
  logics.  Here, we present a generic canonical model construction in
  the semantic framework of coalgebraic modal logic, which pinpoints
	coherence conditions between syntax and semantics of modal logics
	that guarantee strong completeness.
  We apply this method to reconstruct canonical
  model theorems that are either known or folklore, and moreover instantiate 
  our method to obtain new strong
  completeness results. In particular, we prove strong completeness of
  graded modal logic with finite multiplicities, and of the modal
  logic of exact probabilities.
\end{abstract}

\maketitle



\noindent In modal logic, completeness proofs come in two flavours:
\emph{weak} completeness, i.e.\ derivability of all universally
valid formulas,
is often proved using \emph{finite model} constructions, and
\emph{strong} completeness, which additionally allows for a possibly
infinite set of assumptions. The latter entails recursive
enumerability of the set of consequences of a recursively enumerable
set of assumptions, and is usually established using (infinite)
\emph{canonical models}. The appeal of the first method is that it
typically entails decidability.  The second method yields a stronger
result and has some advantages of its own. First, it applies in some
cases where finite models fail to exist, which often means that the
logic at hand is undecidable. In such cases, a completeness proof via
canonical models will at least salvage recursive
enumerability. Second, it allows for schematic axiomatisations, e.g.\
pertaining to the infinite evolution of a system or to observational
equivalence, i.e.\ statements to the effect that certain states cannot
be distinguished by any formula.


In the realm of Kripke semantics, canonical models exist for a large
variety of logics and are well understood, see e.g.\
\cite{BlackburnEA01}. But there is more to modal logic than Kripke
semantics, and indeed the natural semantic structures used to
interpret a  large
class of modal logics go beyond pure relations. This includes e.g.\
the selection function semantics of conditional logics
\cite{Chellas80}, the semantics of probabilistic logics in terms of
probability distributions, and the game frame semantics of coalition 
logic~\cite{Pauly02}.  
To date, there is very little research that provides systematic
criteria, or at least a methodology, for establishing strong
completeness for logics not amenable to Kripke semantics.  This is made worse
as the question of strong completeness crucially depends on the chosen
semantic domain, which as illustrated above may differ widely. It is
precisely this variety in semantics that makes it hard to employ the
strong-completeness-via-canonicity approach, as in many cases there is
no readily available notion of canonical model.  The present work
improves on this situation by providing a widely applicable generic
canonical model construction. More precisely,  we establish the
existence of quasi-canonical models, that is, models based on the
set of maximally consistent sets of formulas that satisfy the truth
lemma, as there may be no unique, or canonical, such model in our
more general case.
In order to cover the large span of semantic structures, we avoid a
commitment to a particular class of models, and instead work within
the framework of coalgebraic modal logic~\cite{Pattinson03} which
precisely provides us with a semantic umbrella for all of the examples
above.  This is achieved by using coalgebras for an endofunctor  as
the semantic domain for modal languages. As we illustrate in examples,
the semantics of particular logics is then obtained by particular
choices of~. Coalgebraic modal logic serves in particular as a
general semantic framework for non-normal modal
logics. As such, it improves on neighbourhood semantics in that it
retains the full semantic structure of the original models
(neighbourhood semantics offers only very little actual semantic
structure, and in fact may be regarded as constructed from syntactic
material~\cite{SchroderPattinson07mcs}).



In this setting, our criterion can be formulated as a set of coherence
conditions that relate the syntactic component of a logic to its
coalgebraic semantics, together with a purely semantic condition
stating that the endofunctor~ that defines the semantics needs to
preserve inverse limits weakly, and thus allows for a passage from the
finite to the infinite. We are initially concerned with the existence
of quasi-canonical models relative to the class of \emph{all}
-coalgebras, that is, whith logics that are axiomatisable by
formulas of modal depth uniformly equal to one \cite{Schroder07}.  As
in the classical theory, the corresponding result for logics with
extra frame conditions requires that the logic is canonical, i.e. the
frame that underlies a quasi-canonical model satisfies the frame
conditions, which holds in most cases, but for the time being needs to
be established individually for each logic.

Our new criterion is then used to obtain both previously known and
novel strong completeness results. In addition to positive results, we
dissect a number of logics for which strong completeness fails and
show which assumption of our criterion is violated. In particular,
this provides a handle on adjusting either the syntax or the semantics
of the logic at hand to achieve strong completeness.
For example, we demonstrate that the failure of strong
completeness for probabilistic modal logic (witnessed e.g.\ by the set
of formulas assigning probability  to an event for all 
but excluding probability ) disappears in the logic of exact
probabilities.  Moreover, we show that graded modal logic, and more
generally any description logic~\cite{BaaderEA03} with qualified
number restrictions, role hierarchies, and reflexive, transitive, and
symmetric roles, is strongly complete over the multigraph model of
\cite{DAgostinoVisser02}, which admits infinite multiplicities. While
strong completeness fails for the naive restriction of this model to
multigraphs allowing only finite multiplicities, we show how to
salvage strong completeness using additive
\mbox{(finite-)integer}-valued measures. Finally, we prove strong
completeness of several conditional logics w.r.t.\ conditional frames
(also known as selection function models); for at least one of these
logics, strong completeness was previously unknown.




\section{Preliminaries and Notation}

\noindent Our treatment of strong completeness is parametric in both
the syntax and the semantics of a wide range of modal logics. On the
syntactic side, we fix a \emph{modal similarity type} 
consisting of modal operators with associated arities. Given a
similarity type  and a countable set  of
atomic propositions, the set  of
\emph{-formulas} is inductively defined by the grammar

where  and  is -ary; further boolean
operators (, , , ) are defined as
usual.  Given any set  (e.g.\ of formulas, atomic propositions, or
sets (!)), we write  for the set of propositional formulas
over  and n for the set
of formulas arising by applying exactly one operator to elements of
.  We instantiate our results to a variety of settings later with
the following similarity types:
\begin{exas} \label{expl:sim-types}
\begin{sparenumerate}
\item The similarity type  of standard modal logic consists of a
single unary operator .
\item Conditional logic \cite{Chellas80} is defined over the
  similarity type  where the binary
  operator  is read as a non-monotonic conditional (default,
  relevant etc.), usually written in infix notation.
\item Graded modal operators~\cite{Fine72} appear in expressive
  description logics~\cite{BaaderEA03} in the guise of so-called
  qualified number restrictions; although we discuss only modal
  aspects, we use mostly description logic notation and terminology
  below. The operators of graded modal logic (GML) are  with  unary.
  We write  instead of . A formula
   is read as `at least  successor states satisfy
  ', and we abreviate .
\item The similarity type  of probabilistic modal logic
  (PML)~\cite{LarsenSkou91} contains the unary modal operators 
  for , read as `with probability at
  least , \dots'.
\end{sparenumerate}
\end{exas}
\noindent We split axiomatisations of modal logics into two parts: the
first group of axioms is responsible for axiomatising the logic
w.r.t. the class of \emph{all} (coalgebraic) models, whereas the
second consists of frame conditions that impose additional conditions
on models. As the class of all coalgebraic models, introduced below,
can always be axiomatised by formulas of \emph{rank }, i.e.\
containing exactly one level of modal operators~\cite{Schroder07} (and
conversely, every collection of such axioms admits a complete
coalgebraic semantics~\cite{SchroderPattinson07mcs}), we restrict the
axioms in the first group accordingly.  More formally:
\begin{defi}
  A \emph{(modal) logic} is a triple 
  where  is a similarity type,  is a set of \emph{rank-1 axioms},
  and  is a set of \emph{frame
    conditions}. We say that  is a \emph{rank-1 logic} if
  . If , we write
   if  can be derived from  with the help of propositional reasoning, uniform
  substitution, and the congruence rule: from  infer  whenever  is
  -ary. For a set  of assumptions,
  we write  if  for (finitely many) . A set  is \emph{-inconsistent}
  if , and otherwise
  \emph{-consistent}.
\end{defi}


\begin{exas}\label{expl:axioms}
\begin{sparenumerate}
\item \label{item:ax-kripke} The modal logic  comes about as the
  rank-1 logic  where . The logics
   arise as  where
   contains the additional axioms that define the respective
  logic~\cite{BlackburnEA01}, e.g.\ 
  in the case of .
\item\label{item:ax-cond} For conditional logic, we take the
  similarity type  together with rank-1 axioms ,  stating that
  the binary conditional is normal in its second argument. Typical
  additional rank-1 axioms are
  
  which together form the so-called \emph{System C}, a modal version of the
  well-known KLM (Krauss/Lehmannn/Magidor) axioms of default reasoning
  due to Burgess~\cite{Burgess81}.
\item\label{item:ax-gml} The axiomatisation of GML
  given in~\cite{Fine72} consists of the rank-1 axioms
\begin{itemize}
\item[] 
\item[]  for 
\item[] 
\item[] 
\end{itemize}
Frame conditions of interest include e.g. reflexivity (), symmetry (), and transitivity
().
\end{sparenumerate}
\end{exas}
\noindent To keep our results parametric also in the semantics of
modal logic, we work in the framework of \emph{coalgebraic modal
  logic} in order to achieve a uniform and coherent presentation.  In
this framework, the particular shape of models is encapsulated by an
endofunctor , the \emph{signature functor} (recall
that such a functor maps every set  to a set , and every map
 to a map  in such a way that composition and
identities are preserved), which may be thought of as a parametrised
data type. We fix the data , ,  etc.\ throughout
the generic part of the development. The role of models in then played
by -coalgebras:
\begin{defi}
A \emph{-coalgebra} is a pair  where  is a set
(the \emph{state space} of )
and  is a
function, the transition structure of .
\end{defi}
\noindent We think of  as a type of successors,
polymorphic in . The
transition structure 
associates a structured collection of successors 
 to each state .
\noindent The following choices of signature functors give rise to the
semantics of the modal logics discussed in Expl.
\ref{expl:axioms}.
\begin{exas}\label{expl:coalgml}
\begin{sparenumerate}
\item \label{item:Kripke} Coalgebras for the covariant powerset
  functor  defined on sets  by  and on maps  by  are Kripke
  frames, as relations  on a set  of worlds
  are in bijection with functions of type .
  Restricting the powerset functor to \emph{finite} subsets,
  i.e. putting , one obtains the class of image finite Kripke
  frames as -coalgebras. 
\item\label{item:cond} The semantics of conditional logic is captured
  coalgebraically by the endofunctor  that maps a set  to the
  set  of selection functions over  (the
  action of  on functions  is given by
  ).  The ensuing -coalgebras
  are precisely the conditional frames of \cite{Chellas80}.
\item\label{item:gml} The \emph{(infinite) multiset functor}
   maps a set  to the set  of multisets
  over , i.e.\ functions of type .
Accordingly, -coalgebras are \emph{multigraphs} (graphs
  with edges annotated by multiplicities). Multigraphs provide an
  alternative semantics for GML which is in many
  respects more natural than the original Kripke
  semantics~\cite{DAgostinoVisser02}, as also confirmed by new
  results below.
\item\label{item:pml} Finally, if  is the support of a function  and  is the set of finitely
  supported probability distributions on , then -coagebras
  are probabilistic transition systems, the semantic domain of
  PML.
\end{sparenumerate}
\end{exas}
\noindent The link between coalgebras and modal languages is provided
by predicate liftings~\cite{Pattinson03}, which are used to interpret
modal operators. Essentially, predicate liftings convert predicates on
the state space  into predicates on the set  of structured
collections of states:
\begin{defi}\label{def:lifting}\cite{Pattinson03}
  An \emph{-ary predicate lifting} () for 
  is a family of maps , where 
  ranges over all sets, satisfying the \emph{naturality} condition
  
  for all , . (For the
  categorically minded,  is a natural transformation
  , where  denotes
  contravariant powerset.) A \emph{structure} for a similarity type
   over an endofunctor  is the assignment of an -ary
  predicate lifting  to every -ary modal operator .
\end{defi}\noindent
\noindent Given a valuation  of the
propositional variables and a -coalgebra , a structure
for  allows us to define a satisfaction relation
 between states of  and formulas  by stipulating that  iff
 and

where .
An \emph{-model} is now a \emph{model}, i.e.\ a triple  as above, such that  for
all all  and all substitution instances  of .  An \emph{-frame} is a -coalgebra  such that
 is an -model for all valuations~.  The
reader is invited to check that the following predicate liftings
induce the standard semantics for the modal languages introduced in
Expl. \ref{expl:sim-types}.
\begin{exas}\label{expl:structure}
\begin{sparenumerate}
\item A structure for  over the covariant powerset functor
   is given by . The frame classes defined by the frame
  conditions mentioned in Expl.~\ref{expl:axioms}.\ref{item:ax-kripke}
  are well-known; e.g.\ a Kripke frame  is a -frame iff 
  is transitive.
\item Putting  reconstructs the semantics of conditional
  logic in a coalgebraic setting. 
\item\label{item:gml-struct} A structure for GML over
   is given by . The frame conditions mentioned in
  Expl.~\ref{expl:axioms}.\ref{item:ax-gml} correspond to conditions
  on multigraphs that can be read off directly from the logical
  axioms. E.g.\ a multigraph satisfies the transitivity axiom  iff whenever  has non-zero
  transition multiplicity to  and  has transition multiplicity
  at least  to , then  has transition multiplicity at least
   to .
\item The structure over  that captures PML coalgebraically is
  given by the the predicate lifting  for .
\end{sparenumerate}
\end{exas}
\noindent From now on, \emph{fix a modal logic  and a structure for  over a functor~}. 
We say that  is \emph{strongly complete} for some class of
models if every -consistent set of formulas is satisfiable in
some state of some model in that class. Restricting to \emph{finite}
sets  defines the notion of \emph{weak completeness}; many
coalgebraic modal logics are only weakly complete~\cite{Schroder07}.
\begin{defi}
  Let  be a set. If  and  is a valuation, we write  for the result of
  substituting  for  in , with propositional
  subformulas evaluated according to the boolean algebra structure of
  . (Hence,  is a formula over the set  of
  atoms.)  A formula  is
  \emph{one-step -derivable}, denoted ,
  if  is propositonally entailed by the set . A set  is \emph{one-step
    -consistent} if there do not exist formulas  such that . Dually, the \emph{one-step
    semantics}  of a formula  is defined inductively by  for
  . A set  is \emph{one-step satisfiable} if
  .  We
  say that  (or ) is \emph{separating} if  is
  uniquely determined by the set . We call  (or ) \emph{one-step
    sound} if every one-step derivable formula  is one-step valid, i.e. .
\end{defi}
\noindent \emph{Henceforth, we assume that  is one-step sound},
so that every -coalgebra satisfies the rank-1 axioms; in the
absence of frame conditions (), this means in
particular that every -coalgebra is an -frame. The above
notions of one-step satisfiability and one-step consistency are the
main concepts employed in the proof of strong completeness in the
following section. 

Given a structure for  over , every set  of rank-1
axioms over  defines a subfunctor  of  with
. This functor induces a structure for which  is one-step
sound.
\begin{exa}\label{expl:subfunctors}
  The additional rank-1 axioms of
  Expl.~\ref{expl:axioms}.\ref{item:ax-cond} induce subfunctors
   of the functor  of
  Expl.~\ref{expl:coalgml}.\ref{item:cond}. E.g.\ we have
  
(it is an amusing exercise to verify the last claim).
\end{exa}

\section{Strong Completeness Via Quasi-Canonical Models}

\noindent 
We wish to establish strong completeness of  by defining a
suitable -coalgebra structure  on the set~ of maximally
-consistent subsets of , equipped with the
standard valuation . The crucial
property required is that  be \emph{coherent}, i.e.

where , for 
-ary, , and ,
as this allows proving, by a simple induction over the structure of
formulas,
\begin{lem}[Truth lemma]
  If  is coherent, then for all formulas ,
   iff .
\end{lem}
\noindent We define a \emph{quasi-canonical model} to be a model
 with  coherent; the term quasi-canonical serves
to emphasise that the coherence condition does not determine the
transition structure  uniquely. By the truth lemma,
quasi-canonical models for  are -models, i.e.\ satisfy
all substitution instances of the frame conditions. The first question
is now under which circumstances quasi-canonical models exist; we
proceed to establish a widely applicable criterion. This criterion has
two main aspects: a \emph{local} form of strong completeness involving
only finite sets, and a preservation condition on the functor enabling
passage from finite sets to certain infinite sets. We begin with the
latter part:
\begin{defi}
  A \emph{surjective -cochain (of finite sets)} is a sequence
   of (finite) sets equipped with surjective
  functions  called \emph{projections}. The
  \emph{inverse limit}  of  is the set
   of
  \emph{coherent} families . The \emph{limit projections} are
  the maps , ; note that the
   are surjective, i.e.\ every  can be extended to a
  coherent family. Since all set functors preserve surjections,
   is a surjective -cochain with projections
  . The functor  \emph{weakly preserves inverse limits of
    surjective -cochains of finite sets} if for every
  surjective -cochain  of finite sets, the canonical
  map  is surjective, i.e.\ every
  coherent family  in  is \emph{induced} by a (not
  necessarily unique)  in the sense that
   for all .
\end{defi}

\begin{exa}\label{exa:cochains}
  Let  be a finite alphabet; then the sets , , form
  a surjective -cochain of finite sets with projections
  , . The inverse limit  is the set
   of infinite sequences over . The covariant powerset
  functor  preserves this inverse limit weakly: given a coherent
  family of subsets , i.e.\  for
  all , we define the set  as the set of all
  infinite sequences  such that  for all ; it is easy to check that indeed  induces the
  , i.e.\ . However,  is by no means uniquely
  determined by this property: Observe that  as just defined is a
  safety property. The intersection of  with any liveness property
  , e.g.\ the set  of all infinite sequences containing
  infinitely many occurrences of a fixed letter in , will also
  satisfy  for all .
\end{exa}
\noindent The second part of our criterion is an infinitary version of
a local completeness property called one-step completeness, which has
been used previously in \emph{weak} completeness
proofs~\cite{Pattinson03,Schroder07}.
\begin{defi}
  We say that  is \emph{strongly one-step complete over finite
    sets} if for finite , every one-step consistent subset 
  of  is one-step satisfiable.
\end{defi}
\noindent 
The difference with plain one-step completeness is that  above
may be infinite. Consequently, strong and plain one-step completeness
coincide in case the modal similarity type  is finite, since
in this case,  is, for finite , finite up
to propositional equivalence. The announced strong completeness
criterion is now the following.
\begin{thm}\label{thm:can-model}
  If  is strongly one-step complete over finite sets and
  separating,  is countable, and  weakly preserves inverse
  limits of surjective -cochains of finite sets, then 
  has a quasi-canonical model.
\end{thm}
\proof[Proof sketch]
  The most natural argument is via the dual adjunction between sets
  and boolean algebras that associates to a set the boolean algebra of
  its subsets, and to a boolean algebra the set of its
  ultrafilters. For economy of presentation, we outline a direct
  proof instead: we prove that
  \begin{itemize}
  \item[()] every maximally one-step consistent
     is one-step satisfiable,\\ where
    .
  \end{itemize}
  The existence of the required coherent coalgebra structure 
  on  follows immediately, since the coherence requirement for
  , , amounts to one-step satisfaction of
  a maximally one-step consistent subset of . 

  To prove (), let , let
  , let  denote the set of
  -formulas of modal nesting depth at most  that employ
  only modal operators from  and only the
  atomic propositions , and let  be the set of
  maximally consistent subsets of . Then  is (isomorphic
  to) the inverse limit , where the projections
   and the limit projections  are just
  intersection with . As the sets  are finite, we
  obtain by strong one-step completeness  such that
  , where
  . By separation,
   is coherent, and hence is induced by some  by weak preservation of inverse limits; then,
  .\qed
\noindent
Together with the Lindenbaum Lemma we obtain strong completeness as
a corollary.
\begin{cor} \label{cor:str-comp}
Under the conditions of Thm.~\ref{thm:can-model},

is strongly complete for -models.
\end{cor}
\noindent Both Thm.~\ref{thm:can-model} and
Cor.~\ref{cor:str-comp} do apply to the case that  has
frame conditions. When  is of rank~1 (i.e.\
), Cor.~\ref{cor:str-comp} implies that 
is strongly complete for (models based on) -frames.  In the
presence of frame conditions, the underlying frame of an -model
need not be an -frame, so that the question arises whether
 is also strongly complete for -frames. In applications,
positive answers to this question, usually referred to as the
canonicity problem, typically rely on a judicious choice of
quasi-canonical model to ensure that the latter is an -frame,
often the largest quasi-canonical model under some ordering on
. Detailed examples are given in Sec.~\ref{sec:examples}.
\begin{rem}
  It is shown in~\cite{KurzRosicky} that  admits a strongly
  complete modal logic if  weakly preserves (arbitrary) inverse
  limits \emph{and preserves finite sets}. The essential contribution
  of the above result is to remove the latter restriction, which fails
  in important examples. Moreover, the observation that we need only
  consider \emph{surjective} -cochains is relevant in some
  applications, see below.
\end{rem}

\begin{rem}
  A last point that needs clearing up is whether strong completeness
  of coalgebraic modal logics can be established by some more general
  method than quasi-canonical models of the quite specific shape used
  here. The answer is negative, at least in the case of rank-1 logics
  : it has been shown in~\cite{KurzPattinson05} that every such
   admits models which consist of the maximally
  \emph{satisfiable} sets of formulas and obey the truth lemma. Under
  strong completeness, such models are quasi-canonical.

  This seems to contradict the fact that some canonical model
  constructions in the literature, notably the canonical Kripke models
  for graded modal logics~\cite{Fine72,DeCaro88}, employ state spaces
  which have multiple copies of maximally consistent sets.  The above
  argument indicates that such logics fail to be coalgebraic, and
  indeed this is the case for GML with Kripke
  semantics. As mentioned above, GML has an alternative
  coalgebraic semantics over multigraphs, and we show below that this
  semantics does admit quasi-canonical models in our sense.
\end{rem}

\section{Examples}\label{sec:examples}

\noindent We now show how the generic results of the previous section
can be applied to obtain canonical models and associated strong
completeness and compactness theorems for a large variety of
structurally different modal logics. We have included some negative
examples where canonical models necessarily fail to exist due to
non-compactness, and we analyse which conditions of
Thm.~\ref{thm:can-model} fail in each case. We emphasise that in
the positive examples, the verification of said conditions is entirely
stereotypical. Weak preservation of inverse limits of surjective
-cochains usually holds without the finiteness assumption,
which is therefore typically omitted.

\begin{exa}[Strong completeness of Kripke semantics for ]
Recall from Expl.~\ref{expl:coalgml}.\ref{item:Kripke} that Kripke
  frames are coalgebras for the powerset functor .
  Strong completeness of  with respect to Kripke semantics is, of
  course, well known.  We briefly illustrate how this can be derived
  from our coalgebraic treatment.  To see that  is strongly
  one-step complete over finite sets , let
   be maximally one-step
  consistent. It is easy to check that  satisfies .  To prove that the
  powerset functor weakly preserves inverse limits, let  be an
  -cochain, and let  be a coherent
  family. Then  is itself a cochain, and the set  induces  (w.r.t.\ the subset
  ordering on ). Separation is clear. By
  Thm.~\ref{thm:can-model}, there exists a quasi-canonical Kripke
  model for all normal modal logics.  In particular, the standard
  canonical model~\cite{Chellas80} is quasi-canonical; it witnesses
  strong completeness (w.r.t.\ frames) of all canonical logics such as
  , , .
\end{exa}

\begin{exa}[Failure of strong completeness of  over finitely
branching models]
As seen in Expl.~\ref{expl:coalgml}.\ref{item:Kripke}, finitely
branching Kripke frames are coalgebras for the finite powerset functor
. It is clear that quasi-canonical models fail to exist
in this case, as compactness fails over finitely branching frames: one
can easily construct formulas  that force a state to have at
least  different successors.  The obstacle to the application of
Thm.~\ref{thm:can-model} is that the finite powerset functor fails to
preserve inverse limits weakly, as the inverse limit of an
-cochain of finite sets may fail to be finite.
\end{exa}

\begin{exa}[Conditional logic]
  Recall from Expl.~\ref{expl:coalgml}.\ref{item:cond} that the
  conditional logic  is interpreted over the functor
  . To prove strong one-step completeness
  over finite sets , let 
  be maximally one-step consistent. Define  by
  ; it is
  mechanical to check that . To see that  weakly
  preserves inverse limits, let  be a surjective
  -cochain, let , and let 
  be coherent. Define  by letting 
  for a coherent family  iff whenever 
  for some  and some , then . Using
  surjectivity of the projections of , it is straightforward to
  prove that  induces . Finally, separation is clear. By
  Thm.~\ref{thm:can-model}, it follows that the conditional logic
   has a quasi-canonical model, and hence that  is strongly
  complete for conditional frames.  In the case of the additional
  rank-1 axioms mentioned in
  Expl.~\ref{expl:axioms}.\ref{item:ax-cond} and the corresponding
  subfunctors of  described in Expl.~\ref{expl:subfunctors}, the
  situation is as follows.

  \textbf{Identity:} The functor  weakly preserves
  inverse limits of surjective -cochains. In the notation
  above, put  iff the condition above holds and
  .

  \textbf{Identity and disjunction:} The functor
   weakly preserves inverse limits of
  surjective -cochains: put  iff 
  and whenever , then .

  \textbf{System C:} It is open whether the the functor
   weakly preserves inverse
  limits of surjective -cochains, and whether System C is
  strongly complete over conditional frames.

  Indeed it appears to be an open problem to find \emph{any} semantics
  for which System C is strongly complete, other than the generalised
  neighbourhood semantics as described e.g.\
  in~\cite{SchroderPattinson07mcs}, which is strongly complete for
  very general reasons but provides little in the way of actual
  semantic information. The classical preference semantics according
  to Lewis is only known to be weakly
  complete~\cite{Burgess81}. Friedman and
  Halpern~\cite{FriedmanHalpern01} do silently prove strong
  completeness of System C w.r.t.\ plausibility measures; however, on
  close inspection the latter turn out to be essentially equivalent to
  the above-mentioned generalised neighbourhood semantics. Moreover,
  Segerberg~\cite{Segerberg89} proves strong completeness for a whole
  range of conditional logics over \emph{general} conditional frames,
  where, in analogy to corresponding terminology for Kripke frames, a
  general conditional frame is equipped with a distinguished set of
  \emph{admissible propositions} limiting both the range of valuations
  and the domain of selection functions. In contrast, our method
  yields full conditional frames in which the frame conditions hold
  for \emph{any} valuation of the propositional variables. While in
  the case of  and its extension by  alone, these models
  differ from Segerberg's only in that they insert default values for
  the selection function on non-admissible propositions, the
  canonical model for the extension of  by 
  has non-trivial structure on non-admissible propositions, and we
  believe that our strong completeness result for this logic is
  genuinely new.
\end{exa}


\begin{exa}[Strong completeness of GML over
  multigraphs]\label{expl:gml-infty}
  Recall from Expl.~\ref{expl:coalgml}.\ref{item:gml} that graded
  modal logic (GML) has a coalgebraic semantics in terms of the multiset
  functor . To prove strong one-step completeness over
  finite sets , let  be
  maximally one-step consistent. We define  by
  ; it is easy to check that  is
  well-defined and additive. To prove weak preservation of inverse
  limits, let  be an -cochain, let , and
  let  be coherent. Then define
   pointwise by

noting that the sequence  is decreasing by coherence. A
straightforward computation shows that  induces . Separation
is clear.

By the above and Thm.~\ref{thm:can-model}, all extensions of GML have
quasi-canonical multigraph models. While the technical core of the
construction is implicit in the work of Fine~\cite{Fine72} and
de~Caro~\cite{DeCaro88}, these authors were yet unaware of multigraph
semantics, and hence our result that \emph{GML is strongly complete
  over multigraphs} has not been obtained previously. 



The standard frame conditions for reflexivity, symmetry, and
transitivity (Expls.~\ref{expl:coalgml}.\ref{item:ax-gml}
and~\ref{expl:structure}.~\ref{item:gml-struct}) and arbitrary
combinations thereof are easily seen to be satisfied in the
quasi-canonical model constructed above. We point out that this
contrasts with Kripke semantics in the case of the graded version of
, i.e.\ GML extended with the reflexivity and transitivity axioms
of Expl.~\ref{expl:coalgml}.\ref{item:ax-gml}: as shown
in~\cite{FattorosiBarnabaCerrato88}, the complete axiomatisation of
graded modal logic over transitive reflexive Kripke frames includes
two rather strange combinatorial artefacts, which by the above
disappear in the multigraph semantics. The reason for the divergence
(which we regard as an argument in favour of multigraph semantics) is
that, while in many cases multigraph models are easily transformed
into equivalent Kripke models by just making copies of states, no such
translation exists in the transitive reflexive case (transitivity
alone is unproblematic).

Observe moreover that the above extends straightforwardly to
decription logics  with qualified number restrictions
and a role hierarchy  where roles may be distinguished as, in
any combination, transitive, reflexive, or symmetric. As shown
in~\cite{HorrocksEA99,KazakovEA07},  is undecidable for
many , even when only transitive roles are considered. For
undecidable logics, completeness is in some sense the `next best
thing', as it guarantees if not recursiveness then at least recursive
enumerability of all valid formulas, and hence enables automatic
reasoning. Essentially, our results show that the natural
axiomatisation of \emph{ with transitive, symmetric and
  reflexive roles is strongly complete over multigraphs}, a result
which fails for the standard Kripke semantics.
\end{exa}





\begin{exa}[Failure of strong completeness of image-finite GML]
\label{expl:gml-finite}
Similarly to the case of image-finite Kripke frames, one can model an
image-finite version of graded modal logic coalgebraically by
exchanging the functor  for the \emph{finite multiset
  functor} , where  consists of all maps 
with finite support. Of course, the resulting logic is non-compact and
hence fails to admit a canonical model. This is witnessed not only by
the same family of formulas as in the case of image-finite Kripke
semantics, which targets finiteness of the number of different
successors, but also by the set of formulas , which targets finiteness of multiplicities. Analysing the
conditions of Thm.~\ref{thm:can-model}, we detect two violations: not
only does weak preservation of inverse limits fail, but there is also
no way to find an axiomatisation which is strongly one-step complete
over finite sets (again, consider sets ).
\end{exa}

\noindent Strong completeness of image-finite GML can be recovered by
slight adjustments to the syntax and semantics. We formulate a more
general approach, as follows.

\begin{exa}[Strong completeness of the logic of additive measures]
\label{expl:additive-measures}
We fix an at most countable commutative monoid~ (e.g.\
). We think of the elements of  as describing the measure
of a set of elements. To ensure compactness, we have to allow some
sets to have undefined measure. That is, we work with coalgebras for
the endofunctor  defined by
 
The modal logic of additive -valued measures is given by the
similarity type  where
 expresses that  has measure , i.e.\

 is clearly separating. The logic is axiomatised by the
following two axioms:

These axioms are strongly one-step complete over finite sets : if
 is maximally one-step
consistent, then  where  iff
 for some necessarily unique , in which case
. Moreover,  weakly preserves inverse limits
, with finite : a coherent family
 is induced by
, where  and  is easily seen
to be well-defined and additive. Theorem~\ref{thm:can-model} now
guarantees existence of quasi-canonical models. A simple example is , which induces a logic of even and odd.

For the case , we obtain a variant of graded modal logic
with finite multiplicities, where we code  as . However, it may still be the case that
a state has a family of successor sets of unbounded measure, so that
undefinedness of the measure of the entire state space just hides an
occurrence of infinity. This defect is repaired by insisting that the
measure of the whole state space is finite at the expense of
disallowing the modal operator  in the language, as follows.
\end{exa}







\newcommand{\GMLm}{\mathrm{GML}^-}
\begin{exa}[Strong completeness of finitely branching
  ]\label{expl:gmlm}

  To force the entire state space to have finite measure, we
  additionally introduce a \emph{measurability} operator~,
  interpreted by , and impose
  obvious axioms guaranteeing that measures on  are defined on
  boolean subalgebras of , in particular  (i.e.\
   is finite), and . In order to achieve
  compactness, we now leave a bolt hole on the syntactical side and
  exclude the operator . In other words, the syntax of  is
  given by the similarity type
,
and we interpret  over coalgebras for the functor 
 defined by

Separation is clear. 
The axiomatisation of  is given by the axiomatisation of the
modal logic of additive measures, the above-mentioned axioms on ,
and the additional axiom 

which compensates for the absence of . Strong one-step
completeness over finite sets and weak preservation of inverse limits
is shown analogously as in Expl.~\ref{expl:additive-measures}, so that
we obtain a \emph{strongly complete finitely branching graded modal
  logic }. The tradeoff is that the operator 
	is no longer expressible as  in  which only allows to formulate the implication
.
\end{exa}



\begin{exa}[Failure of strong completeness for PML over finitely
supported probability distributions]
Like image-finite graded modal logic, probabilistic modal logic as
introduced in Expl.~\ref{expl:coalgml}.\ref{item:pml} fails to be
compact, and violates the conditions of Thm.~\ref{thm:can-model} on
two counts, namely weak preservation of inverse limits and strong
one-step completeness over finite sets. The first issue is related to
image-finiteness, while the second is rooted in the structure of the
real numbers: e.g.\ the set  is finitely satisfiable but not satisfiable. 
\end{exa}

\newcommand{\PMLe}{\mathrm{PML}_e}
\begin{exa}[Strong completeness of the logic of exact
probabilities]
In order to remove the above-mentioned failure of compactness, we
consider the fragment of probabilistic modal logic containing only
operators  stating that a given event has probability exactly
. (This is, of course, less expressive than the operators  but
still allows reasonable statements such as that rolling a six on a die
happens with probability .) Moreover, we require probabilities to
be rational and allow probabilities to be undefined, thus following
the additive measures approach as outlined above, where we consider a
subfunctor of  defined by the requirement that the
whole set has measure~. However, we are able to impose stronger
conditions on the domain  of a probability
measure  on : we require that  and that ,
 imply , which is reflected in the additional
axioms  and . It is natural that we cannot force closure
under intersection, as there is in general no way to infer the exact
probability of  from the probabilities of  and .
Along the same lines as above, we now obtain quasi-canonical models,
and hence strong completeness and compactness, of the arising modal
logic of exact probabilities.

\end{exa}




\section{Conclusion}

\noindent We have laid out a systematic method of proving existence of
canonical models in a generic semantic framework encompassing a wide
range of structurally different modal logics. We have shown how this
method turns the construction of canonical models into an entirely
mechanical exercise where applicable, and points the way to obtaining
compact fragments of non-compact logics. As example applications, we
have reproved a number of known strong completeness result and
established several new results of this kind; specifically, the latter
includes strong completeness of the following logics.
\begin{sparitemize}
\item The modal logic of exact probabilities, with operators 
  `with probability exactly '.
\item Graded modal logic over transitive reflexive multigraphs, i.e.\
  the natural graded version of , and more generally description
  logic with role hierarchies including transitive, reflexive, and
  symmetric roles and qualified number restrictions also on non-simple
  (e.g.\ transitive) roles.
\item The conditional logic , i.e.\ with the
  standard axioms of identity and disjunction, interpreted over
  conditional frames.
\end{sparitemize}
A number of interesting open problems remain, e.g.\ to find further
strongly complete variants of probabilistic modal logic or to
establish strong completeness of the full set of standard axioms of
default logic, Burgess' System C~\cite{Burgess81}, over the
corresponding class of conditional frames.

\bibliographystyle{myabbrv}\bibliography{coalgml}

\iffalse
\newpage\appendix
\section{Omitted Proofs}


\subsection*{Full Proof of Theorem~\ref{thm:can-model}}

\noindent For a set  and , we write  if .

To begin, we reduce the claim of the theorem to
  \begin{itemize}
  \item[()] every maximally one-step consistent
     is one-step satisfiable,\\ where
    .
  \end{itemize}
  Let . We have to prove that there exists
   such that 
  
  for  -ary, , and
  . This means that
  , where the set
   is obtained as follows. Let
   denote the set of propositional variables , where 
  ranges over , let  be the -valuation
  defined by , and let  be the
  substitution defined by . Then  consists
  of all formulas  where
   is such that . As
  shown in (the full version of)~\cite{SchroderPattinson07mcs},
   is (maximally) one-step consistent, hence satisfiable by
  strong one-step completeness, so that  exists as
  required.

  To prove (), let , let
  , let  denote the set of
  -formulas of modal nesting depth at most  that employ
  only modal operators from  and only the
  atomic propositions , and let  be the set of
  maximally consistent subsets of . Then  is (isomorphic
  to) the inverse limit , where the projections
   and the limit projections  are
  just intersection with : it is clear that  is
  uniquely determined by all its projections , and conversely, a coherent family  for
  , i.e.\ 
  comes from  defined by :
  . As
  the sets  are finite, we obtain by strong one-step completeness
   such that , where . We use separation to show that
   is coherent: It suffices to show that
  
  for  -ary and .  By
  naturality of , the left hand side is equivalent to
  
  By finiteness of the , there exists, for each , a formula
   such that . Then
  . Thus, both sides of ()
  are equivalent to .

  Since  weakly preserves the inverse limit of , it follows
  that the family  is induced by some . Then,
  : we have to show
  
  for  -ary, . We have 
  and  such that ,
  . Thus,  for all
  , so that the left hand side is by naturality of 
  equivalent to
  
  which is equivalent to the right hand side by construction of 
  .  \qed
\fi

\end{document}
