\documentclass[a4paper]{llncs}


\usepackage{graphicx}
\usepackage{amssymb}
\usepackage{amsmath}
\usepackage{enumerate}
\usepackage{scalefnt}
\usepackage{setspace} 

\usepackage{url}
\newcommand{\keywords}[1]{\par\addvspace\baselineskip
\noindent\keywordname\enspace\ignorespaces#1}
\newcommand{\bm}[1]{\mbox{\boldmath{}}}
\newcommand{\bms}[1]{\mbox{\boldmath{{\footnotesize}}}}
\newcommand{\bmss}[1]{\mbox{\boldmath{{\tiny}}}}
\newcommand{\argmax}{\operatornamewithlimits{argmax}}
\newcommand{\argmin}{\operatornamewithlimits{argmin}}

\newtheorem{defi}{Definition}
\newtheorem{thm}{Theorem}
\newtheorem{lem}{Lemma}
\newtheorem{clm}{Claim}
\newtheorem{coro}{Corollary}
\newtheorem{rmk}{Remark}
\newtheorem{fct}{Fact}
\newtheorem{cjt}{Conjecture}
\newtheorem{pro}{Proposition}

\begin{document}

\setstretch{0.96}

\mainmatter

\title{Minimax Regret 1-Median Problem \\ in Dynamic Path Networks}

\author{
Yuya Higashikawa \inst{1,6}
\and Siu-Wing Cheng \inst{2}
\and Tsunehiko Kameda \inst{3}
\and Naoki Katoh \inst{4,6}\thanks{Supported by JSPS Grant-in-Aid for Scientific Research(A)(25240004)}
\and Shun Saburi \inst{5}
}

\institute{
{Department of Information and System Engineering, Chuo University, Japan, 
higashikawa.874@g.chuo-u.ac.jp}
\and
{Department of Computer Science and Engineering, The Hong Kong University of Science and Technology, Hong Kong, 
scheng@cse.ust.hk}
\and
{School of Computing Science, Simon Fraser University, Canada,
tiko@sfu.ca}
\and
{Department of Informatics, Kwansei Gakuin University, Japan,
naoki.katoh@gmail.com}
\and
{Department of Architecture and Architectural Engineering, Kyoto University, Japan,
as-saburi@archi.kyoto-u.ac.jp}
\and
{CREST, Japan Science and Technology Agency (JST), Japan}
}


\maketitle


\begin{abstract}
This paper considers the minimax regret 1-median problem in dynamic path networks.
In our model, we are given a dynamic path network consisting of an undirected path with positive edge lengths, uniform positive edge capacity,
and nonnegative vertex supplies.
Here, each vertex supply is unknown but only an interval of supply is known.
A particular assignment of supply to each vertex is called a {\it scenario}. 
Given a scenario  and 
a sink location
 in a dynamic path network,
let us consider the evacuation time to  of a unit supply
given on a vertex by . 
The cost of  under  is defined as the sum of evacuation times to  for all supplies given by , 
and the {\it median} under  is defined as a sink location which minimizes this cost.
The regret for  under  is defined as the cost of  under  minus the cost of the median under . 
Then, the problem is to find a sink location such that the maximum regret for all possible scenarios is minimized.
We propose an  time algorithm for the minimax regret 1-median problem in dynamic path networks with uniform capacity, 
where  is the number of vertices in the network. 
\keywords{minimax regret, sink location, dynamic flow, evacuation planning}
\end{abstract}

\section{Introduction}
The Tohoku-Pacific Ocean Earthquake happened in Japan on March 11, 2011, 
and many people failed to evacuate and lost their lives due to severe attack by tsunamis. 
From the viewpoint of disaster prevention from city planning and evacuation planning,  
it has now become extremely important to establish effective evacuation planning systems against large scale disasters in Japan. 
In particular, arrangements of tsunami evacuation buildings in large Japanese cities near the coast has become an urgent issue. 
To determine appropriate tsunami evacuation buildings, we need to consider where evacuation buildings are located 
and how to partition a large area into small regions so that one evacuation building is designated in each region. 
This produces several theoretical issues to be considered. 
Among them, this paper focuses on the location problem of the evacuation building assuming that we fix the region such that all evacuees in the region are planned to evacuate to this building. 
In this paper, we consider the simplest case for which the region consists of a single road. \\
\indent
In order to represent the evacuation, we consider the {\it dynamic} setting in graph networks, which was first introduced by Ford et al.~\cite{ff58}.
In a graph network under the dynamic setting, each vertex is given supply and each edge is given length and capacity which limits the rate of the flow into the edge per unit time.
We call such networks under the dynamic setting {\it dynamic networks}.
Unlike in static networks, the time required to move supply from one vertex to a sink can be increased due to congestion caused by the capacity constraints, 
which require supplies to wait at vertices until supplies preceding them have left.
In this paper, we consider the flow on dynamic networks as continuous, that is, each input value is given as a real number, and supply, flow and time are defined continuously.
Then each supply can be regarded as fluid, and edge capacity is defined as the maximum amount of supply which can enter an edge per unit time.
The {\it 1-sink location problem in dynamic networks} is defined as the problem which requires to find the optimal location of a sink in a given dynamic network 
so that all supplies are sent to the sink as quickly as possible. \\
\indent
In order to evaluate an evacuation, we can naturally consider two types of criteria: {\it completion time criterion} and {\it total time criterion}.
In this paper we adopt the latter one (for the former one, refer to \cite{h14,hgk14_2,hgk14_4,mumf06}).
We here define a {\it unit} as an infinitesimally small portion of supply.
Given a sink location  in a dynamic network, let us consider an evacuation to  starting at time  
and define the {\it evacuation time} of a unit to  as the time at which the unit reaches  in the evacuation.
The total time for the evacuation to  is defined as the sum of evacuation times over all infinitesimal units to .
Then, the minimum total time for all possible evacuations to  could be the criterion for the optimality of sink location, which we adopt.
Given a dynamic network, we define the {\it 1-median problem} as the problem which requires to find a sink location minimizing the minimum total time, 
and the optimal solution is called the {\it median}.\\
\indent
Although the above criterion is reasonable for the sink location, it may not be practical since the number of evacuees in an area may vary depending on the time
(e.g., in an office area in a big city, there are many people during the daytime on weekdays while there are much less people on weekends or during the night time). 
So, in order to take into account the uncertainty of population distribution,
we consider the {\it maximum regret} for a sink location as another evaluation criterion assuming that for each vertex, we only know an interval of vertex supply. 
Then, the {\it minimax regret 1-median problem in dynamic path networks} is formulated as follows.
A particular assignment of supply to each vertex is called a {\it scenario}.
Here, for a sink location  and a scenario , we denote the minimum total time by .
Also let  denote the median under .
The problem can be understood as a 2-person Stackelberg game as follows.
The first player picks a sink location  and the second player chooses a scenario  that maximizes the {\it regret} defined as .
The objective of the first player is to choose  that minimizes the maximum regret. \\
\indent
Related to the minimax regret facility location in graph networks, especially for trees, some efficient algorithms have been presented by~\cite{ab00,bk12,bks13,bgk08,cl98,c08}.
For dynamic networks, Cheng et al.~\cite{chknsx13} first studied the
{\it minimax regret 1-center problem} in path networks,
which requires to find a sink location in a path that minimizes the maximum regret where the completion time criterion is adopted instead of the total time one.
They presented an  time algorithm.
Higashikawa et al.~\cite{hacknsx14} improved the time bound by \cite{chknsx13} to , and also Wang~\cite{w13} independently achieved the same time bound of  with better space complexity.
Very recently, Bhattacharya et al.~\cite{bk14} have improved the time bound to .
The above problem was extended to the multiple sink location version by Arumugam et al.~\cite{aags14}.
For the minimax regret -center problem in dynamic path networks with uniform capacity, they presented an  time algorithm, and this time bound was improved to  recently~\cite{h14}.
On the other hand, for dynamic tree networks, only the minimax regret 1-center problem was solved in  time~\cite{hgk14,hgk14_3}.

This paper first considers the minimax regret median problem in dynamic networks while all the above works for dynamic networks treated center problems.
In this paper, we address the minimax regret 1-median problem in dynamic path networks with uniform capacity and present an  time algorithm. \\





















\section{Preliminaries}
\label{sec:pre}

\subsection{Dynamic path networks under uncertain supplies}
\label{subsec:dpn}
Let  be an undirected path with ordered vertices   and  edges  where
  for .
Let  be a dynamic network 
with the underlying path graph ; 
 is a function that associates each edge  with positive length , 
 is a function that associates each vertex  with positive weight , amount of supply at ;
 is the capacity, a positive constant representing the amount of supply which can enter an edge per unit time;
 is a positive constant representing the time required for a flow to travel a unit distance.
In our model, instead of the weight function  on vertices, we are given the weight interval function 
that associates each vertex  with an interval of supply  with .
We call such a network  with path structures {\it a dynamic path network under uncertain supplies}.

In the following, we write  to indicate that a point is a vertex of  or lies on one of the edges of .  
For any point ,
we abuse this notation by also letting  denote the distance from  to . 
Informally we can regard  as being embedded on a real line with .
For two points  satisfying , 
let  (resp. ,  and ) denote an interval in  consisting of all points  
such that  (resp. ,  and ).

\subsection{Scenarios}
\label{subsec:scn}
Let  denote the Cartesian product of all  for :

An element of , i.e., a particular assignment of weight to each vertex, is called a {\it scenario}.
Given a scenario , we denote by  the weight of a vertex  under .


\subsection{Total evacuation time}
\label{subsec:to}
In our model, the supply is defined continuously.
We define a unit as an infinitesimally small portion of supply.
Given a sink location  and a scenario , 
without loss of generality, an evacuation to  under  is assumed to satisfy the following assumptions.
When a unit arrives at a vertex  on its way to , it has to wait for the departure if there are already some units waiting for leaving . 
All units waiting at  for leaving  are processed in the first-come first-served manner. 

For a given  and , let us consider an evacuation to  under  starting at time 
and define the evacuation time of a unit to  under  as the time at which the unit reaches .
Let  denote the sum of evacuation times over all infinitesimal units to  under .
Also let  (resp. ) denote the sum of evacuation times to  under  for all units on  (resp. ).
Then,  is obviously the sum of  and , i.e., 

Without loss of generality, we assume  and .

We will show the formula of  that has been proved in \cite{hgk14_2,hgk14_4}.
Suppose that  is located in an open interval  with , i.e., .
We here show only the formula of  (the case of  is symmetric).
First, let us define the vertex indices  inductively as

where .
Obviously  holds.
We then call a set of all units on  {\it i-th cluster},
and a vertex  the {\it head} of -th cluster (see Fig.~\ref{fig:cl}).
\begin{figure}[b]
\centering
\includegraphics[width=90mm,clip]{clusters.eps} 
\caption{Illustration of left clusters for  where -th cluster is headed by a vertex }
\label{fig:cl}
\end{figure}

\noindent
Also, for , we define  as ,
which is called the weight of -th cluster.
The interpretation can be derived from \cite{hgk14_2,hgk14_4} as follows.
The first unit on each  does not encounter any congestion on its way to . 
Here, although  may not be uniquely determined by (\ref{ms2.5}),
we choose the maximum index as .
By this assumption and (\ref{ms2.5}), the following inequality holds for :

In other words, even if we transform the input so that all units on  are moved to  for ,
the sum of evacuation times over  never changes.
In the following, we call such a transformation the {\it left-clustering} for  and
clusters obtained by the left-clustering for  {\it left clusters} for 
(the right-clustering and right clusters are symmetric).
Then as in \cite{hgk14_2,hgk14_4},  is written as 










\subsection{Minimax regret formulation}
\label{subsec:mi}
For a scenario , let  be a point in  that minimizes  over , called the {\it median} under . 
We now define the {\it regret} for  under  as

Moreover, we also define the {\it maximum regret} for  as

If , 
we call  the {\it worst case scenario} for .
The goal is to find a point , called the {\it minimax regret median}, that minimizes  over ,
i.e., the objective is to





\subsection{Known properties for the fixed scenario case}
\label{subsec:kp}
We here show some properties on the 1-median problem in a dynamic path network  when a scenario  is given, which were basically presented in \cite{hgk14_2,hgk14_4}.
We first introduce the following two lemmas.


\begin{lem}\hspace{-1mm}{\rm \cite{hgk14_2,hgk14_4}}
For a scenario ,  is at a vertex in .
\label{lem:ms}
\end{lem}



\begin{lem}\hspace{-1mm}{\rm \cite{hgk14_2,hgk14_4}}
For a scenario , all  over  can be computed in  time in total.
\label{lem:pht}
\end{lem}


\noindent
We then can see a corollary of these lemmas.

\begin{coro}\hspace{-1mm}{\rm \cite{hgk14_2,hgk14_4}}
For a scenario ,  and  can be computed in  time.
\label{coro:mst}
\end{coro}



Now let us look at the formula of (\ref{eq:ms3}).
Even if  is moving on an edge  (not including endpoints  and ), the formation of left clusters for  does not change over .
Therefore,  is a linear function of , and symmetrically,  is also a linear function.
For , letting  and  be the values such that for ,

we can derive the following lemma from \cite{hgk14_2,hgk14_4}.

\begin{lem}
For a scenario , all  and  over  can be computed in  time in total.
\label{lem:abt}
\end{lem}





\section{Properties of worst case scenarios}
\label{sec:pwcs}

In this section, we show the important properties which worst case scenarios have.
In our problem, a main difficulty lies in evaluating  over  to compute  even for a fixed  since 
the size of  is infinite.
We thus aim to find a scenario set with a finite size (in particular, a polynomial size)
which includes a worst case scenario for any .
In order to do this, we introduce a new concept, the {\it gap} between two points  under a scenario , defined by

By Lemma \ref{lem:ms} and the definition of (\ref{eq:rg}), we have

and by (\ref{eq:rmax}) and (\ref{eq:rg2}),

From (\ref{eq:rmax2}), if we can compute  for a fixed pair ,
 can also be computed by repeating the same maximization over .
We call a scenario that maximizes  for a fixed  a worst case scenario for .
In the following, we show a scenario set of size  that includes a worst case scenario for a fixed ,
which implies a scenario set of size  that includes a worst case scenario for a fixed . 


\subsection{Bipartite scenario}
\label{subsec:bs}
We first introduce the concept of the {\it bipartite scenario}, which was originally introduced as the {\it dominant scenario} in \cite{chknsx13,hacknsx14}.
Let us consider a scenario .
A scenario  is said to be {\it left-bipartite} (resp. {\it right-bipartite})
if  (resp. ) over  and 
 (resp. ) over  for some .
Obviously the number of such scenarios is .
The authors of \cite{chknsx13,hacknsx14} treated the minimax regret 1-center problem in dynamic path networks,
which requires to find a sink location in a path that minimizes the maximum regret similarly defined as (\ref{eq:rmax})
where the completion time criterion is adopted instead of the total time one.
They proved that for any point in an input path, at least one worst case scenario is left-bipartite or right-bipartite.


\subsection{Pseudo-bipartite scenario}
\label{subsec:pbs}
We here introduce the concept of the {\it pseudo-bipartite scenario}.
A scenario  is said to be {\it left-pseudo-bipartite} (resp. {\it right-pseudo-bipartite})
if  (resp. ) over  and 
 (resp. ) over  for some .
In this definition, we do not care about the weight of a vertex , called the {\it intermediate vertex}.
Given a pseudo-bipartite scenario with the intermediate vertex , we call intervals  and  the {\it left part} and the {\it right part}, respectively.
Let  (resp. ) denote a set of all left-pseudo-bipartite scenarios (resp. right-pseudo-bipartite scenarios).
We then prove the following lemma (the proof is given in Appendix A).
\begin{lem}
Given a pair  satisfying  (resp. ), there exists a worst case scenario for  belonging to  (resp. ) such that  (resp. ) is in the left part and  (resp. ) is in the right part.
\label{lem:wcsxy}
\end{lem}












\subsection{Critical pseudo-bipartite scenario}
\label{subsec:cpbs}
By Lemma \ref{lem:wcsxy}, we studied the property of a worst case scenario for a fixed ,
however 
the sizes of  and 
are still infinite since the weight of the intermediate vertex in a pseudo-bipartite scenario is not fixed.
In the rest of this section, we focus on the weight of the intermediate vertex in a pseudo-bipartite scenario
which is worst for .

Given a pair  satisfying ,
let us consider a scenario  
such that the intermediate vertex is  and .
Suppose that the weight of  is set as the minimum, i.e., .
Performing the right-clustering for  under  (mentioned in Section \ref{subsec:to}), 
we will get right clusters for  such that 
for ,
the head of -th cluster is 
and the weight of -th cluster is .
Then, suppose that the intermediate vertex  belongs to -th cluster.


Now let us increase the weight of , little by little, without changing the weight of any other vertex.
Let 
 be a scenario in  such that the intermediate vertex is  whose weight is .
Suppose that when the weight of  reaches some value , the following equality holds:

Note that  corresponds to the weight of -th cluster under .
At that moment, referring to (\ref{eq:ms2.6}), -th cluster is merged to -th cluster.
We then call  a {\it critical left-pseudo-bipartite scenario} for .
Also, 
 and  are assumed to be critical left-pseudo-bipartite scenarios for 
even if any merge does not occur at those moments.
{\it Critical right-pseudo-bipartite scenarios} for  are symmetrically defined.
Let  denote a set of all critical left-pseudo-bipartite scenarios and critical right-pseudo-bipartite scenarios for ,
and .
We will show two lemmas
(the proof of Lemma \ref{lem:wcsxy2} is given in Appendix B).

\begin{lem}
Given a pair , there exists a worst case scenario for  belonging to .
\label{lem:wcsxy2}
\end{lem}
\begin{lem}
For a vertex , 
the size of  is ,
and all scenarios in  can be computed in  time.
\label{lem:csy}
\end{lem}
\begin{proof}
We first prove that 
the number of critical left-pseudo-bipartite scenarios for  is 
(the case of critical right-pseudo-bipartite scenarios is symmetric).
Suppose that .
For  and , let 
 be a scenario in 
such that the intermediate vertex is  whose weight is .
Here, let us define the order between two scenarios  and :
 holds if and only if 
(a)  or 
(b)  and . 
For , we also define  and  as follows.
Let  be the number of critical left-pseudo-bipartite scenarios for  such that the intermediate vertex is 
(including  and ).
Let  be, under a scenario , the number of right clusters for  that follow a cluster including .



Let us consider computing all critical left-pseudo-bipartite scenarios for  in ascending order,
and suppose that the weight of  now increases from  to .
While it increases,
since  critical left-pseudo-bipartite scenarios for  occur (except  and ),
and at each such scenario, one or more clusters are merged into one of ,
at least  clusters are merged into one of  in total.
We thus have  for , i.e.,

Note that the total number of critical left-pseudo-bipartite scenarios for  is exactly .
By (\ref{eq:lem1.21}), we have 

which is  since  and .

In the rest of the proof, we show that all critical left-pseudo-bipartite scenarios for  can be computed in  time.
Recall that all critical left-pseudo-bipartite scenarios for  are computed in ascending order.
The algorithm first gets ,
and performs the right clustering for  under .
As claimed in \cite{hgk14_2,hgk14_4}, it is easy to see that the right clustering for a fixed  can be done in  time.



Suppose that for particular  and ,  is critical for ,
and the algorithm has already obtained  and the right clusters for .
We then show how to compute the subsequent critical left-pseudo-bipartite scenario.
Let  be a right cluster for  including  and  be a right cluster for  immediately following .
Also, let  (resp. ) be the index of a vertex that corresponds to the head of  (resp. ),
and  (resp. ) be the weight of  (resp. ).

There are two cases: 
[Case 1] ; 
[Case 2] .
For Case 2,
we notice that  is equivalent to .
Therefore, this case immediately results in Case 1 by letting  be  and  be 
(although a right cluster for  including  may be , not ).
We thus consider only Case 1 in the following.

The algorithm will compute the subsequent critical left-pseudo-bipartite scenario 
where  satisfies .
In order to compute , the algorithm test if there exists  such that

which is similar to (\ref{eq:cpbs1}).
If yes, for such , the algorithm returns  and updates the right clusters for  by merging  into .
Otherwise,  is just returned. 
Such testing and updating are done in  time.
Since the number of critical left-pseudo-bipartite scenarios for  is  and each of those is computed in  time,
we completes the proof.
\qed
\end{proof}
By (\ref{eq:rmax2}), we have a corollary of Lemma \ref{lem:wcsxy2}.
\begin{coro}
Given a point , there exists a worst case scenario for  belonging to .
\label{coro:wcsx}
\end{coro}
Also, a corollary of Lemma \ref{lem:csy} immediately follows.
\begin{coro}
The size of  is ,
and all scenarios in  can be computed in  time.
\label{coro:cusy}
\end{coro}

\section{Algorithm}
\label{sec:al}
In this section, we show an algorithm that computes the minimax regret median, which minimizes  over .
The algorithm basically consists of two phases: \\
{\bf [Phase 1]} Compute  over , and \\
{\bf [Phase 2]} Compute  over . \\
After these, the algorithm evaluates all the  values obtained and finds the minimax regret median in  time.

By Corollary \ref{coro:wcsx}, we only have to consider scenarios in  to compute  for any .
Therefore, the algorithm computes all scenarios in  in advance, which can be done in  time by Corollary \ref{coro:cusy}.
Subsequently, it computes all the values  over  for Phase 1 and Phase 2.
By Corollaries \ref{coro:mst} and \ref{coro:cusy}, this can be done in  time in total.  

First let us see details in Phase 1. 
For a fixed scenario ,
since all  over  can be computed in  time by Lemma \ref{lem:pht}
and  has already been computed before Phase 1,
all  over  can also be computed in  time (refer to (\ref{eq:rg})).
After the algorithm obtains  over  in  time,
for each ,
 over  are evaluated to obtain .
Thus, it is easy to see that Phase 1 can be done in  time in total.


We next focus on Phase 2. 
As mentioned at the end of Section \ref{subsec:kp}, 
for a fixed scenario , 
 is a linear function of  for each  (not including  and ).
Therefore,  is also linear for  for each .
Referring to (\ref{eq:aibi}), a function  on an edge  is written as

Recall that  has already been computed.
Then, by Lemma \ref{lem:abt},  on  over  can be computed in  time.
After the algorithm does the same computation over  in  time,
on each edge ,
we have  linear functions  over .
By the definition of (\ref{eq:rmax}),  can be obtained by solving a linear programming problem in two dimensions with  constraints,
i.e.,

This problem can be solved in  time by \cite{d84}.
Repeating the same operations over , Phase 2 is completed in  time.

\begin{thm}
The minimax regret 1-median problem in dynamic path networks with uniform capacity can be solved in  time.
\label{thm:1}
\end{thm}

\section{Conclusion}
\label{sec:co}
In this paper, we address the minimax regret 1-median problem in dynamic path networks with uniform capacity and present an  time algorithm. 
Additionally, this is the first work that treats the minimax regret facility location problem in dynamic networks where the total time criterion is adopted.
Two natural questions immediately follow.
The first one is whether we can reduce the number of scenarios to be considered.
The other one is whether we can extend the problem to the -median version with ,
or the problem in more general networks.  

\begin{thebibliography}{99}

	\bibitem{aags14}
		G. P. Arumugam, J. Augustine, M. J. Golin, P. Srikanthan,
		\newblock ``A polynomial time algorithm for minimax-regret evacuation on a dynamic path",
		\newblock {\em CoRR abs/1404.5448},
		\newblock arXiv:1404.5448.

	\bibitem{ab00} 
		I.~Averbakh and O.~Berman, 
		\newblock ``Algorithms for the robust -center problem on a tree'', 
		\newblock {\em European Journal of Operational Research},
		\newblock 123(2), pp.~292-302, 2000.
	
	\bibitem{bk12}
		B.~Bhattacharya and T.~Kameda,
		\newblock ``A linear time algorithm for computing minmax regret -median on a tree",
		\newblock {\em Proc. the 18th Annual International Computing and Combinatorics Conference} (COCOON 2012),
		\newblock LNCS 7434, pp.~1-12, 2012. 
		
	\bibitem{bk14}
		B. Bhattacharya, T. Kameda,
		\newblock ``Improved algorithms for computing minmax regret 1-sink and 2-sink on path network",
		\newblock {\em Proc. The 8th Combinatorial Optimization and Applications} (COCOA 2014),
		\newblock LNCS 8881, pp.~146-160, 2014.
	
	\bibitem{bks13}
		B.~Bhattacharya, T.~Kameda and Z.~Song,
		\newblock ``A linear time algorithm for computing minmax regret -median on a tree network",
		\newblock {\em Algorithmica},
		\newblock pp.~1-20, 2013. 

	\bibitem{bgk08}
		G.~S.~Brodal, L.~Georgiadis and I.~Katriel,
		\newblock ``An  version of the Averbakh-Berman algorithm for the robust median of a tree",
		\newblock {\em Operations Research Letters},
		\newblock 36(1), pp.~14-18, 2008. 

	\bibitem{cl98}
		B.~Chen and C.~Lin, 
		\newblock ``Minmax-regret robust 1-median location on a tree'', 
		\newblock {\em Networks},
		\newblock 31(2), pp.~93-103, 1998.



	\bibitem{chknsx13}
		S.~W.~Cheng, Y.~Higashikawa, N.~Katoh, G.~Ni, B.~Su and Y.~Xu,
		\newblock ``Minimax regret 1-sink location problems in dynamic path networks",
		\newblock {\em Proc. The 10th Annual Conference on Theory and Applications of Models of Computation} (TAMC 2013),
		\newblock LNCS 7876, pp.~121-132, 2013.



	\bibitem{c08} 
		E.~Conde, 
		\newblock ``A note on the minmax regret centdian location on trees'',
		\newblock {\em Operations Research Letters},
		\newblock 36(2), pp.~271-275, 2008.

	\bibitem{d84} M.~E.~Dyer, 
		\newblock ``Linear time algorithms for two- and three-variable linear programs'',
		\newblock {\em SIAM Journal on Computing},
		\newblock 13(1), pp.~31-45, 1984.
		
	\bibitem{ff58}
		L.~R.~Ford~Jr., D.~R.~Fulkerson,
		\newblock ``Constructing maximal dynamic flows from static flows",
		\newblock {\em Operations Research},
		\newblock 6, pp.~419-433, 1958.
	
	\bibitem{h14}
		Y.~Higashikawa,
		\newblock ``Studies on the Space Exploration and the Sink Location under Incomplete Information towards Applications to Evacuation Planning",
		\newblock {\em Doctoral Dissertation},
		\newblock Kyoto University, 2014.
			
	\bibitem{hacknsx14}
		Y.~Higashikawa, J.~Augustine, S.~W.~Cheng, N.~Katoh, G.~Ni, B.~Su and Y.~Xu,
		\newblock ``Minimax Regret 1-Sink Location Problem in Dynamic Path Networks",
		\newblock {\em Theoretical Computer Science},
		\newblock DOI: 10.1016/j.tcs.2014.02.010, 2014.
	
	\bibitem{hgk14}
		Y.~Higashikawa, M.~J.~Golin, N.~Katoh,
		\newblock ``Minimax Regret Sink Location Problem in Dynamic Tree Networks with Uniform Capacity",
		\newblock {\em Proc. The 8th International Workshop on Algorithms and Computation} (WALCOM 2014),
		\newblock LNCS 8344, pp.~125-137, 2014.

	\bibitem{hgk14_2}
		Y.~Higashikawa, M.~J.~Golin, N.~Katoh,
		\newblock ``Multiple sink location problems in dynamic path networks",
		\newblock {\em Proc. The 10th International Conference on Algorithmic Aspects of Information and Management} (AAIM 2014),
		\newblock LNCS 8546, pp.~149-161, 2014.
		
	\bibitem{hgk14_3}
		Y.~Higashikawa, M.~J.~Golin, N.~Katoh,
		\newblock ``Minimax Regret Sink Location Problem in Dynamic Tree Networks with Uniform Capacity",
		\newblock {\em Journal of Graph Algorithms and Applications},
		\newblock 18(4), pp.~539-555, 2014.

	\bibitem{hgk14_4}
		Y.~Higashikawa, M.~J.~Golin, N.~Katoh,
		\newblock ``Multiple Sink Location Problems in Dynamic Path Networks",
		\newblock {\em Theoretical Computer Science},
		\newblock DOI:10.1016/j.tcs.2015.05.053.
	






	\bibitem{mumf06}
		S.~Mamada, T.~Uno, K.~Makino and S.~Fujishige,
		\newblock ``An  algorithm for the optimal sink location problem in dynamic tree networks",
		\newblock {\em Discrete Applied Mathematics},
		\newblock 154(16), pp.~2387-2401, 2006. 





	\bibitem{w13}
		H.~Wang, 
		\newblock ``Minmax regret 1-facility location on uncertain path networks",
		\newblock {\em Proc. The 24th International Symposium on Algorithms and Computation (ISAAC 2013)},
		\newblock LNCS 8283, pp.~733-743, 2013.

\end{thebibliography}

\clearpage

\section*{Appendices}

\subsection*{Appendix A: Proof of Lemma \ref{lem:wcsxy}}
\label{app:a}

We only treat the case of  satisfying  since the other case is symmetric.

We first prove the following claim.
\begin{clm}
Given a pair  satisfying ,
there exists a worst case scenario for  such that 
the weight of every vertex  is .
\label{clm:2}
\end{clm}

\noindent
{\it Proof of Claim \ref{clm:2}.} \
We here assume : if , the proof is straightforward.
Let  be a worst case scenario for .
If there are more than one worst case scenario, 
we choose the one such that all weights are lexicographically maximized in the order of ascending indices
among all worst case scenarios.
We prove by contradiction: suppose  for some vertex . 
Under a scenario , let us perform the left-clustering for  and , respectively (refer to Section \ref{subsec:to}).
Performing the left-clustering for ,
let  be a left cluster for  including , 
 be the index of a vertex that corresponds to the head of ,
and  be the weight of .
Also, performing the left-clustering for , ,  and  are similarly defined. 
When  evacuates to , 
by the definition of a cluster, the first unit of  does not encounter any congestion in the interval ,
however, it may encounter congestions at some vertices in .
If it does not encounter any congestion in , 
 is never merged to any other cluster for ,
which implies  and .
Otherwise,  is eventually merged into a left cluster for  headed by a vertex , i.e., .
We thus consider two cases: 
[Case 1]  and ; 
[Case 2]  and .



Now let  be a scenario obtained from  by increasing the weight of  by infinitesimally small , 
i.e.,  and  for .
We then show the following claim for the proof of Claim \ref{clm:2}.
\begin{clm}
While  changes to , 
the formation of left clusters for  (resp. ) remains the same.
\label{clm:1}
\end{clm}
{\it Proof of Claim \ref{clm:1}.}  \
Performing the left-clustering for  under ,
let  be a left cluster for  immediately following ,
and 
be the index of a vertex that corresponds to the head of .
Referring to (\ref{eq:ms2.6}), the following inequality holds:

Then, for a sufficiently small , we have

The inequality of (\ref{eq:clm1.1}) means that after performing the left-clustering for  under , the first unit of  does not catch up with the last unit of  at ,
and by (\ref{eq:clm1.2}), this remark also holds even for . 
Under , if  is not merged to , any other merge never occurs.
Thus, the formation of left clusters for  does not change,
and similarly, it does not change for . 
\qed


\medskip

By Claim \ref{clm:1} and the definitions of (\ref{eq:ms1}) and (\ref{eq:ms3}), we have 

and similarly, 

Also by the definition of (\ref{eq:gap}), we have

From (\ref{eq:lem1.1}), (\ref{eq:lem1.2}) and (\ref{eq:lem1.3}), we can derive 


If Case 1 occurs, 
we can immediately see that the right side of (\ref{eq:lem1.4}) is greater than zero, i.e., ,
which contradicts that  is a worst case scenario for .

If Case 2 occurs, 
performing the left-clustering for  merges  into  as mentioned above,
and then, all units on  are also merged into .
Therefore, if we consider an input such that all units on  are moved to ,
the first unit of  must catch up with the last unit of supply at ,
i.e.,

Since  includes  and all units on , we have

Note that in (\ref{eq:lem1.6}), the left side is less than the right side when  also includes some clusters for  following .
From (\ref{eq:lem1.4}), (\ref{eq:lem1.5}) and (\ref{eq:lem1.6}), we can derive 

which contradicts that  is a worst case scenario for .
\qed


\medskip

If we consider a worst case scenario for  such that the weight of every vertex  is  and weights of all other vertices in  are lexicographically minimized in the order of descending indices,
the following claim is also proved in a similar manner as in the proof of Claim \ref{clm:2}.
\begin{clm}
Given a pair  satisfying ,
there exists a worst case scenario for  such that 
the weight of every vertex  is  and 
the weight of every vertex  is .
\label{clm:3}
\end{clm}

Now, let  be a worst case scenario for  such that the weight of every vertex  is , 
the weight of every vertex  is , and
weights of all other vertices in the open interval  are lexicographically maximized in the order of ascending indices.
Then,  can be proved.
We prove by contradiction: there exist two vertices  satisfying  such that  and .
Let  be a scenario obtained from  by increasing the weight of  by infinitesimally small  and
decreasing the weight of  by the same , 
i.e., ,  and  for .
Then, we immediately see  and , therefore

The inequality of (\ref{eq:lem1.8}) implies that  is also a worst case scenario for ,
which contradicts the lexicographical maximality of weights on the open interval  under .
\qed






\subsection*{Appendix B: Proof of Lemma \ref{lem:wcsxy2}}
\label{app:b}
For a fixed pair  satisfying , 
let us consider a worst case scenario 
in  such that the intermediate vertex is  ().
We now consider the weight of  as a variable ,
and let 
 be a scenario in 
such that the intermediate vertex is  whose weight is .
Then, let  denote the gap between  and  under , i.e.,

Suppose that  is critically left-pseudo-bipartite for  when ,
where  is a positive integer and .
In the following, we prove that a function  is convex and piecewise-linear for  for every .


Under a scenario , let us perform the left-clustering for  and the right-clustering for , respectively.
Performing the left-clustering for ,
let  be a left cluster for  including , 
 be the index of a vertex that corresponds to the head of ,
and  be the weight of .
Also, performing the right-clustering for , ,  and  are similarly defined. 

We first show the following claim.
\begin{clm}
 is continuous for .
\label{clm:4}
\end{clm}
\noindent
{\it Proof of Claim \ref{clm:4}.} \
The statement is equivalent to

We here prove (\ref{eq:lem1.10}) (the case of (\ref{eq:lem1.11}) is similarly treated).
By (\ref{eq:lem1.9}), we only have to prove

Note that, similarly as in Claim \ref{clm:1}, while  changes to , 
the formation of left clusters for  remains the same.
Therefore, by the definitions of (\ref{eq:ms1}) and (\ref{eq:ms3}), we have 

which leads (\ref{eq:lem1.12}) by letting  go to positive zero.
\qed

\medskip

For an integer , we consider the right-derivative of  for , i.e.,

Similarly to (\ref{eq:lem1.13}), we have

From (\ref{eq:lem1.9}), (\ref{eq:lem1.13}) and (\ref{eq:lem1.15}), we derive

and by (\ref{eq:lem1.14}) and (\ref{eq:lem1.16}),

We here notice that as  increases,  and  never change (even if left clusters for  following  and right clusters for  following  are merged to  and , respectively).
Also, since both of  and  include , 
 and  can be represented as follows:

where  and  are functions of .
In addition,  increases only if a left cluster for  following  is merged to , 
i.e.,  is an increasing step function,
and  is a constant function of  since the formation of right clusters for  does not change over 
(recall the definition of critical left-pseudo-bipartite scenarios for  in Section \ref{subsec:cpbs}).
From the above observations, and (\ref{eq:lem1.17}), (\ref{eq:lem1.18}) and (\ref{eq:lem1.19}), we derive that for ,

which is an increasing step function.
By this fact and the continuity of  by Claim \ref{clm:4}, we have the following claim.
\begin{clm}
For an integer ,  is convex and piecewise-linear for .
\label{clm:5}
\end{clm}

By Claim \ref{clm:5}, a solution that maximizes  must be in ,
i.e., a worst case scenario for  is critically left-pseudo-bipartite for .
\qed









\end{document}