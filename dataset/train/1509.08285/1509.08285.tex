\section{Experiments}
\label{sec:experiments}

We evaluate Algorithm~\ref{alg:ip}.
It can solve large instances within minutes on a standard desktop computer;
our filtering techniques prove critical to success.
Especially \pointguardfilter and \witnessfilter, see Sections~\ref{sec:filters-pointguards} and~\ref{sec:filters-witnesses}, significantly increase the solvable instance size.
Our instances, the tested configurations of Algorithm~\ref{alg:ip}, the experimental setup, and our findings are described in Sections~\ref{sec:experiments-instance-description}, \ref{sec:configurations}, \ref{sec:experiments-setup}, and~\ref{sec:experiments-results}, respectively.

\subsection{Instances}
\label{sec:experiments-instance-description}

\begin{figure}
	\centering
	\subfigure[\walk: Random walk with uniform step width.]{
		\includegraphics[width=.95\linewidth]{pdf/walk.pdf}
		\label{fig:instance-walk}
	} \\
	\subfigure[\sinewalk: Sum of a sine wave and a random walk.]{
		\includegraphics[width=.95\linewidth]{pdf/sinewalk.pdf}
		\label{fig:instance-sinewalk}
	} \\
	\subfigure[\parabolawalk: Sum of a pa\-rab\-o\-la and a random walk.]{
		\includegraphics[width=.45\linewidth]{pdf/parabolawalk.pdf}
		\label{fig:instance-parabolawalk}
	}\hspace{.03\linewidth}\subfigure[\concavevalleys: Optimal solutions require point guards in the valley centers.]{
		\includegraphics[width=.45\linewidth]{pdf/concavevalleys.pdf}
		\label{fig:instance-concavevalleys}
	}
	\caption{Four classes of randomly generated test instances available in the \acs{TGPIL}~\cite{tgpil}.}
	\label{fig:instance}
\end{figure}

We test four classes of random terrains from the 2015-08-06 version of the \ac{TGPIL}~\cite{tgpil}, see Figure~\ref{fig:instance} for an overview.
Each class comprises 20 instances with , , , , and  vertices each, yielding 400 instances.

A \walk, see Figure~\ref{fig:instance-walk}, has  vertices with -coordinates , and the -th -coordinate is a random offset from the -th.
\sinewalk and \parabolawalk, see Figures~\ref{fig:instance-sinewalk} and~\ref{fig:instance-parabolawalk}, are the sum of a \walk and a properly scaled sine or parabola, respectively.
Both classes pose a challenge because many points see a large slope, highly fragmented by shadows of local features.

Preliminary experiments revealed that the above classes hardly require non-vertex guards for optimal solutions.
Hence we propose the \concavevalleys class, see Figure~\ref{fig:instance-concavevalleys}, which encourages point guards.
An instance starts as a \walk.
Iteratively pick a random edge and replace it by a valley with concave slopes.
Connect the slopes by a bottom edge, such that a point in its interior covers both slopes as in Figure~\ref{fig:tgp-point}.
An optimal solution for such a terrain usually requires guards in bottom edges' interiors.

None of the above classes deliberately provokes the NP-hardness of the \ac{TGP};
they are not designed to contain a reduction of hard instances of e.g.\ {PLANAR~3SAT}, as used in the NP-hardness proof of King and Krohn~\cite{kk-tginph-11}.
In our case, testing such instances is out of scope:
We provide and evaluate the means to transform a terrain into a small discretization that can be handed to a solver; \ac{IP}, \ac{PTAS}, {SAT}, or other.
The transformation has to be efficient and our experiments are designed to verify just that.
Combinatorially hard instances merely benchmark the underlying solver.

\subsection{Configurations}
\label{sec:configurations}

We test Algorithm~\ref{alg:ip} in seven configurations, see Table~\ref{tab:config}, to individually assess the impact of each filtering technique from Section~\ref{sec:filters}.
\vdefault, \vnodom, and \vnow test the \vertexguardmode mode;
\pdefault, \pnoedge, \pnodom, and \pnow test the considerably harder \pointguardmode mode.
Recall that \pointguardfilter does not apply to \vertexguardmode mode.
Since we test 400 instances, this results in 2800 test runs.

\begin{table}
	\centering
	\small
	\begin{tabular}{|l|c|c|c|c|}
		\hline
		Configuration & Mode & \pointguardfilter & \domfilter & \witnessfilter \\
		\hline
		\hline
		\vdefault & \vertexguardmode & n/a & yes & yes \\
		\hline
		\vnodom   & \vertexguardmode & n/a &  no & yes \\
		\hline
		\vnow     & \vertexguardmode & n/a & yes &  no \\
		\hline
		\hline
		\pdefault & \pointguardmode  & yes & yes & yes \\
		\hline
		\pnoedge  & \pointguardmode  &  no & yes & yes \\
		\hline
		\pnodom   & \pointguardmode  & yes &  no & yes \\
		\hline
		\pnow     & \pointguardmode  & yes & yes &  no \\
		\hline
	\end{tabular}
	\caption{Algorithm configurations, \acs{VTGP} above and \acs{CTGP} below.}
	\label{tab:config}
\end{table}

\subsection{Experimental Setup}
\label{sec:experiments-setup}

We used eight identical Linux 3.13 machines with Intel Core i7-3770 CPUs running at 3.4\,GHz, provided with 8\,MB of cache and 16\,GB of main memory.
Every run was limited to 15~minutes of CPU time and 14\,GB of memory.
Our software, except solving \acp{IP} with CPLEX, is not parallelized.
Refer to Section~\ref{sec:implementation-frameworks} for details regarding the toolchain.

\subsection{Results}
\label{sec:experiments-results}

The solution rates and median solution times of every combination of configuration, instance class, and instance complexity are listed in Tables~\ref{tab:percentage} and~\ref{tab:time}; each cell corresponds to 20 test runs.
Due to the imposed time and memory limits, not all test runs succeeded;
we account for unfinished test runs with an infinite completion time.
Hence, we use the median instead of the mean throughout the analysis.
More fine-grained timing information is presented in Figure~\ref{fig:time-boxplot}.
Except in \pnoedge mode, all unsolved instances were caused by running out of memory, see Section~\ref{sec:experiments-memory}.
We dedicate one subsection each to the relative hardness of the instance classes (Section~\ref{sec:experiments-instances}), an overview of \vertexguardmode and \pointguardmode modes (Sections~\ref{sec:experiments-vertex} and~\ref{sec:experiments-point}), the impact of \pointguardfilter, \domfilter and \witnessfilter (Sections~\ref{sec:experiments-edge}, \ref{sec:experiments-dom} and~\ref{sec:experiments-witnesses}), timing behavior (Section~\ref{sec:experiments-time}), and memory consumption (Section~\ref{sec:experiments-memory}).

\begin{table}
	\centering
	\small
	\begin{tabular}{|l|l|ccccc|}
		\hline
		\multirow{2}{*}{Configuration} & \multirow{2}{*}{Instance} & \multicolumn{5}{|c|}{\#vertices} \\
		& &  &  &  &  &  \\
		\hline
		\hline
		\multirow{4}{*}{\vdefault}
			& \walk           & 100\,\% & 100\,\% & 100\,\% & 100\,\% & 100\,\% \\
			& \sinewalk       & 100\,\% & 100\,\% & 100\,\% &   0\,\% &   0\,\% \\
			& \parabolawalk   & 100\,\% & 100\,\% & 100\,\% &   0\,\% &   0\,\% \\
			& \concavevalleys & 100\,\% & 100\,\% & 100\,\% & 100\,\% & 100\,\% \\
		\hline
		\multirow{4}{*}{\vnodom}
			& \walk           & 100\,\% & 100\,\% & 100\,\% & 100\,\% & 100\,\% \\
			& \sinewalk       & 100\,\% & 100\,\% & 100\,\% &   0\,\% &   0\,\% \\
			& \parabolawalk   & 100\,\% & 100\,\% & 100\,\% &   0\,\% &   0\,\% \\
			& \concavevalleys & 100\,\% & 100\,\% & 100\,\% & 100\,\% & 100\,\% \\
		\hline
		\multirow{4}{*}{\vnow}
			& \walk           & 100\,\% & 100\,\% & 100\,\% &   0\,\% &   0\,\% \\
			& \sinewalk       & 100\,\% & 100\,\% &   0\,\% &   0\,\% &   0\,\% \\
			& \parabolawalk   & 100\,\% &  25\,\% &   0\,\% &   0\,\% &   0\,\% \\
			& \concavevalleys & 100\,\% & 100\,\% & 100\,\% &   0\,\% &   0\,\% \\
		\hline
		\hline
		\multirow{4}{*}{\pdefault}
			& \walk           & 100\,\% & 100\,\% & 100\,\% & 100\,\% & 100\,\% \\
			& \sinewalk       & 100\,\% & 100\,\% & 100\,\% &   0\,\% &   0\,\% \\
			& \parabolawalk   & 100\,\% & 100\,\% & 100\,\% &   0\,\% &   0\,\% \\
			& \concavevalleys & 100\,\% & 100\,\% & 100\,\% & 100\,\% & 100\,\% \\
		\hline
		\multirow{4}{*}{\pnoedge}
			& \walk           & 100\,\% & 100\,\% & 100\,\% &  55\,\% &   0\,\% \\
			& \sinewalk       & 100\,\% & 100\,\% &   0\,\% &   0\,\% &   0\,\% \\
			& \parabolawalk   & 100\,\% & 100\,\% &   0\,\% &   0\,\% &   0\,\% \\
			& \concavevalleys & 100\,\% & 100\,\% & 100\,\% &  90\,\% &   0\,\% \\
		\hline
		\multirow{4}{*}{\pnodom}
			& \walk           & 100\,\% & 100\,\% & 100\,\% & 100\,\% & 100\,\% \\
			& \sinewalk       & 100\,\% & 100\,\% & 100\,\% &   0\,\% &   0\,\% \\
			& \parabolawalk   & 100\,\% & 100\,\% & 100\,\% &   0\,\% &   0\,\% \\
			& \concavevalleys & 100\,\% & 100\,\% & 100\,\% & 100\,\% & 100\,\% \\
		\hline
		\multirow{4}{*}{\pnow}
			& \walk           & 100\,\% & 100\,\% & 100\,\% &   0\,\% &   0\,\% \\
			& \sinewalk       & 100\,\% & 100\,\% &   0\,\% &   0\,\% &   0\,\% \\
			& \parabolawalk   & 100\,\% &  20\,\% &   0\,\% &   0\,\% &   0\,\% \\
			& \concavevalleys & 100\,\% & 100\,\% & 100\,\% &   0\,\% &   0\,\% \\
		\hline
	\end{tabular}
	\caption{Solution rates for each configuration, instance class, and instance complexity.}
	\label{tab:percentage}
\end{table}

\begin{table}
	\centering
	\small
	\begin{tabular}{|l|l|ccccc|}
		\hline
		\multirow{2}{*}{Configuration} & \multirow{2}{*}{Instance} & \multicolumn{5}{|c|}{\#vertices} \\
		& &  &  &  &  &  \\
		\hline
		\hline
		\multirow{4}{*}{\vdefault}
			& \walk           &   0.0\,s &   0.2\,s &   2.9\,s &  18.0\,s &  40.3\,s \\
			& \sinewalk       &   0.0\,s &   0.9\,s &  13.8\,s &      n/a &      n/a \\
			& \parabolawalk   &   0.1\,s &   1.8\,s &  21.7\,s &      n/a &      n/a \\
			& \concavevalleys &   0.2\,s &   2.4\,s &  24.2\,s & 177.2\,s & 337.6\,s \\
		\hline
		\multirow{4}{*}{\vnodom}
			& \walk           &   0.0\,s &   0.3\,s &   3.6\,s &  21.8\,s &  48.6\,s \\
			& \sinewalk       &   0.1\,s &   1.3\,s &  18.2\,s &      n/a &      n/a \\
			& \parabolawalk   &   0.1\,s &   2.5\,s &  27.8\,s &      n/a &      n/a \\
			& \concavevalleys &   0.2\,s &   2.4\,s &  24.3\,s & 177.1\,s & 411.0\,s \\
		\hline
		\multirow{4}{*}{\vnow}
			& \walk           &   0.0\,s &   0.4\,s &   8.8\,s &      n/a &      n/a \\
			& \sinewalk       &   0.1\,s &   6.4\,s &      n/a &      n/a &      n/a \\
			& \parabolawalk   &   0.4\,s &      n/a &      n/a &      n/a &      n/a \\
			& \concavevalleys &   0.2\,s &   2.7\,s &  32.0\,s &      n/a &      n/a \\
		\hline
		\hline
		\multirow{4}{*}{\pdefault}
			& \walk           &   0.0\,s &   0.3\,s &   4.6\,s &  27.9\,s &  62.4\,s \\
			& \sinewalk       &   0.1\,s &   1.7\,s &  26.3\,s &      n/a &      n/a \\
			& \parabolawalk   &   0.2\,s &   3.3\,s &  45.1\,s &      n/a &      n/a \\
			& \concavevalleys &   0.1\,s &   0.8\,s &  10.3\,s &  68.2\,s & 137.4\,s \\
		\hline
		\multirow{4}{*}{\pnoedge}
			& \walk           &   0.2\,s &   4.5\,s &  79.7\,s & 833.3\,s &      n/a \\
			& \sinewalk       &   1.0\,s &  58.8\,s &      n/a &      n/a &      n/a \\
			& \parabolawalk   &   4.1\,s & 220.3\,s &      n/a &      n/a &      n/a \\
			& \concavevalleys &   0.2\,s &   3.2\,s &  58.3\,s & 652.3\,s &      n/a \\
		\hline
		\multirow{4}{*}{\pnodom}
			& \walk           &   0.0\,s &   0.4\,s &   5.0\,s &  31.5\,s &  70.4\,s \\
			& \sinewalk       &   0.1\,s &   2.0\,s &  31.1\,s &      n/a &      n/a \\
			& \parabolawalk   &   0.2\,s &   4.0\,s &  51.8\,s &      n/a &      n/a \\
			& \concavevalleys &   0.1\,s &   0.9\,s &  10.7\,s &  68.0\,s & 145.5\,s \\
		\hline
		\multirow{4}{*}{\pnow}
			& \walk           &   0.0\,s &   0.6\,s &  10.4\,s &      n/a &      n/a \\
			& \sinewalk       &   0.1\,s &   7.8\,s &      n/a &      n/a &      n/a \\
			& \parabolawalk   &   0.6\,s &      n/a &      n/a &      n/a &      n/a \\
			& \concavevalleys &   0.1\,s &   1.1\,s &  20.8\,s &      n/a &      n/a \\
		\hline
	\end{tabular}
	\caption{Median solution times per configuration, instance class, and instance complexity.}
	\label{tab:time}
\end{table}

\subsubsection{Overview: Instances}
\label{sec:experiments-instances}

Tables~\ref{tab:percentage} and~\ref{tab:time} clearly reveal that \sinewalk and \parabolawalk are harder to solve than \walk and \concavevalleys.
This is to be expected, since \sinewalk and \parabolawalk contain a \walk as additive noise, and since the sole purpose of \concavevalleys is to encourage placing non-vertex guards, see Section~\ref{sec:experiments-instance-description}.
Furthermore, \sinewalk and \parabolawalk comprise facing valleys, resulting in highly fragmented visibility regions and complex visibility overlays in which a large portion of the guards and witnesses cannot be filtered out.
This induces time and memory intensive calculations and a complex \ac{IP}, making \sinewalk and \parabolawalk challenging instance classes.

\subsubsection{Overview: Vertex Guards}
\label{sec:experiments-vertex}

\vdefault and \vnodom solve all instances of \walk and \concavevalleys, and the \sinewalk and \parabolawalk instances of up to  vertices;
\vnow can solve instances which are smaller by about a factor of~10, see Table~\ref{tab:percentage}.
This already demonstrates the importance of \witnessfilter.
\vdefault and \vnodom have comparable running times, with a slight advantage for \vdefault, see Table~\ref{tab:time}; \vnow is slower.

\subsubsection{Overview: Point Guards}
\label{sec:experiments-point}

In terms of solved instances, refer to Table~\ref{tab:percentage}, \pdefault and \pnodom are the strongest configurations, solving all \walk and \concavevalleys instances as well as the \sinewalk and \parabolawalk instances with up to  vertices.
\pnow and \pnoedge are much weaker.
Table~\ref{tab:time} indicates that \pdefault is slightly faster than \pnodom.
It is clear from the performance of \pnoedge that \pointguardfilter is crucial in \pointguardmode mode.

\subsubsection{Impact of Filtering Edge-Interior Guards}
\label{sec:experiments-edge}

\begin{table}
	\centering
	\small
	\begin{tabular}{|l|l|ccccc|}
		\hline
		\multirow{2}{*}{Configuration} & \multirow{2}{*}{Instance} & \multicolumn{5}{|c|}{\#vertices} \\
		& &  &  &  &  &  \\
		\hline
		\hline
		\multirow{4}{*}{\pnodom}
			& \walk           & 80.1\,\% & 86.8\,\% & 89.5\,\% & 91.0\,\% & 91.7\,\% \\
			& \sinewalk       & 92.7\,\% & 97.6\,\% & 98.3\,\% &      n/a &      n/a \\
			& \parabolawalk   & 97.5\,\% & 98.8\,\% & 98.9\,\% &      n/a &      n/a \\
			& \concavevalleys & 65.8\,\% & 72.5\,\% & 77.7\,\% & 79.9\,\% & 80.6\,\% \\
		\hline
	\end{tabular}
	\caption{Median percentage of guard candidates removed by \pointguardfilter.}
	\label{tab:pointguardfilter}
\end{table}

\begin{figure}
	\subfigure[Unfiltered guard candidates.]{
		\includegraphics[width=.45\linewidth]{pdf/guard_candidates_unfiltered.pdf}
		\label{fig:filtering-effect-off}
	}\hfill
	\subfigure[Filtered guard candidates.]{
		\includegraphics[width=.45\linewidth]{pdf/guard_candidates_filtered.pdf}
		\label{fig:filtering-effect-on}
	}
	\caption{The effect of \pointguardfilter (excerpt of a -vertex \parabolawalk).
		White circles represent guard candidates.}
	\label{fig:filtering-effect}
\end{figure}

Recall that \pointguardfilter, see Section~\ref{sec:filters-pointguards}, only applies to \pointguardmode mode.
Table~\ref{tab:pointguardfilter} depicts the percentage of guards it removes in the \pnodom configuration\dash---the only configuration without interfering guard filters\dash---and Figure~\ref{fig:filtering-effect} illustrates its effectiveness using a -vertex \parabolawalk as example.

\pointguardfilter proves to be our most effective guard filter by removing roughly 90\,\% (80\,\%) of the guard candidates in the  vertex \walk (\concavevalleys) instances, and well above 95\,\% in the largest solved \sinewalk and \parabolawalk instances.
Tables~\ref{tab:percentage} and~\ref{tab:time} demonstrate that \pointguardfilter massively improves performance in terms of solution rates and median solution times.
This makes it the key success factor when solving the \ac{CTGP}, removing the computational barrier between \ac{VTGP} and \ac{CTGP}:
Without it, \pnoedge would be the state of the art, and solvable instances of the \ac{CTGP} would be smaller by at least a factor of 10 than for \ac{VTGP} and would take much more time.

\subsubsection{Impact of Filtering Dominated Guards}
\label{sec:experiments-dom}

\begin{table}
	\centering
	\small
	\begin{tabular}{|l|l|ccccc|}
		\hline
		\multirow{2}{*}{Configuration} & \multirow{2}{*}{Instance} & \multicolumn{5}{|c|}{\#vertices} \\
		& &  &  &  &  &  \\
		\hline
		\hline
		\multirow{4}{*}{\vdefault}
			& \walk           & 65.8\,\% & 64.1\,\% & 63.6\,\% & 63.5\,\% & 63.5\,\% \\
			& \sinewalk       & 58.5\,\% & 63.0\,\% & 63.6\,\% &      n/a &      n/a \\
			& \parabolawalk   & 59.9\,\% & 63.3\,\% & 63.6\,\% &      n/a &      n/a \\
			& \concavevalleys & 13.4\,\% & 13.1\,\% & 13.3\,\% & 13.3\,\% & 13.3\,\% \\
		\hline
		\hline
		\multirow{4}{*}{\pnoedge}
			& \walk           & 92.9\,\% & 94.7\,\% & 95.6\,\% & 95.8\,\% &      n/a \\
			& \sinewalk       & 88.1\,\% & 96.7\,\% &      n/a &      n/a &      n/a \\
			& \parabolawalk   & 93.2\,\% & 98.0\,\% &      n/a &      n/a &      n/a \\
			& \concavevalleys & 77.1\,\% & 80.5\,\% & 83.9\,\% & 85.0\,\% &      n/a \\
		\hline
	\end{tabular}
	\caption{Median percentage of guard candidates removed by \domfilter.}
	\label{tab:domfilter}
\end{table}

Table~\ref{tab:domfilter} displays the percentage of guard candidates that \domfilter, refer to Section~\ref{sec:filters-dominated}, filtered out in the \vdefault and \pnoedge configurations.
Observe that we need to obtain these numbers in configurations without other active filters.

\domfilter has no impact on the solution rates within the limits imposed by our setup (see Section~\ref{sec:experiments-setup}).
\vdefault and \pdefault are slightly faster than \vnodom and \pnodom, respectively.
However, the key advantage of \domfilter is that it saves memory by deleting dominated guard candidates;
this is important since memory consumption is the bottleneck of our implementation, see Section~\ref{sec:experiments-memory}.

\subsubsection{Impact of Filtering Witnesses}
\label{sec:experiments-witnesses}

\begin{table}
	\centering
	\small
	\begin{tabular}{|l|l|ccccc|}
		\hline
		\multirow{2}{*}{Configuration} & \multirow{2}{*}{Instance} & \multicolumn{5}{|c|}{\#vertices} \\
		& &  &  &  &  &  \\
		\hline
		\hline
		\multirow{4}{*}{\vdefault}
			& \walk           & 91.9\,\% & 94.3\,\% & 95.4\,\% & 96.1\,\% & 96.4\,\% \\
			& \sinewalk       & 95.8\,\% & 98.8\,\% & 99.3\,\% &      n/a &      n/a \\
			& \parabolawalk   & 98.6\,\% & 99.4\,\% & 99.5\,\% &      n/a &      n/a \\
			& \concavevalleys & 76.5\,\% & 80.5\,\% & 83.7\,\% & 85.1\,\% & 85.5\,\% \\
		\hline
		\hline
		\multirow{4}{*}{\pdefault}
			& \walk           & 91.9\,\% & 94.3\,\% & 95.5\,\% & 96.1\,\% & 96.4\,\% \\
			& \sinewalk       & 96.3\,\% & 98.9\,\% & 99.3\,\% &      n/a &      n/a \\
			& \parabolawalk   & 98.7\,\% & 99.5\,\% & 99.5\,\% &      n/a &      n/a \\
			& \concavevalleys & 80.3\,\% & 84.2\,\% & 87.4\,\% & 88.7\,\% & 89.1\,\% \\
		\hline
	\end{tabular}
	\caption{Median percentage of witnesses removed by \witnessfilter.}
	\label{tab:witnessfilter}
\end{table}

The percentage of witnesses removed by \witnessfilter, see Section~\ref{sec:filters-witnesses}, in the \vdefault and \pdefault configurations is displayed in Table~\ref{tab:witnessfilter}.
Throughout our instances, \witnessfilter removes the vast majority of witnesses, often more than~95\,\%.
Furthermore, Table~\ref{tab:percentage} clearly shows that disabling it reduces the solvable instance size by at least a factor of~10.
All of this renders \witnessfilter simple, fast, and useful.

\subsubsection{Timing Behavior}
\label{sec:experiments-time}

\begin{figure}
	\centering
	\includegraphics[width=.8\linewidth]{pdf/timing_barchart_100000_walk.pdf}
	\caption{Median CPU time spent by subroutine regarding -vertex \walk instances.}
	\label{fig:time-bar}
\end{figure}

\begin{figure}
	\centering
	\includegraphics[width=.8\linewidth]{pdf/timing_boxplot_100000_walk.pdf}
	\caption{Box plot of solution times regarding -vertex \walk instances.
		Boxes comprise first, second, and third quartile;
		whiskers indicate the 10th and 90th percentile.}
	\label{fig:time-boxplot}
\end{figure}

Figures~\ref{fig:time-bar} and~\ref{fig:time-boxplot} show how much CPU time is spent in each part of Algorithm~\ref{alg:ip} and the distribution of the solution times, respectively.
For a comparison, we pick an instance class and a complexity that was solved by every configuration:
\walk with  vertices\dash---the other combinations are left out, as they permit the same interpretation.

The strongest impact is that of \pointguardfilter, which is disabled in the rightmost bar in Figure~\ref{fig:time-bar}.
Without \pointguardfilter there is a computational gap between \pointguardmode and \vertexguardmode mode, as determining visibility regions of unneeded guards dominates the CPU time.
Guard-filtering time also increases in \pnoedge mode because \domfilter is active and has more guards to compare.

\witnessfilter has the second-most important impact.
In its absence, CPU times roughly double due to increased \ac{IP} solution times:
In the \ac{IP}, non-inclusion-minimal witnesses form constraints that are dominated by the inclusion-minimal witnesses' constraints.
The \ac{IP} solver eliminates dominated constraints in a preprocessing phase, but \witnessfilter does so more efficiently since it exploits the underlying geometry.

The timing behavior is hardly influenced by \domfilter.
It is, however, beneficial w.r.t.\ memory consumption, see Section~\ref{sec:experiments-memory}.

A general observation is that the filtering mechanisms significantly reduce the computational overhead.
One would expect \ac{IP} solution times to dominate an exact solver for an NP-hard problem.
This, however, tends not to be the case in geometric optimization problems like the \ac{AGP} and its relatives~\cite{rsfhkt-eag-14}.
It stems from the \ac{EGC} paradigm:
Instead of floating-point arithmetic, exact number types must be used to ensure correct results.
\vdefault and \pdefault still are not clearly dominated by \ac{IP} solution times, but much closer to it than unfiltered approaches.

\subsubsection{Memory Consumption}
\label{sec:experiments-memory}

Within our experimental setup, using the \vdefault and \pdefault modes, even instances with  vertices are solved within minutes.
All unsolved configuration/instance pairs run out of memory, not time;
only \pnoedge occasionally runs out of time\dash---owed to the large number of unnecessary visibility calculations otherwise prevented by \pointguardfilter, see Sections~\ref{sec:experiments-edge} and~\ref{sec:experiments-time}.
So as long as instances are not designed to reveal the NP-hardness of the \ac{TGP}, the limiting resource is memory.

Two phases of Algorithm~\ref{alg:ip} generate a significant amount of data which persists in memory.
The first phase is the computation of all visibility regions of all vertices  in line~\ref{alg:visi-v}.
This stores  -coordinates in memory which define the unfiltered guard candidate set~.
Filters remove the vast majority of candidates from~.
The second phase determines the visibility regions of the remaining points in , generating the largest chunk of data in line~\ref{alg:visi-u}:
At this point we keep  -coordinates in memory.
We conjecture that simultaneously holding all these visibility regions in memory cannot be avoided since a guard at the far right of the terrain may still see a region at its very left;
hence we do not see a way to apply \witnessfilter before knowing all extremal points.
As a lower bound, observe that a discretization with guards  and witnesses  yields a constraint matrix  of the \acs{IP}~\eqref{eq:ip-begin}--\eqref{eq:ip-end} with , i.e., one with  entries for the \ac{CTGP}.

We remark that the memory bottleneck is amplified by the fact that we follow the \ac{EGC} paradigm, which ensures a correct and consistent representation of all visibility regions and a correct order of all visibility events.
Specifically, we do not store coordinates of points using floating-point arithmetic.
Neither do we use the other extreme, i.e., an exact representation by arbitrary precision rationals as e.g.\ provided by the GMP~\cite{gmp} library.
Instead, we rely on \ac{CGAL}~\cite{cgal}, more precisely on \texttt{CGAL::Exact\_predicates\_exact\_constructions\_kernel}, which provides lazy constructions:
Each coordinate is initially represented by two doubles that encode an interval containing the actual coordinate.
This suffices for many decisions, for instance, a comparison with another coordinate.
However, in cases in which intervals overlap, the exact coordinates are computed with GMP.
Compared to the pure exact approach this usually yields a significant advantage regarding speed and memory~\cite{pf-aglesfegc-2011}.
