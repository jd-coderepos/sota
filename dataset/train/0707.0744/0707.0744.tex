\documentclass{pseudoelsart} 
\usepackage{latexsym} 
\usepackage{alltt}
\usepackage[all]{xy} 
\CompileMatrices 
\xyoption{frame}
\usepackage{amssymb}
\usepackage{amscd} 
\usepackage{amsmath}
\usepackage{theorem} 
{\theorembodyfont{\rmfamily}
\newtheorem{example}[thm]{Example}} 
{\theorembodyfont{\rmfamily}
\newtheorem{projection}[thm]{Projection}} 
{\theorembodyfont{\rmfamily}
\newtheorem{notation}[thm]{Notation}}
{\theorembodyfont{\rmfamily}
\newtheorem{definition}[thm]{Definition}}  

\newcommand{\Nat}{{\mathbb N}}
\newcommand{\tr}{{\mathtt{true}}}
\newcommand{\fa}{{\mathtt{false}}}
\newcommand{\pgaEA}{\textrm{pgaEA}}
\newcommand{\sem}[1]{[|#1|]}
\newcommand{\rpar}[1]{r_{\mathit{#1}}}
\newcommand{\spar}[1]{s_{\mathit{#1}}}
\newcommand{\sdef}[1]{\spar{default}(#1)}
\newcommand{\rdef}[1]{\rpar{default}(#1)}
\newcommand{\service}{\mathit{serv}}
\newcommand{\prom}[3]{#1 \overset{\pi:#2}{- \!\!\!- \!\!\!- \!\!\!- \!\!\!\!\!\!\!\!\longrightarrow} #3}
\newcommand{\lprom}[3]{pi_{#1,#2}(#3)}
\newcommand{\lwprom}[3]{pw_{#1,#2}(#3)}
\newcommand{\geb}{{\sim}}

\begin{document} 
\begin{frontmatter} 
\title{A process algebra based framework for promise theory}
\author[label1,label2]{Jan  Bergstra\thanksref{email1}} 
\author[label1]{Inge Bethke\thanksref{email2}\corauthref{cor}}
\author[label3]{Mark Burgess\thanksref{email3}}
\corauth[cor]{Corresponding author. Address: Kruislaan 403, 1098 SJ Amsterdam, The Netherlands }
\thanks[email1]{E-mail: \texttt{janb@}\texttt{phil.uu.nl, science.uva.nl}}
\thanks[email2]{E-mail:  \texttt{inge@science.uva.nl}}
\thanks[email3]{E-mail:  \texttt{mark@iu.hio.no}\\
\mbox{}\\
An earlier version 
of this paper also appeared as report PRG0701, Section Theoretical Software Engineering, 
Informatics Institute, 
Faculty of Science, University of Amsterdam.}
\address[label1]{University of Amsterdam, Faculty of Science, Section Theoretical
Software
Engineering (former Programming Research Group)} 
\address[label2]{Utrecht University, Department of Philosophy, Applied Logic
Group}
\address[label3]{University College Oslo, Faculty of Engineering}
\begin{abstract}
We present a process algebra based approach to formalize the interactions of computing devices
such as the representation of policies and the resolution of conflicts. As an example we specify how
promises may be used in coming to an agreement regarding a simple though
practical transportation problem.
\end{abstract}
\begin{keyword}
Software/program verification, formal methods D.2.4
\end{keyword}
\end{frontmatter}
\section{Introduction}
The mechanism of an autonomous agent announcing a promise towards another agent
is a powerful organizational principle in the setting of computer networks.
Several approaches have been used in the past to formalize such interactions of
computing devices as a
representation of policies and a resolve of conflicts: 
Burgess and Fagernes \cite{BF06}
represent them as graphs, Prakken and Sergot \cite{PS96,PS97} use temporal
deontic logic, Lupu and Sloman \cite{LS97} propose role theory, Glasgow et al.
\cite{GMP92} modal logic, Bandera et al. \cite{BLMR04} event
calculus and  Lafuente and Montanari \cite{LM05} model checking.
In this paper we use process algebra \cite{BPS01} for the formalization of a
restricted set of aspects of promises paying attention to the sequential
ordering of promises between a number of parties. As an example we specify how
promises may be used in coming to an agreement regarding a simple though
practical transportation problem.

In the world of process algebra we can label certain communications as promises
if that makes sense intuitively. Process algebra formalisms will not provide very
sharp distinctions that set apart \emph{promise acts} from all other conceivable
actions, however. Modal logics are in principle better suited for the task to
capture what is specific about promises, but process algebras may be more
helpful to formalize the role that  promises can play in specific multi-agent
systems. The justification of the
process algebra framework for promises is therefore as follows:
\begin{enumerate}
\item to provide clear and formalized cases of the use of promises in some
protocols that occur within multi-agent systems,
\item to support the design and analysis of distributed protocols that make use
of promises made by autonomous agents.
\end{enumerate}
The process algebra framework cannot, by nature, characterize the concept of a
promise in its logical essence. That is a much harder task and requires the
design of specific versions of deontic logic.






\section{A data type for task bodies}
The data type for task bodies is depicted in  Figure \ref{TB}.
\begin{figure}[htbp] 
\centering
\fbox{ 
\mbox{ 
\xy 
\UseComputerModernTips 
\xymatrix@=40pt@!{ 
&{\mathcal{T}}\\
{\mathcal{TB}} \ar@(l,u)[]^{\geb} \ar@(l,d)[]_{\neg}\ar[ur]^{t}
\ar@{>->}[dr]|{\#} \ar@/^1pc/[dr]^p \ar@/_1pc/[dr]_{s}&\ar[l]_\gamma\\
&{\mathbb{B}}\ar@(r,u)[]_\neg
\\
&\ar@/^1pc/[u]^\tr \ar@/_1pc/[u]_\fa\\
} 
\endxy 
}} 
\caption{Data type for task bodies}\label{TB} 
\end{figure}



 is assumed to be a finite set of primitives which fall into two
basic complementary
categories, namely into tasks for giving or taking, or \emph{services} and \emph{usage}.
We distinguish the atom |the special task of compliance.
Atomic tasks are assumed services rather than uses and positive, i.e.\ , not 
negated . Two operations  on 
tasks
are then considered:
\begin{enumerate}
\item (\emph{usage}) if  is a task then \footnote{In \cite{B06,BF06}
the use of a  is denoted by  or  instead of .} is the task of making use of  as performed by
another agent, and
\item (\emph{negation}) if  is a task then  is the task of not doing .
\end{enumerate}
Moreover,  specify the properties \emph{service} and 
\emph{positive}.
The interaction of these operations satisfies the laws in Figure \ref{usage_negation}.
\begin{figure}[htbp]
2mm]
&\neg \neg x & = & x&&&s(\neg x) & =& s(x)&&& p(\neg x)&=& \neg p(x)&\2mm]
\hspace{.5cm}&&&&&&&&&&&&&&\hspace{.5cm}\\
\hline
\end{array}

t(\geb  x)=t(x)=t(\neg x).

x \# \neg x\ \ \ \ \ \frac{x\# y}{y \# x}
\ \ \ \ \ \frac{x \# y}{t(x)=t(y)}
\ \ \ \ \ \ \frac{x\# y}{\neg ( x \# \neg y)}

\neg (x \# x)
\prom{a}{x}{b}.
\begin{array}{|cll|}
\hline&&\\
\hspace{0.5cm}&(1)\hspace{1 cm}&\prom{a}{x}{b}\hspace{1.5cm}\4mm]
&(3)&\prom{a}{\neg x}{b}\4mm]
\hline
\end{array}

x=y \Rightarrow \prom{a}{x}{b}=\prom{a}{y}{b}.

\prom{a[c]}{x}{b[d]}

\prom{a[a]}{x}{b[b]}.

\prom{c}{\gamma}{a},

\prom{a[c]}{x}{b[d]},\prom{c}{\gamma}{a} \Longrightarrow \prom{c}{x}{d}.

\prom{a[c_1,\ldots, c_n]}{x}{b[d_1,\ldots, d_m]}
\prom{a}{x}{b}
pi_{a, b}(x)\ \frac{S}{S\oplus \prom{a}{x}{b}}\ \ \ \ \ \textit{promise introduction}

pw_{a, b}(x)\ \frac{S}{S\ominus \prom{a}{x}{b}}\ \ \ \ \ \textit{promise withdrawal}

pi_{a[c]\rightarrow b[d]}(x)\ \frac{S\oplus \prom{c}{\gamma}{a}}{S\oplus \prom{c}{\gamma}{a}\oplus \prom{c}{x}{d}}

(E(x) \rightarrow \forall c\neq b\ \neg p_{a,c}(x)) \Longrightarrow pi_{a, b}(x)\ \frac{S}{S\oplus \prom{a}{x}{b}},

\begin{array}{rl}
P_{a,b}(x) = pi_{a,b}(x) \cdot ( & (E(\geb x)\rightarrow \forall c\neq a\ \neg p_{b,c}(\geb x)):
\rightarrow pi_{b,a}(\geb x)\\
& + \\
&pi_{b,a}(\neg \geb x) \cdot (pw_{a,b}(x) \parallel pw_{b,a}(\neg \geb x))\\
)&
\end{array}

E(\geb \mathit{tbc2JUB}). 

\begin{array}{ll}
pi_{ja,ma}(\mathit{tbc2JUB})&\cdot\\
pi_{ma,ja}(\mathit{\geb tbc2JUB})&\cdot\\
pi_{ju,ma}(\mathit{tbc2JUB})&\cdot\\
pi_{ma,ju}(\mathit{\neg \geb tbc2JUB})&\cdot\\
pw_{ju,ma}(\mathit{tbc2JUB})&\cdot\\
pw_{ma,ju}(\mathit{\neg \geb tbc2JUB}).&\\
\end{array}

On an intuitive level, the trace can be described as follows:
Initially Jan promises Mark a lift to JUB which Mark accepts. Then 
J\"urgen makes this promise too which Mark|because of the exclusiveness of this
task|declines. Thereupon J\"urgen withdraws his offer and Mark his declination.

This kind of example can typically be found in data centre management:
renaming the agents and tasks to 
\begin{enumerate}
\item , and
\item 
\end{enumerate}
we derive an example of choosing a supplier for e.g.\ packet transport,
power/elec\-tri\-ci\-ty etc. Promises are then exclusive if ISPA and ISPB are
competitors, for instance.

\section{Conclusion}
We have provided the outline of a process algebra based framework for promise theory.
Using this algebra in combination with conditional guards one can formalize|as other approaches do|how promises might be used
in coming to an agreement. However, in contrast to the static approaches to promise theory mentioned in the introduction,
in the here chosen framework|the algebra of communicating processes ACP| the interaction of promises and the resolution of conflicts can be modelled in a dynamic way.

This formalization is treating promises at a meta-level. There are also
underlying events or processes that the promises suppress|we do not talk about
how the promises are kept, or comment on their reliability; that is a different
matter. Thus our description is at a \emph{promise management level}. At that
level we could say it describes an autonomous process.
\section*{Acknowledgements}
Jan Bergstra and Mark Burgess acknowledge helpful discussions with J\"urgen
Sch\"onw\"alder, School of Engineering and Science, during a working visit  of one week to the Jacobs University
Bremen in October 2006.

\begin{thebibliography}{99}

\bibitem{BB92}
J.C.M. Baeten and J.A. Bergstra.
\newblock Process algebra with signals and conditions.
\newblock In \cite{B92}, 273--323 (1992).

\bibitem{BW90}
J.C.M. Baeten and W.P. Weijland.
\newblock \emph{Process algebra}.
\newblock Cambridge Tracts in Theoretical Computer Science 18, Cambridge University Press (1990).

\bibitem{BLMR04}
A.K. Bandara, E.C. Lupu, J. Moffet, and A. Russo.
\newblock Using event calculus to formalise policy specification and analysis.
\newblock \emph{Proceedings of the 5th International Workshop on Policies
for Distributed Systems and Networks (POLICY 2004)}, IEEE Computer Society, 229--239 (2004).
 
\bibitem{BPS01}
J.A. Bergstra, A. Ponse, and S.A. Smolka (eds).
\emph{The Handbook of Process Algebra}, Elsevier (2001).

\bibitem{B92}
M. Broy (ed).
\newblock \emph{Programming and Mathematical Method, Marktoberdorf Summer School 1990},
NATO ASI series F 88, Springer Verlag (1992).

\bibitem{B06}
Mark Burgess.
\newblock A promise theory approach to collaborative power reduction in a pervasive computing
environment.
\newblock In \cite{MJYT06}, 615--624 (2006).

\bibitem{BF06}
Mark Burgess and Siri Fagernes.
\newblock Promise theory|a model of autonomous objects for pervasive computing and swarms.
\newblock \emph{ICNS}, 118 (2006).

\bibitem{F00}
Wan Fokkink.
\newblock \emph{Introduction to Process Algebra}.
\newblock Texts in Theoretical Computer Science. An EATCS Series. Springer (2000)

\bibitem{GMP92}
J. Glasgow, G. MacEwan, and P. Panagaden.
\newblock A logic for reasoning about security.
\newblock \emph{ACM Transactions on Computer Science} 10, 226--264 (1992).

\bibitem{LS97}
E. Lupu and M. Sloman.
\newblock Conflict analysis for management policies.
\newblock \emph{Proceedings of the Vth International Symposium on Integrated
Network Management IM'97}, Chapman \& Hall, 1--14 (1997).

\bibitem{LM05}
A.L. Lafuente and U. Montanari.
\newblock Quantitative mu-calculus and ctl defined over constraint semi-rings.
\newblock \emph{Electronic Notes on Theoretical Computing Systems QAPL}, 
1--30 (2005).

\bibitem{MJYT06}
J. Ma, H. Jin, L.T. Yang, and J. J.-P. Tsai (eds).
\emph{Ubiquitous Intelligence and Computing: Third International Conference}, 
Lecture Notes in Computer Science, volume 4159, Springer, (2006).

\bibitem{PS96}
H. Prakken and M. Sergot.
\newblock Contrary-to-duty obligations.
\newblock \emph{Studia Logica} 57, 91--115, (1996).

\bibitem{PS97}
H. Prakken and M. Sergot.
\newblock Dyadic deontic logic and contrary-to-duty obligations.
\newblock In \cite{N97}, 223--262 (1997).

\bibitem{N97}
D.N. Nute (ed).
\newblock \emph{Defeasible Deontic Logic}, Synthese Library, Kluwer (1997).


\end{thebibliography}


 \end{document} 
