\subsection{Dataset Statistics}

Statistics of the RE datasets are listed in \Cref{tab:dataset-statistics}.

\begin{table}[h]
    \centering
    \small
    \begin{threeparttable}
    \resizebox{\linewidth}{!}{
    \begin{tabular}{ccccc}
    \toprule
    Dataset & \#Train & \#Dev & \#Test & \#Relation \\
    \midrule
    SemEval & 8000 & - & 2717 & 19 \\
    TACRED & 68124 & 22631 & 15509 & 42 \\
    TACREV & 68124 & 22631 & 15509 & 42 \\
    \midrule
    TACRED(1\%) & 682 & 227 & \multirow{3}{*}{15509} & \multirow{3}{*}{42}\\
    TACRED(5\%) & 3407 & 1133 & & \\
    TACRED(10\%) & 6815 & 2265 & & \\
    \bottomrule
    \end{tabular}
    }
    \end{threeparttable}
\caption{Statistics of datasets}
    \label{tab:dataset-statistics}
\end{table}



\subsection{Analysis of type constrained decoding} \label{ssec:type_constrained_decoding}

In this ablation study, we make comparisons between \modelname with and without typed constrained decoding. The results is demonstrated in \Cref{tab:result-type-constraint}. Type constraint brings improvement for 0.2\% in average in TACRED and has comparable performance on TACREV. Type-relation mapping is inherently involved in training data, so this ablation study proves \modelname can learn type-relation mapping from data.

\begin{table}[h]
    \centering
    \begin{threeparttable}
    \small
    \begin{tabular}{ccc}
    \toprule
    Dataset & TACRED & TACREV\\
    \midrule
    bart-large-cnn & 73.6 & 81.0 \\
    - type constraint & 73.4 & 81.0 \\
    \midrule
    bart-large-xsum & 73.3 & 81.0 \\
    - type constraint & 73.1 & 81.0 \\
    \midrule
    pegasus-large & 75.1 & 83.3 \\
    - type constraint & 75.0 & 83.3\\
    \midrule
    average gap & -0.2 & 0\\
    \bottomrule
    \end{tabular}
    \end{threeparttable}
    \caption{Comparison between \modelname with and without type constraint decoding. We report micro F1 on TACRED and TACREV.}
    \label{tab:result-type-constraint}
\end{table}

\subsection{Analysis of calibration for NA relation} \label{ssec:analysis_of_calibration}

In this ablation study, we make comparisons between \modelname with and without calibration for NA relation. The results is demonstrated in \Cref{tab:analysis_of_calibration}. Calibration brings improvement for 0.3\% in average on TACRED and 0.1\% in average on TACREV.

\begin{table}[h]
    \centering
    \begin{threeparttable}
    \small
    \begin{tabular}{ccc}
    \toprule
    Dataset & TACRED & TACREV \\
    \midrule
    bart-large-cnn & 73.6 & 81.0 \\
    - calibration & 73.2 & 80.9 \\
    \midrule
    bart-large-xsum & 73.3 & 81.0 \\
    - calibration & 72.9  & 80.9 \\
    \midrule
    pegasus-large & 75.1 & 83.3\\
    - calibration & 74.8 & 83.2\\
    \midrule
    average gap &  -0.3 & -0.1 \\
    \bottomrule
    \end{tabular}
    \end{threeparttable}
    \caption{Comparison between \modelname with and without calibration for NA relation We report micro F1 on TACRED and TACREV.}
    \label{tab:analysis_of_calibration}
\end{table}


\subsection{Hyper-parameters and reimplementation}\label{ssec:hyperparameter}

This section details the training and inference processes of baselines and our models. We train and inference all models with PyTorch and Huggingface Transformers on GeForce RTX 2080 or NVIDIA RTX A5000 GPUs. All optimization uses Adam and linear scheduler. A weight decay is used for regularization. We run all experiments on three seeds  and report the median.
Specifically, the best hyperparameters for full training setting with \emph{pegasus-large} are listed below:

\begin{itemize}[leftmargin=1em]
    \setlength\itemsep{0em}
    \item learning rate: 1e-4
    \item weight decay: 5e-6
    \item epoch number: 20
    \item max source length: 256
    \item max target length: 64
    \item gradient accumulation steps: 2
    \item warm up steps: 1000
\end{itemize}

The best hyperparameters for low-resource setting with \emph{pegasus-large} are listed below:

\begin{itemize}[leftmargin=1em]
    \setlength\itemsep{0em}
    \item learning rate: 1e-5
    \item weight decay: 5e-6
    \item epoch number: 100
    \item max source length: 256
    \item max target length: 64
    \item gradient accumulation steps: 2
    \item warm up steps: 0
\end{itemize}

With the 1/5/10\% indices of TACRED provided by \cite{sainz-etal-2021-label}\footnote{Github repository of low-resource indices: \url{https://github.com/osainz59/Ask2Transformers}}, we re-implement RECENT~\cite{lyu-chen-2021-relation} and test it under both low-resource and full-training scenarios\footnote{Github repository of RECENT: \url{https://github.com/Saintfe/RECENT}}. However, we find the origin evaluation scripts provided by the author has a serious issue which wrongly corrects all false positive samples of the binary classifier as true negative samples. So the recall of NA is always 100\% and precision of positive relations is unreasonably high. We correct this issue and the test results significantly differ from origin results reported by previous study. We also test IRE~\cite{zhou2021improved} and KnowPrompt~\cite{chen2021knowprompt} on low resource datasets and search the best hyperparameter with grid searching\footnote{Github repository of IRE: \url{https://github.com/wzhouad/RE\_improved\_baseline}}\footnote{Github repository of KnowPrompt: \url{https://github.com/zjunlp/KnowPrompt}}. The previous work of KnowPrompt reports the micro F1 score on SemEval. We train KnowPrompt on SemEval with codes and hyper-parameters provided by authors and re-evaluate it with official macro-F1 scoring method.





\begin{table*}[h]
    \centering
    \small
    \begin{threeparttable}
    \setlength{\tabcolsep}{5pt}
    \resizebox{\textwidth}{!}{
    \begin{tabular}{c|c|cccc}
    \toprule
    Technique & Example & M\tnote{1} & S\tnote{1} & T\tnote{1} & TS\tnote{1} \\
    \midrule
    Entity information verbalization & \tabincell{l}{\texttt{Context}\tnote{2} . ... \{subj\} ... \{obj\} ...} & \Checkmark & \XSolidBrush & \Checkmark & \Checkmark\\
\midrule
    \tabincell{c}{Entity marker \\\cite{zhang-etal-2019-ernie}} & \tabincell{l}{... <e1> \{subj\} </e1> ... <e2> \{obj\} </e2> ...} & \Checkmark & \Checkmark & \XSolidBrush & \XSolidBrush\\
\tabincell{c}{Entity typed marker \\\cite{zhong-chen-2021-frustratingly}} & \tabincell{l}{... <e1-\{subj-type\}> \{subj\} </e1-\{subj-type\}> \\... <e2-\{obj-type\}> \{obj\} </e2-\{obj-type\}> ...} & \Checkmark & \Checkmark & \Checkmark & \XSolidBrush\\
\tabincell{c}{Entity typed marker (punct) \\\cite{zhou2021improved}} & \tabincell{l}{... @ * \{subj-type\} * \{subj\} @ ... \#  \{obj-type\}  \{obj\} \# ...} & \Checkmark & \Checkmark & \Checkmark & \Checkmark \\
\midrule
    \tabincell{c}{Entity information verbalization + Entity typed marker} & \tabincell{l}{\texttt{Context}\tnote{2} ... @ * \{subj-type\} * \{subj\} @ \\... \#  \{obj-type\}  \{obj\} \# ...} & \Checkmark & \Checkmark & \Checkmark & \Checkmark \\
\tabincell{c}{Entity information verbalization + Entity typed marker (punct)} & \tabincell{l}{\texttt{Context}\tnote{2} ... @ * \{subj-type\} * \{subj\} @ \\... \#  \{obj-type\}  \{obj\} \# ...} & \Checkmark & \Checkmark & \Checkmark & \Checkmark \\
    \bottomrule
    \end{tabular}
    }
    \begin{tablenotes}
        \footnotesize 
        \item[1] Column names are short for mentions, spans, types and type semantics, respectively.
        \item[2] Augmented context are generated with the template: ``\emph{The \{subj\} entity is \{subj\} .\\ The \{obj\} entity is \{obj\} . The type of \{subj\} is \{subj-type\} . The type of \{obj\} is \{obj-type\} }''
      \end{tablenotes}
    \end{threeparttable}
    \caption{Sentence processing techniques for incorporating entity information.}
    \label{tab:entity-information-injection-techniques}
\end{table*}

\begin{table*}[h]
    \centering
    \begin{threeparttable}
    \small
    \begin{tabular}{ccccccc}
    \toprule
    \multirow{2}{*}{Technique} & \multicolumn{3}{c}{TACRED} & \multicolumn{3}{c}{TACREV} \\
    \cline{2-7}
    & P & R & F1 & P & R & F1 \\
    \midrule
    Entity information verbalization & 71.8 & 75.1 & 73.4 & 78.4 & 83.8 & 81.0 \\ 
    Entity tag & 71.3 & 71.0 & 71.1 & 81.0 & 75.2 & 78.0 \\ 
    Entity typed tag & 73.5 & 69.7 & 71.5 & 79.9 & 79.6 & 79.8 \\ 
    Entity typed tag(punct) & 70.3 & 74.0 & 72.1 & 81.1 & 79.3 & 80.2 \\ 
    Entity infromation verbalization + Entity tag & 73.4 & 71.8 & 72.6 & 78.7 & 81.7 & 80.2 \\ 
    Entity infromation verbalization + Entity typed tag        & 70.6 & 73.5 & 72.1 & 82.8 & 80.0 & \textbf{81.4} \\
    Entity infromation verbalization + Entity typed tag(punct) & 72.6 & 74.5 & \textbf{73.6} & 82.1 & 79.8 & 81.0 \\
    \bottomrule
    \end{tabular}
    \end{threeparttable}
    \caption{Analysis of different input formulation techniques on \emph{bart-large-cnn}. We report micro F1 scores on both datasets. The best F1 score is identified with \textbf{bold}.}
    \label{tab:result-input-formulation-bart-large-cnn}
\end{table*}

\subsection{Manual-constructed templates}

In this subsection, we display our manual-constructed templates for SemEval (\Cref{tab:semantic1-templates-semeval}), and three templates designed for TACRED, which are \textit{Semantic1} (\Cref{tab:semantic1-templates}), \textit{Semantic2}(\Cref{tab:semantic2-templates}), and \textit{Structural}(\Cref{tab:structural-templates}).

\begin{table*}[h]
    \centering
    \begin{threeparttable}
    \small
    \begin{tabular}{cc}
    \toprule
    Relation & Template \\
    \midrule
   org:country\_of\_headquarters & \{subj\} has a headquarter in the country \{obj\}\\
org:parents & \{subj\} has the parent company \{obj\}\\
per:stateorprovince\_of\_birth & \{subj\} was born in the state or province \{obj\}\\
per:spouse & \{subj\} is the spouse of \{obj\}\\
per:origin & \{subj\} has the nationality \{obj\}\\
per:date\_of\_birth & \{subj\} has birthday on \{obj\}\\
per:schools\_attended & \{subj\} studied in \{obj\}\\
org:members & \{subj\} has the member \{obj\}\\
org:founded & \{subj\} was founded in \{obj\}\\
per:stateorprovinces\_of\_residence & \{subj\} lives in the state or province \{obj\}\\
per:date\_of\_death & \{subj\} died in the date \{obj\}\\
org:shareholders & \{subj\} has shares hold in \{obj\}\\
org:website & \{subj\} has the website \{obj\}\\
org:subsidiaries & \{subj\} owns \{obj\}\\
per:charges & \{subj\} is convicted of \{obj\}\\
org:dissolved & \{subj\} dissolved in \{obj\}\\
org:stateorprovince\_of\_headquarters & \{subj\} has a headquarter in the state or province \{obj\}\\
per:country\_of\_birth & \{subj\} was born in the country \{obj\}\\
per:siblings & \{subj\} is the siblings of \{obj\}\\
org:top\_members/employees & \{subj\} has the high level member \{obj\}\\
per:cause\_of\_death & \{subj\} died because of \{obj\}\\
per:alternate\_names & \{subj\} has the alternate name \{obj\}\\
org:number\_of\_employees/members & \{subj\} has the number of employees \{obj\}\\
per:cities\_of\_residence & \{subj\} lives in the city \{obj\}\\
org:city\_of\_headquarters & \{subj\} has a headquarter in the city \{obj\}\\
per:children & \{subj\} is the parent of \{obj\}\\
per:employee\_of & \{subj\} is the employee of \{obj\}\\
org:political/religious\_affiliation & \{subj\} has political affiliation with \{obj\}\\
per:parents & \{subj\} has the parent \{obj\}\\
per:city\_of\_birth & \{subj\} was born in the city \{obj\}\\
per:age & \{subj\} has the age \{obj\}\\
per:countries\_of\_residence & \{subj\} lives in the country \{obj\}\\
org:alternate\_names & \{subj\} is also known as \{obj\}\\
per:religion & \{subj\} has the religion \{obj\}\\
per:city\_of\_death & \{subj\} died in the city \{obj\}\\
per:country\_of\_death & \{subj\} died in the country \{obj\}\\
org:founded\_by & \{subj\} was founded by \{obj\}"\\
    \bottomrule
    \end{tabular}
    \end{threeparttable}
    \caption{First semantic templates for TACRED, where \{subj\} and \{obj\} are the placeholders for subject and object entities.}
    \label{tab:semantic1-templates}
\end{table*}

\begin{table*}[h]
    \centering
    \begin{threeparttable}
    \small
    \begin{tabular}{cc}
    \toprule
    Relation & Template \\
    \midrule
    no\_relation &     The relation between \{subj\} and \{obj\} is not available\\
    per:stateorprovince\_of\_death &     The relation between \{subj\} and \{obj\} is state or province of death\\
    per:title &     The relation between \{subj\} and \{obj\} is title\\
    org:member\_of &     The relation between \{subj\} and \{obj\} is member of\\
    per:other\_family &     The relation between \{subj\} and \{obj\} is other family\\
    org:country\_of\_headquarters &     The relation between \{subj\} and \{obj\} is country of headquarters\\
    org:parents &     The relation between \{subj\} and \{obj\} is parents of the organization\\
    per:stateorprovince\_of\_birth &     The relation between \{subj\} and \{obj\} is state or province of birth\\
    per:spouse &     The relation between \{subj\} and \{obj\} is spouse\\
    per:origin &     The relation between \{subj\} and \{obj\} is origin\\
    per:date\_of\_birth &     The relation between \{subj\} and \{obj\} is date of birth\\
    per:schools\_attended &     The relation between \{subj\} and \{obj\} is schools attended\\
    org:members &     The relation between \{subj\} and \{obj\} is members\\
    org:founded &     The relation between \{subj\} and \{obj\} is founded\\
    per:stateorprovinces\_of\_residence &     The relation between \{subj\} and \{obj\} is state or province of residence\\
    per:date\_of\_death &     The relation between \{subj\} and \{obj\} is date of death\\
    org:shareholders &     The relation between \{subj\} and \{obj\} is shareholders\\
    org:website &     The relation between \{subj\} and \{obj\} is website\\
    org:subsidiaries &     The relation between \{subj\} and \{obj\} is subsidiaries\\
    per:charges &     The relation between \{subj\} and \{obj\} is charges\\
    org:dissolved &     The relation between \{subj\} and \{obj\} is dissolved\\
    org:stateorprovince\_of\_headquarters &     The relation between \{subj\} and \{obj\} is state or province of headquarters\\
    per:country\_of\_birth &     The relation between \{subj\} and \{obj\} is country of birth\\
    per:siblings &     The relation between \{subj\} and \{obj\} is siblings\\
    org:top\_members/employees &     The relation between \{subj\} and \{obj\} is top members or employees\\
    per:cause\_of\_death &     The relation between \{subj\} and \{obj\} is cause of death\\
    per:alternate\_names &     The relation between \{subj\} and \{obj\} is person alternative names\\
    org:number\_of\_employees/members &     The relation between \{subj\} and \{obj\} is number of employees or members\\
    per:cities\_of\_residence &     The relation between \{subj\} and \{obj\} is cities of residence\\
    org:city\_of\_headquarters &     The relation between \{subj\} and \{obj\} is city of headquarters\\
    per:children &     The relation between \{subj\} and \{obj\} is children\\
    per:employee\_of &     The relation between \{subj\} and \{obj\} is employee of\\
    org:political/religious\_affiliation &     The relation between \{subj\} and \{obj\} is political and religious affiliation\\
    per:parents &     The relation between \{subj\} and \{obj\} is parents of the person\\
    per:city\_of\_birth &     The relation between \{subj\} and \{obj\} is city of birth\\
    per:age &     The relation between \{subj\} and \{obj\} is age\\
    per:countries\_of\_residence &     The relation between \{subj\} and \{obj\} is countries of residence\\
    org:alternate\_names &     The relation between \{subj\} and \{obj\} is organization alternate names\\
    per:religion &     The relation between \{subj\} and \{obj\} is religion\\
    per:city\_of\_death &     The relation between \{subj\} and \{obj\} is city of death\\
    per:country\_of\_death &     The relation between \{subj\} and \{obj\} is country of death\\
    org:founded\_by &     The relation between \{subj\} and \{obj\} is founded by"\\
    \bottomrule
    \end{tabular}
    \end{threeparttable}
    \caption{Second semantic templates for TACRED, where \{subj\} and \{obj\} are the placeholders for subject and object entities.}
    \label{tab:semantic2-templates}
\end{table*}

\begin{table*}[h]
    \centering
    \begin{threeparttable}
    \small
    \begin{tabular}{cc}
    \toprule
    Relation & Template \\
    \midrule
    no\_relation & \{subj\} no relation \{obj\} \\
    per:stateorprovince\_of\_death & \{subj\} person state or province of death \{obj\} \\
    per:title & \{subj\} person title \{obj\} \\
    org:member\_of & \{subj\} organization member of \{obj\} \\
    per:other\_family & \{subj\} person other family \{obj\} \\
    org:country\_of\_headquarters & \{subj\} organization country of headquarters \{obj\} \\
    org:parents & \{subj\} organization parents \{obj\} \\
    per:stateorprovince\_of\_birth & \{subj\} person state or province of birth \{obj\} \\
    per:spouse & \{subj\} person spouse \{obj\} \\
    per:origin & \{subj\} person origin \{obj\} \\
    per:date\_of\_birth & \{subj\} person date of birth \{obj\} \\
    per:schools\_attended & \{subj\} person schools attended \{obj\} \\
    org:members & \{subj\} organization members \{obj\} \\
    org:founded & \{subj\} organization founded \{obj\} \\
    per:stateorprovinces\_of\_residence & \{subj\} person state or provinces of residence \{obj\} \\
    per:date\_of\_death & \{subj\} person date of death \{obj\} \\
    org:shareholders & \{subj\} organization shareholders \{obj\} \\
    org:website & \{subj\} organization website \{obj\} \\
    org:subsidiaries & \{subj\} organization subsidiaries \{obj\} \\
    per:charges & \{subj\} person charges \{obj\} \\
    org:dissolved & \{subj\} organization dissolved \{obj\} \\
    org:stateorprovince\_of\_headquarters & \{subj\} organization state or province of headquarters \{obj\} \\
    per:country\_of\_birth & \{subj\} person country of birth \{obj\} \\
    per:siblings & \{subj\} person siblings \{obj\} \\
    org:top\_members/employees & \{subj\} organization top members or employees \{obj\} \\
    per:cause\_of\_death & \{subj\} person cause of death \{obj\} \\
    per:alternate\_names & \{subj\} person alternate names \{obj\} \\
    org:number\_of\_employees/members & \{subj\} organization number of employees or members \{obj\} \\
    per:cities\_of\_residence & \{subj\} person cities of residence \{obj\} \\
    org:city\_of\_headquarters & \{subj\} organization city of headquarters \{obj\} \\
    per:children & \{subj\} person children \{obj\} \\
    per:employee\_of & \{subj\} person employee of \{obj\} \\
    org:political/religious\_affiliation & \{subj\} organization political or religious affiliation \{obj\} \\
    per:parents & \{subj\} person parents \{obj\} \\
    per:city\_of\_birth & \{subj\} person city of birth \{obj\} \\
    per:age & \{subj\} person age \{obj\} \\
    per:countries\_of\_residence & \{subj\} person countries of residence \{obj\} \\
    org:alternate\_names & \{subj\} organization alternate names \{obj\} \\
    per:religion & \{subj\} person religion \{obj\} \\
    per:city\_of\_death & \{subj\} person city of death \{obj\} \\
    per:country\_of\_death & \{subj\} person country of death \{obj\} \\
    org:founded\_by & \{subj\} organization founded by \{obj\} \\
    \bottomrule
    \end{tabular}
    \end{threeparttable}
    \caption{Structural templates for TACRED, where \{subj\} and \{obj\} are the placeholders for subject and object entities.}
    \label{tab:structural-templates}
\end{table*}

\begin{table*}[h]
    \centering
    \begin{threeparttable}
    \small
    \begin{tabular}{cc}
    \toprule
    Relation & Template \\
    \midrule
    Other & \{subj\} is not related to \{obj\}\\
    Component-Whole(e1,e2) & \{subj\} is the component of \{obj\}\\
    Component-Whole(e2,e1) & \{subj\} has the component \{obj\}\\
    Instrument-Agency(e1,e2) & \{subj\} is the instrument of \{obj\}\\
    Instrument-Agency(e2,e1) & \{subj\} has the instrument \{obj\}\\
    Member-Collection(e1,e2) & \{subj\} is the member of \{obj\}\\
    Member-Collection(e2,e1) & \{subj\} has the member \{obj\}\\
    Cause-Effect(e1,e2) & \{subj\} has the effect \{obj\}\\
    Cause-Effect(e2,e1) & \{subj\} is the effect of \{obj\}\\
    Entity-Destination(e1,e2) & \{subj\} locates in \{obj\}\\
    Entity-Destination(e2,e1) & \{subj\} is the destination of \{obj\}\\
    Content-Container(e1,e2) & \{subj\} is the content in \{obj\}\\
    Content-Container(e2,e1) & \{subj\} contains \{obj\}\\
    Message-Topic(e1,e2) & \{subj\} is the message on \{obj\}\\
    Message-Topic(e2,e1) & \{subj\} is the topic of \{obj\}\\
    Product-Producer(e1,e2) & \{subj\} is the product of \{obj\}\\
    Product-Producer(e2,e1) & \{subj\} produces \{obj\}\\
    Entity-Origin(e1,e2) & \{subj\} origins from \{obj\}\\
    Entity-Origin(e2,e1) & \{subj\} is the origin of \{obj\}\\
 \bottomrule
    \end{tabular}
    \end{threeparttable}
    \caption{Semantic templates for SemEval, where \{subj\} and \{obj\} are the placeholders for subject and object entities.}
    \label{tab:semantic1-templates-semeval}
\end{table*}

\begin{comment}
\subsection{Limitations and Potential Risks}

\xhdr{Limitations} \modelname assumes that the annotation of summarization and manual-designed templates of relations are cheap to obtain. This assumption is hold in the most common domains. However, they can be probably counterfactual in specific domains. For eaxmple, designing templates for biomedical relation extraction needs expert knowledge, which will increase manual efforts in \modelname.

\xhdr{Potential risks} \modelname are based on pretrained models, such as BARTs~\cite{lewis-etal-2020-bart}. Therefore, the summarization of our models can be influenced by the training corpus of these pretrained models, which may raise potential risks of generating malicious, counterfactual, and biased sentences and causing ethics concerns. We suggest examing these issues before applying \modelname in real-world.
\end{comment}