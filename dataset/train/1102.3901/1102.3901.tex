
\documentclass[preprint,12pt]{elsarticle}
\usepackage{multicol}

\usepackage{amsmath}
\usepackage{amsfonts}
\usepackage{amssymb}
\usepackage{amsthm}
\usepackage{verbatim}

\usepackage{graphicx}
\usepackage{epic}
\usepackage{eepic}
\usepackage{epsfig,float}
\usepackage{pdfsync}
\usepackage{bm}


\DeclareGraphicsRule{.tif}{png}{.png}{`convert #1 `dirname #1`/`basename #1 .tif`.png}

\renewcommand{\le}{\leqslant}
\renewcommand{\ge}{\geqslant}

\newcommand{\co}{companion}
\newcommand{\ol}{\overline}
\newcommand{\eps}{\varepsilon}
\newcommand{\emp}{\emptyset}
\newcommand{\rhoR}{R}
\newcommand{\Sig}{\Sigma}
\newcommand{\sig}{\sigma}
\newcommand{\noin}{\noindent}
\newcommand{\ur}{uniquely reachable}
\newcommand{\bi}{\begin{itemize}}
\newcommand{\ei}{\end{itemize}}
\newcommand{\be}{\begin{enumerate}}
\newcommand{\ee}{\end{enumerate}}
\newcommand{\bd}{\begin{description}}
\newcommand{\ed}{\end{description}}
\newcommand{\eq}{\end{quote}}
\newcommand{\txt}[1]{\mbox{ #1 }}

\newcommand{\etc}{\mbox{\it etc.}}
\newcommand{\ie}{\mbox{\it i.e.}}
\newcommand{\eg}{\mbox{\it e.g.}}

\newcommand{\inv}[1]{\mbox{}}

\newcommand{\stress}[1]{{\fontfamily{cmtt}\selectfont #1}}

\def\shu{\mathbin{\mathchoice
{\rule{.3pt}{1ex}\rule{.3em}{.3pt}\rule{.3pt}{1ex}
\rule{.3em}{.3pt}\rule{.3pt}{1ex}}
{\rule{.3pt}{1ex}\rule{.3em}{.3pt}\rule{.3pt}{1ex}
\rule{.3em}{.3pt}\rule{.3pt}{1ex}}
{\rule{.2pt}{.7ex}\rule{.2em}{.2pt}\rule{.2pt}{.7ex}
\rule{.2em}{.2pt}\rule{.2pt}{.7ex}}
{\rule{.3pt}{1ex}\rule{.3em}{.3pt}\rule{.3pt}{1ex}
\rule{.3em}{.3pt}\rule{.3pt}{1ex}}\mkern2mu}}

\newcommand{\bA}{{\mathbf A}}
\newcommand{\ba}{{\mathbf a}}
\newcommand{\bB}{{\mathbf B}}
\newcommand{\bC}{{\mathbf C}}
\newcommand{\bD}{{\mathbf D}}
\newcommand{\bE}{{\mathbf E}}
\newcommand{\bF}{{\mathbf F}}
\newcommand{\bG}{{\mathbf G}}
\newcommand{\bH}{{\mathbf H}}
\newcommand{\bI}{{\mathbf I}}
\newcommand{\bJ}{{\mathbf J}}
\newcommand{\bK}{{\mathbf K}}
\newcommand{\bQ}{{\mathbf Q}}
\newcommand{\bq}{{\mathbf q}}
\newcommand{\bR}{{\mathbf R}}
\newcommand{\bS}{{\mathbf S}}

\newcommand{\bbA}{{\mathbb A}}
\newcommand{\bbB}{{\mathbb B}}
\newcommand{\bbC}{{\mathbb C}}
\newcommand{\bbD}{{\mathbb D}}
\newcommand{\bbE}{{\mathbb E}}
\newcommand{\bbI}{{\mathbb I}}
\newcommand{\bbK}{{\mathbb K}}
\newcommand{\bbL}{{\mathbb L}}
\newcommand{\bbM}{{\mathbb M}}
\newcommand{\bbN}{{\mathbb N}}
\newcommand{\bbP}{{\mathbb P}}
\newcommand{\bbR}{{\mathbb R}}
\newcommand{\bbS}{{\mathbb S}}
\newcommand{\bbT}{{\mathbb T}}
\newcommand{\bbU}{{\mathbb U}}
\newcommand{\bbZ}{{\mathbb Z}}

\newcommand{\bmA}{\bm{A}}
\newcommand{\bmB}{\bm{B}}
\newcommand{\bmF}{\bm{F}}
\newcommand{\bmI}{\bm{I}}
\newcommand{\bmK}{\bm{K}}
\newcommand{\bmS}{\bm{S}}

\newcommand{\fA}{{\mathfrak A}}
\newcommand{\fB}{{\mathfrak B}}
\newcommand{\fC}{{\mathfrak C}}
\newcommand{\fD}{{\mathfrak D}}
\newcommand{\fK}{{\mathfrak K}}
\newcommand{\fL}{{\mathfrak L}}
\newcommand{\fN}{{\mathfrak N}}
\newcommand{\fM}{{\mathfrak M}}
\newcommand{\fQ}{{\mathfrak Q}}
\newcommand{\fR}{{\mathfrak R}}

\newcommand{\cA}{{\mathcal A}}
\newcommand{\cB}{{\mathcal B}}
\newcommand{\cC}{{\mathcal C}}
\newcommand{\cD}{{\mathcal D}}
\newcommand{\cE}{{\mathcal E}}
\newcommand{\cI}{{\mathcal I}}
\newcommand{\cL}{{\mathcal L}}
\newcommand{\cM}{{\mathcal M}}
\newcommand{\cN}{{\mathcal N}}
\newcommand{\cP}{{\mathcal P}}
\newcommand{\cR}{{\mathcal R}}
\newcommand{\cS}{{\mathcal S}}
\newcommand{\cT}{{\mathcal T}}
\newcommand{\cU}{{\mathcal U}}
\newcommand{\cX}{{\mathcal X}}
\newcommand{\spd}{}
\newcommand{\one}{{\mathbf 1}}

\newcommand{\Lra}{{\hspace{.1cm}\Leftrightarrow\hspace{.1cm}}}
\newcommand{\lra}{{\hspace{.1cm}\leftrightarrow\hspace{.1cm}}}
\newcommand{\la}{{\hspace{.1cm}\leftarrow\hspace{.1cm}}}
\newcommand{\raL}{{\hspace{.1cm}{\rightarrow_L} \hspace{.1cm}}}
\newcommand{\lraL}{{\hspace{.1cm}{\leftrightarrow_L} \hspace{.1cm}}}

\newcommand{\sn}{{semiautomaton}}
\newcommand{\sa}{{semiautomata}}
\newcommand{\Sn}{{Semiautomaton}}
\newcommand{\Sa}{{Semiautomata}}
\newcommand{\se}{{settable}}
\newcommand{\Se}{{Settable}}
\newcommand{\pc}{{prefix-continuous}}

\newcommand{\cover}{{version}}

\newcommand{\com}{\mathbb{C}}
\newcommand{\rev}{R}
\newcommand{\deter}{D}
\newcommand{\mini}{M}
\newcommand{\trim}{T}

\newcommand{\qedb}{\hfill} 

\newtheorem{proposition}{Proposition}
\newtheorem{example}{Example}
\newtheorem{lemma}{Lemma}
\newtheorem{theorem}{Theorem}
\newtheorem{corollary}{Corollary}
\newtheorem{definition}{Definition}

\journal{Theoretical Computer Science}

\begin{document}

\begin{frontmatter}

\title{Theory of \'Atomata\tnoteref{support}}
\tnotetext[support]{This work was supported 
by the Natural Sciences and Engineering Research Council of Canada under grant No.~OGP0000871, 
the ERDF funded Estonian Center of Excellence in Computer Science, EXCS, 
the Estonian Science Foundation grant 7520, and the Estonian Ministry of Education and 
Research target-financed research theme no. 0140007s12.}

\author[JB]{Janusz~Brzozowski}
\ead{brzozo@uwaterloo.ca}
\author[HT]{Hellis~Tamm}
\ead{hellis@cs.ioc.ee}
\address[JB]{David R. Cheriton School of Computer Science, University of Waterloo,
Waterloo, ON, Canada N2L 3G1}
\address[HT]{Institute of Cybernetics, Tallinn University of Technology,
Akadeemia tee 21, 12618 Tallinn, Estonia}

\begin{abstract}
We show that every regular language defines a unique nondeterministic finite 
automaton (NFA), which we call ``\'atomaton'', whose states are the ``atoms'' of 
the language, that is, non-empty intersections of complemented or uncomplemented 
left quotients of the language.
We describe methods of constructing the \'atomaton, and prove that it is 
isomorphic to the reverse automaton of the minimal deterministic finite automaton 
(DFA) of the reverse language.
We study ``atomic'' NFAs in which the right language of every state 
is a union of atoms. We generalize Brzozowski's double-reversal method for 
minimizing a deterministic finite automaton (DFA), showing that the result of 
applying the subset construction to an NFA is a minimal DFA if and only if 
the reverse of the NFA is atomic.
We prove that Sengoku's claim that his method always finds a minimal NFA is false.
\end{abstract}

\begin{keyword}
regular languages \sep left quotients
\end{keyword}

\end{frontmatter}

\section{Introduction}

Nondeterministic finite automata (NFAs) were introduced by 
Rabin and Scott~\cite{RaSc59} in 1959 and still
play a major role in the theory of automata. 
For many purposes it is necessary to convert an NFA to a deterministic finite automaton (DFA).
In particular, for each NFA there exists a minimal DFA, unique up to isomorphism.
This DFA is uniquely defined by every regular language, and uses the left quotients 
of the language as states. As well, it is  possible to associate an NFA with each DFA, 
and this is the subject of the present paper. Our NFA is also uniquely defined by 
every regular language, and uses non-empty intersections of complemented and 
uncomplemented quotients---the ``atoms'' of the language---as states. 

It appears that the NFA most often associated with a regular language is 
the universal automaton, sometimes  appearing under different names. 
A substantial survey by Lombardy and Sakarovitch~\cite{LoSa07} on the subject 
of the universal automaton contains its history and a detailed discussion 
of its properties. We refer the reader to that paper, and mention only that 
research related to the universal automaton goes back to the 
1970's: \eg, in~\cite{Car70} as reported in~\cite{ADN92}, \cite{Con71,KaWe70}.

We call our NFA the ``\'atomaton'' because it is based on the atoms of a regular language; 
we add the accent\footnote{The word should be pronounced with 
the accent on the first a.} to minimize the possible confusion between ``automaton'' and ``atomaton''.
We prove that the \'atomaton of a regular language  is isomorphic to the reverse 
automaton of the minimal DFA of the reverse language .

We  introduce ``atomic'' automata, in which the right language of any state is 
a union of some atoms. 
This generalizes residual automata~\cite{DLT02} 
in which the right language of any state is a left quotient (which we prove 
to be a union of atoms), and includes also  \'atomata (where the right language 
of any state is an atom), DFAs, and universal automata.

We characterize the class of NFAs for which the subset construction 
yields a minimal DFA. More specifically, we show that the subset construction 
applied to an NFA produces a minimal DFA if and only if the reverse automaton 
of that NFA is atomic. 
This generalizes Brzozowski's method for DFA minimization by double reversal~\cite{Brz63}.

We study reduced atomic NFAs associated with a given regular language. We formalize Sengoku's approach~\cite{Sen92} to finding minimal NFAs, and prove that it does not always work, since there exist languages for which no atomic NFA is minimal.

Section~\ref{sec:LAE} recalls properties of regular languages, finite automata, and systems of language equations.
In Section~\ref{sec:partial}, we study the right languages of the states of any NFA; we call these languages ``partial quotients". We define partial atoms and partial \'atomata and study their properties.
Quotients and atoms of a regular language and the \'atomaton are introduced and 
studied in Section~\ref{sec:atoms}.
In Section~\ref{sec:atomic}, we examine NFAs in which the right language of every state is a union of atoms, and we extend
Brzozowski's method~\cite{Brz63} of DFA minimization.
In Section~\ref{sec:reduced}, we study reduced atomic NFAs accepting a given regular language, and prove that Sengoku's claim~\cite{Sen92} that his method always finds a minimal NFA is false.
Section~\ref{sec:conc} closes the paper.


A much shorter version of some of the results presented here has previously 
appeared in~\cite{BrTa11}. Note that here we use a definition of an atom 
which is slightly different from that of~\cite{BrTa11} for reasons explained 
at the end of Section~\ref{sec:atoms}.



\section{Languages, Automata and Equations}
\label{sec:LAE}
If  is a non-empty finite alphabet, then  is the free monoid generated by .
 A \emph{word} is any element of , and the empty word is . The length of a word  is .
A \emph{language} over  is any subset of . 

The following   operations are defined on languages over :  
\emph{complement} \linebreak (),  
\emph{union}  (),  
\emph{intersection} (),  
\emph{product}, usually called  
\emph{concatenation} or \emph{catenation},   
(), \emph{positive closure} (), and  \emph{star} ().
The \emph{reverse  of a word}  is defined as follows: 
, and , where .
The \emph{reverse of a language}  is denoted 
by  and defined as .

A~\emph{nondeterministic finite automaton} is a quintuple 
, where 
 is a finite, non-empty set of \emph{states}, 
 is a finite non-empty \emph{alphabet}, 
 is the  \emph{transition function},
 is the set of  \emph{initial states},
and  is the set of \emph{final states}.
As usual, we extend the transition function to functions 
, and .
We do not distinguish these functions notationally, but use  for all three.
The \emph{language accepted} by an NFA  is 
.
Two NFAs are \emph{equivalent} if they accept the same language. 
The  \emph{left language} of a state  of  is 
. 
The \emph{right language} of  is 
.
The \emph{right language} of a set  of states of  is
; hence
.
A state is \emph{unreachable} if its left language is empty.
A state is \emph{empty} if its right language is empty.
An NFA is \emph{trim} if it has no empty or unreachable states.
Two states  of an NFA are \emph{equivalent} if their right languages are identical. 
An NFA is \emph{reduced} if it has no equivalent states. 
An NFA is \emph{minimal} if it has the minimal number of states among all
the equivalent NFAs.

A \emph{deterministic finite automaton} is a quintuple 
, where
, , and  are as in an NFA, 
 is the transition function, 
and  is the initial state. 
A DFA is an NFA in which the set of initial states is  and 
the range of the transition function is restricted to singletons , .

A DFA is \emph{minimal} if it has no unreachable states and 
no two of its states are equivalent.



We use the following operations on automata: \\
\hglue 10 pt 1.
\emph{Determinization} () 
applied to an NFA  yields a DFA  obtained by the well-known subset construction, where only subsets (including the empty subset) reachable from the initial subset of  are used. \\
\hglue 10 pt 2.
 \emph{Reversal} () applied to an NFA  yields an NFA , where initial and final states of  
are interchanged in  and all the transitions  
are reversed. \\
\hglue 10 pt 3.
\emph{Trimming} () applied to an NFA  accepting 
a non-empty language deletes from  all unreachable and empty states,
along with the incident transitions, yielding an NFA . \\
\hglue 10 pt 4.
\emph{Minimization}  () applied to a DFA  yields 
the minimal DFA  equivalent to .

A trim DFA  is \emph{bideterministic} if also  is a DFA.
A language is \emph{bideterministic} if it is accepted by a bideterministic 
DFA.

\begin{example}
\label{ex:aut_ops}
Figure~\ref{fig:aut_ops}~(a) shows an NFA , where the initial states are indicated by incoming arrows and final states by double circles. Its determinized DFA  is in 
Fig.~\ref{fig:aut_ops}~(b), where braces around sets are omitted.  
The minimal equivalent DFA    of   
is in Fig.~\ref{fig:aut_ops}~(c),
where the equivalent states , ,  and  are represented by 
. The reversed and trimmed version  of  
is in Fig.~\ref{fig:aut_ops}~(d). \qedb
\end{example}
\begin{figure}[hbt]
\begin{center}
\setlength{\unitlength}{0.00043745in}
\begingroup\makeatletter\ifx\SetFigFont\undefined \gdef\SetFigFont#1#2#3#4#5{\reset@font\fontsize{#1}{#2pt}\fontfamily{#3}\fontseries{#4}\fontshape{#5}\selectfont}\fi\endgroup {\renewcommand{\dashlinestretch}{30}
\begin{picture}(10284,4052)(0,-10)
\put(8183,1622){\makebox(0,0)[lb]{\smash{{\SetFigFont{10}{12.0}{\familydefault}{\mddefault}{\updefault}}}}}
\put(5186.000,3444.000){\arc{364.664}{2.4959}{6.9871}}
\blacken\path(5339.259,3448.869)(5325.000,3326.000)(5395.402,3427.703)(5339.259,3448.869)
\put(8262.000,3455.000){\arc{364.664}{2.4959}{6.9871}}
\blacken\path(8415.259,3459.869)(8401.000,3337.000)(8471.402,3438.703)(8415.259,3459.869)
\put(1196.691,3090.686){\arc{364.730}{2.4963}{6.9846}}
\blacken\path(1350.223,3095.873)(1336.000,2973.000)(1406.372,3074.724)(1350.223,3095.873)
\put(3724.691,3082.686){\arc{364.730}{2.4963}{6.9846}}
\blacken\path(3878.223,3087.873)(3864.000,2965.000)(3934.372,3066.724)(3878.223,3087.873)
\put(10009.686,1740.000){\arc{364.258}{4.0681}{8.5565}}
\blacken\path(10014.856,1586.630)(9892.000,1601.000)(9993.640,1530.506)(10014.856,1586.630)
\put(3316.686,576.000){\arc{364.258}{4.0681}{8.5565}}
\blacken\path(3321.856,422.630)(3199.000,437.000)(3300.640,366.506)(3321.856,422.630)
\put(5182,1706){\ellipse{720}{720}}
\put(5186,440){\ellipse{720}{720}}
\put(6446,1707){\ellipse{720}{720}}
\put(6453,454){\ellipse{720}{720}}
\put(5184,437){\ellipse{630}{630}}
\put(5195,2999){\ellipse{720}{720}}
\put(8265,1729){\ellipse{720}{720}}
\put(9529,1730){\ellipse{720}{720}}
\put(8269,3005){\ellipse{720}{720}}
\put(1194,2635){\ellipse{720}{720}}
\put(1194,2635){\ellipse{630}{630}}
\put(3718,2637){\ellipse{720}{720}}
\put(2458,2638){\ellipse{720}{720}}
\put(3718,2637){\ellipse{630}{630}}
\put(1583,590){\ellipse{720}{720}}
\put(2842,588){\ellipse{630}{630}}
\put(2843,589){\ellipse{720}{720}}
\put(6441,1709){\ellipse{630}{630}}
\put(6452,452){\ellipse{630}{630}}
\put(9536,1726){\ellipse{630}{630}}
\path(5550,446)(6082,446)
\blacken\path(5962.000,416.000)(6082.000,446.000)(5962.000,476.000)(5962.000,416.000)
\path(6450,1339)(6450,807)
\blacken\path(6420.000,927.000)(6450.000,807.000)(6480.000,927.000)(6420.000,927.000)
\path(5003,1384)(5003,754)
\blacken\path(4973.000,874.000)(5003.000,754.000)(5033.000,874.000)(4973.000,874.000)
\path(5363,751)(5363,1381)
\blacken\path(5393.000,1261.000)(5363.000,1381.000)(5333.000,1261.000)(5393.000,1261.000)
\path(6188,723)(5445,1451)
\blacken\path(5551.709,1388.445)(5445.000,1451.000)(5509.718,1345.588)(5551.709,1388.445)
\path(6085,2459)(6333,2038)
\blacken\path(6246.245,2126.168)(6333.000,2038.000)(6297.942,2156.621)(6246.245,2126.168)
\path(5190,2081)(5190,2621)
\blacken\path(5220.000,2501.000)(5190.000,2621.000)(5160.000,2501.000)(5220.000,2501.000)
\path(9168,2482)(9416,2061)
\blacken\path(9329.245,2149.168)(9416.000,2061.000)(9380.942,2179.621)(9329.245,2149.168)
\path(8258,2099)(8258,2639)
\blacken\path(8288.000,2519.000)(8258.000,2639.000)(8228.000,2519.000)(8288.000,2519.000)
\path(1509,2815)(2139,2815)
\blacken\path(2019.000,2785.000)(2139.000,2815.000)(2019.000,2845.000)(2019.000,2785.000)
\path(2139,2455)(1509,2455)
\blacken\path(1629.000,2485.000)(1509.000,2455.000)(1629.000,2425.000)(1629.000,2485.000)
\path(2004,3348)(2252,2927)
\blacken\path(2165.245,3015.168)(2252.000,2927.000)(2216.942,3045.621)(2165.245,3015.168)
\path(3272,3329)(3520,2908)
\blacken\path(3433.245,2996.168)(3520.000,2908.000)(3484.942,3026.621)(3433.245,2996.168)
\path(3353,2633)(2813,2633)
\blacken\path(2933.000,2663.000)(2813.000,2633.000)(2933.000,2603.000)(2933.000,2663.000)
\path(6075,1698)(5565,1698)
\blacken\path(5685.000,1728.000)(5565.000,1698.000)(5685.000,1668.000)(5685.000,1728.000)
\path(9234,1923)(8604,1923)
\blacken\path(8724.000,1953.000)(8604.000,1923.000)(8724.000,1893.000)(8724.000,1953.000)
\path(8589,1563)(9219,1563)
\blacken\path(9099.000,1533.000)(9219.000,1563.000)(9099.000,1593.000)(9099.000,1533.000)
\path(2397,1281)(2645,860)
\blacken\path(2558.245,948.168)(2645.000,860.000)(2609.942,978.621)(2558.245,948.168)
\path(2540,400)(1910,400)
\blacken\path(2030.000,430.000)(1910.000,400.000)(2030.000,370.000)(2030.000,430.000)
\path(1887,768)(2517,768)
\blacken\path(2397.000,738.000)(2517.000,768.000)(2397.000,798.000)(2397.000,738.000)
\path(1276,2283)(1277,2282)(1279,2281)
	(1283,2278)(1289,2273)(1298,2267)
	(1310,2259)(1325,2249)(1342,2236)
	(1363,2222)(1387,2206)(1414,2188)
	(1443,2169)(1475,2149)(1509,2128)
	(1544,2107)(1581,2085)(1620,2063)
	(1660,2042)(1701,2020)(1744,1999)
	(1788,1979)(1833,1959)(1880,1941)
	(1929,1923)(1980,1906)(2033,1891)
	(2088,1877)(2146,1864)(2206,1853)
	(2269,1844)(2334,1837)(2401,1833)
	(2469,1832)(2537,1834)(2603,1838)
	(2667,1845)(2728,1855)(2787,1866)
	(2842,1879)(2895,1893)(2945,1909)
	(2993,1926)(3039,1944)(3083,1963)
	(3125,1983)(3166,2004)(3205,2025)
	(3243,2046)(3279,2068)(3314,2090)
	(3347,2112)(3379,2134)(3409,2155)
	(3438,2175)(3464,2194)(3488,2212)
	(3509,2228)(3527,2243)(3543,2255)
	(3556,2266)(3567,2274)(3574,2280)(3586,2290)
\blacken\path(3513.019,2190.131)(3586.000,2290.000)(3474.608,2236.225)(3513.019,2190.131)
\put(7149,349){\makebox(0,0)[lb]{\smash{{\SetFigFont{10}{12.0}{\familydefault}{\mddefault}{\updefault}}}}}
\put(5438,859){\makebox(0,0)[lb]{\smash{{\SetFigFont{10}{12.0}{\familydefault}{\mddefault}{\updefault}}}}}
\put(5881,1099){\makebox(0,0)[lb]{\smash{{\SetFigFont{10}{12.0}{\familydefault}{\mddefault}{\updefault}}}}}
\put(6547,940){\makebox(0,0)[lb]{\smash{{\SetFigFont{10}{12.0}{\familydefault}{\mddefault}{\updefault}}}}}
\put(5107,333){\makebox(0,0)[lb]{\smash{{\SetFigFont{10}{12.0}{\familydefault}{\mddefault}{\updefault}}}}}
\put(5092,1592){\makebox(0,0)[lb]{\smash{{\SetFigFont{10}{12.0}{\familydefault}{\mddefault}{\updefault}}}}}
\put(4695,880){\makebox(0,0)[lb]{\smash{{\SetFigFont{10}{12.0}{\familydefault}{\mddefault}{\updefault}}}}}
\put(5723,123){\makebox(0,0)[lb]{\smash{{\SetFigFont{10}{12.0}{\familydefault}{\mddefault}{\updefault}}}}}
\put(5108,2906){\makebox(0,0)[lb]{\smash{{\SetFigFont{10}{12.0}{\familydefault}{\mddefault}{\updefault}}}}}
\put(5295,2223){\makebox(0,0)[lb]{\smash{{\SetFigFont{10}{12.0}{\familydefault}{\mddefault}{\updefault}}}}}
\put(8363,2250){\makebox(0,0)[lb]{\smash{{\SetFigFont{10}{12.0}{\familydefault}{\mddefault}{\updefault}}}}}
\put(1104,3363){\makebox(0,0)[lb]{\smash{{\SetFigFont{10}{12.0}{\familydefault}{\mddefault}{\updefault}}}}}
\put(3623,3348){\makebox(0,0)[lb]{\smash{{\SetFigFont{10}{12.0}{\familydefault}{\mddefault}{\updefault}}}}}
\put(1682,2898){\makebox(0,0)[lb]{\smash{{\SetFigFont{10}{12.0}{\familydefault}{\mddefault}{\updefault}}}}}
\put(1771,2170){\makebox(0,0)[lb]{\smash{{\SetFigFont{10}{12.0}{\familydefault}{\mddefault}{\updefault}}}}}
\put(2326,1923){\makebox(0,0)[lb]{\smash{{\SetFigFont{10}{12.0}{\familydefault}{\mddefault}{\updefault}}}}}
\put(3001,2371){\makebox(0,0)[lb]{\smash{{\SetFigFont{10}{12.0}{\familydefault}{\mddefault}{\updefault}}}}}
\put(5730,1810){\makebox(0,0)[lb]{\smash{{\SetFigFont{10}{12.0}{\familydefault}{\mddefault}{\updefault}}}}}
\put(10269,1640){\makebox(0,0)[lb]{\smash{{\SetFigFont{10}{12.0}{\familydefault}{\mddefault}{\updefault}}}}}
\put(7981,3780){\makebox(0,0)[lb]{\smash{{\SetFigFont{10}{12.0}{\familydefault}{\mddefault}{\updefault}}}}}
\put(4906,3782){\makebox(0,0)[lb]{\smash{{\SetFigFont{10}{12.0}{\familydefault}{\mddefault}{\updefault}}}}}
\put(8852,2036){\makebox(0,0)[lb]{\smash{{\SetFigFont{10}{12.0}{\familydefault}{\mddefault}{\updefault}}}}}
\put(8844,1258){\makebox(0,0)[lb]{\smash{{\SetFigFont{10}{12.0}{\familydefault}{\mddefault}{\updefault}}}}}
\put(2088,870){\makebox(0,0)[lb]{\smash{{\SetFigFont{10}{12.0}{\familydefault}{\mddefault}{\updefault}}}}}
\put(2081,100){\makebox(0,0)[lb]{\smash{{\SetFigFont{10}{12.0}{\familydefault}{\mddefault}{\updefault}}}}}
\put(3580,468){\makebox(0,0)[lb]{\smash{{\SetFigFont{10}{12.0}{\familydefault}{\mddefault}{\updefault}}}}}
\put(8184,2910){\makebox(0,0)[lb]{\smash{{\SetFigFont{10}{12.0}{\familydefault}{\mddefault}{\updefault}}}}}
\put(9240,2913){\makebox(0,0)[lb]{\smash{{\SetFigFont{10}{12.0}{\familydefault}{\mddefault}{\updefault}{\bf (c)}}}}}
\put(6135,2913){\makebox(0,0)[lb]{\smash{{\SetFigFont{10}{12.0}{\familydefault}{\mddefault}{\updefault}{\bf (b)}}}}}
\put(15,2553){\makebox(0,0)[lb]{\smash{{\SetFigFont{10}{12.0}{\familydefault}{\mddefault}{\updefault}{\bf (a)}}}}}
\put(15,483){\makebox(0,0)[lb]{\smash{{\SetFigFont{10}{12.0}{\familydefault}{\mddefault}{\updefault}{\bf (d)}}}}}
\put(1118,2545){\makebox(0,0)[lb]{\smash{{\SetFigFont{10}{12.0}{\familydefault}{\mddefault}{\updefault}}}}}
\put(2379,2545){\makebox(0,0)[lb]{\smash{{\SetFigFont{10}{12.0}{\familydefault}{\mddefault}{\updefault}}}}}
\put(3631,2546){\makebox(0,0)[lb]{\smash{{\SetFigFont{10}{12.0}{\familydefault}{\mddefault}{\updefault}}}}}
\put(1511,482){\makebox(0,0)[lb]{\smash{{\SetFigFont{10}{12.0}{\familydefault}{\mddefault}{\updefault}}}}}
\put(2602,498){\makebox(0,0)[lb]{\smash{{\SetFigFont{10}{12.0}{\familydefault}{\mddefault}{\updefault}}}}}
\put(6202,1595){\makebox(0,0)[lb]{\smash{{\SetFigFont{10}{12.0}{\familydefault}{\mddefault}{\updefault}}}}}
\put(6210,348){\makebox(0,0)[lb]{\smash{{\SetFigFont{10}{12.0}{\familydefault}{\mddefault}{\updefault}}}}}
\put(9286,1632){\makebox(0,0)[lb]{\smash{{\SetFigFont{10}{12.0}{\familydefault}{\mddefault}{\updefault}}}}}
\put(6897.686,464.000){\arc{364.258}{4.0681}{8.5565}}
\blacken\path(6902.856,310.630)(6780.000,325.000)(6881.640,254.506)(6902.856,310.630)
\end{picture}
}
 \end{center}
\caption{(a) An NFA ; (b) ; (c) ; 
(d) .} 
\label{fig:aut_ops}
\end{figure}

The \emph{left quotient}, or simply \emph{quotient,} of a language  
by a word  is  the language . 
Left quotients are also known as \emph{right residuals}.
Dually, the \emph{right quotient} of a language  by a word  is  
the language . 
Evidently, if  is an NFA and  is in ,  
then .

The \emph{quotient DFA} of a regular language  is 
, where , 
, 
,  and
.
The quotient DFA of  is the minimal DFA for .


The following is from~\cite{LoSa07}:
If , a \emph{subfactorization} 
of  is  a pair  of languages 
over  such that .
A~\emph{factorization} of  is 
a subfactorisation  such that, if ,  ,
and  for any pair , then  and .
The \emph{universal automaton} of  is 
 where  is the set of all factorizations
of , ,
, and
 if and only if .


For any language  let  if  and 
otherwise.
Also, let  and let .
A~\emph{nondeterministic system of equations (NSE)} with  variables 
 is a set of language equations \\

where ,  together with an 
\emph{initial set of variables}  , where  is an index set.
The equations are assumed to have been simplified by the rules

Let ; then  is the left quotient of  by .
The language defined by an NSE is .

Each NSE defines a unique NFA  and \emph{vice versa}. States of  correspond 
to the variables , there is a transition   
in  if and only if , the set of initial states of  is 
, and the set of final states is .

If each  is a left quotient (that is, a right residual) of the language 
, then the NSE and the corresponding NFA are called 
\emph{residual}~\cite{DLT02}.

A~\emph{deterministic system of equations (DSE)} 
is an NSE

where , , and 
the empty language  is retained if it appears. 

Each DSE defines a unique DFA  and \emph{vice versa}. Each state of  
corresponds to a variable , there is a transition  
 in  if and only if 
,
the initial state of  corresponds to , and the set of final states is 
.
In the special case when  is minimal, its DSE constitutes its 
\emph{quotient equations}, where every  is a quotient of the initial 
language .

To simplify the notation, we write  instead of  in  equations.
\begin{example}
\label{ex:NSE}
For the NFA of Fig.~\ref{fig:aut_ops}~(a), we have the NSE 

with the initial set 
.
The language  accepted by the DFA of Fig.~\ref{fig:aut_ops}~(b) 
is obtained  from this NSE as shown by the equations below on the left. 
Renaming the unions of variables by new variables corresponding to subsets in 
the subset construction, we get the equations on the right;
for example,  is renamed as .
This is
the DSE for the DFA of Fig.~\ref{fig:aut_ops}~(b). 

 Noting that , , and  are equivalent, 
we get the quotient equations for the DFA of Fig.~\ref{fig:aut_ops}~(c), 
where , , \etc
  
\qedb\end{example}



\section{Partial Quotients, Partial Atoms and Partial \'Atomata}
\label{sec:partial}
\subsection{Introduction and Motivation}
In 1992, Sengoku~\cite{Sen92} defined an NFA to be \emph{disjoint} if the right languages of any two distinct states are disjoint. He noted that a disjoint NFA  has exactly one final state, and proved that an NFA is disjoint if and only if  is deterministic. It follows that if we reverse, determinize\footnote{The reader should note that Sengoku's DFAs are incomplete, and he does not include the empty
state in the  determinized version of an NFA.}, and reverse , the resulting NFA  is a disjoint NFA equivalent to . 

We shall show that another NFA obtained from  by a completely different process turns out to be isomorphic to . In our approach we start with the right languages of states of , which we call \emph{partial quotients of }. This terminology is logical, since the right language of a state of  that is reached by word  is always a subset of the quotient of  by . Next we construct all nonempty intersections of complemented and uncomplemented partial quotients of , and refer to these languages as 
\emph{partial atoms of }. These partial atoms become states of an NFA which we call 
\emph{partial \'atomaton of }. We then prove that the partial \'atomaton of  is isomorphic to .

We begin with a simple example to illustrate the formal ideas that follow.

\begin{figure}[t]
\begin{center}
\setlength{\unitlength}{0.00043745in}
\begingroup\makeatletter\ifx\SetFigFont\undefined \gdef\SetFigFont#1#2#3#4#5{\reset@font\fontsize{#1}{#2pt}\fontfamily{#3}\fontseries{#4}\fontshape{#5}\selectfont}\fi\endgroup {\renewcommand{\dashlinestretch}{30}
\begin{picture}(7860,3207)(0,-10)
\put(7332,1540){\makebox(0,0)[lb]{\smash{{\SetFigFont{10}{12.0}{\familydefault}{\mddefault}{\updefault}}}}}
\put(4973.000,2107.000){\arc{364.664}{2.4959}{6.9871}}
\blacken\thicklines
\path(5118.214,2128.974)(5112.000,1989.000)(5189.516,2105.715)(5118.214,2128.974)
\thinlines
\put(2085.000,3000.000){\arc{364.664}{2.4959}{6.9871}}
\blacken\thicklines
\path(2230.214,3021.974)(2224.000,2882.000)(2301.516,2998.715)(2230.214,3021.974)
\thinlines
\put(2079.309,243.314){\arc{364.730}{5.6379}{10.1262}}
\blacken\thicklines
\path(1934.779,220.986)(1940.000,361.000)(1863.313,243.738)(1934.779,220.986)
\thinlines
\put(6237,737){\ellipse{720}{720}}
\put(6237,2530){\ellipse{720}{720}}
\put(6235,739){\ellipse{630}{630}}
\put(7492,1644){\ellipse{720}{720}}
\put(815,1630){\ellipse{720}{720}}
\put(2082,730){\ellipse{720}{720}}
\put(2082,2523){\ellipse{720}{720}}
\put(2080,732){\ellipse{630}{630}}
\put(4970,1637){\ellipse{720}{720}}
\path(6057,2222)(6057,1097)
\blacken\thicklines
\path(6019.500,1232.000)(6057.000,1097.000)(6094.500,1232.000)(6019.500,1232.000)
\thinlines
\path(6417,1052)(6417,2177)
\blacken\thicklines
\path(6454.500,2042.000)(6417.000,2177.000)(6379.500,2042.000)(6454.500,2042.000)
\thinlines
\path(6556,2314)(7186,1864)
\blacken\thicklines
\path(7054.350,1911.952)(7186.000,1864.000)(7097.942,1972.982)(7054.350,1911.952)
\blacken\path(5378.650,1319.048)(5247.000,1367.000)(5335.058,1258.018)(5378.650,1319.048)
\thinlines
\path(5247,1367)(5877,917)
\path(5015,857)(5015,1262)
\blacken\thicklines
\path(5052.500,1127.000)(5015.000,1262.000)(4977.500,1127.000)(5052.500,1127.000)
\thinlines
\path(7543,864)(7543,1269)
\blacken\thicklines
\path(7580.500,1134.000)(7543.000,1269.000)(7505.500,1134.000)(7580.500,1134.000)
\thinlines
\path(12,1630)(417,1630)
\blacken\thicklines
\path(282.000,1592.500)(417.000,1630.000)(282.000,1667.500)(282.000,1592.500)
\thinlines
\path(1902,2215)(1902,1090)
\blacken\thicklines
\path(1864.500,1225.000)(1902.000,1090.000)(1939.500,1225.000)(1864.500,1225.000)
\thinlines
\path(2262,1045)(2262,2170)
\blacken\thicklines
\path(2299.500,2035.000)(2262.000,2170.000)(2224.500,2035.000)(2299.500,2035.000)
\thinlines
\path(1092,1360)(1722,910)
\blacken\thicklines
\path(1590.350,957.952)(1722.000,910.000)(1633.942,1018.982)(1590.350,957.952)
\thinlines
\path(1100,1847)(1737,2289)
\blacken\thicklines
\path(1647.464,2181.229)(1737.000,2289.000)(1604.708,2242.849)(1647.464,2181.229)
\thinlines
\path(5283,1865)(5920,2307)
\blacken\thicklines
\path(5830.464,2199.229)(5920.000,2307.000)(5787.708,2260.849)(5830.464,2199.229)
\thinlines
\path(7230,1373)(6593,931)
\blacken\thicklines
\path(6682.536,1038.771)(6593.000,931.000)(6725.292,977.151)(6682.536,1038.771)
\put(6147,647){\makebox(0,0)[lb]{\smash{{\SetFigFont{10}{12.0}{\familydefault}{\mddefault}{\updefault}}}}}
\put(6162,2417){\makebox(0,0)[lb]{\smash{{\SetFigFont{10}{12.0}{\familydefault}{\mddefault}{\updefault}}}}}
\put(6530,1494){\makebox(0,0)[lb]{\smash{{\SetFigFont{10}{12.0}{\familydefault}{\mddefault}{\updefault}}}}}
\put(5323,2192){\makebox(0,0)[lb]{\smash{{\SetFigFont{10}{12.0}{\familydefault}{\mddefault}{\updefault}}}}}
\put(4760,2380){\makebox(0,0)[lb]{\smash{{\SetFigFont{10}{12.0}{\familydefault}{\mddefault}{\updefault}}}}}
\put(7242,2364){\makebox(0,0)[lb]{\smash{{\SetFigFont{10}{12.0}{\familydefault}{\mddefault}{\updefault}}}}}
\put(6889,827){\makebox(0,0)[lb]{\smash{{\SetFigFont{10}{12.0}{\familydefault}{\mddefault}{\updefault}}}}}
\put(4827,205){\makebox(0,0)[lb]{\smash{{\SetFigFont{10}{12.0}{\familydefault}{\mddefault}{\updefault}{\bf (b)}}}}}
\put(5727,1509){\makebox(0,0)[lb]{\smash{{\SetFigFont{10}{12.0}{\familydefault}{\mddefault}{\updefault}}}}}
\put(5284,932){\makebox(0,0)[lb]{\smash{{\SetFigFont{10}{12.0}{\familydefault}{\mddefault}{\updefault}}}}}
\put(6793,2146){\makebox(0,0)[lb]{\smash{{\SetFigFont{10}{12.0}{\familydefault}{\mddefault}{\updefault}}}}}
\put(732,1540){\makebox(0,0)[lb]{\smash{{\SetFigFont{10}{12.0}{\familydefault}{\mddefault}{\updefault}}}}}
\put(192,145){\makebox(0,0)[lb]{\smash{{\SetFigFont{10}{12.0}{\familydefault}{\mddefault}{\updefault}{\bf (a)}}}}}
\put(1992,640){\makebox(0,0)[lb]{\smash{{\SetFigFont{10}{12.0}{\familydefault}{\mddefault}{\updefault}}}}}
\put(2352,100){\makebox(0,0)[lb]{\smash{{\SetFigFont{10}{12.0}{\familydefault}{\mddefault}{\updefault}}}}}
\put(2007,2410){\makebox(0,0)[lb]{\smash{{\SetFigFont{10}{12.0}{\familydefault}{\mddefault}{\updefault}}}}}
\put(2375,1487){\makebox(0,0)[lb]{\smash{{\SetFigFont{10}{12.0}{\familydefault}{\mddefault}{\updefault}}}}}
\put(1572,1502){\makebox(0,0)[lb]{\smash{{\SetFigFont{10}{12.0}{\familydefault}{\mddefault}{\updefault}}}}}
\put(1168,2185){\makebox(0,0)[lb]{\smash{{\SetFigFont{10}{12.0}{\familydefault}{\mddefault}{\updefault}}}}}
\put(1144,925){\makebox(0,0)[lb]{\smash{{\SetFigFont{10}{12.0}{\familydefault}{\mddefault}{\updefault}}}}}
\put(1587,2927){\makebox(0,0)[lb]{\smash{{\SetFigFont{10}{12.0}{\familydefault}{\mddefault}{\updefault}}}}}
\put(4819,1533){\makebox(0,0)[lb]{\smash{{\SetFigFont{10}{12.0}{\familydefault}{\mddefault}{\updefault}}}}}
\thinlines
\put(7486.000,2129.000){\arc{364.664}{2.4959}{6.9871}}
\blacken\thicklines
\path(7631.214,2150.974)(7625.000,2011.000)(7702.516,2127.715)(7631.214,2150.974)
\end{picture}
}
 \end{center}
\caption{(a) An NFA ; (b) partial \'atomaton  of .} 
\label{fig:partatom}
\end{figure}

\begin{example}
\label{ex:partial}
Consider the NFA  of Fig.~\ref{fig:partatom}~(a) recognizing a language  
over alphabet . The right language of state  is , 
which is the set of all words having an odd number of s, 
and that of state  is , which is the set of all words 
having an even number of s. It follows that  .
Each language  is a partial quotient of , 
since , and .
The NSE for  is the set of equations 

with initial set .

Next, we construct the partial atoms of . 
Since , the intersections containing  and 
 are empty. 
We also note that  and . 
The partial atoms are the non-empty intersections 
, 
, 
, and 
, and 
they obey the following equations:


If we identify  with , 
 with , 
 with , and 
 with , 
and use  as the initial set,
we obtain the partial \'atomaton
 of Fig.~\ref{fig:partatom}~(b).


From Fig.~\ref{fig:partatom}, we find that  () is the set of all words that begin with  and have an odd (even) number of s.
Also,  is the set of all words that begin with  and have an odd number of s, and a word is in  if it is empty or begins with  and has an even number of s.

Now we construct . The steps are shown in Tables~\ref{tab:N}--\ref{tab:Nrdr}, 
where initial (final) states are denoted by right (left) arrows.
Note that Table~\ref{tab:Nrdr} corresponds precisely to Fig.~\ref{fig:partatom}~(b).
\qedb
\end{example}

\begin{table}[hbt]
\begin{minipage}[b]{0.45\linewidth}
\caption{NFA .}
\label{tab:N}
{\footnotesize
\begin{center}

\end{center}
}
\end{minipage}
\hspace{.3cm}
\begin{minipage}[b]{0.45\linewidth}
\caption{NFA .}
\label{tab:Nr}
{\footnotesize
\begin{center}

\end{center}}
\end{minipage}
\end{table}

\begin{table}[hbt]
\begin{minipage}[b]{0.4\linewidth}
\caption{DFA .}
\label{tab:Nrd}
{\footnotesize
\begin{center}

\end{center}}
\end{minipage}
\hspace{.5cm}
\begin{minipage}[b]{0.4\linewidth}
\caption{NFA .}
\label{tab:Nrdr}
{\footnotesize
\begin{center}

\end{center}}
\end{minipage}
\end{table}
In summary, in this section we present the following contributions:
\bi
\item
We define partial \'atomata and study their properties.
\item
We prove that partial \'atomata are isomorphic to Sengoku's disjoint NFAs, obtained from any NFA  by finding .
\item
We prepare the ground for definitions of \'atomata in the next section.
\'Atomata are special cases of partial \'atomata and have even nicer properties than partial \'atomata.
\ei

\subsection{Partial \'Atomata}
Let  be a non-empty regular language and 
let  be any NFA accepting , with
state set .
Let   ,  be 
the  \emph{partial quotients} of .
A partial quotient  is \emph{initial} if  is an initial state of ,
it is \emph{final} if  is a final state.


A \emph{partial atom} of  is any non-empty language 
of the form 
, 
where  is either  or .
A partial atom is \emph{initial} if it has some initial partial quotient 
 as a term in its intersection, 
it is \emph{final} if and only if it contains .
Since  is non-empty,  has at least one partial quotient containing . 
Hence it has exactly one final partial atom
, where 
 if , and  otherwise.

If the intersection 
 is non-empty, then we call it 
the \emph{negative} partial atom; all the other partial atoms are \emph{positive}. 
Let the set of partial atoms be . 
Let the number of positive partial atoms be ; this number is either  or .
By convention,  is the set of initial partial atoms,
 is the final partial atom, and  is the negative partial atom, if  present.
The negative partial atom can never be final, 
since there must be at least one complemented final partial quotient in its intersection.

In the following definition we use a one-one correspondence 
 between partial atoms  and the states 
 of the NFA  defined below.

\begin{definition}
\label{def:partial_atomaton}
The \emph{partial \'atomaton} of  , is the NFA defined by 
,
 where ,
 , 
 and  if and only if 
, 
for all  and .
\end{definition}


The partial \'atomaton can be constructed directly from the NSE corresponding to , as illustrated in Example~\ref{ex:partial}.


\begin{proposition}
\label{prop:quotient_partial}
The following properties hold for partial atoms:\\
1. Partial atoms are pairwise disjoint, that is,  for all 
, .\\
2. The quotient  of  by  is a (possibly empty) union of 
partial atoms.\\
3. The quotient  of  by  is 
a (possibly empty) union of partial atoms.\\
4. Partial atoms define a partition of .
\end{proposition}
\begin{proof}
1. If , then there exists  such that 
 is a term of  and  is a term of , or \emph{vice versa}. 
Hence .\\
\hglue10pt 
2. The empty quotient, if present, is the empty union of partial atoms. 
Every non-empty quotient  is a union of some partial quotients. 
As well, every  is the union of all the  
intersections that have  as a term. 
This includes all partial atoms that have  as a term, and possibly 
some empty intersections.\\
\hglue10pt 
3. The quotient of a partial atom  by a letter  is an intersection 
of quotients of uncomplemented or complemented partial quotients of . 
Since a quotient of a partial quotient is a union of partial quotients, 
and a quotient of a complemented partial quotient is an intersection of complemented 
partial quotients, the quotient of  by  is a union of intersections of
complemented or uncomplemented partial quotients of .
If a partial quotient  does not appear as a term in some intersection  
of this union, then we ``add it in'' by using the fact that 
.
After all the missing partial quotients are so added, we obtain a union of 
partial atoms.
It  follows that  is a union of partial atoms of  for every .\\
\hglue10pt 
4. Since the union of all the intersections of complemented and uncomplemented partial atoms is , the claim follows.
\end{proof}

\begin{lemma}
\label{lem:inclusion_partial}
Let .
If  and  then , for .
\end{lemma}
\begin{proof}
Assume that  and , but suppose  for some  
. 
Then  and .
By Proposition~\ref{prop:quotient_partial}, Part 3,  is a union of 
partial atoms. So, on the one hand,  and  together imply 
. On the other hand, from  and , 
we get . So if , we have a contradiction.
Hence, . 
\end{proof}

\begin{lemma}
\label{lem:path_partial}
For , 
 if and only if 
, for .
\end{lemma}
\begin{proof}  
The proof is by induction on the length of . 
If  and , then  and 
. If  and , then , 
since partial atoms are disjoint; hence .
If , then the lemma holds by Definition~\ref{def:partial_atomaton}.  

Now, let , where  and , and assume that 
lemma holds for . 
Suppose that .
Then there exists some state  such that
  and 
.
Thus,   by the definition of partial \'atomaton, 
and  by the induction assumption, implying that
.
 
Conversely, let . Then .
Let . Then . 
Since by Proposition~\ref{prop:quotient_partial}, Part 3,  is a union 
of partial atoms, there exists a partial atom  such that .
Since ,  by Lemma~\ref{lem:inclusion_partial} we get 
.
Furthermore, because  and , we have 
. Since , then  by 
Lemma~\ref{lem:inclusion_partial}.

As the lemma holds for  and ,  implies
, and 
 implies
, showing that
.
\end{proof}

\begin{proposition}
\label{prop:right_lang_partial}
The right language of state  of partial \'atomaton  is 
the partial atom , that is, 
, for all .
\end{proposition}
\begin{proof}  
Let ; then 
.
By Lemma~\ref{lem:path_partial}, we have .  
Since ,  we have . 

Now suppose that . Then , and since ,
by Lemma~\ref{lem:inclusion_partial} we get . 
By Lemma~\ref{lem:path_partial}, 
, that is, 
. 
\end{proof}

\begin{proposition}
\label{prop:lang_partial_atomaton}
The language accepted by partial \'atomaton  of  is ,
that is, .
\end{proposition}
\begin{proof}
We have , by Proposition~\ref{prop:right_lang_partial}.
Since  is the set of all partial atoms that have some  as a term 
such that , we also have . 
\end{proof}


Next, we will show that  is isomorphic to the NFA . 
To prove this result, we use the automata 
,
, and
.
This is a generalization of the isomorphism result in~\cite{BrTa13}.

\begin{proposition}[Isomorphism]
\label{prop:isomorphism}
Let  be the mapping assigning to state 
, given by
 of , the set 
.
Then  is an NFA isomorphism between  and . 
\end{proposition}
\begin{proof}
Every initial state  of  is mapped to a subset  of , 
corresponding to the set of uncomplemented 's in , 
having a property . 
Then  is a final state of  and therefore, 
an initial state of .

The final state  of  is mapped to 
the set of all 's such that , that is, 
the set of final states of , 
which is the initial state of , 
and thus the final state of .

We also have to demonstrate that  if and 
only if 
for all  
and .

Let  be a state of .
The left language of state  consists of all words  such that 
 for every , 
but  for any .
We get that, , and so
.
Also, given a state  of  (as well as ), 
similar equations hold for . 
Then,  for some  if and only if
.
This is equivalent to having  if and only if
.
Considering above, the latter is equivalent to
.

Let  and .
Then we have that 
 
if and only if
.
By denoting  
and , 
we get that  
if and only if .
According to the definition of , the latter is equivalent to 
.
\end{proof}


\begin{corollary}
\label{cor:nfa_isomorphism}
The mapping  is a DFA isomorphism between  and 
.
\end{corollary}



\section{Quotients, Atoms and \'Atomata}
\label{sec:atoms}

\subsection{Background}
Sengoku~\cite{Sen92} studied the NFA  obtained from any NFA  by reversal, determinization, minimization and  reversal.  He called  the \emph{normal} NFA equivalent to . He defined an NFA  to be in \emph{standard form}
if  is minimal. 

Recall a (slightly modified version of a) theorem from~\cite{Brz63}:
\begin{theorem}
\label{thm:Brz}
If  an NFA  has no empty states and  is deterministic, 
then  is minimal.
\end{theorem}

Suppose instead of starting with an NFA, we start with a minimal DFA . 
Since  has no unreachable states,  has no empty states, and 
so Theorem~\ref{thm:Brz} applies to . Thus 
 is minimal, that is, , and .
Since  is minimal,  is in standard form.
The NFA  also appeared in the work of Matz and Potthoff~\cite{MaPo95}.

As in the case of disjoint NFAs discussed in the previous section, we introduce a completely different definition of an NFA (which we call an \'atomaton) defined by a given minimal DFA---or equivalently, by any regular language---and prove that that NFA is isomorphic to .


The concepts used here are special cases of those of Section~\ref{sec:partial}: 
here, instead of using an arbitrary NFA, we start with the minimal DFA of a regular language .

\subsection{\'Atomata}

Let  be the minimal DFA of , with
state set .
It is well known that the right language of every state  of  
is a quotient  of , . 


An \emph{atom} of  is any non-empty language of the form 
, 
where  is either  or . 
Let the set of atoms be . 
Thus atoms of  define a partition of , and 
 has at most  atoms.

An atom is \emph{initial} if it has  (rather than ) as a term;
it is \emph{final} if and only if it contains .
Since  is non-empty, it has at least one quotient containing . 
Hence it has exactly one final atom, the atom 
, where 
 if , and  otherwise.

If the intersection 
 is non-empty, then we call it 
the \emph{negative} atom; all the other atoms are \emph{positive}. 
Let the number of positive atoms be ; this number is either  or .
By convention,  is the set of initial atoms,  is the final atom, 
and the negative atom, if present, is .
The negative atom can never be final, 
since there must be at least one complemented final quotient in its intersection.


Evidently, the set of partial atoms of the quotient DFA  of the language  is 
the set of atoms of .
Since atoms of  are a special case of partial atoms, all the results of 
Section~\ref{sec:partial} about partial atoms hold for atoms.

\vskip1em

Let  be an NFA accepting , with 
partial quotients ,
partial atoms , and 
partial \'atomaton 
.


\begin{proposition}
\label{prop:atomsubset}
For every , where , there exists some atom , 
, such that . 
\end{proposition}
\begin{proof}
Let , where .
Let  be an atom that has a quotient  
uncomplemented if and only if there is some  
such that , and all the other quotients complemented.
We claim that .
On the one hand, from the choice of atom , it is clear that 
.
On the other hand, it has to be the case that every quotient  that is 
complemented in , is a union of some 's from the set 
, or otherwise  would be included
as an uncomplemented quotient.
Therefore, ,
implying
.
It follows that .
Thus, .
\end{proof}

\begin{proposition}
\label{prop:union}
Every atom  is a disjoint union of some s.
\end{proposition}
\begin{proof}
The set , as well 
the set  of atoms is a partition of .
By Proposition~\ref{prop:atomsubset}, every  is a subset of
some atom ; hence we conclude that every atom is a disjoint union of 
some s.  
\end{proof}


We define the \'atomaton of  as a special case of a partial \'atomaton
that uses a one-one correspondence 
 between atoms  and the states 
 of the NFA  as follows:

\begin{definition}
\label{def:atomaton}
The \emph{\'atomaton} of 
 is the NFA 
 where ,
 , 
 and  if and only if 
, for all  and 
.
\end{definition}


\begin{proposition}
\label{prop:partial_atomaton}
Suppose  is an NFA accepting  and  is its partial \'atomaton; 
then   is the \'atomaton of  if and only if  is the set of atoms.
\end{proposition}
\begin{proof}
If  is the \'atomaton of , then the set  must be the set of atoms.

Conversely, let  be the set of atoms. 
We show that in this case,  is the set of initial atoms, and 
 is the final atom.
By definition of ,  if and only if  has some term 
 such that  for some initial state  of . 
So if , then there is some 
such that  holds, implying 
that  is an initial atom.
On the other hand, if  is an initial atom, then it must have some term
 such that  and , 
implying .

Also, since ,  is the final atom. 
One can verify now that the partial \'atomaton  of 
Definition~\ref{def:partial_atomaton} is the \'atomaton of . 
\end{proof}



We illustrate the computation of the \'atomaton using quotient equations.

\begin{example}
\label{ex:atoms1}
Consider the language  of Example~\ref{ex:partial}.
It is defined by the following quotient equations:
  
We find the atoms using these quotient equations in the same way 
as we found partial atoms from the equations for partial quotients. 
Note that all intersections having  as a term are empty, 
as are those containing .
Hence there are only two atoms: 
, and
. 
Thus we find the atom equations

where  and .
By Proposition~\ref{prop:union}, every atom is a union of partial atoms. 
Indeed one verifies that  and  , 
where the  are defined in Example~\ref{ex:partial}.
\qedb
\end{example}


We now relate a number of concepts associated with regular languages:

\begin{theorem}
\label{thm:atomaton}
Let  be a regular language, let  be its minimal DFA, 
and let  be its \'atomaton. Then the following statements hold:\\
1.  is isomorphic to .\\
2. The reverse  of  is the minimal DFA of .\\
3. The DFA  is the minimal DFA of .\\
4. For any NFA  accepting ,  is isomorphic to  .\\
5.  is isomorphic to  if and only if  is bideterministic.\\
\end{theorem}
\begin{proof}
Claim 1 follows by Propositions~\ref{prop:isomorphism} and~\ref{prop:partial_atomaton}.
Claim 2 follows from Claim 1 and  Theorem~\ref{thm:Brz}.
Since  is deterministic and minimal, it has no unreachable states. 
Hence  has no empty states and Theorem~\ref{thm:Brz} applies. 
Therefore  is the minimal DFA accepting , and Claim 3 follows. 
Claim 4 holds because  is the minimal DFA of .

To prove Claim 5, first suppose that  is isomorphic to .
DFA  must be trim, because all states of \'atomaton  are non-empty.  
Since   is isomorphic to ,  itself is a trim DFA. 
By Claim~2,  is a DFA. 
Hence , and so also , are bideterministic.

Conversely, let  be a bideterministic DFA accepting . 
Since  is a trim DFA,   is minimal by Theorem~\ref{thm:Brz}.
Since  is deterministic, we get .
Thus 
is isomorphic to  by Claim~4.
On the other hand, 
since  is deterministic,  is minimal 
by Theorem~\ref{thm:Brz}. Hence  is isomorphic to .
Since  is isomorphic both to  and , 
 is isomorphic to .
\end{proof}

An NFA  isomorphic to the trim \'atomaton 
is considered in~\cite{MaPo95}. It is noted there that for each word  in  
there is a unique path in  accepting , 
and deleting any transition from  results 
in a smaller accepted language. 
It is also stated in~\cite{MaPo95} without proof that the
right language  of any state  of an NFA  accepting  is 
a subset of a union of atoms. This holds because  is a subset 
of some quotient of , and quotients are unions of atoms by 
Proposition~\ref{prop:quotient_partial}, Part 2.

Theorem~\ref{thm:atomaton} provides another method of finding the \'atomaton 
of : simply reverse the quotient DFA of .

\vskip1em
 
To end this section, we explain the differences between our present  
definition of an atom and that of~\cite{BrTa11}.
The definition  in \cite{BrTa11} did not consider 
the intersection of all the complemented quotients to be an atom, and
so all atoms were positive.
It was shown in~\cite{BrTa11} that the reverse of the \'atomaton
with only positive atoms is the trim version of the minimal DFA of . 
With the negative atom, we avoid the trimming operation;
so the reverse of the \'atomaton is the minimal DFA of . 
Also, with the negative atom, a language  and its complement language  
have the same atoms. Finally, we have symmetry between the atoms with 
0 and  complemented quotients, and the same upper bounds on quotient 
complexity for both, as was shown in \cite{BrTa13}.


\section{Atomic NFAs}
\label{sec:atomic}

\subsection{Basic Properties}
We now introduce a new class of NFAs and study their properties.

\begin{definition}
\label{def:atomic2}
An NFA  is \emph{atomic} if for every state 
, the right language  of  is a union of some atoms of . 
\end{definition}
Note that, if , then it is the union of zero atoms.

Recall that an NFA  is residual, if  is a (left) quotient of 
 for every .
Since every quotient is a union of atoms (see Proposition~\ref{prop:quotient_partial}, 
Part~2), every residual NFA is atomic. 
However, the converse is not true: there exist atomic NFAs which are not residual.
For example, the \'atomaton of a language  is atomic, but not necessarily residual, because
in a general case, atoms are different from quotients. 
Note also that every DFA with only reachable states is atomic because
the right language of every state of such DFA is some quotient. 

Let us now consider the universal automaton  of
a language . We state some basic properties of this automaton from~\cite{LoSa07}.
Let  be a factorization of .
Then \\
\hglue 10pt (1)
 
and .\\
\hglue 10pt (2)
 and 
.\\
\hglue 10pt (3) The universal automaton  accepts .


\begin{theorem}
\label{thm:univ_atomic}
Let  be any regular language. The following automata accepting  
are atomic:
\newline
1. The \'atomaton .
\newline
2. Any DFA with no unreachable states.
\newline
3. Any residual NFA. 
\newline
4. The universal automaton .
\end{theorem}
\vspace{-.3cm}
\begin{proof} 
1.  is atomic because the right language of every state of  is an atom of .\\
\hglue10pt 
2. The right language of every state of any DFA accepting  
that has no unreachable states, 
is a quotient of~. Since every quotient is a union of atoms, every such DFA 
is atomic.\\
\hglue10pt 
3. The right language of every state of any residual NFA of  
is a quotient of~, and hence a union of atoms. Thus, any residual NFA is 
atomic.\\
\hglue10pt 
4. We show that the right language of every state  of  is a union 
of atoms of . Let  be any state of .
Since by property (2) above,  holds, it is enough to show 
that  is a union of atoms. 

By (1), . 
We note that if , then  is the union of zero atoms.
We also note that if , then , and so  is the union of 
all atoms.

Let  be the quotients of . 
Then for some , .
Now , where  is either  or . 
Thus,  is a union of atoms of . 
\end{proof}


\subsection{Atomicity of States and NFAs}
\label{sec:detecting}

Let  be any NFA accepting .
We call a state  of  \emph{atomic} if its right language
 is a union of atoms of . 
We now present a method of detecting which states of an NFA
are atomic.

Consider the DFA  and the NFA ;
these two automata have the same set  of states. 
By Proposition~\ref{prop:isomorphism}, there is an isomorphism 
between the partial \'atomaton  of  and the NFA . 
Since there is a one-one correspondence between states  of 
 and partial atoms  of , we can also establish a one-one 
correspondence  between partial atoms  of  and 
states  of  as follows:

\begin{definition}
\label{def:mapping}
Let  be the mapping such that for every  and 
,  if and only if . 
\end{definition}


DFA  is not necessarily minimal. 
Let  be the sets of equivalent states 
of ; that is, every  is an equivalence class of 
the states of . 
The following proposition holds:

\begin{proposition}
\label{prop:atom_equiv}
Let  be a set of partial atoms of , and 
let  be the corresponding set of states of 
according to mapping .
An equality  holds for some atom  if and only if
 is equal to some equivalence class .  
\end{proposition}
\begin{proof}
Let  be a set of partial atoms of , and 
let  be 
the corresponding set of states of .

Consider the minimal DFA  of .  
It is well known that this DFA can be obtained 
by ``merging'' the states of each set  of the states of the DFA
, into a state  of .
Similarly, , the reverse NFA of the minimal DFA of ,
can be obtained by merging the corresponding states of , 
or equivalently, its isomorphic partial \'atomaton .
Since by Proposition~\ref{prop:right_lang_partial},
the right language of every state of  is some partial atom, 
the right language of the state  of  is
the union of partial atoms  of  such that their corresponding states
 belong to . 

On the other hand, according to Theorem~\ref{thm:atomaton}, Part 4, 
the NFA  is isomorphic to the \'atomaton of , and 
by Proposition~\ref{prop:right_lang_partial},
the right language of every state of the \'atomaton is some atom.
We conclude that the union of partial atoms  such that 
 belong to , is some atom . 
Since partial atoms are disjoint, no other union of partial atoms
of  can be equal to . Thus, the claim of the proposition holds.
\end{proof}


We use the equivalence classes  of the states of 
the DFA  to detect which states of  are atomic.

\begin{theorem}
\label{thm:atomic_state}
A state  of an NFA  is atomic if and only if 
the subset  of states of 
 is a union of some equivalence classes of .
\end{theorem}
\begin{proof}
Consider a state  of an NFA  with right language .
Let  be the set of partial atoms  of  
that have  uncomplemented in the intersection representing , and
let  be the corresponding set of states of the partial \'atomaton 
 of .
Let  be the set of states of  that are assigned to
the states in  by the mapping  of 
Proposition~\ref{prop:isomorphism}. Clearly,  consists of
exactly those states  of  such that . 

Now suppose that  is a union of atoms.
Since  is the set of partial atoms of  with  uncomplemented, 
 is equal to the union of all partial atoms in . 
So the union of all partial atoms in  is a union of atoms. 
By Definition~\ref{def:mapping}, partial atoms in  are 
mapped by  exactly to the states in . 
By Proposition~\ref{prop:atom_equiv},
 is a union of some equivalence classes of . 

Conversely, if  is not a union of atoms, then the union of 
partial atoms in  is not a union of atoms either. 
Contrarily to the case above, the set  cannot be a union of 
any equivalence classes of . 
\end{proof}

\begin{example}
\label{ex:atomic_state}
Consider the NFA  of Table~\ref{tab:N} and the DFA  of Table~\ref{tab:Nrd}. 
The equivalence classes of the states of  are 
 and . 
Since  appears in both states of  and does not appear in the states of , 
state  of  is atomic.
However,  appears in the set , which is not a union of 
equivalence classes; hence state  of  is not atomic. Similarly, state  
is not atomic.
\qedb
\end{example}

The following result is a consequence of Theorem~\ref{thm:atomic_state}:
\begin{corollary}
\label{cor:atomic}
An NFA  is atomic if and only if  is minimal. 
\end{corollary}
\begin{proof}
If  is minimal, the equivalence classes of its states  are singletons. 
So the set of states of   in which a state  of  appears is a union of equivalence classes.
By Theorem~\ref{thm:atomic_state},
every state is atomic, and  so is .

Conversely, suppose  is atomic, but  is not minimal.
Then there are two states  and  of   which are equivalent.
Without loss of generality, suppose that ; 
then the set of states in which  appears cannot be a union of equivalence classes. 
By Theorem~\ref{thm:atomic_state} again,  is not atomic, and neither is .
\end{proof}

We also have the following corollary:

\begin{corollary}
\label{cor:atomic2}
An NFA  is atomic if and only if the partial atoms of  are the atoms of . 
\end{corollary}
\begin{proof}
By Corollary~\ref{cor:atomic}, an NFA  is atomic if and only if 
 is minimal. 
But, if  is minimal, then  and 
.
By Theorem~\ref{thm:atomaton}, Part 4, 
 is isomorphic to the \'atomaton  of . 
Since by Proposition~\ref{prop:isomorphism}, the partial \'atomaton  
of  is isomorphic to ,  and  
are isomorphic. According to Proposition~\ref{prop:partial_atomaton},
this means that the partial atoms of  are the atoms of . 
\end{proof}


\begin{example}
\label{ex:atomicity}
All three possibilities for the atomic nature of  and  exist:
 of Table~\ref{tab:Na} and its reverse are not atomic.
 of Table~\ref{tab:Nb} is atomic, but its reverse is not.
 of Table~\ref{tab:Nc} and its reverse are both atomic.
Note that all three of these NFAs accept , where .
\qedb
\begin{table}[b]
\begin{minipage}[b]{0.3\linewidth}
\caption{.}
\label{tab:Na}
{\footnotesize
\begin{center}

\end{center}}
\end{minipage}
\hspace{0.3cm}
\begin{minipage}[b]{0.3\linewidth}
\caption{.}
\label{tab:Nb}
{\footnotesize
\begin{center}

\end{center}}
\end{minipage}
\hspace{0.55cm}
\begin{minipage}[b]{0.3\linewidth}
\caption{.}
\label{tab:Nc}
{\footnotesize
\begin{center}

\end{center}}
\end{minipage}

\end{table}

\end{example}

\subsection{Extension of Brzozowski's Theorem on DFA Minimization}
\label{sec:extension}

Theorem~\ref{thm:Brz} is the basis for Brzozowski's ``double-reversal'' 
minimization algorithm~\cite{Brz63}:
Given any DFA , reverse it to get , determinize 
 to get , reverse  to get 
, and then determinize  to get 
. This last DFA is guaranteed to be minimal 
by Theorem~\ref{thm:Brz}, since  is deterministic and
 has no empty states.
Hence  is the minimal DFA equivalent to .


Since this conceptually very simple algorithm carries out two 
determinizations, its complexity is exponential in the number of states 
of the original automaton in the worst case. But its 
performance is good in practice, often better than Hopcroft's 
algorithm \cite{TaV05,Wat95}.
Furthermore, this algorithm applied to an NFA still yields an equivalent 
minimal DFA; see~\cite{Wat95}, for example.



As a consequence  of Corollary~\ref{cor:atomic}, we can now generalize 
Theorem~\ref{thm:Brz}:

\begin{theorem}
\label{thm:extension}
For any NFA ,   is minimal if and only if  is atomic. 
\end{theorem}

\begin{corollary}
If  is a non-minimal  DFA, then  is not atomic. 
\end{corollary}



\section{Reduced Atomic NFAs of a Given Regular Language}
\label{sec:reduced}

The following properties of reduced atomic NFAs were proved in~\cite{BrTa13}. 
A similar approach was used more informally by Sengoku~\cite{Sen92}.

\begin{theorem}[Legality]
\label{thm:unions}
Suppose  is a regular language,  its \'atomaton is
, and
 is a trim NFA, where 
 is a collection of 
sets of positive atom symbols and .
If , define 
 to be the set of atom symbols 
appearing in the sets   of . 
Then  is a reduced atomic NFA of  if and only if it satisfies the following
conditions:
\be
\item
\label{cond:in}
.
\item
\label{cond:trans}
For all , .
\item
\label{cond:out}
For all , we have  
if and only if .
\ee
\end{theorem}


\subsection{Enumerating Reduced Atomic NFAs}
If we allow equivalent states, there is an infinite number of atomic NFAs 
accepting a given regular language, but their behaviours are not all distinct.
Hence we consider only reduced atomic NFAs.
The number of trim reduced atomic NFAs can be very large. 
There can be such NFAs with as many as  non-empty states, 
since there are that many non-empty sets of positive atoms. 

From now on, we drop the curly brackets and commas when representing 
sets of states or sets of atoms in tables.
For example,  stands for , and 
 is used instead of .

\begin{example}
\label{ex:reducedatomic}
The DFA of Table~\ref{tab:dkw} was used in~\cite{KaWe70}. 
It accepts the language , where . 
The quotients of  are
, 
, and
.
NFA  and the isomorphic trim \'atomaton  with states renamed  are shown in Tables~\ref{tab:drdrkw} and~\ref{tab:akw}.
The positive atoms are
,  and , and
, 
,
and .


\begin{table}[b]
\begin{minipage}[b]{0.25\linewidth}
\caption{.}
\label{tab:dkw}
{\footnotesize
\begin{center}

\end{center}}
\end{minipage}
\hspace{0.2cm}
\begin{minipage}[b]{0.3\linewidth}
\caption{.}
\label{tab:drdrkw}
{\footnotesize
\begin{center}

\end{center}}
\end{minipage}
\hspace{1cm}
\begin{minipage}[b]{0.25\linewidth}
\caption{.}
\label{tab:akw}
{\footnotesize
\begin{center}

\end{center}}
\end{minipage}
\end{table}


\begin{table}[hbt]
\begin{minipage}[b]{0.45\linewidth}
\caption{NFA .}
\label{tab:b1}
{\footnotesize
\begin{center}

\end{center}}
\end{minipage}
\hspace{0.4cm}
\begin{minipage}[b]{0.45\linewidth}
\caption{NFA .}
\label{tab:b2}
{\footnotesize
\begin{center}

\end{center}}
\end{minipage}
\end{table}




\begin{table}[t]
\begin{minipage}[b]{0.45\linewidth}
\caption{A 5-state NFA.}
\label{tab:b3}
{\footnotesize
\begin{center}

\end{center}}
\end{minipage}
\hspace{.4cm}
\begin{minipage}[b]{0.45\linewidth}
\caption{A 7-state NFA.}
\label{tab:b4}
{\footnotesize
\begin{center}

\end{center}}
\end{minipage}
\end{table}

Since  is not of the form , where , 
no 1-state NFA exists for .
\be
\item 
For the initial state we could pick state  with two atoms.  From there, the \'atomaton reaches 
 under , and   under . 
        \be
        \item
If we pick 
as the second state,  we can cover  by  and 
, as  in Table~\ref{tab:b1}. This minimal
atomic NFA turns out to be unique; it is also minimal among all NFAs.
        \item
        We can use  as a state and 
        for the transition under . This gives an NFA  isomorphic to the DFA of Table~\ref{tab:dkw}.
        \item
        We can use state 
        as shown in Table~\ref{tab:b2}.
        \ee
\item
We can pick two initial states,  and . 
        \be
        \item
        If we add , this leads to the  \'atomaton of Table~\ref{tab:akw}.
        \item
        A 5-state solution is shown in Table~\ref{tab:b3}.
        \ee
\item
We can use three initial states, ,  and . 
        A 7-state NFA is shown in  Table~\ref{tab:b4}. This 
           is a largest possible reduced solution.\qedb
\ee

\end{example}




The number of minimal atomic NFAs can also be very large. 
\begin{example}
\label{ex:atomicminimal}
Let  and consider the language .
The quotients of  are ,  and .
The quotient DFA of  is shown in Table~\ref{tab:d}, and its \'atomaton, in Tables~\ref{tab:a} and~\ref{tab:a_relabel} (where the atoms have been relabelled). 
The atoms  are ,  and , and there is no negative atom.
Thus the quotients are , , and .

We find all the minimal atomic NFAs of .
Obviously, there is no 1-state solution.
The states of any atomic NFA are sets of atoms, and 
there are seven non-empty sets of atoms to choose from. 
Since there is only one initial atom, there is no choice: we must take .
For the transition , we can add  or . 
If there are only two states, atom  cannot be reached. So there is no  2-state atomic NFA.
The results for 3-state atomic NFAs  are summarized in Proposition~\ref{prop:281}. 
\qedb
\end{example}


\begin{table}[hbt]
\begin{minipage}[b]{0.2\linewidth}
\caption{DFA .}
\label{tab:d}
{\footnotesize
\begin{center}

\end{center}}
\end{minipage}
\hspace{1cm}
\begin{minipage}[b]{0.32\linewidth}
\caption{\'Atomaton .}
\label{tab:a}
{\footnotesize
\begin{center}

\end{center}}
\end{minipage}
\hspace{0.35cm}
\begin{minipage}[b]{0.35\linewidth}
\caption{ relabelled.}
\label{tab:a_relabel}
{\footnotesize
\begin{center}

\end{center}}
\end{minipage}
\end{table}


\begin{proposition} 
\label{prop:281}
The language  has 281 minimal atomic NFAs. 
\end{proposition}
\begin{proof}

The only initial state of the \'atomaton  corresponds to atom , 
so  must be included.
To implement the transition
 from ,
either  or  must be chosen. 
\be
\item
If  is chosen, then there must be a set containing  but not ; otherwise 
the transition 
  cannot be realized.
        \be
        \item
        If  is taken, then  must be taken, and this makes four states.
        \item
         Hence  must be chosen, yielding the \'atomaton .
         \ee
\item
If  is chosen, then we could choose ,  or , 
since  would also require . Thus there are three cases:
        \be
        \item
         yields  of Table~\ref{tab:fn1}, if the minimal number of 
        transitions is used. 
        The following transitions can also be added: 
        , , .
        Since these can be added independently, we have eight more NFAs. 
        Using the maximal number of transitions, we get  of Table~\ref{tab:fn9}.
        \item
         results in  with the minimal number 
          of transitions, and  with the maximal one.
        \item
         results in  (the quotient DFA) 
          with the minimal number of transitions, and   with the maximal one.
        \ee
\ee

\begin{table}[h]
\begin{minipage}[b]{0.45\linewidth}
\caption{NFA .}
\label{tab:fn1}
{\footnotesize
\begin{center}

\end{center}}
\end{minipage}
\hspace{0.1cm}
\begin{minipage}[b]{0.45\linewidth}
\caption{NFA .}
\label{tab:fn9}
{\footnotesize
\begin{center}

\end{center}}
\end{minipage}
\end{table}


\begin{table}[hbt]
\begin{minipage}[b]{0.45\linewidth}
\caption{NFA .}
\label{tab:fn10}
{\footnotesize
\begin{center}

\end{center}}
\end{minipage}
\hspace{0.2cm}
\begin{minipage}[b]{0.45\linewidth}
\caption{NFA .}
\label{tab:fn25}
{\footnotesize
\begin{center}

\end{center}}
\end{minipage}
\end{table}

\begin{table}[h]
\begin{minipage}[b]{0.45\linewidth}
\caption{NFA .}
\label{tab:fn26}
{\footnotesize
\begin{center}

\end{center}}
\end{minipage}
\hspace{0.2cm}
\begin{minipage}[b]{0.45\linewidth}
\caption{NFA .}
\label{tab:fn281}
{\footnotesize
\begin{center}

\end{center}}
\end{minipage}
\end{table}


Also  has 3-state non-atomic NFAs.
The determinized version of NFA  of Table~\ref{tab:fn10} is not minimal.
By Theorem~\ref{thm:extension},  is not atomic. But ;
hence we obtain a non-atomic 3-state NFA for  by reversing  and interchanging  and . There are other non-atomic 3-state solutions.
\end{proof}

One can verify that there is no NFA with fewer than 3 states which
accepts the language .
This implies that every minimal atomic NFA of  is also 
a minimal NFA of .
However, this is not the case with all regular languages, as we will see next. 



\subsection{Atomic Minimal NFAs}
 
Recall that Sengoku 
defines an NFA  to be in \emph{standard form}~\cite{Sen92}(p.~19) 
if  is minimal.
By our Corollary~\ref{cor:atomic}, such an  is atomic.
Sengoku makes the following claim~\cite{Sen92}(p.~20):
\begin{quote}
\vskip-0.1cm
\emph{We can transform the nondeterministic automaton into its standard form 
by adding some extra transitions to the automaton. Therefore the number of 
states is unchangeable.}
\end{quote}
\vskip-0.1cm
This claim amounts to stating that any NFA can be transformed to an equivalent  
atomic NFA by adding some transitions. Unfortunately, it is false:
\begin{theorem}
\label{thm:Sengoku}
There exists a language for which no minimal NFA is atomic.
\end{theorem}
\begin{proof}
\vskip-0.1cm
The regular language  accepted by DFA  of Table~\ref{tab:d_mp}  
is the same as that of an NFA considered in~\cite{MaPo95}(p.~80, Sect.~3). 
NFA  and its isomorphic \'atomaton  
with relabelled states are in Tables~\ref{tab:drdr_mp} and~\ref{tab:a_mp}, 
respectively (there is no negative atom). 

Recall that a ``fooling set'' for a regular language  is a set  such that  for all , and either  or  for all . It is known that every NFA of  needs at least  states, if it has a fooling set of cardinality ~\cite{Bir92}.
One verifies that  is a fooling set for . 
Hence every NFA for  requires at least four states.

A minimal NFA  of  having four states is shown in 
Table~\ref{tab:n_mp}; it is not atomic and it is not unique. 
We try to construct a 4-state atomic NFA  equivalent to . 
\begin{table}[hbt]
\begin{minipage}[b]{0.19\linewidth}
\caption{.}
\label{tab:d_mp}
{\footnotesize
\begin{center}

\end{center}}
\end{minipage}
\hspace{0.03cm}
\begin{minipage}[b]{0.38\linewidth}
\caption{.}
\label{tab:drdr_mp}
{\footnotesize
\begin{center}

\end{center}}
\end{minipage}
\hspace{0.83cm}
\begin{minipage}[b]{0.32\linewidth}
\caption{ .}
\label{tab:a_mp}
{\footnotesize
\begin{center}

\end{center}}
\end{minipage}
\end{table}
First, we note that quotients corresponding to the states of  can be expressed 
as sets of atoms as follows:
, , , ,
, , , , and
. One can verify that these are the states of the determinized 
version of the \'atomaton, which is isomorphic to the original DFA . 
Now, every state of  must be a subset of a set of atoms of some quotient, 
and all these sets of atoms of quotients must be covered by the states of .
We note that quotients , , and 
do not contain any other quotients as subsets, while all the other quotients do.
It is easy to see that there is no combination of three or fewer sets of atoms, 
other than these three sets, that can cover these quotients. 
Since our aim is to find a four-state atomic NFA, and because we also need a set 
containing the atom , we have to use these three sets as states of . 
To use only one set of atoms with , that set has to be
a subset of every quotient having . So it must be
a subset of . If we use  as a state, then by the transition 
table of the \'atomaton, there must be at least one more state to cover 
. Similarly, if we use , then we must have another state to cover 
. If we use , then we must have a state to cover 
. And if we use , then we must have a state to cover 
. We conclude that a smallest atomic NFA has at least five states.
There is a five-state atomic NFA, as 
shown in Table~\ref{tab:n5_mp}. It is not unique. 

Since there does not exist a four-state atomic NFA equivalent to the DFA ,
it is not possible to convert the non-atomic 
minimal NFA  to an atomic NFA by adding transitions.
\end{proof}

\begin{table}[t]
\begin{minipage}[b]{0.3\linewidth}
\caption{NFA .}
\label{tab:n_mp}
{\footnotesize
\begin{center}

\end{center}}
\end{minipage}
\hspace{0.5cm}
\begin{minipage}[b]{0.45\linewidth}
\caption{.}
\label{tab:n5_mp}
{\footnotesize
\begin{center}

\end{center}}
\end{minipage}
\vskip-0.3cm
\end{table}

\vskip-0.1cm
In summary, Sengoku's method cannot always find the minimal NFAs, but 
 it is able to find all atomic minimal NFAs.
His minimization algorithm proceeds by 
``merging some states of the normal nondeterministic automaton''.
This is similar to our search for subsets of atoms that satisfy 
Theorem~\ref{thm:unions}.

\section{Conclusions}
\label{sec:conc}
For any NFA , we introduced a natural set of languages, the partial atoms, and constructed a new NFA, which we proved to be isomorphic to ---an NFA studied by Sengoku.
For any regular language , we introduced a natural set of languages, the atoms of ; we then constructed an NFA , the \'atomaton of , which we proved to be isomorphic to the NFA , also studied by Sengoku.
 We  introduced atomic automata, and generalized 
Brzozowski's method of minimization of DFAs by double reversal.
We studied atomic NFAs associated with a given regular language and,
contrarily to Sengoku's claim, proved that not every language has an atomic 
minimal NFA.

For completeness we mention that the quotient complexity (equivalent to 
state complexity) of atoms of regular languages was studied in~\cite{BrTa13} and~\cite{BrDa13}. 

\begin{thebibliography}{16}
\expandafter\ifx\csname natexlab\endcsname\relax\def\natexlab#1{#1}\fi
\providecommand{\bibinfo}[2]{#2}
\ifx\xfnm\relax \def\xfnm[#1]{\unskip,\space#1}\fi
\bibitem[{Arnold et~al.(1992)Arnold, Dicky and Nivat}]{ADN92}
\bibinfo{author}{A.~Arnold}, \bibinfo{author}{A.~Dicky},
  \bibinfo{author}{M.~Nivat}, \bibinfo{title}{A note about minimal
  non-deterministic automata}, \bibinfo{journal}{Bull. EATCS}
  \bibinfo{volume}{47} (\bibinfo{year}{1992}) \bibinfo{pages}{166--169}.
\bibitem[{Birget(1992)}]{Bir92}
\bibinfo{author}{J.C. Birget}, \bibinfo{title}{Intersection and union of
  regular languages and state complexity}, \bibinfo{journal}{Inform. Process.
  Lett.} \bibinfo{volume}{43} (\bibinfo{year}{1992}) \bibinfo{pages}{185--190}.
\bibitem[{Brzozowski(1963)}]{Brz63}
\bibinfo{author}{J.~Brzozowski}, \bibinfo{title}{Canonical regular expressions
  and minimal state graphs for definite events}, in:
  \bibinfo{booktitle}{Proceedings of the Symposium on Mathematical Theory of
  Automata}, volume~\bibinfo{volume}{12} of \textit{\bibinfo{series}{MRI
  Symposia Series}}, \bibinfo{publisher}{Polytechnic Institute of Brooklyn,
  N.Y.}, \bibinfo{year}{1963}, pp. \bibinfo{pages}{529--561}.
\bibitem[{Brzozowski and Davies(2013)}]{BrDa13}
\bibinfo{author}{J.~Brzozowski}, \bibinfo{author}{G.~Davies},
  \bibinfo{title}{Maximal syntactic complexity of regular languages implies
  maximal quotient complexities of atoms.}, \bibinfo{year}{2013}.
  \bibinfo{note}{{\tt http://arxiv.org/abs/1302.3906}}.
\bibitem[{Brzozowski and Tamm(2011)}]{BrTa11}
\bibinfo{author}{J.~Brzozowski}, \bibinfo{author}{H.~Tamm},
  \bibinfo{title}{Theory of \'atomata}, in: \bibinfo{editor}{G.~Mauri},
  \bibinfo{editor}{A.~Leporati} (Eds.), \bibinfo{booktitle}{Proceedings of the
  15th International Conference on Developments in Language Theory
  DLT\/}, volume \bibinfo{volume}{6795} of
  \textit{\bibinfo{series}{Lecture Notes in Computer Science}},
  \bibinfo{publisher}{Springer}, \bibinfo{year}{2011}, pp.
  \bibinfo{pages}{105--116}.
\bibitem[{Brzozowski and Tamm(2013)}]{BrTa13}
\bibinfo{author}{J.~Brzozowski}, \bibinfo{author}{H.~Tamm},
  \bibinfo{title}{Complexity of atoms of regular languages},
  \bibinfo{journal}{Int. J. Found. Comput. Sc.}  (\bibinfo{year}{2013}).
  \bibinfo{note}{To appear}.
\bibitem[{Carrez(1970)}]{Car70}
\bibinfo{author}{C.~Carrez}, \bibinfo{title}{On the minimalization of
  non-deterministic automaton}, \bibinfo{type}{Technical Report}, Lille
  University, \bibinfo{address}{Lille, France}, \bibinfo{year}{1970}.
\bibitem[{Conway(1971)}]{Con71}
\bibinfo{author}{J.~Conway}, \bibinfo{title}{Regular Algebra and Finite
  Machines}, \bibinfo{publisher}{Chapman and Hall}, \bibinfo{address}{London},
  \bibinfo{year}{1971}.
\bibitem[{Denis et~al.(2002)Denis, Lemay and Terlutte}]{DLT02}
\bibinfo{author}{F.~Denis}, \bibinfo{author}{A.~Lemay},
  \bibinfo{author}{A.~Terlutte}, \bibinfo{title}{Residual finite state
  automata}, \bibinfo{journal}{Fund. Inform.} \bibinfo{volume}{51}
  (\bibinfo{year}{2002}) \bibinfo{pages}{339--368}.
\bibitem[{Kameda and Weiner(1970)}]{KaWe70}
\bibinfo{author}{T.~Kameda}, \bibinfo{author}{P.~Weiner}, \bibinfo{title}{On
  the state minimization of nondeterministic automata}, \bibinfo{journal}{IEEE
  Trans. Comput.} \bibinfo{volume}{C-19} (\bibinfo{year}{1970})
  \bibinfo{pages}{617--627}.
\bibitem[{Lombardy and Sakarovitch(2008)}]{LoSa07}
\bibinfo{author}{S.~Lombardy}, \bibinfo{author}{J.~Sakarovitch},
  \bibinfo{title}{The universal automaton}, in: \bibinfo{editor}{J.~Flum},
  \bibinfo{editor}{E.~Gr\"adel}, \bibinfo{editor}{T.~Wilke} (Eds.),
  \bibinfo{booktitle}{Logic and Automata: History and Perspectives},
  volume~\bibinfo{volume}{2} of \textit{\bibinfo{series}{Texts in Logic and
  Games}}, \bibinfo{publisher}{Amsterdam University Press},
  \bibinfo{year}{2008}, pp. \bibinfo{pages}{457--504}.
\bibitem[{Matz and Potthoff(1995)}]{MaPo95}
\bibinfo{author}{O.~Matz}, \bibinfo{author}{A.~Potthoff},
  \bibinfo{title}{Computing small finite nondeterministic automata}, in:
  \bibinfo{editor}{U.H. Engberg}, \bibinfo{editor}{K.G. Larsen},
  \bibinfo{editor}{A.~Skou} (Eds.), \bibinfo{booktitle}{Proc. of the Workshop
  on Tools and Algorithms for Construction and Analysis of Systems}, BRICS
  Note, \bibinfo{publisher}{BRICS}, \bibinfo{address}{Aarhus, Denmark},
  \bibinfo{year}{1995}, pp. \bibinfo{pages}{74--88}.
\bibitem[{Rabin and Scott(1959)}]{RaSc59}
\bibinfo{author}{M.~Rabin}, \bibinfo{author}{D.~Scott}, \bibinfo{title}{Finite
  automata and their decision problems}, \bibinfo{journal}{IBM J. Res.\ and
  Dev.} \bibinfo{volume}{3} (\bibinfo{year}{1959}) \bibinfo{pages}{114--129}.
\bibitem[{Sengoku(1992)}]{Sen92}
\bibinfo{author}{H.~Sengoku}, \bibinfo{title}{Minimization of nondeterministic
  finite automata}, Master's thesis, Kyoto University,
  \bibinfo{address}{Department of Information Science, Kyoto University, Kyoto,
  Japan}, \bibinfo{year}{1992}.
\bibitem[{Tabakov and Vardi(2005)}]{TaV05}
\bibinfo{author}{D.~Tabakov}, \bibinfo{author}{M.~Vardi},
  \bibinfo{title}{Experimental evaluation of classical automata constructions},
  in: \bibinfo{booktitle}{Proceedings of the 12th International Conference on
  Logic for Programming, Artificial Intelligence, and Reasoning LPAR\/},
  volume \bibinfo{volume}{3835} of \textit{\bibinfo{series}{LNAI}},
  \bibinfo{publisher}{Springer}, \bibinfo{year}{2005}, pp.
  \bibinfo{pages}{396--411}.
\bibitem[{Watson(1995)}]{Wat95}
\bibinfo{author}{B.W. Watson}, \bibinfo{title}{Taxonomies and toolkits of
  regular language algorithms}, Ph.D. thesis, Faculty of Mathematics and
  Computing Science, Eindhoven University of Technology, Eindhoven, The
  Netherlands, \bibinfo{year}{1995}.

\end{thebibliography}

\end{document}
