\documentclass[12pt]{article}
\usepackage{Tillpack}

\addtolength{\textheight}{30mm}
\addtolength{\topmargin}{-10mm}
\addtolength{\textwidth}{15mm}

\title{Trees in simple Polygons}
\author{ Tillmann Miltzow\footnote{Institute of Computer Science, Freie Universit\"at Berlin, Germany. \texttt{t.miltzow@gmail.com}}} 


\date{}


\begin{document}
\maketitle


\begin{abstract} 
	We prove that every simple polygon contains a degree  tree encompassing a prescribed set of vertices.
	We give tight bounds on the minimal number of degree  vertices. We apply this result to reprove a result from Bose~\emph{et al.}~\cite{DBLP:journals/dcg/BoseHT01} that every set of disjoint line segments in the plane admits a binary tree.
\end{abstract}

\paragraph{Introduction}
Recently many papers have been published regarding the augmentation of discrete geometric objects, in particular a set of non-crossing segments in the plane. While these problems are studied, not just many open problems could be solved but also new tools developed. Among these tools is the simple and easy to prove lemma \emph{about matchings in polygons} by Manuel Abellanas, Alfredo Garc\'{\i}a Olaverri, Ferran Hurtado, Javier Tejel and Jorge Urrutia~\cite{DBLP:journals/comgeo/AbellanasOHTU08}. It gives sufficient conditions when a predefined set of vertices of a polygon can be \emph{geometrically} matched \emph{within} this polygon. 
It was recently used to show that the bichromatic compatible matching graph is connected~\cite{2012arXiv1207.2375A}.

The main motivation of this note is to find a useful variation of this auspicious lemma about matchings in polygons. Trees with low maximal degree seemed curious. We show the existence of a max degree  tree spanning a prescribed set of vertices \emph{in} a given polygon(to be made precise soon). We give explicit tight bounds on the number of degree  vertices in the attained tree. To the best of our knowledge this question has not been studied before.

As an application we give a new easy proof to an old result from Prosenjit Bose, Michael E. Houle and Godfried T. Toussaint~\cite{DBLP:journals/dcg/BoseHT01} that every set of non-crossing line segments in the plane admits a binary encompassing geometric tree.
Later Michael Hoffmann, Bettina Speckmann and Csaba D. T\'{o}th gave an alternative proof~\cite{Hoffmann201035}. A comparison of the their proof and ours reveals similarities, which will be explained later. 

It is important to note that we assume general position throughout this note in the sense that no three segment endpoints are on a common line. Theorem \ref{thm:main} also holds without this assumption.


\paragraph{Main Result}
A PSLG (planar straight line graph) is a planar non-crossing embedding of a graph with straight edges. 
A simple polygon is a PSLG which is a cycle. We say that a PSLG  is \emph{inscribed in} a simple polygon  if   is contained in the closed bounded region of  and for every edge  of  either:  is in the interior of  or
  is also an edge of .

We say a vertex of a simple polygon is \emph{reflex} if the interior angle to its incident edges is larger than  and \emph{convex} otherwise. 
	\begin{figure}[h]
			\begin{center}
			\includegraphics[width = 0.4\textwidth]{convexCase}
				\caption{The Theorem is trivial in the convex case.}
				\label{fig:convexCase}
			\end{center}
	\end{figure} 
\begin{theorem}\label{thm:main}
	Let  be a simple polygon,  the set of reflex vertices of  and  some superset of vertices of . Further we can specify two vertices  such that the following holds:
	There exists a PSLG  with vertex set  inscribed in , which is a tree. The tree  satisfies the following degree restrictions:

	For every natural number  exists a polygon with  reflex vertices
	such that any tree with the properties above has  reflex vertices.
\end{theorem}
\begin{proof}
	The proof goes by induction on the number of reflex vertices of .
	For the induction basis assume  is convex. 
	We aim to construct a path from  to  traversing
	the vertices of . Consider the only two paths  and  on  connecting 
	 and . One possible path from 
	 to  traverses at first all vertices of  on  in the order 
	they appear on  and traverses thereafter the remaining vertices on ,
	before it reaches .
	Clearly,  and all other vertices have degree , see Figure \ref{fig:convexCase}. 	It is an instructive exercise to count the number of such paths in a given convex polygon.
	\begin{figure}[h]
			\begin{center}
			\includegraphics[width = \textwidth]{inductionStep}
				\caption{Either both marked vertices are in  or in .}
				\label{fig:inductionStep}
			\end{center}
	\end{figure} 
	
	For the induction step assume that  has some reflex vertices and let  be such a reflex vertex of , as in Figure \ref{fig:inductionStep}. We shoot a ray from  into the interior of  in such a way that both angles at  become convex. We denote by  the first point on the boundary of   hit by the ray. We further assume that  is not a vertex of . The segment  splits  into two simple polygons  and . The vertices  and  are convex in  and . Both  and  have fewer reflex vertices than . We denote the vertices of  in  with . We list the vertex  in  and . We distinguish two cases. 
	
	\textbf{Case 1} Both  and  specified in the Lemma are in the same polygon, say . Then we apply the induction hypothesis on  with the  and  and  as specified vertices. We receive a PSLG  with the above properties. In particular  has degree  in . We also apply the induction hypothesis on  and  with  and any other vertex of  specified. We receive a second tree  with the above properties and  has degree  in . The union  has all the above properties and in particular  has degree at most .
	
	\textbf{Case 2} The specified vertices  and  are in different polygons  and  respectively. We apply the induction hypothesis on  with  and   and  as specified vertices. We receive a PSLG  with all the above properties and the degree of  in  is exactly one. Similarly there exists a PSLG  in  such that  has degree  in  the union   satisfies all the properties in the Lemma and the degree of  is exactly .
				\begin{figure}[h]
			\begin{center}
			\includegraphics[width = \textwidth]{niceExample}
				\caption{a) Is the drawing of a regular polygon on  vertices. b) Is the transformed vertices with  spikes added. The shaded region indicates the visibility region of the convex vertex of the spikes. c) The only tree that satisfies the conditions of Theorem \ref{thm:main} is drawn with fat dashed lines. All reflex vertices must have degree .}
				\label{fig:tightExample}
			\end{center}
	\end{figure} 
	It remains to give an example of a polygon with  reflex vertices such that at any reflex vertex we must have degree . We start the construction with a regular polygon on  vertices. We add a spike to the vertices , as shown in Figure~\ref{fig:tightExample}~b).
	This creates new reflex vertices  and new convex vertices  and destroys the corresponding original vertices of the polygon. For any new convex vertex  only the vertices  and  are visible, by construction. We define  formally as 
	 We choose   and  as the marked vertices  and . It is clear, that in any encompassing PSLG as in Theorem \ref{thm:main} the vertices at the spikes must have degree  as well as  and . Since the sum of the degrees is constant, exactly the reflex vertices must have degree .
\end{proof}

The just proven Theorem needed the marked vertices  and  mainly for the induction to work. One could formulate the Theorem without mentioning them and only consider them in the proof, because they strengthen the result only slightly. 
To do so consider the case that  contains at least  convex vertices and mark them as  and . In the other case consider
vertices with smallest and largest -coordinate.  They clearly are convex. Mark them as  and . They can be remove later from the tree, as they are leaves.



\paragraph{Application}In this paragraph we want to prove the existence of a binary tree spanning a set of disjoint line segments in the plane.
It was proven first by Bose~\emph{et al.}~\cite{DBLP:journals/dcg/BoseHT01}. Later Hoffmann~\emph{et al.} gave a simpler alternative proof~\cite{Hoffmann201035}. We will give a third proof, as an application of Theorem \ref{thm:main}. Our proof resembles the proof in \cite{Hoffmann201035}. Roughly speaking Hoffmann~\emph{et al.} use a convex subdivision and the \emph{tunnel graph} to assign the segment endpoints to neighboring cells in a clever way. We only construct one polygon, apply Theorem \ref{thm:main} and delete some superfluous edges. Implicitly we also subdivide the plane into convex regions, but our assignment to the regions is more crude. That is why we have an additional clean up step in the end.

A PSLG is an \emph{encompassing tree} of a set  of line segments in the plane if it is a tree as a graph and every line segment of  is an edge of the tree.
\begin{theorem}[\cite{DBLP:journals/dcg/BoseHT01}]
	Let  be a set of  disjoint line segments in the plane with no  segment endpoints on a common line. Then there exists 
a max degree- planar encompassing tree of . 
\end{theorem}
\begin{proof}
	The idea of the proof is to construct a simple polygon out of the set of line segments, apply Theorem \ref{thm:main} and make some minor changes in order to  attain the desired tree, see Figure~\ref{fig:overview}.
	\begin{figure}[h]
			\begin{center}
			\includegraphics[width = \textwidth]{overview}
				\caption{a) Line segments are surrounded by a bounding box. b) The polygon  is attained after the segment extensions, note that  can only be extended after . c) The simplified polygon is colored grey. d) The tree attained by Theorem \ref{thm:main} is depicted. 
				e) The corresponding tree of the line segments.
				 f) Adding the original line segments and removing superfluous edges in grey results in the desired binary tree.}
				\label{fig:overview}
			\end{center}
	\end{figure} 
	The first step is to draw a bounding box  around . Next we pick a segment  of  which has a point on the boundary of the convex hull of . We extend  till it reaches the bounding box and becomes part of the new bounding polygon  itself.
	We continue iteratively. The set  is defined to be  and  is a segment which has a point on the boundary of the convex hull of . We extend  till it becomes part of the new bounding polygon  itself. See Figure~\ref{fig:overview}~b) for an illustration.
Clearly  is a polygon, though not simple. In the next step depicted in Figure~\ref{fig:overview}~c), we draw a simple polygon  in the interior of . We want that the set of reflex vertices of  and  agree and that  is very close to  in the Hausdorff sense. As this is very intuitive we hope that the reader will be satisfied to know that
this simplification step was used also for instance in \cite{2012arXiv1207.2375A} lemma  and we refer to them for a rigorous proof.
We want to apply Theorem \ref{thm:main} on .
Note that each line segment has exactly one reflex vertex in  and these are the only reflex vertices of . We define  as the set of reflex vertices of  and for each extended endpoint of a line segment we choose a closest point on  to belong to , see Figure~\ref{fig:overview}~d). Thus  corresponds to the segment endpoints. The vertices  and  can be chosen arbitrarily.
Now Theorem \ref{thm:main} grants us the existence of a tree  in . We define  to be the graph which we attain when we connect the endpoints of the line segments whenever the corresponding points in  where connected by an edge in . To see that this is possible consider any  segment endpoints  and the corresponding points in  let us denote them by  and  respectively(not necessarily of the same segment). Observe that if  and  are close enough that  sees  iff  sees  (Remember that we assume general position).
See Figure~\ref{fig:overview}~e). 
At last we add the line segments successively to . The tree  is defined to be  whenever the segment  is already an edge of . Otherwise  is attained from  by adding segment  and removing one of the adjacent edges, which closes a cycle with . We choose to remove the edge which is incident to the vertex with the highest degree. Figure~\ref{fig:overview}~f) shows .
It is clear that  is an encompassing PSLG which is a tree as a graph and contains all the given line segments. It remains to show that it has max degree . 
Let us consider a line segment . Clearly, if  already belonged to  the endpoints have max degree . Otherwise we note that one of the endpoints  has max degree  and the other  has max degree  in . When we add  both  and  have max degree  and  respectively and we close some cycle, which contains an edge incident to  and another incident to . If  had indeed degree  we would have remove that edge and  has degree . Thus both have max degree  in  and henceforth max degree  in .
\end{proof}


\paragraph{Acknowledgments}
The author wants to thank Andrei Asinowski and G\"{u}nter Rote for their eagerness to listen and discuss important and non-important issues~\smiley. Special thanks goes to Matthias Henze for proofreading.


\bibliography{TillLib}
\bibliographystyle{plain}

\end{document}