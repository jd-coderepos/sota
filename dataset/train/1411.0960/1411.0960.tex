We will use the operations from the previous section to obtain a dynamic algorithm for \BP with respect to large items. The operations insert and delete are designed to process the input depending of whether an item is to be inserted or removed. Keep in mind that the parameter  changes over time as  may increase or decrease. In order to fulfill the properties  and , we need to adapt the number of items per group whenever  changes. The shiftA and shiftB operations are thus designed to manage the dynamic number of items in the groups as  changes.
Note that a group in the -block with parameter  has by definition the same number of items as a group in the -block with parameter  assuming they are in the same size category. If  increases, the former  block is treated as the new  block in order to fulfill the properties  and  while a new empty  block is introduced. To be able to rename the blocks, the  block needs to be empty. Accordingly the  block needs to be empty if  decreases in order to treat the old  block as new  block. Hence we need to make sure that there are no groups in the -block if  increases and vice versa, that there are no groups in the -block if  decreases.

We denote the number of all groups in the -blocks at time t by  and the number of groups in -blocks at time  by . To make sure that the -block (respectively the -block) is empty when  increases (decreases) the ratio  needs to correlate to the fractional digits of  at time  denoted by . Hence we partition the interval  into exactly  smaller intervals . We will make sure that  iff . Note that the term  is  if the -block is empty and the term is  if the -block is empty. This way, we can make sure that as soon as  increases, the number of -blocks is close to  and as soon as  decreases, the number of -blocks is close to . Therefore, the -block can be renamed whenever  changes. The algorithm uses shiftA and shiftB operations to adjust the number of - and -blocks. Recall that a shiftA operation reduces the number of groups in the -block by  and increases the number of groups in the -block by  (shiftB works vice versa). Let  be the number of shiftA/shiftB operations that need to be performed to adjust . 



\begin{figure}[ht]
  \begin{subfigure}{.99\textwidth}
\centering
\scalebox{0.9}{
\begin{tikzpicture}
\draw (0,0) -- (10,0);
\draw[very thick] (0,-7pt) -- (0,7pt);
\node at (0,-15pt) {};
\draw[very thick] (5,-7pt) -- (5,7pt);
\node at (5,-15pt) {};

\draw[very thick] (10,-7pt) -- (10,7pt);
\node at (10,-15pt) {};

\foreach \x in {0,1,...,9}{
  \draw (\x+1,-2pt) -- (\x+1,2pt);
}
\node at (0.5,10pt) {};
\node at (1.5,10pt) {};
\node at (2.5,10pt) {};
\node at (3.5,10pt) {};
\node at (4.5,10pt) {};



\node[rectangle,draw,minimum width=8cm, minimum height=1cm] (A) at
(4,-2.7) {};
\node[below = of A,yshift=1cm] {98\%};
\node[right = of A, rectangle,draw,minimum width=1cm, minimum
height=1cm,xshift=-0.9cm] (B) {};
\node[below = of B,yshift=1cm] {2\%};

\node at (3.2,-23pt) (dt) {};
\draw[thick,->] (dt) to (3.8,0);
\end{tikzpicture}
}
\caption{Before Insert}
  \end{subfigure}

  \begin{subfigure}{.99\textwidth}
\centering
\scalebox{0.9}{
\begin{tikzpicture}[every text node part/.style={align=center}]
\draw (0,0) -- (10,0);
\draw[very thick] (0,-7pt) -- (0,7pt);
\node at (0,-25pt) {\\  \\};
\draw[very thick] (5,-7pt) -- (5,7pt);
\node at (5,-25pt) {\\ \\};

\draw[very thick] (10,-7pt) -- (10,7pt);
\node at (10,-25pt) {\\\\};

\foreach \x in {0,1,...,9}{
  \draw (\x+1,-2pt) -- (\x+1,2pt);
}
\node at (5.5,10pt) {};
\node at (6.5,23pt) {\\  \\ };
\node at (7.5,10pt) {};
\node at (8.5,10pt) {};
\node at (9.5,10pt) {};



\node[rectangle,draw,minimum width=1cm, minimum height=1cm] (A) at
(1,-2.7) {};
\node[below = of A,yshift=1cm] {1\%};
\node[right = of A, rectangle,draw,minimum width=8cm, minimum
height=1cm,xshift=-0.9cm] (B) {};
\node[below = of B,yshift=1cm] {99\%};

\node at (6.0,-23pt) (dt) {};
\draw[thick,->] (dt) to (6.7,0);
\end{tikzpicture}
}
    
\caption{After Insert}
    \end{subfigure}
  \caption{Comparison of the situation before and after an Insert Operation}
\end{figure}

In the following algorithm we make use of an algorithm called \textsc{improve}, which was developed in \cite{jansen2013binpacking} to reduce the number of used bins. Using \textsc{improve}(x) on a packing  with approximation guarantee  for some  and some additive term  yields a new packing  with approximation guarantee . We use the operations in combination with the improve algorithm to obtain a fixed approximation guarantee. 




\begin{algo}[\ac{afptas} for large items] \label{alg-afptas}
  \ 
  \begin{small}
    

    \begin{algorithm}[H]
    \TitleOfAlgo{Insertion}
    \If{SIZE( or 
	SIZE}{use offline Bin Packing}
    \Else{
    \textsc{improve}(2);
    insert()\;
    \tcp{Shifting to the correct interval}
    Let  be the interval containing \;
    Let  be the interval containing \;
    Set \;
    \If(\tcp*[h]{Modulo  when  increases}){}{
         =  + \;
    }
    \tcp{Shifting  groups from  to }
    \For{ to }{
        \If{i+p = A(t) + B(t)}{Rename();}
        \textsc{improve}(1);
        shiftA\;
    }
    }
\end{algorithm}
  \end{small}
  \begin{small}
    

\begin{algorithm}[H]
    \TitleOfAlgo{Deletion}
    \If{SIZE( or 
	SIZE}{use offline Bin Packing}
    \Else{
    \tcp{Departing item }
    \textsc{improve}(4);
    delete()\;
    \textsc{ReduceComponents}\;
    \tcp{}
    \tcp{Shifting to the correct interval}
    Let  be the interval containing \;
    Let  be the interval containing \;
    Set \;
    \If(\tcp*[h]{Modulo  when  decreases}){}{
        d = d - (A(t)+B(t))\;
    }
    \tcp{Shifting  groups from A to B}
    \For{ to }{
        \If{i-p = 0}{Rename(A,B);}
        \textsc{improve}(3);
        shiftB\;
    }
    }\end{algorithm}
  \end{small}    
\end{algo}
Note that as exactly  groups are shifted from  to  (or  to ) we have by definition that  at the end of the algorithm. Note that  can be bounded by .

\begin{lemma}
\label{lem:disbounded}
At most  groups are shifted from  to  (or  to ) in Algorithm \ref{alg-afptas}.
\end{lemma}

\begin{proof}
Since the value  changes at most by  we can bound  by  to obtain the change in the fractional part. By  Lemma \ref{lem1} the number of intervals (=the number of groups) is bounded by . Using  and the fact that the number of groups  increases or decreases at most by , we can give a bound for the parameter  in both cases by

Hence, the number of shiftA/shiftB operations is bounded by .
\end{proof}


\begin{lemma}
\label{lem:binpackingalg}
    Every rounding function  produced by Algorithm \ref{alg-afptas} fulfills properties  to  with parameter .
\end{lemma}

\begin{proof}
Since Algorithm \ref{alg-afptas} uses only the operations insert, delete, shiftA and shiftB, the properties to are always fulfilled by Lemma  and the \ac{lp}/\ac{ilp} solutions  correspond to the rounding function by Lemma . 

Furthermore, the algorithm is designed such that whenever  increases the -block is empty and the -block is renamed to be the new -block. Whenever  decreases the -block is empty and the -block is renamed to be the new -block. Therefore the number of items in the groups is dynamically adapted to match with the parameter .
\end{proof}


\subsection{Large items}
In this section we prove that Algorithm \ref{alg-afptas} is a dynamic robust \ac{afptas} for the \BP problem if all items have size at least . The treatment of small items is described in Section \ref{sec:small} and the general case is described in Section \ref{sec:general}. 

We will prove that the migration between packings  and  is bounded by  and that we can guarantee an asymptotic approximation ratio such that  for a parameter  and for every .
The Algorithm \textsc{improve} was developed in \cite{jansen2013binpacking} to improve the objective value of an \ac{lp} with integral solution  and corresponding fractional solution . For a vector  let  be the set of all integral vectors  such that .

Let  be an approximate solution of the \ac{lp}  with  inequalities and let  and , where  denotes the fractional optimum of the \ac{lp} and  is part of the input of the algorithm (see Jansen and Klein \cite{jansen2013binpacking}). 
Let  be an approximate integer solution of the \ac{lp} with  for some value 
 and with . 
Suppose that both  and  have only  non-zero components. For every component  we suppose that . Furthermore we are given indices , such that the non-zero components  are sorted in non-decreasing order, i.\,e., .
\begin{algo}[\textsc{improve}]\label{improve}
\ 
  \begin{enumerate}
   \item Set ,  and 
   
    \item Compute an approximate solution  of the \ac{lp} 
	with ratio 
	\item If  then set , 
	 and goto step 9
  \item Choose the largest  such that the sum of the smallest components  is bounded by
  
	\item For all  set  
	and 
	\item Set  where  is a
          vector consisting of the components 
	. Reduce the number of non-zero components to at most .
  \item 
  \item For all non-zero components  set 
	\item If possible choose  such that  otherwise
  choose  such that  is maximal.
  \item Return 
  \end{enumerate}
\end{algo}


In the following we prove that the algorithm \textsc{improve} applied to the \BP \ac{ilp} actually generates a new improved packing  from the packing  with corresponding \ac{lp} and \ac{ilp} solutions  and . We therefore use Theorem \ref{thm-improve} and Corollary \ref{cor-improve} that were proven in \cite{jansen2013binpacking}.

\begin{theorem}\label{thm-improve}
    Let  be a solution of the \ac{lp} with  and furthermore . Let  be an integral
	 solution of the \ac{lp} with   for some value 
	 and with .
	 Solutions  and  have the same number of non-zero components and for each component we have 
	 .
	The Algorithm  then returns a fractional solution  with  and an integral solution
	 where one of the two properties hold:
	  or . 
	 Both,  and  have at most 
	non-zero components and the distance between  and  is bounded by .
\end{theorem}
\begin{corollary}\label{cor-improve}
    Let  for some  and 
	and let  for some  and . 
	Solutions  and  have the same number of non-zero components and for each component we have 
	 .
	 Then Algorithm  returns a fractional solution
	 with  and integral solution  where one of the two properties hold:
	  or . 
	Both,  and  have at most 
	non-zero components and the distance between  and  is bounded by .
\end{corollary}
Let  and .

\begin{theorem}\label{thm-packing}
    Given a rounding function  and an \ac{lp} defined for , let  be a fractional solution of the \ac{lp} with
	,  and  for some . Let  be an integral solution of the \ac{lp} with  and corresponding packing  such that .
	Suppose  and  have the same number  of non-zero components and for all components  we have
	.	Then Algorithm  on  and  returns a new fractional solution  with  and also a new integral solution  with corresponding packing  such that
	
	Further, both solutions  and  have the same number  of non-zero components and for each component we have
	. The number of changed bins from the packing
         to the packing  is bounded by .
    \end{theorem}
\begin{proof}
	

    To use Theorem \ref{thm-improve} and Corollary \ref{cor-improve} we have to prove that certain conditions follow from the requisites of Theorem \ref{thm-packing}.
    We have  by condition. Since
	 we obtain for the integral solution  that
	.
	Hence by definition of  we get . This is one requirement to use Theorem \ref{thm-improve}
	or Corollary \ref{cor-improve}.
	We distinguish the cases where   and  and look at them separately.
	
	Case 1: .
    For the parameter  we give a lower bound by the inequality . Lemma \ref{lem2} shows that  and therefore yields  
    
    and hence . We can therefore use Theorem \ref{thm-improve}.
    
	Algorithm \textsc{improve} returns by Theorem \ref{thm-improve} a  with  and an integral solution  with	 or . Using that  we can conclude . In the case where  we can bound the number of bins of the new packing  by	.
	In the case that  we obtain . Furthermore we know by Theorem \ref{thm-improve} that  and  have at most  non-zero components.
	
	Case 2: .
    First we prove that  is bounded from below. Since  we obtain that , which is a requirement to use Corollary \ref{cor-improve}.
	By using Algorithm \textsc{improve} on solutions  with  and  with  we obtain by Corollary \ref{cor-improve} a fractional solution  with
	 and an integral solution  with either  or .
	So for the new packing  we can guarantee that  if . In the case that , we can guarantee that . Furthermore we know by Corollary \ref{thm-improve} that  and  have at most  non-zero components.
    
Theorem \ref{thm-improve} as well as Corollary \ref{cor-improve} state that the distance  is bounded by . Since  corresponds directly to the packing  and the new integral solution  corresponds to the new packing , we know that only  bins of  need to be changed to obtain packing .
\end{proof}

In order to prove correctness of Algorithm \ref{alg-afptas}, we will make use of the auxiliary Algorithm \ref{reducecomponents} (\textsc{ReduceComponents}). Due to a delete-operation, the value of the optimal solution  might decrease. Since the number of non-zero components has to be bounded by , the number of non-zero components might have to be adjusted down. The following algorithm describes how a fractional solution  and an integral solution  with reduced number of non-zero components can be computed such that  is bounded. The idea behind the algorithm is also used in the \textsc{Improve} algorithm. The smallest  components are reduced to  components using a standard technique presented for example in \cite{beling1998}. Arbitrary many components of  can thus be reduced to  components without making the approximation guarantee worse.
\begin{algo}[\textsc{ReduceComponents}]\label{reducecomponents}
\ 
  \begin{enumerate}
  \item Choose the smallest non-zero components .
  \item If  then return  and 
  
  \item Reduce the components  to  components  with .
  \item For all  set i= b_ji= a_j 
	 
	
    and i= b_ji= a_j
  \item If possible choose  such that  otherwise
  choose  such that  is maximal.
  \item Return 
  \end{enumerate}
\end{algo}
The following theorem shows that the algorithm above yields a new fractional solution  and a new integral solution  with a reduced number of non-zero components.
\begin{theorem}
\label{thm:reduce}
    Let  be a fractional solution of the \ac{lp} with	. Let  be an integral solution of the \ac{lp} with . Suppose  and  have the same number  of non-zero components and for all components  we have
	.	Using the Algorithm  on  and  returns a new fractional solution  with  and a new integral solution  with . Further, both solutions  and  have the same number of non-zero components and for each component we have . The number of non-zero components can now be bounded by . Furthermore, we have that .
\end{theorem}
\begin{proof}
    Case 1: . We will show that in this case,  and  already have  non-zero components. In this case the algorithm returns  and . Since  the components  have an average size of at least  and since  are the smallest components, all components of  have average size at least . The size  is bounded by . Hence the number of non-zero components can be bounded by .

Case 2: . We have to prove different properties for the new fractional solution  and the new integral solution .

\textbf{Number of non-zero components}: The only change in the number of non-zero components is in step 3 of the algorithm, where the number of non-zero components is reduced by . As  have at most  non-zero components,  have at most  non-zero components. In step 4 of the algorithm,  is defined such that . In step 5 of the algorithm  is chosen such that . Hence we obtain that .

\textbf{Distance between  and }: The only steps where components of  changes are in step 4 and 5. The distance between  and  is bounded by the sum of the components that are set to , i.\,e.,  and the sum of the increase of the increased components . As , we obtain that the distance between  and  is bounded by . Using that , the distance between  and  is bounded by .

\textbf{Approximation guarantee}: The fractional solution  is modified by condition of step 3 such that the sum of the components does not change. Hence .\\
Case 2a: . Since  is chosen maximally we have for every non-zero component that . Since there are at most  non-zero components we obtain that .
Case 2b: .
By definition of  we have . We obtain for  that .
\end{proof}

\begin{theorem}
\label{thm-main}
Algorithm \ref{alg-afptas} is an \ac{afptas} with migration factor at most  for the fully dynamic \BP problem with respect to large items.
\end{theorem}

\begin{proof}
Set . Then . We assume in the following that  (which holds for ).

We prove by induction that four properties hold for any packing  and the corresponding \ac{lp} solutions. Let  be a fractional solution of the \ac{lp} defined by the instance  and  be an integral solution of this \ac{lp}.
    The properties  to  are necessary to apply Theorem \ref{thm-packing} and property  provides the wished approximation ratio for the \BP problem.
	\begin{enumerate}
		\item[(1)\label{prop:1}]  (the number of bins is bounded)
		\item[(2)\label{prop:2}] 
		\item[(3)\label{prop:3}] for every configuration  we have 
		\item[(4)\label{prop:4}]  and  have the same number of non-zero components and that number is bounded by 
	\end{enumerate}
	To apply Theorem \ref{thm-packing} we furthermore need a guaranteed minimal size for  and .
	According to Theorem \ref{thm-packing} the integral solution  needs  and 
	 as we set .
	By condition of the while-loop the call of \textsc{improve} is made iff  and . Since  the requirements for the minimum size are fulfilled. As long as the instance is smaller than  or  an offline algorithm for \BP is used. Note that there is an offline algorithm which fulfills properties  to  as shown by Jansen and Klein \cite{jansen2013binpacking}.
	
    Now let  be a packing with  and  for instance  with solutions  and  of the \ac{lp} defined by . Suppose by induction that the properties  to  hold for the instance . We have to prove that these properties also hold for the instance  and the corresponding solutions  and . The packing  is created by the repeated use of an
	call of \textsc{improve} for  and  followed by an operation (insert, delete, shiftA or shiftB).
    We will prove that the properties  to  hold after a call of \textsc{improve} followed by an operation.
	\\{\bf improve:} Let  be the resulting fractional solution of Theorem \ref{thm-packing}, let  be the resulting integral solution of Theorem \ref{thm-packing} and let  be the corresponding packing. Properties  to  are fulfilled 
	for ,  and  by induction hypothesis. Hence all conditions are fulfilled to use Theorem \ref{thm-packing}. 
	By Theorem \ref{thm-packing} the properties  to  are still fulfilled for ,  and  and moreover we get
	 and  for chosen parameter . Let  and  be the fractional and integral solution after an operation is applied to  and . We have to prove that the properties  to  are also fulfilled for  and .
	\\{\bf operations:} First we take a look at how the operations modify  and . By construction of the insertion operation,  and  are increased at most by . By construction of the delete operation,  and  are increased by . By construction of the shiftA and shiftB operation,  and  are increased by .
	An \textsc{improve}(2) call followed by an insertion operation therefore yields  since .
    An \textsc{improve}(4) call followed by a delete operation yields  since  (an item is removed) and . In the same way we obtain that  for an \textsc{improve}(1)/\textsc{improve}(3) call followed by a shiftA/shiftB operation. This concludes the proof that property  is fulfilled for .
	The proof that property  holds is analog since  increases in the same way as  and .
	For property  note that in the operations a configuration  of the fractional solution is increased by  if and only if a configuration  is increased by . Therefore the property that for all configurations  retains from  and . By Theorem \ref{thm-packing} the number of non-zero components of  and  is bounded by  in case of an insert operation. If an item is removed, the number of non-zero components of  and  is bounded by . By Theorem \ref{thm:reduce} the algorithm \textsc{ReduceComponents} guarantees that there are at most  non-zero components. By construction of the shift-operation,  and  might have two additional non-zero components. But since these are being reduced by Algorithm \ref{alg-afptas} (note that we increased the number of components being reduced in step 6 by  to- see \cite{jansen2013binpacking} for details), the \ac{lp} solutions  and  have at most  non-zero components which proves property . Algorithm \ref{alg-afptas} therefore has an asymptotic approximation ratio of .
    
    We still need to examine the migration factor of Algorithm \ref{alg-afptas}. In the case that the offline algorithm is used, the size of the instance is smaller than  or smaller than . Hence the migration factor in that case is bounded by . If the instance is bigger the call of \textsc{improve} repacks at most  bins by Theorem \ref{thm-packing}. Since every large arriving item has size  and  we obtain a migration factor of  for the Algorithm \textsc{improve}. Since the migration factor of each operation is also bounded by , we obtain an overall migration factor of . 
    
    The main complexity of Algorithm \ref{alg-afptas} lies in the use of
    Algorithm \textsc{improve}. As described by Jansen and Klein
    \cite{jansen2013binpacking} the running time of \textsc{improve} is
    bounded by , where  is the time needed to
    solve a system of  linear equations. By using heap structures to
    store the items, each operation can be performed in time
     at
    time , where  denotes the number of items in the instance
    at time . As the number of non-zero components is bounded by
    , the total running time of the
    algorithm is bounded by . The best known running time for the dynamic \BP problem
    \emph{without} removals was
     and is due to
    Jansen and Klein \cite{jansen2013binpacking}.  As this is polynomial
    in  and in  we can conclude that
    Algorithm \ref{alg-afptas} is an \ac{afptas}.
\end{proof}

If no deletions are present, we can use a simple FirstFit algorithm (as described by Jansen and Klein \cite{jansen2013binpacking}) to pack the small items into the bins. This does not change the migration factor or the running time of the algorithm and we obtain a robust \ac{afptas} with  migration for the case that no items is removed. This improves the best known migration factor of  \cite{jansen2013binpacking}.



















