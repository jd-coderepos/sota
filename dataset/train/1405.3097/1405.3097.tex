\documentclass{article} \usepackage[utf8]{inputenc}
\usepackage[T1]{fontenc}
\usepackage[unicode]{hyperref}
\usepackage{xspace}
\usepackage{booktabs}
\usepackage{amsmath}
\usepackage{named}
\usepackage{amssymb}
\usepackage{amsthm}
\usepackage{times}

\newcommand*{\Amc}{\ensuremath{\mathcal{A}}}
\newcommand*{\csf}{\ensuremath{{s}}}
\newcommand*{\ssf}{\ensuremath{{s}}}
\newcommand*{\Sinz}{\ensuremath{\Phi}}
\newcommand*{\Eq}{\ensuremath{\Psi}}
\newcommand*{\Sym}{\ensuremath{\phi_{symm}}}
\newcommand*{\vars}{\ensuremath{\mathrm{vars}}}
\newcommand{\nb}[1]{\marginpar{\bf\scriptsize #1}}
\newcommand*{\CBound}{\ensuremath{\mathsf{CBound}}}
\newcommand*{\edp}{\ensuremath{\mathsf{edp}}}
\newcommand*{\natP}{\ensuremath{\mathbb{N}^{+}}}
\newcommand*{\Mult}{\ensuremath{\mathsf{edp_{m}}}}
\newcommand*{\CMult}{\ensuremath{\mathsf{edp_{cm}}}}
\newcommand*{\cmult}{\ensuremath{\mathsf{cmult}}}
\newcommand*{\Treengeling}{\texttt{Treengeling}\xspace}
\newcommand*{\Lingeling}{\texttt{Lingeling}\xspace}
\newcommand*{\Glucose}{\texttt{Glucose}\xspace}
\newcommand*{\druptrim}{\texttt{drup-trim}\xspace}

\renewcommand*{\mu}{\ensuremath{\textunderscore}}

\newtheorem{theorem}{Theorem}
\newtheorem{example}[theorem]{Example}
\newtheorem{proposition}[theorem]{Proposition}


\begin{document}


\title{Computer-Aided Proof of Erd\H{o}s Discrepancy Properties}
\author{Boris Konev    and
Alexei Lisitsa \\{Liverpool University}
}

\date{}
\maketitle


\begin{abstract}
In 1930s Paul Erd\H{o}s conjectured that for any positive integer  in any
infinite  sequence  there exists a subsequence , for some positive integers  and , such that
.  
The conjecture has been referred to  as one of the major open problems in 
combinatorial number theory and discrepancy theory.  
For the particular case of  a human proof of the conjecture exists; for
 a bespoke computer program had generated sequences of length 
of discrepancy ,
but the status of the conjecture remained open even for such a small bound.
We show that by encoding the problem into Boolean satisfiability and applying the 
state of the art SAT solvers, one can obtain a discrepancy  sequence of length 
and a \emph{proof} of the Erd\H{o}s discrepancy conjecture for
, claiming that no discrepancy 2 sequence of length , or more, exists.    
In the similar way, we obtain a precise bound of  on the maximal
lengths of both multiplicative and completely multiplicative sequences of discrepancy
.
\end{abstract}
\section{Introduction}
Discrepancy theory is a branch of mathematics dealing with 
irregularities of distributions of points in some space in combinatorial, measure-theoretic
and geometric settings~\cite{Beck,Chazelle,Matusek,BeckSos}. 
The paradigmatic combinatorial discrepancy theory  problem can be described in
terms of a hypergraph , that is, a set   and a family
of its subsets . 
Consider a colouring  of the elements of  in  
\emph{blue}  and \emph{red}  () colours. Then one may ask whether there 
exists a colouring of the elements of  such that colours are
distributed uniformly in every element of  or a discrepancy of colours is always inevitable.
Formally, the discrepancy (deviation from a uniform distribution) of a hypergraph  is defined as
.
Discrepancy theory has found applications 
in {computational complexity}~\cite{Chazelle},
complexity of communication~\cite{DBLP:journals/dam/Alon92}
and {differential privacy}~\cite{Muthukrishnan:2012:OPH:2213977.2214090}. 

One of the oldest problems of discrepancy theory is the 
discrepancy of hypergraphs over sets of natural numbers with the  subsets
(hyperedges) forming arithmetical progressions over these sets~\cite{MS96}. 
 Roth's theorem~\cite{Roth64}, one of the main results in the area, states that
for the hypergraph formed by the arithmetic progressions in , that is
, where  and
elements of  being of the form  for arbitrary ,
the discrepancy grows at least as .

\vspace*{1pt}

Surprisingly, for the more restricted case of \emph{homogeneous} arithmetic progressions of the form
, the question of the discrepancy bounds is open for more than eighty years.   
In 1930s Paul Erd\H{o}s conjectured~\cite{Unsolved} that
discrepancy is unbounded. Independently the same conjecture has been raised by 
\cite{chudakov}.
Proving or  disproving this conjecture became one of the major open problems in
combinatorial number theory and discrepancy theory. It has been referred to as the
\emph{Erd\H{o}s discrepancy problem} (EDP)~\cite{Beck,BeckSos,NiTa}.


The  expected value of the discrepancy of 
random   sequences of length  grows as
 and the explicit constructions of a sequence with slowly growing
discrepancy at the rate of  have been demonstrated~\cite{gowers,BCC10}. 
By considering cases, one can see  that
any  sequence containing  or more elements has discrepancy at least ;
that
is, Erd\H{o}s's conjecture holds for the particular case  
(also implied by a stronger result of Mathias~\shortcite{Mathias}). 
Until recently the status of the conjecture remained unknown for all other
values of .  Although widely believed not to be the case, there was still a
possibility that an infinite sequence of discrepancy 2 existed. 

The conjecture whether the discrepancy of an arbitrary  sequence is
unbounded is equivalent to the question whether the discrepancy of a completely
multiplicative  sequence is unbounded, where a sequence is completely
multiplicative if  for any ~\cite{Unsolved}.
For completely multiplicative sequences the choices of how the sequence can be
constructed are severely limited as the entire sequence is defined by the values of
 for prime . The longest completely multiplicative sequence of
discrepancy  has length ~\cite{Polymath2}.  For discrepancy  the
bound was not known.

The EDP has attracted renewed interest in 2009-2010  as it became a topic of the fifth
Poly\-math 
project~\shortcite{Polymath}
a widely publicised endeavour in
collective math initiated by T.~Gowers~\shortcite{gowersblog}.   As part of this
activity  an attempt has been made
to attack the problem using computers (see discussion in \cite{Polymath}). A
purposely written computer program had successfully found  sequences of
length  and discrepancy ; however, no further progress has been made
leading to a claim ``given how long a finite sequence can
be, it seems unlikely that we could answer this question just by a clever
search of all possibilities on a computer''~\cite{Polymath}.


The status of the Erd\H{o}s discrepancy conjecture for  has been
settled by the authors of this article 
\cite{KLArx14}, \cite{KLSAT14} by reduction to SAT.  The method is
based on establishing the correspondence between  sequences that violate
a given discrepancy bound and words accepted by of a finite automaton.  Traces
of this automaton are represented then by a propositional formula and state
of the art SAT solvers are used to prove that the longest  sequence of
discrepancy  contains  elements. 
A  long  sequence of discrepancy  was also constructed.

This article is a revised and extended version of~\cite{KLSAT14}.  We use a
different smaller SAT encoding of the Erd\H{o}s discrepancy problem, which is
based on the sequential counter encoding of the \emph{at
most} cardinality constraints\footnote{We are grateful to Donald E. Knuth for
pointing us in that direction.}. The impact of the new encoding is twofold.
Firstly, it allows us to significantly reduce the size of the machine-generated
proof of the fact that any sequence longer than   has discrepancy at
least .
Secondly, by combining the new encoding with additional restrictions that the
sequence is multiplicative, or completely multiplicative, we improve
significantly the lower bound on the length of sequences of discrepancy . We
prove the surprising result that , the length  of the longest completely multiplicative
sequence of discrepancy , is also the maximal length of a 
multiplicative sequence of discrepancy , which is not the case for  and
.  The article also contains
detailed argumentation, examples and complete proofs.

The article is organised as follows. In Section~\ref{sec:preliminaries} we
introduce the main terms and definition. In Section~\ref{sec:encoding} we
describe the new SAT encoding of the Erd\H{o}s discrepancy problem.
Results and conclusions are discussed in Sections ~\ref{sec:experiments}
and~\ref{sec:conclusion} respectively.  To improve readability a number of
technical proofs have been deferred to an appendix.




\section{Preliminaries}\label{sec:preliminaries}
We divide this section into three parts: main definitions for the Erd\H{o}s 
discrepancy problem, some background and definitions for SAT solving, and
sequential counter-based SAT encoding of cardinality constraints.

Since number  is used both as an element of  sequences and 
as the logical value \emph{true}, to avoid confusion,  in what follows we write
 to refer to the logical value \emph{true} and  to refer to elements of
 sequences.  We also use the following naming convention: we write
 for  sequences,  for sequences of
propositions, and  for  Boolean sequences.


\subsection{Discrepancy of 1 Sequences}
A  sequence of length  is a function .  An
infinite  sequence is a function , where  is the
set of positive natural numbers.  We write  to denote a finite
 sequence of length , and  to denote an infinite sequence. We
refer to  the -th element of a sequence , that is the value of ,
as . 
A (finite or infinite)  sequence  is \emph{completely multiplicative} \cite{apostol} if

The sequence is \emph{multiplicative} if (\ref{eq:multdef1}) is only required
for 
coprime  and .

It is easy to see that a sequence  is
completely multiplicative if, and only if,  and for
the canonical representation  , where  are primes and , 
we have .
This observation leads to a more computationally friendly definition of
completely multiplicative sequences:  is completely multiplicative if,
and only if,


\smallskip

The Erd\H{o}s discrepancy problem can be naturally described in terms of 
sequences (and this is how Erd\H{o}s himself introduced it~\shortcite{Unsolved}).
Erd\H{o}s's conjecture states that for any  in any infinite 
sequence  there exists a subsequence , for some positive integers  and , such that . 

The general definition of discrepancy given above can be specialised in this
case as follows. The discrepancy of a finite   sequence
 of length  can be defined as . For an infinite
sequence  its discrepancy is the supremum of discrepancies of all its
initial finite fragments. 


\begin{example}\label{ex:edp1}
It is easy to see why any  sequence containing  elements has
discrepancy at least . For the proof of contradiction, suppose that 
the discrepancy of some  sequence  is . 
Assume that  is .  
We write 

to track progress in this example, that is,  we put specific values 
 or  into positions , to indicate decisions on 
which have been taken so far, and mark positions , for which no decision has 
been made by an underscore.

Notice that  must be  for otherwise . So, we progress to

Then the th element of the sequence must be  for otherwise 
for  the sum . So we progress to

Then the rd element of the sequence must be  for otherwise  and
so we come to

Repeating the reasoning above for 
 and  followed by ,
for  and  followed by , 
for  and  followed by 
and finally for  and  followed by  we progress to

But then for  we have . 
So we derive a contradiction.
It can be checked  in a similar way that the other possibility of  being
 also leads to a contradiction.

The first eleven elements of the sequence (\ref{eq:example1}) form a discrepancy  sequence.
It is multiplicative but not completely multiplicative as   is .
Reasoning similar to the one above shows that there
exists a unique longest completely multiplicative  sequence of
discrepancy  which has nine elements:

\end{example}

\subsection{Propositional Satisfiability Problem}
We assume standard definitions for propositional logic (see, for example,
\cite{rautenberg}).  Propositional formulae are defined over Boolean constants
\emph{true} and \emph{false}, denoted by  and , respectively, and the set
of Boolean variables (or propositions)  as follows: Boolean constants 
and , as well as the elements of , are formulae; if  and 
are formulae then so are  (conjunction), 
(disjunction),  (implication), 
(equivalence) and  (negation).  We
typically use letters ,  and  to denote propositions and 
capital Greek letters  and  to denote propositional formulae. Whenever
necessary, subscripts and superscripts are used. We use  to denote
the set of all propositions occurring in the formula .

Every propositional formula can be reduced to conjunctive normal form.  Propositions and
negations of propositions are called \emph{literals}.
When the negation is applied to a literal, double negations are implicitly 
removed, that is, if  is  then  is .
A disjunction of literals
is called a \emph{clause}. A clause containing exactly one literal is called
a unit clause. A conjunction of clauses is called a
propositional formula in \emph{conjunctive normal form}, a CNF formula for short. 
A clause can be represented by the set of its literals and the empty 
clause correspond to  (\emph{false}).
A CNF formula can be represented by the set its clauses. These
representations are used interchangeably. 
We typically use meaningful terms typeset in sans serif font,
for example  or ,
to highlight the fact that a propositional formula is a CNF formula of interest.

For a propositional formula , we write  to indicate
that . Propositions 
are designated as `input' propositions in this case, and the intended meaning
is that formula  encodes some property of . Then the
expression  denotes the result of simultaneous replacement
of every occurrence of  in   with , for .

The semantics of propositional formulae is given by interpretations (also termed
assignments).  An
interpretation  is a mapping  extended to literals, clauses,
CNF formulae and propositional formulae in general in the usual way.
For an assignment  and a formula  we say that  satisfies  
(or  is a model of ) if .
A formula  is satisfiable if there exists an assignment that satisfies
it, and unsatisfiable otherwise.

Despite the non-tractability of the satisfiability problem, the tremendous
progress in recent years made it possible to solve many interesting hard
problems by first expressing them as a propositional formula and then
using a {SAT solver} for obtaining a solution~\cite{satHB}. In addition to
returning a satisfying assignment, if the input formula is satisfiable, some
SAT solvers are also capable to return a proof (or certificate) of
unsatisfiability. 

Reverse Unit Propagation (RUP) proofs constitute a compact representation of 
the resolution refutation of the given formula~\cite{GoldbergN03} in the following sense.
\emph{Unit propagation} is a CNF formula transformation
technique, which simplifies the formula by fixing the values of 
propositions occurring to its unit clauses to satisfy these clauses. 
That is, if the unit clause  occurs in the CNF formula then 
all occurrences of  are replaced by  and if the unit clause
  occurs in the CNF formula, all occurrences of 
 are replaced by . Then the CNF formula is simplified in the obvious way.
A clause  is a RUP inference from the input CNF formula
 if adding the unit clauses  to  makes the whole
formula refutable by unit propagation.  A RUP unsatisfiability certificate is
the sequence of clauses  such that for every  
the clause  is a RUP inference from  and
 is the empty clause.  Every unsatisfiable CNF formula has a RUP
unsatisfiability certificate~\cite{GoldbergN03}.

Delete Reverse Unit Propagation (DRUP) proofs extend  RUP proofs by including
extra information about the proof search process, namely clauses that have been
discarded by the solver.  Eliminating this extra information from a DRUP proof
converts it to a valid RUP proof.  DRUP proofs are somewhat longer but they are
significantly faster to verify than a RUP proof~\cite{druptrim}. 


\subsection{Sequential Counter-Based SAT Encoding of Cardinality Constraints}\label{sec:sinz_card}
Cardinality constraints~\cite{cardconstrHB} are expressions that impose restrictions on
interpretations by specifying numerical bounds on the number of propositions,
from a fixed set of propositions, that can be assigned value . The \emph{at
most } constraint over the set of propositions , written
as , holds for an interpretation  if, and only if,
at most  propositions among  are true under .

A SAT encoding for cardinality constraints of the
form  based on a sequential counter circuit has been
suggested by Sinz~\shortcite{Sinz05}. In this encoding, auxiliary propositions
 are introduced to represent a unary counter storing the partial sums
of prefixes of  so that whenever , for
some , we have .  


We slightly simplify the presentation of~\cite{Sinz05} as follows.
Let formula 
be the conjunction of 
Recall that we write  to highlight the fact that 
 are designated `input' propositions; the set of 
all propositions 
of  is
.

Notice that instead of including formulae (\ref{eq:sinz_card_2}) and
(\ref{eq:sinz_card_3}) explicitly in the encoding, one can directly
modify (\ref{eq:sinz_card_1}) by replacing  
all occurrences of , for , with  (the truth
value \emph{false}) and all occurrences of , for  and all 
, with  (the truth value \emph{true}).
Then, for example, for  formula (\ref{eq:sinz_card_1}) simplifies to
. We write (\ref{eq:sinz_card_2}) and
(\ref{eq:sinz_card_3})  explicitly for the exposition purposes.


The proof of the following statement can be extracted from~\cite{Sinz05}. 
It is based on the observation that the sum of the first  elements of the  sequence
 exceeds  if, and only if, either the sum of the first 
 elements already exceeds , or the sum of the first  elements is  and 
the -th element of the sequence is .
We give the formal proof in an appendix for completeness of the presentation.
\begin{proposition}\label{prop:sinz}
Let  be as defined above.  Then 
\begin{enumerate}
\item[(i)] For any assignment  such that  satisfies ,
any  and 
we have 

\item[(ii)] For any -sequence   
 there exists an assignment  such that 
  , for ;
   satisfies ; and 
  for any  and  if  then , for .
\end{enumerate}
\end{proposition}
It follows from Proposition~\ref{prop:sinz} that the formula
 enforces the cardinality
constraint . 

Formula (\ref{eq:sinz_card_1}) can be equivalently rewritten into clausal form:

{where , }.

\smallskip

Notice that the original encoding in~\cite{Sinz05} only contains clauses
(\ref{eq:equiv_clause_3}) and (\ref{eq:equiv_clause_4}) due to a 
polarity-based optimisation based on Tseitin's~\shortcite{Tseitin70} renaming techniques.
One can see that
by unit propagation of clauses (\ref{eq:sinz_card_2}), (\ref{eq:sinz_card_3}) and
 into (\ref{eq:equiv_clause_3}) and
(\ref{eq:equiv_clause_4}) we obtain exactly the set of clauses used
in~\cite{Sinz05}, which consists of  clauses and requires 
auxiliary propositions. 
This optimisation leads to a relaxation in
item~(\textit{i}) of Proposition~\ref{prop:sinz}: Suppose that an assignment
, which
satisfies 
(\ref{eq:sinz_card_2}),
(\ref{eq:sinz_card_3}),
(\ref{eq:equiv_clause_3}) and
(\ref{eq:equiv_clause_4}),
is such that .  Then the value of
 under  can still be true, as long as , while not
violating the cardinality constraint .  

For our purposes we require an unoptimised version of the encoding with 
a tighter restriction on the values of  stated in Proposition~\ref{prop:sinz}.





\section{SAT Encoding of the discrepancy problem}\label{sec:encoding}

We say that a  sequence  is \emph{-bounded}, for some ,
if  , for all .
We extend the notion of -boundedness to Boolean  sequences using the 
relation between  sequences  and Boolean
-sequences  defined as follows:  (in other
words,  is encoded by the Boolean value \emph{true}, and  is encoded by
the Boolean value \emph{false}). Then a  sequence 
 is -bounded if for every  
the disbalance between the number of s in  and the 
number of s in  is at most .

We build our SAT encoding of the Erd\H{o}s discrepancy problem on 
the sequential counter-based encoding of cardinality constraints described
in Section~\ref{sec:sinz_card}. We illustrate our approach by the following consideration.
By Proposition~\ref{prop:sinz}, an arbitrary 
assignment of  values to propositions  can be uniquely 
extended to a model of . We 
denote this model as .
Suppose that the value of some  under  is false.
Then, by Proposition~\ref{prop:sinz}, the sequence  
contains at most  occurrences of  and so it contains at least 
occurrences of . Therefore, the disbalance between the number of occurrences
of  and the number of occurrences of  is at least .  If
 then the sequence  is not -bounded. 
Thus, in our encoding of  -bounded sequences we need to exclude such a possibility.
This can be achieved by conjoining formula   with


Similarly, if the value of  is true, for some , then 
the number of s exceeds the number of s by more than , so these
possibilities should also be excluded by


To summarise, let propositional formula  be the conjunction of ,
with (\ref{eq:equiv_clause_8}) and
(\ref{eq:equiv_clause_6}).
Notice that, as in the case of  , 
we write (\ref{eq:equiv_clause_8}) and (\ref{eq:equiv_clause_6}) explicitly
for the ease of explanation. The proof of the following theorem can 
be found in Appendix~\ref{sec:proofs}.
\begin{theorem}\label{th:cbounded}
For any assignment  the following holds: 
there exists an extension of  to  that is 
a model of  if, and only if, 
the sequence  is -bounded.
\end{theorem}



As (\ref{eq:sinz_card_1}) is logically equivalent to the set of clauses
(\ref{eq:equiv_clause_1})--(\ref{eq:equiv_clause_4}), formula 
  is logically equivalent to 
the set of clauses  consisting of 
(\ref{eq:equiv_clause_1})--(\ref{eq:equiv_clause_4}), 
(\ref{eq:sinz_card_2}), 
(\ref{eq:sinz_card_3}), 
(\ref{eq:equiv_clause_8})
and
(\ref{eq:equiv_clause_6}). 
Let  be the result of applying unit propagation to  in an 
exhaustive manner.
One can see that
the set  contains less than  auxiliary propositions
and less than  clauses.

\begin{example}
We construct ,
a clausal representation of the statement that the sequence
 is -bounded.
We also demonstrate how clauses (\ref{eq:sinz_card_2}), (\ref{eq:sinz_card_3}),  
(\ref{eq:equiv_clause_8})
and 
(\ref{eq:equiv_clause_6})
are unit propagated into clauses (\ref{eq:equiv_clause_1})--(\ref{eq:equiv_clause_4}).


Notice that for  every instance of clause (\ref{eq:equiv_clause_1}) contains 
literal , the only literal of the unit clause (\ref{eq:sinz_card_3}).
Thus, every instance of clause (\ref{eq:equiv_clause_1}) for  is redundant.

For  and , clause (\ref{eq:equiv_clause_1}) contains , the
only literal of the unit clause (\ref{eq:sinz_card_2}), so for 
 and , the instance of clause (\ref{eq:equiv_clause_1})  is also redundant.

For  and ,  an instance of
the unit clause (\ref{eq:sinz_card_2}), namely , unit propagates into 
(\ref{eq:equiv_clause_1}) resulting in a 2-CNF clause 
.


For  and , (\ref{eq:equiv_clause_1}) instantiates to 
.

Finally, for   and 
for , clause (\ref{eq:equiv_clause_1}) contains , 
the only literal of the unit clause  (\ref{eq:equiv_clause_8}). Thus, for 
 and  the instances of clause (\ref{eq:equiv_clause_1}) are redundant.

By a further consideration of cases, one can see that the set of all
non-redundant simplified instance of clause (\ref{eq:equiv_clause_1}), for
 and , consists of 
2pt]
(\lnot \ssf^2_3 \lor \ssf^2_2 \lor \ssf^1_2)  &  (\lnot \ssf^3_5 \lor \ssf^3_4 \lor \ssf^2_4).
\end{array}
\label{eq:ex1.2}
\begin{array}{l@{\qquad}l@{\qquad}l}
(\lnot \ssf^1_1 \lor p_1 )              &  (\lnot \ssf^2_2 \lor p_2)               & (\lnot \ssf^3_4 \lor p_4)               \2pt]
(\ssf^1_2 \lor p_3 )                    &  (\lnot \ssf^2_4 \lor \ssf^2_3 \lor p_4) &\2pt]
\end{array}
\label{eq:ex1.3}
\begin{array}{lll}
(\lnot \ssf^1_1 \lor \ssf^1_2) & (\lnot \ssf^2_2 \lor \ssf^2_3)     & (\lnot \ssf^3_4 \lor \ssf^3_5)\
and set of all non-redundant simplified instances of clause (\ref{eq:equiv_clause_4}) consists of
2pt]
(\lnot p_2 \lor \ssf^1_2) &  (\lnot \ssf^1_2 \lor \lnot p_3 \lor \ssf^2_3) &  (\lnot \ssf^2_3 \lor \lnot p_4 \lor \ssf^3_4)&\2pt]
\end{array}
\label{eq:edp}
\edp(C,n) = \bigwedge_{d=1}^{\lfloor\frac{n}{C+1}\rfloor}\CBound^C(p_d, p_{2d},\dots, p_{\lfloor n/d\rfloor\cdot d}).
\label{eq:mult2}
\textsf{prod}_{j,k} = \{
(\lnot p_{j}\lor \lnot p_{k}\lor p_{j\cdot k}),
(p_{j}\lor p_{k}\lor p_{j\cdot k}),
(\lnot p_{j}\lor  p_{k}\lor \lnot p_{j\cdot k}),
(p_{j}\lor \lnot p_{k}\lor \lnot p_{j\cdot k})\} 
\cmult_i = 
\left\{
\begin{array}{ll}
 \emptyset  & \textrm{ if  is prime}\5pt]
\mathsf{prod}_{j,k}\quad &\textrm{ if none of the cases above applies}\\
& \textrm{ and  are some non-trivial divisors of .}
\end{array}
\right.

\Mult(C,n) = \edp(C,n)\cup\mathop{\bigcup}_{{\substack{1< j\leq n, 1< k \leq n\\ j,k\textrm{ are coprime}\\j\cdot k \leq n}}} \textsf{prod}_{j,k}

\CMult(C,n) = \edp(C,n)\cup\bigcup_{i=1}^n \cmult_i.

    I(\csf^k_j) = 1\quad \textrm{if, and only if,}\quad \quad\sum_{i=1}^j I(p_i) \geq k.
\label{eq:proof2}
    \sum_{i=1}^{j}I'(p_{i}) < k.

\sum_{i=1}^{j}I'(p_{i}) - \sum_{i=1}^{j}(1-I'(p_{i})) = 2( \sum_{i=1}^{j}I'(p_{i})) - j \ge 2k - j > 2k- (2k-C) = C.

\sum_{i=1}^{j} I'(p_{i}) - \sum_{i=1}^{j}(1 - I'(p_{i})) > C,

    \sum_{i=1}^{j}(1- I'(p_{i})) - \sum_{i=1}^{j}I'(p_{i}) > C,
 
that is,  and .  
Then, by the satisfaction condition for  (\ref{eq:equiv_clause_6}), we have 
   and, , we have  .  A contradiction. 
\end{enumerate}
\mbox{}
\end{proof}



 


\section{A sequence of length 1160 and discrepancy 2}\label{sec:sequence}
We give a graphical representation of one of the sequences of length  
obtained from the satisfying assignment computed with the \Treengeling solver.
Here  stands for  and  for , respectively.

\smallskip 

\noindent{}{\tt 
- + + - + - - + + - + + - + - - + - - + + - + - - + - - +\\
+ - + - - + + - + + - + - + + - - + + - + - - - + - + + - \\
+ - - + - - + + + + - - + - - + + - + - - + + - + + - - - \\
- + + - + + - + - + + - - + + - + - + - - - + + - + - - +\\
+ - + + - + - - + + - + - - + - - - + - + + - + - - + + - \\
+ + - + - - + - - + + - + + - + - - + + - + - - + + + - +\\
- + - - - - + + + - + - - + - - + + + - - - + + - + + - + \\
- - + - - + + + - - + - + - + - - + - + + + - + + - + - - \\
+ - - + + - + - - + + - + + - + - - + - - + + - - + + + -\\
- - + + + - + - - - + + - + - - + + - - + - + - - + - + + \\
+ - + - - + + - + + - + - - + + - + - - + - - + + - + - -\\
+ + - - + - + + - + - + - - + - + - + + - + - - + + - + -\\
- + - - + + - + - + - + + - + - + - + + - - - + - + - - + \\
+ + + - - + - - - + + - + - + + - + - - + + - + - - + - -\\
+ + - + - - + + + + - - + - - - + - + + + + - - + - - + +\\
- + + - + - - + + - + - - + - - + + - + - - + + - + + - +\\
- - + + - + - - + - - + + - + + - + - - - - + + + - + - - \\
+ + - - + + + - - - + - + + - + - - + - + + - - - + - + +\\
- + + - + - - + - - + + - - + + + + - + - - + - - + - - + \\
+ + + - - + - - + + + - - - + + - + + - + - - + + - - + - \\
+ - - + - - + + - + + - + - - + - - + - + + + - + + - + - \\
- + - - + + - - + - + + - + + - + - - + - - + - - + + - + \\
+ - + - - + + + - - - + + - + - - + + - + + - - - + + + - \\
- - + + - + + - - - - + + + - - + - + + - + - - + - - + +\\
- + - - + + - + + - + - + + - - + + - - + + - - - - + + + \\
- + + - - + + - - - - + + - + + + - - + + - - - + + + - -\\
- - + - + - + + - + + - + + - + - + - - - - + + + - - + +\\
- + - - + + - + + - + - - + - - + - - + + - + - - + + - + \\
+ - + - - + + - - + - + - - + - + - + - + + + + - - - + -\\
+ - + + - - + - - + - + - + - + + - + - + + + - - + - + - \\
- + - - + - + + + - - + - + + + - - - + + - + - - + - - +\\
+ - + + - - + + - - - + + - + - + + - - + + - + - - - + - \\
+ + - + - - + - + + - - + + - + - - + + - + - - + - + + + \\
- + - - + + - - + - + - + + + - - + - + - - + + - + + - +\\
- - + - - + - + + - - - + - + + - + - + + - - + + - + - -\\
+ + + - + - - - - + + - - + - + + - + - + + - - + + - + - \\
- + + - + - + + - - + + - + - - - + - + + - + - - + + + - \\
- - - + - + - + + - - + + - + - - + + - + + - + + - + - - \\
+ - - + - - + + + + - - - + + - - - + - + - + + - + - + + \\
+ - - + - + + - - + - + - - + - + - + + - - - + + + - + + \\
}











\end{document}
