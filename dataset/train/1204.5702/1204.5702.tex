\documentclass[11pt,3p,times]{elsarticle}
\newenvironment{myproof}{\begin{proof}}{\end{proof}}
\usepackage{amsfonts,amssymb,amscd,amsthm}
\usepackage{latexsym, graphicx}

\newcommand{\La}{\ensuremath{\llcorner}} \newcommand{\Lb}{\ensuremath{\ulcorner}} \newcommand{\Lc}{\ensuremath{\lrcorner}} \newcommand{\Ld}{\ensuremath{\urcorner}} 

\newcommand{\proofBox}{\hfill }

\newtheorem{theorem}{Theorem}
\newtheorem{observation}[theorem]{Observation}
\newtheorem{lemma}[theorem]{Lemma}

\newtheorem{definition}{Definition}
\newtheorem{conjecture}{Conjecture}

\graphicspath{{figures/}}





\journal{Discrete Applied Mathematics: Special Issue for LAGOS'13}

\begin{document}


\author[author1]{Kathie Cameron}
\author[author2]{Steven Chaplick\corref{cor}}
\author[author3]{Ch\'inh T. Ho\`ang}	

\address[author1]{Department of Mathematics, Wilfrid Laurier University, Waterloo, ON, Canada. Email: {\tt kcameron@wlu.ca}}

\address[author2]{Institut f\"ur Mathematik, Technische Universit\"at Berlin, Berlin, Germany. Email: {\tt chaplick@math.tu-berlin.de}}

\address[author3]{Department of Physics and Computer Science, Wilfrid Laurier University, Waterloo, ON, Canada. Email: {\tt choang@wlu.ca} }

\cortext[cor]{Corresponding author.  Phone: {\tt +49 30 314 28706}. Fax: {\tt +49 30 314 25191}}

\begin{frontmatter}


\title{Edge Intersection Graphs of -Shaped Paths in Grids\footnote{An extended abstract of this paper appeared at LAGOS'13 \cite{Cameron2013363}}}



\begin{abstract}
In this paper we continue the study of the edge intersection graphs of one (or zero) bend paths on a rectangular grid. That is, the edge intersection graphs where each vertex is represented by one of the following shapes: , and we consider zero bend paths (i.e.,  and --) to be degenerate 's. 
These graphs, called -EPG graphs, were first introduced by Golumbic et al (2009). 
We consider the natural subclasses of -EPG formed by the subsets of the four single bend shapes (i.e.,  and ) and we denote the classes by  and  respectively. Note: all other subsets are isomorphic to these up to 90 degree rotation.
We show that testing for membership in each of these classes is NP-complete and observe the expected strict inclusions and incomparability (i.e., -EPG
and  is incomparable with ). Additionally, we give characterizations and polytime recognition algorithms for special subclasses of \textit{Split}~~. 
\end{abstract}

\begin{keyword}
Edge Intersection Graphs \sep
Grid Paths \sep
Split Graphs \sep
NP-completeness \sep
Recognition Algorithms \sep
L-graphs
\end{keyword}

\end{frontmatter}

\section{Introduction}
A graph  is called an {\em EPG graph} if  is  the
intersection graph of paths on a rectilinear grid, where each vertex in 
corresponds to a path on the grid and two vertices are adjacent in
 if and only if the corresponding paths share an edge on the grid. EPG
graphs were introduced by Golumbic et al \cite{Gol2009}. The
motivation for studying these graphs comes from circuit layout
problems \cite{Ban1990}. Golumbic et al \cite{Gol2009} defined a {\em
-EPG graph} to be the edge intersection graph of paths on a
grid where the paths are allowed to have at most  bends
(turns). The -EPG graphs are exactly the well studied {\em
interval graphs} (the intersection graphs of intervals on a line).

Golumbic and Jamison \cite{Gol1985}
proved that  the recognition problem for the edge intersection
graphs of paths in trees (EPT) is NP-complete even when restricted
to chordal graphs (i.e., graphs without induced -cycles for ). Heldt et al \cite{Hel2010} proved that the recognition problem for
-EPG is NP-complete.
In a recent paper Epstein et al \cite{EpsteinGM13} have demonstrated that both the
coloring problem and the independent set problem are NP-complete on -EPG graphs.
They have further shown that these problems can be 4-approximated in polynomial time
when a -EPG representation is given and that the clique problem can be solved
optimally in polynomial time even without a given EPG representation.

A graph is {\em chordal} if it does not
contain a chordless cycle with at least four vertices as an
induced subgraph. A graph is a {\em split graph} if its vertices
can be partitioned into a clique and a stable set; {\em Split}
denotes the class of split graphs. Asinowski and
Ries \cite{Asi2012} characterized special subclasses of chordal
-EPG graphs.

Consider a -EPG graph  with a path representation on a grid.
The paths can be of the following four shapes: . In this paper, we  study  -EPG graphs whose paths on
the grid belong to a proper subset of the four shapes.   
If 
is  a subset of , then 
denotes the class of graphs that can be represented by paths whose
shapes belong to . It is important to note that we also 
zero-bend paths (i.e., vertical and horizontal line segments) to be s. 
In particular, an -representation of a graph may have some of its vertices represented as zero-bend paths. 
We are especially interested in the class
 of -EPG graphs whose paths are of the type . Our
main results are:
\begin{itemize}
 \item Establishment of expected separation between the classes: -EPG
and the incomparability between  and .
 \item A proof of NP-completeness of recognition of  and
 of each of the other subclasses of -EPG mentioned above.
 \item Characterizations of, and recognition algorithms for
 gem-free split -graphs and bull-free split -graphs.
\end{itemize}
In Section~\ref{sec:properties}, we discuss background results and
establish some properties of -EPG graphs. In
Section~\ref{sec:np-hard}, we show that recognition of 
and of each of the other subclasses is
an NP-complete problem. In Section~\ref{sec:split}, we give
polytime recognition algorithms for the classes of
gem-free split -graphs and of bull-free split -graphs. We
conclude with some open questions in Section~\ref{sec:conclusion}.
\section{Properties of -EPG graphs}\label{sec:properties}
Let  be a collection of nontrivial simple paths on a rectilinear
grid . (The end-points of each path are grid points.)
The \textit{edge intersection graph}
 has a vertex  for each path  and two vertices are adjacent in 
if the corresponding paths in  share an edge of
. For any grid edge , the set of paths containing
 is a clique in ; such a clique is called an
\textit{edge-clique} \cite{Gol2009}. A \textit{claw} in a grid
consists of three grid edges meeting at a grid point.  The set of
paths which contain two of the three edges of a claw is a clique;
such a clique is called a \textit{claw-clique} \cite{Gol2009} (see
Figure~\ref{fig:edgeclique}).
\begin{figure}[h]
\hfill
\includegraphics[scale=1.5]{edge-clique}
\hfill
\includegraphics[scale=1.5]{claw-clique}
\hfill \ \caption{Left: An edge-clique. Right: A
claw-clique.}\label{fig:edgeclique}
\end{figure}
\begin{lemma}[\cite{Gol2009}]\label{lem:Gol2009}
Consider a -EPG representation
of a graph .
Every clique in  corresponds to either an edge-clique or a claw-clique.
\end{lemma}


The \textit{neighborhood}  of a vertex  is the set of vertices adjacent to .
A set of vertices is \textit{stable} if no two are adjacent. An \textit{asteroidal triple}
(AT) is a stable set of size three such
that for every pair, there is a path between them which avoids the
neighborhood of the other vertex.

\begin{lemma}[AT Lemma \cite{Asi2012}, Theorem 9]\label{lem:AT}
In a -EPG graph, no vertex can have an AT in its
neighborhood.
\end{lemma}

Let  denote the chordless cycle  on four vertices.
Golumbic et al \cite{Gol2009} proved that any -EPG
representation of  corresponds to what they call a ``true
pie", a ``false pie", or a ``frame".  True and false pies require
paths other than 's. A \textit{frame} is a rectangle in the
grid  such that each corner is the bend-point for one
of  and ;  , and  each contain at least one grid edge; and   and  each do not contain an grid edge. Consider
the  and four representations of it shown in
Figure~\ref{fig:C4-K23}. The first three representations are frames,
the fourth is a false pie, and the fifth is a true pie. It follows
that:

\begin{figure}[bth]
\hfill
\includegraphics[scale=1]{C4}
\hfill
\includegraphics[scale=1]{K23}
\hfill \ \caption{Left:  and representations of it. Right:
 and representations of it.}\label{fig:C4-K23}
\end{figure}
\begin{lemma}[ Lemma]\label{lem:C4}
In an - or -representation of a , every ,  and  has a neighbor on both its vertical segment and on its horizontal segment.
\end{lemma}

\begin{observation}
 is in .
\end{observation}
\begin{myproof}
See Figure~\ref{fig:C4-K23} for an -representation of .
\end{myproof}

\begin{lemma}[ Lemma]\label{lem:K2,3}
In an -representation of a  every , and  has a neighbor on both its vertical segment and on its horizontal segment.
\end{lemma}
\begin{myproof}
Consider 
to be the complete bipartite graph with bipartition 
. Note that each of the following is a :
; ; and . As noted above, any
-EPG representation of  corresponds to  a ``true pie", a
``false pie", or a ``frame".  True pies require paths of all four
types, but false pies and frames can be made from just 's and
's.









If an -representation of a  corresponds to a
frame, then every  (and ) has a neighbor on both
its vertical segment and on its horizontal segment. Consider an
-representation of a . Either at least two of the 's
correspond to frames or at least two of the 's correspond to false pies.
The latter is clearly not possible.

Suppose that both   and 
correspond to frames.  Then  and  must have the same
bend-point, and this bend-point must be an  intersection point of
 and . Since  and  are not adjacent, one of 
and  is an  and the other is a . It follows that
every  (and ) has a neighbor on both its vertical
segment and on its horizontal segment. Note that  corresponds
to a false pie.
\end{myproof}
\begin{observation}\label{obs:K2,3}
 is in  but not in . 
\end{observation}
\begin{myproof}
Again, recall
that  and  are 's in . True and
false pies are not representable using just 's and 's.  So
both of these must be represented as frames.  As argued above,
 and  must have the same bend-point.  But since  and
 are not adjacent, if  is an , then  must be an
 and vice versa. It follows that  is not in
.
\end{myproof}
\begin{figure}[bth]
\hfill
\includegraphics[scale=1.2]{3-sun}
\hfill
\includegraphics[scale=1.5]{W4}
 \hfill \ \caption{Left: 3-sun and a
representation of it. Right: 4-wheel and a
representation of it.}\label{fig:3-sun-and4-wheel}
\end{figure}
\begin{observation}\label{obs:3-sun}
The 3-Sun is in  but not in .
\end{observation}
\begin{myproof}
See Figure~\ref{fig:3-sun-and4-wheel} for the 3-sun and an
-representation of it. To see that the 3-sun
does not have an -representation, recall that in a
-EPG graph, every clique is an edge-clique or a claw-clique.
The vertices of the 3-sun can be partitioned into a clique with
vertices  and a stable set with vertices  with edges
.  It is easy to see that if the clique
 is an edge-clique, then only two of  can be
represented regardless of which types of 1-bend paths are used. So
the clique  is a claw-clique. But 's and 's
can not form a claw-clique.
\end{myproof}
\begin{observation}\label{obs:4-wheel}
The 4-wheel is in  but not in  or
.
\end{observation}
\begin{myproof}
See Figure~\ref{fig:3-sun-and4-wheel} for the 4-wheel  and an
-representation of it. Lemma 3 in
\cite{Asi2012} shows that in a -representation of , the
 corresponds to a true pie or a false pie. Since the true pie
requires four shapes, we may assume the  of the  is
represented by a false pie. So,  is not an -graph.
Consider the vertex  of  that is adjacent to all vertices
of the . If  is of type  or , then  can
not share a grid edge with all four paths of the . So, the
 is not an -graph.
\end{myproof}

\begin{figure}[bth]
\centering
\includegraphics[scale=1.5]{Y6}
\caption{ and an  representation of it.}
\label{fig:Y6}
\end{figure}


Let  denote the graph shown in Figure
\ref{fig:Y6}.
Graph  consists of  with bipartition 
together with a vertex  adjacent to all the other vertices except . Note
that  contains both  and  as induced subgraphs, and thus is
not representable in  or . Figure
\ref{fig:Y6}
gives an -representation of .

\begin{lemma}[ Lemma]\label{lem:Y6}
In any -representation of  every , , and  has a neighbor on both its vertical segment and on its horizontal segment.
\end{lemma}
\begin{myproof}
As mentioned above, Lemma 3 in
\cite{Asi2012} implies that in an -representation of , the
 is represented by a false pie. Consider an -representation of .
The induced  of  is represented as in Figure
\ref{fig:Y6Proof}
(i). The  and  of Figure 
\ref{fig:Y6Proof}
(i) are either  and  or  and . Vertex  of  is adjacent to vertices  and  only.
It follows that the  and  of Figure
\ref{fig:Y6Proof}
(i) are  and , and that  and  intersect in a second point , which is the bend-point of 
(see Figure
\ref{fig:Y6Proof}
(ii) ). The representation is unique up to whether  is an  and  is a  or vise versa,
and the shape of  (see Figure
\ref{fig:Y6Proof}
(iii) for example). In any case, each ,  and  has a neighbor on its vertical segment
and on its horizontal segment.

\begin{figure}[h]
\hfill
\begin{tabular}{c}
\includegraphics[scale=1.5]{Y6-lemma-i}\\
(i)
\end{tabular}
\hfill
\begin{tabular}{c}
\includegraphics[scale=1.5]{Y6-lemma-ii}\\
(ii)
\end{tabular}
\hfill
\begin{tabular}{c}
\includegraphics[scale=1.5]{Y6-lemma-iii}\\
(iii)
\end{tabular}
\hfill \
\caption{Visual aids for the proof of Lemma \ref{lem:Y6}.}
\label{fig:Y6Proof}
\end{figure}
\end{myproof}

\section{NP-Hardness: Recognition of  and of Other Subclasses of -EPG }
\label{sec:np-hard}

It is well-known that interval graphs (i.e., -EPG graphs) can
be recognized in polynomial time \cite{Boo1976}. The complexity of
the recognition problem for -EPG () was given as an
open problem in the paper introducing EPG graphs \cite{Gol2009}.
The recognition problem for -EPG has been shown to be
NP-complete in a recent paper \cite{Hel2010}. In this section we
consider the complexity of recognizing the simplest natural
subclass of -EPG which is a superclass of -EPG; namely,
. Specifically, we show that it is NP-complete to decide
membership in . We will call the classes  and  \textit{the natural
subclasses of} -EPG. We show that it is NP-complete to
decide membership in each of these classes.

We use  to denote the subgraph of  induced by the set 
of vertices.

\begin{theorem}
Deciding membership in each of  and   is NP-complete.
\end{theorem}
\begin{myproof}
A given  model is easily verified, so  recognition
is in NP, and the same is true for each of the other classes.
For NP-hardness we demonstrate a reduction from the
usual 3-SAT problem (defined below). Our reduction is inspired by
the NP-completeness proof for -EPG \cite{Hel2010}.

The essential ingredients of our construction are described in the
following observations. In a -EPG-representation  of a
graph  containing vertices  and , we say that  is an
\emph{internal neighbor} of  in  when:  is adjacent to
, 's bend-point is not contained in  and w.l.o.g.
's horizontal contains 's horizontal (see Figure
\ref{fig:internal-external}(i)). We also say that  is an
\emph{external neighbor} of  when  is adjacent to  but
 is not an internal neighbor of . Notice that, in any
-EPG-representation of a graph, a vertex can have at most four
stable external neighbors (as depicted in Figure
\ref{fig:internal-external}(ii)). Additionally, if a vertex  is
an internal neighbor of a vertex , then  can have at most
two stable external neighbors which are not adjacent to  (see
Figure \ref{fig:internal-external}(iii)). Finally, we say that a
vertex  is \emph{adjacent to a graph } when  is adjacent to
exactly one vertex in an induced  (see Figure
\ref{fig:internal-external}(iv) where ).

Let  denote
the set of natural subclasses of -EPG.  We will use  to denote
an arbitrary natural subclass.  For each natural subclass
, we define a special graph :
  

Recall that for each , in any
-representation of , every ,  and
 of the representation has a neighbor with edge-intersection on
its vertical and a neighbor with edge-intersection on its horizontal (by
Lemmas \ref{lem:C4}, \ref{lem:K2,3} and \ref{lem:Y6}).

Consider a graph  with a vertex  that is adjacent to an
, and let  be 's neighbor in . It follows
from the previous paragraph that in any -representation of ,  is
necessarily an external neighbor of .

With these observations
in mind we can now describe the structure of our graph .
\begin{figure}[h]
\begin{tabular}{c}
\includegraphics[scale=1]{internal-neighbours} \\
(i)
\end{tabular}
\hfill
\begin{tabular}{c}
\includegraphics[scale=1]{external-neighbours}\\
(ii)
\end{tabular}
\hfill
\begin{tabular}{c}
\includegraphics[scale=1]{internal+external-neighbours}\\
(iii)
\end{tabular}
\hfill
\begin{tabular}{c}
\includegraphics[scale=1]{adjacent-4-cycle}\\
(iv)
\end{tabular}
\caption{(i):  is an internal neighbor of  (left:
\emph{internal horizontal  neighbor}; right: \emph{internal
vertical neighbor}). (ii):  with stable external neighbors
. (iii):  is an internal neighbor of , and
 has two stable external neighbors  which are not adjacent
to . (iv):  adjacent to one  and  adjacent to two
adjacent 's.} \label{fig:internal-external}
\end{figure}
A \emph{3-SAT} formula  is a boolean formula over variables
 where  is a conjunction of  clauses
, each clause  () is a
disjunction of three literals ,
and each literal  () is either
some variable  ()
or its negation.
Given a 3-SAT formula , it is well known that it is
NP-complete to decide whether there exists an assignment to the
variables of  that satisfies  \cite{Kar1972}.

Given a 3-SAT formula  and ,
we will construct a graph 
such that  is in  if and only if  can be
satisfied. Graph  consists of an induced subgraph  for
each clause  of  and a variable gadget to identify the
clauses with their corresponding literals. The general form of
these gadgets where  or  and thus
 is given in Figure \ref{fig:gadgets}.
\begin{figure}[h]
\centering
\hfill
\includegraphics[scale=1]{variable-gadget}
\hfill
\includegraphics[scale=1]{clause-gadget}
\hfill \ 
\caption{The general form of  when  or
. On the left is the main construction of  where the clause gadgets (depicted on the right) are drawn in the shaded region. Also, the shaded region of the depiction of the single clause gadget (on the right) corresponds to the induced subgraph  of  and the dotted box contains the remainder of . Note: the literals (i.e., vertices of the form  or
) of the ith clause () are matched to  in the clause gadget
.} \label{fig:gadgets}
\end{figure}


For , each of the 's that , , and the
's are adjacent to is replaced by  (so that , , and the
's are adjacent to a degree 2 vertex of ; also, for each j,
, we add a vertex  which is adjacent to
 and  (thus turning the  induced by
 into a .

Similarly, for , each of the 's that , , and the
's are adjacent to is replaced by  (so that , , and the
's are adjacent to the degree 2 vertex of ); also, for each j,
, we add a vertex  which is adjacent to
 and  as above and a vertex  adjacent to
 and   (thus turning the  induced by
 into a ).

We begin by describing the structure of the -representation
of the variable gadget. Notice that the vertex  is adjacent to
four 's. Thus, as we have observed,  will have four
external neighbors in any -representation of .
Furthermore, since the neighborhood of  is a stable set, the
vertices  are all internal neighbors of . Without loss of
generality, we will assume that  is represented by an . Finally,
suppose that  is an internal horizontal neighbor of . When 
is  or ,
since  is a ,
which can only be represented as a frame,
 is necessarily an internal vertical neighbor of
. For ,
 is ,
with bipartition  and . In an
-representation of , only a vertex of the size 2 partite set
can have two internal horizontal neighbors.  So  is necessarily an
internal vertical neighbor of . For ,
 is  where
 is the vertex of degree 2.  In an -representation of , only
the neighbors of the degree 2 vertex can have two internal horizontal neighbors.
So again,  is necessarily an
internal vertical neighbor of . Similarly, if  were to be an internal vertical neighbor
of ,  would necessarily be an internal
horizontal neighbor of  (\footnote{We will later use the
location (i.e., as an internal horizontal or internal vertical
neighbor of ) as a variable's truth value.}).

From these
observations, in Figure \ref{fig:var-gadget}, for ,
we depict the general structure of a
-representation of the subgraph of  induced by
     
    and the 's adjacent to
these vertices.

Now, w.l.o.g., suppose that  is an internal horizontal neighbor
of . Notice that  is adjacent to two 's, that  is an internal
horizontal neighbor of , and that the neighborhoods of  and 
are disjoint. Thus, since the neighborhood of  is a stable set,
the vertices  are internal vertical neighbors of
. Similarly, for each ,  is an internal
horizontal neighbor of  since each  is an internal
vertical neighbor of  and each  is adjacent to two 's.
These observations provide the general structure of a
-representation of the subgraph of  induced by
        and the
's adjacent to these vertices (as seen in Figure
\ref{fig:var-gadget} for ).
\begin{figure}[h]
\hfill
\includegraphics[scale=1]{variable-gadget-model1}
\hfill
\includegraphics[scale=1]{variable-gadget-model2}
\hfill \ 
\caption{Left: The possible -representations of 
induced by      
    and the 's adjacent to
these vertices (note:  and
, and  is a
permutation on ). Right: The possible
-representations of  induced by   
     and the 's adjacent to
these vertices (note:  is a permutation on ).} \label{fig:var-gadget}
\end{figure}



With the restricted structure of the variable gadget in mind, we
now turn our attention to the clause gadget of a clause . Notice that  is
a clique (i.e.,  have pairwise edge-intersections
in any -representation of ).
Furthermore, ,  and  intersect
's vertical only since  is an internal horizontal
neighbor of . Only the vertical with the highest top-point
and the vertical with the lowest bottom-point are not contained
in the union of the other three verticals.  W.l.o.g., assume that the
vertical of  is contained in the union of the verticals of
,  and . Since , 
and  are not adjacent to , paths  and
 must not intersect in a vertical grid edge, but rather in a
horizontal grid edge (see Figure \ref{fig:clause-gadget}).
Additionally, observe that, when
\footnote{Remember,  is some  or
 ().} is an internal vertical
neighbor of ,  is necessarily an internal horizontal
neighbor of  since  is adjacent to two 's.
Similarly, when  is an internal horizontal neighbor of
,  is an internal vertical neighbor of . However,
 cannot be an internal horizontal neighbor of  since
 is not adjacent to  and  and  have
a horizontal grid edge in common. Thus, it is not possible for all
three literals to be internal vertical neighbors of . On the
other hand, when at most two literals are internal vertical
neighbors of , we can always construct the
-representation of the clause gadget. In particular, this
can be done using one of the three templates depicted in Figure
\ref{fig:clause-gadget}. Note, to form an -representation
of , the placement of the -representations of the
clause gadgets from Figure \ref{fig:clause-gadget} can be described as
follows:
\begin{itemize}
\item When at most one literal is an
internal vertical neighbor of , (i.e. for type (i) and (ii) of
Figure \ref{fig:var-gadget}), we place the
-representation of the clause gadget ``below''
 and to the ``left'' of  (with
respect to the depiction in Figure \ref{fig:var-gadget}).
\item When two literals  and  are
internal horizontal neighbors of , (i.e. for type (iii)
of Figure \ref{fig:var-gadget}), we need to place the
-representation of the clause gadget ``between''
 and  and to the ``left'' of
 (with respect to the depiction in Figure
\ref{fig:var-gadget}).
\end{itemize}
\begin{figure}[h]
\begin{tabular}{c}
\includegraphics[scale=1]{clause-gadget-model1} \\
(i)
\end{tabular}
\hfill
\begin{tabular}{c}
\includegraphics[scale=1]{clause-gadget-model2} \\
(ii)
\end{tabular}
\hfill
\begin{tabular}{c}
\includegraphics[scale=1]{clause-gadget-model3} \\
(iii)
\end{tabular}
\caption{-representations of the clause gadget for a clause
 inside an -representation of
. (i) ; (ii)  and ; (iii)  and .} \label{fig:clause-gadget}
\end{figure}
We can now see that a literal being an internal vertical neighbor
of  corresponds to when that literal is \emph{false} (since at
most two literals can be internal vertical neighbors of ) and a
literal being an internal horizontal neighbor of  corresponds
to when that literal is \emph{true}. Thus, since  and
 cannot both be internal vertical (or horizontal)
neighbors of , the -representations of 
 correspond to satisfying assignments of .
\end{myproof}




\noindent We conjecture that a similar approach can be used to prove
that recognizing -EPG is NP-hard for . 


\section{Characterization and Recognition of {\em Split}  }
\label{sec:split}
Recall that recognizing chordal EPT graphs is NP-complete \cite{Gol1985}. We have
just shown that recognizing -graphs is NP-complete. Thus,
it is of interest to characterize the class {\em Chordal} . A first step in this direction would be to study {\em
Split} , that is, the class of split
-graphs. We divide this discussion into three parts. In the
first part, we establish some properties of split -graphs.
In the latter two parts, we characterize two special subclasses of
split -graphs.
\subsection{Properties of {\em Split}  }
\label{sec:properties-of-split}
In this section, we will establish some properties of the class
{\em Split} . We conjecture a
characterization of this class. First, we need to introduce a few
definitions.

Recall that  denotes the set of vertices adjacent to vertex .
Vertices  and  are called {\em twins} if either they are non-adjacent
and  or if they are adjacent and .
A vertex  {\em dominates} a vertex  if . The domination relation is reflexive and transitive, but need not be
antisymmetric - twins dominate each other.   Two vertices are
{\em comparable} if one dominates the other. A vertex is called {\em maximal}
if it is not dominated by any other vertex.

Let  be a subset of vertices of . A vertex which belongs
to  is called an {\em -vertex}. We use  to denote the set of
vertices not in  which have at least one neighbor in .  We use   to denote the
subgraph of  induced by the vertices of  which are not in .


We say that an -path lies on a horizonal (vertical) line  if its horizontal (vertical) part
intersects  in a grid edge. An -path  lies on another -path  if part of 
lies on part of . We say an -path  lies above (below) another -path or horizontal
line  if the y-coordinate of the horizontal part of  is greater (less) than the y-coordinate
of the horizontal part of . Lying to the left or right is defined similarly.

A {\em split partition}  of a graph  is a partition of
its vertices into a clique  and a stable set . We will
enumerate the vertices of  as .

Let  be an -graph  with a split partition . Consider an
-representation of  on the grid. It
follows from Lemma \ref{lem:Gol2009} that  corresponds to an
edge-clique. We may assume without loss of generality that the edge of
the grid that belongs to all -paths of  is vertical. The horizontal
parts of -paths of  are called {\em branches}. Let  be the vertical
line-segment which is the union of the vertical parts of all -paths of .
The part of  below the first (top) branch is called the {\em trunk}. The
part of  above the first branch is called the {\em
crown} (see Figure~\ref{fig:L-representation}). All -paths of  contain the
lowest grid-edge of the crown; call this the {\em base} of the crown.





\begin{observation}\label{obs:comparable}
The -vertices whose -paths lie on the same branch (or
on the crown) are pairwise comparable. An -vertex whose -path lies
on the trunk dominates all -vertices whose -paths lie below it
in the representation. \proofBox
\end{observation}
See Figure~\ref{fig:L-representation} for an illustration of
Observation~\ref{obs:comparable}.
\begin{figure}[h]
\hfill
\includegraphics[scale=1]{split-ex-graph}
\hfill
\includegraphics[scale=1]{split-ex-rep}
\hfill \ \caption{A Split  graph (left) and an
-representation of it (right).}\label{fig:L-representation}
\end{figure}


The {\em gem} is the graph with vertices , edges
. The {\em bull} is the graph with vertices
, edges ; vertex  is called the {\em nose}
of the bull.  In a split partition  of the path  on four
vertices, the degree 1 vertices are in  and the degree 2 vertices
are in . It follows that any split partition of the gem has  and 
in  and ,  and  in .  In a split partition of the bull,
 and  are in  and  and  are in , but the
nose  may be in either  or . When the nose is in , the
bull is called an {\em -bull}; that is,

\begin{definition}\label{def:S-bull}
An {\em S-bull} is a bull such that the three vertices of degrees
less than three in the bull are in .
\end{definition}
In Figure~\ref{fig:L-representation},  is an
-bull but  is not an -bull even though it is
a bull. 

Note: in the remainder of this paper, for a graph  with an 
-representation , we will use  to denote the grid path of 
the vertex  of  in .


\begin{observation}\label{obs:gem-on-crown}
Let  be a split graph  with a split partition .
If  admits an -representation and contains a gem, then
exactly one of the gem's -vertices has its -path lying
on the crown of the representation.
\end{observation}
\begin{myproof}
Let the vertices of the gem be , , , ,  with ,
,  , ,   and , , ,
 . Assume that neither  nor  lies on the
crown. Since  and  are incomparable, by
Observation~\ref{obs:comparable}, we may assume  lies on
a branch. Since  is adjacent to ,  must lie on
the vertical segment of  and lie above  in the
representation. By our assumption,  must lie on the trunk.
By Observation~\ref{obs:comparable},  dominates , a
contradiction. Thus, we may assume  lies on the crown.
Since  is incomparable with ,  cannot lie on the
crown.
\end{myproof}
\begin{observation}\label{obs:bull-on-vertical}
Let  be a split graph  with a split partition .
If  admits an -representation and contains an -bull,
then some -vertices of this bull have their paths lying on
either the crown or trunk of the representation. \proofBox
\end{observation}
\begin{observation}\label{obs:universal}
Let  be a split graph  with a split partition .
Suppose there is a vertex  in  with . Then
 is an -graph if and only if  is. 
\end{observation}
\begin{myproof}
Note that  has no neighbor in .  Suppose 
has an -representation.  All -paths of vertices of 
contain the base of the crown. We can place  so the that it lies at
the top of the base of the crown -- and if necessary, move paths of 
on the crown up -- to obtain a representation of . Note that
no -vertices were placed on the trunk since  is inserted 
between the base of the crown and the crown without its base. 
\end{myproof}

\noindent \textbf{Remark:} ``Moving an -path up" in an -representation
may require inserting a row into the grid since -paths start and end
at vertices of the grid.
\begin{observation}\label{obs:twin}
Let  be a split graph  with a split partition .
Suppose  contains  twins  and . Then  is an -graph
if and only if  is. 
\end{observation}
\begin{myproof}
Suppose  is adjacent to . Suppose further that  is in .  Then 
is in  and it follows that  is adjacent to all vertices of
. So, we are done by Observation~\ref{obs:universal}. Thus,
we can assume that  both  and  are in . Consider an -representation of
. By making  an exact copy of , we obtain a
representation for .

Now assume  is not adjacent to  . Suppose both 
and  are in . Consider an -representation of .
Then  lies on a branch, on the trunk, or on the crown.
We can assume  does not lie on both a branch and the crown or trunk
by moving it up if necessary.  By placing 
so it lies next to  on the branch (or on the trunk, or on the crown, respectively)
that  lies on, so that  intersects the same -paths that  does,
we obtain a representation for .
Now, we may assume  is in  and  is in
. It follows that  has no neighbor in . But then we are
done by Observation~\ref{obs:universal}.
\end{myproof}
\begin{observation}\label{obs:threshold}
Let  be a split graph  with a split partition .
Suppose there is a subset  of  such that the vertices of
 are pairwise comparable and .
Then  is an -graph if and only if  is.
Further,  can be constructed from  so that no -vertex
is placed on the trunk.
\end{observation}
\begin{myproof}
Suppose
there is an -representation of . Vertices of
 will be represented by -paths starting with the base 
of the crown so that they all have the same bend-point,
just below the first (highest) branch.
Recall that the -paths of  all contain the base  of the crown.
We can move the -paths of the -vertices which lie on the crown up so they
do not intersect with the vertical parts of the paths of . We can place the
paths corresponding to vertices  of  so that they lie on this new branch (and
thus not on the trunk).
\end{myproof}

\begin{observation}\label{obs:neighbor-degree-one}
Let  be a split graph  with a split partition . Suppose
some vertex  is such that all of its neighbors in 
have degree one. Then  is an -graph if and only if  is. \proofBox
\end{observation}
\begin{observation}\label{obs:gem-comparable}
Let  be a gem-free graph with a split partition . Then
any two vertices of  with a common neighbor in  are
comparable. \proofBox
\end{observation}
\begin{observation}\label{obs:gem-domination}
Let  be a gem-free graph with a split partition . Let
 be a maximal vertex in  and  be a vertex in  with a
common neighbor with . Then  dominates . 
\proofBox
\end{observation}

Consider the nine graphs shown in
Figure~\ref{fig:characterization}. We believe that they are the
only minimal forbidden obstructions for a split graph to be an
-graph. We pose this as a conjecture.

\begin{conjecture}
A split graph is an -graph if and only if it does not contain any of
the nine graphs in Figure~\ref{fig:characterization} as an induced
subgraph.
\end{conjecture}
Theorems~\ref{thm:bull-free}~and~\ref{thm:gem-free} (proved in the
next sections) can be seen as first steps in this direction.

\begin{figure}
\centering \hfill
\begin{tabular}{c}
\includegraphics[scale=1]{U1} \\ 
\end{tabular}
\hfill
\begin{tabular}{c}
\includegraphics[scale=1]{U2} \\ 
\end{tabular}
\hfill \
\par
\begin{tabular}{c}
\includegraphics[scale=1]{S1} \\ 
\end{tabular}
\hfill
\begin{tabular}{c}
\includegraphics[scale=1]{S2} \\ 
\end{tabular}
\hfill
\begin{tabular}{c}
\includegraphics[scale=1]{S3} \\ 
\end{tabular}
\hfill
\begin{tabular}{c}
\includegraphics[scale=1]{S4} \\ 
\end{tabular}
\hfill
\begin{tabular}{c}
\includegraphics[scale=1]{S5} \\ 
\end{tabular}
\hfill
\begin{tabular}{c}
\includegraphics[scale=1]{S6} \\ 
\end{tabular}
\hfill
\begin{tabular}{c}
\includegraphics[scale=1]{S7} \\ 
\end{tabular}
\caption{In  and , the vertex  is adjacent to all
remaining vertices.}\label{fig:characterization}
\end{figure}
\begin{lemma}\label{lem:9graphs}
None of the nine graphs shown in Figure~\ref{fig:characterization}
is an -graph.
\end{lemma}
\begin{myproof}
By Lemma~\ref{lem:AT}, the graphs  and  do not
admit -representations.

Consider the graph  with the
split partition  where , , and  with the
subscripts taken modulo 3. Each pair of -vertices is in a gem.
Observation~\ref{obs:gem-on-crown} says that in an -representation
of a gem, exactly one of its two -vertices lies on the crown.  This
is not possible. So,  is not an -graph. Similarly,
Observations~\ref{obs:gem-on-crown} and
\ref{obs:comparable} show that , , and
 are not -graphs.

Consider the graph . Suppose
 admits an -representation. Let  be the
three -bulls of . By
Observation~\ref{obs:bull-on-vertical}, each  contains an
-vertex  such that  lies on the trunk or crown.
Without loss of generality, we may assume the trunk contains 
and . The fact that  is incomparable with 
contradicts Observation~\ref{obs:comparable}. Similar arguments
show that  and  are not -graphs.

Finally, it is a
routine but tedious matter to show that all proper induced
subgraphs of the graphs in Figure~\ref{fig:characterization} are
-graphs.
\end{myproof}


The {\em k-sun} () is the graph obtained by taking a
cycle on  vertices and joining every pair of odd-indexed vertices by an edge.
So, a 3-sun is the graph , a 4-sun is the graph , and
 occurs in any -sun with . A graph is {\em
strongly chordal} if it is chordal and contains no -sun. The
following follows from Lemma~\ref{lem:9graphs}.
\begin{observation}
Chordal   = Strongly Chordal . \proofBox
\end{observation}
\subsection{Split graphs without S-bulls}
In this section, we give a characterization by forbidden induced
subgraphs of split -graphs without -bulls. This provides
a polytime algorithm for recognizing split -graphs without
-bulls.
\begin{observation}\label{obs:bull}
Let  be two incomparable vertices in . If  does
not contain an -bull, then no vertex  is adjacent to
some vertex  of  and to some vertex  of
.
\end{observation}
\begin{myproof}
If such a vertex  exists, then  induces a -bull.
\end{myproof}
\begin{theorem}\label{thm:S-bull-gem-free}
All -bull-free, gem-free split graphs are -graphs.
\end{theorem}
\begin{myproof}
By induction on the number of vertices. Let  be a graph with a
split partition  and with no -bull. Let  be a maximal -vertex.
If , we are done by Observation~\ref{obs:universal}. So
assume . Let  be the set of -vertices which have a neighbor
in common with . Since  is gem-free, by Observation~\ref{obs:gem-domination},
 dominates all vertices in . So . Since  is -bull
free, by Observation~\ref{obs:bull}, the vertices of  are pairwise-comparable.
Let . Then , since if vertex 
has a neighbor , then since , vertices  and  have
common neighbor , so . By the induction hypothesis,
 is an -graph.  By Observation~\ref{obs:threshold}, 
is an -graph.
\end{myproof}

\begin{theorem}\label{thm:bull-free}
Let  be a graph with a split partition  and with no
-bull. Then  admits an -representation if and only if  does
not contain  or  as an induced subgraph.
\end{theorem}
\begin{myproof}
By induction on the number of vertices. We
only need to prove the ``if'' part. Let  be a graph with a
split partition  and with no -bull, , or .
If  has no gem, the result follows from Theorem~\ref{thm:S-bull-gem-free}.
So we assume that  contains a gem; that is,
there are two incomparable -vertices with a common
neighbor. Let  be two incomparable -vertices with a
common neighbor such that  is largest, where
 denotes the degree of vertex . Define .
The following two facts are easy to establish.

Suppose  is incomparable
to both  and . Vertex  has no neighbors in , for otherwise it can be shown that  contains
an -bull or . Without loss of generality, we may assume
 has a neighbor  in . Now, there is a
-bull with vertices , and some . We have established (\ref{e:domination}).



Consider a vertex
 with a neighbor in . Suppose  has a neighbor
. By (\ref{e:domination}), we may assume  is
comparable to . The existence of  implies  dominates
. It follows that  is comparable to , for
otherwise, , contradicting our choice of  and .
Thus,  dominates . By Observation~\ref{obs:bull}, with
 and ,  contains an -bull, a
contradiction. So, we have . By
Observation~\ref{obs:bull},  has either no neighbor in  or no neighbor in . Thus,
(\ref{e:neighborhood-inclusion}) is established.

The paths of  will lie on the
crown and first branch. The paths of  will lie on
branches below that first branch. By
(\ref{e:neighborhood-inclusion}), the vertices of  
can be partitioned into two sets  and  such that 
is in  and dominates every vertex in . Now, we
claim that

If some two vertices  are incomparable, then by
Observation~\ref{obs:bull},  contains an -bull. So,
(\ref{e:Di-comparable}) holds.

It follows that the vertices of
 are pairwise comparable in the subgraph of  induced by
  (and in the subgraph of  induced by ).
Vertices of  will be
represented by -paths with the same bend-point. Place the
paths representing  so they lie on the crown with  being above 
if  is dominated by . (If two vertices dominate each other,
place one so that it lies above the other.) Place the paths representing 
so they lie on
the first branch with  to the right of  if  is
dominated by . (If two vertices dominate each other,
place one so that it lies to the right of the other.) For any two vertices  of  (respectively,
), if  dominates  in (respectively, ), then every
-path of a -vertex must pass through an edge of  to
reach . This completes the description of the representation
of .



Define . By (\ref{e:neighborhood-inclusion}), there
is no vertex in  with a neighbor in  and one in . The
set  contains no gem, for otherwise, 
contains .
It follows from Observations~\ref{obs:gem-comparable}
and~\ref{obs:gem-domination} that the set  can be partitioned
into sets  () such that, for each
, the vertices in  are pairwise comparable, and
no -vertex has a neighbor in  and one in , for  (in particular, for each , there is a maximal
-vertex  with ). Define . By the induction hypothesis,  is an
-graph. By Observation~\ref{obs:threshold},  is  an
-graph. 
\end{myproof}

We note that a polytime algorithm to construct an
-representation for the input graph can be extracted from
the proofs above. The algorithm is certifying in the sense that it
produces either an -representation, or an obstruction.

\subsection{Split graphs without gems}
In this section, we give a characterization by forbidden induced
subgraphs of split -graphs without gems. This provides a
polytime algorithm for recognizing split -graphs without
gems. First, we need to introduce a definition.

\begin{figure}[h]
\centering
\includegraphics[scale=1]{G8}
\caption{The graph . Note: this graph contains two disjoint -bulls.}
\label{fig:G8}
\end{figure}

\begin{lemma}\label{lem:two-bull}
Let  be a gem-free graph with a split partition .
Suppose  does not contain the graph  of Figure \ref{fig:G8}
as an induced subgraph. Then,
there is an -representation of  with no -vertices
having their -paths lying on the trunk. 
\end{lemma}
\begin{myproof}
By induction on the number of vertices.

We can assume that no vertex  has  by applying induction and
Observation~\ref{obs:universal}.

It follows that every vertex  has a neighbor in  and that
if  then . We can also assume that no
-vertex is isolated.


We partition the vertices of  in the following way.  Let  be a
maximal -vertex, let  be the set of -vertices with a neighbor
in common with , and let . In general, for each , ,
let  be a maximal vertex in ,
let  be the set of -vertices which have a neighbor in common with ,
and let . By definition of , for each i,
. Since  is gem-free, by Observation~\ref{obs:gem-domination},
for each ,  dominates all vertices in , so .
Since , there are at least two sets  (that is, .
Since  does not contain , at least one of the subgraphs induced by
  (say, ), does not contain an -bull. Then by
Observation~\ref{obs:bull}, the vertices of  are pairwise comparable.  By
the induction hypothesis,  admits an -representation
with no paths representing the vertices of  lying on the trunk.  Then
by Observation~\ref{obs:threshold},  has an -representation with no paths
representing -vertices lying on the trunk.
\end{myproof}
\begin{theorem}\label{thm:gem-free}
Let  be a gem-free graph with a split partition . Then
 admits an -representation if and only if  does not contain
 as an induced subgraph.
\end{theorem}
\begin{myproof}
By induction on the number of vertices. We
only need to prove the ``if'' part. Let  be a gem-free graph
with a split partition  and not containing .
As in the proof of Lemma~\ref{lem:two-bull}, we can assume that
every vertex  has a neighbor in  and that if ,
then . We can also assume that no
-vertex is isolated.
Define  and  as in the proof of Lemma~\ref{lem:two-bull}.
Then for all , . Since  is gem-free,
 dominates all vertices in , and so . If
the vertices of some  are pairwise comparable, then we are done
by the induction hypothesis and Observation~\ref{obs:threshold}.
Therefore, for each ,  must contain two incomparable vertices,
that is, the subgraph  must contain an -bull.  Since
 does not contain , it follows that  and also that
 does not contain .   By
Lemma~\ref{lem:two-bull}, there is an -representation of
 with no vertices of  on the trunk. By the induction
hypothesis, the graph  has an
-representation. We place the branches of  under those
of of  and extend the vertical segments of the paths of
 to the crown of . The adjacency of  is preserved
because  has no -vertices on the trunk in the
-representation.
\end{myproof}
\section{Concluding Remarks and Open Problems}\label{sec:conclusion}
In this paper, we considered the edge intersection graphs of
-shaped paths on a grid. We showed that recognizing such
graphs is NP-complete. We considered the open problem of
characterizing chordal -graphs. As first steps in solving
this problem, we found characterizations of split gem-free
-graphs and split -graphs without -bulls (a class
more general than split bull-free -graphs). Our
characterizations imply polytime algorithms for recognizing these
two classes of graphs. We posed a conjecture on the
characterization of split -graphs. This conjecture would
imply a polytime recognition algorithm for split -graphs. The
following open problems related to our works arise: (1) Extending
the observations in Section~\ref{sec:split} to other subclasses of
-EPG graphs; (2) Find a polytime algorithm for recognizing {\em Chordal}
; (3) Establish NP-completeness of recognizing
-EPG graphs for every  at least 2.

\section{Acknowledgements}
This research was supported by the Natural Sciences and Engineering Research Council of Canada. 
Additionally, Steven Chaplick was partially supported by the ESF GraDR EUROGIGA grant as project GACR GIG/11/E023.




\bibliographystyle{acm}
\bibliography{B1-EPG}

























\end{document}
