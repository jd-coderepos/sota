\documentclass[conference]{IEEEtran}

\usepackage{graphicx} \usepackage{epstopdf}  \usepackage{amssymb} \usepackage{amsmath} \usepackage{amsthm} \usepackage{algorithm} \usepackage{algorithmic} \usepackage{subfigure} \usepackage{url}

\newtheorem{definition}{Definition} \newtheorem{theorem}{Theorem} \newtheorem{remark}{Remark} \newtheorem{corollary}{Corollary}

\graphicspath{{./}}

\begin{document}

\title{Randomization Improving Online Time-Sensitive Revenue Maximization for Green Data Centers}

\author{\IEEEauthorblockN{Huangxin Wang}
\IEEEauthorblockA{
George Mason University\\
Fairfax, VA 22030\\
hwang14@gmu.edu}
\and
\IEEEauthorblockN{Jean X. Zhang}
\IEEEauthorblockA{
Virginia Commonwealth University\\
Richmond, VA 23284\\
jxzhang@vcu.edu}
\and
\IEEEauthorblockN{Bo Yang}
\IEEEauthorblockA{
Jilin University\\
Changchun, China 130012\\
ybo@jlu.edu.cn}
\and
\IEEEauthorblockN{Fei Li}
\IEEEauthorblockA{
George Mason University\\
Fairfax, VA 22030\\
fli4@gmu.edu}
}
\maketitle



\begin{abstract}
Green data centers have become more and more popular recently due to their sustainability. The resource management module within a green data center, which is in charge of dispatching jobs and scheduling energy, becomes especially critical as it directly affects a center's profit and sustainability. The thrust of managing a green data center's machine and energy resources lies at the uncertainty of incoming job requests and future showing-up green energy supplies. Thus, the decision of scheduling resources has to be made in an online manner. Some heuristic deterministic online algorithms have been proposed in recent literature. In this paper, we consider online algorithms for green data centers and introduce a randomized solution with the objective of maximizing net profit. Competitive analysis is employed to measure online algorithms' theoretical performance. Our algorithm is theoretical-sound and it outperforms the previously known deterministic algorithms in many settings using real traces. To complement our study, optimal offline algorithms are also designed.
\end{abstract}



\section{Introduction}

In this paper, we study the problem of scheduling jobs and energy in data centers. A \emph{data center} is a computing facility used to house computing systems and their associated components such as communication and storage subsystems. Usually, a data center stores data and provides computing functionalities to its customers. Through charging fees for data access and server services, a data center gains revenue~\cite{amazonprice}. At the same time, to maintain its running structure, a data center has to pay \emph{operational costs} including hardware costs (for example, those of upgrading computing and storage devices and air conditioning facilities), electrical bills for power supply, network connection costs, and personnel costs. To maximize a data center's net profit, we expect to increase the revenue gathered and simultaneously decrease the operational costs paid.

Unfortunately, the ever increasing power costs and energy consumption in data centers have brought many serious economic and environmental problems to our society and evoked significant attention recently. As reported, the estimates of annual power costs for U.S. data centers in 2010 reached as high as  billion dollars~\cite{EnergyCost}. As a concrete example, in a modern high-scale data center with 45,000 to 50,000 servers, more than 70\% of its operational cost (around half a billion dollars per year)~\cite{usage} goes to maintaining the servers and providing power supply. The energy spending in data centers in 2014 is \0.861.4412.011.002M \in \mathbb{Z^+}100jr_j \in \mathbb{Z}^+p_j \in \mathbb{Z}^+d_j \in \mathbb{Z}^+q_j \in [1, M]jjq_j \times p_jjr_jd_j - p_j(r_j, p_j, q_j)jj = (r_j, p_j, q_j)s_jjc_jc_j := s_j + p_jjl_jjl_j := \frac{p_j}{c_j - r_j}c_j \ge r_j + p_jv_j :=
\begin{cases}
\0, & \mbox{otherwise}
\end{cases}
\betaL_jL_j \in (0, 1]jv_jj\frac{p_j}{c_j - r_j} \ge L_jd_jjd_j := r_j + \frac{p_j}{L_j}j\ when at on-peak (usually at daytime) and price B^nB^d = \ and 0.08/kWhk1 / kkk := \max \limits_{I} \frac{OPT - \delta}{E[ON]}\deltaE[ON]jp_jq_jc \cdot p_j \cdot q_j - \int_t P(t)P(t)0B^dB^nP(t)j\frac{c \cdot p_j \cdot q_j - \int_t P(t)}{c \cdot p_j \cdot q_j} = 1 - \int_t \frac{P(t)}{c \cdot p_j \cdot q_j}1 - \frac{P(t)}{c \cdot p_j \cdot q_j}v_{on}v_{off}v_gjjj1 - \frac{P(t)}{c \cdot p_j \cdot q_j}(0, 1]00 < v_{on} < v_{off} < v_g = 1v_{on}v_{off}v_g\max \left\lbrace\frac{v_{off}}{v_{on}}, \frac{v_g}{v_{on}}\right\rbracep_j = 1q_j = M(r, d)rdMt_1t_20Mj = (t_1, t_2)jt_1v_{on}jt_2v_g\frac{OPT}{FF} = \frac{v_g}{v_{on}}t_1t_2\frac{OPT}{FF} = \frac{v_{off}}{v_{on}}\max\left\lbrace \frac{v_{off}}{v_{on}}, \frac{v_g}{v_{on}}\right\rbrace\max\left\lbrace 1+\frac{v_{on}}{v_{off}}, 1 + \frac{v_{off}}{v_g}\right\rbracep_j = 1Mt_1t_2t_1t_2j_1 = (t_1, t_2)j_2 = (t_2, t_2)j_1t_2j_1j_2v_{off}j_1j_2t_1t_2v_{on} + v_{off}\frac{OPT}{BF} = 1 + \frac{v_{on}}{v_{off}}t_1t_20Mt_1t_2\frac{OPT}{BF} = 1 + \frac{v_{off}}{v_g}\max\left\lbrace 1 + \frac{v_{on}}{v_{off}}, 1 + \frac{v_{off}}{v_g}\right\rbrace0 < v_{on} < v_{off} < v_g = 122ppppjjjpj1 - pjpc = \max \left\lbrace 1 + \frac{v_{on}}{v_{off}} - \left(\frac{v_{on}}{v_{off}}\right)^2, 1 + \frac{v_{off}}{v_g} - \left(\frac{v_{off}}{v_g}\right)^2 \right\rbracec1.25\frac{OPT}{RF} \le 1.25t_1t_1cc = 1.25t_1tc(r, d)rdt_1t_2j_1 = (t_1, t_2)j_1t_2v_{off}j_1t_1pt_21 - pp \cdot v_{on} + (1 - p) \cdot v_{off}\frac{OPT}{RF} = \frac{v_{off}}{p \cdot v_{on} + (1 - p) \cdot v_{off}}j_1 = (t_1, t_2)j_2 = (t_2, t_2)j_1t_1j_2t_2v_{on} + v_{off}j_1t_1pj_1j_2t_2p \cdot v_{on} + v_{off}\frac{OPT}{RF} = \frac{v_{on} + v_{off}}{p \cdot v_{on} + v_{off}}\min_p \left\{\max\left\{\frac{v_{off}}{p \cdot v_{on} + (1 - p) \cdot v_{off}}, \frac{v_{on} + v_{off} }{p \cdot v_{on} + v_{off}}\right\}\right\}p = \frac{x}{1 + x - x^2}x = \frac{v_{on}}{v_{off}}\frac{OPT}{RF} = 1 + x - x^2 = 1 + \frac{v_{on}}{v_{off}} - \left(\frac{v_{on}}{v_{off}}\right)^2 \le 1.25t_1t_2j_1 = (t_1, t_2)j_1\frac{OPT}{RF} = \frac{v_g}{p \cdot v_{off} + (1 - p) \cdot v_g}j_1 = (t_1, t_2)j_2 = (t_2, t_2)\frac{OPT}{RF} = \frac{v_{off} + v_g}{p \cdot v_{off} + v_g}\min_p \left\lbrace\max\left\lbrace \frac{v_g}{p \cdot v_{off} + (1 - p) \cdot v_g} , \frac{v_g + v_{off} }{p \cdot v_{off} + v_g}\right\rbrace \right\rbracep = \frac{y}{1 + y - y^2}y = \frac{v_{off}}{v_g}0 < y < 1\frac{OPT}{RF} = 1 + y - y^2 = 1 + \frac{v_{off}}{v_g} - \left(\frac{v_{off}}{v_g}\right)^2  \leq 1.25\max \left\lbrace\frac{v_{off}}{v_{on}}, \frac{v_g}{v_{on}}\right\rbrace\max\left\lbrace 1 + \frac{v_{on}}{v_{off}}, 1 + \frac{v_{off}}{v_g}\right\rbrace\max\left\{1 + \frac{v_{on}}{v_{off}} - \left(\frac{v_{on}}{v_{off}}\right)^2, 1 + \frac{v_{off}}{v_g} - \left(\frac{v_{off}}{v_g}\right)^2\right\}\frac{OPT}{RF} < \frac{OPT}{BF}1 + k - k^2 < 1 / k0 < k <11 + \frac{v_{on}}{v_{off}} - \left(\frac{v_{on}}{v_{off}}\right)^{2} < \frac{v_{off}}{v_{on}}1 + \frac{v_{off}}{v_g} - \left(\frac{v_{off}}{v_g}\right)^2  < \frac{v_g}{v_{off}}\frac{OPT}{RF} < \frac{OPT}{FF}\begin{cases}
p = \frac{x}{1 + x - x^2}, & x = \frac{v_{on}}{v_{off}}\\
p' = \frac{y}{1 + y - y^2}, & y = \frac{v_{off}}{v_g}
\end{cases}\alphac\alpha1 / c\alphac\alpha\max\{ \frac{v_g}{v_{on}}, 1 + \frac{v_{on}}{v_{off}}, 1 + \frac{v_{off}}{v_g}\}100140550.13/kWh0.08/kWh\ per machine.

\paragraph*{Workloads}

We use real workload traces \emph{Grid5k} as the workload input in our simulation. Grid5k~\cite{Grid5k} is a real workload trace which was collected from Grid'5000 system~\cite{Grid5000Platform}, a- node experimental grid platform consisting of 9 sites geographically distributed in France, from May 2004 to November 2006.

We randomly select 2 five-day-period workloads, denoted as \emph{Grid5k-1} and \emph{Grid5k-2} as the workload input in the simulation. Note that in order to simulate various workload utilization, we random sample jobs to create simulation workloads. Also, the job processing time and node requirements are re-scaled to meet the size of the simulated data center.







\subsection{Methodology}

We evaluate the performance of the online algorithm under various types of workloads and various value of \emph{least service quality} . We set the workload utilization range from  to . Note that as Random-Fit has its randomness factor internal to the algorithm, we do not need to tune its randomness. Each simulation is repeated for  times and we compare the average values. To evaluate the algorithms, we conduct large scale simulations ( machine nodes) to thoroughly compare the performance of the online algorithms. Due to the high running time demand of the offline algorithm, we simulate with relative smaller scale parameters ( machine nodes) when compare the online algorithms with the optimal offline algorithm.



\subsection{Result and analysis}

We first present the evaluation of the online algorithms Random-Fit, First-Fit, Best-Fit and GreenSlot under various settings. Then we show the comparison of the online algorithms with the optimal offline algorithm to confirm with the theoretical competitive ratio analysis.

\subsubsection{Comparison of online algorithms}


















We compare the online algorithm on the profits they achieve. In order to compare the competitive ratio of the online algorithms, we normalize the profits of each algorithm by the best-performed algorithm under each setting as detailed below.


First, we set the most profitable algorithm at each setting (under various workload utilizations and least service quality  values) as an optimal performance . Then we compute the lower bound of competitive ratio using  where  is the net profit gained by an online algorithm. As  is usually lower than the true optimum, therefore, the competitive ratio derived is only a lower bound of the real competitive ratios. It is fair enough to show that our designed  algorithm has better worst-case competitive ratios than First-Fit, Best-Fit, and GreenSlot.



\begin{figure}
  \centering
  \subfigure[Grid5k-1]{\includegraphics[width=.26\textwidth, clip = true, trim = {0mm 0mm 20mm 5mm}]{/Com_CR_Profits_Grid5kOne_Uniform_L_20.pdf}}
\subfigure[Grid5k-2]{\includegraphics[width=.26\textwidth, clip = true, trim = {0mm 0mm 20mm 5mm}]{/Com_CR_Profits_Grid5kTwo_Uniform_L_20.pdf}}
\caption{Lower bounds of competitive ratio  under different workloads \newline with }
\label{fig:prfits_0.2}
\end{figure}


\begin{figure}
  \centering
  \subfigure[Grid5k-1]{\includegraphics[width=.26\textwidth, clip = true, trim = {0mm 0mm 20mm 5mm}]{/Com_CR_Profits_Grid5kOne_Uniform_L_5.pdf}}
\subfigure[Grid5k-2]{\includegraphics[width=.26\textwidth, clip = true, trim = {0mm 0mm 20mm 5mm}]{/Com_CR_Profits_Grid5kTwo_Uniform_L_5.pdf}}
\caption{Lower bound of competitive ratio under different workloads \newline with }
\label{fig:prfits_0.05}
\end{figure}






Figure~\ref{fig:prfits_0.2} and Figure~\ref{fig:prfits_0.05} show the lowered bound of competitive ratios of the algorithms under various workloads settings and with least service quality  as  and  respectively. From these figures, we observe that Best-Fit tends to gain a better profit when the data center utilization is lower than , while First-Fit is better when the data center utilization is higher than about . In whatever data center utilization, our proposed algorithms always guarantee a better worst-case performance. Note that an online algorithm cannot predict precisely a data center's long-time utilization at both fine-grained and coarse-grained levels. Therefore, alternatively employing the two algorithms First-Fit and Best-Fit cannot achieve a better worst-case performance than Random-Fit.

Best-Fit is less profitable when the data utilization is high because Best-Fit tends to delay scheduling jobs in order to consume less expensive energy. This delayed scheduling behavior results in many jobs missing their deadlines and thus achieving a lower profit. While First-Fit always schedules jobs to the first available time slots thus it could schedule more jobs than other algorithms. In the simulation, we observe First-Fit schedules  more workloads then Best-Fit and GreenSlot, and around  more workloads than Random-Fit with moderate workload (has utilization ). But it cannot make a good use of green energy when the data center is of low utilization. Its green energy utilization is less than  of that of Best-Fit when workload utilization is around . While Random-Fit can strike a balance between the amount of workload scheduled and the amount of green energy consumed, and thus tends to have better competitive ratio.

Taking the above analysis one step further, we conclude that if the data center utilization is predictable, then an adaptive scheduling algorithm which dynamically switches between Best-Fit and First-Fit according to the data center's utilization in a long-enough scheduling window would have a better performance. However, the data center utilization is usually hard to be predicted~\cite{MeisnerW10}.

In the simulation, we also find GreenSlot is sensitive to the value of the least service quality . It has performance very close to Best-Fit when  is relatively small, i.e., the job span is relatively large. It is because the penalty of delaying scheduling jobs will not be effective when the jobs have relatively small least service quality, as the penalty will be imposed only when a job is about to miss its deadline (for example, 20\% of its required processing time ahead of its deadline).

The running time of these algorithms in scheduling a job is in the order of several milliseconds which is negligible compare to the job's processing time, usually at several minutes or hours. In specific, First-Fit runs fastest, Random-Fit is the second, while GreenSlot and Best-Fit almost have the same running time.

Based on our simulation results, we remark that Random-Fit is the best algorithm (in terms of competitive ratio and profit maximization).

\subsubsection{Comparison with offline algorithm.}

We further conduct simulations to confirm with the theoretic result that Random-Fit has a better worst-case competitive ratio when jobs are of the same lengths and sizes. We implement an optimal offline algorithm to show the real experimental competitive ratios. The offline algorithm is formulated using a binary integer program and it is run by the LINDO solver. Note that the optimal algorithm is very time consuming, thus we shrink the nodes in the data center from  to  in order to get the optimal result within a reasonable time.

In the simulation, we simulate  uniform workloads with utilization  and  respectively. We compare the online algorithms against the optimal offline algorithm using competitive ratio. For ease of presentation, we abbreviate the algorithms First-Fit, Best-Fit, Random-Fit, GreenSlot and offline optimal as: FF, BF, RF, GS and OPT respectively.



\begin{table}[!ht]
\centering
\caption{competitive ratio of online algorithms}
\begin{tabular}{|l|l|l|l|l|l|l|}  \hline
matrix & FF & BF & RF & GS \\ \hline \hline
competitive ratio (workload = 10\%) &1.56&	1.03&	1.16&	1.03\\ \hline
competitive ratio (workload = 100\%)      &   1.05 &   1.29   & 1.24 &   1.27   \\ \hline
\end{tabular}
\label{tb_com_opt_20}
\end{table}



Table~\ref{tb_com_opt_20} shows the competitive ratio of various online algorithm under different workload utilization. We conclude that First-Fit, Best-Fit and GreenSlot have competitive ratios worse than the theoretical upper bound () of Random-Fit. This conclusion confirms our theoretical results.



\section{Conclusions}

In this work, we study online scheduling of energy and jobs in green data centers with the objective of maximizing net profit. In our problem setting, energy costs are time-sensitive and so is the net profit. Prior work employs deterministic approaches only and the underlying algorithmic ideas are either First-Fit or Best-Fit; furthermore no theoretical analysis has been given. In this paper, competitive analysis is used to measure an online algorithm's theoretical performance. We conclude that randomness plays an important role in maximizing net profit. Experiments on real workload traces have shown that our algorithm indeed outperforms the previous ones, as what the theory indicates.



\bibliographystyle{IEEEtran}
\bibliography{greenSlot-short}



\appendix
\renewcommand{\thesubsection}{\Alph{subsection}}



\subsection{Offline Algorithm}
\label{Appendix_offline}
We formulate a linear program for the special cases when jobs have same processing times and node requirements. Let  denotes the amount of green energy arrive at time  and let  denotes the unit brown energy price at time . Assume all jobs have the same processing time slots  and node requirement . Let  denote the revenue earned by scheduling job . Let  be an indicator variable indicates whether a job is scheduled () or not (). Let  be an indicator variable denotes whether job  is started at time  () or not (). Let  denotes the number of jobs started at time . Let  denotes the energy demand at time .

We have the following formulation.



\subsection{Hardness of the Problem GDC-RM}
\label{appendix_GDC-RM_hardness}
Note that GDC-RM essentially is not an offline problem since the jobs and the green energy cannot be modeled and predicted precisely at all the time. However, understanding the hardness of the offline version may be useful to us in evaluating an online algorithm's theoretical and empirical performance. We prove that the offline version of GDC-RM is NP-hard, using a reduction from the well-known NP-hard problem `Knapsack'~\cite{GareyJ79}.

\begin{theorem}
The offline version of the problem GDC-RM is NP-hard.
\label{thm:nphard}
\end{theorem}

\begin{IEEEproof}
Given a candidate solution, it takes polynomial-time for us to verify whether this solution is feasibly scheduled or not. Thus, the problem GDC-RM belongs to NP. In the following, we prove that GDC-RM is NP-hard by showing a polynomial-time reduction from the Knapsack problem to it. In the Knapsack problem, there are a knapsack of capacity  and  items with each one has size . The goal is to make the knapsack as full as possible. The Knapsack problem is known NP-hard~\cite{GareyJ79}.

Consider the problem GDC-RM. Assume the produced green energy has a budget of  in a scheduling window and the brown energy's costs ( and ) are high enough such that any use of brown energy makes no positive net profit at all. Therefore, to maximize the net profit, we would like to find a set of jobs such that these jobs consume as much as close to but no more than the green energy budget  without using any amount of the brown energy. Particularly, we restrict that the green energy is available within a scheduling window  and all jobs  have the same release time  and (maybe different) deadlines   to ensure the same service qualities  ---  and  are the boundaries of this scheduling window and  is the latest time where green energy is still available. Let . Also, we restrict that each job  has . This conversion takes linear time of the number of jobs.

If we have a polynomial-time optimal solution to the problem GDC-RM with the special input instance as created as in the above, then we have an optimal solution to the following Knapsack problem: The knapsack has its capacity of  and each item  has its size of . As the Knapsack problem is NP-hard, then the problem GDC-RM is NP-hard.
\end{IEEEproof}











\end{document}
