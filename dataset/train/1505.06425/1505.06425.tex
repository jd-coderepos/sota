

\documentclass{jtacs}


\usepackage{arydshln}
\usepackage{layout}
\usepackage{amsmath}
\usepackage{textcomp}
\usepackage{adjustbox}
\usepackage{graphicx}
\usepackage{placeins}

\DeclareMathOperator*{\oplusplus}{\oplus}

\newtheorem{thm}{Theorem}[section]
 \newtheorem{cor}[thm]{Corollary}
 \newtheorem{lem}[thm]{Lemma}
 \newtheorem{prop}[thm]{Proposition}
\newtheorem{defn}[thm]{Definition}
\newtheorem{rem}[thm]{Remark}
\numberwithin{equation}{section}

\begin{document}



\title{An algorithm for multipication of Kaluza numbers}
\maketitle
\xauthor{A.~Cariow}

\xaffiliation{West Pomeranian University of Technology, Szczecin, \\
Faculty of Computer Science and Information Technology, \\
\.{Z}o\l{}nierska 49, 71-210 Szczecin, Poland}

\xemail{acariow@wi.zut.edu.pl}

\xauthor{G.~Cariowa}
\xaffiliation{West Pomeranian University of Technology, Szczecin, \\
Faculty of Computer Science and Information Technology, \\
\.{Z}o\l{}nierska 49, 71-210 Szczecin, Poland}
\xemail{gcariowa@wi.zut.edu.pl}

\xauthor{R.~\L{}entek}
\xaffiliation{West Pomeranian University of Technology, Szczecin, \\
Faculty of Computer Science and Information Technology, \\
\.{Z}o\l{}nierska 49, 71-210 Szczecin, Poland}
\xemail{rlentek@wi.zut.edu.pl}





\begin{abstract}
This paper presents the derivation of a new algorithm for multiplying of two Kaluza numbers.
Performing this operation directly requires 1024 real multiplications and 992 real additions.  The proposed algorithm can compute the same result with only 512 real multiplications and 576 real additions. The derivation of our algorithm is based on utilizing the fact that multiplication of two Kaluza numbers can be expressed as a matrix–vector product. The matrix multiplicand that participates in the product calculating has unique structural properties. Namely exploitation of these specific properties leads to significant reducing of the complexity of Kaluza numbers multiplication.

\end{abstract}

\begin{keywords}
Kaluza numbers, multiplication of hypercomplex numbers, fast algorithms
\end{keywords}






\section{Introduction}

Today hypercomplex numbers \cite{1} are used in various fields of data processing including digital signal and image processing, machine graphics, telecommunications and cryptography \cite{2,3,4,5,6,7,8,9,10}. The most popular are quaternions, octonions and sedenions \cite{1}. Perhaps the less popular are the Pauli, Dirac and Kaluza numbers. This numbers are mostly used in solving different physical problems in electrodynamics, field
theory, etc. Somehow or other, hypercomplex number systems and their applications to data processing are beautiful enough to be worth studying simply for the pleasure of it.

The multiplication of two hypercomplex numbers is often one of the more time consuming operations. The reason for this is, because the addition of -dimensional hypercomplex
numbers requires  real additions, while the multiplication of these numbers already
requires  real additions and  real multiplication. It is easy to see that the increasing
of dimensions of hypercomplex number increases the computational complexity of the
multiplication. Therefore, reducing the computational complexity of the multiplication of
hypercomplex numbers is an important scientific and engineering problem.

Efficient algorithms for the multiplication of various hypercomplex numbers already exist \cite{12,13,14,15,16,17,18,19,20,21,22,23,24}. No such algorithms for the multiplication of Kaluza numbers have been proposed. The aim of the present paper is to suggest an efficient algorithm for this purpose.

\section{Preliminary Remarks}

A Kaluza number is defined as follows:


\begin{adjustbox}{margin = 0 10 0 10}
\\
\end{adjustbox}
where   and  ,  are real numbers, and
,  are the imaginary units. The results of all possible products of Kaluza numbers imaginary units can be summarized in the following table:

\begin{adjustbox}{margin = 0 20 0 25}
\\
\end{adjustbox}

\begin{table}[h]
\caption{North-west quadrant of the table for Kaluza numbers imaginary units multiplication}
\begin{adjustbox}{width=0.99\textwidth, margin = 0 0 0 10}

\end{adjustbox}
\label{tab1}
\end{table}\begin{table}[h]
\caption{North-east quadrant of the table for Kaluza numbers imaginary units multiplication}
\begin{adjustbox}{width=0.99\textwidth}

\end{adjustbox}
\label{tab2}
\end{table}
\begin{table}[h]
\caption{South-west quadrant of the table for Kaluza numbers imaginary units multiplication}
\begin{adjustbox}{width=0.99\textwidth}

\end{adjustbox}
\label{tab3}
\end{table}
\begin{table}[h]\caption{South-east quadrant of the table for Kaluza numbers imaginary units multiplication }
\begin{adjustbox}{width=0.99\textwidth}

\end{adjustbox}
\label{tab4}
\end{table}
\FloatBarrier
Suppose we want to compute the product of two Kaluza numbers.


where



The operation of multiplication of Kaluza numbers can be represented more compactly in
the form of vector-matrix product:

where





\\*
\begin{adjustbox}{width=\textwidth, margin= 3 15 0 5}

\end{adjustbox}
\\*
\begin{adjustbox}{width=\textwidth, margin= 3 15 0 5}

\end{adjustbox}
\\
\\*
\begin{adjustbox}{width=\textwidth, margin= 3 15 0 5}

\end{adjustbox}
\\
\\*
\begin{adjustbox}{width=\textwidth, margin= 3 15 0 5}

\end{adjustbox}


\section{Synthesis of a rationalized algorithm for computing Kaluza numbers
product}

The direct multiplication of two Kaluza requires 1024 real multiplications and 992 real additions. We shall present the algorithm, which reduce arithmetical complexity to 512 real multiplications and 576 real additions.

At first, we rearrange the rows of the matrix in the following order \{1, 2, 3, 7, 5, 9, 4, 8, 6, 10, 11, 17, 13, 19, 15, 21, 12, 18, 14, 20, 16, 22, 23, 27, 25, 29, 24, 28, 26, 30, 31, 32\}. Next, we rearrange the columns of obtained matrix in the same manner. As a result, we obtain the following matrix:

Then we can rewrite expression (2.3) in following form:


where



\begin{center}

\begin{adjustbox}{width=0.55\textwidth, margin= 0 5 0 0}

\end{adjustbox}\end{center}

\begin{center}

\begin{adjustbox}{width=0.55\textwidth, margin= 0 0}

\end{adjustbox}\end{center}

\begin{center}

\begin{adjustbox}{width=0.55\textwidth, margin= 0 0}

\end{adjustbox}\end{center}

\begin{center}

\begin{adjustbox}{width=0.55\textwidth, margin= 0 0}

\end{adjustbox}\end{center}


and



\\*
\begin{adjustbox}{width=\textwidth, margin=  3 15 0 5}

\end{adjustbox}


\\*
\begin{adjustbox}{width=\textwidth, margin= 3 15 0 5}

\end{adjustbox}


\\*
\begin{adjustbox}{width=\textwidth, margin= 3 15 0 5}

\end{adjustbox}

\\*
\begin{adjustbox}{width=\textwidth, margin= 3 15 0 5}

\end{adjustbox}

If we interpret the resulting matrix as a block-matrix, it is easy to see that each block of this matrix (or more precisely, each ()-submatrix) is bisymmetric, i.e. has the properties of pair wise symmetry about both of its main diagonals. There is an effective method of factorization of this type matrices, which allows during the calculation of the vector-matrix products halve the number of multiplications, that is to perform only two, not four multiplications at the expense of triple increasing the number of additions (from two to six) \cite{25,26}:
\begin{center}

\end{center}
where  is the () Hadamard matrix.

Fig. 1. shows a data flow diagram of the rationalized algorithm for () bisymmetric
matrix-vector multiplication in according to (3.2). The circles in this figure show the operation
of multiplication by a value inscribed inside a circle. In turn, the rectangles indicate the
matrix-vector multiplications with matrices inscribed inside rectangles. In this paper the data flow diagrams are oriented from left to right. Straight lines in the figure denote the operation of data transfer. We use the solid lines without any arrows, so as not to clutter up the presented diagrams.

Now it is easy to show that using the above method of ()-bisymmetric matrix factorization
we can construct an efficient algorithm for multiplying Kaluza numbers. In such algorithm the number of real multiplications will be reduced by half  compared with naive method of calculations. However, if we directly use an expansion (3.2) for
each of the submatrices of matrix , the number of real additions will increases dramatically.
In this case, the total value of computation complexity becomes even greater than in
the case of naive method of computations. However, if we utilize the fact that the
multiplication of the () -Hadamard matrix on the corresponding subvector (the fig. 1 is
positioned before of multiplications) is a common operation for all submatrices disposed in
a same vertical of matrix , and a similar operation (in the fig.1, it follows by multiplications) is a common operation for all submatrices disposed in a same horizontal of matrix , the number of real additions can be significantly reduced. Consider the synthesis of an efficient algorithm for multiplying Kaluza numbers in more detail.

\begin{figure}[ht]
\centering
  \includegraphics[width=0.3\textwidth]{D2.pdf}
  \caption{Data flow diagram of the rationalized algorithm for ()-bisymmetric matrix-vector
multiplication in according to (3.2)}
  \label{fig1}
\end{figure}

Let us first introduce some matrices


where  is an identity  matrix, signs ``'' and ``'' denote tensor product and direct sum of two matrices respectively, and   is an integer matrix consisting of all 1s [26],


where

\begin{center}
\begin{adjustbox}{width=0.54\textwidth, margin=0 0 0 10}

\end{adjustbox}
\end{center}

\begin{center}
\begin{adjustbox}{width=0.57\textwidth, margin=0 0 0 0}

\end{adjustbox}
\end{center}

\begin{center}
\begin{adjustbox}{width=0.57\textwidth, margin=0 0 0 0}

\end{adjustbox}
\end{center}

\begin{center}
\begin{adjustbox}{width=0.57\textwidth, margin=0 0 0 0}

\end{adjustbox}
\end{center}

\begin{center}
\begin{adjustbox}{width=0.57\textwidth, margin=0 0 0 0}

\end{adjustbox}
\end{center}

\begin{center}
\begin{adjustbox}{width=0.57\textwidth, margin=0 0 0 0}

\end{adjustbox}
\end{center}

\begin{center}
\begin{adjustbox}{width=0.57\textwidth, margin=0 0 0 0}

\end{adjustbox}
\end{center}

\begin{center}
\begin{adjustbox}{width=0.57\textwidth, margin=0 0 0 0}

\end{adjustbox}
\end{center}

\begin{center}
\begin{adjustbox}{width=0.57\textwidth, margin=0 0 0 0}

\end{adjustbox}
\end{center}

\begin{center}
\begin{adjustbox}{width=0.57\textwidth, margin=0 0 0 0}

\end{adjustbox}
\end{center}

\begin{center}
\begin{adjustbox}{width=0.57\textwidth, margin=0 0 0 0}

\end{adjustbox}
\end{center}

\begin{center}
\begin{adjustbox}{width=0.57\textwidth, margin=0 0 0 0}

\end{adjustbox}
\end{center}

\begin{center}
\begin{adjustbox}{width=0.57\textwidth, margin=0 0 0 0}

\end{adjustbox}
\end{center}

\begin{center}
\begin{adjustbox}{width=0.57\textwidth, margin=0 0 0 0}

\end{adjustbox}
\end{center}

\begin{center}
\begin{adjustbox}{width=0.57\textwidth, margin=0 0 0 0}

\end{adjustbox}
\end{center}

\begin{center}
\begin{adjustbox}{width=0.57\textwidth, margin=0 10 0 0}

\end{adjustbox}
\end{center}

and

\begin{center}
\begin{adjustbox}{width=0.77\textwidth, margin=0 10 0 10}

\end{adjustbox}
\end{center}

Using the above matrices and relations the computational procedure for calculating Kaluza numbers product can be written as follows:



\begin{adjustbox}{margin = 0 0 0 8}
\end{adjustbox}

It is easy to see that the diagonal matrix  contains only 32 elements differing by its
value. The remaining 480 elements coincide with these thirty two elements up to a sign. Let us create a column vector , consisting of the all 32 elements of matrix  which possess different values. Then it is easy to see that the elements ,  can be calculated using the following vector–matrix procedure:





\begin{figure}[b!]
 \centering
  \includegraphics[width=0.8\textwidth]{Ktest.pdf}
  \caption{Data flow diagram for rationalized Kaluza numbers multiplication algorithm in accordance
with the procedure (3.3)}
  \label{fig2}
\end{figure}
\FloatBarrier

Fig. 2 shows a data flow diagram representation of the rationalized algorithm for computation
of Kaluza numbers product and Fig. 3 shows a data flow diagram of the process
for calculating the vector  elements. In Fig. 2, the points where lines converge denote
summation. As follows from Fig. 3, calculation of elements of diagonal matrix  requires
performing only trivial multiplications by the power of two. Such operations may be implemented using convention arithmetic shift operations, which have simple realization and hence may be neglected during computational complexity estimation \cite{24}.

\begin{figure}[hb]
 \centering
  \includegraphics[width=0.25\textwidth]{KKa.pdf}
  \caption{The data flow diagram describing the process of calculating elements of the vector  }
  \label{fig3}
\end{figure}
\FloatBarrier

\section{Evaluation of computational complexity}

We calculate how many real multiplications (excluding multiplications by power of two)
and real additions are required for realization of the proposed algorithm, and compare it
with the number of operations required for a direct evaluation of matrix–vector product in
Eq. (2.3). As already mentioned the number of real multiplications required using the proposed
algorithm is 512. Thus using the proposed algorithm the number of real multiplications
to calculate Kaluza number product is halved. The number of real additions required
using our algorithm is 576. Therefore, the total number of arithmetic operations for proposed
algorithm is approximately 46\% less than that of the direct evaluation.

\section{Conclusions}

In this paper, we have presented an original algorithm allowing to multiply Kaluza numbers with reduced both multiplicative and additive complexity. Furthermore, the total number of operations in our algorithm is almost two times less than the total number of operations in the compared algorithm. Therefore, the proposed algorithm is better than the naive algorithm, both in terms of hardware implementation and in terms of its software implementation on a conventional computer too.
\\
\\
\\
\\
\begin{thebibliography}{26}

\bibitem{1} Kantor I., Solodovnikov A. Hypercomplex numbers, Springer-Verlag, New York, (1989).

\bibitem{2} B\"{u}low T., Sommer G. Hypercomplex signals – a novel extension of the analytic signal to the
multidimensional case, IEEE Trans. Sign. Proc., Vol. SP--49, No 11, 2844--2852, (2001).

\bibitem{3} Alfsmann D. On families of -dimensional hypercomplex algebras suitable for digital signal
processing, in Proc. European Signal Processing Conf. (EUSIPCO 2006), Florence, Italy,
(2006).

\bibitem{4} Alfsmann D., G\"{o}ckler H. G., Sangwine S. J., Ell T. A. Hypercomplex Algebras in Digital Signal
Processing: Benefits and Drawbacks (Tutorial). Proc. EURASIP 15th European Signal Processing
Conference (EUSIPCO 2007), Pozna\'{n}, Poland, 1322--1326, (2007).

\bibitem{5} Sangwine S. J., Bihan N. Le Hypercomplex analytic signals: extension of the analytic signal
concept to complex signals, Proc. EURASIP 15th European Signal Processing Conference
(EUSIPCO 2007), Pozna\'{n}, Poland, 621--624, (2007).

\bibitem{6} Moxey C. E., Sangwine S. J., Ell T. A. Hypercomplex correlation techniques for vector images,
IEEE Trans. Signal Processing, Vol. 51, No 7, 1941--1953, (2003).

\bibitem{7} Bayro-Corrochano E. Multi-resolution image analysis using the quaternion wavelet transform,
Numerical Algorithms, Vol. 39, No 1--3, 35--55, (2005).

\bibitem{8} Calderbank R., Das S., Al-Dhahir N., Diggavi S. Construction and analysis of a new quaternionic
Space-time code for 4 transmit antennas, Communications in information and systems Vol. 5, No 1, 1--26, (2005).

\bibitem{9} Belfiore J.-C., Rekaya G. Quaternionic lattices for space-time coding, Proceedings of the Information
Theory Workshop. IEEE, Paris 31 March -- 4 April 2003, 267--270, (2003).

\bibitem{10} Ertu\v{g} \"{O}. Communication over Hypercomplex Kahler Manifolds: Capacity of Dual-Polarized
Multidimensional-MIMO Channels. Wireless Personal Communications, vol. 41, No 1, 155--168, (2007).

\bibitem{11}Silvestrov V. V. Number Systems, Soros Educational Journal, No 8, 121--127, (1998).

\bibitem{12} Makarov O. An algorithm for the multiplication of two quaternions, Zh. Vychisl. Mat. Mat.
Fiz., Vol. 17, No 6, 1574--1575, (1977).

\bibitem{13} Cariow A., Cariowa G., “Algorithm for multiplying two octonions”, Radioelectronics and Communications Systems (Allerton Press, Inc. USA), Vol. 55, Issue 10, 464--473, (2012).

\bibitem{14} Cariow A., Cariowa G., An algorithm for fast multiplication of sedenions, Information Processing Letters, 113,  324--331, (2013).

\bibitem{15} Cariow A., Cariowa G., "An algorithm for multiplication of Dirac numbers", Journal of Theoretical and Applied Computer Science, vol. 7, No 4, 26--34, (2013).

\bibitem{16} Cariow A., Cariowa G., An Algorithm for Fast Multiplication of Pauli Numbers. Advances in Applied Clifford Algebras, Advances in Applied Clifford Algebras, Vol. 25, No 1, 53--63, (2015).
\bibitem{17} Cariow A., Cariowa G., On the Multiplication of Biquaternions, Soft Computing in Computer and Information Science: Advances in Intelligent Systems and Computing, Vol. 342, 423--434, (2015).
\bibitem{18} Cariow A., Cariowa G., Kubsik B., An algorithm for multiplication of split-octonions, CoRR abs/arXiv:1503.01058v1, 1--14, (2015).
\bibitem{19} Cariow A., Cariowa G., Knapi\'{n}ski J., Derivation of a low multiplicative complexity algorithm for multiplying hyperbolic octonions, CoRR abs/arXiv:1502.06250v1, 1--15, (2015).
\bibitem{20} Cariow A., Cariowa G., A new algorithm for multiplying two Dirac numbers, CoRR abs/arXiv:1501.00828v1, 1--14, (2015).
\bibitem{21} Cariow A., Cariowa G., An algorithm for multiplication of trigintaduonions. Journal of Theoretical and Applied Computer Science, Vol. 8, No 1, 50--75, (2014).
\bibitem{22} \c{T}ariov A., \c{T}ariova G., „Zracjonalizowany algorytm mno\.{z}enia dw\'{o}ch kwaternion\'{o}w. Przegl\c{a}d Elektrotechniczny, No 9, 137--140, (2010).
\bibitem{23} \c{T}ariova G., \c{T}ariov A., „Aspekty algorytmiczne redukcji liczby blok\'{o}w mno\.{z}\c{a}cych w uk\l{}adzie do obliczania iloczynu dw\'{o}ch kwaternion\'{o}w”, w czasopi\'{s}mie Pomiary, Automatyka, Kontrola, No 7, 668--690, (2010).
\bibitem{24} \c{T}ariov A., \c{T}ariova G., "Aspekty algorytmiczne organizacji uk\l{}adu procesorowego do mno\.{z}enia liczb Cayleya", Elektronika, No 11, 137--140, (2010).

\bibitem{25} Cariow A. Strategie racjonalizacji oblicze\'{n} przy wyznaczaniu iloczyn\'{o}w macierzowo-wektorowych,
Metody Informatyki Stosowanej, No 1, 147--158, (2008).

\bibitem{26} Cariow A. Algorytmiczne aspekty racjonalizacji oblicze\'{n} w cyfrowym przetwarzaniu
sygna\l{}\'{o}w, Wydawnictwo Zachodniopomorskiego Uniwersytetu Technologicznego, (2011).

\end{thebibliography}

\end{document}
