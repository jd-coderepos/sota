\documentclass{sig-alternate-2013}

\newfont{\mycrnotice}{ptmr8t at 7pt}
\newfont{\myconfname}{ptmri8t at 7pt}
\let\crnotice\mycrnotice \let\confname\myconfname 

\permission{Permission to make digital or hard copies of all or part of this work for personal or classroom use is granted without fee provided that copies are not made or distributed for profit or commercial advantage and that copies bear this notice and the full citation on the first page. Copyrights for components of this work owned by others than ACM must be honored. Abstracting with credit is permitted. To copy otherwise, or republish, to post on servers or to redistribute to lists, requires prior specific permission and/or a fee. Request permissions from permissions@acm.org.}
\conferenceinfo{PODC'15,}{July 21--23, 2015, Donostia-San Sebasti\'an, Spain.}
\copyrightetc{Copyright \copyright~2015 ACM \the\acmcopyr}
\crdata{978-1-4503-3617-8 /15/07\ ...\nnnnnvr(v)n\Omega(\log^*n)nO(\log^*n)nnn/2a(p)pa(n)a(0)=0a(1)=1\theta(n\ln(n))nO(\log^*n)\Omega(\log^*n)n\Omega(\log^*n)AA'AA'AGxykAxy\max\{r(x),r(y)\}+kGxykAxy\max\{r(x),r(y)\}+kNxyxyAA'NvAvNxyA'v\max\{r(x),r(y)\}+kID(x)>ID(y)dvxdv\{1,2\}v\{3,4\}A'A'AAvrAr/2v\Omega(r)d\{1,2,..., \lfloor{} r/2 \rfloor{} \}u_dw_ddvdrr/2vrn/2\frac{1}{2}.\log^*(n/2)A\pi\Omega(\log^*n)n\frac{1}{2}.\log^*(n/2)\frac{1}{2}.\log^*(n/2)\pi\frac{1}{2}.\log^*(n/2)\pin/2\pi\frac{1}{2}\log^*(n/2)\piA\frac{1}{2}\log^*(n/2)\Omega(\log^*(n/2))\Omega(\log^*(n))\pi\Omega(\log^*(n))$. This concludes the proof of the theorem.

\section{Conclusion and further work}

This paper presents a new measure of the locality. It is close to the classic measure for some problems, and very different for others. It would be interesting to characterise the problems of the first and second types. Also, we only consider the cycle topology, and results for more general graphs are missing. Last, as the average we consider is over the nodes, but it would also be interesting to begin to study the expectancy of the running time on graphs where the permutation of the identifiers is taken uniformly at random, for both the classic and the new measure.

\section*{Acknowledgements}
I would like to thank Pierre Fraigniaud, Juho Hirvonen, Tuomo Lempi\"ainen and Jukka Suomela for helpful discussions.

\begin{thebibliography}{1}

\bibitem{ColeV86}
Richard Cole, and Uzi Vishkin.
\newblock Deterministic Coin Tossing with Applications to Optimal Parallel List Ranking.
\newblock {\em Information and Control}, 70(1):32--53,1986.

\bibitem{KormanSV13}
Amos Korman, Jean{-}S{\'{e}}bastien Sereni, and Laurent Viennot.
\newblock Toward more localized local algorithms: removing assumptions
  concerning global knowledge.
\newblock {\em Distributed Computing}, 26(5-6):289--308, 2013.

\bibitem{Linial92}
Nathan Linial.
\newblock Locality in distributed graph algorithms.
\newblock {\em SIAM J. Comput.}, 21(1):193--201, 1992.

\bibitem{Musto11}
Topi Musto.
\newblock Knowledge of degree bounds in local algorithms.
\newblock Master's thesis, University of Helsinki, 2011.

\bibitem{Peleg00}
David Peleg.
\newblock Distributed Computing: A Locality- Sensitive Approach. 
\newblock SIAM, Philadelphia, PA, 2000.

\bibitem{oeisA000788}
N.~J.~A. Sloane.
\newblock The {O}n-{L}ine {E}ncyclopedia of {I}nteger {S}equences.
\newblock \href{http://oeis.org/A000788}{A000788}.



\end{thebibliography}

\end{document}
