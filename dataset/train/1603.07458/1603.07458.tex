\documentclass{article}
\usepackage[utf8]{inputenc}
\usepackage[T1]{fontenc}
\usepackage{graphics}
\usepackage{alltt}
\usepackage{url}
\usepackage{epic}
\usepackage{latexsym}
\usepackage[all]{xy}
\usepackage{amssymb}
\usepackage{amsmath}
\usepackage{comment}
\usepackage{enumitem} 
\newlist{itemrond}{itemize}{2}
\setlist[itemrond,1]{label=\textbullet}
\setlist[itemrond,2]{label= \alph*}


\usepackage{pslatex}
\usepackage[pdftex,bookmarksnumbered=true]{hyperref}
\title{The method "Model Elimination" of D.W.Loveland explained}
\author{Michel Lévy}

\newtheorem{theoreme}{Theorem}\newtheorem{lemme}[theoreme]{Lemma}
\newtheorem{corollaire}[theoreme]{Corollary}
\newtheorem{remarque}[theoreme]{Remark}
\newtheorem{exemple}[theoreme]{Example}
\newtheorem{definition}[theoreme]{Definition}
\newtheorem{propriete}[theoreme]{Property}

\newenvironment{preuve}{\noindent {\em Proof :}\ }{{\hfill
    }\vspace{.5pc}} \newcommand{\sg}{\!\!<\!\!}


\begin{document}

\maketitle
\tableofcontents
The method "Model Elimination is a proof method very easy to implement and it is the reason of his success.
To present it, we use the following references
 ~\cite{Loveland1997}, ~\cite{Loveland1978} et  ~\cite{CSC648}.

The last document presents clearly and concisely the production of the lemma. Without this help, it would have been impossible
to write this explanation of the method of D.W.Loveland.




\section{Basis of the method}

The opposite of the literal  is  if  is an atom and  if .
In the following, we note  
 , the opposite of the literal .

A chain is a list of B-literals and A-literals (also called ancestor literals). An A-literal is represented by a literal
enclosed in brackets. A B-literal is a literal in the usual sense.

The empty list is written .

An elementary chain is a list of B-litterals.

An acceptable chain is a chain beginning (at the left side) by a B-literal.

On the chains, we define three operations, reduction, extension and removal.

In order to make easier the understanding of the Model Elimination method (abbreviated form ME), we will give a version of
this method for the propositional logic and another version for the first order logic.


\subsection{Model Elimination for the propositional logic}

\begin{description}
\item [Extension :]
Let  be a set of elementary chains. 

Let  an \emph{acceptable} chain where  is the B-literal on the left of the chain.

Let  be a chain member of .
The chain   is produced by \emph{extension} of the chain  with .


\item [Reduction:]
Let  be an acceptable chain, where  is a B-literal.

The chain  is obtained by reduction of this chain.


\item [Removal :]
Let  a chain beginning with A-literal .

The chain  is obtained by removal on the chain .


\end{description}

\begin{definition}[Derivation]\label{derivation1}
Let  be a set of elementary chains. A derivation from  is a sequence of chains  where 
such that  and for  from  to , the chain  is obtained by extension of the chain  with
a elementary chain from , by a reduction of the chain  or by a removal on the chain .\\
A chain is derivable from , if there exists a derivation from , finishing with this chain.
\end{definition}

Let  be a chain. We associate it with a normal form ,which gives the meaning of the chain. 

\begin{definition}[normal form associated with a chain]\ 
\begin{itemrond}
\item  where  is the always false formula.
\item  where  is the logical disjunction,  is a B-literal and  is a chain.
\item  where  is the logical conjunction,  is a A-literal and  is a chain.
\end{itemrond}
\end{definition}

Note that, following this definition, an elementary chain is a disjunction of his B-literals.

When there is no ambiguity, we identify a chain and the normal form associated with the chain. Let us take an example.
Let  be a literal, . As a consequence, the formula  is equivalent to , 
which is the meaning of the empty chain. So, identifying formula and chain, we may write .

We show below, that the chains derived from  are logical consequences from . This property is called the 
coherence of the method.

During a derivation, we can create lemma, which are elementary chains logical consequences of .
It's clear that, to obtain consequences of , we can use extensions from  and from the lemmas created during
the derivations from .

In the references \cite{Loveland1997} and \cite{Loveland1978}, the removal is built-in with extension and reduction.
At the end of an extension or a reduction, we make all the removals necessary to obtain again an acceptable chain. 
But it seems to me useful, following the example of \cite{CSC648}, to distinguish this operation to facilitate the 
understanding of the creation and use of the lemmas.


\begin{lemme}[monotony of chains]\label{monotonie-chaine}
Let  be three chains. Let  be a set of formulas.
Let us suppose that 
 . Then . 
\end{lemme}

\begin{preuve}
Let us suppose that . \\
We show the conclusion by recurrence on the length of . \\
It's clear if  is the empty chain.\\
Let us suppose that , where  is a literal.\\
By the recurrence hypothesis,  .\\
By definition of the meaning of the chains,  and .\\
From the monotony of the disjunction, it follows that, , \\thus 
.\\
The case where  is similar, because the conjunction is also monotonous.
\end{preuve}

\begin{corollaire}[monotony of chains]\label{monotonie-chaine-egal}
Let  be three chains. Let  be a set of formulas.
Let us suppose that . then .
\end{corollaire}

This corollary is an immediate consequence of the previous lemma, because the equivalence  is the
conjunction of 
  and .


\begin{lemme}[coherence of removal]\label{coherence-enlevement}
Let  be a literal and  be a chain. We have : 
\end{lemme}

\begin{preuve}
Above we have seen that . From the corollary \ref{monotonie-chaine-egal}, it follows that 
\end{preuve}

\begin{lemme}[coherence of reduction]\label{coherence-reduction}
Let  be a literal and  be two chains. We have .
\end{lemme}

\begin{preuve}
From the meaning of the chains, .\\
From the meaning of the negation, .\\
From the lemme \ref{monotonie-chaine}, we deduce that .\\
Consequently  and , thus .\\
Because , and by the property of the implication, .\\
From the lemma \ref{monotonie-chaine},.
\end{preuve}




\begin{lemme}[coherence of extension]\label{coherence-expansion}
Let  be a set of elementary chains. Let  be a chain giving by extension with  the chain .\\
We have : .
\end{lemme}

\begin{preuve}
By definition of the extension, there is a literal  and a chain  such that .
And there is a chain belonging to , which is written   and  .\\
The elementary chain  is equivalent to .\\
It follows that , and also 
.\\
From the meaning of the chains, , thus 
.\\ 
From the lemma \ref{monotonie-chaine}, we deduce that .
\end{preuve}


\begin{theoreme}[coherence of the method]
Let  be a set of elementary chains and  a chain derivable from .
We have : .
\end{theoreme}

\begin{preuve}
Let  où  be a derivation (see \ref{derivation1}) of the chain  from .\\
Because , we have .\\
From the lemmas \ref{coherence-expansion}, \ref{coherence-reduction}, \ref{coherence-enlevement}, it results that :
for all  between  and , .\\
Thus, by recurrence on the length of derivation  :
for all  where , .\\
Because  is the last chain of the derivation, we have .
\end{preuve}

\begin{corollaire}[proof of unsatisfiability]\label{coherence-ME}
Let  be a set of elementary chains. If  is derivable from , then  is unsatisfiable.
\end{corollaire}

\begin{preuve}
Let us suppose that  is derivable from . From the theorem above, it follows that .
Because  is the formula false (without model),  has no model.
\end{preuve}


\subsection{Model Elimination for the first order logic}

\begin{description}
\item [Extension :]
Let  be a set of elementary chains. 

Let  an \emph{acceptable} chain where  is the B-literal to the left of the chain.

Let  a \emph{copy} of a chain belonging to , whose variables  \emph{do not appear}
in .  

Let us suppose that there exists  a \emph{most general unifier} of  and the opposite of literal . 
Then the chain   is obtained by \emph{extension} of the chain  from  .


\item [Reduction:]
Let  be an acceptable chain, where  is the left most B-literal of the chain and  a A-literal,
such that there exists a \emph{most general unifier}  between  and the opposite of .

The chain  is obtained by reduction of the chain .

\item [Removal :]
Let  be a chain beginning by the A-literal .

The chain  is obtained by removal on the chain  

\end{description}


The universal closure of a formula  is written . It is the formula obtained while universally quantifying all the
free variables of .

Let  be a set of formulas. The universal closure of , written  
is the set of the universal closure of the formulas belonging to .

In the following, we use the notion of logical consequence in its most usual sense. A formula is logical consequence of a set
of formulas, if every model of the set (giving values to the function symbols, relation symbols \emph{and to the variables})
is model of the formula.

With this notion of logical consequence, we have 
 , and we have .

We will see that the chains, derivables from a set  of elementary chains, are consequences of  : it is
this property, which is, for the first order logic, called the coherence of the method.

During a derivation, we can produce lemmas, which are elementary chains consequences of  .
It's clear that, in order to obtain consequences of , we can use extensions from  or from
the lemma produced during the derivations from .

Let  be a substitution. We write  the formula obtained while replacing all the free variables of 
by their values in the substitution. When the formula  has no quantifier, we have  .

\begin{lemme}[coherence of reduction]\label{coherence-reduction-1}
Let  be a chain and  be a chain produced by reduction of .
We have : .
\end{lemme}

\begin{preuve}
By definition of the reduction, there exists two literals  and , two chains  and , a substitution 
such that 
 , the literals   et  are opposite  and
.

From the properties of the universal closure, we have : .\\
Because  and
from the coherence of reduction for propositional logic \ref{coherence-reduction}, 
 we have : .
Thus . \\
From the properties of the logical consequence, we conclude: .
\end{preuve}



\begin{lemme}[coherence of extension]\label{coherence-expansion-1}
Let  be a set of elementary chains. Let  be a chain giving by extension with  the chain .
We have : .
\end{lemme}

\begin{preuve}
By definition of extension, there exists a literal  and a chain  such that  and there exists a chain of , 
which is written  and a substitution  such that  and  are two opposite literals and
 .\\
Because the literals   et  are opposite, the elementary chain  is equivalent to
 .\\
It follows that , and thus
.\\
From the sense of the chains, , thus 
.\\ 
From the chains monotony \ref{monotonie-chaine}, we deduce that .\\
From the property of the universal closure, we have .\\
From the property of the logical consequence, .\\
Because the hypothesis have no free variables, we have : .\\
So we conclude that :  .
\end{preuve}


\begin{theoreme}[coherence of the method]
Let  be a set of elementary chains and  be a chain derivable from .
We have : .
\end{theoreme}

\begin{preuve}
Let  where  be a derivation (see \ref{derivation1}) of  from .\\
Because  and from the property of the logical consequence, we have :\\ .\\
From the lemmas \ref{coherence-expansion-1}, \ref{coherence-reduction-1}, \ref{coherence-enlevement}, it follows that :\\
for all  between  and , .\\
Thus by recurence of the length of derivations :\\
for all  such that , .\\
Because  is the last chain of the derivation, .
\end{preuve}

\begin{corollaire}[proof of unsatisfiability]\label{coherence-ME-1}
Let  be a set of elementary chains. If  is derivable from , then  is unsatisfiable.
\end{corollaire}

\begin{preuve}
Let us suppose that  is derivable from . Then, by the theorem above, 
 .
Because  has no model,   has no model.
\end{preuve}

\section{Production of lemmas in propositional logic}

To each A-literal of a chain, we associate an integer, the scope of the literal.

During a extension, the scope of the new A-literal is zero.

During a reduction, the scope of the A-literal which is used by the reduction \emph{can be modified}. If the number of
A-literals to the left of this A-literal is greater than its actual scope, its scope \emph{becomes} this number.

During the removal of an A-literal, \emph{a lemma} is generated which is an elementary chain whose elements are the opposite of
all the A-literals whose scope is equal to the number of A-literals to their left. The not zero scope of these A-literals are 
decremented.


Note that, during a derivation, the scope of an A-literal is at most equal to the number of A-literal to its left.
This property is true for the first chain of a derivation, because this chain has no A-literal, and it is clearly maintained
by extension (the new A-literal has the scope zero), by reduction (the only A-literal whose scope is modified, has ist scope
equal to the number of A-literal to its left) and by removal. 

From this remark, it results that, when we remove an A-literal, the first in its chain, its scope is zero, thus its opposite
is member of the lemma produced.

In order to make easier the understanding of the creation of lemmas and the proof of their correctness, we repeat what we said
above, by defining again the three operations extension, reduction and removal, while adding the calculus of the scopes. 


\begin{description}
\item [Extension :]
Let  be a set of elementary chains. \\
Let  be a chain   where  is the leftmost B-literal. \\
Let  a chain belonging to . The chain   
is obtained by \emph{extension} of the chain  from .\\
The scope of the new A-literal  is zero.

\item [Reduction :]
Let   an acceptable chain, where  is the leftmost literal of the chain and 
  an ancestor literal. The chain 
   is obtained by reduction of the chain .\\
If the number of A-literals strictly to the left of this ancestor literal is greater than its scope before reduction, its scope
becomes this number.

\item [Removal :] 
Let   be a chain beginning with the A-literal . The chain  is obtained by removal from the chain
 .

A lemma is produced which is the disjunction of this A-literal and of all the other A-literals 
whose scope is equal to the number
of A-literals to their left. The not-zero scopes of these A-literals are decremented.
\end{description}


The addition of lemmas can \emph{make easier or make harder} the derivations. It can make them easier, because the use of a lemma
can avoid to do again the derivation which has produced this lemma. 
It can make them harder, because it can add too many lemmas and unnecessary lemmas.

There is several policies for the use of lemmas. We can add them during a derivation or during the 
construction of a derivation's tree ( a derivation can add lemmas used in another derivation). 
We can select the "best" lemmas, for example, the shortest lemmas. We can also replace some entry chains by lemmas subsuming these
chains.

We do not consider these policies of use of lemmas, which was the subject of many papers. We content ourselves to prove
that the lemmas generated during a derivation are really consequences of the entry chains of the derivation.

We present a property of the chains, verified by a chain without A-litteral, and kept by extension, reduction, removal.
Thus its property is verified by each chain derived and allows us to prove the correctness of lemmas.



\begin{definition}[Property of the derived chains]\label{prop-chaine-derivee}
Let  be a set of elementary chains and let  be a chain.

There exists , where ,  some literals  where , 
some integer  where , where  is the scope of the A-literal 
and some elementary chains
  where  such that \\.

Let  be the set of A-literals defined by 
.
We identify the set  with the  \emph{conjunction} of its elements.

 verify the property of the derived chains with respect to  if
for   where ,  and .
\end{definition}

The A-literal  is used in the reduction of the descendants of the A-literal . 
For , it is used to reduce the descendants of  et for  to reduce the descendants of .
Thus  is the set of A-literals used to reduce the descendants of .

I have to recognize that I was unable to understand the proof of the correctness of lemmas with the only reading 
of the book of  D.W.Loveland \cite{Loveland1978}. 

It is the main reason which impulses me to write this \emph{explanation} of the model elimination method.
The most difficult part was to find the property of derived chains, which is invariant during a derivation and which allows
to explain the correctness of the lemmas.



\begin{lemme}[Invariance of the property of the derived chains]\label{invariant-derivation}
Let  a set of elementary chains and  a chain verifying the property of the derived chains with respect to .
Then this same property is also verified by the chain  obtained from the chain  by extension with , reduction 
or removal.
Furthermore the lemma produced during the removal is consequence of .
\end{lemme}

\begin{preuve}

For the chain , we take again the notations of the property above \ref{prop-chaine-derivee}.

Because  is a chain,  
il existe , où ,  some literals  où , some integer  where , 
some elementary chains 
  où  such that .
For  where ,  is the scope of the literal .

Let  be the set of literals defined by 
.

\begin{itemrond}
\item Let us suppose that the chain  was produced by extension of  with . 

Let us suppose that  begins with the B-literal  and that the extension is produced with the chain 
element of . 

Note that .
Because a new A-literal   is added, we have for  where , .

Because the scopes are not changed (except for the new A-literal), we have for  where  ,
. 
Clearly the scope of the  A-literal of  is the same as the scope of the  literal
of . Thus for  , we have : . 

Because the new A-literal is introduced as   with the scope zero, by definition of 
 , we have :\\
\textbf{(a) : } 

From the hypothesis on , we have : for   where , . \\ 
Because  and , we have : for   where , .\\
By replacing  by  and  by , we have : for  where  , . 
By adding the condition \textbf{(a)} we obtain :\\
\textbf{(b) : } for  where , 

It is the first part of the property that must verify . It remains to verify that 
for  where , .

By the properties of derived chains of , we have for  where
  où , 
. 

We said above that for  where , .\\
Thus, the property on  can be translated in \\
\textbf{(c) : } for  where , 

Because  and that this chain is equivalent to  , we have
. Let remind us that . 
From the lemma monotony of chains \ref{monotonie-chaine}, we deduce that:\\
\textbf{(d) : } .

From \textbf{(c)} and \textbf{(d)}, we deduce that for  where 
, \\
By replacing  with , we obtain :\\
\textbf{(e) : } for  where , 

We know already that   belongs to the conjunction , thus .
Consequently for  where , .  
That finishes the proof that , obtained from  by extension, keeps the property of the derived chains.



\item Let us suppose that  was obtained by reduction of .

In this case,  and the A-literals are not changed. Only the part  of the chain  is modified.

Thus we have for  from  to ,   and for   from  to , .

The chain  is written  and there is a A-literal  where  and  et .

By definition of the reduction,  the scope of 
   is   (the number of A-literals to the left of ) in . 
In the following we reserve  as the index of this A-literal causing the reduction.

Because for all  such that  and ,  and that , we have :
 for all  where , .

Because  verify for all  where , , that for all  where , 

and , we have :\\
for  where ,  is verified by .

Thus  verify the first part of the property of the derived chains. It remains us to prove that
for  from  to , .


The A-literal , which is used to reduce the descendants of , belongs to all the 
conjunctions , thus \\
\textbf{ (a) : }for  where , .

From the lemma \ref{monotonie-chaine}, we have :\\
\textbf{ (b) :} 

From the propositions \textbf{(a)} and \textbf{(b)}, we deduce :\\
\textbf{ (c) :} for  where  , .

Because for all  , where , , and because these sets are considered as 
conjunction of their members, we have :
for  where , .

As  verify the property of the derived chains, we have  : \\.

Because  implies , we have : \\
\textbf{ (d) :}.

From \textbf{(c)}, \textbf{(d)} and because for , , we have :\\
for  where .\\
Consequently the chain  produced by reduction on , verifies also the property of the derived chains.

\item Let us suppose that the chain  is obtained by removal on the chain .

In the first place, we show that the lemma created during the removal is consequence of .
During this removal .

Because  verify the property of the derived chains, we have : \\for  where , 
.

Thus . 

Let us note that  is the \emph{conjunction}  of all the literals  whose scope is , id est the
number of A-literals to the left of .

The formula  is equivalent to the \emph{disjunction} of the opposite of these literals.
This is the lemma added by the removal. Thus this lemma is consequence of .

The decrementation of the scopes, \emph{after} the removal of the first A-literal of , makes that the other A-literals of  
whose scope were equal to the number of their A-literals to their left, remain the same in . 
Formaly that means that when in , we had  (the scope of literal  equal to the number of A-literal to its left), 
we have  in . This remark implies that  for  where , .
 
In the case of removal, ,  and from the notations of , 
for  where , , for  from , . 

The chain  verify that : 

for  where ,  and .

Because  and , we have 
for  where ,  and 

By replacing  by  where  and knowing that , we conclude that :
for  where ,  and .\\
So the property of the derived chains is kept by removal
\end{itemrond}
\end{preuve}


\begin{theoreme}
Let  be a set of elementary chains. Every chain of a derivation from  verifies the property of the derived chains
\ref{prop-chaine-derivee} and the lemmas produced during this derivation are consequences of .
\end{theoreme}

\begin{preuve}
The chain origin of a derivation, having no A-literal, verify the property of the derived chains.
This property being kept by each step of a derivation, by \ref{invariant-derivation}, every chain of the derivation has this 
property. Because every chain of a derivation verify this property, during each removal, 
as we prove in \ref{invariant-derivation}, the lemmas produced are consequence of . 
\end{preuve}




\section{Production of lemmas in first order logic}

The scope's calculus is nearly the same as in the propositional case.
During an extension, the scope of the new A-literal is zero.
During a reduction, the scope of the A-literal used in the reduction \emph{can be modified}.
If the number of A-literals to the left of this A-literal is greater that its scope, this scope \emph{becomes}
this number.
During the removal of an A-literal, \emph{a lemma} consisting in the opposite of all the A-literals whose scope is equal to the
number of A-literals to their left is produced. 
The not zero scopes of these A-literals are decremented.
To avoid any ambiguity, we define again the three operations extension, reduction and removal, while adding the 
calculus of the scopes.


\begin{description}
\item [Extension :]

Let  be a set of elementary chains.\\
Let  be an \emph{acceptable} chain where  is the leftmost B-literal.\\
Let  be a \emph{copy} of a chain belonging to , whose variables \emph{do not appear} in .\\
Let us suppose that there exists a \emph{most general unifier} of  and the opposite of the literal .
Then the chain   is obtained by extension of the chain  from .\\
The scope of the new A-literal  is zero.\\
We note also that the scopes defined in  and  are kept, more precisely, the scopes of the th literal 
of the chain  and of the chain  are equal. Briefly, the scopes are preserved by substitution.


\item [Reduction :]

Let  be an acceptable chain, where  is the leftmost B-literal, and  an A-literal, such that there is
a \emph{most general unifier} between  and the opposite of . Then the chain  is obtained
by reduction of the chain  .\\
If the number of A-literals to the left of the A-literal used for the reduction, is greater than its scope before reduction,
this scope becomes this number.\\
As for the extension, the scope of the other A-literals are preserved by subsitution.


\item [Removal :] 

Let  be a chain beginning by the A-literal . The chain  is obtained by removal of the chain .

A lemma, consisting in the opposite of this A-literal and of all other A-literals whose scope is equal to the number
of A-literals to their left, is produced. The not zero scopes of theses A-literals are decremented.

\end{description}

In the first order case, we do not make all the proofs necessary to establish the correctness of the lemmas produced during
the removal.
We give only below the property of the derived chains, invariant during the derivations and we admit this invariance.
The only difference with the propositional case, is the replacement, in the last line
of this property,  of  by the universal closure .

The proof of this invariance is similar to that of the propositional logic, but complicated by the substitutions.
We leave this proof of invariance to the courageous reader.




\begin{definition}[Property of the derived chains]\label{prop-chaine-derivee-1}

Let  be a set of elementary chains and let  be a chain.

There exists , where ,  some literals  where , some integer  where 
and some elementary chains
  where  such that \\.
For  where ,  is the scope of the litteral .

Let  be defined by 
 .
We identify the set  with the  \emph{conjunction} of its elements.


 verify the property of the derived chains with respect to  if
for   where ,  and .
\end{definition}



\begin{theoreme}
Let  be a set of elementary chains.
Every lemma produced during a derivation from  is a consequence of .
.
\end{theoreme}

\begin{preuve}
Let  a chain derived from , beginning by an A-literal. The lemma produced by the removal of this A-litteral is
the elementary chain composed with all the opposites of the 
A-litterals of the chain whose scope is equal to the number of literals to their left.

From the invariance of the property of the derived chains, we 
know that  verify this property. Thus , where  is the conjunction
of A-literals of the chain, whose scope is equal to the number of A-literals to their left. The lemma is equivalent to
the formula , thus consequence of .

\end{preuve}

\section{Method's Completeness}

We show the completeness of the method. Let  a set of elementary chains. In the propositional case, we show
that, if  is unsatisfiable, then the empty chain can be derived from . In the first order case,
we show that, if  is unsatisfiable, then the empty chain can be derived from .

\subsection{Propositional completeness}

\begin{propriete}\label{concatenate-property}
Let  be a set of elementary chains. Let  be a chain and  be a derivation from .
Then  is also a derivation from .

\end{propriete}

\begin{preuve}
It's enough to verify that if the chain  gives  by extension from  (respectively reduction or removal), 
then  gives  by extension from  (respectively reduction or removal).
\end{preuve}

\begin{theoreme}\label{completeness-minimal-propositional}
Let  be a minimally unsatisfiable set of elementary chains. For every , there is a propositional derivation
(in the sense of \ref{derivation1}) from , starting with  of the empty clause.
\end{theoreme}

\begin{preuve}
Let us call length of a set of chains, the sum of the lengths of the chains belonging to the set. The proof is done by 
recurrence on the length of . Let us suppose the the theorem is verified when the length of  is less than .
Let  the length of . We prove that the theorem is still verified.

Let  be a clause element of . We consider two cases as  is a unitary clause or not.
\begin{itemrond}
\item  is unitary, i.e. a chain of length . Let  the literal of the chain.

In , there exists a chain  where . Let us suppose, on the contrary, that no chain of 
contains the literal .  If   had a model ,  would be model of .
Since  has no model, it results that  has no model, which contradicts that  is minimaly 
unsatisfiable.

Let  and . It is easy to verify that  and  are equivalent.
Since  is minimaly unsatisfiable,  is satisfiable. Therefore,  every minimaly unsatisfiable
subset from  includes . Let  be such a set. Since the length of  is less than , the length of
 is also less than  and the hypothesis of recurrence can be applied to .
Therefore there exists a derivation  of the empty clause beginning with  from .
From the property \ref{concatenate-property}, it results that  is a derivation of the chain  beginning
with  from .

The clause  where  gives by extension with , the chain . Therefore  is a derivation
beginning with , of the empy chain from .

\item  is not an unitary, therefore  where  is a literal and  is a not empty chain.

Let  a subset minimaly unsatisfiable of  and  a subset minimaly unsatisfiable
from . Since  is minimaly unsatisfiable,  and .

Since the lengths of  and  are less than , by hypothesis of recurrence, there is a derivation 
 beginning with  and ending with the empy clause from  
and also a derivation  beginning with 
and ending with emmpty clause from .

From the property \ref{concatenate-property}, it results that  is a derivation beginning with 
and ending with  from . Therefore  is a derivation beginning with  and
ending with the empty clause from .

\end{itemrond}

\end{preuve}

\begin{corollaire}\label{completeness-propositional}
Let  an unsatisfiable set of elementary chains. The empty chain can be propositionaly derived from .
\end{corollaire}

\begin{preuve}
Since  is unsatisfiable, it contains a subset  minimaly unsatisfiable. 
From the theorem \ref{completeness-minimal-propositional}, the empty chain can be derived from  therefore also
from .
\end{preuve}





\subsection{First order completeness}

The first order completeness proof follows the usual method. Let  a set of elementary chains.
Let us suppose that  is unsatisfiable. From the Herbrand works, we conclude   
that there exists a set  finite, unsatisfiable of instances  of the chains of  on the Herbrand domain
associated to . From the previous subsection, we conclude that there exists a derivation of the empty chain 
from . We show that this propositional derivation can be lifted in a first order derivation of the empty chain from 
.

\begin{lemme}[Lifting of an extension]\label{lifting-extension}
Let  a chain and  a elementary chain. Let  a instance without variable of ,  an instance without variable
of  and  an \emph{propositional} extension of  with . There exists a first order extension  of  with 
whose instance is .
\end{lemme}

\begin{preuve}
Because  is an extension of  with , the chain  can be written  where  is a literal, the chain  is
written  and .

Because  is an instance of , there exists a substitution  such that  where  is a literal such that
 and  is a chain such that .

Because  is an instance of , there exists a substitution  such that  where 
is a literal such that ,  is a chain such that  and  is a  chain
such that .

Let  a renaming od  such that  and  have no common variables.  is a bijection between the variables
od  and the variables od . Let us note  the inverse of  on the variables of .
Let  be the following substitution :
\begin{itemize}
\item for  variable of , 
\item for  variable of , 
\item for other variable , 
\end{itemize}

Because  and  have no common variable, the substitution  is well defined.
By definition of ,  we have . Because  is the identity on the variables of , 
we have . By definition de , .
Thus , i.e.  unify  and . 

Let  the main unifier of these two literals. There exists a substitution  such that .
Let . The chain  is a first order extension of  with  and , i.e.
 is an instance of , actually, in more detail :
\begin{itemize}
\item 
\item 
\item 
\end{itemize}

\end{preuve}

\begin{lemme}[Lifting of a reduction]\label{lifting-reduction}
Let  be a chain,  an instance of  without variable and  produced by propositional reduction of .
There exists  a first order reduction of  having  as an instance.
\end{lemme}

\begin{preuve}
The proof (easy) is left to the reader.
\end{preuve}

\begin{lemme}[lifting of a removal]\label{lifting-removal}
Let  a chain,  an instance of  without variable and  produced by removal on .
There exists  obtained by removal on  having  as an instance.
\end{lemme}

\begin{preuve}
The proof (trivial) is left to the reader.
\end{preuve}

\begin{theoreme}[lifting of a derivation]\label{lifting-derivation}
Let  a set of elementary chains,  a set of instances without variable of the chains of  and
let  a propositional derivation from  beginning with a chain of .
There exists a first order derivation  from  beginning with a chain of , such that,
for  such that , the chain  is an instance of .
\end{theoreme}

\begin{preuve}
The proof is done by recurrence on .
For , the theorem results from the fact that  is an instance of a chain of . 
Suppose the theorem verified for . Let  a propositional derivation from 
beginning with a chain of .

By hypothesis of recurrence, there exists a first order derivation  from  beginning with a chain from 
, such that for  such that , the chain  is an instance of .

Let us suppose that  is produced by extension of  with a chain of . Because  is an instance of  and
because a chain of  is an instance of a chain of , by the lemma \ref{lifting-extension}, there exists  a 
first order extension of  with a chain of , having the instance . We put .

With the aid of the lemmas \ref{lifting-reduction} and \ref{lifting-removal}, the cases where  is produced by reduction
or removal, are analog.
\end{preuve}


\begin{corollaire}[Completeness of first order model elimination]
Let  a set of elementary chains, such that  is unsatisfiable. There exists a first order derivation
of the empty chain from .
\end{corollaire}

\begin{preuve}
Because  is unsatisfiable, from the work of Herbrand, there exists a set  finite, unsatisfiable of
chains instances of chains of .

From the corollary \ref{completeness-propositional}, there exists a \emph{propositional} derivation of the empty clause.
By the theorem \ref{lifting-derivation}, there exists a first order derivation beginning with a chain of 
and ending with a chain whose empty clause is an instance. The last chain of this first order derivation is necessarely the
empty clause. Thus the empty clause is derived at the first order from 
\end{preuve}


\section*{Conclusion}
What is so difficult, in the reading of the book of D.W.Loveland \cite{Loveland1997}, is that he has not separated the 
propositional case and the first order case. By doing this separation, I hope to have clarified the method of Model 
Elimination, especially the proof that the lemmas generated by this method are correct.








\bibliographystyle{alpha}
\bibliography{me-english}



\end{document}