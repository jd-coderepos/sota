\documentclass{article} 
\RequirePackage{amsfonts}       
\usepackage{amssymb}
\usepackage{amsthm}




\newcommand{\Omit}[1]{}


\newtheorem{definition}{Definition}
\newtheorem{theorem}{Theorem}
\newtheorem{corollary}{Corollary}
\newtheorem{lemma}{Lemma}
\newtheorem{proposition}{Proposition}
\newenvironment{remark}{\medskip\noindent \textsc{Remark.}} {\medskip}


\newenvironment{pf}{\em \medskip\noindent \textsc{Proof.}}
{\hspace*{\fill}\nolinebreak[2]\hspace*{\fill}\medskip}

\newenvironment{hypothesis}{\medskip\noindent \textsc{Hypothesis.}\itshape} {\upshape\medskip}

\newbox\itembox
\def\itemlistlabel#1{#1\hfill}
\def\itemlist#1{\setbox\itembox=\hbox{#1}\list{}{\labelwidth\wd\itembox
                             \leftmargin\labelwidth
                             \advance\leftmargin by\itemindent
                             \advance\leftmargin by\labelsep
                             \let\makelabel\itemlistlabel}}
\let\enditemlist\endlist





\renewcommand{\phi}{\varphi}
\newcommand{\eqdef}{\stackrel{\mathrm{def}}{=}}                     \newcommand{\tuple}[1]{\langle #1 \rangle}




\newcommand{\card}[1]{\mathit{Card}(#1)}           \newcommand{\ext}[1]{|#1|}




\newcommand{\eqv}{\leftrightarrow}      \newcommand{\imp}{\rightarrow}          \newcommand{\subfml}{\mathit{sf}}
\newcommand{\lngth}[1]{|\!|#1|\!|}



\newcommand{\nxt}{\mathit{X}}                            \newcommand{\henceforth}{\mathit{G}}                     \newcommand{\eventually}{\mathit{F}}                     




\newcommand{\cstit}[1]{[{#1}]}           \newcommand{\poscstit}[1]{\langle {#1} \rangle}    \newcommand{\dstit}[2]{[{#1}\ \mathit{dstit}\! :{#2}]}

\newcommand{\InclBox}[1]{}      


\newcommand{\actAgt}[2]{{#2}\!\!:\!\!{#1}}  

\newcommand{\after}[1]{[{#1}]}                          \newcommand{\done}[1]{\mathit{Done}_{#1}}               \newcommand{\before}[1]{\mathit{Before}_{#1}}           \newcommand{\feasible}[1]{\langle{#1}\rangle}           



\newcommand{\bel}[1]{\mathit{Bel}_{#1}}       



\newcommand{\accessRel}[1]{{\cal #1}}

\newcommand{\rafter} [1]{\cal{R}_{#1}}     \newcommand{\rbel}   [1]{\cal{B}_{#1}}     \newcommand{\rchx}[1]{\cal{C}_{#1}}     \newcommand{\rhenceforth}{\cal{R}_{\henceforth}}            




\newcommand{\atmset}{\ensuremath{\mathit{ATM}}}        \newcommand{\actset}{\ensuremath{\mathit{ACT}}}        \newcommand{\evtset}{\ensuremath{\mathit{EVT}}}
\newcommand{\agtset}{\ensuremath{\mathit{AGT}}}        




\newcommand{\STIT} {{\textsf{STIT}}}              \newcommand{\CSTIT}{{\textsf{CSTIT}}}            \newcommand{\DSTIT}{{\textsf{DSTIT}}}            

\newcommand{\LCSTIT}{}
\newcommand{\LDSTIT}{}




\newcommand{\Light}{\mathit{Light}}         \newcommand{\Heads}{\mathit{Heads}}         \newcommand{\Tails}{\mathit{Tails}}         





\title{Alternative axiomatics and complexity of deliberative \STIT\ theories}


\author{Philippe Balbiani, Andreas Herzig and Nicolas Troquard
\\ Institut de recherche en informatique de Toulouse (IRIT), France
\\ \small{\texttt{balbiani,herzig,troquard@irit.fr}}
}

\date{}


\begin{document}

\maketitle

\begin{abstract}
We propose two alternatives to Xu's axiomatization of the Chellas \STIT.
The first one also provides an alternative axiomatization of
the deliberative \STIT.
The second one starts from the idea that the historic necessity operator
can be defined as an abbreviation of operators of agency, and
can thus be eliminated from the logic of the Chellas \STIT.
The second axiomatization also allows us to establish
that the problem of deciding the satisfiability of a \STIT\ formula
without temporal operators is NP-complete in the single-agent case, and
is NEXPTIME-complete in the multiagent case, both for the deliberative
and the Chellas' \STIT.
\end{abstract}

\tableofcontents

\section{Introduction}


\STIT\ theory is one of the most prominent accounts of agency in
philosophy of action. It is the logic of constructions of the form
`agent  sees to it that  holds'.
While \STIT\ has played an important role in philosophical logic
since the 80ies, it seems to be fair to say that its mathematical aspects
have not been developed to the same extent. Most probably the reason is that
\STIT's models of agency are much more complex than
those existing for other modal concepts (such as say necessity, belief,
or knowledge): first, the `seeing-to-it-that' modalities interact
(or perhaps better: must be guaranteed not to interact)
because the agents' choices are supposed to be independent;
second there is another kind of modality involved, viz.\
the `master modality' of historic necessity.
There are also temporal modalities, but just as most of the other
proof-theoretic approaches to \STIT, we do not investigate
these here.

As a consequence, proof systems for \STIT\ are rather complex, too.
To our knowledge the following have been proposed in the literature.

\begin{itemize}
\item Xu provides Hilbert-style axiomatizations in terms of
the historic necessity operator and Chellas' \STIT\ operator
\cite[Chap.\ 17]{belnap01facing},
without considering temporal operators.
As the deliberative \STIT-operator can be expressed in terms of Chellas'
(together with the historic necessity operator),
the axiomatization transfers to the deliberative \STIT.
Xu proves their completeness (without considering the temporal dimension),
by means of canonical models, and proves decidability by means of filtration.
Besides, Xu also gives a complete axiomatization of the one-agent
achievement \STIT\ \cite[Chap.\ 16]{belnap01facing}.

\item Wansing provides a tableau proof system for the deliberative \STIT\ \cite{wansing06aiml}.
The system is complete, but does not guarantee
termination, and thus ``is not tailored for defining tableau
algorithms'' \cite{wansing06aiml}.

\item D\'{e}gremont gives a dialogical proof procedure for the
deliberative \STIT\ \cite{Degremont}. Again, the system is complete,
but does not guarantee termination, and can therefore
only be used to build proofs by hand.
\end{itemize}

In this note, we focus on the so-called Chellas \STIT\ named after his
proponent \cite{chellas69phd, chellas92agency}. The original operator
defined by Chellas is nevertheless notably different since it does not
come with the principle of independence of agents that plays a central
role here. Following its presentation in \cite{horty95deliberative},
we use the term \CSTIT\ to refer to the logic of that modal operator.
We show that Xu's axiomatics of the logic of the Chellas \STIT\ can be
greatly simplified. After recalling it (Section \ref{sec:background})
we propose an alternative one and prove its completeness  (Section \ref{sec:alterAx}).
Based on the latter we show that in presence of at least two agents,
the modal operator of historic necessity
can be defined as an abbreviation (Section \ref{sec:unsettling}).
This leads to a simplified semantics (Section \ref{sec:newSemantics}),
and to characterizations of the complexity of satisfiability
(Section \ref{sec:complexity}).



\section{Xu's axioms for the \CSTIT}
\label{sec:background}


Some preliminary remarks are due. In \cite[Chap.\ 17]{belnap01facing},
Ming Xu presents , an axiomatization for the basic
(that is, without temporal operators) \emph{deliberative} \STIT\ logic.
As pointed out, deliberative \STIT\ logic and Chellas' \STIT\ logic are
interdefinable and just differ in the choice of primitive operators.
Following Xu we refer to these two logics as the
\emph{deliberative \STIT\ theories}.
We here mainly focus on  with the Chellas \STIT\ operator as
primitive.

\subsection{Language}
The language of Chellas' \STIT\ logic is built from
a countably infinite set of atomic propositions  and
a countable set of agents .
To simplify notation we suppose that  is an initial subset
 of  (possibly  itself).

Formulas are built by means of the boolean connectives together
with modal operators of historic necessity and of agency in the standard way.
Usually these modal constructions are noted
 (` is settled') and
 (` sees to it that '),
where
. For reasons of conciseness we here prefer to use
 instead of ,
and 
instead of .
The language \LCSTIT\ of the Chellas \STIT\ is therefore defined by the following BNF:

where  ranges over  and  ranges over .
This provides a standard notation for the dual constructions
 and , respectively abbreviating
 and .

The language \LDSTIT\ of the deliberative \STIT\ is defined by:

Note that neither \LCSTIT\ nor \LDSTIT\ contain temporal operators.

The following function will be useful to compute
the number of symbols that are necessary to write down .

\begin{definition}
We define recursively a mapping  from formulas of \LCSTIT  \LDSTIT\
to  :
,
,
,
, and
.
\end{definition}


\goodbreak
\subsection{Semantics}
The semantics of the \CSTIT\ is extensively studied in Belnap et al.\ \cite{belnap01facing}.
It consists of a branching-time structure (BT)
augmented by the set of agents and a choice function (AC).
Here, we refer to BT + AC models as \STIT-models.

A \emph{BT structure} is of the form , where
 is a nonempty set of moments, and
 is a tree-like ordering of these moments: for any
,  and  in , if  and ,
then either  or  or .

A maximal set of linearly ordered moments from  is a \emph{history}.
When  we say that moment  is \emph{on} the history .
 is the set of all histories.
 denotes the set of histories passing through .
An \emph{index} is a pair , consisting of a moment  and a
history  from  (i.e., a history and a moment in that history).

A \emph{BT+AC model} is a tuple , where:
\begin{itemize}
\item  is a BT structure;

\item 
is a function mapping each agent and each moment  into a partition of ,
such that
    \begin{itemize}
    \item ;
    \item  for every ;
    \item for all  and all mappings 
          such that ,
          we have .
    \end{itemize}

\item  is valuation function .
\end{itemize}
The equivalence classes belonging to  can be thought of as possible choices
that are available to agent  at .
Given a history ,  represents the particular choice
from  containing , or in other words, the particular action
performed by  at the index .
We call the constraint of nonempty intersection of all possible simultaneous choices of agents
(or: strategy profile) the \emph{superadditivity constraint}.

A formula is evaluated with respect to a model and an index.\\

\begin{tabular}{lcl}
 & iff  & \\
 & iff & \\
 & iff &  and
                                                  \\
 & iff & \\
 & iff & \\

 & iff & \\
                      && and 
\end{tabular}\\

Hence historical necessity (or inevitability) at a moment  in a history is
truth in all histories passing through .
According to Chellas, an agent  sees to it that  in a moment-history pair 
if  holds on all histories that agree with 's current choice.

\emph{Validity in BT+AC structures}
is defined as truth at every moment-history pairs of every BT+AC-models.
A formula  is satisfiable in BT+AC structures if  is not valid
in BT+AC structures.

The following valid equivalences justify the interdefinability of
our \STIT-operators:
\begin{center}
\begin{tabular}{lcl}
 &  &  \\
 &  & 
\end{tabular}
\end{center}



\goodbreak
\subsection{Axiomatics}
Xu gave the following axiomatics of Chellas' \CSTIT:
\begin{itemlist}{(AAIA)}
  \item[S5()]     the axiom schemas of S5  for 
  \item[S5()]        the axiom schemas of S5 for every 
  \item[(\InclBox{i})] 
  \item[(AIA)]
  
\end{itemlist}
The last item is a family of \emph{axiom schemes for
independence of agents} that is parameterized by the integer .\footnote{
Xu's original formulation of (AIA) is
       for .
The difference predicates  express that
 are all distinct.
They are defined from an equality predicate  whose domain is .
Formally we have to add the axioms:
, and
\\ \centerline{
.
}
In consequence Xu's axiomatics has to contain axioms for equality.
We here preferred not to introduce equality in order to stay with the same
logical language throughout.

Clearly, each of our (AIA) can be proved from Xu's original (AIA).
The other way round, given  and pairwise different ,
suppose w.l.o.g.\ that  for .
Then one can prove Xu's (AIA)
\\ \centerline{
  
}
from our (AIA)
\\ \centerline{
  
}
by appropriately choosing  to be  for all those 
that are not among : as 
and  hold, these conjuncts can be dropped
from our (AIA).

}

\begin{remark}
As (AIA) implies (AIA), the family of schemas can be replaced by
the single (AIA) when  is finite.
\end{remark}



Xu's system has the standard inference rules of modus ponens and
necessitation for . From the latter necessitation rules for
every  follow by axiom (\InclBox{i}).

\begin{theorem}[\mbox{\cite[Chapter 17]{belnap01facing}}]\label{XuCompleteness}
A formula  of \LCSTIT\ is valid in BT+AC structures iff  is provable from
the schemas S5(), S5(), (\InclBox{i}), and (AIA) by the rules of
modus ponens and -necessitation.
\end{theorem}


Xu's decidability proof proceeds by building a canonical model
followed by filtration \cite[Theorems 17-18]{belnap01facing}.
Although he does not mention complexity issues,
when decidability is proved by canonical model construction
from which a finite model is obtained by filtration, then
``a NEXPTIME algorithm is usually being employed''
\cite[Appendix C, p.\ 515]{Blackburn:2001:ML}.
Therefore it can be expected that the problem of deciding
the satisfiability of a given formula of \LCSTIT\
is in NEXPTIME.
We shall characterize complexity precisely in Section \ref{sec:complexity}.





\goodbreak
\section{An alternative axiomatics }\label{sec:alterAx}


We now prove that (AIA) can be replaced by the family of axiom schemes
\begin{itemlist}{(\InclBox{i})}
  \item[(AAIA)]
     \hfill for 
\end{itemlist}
We call (AAIA) the \emph{alternative axiom schema for independence of agents}.
Just as Xu's (AIA), (AAIA) involves  agents.

\begin{lemma}[validity of AAIA]\label{lem:validAaia}
For each ,

is valid in BT+AC structures.


\begin{pf}
See Annex.
\end{pf}
\end{lemma}

To warm up, we first prove that our (AAIA) implies Xu's (AIA).

\begin{lemma}\label{ortho-theo}
The schema (AIA) is provable from
S5(), S5(), (\InclBox{i})
and:
\begin{itemlist}{(\InclBox{i})}
  \item[(AAIA)]

\end{itemlist}
by modus ponens and -necessitation.


\begin{pf}
We establish the following deduction:
\begin{enumerate}
 \item  \hfill from axiom (AAIA), substituting  for 
\item  \hfill from previous line by S5()
\item     \hfill from previous line by S5()
\item  \hfill from previous line by K()
\item  \\
 \hfill from previous line by -necessitation and K()
\item  \hfill from previous line by S5()
\item  \\    
\hfill from previous line by (\InclBox{i}) axiom
and S5()

\end{enumerate}
\end{pf}

\end{lemma}

We turn back to an arbitrary number of agents.

\begin{lemma}
\label{provable-lemma}
Every schema (AIA) is provable from
S5(), S5(), (\InclBox{i})
and (AAIA)
by the rules of modus ponens and -necessitation.

\begin{pf}
We proceed by induction on . The base case  is settled
by \mbox{Lemma \ref{ortho-theo}}. Now, suppose AIA is
provable:

We prove AIA with the following steps.
\begin{enumerate}
\item   \hfill by induction
hypothesis (AIA)
\item   \hfill from
previous line by (AAIA)
\item  
\hfill from previous line by K()
\item \\
 \hfill from previous line by S5()
\item   \hfill from
previous line by S5()
\item \\
 \hfill from previous line by -necessitation and K()
\item \\
 \hfill from previous line by (\InclBox{i}) axiom and S5()
\item   \hfill from
previous line by S5()
\end{enumerate}

\end{pf}

\end{lemma}

\begin{theorem}
A formula of \LCSTIT\ is valid in BT+AC structures iff it is provable from the
axiom schemas S5(), S5(), (\InclBox{i}) and (AAIA) by the rules
modus ponens and -necessitation.

\begin{pf}
First, observe that Xu's axiomatics and ours only differ by the schemas
(AIA) and (AAIA).

Soundness follows from:
\begin{enumerate}
\item the validity of our schemas AAIA (see Lemma \ref{lem:validAaia}),
\item the validity of the rest of the axioms, and
\item the fact that modus ponens and -necessitation preserve validity.
\end{enumerate}
The last two points are warranted by the soundness of Xu's axioms
(Theorem~\ref{XuCompleteness}).

Completeness follows from provability of Xu's (AIA) from our (AAIA)
(see Lemma \ref{provable-lemma}).
As observed above, the rest of Xu's axioms is directly present in our axiomatics.
\end{pf}

\end{theorem}

An alternative axiomatics of the deliberative \STIT\ is obtained viewing
 as an abbreviation of .



\goodbreak
\section{Historic necessity is superfluous
         in presence of two agents or more}\label{sec:unsettling}


In this section, we suppose that , i.e.\
there are at least agents  and~.

The equivalence 
is provable from (AAIA), (\InclBox{i}) and S5().
This suggests that  can be viewed as an abbreviation of
.
Let us take this as an axiom schema.

\begin{itemlist}{(\InclBox{i})}
  \item[Def()]
  
\end{itemlist}

Pushing this further we can prove that under Def(), axiom (AAIA) can be
replaced by the family of axiom schemas of general permutation:
\begin{itemlist}{(\InclBox{i})}
\item[(GPerm)]
    \hfill for 
\end{itemlist}
Note that similar to Xu's axiomatization, if  is finite then the single
schema (GPerm) is sufficient.

The next lemma establishes soundness.

\begin{lemma}\label{lem:validGperm}
(GPerm) is valid in BT+AC structures.

\begin{pf}
See Annex.
\end{pf}
\end{lemma}

Now we prove that the principles of the preceding section can be derived.

\begin{lemma}\label{lem:BoxIsS5}
The axiom schemas of S5(),
and the schemas (\InclBox{i}) and (AAIA)
are provable from Def(), S5() and (GPerm)
by the rules of modus ponens and -necessitation,
and -necessitation is derivable.

\begin{pf}
First let us prove that the logic of  is S5.
Clearly the K-axiom

is provable using standard modal principles, and the T-axiom
 follows from S5() and S5().
It remains to prove the 5-axiom :
\begin{enumerate}
\item
  \hfill by S5();
\item
  \hfill by (GPerm)
                                                      and K();
\item
  \hfill by S5()
                                                      and K();
\item
  \hfill by (GPerm);
\item
  \hfill from lines 1-4.
\end{enumerate}
Finally, -necessitation is derivable
by applying first -necessitation and then -necessitation.

Concerning (AAIA) it is easy to see that under Def()
it is an instance of (GPerm), for all .
It remains to prove (\InclBox{i}).
Let us show that :
\begin{enumerate}
\item              \hfill by S5();
\item  \hfill by (GPerm);
\item  \hfill from lines 1-2.
\end{enumerate}

\end{pf}

\end{lemma}


\begin{theorem}\label{cool-theo}
Suppose .
Then a formula of \LCSTIT\ is valid in BT+AC structures
iff it is provable from
S5(),        
Def(),    
and (GPerm)                 
by the rules of modus ponens and -necessitation.
\end{theorem}

\begin{remark}
If  then the validities of \LCSTIT\ are axiomatized by Def(),
S5(), S5(), and
.
Moreover, the Church-Rosser axiom
.
can be proved from S5(), S5() and (GPerm).
Therefore  logic with two agents is a so-called product logic,
alias a two-dimensional modal logic \cite{Marx99,GabbayEtAl03}.
Such product logics are characterized by the permutation axiom

together with the Church-Rosser axiom.
Hence the logic of the two-agent \STIT\ is nothing but the product
S5 = S5S5.
\end{remark}





\goodbreak
\section{A simpler semantics}\label{sec:newSemantics}


All axiom schemes are in the Sahlqvist class \cite{Blackburn:2001:ML},
and therefore have a standard possible worlds semantics.

\emph{Kripke models} are of the form
, where
 is a nonempty set of possible worlds,
 is a mapping associating to every  an equivalence relation  on ,
and  is a mapping from  to the set of subsets of .
We impose that  satisfies the following property:

\begin{definition}[general permutation property]
We say that  satisfies the \emph{general permutation property} iff
for all  and for all ,
if  then there is  such that:
 and 
for every .
\end{definition}


We have the usual truth condition:

and the usual definitions of validity and satisfiability.

\begin{lemma}\label{lem:prop-rel}
For every , and every ,  satisfies
the following properties:
\begin{enumerate}
\item If  then .

\item  is an equivalence relation for every .

\item .
\end{enumerate}

\begin{pf}
(1) follows from the validity of
 (due to (GPerm)), and
the validity of
 (due to (GPerm),
given that ).

(2) follows from (1) and the fact that the S5-axioms are valid for 
(see Lemma \ref{lem:BoxIsS5}).

In (3), the right-to-left inclusion

follows from the inclusion
.
For the left-to-right inclusion
suppose .
Hence there are  such that
.
As all the  are equivalence relations we may suppose w.l.o.g.\ that
.
\begin{itemize}
\item If  is odd then
 by (1).
The latter is equal to  by (2).

\item If  is even then
 by (1) and (2).
The latter is equal to                 again by (1),
and to  by (2),
which is equal to  because  is an equivalence relation.
\end{itemize}
It follows that .

\end{pf}
\end{lemma}

\begin{theorem}\label{theo:Sahlqvist}
A formula of \LCSTIT\ is valid in Kripke models satisfying the general permutation property
iff it is provable from
\begin{itemlist}{(GPerm)}
\item[S5()]     the axiom schemas of S5 for every 
\item[Def()] 
\item[(GPerm)] 
                    \hfill for 
\end{itemlist}
by the rules of modus ponens and -necessitation.

\begin{pf}
If  is finite then
Sahlqvist's Theorem warrants that our axiomatics of Section \ref{sec:unsettling}
is sound and complete w.r.t.\ Kripke models satisfying the general permutation property.
We show in the annex that this can be extended to the infinite case.\end{pf}
\end{theorem}





\goodbreak
\section{Complexity}\label{sec:complexity}


The axiom system of the preceding section allows us to characterize the complexity
of satisfiability of \STIT\ formulas.
We study separately the cases of Chellas' \STIT\ and of the 
deliberative \STIT.

\subsection{Complexity of Chellas' \STIT}

First, satisfiability of \CSTIT-formulas
can be decided in nondeterministic exponential time.

\begin{lemma}\label{lem:ComplexCstitUpper}
The problem of deciding satisfiability of a formula of \LCSTIT\ 
is in NEXPTIME.

\begin{pf}
This can be proved by the standard filtration construction,
which establishes that in order to know whether a formula 
is satisfiable in the Kripke models of Section \ref{sec:newSemantics}
it suffices to consider models having at most  possible worlds.
See the annex for details.
\end{pf}

\end{lemma}

In the rest of the section we show that the upper bound is tight
if there are at least two agents. As usual we start with the two-agents case.

\begin{lemma}\label{lem:ComplexCstitLowerTwoAgents}
If  then the problem of deciding satisfiability of
a formula of \LCSTIT\ is NEXPTIME-hard.

\begin{pf}
Remember our observation at the end of Section \ref{sec:unsettling}:
when  then 
is nothing but the product logic S5S5.
We can then apply a result of Marx in \cite{Marx99}, who proved that the problem of
deciding membership of  in S5S5 is NEXPTIME-hard.
(Actually Marx also proved membership in NEXPTIME.)
\end{pf}
\end{lemma}

Hence two-agent \CSTIT\ logic is NEXPTIME-complete.
Now we state NEXPTIME-completeness for any number of agents greater than .



\begin{theorem}\label{theo:cstitNexptComplete}
If  then the problem of deciding satisfiability
of a formula of \LCSTIT\ is NEXPTIME-complete.

\begin{pf}
See Annex.
\end{pf}

\end{theorem}

It remains to establish the complexity of single-agent \CSTIT.
It turns out that it has the same complexity as S5.

\begin{theorem}\label{theo:cstitSingleagentNpComplete}
If  then the problem of deciding satisfiability
of a formula of \LCSTIT\ is NP-complete.

\begin{pf}
This can be proved by establishing an upper bound on the size of the models that is
quadratic in the length of the formula under concern.
\end{pf}

\end{theorem}

\begin{remark}
Intriguingly, while one-agent \STIT\ has the same complexity as S5, and
two-agent \STIT\ has the same complexity as S5,
-agent \STIT\ \emph{does not} have the same complexity as S5:
while Xu's proof establishes decidability of \LCSTIT-formulas for any number of agents,
it was proved by Maddux that S5 is undecidable \cite{MarxMikulas00}.
\end{remark}

Thus we have characterized the complexity of satisfiability of \CSTIT\
formulas for all cases.


\goodbreak
\subsection{Complexity of the deliberative \STIT}

The complexity results for Chellas' \STIT\ do not immediately transfer to \DSTIT.
Indeed, the definition of the deliberative \STIT\ from the \CSTIT\ through
 = 
does not directly provide a lower bound for the deliberative \STIT\
because this is not a polynomial transformation.
We now establish these results by giving polynomial translations
from \CSTIT\ to \DSTIT and vice versa.


Let  be any formula of \LDSTIT, and
let  be the set of subformulas of .
Let  be a set of (pairwise distinct) atoms none of
which occurs in .
Every  abbreviates the subformula  of .
We recursively define equivalences (`biimplications') that capture
the logical relation between  and .

\begin{definition}
We define:
\begin{center}
\begin{tabular}{lll}
 &         = &  ()\\
 & = & ()\\
& =  & ()\\
 & = & ()\\
 & = & (
\end{tabular}
\end{center}
\end{definition}

\begin{definition}
We define the translation  from \DSTIT\ formulas to \CSTIT\ formulas as:
.
\end{definition}


\begin{theorem}\label{theo:dstitNexptHard}
 is a polynomial translation from \LDSTIT\ to \LCSTIT, and
for every formula  of \LDSTIT,
 is satisfiable iff  is satisfiable.

\begin{pf}
See Annex.
\end{pf}
\end{theorem}

It follows that the problem of deciding whether a formula of \LDSTIT\ is
satisfiable is in NEXPTIME.
We now prove that this bound is tight.

\begin{definition}
We define equivalences  such that

and  if  is an atomic formula or if its main
logical connector is boolean.
\end{definition}

\begin{definition}
We define the translation  from \LCSTIT\ to \LDSTIT\ as:
.
\end{definition}

\begin{theorem}\label{theo:dstitInNexpt}
 is a polynomial translation from \LCSTIT\ to \LDSTIT, and
for every formula  of \LCSTIT,
 is satisfiable iff  is satisfiable.

\begin{pf}
The proof is analogous to that of Theorem \ref{theo:dstitNexptHard}.
\end{pf}
\end{theorem}

Together, Theorems \ref{theo:cstitNexptComplete}, \ref{theo:cstitSingleagentNpComplete},
\ref{theo:dstitNexptHard} and \ref{theo:dstitInNexpt} entail:

\begin{corollary}
The problem of deciding whether a formula of \LDSTIT\ is satisfiable is
NEXPTIME-complete if , and
it is NP-complete if .
\end{corollary}









\section{Conclusion}\label{sec:conclusion}


In this note we have established NEXPTIME-completeness of the satisfiability problem
of formulas of Chellas' \STIT\ and of the deliberative \STIT\
for the case of two or more agents.
All our complexity results appear to be new.

Our new axiom system for \STIT\ of Section \ref{sec:alterAx}
is an interesting alternative to Xu's.
It highlights the central role of the well-known equivalences
 and
, for 
in theories of agency:
as we have shown, they allow to capture independence of agents just as
Xu's schema (AIA) does.

For the case of more than two agents, Section \ref{sec:unsettling}
provides a quite simple axiom system that is made up of very basic
modal principles, and moreover, does without historic necessity.

As we have pointed out in Section \ref{sec:alterAx},
an alternative axiomatics for the deliberative \STIT\ follows
straightforwardly.
We do not know whether the redundancy of historic necessity that we have established
for the \CSTIT\ in Section \ref{sec:unsettling} transfers to the deliberative \STIT.

\section*{Acknowledgements}


Thanks to Olivier Gasquet for comments and discussions.

\bibliographystyle{alpha}
\bibliography{biblio}


\addcontentsline{toc}{section}{Annex A: Proofs}
\section*{Annex: Proofs}


\subsection*{A.1: Proof of Lemma \ref{lem:validAaia}}

In order to prove the validity of every schema
\begin{itemlist}{(\InclBox{i})}
  \item[(AAIA)]
     \hfill for 
\end{itemlist}
in BT+AC structures, we show that for every ,  and
 there is  such that
  for every .

Consider the strategy  such that
, and
 for every .
By the superadditivity constraint there is some  such that
. Hence
, and
 for .


\subsection*{A.2: Proof of Lemma \ref{lem:validGperm}}

We have to prove the validity of every schema
\begin{itemlist}{(\InclBox{i})}
\item[(GPerm)]
  \hfill for 
\end{itemlist}
in BT+AC structures.

A look at the proof of Lemma \ref{lem:validAaia} shows that

is valid in BT+AC structures.
It therefore suffices to show the validity of
.
The latter is the case because
(1)  is valid
(due to validity of axiom (\InclBox{i})), and
(2)  is valid
(due to validity of S5()).


\subsection*{A.3: Proof of Theorem \ref{theo:Sahlqvist}}

We prove the theorem for the infinite case, i.e.\ .
In this case the general permutation property is no longer a first-order property,
and Sahlqvist's result does not apply, i.e.\ the canonical model does not
necessarily satisfy the general permutation property.

Let  be a formula that is consistent w.r.t.\ the axiomatic system
of Section \ref{sec:unsettling}.
Let  be the canonical model associated to this system.
By arguments following the lines of those in the proof of Lemma \ref{lem:prop-rel} we have:
\begin{itemize}
\item ,  is an equivalence relation;
\item  such that , ;
\item .
\end{itemize}
By the truth lemma we may suppose that  is generated via 
from a possible world  such that .
Let  be the filtration of  w.r.t.\ 
(just as done in Annex A.4).
Note that  for all  not occurring in .
This allows us to show that  satisfies the general permutation property.
From this completeness follows (via the filtration lemma).




\subsection*{A.4: Proof of Lemma \ref{lem:ComplexCstitUpper}}

Let   be a Kripke model such that
every  is an equivalence relation and
 satisfies the general permutation property.
Let  be a world and  a formula of \LCSTIT\ such that
.
Suppose that  is generated from  through .
(This can be supposed w.l.o.g.\ because of Lemma \ref{lem:prop-rel}
of Section \ref{sec:newSemantics}.)
 being the set of all subformulas of ,
we say  and  are -equivalent iff
, and note .
Let  denote
the equivalence class of  modulo .

We construct  such that:
\begin{itemize}
\item 
\item         iff
      
\item  for all 
\end{itemize}
Remark that for all ,
if  does not occur in  then .

We must check that every  is an equivalence relation,
that  verifies the general permutation property,
that for all  and ,
 iff , and
that  is exponential in the length of :
\begin{enumerate}
\item Every  is an equivalence relation, and
 satisfies the general permutation property.

This follows from the definition of .

\item .

This follows from the filtration lemma (see \cite{Blackburn:2001:ML} for details).


\item 

Note that members of  are subsets of states of  satisfying
exactly the same formulas of .
Thus  corresponding to the set
of subsets of .
We can show by induction on  that 
and then conclude.

\end{enumerate}

Hence,  \LCSTIT,
if  is satisfiable then
 such that  and
there is  such that .
It allows us to propose a decision procedure with input  \LCSTIT,
and which works as follows:
guess an integer  and a model 
such that ;
then check whether there is a  such that .




\subsection*{A.5: Proof of Theorem \ref{theo:cstitNexptComplete}}

The upper bound is given by Lemma \ref{lem:ComplexCstitUpper}.

To establish the lower bound consider 
the set of formulas where only the agent symbols  and  occur.
We show that deciding satisfiability of any formula of that fragment
is NEXPTIME-hard, for any  such that .
If  is just  this holds
by Lemma \ref{lem:ComplexCstitLowerTwoAgents}.
Else we prove that if  then
the logic of Kripke models for  is a conservative extension of
that for .


Let  be any formula containing only  and .

For the left-to-right direction, suppose  is valid in all Kripke models
for the set of agents .
By Theorem \ref{cool-theo},  can then be proved from
axioms (GPerm), (Perm01), S5() and S5() with
the rules of modus ponens, - and  -necessitation.
Therefore  is also provable from the `bigger' axiomatics for .

For the right-to-left direction, suppose there is a Kripke model
 for the set of agents  and a  such that
, where 
associates to every  an equivalence relation  on .
We are going to build a Kripke model  for the bigger set of agents 
such that .
Let  such that
 with
,
 and
 for .
Clearly , too.
It remains to show that  is indeed a Kripke model
as required in Section \ref{sec:newSemantics}.
By item  of Lemma \ref{lem:prop-rel}
every  is an equivalence relation, so we only have to show that
the general permutation property holds in :
if  then
there is  such that:
 and 
for every 
(cf.\ Lemma \ref{lem:validGperm}).
First we show that for every  and 
we have .
\begin{itemize}
\item If  and  then trivially
.
\item If  and  then

\item If  and  then

\item If  and  then

\item If  and  then

\item If  and  then

\item if  and  then

\end{itemize}
(The identities in all these items hold because
 and  permute by item  of Lemma \ref{lem:prop-rel}, and
because  and  are equivalence relations.)
Thus  implies .
We have to show that for every  there is  such that:
 and , for every .
\begin{itemize}
\item
For ,  implies that
 by item  of Lemma \ref{lem:prop-rel},
and the latter implies that .
Therefore there is a  such that  and .
\item
For , take :
 implies that  by definition of , and
we have  because every  is an equivalence relation
(for  this is the case by item  of Lemma \ref{lem:prop-rel}).
\end{itemize}




\subsection*{A.5: Proof of Theorem \ref{theo:dstitNexptHard}}

The proof is done via the following lemmata.

\begin{lemma}\label{lem:dsat2trcsat}
For all formulas  in the language of \DSTIT,
if    is satisfiable
then  is satisfiable.

\begin{pf}
Suppose there is  such that  .
We build a model  such that 
by setting
 for all atoms  appearing in , and

for all .

By induction on the structure of  we show that 
for all  and all .
(Details left to the reader.)

Hence , and also
      .
Since , we have  by construction of .
Thus  ,
in other words
      .
\end{pf}
\end{lemma}

\begin{lemma}\label{lem:trdsat2csat}
For all formulas  in the language of \DSTIT,
if    is satisfiable
then  is satisfiable.

\begin{pf}
Suppose there is 
such that  . Thus
.
By induction on the structure of  we show that

for all  and all .
(Details left to the reader.)

Thus  , and .
Hence .
\end{pf}
\end{lemma}


\begin{lemma}
 is a polynomial transformation.

\begin{pf}
We easily show that  and
.
Then, .
We conclude that .
Remark that .
Moreover, for every formula  in the language of \CSTIT,
.
As a result,
.

\end{pf}
\end{lemma}





\end{document}
