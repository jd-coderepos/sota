\section{Results}\label{results}

In this section, we discuss the results of our user study. We label the login performance of participants in session 1 and session 2 as \textit{login 1} and \textit{login 2}, respectively. We evaluated the usability of CuedR via all metrics suggested in the literature~\cite{design_space}: memorability, login time, number of login attempts, and user feedback. In addition, we analyzed the impact of portfolios on login performance and user perceptions on the effectiveness of different cues. We also discuss the results of pilot study on the memorability of multiple CuedR passwords.

\subsection{Memorability}\label{memory}

We observed a $100$\% login success rate for CuedR in both \textit{login 1} and \textit{login 2}. In \textit{login 1}, all the participants successfully recognized the keywords on the first attempt. In \textit{login 2}, $89$\% of participants succeeded on the first attempt to recognize all six keywords. The other four participants ($11$\%) recognized five of out of six keywords on the first attempt. Three participants corrected their mistake on the second attempt, and the other participant succeeded on the third attempt. 

\subsection{Registration and Login Time}\label{time} 

The mean time for registration was $31.2$ seconds (median: $30$ seconds, SD: $10.5$ seconds). The mean time for successful login were $25.7$ seconds (median: $24.0$ seconds, SD: $8.3$ seconds) in \textit{login 1}, and $38.0$ seconds (median: $39.0$ seconds, SD: $11.4$ seconds) in \textit{login 2}. A paired-samples t-test reveals that login time in \textit{login 1} was significantly less than that in \textit{login 2}, $t(36)=7.81$, $p<0.01$. This was expected, as participants performed {\em login 1} shortly after learning the keywords. To note, the reported registration and login time include the time to download images.

The login time in CuedR is in line with that in prior recognition based schemes offering $28$ bits of entropy~\cite{image_type,text_recog}. We note that our results for login time are likely conservative, since they measure initial use. A recent field study~\cite{geopass2} reveals that login time decreases with the frequent use of a scheme due to training effects. These findings are in agreement with our user feedback, where the participants reported that with practice, they could quickly recognize the keywords (see Table~\ref{tab:usability}). 

\begin{figure}[t]
\centering
\includegraphics[width=70mm, height=36mm]{graphs/cues}
\caption{Responses to the question: ``How often did the following cues assist you in recognizing keywords in CuedR?"}
\label{fig:cues}
\end{figure}


\begin{table*}[t]
\renewcommand{\arraystretch}{1.3}
\caption{Questionnaire responses for the usability of CuedR. Scores are
  out of 10. * indicates that scale was reversed. SD: Standard Deviation}  \centering
\vspace{0.2cm}
\begin{tabular}{c@{}m{8.5cm}rrrr@{}}
\hline
\hspace{0.2cm}&\multicolumn{1}{c}{Questions}&\multicolumn{1}{c}{Mode}&\multicolumn{1}{c}{Median}&\multicolumn{1}{c}{Mean}&\multicolumn{1}{c}{SD}\\ 
\cline{2-6}
& I could easily sign up with CuedR&$10$\phantom{a}&$9.0$\phantom{ab}&$9.0$\phantom{a}&$1.3$\phantom{m}\\
\cline{2-6}
& The login using CuedR was easy&$10$\phantom{a}&$10.0$\phantom{ab}&$9.5$\phantom{a}&$0.7$\phantom{m}\\ 
\cline{2-6}
& Keywords are easy to remember in CuedR &$10$\phantom{a}&$10.0$\phantom{ab}&$9.4$\phantom{a}&$0.8$\phantom{m}\\ 
\cline{2-6}
& *I found CuedR too time-consuming \newline (i.e., I did not find CuedR too time consuming) &$10$\phantom{a}&$7.0$\phantom{ab}&$6.4$\phantom{a}&$2.6$\phantom{m}\\
\cline{2-6}
& With practice, I could quickly enter my password in CuedR &$10$\phantom{a}&$10.0$\phantom{ab}&$9.8$\phantom{a}&$0.6$\phantom{m}\\
\cline{2-6}
& I could easily use CuedR every day&$10$\phantom{a}&$9.0$\phantom{ab}&$8.8$\phantom{a}&$1.3$\phantom{m}\\ 
\cline{2-6}
& I could easily use CuedR every week&$10$\phantom{a}&$9.0$\phantom{ab}&$9.0$\phantom{a}&$1.3$\phantom{m}\\ 
\hline
\end{tabular}
\label{tab:usability}
\end{table*}


\subsection{Impact of Portfolios on Usability}\label{im_portfolio}

In our study, all the participants succeeded to recognize their keywords irrespective of the type of portfolios in both \textit{login 1} and \textit{login 2}. In \textit{login 1}, no participant made any mistake in any portfolio, and thus there was no difference among portfolios for the number of attempts to succeed. In~\textit{login 2}, four participants ($11$\%) required multiple attempts to succeed (see the results for \textit{Memorability}), where one-way ANOVA test results show that there was no significant difference among portfolios in terms of the number of attempts required to successfully recognize the keywords, $F(17,220)=1.16$, $p=0.31$. In addition, we conducted a post-hoc pairwise comparison using Tukey's HSD (Honestly Significant Difference), which reveals no significant difference between any pair of portfolios for the number of attempts to succeed. 

Our one-way ANOVA test results demonstrate that there was no significant difference among different portfolios in terms of the time to learn the keyword during registration, $F(17,220)=0.76$, $p=0.71$, or recognize the keyword either in \textit{login 1}, $F(17,220)=1.16$, $p=0.31$, or in \textit{login 2}, $F(17,220)=0.59$, $p=0.87$. In addition, we conducted a post-hoc pairwise comparison using Tukey's HSD, which did not find any significant difference between any pair of portfolios in either registration time or in login time. These findings indicate that the usability in recognizing keywords did not vary significantly across different portfolios used in our study.

\subsection{User Perception on the Efficacy of Different Cues}\label{effect_cues} 

To understand user perception on the importance of different cues in aiding recognition, we asked them at the end of second session, ``How often did the following cues assist you in recognizing keywords in CuedR?" In response, for each cue they selected one of five options: \textit{Never}, \textit{Rarely}, \textit{Sometimes}, \textit{Often}, or \textit{Always}. Our results show that participants report using multiple cues to varying degrees to help recognize their keywords (see Figure~\ref{fig:cues}). In particular, $92$\% of participants reported that the images were \textit{always} or \textit{often} helpful to recognize keywords, while $62$\%, $40$\%, and $14$\% of participants, respectively reported that spatial, phrase, and numerical cues were \textit{always} or \textit{often} helpful in recognizing keywords. The participants' diverse choices for cues to aid recognition and their high login success rate support our anticipation that letting users choose the appropriate cue(s) to their learning process aids the memorability for system assigned random passwords. 

\subsection{User Feedback on Usability and Applicability}\label{feedback}

We asked the participants to answer two sets of $10$-point Likert-scale questions ($1$: \textit{strong disagreement}, $10$: \textit{strong agreement}) at the end of the second session. We reversed some of the questions to avoid bias; the scores marked with (*) were reversed before calculating the modes, medians, and means. So, a higher score always indicates a more positive result for CuedR. To design the questionnaire, we carefully followed the guidelines provided in the existing password literature~\cite{geopass,ccp,passpoint3}, including using nearly identical questions to those from other studies.

\begin{table}[b]
\renewcommand{\arraystretch}{1.3}
\caption{The applicability of CuedR for different online
  accounts. Scores are out of 10.}
\vspace{0.2cm}
\centering
\begin{tabular}{c@{}lcccc}
\hline
\hspace{0.2cm}&\multicolumn{1}{c}{Online accounts}&Mode&Median&Mean&SD\\ 
\hline
& Bank&$10$&$8.0$&$7.4$&$2.6$\\ 
\cline{2-6}
& E-mail&$10$&$9.0$&$8.1$&$2.1$\\ 
\cline{2-6}
& Social Networking&$10$&$9.0$&$7.7$&$2.4$\\ 
\cline{2-6}
& University Portal&$10$&$8.0$&$8.2$&$1.9$\\ 
\cline{2-6}
& E-commerce&$10$&$9.0$&$7.8$&$2.5$\\ 
\hline
\end{tabular}
\label{tab:applicable}
\end{table}


\subsubsection{Usability} Participants showed a high degree of satisfaction with the usability (e.g., memorability, ease of login, ease of using either weekly or daily) of CuedR. Their feedback was also positive (mode, median, and mean higher than neutral) regarding login time, and they indicated that with practice they could log in quickly using CuedR (see~Table~\ref{tab:usability}). In our study, we could not test the usability of implicit feedback for CuedR, since most users did not make enough login mistakes to gain experience with it.

\subsubsection{Applicability} At the end of second session, we asked $31$ of the participants,\footnote{We failed to ask the first six participants.} ``Do you want to use CuedR in real life as a replacement to traditional textual passwords?" $84$\% responded `Yes', $10$\% responded `Maybe', and two participants responded `No', where both of them mentioned that they would prefer traditional textual passwords in real life as they did not find any problems with them. User feedback about the applicability of CuedR in different online accounts is illustrated in Table~\ref{tab:applicable}.

\subsection{Pilot study: Memorability for Multiple CuedR Passwords}
It is common in password research to report a single-password study in the first article of a new authentication scheme, which helps to establish performance bounds and figure out whether multiple-passwords tests are worthwhile in future research. A recent survey~\cite{survey} reported that out of $25$ graphical password schemes proposed to date, only three have been evaluated through a multiple-password study, and none of these study results was reported in the first article. Since the use of multiple passwords is an important issue for deployment, however, we conducted a pilot study for multiple passwords, in addition to reporting the detailed results of a single-password study.

The study procedure was same as that in our single-password study, except that each participant was assigned three CuedR passwords ($18$ keywords, in total) instead of one. To administer this experiment, we created three different websites outfitted with CuedR and presented the sites to participants as tabs in an open browser window. Participants were free to select the order of websites at registration and login, but the tabs were arranged the same way every time. For this study, we recruited $11$ students ($9$ men, $2$ women) who came from various majors of our university. We believe that $11$ represents a suitable sample size for a pilot study~\cite{pccp_pilot}. 

In this study, all of the participants were able to log in successfully within three attempts in both \textit{login 1} (same day of registration) and in \textit{login 2} (one week after registration). In \textit{login 1}, nine participants ($82$\%) succeeded on the first attempt for all three CuedR passwords. One participant ($9$\%) succeeded to log in using two CuedR passwords on the first attempt, where she recognized $17$ keywords on the first attempt and corrected the lone mistake on the second attempt. Another participant succeeded on the first attempt for one CuedR password, where she successfully recognized $15$ keywords on the first attempt and corrected the mistakes on the second attempt.

In \textit{login 2}, six participants ($55$\%) succeeded on the first attempt to recognize all $18$ keywords. Four participants ($36$\%) successfully recognized $17$ keywords on the first attempt, i.e., they succeeded to log in using two CuedR passwords on the first attempt. For another CuedR password, two ($18$\%) of these four participants succeeded on the second attempt and other two participants succeeded on the third attempt. One participant ($9$\%) successfully recognized $16$ keywords on the first attempt. In particular, she succeeded to log in using one CuedR password on the first attempt and succeeded on the second attempt for other two CuedR passwords. 











