\documentclass[twocolumn]{article}
\usepackage{latexsym}	      \usepackage{hyperref}	      
\usepackage{makeidx}
\newcommand{\out}[1]{}        \newcommand{\ams}[1]{#1}      \usepackage[preserveurlmacro]{breakurl}
\makeindex

\setlength{\textwidth}{6.875in}
\setlength{\columnsep}{0.375in}
\setlength{\oddsidemargin}{-.1875in}
\setlength{\textheight}{8.875in}
\setlength{\topmargin}{0in}
\setlength{\headheight}{0in}
\setlength{\headsep}{0in}
\setlength{\parindent}{12pt}
\makeatletter
\def\section{\@startsection {section}{1}{0pt}{-3.25ex plus -1ex minus 
   -.2ex}{1.5ex plus .2ex}{\large\bf}}
\def\subsection{\@startsection {subsection}{2}{0pt}{-2ex plus -1ex minus 
   -.2ex}{1.5ex plus .2ex minus .3ex}{\@setfontsize\large\@xipt{13}\bf}}
\def\paragraph{\@startsection
   {paragraph}{4}{\z@}{2ex plus 1ex minus .2ex}{-1em}{\normalsize\bf}}
\long\def\@makecaption#1#2{
   \vskip 10pt 
   \setbox\@tempboxa\hbox{\sc #1: \it #2}
   \ifdim \wd\@tempboxa >\hsize   \sc #1: \it #2\par         \else                        \hbox to\hsize{\hfil\box\@tempboxa\hfil}  
   \fi}
\renewenvironment{theindex}
      {\if@twocolumn
        \@restonecolfalse
      \else
        \@restonecoltrue
      \fi
      \columnseprule \z@
      \columnsep 35\p@
      \twocolumn[\section*{\raisebox{4pt}[0pt][0pt]{\Large\bf \indexname}}]\@mkboth{\leftheader}{\indexname}\parindent\z@
      \parskip\z@ \@plus .3\p@\relax
      \let\item\@idxitem}
     {\if@restonecol\onecolumn\else\clearpage\fi}
\makeatother
\unitlength             .8 mm   

\newtheorem{defi}{Definition}[section]
\newtheorem{theo}{Theorem}
\newtheorem{prop}{Proposition}[section]
\newtheorem{lemm}{Lemma}
\newtheorem{coro}{Corollary}
\newtheorem{obs}{Observation}[section]
\newtheorem{exam}{Example}

\newenvironment{definition}[1]{\begin{defi} \rm \label{df-#1} }{\end{defi}}
\newenvironment{theorem}[1]{\begin{theo} \rm \label{th-#1} }{\end{theo}}
\newenvironment{proposition}[1]{\begin{prop} \rm \label{pr-#1} }{\end{prop}}
\newenvironment{lemma}[1]{\begin{lemm} \rm \label{lem-#1} }{\end{lemm}}
\newenvironment{corollary}[1]{\begin{coro} \rm \label{cor-#1} }{\end{coro}}
\newenvironment{observation}[1]{\begin{obs} \rm \label{obs-#1} }{\end{obs}}
\newenvironment{example}[1]{\begin{exam} \rm \label{ex-#1} }{\end{exam}}
\newenvironment{proof}{\begin{trivlist} \item[\hspace{\labelsep}\bf
Proof:]}{\hfill \end{trivlist}}
\newenvironment{remark}{\begin{trivlist} \item[\hspace{\labelsep}\bf Remark:]}{\hfill \end{trivlist}}

\newcommand{\df}[1]{Definition~\ref{df-#1}}
\newcommand{\thm}[1]{Theorem~\ref{th-#1}}
\newcommand{\pr}[1]{Proposition~\ref{pr-#1}}
\newcommand{\lem}[1]{Lemma~\ref{lem-#1}}
\newcommand{\cor}[1]{Corollary~\ref{cor-#1}}
\newcommand{\ob}[1]{Observation~\ref{obs-#1}}
\newcommand{\ex}[1]{Example~\ref{ex-#1}}

\newenvironment{itemise}{\begin{list}{}{\leftmargin 18pt
                        \labelwidth\leftmargini\advance\labelwidth-\labelsep
                        \topsep 4pt \itemsep 2pt \parsep 2pt}}{\end{list}}
\newenvironment{itemise2}{\begin{list}{{\bf --}}{\leftmargin 15pt
                        \labelwidth\leftmargini\advance\labelwidth-\labelsep
                        \topsep 2pt \itemsep 1pt \parsep 1pt}}{\end{list}}
\newenvironment{itemise3}{\begin{list}{*}{\leftmargin 12pt
                        \labelwidth\leftmargini\advance\labelwidth-\labelsep
                        \topsep 0pt \itemsep 0pt \parsep 0pt}}{\end{list}}

\newcommand{\phrase}[1]{\index{#1}{\em #1}}		\newcommand{\implies}{\Rightarrow}
\newcommand{\turn}{\vdash}                              \newcommand{\dbigcup}{\bigcup_{\uparrow}}		\newcommand{\nbigcup}{\bigcup_{\bullet}}		\newcommand{\nbigcap}{\bigcap_{\bullet}}		\newcommand{\bbigcup}{\overline{\bigcup}}		\newcommand{\bbigcap}{\overline{\bigcap}}		\newcommand{\nbbigcap}{\bbigcap_{\bullet}}		\newcommand{\fbbigcup}{\overline{\bigcup}^f}		\newcommand{\bbbigcup}{\overline{\bigcup}^2}		\newcommand{\dcup}{~~\makebox[0pt]{\LARGE}\makebox[0pt]{}~~}
\newcommand{\monus}{~~\makebox[0pt]{\LARGE}\makebox[0pt]{}~~}
\newcommand{\bigdvee}{~\makebox[0pt]{\Huge}\makebox[0pt]{}~}
\newcommand{\dvee}{\cdot\vee}
\newcommand{\dl}[1]{\mbox{\rm I\hspace{-0.75mm}#1}}     \newcommand{\dc}[1]{\mbox{\rm {\raisebox{.4ex}{\makebox [0pt][l]{\hspace{.2em}\scriptsize }}}#1}}
\newcommand{\IZ}{\mbox{\bf Z}}                          \newcommand{\pow}{{\cal P}}                             \ams{\newfont{\open}{msbm10}                            \newcommand{\IT}{\mbox{\open T}}                        \renewcommand{\IZ}{\mbox{\open Z}}                      \newfont{\fsc}{eusm10}                                  \renewcommand{\pow}{\mbox{\fsc P}}}                     \newcommand{\plat}[1]{\raisebox{0pt}[0pt][0pt]{}}   \newcommand{\tuple}[1]{\plat{			 	\stackrel{\mbox{\tiny }}
	{\raisebox{-.3ex}[.3ex]{\tiny }}
	\!\!#1\!\!
	\stackrel{\mbox{\tiny }}
	{\raisebox{-.3ex}[.3ex]{\tiny }} }}
\newcommand{\Id}[1]{[\hspace{-1.4pt}[#1]\hspace{-1.2pt}]} \newcommand{\goto}[1]{\stackrel{#1}{\longrightarrow}}   \newcommand{\gonotto}[1]{\hspace{4pt}\not\hspace{-4pt}  \stackrel{#1\ }{\longrightarrow}}
\newcommand{\trans}{\mbox{}}                    \newcommand{\bis}[1]{ \;                                \raisebox{.3ex}{\leftrightarrow}
                  \,_{#1}\,}
\newcommand{\nobis}[1]{\mbox{\underline{\makebox[.7em]{}}}}
\newcommand{\pre}[1]{\mbox{}}	\newcommand{\Co}{{\it Co}}			\newcommand{\Con}{{\it Con}}			\newcommand{\ConGP}{{\it Con}}    		\newcommand{\fCon}{{\it fCon}}			\newcommand{\Cn}[1]{{\it Cn}(#1)}		\newcommand{\fCn}[1]{{\it Cn}_{\it fin}(#1)}	\newcommand{\bCn}[1]{{\it Cn}_2(#1)}		\newcommand{\defeq}{:=}	                        \newcommand{\bsmallcupf}{\overline{\cup}_f}     \newcommand{\bsmallcap}{\overline{\cap}}        \newcommand{\bbsmallcupf}{\overline{\cup}_f^2} 
\newcommand{\bbsmallcap}{\overline{\cap}^2}  
\newcommand{\pf}{{\bf Proof:\ }}                

\newcount\PLv\newcount\PLw\newcount\PLx\newcount\PLy\newdimen\PLyy\newdimen\PLX
\newbox\PLdot \setbox\PLdot\hbox{\tiny.} \def\scl{.08} \def\PLot#1{\PLx`#1\advance\PLx-42\PLy\PLx\PLv\PLx\divide\PLy9\PLw\PLy\multiply
\PLw9\advance\PLx-\PLw\advance\PLx-4\PLy-\PLy\advance\PLy4\PLX=\the\PLx pt
\advance\PLyy\the\PLy pt\wd\PLdot=\scl\PLX\raise\scl\PLyy\copy\PLdot}
\def\draw#1{\ifx#1\end\let\next=\relax\else\PLot#1\let\next=\draw\fi\next}
\def\pa{\mbox{a\hspace{-2pt}\raisebox{-.7pt}{\draw Wabcdefgh_VVM\end}\hspace{1pt}}}

\newcommand{\IN}{\dl{N}}                        \newcommand{\IQ}{\dc{Q}}                        \newcommand{\IC}{\dc{C}}                        \newcommand{\IE}{\dl{E}}                        \newcommand{\IG}{\dc{G}}                        \newcommand{\fC}{{\cal C}}                      \newcommand{\fE}{{\cal E}}                      \newcommand{\fG}{{\cal G}}                      \newcommand{\fN}{{\cal N}}                      \newcommand{\fF}{{\cal F}}                      \newcommand{\fL}{{\cal L}}                      \newcommand{\fM}{{\cal M}}                      \newcommand{\fS}{{\cal S}}                      \newcommand{\fR}{{\cal R}}                      \newcommand{\eC}{{\rm C}}                       \newcommand{\eD}{{\rm D}}                       \newcommand{\eE}{{\rm E}}                       \newcommand{\eF}{{\rm F}}                       \newcommand{\eG}{{\rm G}}                       \newcommand{\eH}{{\rm H}}                       \newcommand{\eK}{{\rm K}}                       \newcommand{\eL}{{\rm L}}                       \newcommand{\eN}{{\rm N}}                       \newcommand{\eP}{{\rm P}}                       \newcommand{\eM}{{\rm M}}                       \newcommand{\eT}{{\rm T}}                       \newcommand{\fT}{{\cal T}}                      

\hyphenation{arc-weights}

\begin{document}
\bibliographystyle{plain}
 
\title{\Large\bf Configuration Structures, Event Structures and Petri Nets\thanks{This work was supported by ONR grant N00014-92-J-1974 (when
 both authors were at Stanford), 
 by EPSRC grant GR/S22097 (when both authors were at the
 University of Edinburgh) and by a Royal Society-Wolfson Award.}}
\author{R.J. van Glabbeek\\
 \normalsize NICTA, Sydney, Australia\-3pt]
 \normalsize Stanford University, USA\-3pt]
 \normalsize School of Informatics, University of Edinburgh, UK\-3pt]
 \normalsize \tt gdp@inf.ed.ac.uk}
\date{}
\maketitle
{\small\noindent In this paper the correspondence between safe Petri
nets and event structures, due to Nielsen, Plotkin and Winskel, is
extended to arbitrary nets without self-loops, under the collective
token interpretation.  To this end we propose a more general form of event
structure, matching the expressive power of such nets.  These new event
structures and nets are connected by relating both notions with
\phrase{configuration structures}, which can be regarded as
representations of either event structures or nets that capture their
behaviour in terms of action occurrences and the causal relationships
between them, but abstract from any auxiliary structure.

A configuration structure can also be considered logically, as a class
of propositional models, or---equivalently---as a propositional theory
in disjunctive normal from.  Converting this theory to conjunctive
normal form is the key idea in the translation of such a structure
into a net.

For a variety of classes of event structures we characterise the
associated classes of configuration structures in terms of their
closure properties, as well as in terms of the axiomatisability of the
associated propositional theories by formulae of simple prescribed
forms, and in terms of structural properties of the associated Petri
nets.}
\section*{Introduction}\label{introduction}

The aim of this paper is to connect several models of concurrency, by
providing behaviour preserving translations between them.

\begin{figure}[h]\vspace{-2ex}
\caption{Behaviour preserving translations in \cite{NPW81}}
\expandafter\ifx\csname graph\endcsname\relax \csname newbox\endcsname\graph\fi
\expandafter\ifx\csname graphtemp\endcsname\relax \csname newdimen\endcsname\graphtemp\fi
\setbox\graph=\vtop{\vskip 0pt\hbox{\special{pn 8}\special{pa 125 1100}\special{pa 875 1100}\special{pa 875 600}\special{pa 125 600}\special{pa 125 1100}\special{fp}\graphtemp=\baselineskip\multiply\graphtemp by -1\divide\graphtemp by 2
    \advance\graphtemp by .5ex\advance\graphtemp by 0.850in
    \rlap{\kern 0.500in\lower\graphtemp\hbox to 0pt{\hss Occurrence\hss}}\graphtemp=\baselineskip\multiply\graphtemp by 1\divide\graphtemp by 2
    \advance\graphtemp by .5ex\advance\graphtemp by 0.850in
    \rlap{\kern 0.500in\lower\graphtemp\hbox to 0pt{\hss nets\hss}}\special{pa 1125 1100}\special{pa 1925 1100}\special{pa 1925 600}\special{pa 1125 600}\special{pa 1125 1100}\special{fp}\graphtemp=\baselineskip\multiply\graphtemp by -2\divide\graphtemp by 2
    \advance\graphtemp by .5ex\advance\graphtemp by 0.850in
    \rlap{\kern 1.525in\lower\graphtemp\hbox to 0pt{\hss Prime event\hss}}\graphtemp=.5ex\advance\graphtemp by 0.850in
    \rlap{\kern 1.525in\lower\graphtemp\hbox to 0pt{\hss structures\hss}}\graphtemp=\baselineskip\multiply\graphtemp by 2\divide\graphtemp by 2
    \advance\graphtemp by .5ex\advance\graphtemp by 0.850in
    \rlap{\kern 1.525in\lower\graphtemp\hbox to 0pt{\hss {\footnotesize w.\ bin.\ conflict}\hss}}\special{pa 2125 1100}\special{pa 3225 1100}\special{pa 3225 600}\special{pa 2125 600}\special{pa 2125 1100}\special{fp}\graphtemp=\baselineskip\multiply\graphtemp by -1\divide\graphtemp by 2
    \advance\graphtemp by .5ex\advance\graphtemp by 0.850in
    \rlap{\kern 2.675in\lower\graphtemp\hbox to 0pt{\hss Prime algebraic\hss}}\graphtemp=\baselineskip\multiply\graphtemp by 1\divide\graphtemp by 2
    \advance\graphtemp by .5ex\advance\graphtemp by 0.850in
    \rlap{\kern 2.675in\lower\graphtemp\hbox to 0pt{\hss coherent domains\hss}}\special{pa 1575 400}\special{pa 2475 400}\special{pa 2475 0}\special{pa 1575 0}\special{pa 1575 400}\special{fp}\graphtemp=\baselineskip\multiply\graphtemp by -1\divide\graphtemp by 2
    \advance\graphtemp by .5ex\advance\graphtemp by 0.200in
    \rlap{\kern 2.025in\lower\graphtemp\hbox to 0pt{\hss Families of\hss}}\graphtemp=\baselineskip\multiply\graphtemp by 1\divide\graphtemp by 2
    \advance\graphtemp by .5ex\advance\graphtemp by 0.200in
    \rlap{\kern 2.025in\lower\graphtemp\hbox to 0pt{\hss configurations\hss}}\special{pa 875 700}\special{pa 1125 700}\special{fp}\special{sh 1.000}\special{pa 1055 675}\special{pa 1125 700}\special{pa 1055 725}\special{pa 1055 675}\special{fp}\special{pa 1125 1000}\special{pa 875 1000}\special{fp}\special{sh 1.000}\special{pa 945 1025}\special{pa 875 1000}\special{pa 945 975}\special{pa 945 1025}\special{fp}\special{pa 1825 600}\special{pa 1825 400}\special{fp}\special{sh 1.000}\special{pa 1800 470}\special{pa 1825 400}\special{pa 1850 470}\special{pa 1800 470}\special{fp}\special{pa 2225 400}\special{pa 2225 600}\special{fp}\special{sh 1.000}\special{pa 2250 530}\special{pa 2225 600}\special{pa 2200 530}\special{pa 2250 530}\special{fp}\special{pa 2125 1000}\special{pa 1925 1000}\special{fp}\special{sh 1.000}\special{pa 1995 1025}\special{pa 1925 1000}\special{pa 1995 975}\special{pa 1995 1025}\special{fp}\special{ar 500 850 500 500 0 6.28319}\special{pa 500 350}\special{pa 500 600}\special{fp}\special{sh 1.000}\special{pa 525 530}\special{pa 500 600}\special{pa 475 530}\special{pa 525 530}\special{fp}\graphtemp=.5ex\advance\graphtemp by 0.475in
    \rlap{\kern 0.500in\lower\graphtemp\hbox to 0pt{\hss {\footnotesize unfol ding\ }\hss}}\graphtemp=.5ex\advance\graphtemp by 0.250in
    \rlap{\kern 0.500in\lower\graphtemp\hbox to 0pt{\hss Safe Petri Nets\hss}}\hbox{\vrule depth1.350in width0pt height 0pt}\kern 3.225in
  }}

 \centerline{\raise 1ex\box\graph}\vspace{-2ex}
\end{figure}

In {\sc Nielsen, Plotkin \& Winskel} \cite{NPW81} \phrase{event
structures} were introduced as a stepping stone between \phrase{Petri
nets} and \phrase{Scott domains}. It was established that every
\phrase{safe} Petri net can be unfolded into an \phrase{occurrence
net}; the occurrence nets are then in correspondence with event
structures; and they in turn are in correspondence with
\phrase{prime algebraic coherent Scott domains}.  In {\sc Winskel}
\cite{Wi87a} a more general notion of event structure was proposed,
corresponding to a more general kind of Scott domain.  The event
structures from \cite{NPW81} are now called \phrase{prime event
structures with binary conflict}.

The translation from event structures to domains passes through a
stage of \phrase{families of configurations of event structures}.
{\sc Winskel} \cite{Wi82} and {\sc Van Glabbeek {\small\&} Goltz} \cite{GG90}
found it convenient to use such families as a model of concurrency in
its own right. In this context the families were called
\phrase{configuration structures} \cite{GG90}.\vspace{-2ex}

\begin{figure}[htb]
\caption{Our main contribution: behaviour preserving translations
between four models of concurrency}
\expandafter\ifx\csname graph\endcsname\relax \csname newbox\endcsname\graph\fi
\expandafter\ifx\csname graphtemp\endcsname\relax \csname newdimen\endcsname\graphtemp\fi
\setbox\graph=\vtop{\vskip 0pt\hbox{\special{pn 8}\special{ar 600 1500 400 400 0 6.28319}\graphtemp=\baselineskip\multiply\graphtemp by -2\divide\graphtemp by 2
    \advance\graphtemp by .5ex\advance\graphtemp by 1.500in
    \rlap{\kern 0.600in\lower\graphtemp\hbox to 0pt{\hss  \hss}}\graphtemp=.5ex\advance\graphtemp by 1.500in
    \rlap{\kern 0.600in\lower\graphtemp\hbox to 0pt{\hss 1-occurrence\hss}}\graphtemp=\baselineskip\multiply\graphtemp by 2\divide\graphtemp by 2
    \advance\graphtemp by .5ex\advance\graphtemp by 1.500in
    \rlap{\kern 0.600in\lower\graphtemp\hbox to 0pt{\hss nets\hss}}\special{ar 1600 500 500 500 0 6.28319}\graphtemp=\baselineskip\multiply\graphtemp by -2\divide\graphtemp by 2
    \advance\graphtemp by .5ex\advance\graphtemp by 0.500in
    \rlap{\kern 1.600in\lower\graphtemp\hbox to 0pt{\hss  \hss}}\graphtemp=.5ex\advance\graphtemp by 0.500in
    \rlap{\kern 1.600in\lower\graphtemp\hbox to 0pt{\hss Configuration\hss}}\graphtemp=\baselineskip\multiply\graphtemp by 2\divide\graphtemp by 2
    \advance\graphtemp by .5ex\advance\graphtemp by 0.500in
    \rlap{\kern 1.600in\lower\graphtemp\hbox to 0pt{\hss structures\hss}}\special{ar 2600 1500 500 500 0 6.28319}\graphtemp=\baselineskip\multiply\graphtemp by -2\divide\graphtemp by 2
    \advance\graphtemp by .5ex\advance\graphtemp by 1.500in
    \rlap{\kern 2.600in\lower\graphtemp\hbox to 0pt{\hss  \hss}}\graphtemp=.5ex\advance\graphtemp by 1.500in
    \rlap{\kern 2.600in\lower\graphtemp\hbox to 0pt{\hss Pure event\hss}}\graphtemp=\baselineskip\multiply\graphtemp by 2\divide\graphtemp by 2
    \advance\graphtemp by .5ex\advance\graphtemp by 1.500in
    \rlap{\kern 2.600in\lower\graphtemp\hbox to 0pt{\hss structures\hss}}\special{ar 1600 2500 500 500 0 6.28319}\graphtemp=\baselineskip\multiply\graphtemp by -1\divide\graphtemp by 2
    \advance\graphtemp by .5ex\advance\graphtemp by 2.500in
    \rlap{\kern 1.600in\lower\graphtemp\hbox to 0pt{\hss Propositional\hss}}\graphtemp=\baselineskip\multiply\graphtemp by 1\divide\graphtemp by 2
    \advance\graphtemp by .5ex\advance\graphtemp by 2.500in
    \rlap{\kern 1.600in\lower\graphtemp\hbox to 0pt{\hss theories\hss}}\special{ar 2800 500 400 400 -0.062500 0.062500}\special{ar 2800 500 400 400 -0.286899 -0.161899}\special{ar 2800 500 400 400 -0.511299 -0.386299}\special{ar 2800 500 400 400 -0.735698 -0.610698}\special{ar 2800 500 400 400 -0.960098 -0.835098}\special{ar 2800 500 400 400 -1.184497 -1.059497}\special{ar 2800 500 400 400 -1.408897 -1.283897}\special{ar 2800 500 400 400 -1.633296 -1.508296}\special{ar 2800 500 400 400 -1.857696 -1.732696}\special{ar 2800 500 400 400 -2.082095 -1.957095}\special{ar 2800 500 400 400 -2.306495 -2.181495}\special{ar 2800 500 400 400 -2.530894 -2.405894}\special{ar 2800 500 400 400 -2.755294 -2.630294}\special{ar 2800 500 400 400 -2.979693 -2.854693}\special{ar 2800 500 400 400 -3.204093 -3.079093}\special{ar 2800 500 400 400 -3.428492 -3.303492}\special{ar 2800 500 400 400 -3.652892 -3.527892}\special{ar 2800 500 400 400 -3.877291 -3.752291}\special{ar 2800 500 400 400 -4.101691 -3.976691}\special{ar 2800 500 400 400 -4.326090 -4.201090}\special{ar 2800 500 400 400 -4.550490 -4.425490}\special{ar 2800 500 400 400 -4.774889 -4.649889}\special{ar 2800 500 400 400 -4.999288 -4.874288}\special{ar 2800 500 400 400 -5.223688 -5.098688}\special{ar 2800 500 400 400 -5.448087 -5.323087}\special{ar 2800 500 400 400 -5.672487 -5.547487}\special{ar 2800 500 400 400 -5.896886 -5.771886}\special{ar 2800 500 400 400 -6.121286 -5.996286}\graphtemp=\baselineskip\multiply\graphtemp by -2\divide\graphtemp by 2
    \advance\graphtemp by .5ex\advance\graphtemp by 0.500in
    \rlap{\kern 2.800in\lower\graphtemp\hbox to 0pt{\hss  \hss}}\graphtemp=.5ex\advance\graphtemp by 0.500in
    \rlap{\kern 2.800in\lower\graphtemp\hbox to 0pt{\hss Transition\hss}}\graphtemp=\baselineskip\multiply\graphtemp by 2\divide\graphtemp by 2
    \advance\graphtemp by .5ex\advance\graphtemp by 0.500in
    \rlap{\kern 2.800in\lower\graphtemp\hbox to 0pt{\hss systems\hss}}\special{pa 350 1375}\special{pa 850 1375}\special{pa 850 1225}\special{pa 350 1225}\special{pa 350 1375}\special{fp}\graphtemp=.5ex\advance\graphtemp by 1.300in
    \rlap{\kern 0.600in\lower\graphtemp\hbox to 0pt{\hss {\footnotesize occ.\ nets}\hss}}\special{pa 1350 375}\special{pa 1850 375}\special{pa 1850 225}\special{pa 1350 225}\special{pa 1350 375}\special{fp}\graphtemp=.5ex\advance\graphtemp by 0.300in
    \rlap{\kern 1.600in\lower\graphtemp\hbox to 0pt{\hss {\footnotesize families}\hss}}\special{pa 2550 375}\special{pa 3050 375}\special{pa 3050 225}\special{pa 2550 225}\special{pa 2550 375}\special{fp}\graphtemp=.5ex\advance\graphtemp by 0.300in
    \rlap{\kern 2.800in\lower\graphtemp\hbox to 0pt{\hss {\footnotesize domains}\hss}}\special{pa 2400 1375}\special{pa 2800 1375}\special{pa 2800 1225}\special{pa 2400 1225}\special{pa 2400 1375}\special{fp}\graphtemp=.5ex\advance\graphtemp by 1.300in
    \rlap{\kern 2.600in\lower\graphtemp\hbox to 0pt{\hss {\footnotesize prime}\hss}}\special{pa 883 1217}\special{pa 1246 854}\special{fp}\special{sh 1.000}\special{pa 1179 885}\special{pa 1246 854}\special{pa 1215 921}\special{pa 1179 885}\special{fp}\special{pa 1246 2146}\special{pa 883 1783}\special{fp}\special{sh 1.000}\special{pa 1215 2079}\special{pa 1246 2146}\special{pa 1179 2115}\special{pa 1215 2079}\special{fp}\special{sh 1.000}\special{pa 915 1850}\special{pa 883 1783}\special{pa 950 1815}\special{pa 915 1850}\special{fp}\special{pa 1954 854}\special{pa 2246 1146}\special{fp}\special{sh 1.000}\special{pa 1985 921}\special{pa 1954 854}\special{pa 2021 885}\special{pa 1985 921}\special{fp}\special{sh 1.000}\special{pa 2215 1079}\special{pa 2246 1146}\special{pa 2179 1115}\special{pa 2215 1079}\special{fp}\special{pa 1954 2146}\special{pa 2246 1854}\special{fp}\special{sh 1.000}\special{pa 2021 2115}\special{pa 1954 2146}\special{pa 1985 2079}\special{pa 2021 2115}\special{fp}\special{sh 1.000}\special{pa 2179 1885}\special{pa 2246 1854}\special{pa 2215 1921}\special{pa 2179 1885}\special{fp}\special{pa 1600 1000}\special{pa 1600 2000}\special{fp}\special{sh 1.000}\special{pa 1575 1070}\special{pa 1600 1000}\special{pa 1625 1070}\special{pa 1575 1070}\special{fp}\special{sh 1.000}\special{pa 1625 1930}\special{pa 1600 2000}\special{pa 1575 1930}\special{pa 1625 1930}\special{fp}\graphtemp=\baselineskip\multiply\graphtemp by 3\divide\graphtemp by 2
    \advance\graphtemp by .5ex\advance\graphtemp by 1.500in
    \rlap{\kern 1.600in\lower\graphtemp\hbox to 0pt{\hss ~~~~~{\footnotesize\cite{Pr94a}}\hss}}\special{pa 1000 1500}\special{pa 2100 1500}\special{fp}\special{sh 1.000}\special{pa 2030 1475}\special{pa 2100 1500}\special{pa 2030 1525}\special{pa 2030 1475}\special{fp}\special{pa 850 1300}\special{pa 2400 1300}\special{fp}\special{sh 1.000}\special{pa 920 1325}\special{pa 850 1300}\special{pa 920 1275}\special{pa 920 1325}\special{fp}\special{sh 1.000}\special{pa 2330 1275}\special{pa 2400 1300}\special{pa 2330 1325}\special{pa 2330 1275}\special{fp}\graphtemp=\baselineskip\multiply\graphtemp by -1\divide\graphtemp by 2
    \advance\graphtemp by .5ex\advance\graphtemp by 1.300in
    \rlap{\kern 1.625in\lower\graphtemp\hbox to 0pt{\hss ~~~~~~~~~~~{\footnotesize\cite{NPW81}}\hss}}\special{ar 600 1500 600 600 0 6.28319}\special{pa 0 1500}\special{pa 200 1500}\special{fp}\special{sh 1.000}\special{pa 130 1475}\special{pa 200 1500}\special{pa 130 1525}\special{pa 130 1475}\special{fp}\graphtemp=\baselineskip\multiply\graphtemp by -1\divide\graphtemp by 2
    \advance\graphtemp by .5ex\advance\graphtemp by 1.500in
    \rlap{\kern 0.100in\lower\graphtemp\hbox to 0pt{\hss \footnotesize 1-\hss}}\graphtemp=\baselineskip\multiply\graphtemp by 1\divide\graphtemp by 2
    \advance\graphtemp by .5ex\advance\graphtemp by 1.500in
    \rlap{\kern 0.100in\lower\graphtemp\hbox to 0pt{\hss {\footnotesize unf}\hss}}\graphtemp=.5ex\advance\graphtemp by 0.800in
    \rlap{\kern 0.600in\lower\graphtemp\hbox to 0pt{\hss Pure Petri Nets\hss}}\special{pa 2100 500}\special{pa 2400 500}\special{da 0.050}\special{sh 1.000}\special{pa 2170 525}\special{pa 2100 500}\special{pa 2170 475}\special{pa 2170 525}\special{fp}\special{sh 1.000}\special{pa 2330 475}\special{pa 2400 500}\special{pa 2330 525}\special{pa 2330 475}\special{fp}\graphtemp=\baselineskip\multiply\graphtemp by -1\divide\graphtemp by 2
    \advance\graphtemp by .5ex\advance\graphtemp by 0.500in
    \rlap{\kern 2.250in\lower\graphtemp\hbox to 0pt{\hss \raisebox{1pt}{\footnotesize\cite{vG95c}}\hss}}\graphtemp=\baselineskip\multiply\graphtemp by 1\divide\graphtemp by 2
    \advance\graphtemp by .5ex\advance\graphtemp by 0.500in
    \rlap{\kern 2.250in\lower\graphtemp\hbox to 0pt{\hss  \hss}}\hbox{\vrule depth3.000in width0pt height 0pt}\kern 3.200in
  }}


 \centerline{\box\graph}\vspace{-3.6em}
\end{figure}
\vfill

\noindent
The present paper \hfill generalises the cor-\\ respondence between
\hfill safe \,Petri \,nets \,and\\ configuration structures to
(possibly unsafe) nets without self-loops (the \phrase{pure} nets).
\pagebreak
For this purpose we use a more general kind of
configuration structure than in \cite{GG90}, the \phrase{set
systems}. These have an attractive alternative presentation as
\phrase{propositional theories} \cite{Pr94a}, which is exploited in
their translation to nets. We also generalise the event structures of
\cite{Wi87a}, so that, again, our configuration structures arise as
their families of configurations.  The connection between
configuration structures and Scott domains is generalised in {\sc Van
Glabbeek} \cite{vG95c}, who proposes \phrase{transition systems} as
alternative presentations of domains; we do not consider these matters
further in the present paper.

The relationship between configuration structures, infinitary
propositional theories, event structures and Petri nets is described
in Section~\ref{four models}. We \emph{1-unfold} pure nets
into \emph{pure 1-occurrence nets}, which generalise the occurrence
nets of \cite{NPW81}, and argue that this 1-unfolding preserves the
causal and branching time behaviour of the represented system. This
allows us to restrict attention to pure 1-occurrence nets in the rest
of the paper.  Moreover, we give translations showing that
configuration structures, propositional theories and event structures
are equivalent up to so-called \phrase{configuration equivalence}
(which is defined as being mapped to the same configuration
structures) and that, with a slight restriction, all four models are
equivalent up to \emph{finitary equivalence}.

Section~\ref{computational} introduces a computational interpretation
of configuration structures, Petri nets and event structures in terms
of associated transition relations;
restricted to pure Petri nets and pure event structures, these
transition relations can be derived from the relevant sets of
configurations, but not in general.  With that,
Section~\ref{equivalence} provides definitions of notions
of \emph{reachable} and \index{secured configurations}\emph{secured}
(reachable in the limit) configurations and considers corresponding
notions of equivalence by restricting to reachable or secured, and
possibly finite, configurations.

With the general framework thus provided, Section~\ref{brands} considers
the various brands of event structures introduced by Winskel and his
co-workers. They are shown to correspond to natural restrictions on
the general notion of event structure, adapting the comparisons, on the
one hand, to the original notion of configuration and, on the other
hand, to the relevant one from the general theory. These comparisons
are summarised in Table~\ref{7 classes}.

It is then natural to enquire how the event structure restrictions are
reflected in corresponding restrictions on configurations structures
and so on; this is the subject of Section~\ref{ComparingModels}.
Sections~\ref{EvsC} and~\ref{theories} provide such comparisons,
summarised in Table~\ref{correspondence}, for configuration structures
and propositional theories up to configuration equivalence.  The
restrictions on configuration structures are natural closure
properties, and those on propositional theories concern the form of
the formulae occurring in an axiomatisation. Section~\ref{EvsC-secured}
does the same, see Table~\ref{correspondence secured}, but now with
the comparison based on the secured configurations.

Section~\ref{finitary comparisons} concerns the finitary case, with
general comparisons being summarised in Table~\ref{correspondence_finite}
and the restriction to the finite reachable configurations summarised in
Table~\ref{correspondence_finite_reachable}. Section~\ref{Tie-nets}
ties in Petri nets, providing corresponding structurally defined
subclasses; however we were not successful in doing this in all cases. The
main mathematical work is done in Sections~\ref{EvsC} and~\ref{theories},
with the rest of the section adapting this work to the various cases at hand.

Section~\ref{related work} contains a discussion of related work and
presents some possibilities for future research. Finally, there is an
index for the many technical terms introduced in the course of the paper.

The papers \cite{GP95} and \cite{GP04} contain extended abstracts of
parts of this work, together with additional material.

\section{Four models of concurrency}\label{four models}

In this section we present the four models of concurrency mentioned in
the introduction, and provide translations between them.

\subsection{Configuration structures}

\begin{definition}{set system}\index{set systems}
A \emph{set system} is a pair  with  a
set and  a collection of subsets.
\end{definition}
When a set system is used to represent a concurrent system, we call it
a \phrase{pure configuration structure}\index{configuration
structures}, but generally drop the word ``pure''. (We envision
introducing a broader class of configuration structures in the future,
matching the expressive power of impure nets.) The elements of  are
then called \phrase{events} and the elements of 
\phrase{configurations}. An event represents an occurrence of an
action the system may perform; a configuration  represents a state
of the system, namely the state in which the events in  have
occurred.

\subsection{Propositional theories}

A set system can also be considered from a logical point of view: 
is thought of as a collection of \phrase{propositions} and  as the
collection of \phrase{models}. Connecting with the computational point
of view, we associate with an event the proposition that it has
happened.  This point of view is due to {\sc Pratt}~\cite{GP93a,Pr94a}.
We can now represent a set system by the valid sentences, those holding
in all models; these are the \phrase{laws of }.

To make this precise, we choose a language: infinitary propositional
logic.  Given a set  of \emph{(propositional) variables}, the
\phrase{formulae over } form the least class including  and
closed under  (negation) and  (conjunction of sets of
formulae). We make free use of other standard connectives such as
: they are all definable from 
and .  As usual, an \phrase{interpretation} of  is just
a subset of  and one defines in the standard way when an
interpretation makes a formula true.

\begin{definition}{theory}\index{propositional theories}
An \emph{(infinitary)} \emph{propositional theory} is a pair  with  a set of propositional variables and  a class
of infinitary propositional formulae over .
\end{definition}
A formula  over  is \phrase{valid} in a set system  iff it is true in all elements of ; the \emph{theory
associated to}\index{propositional theory associated to} C
is , where 
denotes the class of formulae valid in .  Equally, given a
propositional theory , its \phrase{associated set
system} is , where  is the set of
\phrase{models} of , those interpretations of  making every formula
in  true.  We say that T \phrase{axiomatises} .
A formula  over  is a \phrase{logical consequence} of a
theory  if  is true in any model of ; a formula
 over  \phrase{implies}  iff the latter is a logical
consequence of the theory .  Two propositional
theories T and  are \phrase{logically equivalent} if , which is easily seen to be the case iff they have the same
logical consequences.

\begin{theorem}{CtoTtoC}
Let  be a set system. Then   axiomatises
, i.e., .
\end{theorem}

\begin{proof}
The single formula  already constitutes an axiomatisation of . It is
called the \phrase{disjunctive normal form} of .
\end{proof}
Thus  and  provide a bijective correspondence between set
systems and infinitary propositional theories up to logical
equivalence.  For any two subsets , of , let the \phrase{clause}
 abbreviate the implication ; we say that the elements of  are the \phrase{antecedents}
of the clause, and those of  its \phrase{consequents}.  Then for any
set system , the set of clauses  constitutes another axiomatisation of .  A
theory consisting of a set of clauses is said to be in
\phrase{conjunctive normal form}.

\subsection{Event structures}\label{event structures}

\begin{definition}{event structure}\index{event structures}
An \emph{event structure} is a pair  with
\begin{itemise}
\item  a set of \phrase{events},
\item , the
      \phrase{enabling relation}.
\end{itemise}
\end{definition}
Like a configuration structure, an event structure describes a
concurrent system in which the events represent action occurrences.
In previous notions of event structure \cite{Wi87a,Wi89}, one only had
singleton enablings: . Here we
generalise  to a relation between sets of events. As before,
the enabling relation places some restrictions on which events can
happen when. The idea here is that when  is the set of events that
happened so far, an additional set  of events can happen
(concurrently) iff every subset of  is enabled by a set of
events that happened before, i.e., a subset of .

\begin{example}{ternary conflict}
Let  and the enabling relation be given by
 for any  with .
In the initial state of the event structure ,
each of the events ,  and  can happen, and any two of them
can happen concurrently. However, there is no way all three events can
ever happen, because there is no set of events  with . This is a case of \phrase{ternary conflict}.
\end{example}

\begin{example}{resolved conflict}\index{resolvable conflict}
Let  and the enabling relation be given by  and  for any  with . Initially, each of the events ,  and  can occur,
and the events  and  can even happen concurrently.  The events
 and , on the other hand, can initially not happen concurrently,
for we do not have . However, as soon as 
occurs, the events  and  can occur in parallel. We say that the
\emph{conflict} between  and  is \emph{resolved} by the
occurrence of .
\end{example}

\begin{example}{asymmetric conflict}
Let  and the enabling relation be given by 
and  for any  with .
Initially,  and  can both occur, but not in parallel.
After  has happened,  may follow, but when  happens first,
 cannot follow. The reason is that we do not have  for some . This is a case of
\phrase{asymmetric conflict} \cite{Lk92,PP95}.
\end{example}
In Section \ref{brands} we will explain how these event structures
generalise the ones of \cite{NPW81,Wi87a,Wi89}.  In those papers the
behaviour of an event structure is formalised by associating to it a
family of configurations. However, there are several ways to do so
(cf.\ Section~\ref{equivalence}); here we only consider the
simplest variant.

\begin{definition}{EtoC}
Let  be an event structure.  The set 
of \phrase{left-closed configurations} of E is given by\vspace{-2ex}

The \phrase{left-closed configuration structure associated to} E is
.
Two event structures E and F are \phrase{-equivalent}
if .
\end{definition}
In Section~\ref{computational} we provide a computational
interpretation of event structures with the property that the
left-closed configurations of an event structure adequately represent
the behaviour of the represented system for the following class of
``pure'' event structures:

\begin{definition}{pure es}
An event structure is \phrase{pure} if  only if .
\end{definition}
The event structures of Examples~\ref{ex-ternary conflict}
and~\ref{ex-resolved conflict} are pure, but the one of
\ex{asymmetric conflict} is not.

We now show that any configuration structure can be obtained as the
left-closed configuration structure associated to a pure event
structure.

\begin{definition}{CtoE}
Let  be a configuration structure.  The \phrase{event
structure associated to} C is , with
 iff .
\end{definition}

\begin{theorem}{CtoEtoC}
Let C be a configuration structure. Then  is pure and
.
\end{theorem}

\begin{proof}
Let  and . Suppose . For any  take . Then 
and . So .  Conversely, suppose . Then there is a  such that . (In fact, .) By construction, .
\end{proof}
Hence,  and  provide a bijective correspondence between
configuration structures and (pure) event structures up to
-equivalence.

\subsubsection*{Event structures vs.\ propositional theories}

By combining Theorems~\ref{th-CtoTtoC} and~\ref{th-CtoEtoC} we find
that  and  constitute a bijective
correspondence between (pure) event structures up to -equivalence
and propositional theories up to logical equivalence. Below we provide
direct translations between them.


To any event structure  we associate the
propositional theory , where

This logical view of event structures corresponds exactly with their
left-closed interpretation:

\begin{proposition}{EtoT}  for any event
structure E.
\end{proposition}

\begin{proof}
Immediate from the definitions.
\end{proof}
Similarly, to any propositional theory  in conjunctive
normal form we associate the (not necessarily pure) event structure
, where


\begin{proposition}{TtoE}  for any theory T
in conjunctive normal form.
\end{proposition}

\begin{proof}
Let . To establish  we take  and show . Let  be such that
. As  is true in  we have
. It follows that .

Now let . To establish  we take .  We have to show that  is true
in . So suppose .  Then there must be a  with , hence . It follows
that , which had to be shown.
\end{proof}
Thus  and  provide a bijective correspondence between event
structures up to -equivalence and propositional theories up to
logical equivalence.

\begin{definition}{pure PT}
A propositional theory in conjunctive normal form is \phrase{pure} if it
only contains clauses  with .
\end{definition}
Clearly every propositional theory is logically equivalent to a pure one, as
impure clauses are tautologies, i.e., they hold in all interpretations.
In case  is a pure theory, we can define the
associated pure event structure  by
.
Note that  is pure and .

\subsection{Petri nets}\label{PN}

\begin{definition}{petri}\hfill\mbox{}\index{Petri nets}\\
A \emph{Petri net} is a tuple  with
\begin{itemise}\vspace{-1ex}
\item  and  two disjoint sets of \phrase{places}
and \phrase{transitions} (\emph{Stellen} and \emph{Transitionen} in German),
\vspace{-1ex}\item ,
the \phrase{flow relation},
\vspace{-1ex}\item and , the \phrase{initial marking}.
\end{itemise}\vspace{-1ex}
\end{definition}
Petri nets are pictured by drawing the places as circles and the
transitions as boxes. For  there are 
\phrase{arcs} from  to .  A net is said to be \emph{without
arcweights}\phrase{arcweights} if the range of  is .

When a Petri net represents a concurrent system, a global state of
such a system is given as a \phrase{marking}, which is a multiset over
, i.e., a function .
Such a state is depicted by placing  dots (\phrase{tokens})
in each place . The initial state is given by the marking .  In
order to describe the behaviour of a net, we describe the \phrase{step
transition relation} between markings.

\begin{definition}{multiset}
For two multisets  and  over , or more generally for functions
, write  if  for all ;
 is the function given by , and
\mbox{} the one with  for all ;
\mbox{} is given by .

A multiset  over  is \phrase{finite} if  is finite.  A multiset  with  for all
 is identified with the set .  
\end{definition}
Note that for multisets  and , the function  need not be a
multiset.

\begin{definition}{firing}
For a finite multiset  of transitions in a Petri net, let
 be the multisets of
\index{preplaces}\emph{pre-} and \phrase{postplaces} of , given by
 for .
We say that  is \phrase{enabled} under a marking  if .
In that case  can \phrase{fire} under , yielding the marking
^\bullet U, written \plat{M\goto{U}M'}.

A chain  is called
a \phrase{firing sequence}.  A marking  is
\emph{reachable}\index{reachable marking} if there is
such a sequence ending in .
\end{definition}
If a multiset  of transitions fires, for every transition  in  and
every arc from a place  to , a token moves along that arc
from  to . These tokens are consumed by the firing, but also new
tokens are created, namely one for every outgoing arc of . These end up
in the places at the end of those arcs. If  occurs several times in
, all this happens several times (in parallel) as well.
The firing of  is only possible if there are sufficiently many
tokens in the preplaces of  (the places where the incoming arcs
come from). In Section~\ref{unbounded par} we explain why we consider
the firing of finite multisets only.

\subsubsection*{From Petri nets to configuration structures}

As for event structures, the behaviour of a net can be captured by
associating to it a family of configurations. 

\begin{definition}{configuration} 
\index{configurations!of a Petri net}
A \emph{(finite) configuration} of a Petri net  is
any finite multiset  of transitions with the property that the
function  given by ^\bullet
\!X is a marking, i.e., .
Let  denote the set of\out{ finite} configurations of .
\end{definition}
Note that  is a configuration and ; note
further that if  is a configuration and \plat{M_x \goto{~U} M'} then
 is a configuration and . So if
\plat{I \goto{U_1} M_1 \goto{U_2} \cdots \goto{U_n} M_n} is a
firing sequence, then  is a configuration and
. In general, when  then  is the marking
that would result from firing all transitions in , if possible,
regardless of the order in which they fire.

Next we will determine which nets can be faithfully described in this
way by means of set systems.

\begin{definition}{1-occurrence}
A \phrase{1-occurrence net} is a net in which every configuration is a set.
\end{definition}
This implies that any transition can fire at most once,
i.e., in every firing sequence \plat{M_0 \goto{U_1} \cdots \goto{U_n} M_n}
the multisets  are sets and disjoint.
When dealing with a 1-occurrence net, typically presented as a tuple
, we call its transitions \phrase{events}.

\begin{definition}{loops}
A net  is \phrase{pure} if there is no  in 
and  in  with  and , i.e., if it is without
\phrase{self-loops}.
\end{definition}
In Section~\ref{computational} we will argue that the configurations
of a 1-occurrence net adequately represent the behaviour of the
represented system only in the case of pure nets.
Therefore we will restrict attention to pure 1-occurrence nets.

\begin{definition}{NtoC}
Let  be a pure 1-oc\-cur\-rence net. Its
\phrase{associated\out{ (finitary)} configuration structure}
 is . Two such nets N and  are
\emph{configuration equivalent}\index{configuration
equivalence}---written ---if .
\end{definition}

\subsubsection*{Individual vs.\ collective tokens}

\index{collective token interpretation}
\index{individual token interpretation}
There are two different schools of thought in interpreting the causal
behaviour of Petri nets, which can be described as the \emph{individual} and
\emph{collective token} philosophy \cite{GP95,vG05}.\footnote{The
individual token interpretation of ordinary nets should not be
confused with the concept of \phrase{Petri nets with individual
tokens} \cite{Rei85} such as \phrase{predicate/transition nets} or
\phrase{coloured Petri nets}; there the individuality is hardwired into the
syntax of nets.} The following example illustrates their difference.

\begin{figure}[htp]
\expandafter\ifx\csname graph\endcsname\relax \csname newbox\endcsname\graph\fi
\expandafter\ifx\csname graphtemp\endcsname\relax \csname newdimen\endcsname\graphtemp\fi
\setbox\graph=\vtop{\vskip 0pt\hbox{\graphtemp=.5ex\advance\graphtemp by 0.125in
    \rlap{\kern 0.000in\lower\graphtemp\hbox to 0pt{\hss A:\hss}}\special{pn 8}\special{ar 375 125 125 125 0 6.28319}\graphtemp=.5ex\advance\graphtemp by 0.125in
    \rlap{\kern 0.375in\lower\graphtemp\hbox to 0pt{\hss \hss}}\special{pa 500 125}\special{pa 750 125}\special{fp}\special{sh 1.000}\special{pa 650 100}\special{pa 750 125}\special{pa 650 150}\special{pa 650 100}\special{fp}\special{pa 750 250}\special{pa 1000 250}\special{pa 1000 0}\special{pa 750 0}\special{pa 750 250}\special{fp}\graphtemp=.5ex\advance\graphtemp by 0.125in
    \rlap{\kern 0.875in\lower\graphtemp\hbox to 0pt{\hss \hss}}\special{pa 1000 125}\special{pa 1250 125}\special{fp}\special{sh 1.000}\special{pa 1150 100}\special{pa 1250 125}\special{pa 1150 150}\special{pa 1150 100}\special{fp}\special{ar 1375 125 125 125 0 6.28319}\graphtemp=.5ex\advance\graphtemp by 0.125in
    \rlap{\kern 1.375in\lower\graphtemp\hbox to 0pt{\hss \hss}}\special{pa 1500 125}\special{pa 1750 125}\special{fp}\special{sh 1.000}\special{pa 1650 100}\special{pa 1750 125}\special{pa 1650 150}\special{pa 1650 100}\special{fp}\special{pa 1750 250}\special{pa 2000 250}\special{pa 2000 0}\special{pa 1750 0}\special{pa 1750 250}\special{fp}\graphtemp=.5ex\advance\graphtemp by 0.125in
    \rlap{\kern 1.875in\lower\graphtemp\hbox to 0pt{\hss \hss}}\special{pa 2000 125}\special{pa 2250 125}\special{fp}\special{sh 1.000}\special{pa 2100 150}\special{pa 2000 125}\special{pa 2100 100}\special{pa 2100 150}\special{fp}\special{ar 2375 125 125 125 0 6.28319}\graphtemp=.5ex\advance\graphtemp by 0.125in
    \rlap{\kern 2.375in\lower\graphtemp\hbox to 0pt{\hss \hss}}\hbox{\vrule depth0.250in width0pt height 0pt}\kern 2.500in
  }} \centerline{\raise 1em\box\graph}
\end{figure}
\noindent
In this net, the transitions  and  can fire once each.
After  has fired, there are two tokens in the middle place.
According to the individual token philosophy, it makes a
difference which of these tokens is used in firing . If the token
that was there already is used (which must certainly be the case if
 fires before the token from  arrives), the transitions 
and  are causally independent. If the token that was produced by
 is used,  is causally dependent on . Thus, the net A above
has two maximal computations, that can be characterised by partial orders:
\begin{picture}(11,0)
\put(0,0){}
\put(3,1){\vector(1,0){5}}
\put(9,0){}
\end{picture}
and the trivial one
\plat{\begin{array}{@{}c@{}}\,\,a\!\-3pt]b\;\,\end{array}}.
However, in the collective token philosophy A does not have a run
\begin{picture}(11,0)
\put(0,0){}
\put(3,1){\vector(1,0){5}}
\put(9,0){}
\end{picture}
and can therefore not be equivalent to C in any causality preserving way.

Thus, capturing the behaviour of nets by means of our mapping  to
configuration structures is compatible with the collective token
interpretation only. In the remainder of this paper, we therefore take
the collective token approach.

\subsubsection*{Rooted structures and finitary equivalence}

The configuration structure associated to a pure 1-occurrence net is
always \phrase{finitary}, meaning that all configurations are finite, and
\phrase{rooted}, meaning that the empty set of events is a configuration.
In order to translate between the models of concurrency seen before
and Petri nets, we therefore restrict attention to rooted structures,
and ignore infinite configurations.

\begin{definition}{rooted}
A configuration structure  is \phrase{rooted} if
.  A propositional theory is \phrase{rooted} if it has
no clause of the form  as a logical consequence.
An event structure  is \phrase{rooted} if
.
\end{definition}

\begin{proposition}{rooted}
If C is rooted, then so are  and .
If T is rooted, then so are  and .
If E is rooted, then so are  and .
\end{proposition}

\begin{proof}
Straightforward.
\end{proof}

\begin{definition}{finitary equivalence}
Given a configuration structure , let  be the
configuration structure with the same events but with only the finite
configurations of .
Two configuration structures  and  are \emph{finitarily
equivalent}\index{finitary equivalence}---written ---if
. 
\end{definition}
Instead of considering configuration structures up to finitary
equivalence, we could just as well restrict attention to finitary
configuration structures, thereby taking a normal form in each
equivalence class.  However, on the level of propositional theories
this involves adding clauses  for every infinite
set of events , which would needlessly complicate the forthcoming
\pr{NtoT}.  Moreover, the fact that  is finitary for every
pure 1-occurrence net  is more a consequence of not considering
infinite configurations of Petri nets than of there not being any
(cf.\ Section~\ref{unbounded par}).

\out{
 A configuration structure is \phrase{finitary} if all its configurations
 are finite.  A propositional theory is \phrase{finitary} if, for every
 infinite , it has the clause  as
 a logical consequence.  An event structure  is \phrase{
 finitary} if  only holds for finite sets .

 \begin{proposition}{finitary}
 If C is finitary, then so are  and .
 If T is finitary, then so are  and .
 If E is finitary, then so are  and .
 \end{proposition}

 \begin{proof}
 Straightforward.
 \end{proof}
}

\subsubsection*{From configuration structures to Petri nets}

We now proceed to show that, up to finitary equivalence, every rooted
configuration structure can be obtained as the image of a pure
1-occurrence net.

\begin{definition}{TtoN}
Let  be a rooted propositional theory in
conjunctive normal form. We define the associated Petri net
 as follows.  As transitions of the net we take the events
from . For every transition we add one place,
containing one initial token, that has no incoming arcs,
and with its only outgoing arc going to that transition.  These
\phrase{1-occurrence places} make sure that every transition fires at
most once.  For every clause  in  with  finite,
we introduce a place in the net. This place has outgoing arcs to each
of the transitions in , and incoming arcs from each of the
transitions in . Let  be the cardinality of . As  is
rooted, .  We finish the construction by putting 
initial tokens in the created place:
\expandafter\ifx\csname graph\endcsname\relax \csname newbox\endcsname\graph\fi
\expandafter\ifx\csname graphtemp\endcsname\relax \csname newdimen\endcsname\graphtemp\fi
\setbox\graph=\vtop{\vskip 0pt\hbox{\special{pn 8}\special{ar 866 610 98 98 0 6.28319}\graphtemp=.5ex\advance\graphtemp by 0.583in
    \rlap{\kern 0.831in\lower\graphtemp\hbox to 0pt{\hss \hss}}\graphtemp=.5ex\advance\graphtemp by 0.583in
    \rlap{\kern 0.902in\lower\graphtemp\hbox to 0pt{\hss \hss}}\graphtemp=.5ex\advance\graphtemp by 0.650in
    \rlap{\kern 0.866in\lower\graphtemp\hbox to 0pt{\hss \hss}}\special{pa 0 157}\special{pa 157 157}\special{pa 157 0}\special{pa 0 0}\special{pa 0 157}\special{fp}\special{ar 293 654 591 591 -1.802089 -0.242404}\special{sh 1.000}\special{pa 202 94}\special{pa 157 79}\special{pa 190 45}\special{pa 202 94}\special{fp}\special{pa 0 433}\special{pa 157 433}\special{pa 157 276}\special{pa 0 276}\special{pa 0 433}\special{fp}\special{ar 295 1130 787 787 -1.746343 -0.862746}\special{sh 1.000}\special{pa 201 372}\special{pa 157 354}\special{pa 192 323}\special{pa 201 372}\special{fp}\special{pa 0 709}\special{pa 157 709}\special{pa 157 551}\special{pa 0 551}\special{pa 0 709}\special{fp}\special{ar 472 1273 787 787 -1.982313 -1.159279}\special{sh 1.000}\special{pa 204 558}\special{pa 157 551}\special{pa 184 513}\special{pa 204 558}\special{fp}\special{pa 0 984}\special{pa 157 984}\special{pa 157 827}\special{pa 0 827}\special{pa 0 984}\special{fp}\special{ar 703 1395 787 787 -2.335515 -1.488036}\special{sh 1.000}\special{pa 203 818}\special{pa 157 827}\special{pa 169 781}\special{pa 203 818}\special{fp}\special{pa 1575 157}\special{pa 1732 157}\special{pa 1732 0}\special{pa 1575 0}\special{pa 1575 157}\special{fp}\special{pa 936 541}\special{pa 1280 138}\special{pa 1575 79}\special{sp}\special{sh 1.000}\special{pa 980 527}\special{pa 936 541}\special{pa 942 494}\special{pa 980 527}\special{fp}\special{pa 1575 433}\special{pa 1732 433}\special{pa 1732 276}\special{pa 1575 276}\special{pa 1575 433}\special{fp}\special{ar 1069 -169 787 787 0.872902 1.703581}\special{sh 1.000}\special{pa 1000 640}\special{pa 965 610}\special{pa 1007 591}\special{pa 1000 640}\special{fp}\special{pa 1654 433}\special{pa 1654 827}\special{dt 0.079}\special{pa 1575 984}\special{pa 1732 984}\special{pa 1732 827}\special{pa 1575 827}\special{pa 1575 984}\special{fp}\special{ar 1492 122 787 787 1.465367 2.355163}\special{sh 1.000}\special{pa 946 725}\special{pa 936 680}\special{pa 981 690}\special{pa 946 725}\special{fp}\graphtemp=.5ex\advance\graphtemp by 0.827in
    \rlap{\kern 0.728in\lower\graphtemp\hbox to 0pt{\hss \hss}}\hbox{\vrule depth0.984in width0pt height 0pt}\kern 1.732in
  }} 
\end{definition}
The place belonging to the clause  does not place any
restrictions on the firing of the first \mbox{} transitions in
.  However, the last one can only fire after an extra token arrives
in the place.  This can happen only if one of the transitions in 
fires first.  The firing of more transitions in  has no adverse
effects, as each of the transitions in  can fire only once.  Thus
this place imposes the same restriction on the occurrence of events as
does the corresponding clause.

\begin{theorem}{PtoNtoC}
Let T be a rooted propositional theory in conjunctive normal
form. Then 
\end{theorem}

\begin{proof}
 iff  is finite and  for any
place . We have  for all 1-occurrence places 
exactly when no transition fires twice in , i.e., when  is a set.
For a place  belonging to the clause  we have
 iff either one of the transitions in  has fired, or
not all of the transitions in  have fired, i.e., when 
holds in , seen as a model of propositional logic. The clauses
 of  with  infinite surely hold in any finite
configuration .
Thus,  iff  is a finite model of .
\end{proof}
The net  is always without arcweights.
Moreover, in case  is pure (cf.\ \df{pure PT}), so is the net
. As any rooted configuration structure can be axiomatised
by a pure rooted propositional theory in conjunctive normal form,
it follows that
\begin{corollary}{CtoN}
For every rooted configuration structure there exists a pure
1-occurrence net without arcweights with the same finite configurations.
\hfill 
\end{corollary}
Thus we have established a bijective correspondence between rooted
configuration structures up to finitary equivalence and pure
1-oc\-cur\-rence nets up to configuration equivalence.
Moreover, every pure 1-occurrence net is configuration
equivalent to a pure 1-occurrence net without arcweights.

\begin{example}{ternary-PN}
The event structure with ternary conflict of \ex{ternary conflict} can
be represented by the propositional theory 
The Petri net associated to\\
this theory by \df{TtoN} is:\hfill
\expandafter\ifx\csname graph\endcsname\relax \csname newbox\endcsname\graph\fi
\expandafter\ifx\csname graphtemp\endcsname\relax \csname newdimen\endcsname\graphtemp\fi
\setbox\graph=\vtop{\vskip 0pt\hbox{\special{pn 8}\special{ar 492 98 98 98 0 6.28319}\graphtemp=.5ex\advance\graphtemp by 0.098in
    \rlap{\kern 0.492in\lower\graphtemp\hbox to 0pt{\hss \hss}}\special{pa 0 591}\special{pa 197 591}\special{pa 197 394}\special{pa 0 394}\special{pa 0 591}\special{fp}\graphtemp=.5ex\advance\graphtemp by 0.492in
    \rlap{\kern 0.098in\lower\graphtemp\hbox to 0pt{\hss \hss}}\special{pa 423 168}\special{pa 197 394}\special{fp}\special{sh 1.000}\special{pa 242 384}\special{pa 197 394}\special{pa 207 348}\special{pa 242 384}\special{fp}\special{pa 394 591}\special{pa 591 591}\special{pa 591 394}\special{pa 394 394}\special{pa 394 591}\special{fp}\graphtemp=.5ex\advance\graphtemp by 0.492in
    \rlap{\kern 0.492in\lower\graphtemp\hbox to 0pt{\hss \hss}}\special{pa 492 197}\special{pa 492 394}\special{fp}\special{sh 1.000}\special{pa 517 354}\special{pa 492 394}\special{pa 467 354}\special{pa 517 354}\special{fp}\special{pa 787 591}\special{pa 984 591}\special{pa 984 394}\special{pa 787 394}\special{pa 787 591}\special{fp}\graphtemp=.5ex\advance\graphtemp by 0.492in
    \rlap{\kern 0.886in\lower\graphtemp\hbox to 0pt{\hss \hss}}\special{pa 562 168}\special{pa 787 394}\special{fp}\special{sh 1.000}\special{pa 777 348}\special{pa 787 394}\special{pa 742 384}\special{pa 777 348}\special{fp}\special{ar 886 846 98 98 0 6.28319}\graphtemp=.5ex\advance\graphtemp by 0.846in
    \rlap{\kern 0.886in\lower\graphtemp\hbox to 0pt{\hss \hss}}\special{pa 886 748}\special{pa 886 591}\special{fp}\special{sh 1.000}\special{pa 861 630}\special{pa 886 591}\special{pa 911 630}\special{pa 861 630}\special{fp}\special{ar 492 846 98 98 0 6.28319}\graphtemp=.5ex\advance\graphtemp by 0.846in
    \rlap{\kern 0.492in\lower\graphtemp\hbox to 0pt{\hss \hss}}\special{pa 492 748}\special{pa 492 591}\special{fp}\special{sh 1.000}\special{pa 467 630}\special{pa 492 591}\special{pa 517 630}\special{pa 467 630}\special{fp}\special{ar 98 846 98 98 0 6.28319}\graphtemp=.5ex\advance\graphtemp by 0.846in
    \rlap{\kern 0.098in\lower\graphtemp\hbox to 0pt{\hss \hss}}\special{pa 98 748}\special{pa 98 591}\special{fp}\special{sh 1.000}\special{pa 73 630}\special{pa 98 591}\special{pa 123 630}\special{pa 73 630}\special{fp}\hbox{\vrule depth0.945in width0pt height 0pt}\kern 0.984in
  }} \raisebox{1.5cm}[0pt][0pt]{\box\graph}
\vspace{.8cm}
\end{example}

\begin{example}{resolvable}
Below are the event structure with resolvable conflict from \ex{resolved conflict},
its representation as a propositional theory, and the associated Petri
net, as well as its configurations, ordered by inclusion.
\\
\expandafter\ifx\csname graph\endcsname\relax \csname newbox\endcsname\graph\fi
\expandafter\ifx\csname graphtemp\endcsname\relax \csname newdimen\endcsname\graphtemp\fi
\setbox\graph=\vtop{\vskip 0pt\hbox{\special{pn 8}\special{ar 1811 98 98 98 0 6.28319}\graphtemp=.5ex\advance\graphtemp by 0.098in
    \rlap{\kern 1.811in\lower\graphtemp\hbox to 0pt{\hss \hss}}\special{pa 1811 197}\special{pa 1811 394}\special{fp}\special{sh 1.000}\special{pa 1836 354}\special{pa 1811 394}\special{pa 1786 354}\special{pa 1836 354}\special{fp}\special{pa 1713 591}\special{pa 1909 591}\special{pa 1909 394}\special{pa 1713 394}\special{pa 1713 591}\special{fp}\graphtemp=.5ex\advance\graphtemp by 0.492in
    \rlap{\kern 1.811in\lower\graphtemp\hbox to 0pt{\hss \hss}}\special{pa 1811 591}\special{pa 1811 787}\special{fp}\special{sh 1.000}\special{pa 1836 748}\special{pa 1811 787}\special{pa 1786 748}\special{pa 1836 748}\special{fp}\special{ar 1811 886 98 98 0 6.28319}\graphtemp=.5ex\advance\graphtemp by 0.886in
    \rlap{\kern 1.811in\lower\graphtemp\hbox to 0pt{\hss \hss}}\special{ar 1417 886 98 98 0 6.28319}\graphtemp=.5ex\advance\graphtemp by 0.886in
    \rlap{\kern 1.417in\lower\graphtemp\hbox to 0pt{\hss \hss}}\special{pa 1417 984}\special{pa 1417 1181}\special{fp}\special{sh 1.000}\special{pa 1442 1142}\special{pa 1417 1181}\special{pa 1392 1142}\special{pa 1442 1142}\special{fp}\special{pa 1319 1378}\special{pa 1516 1378}\special{pa 1516 1181}\special{pa 1319 1181}\special{pa 1319 1378}\special{fp}\graphtemp=.5ex\advance\graphtemp by 1.280in
    \rlap{\kern 1.417in\lower\graphtemp\hbox to 0pt{\hss \hss}}\special{pa 1741 955}\special{pa 1516 1181}\special{fp}\special{sh 1.000}\special{pa 1561 1171}\special{pa 1516 1181}\special{pa 1526 1136}\special{pa 1561 1171}\special{fp}\special{ar 2205 886 98 98 0 6.28319}\graphtemp=.5ex\advance\graphtemp by 0.886in
    \rlap{\kern 2.205in\lower\graphtemp\hbox to 0pt{\hss \hss}}\special{pa 2205 984}\special{pa 2205 1181}\special{fp}\special{sh 1.000}\special{pa 2230 1142}\special{pa 2205 1181}\special{pa 2180 1142}\special{pa 2230 1142}\special{fp}\special{pa 2106 1378}\special{pa 2303 1378}\special{pa 2303 1181}\special{pa 2106 1181}\special{pa 2106 1378}\special{fp}\graphtemp=.5ex\advance\graphtemp by 1.280in
    \rlap{\kern 2.205in\lower\graphtemp\hbox to 0pt{\hss \hss}}\special{pa 1881 955}\special{pa 2106 1181}\special{fp}\special{sh 1.000}\special{pa 2096 1136}\special{pa 2106 1181}\special{pa 2061 1171}\special{pa 2096 1136}\special{fp}\graphtemp=.5ex\advance\graphtemp by 0.295in
    \rlap{\kern 0.787in\lower\graphtemp\hbox to 0pt{\hss \hss}}\graphtemp=.5ex\advance\graphtemp by 1.083in
    \rlap{\kern 0.433in\lower\graphtemp\hbox to 0pt{\hss \hss}}\hbox{\vrule depth1.378in width0pt height 0pt}\kern 2.303in
  }} \raisebox{3.8cm}{\box\graph}
\hfill\begin{picture}(25,50)(-10,-14)
\put(0,0){\makebox[0pt]{}} \put(-2,3){\line(-1,1){5}}
\put(-10,10){\makebox[0pt]{}}  \put(2,3){\line(1,1){5}}
\put(10,10){\makebox[0pt]{}}   \put(-7.5,23.5){\line(1,1){5}}
\put(0,10){\makebox[0pt]{}}    \put(2.5,13.5){\line(1,1){5}}
\put(0,4){\line(0,1){4}}            \put(-2.5,13.5){\line(-1,1){5}}
\put(-10,20){\makebox[0pt]{}}  \put(-10,14){\line(0,1){5}}
\put(10,20){\makebox[0pt]{}}   \put(10,14){\line(0,1){5}}
\put(0,30){\makebox[0pt]{}}  \put(7.5,23.5){\line(-1,1){5}}
\end{picture}
\end{example}

\begin{example}{disjunctive}
Below is a propositional theory describing a system in which either
 or  is sufficient to enable the event ; this is sometimes
called \phrase{disjunctive causality}.  We also display the associated
Petri net, and its representation as an event structure and a
configuration structure.
\\
\expandafter\ifx\csname graph\endcsname\relax \csname newbox\endcsname\graph\fi
\expandafter\ifx\csname graphtemp\endcsname\relax \csname newdimen\endcsname\graphtemp\fi
\setbox\graph=\vtop{\vskip 0pt\hbox{\special{pn 8}\special{ar 1654 886 98 98 0 6.28319}\special{pa 1654 984}\special{pa 1654 1181}\special{fp}\special{sh 1.000}\special{pa 1679 1142}\special{pa 1654 1181}\special{pa 1629 1142}\special{pa 1679 1142}\special{fp}\special{pa 1555 1378}\special{pa 1752 1378}\special{pa 1752 1181}\special{pa 1555 1181}\special{pa 1555 1378}\special{fp}\graphtemp=.5ex\advance\graphtemp by 1.280in
    \rlap{\kern 1.654in\lower\graphtemp\hbox to 0pt{\hss \hss}}\special{ar 1260 98 98 98 0 6.28319}\graphtemp=.5ex\advance\graphtemp by 0.098in
    \rlap{\kern 1.260in\lower\graphtemp\hbox to 0pt{\hss \hss}}\special{pa 1260 197}\special{pa 1260 394}\special{fp}\special{sh 1.000}\special{pa 1285 354}\special{pa 1260 394}\special{pa 1235 354}\special{pa 1285 354}\special{fp}\special{pa 1161 591}\special{pa 1358 591}\special{pa 1358 394}\special{pa 1161 394}\special{pa 1161 591}\special{fp}\graphtemp=.5ex\advance\graphtemp by 0.492in
    \rlap{\kern 1.260in\lower\graphtemp\hbox to 0pt{\hss \hss}}\special{pa 1358 591}\special{pa 1584 816}\special{fp}\special{sh 1.000}\special{pa 1574 771}\special{pa 1584 816}\special{pa 1538 806}\special{pa 1574 771}\special{fp}\special{ar 2047 98 98 98 0 6.28319}\graphtemp=.5ex\advance\graphtemp by 0.098in
    \rlap{\kern 2.047in\lower\graphtemp\hbox to 0pt{\hss \hss}}\special{pa 2047 197}\special{pa 2047 394}\special{fp}\special{sh 1.000}\special{pa 2072 354}\special{pa 2047 394}\special{pa 2022 354}\special{pa 2072 354}\special{fp}\special{pa 1949 591}\special{pa 2146 591}\special{pa 2146 394}\special{pa 1949 394}\special{pa 1949 591}\special{fp}\graphtemp=.5ex\advance\graphtemp by 0.492in
    \rlap{\kern 2.047in\lower\graphtemp\hbox to 0pt{\hss \hss}}\special{pa 1949 591}\special{pa 1723 816}\special{fp}\special{sh 1.000}\special{pa 1769 806}\special{pa 1723 816}\special{pa 1733 771}\special{pa 1769 806}\special{fp}\special{ar 1260 886 98 98 0 6.28319}\graphtemp=.5ex\advance\graphtemp by 0.886in
    \rlap{\kern 1.260in\lower\graphtemp\hbox to 0pt{\hss \hss}}\special{pa 1329 955}\special{pa 1555 1181}\special{fp}\special{sh 1.000}\special{pa 1545 1136}\special{pa 1555 1181}\special{pa 1510 1171}\special{pa 1545 1136}\special{fp}\graphtemp=.5ex\advance\graphtemp by 1.083in
    \rlap{\kern 0.709in\lower\graphtemp\hbox to 0pt{\hss \hss}}\graphtemp=.5ex\advance\graphtemp by 0.295in
    \rlap{\kern 0.354in\lower\graphtemp\hbox to 0pt{\hss \hss}}\hbox{\vrule depth1.378in width0pt height 0pt}\kern 2.146in
  }} \raisebox{3.8cm}{\box\graph}
\unitlength .9mm
\hfill\begin{picture}(24,20)(-10,-4)
\put(0,0){\makebox[0pt]{}} \put(-2,3){\line(-1,1){5}}
\put(-10,10){\makebox[0pt]{}}  \put(2,3){\line(1,1){5}}
\put(10,10){\makebox[0pt]{}}   \put(-7.5,23.5){\line(1,1){5}}
\put(0,20){\makebox[0pt]{}}  \put(-7.5,13.5){\line(1,1){5}}
\put(0,24){\line(0,1){4}}             \put(7.5,13.5){\line(-1,1){5}}
\put(-10,20){\makebox[0pt]{}}  \put(-10,14){\line(0,1){5}}
\put(10,20){\makebox[0pt]{}}   \put(10,14){\line(0,1){5}}
\put(0,30){\makebox[0pt]{}}  \put(7.5,23.5){\line(-1,1){5}}
\end{picture}
In case we modify the event structure by omitting the enabling
, the propositional theory gains a clause
, the Petri net gains a marked place
with arrows to  and , and the configuration structure loses the
configurations  and  .
\end{example}

\subsubsection*{From nets to theories and event structures}
\label{NtoT}

We know already how to translate pure 1-occurrence nets into
propositional theories and event structures, namely through the
intermediate stage of configuration structures. Below we provide
direct translations that might shed more light on the relationships
between these models of concurrency.

Let  be a 1-occurrence net.
For any place  let  be
its set of \phrase{posttransitions} and  its set of \phrase{pretransitions}.
For any finite set  of posttransitions of ,
 is the number of tokens needed in place  for
all transitions in  to fire,\footnote{In case  is without
arcweights,  is simply  (cf.\ \cite{GP04}).}
so , if positive, is the number of tokens that
have to arrive in  before all transitions in  can fire.
Furthermore, for , let \mbox{^\bullet\!s} be the collection of
sets  of pretransitions of , such that if all transitions in 
fire, at least  tokens will arrive in .
Write  for \plat{^{^\bullet Y(s)-I(s)}\!s}. One
of the sets of transitions in  has to fire entirely before
all transitions in  can fire.

The formula 
expresses which transitions need to fire for  tokens to arrive in
. The formula  expresses that one of the sets of transitions in
 has to fire entirely before all transitions in  can
fire. The \phrase{propositional theory associated to}  is defined
as , where  consists of all
formulae  with
 and . It follows that

\begin{proposition}{NtoT}
 for any pure 1-occurrence net .
\hfill
\end{proposition}

\begin{proof}
Let  be a pure 1-occurrence net and
 be a finite set of transitions of .
Then  iff for all  and  the formula \plat{\bigwedge Y \Rightarrow
\varphi^{^\bullet Y(s)-I(s)}_s} is true in , which is
the case iff , or 
In the latter formula the clause ^\bullet s can just
as well be deleted, as transitions in  that are not in 
do not make a contribution to  anyway. Thus this formula
is equivalent to 
Likewise, requiring this implication to merely hold for sets of
transitions  with  is moot.
Hence 
\end{proof}

For any finite set of transitions , let  be the set of
places  with  and .
Now write  whenever  with
. We also write  whenever
 is infinite. The \phrase{event structure associated to}  is
defined as .
Note that if  is pure, then so is .

\begin{proposition}{NtoE}
Let  be a pure 1-occurrence net. Then
.
\hfill
\end{proposition}

\begin{proof}
Let  be a pure 1-occurrence net and
 be a finite set of transitions of .
Then\vspace{-2em}








\end{proof}
The size of  and  can be reduced by redefining
 to consist of the \emph{minimal} subsets  of
\mbox{} with . This does not affect
the truth of Propositions~\ref{pr-NtoT} and~\ref{pr-NtoE}, although it
slightly complicates their proofs. Likewise, in the definition of
 only those formulae \plat{\bigwedge Y \Rightarrow
\varphi^{^\bullet Y(s)-I(s)}_s}
are needed for which  (the remaining formulae
being tautologies). This yields the maps of \cite{GP04}.

\out{
 Let  denote the thusly
 modified map from nets to propositional theories. The translations
  and  between propositional theories in conjunctive normal
 form and Petri nets preserve even more information than finite
 models/configurations:

 \begin{proposition}{TtoNtoT}
 Let  be a rooted propositional theory WITH FINITE CONFLICT in
 conjunctive normal form. Then .
 \end{proposition}

 \begin{proof}
 The pairs  with  a place in  and  satisfying  are exactly those
 with  the place belonging to a clause  in , and
 for all such pairs we have . The formula
 generated by  is again .
 \end{proof}
}

\subsubsection*{1-Unfolding}

Below we show that the restriction to 1-occurrence nets is not very
crucial; every net can be ``unfolded'' into a 1-occurrence net
without changing its behaviour in any essential way. However, the
unfolding cannot be configuration equivalent to the original, as
the identity of transitions cannot be preserved.

\begin{definition}{1-unfolding}
Let  be a Petri net. Its \phrase{1-unfolding}
 into a 1-occurrence net is given by
(for , , )
\begin{itemise}
\item ,
\item ,
\item  and ,
\item  and
       for  with ,
\item  and .
\end{itemise}
\end{definition}
Thus, every transition is replaced by countably many copies, each of
which is connected with its environment (though the flow relation) in
exactly the same way as the original. Furthermore, for every such copy
 an extra place  is created, containing one initial token, and
having no incoming arcs and only one outgoing arc, going to .
This place guarantees that  can fire only once.


\begin{figure}[htb]
\expandafter\ifx\csname graph\endcsname\relax \csname newbox\endcsname\graph\fi
\expandafter\ifx\csname graphtemp\endcsname\relax \csname newdimen\endcsname\graphtemp\fi
\setbox\graph=\vtop{\vskip 0pt\hbox{\special{pn 8}\special{pa 500 650}\special{pa 650 650}\special{pa 650 500}\special{pa 500 500}\special{pa 500 650}\special{fp}\special{ar 75 75 75 75 0 6.28319}\special{pa 128 128}\special{pa 500 500}\special{fp}\special{sh 1.000}\special{pa 482 447}\special{pa 500 500}\special{pa 447 482}\special{pa 482 447}\special{fp}\special{ar 75 1075 75 75 0 6.28319}\special{pa 128 1022}\special{pa 500 650}\special{fp}\special{sh 1.000}\special{pa 447 668}\special{pa 500 650}\special{pa 482 703}\special{pa 447 668}\special{fp}\special{ar 1075 575 75 75 0 6.28319}\special{pa 650 575}\special{pa 1000 575}\special{fp}\special{sh 1.000}\special{pa 950 550}\special{pa 1000 575}\special{pa 950 600}\special{pa 950 550}\special{fp}\special{pa 2500 400}\special{pa 2650 400}\special{pa 2650 250}\special{pa 2500 250}\special{pa 2500 400}\special{fp}\graphtemp=.5ex\advance\graphtemp by 0.075in
    \rlap{\kern 2.575in\lower\graphtemp\hbox to 0pt{\hss \hss}}\special{ar 2575 75 75 75 0 6.28319}\special{pa 2575 150}\special{pa 2575 250}\special{fp}\special{sh 1.000}\special{pa 2600 200}\special{pa 2575 250}\special{pa 2550 200}\special{pa 2600 200}\special{fp}\special{pa 2500 900}\special{pa 2650 900}\special{pa 2650 750}\special{pa 2500 750}\special{pa 2500 900}\special{fp}\graphtemp=.5ex\advance\graphtemp by 0.575in
    \rlap{\kern 2.575in\lower\graphtemp\hbox to 0pt{\hss \hss}}\special{ar 2575 575 75 75 0 6.28319}\special{pa 2575 650}\special{pa 2575 750}\special{fp}\special{sh 1.000}\special{pa 2600 700}\special{pa 2575 750}\special{pa 2550 700}\special{pa 2600 700}\special{fp}\special{pa 2500 1400}\special{pa 2650 1400}\special{pa 2650 1250}\special{pa 2500 1250}\special{pa 2500 1400}\special{fp}\graphtemp=.5ex\advance\graphtemp by 1.075in
    \rlap{\kern 2.575in\lower\graphtemp\hbox to 0pt{\hss \hss}}\special{ar 2575 1075 75 75 0 6.28319}\special{pa 2575 1150}\special{pa 2575 1250}\special{fp}\special{sh 1.000}\special{pa 2600 1200}\special{pa 2575 1250}\special{pa 2550 1200}\special{pa 2600 1200}\special{fp}\graphtemp=.5ex\advance\graphtemp by 1.575in
    \rlap{\kern 2.575in\lower\graphtemp\hbox to 0pt{\hss .\hss}}\graphtemp=.5ex\advance\graphtemp by 1.650in
    \rlap{\kern 2.575in\lower\graphtemp\hbox to 0pt{\hss .\hss}}\graphtemp=.5ex\advance\graphtemp by 1.725in
    \rlap{\kern 2.575in\lower\graphtemp\hbox to 0pt{\hss .\hss}}\special{ar 2075 75 75 75 0 6.28319}\special{pa 2144 104}\special{pa 2500 250}\special{fp}\special{sh 1.000}\special{pa 2463 208}\special{pa 2500 250}\special{pa 2444 254}\special{pa 2463 208}\special{fp}\special{pa 2115 138}\special{pa 2500 750}\special{fp}\special{sh 1.000}\special{pa 2495 694}\special{pa 2500 750}\special{pa 2452 721}\special{pa 2495 694}\special{fp}\special{pa 2101 146}\special{pa 2500 1250}\special{fp}\special{sh 1.000}\special{pa 2507 1194}\special{pa 2500 1250}\special{pa 2459 1211}\special{pa 2507 1194}\special{fp}\special{ar 2075 1075 75 75 0 6.28319}\special{pa 2115 1012}\special{pa 2500 400}\special{fp}\special{sh 1.000}\special{pa 2452 429}\special{pa 2500 400}\special{pa 2495 456}\special{pa 2452 429}\special{fp}\special{pa 2144 1046}\special{pa 2500 900}\special{fp}\special{sh 1.000}\special{pa 2444 896}\special{pa 2500 900}\special{pa 2463 942}\special{pa 2444 896}\special{fp}\special{pa 2135 1121}\special{pa 2500 1400}\special{fp}\special{sh 1.000}\special{pa 2475 1350}\special{pa 2500 1400}\special{pa 2445 1389}\special{pa 2475 1350}\special{fp}\special{ar 3075 575 75 75 0 6.28319}\special{pa 2642 359}\special{pa 3008 541}\special{fp}\special{sh 1.000}\special{pa 2974 497}\special{pa 3008 541}\special{pa 2952 541}\special{pa 2974 497}\special{fp}\special{pa 2642 791}\special{pa 3008 609}\special{fp}\special{sh 1.000}\special{pa 2952 609}\special{pa 3008 609}\special{pa 2974 653}\special{pa 2952 609}\special{fp}\special{pa 2617 1263}\special{pa 3033 637}\special{fp}\special{sh 1.000}\special{pa 2985 665}\special{pa 3033 637}\special{pa 3026 693}\special{pa 2985 665}\special{fp}\hbox{\vrule depth1.725in width0pt height 0pt}\kern 3.150in
  }} \centerline{\raise 1em\box\graph}\vspace{1em}
~~~~\emph{A net fragment \hfill and its 1-unfolding}~~~
\vspace{-.65em}
\end{figure}

We argue that the causal and branching time behaviour of the
represented system is preserved under 1-unfolding. When dealing with
labelled Petri nets, all copies  of a transition  carry the
same label as . In this setting, common semantic equivalences like
the \phrase{fully concurrent bisimulation} \emph{equivalence} \cite{BDKP91}
or the \emph{(hereditary) history preserving bisimulation
equivalence}\index{history preserving bisimulation} \cite{vG06} under
either the individual or collective token interpretation identify a
net and its 1-unfolding.

Note that the construction above does not introduce self-loops. Thus
unfoldings of pure nets remain pure.
We therefore have translations between arbitrary pure nets, event
structures, configuration structures and propositional theories, as
indicated in the introduction.

\index{self-sequential interpretation of Petri nets}
It is possible to give a slightly different interpretation of nets,
namely by excluding transitions from firing concurrently with
themselves (cf.\ \cite{GR83}).\footnote{This distinction is
independent of the individual--collective token dichotomy, thus
yielding four computational interpretations of nets \cite{vG05}.}
This amounts to simplifying \df{firing} by requiring  to be a set
rather than a multiset. Under this interpretation our unfolding
could introduce concurrency that was not present before. However, for
this purpose \df{1-unfolding} can be adapted by removing the initial
tokens from the places  for  and  (but leaving
the token in ), and adding an arc from transition 
to place  for every  and .

\out{
 \subsection{Self-concurrency}

 In older papers on Petri nets a multiset of transitions was allowed to
 fire only if it was a set, i.e., no transition could fire multiple
 times concurrent with itself. The argument for this restriction was
 that a transition could be thought of as a subsystem like a printer,
 that can only print one file at a time. When there are enough tokens in
 its preplaces (representing print-requests and other preconditions for
 printing) to handle two files, these have to be printed one by one.
 This argument has been convincingly rebutted in {\sc Goltz \& Reisig}
 \cite{GR83}, and since then multisets are generally allowed to fire.
 In any case, the behaviour of nets under the \phrase{self-sequential}
 firing rule can easily be encoded into the behaviour of nets
 under the \phrase{self-concurrent} firing rule of \df{firing} by the
 following proposition.

 \begin{proposition}{self-sequential}
 Any net N can be transformed into a net  such that
 \begin{itemise}
 \item under the self-sequential firing rule  behaves the same as N,
 \item in N' we have  only if  is a set.
 \end{itemise}
 \end{proposition}

 \begin{proof}
 For any transition  in N add a \phrase{self-loop}, consisting of a place
  with  and . This yields the required net .
 \end{proof}
 Further on we assume the firing rule of \df{firing}, but indicate when
 necessary what has to be changed in case the self-sequential firing
 rule is assumed.
}

\out{
 \subsubsection*{Multiset systems}

 \begin{definition}{multiset system}
 A \phrase{multiset system} is a pair  with  a
 set and  a collection of multisets over .
 \end{definition}
 A multiset system is a generalised configuration structure in which
 events may occur multiple times in the same system run. As indicated in
 \df{configuration}, the configuration structure associated to any pure
 Petri net  is a multiset system; it is a set system only when
  is a 1-occurrence net.  We have no direct method of translating
 multiset systems back into Petri nets, nor do we have generalisations
 of event structures or propositional theories matching the expressive
 power of multiset systems.  However, every multiset system can be
 ``1-unfolded'' into a set system representing the same behaviour (cf.\
 Section~\ref{computational}), thereby making the extra expressive
 power of multiset systems less relevant.

 \begin{definition}{multi2set}
 Let  be a multiset system. The \phrase{1-unfolding}
  of  into a set system is given by
 \begin{itemise}
 \item ,
 \item  iff , for , where  is given by .
 \end{itemise}
 \end{definition}
 The above definitions assume that multisets may have only finitely
 many occurrences of the same event, an assumption that holds for the
 multiset systems associated to Petri nets. However, the generalisation
 to multisets with arbitrary many occurrences of the same event is
 fairly straightforward.

 Now we have two translation from Petri nets into configuration
 structures, as indicated below.
 
 The following proposition, whose proof is straightforward, says that
 this diagram commutes.
 \begin{proposition}{1-unfolding}
 Let  be a Petri net.
 The configuration structure associated to the 1-unfolding of 
 equals the 1-unfolding of the configuration structure associated to
 . \hfill 
 \end{proposition}
}

\out{
 \subsection{Marking equivalence (to be omitted)}

 One often is interested in the behaviour of nets as far as it can be
 expressed in terms of transition firings. The places etc.\ are then
 seen as just a tool in specifying such behaviour. In this view, one of the
 most discriminating behavioural equivalences we can think of in the
 collective token framework is the following notion of \phrase{marking
 equivalence}:
 \begin{definition}{marking equivalence}
 Two nets  and  are \phrase{marking equivalent} if 
 and there exists a bijection  between their reachable markings,
 such that the initial markings are related, , and .
 \end{definition}
 Note that marking equivalence preserves all causal information
 present in the net representation of a concurrent system. Whether two
 transitions are causally independent is a context-sensitive matter. It
 varies with the markings enabling them both. In such a marking two
 transitions are independent iff they can fire in one step. This
 kind of information is present in the step transition relation \plat{\goto{U}}.
 The nets A and B are marking equivalent. However,
 for many purposes marking equivalence is too fine. 
 It distinguishes for instance the nets P and Q below, as well as M and N\@.

 \begin{figure}[t]
  \input{marking}
  \centerline{\raise 1em\box\graph}
  \end{figure}



 \phrase{Reachable configuration equivalence} is defined similarly and is
 strictly coarser than (reachable) marking equivalence:
 The nets P and Q are (reachable) configuration equivalent.
 However M and N below are not.  The reason is that although in N all
 transitions have a different identity, even though they have the same label.
 \out{
  \begin{figure}[htb]
  \input{identity}
  \centerline{\raise 1em\box\graph}
  \end{figure}
  }
}

\section{Computational interpretation}\label{computational}

In this section we formalise the dynamic behaviour of configuration
structures, Petri nets and event structures, by defining a transition
relation between their configurations. This transition relation tells
how a represented system can evolve from one state to another.  We
prove that on the classes of pure 1-occurrence nets and pure event
structures the translations of Section~\ref{four models} preserve
these transition relations, and show that this result does not extend
to impure 1-occurrence nets or impure event structures.

We indicate that impure nets and event structures may be captured by
considering configuration structures upgraded with an explicit
transition relation between their configurations. However, the
methodology of the present paper is incapable of providing transition
preserving translations between general event structures, 1-occurrence
nets and the upgraded configuration structures.  It is for this reason
that we focus on pure nets and pure event structures.

Our transition relation for Petri nets is derived directly from the
firing rule, which constitutes the standard computational
interpretation of nets. The idea of explicitly defining a transition
relation between the configurations of an event structure may be new,
but we believe that our transition relation is the only natural
candidate that is consistent with the notion of configuration employed
in {\sc Winskel} \cite{Wi87a,Wi89}
(cf.~Sections~\ref{equivalence} and~\ref{brands}).
Our transition relation on configuration structures is chosen so as to
match the ones on nets and on event structures, and formalises a
computational interpretation of configuration structures which we call
the \phrase{asynchronous interpretation}.

We briefly discuss two alternative interpretations of configuration
structures, formalised by alternative transition relations.
The first is the computational interpretation of Chu spaces from {\sc
Gupta \& Pratt} \cite{GP93a,Gup94,Pr94a}.  The second
is a variant of our asynchronous interpretation, based on the
assumption that only finitely many events can happen in a finite time.
This \phrase{finitary asynchronous interpretation} matches the
standard computational interpretation of Petri nets better than does
the asynchronous interpretation, although it falls short in explaining
uncountable configurations of event structures \cite{Wi87a,Wi89}.  We
point out some problems that stand in the way of lifting the
computational interpretation of nets to the infinitary level.

\subsection{The asynchronous interpretation}\label{asynchronous}



\begin{definition}{transitions}
Let  be a configuration structure. For  in 
write  if 
and 
The relation  is called the \phrase{step transition relation}.
\end{definition}
Here  indicates that the represented system can go from
state  to state  by concurrently performing a number of events
(namely those in ). The first requirement is unavoidable.
The second one
says that a number of events can be performed concurrently, or
simultaneously, only if they can be performed in any order. This
requirement represents our postulate that different events do not
synchronise in any way; they can happen in one step only if they are
causally independent. Hence our transition relation  and
the corresponding computational interpretation of configuration
structures is termed \phrase{asynchronous}.

The \phrase{single-action transition relation}  on
 is given by  iff  and
 is a singleton.  In pictures we omit transitions of the form
, that exists for every configuration , we indicate
the single-action transition relation by solid arrows, and the rest of
the step transition relation by dashed ones.

\begin{example}{transitions}
These are the transition relations for
 and
two structures E and F.
\begin{center}
\begin{picture}(23,25)(-13,-5)
\put(0,0){\makebox[0pt]{}} \put(-2,3){\vector(-1,1){5}}
\put(-10,10){\makebox[0pt]{}}  \put(2,3){\vector(1,1){5}}
\put(10,10){\makebox[0pt]{}}   \put(-8,13){\vector(1,1){5}}
\put(0,20){\makebox[0pt]{}}  \put(8,13){\vector(-1,1){5}}
\put(0,16.6){\vector(0,1){1}}
\put(0,4){\dashbox{1}(0,12){}}
\put(-13,-5){\makebox[0pt][l]{}}
\end{picture}
\hfill
\begin{picture}(22,30)(-10,0)
\put(0,0){\makebox[0pt]{}} \put(-2,3){\vector(-1,1){5}}
\put(-10,10){\makebox[0pt]{}}  \put(2,3){\vector(1,1){5}}
\put(10,10){\makebox[0pt]{}}   \put(-8,23){\vector(1,1){5}}
\put(-10,20){\makebox[0pt]{}}  \put(-10,14){\vector(0,1){5}}
\put(10,20){\makebox[0pt]{}}   \put(10,14){\vector(0,1){5}}
\put(0,30){\makebox[0pt]{}}  \put(8,23){\vector(-1,1){5}}
\put(-15,0){\makebox[0pt]{}}
\end{picture}
\hfill
\begin{picture}(28,30)(-10,0)
\put(0,0){\makebox[0pt]{}} \put(-2,3){\vector(-1,1){5}}
\put(-10,10){\makebox[0pt]{}}  \put(2,3){\vector(1,1){5}}
\put(10,10){\makebox[0pt]{}}   \put(-10,24){\vector(0,1){5}}
\put(-10,20){\makebox[0pt]{}}  \put(-10,14){\vector(0,1){5}}
\put(10,20){\makebox[0pt]{}}   \put(10,14){\vector(0,1){5}}
\put(-10,30){\makebox[0pt]{}} \put(10,24){\vector(0,1){5}}
\put(10,30){\makebox[0pt]{}} 
\put(-15,0){\makebox[0pt]{}}
\end{picture}
\end{center}
\end{example}
Such pictures of configuration structures are somewhat misleading
representations, as they suggest a notion of global time, under which
at any time the represented system is in one of its states, moving
from one state to another by following the transitions. Although this
certainly constitutes a valid interpretation, we favour a more truly
concurrent view, in which all events can be performed independently,
unless the absence of certain configurations indicates otherwise.
Under this interpretation, the configurations can be thought of as
\emph{possible} states the system can be in, \emph{from the point of
view of a possible observer}. They are introduced only to indicate (by
their absence) the dependencies between events in the represented system.

In particular, in the structure D above, the events 
and  are completely independent, and there is no need to assume
that they are performed either simultaneously or in a particular
order.  The ``diagonal'' in the picture serves merely to remind us of
the independence of these events.  In terms of higher dimensional
automata \cite{Pr91a} it indicates that ``the square is filled in''.

On the other hand, the absence of any ``diagonals'' in E indicates two
distinct linearly ordered computations. In one the event  can only
happen after event , and  in turn has to wait for ; the other
has a causal ordering . There is no way to view  and  as
independent; if there were, there should be a transition
\plat{\emptyset \goto{}_\eC \{d,e\}}.  In labelled versions of
configuration structures, a computationally motivated semantic
equivalence would identify the structures E and F, provided the events
 and  carry the same label. We do not address such semantic
equivalences in this paper, however.

The configuration structure E is completely axiomatised by the two
clauses  \vspace{-2em} 
indicating the absence of configurations  and , respectively.
On the other hand, D has the empty axiomatisation.
An event structure representing D is given by the enabling relation
, ,
 and , whereas
an enabling relation
for  is , ,
, , ,
, ,
 and .
Petri net representations of D and E are given below.
\expandafter\ifx\csname graph\endcsname\relax \csname newbox\endcsname\graph\fi
\expandafter\ifx\csname graphtemp\endcsname\relax \csname newdimen\endcsname\graphtemp\fi
\setbox\graph=\vtop{\vskip 0pt\hbox{\special{pn 8}\special{ar 89 89 89 89 0 6.28319}\graphtemp=.5ex\advance\graphtemp by 0.089in
    \rlap{\kern 0.089in\lower\graphtemp\hbox to 0pt{\hss \hss}}\special{pa 0 536}\special{pa 179 536}\special{pa 179 357}\special{pa 0 357}\special{pa 0 536}\special{fp}\graphtemp=.5ex\advance\graphtemp by 0.446in
    \rlap{\kern 0.089in\lower\graphtemp\hbox to 0pt{\hss \hss}}\special{pa 89 179}\special{pa 89 357}\special{fp}\special{sh 1.000}\special{pa 114 321}\special{pa 89 357}\special{pa 64 321}\special{pa 114 321}\special{fp}\special{ar 446 89 89 89 0 6.28319}\graphtemp=.5ex\advance\graphtemp by 0.089in
    \rlap{\kern 0.446in\lower\graphtemp\hbox to 0pt{\hss \hss}}\special{pa 357 536}\special{pa 536 536}\special{pa 536 357}\special{pa 357 357}\special{pa 357 536}\special{fp}\graphtemp=.5ex\advance\graphtemp by 0.446in
    \rlap{\kern 0.446in\lower\graphtemp\hbox to 0pt{\hss \hss}}\special{pa 446 179}\special{pa 446 357}\special{fp}\special{sh 1.000}\special{pa 471 321}\special{pa 446 357}\special{pa 421 321}\special{pa 471 321}\special{fp}\graphtemp=.5ex\advance\graphtemp by 0.804in
    \rlap{\kern 0.268in\lower\graphtemp\hbox to 0pt{\hss D\hss}}\special{ar 1875 89 89 89 0 6.28319}\graphtemp=.5ex\advance\graphtemp by 0.089in
    \rlap{\kern 1.875in\lower\graphtemp\hbox to 0pt{\hss \hss}}\special{ar 1875 804 89 89 0 6.28319}\special{pa 1429 536}\special{pa 1607 536}\special{pa 1607 357}\special{pa 1429 357}\special{pa 1429 536}\special{fp}\graphtemp=.5ex\advance\graphtemp by 0.446in
    \rlap{\kern 1.518in\lower\graphtemp\hbox to 0pt{\hss \hss}}\special{ar 1866 437 357 357 -2.915184 -1.797205}\special{sh 1.000}\special{pa 1550 328}\special{pa 1518 357}\special{pa 1502 317}\special{pa 1550 328}\special{fp}\special{ar 1866 456 357 357 1.797205 2.915184}\special{sh 1.000}\special{pa 1757 771}\special{pa 1786 804}\special{pa 1745 820}\special{pa 1757 771}\special{fp}\special{ar 1161 446 89 89 0 6.28319}\graphtemp=.5ex\advance\graphtemp by 0.446in
    \rlap{\kern 1.161in\lower\graphtemp\hbox to 0pt{\hss \hss}}\special{pa 1250 446}\special{pa 1429 446}\special{fp}\special{sh 1.000}\special{pa 1393 421}\special{pa 1429 446}\special{pa 1393 471}\special{pa 1393 421}\special{fp}\special{pa 1786 536}\special{pa 1964 536}\special{pa 1964 357}\special{pa 1786 357}\special{pa 1786 536}\special{fp}\graphtemp=.5ex\advance\graphtemp by 0.446in
    \rlap{\kern 1.875in\lower\graphtemp\hbox to 0pt{\hss \hss}}\special{pa 1875 357}\special{pa 1875 179}\special{fp}\special{sh 1.000}\special{pa 1850 214}\special{pa 1875 179}\special{pa 1900 214}\special{pa 1850 214}\special{fp}\special{pa 1875 714}\special{pa 1875 536}\special{fp}\special{sh 1.000}\special{pa 1850 571}\special{pa 1875 536}\special{pa 1900 571}\special{pa 1850 571}\special{fp}\special{pa 2143 536}\special{pa 2321 536}\special{pa 2321 357}\special{pa 2143 357}\special{pa 2143 536}\special{fp}\graphtemp=.5ex\advance\graphtemp by 0.446in
    \rlap{\kern 2.232in\lower\graphtemp\hbox to 0pt{\hss \hss}}\special{ar 1884 437 357 357 -1.344388 -0.226408}\special{sh 1.000}\special{pa 2248 317}\special{pa 2232 357}\special{pa 2200 328}\special{pa 2248 317}\special{fp}\special{ar 1884 456 357 357 0.226408 1.344388}\special{sh 1.000}\special{pa 2005 820}\special{pa 1964 804}\special{pa 1993 771}\special{pa 2005 820}\special{fp}\special{ar 2589 446 89 89 0 6.28319}\graphtemp=.5ex\advance\graphtemp by 0.446in
    \rlap{\kern 2.589in\lower\graphtemp\hbox to 0pt{\hss \hss}}\special{pa 2500 446}\special{pa 2321 446}\special{fp}\special{sh 1.000}\special{pa 2357 471}\special{pa 2321 446}\special{pa 2357 421}\special{pa 2357 471}\special{fp}\graphtemp=.5ex\advance\graphtemp by 0.804in
    \rlap{\kern 1.161in\lower\graphtemp\hbox to 0pt{\hss E\hss}}\hbox{\vrule depth0.893in width0pt height 0pt}\kern 2.679in
  }} \centerline{\box\graph}

\begin{example}{absence}
Take the system G, represented below as a configuration structure with
a transition relation, a propositional theory, an event structure and
a Petri net.
\begin{figure}[ht]
\begin{picture}(65,40)(-13,-2)
\put(0,0){\makebox[0pt]{}} \put(-2,3){\vector(-1,1){5}}
\put(-10,10){\makebox[0pt]{}}  \put(2,3){\vector(1,1){5}}
\put(10,10){\makebox[0pt]{}}   \put(-8,13){\vector(1,1){5}}
\put(0,20){\makebox[0pt]{}}  \put(8,13){\vector(-1,1){5}}
\put(0,16.6){\vector(0,1){1}}
\put(0,4){\dashbox{1}(0,12){}}
\put(20,20){\makebox[0pt]{}} \put(12,13){\vector(1,1){5}}
\put(10,30){\makebox[0pt]{}} \put(2,23){\vector(1,1){5}}
\put(10,26.6){\vector(0,1){1}}
\put(10,14){\dashbox{1}(0,12){}}      \put(18,23){\vector(-1,1){5}}
\put(-13,30){\makebox[0pt][l]{}}
\put(34,30){\makebox{}}
\put(34,7){\makebox{}}
\put(34,1){\makebox{}}
\put(34,-5){\makebox{ for }}
\end{picture}~~~~~~~~~~~~~~~~
\expandafter\ifx\csname graph\endcsname\relax \csname newbox\endcsname\graph\fi
\expandafter\ifx\csname graphtemp\endcsname\relax \csname newdimen\endcsname\graphtemp\fi
\setbox\graph=\vtop{\vskip 0pt\hbox{\special{pn 8}\special{ar 89 446 89 89 0 6.28319}\graphtemp=.5ex\advance\graphtemp by 0.446in
    \rlap{\kern 0.089in\lower\graphtemp\hbox to 0pt{\hss \hss}}\special{pa 0 893}\special{pa 179 893}\special{pa 179 714}\special{pa 0 714}\special{pa 0 893}\special{fp}\graphtemp=.5ex\advance\graphtemp by 0.804in
    \rlap{\kern 0.089in\lower\graphtemp\hbox to 0pt{\hss \hss}}\special{pa 89 536}\special{pa 89 714}\special{fp}\special{sh 1.000}\special{pa 114 679}\special{pa 89 714}\special{pa 64 679}\special{pa 114 679}\special{fp}\special{ar 446 89 89 89 0 6.28319}\graphtemp=.5ex\advance\graphtemp by 0.089in
    \rlap{\kern 0.446in\lower\graphtemp\hbox to 0pt{\hss \hss}}\special{pa 357 536}\special{pa 536 536}\special{pa 536 357}\special{pa 357 357}\special{pa 357 536}\special{fp}\graphtemp=.5ex\advance\graphtemp by 0.446in
    \rlap{\kern 0.446in\lower\graphtemp\hbox to 0pt{\hss \hss}}\special{pa 446 179}\special{pa 446 357}\special{fp}\special{sh 1.000}\special{pa 471 321}\special{pa 446 357}\special{pa 421 321}\special{pa 471 321}\special{fp}\special{ar 446 804 89 89 0 6.28319}\special{pa 446 536}\special{pa 446 714}\special{fp}\special{sh 1.000}\special{pa 471 679}\special{pa 446 714}\special{pa 421 679}\special{pa 471 679}\special{fp}\special{pa 357 1250}\special{pa 536 1250}\special{pa 536 1071}\special{pa 357 1071}\special{pa 357 1250}\special{fp}\graphtemp=.5ex\advance\graphtemp by 1.161in
    \rlap{\kern 0.446in\lower\graphtemp\hbox to 0pt{\hss \hss}}\special{pa 446 893}\special{pa 446 1071}\special{fp}\special{sh 1.000}\special{pa 471 1036}\special{pa 446 1071}\special{pa 421 1036}\special{pa 471 1036}\special{fp}\hbox{\vrule depth1.250in width0pt height 0pt}\kern 0.536in
  }} {\raise 8em\box\graph}
\end{figure}
There is no need to assume, as following the transitions might
suggest, that in any execution of G the event  happens either after
 or before ; when actions may have a duration,  may overlap
with both  and . The configuration structure, with its step
transition relation, is not meant to order  with respect to  and
.\linebreak All it does is specify that  comes after , and it
does so by not including configurations  and . This is
concisely conveyed by the representation of G as a propositional
theory in conjunctive normal form.
\end{example}

\subsection{Petri nets}\label{PN computational}

The firing relation between markings induces a transition relation
between the configurations of a net:
\begin{definition}{transitions-PN}
The \phrase{step transition relation}  between the
configurations ,  of a net N is given by\vspace{-1ex}

\end{definition}
We now show that on pure 1-occurrence nets this step transition
relation matches the one on configuration structures defined above.

\begin{proposition}{transitions-PN}
In a pure net  we have
\begin{center}  iff 
\end{center}
for all  in . (In case N is a pure 1-occurrence net, the
right-hand side can be written as .)
\end{proposition}

\begin{proof}``Only if'': Let  for .\\ Then  and  is
enabled under , i.e., . Now let
. Then ^\bullet
(y-x), so\\ ^\bullet\!x^\bullet (Z\!-\!x)^\bullet (Z\!-\!x) i.e.,  is a
configuration of . Note that for this direction pureness is not needed.

``If'': 
Suppose  and , but .
Then \mbox{} is not enabled under , i.e., there is a place
, such that . Let  be the multiset of
those transitions  in  for which . Then ^\bullet (y-x). As N is pure, for all
transitions  we have , i.e., .
Hence  i.e., . Yet .
\end{proof}
It follows that the step transition relation on a pure net  is
\pagebreak[2]
completely determined by the set of configurations of , and that
for pure 1-occurrence nets this transition relation exactly matches
the one of \df{transitions}.  This makes  an acceptable
abstract representation of a pure 1-occurrence net .

\out{ Moreover, the transition relation of \df{transitions}
 generalises verbatim to multiset systems (reading  for
  and  for ), and by \pr{transitions-PN} this
 generalised transition relation exactly matches the one on pure Petri nets.
}

On an impure net  the step transition relation is in general not
determined by the set of configurations of .  The 1-occurrence
nets P and M below have very different behaviour: in P the transitions
 and  can be done in parallel (there is a transition ), whereas in M there is mutual exclusion.
\begin{figure}[htb]
\expandafter\ifx\csname graph\endcsname\relax \csname newbox\endcsname\graph\fi
\expandafter\ifx\csname graphtemp\endcsname\relax \csname newdimen\endcsname\graphtemp\fi
\setbox\graph=\vtop{\vskip 0pt\hbox{\graphtemp=.5ex\advance\graphtemp by 0.089in
    \rlap{\kern 0.000in\lower\graphtemp\hbox to 0pt{\hss P:\hss}}\special{pn 8}\special{ar 250 89 89 89 0 6.28319}\graphtemp=.5ex\advance\graphtemp by 0.089in
    \rlap{\kern 0.250in\lower\graphtemp\hbox to 0pt{\hss \hss}}\special{pa 250 179}\special{pa 250 357}\special{fp}\special{sh 1.000}\special{pa 275 321}\special{pa 250 357}\special{pa 225 321}\special{pa 275 321}\special{fp}\special{pa 161 536}\special{pa 339 536}\special{pa 339 357}\special{pa 161 357}\special{pa 161 536}\special{fp}\special{pa 250 536}\special{pa 250 714}\special{fp}\special{sh 1.000}\special{pa 275 679}\special{pa 250 714}\special{pa 225 679}\special{pa 275 679}\special{fp}\special{ar 250 804 89 89 0 6.28319}\graphtemp=.5ex\advance\graphtemp by 0.446in
    \rlap{\kern 0.250in\lower\graphtemp\hbox to 0pt{\hss \hss}}\special{ar 607 89 89 89 0 6.28319}\graphtemp=.5ex\advance\graphtemp by 0.089in
    \rlap{\kern 0.607in\lower\graphtemp\hbox to 0pt{\hss \hss}}\special{pa 607 179}\special{pa 607 357}\special{fp}\special{sh 1.000}\special{pa 632 321}\special{pa 607 357}\special{pa 582 321}\special{pa 632 321}\special{fp}\special{pa 518 536}\special{pa 696 536}\special{pa 696 357}\special{pa 518 357}\special{pa 518 536}\special{fp}\special{pa 607 536}\special{pa 607 714}\special{fp}\special{sh 1.000}\special{pa 632 679}\special{pa 607 714}\special{pa 582 679}\special{pa 632 679}\special{fp}\special{ar 607 804 89 89 0 6.28319}\graphtemp=.5ex\advance\graphtemp by 0.446in
    \rlap{\kern 0.607in\lower\graphtemp\hbox to 0pt{\hss \hss}}\graphtemp=.5ex\advance\graphtemp by 0.089in
    \rlap{\kern 1.071in\lower\graphtemp\hbox to 0pt{\hss M:\hss}}\special{ar 1321 89 89 89 0 6.28319}\graphtemp=.5ex\advance\graphtemp by 0.089in
    \rlap{\kern 1.321in\lower\graphtemp\hbox to 0pt{\hss \hss}}\special{pa 1321 179}\special{pa 1321 357}\special{fp}\special{sh 1.000}\special{pa 1346 321}\special{pa 1321 357}\special{pa 1296 321}\special{pa 1346 321}\special{fp}\special{pa 1232 536}\special{pa 1411 536}\special{pa 1411 357}\special{pa 1232 357}\special{pa 1232 536}\special{fp}\special{pa 1321 536}\special{pa 1321 714}\special{fp}\special{sh 1.000}\special{pa 1346 679}\special{pa 1321 714}\special{pa 1296 679}\special{pa 1346 679}\special{fp}\special{ar 1321 804 89 89 0 6.28319}\special{ar 1679 446 89 89 0 6.28319}\graphtemp=.5ex\advance\graphtemp by 0.446in
    \rlap{\kern 1.679in\lower\graphtemp\hbox to 0pt{\hss \hss}}\graphtemp=.5ex\advance\graphtemp by 0.446in
    \rlap{\kern 1.321in\lower\graphtemp\hbox to 0pt{\hss \hss}}\special{ar 1501 468 143 143 -2.250941 -0.636548}\special{sh 1.000}\special{pa 1614 340}\special{pa 1615 383}\special{pa 1574 369}\special{pa 1614 340}\special{fp}\special{ar 1501 425 143 143 0.636548 2.250941}\special{sh 1.000}\special{pa 1423 578}\special{pa 1411 536}\special{pa 1454 539}\special{pa 1423 578}\special{fp}\special{ar 2036 89 89 89 0 6.28319}\graphtemp=.5ex\advance\graphtemp by 0.089in
    \rlap{\kern 2.036in\lower\graphtemp\hbox to 0pt{\hss \hss}}\special{pa 2036 179}\special{pa 2036 357}\special{fp}\special{sh 1.000}\special{pa 2061 321}\special{pa 2036 357}\special{pa 2011 321}\special{pa 2061 321}\special{fp}\special{pa 1946 536}\special{pa 2125 536}\special{pa 2125 357}\special{pa 1946 357}\special{pa 1946 536}\special{fp}\special{pa 2036 536}\special{pa 2036 714}\special{fp}\special{sh 1.000}\special{pa 2061 679}\special{pa 2036 714}\special{pa 2011 679}\special{pa 2061 679}\special{fp}\special{ar 2036 804 89 89 0 6.28319}\graphtemp=.5ex\advance\graphtemp by 0.446in
    \rlap{\kern 2.036in\lower\graphtemp\hbox to 0pt{\hss \hss}}\special{ar 1857 468 143 143 -2.505044 -0.890651}\special{sh 1.000}\special{pa 1783 369}\special{pa 1742 383}\special{pa 1743 340}\special{pa 1783 369}\special{fp}\special{ar 1857 425 143 143 0.890651 2.505044}\special{sh 1.000}\special{pa 1903 539}\special{pa 1946 536}\special{pa 1934 578}\special{pa 1903 539}\special{fp}\hbox{\vrule depth0.893in width0pt height 0pt}\kern 2.125in
  }} \centerline{\raise 1em\box\graph}
\end{figure}
Yet their configurations are the same: .
Therefore it is not a good idea to represent each 1-occurrence net
 by the configuration structure .


\subsection{Event structures}\label{ES computational}

\begin{definition}{transitions-ES}
The \phrase{step transition relation}  between
configurations  of an event structure
 is given by

\end{definition}
This formalises the intuition provided in Section~\ref{event structures}.
The following proposition says that for pure event structures
this transition relation also exactly matches the one of \df{transitions}.
\begin{proposition}{transitions-ES}
Let  be a pure event structure, and
. Then  iff .
\end{proposition}
\begin{proof}
We have to establish that
\begin{center}
 iff .
\end{center}
``Only if'' follows immediately from the definitions.
For ``if'' let  and . Let .
Then , so .
Hence, by \df{EtoC}, .
As  is pure, , hence , as
required.
\end{proof}
This makes  an acceptable abstract representation of a pure
event structure .

As for Petri nets, \pr{transitions-ES} does not
generalise to impure event structures, with again the systems 
and  serving as a counterexample. An event structure
representation for  is , with  and
 given by , ,
,  and .
Another counterexample is the event structure, say H, of \ex{asymmetric
conflict}. The transition relations of P, M and H are
\begin{center}
\begin{picture}(31,22)(-18,0)
\put(0,0){\makebox[0pt]{}} \put(-2,3){\vector(-1,1){5}}
\put(-10,10){\makebox[0pt]{}}  \put(2,3){\vector(1,1){5}}
\put(10,10){\makebox[0pt]{}}   \put(-8,13){\vector(1,1){5}}
\put(0,20){\makebox[0pt]{}}  \put(8,13){\vector(-1,1){5}}
\put(0,16.6){\vector(0,1){1}}
\put(0,4){\dashbox{1}(0,12){}}
\put(-18,20){\makebox[0pt][l]{P:}}
\end{picture}
\hfill
\begin{picture}(31,22)(-18,0)
\put(0,0){\makebox[0pt]{}} \put(-2,3){\vector(-1,1){5}}
\put(-10,10){\makebox[0pt]{}}  \put(2,3){\vector(1,1){5}}
\put(10,10){\makebox[0pt]{}}   \put(-8,13){\vector(1,1){5}}
\put(0,20){\makebox[0pt]{}}  \put(8,13){\vector(-1,1){5}}
\put(-18,20){\makebox[0pt][l]{M:}}
\end{picture}
\hfill
\begin{picture}(31,22)(-18,0)
\put(0,0){\makebox[0pt]{}} \put(-2,3){\vector(-1,1){5}}
\put(-10,10){\makebox[0pt]{}}  \put(2,3){\vector(1,1){5}}
\put(10,10){\makebox[0pt]{}}   \put(-8,13){\vector(1,1){5}}
\put(0,20){\makebox[0pt]{}}  
\put(-18,20){\makebox[0pt][l]{H:}}
\end{picture}
\end{center}

\subsection{The impure case}\label{sec-impure}

In order to provide an adequate abstract representation of impure
1-occurrences nets or impure event structures one could use triples
 with  a configuration
structure and  an explicitly
defined transition relation between its configurations. To capture
arbitrary Petri nets one could further allow the configurations to be
multisets of events, rather than sets.

\begin{definition}{multiset-transition system}
A \phrase{multiset transition system} is a tri\-ple
 with  a set,  a
collection of multisets over  and .

For a configuration structure ,
an event structure  and
a Petri net ,
the \phrase{associated multiset transition system}
is given by , \hfill
 \hfill and
, respectively.

Two structures  and  that may be configuration structures,
event structures and/or Petri nets are \phrase{transition equivalent}
if .
\end{definition}
By Propositions~\ref{pr-transitions-PN} and~\ref{pr-transitions-ES},
for pure 1-occurrence nets transition equivalence coincides with
configuration equivalence, and for pure event
structures it coincides with -equivalence.

We conjecture that there exist maps between 1-occurrence nets
and event structures that preserve transition equivalence.  However,
the set-up of the present paper, that uses propositional theories up
to logical equivalence as a stepping stone in the translation from
event structures to Petri nets, is insufficient to establish this
beyond the pure case. It is for this reason that we focus on pure nets
and pure event structures.

\subsection{The Gupta-Pratt
interpretation}\label{GP}\index{Gupta-Pratt interpretation}

Our configuration structures are, up to isomorphism, the
\emph{extensional} \phrase{Chu spaces} of {\sc Gupta \& Pratt}
\cite{GP93a,Gup94,Pr94a}. It was in their work that the idea arose of
using the full generality of such structures in modelling
concurrency. It should be noted however that the computational
interpretation they give in \cite{GP93a,Gup94,Pr94a} differs somewhat
from the asynchronous interpretation above;
it can be formalised by means of the step transition relation given by
 \cite{GP93a,Pr94a}, thereby
dropping the asynchronicity requirement of \df{transitions}.  This
allows a set of events to occur in one step even if they cannot happen
in any order.

When using the translations between configuration structures and Petri
nets described in Section~\ref{PN}, the Gupta-Pratt interpretation of
configuration structures matches a firing rule on Petri nets
characterised by the possibility of borrowing tokens during the execution
of a multiset of transitions: a multiset  of transitions would be
enabled under a marking  when ^\bullet U. In that case  can fire under , yielding .
Thus the requirement that ^\bullet U is dropped;
tokens that are consumed by the transitions in  may be borrowed
when not available in , as long as they are returned ``to the
bank'' when reproduced by the firing of .

\begin{example}{causality}\index{causality}
The configuration structure  with  and
 models a system in which the
events  and  jointly cause  as their \emph{immediate} effect,
as it is impossible to have done both  and  without doing  also.
Below are the representations of the same system as a propositional
theory, an event structure and a Petri net.
\\
X
\hfill\hfill
\expandafter\ifx\csname graph\endcsname\relax \csname newbox\endcsname\graph\fi
\expandafter\ifx\csname graphtemp\endcsname\relax \csname newdimen\endcsname\graphtemp\fi
\setbox\graph=\vtop{\vskip 0pt\hbox{\special{pn 8}\special{ar 89 89 89 89 0 6.28319}\graphtemp=.5ex\advance\graphtemp by 0.089in
    \rlap{\kern 0.089in\lower\graphtemp\hbox to 0pt{\hss \hss}}\special{pa 0 536}\special{pa 179 536}\special{pa 179 357}\special{pa 0 357}\special{pa 0 536}\special{fp}\graphtemp=.5ex\advance\graphtemp by 0.446in
    \rlap{\kern 0.089in\lower\graphtemp\hbox to 0pt{\hss \hss}}\special{pa 89 179}\special{pa 89 357}\special{fp}\special{sh 1.000}\special{pa 114 321}\special{pa 89 357}\special{pa 64 321}\special{pa 114 321}\special{fp}\special{ar 89 804 89 89 0 6.28319}\special{pa 89 536}\special{pa 89 714}\special{fp}\special{sh 1.000}\special{pa 114 679}\special{pa 89 714}\special{pa 64 679}\special{pa 114 679}\special{fp}\special{ar 804 89 89 89 0 6.28319}\graphtemp=.5ex\advance\graphtemp by 0.089in
    \rlap{\kern 0.804in\lower\graphtemp\hbox to 0pt{\hss \hss}}\special{pa 714 536}\special{pa 893 536}\special{pa 893 357}\special{pa 714 357}\special{pa 714 536}\special{fp}\graphtemp=.5ex\advance\graphtemp by 0.446in
    \rlap{\kern 0.804in\lower\graphtemp\hbox to 0pt{\hss \hss}}\special{pa 804 179}\special{pa 804 357}\special{fp}\special{sh 1.000}\special{pa 829 321}\special{pa 804 357}\special{pa 779 321}\special{pa 829 321}\special{fp}\special{ar 804 804 89 89 0 6.28319}\special{pa 804 536}\special{pa 804 714}\special{fp}\special{sh 1.000}\special{pa 829 679}\special{pa 804 714}\special{pa 779 679}\special{pa 829 679}\special{fp}\special{ar 446 446 89 89 0 6.28319}\graphtemp=.5ex\advance\graphtemp by 0.446in
    \rlap{\kern 0.446in\lower\graphtemp\hbox to 0pt{\hss \hss}}\special{pa 357 893}\special{pa 536 893}\special{pa 536 714}\special{pa 357 714}\special{pa 357 893}\special{fp}\graphtemp=.5ex\advance\graphtemp by 0.804in
    \rlap{\kern 0.446in\lower\graphtemp\hbox to 0pt{\hss \hss}}\special{pa 446 714}\special{pa 446 536}\special{fp}\special{sh 1.000}\special{pa 421 571}\special{pa 446 536}\special{pa 471 571}\special{pa 421 571}\special{fp}\special{pa 714 804}\special{pa 536 804}\special{fp}\special{sh 1.000}\special{pa 571 829}\special{pa 536 804}\special{pa 571 779}\special{pa 571 829}\special{fp}\special{pa 179 804}\special{pa 357 804}\special{fp}\special{sh 1.000}\special{pa 321 779}\special{pa 357 804}\special{pa 321 829}\special{pa 321 779}\special{fp}\special{pa 536 446}\special{pa 714 446}\special{fp}\special{sh 1.000}\special{pa 679 421}\special{pa 714 446}\special{pa 679 471}\special{pa 679 421}\special{fp}\special{pa 357 446}\special{pa 179 446}\special{fp}\special{sh 1.000}\special{pa 214 471}\special{pa 179 446}\special{pa 214 421}\special{pa 214 471}\special{fp}\special{ar 446 1161 89 89 0 6.28319}\graphtemp=.5ex\advance\graphtemp by 1.161in
    \rlap{\kern 0.446in\lower\graphtemp\hbox to 0pt{\hss \hss}}\special{pa 446 1071}\special{pa 446 893}\special{fp}\special{sh 1.000}\special{pa 421 929}\special{pa 446 893}\special{pa 471 929}\special{pa 421 929}\special{fp}\hbox{\vrule depth1.250in width0pt height 0pt}\kern 0.893in
  }} {\box\graph}
\hfill\mbox{}\\emptyset = x_0 \goto{}_\eC x_1 \goto{}_\eC \ldots \goto{}_\eC x_n =x.\begin{array}[b]{ccc}
 \simeq	&\Rightarrow	&\simeq_{\fR}   \\
 \Downarrow&		&\Downarrow	\\
 \simeq_f&\Rightarrow	&\simeq_{\fR_f}	\\
 \end{array}.\emptyset = x_0 \goto{}_\eC x_1 \goto{}_\eC x_2 \goto{}_\eC \ldots.\fR(\fF(\eC)) = \fR(\fF(\eD)) \mbox{~~iff~~} \fS(\fF(\eC)) = \fS(\fF(\eD)).{I \goto{U_1} M_1 \goto{U_2} \cdots \goto{U_n} M_n}.\forall i\in\IN.~ X_i \subseteq X_{i+1} \wedge
\forall Y \subseteq X_{i+1}.~ \exists Z \subseteq X_{i}.~ Z \turn Y.\forall i<n.~ X_i \subseteq X_{i+1} \wedge
\forall Y \subseteq X_{i+1}.~ \exists Z \subseteq X_{i}.~ Z \turn Y.\forall i\leq n.~ \forall Y \subseteq \{e_1,...,e_{i}\}.~
\exists Z \subseteq \{e_1,...,e_{i-1}\}.~ Z \turn Y.\forall i\in\IN.~ \forall Y \subseteq \{e_1,...,e_{i}\}.~
\exists Z \subseteq \{e_1,...,e_{i-1}\}.~ Z \turn Y.\emptyset=X_0\goto{}_\eE X_1\goto{}_\eE X_2\goto{}_\eE\ldots\fR(\fS(\fL(\eE)))=\fR(\fS(\eE))=\fS(\eE)=\fS(\fL((\eE)).
\vspace{-1.5ex}1ex]
\mbox{~~~~}\hfill.
\end{proof}

\begin{corollary}{hyperconnected}
Let  be a pure and -secure event structure.
Then  is hyperconnected.
Conversely, if  is a hyperconnected configuration structure, then  is a pure and
-secure event structure.\hfill
\end{corollary}
Using \thm{CtoEtoC}, \pr{pure ES commute} yields

\begin{proposition}{CtoE-reachable}
Let  be a connected configuration structure. Then
.
\end{proposition}
\begin{proof}
.
\end{proof}
Likewise, if  is a hyperconnected configuration structure
then ; if  is a finitary connected
configuration structure then ; and if  is a
configuration structure of the form  with  a finitary
configuration structure then .

Thus  and  provide a bijective correspondence between pure
event structures up to reachable equivalence and connected
configuration structures (using \pr{pure ES commute}, \df{reachable},
\thm{CtoEtoC} and the above).  Likewise,  and  provide a
bijective correspondence between pure and -secure event
structures up to reachable equivalence and hyperconnected configuration
structures (additionally using Corollaries~\ref{cor-hyperconnected}
and \ref{cor-secure} and \pr{secured equivalence});
 and  provide a bijective correspondence between pure
event structures up to finitary reachable equivalence and finitary
connected configuration structures; and  and  provide a
bijective correspondence between pure event structures up to finitary
reachable equivalence and configuration structures of the form
 with  finitary (additionally using \pr{secured
finitary equivalence}).

\subsubsection*{Impure event structures}

\pr{pure ES commute} does not extend to impure event structures. For those,
their reachable configurations are not determined by their left-closed ones.

\begin{example}{impure}
Let .
Then ,
whereas .
Both configuration structures are connected.

Let .
Then we have  but .
\end{example}
When the step transition relation of \df{transitions-ES} is taken to
be part of the meaning of an event structure, neither the left-closed
nor the reachable configurations capture the meaning of impure event
structures faithfully, as illustrated by the systems P and M mentioned
in Section~\ref{ES computational}. When, on the other hand, the
behaviour of an event structure is deemed to be determined by its
configurations, then on impure event structures  and 
represent mutually inconsistent interpretations.
However, under either interpretation the impure event structures are
redundant: for every event structure there exists a pure one with the
same configurations.  Obviously, which one depends on whether the
left-closed or the reachable configurations are to be preserved.

\begin{proposition}{purification}
For any event structure E there is
a pure event structure  with ,
and a pure event structure  with .
\end{proposition}

\begin{proof}
One can take  to be 
and   to be .
\end{proof}
A structure  can also be directly
obtained by putting .

\pr{purification} shows that any event structure could be
transformed into a pure one, while preserving its reachable
configurations.  However, there is no way to purify any event
structure while preserving its secured configurations:

\begin{example}{unpurifiable}
Let  be given by ,
 and  for 
and \mbox{} for  not of the form  or .
Then\linebreak
,
 and .
The configuration  is not secured, because countably many
stages are needed to perform the events in , and whenever both
 and  happen,  needs to happen first.
Using \pr{pure ES commute},  cannot be the set of secured
configurations of a pure event structure, because  is not of
the form : as  would contain all sets
 for ,  would also
contain their limit .
\end{example}
As  above is -secure, \ex{unpurifiable} also shows that
\cor{hyperconnected} does not extend to impure structures.

\subsubsection*{Reachably pure event structures}\label{reachably pure}

\out{
 As we will see in Section~\ref{brands}, those event structures
 appearing in the literature for which a notion of configuration
 has been studied that corresponds to our concept of a left-closed
 configuration translate to event structures in our sense that are pure.
 The notion of a secured configuration on the other hand has been
 applied to impure event structures as well. The event structures of
 \ex{impure} for instance can easily be cast as event structures in the
 sense of {\sc Winskel} \cite{Wi87a,Wi89}. However, the event structures
 studied before correspond to even structures in our sense that are
 \phrase{singular}, meaning that whenever , either 
 or  is a singleton. For such event structures impureness is
 harmless, because, as is straightforward to check, the impure bits of
 the enabling relation can be deleted without changing the set of
 secured configurations or the transition relation between them in any way.
}

For impure event structures, the functions , , ,
 and  do not reflect the step transition relation
between configurations and hence may translate an event structure into
a configuration structure with a different computational interpretation.
This is illustrated by the event structure M of Section~\ref{ES
computational}, for which we have 
but .
We now extend the class of pure event structures to a slightly
larger class of \phrase{reachably pure} event structures, on which the
functions , ,  and , but not , still
preserve the computational interpretation of event structures.
This extension is necessary in order to cast the event structures of
{\sc Winskel} \cite{Wi87a,Wi89} as special cases of ours, for they
translate into our framework as event structures that are reachably
pure but not pure.

\begin{definition}{reachably pure}
An event structure is \phrase{reachably pure} if  only if
either  or .
\end{definition}
The event structure  of \ex{impure} for instance is reachably
pure, but not pure.

\begin{proposition}{reachably pure}
For every reachably pure event structure  there exists a pure
event structure  such that  iff
 for all  and
 with .
Also, if  is rooted, so is .
\end{proposition}
\begin{proof}
Obtain  by omitting all enablings  with . Apply \df{transitions-ES}.
\end{proof}
\begin{corollary}{reachably pure}
For any reachably pure event structure  one has
, ,
 and \plat{\fS_f(\hat\eE)\mathbin=\fS_f(\eE)}.
Moreover,  is -secure iff  is.
\end{corollary}
However, in \ex{impure} we have .

With the above results and \pr{pure ES commute}, all results for
configuration structures in this section, namely
Propositions~\ref{pr-reachable transitions}--\ref{pr-secured finitary
equivalence} and \ob{secure}, lift to reachably pure event structures:

\begin{corollary}{reachable transition equivalence ES}
Let  be a reachably pure event structure,
 and .
Then  iff .
\hfill 
\end{corollary}

\begin{corollary}{secured transition equivalence ES}
A reachably pure event structure
 is -secure
iff for all  and all  one has
 iff .
\hfill 
\end{corollary}

\begin{corollary}{ES analogies}
For any reachably pure event structure  it holds that
, ,
 and .
\hfill 
\end{corollary}

\begin{corollary}{ES analogies secure}
For any reachably pure and -secure event structure  it holds that
, i.e.,  is hyperconnected.
\hfill 
\end{corollary}

\begin{corollary}{reachable equivalence ES}
Let  and  be reachably pure and -secure event structures.
Then  iff .
\hfill 
\end{corollary}

\begin{corollary}{finitary reachable equivalence ES}
Let  be reachably pure event structures.
Then  iff .
\hfill 
\end{corollary}
We call two reachably pure event structures  and 
\emph{reachably equivalent}\index{reachable equivalence} iff  and
\emph{finitarily reachably equivalent}\index{finitary reachable equivalence}
iff .  Restricted to pure event structures
these definitions agree with \df{reachable}.

\subsubsection*{Secure event structures}\label{secure}

When dealing with secured configurations, we will mainly be interested
in event structures  that are reachable pure and
-secure, and satisfy .
The third property says that all secured configurations of  are in
fact left-closed configurations. Together, these three properties ensure that the
computational behaviour of  is adequately represented by .
An event structure with these properties is called \phrase{secure}.

\begin{proposition}{hyperconnected secure}
If  is a hyperconnected configuration structure, then 
is secure.
\end{proposition}

\begin{proof}
Hyperconnected configuration structures are -secure, so
that  is pure and -secure follows from \cor{secure}.
Moreover, using \pr{pure ES commute},
.
\end{proof}
Thus  and  provide a bijective correspondence between secure
event structures up to reachable equivalence and hyperconnected
configuration structures.

\out{
 \begin{example}{insecure analogies}
 Take  with , ,  for all finite  with , and  for  infinite. This event structure is secure.
 However, , so
  does not satisfy .
 \end{example}
}

\paragraph{Remark}
For reachably pure event structures , unlike for configuration
structures, the requirement  does not imply
-security. Moreover, this requirement would be insufficient in
\cor{reachable equivalence ES}.

\begin{example}{insecure ES}
Take  with  for
 finite, and  otherwise. This event structure is reachably pure
and satisfies . However, ,
yet .

Take  and  for all
.  The event structures  and  have the same secured
configurations, yet are not reachably equivalent.
\end{example}

\section{Other brands of event structures}\label{brands}

Event structures have been introduced in {\sc Nielsen, Plotkin \&
Winskel} \cite{NPW81} as triples , in {\sc Winskel}
\cite{Wi87a} as triples  and ,
and in {\sc Winskel} \cite{Wi89} as triples  and
 ---a special case of those in \cite{NPW81}.  Here
we will explain how our event structures generalise these previous proposals.
The components , ,  and  that occur in the
triples mentioned above can be defined in terms of our event
structures as follows.

\begin{definition}{consistency}
Let  be an event structure.
A set of events  is \phrase{consistent}, written , if\vspace{-1ex}
        
The binary \phrase{conflict relation}  is given by
 iff .
Write  for `` is finite and consistent''---this is our
rendering of the component  in \cite{Wi87a}.
For  and , write  for

The \phrase{direct causality relation}  is given by

We take the \phrase{causality relation}, , to be the reflexive
and transitive closure of .
\end{definition}
The next definition gives various properties of our event structures
which, in suitable combinations, determine subclasses corresponding to
the various event structures in \cite{NPW81,Wi87a,Wi89}.

\begin{definition}{event-properties}
An event structure  is
\begin {itemise}
\item \phrase{singular} if ,
\item \phrase{conjunctive} if ,
\item \phrase{locally conjunctive} if ,
\item \phrase{-ir\-re\-dun\-dant} if every event occurs in a secured
      configuration, i.e., ,
\item \phrase{-irredundant} if every event occurs in a left-closed
      configuration, i.e., ,
\item and \phrase{cycle-free} if there is no chain\\
     \mbox{}\hfill\hfill\mbox{}
\end{itemise}
and has
\begin{itemise}
\item \phrase{finite causes} if   finite,
\item \phrase{finite conflict} if  infinite 
\item and \phrase{binary conflict} if .
\pagebreak[3]
\end{itemise}
\end{definition}
As we will explain below,
the event structures of \cite{NPW81,Wi87a,Wi89} all correspond to
event structures in our sense that are rooted, singular and with
finite conflict. The event structures given as triples involving 
even have binary conflict, the ones from \cite{Wi87a,Wi89} have finite
causes, and the ones involving  are conjunctive,
-irredundant and cycle-free.  The event structures of
\cite{Wi87a,Wi89} that involve  are moreover -irredundant,
a property that implies -irredundancy and cycle-freeness.
The requirement of \emph{stability} in \cite{Wi87a,Wi89} corresponds to
our notion of local conjunctivity.

Each of the correspondences above will be established by means of
evident translations from the class of event structures from
\cite{NPW81,Wi87a,Wi89} under consideration to the class of our event
structures with the mentioned properties, and vice versa. These
translations will preserve the sets of events of related structures as
well as their configurations. However, which configurations will be
preserved varies, as indicated in Table~\ref{7 classes}.
\begin{table}
\mbox{}\hfill\small
\begin{tabular}{@{}|@{~}l@{}r@{~}|@{~}l@{~\,}r@{~}|@{}}
\hline
ev.\,str.\,\cite{Wi87a} &  & rtd, sing, f.causes \& f.conflict& \\
stable     \cite{Wi87a} &  & same \& locally conjunctive& \\
prime      \cite{Wi87a} &  & same \& conjunctive \& -irr.& \\
ev.\,str.\,\cite{Wi89}  &  &  rtd, sing, f.causes \& bin.conflict& \\
stable     \cite{Wi89}  &  & same \& locally conjunctive& \\
prime      \cite{Wi89}  &  & same \& conjunctive \& -irr.& \\
ev.\,str.\,\cite{NPW81} &  & rtd, sing, b.c., conj, -irr \& c.-f.& \\
\hline
\end{tabular}\hfill\mbox{}\vspace{-1ex}
\caption{7 corresponding classes of event structures}
\label{7 classes}
\vspace{-.7em}
\end{table}
The configurations employed in \cite{Wi87a,Wi89} correspond to our
secured configurations, whereas the configurations employed for event
structures involving  correspond to our left-closed
configurations. In the intersection of those two situations, the
secured and left-closed configurations of event structures coincide.

\begin{definition}{manifestly conjunctive}
An event structure is \phrase{manifestly conjunctive} if for every set of
events  there is at most one set  with .
\end{definition}
Every conjunctive event structure can be made manifestly conjunctive
by deleting from , for every set , all but the smallest 
for which . The property of conjunctivity implies that such
a smallest  exists. This normalisation preserves -equivalence
and even transition equivalence (cf.~\df{multiset-transition system})
and all properties of \df{event-properties}.
The event structures in our sense that arise as translations of event
structures from \cite{NPW81,Wi87a,Wi89} that involve  are all
manifestly conjunctive.

\begin{observation}{manifestly conjunctive}
Any singular, cycle-free, manifestly conjunctive event structure is pure.
\end{observation}
Hence the translations between the event structures from
\cite{NPW81,Wi87a,Wi89} involving  and subclasses of our event
structures will preserve not only -equivalence, but even transition
equivalence.

\begin{lemma}{finite conflict secure}
If  has finite conflict, then .
\end{lemma}
\begin{proof}
Let  and let  be a stepwise securing of .
Let . Then either  is infinite and  or
 is finite and hence contained in  for some .
In the latter case  with .
\end{proof}

\begin{observation}{singular}~\vspace{-6pt}
\item Any singular event structure is reachably pure.
\end{observation}

\begin{proposition}{finite conflict secure}
Any singular event structure with finite conflict is \hyperref[secure]{secure}.
\end{proposition}
\begin{proof}
Let  be a singular event structure with finite conflict.
Then the event structure , as defined in the proof of
\pr{reachably pure}, is pure and with finite conflict. \lem{finite
conflict secure} yields .  Hence,
.
The other direction follows from \cor{ES analogies}.
\end{proof}
As all event structures of \cite{NPW81,Wi87a,Wi89} correspond to event
structures in our sense that are singular and with finite conflict,
they all fall in the scope of Corollaries~\ref{cor-reachable
transition equivalence ES} and~\ref{cor-reachable equivalence ES}, so
reachable equivalence preserves the computational interpretation of
event structures and is characterised by having the same secured
configurations. Hence the translations between the event structures
from \cite{Wi87a,Wi89} and subclasses of our event structures will
preserve reachable equivalence. We will show that they also preserve
-equivalence, and even transition equivalence
(cf.~\df{multiset-transition system}); however, this involves {\em
defining} the left-closed configurations and a transition relation on
the structures of \cite{Wi87a,Wi89}.

\subsection{\normalsize Left-closed configurations and transitions}

For singular event structures , the enabling relation consists
of two parts: enablings of the form  with , and enablings
of the form . When  has finite conflict, the first
part can be fully expressed in terms of , at least to the extent
to which it determines which sets of events are configurations. When
 has finite causes, the second part can similarly be expressed in
terms of . One obtains the following.

\begin{observation}{Wi87a}
Let  be a singular event structure with finite causes and
finite conflict. Then\1.5ex]
\mbox{~}\hfill \hfill
\end{observation}
It follows that such event structures can alternatively be represented
as triples  with 
symmetric and irreflexive and , as
are the structures of \cite{Wi89}.

When , any configuration containing  also contains .
When  is conjunctive and satisfies
 for all , then for any event
 there is a smallest set  with .
In that case, the part of the enabling relation consisting of enablings  is in essence completely determined by the causality relation
.

\begin{observation}{prime}
Let  be a singular, conjunctive event structure
with finite conflict, such that  for all . Then

If  moreover is rooted and with binary conflict, then\X \in L(\eE) \Leftrightarrow \left\{\begin{array}{@{}l@{}}
 \forall d,e \in X.~ \neg (d \# e) \, \wedge\\
 \forall d,e \in E.~ d \leq e \in X  \Rightarrow d \in X
 \end{array}\right.X \in R_f(\eE) \Leftrightarrow \left\{\begin{array}{@{}l@{}}
\fCon(X) \, \wedge\\
\exists e_1, \ldots, e_n \in X.~X = \{e_1,...,e_n\}\,
\wedge \\ \forall i\leq n.~ \{e_1,...,e_{i-1}\} \turn_s e_i.
\end{array}\right.X \in R_f(\eE) \Leftrightarrow \left\{\begin{array}{@{}l@{}}
\forall d,e \in X.~ \neg (d \# e) \, \wedge\\
\exists e_1, \ldots, e_n \in X.~X = \{e_1,...,e_n\}\,
\wedge \\ \forall i\leq n.~ \{e_1,...,e_{i-1}\} \turn_s e_i.
\end{array}\right.X \in R_f(\eE) \Leftrightarrow \left\{\begin{array}{@{}l@{}}
 \fCon(X) \, \wedge\\
 \forall d,e \in E.~ d \leq e \in X  \Rightarrow d \in X.
 \end{array}\right.X \in R_f(\eE) \Leftrightarrow \left\{\begin{array}{@{}l@{}}
 X \mbox{ is finite} \, \wedge
 \forall d,e \in X.~ \neg (d \# e) \, \wedge\\
 \forall d,e \in E.~ d \leq e \in X  \Rightarrow d \in X.
 \end{array}\right.X \in S(\eE) \Leftrightarrow \forall Y \subseteq_{\it fin} X.~
\exists Z \in R_f(\eE).~ Y \subseteq Z \subseteq X,X \in S(\eE) \Leftrightarrow \left\{\begin{array}{@{}l@{}}
\forall Y \subseteq_{\it fin} X.~ \fCon(Y) \, \wedge\\
\forall e \in X.~ \exists e_0, \ldots, e_n \in X.~ e=e_n\,
\wedge \\ \forall i\leq n.~ \{e_0,...,e_{i-1}\} \turn_s e_i.
\end{array}\right.\fR(\eE)=\fR(\eF) ~\mbox{ iff }~ \fS(\eE)=\fS(\eF) ~\mbox{ iff }~
\fR_f(\eE)=\fR_f(\eF).\vspace{-3pt}1.5ex]
\mbox{~}\hfill \hfill
\end{observation}
For  a left-closed configuration of a singular, conjunctive event
structure and  we say that  can happen at stage , if
there is no chain .  Now we have
\pagebreak[2]
 iff each event in  can happen at some finite stage.
It follows that:

\begin{observation}{prime-secured}
Let  be a singular, conjunctive event structure.
Then
\begin{enumerate}
\item .
\item  is -irredundant iff .
\item  is -irredundant iff  is -irredundant and
for every  there is an  such that there is no chain
.
\item In case  is cycle-free we have\vspace{-1ex}

\item If  is -irredundant then .
\end{enumerate}
\out{
 
 If  moreover is rooted and with binary conflict, then
 
 Let  be a singular, conjunctive event structure.
 In case  is cycle-free, every
 finite set  in  can be seen to be in :
 
 If  moreover is rooted and with binary conflict, then
 
}
\end{observation}
Together with \lem{finite conflict secure} this yields
\begin{corollary}{prime-secured}
Let  be a singular, conjunctive,
-irredundant event structure with finite conflict.
Then .\hfill
\end{corollary}

\out{
 \begin{proposition}{infinitary conjunctive configurations}
 Let  be a singular, conjunctive, -irredundant event
 structure with finite conflict. Then .
 \end{proposition}

 \begin{proof}
 ``'' follows immediately from \df{secured
 configurations}, using that  has finite conflict.

 ``'': For any , let
  be . Any secured or left-closed
 configuration containing  must contain . As  is
 -irredundant,  must occur in a secured configuration. Hence
  is finite, and . Thus  by \ob{prime}.

 Now suppose . For any , let
  be . It must be that
 . Hence .
 Now \lem{consistent unions} implies that .
 Moreover, \downarrow\! Y, so
 direction ``'' of \pr{infinite secured configurations}
 (not using the requirement of finite causes) implies that .
 \end{proof}
}

\subsection[The event structures of Winskel 1987]{The event structures of {\sc Winskel} \cite{Wi87a}}
\label{Wi87a}

These are defined as triples  where
\begin{itemise}
\item  is a set of \phrase{events},
\item  is a nonempty \phrase{consistency
      predicate} such that:
      ,
\item and  is the \phrase{enabling
      relation}, which satisfies .
\end{itemise}
Such an event structure is \phrase{stable} if it satisfies

\index{families of configurations of event structures}The family  of configurations of such an event
structure (written  in \cite{Wi87a}) consists of those
 which are
\begin{itemise}
\item \phrase{consistent}: every finite subset of  is in ,
\item and \phrase{secured}: ,
\end{itemise}
just as in \pr{infinite secured configurations}.
In addition, we define  and  exactly as in
Observation~\ref{obs-Wi87a}, but reading  for  and
 for . Again, we write  for
, and  for .

Here we will show that up to reachable equivalence and even
transition equivalence (cf.~\df{multiset-transition system}) these
event structures are exactly the ones in our sense which are rooted,
singular, with finite causes and with finite conflict; and the stable
event structures of \cite{Wi87a} are the ones which are moreover
locally conjunctive.

For  an event structure as in
\cite{Wi87a}, let the event structure 
be given by\vspace{-1em}

Now, for ,

and whenever  we have


\begin{proposition}{Wi87a to ours}
Let  be an event structure as in \cite{Wi87a}. Then
 is rooted, singular and with finite causes and finite
conflict. If  is stable then  is locally conjunctive.
Moreover,  and .
\end{proposition}

\begin{proof} Let  be an event
structure as in \cite{Wi87a}. As  is nonempty and
subset-closed we have . Thus
, i.e.,  is rooted. By
construction,  is singular and with finite causes and finite
conflict.  That the stability of  implies the local conjunctivity
of  follows because  and
every collection of finite sets has a finite subcollection with the
same intersection. With \pr{infinite secured configurations} and
\ob{Wi87a}, respectively, using (\ref{Con}) and (\ref{enabling}), one
easily checks that  and .
\end{proof}
For  a rooted event structure, the structure , where  and  are given by
\df{consistency}, is clearly an event structure in the sense of
\cite{Wi87a}.

\begin{proposition}{ours to Wi87a}
Let  be a rooted, singular event structure with finite causes and
finite conflict. Then 
and .
Moreover,  is stable if  is locally conjunctive.
\end{proposition}

\begin{proof}
Trivial, with \pr{infinite secured configurations} and \ob{Wi87a}.
\end{proof}

\subsection[The event structures of Winskel 1989]{The event structures of {\sc Winskel} \cite{Wi89}}
\label{Wi89}

These are defined as triples  where
\begin{itemise}
\item  is a set of \phrase{events},
\item  is a symmetric, irreflexive
      \phrase{conflict relation}. Write  for the set of finite,
      conflict-free subsets of , i.e., those finite subsets
       for which 
\item and  is the \phrase{enabling
      relation}, which satisfies .
\end{itemise}
Such an event structure is \phrase{stable} if it satisfies

The family  of configurations of such an event
structure (written  in \cite{Wi89}) consists of those
 which are
\begin{itemise}
\item \phrase{conflict-free}: ,
\item and \phrase{secured}: ,
\end{itemise}
just as in \ob{Wi89-secured}.  Note that a set of events  is
conflict-free iff every finite subset of  is in . In
addition, we define  and  exactly as in
Observation~\ref{obs-Wi89}, reading  for  and 
for .

Say that an event structure  in the sense of
\cite{Wi87a} has \phrase{binary conflict} if for any : 
Clearly, the event structures of \cite{Wi89} are just a reformulation
of the event structures of \cite{Wi87a} that have binary conflict.
A small variation of the arguments from the previous section shows
that, up to reachable equivalence and even transition equivalence, the event
structures of \cite{Wi89} are exactly the ones in our sense which are
rooted, singular, with finite causes and with binary conflict; and the
stable event structures of \cite{Wi89} are the ones which are moreover
locally conjunctive:

For  an event structure as in
\cite{Wi89}, let the event structure 
be given by 
Write  for .
Then equations (\ref{Con}) and (\ref{enabling}) of Section~\ref{Wi87a} hold again.

\begin{proposition}{Wi89 to ours}
Let  be an event structure as in \cite{Wi89}. Then
 is rooted, singular and with finite causes and binary
conflict. If  is stable then  is locally conjunctive.
Moreover,  and .
\end{proposition}

\begin{proof} Let  be an event
structure as in \cite{Wi89}. By construction,  is rooted,
singular and with finite causes and binary conflict.  That the
stability of  implies the local conjunctivity of 
follows exactly as in the proof of \pr{Wi87a to ours}.  With
Observations~\ref{obs-Wi89-secured} and \ref{obs-Wi89}, respectively,
one obtains  and .
\end{proof}
For  a rooted event structure with binary
conflict, the structure ,
where  is given by \df{consistency} and
  iff

is clearly an event structure in the sense of \cite{Wi89}.
\begin{proposition}{ours to Wi89}
Let  be a rooted, singular event structure with finite causes and
binary conflict. Then 
and .
Moreover,  is stable if  is locally conjunctive.
\end{proposition}

\begin{proof}
The first two statements are trivial, with
Observations~\ref{obs-Wi89-secured} and \ref{obs-Wi89}.  Now assume
 is locally conjunctive; we show that  is
stable. So assume ,  and for  it holds that . The latter means that
either  or . We have to show that .
\1ex]
{\sc proof:}
Let . We have to find a  with .
In case  or  we can take , because E is
rooted and with binary conflict.

In case  with , we have .

In case , we use  to
infer that there is an  with .

In case  with , then  and hence .
\X \turn Y \mbox {~iff~} \left\{\begin{array}{@{}l@{}}
\mbox{ and } \\
\mbox{or ,  and } \\
\mbox{or  is infinite and .} \end{array}\right.\forall d,e,f \in E.~ d \leq e \wedge d \# f \Rightarrow e \# f,\{d \in E \mid d \leq e\} \mbox{ is finite for all } e \in E,X \turn Y \mbox {~iff~} \left\{\begin{array}{@{}l@{}}
\mbox{ and } \\
\mbox{or , ,  and } \\
\mbox{or  and .} \end{array}\right.2<|X|<\infty \Rightarrow \emptyset \turn X.\forall Y\!\subseteq_{\it fin}\!X.~ \Cn{Y}\forall Y\!\subseteq\!X.~ (|Y|\!\leq\!2 \implies \Cn{Y}).A \subseteq C \wedge \Cn{\bigcup A} \Rightarrow \bigcup A \in C\emptyset \neq A \subseteq C \Rightarrow \bigcap A \in C\emptyset \neq A \subseteq C \wedge \Cn{\bigcup A} \Rightarrow \bigcap A \in C[\forall Y \!\!\subseteq\! X.~(Y \mbox{ finite} \Rightarrow \exists z
\!\in\! C.~ Y \!\!\subseteq\! z \!\subseteq\! X)] \Rightarrow X \!\!\in\! C[\forall Y \!\!\subseteq\! X.~(|Y| \leq 2 \Rightarrow \exists z
\!\in\! C.~ Y \!\!\subseteq\! z \!\subseteq\! X)] \Rightarrow X \!\!\in\! CA \subseteq C \wedge \fCn{\bigcup A} \Rightarrow \bigcup A \in CA \subseteq C \wedge \bCn{\bigcup A} \Rightarrow \bigcup A \in C\emptyset \neq A \subseteq C \wedge \fCn{\bigcup A} \Rightarrow \bigcap A \in C\emptyset \neq A \subseteq C \wedge \bCn{\bigcup A} \Rightarrow \bigcap A \in C.\begin{array}{ccccc}
\bbbigcup\mbox{-closed}&\implies&\fbbigcup\mbox{-closed}&\implies&
\bbigcup\mbox{-closed}\\
\Downarrow && \Downarrow \\
\mbox{binary conflict} &\implies&\mbox{finite conflict}
\end{array}[\forall x,y\!\in\! A.~\exists z\!\in\!A.~x\cup y \subseteq z]\Rightarrow
\bigcup A \in C[\forall x,y\!\in\! A.~\exists z\!\in\!C.~x\cup y \subseteq z
\subseteq \bigcup A]\Rightarrow \bigcup A \in C\vspace{-2ex} A \subseteq C \wedge \forall F\subseteq_{\it fin}\!A.~\Cn
{\bigcup F} \Rightarrow \bigcup A \in C\vspace{-2ex} A \subseteq C \wedge \forall x,y\!\in\!A.~\Cn
{x\cup y} \Rightarrow \bigcup A \in C\vspace{-2ex} \emptyset\neq A \subseteq C \wedge \forall
F\subseteq_{\it fin}\!A.~\Cn{\bigcup F} \Rightarrow \bigcap A \in C\vspace{-2ex} \emptyset\neq A \subseteq C \wedge \forall
x,y\!\in\!A.~\Cn{x\cup y} \Rightarrow \bigcap A \in C.\forall x,y\!\in\! A.~\exists z\!\in\!A.~x\cup y \subseteq z.\forall Y \!\!\subseteq\! X.~(Y \mbox{finite} \Rightarrow \exists
z\!\in\! C.~ Y \!\!\subseteq\! z\!\subseteq\! X).A \subseteq C \wedge \forall x,y\!\in\!A.~\Cn{x\cup y}.A \subseteq C \wedge \bCn{\bigcup A}.A
\subseteq C \wedge \forall F\subseteq_{\it fin}\!A.~\Cn{\bigcup F}.1ex]
\mbox{}\hfill.\hfill\E := \{a_i \mid i \geq 1\}  \cup \{b,c\} \cup  \{d_i \mid i \geq 1\} \vspace{-1ex}\emptyset,~ \{a_i \mid i \geq 1\} \cup \{b\},~ \{a_i \mid i \geq 1\}
 \cup \{c\}\vspace{-1ex} \{a_1, \ldots, a_{n},d_n,b,c \}.\vspace{-1ex}(\bigvee_{j\in J}\bigwedge Y_{j})\mbox{ }\wedge \bigwedge_{j,k\in J,~j\neq k} 
         \neg (\bigwedge Z_{j,k} \wedge \bigwedge Z_{k,j})(\bigwedge X \implies \bigvee_{j\in J}\bigwedge Y_{j})
 \wedge \bigwedge_{j,k\in J,~j\neq k} 
         \neg (e_{j,k} \wedge e_{k,j})\varphi_X := (\bigwedge X \Rightarrow \bigvee_{Y\turn X} \bigwedge Y){\dot{\varphi}_X := (\bigwedge X \Rightarrow
\bigvee^{\makebox{\raisebox{-11pt}[0pt][0pt]{\Huge}}}_{Y\turn' X}
\bigwedge Y)}{\cal U}_{x,y} = \{m \subseteq E \mid x \subseteq m, (m \cap y)  =\emptyset\}E:=\{a_i, b_i \mid i\in\IN\}
\cup \{c\},R_f(\eE) = \left.\left\{\begin{array}{@{}l@{}}
\{a_i,b_i\mid i<n\}\\
\{a_i,b_i\mid i<n\}\cup\{a_n\}\\
\{a_i,b_i\mid i<n\}\cup\{c\}
\end{array}\;\right|\; n\in\IN \right\},S(\eE)=R_f(\eE)\cup\{\{a_i,b_i\mid i\in\IN\}\}L(\eE)=S(\eE)\cup\{E\}.~~~~\forall Y \subseteq X.~ (Y
\mbox{ finite} \implies \exists z\!\in\!C.~ Y \subseteq  z \subseteq  X).\forall Y \subseteq X.~
(|Y| \leq 2 \implies \exists z\!\in\!C.~ Y \subseteq  z \subseteq  X).\forall
Y\!\subseteq\!X.~Y\mbox{ finite (resp.\ }|Y|\!\leq\!2) \implies \exists
z\!\in\!C.~ Y \subseteq z \subseteq X.\forall Y \subseteq X.~ Y \mbox{ finite} \implies \exists
z\!\in\!C^f.~ Y \subseteq  z \subseteq  XS((\fS(\eC))^f)\hspace{-.55pt}\subseteq\hspace{-.55pt} S(\eC^f)
\hspace{-.55pt}=\hspace{-.55pt} S(\eC) \hspace{-.55pt}=\hspace{-.55pt}
S(\fS(\eC)) \hspace{-.55pt}\subseteq\hspace{-.55pt} S((\fS(\eC))^f).\forall n\in \IN.~ x_n \subseteq x_{n+1} \wedge \forall Y \subseteq
x_{n+1}.~ \exists Z \subseteq x_{n}.~ Z \turn Y.\forall Y\!\subseteq_{\it fin}\! X.~\exists z\!\in\!C^f\!\!.~Y\!\subseteq\!z
\Leftrightarrow
\forall Y\!\subseteq_{\it fin}\! X.~\exists z\!\in\!C.~Y\!\subseteq\!z,1em]
Moreover, for configuration structures satisfying the axiom of
finiteness and closed under \raisebox{-1.5pt}[0pt]{} we
reformulate the condition of being hyperconnected.

\begin{proposition}{coincidence-freeness-hyperconnectedness}
Let  be a configuration structure closed under
\raisebox{-1.5pt}[0pt]{} and satisfying the axiom of
finiteness.  Then  is hyperconnected iff it is coincidence-free.
\end{proposition}

\begin{proof}
``Only if'' is trivial, so suppose  is
coincidence-free. Closure under \raisebox{-2pt}[0pt]{}
immediately implies that , so it remains to be
shows that .  Let . For any  say
that  can happen at stage  if  is the smallest cardinality of
a subconfiguration of  containing . By the axiom of finiteness,
this cardinality is always finite. Let  be the set of all events
in  that can happen at stage . Then ,
 for  and .  As  is the union of all subconfigurations of  of
size  and  is closed under , we have
 for .\linebreak[3] Let  for some .  It suffices to show that
. For any \mbox{} pick a subconfiguration 
of  of  elements, containing .  Given that  does not
have a proper subconfiguration containing , for any  in
, by coincidence-freeness, there must be subconfiguration 
of  with , showing that . It follows that .  Hence  and as  is closed under \plat{\bbigcup}
we have .
\end{proof}
We now apply the results of this section to characterise the secured
configuration structures associated to the various event structures of
{\sc Winskel} \cite{Wi87a,Wi89}.

\begin{corollary}{characterisation-Wi87a}
A configuration structure arises as the family of (secured)
configurations of an event structure of \cite{Wi87a} iff it satisfies
the axioms of rootedness, finiteness, coincidence-freeness and
finite-completeness.

A configuration structure arises as the family of (secured)
configurations of a stable event structure of \cite{Wi87a} iff it
moreover is closed under .
\hfill 
\end{corollary}
These characterisations were obtained earlier in \cite{Wi87a}.
However, the following one seems to be new.

\begin{corollary}{characterisation-prime-Wi87a}
A configuration structure arises as the family of configurations of a
prime event structure of \cite{Wi87a} iff it satisfies
the axioms of rootedness, finiteness, coincidence-freeness,
finite-completeness, irredundancy and closure under .
\hfill 
\end{corollary}
Recall that for these structures the left-closed and secured
configurations are the same.

\begin{corollary}{characterisation-Wi89}
A configuration structure arises as the family of (secured)
configurations of an event structure of \cite{Wi89} iff it satisfies
the axioms of rootedness, finiteness, coincidence-freeness and
closure under \plat{\,\bbigcup^2}.

A configuration structure arises as the family of (secured)
configurations of a stable event structure of \cite{Wi89} iff it
moreover is closed under .
\hfill 
\end{corollary}
In \cite{Wi87a} the characterisations above were claimed, but using
coherence (cf.\ \df{compatibility}.7) instead of closure under
\plat{\bbigcup^2}.  Arend Rensink [personal communication, around
1996] provided the following counterexample against that characterisation.

\begin{example}{Rensink}
Let  be given by  and\vspace{-1.4ex}

Then  satisfies the axioms of rootedness, finiteness,
coincidence-freeness, closure under  and coherence, but it
is not closed under \plat{\bbigcup^2} (cf.\ \df{properties}.7). By
\cor{characterisation-Wi89} it therefore cannot arise as the family of
configurations of an event structure of \cite{Wi89}.
\end{example}
We now propose the property of closure under \plat{\bbigcup^2} as the
replacement for coherence in this theorem. Using
\pr{directed-unions} we can replace closure under \plat{\bbigcup^2} (and
rootedness) by coherence if we have closure under .
This gives the following, apparently novel, characterisation.

\begin{corollary}{characterisation-prime-Wi89}
A configuration structure arises as the family of configurations of a
prime event structure of \cite{Wi89} iff it satisfies the axioms of
finiteness, coincidence-freeness, coherence, irredundancy and closure
under .
\hfill 
\end{corollary}

\subsubsection*{Propositional theories}

We do not have axiomatic characterisations of properties like
connectedness or hyperconnectedness, and therefore we cannot offer a
third column for Table~\ref{correspondence secured} such that for
 a hyperconnected configuration structure satisfying a package of
properties, a suitably axiomatised  theory can be found for which
. 
As best we could work up to reachable equivalence, and be content with
a theory  such that .
In this context we directly inherit the third column of
Table~\ref{correspondence}---however, only for theories that are
\emph{secure}, a property for which we have no axiomatic characterisation.
\begin{definition}{secure-PT}
A configuration structure  is \phrase{secure} if
, and a propositional theory  is \emph{secure}
if  is.
\end{definition}
Note that if  is secure, then so is any pure event structure
 with .

\begin{corollary}{CtoT-secured}
A hyperconnected configuration structure  has any package of
properties from the second column of Table~\ref{correspondence secured} iff
there is a secure propositional theory  whose formulae are of
one of the combinations of the forms found on the corresponding lines
of Table~\ref{correspondence}, such that .
\end{corollary}

\begin{proof}
Given a hyperconnected configuration structure  satisfying a
package of properties from the second column of
Table~\ref{correspondence secured}, \thm{CtoE-secured} yields a pure
and secure event structure with the corresponding properties such that
. By \thm{EtoT}, there is a theory  whose formulae are of
one of the combinations of the forms found on the corresponding lines
of Table~\ref{correspondence}, such that .
As  is secure, so is  and hence .
Using \pr{pure ES commute} we find .

Conversely, given a package of properties from the second column of
Table~\ref{correspondence secured}, let  be a secure theory whose
formulae are of one of the combinations of the forms found on the
corresponding lines of Table~\ref{correspondence}.
Then \thm{TtoC} yields that  has the corresponding package of
properties from  the second column of Table~\ref{correspondence}
(i.e., skipping ``hyperreachable''), so by \thm{CtoE} there is a pure
event structure  with the corresponding properties such that
. As noted above,  is secure, and by
\thm{EtoC-secured}  has the given package of properties. Furthermore,
.
\end{proof}
In case of a package of properties including finite conflict, or
excluding (local) conjunctivity, the requirement that  be secure
may be dropped. This follows from the remarks following \thm{EtoC-secured}.

We do not have an axiomatic characterisation of irredundancy, nor of
the axiom of finiteness, and hence neither of the event structures
from \cite{Wi87a,Wi89}.

\subsubsection*{Reachable configuration structures}

We were unable to find correspondences of the form of
Table~\ref{correspondence secured} using reachable
configurations instead of secured ones. The problem we encountered is
that the set of reachable configurations of a singular event structure
need not be closed under bounded unions.

\begin{example}{no bounded unions}
Let  be given by ,
 and  for 
and \mbox{} for  not of the form .
This event structure is rooted and singular and has finite conflict.
Its reachable configurations include  for all numbers ,
together with all sets containing .
In particular the set of all events is a reachable configuration,
because after  all other events can happen in one step.
However,  is not a reachable configuration.
Therefore,  fails to be closed under bounded intersections.
\end{example}

\subsection{Two finitary comparisons}\label{finitary comparisons}

In this section we characterise the configuration structures that
arise by taking the finite left-closed configurations of the various
classes of event structures.
We also characterise the corresponding propositional
theories, but working up to finitary equivalence only. Thus, given a
finitary configuration structure  satisfying some relevant closure
properties, we seek a proposition theory  of a particular form such that
; we do not seek a theory  with .
Subsequently, we do the same for the finite \emph{reachable} configurations
of the various classes of event structures.

We can put any event structure  into a ``finitary'' form 
by removing all causal relations  with  or  infinite
and then adding all  for  infinite.  Clearly
 has finite causes and finite conflict, and . Thus, by \thm{CtoE}, any finitary configuration
structures arises as  for an event structure  with
finite causes and finite conflict.\linebreak[2]
Next, since every clause of the form  with  infinite
is satisfied by any finite configuration, up to finitary equivalence
any configuration structure has an axiomatisation by formulae of the
form (fin, any).  Thus, at the level of finitary equivalence, we have
a general correspondence between (pure) event structures (with finite
causes and finite conflict), finitary configuration structures and
this class of propositional theories.

For particular correspondences we again consider the relevant
properties of event structures and their correspondences in
configuration structures and prop\-ositional theories.  We consider
pureness, rootedness, singularity, conjunctivity, local conjunctivity
and binary conflict, as finite conflict is already built in.  For
finitary configuration structures,
the distinctions between  and , and between
 and , disappear, and indeed
we are left with closure conditions,
, , ,  and ,
meaning, respectively: closure under finite bounded unions, binary
intersections, bounded binary intersections, finite pairwise
consistent unions and pairwise consistent binary intersections.

\begin{observation}{finite closure conditions}
A finitary configuration structure
\begin{itemise3}
\item[\bf --] is closed under  iff it is closed under ;
\item[\bf --] is closed under  iff it is closed under ;
\item[\bf --] is closed under  iff it is closed under ;
\item[\bf --] is closed under  iff it is closed under ;
\item[\bf --] is closed under  iff it is closed under .
\end{itemise3}
\end{observation}
Note that a configuration structure is closed under  iff
it is closed under  and is either rooted or empty. We say
that a configuration structure  has \phrase{finite binary
conflict} iff for every finite set of events 

Note that a finitary configuration structure  has finite binary
conflict iff  with  its closure under binary
conflict, as introduced in \df{conflict closure}.

\begin{observation}{finitary closure conditions}
If a configuration structure  is closed under ,
, ,  or , then so is .
Furthermore, if  has binary conflict then  has finite
binary conflict.
\end{observation}
For propositional theories used for comparison up to finitary equivalence
we replace ``ddc'' and ``bddc'' by new forms ``ddfc''  and ``bddfc'', meaning
\emph{finite} conjunctions. The interpretation of the resulting forms  
should be clear; as before, 
they are combined by taking meets in the left and right lattices.
We get the correspondences summarised by Table~\ref{correspondence_finite}.
\begin{table}[htb]
\begin{tabular}{@{}|@{~}l@{\,}|@{~}l@{~}|@{~}l@{~}|@{}}
\hline
Event           & Configuration                 & Propositional         \\
structures      & structures                    & theories              \\
\hline\hline
rooted          & rooted                        & (nef, any)            \\
singular        & closed under     & (1, any), (fin., 0)   \\
conjunctive     & closed under            & (fin., )     \\
locally conj.   & closed under      & (fin., ddfc)          \\
binary conflict & fin.\ bin.\ conflict          & (, any)      \\
\hline
\raisebox{0pt}[12pt]{sing}.\ \& bin.\ con.& closed under   & (1, any), (, 0)\\
loc.\ conj.\ \& b.c.& closed under & (, bddfc)      \\
\hline
\end{tabular}
\caption{Corresponding properties for finite parts \label{correspondence_finite}}
\end{table}

We define a package of properties of configuration structures from the 
table analogously to before. We call a set of properties from the second column of
Table~\ref{correspondence_finite} a {package }if
\begin{itemise2}
\itemsep 0pt
\item it contains the property
``closed under '' iff  it contains the properties
``closed under '' and ``having finite binary conflict'',
and
\item it contains the property
``closed under '' iff it contains the properties
``closed under '' and ``having finite binary conflict.''
\vspace{2pt}
\end{itemise2}
We can now formulate the correspondences explicitly as:
\begin{theorem}{finite}~
\begin{enumerate}
\item Let  be a (pure)  event structure
satisfying any collection of properties
from Table~\ref{correspondence_finite}. Then there is a (pure) propositional
theory  whose axioms
have forms which are combinations of the forms corresponding to the event structure 
properties, such that .
\item Let  be a propositional theory whose axioms
have forms which are combinations of forms from a given collection of rows of 
Table~\ref{correspondence_finite}. Then  has the corresponding 
collection of properties of configuration structures.
\item Let  be a finitary configuration structure satisfying a
given package of properties from Table~\ref{correspondence_finite}.
Then there is a pure event structure  with finite causes and
finite conflict such that  and with the
corresponding collection of properties of event structures.
\end{enumerate}
\end{theorem}
\pf
\begin{enumerate}
\item 
Given a (pure) event structure 
satisfying a given collection of properties of the table, 
Theorem~\ref{th-EtoT} yields an axiomatisation of 
 by a (pure) propositional theory  whose axioms have 
the form of a combination of the forms corresponding to
the  properties given by Table~\ref{correspondence}.

Now we can remove any formulae of the form  with  infinite
from the axiomatisation as they are automatically true in finite
interpretations (i.e., the finite subsets of ).  (Alternatively, we
could have obtained these forms by requiring , without limitation
of generality, to be with finite conflict.)  Next, to any formula of
the form  one can associate a formula of the form
 by removing all infinite disjuncts, and the
associated formula is true in a finite interpretation iff the original
one is; the same holds for  and 
formulae. Making these replacements as necessary, one arrives at the
required (pure) propositional theory

\item 
Given any propositional theory  whose axioms have the form of
combinations of forms given in rows of the table, then, by \thm{TtoC},
 satisfies the corresponding properties of
Table~\ref{correspondence} and so, by Observations~\ref{obs-finitary
closure conditions} and~\ref{obs-finite closure conditions},
 satisfies the corresponding properties of
Table~\ref{correspondence_finite}.

\item 
Let  be a finitary configuration structure with a given package
of properties of the table, not including finite binary conflict.
Then,  by \ob{finite closure conditions},  satisfies the
corresponding package of properties of Table~\ref{correspondence},
and so, by \thm{CtoE}, there is a pure  event structure  satisfying 
the corresponding properties for event structures such that 
.
Now  also satisfies these properties, and
.

In the case of a package of properties which does include finite
binary conflict, by \lem{closure} and \cor{conflict closure}, 
has the same package of properties but with binary conflict instead of
finite binary conflict. Thus, by \thm{CtoE}, there is a pure event
structure  with the corresponding properties such that
.  Now  is pure and also satisfies these properties, and
.
\hfill
\end{enumerate}

\subsubsection*{Comparisons via finite reachable parts}

We now turn to comparisons via finite reachable parts. 
A similar obstacle as in Section~\ref{EvsC-secured} presents itself:
an event structure  may have binary conflict even though
 does not have finite binary conflict.
\begin{example}{pentagon}
Let  be the configuration structure with events 
and with configurations:

and

where the counting is done mod 5. Then  is finitary and has (finite)
binary conflict, but its reachable part has not, as  is not reachable.
Furthermore,  can be given by a pure rooted event structure with
finite causes and binary conflict, namely the one with the enablings

again counting mod 5 (and omitting explicit set parentheses),
plus those needed for rootedness and binary conflict.
\end{example}
Since our primary interest is in characterising natural properties
of event structures we find a suitable weakening of this property of
configuration structures, and proceed analogously to Section~\ref{EvsC-secured}.
\begin{definition}{sbc}
A configuration structure  has \phrase{finite reachable binary
conflict} iff .
\end{definition}
In other words, a configuration structure  has
finite reachable binary conflict iff  exactly
when  can be written as    
so that for every  and  with  there 
is a configuration  such that .

We then obtain the correspondences summarised by Table~\ref{correspondence_finite_reachable}.

\begin{table}[htb]
\begin{tabular}{@{}|@{~}l@{\hspace{3pt}}|@{~}l@{\hspace{3pt}}|@{~}l@{\hspace{3pt}}|@{}}
\hline
Event           & Configuration                 & Propositional         \\
structures      & structures                    & theories              \\
\hline\hline
rooted          & rooted                        & (nef, any)            \\
singular        & closed under     & (1, any), (fin., 0)   \\
conjunctive     & closed under        & (fin., )         \\
locally conj.   & closed under       & (fin., ddfc)         \\
binary conflict & fin.\ reach.\ b.c.             & (, any)     \\
\hline
\raisebox{0pt}[12pt]{sing}.\ \& bin.\ con.& closed under   & (1, any), (, 0)\\
loc.\ conj.\ \& b.c.& closed under & (, bddfc)       \\
\hline
\end{tabular}
\caption{Corresponding properties for finite reachable parts}\label{correspondence_finite_reachable}
\end{table}

\begin{lemma}{G3r} Let  be a pure event structure
with the properties given in one of the rows of
Table~\ref{correspondence_finite_reachable}. Then  has the
corresponding property, as given in the table.
\end{lemma}
\begin{proof} The event structure  has the same properties as
 and in addition has finite conflict. Clearly ,
and by \pr{finite reachable ES} we have .
By \lem{finite conflict secure} . Hence, by
\thm{EtoC-secured},  has the corresponding property given in
Table~\ref{correspondence secured}.  In case the row we started with
was not that of binary conflict, by Observations~\ref{obs-finitary
closure conditions} and~\ref{obs-finite closure conditions}
 has the corresponding property of
Table~\ref{correspondence_finite_reachable}.  In case the row we
started with \emph{was} that of binary conflict, by expanding
\df{hyperreachable conflict} we find that . Now observe that 
for any configuration structure .  Applying
Propositions~\ref{pr-finite reachable ES} and~\ref{pr-idempotence}
this yields 
Hence  has finite reachable binary conflict.
\end{proof}
Note that the purity requirement in \lem{G3r} can be weakened to
reachable purity, and is only needed for the
binary conflict row, namely in the application of \thm{EtoC-secured}.
The following example shows that this requirement cannot be omitted.

\begin{example}{impure finite reachable binary conflict}
Let  be the event structure with
 and the enablings  when
, as well as  where the counting
is done mod 3. Then  has binary conflict,  and
. Hence  does not have finite
reachable binary conflict.
\end{example}

We can now establish the correspondences of Table~\ref{correspondence_finite_reachable}. 
We define packages of properties of configuration structures from the
table just as we did for finitary equivalence, substituting finite reachable
binary conflict for finite binary conflict; and we keep the same form
lattices and their interpretation as just used for finitary
equivalence.
\begin{theorem}{finite_reachable}~
\begin{enumerate}
\item Let  be a (pure) event structure
satisfying any collection of properties
from Table~\ref{correspondence_finite_reachable}. Then there is a (pure) propositional
theory  whose axioms
have forms which are combinations of the forms corresponding to the event structure 
properties such that .
\item Let  be a propositional theory whose axioms
have forms which are combinations of forms from a given collection of rows of 
Table~\ref{correspondence_finite_reachable}. Then 
has the corresponding collection of properties of configuration structures.
\item Let  be a finitary connected configuration structure
satisfying a given package of properties from
Table~\ref{correspondence_finite_reachable}. Then there is a pure
event structure  with finite causes and finite conflict such that
 and with the corresponding collection of properties
of event structures.
\end{enumerate}
\end{theorem}
\pf
\begin{enumerate}
\item This is immediate from part 1 of Theorem~\ref{th-finite}.
\item  Let  be such a theory. It follows from Theorem~\ref{th-finite}
that there is a pure event structure  satisfying the corresponding properties 
from the table such that  . The
conclusion then follows from \lem{G3r} and \pr{pure ES commute}.
\item This follows just as in the proof of \thm{finite}:
Let  be a finitary connected configuration structure with a given package
of properties of the table, not including finite reachable binary conflict.
Then, by \ob{finite closure conditions},  satisfies the
corresponding package of properties of Table~\ref{correspondence},
and so, by \thm{CtoE}, there is a pure  event structure  satisfying 
the corresponding properties for event structures such that 
.
Now  is pure and also satisfies these properties, so \pr{pure ES
commute} yields
.

In the case of a package of properties which does include finite reachable
binary conflict, by \lem{closure} and \cor{conflict closure}, 
has the same package of properties but with binary conflict instead of
finite reachable binary conflict. Thus, by \thm{CtoE}, there is a pure event
structure  with the corresponding properties such that
.  Now  also satisfies these properties, and
.
\hfill
\end{enumerate}



\noindent
We now apply the results of this section to characterise the finitary
configuration structures associated to the various event structures of
{\sc Winskel} \cite{Wi87a,Wi89}.

\begin{corollary}{characterisation-Winskel finitary}
A configuration structure arises as the family of finite left-closed
configurations of an event structure of \cite{Wi87a} iff it is
finitary, rooted and closed under . It arises as the
family of finite left-closed configurations of a stable event
structure of \cite{Wi87a} iff it moreover is closed under .

A configuration structure arises as the family of finite left-closed
configurations of an event structure of \cite{Wi89} iff it is
finitary, rooted and closed under .  It arises as the
family of finite left-closed configurations of a stable event
structure of \cite{Wi89} iff it moreover is closed under .
\hfill 
\end{corollary}
The four classes of configuration structures mentioned in the above
corollary arise as the finite models of propositional
theories whose axioms have the forms
\begin{center}
\begin{tabular}{c}
(1, any), (nef, 0)\\
(1, ddfc), (nef, 0)\\
(1, any), (2, 0)\\
(1, bddfc), (2, 0),
\end{tabular}
\end{center}
respectively.

When dealing with finite \emph{reachable} configurations, the same
characterisations as in \cor{characterisation-Winskel finitary} are
obtained, but now the resulting configuration structures are
additionally connected.

\begin{proposition}{coincidence-freeness-connectedness}
Let  be a finitary configuration structure closed under
.  Then  is connected iff it is coincidence-free.
\end{proposition}

\begin{proof}
Similar to the proof of \pr{coincidence-freeness-hyperconnectedness}.
\pagebreak[3]
\end{proof}

\begin{corollary}{characterisation-Winskel finitary reachable}
A configuration structure arises as the family of finite reachable
configurations of an event structure of \cite{Wi87a} iff it is
finitary, rooted, coincidence-free and closed under .  It
arises as the family of finite reachable configurations of a stable
event structure of \cite{Wi87a} iff it moreover is closed under
.

A configuration structure arises as the family of finite reachable
configurations of an event structure of \cite{Wi89} iff it is
finitary, rooted, coincidence-free and closed under .
It arises as the family of finite reachable configurations of a
stable event structure of \cite{Wi89} iff it moreover is closed under
.
\hfill 
\end{corollary}
The first class of configuration structures in this corollary was the
class of configuration structures originally considered in \cite{GG90}.

We have no characterisation of the finitary configuration structures
associated to the event structures from \cite{NPW81}; in particular,
the property of -irredundancy appears hard to express in terms of
finite configurations. As for the prime event structures from
\cite{Wi87a,Wi89}, recall that their finite left-closed configurations
are the same as their finite reachable or finite secured configurations.

\begin{corollary}{characterisation-prime-Winskel finitary}
A configuration structure arises as the family of finite
configurations of a prime event structure of \cite{Wi87a} iff it is
finitary, rooted, coincidence-free, irredundant and closed under
 and .

A configuration structure arises as the family of finite
configurations of a prime event structure of \cite{Wi89} iff it is
finitary, coherent, coincidence-free, irredundant and closed under
.
\hfill 
\end{corollary}
The two classes of configuration structures mentioned above
arise as the finite reachable models of propositional
theories whose axioms have the forms
\begin{center}
\begin{tabular}{c}
(1, 1), (nef, 0)\\
(1, 1), (2, 0)
\end{tabular}
\end{center}
respectively. We do not have axiomatic characterisations of
connectedness or coincidence-freedom; that lack is circumvented in the
above characterisations by talking about \emph{reachable} models.


\subsection{Tying-in Petri nets}\label{Tie-nets}

The third columns of Tables~\ref{correspondence},~\ref{correspondence_finite},~\ref{correspondence_finite_reachable} also provide characterisations
of classes of Petri nets corresponding to various combinations of
properties of event structures or configuration structures.  The pure
1-occurrence nets correspond up to finite configuration equivalence to
event and configuration structures that are rooted and with finite
conflict. We do not have a structural characterisation of the subclass
of those pure 1-occurrence nets corresponding to locally conjunctive
event structures. However, for each combination of the properties
singular, conjunctive, and binary conflict, the forms  that
characterise the associated propositional theory also provide a
structural characterisation of the associated subclass of pure
1-occurrence nets.  Here  restricts the cardinality of the set of
posttransitions of any given place, and  restricts the cardinality of
its set of pretransitions; we say that the place \emph{has} the form .
For example, rooted singular pure event structures correspond to the
pure 1-occurrence nets each of whose places have either no pretransitions
or exactly one posttransition. The proof of the correspondences goes
via the characterisations of the associated propositional theories.

\begin{theorem}{Petri-correspondence}
Let  be a (pure) rooted propositional theory in conjunctive
normal form, whose clauses are combinations
of the forms found in lines 2, 3 and 5 of Table~\ref{correspondence}.
Then  is a (pure) 1-occurrence net whose places have the
corresponding combinations of forms, as well as (nef, any).

Similarly, if  is a (pure) 1-occurrence net whose places have
combinations of the forms found in lines 2, 3 and 5 of
Table~\ref{correspondence}, then , as defined
\hyperref[NtoT]{at the end} \hyperref[NtoT]{of Section~\ref*{PN}}, is
a (pure) rooted propositional theory axiomatised by clauses obeying
these forms, as well as (nef, any).
\end{theorem}

\begin{proof}
The first statement follows immediately from the construction in \df{TtoN}.
For the second statement, recall that  consists of the formulae

for  and . When converting such
formulae to conjunctive normal form, one obtains clauses 
with  for some place . 
As remarked at the end of Section~\ref*{PN},
one can omit any clauses for , or more generally for
which ,
as then .
Hence all clauses obey the restriction
\mbox{(nef, any)} and  is rooted. By construction,  is
pure when  is.  If  has the form (1, any) or \mbox{(, any)}, then so do the associated clauses.  Furthermore, if
 has no pretransitions, then the associated clauses
have the form , and if  has one pretransition
, then the associated clauses have the form 
for  with 
, 
and  for  with
. Thus, if  has the form
(any, 0) or \mbox{(any, )}, then so do the associated clauses.
\end{proof}
This theorem also holds when using the third columns of
Tables~\ref{correspondence_finite} or~\ref{correspondence_finite_reachable}
(which are the same) instead of the one of Table~\ref{correspondence}.
For these columns are obtained by additionally imposing the condition
(finite, any), a condition that is implied by (nef, any). Furthermore,
the theorem remains true if any place  with  incoming arcs and 
initial tokens is deemed to additionally have the form
``(, ) or (, 0)''.
Namely if  and  then the
transitions in  cannot all happen, so we obtain the clause . Among such clauses one only needs to retain the
ones with  minimal, that is, with .
Finally, places without posttransitions may be ignored.\linebreak
Thus, for example, pure 1-occurrence nets whose places either have
 posttransition, or one incoming arc and no initial
tokens, or no incoming arcs and  initial token correspond to
pure singular event structures with binary conflict.

This theorem, together with \thm{PtoNtoC} and \pr{NtoT}, yields a
bijection up to finitary equivalence between the stated subclasses of
pure rooted propositional theories and the corresponding subclasses of
pure 1-occurrence nets. As the nets are pure, these bijections
also hold up to finitary reachable equivalence.



\section{Related Work}\label{related work}

The notion of a configuration structure as a model of concurrency in
its own right stems from {\sc Winskel} \cite{Wi82}; our configuration
structures are obtained by dropping the requirements imposed in \cite{Wi82}:
coherence, stability, coincidence-freeness and the axiom of finiteness.
The term \emph{configuration structures} stems from \cite{GG90}; their
configuration structures obeyed the requirements of finitariness, rootedness,
coincidence-freeness and closure under , that together
ensured that these structures were exactly the families
of finite configurations of Winskel's event structures \cite{Wi87a}.
Two further partial generalisations of this model were previously proposed by
{\sc Pinna \& Poign\'e} \cite{PP95} and {\sc Hoogers, Kleijn \& Thiagarajan}
\cite{HKT96}. The \phrase{event automata} of \cite{PP95} are rooted
finitary configuration structures together with a transition relation
between the configurations; each transition extends a configuration
with exactly one event. The \phrase{local event structures} of
\cite{HKT96} are rooted, finitary, connected configuration structures
together with a step transition relation  between the
configurations that satisfies
\begin{itemise}
\item ,
\item  implies , and
\item  and  implies .
\end{itemise}
In \cite{HKT96}  is denoted , so that
their notation  implies  and
translates to .

Our configuration structures are, up to isomorphism, the
\emph{extensional} \phrase{Chu spaces} of {\sc Gupta \& Pratt}
\cite{GP93a,Gup94,Pr94a}. It was in their work that the idea arose of
using the full generality of such structures in modelling
concurrency. Also the propositional representation of configuration
structures stems from \cite{GP93a,Pr94a}.  It should be noted however
that the computational interpretation in \cite{GP93a,Gup94,Pr94a}
differs somewhat from that in \cite{Wi87a,GG90,PP95,HKT96} and the
current work.  In particular, in \cite{GP93a,Gup94,Pr94a} unreachable
configurations may be semantically relevant, as witnessed by the
notions of \phrase{causality} and \phrase{internal choice} in
\cite{GP93a,Pr94a} and that of \phrase{history preserving
bisimulation} in \cite{Gup94}.

{\sc Gunawardena} proposes \phrase{causal automata} in \cite{Gun92b} and
\phrase{geometric automata} in \cite{Gun91}. The first are given by a
set of events with, for each event , a boolean expression 
over the set of events. Each event occurrence in  is
interpreted as the proposition that it happened, and  is enabled
when  evaluates to true.  In geometric automata, a more
complicated infinitary logic is used, and the boolean expression is
replaced by two positive logical expressions, one of which must
evaluate to true, and the other to false, in order for the associated
event to be enabled.  Both models can be interpreted in a natural way
in terms of event automata; plain configuration structures are not
sufficient here.

Our event structures are directly inspired by, and generalise, the
ones of {\sc Winskel} \cite{NPW81,Wi87a,Wi89}. Many other variants
of these event structures have been proposed in the literature.

A \phrase{bundle event structure}, as studied in {\sc Langerak}
\cite{Lk92}, is given as a tuple  with  a set of
events,  an irreflexive, symmetric conflict relation,
, the \phrase{bundle
relation}, and  a \emph{labelling function},
labelling events with actions from a given alphabet . When , the events in  should be pairwise in conflict; in this
case  can happen only if one of the events in  occurred
earlier. Ignoring the labelling function, a bundle event structure can
in our framework best be understood as a propositional theory, namely
one whose formulae have the forms \mbox{(2, 0)} and (1, dds). Here ``dds''
stands for \phrase{disjoint disjunction of singletons}; in the right
form lattice of Figure~\ref{form-lattices} it can be positioned right
below ``bddc'', or right below ``bddfc'' of Section~\ref{finitary comparisons}.
The configurations used in
\cite{Lk92} are in our terminology finite reachable configurations.
Using the translations of Section~\ref{ComparingModels}, preserving
finitary reachable equivalence, the bundle event structures map to a
subclass of rooted, singular, locally conjunctive event structures
with binary conflict, and hence to a subclass of stable event
structures as defined in \cite{Wi89} that contains the class of prime
event structures of \cite{Wi89}.\vspace{0pt plus 1pt}

Langerak's notion of an \phrase{extended bundle event structure} on the
other hand does not correspond to an event structure as in
\cite{Wi87a,Wi89}.  Here the symmetric binary conflict relation 
is replaced by an asymmetric counterpart \index{asymmetric
conflict}, a relation that was considered independently in
\cite{PP95}, writing  for .
When , the event  can happen either initially or after ;
however, as soon as  happens,  is blocked. When both  and 
happen,  causally precedes .  Asymmetric conflict 
 can be translated into our framework as , where it is important that  is the \emph{only} set of
events enabling . The absence of both 
and  translates to , and the
conjunction of  and  is simply 
and translates to the absence of any  with .
Under this translation, the configurations of extended bundle event
structures defined in \cite{Lk92} are exactly our finite reachable
configurations of \df{reachable ES}.  Thus, the class of extended
bundle event structures can be regarded as a subclass of our rooted,
locally conjunctive event structures with binary conflict.  However,
they are not pure, and cannot be faithfully represented by
configuration structures as studied in this paper.\vspace{0pt plus 1pt}
The relationship between event structures with asymmetric conflict,
Petri nets, and domains, is studied in \cite{BCM01}.

A \phrase{dual event structure}, as studied in {\sc Katoen} \cite{Ka96},
is like an extended bundle event structure, but without the
requirement that when  the events in  should be
pairwise in conflict. This amounts to generalising the formulae of the
form (1, dds) to \mbox{(1, any)}. They correspond to a subclass of our
rooted event structures with binary conflict.
The same can be said for the \phrase{extended dual event structures} of
\cite{Ka96}. Here the new feature is the irreflexive and symmetric
\phrase{interleaving relation} , modelling mutual 
exclusion of events, i.e., disallowing them to overlap in time. As for
the event structure M in Section~\ref{ES computational},
 can in our framework be modelled as
.\vspace{0pt plus 1pt}

As remarked in the introduction, behaviour preserving translations
from safe Petri nets to a class of event structures, and from there to
configuration structures, are defined in \cite{NPW81}.  In
Section~\ref{prime-bc} we saw that the event structures of
\cite{NPW81} can be seen as a subclass of our event structures, in the
sense that there are translations back and forth that respect the
identify of events and the sets of associated configurations. The
translation in \cite{NPW81} from safe nets to event structures
proceeds in two steps: an \phrase{unfolding} turns every safe net into
an \phrase{occurrence net}---a particular kind of pure safe 1-occurrence
net---and a mapping  takes occurrence nets to event structures.
The transitions of an occurrence net  become the events of the
event structure , and the finite configurations of , as defined in
this paper, equal the finite configurations of  as defined in
\cite{NPW81}. This follows directly from the definitions. Hence the
translation  preserves finitary configuration equivalence.  It is not hard
to check that the unfolding of a safe 1-occurrence net preserves finitary
configuration equivalence as well. Thus, restricted to safe
1-occurrence nets, the translations of \cite{NPW81} are entirely in agreement with ours.

This agreement extends to pure safe nets that are not
1-occurrence nets. However, this cannot be stated in the terminology
of this paper, for the unfolding may make multiple copies of a single
transition, namely one for every possible way in which it can be
fired.  Since the identify of events is thereby not preserved, this
unfolding does not respect the equivalences of this paper.

Define a \phrase{1-reachable-occurrence net} to be a net in which every
\emph{reachable} configuration is a set. This notion is a slight
generalisation of a 1-occurrence net. When working up to reachable
equivalence, all our work generalises without change from 1-occurrence
nets to 1-reachable-occurrence nets. Similarly, define a
1-reachable-occurrence net  to be \phrase{semantically pure} if
there exists a pure net  with the same places and
transitions, and possibly less arcs and less initial tokens, that has
the same reachable configurations and the same step transition
relation between those configurations.  When working up to
reachable equivalence, also preserving the transitions between
reachable configurations, our connections between pure
1-reachable-occurrence nets and pure event structures evidently
generalise to semantically pure 1-reachable-occurrence nets---just as
they did to reachably pure event structures.

{\sc Boudol} \cite{Bo90} provides translations between a class of
1-reachable-occurrence nets, the \phrase{flow nets}, and a class of {\em
flow event structures} that have expressive power strictly between
the bundle event structures of \cite{Lk92} and the stable event
structures of \cite{Wi89}.  His correspondence extends the
correspondence due to \cite{NPW81} between occurrence nets and prime
event structures with binary conflict.  Flow nets are defined to have
the property that transitions that can occur in the same firing
sequence do not share a preplace. This implies that the reachable
configurations of a flow net , as well as the transition relation
 between them, are unaffected by omitting the arcs from a
transition  to a place  for which there also is an arc from 
to . Any flow net can thereby be transformed, in a behaviour
preserving way, into a pure 1-reachable-occurrence net. Hence flow
nets are semantically pure.

As Boudol's translations preserve the notions of event (= transition)
and finite reachable configuration, they are consistent with our
approach. Our translations can thus be regarded as an extension of the
work of \cite{Bo90} to a more general class of Petri nets and event
structures.

Another translation between Petri nets and a model of event
structures has been provided in {\sc Hoogers, Kleijn \& Thiagarajan}
\cite{HKT96}, albeit only for systems without autoconcurrency.  As
mentioned, their event structures are families of configurations with
a step transition relation between them. The translations of
\cite{HKT96} are quite different from ours: even on 1-occurrence nets
an individual transition may correspond to multiple events in the
associated event structure.  We conjecture that the two approaches are
equivalent under a suitable notion of history preserving bisimulation.

\subsubsection*{Future research}

As we have seen, both event structures and Petri nets
have naturally associated transition relations. In the pure case these
transition relations can be derived from their associated sets of
configurations, but this fails more generally. A natural line of future work
is therefore to go beyond the pure case, looking for a suitable notion of
configuration structure equipped with a transition relation and, perhaps, a
suitable notion of propositional theory.

We would also like to connect our models with appropriate versions of
higher dimensional automata \cite{Pr91a}. An embedding up to finitary
reachable equivalence of rooted configuration structures as well as
Petri nets into a form of higher dimensional automata called \phrase{cubical
sets} is proposed in \cite{vG06}. Another form of higher dimensional
automata called \phrase{labelled step transition systems} is
considered in \cite{vG05}.

After the initial work of~\cite{NPW81} it was natural to ask whether their
unfolding could be seen as a universal construction. This led to a
development of categories of event structures, nets and related models, and, in turn, to a
general process algebra whose constructions were natural categorically: see
\cite{Wi87a,WN95,SNW96}. In our case it would be natural to look for
categories of configuration structures and the other models of this paper, so that, for example,
the connections developed in Section 1 became functorial.  The recent work
of~\cite{Win08, WH08} on adding symmetry to structures may prove helpful here.
Proposals for a category of configuration structures can be found in
\cite{Pr94a} and \cite{BMMS98}.

In a different direction, the equivalences considered in this paper are
quite fine and it would be interesting to look at coarser ones, say
along the lines of history preserving bisimulation.  In that connection,
and also the categorical one, it may be useful to consider configuration
structures, and other models, equipped with event labellings.

\paragraph{Acknowledgement} We thank the referees for their helpful comments.



\begin{thebibliography}{10}
\small

\bibitem{BCM01}
{\sc P. Baldan, A. Corradini \& U. Montanari} (2001):
\newblock {\em Contextual Petri nets, asymmetric event structures and processes.}
\newblock Information and Computation 171(1), pp. 1--49.

\bibitem{BD87}
{\sc E.~Best \& R.~Devillers} (1987):
\newblock {\em Sequential and Concurrent Behaviour in {Petri} Net Theory.}
\newblock {\sl Theoretical Computer Science} 55, pp. 87--136.

\bibitem{BDKP91}
{\sc E.~Best, R.~Devillers, A.~Kiehn \& L.~Pomello} (1991):
\newblock {\em Concurrent bisimulations in {P}etri nets.}
\newblock {\sl Acta Informatica} 28, pp. 231--264.

\bibitem{Bo90}
{\sc G.~Boudol} (1990):
\newblock {\em Flow event structures and flow nets.}
\newblock In I.~Guessarian, editor: {\sl Semantics of Systems of Concurrent Processes, Proceedings LITP Spring School on Theoretical Computer Science, {\rm La Roche Posay, France}}, {\sl \rm LNCS} 469, Springer, pp. 62--95.

\bibitem{BMMS98}
{\sc R. Bruni, J. Meseguer, U. Montanari \& V. Sassone} (1998):
\newblock {\em A comparison of Petri net semantics under the
collective token philosophy.}
\newblock In J. Hsiang \& A. Ohori, editors: Proceedings
4 Asian Computing Science Conference {\sl Advances in
Computing Science}, ASIAN'98, Manila,  The Philippines,
LNCS 1538, Springer, pp. 225--244.

\bibitem{En91}
{\sc J.~Engelfriet} (1991):
\newblock {\em Branching processes of petri nets.}
\newblock {\sl Acta Informatica} 28(6), pp. 575--591.

\bibitem{Ga81}
{\sc D.M. Gabbay} (1981):
\newblock {\em Semantic Investigations in {H}eyting's Intuitionistic Logic}, {\sl Synthese Library} 148.
\newblock D. Reidel.

\bibitem{vG95c}
{\sc R.J.~van Glabbeek} (1995):
\newblock {\em History preserving process graphs.}
\newblock Draft available at
\burl{http://theory.stanford.edu/~rvg/abstracts.html#hppg}.

\bibitem{vG05}
{\sc R.J.~van Glabbeek} (2005):
\newblock {\em The Individual and Collective Token Interpretations of Petri Nets.}
\newblock In M. Abadi \& L. de Alfaro, editors: Proceedings 16 International Conference on {\sl Concurrency Theory}, CONCUR'05, San Francisco, USA, LNCS 3653, Springer, pp. 323-337.

\bibitem{vG06}
{\sc R.J.~van Glabbeek} (2006):
\newblock {\em On the Expressiveness of Higher Dimensional Automata.}
\newblock Theoretical Computer Science 368(1-2), pp. 169-194.

\bibitem{GG90}
{\sc R.J.~van Glabbeek \& U.~Goltz} (1990):
\newblock {\em Refinement of actions in causality based models.}
\newblock In J.W. de~Bakker, W.P.~de Roever \& G.~Rozenberg, editors:
{Proceedings REX Workshop on \sl Stepwise Refinement of Distributed
Systems: Models, Formalism, Correctness, {\rm Mook, The Netherlands 1989}},
{\sl \rm LNCS} 430, Springer, pp. 267--300.

\bibitem{GP95}
{\sc R.J.~van Glabbeek \& G.D. Plotkin} (1995):
\newblock {\em Configuration structures (extended abstract).}
\newblock In D.~Kozen, editor: {\sl {\rm Proceedings  Annual IEEE Symposium on} Logic in Computer Science, {\rm LICS'95, San Diego, USA}}, IEEE Computer Society Press, pp. 199--209.

\bibitem{GP04}
{\sc R.J.~van Glabbeek \& G.D.~Plotkin} (2004):
\newblock {\em Event structures for resolvable conflict.}
\newblock In: V.~Koubek \& J.~Kratochvil, editors, {\sl {\rm Proceedings
   International Symposium on} Mathematical Foundations of Computer
  Science, {\rm MFCS'04, Prague, Czech Republic}}, LNCS 3153,
  Springer, pp. 550-561.

\bibitem{GR83}
{\sc U.~Goltz \& W.~Reisig} (1983):
\newblock {\em The non-sequential behaviour of {Petri} nets.}
\newblock {\sl Information and Computation} 57, pp. 125--147.

\bibitem{Gun91}
{\sc J.~Gunawardena} (1991):
\newblock {\em Geometric Logic, Causality and Event Structures}.
\newblock In J.C.M. Baeten \& J.F. Groote, editors: Proceedings 2 International Conference on {\sl Concurrency Theory},
  CONCUR'91, Amsterdam, The Netherlands, LNCS 527, Springer, pp. 266-280.

\bibitem{Gun92b}
{\sc J.~Gunawardena} (1992):
\newblock {\em Causal automata.}
\newblock {\sl Theoretical Computer Science} 101(2), pp. 265--288.

\bibitem{Gup94}
{\sc V.~Gupta} (1994):
\newblock {\em Chu Spaces: A Model of Concurrency}.
\newblock PhD thesis, Stanford University.
\newblock Available at \url{http://boole.stanford.edu/pub/gupthes.ps.gz}.

\bibitem{GP93a}
{\sc V.~Gupta \& V.R. Pratt} (1993):
\newblock {\em Gates accept concurrent behavior.}
\newblock In {Proceedings 34 Annual Symposium on \sl
Foundations of Computer Science\rm, FOCS'93, Palo Alto, USA}, IEEE
Computer Society Press, pp. 62--71.

\bibitem{WH08}
{\sc J. Hayman \& G.~Winskel} (2008):
\newblock {\em The unfolding of general Petri nets.}
\newblock In: R. Hariharan, M. Mukund \& V. Vinay, editors:
 Proceedings IARCS Annual Conference on
 {\sl Foundations of Software Technology and Theoretical Computer
Science}, Bangalore, India 2008.
\newblock Available at \burl{http://drops.dagstuhl.de/opus/volltexte/2008/1755/}.

\bibitem{HKT96}
{\sc P.W. Hoogers, H.C.M. Kleijn \& P.S. Thiagarajan} (1996):
\newblock {\em An event structure semantics for general {Petri} nets.}
\newblock {\sl Theoretical Computer Science} 153, pp. 129--170.

\bibitem{Ka96}
{\sc J.-P. Katoen} (1996):
{\it Quantitative and Qualitative Extensions of Event Structures},
\newblock PhD thesis, Department of Computer Science, University of Twente.

\bibitem{Lk92}
{\sc R.~Langerak} (1992):
\newblock {\em Transformations and Semantics for LOTOS}.
\newblock PhD thesis, Department of Computer Science, University of Twente.

\bibitem{LW91}
{\sc K.G. Larsen \& G.~Winskel} (1991):
\newblock {\em Using information systems to solve recursive domain equations.}
\newblock {\sl Information and Computation} 91(2), pp. 232--258.

\bibitem{MMS92}
{\sc J.~Meseguer, U.~Montanari \& V.~Sassone} (1992):
\newblock {\em On the semantics of {Petri} nets.}
\newblock In W.R. Cleaveland, editor: {Proceedings Third International
Conference on {\sl Concurrency Theory}, CONCUR'92, Stony Brook, NY,
USA}, LNCS 630, Springer, pp. 286--301.

\bibitem{NPW81}
{\sc M.~Nielsen, G.D. Plotkin \& G.~Winskel} (1981):
\newblock {\em Petri nets, event structures and domains, part {I}.}
\newblock {\sl Theoretical Computer Science} 13(1), pp. 85--108.

\bibitem{PP95}
{\sc G.M. Pinna \& A.~Poign\'e} (1995):
\newblock {\em On the nature of events: another perspective in concurrency.}
\newblock {\sl Theoretical Computer Science} 138(2), pp. 425--454.

\bibitem{Plo78}
{\sc G.D. Plotkin} (1978):
\newblock {\em  as a universal domain.}
\newblock {\sl  J.\ Comput.\ Syst.\ Sci.} 17(2), pp. 209--236.

\bibitem{Pr91a}
{\sc V.R. Pratt} (1991):
\newblock {\em Modeling concurrency with geometry.}
\newblock In {Conference Record of the 18 Annual ACM
Symposium on {\sl Principles of Programming Languages}, POPL'91,
Orlando, USA}, pp. 311--322.

\bibitem{Pr94a}
{\sc V.R. Pratt} (1994):
\newblock {\em Chu spaces: complementarity and uncertainty in rational mechanics.}
\newblock Course Notes, TEMPUS Summer School, Budapest.
\newblock Available at \url{http://boole.stanford.edu/pub/bud.pdf}.

\bibitem{Rei85}
{\sc W. Reisig} (1985):
\newblock {\em Petri Nets: An Introduction.}
\newblock Springer.

\bibitem{SNW96}
{\sc V. Sassone, M. Nielsen \& G. Winskel} (1996):
\newblock {\em Models for concurrency: Towards a classification.}
\newblock {\sl Theoretical Computer Science} 170, pp. 297-348.

\bibitem{Sc74}
{\sc D.S. Scott} (1974):
\newblock {\em Completeness and axiomatizability in many-valued logic.}
\newblock In L.~Henkin et~al., editors: {Proceedings \sl Tarski Symposium}, AMS, pp. 411--435.

\bibitem{Wi82}
{\sc G.~Winskel} (1982):
\newblock {\em Event structure semantics for CCS and related languages.}
\newblock In M. Nielsen and E.M. Schmidt, editors: Proceedings
 Colloquium on {\sl Automata, Languages and Programming},
ICALP'82, Aarhus, Denmark, 1982, LNCS 140, Springer, pp. 561-576.

\bibitem{Wi87a}
{\sc G.~Winskel} (1987):
\newblock {\em Event structures.}
\newblock In W.~Brauer, W.~Reisig \& G.~Rozenberg, editors: Proceedings of an Advanced
Course on {\sl Petri Nets: Applications and Relationships to Other
Models of Concurrency}, {\sl Advances in Petri Nets} 1986, Part II, {\rm Bad
Honnef, September 1986}, {\sl \rm LNCS} 255, Springer, pp. 325--392.

\bibitem{Wi89}
{\sc G.~Winskel} (1989):
\newblock {\em An introduction to event structures.}
\newblock In J.W. de~Bakker, W.P.~de Roever \& G.~Rozenberg, editors:
{Proceedings REX School/Workshop on \sl Linear Time, Branching Time and Partial Order
in Logics and Models for Concurrency, {\rm Noordwijkerhout, The
Netherlands 1988}}, {\sl \rm LNCS} 354, Springer, pp. 364--397.

\bibitem{Win08}
{\sc G.~Winskel} (2008):
\newblock {\em Events, Causality and Symmetry.}
\newblock In: E. Gelenbe, S. Abramsky and V. Sassone, editors:
 Proceedings BCS International Academic Conference {\sl Visions in Computer
Science}, London, UK 2008. {\sl Electronic Workshops in Computing}.
\newblock Available at \burl{http://www.bcs.org/server.php?show=ConWebDoc.22872}.

\bibitem{WN95}
{\sc G. Winskel \& M. Nielsen} (1995):
\newblock {\em Models for Concurrency.}
\newblock In: {\sl Handbook of Logic in Computer Science}, volume 4,
Oxford University Press, pp. 1-148.

\end{thebibliography}

\printindex

\end{document}
