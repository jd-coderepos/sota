\documentclass{paper}


\usepackage{amssymb,amsmath,amsthm,amsfonts}
\usepackage [usenames] {xcolor}
\usepackage{enumerate}
\usepackage{cite}
\usepackage{plaatjes}
\usepackage[linenum]{colormath}
\usepackage{fullpage}


\newcommand {\N} {\mathbb {N}}
\newcommand {\eps} {\varepsilon}
\newcommand {\R} {\mathbb {R}}
\newcommand {\Q} {\mathbb {Q}}
\newcommand {\eqdef}{:=}
\newcommand {\Z} {\mathbb {Z}}
\newcommand {\script} [1] {\ensuremath {\mathcal {#1}}}
\newcommand {\etal} {\textit {et al.}}
\setlength{\fboxsep}{.5pt}
\renewcommand\thefootnote{\tiny\protect\framebox{\arabic{footnote}}}

\newcommand {\D} {\script {D}}
\DeclareMathOperator{\ch}{ch}
\newcommand {\onion} {\raisebox{-0.25ex}{\includegraphics{figures/onion}}}

\newtheorem{theorem}{Theorem}[section]
\newtheorem{lemma}[theorem]{Lemma}
\newtheorem{cor}[theorem]{Corollary}
\newtheorem{prop}[theorem]{Proposition}
\newtheorem{obs}[theorem]{Observation}



\title{Unions of Onions: 
   Preprocessing Imprecise Points for Fast Onion
   Decomposition\thanks{A preliminary version appeared as 
   M.~L\"offler and W.~Mulzer. \emph{Unions of Onions: Preprocessing 
   Imprecise Points for Fast Onion Layer Decomposition}.
   Proc. 13th WADS, pp. 487--498, 2013.}
}

\author
{
  Maarten L\"offler\thanks{
    Dept. of Information and Computing Sciences, 
      Universiteit Utrecht, the Netherlands,
    \texttt{m.loffler@uu.nl}}
   \and Wolfgang Mulzer\thanks{
    Institut f\"ur Informatik, Freie Universit\"at Berlin, Germany, 
    \texttt{mulzer@inf.fu-berlin.de}
    }
}

\begin{document}
\maketitle

\begin{abstract}
   Let  be a set of  pairwise disjoint unit disks in the plane.
   We describe how to build a data structure for  so that
   for any point set  containing
   exactly one point from each disk, we can quickly find the
   onion decomposition (convex layers) of .

   Our data structure can be built in  time
   and has linear size. Given , we can find its 
   onion decomposition in  time, where  is the number of layers.
   We also provide a matching lower bound.
   
   Our solution is based on a recursive space decomposition,
   combined with a fast algorithm to compute the union of two disjoint onion 
   decompositions.
\end{abstract}


\section {Introduction}

Let  be a planar -point set. Take the convex hull of  and remove
it; repeat until  becomes empty.
This process is called \emph{onion peeling}, and the resulting
decomposition of  into convex polygons is the
\emph{onion decomposition}, or \emph {onion} for short, of .
It can be computed in  time~\cite {c-clps-85}.
Onions provide a natural, more robust, generalization of the convex hull,
and they have applications in pattern recognition, statistics, and 
planar halfspace range searching~\cite{cgl-tpogd-85,h-rsar-72,sf-clntrpdps-99}.

Recently, a new paradigm has emerged for modeling data
imprecision.
Suppose we need to compute some interesting property of
a planar point set. Suppose further that we have some advance
knowledge about the possible locations of the points, e.g.,
from an imprecise sensor measurement. We would like to preprocess
this information, so that once the precise inputs are available,
we can obtain our structure faster.
We will study  the complexity of computing onions
in this framework.

\subsection{Related Work}

The notion of onion decompositions first appears in the 
computational statistics literature~\cite{h-rsar-72}, and several rather 
brute-force algorithms to compute it have been suggested 
(see \cite{e-chp-82} and the references therein).
In the computational geometry community, Overmars and van 
Leeuwen~\cite{ol-mcp-81} presented the first near-linear time algorithm,  
requiring  time.
Chazelle~\cite{c-clps-85} improved this to an optimal  time 
algorithm.
Nielsen~\cite{n-ospcml-96} gave an output-sensitive algorithm to compute 
only the outermost  layers in  time, where  is the 
number of vertices participating on the outermost  layers.
In , Chan~\cite{c-addsf3cha2nnq-10} described an  
expected time algorithm.

The framework for preprocessing regions that represent points was first 
introduced by Held and Mitchell~\cite {hm-tipps-08}, who show how to store 
a set of disjoint unit disks in a data structure such that any point set 
containing one point from each disk can be triangulated in linear time.
This result was later extended to arbitrary disjoint regions in the plane 
by van Kreveld~\etal~\cite{klm-pipast-10}.
L\"offler and Snoeyink first showed that the Delaunay triangulation (or its 
dual, the Voronoi diagram) can also be computed in linear time after 
preprocessing a set of disjoint unit disks~\cite{ls-dtoipiltap-10}.
This result was later extended by Buchin~\etal~\cite{blmm-pipfdtsae-11},
and Devillers gives a practical alternative~\cite{d-dtoippaagafqt-11}.
Ezra and Mulzer~\cite{em-choipitap-13} show how to preprocess a set of 
lines in the plane such that the convex hull of a set of points with 
one point on each line can be computed faster than  time.

These results also relate to the \emph{update complexity} model. 
In this paradigm, the input values or points come with some uncertainty,
but it is assumed that during the execution of the algorithm, the values or 
locations can be obtained exactly, or with increased precision, at a 
certain cost. 
The goal is then to compute a certain combinatorial property or structure 
of the precise set of points, while minimising the cost of the updates 
made by the 
algorithm~\cite{bhkr-eusgcu-05,fggt-chapa-94,hekmr-cmstu-08,tk-itet-11}.

\tweeplaatjes [scale=.99] {intro-example-in} {intro-example-out} 
{ (a) Two disjoint onions. (b) Their union.}

\subsection{Results}

We begin by showing that the union of two disjoint onions can be computed in 
 time, where  is the number of layers in the
resulting onion.

We apply this algorithm to obtain an efficient solution to the
onion preprocessing problem mentioned in the introduction.
Given  pairwise disjoint unit disks that model
an imprecise point set, we build a data structure of size
 such that the onion decomposition of an instance can be 
retrieved in  time, where  is 
the number of layers in the resulting onion.
We present several preprocessing algorithms. 
The first is very simple and achieves 
expected time. The second and third algorithm make this guarantee 
deterministic, at the cost of worse constants and/or a more involved
algorithm.

We also show that the dependence on  is necessary:
in the worst case, any comparison-based algorithm
can be forced to take  time on some instances.

\section {Preliminaries and Definitions}

Let  be a set of  points in . 
The \emph {onion decomposition}, or \emph {onion}, of , is the 
sequence  of nested convex polygons with vertices 
from , 
constructed recursively as follows:
if ,  we set 
,
where  is the convex hull of ;
if , then 
~\cite {c-clps-85}.
An element of  is called a \emph{layer} of .
We represent the layers of  as dynamic balanced binary search
trees, so that operations \emph{split} and \emph{join} can be performed
in  time.

Let  be a set of disjoint unit disks in . We say a point set 
 is a \emph {sample} from  if every disk in  contains
exactly one point from . 
We write  for the logarithm with base .

\section {Main Result}

Our data structure and accompanying query algorithm require several pieces, to be described in the
following sections.

\subsection {Unions of Onions}

Suppose we have two point sets  and , together with their onions. 
We show how to find  quickly, given that 
and  are disjoint, given that  and  do not overlap.
Deleting points can only decrease
the number of layers, so:

\begin {obs} \label {obs:k}
  Let . 
  Then  and  cannot have more layers than 
  . \hfill
\end {obs}

The following lemma constitutes the main ingredient of our onion-union 
algorithm.  A \emph{convex chain} is any connected subset 
of a convex closed curve. 

\begin{lemma} \label{lem:chainhull}
  Let  and  be two non-crossing convex polygonal chains. We can find 
   in  time, where  is the total number
  of vertices in  and .
\end{lemma}

\begin{proof}
  Since  and  do not cross, the pieces of  and  that appear on 
   are both connected. If not, there would be on  
  four points that alternate between , , , and , in that order. 
  However, the points on  must be connected inside  by
  the polygonal chain; 
  the same holds for the points on . Thus, the chains  and  would cross, which 
  contradicts the assumption of the lemma. 

  Since  and  are convex chains, we can compute
   in  time. Furthermore, since  and 
  are disjoint, we can also, in  time, make sure that
  , by removing parts from  or ,
  if necessary.
  Now we can 
  find the bitangents of  and  in 
  logarithmic time~\cite {ks-cctws-95}.
\end{proof}

\begin{lemma} \label{lem:restore}
  Suppose  has  layers. Let  be the outer layer of
  , and  be two vertices of . 
  Let  be the points on  between  and , going counter-clockwise. 
  We can find 
   in  time.
\end{lemma}

\begin{proof}
  The points  and  partition   into two pieces,  and . 
  Let  be the second layer of .
  The outer layer of  is the convex hull of 
  , 
  i.e., the convex hull of  and . 
  By Lemma~\ref {lem:chainhull}, we can find it in  time.
  Let  be the points on  where the outer 
  layer of  connects.
  We remove the part between  and  from , and
  use recursion to compute the remaining layers of  
  in  time; see
  Figure~\ref{fig:union-half-eaten+union-cascaded}.
\end{proof}

\tweeplaatjes{union-half-eaten}{union-cascaded} 
{ (a) A half-eaten onion; (b) the restored 
onion.}

\noindent
We conclude with the main theorem of this section:

\begin{theorem}\label{thm:ounion}
  Let  and  be two planar point sets of total size .
  Suppose that  and  are disjoint.
  We can find the onion  in  
  time,
  where  is the resulting number of layers.
\end{theorem}

\begin{proof}
  By Observation~\ref{obs:k},  and  each have at most 
   layers.
  We use Lemma~\ref{lem:chainhull} to find  
  in  time.
  By Lemma~\ref{lem:restore}, the remainders of 
   and  can be restored to 
  proper onions in  time.
  The result follows by induction.
\end{proof}


\subsection{Space Decomposition Trees}

We now describe how to preprocess the disks in 
for fast divide-and-conquer. 
A \emph{space decomposition tree} (SDT)  
is a rooted binary tree where each node  is associated with a planar region
. The root corresponds to all of ; for each leaf 
 of , the region  intersects only a constant number of
disks in .
Furthermore, each inner node  in  is associated with a directed line
, so that if  is the left child and  the right
child of , then  and 
. Here,  is the 
halfplane to the left of  and  the halfplane
to the right of ; see Figure~\ref{fig:dec-tree}.

Let , and let  be an SDT.
For a node  of , let  denote the number of disks in  that
intersect . We call  an -SDT
for  if for every inner node  we have that
(i) the line  intersects at most  disks that intersect
; and (ii) , where  and  are the children
of .

\eenplaatje{dec-tree}{A space decomposition tree for
 unit disks.}

\begin{lemma}\label{lem:tree_complexity}
Let  be an -SDT. The tree  has height  and
 nodes.
Furthermore, .
\end{lemma}

\begin{proof}
The fact that  has height  is immediate from property
(ii) of an -SDT.
For , let 
, the set of nodes
whose regions intersect between  and  disks.
Note that the sets  constitute a partition of the nodes.
Let  be the nodes in  whose parent
is not in . By property (ii) again, the  
along any root-leaf path in  are monotonically decreasing, so the nodes in 
 are unrelated (i.e., no node in  is 
an ancestor or descendant of another node in 
). 
Furthermore, the nodes in  induce in  a forest  such 
that each tree in  has a root from  and 
constant height (depending on ).

Let .
We claim that for , we have  

for some large enough constant .
Indeed, consider a node .
As noted above,  is the root of a tree  of constant height in the 
forest induced by .
By property (i), any node   in this subtree adds at most 
 additional disk intersections (i.e., 
, where ,  are the children of
).
Since 
 has
constant size, the total increase in disk intersections in 
is thus at most , for some constant . 
Since , it follows that the number of disk intersections
increases multiplicatively by a factor of at most , for some constant .
The trees  partition  and the root intersects
 disks, so for the nodes in , the total number of 
disk intersections
has increased by a factor of at most 
, giving 
(\ref{equ:boundDi}). The product in (\ref{equ:boundDi}) is easily estimated:

since . Hence, each set  has at most
 nodes for . The total size of all 
 is .
Since each   lies in a constant size subtree rooted at 
a , it follows that  has  nodes.
For the same reason, we get that
.
\end{proof}

Now there are several ways to obtain an -SDT 
for .
A very simple construction is based on the following lemma, which 
is an algorithmic version of a result by 
Alon~\etal~\cite[Theorem~1.2]{akp-cddsl-89}. 
See Section~\ref{sec:derandomize} for alternative approaches.

\begin{lemma}\label{lem:line-separator}
There exists a constant , so that for any set  of 
 congruent nonoverlapping disks in the plane, there is a 
line  with at least  disks completely 
to each side of it. We can find  in 
 expected time.
\end{lemma}


\begin{proof}
Our proof closely follows Alon~\etal~\cite[Section~2]{akp-cddsl-89}.
Set ,
and pick a random integer  between  and . 
Find a line  whose angle with the -axis is 
and that has  disk centers on each side.
Given , we can find  in  time by a median computation.
The proof by Alon~\etal\ implies that with probability at least  over the
choice of , the line  intersects at most  disks in
, for some constant .
Thus,  
we need two tries in expectation
to find a good line . The 
expected running time is . 
\end{proof}

To obtain a -SDT  for ,  we apply
Lemma~\ref{lem:line-separator} recursively until the region for
each node intersects only a constant number of disks.
Since the expected running time per node is linear
in the number of intersected disks, 
Lemma~\ref{lem:tree_complexity} shows that the total expected running time
is .

By Lemma~\ref{lem:tree_complexity}, the leaves of  induce a planar
subdivision  with  faces. We add a large enough
bounding box to  and triangulate the resulting graph.
Since  is planar, the triangulation has complexity
 and can be computed in the same time (no need for heavy
machinery---all faces of  are convex). With each disk
in , we store the list of triangles that intersect it
(recall that each triangle intersects a constant number of  disks). 
This again takes  time and space.
We conclude with the main theorem of this section:

\begin{theorem}\label{thm:sdt1}
  Let  be a set of  disjoint unit disks in .
  In  expected time, we can construct
  an  space decompositon tree  for . 
  Furthermore, for each disk , we have
  a list of triangles  that cover the leaf
  regions of  that intersect .\qed
\end{theorem}


\subsection{Processing a Precise Input}

Suppose we have an -SDT together with a point location
structure as in Theorem~\ref{thm:sdt1}.
Let  be a sample
from . Suppose first that we know , the number of 
layers in .
For each input point , let  be the corresponding disk. 
We check all triangles in ,
until we find the one that contains .
Since there are  triangles, and each one intersects 
disks, this takes  total time
for all points in .
Afterwards, we know for each point in  the leaf of  that contains it.

For each node  of , let  be the number of points in the subtree
rooted at . We can compute the 's in total time  by a postorder
traversal of .
The \emph{upper tree}  of  consists of all nodes  with
.
Each leaf of  corresponds to a subset of  with
 points. For each such subset, we use Chazelle's 
algorithm~\cite{c-clps-85}
to find its onion decomposition in  time. 
Since the subsets are disjoint, this takes  total time.
Now, in order to obtain , we perform a postorder traversal of 
, using Theorem~\ref{thm:ounion} in each node to unite the onions of 
its children. This gives  at the root.

The time for the onion union at a node  is
. 
We claim that for , the upper tree
 contains at most  nodes  with .
Given the claim,  the 
total work is proportional to

since the series  is dominated by the
first term . It remains to prove the claim. Fix 
 and let  be the nodes
in  with , whose parents have
. Since the nodes in  represent disjoint subsets
of , we have . 
Furthermore, by property (i) of an -SDT ,
both children  for every node  
have , so that after  levels,
all descendants  of  have . The claim follows.

So far, we have assumed that  is given. Using standard exponential
search, this requirement can be removed. More precisely, 
for , set . Run the above algorithm 
for . If the algorithm
succeeds, report the result. If not, abort as soon as it turns
out that an intermediate onion has more than  layers and try
. The total time is 

as desired. This finally proves our main result.

\begin {theorem}
  Let  be a set of  disjoint unit disks in .
  We can build a data structure that stores , of size , 
  in  expected time, such that given a sample 
   of , we can compute 
   in  time, where  is the number of layers in
  .\hfill
\end {theorem}


\noindent\textbf{Remark.}
Using the same approach, without the exponential search, we can also 
compute the outermost  layers of an onion with arbitrarily many layers 
in  time, for any .
In order to achieve this, we simply abort the union algorithm whenever 
 layers have been found, and note that 
by Observation~\ref {obs:k},
the points in  not on the 
outermost  layers of  will never be part of the 
outermost  layers of  for any .

\section{Deterministic Preprocessing}\label{sec:derandomize}

We now present alternatives to Lemma~\ref{lem:line-separator}. 
First, we describe a very simple construction that
gives a deterministic way to build an -SDT
in  time.

\begin{lemma}
Let  be a set of  non-overlapping unit disks.
Suppose that the centers of  have been sorted in horizontal and vertical direction.
Then we can find in  time a (vertical or horizontal) line , such that
 intersects  disks and such that  has
at least  disks from  completely to each side.
\end{lemma}

\begin{proof}
Let , , ,  be the  left-, right-,
top-, and bottommost disks in , respectively.
We can find these disks in  time, since we know the horizontal
and vertical order of their centers. We call 
 the \emph{outer disks}, and
 the \emph{inner disks}.

Let  be the smallest axis-aligned rectangle that contains
all inner disks. Again,  can be found in linear time. 
There are  inner disks, and all
disks are disjoint, so the area of  must be .
Thus,  has width or height ; assume 
wlog that it has width . Let  be
the rectangle obtained by moving the left boundary of  to the right
by two units, and the right boundary of  to the left by two units.
The rectangle  still has width , and it intersects
no disks from .
There are  vertical lines that intersect
 and that are spaced at least one unit apart.
Each such line has at least  disks completely to each side, and
each disk is intersected by at most one line. 
Hence, there must be a line that intersects  disks, as claimed.
We can find such a line in  time by sweeping the disks from
left to right.
\end{proof}


The next lemma improves the constants of the previous construction. It
allows us to compute an -SDT
tree in deterministic time , but it requires comparatively heavy machinery.
\begin{lemma}\label{lem:sdt_matousek}
Let  be a set of  congruent non-overlapping disks.
In deterministic time , we can find a line  such
that there are at least  disks completely to each side
of .
\end{lemma}

\begin{proof}
Let  be a planar -point set, and let  be a parameter.
A \emph{simplicial -partition of } is 
a sequence  of 
 triangles and a partition  of  
into  pieces such
that  (i) for , we have 
and ; and (ii) every line  intersects 
 triangles .
Matou\v{s}ek showed that a simplicial -partition exists for every 
planar -point set and for every . 
Furthermore, this partition can be found in
 time (provided that , for 
some )~\cite[Theorem~4.7]{Matousek92}.

Let  be two constants to be determined later. Set 
. Let  be the set of centers of the disks in .
We compute a simplicial -partition for  in  time. 
Let  be the resulting triangles and
 the partition of .
Set , and for , let 
be the line through the origin that  forms an angle  with the 
positive -axis.
Let  be the projection of the triangles
 onto .
We interpret  as a set
of weighted intervals, where the weight of an interval is the size  of 
the associated point set for the corresponding triangle.
By the properties of the simplicial partition, the interval set  
has \emph{depth} ,
i.e., every point on  is covered by at most  intervals 
of .

Note that the sets  can be determined in  total time, for  small enough.
Now, for each  , we find a point  on  that has intervals of
total weight   completely to 
each side. Since the depth of
 is , we can find such a point in time  with
binary search, for a total of  time (it would even be
permissible to spend time  on each ). Let  be the line
perpendicular to  through .

The analysis of Alon~\etal\ shows that for each , there are at
most  disks that intersect  and at least one other line
~\cite[Section~2]{akp-cddsl-89}. Thus, it suffices to focus on the 
disks in  that intersect
at most one line . By simple counting, there is a line
 that exclusively intersects at most  disks. 
It remains to find such a line in  time. For this, we compute 
the arrangement  of the strips with width  centered
around each , together with an efficient point location 
structure. For each cell in the arrangement, we store whether it is covered
by , , or more strips. Using standard techniques, the construction 
takes  time. 
We locate for each triangle  the cells of 
that contain the vertices of . This needs 
 steps.
Since every line intersects at most  triangles,
we know that there are at most  
triangles that intersect a cell boundary of . We call these
triangles the \emph{bad} triangles.

For all other triangles , we know that the associated point set
 lies completely in one cell of . Let  be the 
set of corresponding disks. By using the information stored with the cells, we 
can now determine for each disk  in
 time whether  intersects exactly one line .
Thus, we can determine in total time  for each line  the total
number of disks that intersect only  and whose center is not 
associated with
a bad triangle. Let  be the line for which this number is minimum.

In total, it has taken us  steps to find . Let us
bound the number of disks that intersect . First, we know that
there are at most 

disks whose centers lie in bad triangles. Then, there are at most
 disks that intersect  and at least one other line.
Finally, there are at most  disks with a center in a good 
triangle that intersect only . Thus, if we choose, say, 
  and , 
then  crosses at most  disks in . Furthermore, 
by construction,  has at least  disk centers on 
each side. The result follows.
\end{proof}

\noindent\textbf{Remark.} Actually, we can use the approach from 
Lemma~\ref{lem:sdt_matousek} to compute an -SDT in 
total deterministic time . The bottleneck 
lies in finding the simplicial partition for . All other steps take  time.
However, when applying Lemma~\ref{lem:sdt_matousek} recursively, we do not need
to compute a simplicial partition from scratch. Instead, as in Matou\v{s}ek's paper,
we can recursively refine the existing partitions in linear 
time~\cite[Corollary~3.5]{Matousek92} (while duplicating the triangles for the
disks that are intersected by ). Thus, after spending  time
on the simplicial partition for the root, we need only linear time per
node to find the dividing lines, for a total of , by 
Lemma~\ref{lem:tree_complexity}.


\section{Lower Bounds}

\eenplaatje[width=\textwidth] {lowerbound} {The lower bound construction 
consists of  unit disks centered on a horizontal line 
( in the figure), and two groups of  points sufficiently far 
to the left and to the right of the disks.
Distances not to scale.}
\eenplaatje {lowerbound-kgon} { copies of the construction on a 
regular -gon. 
}

We now show that 
our algorithm is optimal in the decision tree model. 
The precise nature of the decisions does not matter, 
as long as each decision extracts only a constant number
of bits of information from the input.
We begin with a lower bound of  for .
Let  be a multiple of , and consider the lines 

Let  consist 
of  disks centered on the -axis at -coordinates 
between  and ;
a group of  disks centered on  at -coordinates 
between  and ;
and a symmetric group of  disks centered on 
 at -coordinates between  and .
Figure~\ref {fig:lowerbound} shows .

\begin{lemma}\label{lem:permute}
Let  be a permutation on  elements. There is a sample  
of  such that  (the point for the th disk from the left in 
the main group) lies 
on layer  of .
\end{lemma}
\begin{proof}
Take  as the  centers of the disks in  on ,
the  centers of the disks in  on , and for each 
disk  on the -axis
the point .
By construction, the outermost layer of  contains at 
least the leftmost point on , the rightmost point on 
, and the highest point (with -coordinate ).
However, it does not contain any more points: the line segments connecting 
these three points have slope at most . The second highest point 
lies  lower, and at most  further to the left or the right.
The lemma follows by induction. 
\end{proof}

There are  permutations ;
so any corresponding decision tree has height
.
We can strengthen the lower bound to 
 by taking  copies of 
and placing them on the sides of a regular 
-gon, see Figure~\ref{fig:lowerbound-kgon}. 
By Lemma~\ref{lem:permute}, we can choose independently for
each side of the -gon one of  permutations.
The onion depth will be , and the number of permutations
is . 


\begin{theorem}\label{thm:lowerb}
  Let  and . There is a set
   of  disjoint unit disks in ,
  such that any decision-based algorithm to compute  for a sample
   of ,
  based only on prior knowledge of ,
  takes  time in the worst case.
\end{theorem}

The lower bound still applies if
the input points come from an appropriate probability distribution (e.g.,
\cite[Claim~2.2]{Ailon11}). Thus, Yao's minimax 
principle~\cite[Chapter~2.2]{MotwaniRa95} 
yields a corresponding lower bound for any randomized algorithm.

\section{Conclusion and Further Work}

Recently, Hoffmann~\etal~\cite{hkm13} showed how to compute 
in linear deterministic time a line that stabs 
 disks in a set of  disjoint unit 
disks and has  centers on each side, for any 
. They can also find a line that stabs  
disks and has exactly  centers on each side. 
Using this, one can improve the running times of 
Lemma~\ref{lem:line-separator} and Lemma~\ref{lem:sdt_matousek} 
to linear deterministic time. Note that this does not 
impact the final running time for our original problem.

It would be interesting to understand how much the parameter
 can vary for a set of imprecise bounds and how
to estimate  efficiently.
Further work includes considering more
general regions, such as overlapping disks, disks of
different sizes, or fat
regions. It would also be interesting to consider the
problem in 3D. Three-dimensional onions are not
well understood. The best general algorithm is due to Chan
and needs  expected time~\cite{c-addsf3cha2nnq-10}, 
giving more room for improvement.

\noindent\textbf{Acknowledgments.}
The authors would like to thank an anonymous reviewer for comments that improved the paper.
M.L.~supported by the Netherlands Organisation for Scientific 
Research (NWO) under grant 639.021.123.  W.M.~supported in part 
by DFG project MU/3501/1.


\bibliographystyle {abbrv}
\bibliography {onions}

\end{document}
