\documentclass[review]{elsarticle}

\def\squarebox#1{\hbox to #1{\hfill\vbox to #1{\vfill}}}
\renewcommand{\qed}{\hspace*{\fill}
            \vbox{\hrule\hbox{\vrule\squarebox{.667em}\vrule}\hrule}\smallskip\newline}

\newtheorem{thm}{Theorem}
\newtheorem{lem}[thm]{Lemma}
\newtheorem{cor}[thm]{Corollary}
\newtheorem{cla}[thm]{Claim}
\newproof{pf}{Proof}
\newdefinition{defi}{Definition}
\usepackage{amssymb, amsmath, enumerate}
\usepackage{graphicx}
\newcommand{\diD}{\vec{D}}
\input{epsf}

\begin{document}
\begin{frontmatter}
\title{A linear time algorithm for the next-to-shortest path problem on undirected graphs with nonnegative edge lengths}
\author[bang]{Bang Ye Wu\corref{cor1}}
\ead{bangye@cs.ccu.edu.tw}
\author[yl]{Jun-Lin Guo}
\author[yl]{Yue-Li Wang}
\address[bang]{Dept. of Computer Science and Information Engineering,
National Chung Cheng University, ChiaYi, Taiwan 621, R.O.C.}
\address[yl]{Department of Information Management, National Taiwan University of Science and Technology, Taipei, Taiwan, R.O.C.}
\cortext[cor1]{corresponding author}

\begin{abstract}
For two vertices  and  in a graph , the
next-to-shortest path is an -path which length is minimum
amongst all -paths strictly longer than the shortest path
length. In this paper we show that, when the graph is undirected and
all edge lengths are nonnegative,  the problem can be solved in
linear time if the distances from  and  to all other vertices
are given.
\end{abstract}

\begin{keyword}
Algorithms, Graphs, Shortest path, Time complexity, Next-to-shortest
path
\end{keyword}
\end{frontmatter}

\section{Introduction}
Let  be an undirected graph with vertex set , edge set
 and edge-length function . We shall use  and  to stand
for  and , respectively. For , a {\em simple
-path} is a path from  to  without repeated vertex in the
path. In this paper, a path always means a simple path. The {\em
length of a path} is the total length of all edges in the path. An
-path is a shortest -path if its length is minimum amongst
all possible -paths. The shortest path length from  to  is
denoted by  which is the length of their shortest path. A
\emph{next-to-shortest} -path is an -path which length is
minimum amongst those the path lengths \emph{strictly larger} than
. And the {\em next-to-shortest path problem} is to find a
next-to-shortest -path for given ,  and . In this
paper, we present a linear time algorithm for solving the
next-to-shortest path problem on graphs with nonnegative edge
lengths, assuming the distances from  and  to all other
vertices are given.


\subsubsection*{History}
The next-to-shortest path problem was first studied by Lalgudi and
Papaefthymiou in the directed version with no restriction to
positive edge length \cite{lal97}. They showed that the problem is
intractable for path and can be efficiently solved for walk
(allowing repeated vertices). Algorithms for the problem on special
graphs were also studied \cite{bar07,mod06}. For undirected graphs
with positive edge lengths, the first polynomial algorithm was
presented in \cite{kra04} with time complexity  time. The
time complexity has been improved several times
\cite{li06,kao11,wu10}. The currently best result is 
\cite{wu10}, and recently the author further improved to linear
time, assuming the distances from  and  to all other vertices
are given. Hence, the positive length version of the
next-to-shortest path problem can be solved with the same time
complexity as the single source shortest paths problem. On the other
hand, the problem becomes more complicated when edges of zero weight
are allowed, and there is no polynomial time algorithm for this
version before this work.

\subsubsection*{Techniques}
An edge of zero-length is called as \emph{zero-edge} and otherwise a
\emph{positive edge}. Let  be the union of all shortest
-paths. Let  be the digraph obtained from  by
orientating all edges toward . That is, for any directed edge
(arc) in , there is a shortest -path in  passing
through this edge with the same direction. Since a next-to-shortest
path either contains an edge in  or not, the problem is
divided into two subproblems: the shortest detour path problem and
the shortest zigzag path problem. The {\em shortest detour path
problem} is to find a shortest -path using at least one edge not
in  while the {\em shortest zigzag path problem} looks for a
shortest -path consisting of only edges in  with at least
one reverse arc of a positive length in . Clearly, the shorter
path found from the above two subproblems is a next-to-shortest
path.

In this paper, we solve the nonnegative length version also by
solving the two subproblems individually. But there are some
difficulties to be overcome. First, the digraph  is not so
easy to construct as in the positive length version. Secondly,
 is no more a DAG (directed acyclic graph) as in the positive
length version, and therefore some properties in \cite{wu10} cannot
be used. Instead , we solve the two subproblems based on a
relaxed digraph  of , in which all zero edges are
regarded as bidirectional. The method to solve the shortest detour
path subproblem is similar to the previous one for the positive
length version, but a special care is taken into consideration for
the zero-edges and the proofs are non-trivial and different from the previous ones.

The shortest zigzag path subproblem is relatively more complicated.
To solve this subproblem efficiently, the most important thing is to
determine for a pair of vertices  if there exists a simple
-path using a path from  to  as a backward subpath. The
previous paper \cite{wu10} showed a necessary and sufficient
condition for the positive length version, but this condition no
more holds when there are zero-edges. To overcome this difficulty,
we use immediate dominators developed in the area of flow analysis.
In addition, we define zero-component in , which are basically
connected components of the subgraph induced by the zero-edges but
any vertex and its dominators are divided into different components.
By shrinking zero-components and orientating the remaining
zero-edges, we construct an auxiliary DAG. With the help of the
auxiliary DAG, we categorize a shortest zigzag path into four types
and derive necessary and sufficient conditions individually.

The main result of this paper is the following theorem, and its
proof is given by Theorems~\ref{thm:back} and \ref{thm:out} in
Sections~3 and 4, respectively.
\begin{thm}\label{thm:main}
A next-to-shortest -path of an undirected graph with nonnegative
edge lengths can be found in linear time if the distances from 
and  to all other vertices are given.
\end{thm}

\subsubsection*{Paper organization}
The paper is divided as follows. In Section 2, the preliminaries are
presented. In addition to the notation used in this paper, in the
preliminaries, we introduce dominators, a method of constructing
, and zero-components. Also we show some basic properties in
this section. In Sections 3 and 4, we discuss the shortest zigzag,
and detour, path problems, respectively. And finally concluding
remarks are given in Section 5.

\section{Preliminaries}
\subsection{Notation and some properties}
Throughout this paper, we shall assume that  is the input graph
and  is the pair of vertices for which a next-to-shortest
path is asked. Furthermore,  is simple, connected and undirected,
and all edge lengths are nonnegative integers.

For a graph ,  and  denote its vertex and edge sets,
respectively. For simplicity, sometimes we abuse the notation of a
subgraph for its vertex set when there is no confusion from the
context. A {\em -path} is a path from  to . For vertices
 and  on path , let  denote the subpath from  to
. We shall use ``a -path'' and a path  alternatively.
For a path , we use  to denote the reverse path of .
For paths  and ,  denotes the path
obtained by concatenating these two paths. Note that, even for an
undirected path, we use  to specify the direction from 
to . For example, by ``the first vertex  of 
satisfying some property'', we mean that  is the first vertex
satisfying the property when we go from  to  along path .
Two paths are {\em internally disjoint} if they have no common
vertex except their endpoints. For a path , let  denote the length of the path. Let  denote the
shortest path length from  to  in , which is also called
the {\em distance} from  to . For convenience, let
 and .


To show the time complexities more precisely, we shall assume the
distances from  and  to all other vertices are given. These
distances can be found by solving the single source shortest paths
(SSSP) problem. For general undirected and nonnegative edge length
graphs (the most general setting of the problem discussed in this
paper), the SSSP problem can be solved in  time
\cite{cor01,fred87}, and more efficient algorithms exist for special
graphs or graphs with restrictions on edge lengths. A shortest path
tree rooted at  can also be constructed in linear time if the
distances from  to all others are given.

\subsection{ and }

Let  be the digraph obtained from  by orientating all
positive edges toward . That is, we treat all zero-edges as
bidirectional even though only one direction of some of them can be
used to form a shortest -path. Our algorithm for finding a
shortest zigzag path works on  for the sake of efficiency.

To construct , we have to construct  first. In the
following, we show how to construct  and  in linear time.
Clearly, for ,  always holds.
Unfortunately, the condition that  is not a
necessary and sufficient condition to determine the set of vertices
in  when there are zero-edges. The reason is described as
follows. Let  be the subgraph of  with
 and  for . A vertex is a
{\em non--cut} if it is a cut vertex and its removal does not
separate  and . For a non--cut , a connected component
 of  is called a {\em knob} if . Since 
is a cut vertex, any -path passing through a vertex in 
repeats at  and cannot be simple. Furthermore, for any vertex 
in , since , it must be connected to  by
a path of zero-length.

\begin{lem}\label{D+}
A vertex  is in  iff  is not in any knob.
\end{lem}
\begin{pf}
By definition,  implies . Furthermore, 
cannot be in any knob since there is no simple -path in 
passing through any vertex in a knob.

Now, we prove the other direction. For any vertex , consider the digraph  obtained from  by
reversing the direction of all positive edges  with
. Also we add a new vertex  as well as two arcs
 and . Then there exists a shortest -path
passing  in  iff there are two disjoint paths from  to
 in , or equivalently there is no non--cut. Obviously
any vertex  with  cannot be an -cut in
, and there exists such a cut node iff  is in a knob.
\qed\end{pf}

\begin{lem}
 can be constructed in linear time if  and  are
given for all .
\end{lem}
\begin{pf}
First we construct  in linear time. By using depth-first search
starting from , all cut vertices can be found in linear time.
According to the order of found cut vertices, all knobs can be
detected by checking the components after removing the cut
vertices.
\qed\end{pf}

\subsection{Dominators in }

We shall use the term ``immediate dominators'' defined in
\cite{als99}. A vertex  is an {\em -dominator} of
another vertex  iff all paths from  to  contain . An
-dominator  of  is an \emph{-immediate-dominator} of
, denoted by , if it is the one closest to , i.e., any
other -dominator of  is an -dominator of . In
, any vertex has a unique -immediate-dominator. The
-dominator is defined symmetrically, i.e.,  is a {\em
-dominator} of  iff any -path contains , and 
stands for the -dominator closest to . Note that  is an
-dominator and  is a -dominator of any other vertex in
.

Finding immediate dominators is one of the most fundamental problems
in the area of global flow analysis and program optimization. The
first algorithm for the problem was proposed in 1969 by Lowry and
Medlock \cite{lor69}, and then had been improved several times
\cite{har85,len79,pur72,tar74}. A linear time algorithm for finding
the immediate dominator for each vertex was given in \cite{als99}.

\subsection{Zero-components}
\mbox{}

\begin{defi} A path  is a 0-path if all
edges in  are zero-edges. A 0-path  is a
-path if  does not contain any vertex in
. A {\em zero-component} is the subgraph
of  induced by a maximal set of vertices in which every two
vertices are connected by a -path. The zero-component which 
belongs to is denoted by .
\end{defi}

A zero-component may contain only one vertex but no edge. All the
zero-components partition  into equivalence classes, i.e.,
 iff . We shall show how to find all
zero-components of  in linear time.

\begin{lem}\label{zero-d}
If , then  and .
\end{lem}
\begin{pf}
If  is not an -dominator of , there is an -path
 avoiding . Since  and  are in the same
zero-component, there is a 0-path  in  avoiding
. Thus, , possibly taking a short-cut if the
path is non-simple, is a path from  to  avoiding , a
contradiction. Therefore  is an -dominator of .
Similarly we can show that  is also an -dominator of
. Consequently  and  dominate each other, and
thus they are the same vertex. The result  can be
shown similarly.
\qed\end{pf}

An \emph{-dominator tree} \cite{als99} of  is a tree 
with root  and vertex set . A vertex  is a child of
 in  iff .
\begin{lem}
The subgraph of  induced by the edge set
 is the union of all zero-components, where
 is the zero-edges set, and  and  are the
edge sets of - and -dominator trees of , respectively.
\end{lem}
\begin{pf}
Since no positive edge is in any zero-component, we only need to
consider the zero-edges . For any vertex , if 
is the last edge of a path from  to , then  is a child
of  in the -dominator tree. After removing  and
, there is no path from any vertex to its - or
-dominator. Therefore, by definition, the induced subgraph is the
union of all zero-components.
\qed\end{pf}

Since a dominator tree can be constructed in linear time
\cite{als99}, the next corollary follows directly from the above
lemma.

\begin{cor}
All zero-components of  can be found in linear time.
\end{cor}

\subsection{Outward and backward subpaths}

A positive edge  is a reverse positive edge if . It implies that  since any positive
edge in  is unidirectional.

\begin{defi}
A {\em backward subpath} of a path in  is a path consisting of at
least one reverse positive edge and possibly some zero-edges. A
semi-path in  with at least one backward subpath is called a
{\em zigzag path}. Two backward subpaths in a zigzag path are
consecutive if there are separated by a sequence of non-reverse
positive edges and zero edges; otherwise, they form a longer
backward subpath indeed.\footnotemark[4]
\footnotetext[4]{Another way to define a
backward subpath is a {\em maximal} subpath consisting of at least
one reverse positive edge and possibly some zero-edges. The
difference is that, by our definition, there may be some zero-edges
preceding or succeeding a backward subpath. Our definition is for
the sake of simplifying some proofs.}
\end{defi}


By definition, a zigzag path is a semi-path in . For
simplicity, we shall use ``path'' instead of ``semi-path'' in the
following.

\begin{defi}
An {\em outward subpath} of an -path in  is a path consisting
of edges in . The both endpoints of an outward subpath are
in  and all its internal vertices, if any, are not in .
An -path is called a {\em detour path} if it contains at least
one outward subpath.
\end{defi}

The {\em shortest detour path problem} is to find a shortest detour
-path while the {\em shortest zigzag path problem} looks for a
shortest zigzag -path consisting of only edges in . Since
a next-to-shortest path either contains an edge in  or not,
the shorter path found from the above two subproblems is a
next-to-shortest path. Since  and  are fixed throughout this
paper, we shall simply use ``zigzag path'' and ``detour path''
instead of ``zigzag -path'' and ``detour -path'',
respectively.

When the edge lengths are all positive, the following result was
shown in \cite{kao11}, and it is also the basis of the algorithms in
this paper. In remaining paragraphs of this subsection, we show
Theorem~\ref{backout} by Lemmas~\ref{oneback} and \ref{oneout}.

\begin{thm}\label{backout}
A shortest zigzag path contains exactly one backward subpath. A
shortest detour path contains exactly one outward subpath and no
backward subpath.
\end{thm}

\begin{lem}\label{oneback}
A shortest zigzag path contains exactly one backward subpath.
\end{lem}
\begin{pf}
Suppose by contradiction that  is a shortest zigzag path in 
with more than one backward subpath. Let , for
, be the consecutive backward subpaths in
 and  where  and . We may assume that the first and the last edges of  are
positive edges (otherwise move  forth or  back
accordingly). Let  be the first vertex on  such that
 and  the last vertex such that 
(see Fig.~\ref{f1back}.(a)). We divide into three cases, and in
either case we show that there exists a shorter zigzag path .

\begin{itemize}
\item There is a path  from  to an internal vertex  of  such that  is disjoint to . Then  is a zigzag path. Since  is a short-cut of ,  is shorter than  (see Fig.~\ref{f1back}.(b)).
\item There is a path  from an internal vertex  of  to  such that  is disjoint to . Similarly,  is a zigzag path shorter than .
\item Otherwise, since the first case does not hold, there exists a path  from  to a vertex  on , which is internally disjoint to . Furthermore, . Similarly, there exists a path  from a vertex  on  to , which is internally disjoint to . And . Then the path  is a zigzag path. Clearly  (see Fig.~\ref{f1back}.(c)).
\end{itemize}
\qed\end{pf}


\begin{figure}[t]
\begin{center}
\includegraphics[scale=1.2]{f_1back.eps}
\caption{Illustrations for {\rm Lemma~\ref{oneback}}. {\rm (a)} A path  with more than one backward subpath.
The bold line is ; {\rm (b)} Case 1; {\rm (c)} Case 3.}
\label{f1back}
\end{center}
\end{figure}

\begin{lem}\label{2paths}
For any two vertices  and  in , there exist an
-path and a -path; or an -path and an -path; which
are disjoint.
\end{lem}
\begin{pf}
The result is trivial for the case . We only need to
show the case . To show the lemma for this case, we
construct an auxiliary directed graph from  by adding a new
vertex  and two bidirectional edges  and . Since
there is no  non--cut, similar to Menger's theorem, there is an
-path passing through  in the auxiliary graph, and the
desired two paths exist.
\qed\end{pf}

\begin{lem}\label{oneout}
A shortest detour path contains exactly one outward subpath and no
backward subpath.
\end{lem}
\begin{pf}
Let  be a shortest detour path, in which  is an outward
subpath. We shall show that if  had another outward subpath or
backward subpath in addition to , we could construct a
detour path  shorter than .

By Lemma~\ref{2paths}, there exist an -path and a -path; or
an -path and an -path in  which are disjoint. In either
case that the two paths exist, we can concatenate the two paths with
 (or its reverse) to form a simple -path. It is clear
that the shorter detour path in the two cases is a shortest detour
path  containing .
\qed\end{pf}

\section{Shortest zigzag path}

\subsection{Basic properties}
By Theorem~\ref{oneback}, a shortest zigzag path has the form
, in which  are
paths in . Since  is required to be simple, the three
subpaths must be simple and disjoint except at the two joint
vertices. Therefore our goal is to find  minimizing

subject to that there exists a simple path  in . Since  is fixed for
a given graph , the objective is to find the minimum of .
If  and  satisfy the constraint, we say ``the pair  is
{\em valid}'' and `` is valid for ''. A valid pair 
with minimum  is an \emph{optimal backward pair}, or simply
{\em optimal pair}, and the corresponding backward subpath is an
{\em optimal backward subpath}. The shortest zigzag path problem is
equivalent to finding an optimal pair.

The auxiliary simple digraph  is obtained from  by
shrinking every zero-component and orientating all the remaining
zero-edges toward . By the definition of zero-component, if
 is a zero-edge in ,  or .
Therefore the orientation can be easily done. For  in ,
let  denote its corresponding vertex in . For
simplicity, since  and  themselves must be zero-components,
the corresponding vertices in  are also denoted by 
and , respectively. For a vertex ,  and 
are again the immediate - and -dominators (but in
). A simple path in  corresponds to a simple path
in  since, without backward subpath, a path cannot
enter a zero-component twice.

\begin{defi}
We define a binary relation on pairs of vertices in :
 or equivalently  iff  and there
exists a path from  to  in . Let  and .
\end{defi}

\begin{defi}
The predicate  is true iff  and .
\end{defi}

\begin{lem}\label{beta1a}
If  is true, .
\end{lem}
\begin{pf}
By definition, , and therefore 
and . If , they are connected by a
0-path but not a -path, i.e., a path containing a vertex in
. Since , a -path
contains neither  nor . Since  and
, , which implies
that any -path in  contains neither  nor ,
a contradiction.
\qed\end{pf}


The notation defined on  will also be used for .
We do not distinguish between them since there will be no confusion
from the context. The next two lemmas appeared in \cite{wu10} for
positive length version, and it is easy to see it also holds for
nonnegative length version. The next lemma show a necessary
condition for the validity of a pair.
\begin{lem}\label{back_nec}
If  is valid, then  is true.
\end{lem}
\begin{pf}
By definition, . If , by the definition
of immediate dominator, any -path and -path contain 
simultaneously and cannot be disjoint. Therefore we have
, and then  by definition. The relation
 can be shown similarly.
\qed\end{pf}

\begin{lem}\label{close}
If , there are two paths from  and ,
respectively, to , which are  disjoint except at .
\end{lem}
\begin{pf}
Let  and  be any -path. By the definition of
immediate dominator, removing any vertex in  cannot separate
 and  and therefore there are two internally disjoint
-paths, say  and . If  is on one of them, say
, we have done since  and  are the
desired paths. Otherwise, let  be any -path and  be the
first vertex on  and also in . W.l.o.g. let
. Then, the path  is a
-path disjoint to .
\qed\end{pf}


\begin{lem}\label{redu_y}
If  is true and there exists a path  from  to
 avoiding , then there exists a vertex valid for .
Furthermore if  satisfies the above condition with minimum
, then there exists a vertex  such that  is
valid and . The same result also holds for the case
that  is true and there exists a path from  to 
avoiding .
\end{lem}
\begin{pf}
We show the first result and the second one can be shown similarly.
Let  be the last vertex of  in . Since 
is also true, we have that  by
Lemma~\ref{beta1a}. By Lemma~\ref{close}, there are a path
 and a path  which are internally disjoint.
Then, the path  is a zigzag path and
therefore  is a valid pair.

Since  is also a vertex satisfying the condition, we have
, otherwise  contradicts the minimality of
.
\qed\end{pf}

\subsubsection*{Types of optimal backward pairs}

By the definition of zero-component, there exists an -path
avoiding  iff  is not an -dominator of .
Similarly, there exists a -path avoiding  iff  is not
a -dominator of . Therefore, all the valid pairs  can
be categorized into the following four types, and the best of the
four types, if any, is an optimal pair.
\begin{itemize}
\item Type I:  is not an -dominator of  and  is not a -dominator of .
\item Type II:  is an -dominator of  and  is a -dominator of .
\item Type III:  is an -dominator of  and  is not a -dominator of .
\item Type IV:  is not an -dominator of  and  is a -dominator of .
\end{itemize}

In the following subsections, we shall derive linear time algorithms
for each of the types. The next theorem concludes the result of this
section, and its proof is given by Lemmas \ref{type1t}, \ref{type2t}
and \ref{type34t} in the following subsections.


\begin{thm}\label{thm:back}
Suppose that  and  are given for all vertices . A
shortest zigzag path can be found in linear time.
\end{thm}


\subsection{Type I}

By definition,  is a DAG. If  is a zero-edge
in , then   or . By this
property, all the properties and the algorithm derived for a
shortest zigzag path in \cite{wu10} also hold for .
\begin{lem}\label{type1t}
Suppose that  and  are given for all vertices . A
shortest zigzag path of type I can be found in linear time.
\end{lem}
\begin{pf}
For  such that  is true, by definition, the
pair  is valid of type I iff  is valid in
. Therefore, a shortest zigzag path of type I in 
can be found by solving the shortest zigzag path problem in
. By the result of \cite{wu10}, it can be done in
linear time.\footnotemark[4]
\qed\end{pf}
\footnotetext[4]{The problem is named {\em the optimal
backward problem} in \cite{wu10}.}

\subsection{Type II}

For a shortest zigzag path of type II, the corresponding path in
 repeats at both  and . The next lemma is not
only for type II.


\begin{figure}[t]
\begin{center}
\includegraphics[scale=1.2]{backfig2.eps}
\caption{Illustrations for {\rm Lemma~\ref{redu_cx}}. The dotted-line depicts a valid zigzag path. {\rm (a)}  and  is valid; {\rm (b)}  intersects  and  is valid.}
\label{f2back}
\end{center}
\end{figure}
\begin{lem}\label{redu_cx}
For any , if  is valid for some , then
there exists some  such that  or  is
valid.
\end{lem}
\begin{pf}
Since  is valid for , there exists a path . By Lemma~\ref{back_nec},
 is true and . Since 
and ,  is also true.

If  or  does not pass any vertex in , by
Lemma~\ref{redu_y},  is valid and the proof is complete.
Otherwise both the two subpaths pass vertices in , and
therefore, in  we can find  and  on  and
, respectively, as well as a 0-path  which is
internally disjoint to  and . Since 
and , . Since  is a path
passing  and avoiding , . Therefore
 by Lemma~\ref{beta1a}.

If  is disjoint to , the path  is a simple path with a
backward subpath from  to . That is,  and 
is valid (Fig.~\ref{f2back}.(a)). Otherwise  intersects .
Let  be the intersection vertex closest to  on . Then, the
path  is a
simple path with a backward subpath from  to . That is,  and  is valid (Fig.~\ref{f2back}.(b)).
\qed\end{pf}

\begin{lem}\label{2p1to2}
For any two vertices  and  such that , there
exist two internally disjoint paths from  to  and ,
respectively.
\end{lem}
\begin{pf}
By definition, there exists no cut vertex whose removal separates
 from  or . By Menger's theorem, such two disjoint
paths exist.
\qed\end{pf}

By definition, if  is valid for type II,  is a
-dominator of  and  is an -dominator of . We
show a stronger condition in the next lemma.


\begin{lem}\label{type2b}
If  is an optimal pair of type II,  and
.
\end{lem}
\begin{pf}
Suppose that  is a shortest zigzag
path of type II, in which  is the backward subpath from
 to . If , both  and  contain
a vertex in . We shall show that , and then by
Lemma~\ref{redu_cx},  is not optimal. The result
 can be handled similarly.

Let  and  be subpaths of  and
, respectively, such that ,
, and no internal vertex of them is in
. We can find a path  in  and two
disjoint paths from  to  and , respectively
(Lemma~\ref{2p1to2}). Then there are two disjoint paths from 
to , and therefore .
\qed\end{pf}

If  and  as well as  is
true, we say that  and  are a {\em candidate component
pair}. By Lemma~\ref{type2b}, to find an optimal pair of type II, we
only need to determine if there exists a valid pair for any
candidate component pair. For a candidate component pair  and
, let  be the digraph with vertex set . The edge set is
, in which  is the subgraph of
 induced by .

\begin{figure}[t]
\begin{center}
\includegraphics[scale=1.2]{f_beta2.eps}
\caption{{\rm (a)} Four cases of the three paths when  is true; {\rm (b)} The case that  and  exist ({\rm Lemma~\ref{type2iff}}); {\rm (c)} The case that  and  exist {\rm (Lemma~\ref{type2iff})}.}
\label{fbeta}
\end{center}
\end{figure}

\begin{defi}
The predicate  is true iff there are three internally
disjoint paths ,  and  in
 satisfying: (1)  and
; and (2)  or  for ; and (3)  and
.
\end{defi}


Fig.~\ref{fbeta}.(a) illustrates the four possible cases of the
three paths when  is true.
\begin{lem}\label{3v2path}
For any three vertices ,  and  in , there are
two disjoint paths  and , or  and
.
\end{lem}
\begin{pf}
Let  and  be any -path. Note that  avoids
. By Lemma~\ref{2p1to2}, there are two disjoint paths
 and . If  is on one of the paths, we
have done. Otherwise, let  be any path in  from 
to , and  be the last vertex of  appeared on  or
. If  is on , then   and
 are the desired two paths. The case that  is on
 can be shown similarly.
\qed\end{pf}
\begin{cor}\label{3v2pathb}
For any three vertices ,  and  in , there are
two disjoint paths  and , or  and
.
\end{cor}

\begin{lem}\label{type2iff}
Suppose that  and  are a candidate component pair. There
exist  and  such that  is a valid
backward pair of type II iff  is true.
\end{lem}
\begin{pf}
It is clear that if  is valid for type II, the three paths
exist and  is true. It remains to prove that such a
valid  exists if  is true. Since
 is true, there are three internally disjoint paths
,  and  in , in
which  and .

By Lemma~\ref{3v2path} and w.l.o.g., there exist disjoint paths
 and . In the case that , 
contains only one vertex but no edge. If , by
Corollary~\ref{3v2pathb}, there exist two disjoint paths
 and ; or  and .
If  and  exist, the path  is a desired path, as shown in
Fig.~\ref{fbeta}.(b). That is,  and . Otherwise
 and  exist, and the path  is the desired path, as
shown in Fig.~\ref{fbeta}.(c), namely,  and .

It remains to consider . By the definition of immediate
dominator, there is a path  from  to  avoiding . The
path  is
a desired path (similar to Fig.~\ref{fbeta}.(c)), in which 
and .
\qed\end{pf}


From , we construct a vertex-capacitated digraph 
as follows. First, if any connected component contains exactly one
vertex  in  and one vertex  in , we replace the
component by an edge . Then, we add a new source  and a
new sink . For each  there is an edge ; and
there is an edge  for each . The capacity of
vertex  is denoted by . The capacities are assigned as
follows: ;  for any ; and  for any other vertex. In the following, ``the
max-flow in '' means the maximum vertex-capacitated flow
from  to  in . Note that a vertex-capacitated
digraph can be easily transformed to an edge-capacitated digraph,
and the maximum flow of a vertex-capacitated digraph can be computed
by traditional maximum-flow algorithms.

\begin{figure}[t]
\begin{center}
\includegraphics[scale=1.2]{backfig3.eps}
\caption{Illustrations for Lemma~\ref{beta22}. {\rm (a)}  and  are disjoint; {\rm (b)}  intersects . The dotted-line depicts a valid zigzag path. }
\label{f3back}
\end{center}
\end{figure}
\begin{lem}\label{beta22}
Suppose that  and  are a candidate component pair and
 is not a valid type-I pair for any  and
. Then,  is true iff the
max-flow in  is at least three.
\end{lem}
\begin{pf}
If  is true, it is easy to see that the max-flow in
 is at least three. We need to show the other direction.
If there is a flow of value three, there are three internally
disjoint paths , . According
to the assigned capacities, there are at least two distinct vertices
in , and so are in . The only
question is that two of the three paths may have the same endpoints.
That is, w.l.o.g.,  and . By the construction of
, the connected component containing the two paths must
contain another vertex in . W.l.o.g. let 
be such a vertex. Then, let  be a path from  to  and 
the first vertex of  intersecting  or . W.l.o.g. let
 be on  (see Fig.~\ref{f3back}).
\begin{itemize}
\item If  and  are disjoint, the three paths ,  and  satisfy the requirement and  is true (Fig.~\ref{f3back}.(a)).
\item Otherwise  and  share a common vertex, possibly . Let  be the last vertex of  on .
\begin{itemize}
\item If , the three paths ,  and  satisfy the requirement and  is true.
\item Otherwise .
There exists a path  is a path from  to
 with a backward subpath of length , and this path can be extended to a zigzag path of
type I, a contradiction to the assumption
(Fig.~\ref{f3back}.(b)). Note that  is true
and therefore  by Lemma~\ref{beta1a}.
\end{itemize}
\end{itemize}
\qed\end{pf}

\begin{cor}\label{betatime}
Under the assumption of Lemma~\ref{beta22},  can be
determined in  time, in which  and
 are the numbers of edges and vertices in ,
respectively.
\end{cor}
\begin{pf}
By Lemma~\ref{beta22},  can be determined by checking
whether the max-flow in  is larger than or equal to three.
Since all the capacities are integral, this max-flow question can be
determined with at most three iterations of the augmentation step of
the Ford-Fulkerson maximum flow algorithm \cite{for56,cor01} or
equivalently at most three breadth-first search on the residue
graphs. Therefore the time complexity is linear.
\qed\end{pf}

\begin{lem}\label{type2t}
Suppose that  is the length of an optimal backward subpath of
type I. In linear time, we can find an optimal backward subpath of
type II with length less than  or determine there is no such
subpath.
\end{lem}
\begin{pf}
By Lemmas \ref{type2iff} and \ref{beta22}, we can determine if there
exists an optimal backward subpath of type II with length less than
. Note that the proofs of Lemmas~\ref{type2iff} and \ref{beta22}
are constructive and they implies a linear time algorithm for
constructing such a zigzag path if it exists. By Corollary
\ref{betatime}, the time complexity is linear to
, in which the summation is taken over all
candidate component pairs. By Lemma~\ref{type2b} and the uniqueness
of immediate dominator, any zero-component is involved in the
max-flow computations at most twice. Therefore the total time
complexity is .
\qed\end{pf}

\subsection{Types III and IV}

Types III and IV are similar to Type II, but simpler. Furthermore,
the two types are symmetric and we shall only explain Type III
briefly. A pair  is valid for type III if  is an
-dominator of  and  is not a -dominator of .
\begin{lem}\label{type3}
If  is an optimal pair of type III, .
\end{lem}
\begin{pf}
By using a similar argument as in Lemma~\ref{type2b}, this lemma
follows.
\qed\end{pf}
In the next lemma,  has the same definition as in type II.
\begin{lem}\label{type3iff}
Suppose that  and  is not a -dominator of
. There exist  and  such that 
is a valid backward pair of type III iff there are two disjoint
paths  and  in  such that
.
\end{lem}
\begin{pf}
It is clear that if  is valid for type III, the two paths
exist. Conversely, if  and  exist, by Lemma~\ref{2p1to2},
there are two disjoint paths  and
, respectively. Since  is not a -dominator
of , we can find a path  from  to  and avoiding
. Let  be the last vertex of  intersecting  or
. Then  is valid for , namely,  and
.
\qed\end{pf}

\begin{lem}\label{type34t}
Suppose that  is the length of an optimal backward subpath of
type I. In linear time, we can find an optimal backward subpath of
type III or IV with length less than  or determine there is no
such subpath.
\end{lem}
\begin{pf}
Similar to Lemma~\ref{beta22}, the necessary and sufficient
condition of type III shown in Lemma~\ref{type3iff} can be
determined by checking whether the max-flow in  is at
least two or not. And the max-flow computations for all candidate
pairs can be done in linear time. The optimal backward subpath of
type IV can be computed similarly.
\qed\end{pf}

\section{Shortest detour path}

In this section we show an efficient algorithm for finding a
shortest detour path. A shortest detour path contains exactly one
outward subpath and has no backward subpath, in which an outward
subpath is a path  such that , both endpoints
of  are in , and any of its internal vertex is not in
. Note that a \emph{simple} -path containing an edge not
in  must have length strictly larger than , or otherwise
it should be entirely in . Our goal is to efficiently find a
minimum length -path with an outward subpath.

In this section  denotes an arbitrary shortest-path tree of 
rooted at  and let  denote the graph obtained by
removing edges in  from . Apparently  is a
forest consisting of subtrees of  and . By the
definition of , any shortest path between two vertices in 
must be included in . For any , the path from 
to  on  must be entirely within  and therefore  must
be a root of a subtree of . Furthermore, the root of any subtree
of  must be in  because the edge between it and its parent
is removed.
\begin{defi}
For any vertex , let  denote the root of the subtree of
 which  belongs to. Let  denote the set of
edges  such that  and .
\end{defi}


Define

Note that, since  is undirected, both  and  denote
the same edge. But  in general.

\begin{lem}\label{out1}
Any detour path  contains an edge in .
Furthermore, if  is an edge on , then
.
\end{lem}
\begin{pf}
By definition  contains an outward subpath. Since the both
endpoints of this outward subpath are in , they must be in
different subtrees of , and  must have an edge in
. The result  directly follows
from definitions.
\qed\end{pf}


For any vertex ,  and the
equality holds iff  is a shortest -path,
in which  is the -path in  and  is an
arbitrary shortest -path in . A vertex  is a
\emph{dangler} if . By definition any
vertex in  is a dangler.
\begin{lem}\label{dangler1}
If  is not a dangler, there exists a detour path  of length at
most .
\end{lem}
\begin{pf}
Let  be the -path in  and  any shortest -path. Let
 be the last vertex of  intersecting , possibly . The
path  is a simple -path, and the length
of  is  . We shall show
. Then, by the definition of ,  and
therefore . Consequently  is a simple path not
entirely in  and thus a detour path.

Suppose to the contrary that . Since
, we have
 and furthermore . Since  is on ,

which is a contradiction to that  is not a dangler.
\qed\end{pf}

\begin{lem}\label{dangler2}
Suppose that  is not a dangler and  is any shortest -path.
For any , .
\end{lem}
\begin{pf}
Let  be a vertex with . We show that
 cannot contain . Since  is on a shortest -path,
, and therefore . Since , by , we have
. Thus, . Since  is not a dangler,
, and therefore  is not on any shortest
-path.
\qed\end{pf}


\begin{figure}[t]
\begin{center}
\includegraphics[scale=1.2]{outf_0a.eps}
\caption{{\rm (a)}  is a dangler while  is not.  is a common
vertex of a shortest -path and a shortest -path; {\rm (b)} The
triangle inequalities when  intersects  (for
Lemma~\ref{outopt}); {\rm (c)} Case 3 in the proof of {\rm Lemma~\ref{outopt}}:
when  is on .  } \label{outf}
\end{center}
\vspace{10pt}
\end{figure}


\begin{lem}\label{outopt}
If  minimizes function  and , then
there exists a simple -path of length  and with one edge
in . Such a path is a shortest detour path.
\end{lem}
\begin{pf}
Since an edge in  is not an edge in , a simple
path containing edge  must have length
strictly larger than . We only need to show the existence of
such a simple path, and then it is a shortest detour path by
Lemma~\ref{out1}.

Let  and  be the shortest paths from  to  and  on
, respectively. Let  be a shortest -path. If  and 
are disjoint,  is a simple path and its
length is clearly . Otherwise let  be the last vertex of
 intersecting . By the triangle inequalities (see
Fig.~\ref{outf}.(b)), we have

By the minimality of , the equality must hold. i.e.,


We divide into three cases according to whether  and  are
danglers.
\begin{itemize}
\item Case 1: Both  and  are danglers. By Lemma~\ref{2paths}, there are two disjoint paths  and ; or  and  in .
If  and  exist, the path  is a simple path. Since  is
a dangler,  is a shortest -path and
the length of the path is . The other case that 
and  exist can be shown similarly.
\item Case 2: Neither  nor  is a dangler. By Lemma~\ref{dangler1}, there exist two detour paths  and  such that  and .
Then

\item Case 3: Either  or  is a dangler. W.l.o.g. assume that  is a dangler but  is not. First we show that  in this case.
Recall that  is a shortest -path intersecting . If
the intersection  is on , by
Lemma~\ref{dangler2}, .
Otherwise  is on  and therefore  is on a
shortest -path. As a result, .
Since  is not a dangler,
. Therefore

Since  is on a shortest -path,
, and equivalently

Since both  and  are in ,
, and we have

Therefore . Thus, any shortest -path
 is disjoint to . The path  is a simple -path with length
.
\end{itemize}
\qed\end{pf}

\begin{thm}\label{thm:out}
For an undirected graph with nonnegative edge lengths, the shortest
detour path problem can be solved in  time if  and
 are given for all .
\end{thm}
\begin{pf}
By Lemma~\ref{outopt}, the length of a shortest detour path is the
minimum value of function . To compute  minimizing , we
first construct  and a shortest path tree , and then find the
edge set . The minimum value of  can be found by
checking both  and  for all edges . The time complexity is linear if the distances
 and  for all  are given. Once  is found,
by the method in the proof of Lemma~\ref{outopt}, the corresponding
path can be constructed in linear time.
\qed\end{pf}

\section{Concluding remarks}
By Theorem~\ref{thm:main}, we have the next corollary.
\begin{cor}
For undirected graphs with nonnegative edge lengths, if the single
source shortest path problem can be solved in  time, the
next-to-shortest path problem can be solved in  time.
\end{cor}
Important graph classes for which the single source shortest path
problem can be solved in linear time include unweighted graphs (by
BFS \cite{cor01}), planar graphs \cite{hen97}, and integral edge
length graphs \cite{thr99}.

As pointed out in \cite{li06,wu10}, it can be easily shown that the next-to-shortest problem is at least as hard as finding a shortest path between two vertices. When negative-weight edges are allowed, the next-to-shortest problem becomes NP-hard because it is polynomial-time reducible from the longest path problem by a similar reduction. 
An interesting problem is how to efficiently find the next-to-shortest paths for single source and multiple destinations. Another open problem is the complexity of the version on directed graphs with positive-weight edges.

\section*{Acknowledgment}
Bang Ye Wu was supported in part by NSC 97-2221-E-194-064-MY3 and
NSC 98-2221-E-194-027-MY3 from the National Science Council, Taiwan,
R.O.C.


\begin{thebibliography}{99}

\bibitem{als99} S.~Alstrup, D.~Harel, P.~W. Lauridsen, and M.~Thorup, Dominators in linear time, SIAM J. Comput., 28(6) (1999), 2117--2132.

\bibitem{bar07} S.~C. Barman, S.~Mondal, and M.~Pal, An efficient algorithm to find next-to-shortest path on trapezoid
graphs, Adv. Appl. Math. Anal., 2 (2007), 97--107.

\bibitem{cor01}
T.~H. Cormen, C.~E. Leiserson, R.~L. Rivest, and C.~Stein,
Introduction to Algorithms, Second ed., MIT Press, 2001.

\bibitem{for56} L.~R. Ford and D.~R. Fulkerson, Maximal flow through a network, Canad. J. Math., 8 (1956), 399--404.

\bibitem{fred87}
M.~L. Fredman and R.~E. Tarjan, Fibonacci heaps and their
uses in improved network optimization algorithms, J. ACM, 34
(1987), 209--221.

\bibitem{har85}
D.~Harel, A linear time algorithm for finding dominators
in flow graphs and related problems, the 17th Annu. ACM Symp.
Theory Comput. (1985), 185--194.

\bibitem{hen97}
M.~R. Henzinger, P.~Klein, S.~Rao, and S.~Subramanian, 
Faster shortest-path algorithms for planar graphs, J. Comput.
System Sci., 53 (1997), 2--23.

\bibitem{kao11}
K.-H. Kao, J.-M. Chang, Y.-L. Wang, and J.~S.-T. Juan, A
quadratic algorithm for finding next-to-shortest paths in graphs,
Algorithmica, to appear, doi:10.1007/s00453-010-9402-4, 2011.

\bibitem{kra04} I.~Krasiko and S.~D. Noble, Finding next-to-shortest paths in a graph, Inform. Process. Lett., 92 (2004), 117--119.

\bibitem{lal97} K.~N. Lalgudi and M.~C. Papaefthymiou, Computing strictly-second shortest paths, Inform. Process. Lett., 63 (1997), 177--181.

\bibitem{len79}
T.~Lengauer and R.~Tarjan, A fast algorithm for finding
dominators in a flowgraph, ACM Trans. Programming Lang. Syst., 1
(1979), 121--141.

\bibitem{li06} S.~Li, G.~Sun, and G.~Chen, Improved algorithm for finding next-to-shortest paths, Inform. Process. Lett., 99 (2006), 192--194.

\bibitem{lor69} E.~S. Lorry and C.~W. Medock, Object code optimization, Comm. ACM, 12 (1969), 13--22.

\bibitem{mod06} S.~Mondal and M.~Pal, A sequential algorithm to solve next-to-shortest path problem on circular-arc graphs, J. Phys. Sci., 10 (2006), 201--217.

\bibitem{pur72}
P.~W. Purdom and E.~F. Moor, Algorithm 430: Immediate
predominators in a directed graph, Comm. ACM, 15 (1972),
777--778.

\bibitem{tar74}
R.~Tarjan, Finding dominators in directed graphs, SIAM
J. Comput., 3 (1974), 62--89.

\bibitem{thr99} M.~Thorup, Undirected single-source shortest paths with positive integer weights in linear time, J. ACM, 46 (1999), 362--394.

\bibitem{wu10} B.~Y. Wu, A simpler and more efficient algorithm for the
next-to-shortest path problem, in Proceeding of the 4th Annual International Conference on Combinatorial Optimization and Applications (COCOA 2010), LNCS 6509, 219--227. An improved and full version of this paper is available by arXiv: 1105.0608v1.

\end{thebibliography}
\end{document}
