 \documentclass{article} 


\usepackage{epstopdf}




\usepackage{amsmath}
\usepackage{amssymb}
\usepackage{amsfonts}
\usepackage{authblk}


\usepackage{array}
\usepackage{color}





\usepackage{multirow}
\usepackage{rotating}






\newcommand{\cA}{{\cal A}}
\newcommand{\cB}{{\cal B}}
\newcommand{\cC}{{\cal C}}
\newcommand{\cG}{{\cal G}}
\newcommand{\cN}{{\cal N}}
\newcommand{\cU}{{\cal U}}
\newcommand{\cT}{{\cal T}}
\newcommand{\cS}{{\cal S}}
\newcommand{\cL}{{\cal L}}
\newcommand{\cV}{{\cal V}}


\newcommand*{\qed}{\hfill\ensuremath{\square}}


\newtheorem{theorem}{Theorem}[section]
\newtheorem{lemma}{Lemma}[section]
\newtheorem{corollary}{Corollary}[section]
\newtheorem{claim}{Claim}[section]
\newtheorem{proposition}{Proposition}[section]
\newtheorem{definition}{Definition}[section]
\newtheorem{fact}{Fact}[section]
\newtheorem{example}{Example}[section]
\newenvironment{proof}{\noindent{\bf Proof:}}{\qed}

\newcommand{\algo}[1]{
\medskip
\noindent \textbf{Algorithm {\tt #1}}\\
\nopagebreak}



\newcommand{\ai}{\alpha_i}
\newcommand{\bi}{\beta_i}
\newcommand{\gi}{\gamma_i}
\newcommand{\di}{\delta_i}

\newcommand{\oai}{\overline{\alpha}_i}
\newcommand{\obi}{\overline{\beta}_i}
\newcommand{\ogi}{\overline{\gamma}_i}
\newcommand{\odi}{\overline{\delta}_i}

\newcommand{\red}[1]{{\color{red}#1}}






\newcommand{\procedure}[1]{
\medskip
\noindent \textbf{Procedure {\tt #1}}
\nopagebreak \\}

\newcommand{\procend}{\hfill \medskip}

\newcommand{\oddRepair}{{\tt Odd\-Repair}}
\newcommand{\Deactivate}{{\tt Deactivate}}
\newcommand{\evenARepair}{{\tt Even\-Al\-most\-Re\-pair}}
\newcommand{\ringThree}{{\tt Ring\-Three\-Co\-lo\-ring}}
\newcommand{\ringLearning}{{\tt Ring\-Lear\-ning}}
\newcommand{\Elect}{{\tt Elect}}


\begin{document}

\title{
Topology recognition with advice\thanks{This research was done during the visit of Andrzej Pelc at  Sapienza, University of Rome,  partially supported by a visiting fellowship from this university. A preliminary version of this paper appeared in the
Proceedings of the 27th International Symposium on Distributed Computing (DISC 2013).}
}

\author[1]{Emanuele G. Fusco}
\author[2]{Andrzej Pelc\footnote{Partially supported by NSERC discovery grant 8136--2013 
and by the Research Chair in Distributed Computing at the
Universit\'e du Qu\'ebec en Outaouais.}}
\author[3]{Rossella Petreschi}
\affil[1]{Department of Computer, Control, and Management Engineering  Antonio Ruberti,
Sapienza, University of Rome,
00185 Rome, Italy.
{\tt fusco@diag.uniroma1.it}.}
\affil[2]{
 D\'epartement d'informatique, Universit\'e du Qu\'ebec en Outaouais, Gatineau,
Qu\'ebec J8X 3X7, Canada. {\tt pelc@uqo.ca}. }
\affil[3]{Computer Science Department,
Sapienza, University of Rome,
00198 Rome, Italy.
{\tt petreschi@di.uniroma1.it}.}
\date{}
\maketitle

\begin{abstract}

In topology recognition,  each node of an anonymous network has to deterministically produce an isomorphic copy of
the underlying graph, with all ports correctly marked. This task is usually unfeasible without any a priori information.
Such information can be provided to nodes as advice. An oracle knowing the network can give a (possibly different)
string of bits to each node, and all nodes must reconstruct the network using this advice, after a given number of rounds
of communication. During each round each node can exchange arbitrary messages with all its neighbors and perform
arbitrary local computations. The time of completing topology recognition is the number of rounds it takes, and the size
of advice is the maximum length of a string given to nodes.

We investigate tradeoffs between the time in which topology recognition is accomplished and the minimum
size of advice that has to be given  to nodes. We provide upper and lower bounds on the minimum size of advice
that is sufficient to perform topology recognition in a given time, in the class of all graphs of size  and diameter
, for any constant . In most cases, our bounds are asymptotically tight.\end{abstract} 


\begin{center}
{\bf Keywords}\\
topology recognition, network, advice, time, tradeoff
\end{center}


\section{Introduction}


\subsection{The model and the problem}

Learning an unknown network by its nodes is one of the fundamental distributed tasks in networks.
Once nodes acquire a faithful labeled map of the network, any other 
distributed task, such as leader election \cite{HS,P}, minimum weight spanning tree construction \cite{A}, 
renaming \cite{ABDKPR}, etc. can be performed by nodes using
only local computations. Knowing topology also simplifies the task of routing. 
More generally, constructing a labeled map converts all distributed problems to centralized ones, 
in the sense that nodes can solve them simulating a central monitor.

If nodes are a priori equipped with unique identifiers, they can deterministically 
construct a labeled map of the network, by exchanging messages, without any additional  information about the network. However,
even if nodes have unique identities, 
relying on them 
for the task of learning the network is not always possible. Indeed, nodes may be reluctant to reveal their identities for security or privacy reasons. 
Hence it is
important to design algorithms reconstructing the topology of the network without assuming any node labels, i.e., for anonymous networks. In this paper we are interested in deterministic solutions.

 Ports at each node of degree  are 
arbitrarily numbered , and there is no assumed coherence between port numbers at different nodes. A node is aware of its degree, and it knows on which port it sends or receives a message. 
The goal is, for each node, to get an isomorphic copy of the graph underlying the
network, with all port numbers correctly  marked. (Port numbers are very useful, e.g., to perform the routing task.)
There are two variants of this task:
a weaker version, that we call
{\em anonymous topology recognition}, in which the nodes of the reconstructed map are unlabeled,
and a stronger version, that we call {\em labeled topology recognition}, in which all nodes construct 
a map of the network assigning distinct labels to all nodes in the same way, and know their position in this map.
Even anonymous topology recognition is not always feasible without any a priori information given to nodes, as witnessed, e.g.,  by the class
of oriented rings in which ports at each node are numbered 0,1 in clockwise order. No amount of information exchange can help nodes to recognize the size of the oriented ring and hence to reconstruct
correctly its topology. Thus, in order to accomplish (even anonymous) topology recognition for arbitrary networks, some information must be provided to nodes. This can be done in the form of {\em advice}. An oracle knowing the network gives a (possibly different) string of bits to each node. Then nodes execute
a deterministic distributed algorithm that does not assume knowledge of the network, but uses message exchange and the advice provided
by the oracle to nodes, in order to reconstruct the topology of the network by each of its nodes. 

In this paper we study tradeoffs between the {\em size of advice} provided to nodes and the {\em time} of topology recognition. The size of advice is defined as the length of the longest string of bits
given by the oracle to nodes. For communication,
we use the extensively studied  model  \cite{Pe}. 
In this model, communication proceeds in synchronous rounds
and all nodes start simultaneously. In each round each node
can exchange arbitrary messages with all its neighbors and perform arbitrary local computations. 
The time of completing a task is the number of rounds it takes. 
The central question of the paper is:

\begin{quotation}
What is the minimum size of advice that enables (anonymous or labeled) topology recognition in a given time ,
in the class of -node networks of diameter ?
\end{quotation}

It should be stressed that the reason why we adopt the  model is to focus on how deeply nodes should probe the network in order to discover the topology. Since 
nodes are anonymous, this depth of probing may often go beyond the diameter of the network, enabling the nodes to ``see'' other nodes several times, along different paths, and deduce some information
from these deeper views. At the same time, as it is always done when using the  model  \cite{Pe}, the size of messages  is ignored.  

\begin{table}
\begin{small}
\begin{center}
   \begin{tabular}{ccc}
 \cline{1-3} 
 
 \multicolumn{2}{|c}{ \multirow{2}{*}{{\sc Time}}} &\multicolumn{1}{|c|}{\multirow{2}{*} {{\sc Optimal Size of Advice} }} \cr 
\multicolumn{2}{|c}{ } &\multicolumn{1}{|c|}{ }\cr  \cline{1-3} 
\multicolumn{2}{|c}{ \multirow{3}{*}{{}}} &\multicolumn{1}{|c|}{\multirow{2}{*} {{} }} \cr 
\multicolumn{2}{|c}{ \multirow{1}{*}{{}}} &\multicolumn{1}{|c|}{\multirow{2}{*} {{[Proposition \ref{ubTime2D+1}]} }} \cr 
\multicolumn{2}{|c}{ } &\multicolumn{1}{|c|}{ }\cr  \cline{1-3} 
\multicolumn{2}{|c}{ \multirow{2}{*}{{ , }}} &\multicolumn{1}{|c|}{\multirow{2}{*} {{Upper B.: , Lower B.: 1} }} \cr
 \multicolumn{1}{|c}{ \multirow{2}{*}{{ for  }}}& \multicolumn{1}{c}{\multirow{1}{*}{{}}} &\multicolumn{1}{|c|}{\multirow{2}{*} {{[Theorem \ref{ubTimeD+k}]} }} \cr 
\multicolumn{1}{|c}{ }& &\multicolumn{1}{|c|}{ }\cr  \cline{1-3} 
\multicolumn{2}{|c}{ \multirow{2}{*}{{ , }}} &\multicolumn{1}{|c|}{\multirow{2}{*} {{} }} \cr 
 \multicolumn{2}{|c}{ \multirow{2}{*}{{ for  }}} &\multicolumn{1}{|c|}{\multirow{2}{*} {{[Corollary \ref{corD+k}]} }} \cr 
\multicolumn{2}{|c}{ } &\multicolumn{1}{|c|}{ }\cr  \cline{1-3} 
\multicolumn{2}{|c}{ \multirow{3}{*}{{}}} &\multicolumn{1}{|c|}{\multirow{2}{*} {{} }} \cr 
\multicolumn{2}{|c}{ \multirow{1}{*}{{}}} &\multicolumn{1}{|c|}{\multirow{2}{*} {{[Corollary \ref{corD}]} }} \cr 
\multicolumn{2}{|c}{ } &\multicolumn{1}{|c|}{ }\cr  \cline{1-3} 
\multicolumn{2}{|c}{ \multirow{2}{*}{{,}}} &\multicolumn{1}{|c|}{\multirow{2}{*} {{} }} \cr 
 \multicolumn{2}{|c}{ \multirow{2}{*}{{for }}} &\multicolumn{1}{|c|}{\multirow{2}{*} {{[Corollary \ref{corD-k}]} }} \cr 
\multicolumn{2}{|c}{ } &\multicolumn{1}{|c|}{ }\cr  \cline{1-3} 
\end{tabular}
 
\caption{\label{summary} The summary of results. All our bounds are tight, except for those in the line with the  when .}
\end{center}
\end{small}
\end{table}

\subsection{Our results}
We provide upper and lower bounds on the minimum size of advice sufficient to perform topology recognition in a given time,
in the class  of all graphs of size  and diameter , for any constant .
All our upper bounds are valid even for the harder task of labeled topology recognition, while our lower bounds also apply to the easier task of anonymous topology recognition. Hence we will only use the term {\em topology recognition} for all our results.
We prove upper bounds  on the minimum size of advice sufficient to perform topology recognition in a given time , for the class , by providing an assignment of advice of size  and an algorithm, using this advice, that accomplishes this task, within time , for any network in .
We prove lower bounds on the minimum size of advice, sufficient for a given time , by constructing graphs in  for which topology recognition within this time is impossible with advice of a smaller size. (Notice that, while our upper bounds would hold even if , our lower bound constructions rely on the constraint .)

The meaningful span of possible times for topology recognition is  between 0 and 
. Indeed, while  advice of size  permits topology recognition in time 0 (i.e., without communication), we show that topology recognition in time   can be done with advice of size 1, which is optimal. 

For most values of the allotted time, our bounds are asymptotically tight.
This should be compared to many results from the literature on the advice paradigm (see, e.g.,~\cite{CFIKP,FGIP,FIP1,FKL,SN}), which often either consider the size of advice  needed for feasibility of a given task, or only give isolated points in the curve 
of tradeoffs between resources (such as time) and the size of advice.

We show that, if the allotted time is , where ,  then the optimal size of advice is  .
If the allotted time is , then this optimal size is .
If the allotted time is , where ,  then the optimal size of advice is .
The only remaining gap between our bounds is for time ,  where .
In this time interval our upper bound remains  , while the lower bound (that holds for any time) is 1. This leaves a gap  if .
See Table~\ref{summary} for a summary of our results.


Our results show how sensitive is the minimum
size of advice to the time allowed for topology recognition: allowing just one round more, from 
to , decreases exponentially the advice needed to accomplish this task.
Our tight bounds on the minimum size of advice also show a somewhat surprising fact that the amount of information that nodes need to reconstruct a labeled map of the network, in a given time, and that needed to reconstruct an anonymous map of the network in this time, are asymptotically the same in most cases.



\subsection{Related work}
{Many papers \cite{AKM01,CFP,CFIKP,DP,EFKR,FGIP,FIP1,FIP2,FKL,FP,GPPR02,IKP,KKKP02,KKP05,SN,TZ05} considered the problem of increasing the efficiency of network tasks by providing nodes with some information of arbitrary kind.}
This approach was referred to as
algorithms using {\em informative labeling schemes}, or equivalently, algorithms with {\em advice}.  
Advice is given either to nodes of the network or to mobile agents performing some network task.
Several authors studied the minimum size of advice required to solve the
respective network problem in an efficient way. Thus the framework of advice permits to quantify the amount of information
needed for an efficient solution of a given network problem, regardless of the type of information that is provided.

In \cite{CFIKP} the authors investigated the minimum size of advice that has to be given to nodes
to permit graph exploration by a robot. 
 In \cite{KKP05}, given a distributed representation of a solution for a problem,
the authors investigated the number of bits of communication needed to verify the legality of the represented solution.
In \cite{FIP1} the authors compared the minimum size of advice required to
solve two information dissemination problems using a linear number of messages. In \cite{FIP2} the authors
established the size of advice needed to break competitive ratio 2 of an exploration algorithm in trees.
In \cite{FKL} it was shown that advice of constant size permits to carry on the distributed construction of a minimum
spanning tree in logarithmic time. 
In \cite{EFKR} the advice paradigm was used for online problems.
In \cite{FGIP} the authors established lower bounds on the size of advice 
needed to beat time 
for 3-coloring of a cycle and to achieve time  for 3-coloring of unoriented trees.  
In the case of \cite{SN} the issue was not efficiency but feasibility: it
was shown that  is the minimum size of advice
required to perform monotone connected graph clearing.
In \cite{IKP} the authors studied radio networks for
which it is possible to perform centralized broadcasting in constant time. They proved that
 bits of advice allow to obtain constant time in such networks, while
 bits are not enough. 

Distributed computation on anonymous networks has been investigated by many authors, e.g.,
\cite{An,ASW,BV,EPSW,ESW,FP2, KKV,Pe,YK3}
for problems ranging from leader election to computing boolean functions
and communication in wireless networks. 
In \cite{EPSW} the authors compared randomized and deterministic algorithms to solve distributed problems in anonymous networks.
In \cite{ESW} a hierarchy of distributed problems in anonymous networks was studied.
Feasibility of topology recognition for anonymous graphs with adversarial port labelings was studied in~\cite{YK3}.
The problem of efficiency of map construction by a mobile agent, equipped with a token, exploring an anonymous graph  has
been studied in \cite{CDK}. In \cite{DP} the authors investigated the minimum size of advice
that has to be given to a mobile agent, in order to enable it to reconstruct  the topology of an anonymous network or to construct its spanning tree.
Notice that the mobile agent scenario makes the problem of map construction much different from our setting. Since all the advice is given to a single agent,
breaking symmetry may be very hard. Even anonymous map construction often requires providing a large amount of information to the agent, regardless of the exploration time. 
To the best of our knowledge, tradeoffs between time and the size of advice for topology recognition have never been studied before.

\section{Preliminaries}
Unless otherwise stated, we use the word {\em graph} to mean a simple undirected connected graph without node labels, and with ports at each node of degree  labeled . 
Two graphs  and  are {\em isomorphic}, if and only if, there exists a bijection  such that the edge
, with port numbers  at  and  at  is in , if and only if, the edge  with port numbers  at  and  at  is in .

The size of a graph is the number of its nodes. Throughout the paper we consider a fixed positive constant  and the class of graphs of size  and diameter . We use  to denote the logarithm to the base .
For a graph , a node  in , and any integer , we denote by  the set of nodes in  at distance at most  from .


We will use the following notion from \cite{YK3}. 
The {\em view} from node  in graph  is the infinite tree   rooted at  with unlabeled nodes and labeled ports, whose branches are infinite sequences of port numbers coding all infinite paths in the graph, starting from node . The {\em truncated view}  is the truncation of this tree to depth . to level , for each .
 
Given a graph , a function  is called a {\em decoration} of .
Notice that an assignment of advice to nodes of  is a decoration of .
For a given decoration  of a graph  we define the {\em decorated graph}  as follows. Nodes of  are ordered pairs , for all nodes  in .
 has an edge  with port numbers  at  and  at , if and only if,
 contains the edge , with port numbers  at  and  at .

We define  the {\em decorated view} at depth  of node  in , according to , as the truncated view at depth  of node  in the decorated graph . Nodes in the decorated view are labeled with the values assigned to the corresponding nodes of  by the decorating function .

The following two lemmas will be used in the proofs of our upper bounds.

\begin{lemma}\label{lemDominating}
Let  be a graph and let  be a positive integer. There exists a set  of nodes in   satisfying the following conditions. 
\begin{itemize}
\item For any node  of  there exists a node  in  such that the distance between  and  is at most .
\item For each pair  of distinct nodes in , the distance between  and  is larger than .
\end{itemize}
\end{lemma}
\begin{proof}
The set proving the lemma is any independent dominating set of the th power of the graph .
\end{proof}

\begin{lemma}\label{lemUniqueLabels}
Let  be a graph of diameter  and let  be an injective  decoration of . Then each node  in  can accomplish topology recognition using its view, decorated according to , at depth , even without knowing  a priori.
\end{lemma}
\begin{proof}
Let  be the minimum integer such that the views at depth   and , decorated according to , contain the same set of values.
Then the view at depth  contains all nodes and all edges of the graph , and thus it is sufficient to reconstruct the graph  and its decoration .
The lemma follows from the fact that the view at depth  contains all nodes of the graph.  
\end{proof}

The following proposition can be easily proved by induction on the round number.
Intuitively it says that, if two nodes executing the same algorithm have the same decorated views at depth , then they behave identically for at least  rounds.

\begin{proposition}\label{propSameHistory}
Let  and  be two graphs, let  be a node of  and let  be a node of . Let  be a decoration of  and let  be a decoration of .
Let  be any topology recognition algorithm.
Let  be the set of triples , where  is the message received by node  in round  through port  when executing algorithm  on the graph , decorated according to . Let  be defined as , but for , , and  instead of , , and .

If the view of  at depth , decorated according to  is the same as the view of  at depth , decorated according to , then .
\end{proposition}

We will use the above proposition to prove our lower bounds as follows.
If the size of advice is too small, then there are two non-isomorphic graphs  and  resulting, for some node  in  and some node  in , in the same decorated view at the depth equal to the time available to perform topology recognition.
Hence either  or  must incorrectly reconstruct the topology (even anonymous) of  or .


\section{Time  }


We start our analysis by constructing a topology recognition algorithm that works in time  and uses advice of size~1.
Since we will show that, for arbitrary , there are networks in which topology recognition without advice is impossible in any time,
this shows that the meaningful time-span to consider for topology recognition is between  and .


\algo{TR-1}
\noindent{\bf Advice:}\\
The oracle assigns bit 1 to one node (call it ), and bit 0 to all others. Let  be this assignment of advice.

\noindent{\bf Node protocol:}\\
In round , each node  sends its view at depth , decorated according to , to all its neighbors;
it receives such views from all its neighbors and constructs its view at depth , decorated according to .
This task continues until termination of the algorithm.

Let  be the smallest round number at which node  sees a node with advice  in its view decorated according to   (at depth ).
Node  assigns to itself a label in round  as follows.
The label  is the lexicographically smallest shortest path, defined as a sequence of consecutive port numbers  (each traversed edge corresponds to a pair of port numbers), from  to any node with advice 1, in its decorated view at depth .
(Notice that since there can be many shortest paths between  and , this node can appear many times in the decorated view at depth  of .)
Let  be the decoration corresponding to the labeling  obtained as above.
(We will show that labels in  are unique.)

After round , node  starts constructing its decorated view, according to decoration . 
In any round , node  sends both its view, decorated according to , at depth , and its view, decorated according to , at the largest possible depth. Messages required to perform this task are piggybacked to those used for constructing views, decorated according to , at increasing depths.
In each round , node  checks for newly discovered values of .
As soon as there are no new values, node  reconstructs the labeled map and outputs it.
Then node  computes the diameter  of the resulting graph and continues to send its views, decorated according to  and according to  , at increasing depths, until round .  After round  node  terminates.
\procend


\begin{proposition}\label{ubTime2D+1}
Algorithm {\tt TR-1} completes topology recognition for all graphs of size  and diameter  in time , using advice of size 1.
\end{proposition}
\begin{proof}
We first prove that the topology reconstructed at each node is correct.

Uniqueness of labels assigned according to    follows from the fact that  the node  with advice  is unique and, for each node ,
the lexicographically smallest shortest path between  and  is unique.
Since, in a graph of diameter , every node is at distance at most  from , each node  computes its label  within  rounds.

Notice that some nodes will acquire their unique label earlier than others. However, since each node acquires its complete decorated view (according to ) at depth  within  rounds after all nodes at distance at most  from it have acquired their unique label, after  rounds of communication all nodes have their decorated view (according to ) at depth .
By making each node continue to participate in the construction of decorated views at increasing depths until round  (even after it reconstructed the topology), the algorithm guarantees that {\em all} nodes will be able to get their views, decorated according to , at depth . Correctness of the reconstructed topology follows from Lemma~\ref{lemUniqueLabels}. Since all nodes stop in round  and the advice provided to each node consists of 1 bit, the proof is complete.
\end{proof}

The following proposition, cf. \cite{YK3}, shows that advice of size 1, as used by Algorithm~{\tt TR-1}, is necessary, regardless of the allotted time.
As opposed to the -node rings mentioned in the introduction as graphs that require at least one bit of advice, but whose diameter is , the class of graphs we will use to prove the proposition allows greater flexibility of the diameter.

\begin{proposition}\label{lb2D+1}
Let  and let  be an even integer.
The size of advice needed to perform topology recognition for the class of all graphs of size  and diameter  is at least 1.
\end{proposition}

\begin{proof}
Consider the two graphs  and  in Figure~\ref{fig.yk}.a that where constructed in~\cite{YK3}. They have the following properties:
\begin{itemize}
\item they are non-isomorphic;
\item both of them have size 6 and diameter 3;
\item all black nodes in both graphs have the same view;
\item the black nodes are adjacent in both graphs.
\end{itemize}

\begin{figure}
\centering

\begin{tabular}{cc}
\cline{1-2}
\multicolumn{1}{|c|}{
\scalebox{0.24}{\begin{picture}(0,0)\includegraphics{G_1.pdf}\end{picture}\setlength{\unitlength}{3947sp}\begingroup\makeatletter\ifx\SetFigFont\undefined \gdef\SetFigFont#1#2#3#4#5{\reset@font\fontsize{#1}{#2pt}\fontfamily{#3}\fontseries{#4}\fontshape{#5}\selectfont}\fi\endgroup \begin{picture}(10802,6002)(0,-6362)
\put(376,-3661){\makebox(0,0)[lb]{\smash{{\SetFigFont{41}{49.2}{\rmdefault}{\mddefault}{\updefault}{\color[rgb]{0,0,0}}}}}}
\end{picture} }
} & \multicolumn{1}{c|}{
\scalebox{0.24}{\begin{picture}(0,0)\includegraphics{G_1_prime.pdf}\end{picture}\setlength{\unitlength}{3947sp}\begingroup\makeatletter\ifx\SetFigFont\undefined \gdef\SetFigFont#1#2#3#4#5{\reset@font\fontsize{#1}{#2pt}\fontfamily{#3}\fontseries{#4}\fontshape{#5}\selectfont}\fi\endgroup \begin{picture}(10802,6002)(0,-6362)
\put(376,-3661){\makebox(0,0)[lb]{\smash{{\SetFigFont{41}{49.2}{\rmdefault}{\mddefault}{\updefault}{\color[rgb]{0,0,0}}}}}}
\end{picture} }
}\\


\multicolumn{1}{|c|}{
\scalebox{0.24}{\begin{picture}(0,0)\includegraphics{G_2.pdf}\end{picture}\setlength{\unitlength}{3947sp}\begingroup\makeatletter\ifx\SetFigFont\undefined \gdef\SetFigFont#1#2#3#4#5{\reset@font\fontsize{#1}{#2pt}\fontfamily{#3}\fontseries{#4}\fontshape{#5}\selectfont}\fi\endgroup \begin{picture}(10802,6232)(0,-6592)
\put(5401,-6361){\makebox(0,0)[b]{\smash{{\SetFigFont{41}{49.2}{\rmdefault}{\mddefault}{\updefault}{\color[rgb]{0,0,0}(a)}}}}}
\put(376,-3661){\makebox(0,0)[lb]{\smash{{\SetFigFont{41}{49.2}{\rmdefault}{\mddefault}{\updefault}{\color[rgb]{0,0,0}}}}}}
\end{picture} }
} & \multicolumn{1}{c|}{
\scalebox{0.24}{\begin{picture}(0,0)\includegraphics{G_2_prime.pdf}\end{picture}\setlength{\unitlength}{3947sp}\begingroup\makeatletter\ifx\SetFigFont\undefined \gdef\SetFigFont#1#2#3#4#5{\reset@font\fontsize{#1}{#2pt}\fontfamily{#3}\fontseries{#4}\fontshape{#5}\selectfont}\fi\endgroup \begin{picture}(10802,6232)(0,-6592)
\put(5401,-6361){\makebox(0,0)[b]{\smash{{\SetFigFont{41}{49.2}{\rmdefault}{\mddefault}{\updefault}{\color[rgb]{0,0,0}(b)}}}}}
\put(376,-3661){\makebox(0,0)[lb]{\smash{{\SetFigFont{41}{49.2}{\rmdefault}{\mddefault}{\updefault}{\color[rgb]{0,0,0}}}}}}
\end{picture} }
}\\

\cline{1-2}
\multicolumn{2}{|c|}{
\scalebox{0.24}{\begin{picture}(0,0)\includegraphics{H.pdf}\end{picture}\setlength{\unitlength}{3947sp}\begingroup\makeatletter\ifx\SetFigFont\undefined \gdef\SetFigFont#1#2#3#4#5{\reset@font\fontsize{#1}{#2pt}\fontfamily{#3}\fontseries{#4}\fontshape{#5}\selectfont}\fi\endgroup \begin{picture}(10502,6082)(900,-7042)
\put(6151,-6811){\makebox(0,0)[b]{\smash{{\SetFigFont{41}{49.2}{\rmdefault}{\mddefault}{\updefault}{\color[rgb]{0,0,0}(c)}}}}}
\put(4426,-1636){\makebox(0,0)[lb]{\smash{{\SetFigFont{29}{34.8}{\rmdefault}{\mddefault}{\updefault}{\color[rgb]{0,0,0} or }}}}}
\end{picture} }
}\\
\cline{1-2}

\end{tabular}


\caption{\label{fig.yk} a) Graphs  and  from~\cite{YK3}. b) Graphs  and . c) Schematic representation of graphs  and , in which the rectangle represents one of the graphs , , , or .  }
\end{figure}


The graphs  and  in Figure~\ref{fig.yk}.b are obtained from   and , respectively, replacing the edge connecting the two black nodes by two paths of length 2.
These graphs have the following properties:
\begin{itemize}
\item they are non-isomorphic;
\item both of them have size 8 and diameter at most 4;
\item all black nodes in both graphs have the same view;
\item the black nodes are at distance 2 in both graphs.
\end{itemize}
For  and even  we construct two graphs  and , of size  and diameter , as follows.
If  is odd and , we obtain  by  connecting two identical copies of a path of length  and of a clique of size  to each black node of the graph , in a symmetric fashion.
We obtain  by doing the same for graph .
If  is even and , we obtain  by  connecting two identical copies of a path of length  and of a clique of size  to each black node of the graph , in a symmetric fashion.
We obtain  by doing the same for graph . (See Figure~\ref{fig.yk}.c.)

Notice that graphs  and  have diameter  and size , they are non-isomorphic both for  odd and for  even, and all black nodes in both graphs have the same view. Hence, in view of Proposition~\ref{propSameHistory}, black nodes cannot correctly perform (even anonymous) topology recognition in  or in  without any advice. 
\end{proof}






\section{Time above }
In this section we study the size of advice sufficient to perform topology recognition in arbitrary time larger than  (but smaller than ), i.e., large enough for allowing each node to see all nodes and edges of the graph.
We first give an algorithm using advice of size  that performs topology recognition in time .

\algo{TR-2}
\noindent{\bf Advice:}\\
Let  be a graph of size  and diameter . Let .

If  then the oracle gives a unique label of size  as advice to each node.

Suppose that .
The oracle picks a set of nodes  satisfying Lemma~\ref{lemDominating}, for .
Then it chooses a unique label  from the set  for each node  in .
For any node  let  be the lexicographically smallest shortest path (coded as a sequence of consecutive port numbers) from  to .
Sort the nodes  in  in the increasing lexicographic order of .
The binary representation of  is partitioned into  consecutive segments, each of length at most .
The oracle assigns the first segment, with a trailing bit ,  as advice to node .
For , the -th segment, with a trailing bit 0, is assigned as advice to the -th  node of . (Notice that some nodes in  could receive only the trailing bit 0 as advice.) All other nodes get the empty string as advice.
Let  be the above assignment of advice.

\noindent{\bf Node protocol:}\\
We first describe the protocol when .
In round , each node  sends its view at depth , decorated according to , to all its neighbors;
it receives such views from all its neighbors and constructs its view at depth , decorated according to .
This task continues until termination of the algorithm.

Each node  whose advice has a trailing bit  assigns to itself a temporary label  as follows.
Let  be the smallest round number at which node  sees a node with advice with a trailing bit  in its view decorated according to   (at depth ).
The label  is the lexicographically smallest shortest path, defined as a sequence of consecutive port numbers, from  to any node with advice with a trailing bit 1 in its  view, decorated according to , at depth .

Let  be a node whose advice has a trailing bit . After reconstructing its label , node  sends  to the node  closest to it, along the lexicographically smallest shortest path that determined label . Nodes along this path relay these messages piggybacking them to any message that they should send in a given round.


Each  node  (having a trailing bit 1 in its advice) computes  as the first depth in which its view, decorated according to  contains nodes without any advice.
In round  each such node reconstructs its label  from messages  it received (which it sorts in the increasing lexicographic order of ), and from .

Let  be the decoration of  where each node  in the set  is mapped to the binary representation of its unique label , and each node outside of  is mapped to the empty string.

Nodes outside of   start constructing their decorated view, according to . This construction is put on hold by a node  in  until the time when it reconstructs its unique label .
Upon reconstructing its label , each node  starts constructing its view decorated according to , hence allowing its neighbors to construct their view, decorated according to , at depth 1.
This process continues for  steps, during which nodes construct and send their views at increasing depth, decorated according to .

Each node  assigns a label  to itself as follows.
Let  be the smallest depth at which the view of , decorated according to , contains a node  with label  and let  be the lexicographically smallest path connecting  to such a node  (coded as a sequence of consecutive port numbers). Node  sets .



Let  be the decoration of  where each node  is mapped to .
(We will prove that  is an injective function.)
Upon computing its value in  each node starts constructing its decorated view, according to .
In each round , node  checks for newly discovered values of .
As soon as there are no new values, node  reconstructs the labeled map and outputs it.
Then node  computes the diameter  of the resulting graph and continues to send its views, decorated according to , according to  , and according to , at increasing depths, until round .  After round  node  terminates.

If , the protocol consists only of the last step described above, with decoration  replaced by the assignment of advice given to nodes by the oracle.
\procend

\begin{theorem}\label{ubTimeD+k}
Let .
Algorithm {\tt TR-2} completes topology recognition for all graphs of size  and diameter  within time , using advice of size .
\end{theorem}
\begin{proof}
Let  be a graph of size  and diameter . If , then  is constant and each node receives advice of size . In this case the proof follows from Lemma~\ref{lemUniqueLabels}.
Hence in the sequel we can assume that .

Let  be the set of nodes satisfying Lemma~\ref{lemDominating}, for ,  selected by the oracle.
Due to the properties of the set , the sets  and consequently the sets 
are pairwise disjoint for distinct nodes .
Hence, no node would receive more than one segment of a label , for some node .

Within round , each node  receives the complete set of segments ,  for , of its label .
Moreover, it can reconstruct the order of the segments according to the lexicographic order of temporary labels  received together with the corresponding segments. 
Hence, by round , all nodes in  know their unique label .

Since each node of  is at distance at most  from some node in  and  is an injective function for nodes in , the view decorated according to  at depth  of each node contains some uniquely labeled node.
Consequently,  can be computed by each node  within  rounds.
The decoration  is an injective function, due to uniqueness of labels  and due to the fact that, if nodes  used the same node  to compute  and  , respectively, then .
In view of  Lemma~\ref{lemUniqueLabels}, all nodes accomplish (labeled) topology reconstruction within additional  rounds after all nodes have computed their value according to decoration . Hence the algorithm terminates within time .

Since the size of the set   is at least , the size of advice is at most 
, which completes the proof of the theorem.
\end{proof}

We now provide a lower bound on the minimum size of advice sufficient to perform topology recognition. This bound matches the upper bound given by Algorithm~{\tt TR-2} in the time-interval .

\begin{theorem}\label{lbD+k}
Let  and .
The size of advice needed to perform topology recognition in time  in the class of graphs of size  and diameter 
is in .
\end{theorem}
\begin{proof}
Our lower bound will be proved using the following classes  of graphs of size  and diameter , called {\it brooms}.
We define these classes for  sufficiently large and for . {(For  Proposition~\ref{lb2D+1} applies.)}
Nodes in a  broom  are partitioned into three sets, called the {\em bristles}, the {\em stick}, and the {\em handle}.
Let  be the largest even integer such that .

The set  bristles consists of  nodes, partitioned into  pairwise disjoint sets , each of size . The stick consists of  nodes, and the remaining  nodes are in the handle.
Hence the bristles, the stick, and the handle are non-empty sets.
We now describe the set of edges in each part.

Edges of the bristles are partitioned into two sets,  and . 
Edges in  connect nodes of each set  into a path with port numbers 0 and 1 at each edge. We call  {\it head} of each set  the endpoint of the path to which port number 1 has been assigned, and {\it tail} the other  endpoint (to which port number 0 has been assigned).
Notice that sets  can be of size 1, in which case heads coincide with tails.
Edges in  form a perfect matching  among tails of the bristles. These edges have port number 1 at both endpoints. 

Edges of the stick form a path of length  with port numbers 0 and 1 at each edge. (Notice that this path is of length 0, i.e., the stick consists of a single node, when .) 
The handle has no edges.


The bristles, the stick, and the handle are connected as follows.
Let  be the endpoint of the stick to which port number 1 has been assigned, and let  be the other endpoint of the stick (to which port number 0 has been assigned). Nodes  and  coincide when .
Node  is connected to the head of each set  by an edge with port numbers  at  and 0 at each head.
Node  is connected to each node in the handle. Port numbers at  corresponding to these connecting edges are numbered , if , and 
, if . Nodes in the handle are of degree 1, so they have a unique port with number 0. See Figure~\ref{fig.fan} for an example of a broom in .
Notice that all brooms in  are defined over the same set of nodes and share the same edges, apart from those in sets forming perfect matchings among tails of the bristles. Moreover notice that growing the length of the bristles above  would result in a graph of diameter larger than  {(due to the distance between any pair of unmatched tails of the bristles)}, which explains the assumption .

\begin{figure}
\centering
\includegraphics[scale=0.45]{broom.pdf}
\caption{\label{fig.fan} A broom in .}
\end{figure}

For two brooms  and  in  we define {\em corresponding} nodes as follows. Let  and  in  and , respectively, be the nodes in the handles whose only incident edge has port number 0 at both endpoints.
Node {} corresponds to node {}, if and only if, the (unique) shortest path (defined as the sequence of  port numbers on consecutive edges) from  to  is the same as the shortest path from  to .

The idea of the proof is to show that if the size of advice is smaller than , for a sufficiently small constant , then there exist two brooms in , whose corresponding nodes receive the same advice, for which the decorated view at depth  of each node in the handle remains the same in both brooms. 
Since different brooms are non-isomorphic, this will imply the theorem, in view of Proposition~\ref{propSameHistory}.

Observe that for , we have that  is constant, and   is a lower bound on the size of advice for topology recognition, regardless of the allowed time.
Hence we do not need to define brooms when  to prove the theorem.


We now provide a lower bound on the size of the class . 
This size depends on the number  of tails of the bristles among which perfect matchings  can be defined.
For given  and , the size of the class  cannot increase when  grows. 
Hence the class is smallest for the largest considered value of , i.e., .
We do the estimation for this value of .

The number of perfect matchings among tails is at least .




Suppose, from now on, that the size of advice is bounded by , for some constant .
Then there are at most  ways of assigning advice to nodes of a broom in . 
Hence there are at least  brooms in  for which corresponding nodes get the same advice. Fix one such assignment  of advice.


We now provide an upper bound on the number of distinct decorated views, at depth , of any node in the handle, when advice is assigned to nodes according to .
Consider two brooms  and  in , decorated according to assignment .
Let  and  be the perfect matchings among tails of the bristles corresponding to brooms  and , respectively.
 and  result in distinct decorated views, at depth , of corresponding nodes in the handle, if and only if, there exist corresponding tails of the bristles  and , such that the decorated path , whose tail   is matched to  in  and the decorated path , whose tail   is  matched to  in , are different.
The number of distinct decorated paths  of length  is at most .
Since , it follows that there are at most  distinct decorated views, at depth , for any node in the handle, for assignment .

We will show that the following inequality

which we denote by (*),
holds for , when  is sufficiently large. 
 Indeed, for sufficiently large  we have ; in view of , taking the logarithms of both sides we have 
 

Inequality (*) implies that 
. Hence the number of brooms from , decorated according to assignment , exceeds the number of distinct decorated views, at depth , of any node in the handle, for these brooms. It follows {from the pigeonhole principle}  that some decorated view corresponds to different brooms from . In view of Proposition~\ref{propSameHistory}, this proves that (even anonymous) topology recognition in time , for the class of graphs of diameter  and size , requires advice of size at least .
\end{proof}

Since the lower bound  on the size of advice holds regardless of time, 
theorems~\ref{ubTimeD+k} and \ref{lbD+k} imply the following corollary.

\begin{corollary}\label{corD+k}
Let  and .
The minimum size of advice sufficient to perform topology recognition in time  in the class of graphs of size  and diameter  is in . 
\end{corollary}
\section{Time }


In this section we provide asymptotically tight upper and lower bounds on the minimum size of advice sufficient to perform topology recognition in time equal to the diameter  of the network.
Together with the upper bound proved in Theorem~\ref{ubTimeD+k}, applied to time , these bounds show an exponential gap in the minimum size of advice due to time difference of only one round.

\algo{TR-3}
\noindent{\bf Advice:}\\
The oracle assigns a unique label  from the set  to each node .
The advice given to each node  consists of the diameter , the label , and the collection of all  edges incident to , coded as quadruples , where  is the port number at node  corresponding to edge , and  is the port number at node  corresponding to this edge.

\noindent{\bf Node protocol:}\\
In round , each node sends to all its neighbors the collection of edges learned in all previous rounds. 
After  rounds of communication each node reconstructs the topology and stops.
\procend

\begin{proposition}\label{ubTimeD}
Algorithm {\tt TR-3} completes topology recognition for all graphs of size  and diameter  in time , using advice of size .
\end{proposition}
\begin{proof}
Since each node is within distance  from all others, each node learns the whole collection of edges of the graph within  rounds.
Since these edges are coded using unique labels of adjacent nodes and contain port numbers, this is enough to reconstruct the (labeled) topology.

Since, in a graph of size , the diameter  can be coded with  bits and there are  edges incident to a node, each edge being coded with  bits, the size of advice is .
\end{proof}




The following lemma will be used for our lower bound.

\begin{lemma}\label{cliques}
There are at least { non-isomorphic cliques of size .}\end{lemma}
\begin{proof}
{Assume first that the nodes are labeled . For each node  there are  ways to assign ports at it. These assignments are independent for each node, hence there are  distinct assignments of port numbers.
Since from one clique without node labels it is possible to obtain at most  distinct cliques with nodes labeled , our lower bound follows.}
\end{proof}

We define the following classes  of graphs of size  and diameter , called {\it lollipops}.
These graphs will be used to prove our lower bounds for time  and below.
Nodes in a  lollipop  are partitioned into two sets, called the {\em candy} and the {\em stick}.
The candy consists of  nodes; for the purpose of describing our construction we will call these nodes . The stick consists of the remaining  nodes {(see fig.~\ref{fig.lollipop} for an example of a lollipop in )}. 

\begin{figure}
\centering
\includegraphics[scale=0.45]{lollipop.pdf}
\caption{\label{fig.lollipop} A lollipop in .}
\end{figure}

Nodes in the candy are connected to form a clique; port numbers for these edges are assigned arbitrarily from the set . 
Edges of the stick form a path of length  with port numbers 0 and 1 at each edge. 

The stick and the candy are connected as follows.
Let  be the endpoint of the stick to which port number 1 has been assigned and let  be the other endpoint of the stick (to which port number 0 has been assigned). Notice that  and  coincide, when .
Node  is connected to all nodes in the candy. The port number, at node , corresponding to edge   is , if . For  this port number is , when  and , when . The port number, at all nodes , corresponding to edge  is .


Since, for , the size of the candy of a lollipop in  is at least , Lemma~\ref{cliques} implies the following corollary:


\begin{corollary}\label{corLollipops}
The size of the class , for  is at least {.}
\end{corollary}

\begin{theorem}\label{lbD}
Let .
The size of advice needed to perform topology recognition in time  in the class of graphs of size  and diameter 
is in .
\end{theorem}
\begin{proof}
If we consider a lollipop of diameter , then there are  edges of the candy that are outside of the view at depth  of the endpoint  of the stick. 
The idea of the proof is based on the fact that  information about these edges has to be assigned to nodes of the graph as advice that will become available to  within time .


First observe that the view at depth  of the endpoint  of the stick is the same for all lollipops in .
Hence, if the size of advice is at most , then the number of distinct decorated views of node  is at most .
We will show that, if {,}
then, for sufficiently large , the number of lollipops in  exceeds this bound.
Indeed,  by Corollary~\ref{corLollipops}, the size of the class  is at least
{}
For sufficiently large  we have
\begin{small} 
{

}
\end{small}
It is enough to show that 
{}
which is immediate to verify by taking the logarithm of both sides.

It follows that the same decorated view at depth  of node  corresponds to different lollipops from . In view of Proposition~\ref{propSameHistory}, this proves that (even anonymous) topology recognition in time , for the class of graphs of size  and diameter , requires advice of size at least 
{.}
\end{proof}

Proposition~\ref{ubTimeD} and Theorem \ref{lbD} imply the following corollary.
\begin{corollary}\label{corD}
Let .
The minimum size of advice sufficient to perform topology recognition in time  in the class of graphs of size  and diameter 
is in .
\end{corollary}


\section{Time below }
In this section we study the minimum size of advice sufficient to perform topology recognition when the time allotted for this task is too short, for some node, to communicate with all other nodes in the network.



\algo{TR-4}
\noindent{\bf Advice:}\\
Let  be a graph of size  and diameter .
The oracle assigns a unique label  from  to each node  in the graph .
It codes all edges of the graph as quadruples , where  and  are the labels of two adjacent nodes  and ,  is the port number at node  corresponding to edge , and  is the port number at node  corresponding to this edge.
Let  be the set of all these codes.

Let {, for , be the time available to complete the topology recognition}.
If , then the advice provided by  the oracle to each node  is: , the collection  of all edges, and the integer .


If , then the oracle picks a set  of nodes in   satisfying Lemma~\ref{lemDominating}, for .
{For each node , l}et . Moreover, let  be a partition of the edges in  into  pairwise disjoint sets of sizes differing by at most 1. 
Let  be an enumeration of nodes in .
The advice given by the oracle to node  consists of the label , of the set , and of the integer .
Every other node  only gets  and  as advice.
Let  be the resulting assignment of advice.

\noindent{\bf Node protocol:}\\
Let  be the integer received by all nodes as part of their advice.

In round , with , each node sends to all its neighbors the collection of edges learned in all previous rounds. 
(In particular, if , then there is no communication.)
After  rounds of communication each node reconstructs the topology and stops.
\procend

\begin{theorem}\label{ubTimeD-k}
Let .
Algorithm {\tt TR-4} completes topology recognition for all graphs of size  and diameter  within time , using advice of size .
\end{theorem}
\begin{proof}
If  then  is constant. In this case each node uses the collection  of edges received as advice to reconstruct the topology in time .
Since there are  edges in a graph of size , and each edge can be coded with  bits, the size of advice is .

Hence we can assume that .
Since each node is at distance at most  from some node in , and the decorations of the nodes in , for each  such node , assigned according to , collectively contain the entire set , after time  each node of the graph can reconstruct the whole (labeled) topology.
It remains to be shown that the size of advice is in . Indeed,  and the size of each set  is at most . Each edge can be coded with  bits and sets , for , are disjoint.
The theorem follows from the fact that .
\end{proof}



The following lower bound shows that the size of advice used by Algorithm~{\tt TR-4} is asymptotically optimal.

\begin{theorem}\label{lbD-k}
Let  and .
The size of advice needed to perform topology recognition in time  in the class of graphs of size  and diameter 
is in .
\end{theorem}
\begin{proof}
The argument follows closely the proof of Theorem~\ref{lbD}. The only difference is that we consider decorated views of the endpoint  of the stick at depth  instead of . Since there are only  nodes in the lollipop at distance  from , if the  size of advice is at most  , then the number of distinct decorated views of node  at depth  is at most  , for sufficiently large .

The same computation as in the proof of Theorem~\ref{lbD} shows that, if , then the same decorated view at depth  of node  corresponds to different lollipops from . In view of Proposition~\ref{propSameHistory}, this proves that (even anonymous) topology recognition in time , for the class of graphs of diameter  and size , requires advice of size at least .
\end{proof}

Theorems~\ref{ubTimeD-k} and \ref{lbD-k} imply the following corollary.
\begin{corollary}\label{corD-k}
Let   and .
The minimum size of advice sufficient to perform topology recognition in time  in the class of graphs of size  and diameter 
is in .
\end{corollary}


\section{Conclusion and open problems}
We presented upper and lower bounds on the minimum size of advice sufficient to perform topology recognition, in a given time , in -node networks of diameter . 
Our bounds are asymptotically tight for time  and, if  for some constant , in the time interval .
Moreover, in the remaining time interval  our bounds are still asymptotically tight if .
Closing the remaining gap between the lower bound  and the upper bound  in this remaining time interval, for graphs of very small diameter , is a natural open problem. In particular, it would be interesting to find the minimum time in which topology recognition can be accomplished using advice of constant size, or even of size exactly 1.

Other open problems remain in the case of networks with very large diameter, those which do not satisfy the assumption  for some constant , or equivalently those for which .
Our upper bounds do not change in this case (we did not use the assumption  in their analysis), while our lower bounds change as follows, using the same constructions.
The lower bound for time above , i.e., when , where , becomes ;
our lower bound for time  becomes ;
the lower bound for time below , i.e., when , where , becomes . It remains to close the gaps between these lower bounds and the upper bounds that we gave for each allotted time.

Another open problem is related to the communication model used. As mentioned in the Introduction, we chose the  model which ignores the size of messages. It would be interesting to 
study how the results change in the  model in which messages are limited to logarithmic size. Some compression of the information transmitted should be possible, e.g., along the lines of \cite{Ta}.

Let us also address the issue of node identities vs. advice given to nodes.
We did our study for unlabeled networks, arguing that nodes may be reluctant to disclose their identities for security or privacy reasons.
As we have seen, however, for anonymous networks some advice has to be given to nodes, regardless of the allotted time.
Does the oracle have to provide new distinct labels to nodes?
Our results show that for time above  this is not the case, as the minimum size of advice enabling topology recognition in this time is too small for assigning a unique identifier to each node.
Hence, in spite of not having been given, a priori, unique identifiers, nodes can perform labeled topology recognition in this time span.
On the other hand for time at most , the minimum size of advice is sufficiently large to provide distinct identifiers to nodes, and indeed our oracles inserted unique identifiers as part of advice. However, this should not raise concerns about security or privacy, as these identifiers may be arbitrary and hence should be considered as ``nicknames'' temporarily assigned to nodes. 

A different set of problems would appear if (unlike in our scenario) randomization were allowed. For many time values the size of required advice
would then be drastically reduced. For example, for time , advice of size  would suffice. Indeed, providing nodes with
 the diameter  and the size  of the network would enable them to first acquire
different labels with high probability (the size , or some bound on it, is needed to know how long the random labels should be, but such labels can be
randomly produced with no communication),
and then learn the topology of the resulting labeled graph in  rounds. We leave the analysis of tradeoffs between time and advice for randomized topology 
recognition as an open problem.  
 
\bibliographystyle{plain}
\begin{thebibliography}{99}


\bibitem{AKM01}
S.~Abiteboul, H.~Kaplan, T.~Milo, Compact labeling schemes for ancestor
queries, Proc. 12th Annual ACM-SIAM Symposium on Discrete
Algorithms (SODA 2001), 547--556.

\bibitem{An}
D.~Angluin, Local and global properties in networks of processors. 
Proc. 12th Annual ACM Symposium on Theory of Computing (STOC 1980), 82--93.

\bibitem{ASW}
H. Attiya, M. Snir, M. Warmuth,
Computing on an anonymous ring,
Journal of the ACM 35, (1988), 845--875.

\bibitem{ABDKPR}
H. Attiya, A. Bar-Noy, D. Dolev, D. Koller, D. Peleg, R. Reischuk,
Renaming in an asynchronous environment,  Journal of the ACM 37 (1990), 524--548.

\bibitem{A}
B. Awerbuch, Optimal distributed algorithms for minimum weight spanning tree, 
counting, leader election and related problems,
Proc. 19th Annual ACM Symposium on Theory of Computing (STOC 1987), 230--240.


\bibitem{BV}
P. Boldi, S. Vigna,
Computing anonymously with arbitrary knowledge,
Proc. 18th ACM Symposium on Principles of Distributed Computing (PODC 1999), 181--188.

\bibitem{CFP}
S. Caminiti, I. Finocchi, R. Petreschi,
Engineering tree labeling schemes: a case study on least common ancestor,
Proc. 16th Annual European Symposium on Algorithms (ESA 2008), 234--245.


\bibitem{CDK}
J. Chalopin, S. Das, A. Kosowski, 
Constructing a map of an anonymous graph: Applications of universal sequences,
Proc. 14th International Conference on Principles of Distributed Systems (OPODIS 2010), 119--134.


\bibitem{CFIKP}
R. Cohen, P. Fraigniaud, D. Ilcinkas, A. Korman, D. Peleg, 
Label-guided graph exploration by a finite automaton, 
ACM Transactions on Algorithms 4 (2008).


\bibitem{DP}
D. Dereniowski, A. Pelc, Drawing maps with advice,  Journal of Parallel and Distributed Computing 72 (2012), 132--143. 



\bibitem{EFKR}
Y. Emek, P. Fraigniaud, A. Korman, A. Rosen, Online computation with advice, Theoretical Computer Science 412 (2011), 2642--2656.

\bibitem{EPSW}
Y. Emek, C. Pfister, J. Seidel, R. Wattenhofer, Anonymous networks: Randomization= 2-hop coloring,
Proc. 32nd ACM Symposium on Principles of Distributed Computing (PODC 2014), 96--105.

\bibitem{ESW}
Y. Emek, J. Seidel, R. Wattenhofer, Computability in anonymous networks: Revocable vs. irrecovable outputs,
Proc. 41th International Colloquium on Automata, Languages and Programming (ICALP 2014), 183-195.


\bibitem{FGIP}
P. Fraigniaud, C. Gavoille, D. Ilcinkas, A. Pelc, 
Distributed computing with advice: Information sensitivity of graph coloring, 
Distributed Computing 21 (2009), 395--403.

\bibitem{FIP1}
P. Fraigniaud, D. Ilcinkas, A. Pelc, 
Communication algorithms with advice, Journal of  Computer and System Sciences 76 (2010), 222--232.


\bibitem{FIP2}
P. Fraigniaud, D. Ilcinkas, A. Pelc, 
Tree exploration with advice, Information and Computation 206 (2008), 1276--1287.


\bibitem{FKL}
P. Fraigniaud, A. Korman, E. Lebhar,
Local MST computation with short advice,
Theory of Computing Systems 47 (2010), 920--933.


\bibitem{FP}
E. Fusco, A. Pelc, Trade-offs between the size of advice and broadcasting time in trees, Algorithmica 60 (2011), 719--734. 

\bibitem{FP2}
E. Fusco, A. Pelc, How much memory is needed for leader election, Distributed Computing 24 (2011), 65--78.

\bibitem{GPPR02}
C.~Gavoille, D.~Peleg, S.~P\'{e}rennes, R.~Raz.
Distance labeling in graphs, 
Journal of Algorithms 53 (2004), 85-112.


\bibitem{HS}
D.S. Hirschberg, J.B. Sinclair,
Decentralized extrema-finding in circular configurations of processes,
 Communications of the ACM 23 (1980), 627--628.


\bibitem{IKP}
D. Ilcinkas, D. Kowalski, A. Pelc, 
Fast radio broadcasting with advice, 
 Theoretical Computer Science, 411 (2012),  1544--1557.

\bibitem{KKKP02}
M.~Katz, N.~Katz, A.~Korman, D.~Peleg, Labeling schemes for flow and
connectivity, 
SIAM Journal of  Computing 34 (2004), 23--40.



\bibitem{KKP05}
A. Korman, S. Kutten, D. Peleg, Proof labeling schemes,
Distributed Computing 22 (2010), 215--233.  



\bibitem{KKV}
E. Kranakis, D. Krizanc,  J. van der Berg,
Computing Boolean functions on anonymous networks,
Information and Computation 114 (1994), 214--236.


\bibitem{SN}
N. Nisse, D. Soguet, Graph searching with advice,
Theoretical Computer Science 410 (2009), 1307--1318.

 \bibitem{Pe}D. Peleg,
  Distributed computing, a locality-sensitive approach,
  SIAM Monographs on Discrete Mathematics and Applications, Philadelphia 2000.

\bibitem{P}
G.L. Peterson, An  unidirectional distributed algorithm
for the circular extrema problem,
 ACM Transactions on Programming Languages and Systems 4 (1982), 758--762.
 
 \bibitem{Ta}
 S. Tani,  Compression of view on anonymous networks - Folded view -. IEEE Trans. Parallel Distrib. Syst. 23 (2012), 255-262.


\bibitem{TZ05}
M.~Thorup, U.~Zwick, Approximate distance oracles,
Journal of the ACM, 52 (2005), 1--24.


\bibitem{YK3}
M. Yamashita, T. Kameda,
Computing on anonymous networks: Part I - characterizing the solvable cases,
IEEE Transactions on Parallel and Distributed Systems 7 (1996), 69--89. 

\end{thebibliography}



\end{document}