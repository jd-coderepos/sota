

\documentclass[conference]{IEEEtran}
\usepackage{graphicx}
\usepackage{color}
\usepackage{multirow}
\usepackage{listings}
\usepackage{url}
\usepackage{balance}

\lstset{language=java, numbers=left, numberstyle=\tiny\color{black}, numbersep=3pt, escapeinside={\^{*,\circ}^{\circ,\dagger}^{*,\ddagger}^*^\dagger^{*,\circ}^{*,\ddagger}^{\circ,\dagger}^*^\circ^\dagger^\ddagger\timesO(10^6)\simnjjnjt\sim\sim7.4\times2.2\times\times\sim1.1\times1.3\times2.3\times2.5\times2.2\times13.4\times1.4\times15.9\times12.5\times2.4\times2.5\times$ faster than C over GlusterFS when running on a cluster of 64 m2.4xlarge Amazon
EC2 instances. Kira SE also has comparable performance (between 18.5\% slower and 75.1\% faster)
to the C version running on the NERSC Edison supercomputer.  

We also demonstrated that Apache Spark can integrate with existing libraries.
This allows users to reuse existing source code to build new analysis pipelines.
A flexible interface, rich dataflow support, task scheduling capacity, locality optimization, and built-in support for fault tolerance make Spark a 
strong candidate to support many-task scientific applications. 
We experimented with Apache Spark as a popular example of a Big Data platform. We learned that
leveraging such a platform would enable scientists to benefit from the rapid pace of innovation 
and large range of systems and technologies that are being driven by wide-spread interest in Big Data analytics.

\section{Acknowledgments}

This research is supported in part by NSF CISE Expeditions Award CCF-1139158, DOE Award SN10040 DE-SC0012463, and DARPA XData Award FA8750-12-2-0331, and gifts from Amazon Web Services, Google, IBM, SAP, The Thomas and Stacey Siebel Foundation, Adatao, Adobe, Apple, Inc., Blue Goji, Bosch, C3Energy, Cisco, Cray, Cloudera, EMC2, Ericsson, Facebook, Guavus, HP, Huawei, Informatica, Intel, Microsoft, NetApp, Pivotal, Samsung, Schlumberger, Splunk, Virdata and VMware. Author Frank Austin Nothaft is supported by a National Science Foundation Graduate Research Fellowship.

This research is also supported in part by the Gordon and Betty Moore
Foundation and the Alfred P. Sloan Foundation together through the
Moore-Sloan Data Science Environment program.

This research used resources of the National Energy Research Scientific Computing Center, a DOE Office of Science User Facility supported by the Office of Science of the U.S. Department of Energy under Contract No. DE-AC02-05CH11231.
\bibliographystyle{abbrv}
\balance
\bibliography{Kira} 





\end{document}
