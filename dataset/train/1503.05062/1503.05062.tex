\documentclass[A4]{article}
\usepackage{graphicx}
\usepackage{amssymb}
\usepackage{amsmath}
\usepackage{amsthm}
\usepackage{color}
\usepackage{fullpage}

\graphicspath{{figures/}}

\newtheorem{theorem}{Theorem}
\newtheorem{lemma}[theorem]{Lemma}
\newtheorem{proposition}[theorem]{Proposition}
\newtheorem{observation}[theorem]{Observation}


\usepackage{ wasysym }
\newcommand{\THqed}{\hspace*{\fill}{\large\sun}}
\renewcommand{\qed}{\THqed}


\usepackage[usenames,dvipsnames]{xcolor}
\newcommand{\todo}[2][ToDo]{\textcolor{red}{*** \textsc{#1:} #2 ***}} \newcommand{\tom}[2][says]{\textcolor{CarnationPink}{\textsc{Tom #1:} #2}}
\newcommand{\ruy}[2][says]{\textcolor{blue}{\textsc{Ruy #1:} #2}}
\newcommand{\figurenames}{\figurename{s}}




\begin{document}


\title{Modem Illumination of Monotone Polygons\thanks{A preliminary version of this work has been presented at EuroCG'09~\cite{Aich}.\newline
O. Aichholzer and B. Vogtenhuber partially supported by the ESF EUROCORES programme EuroGIGA -- CRP `ComPoSe', Austrian Science Fund (FWF): I648-N18. R. Fabila-Monroy partially supported by CONACyT (Mexico), grant 153984. D. Flores-Pe\~naloza partially supported by CONACyT (Mexico), grant 168277, and PAPIIT IA102513 (UNAM, Mexico). T. Hackl supported by the Austrian Science Fund (FWF): P23629-N18 `Combinatorial Problems on Geometric Graphs'. J. Urrutia partially supported by CONACyT (Mexico) grant CB-2007/80268.
  }
 }


\author{Oswin Aichholzer\thanks{Institute for Software Technology, Graz University of Technology, Graz, Austria, \texttt{[oaich|thackl|apilz|bvogt]@ist.tugraz.at}.}
\and Ruy Fabila-Monroy\thanks{Departamento de Matem\'aticas, Cinvestav, D.F. M\'exico, M\'exico, \texttt{ruyfabila@math.cinvestav.edu.mx}.}
\and David Flores-Pe\~naloza\thanks{Departamento de Matem\'aticas, Facultad de Ciencias, Universidad Nacional Aut\'onoma de M\'exico, D.F. M\'exico, M\'exico, \texttt{dflorespenaloza@gmail.com}.}
\and Thomas Hackl
\and Jorge Urrutia\thanks{Instituto de Matem\'aticas, Universidad Nacional Aut\'onoma de M\'exico, D.F. M\'exico, M\'exico, \texttt{urrutia@matem.unam.mx}.}
\and Birgit Vogtenhuber
}


\maketitle


\begin{abstract}
  We study a generalization of the classical problem of the illumination of polygons.
  Instead of modeling a light source we model a wireless device whose radio signal can penetrate a given number~ of walls.
  We call these objects \mbox{-modems} and study the minimum number of \mbox{-modems} sufficient and sometimes necessary to illuminate monotone and monotone orthogonal polygons.
  We show that every monotone polygon with~ vertices can be illuminated with  \mbox{-modems}.
  In addition, we exhibit examples of monotone polygons requiring at least  \mbox{-modems} to be illuminated.

  For monotone orthogonal polygons with~ vertices we show that for~ and for even~, every such polygon can be illuminated with  \mbox{-modems}, while for odd~,  -modems are always sufficient.
  Further, by presenting according examples of monotone orthogonal polygons, we show that both bounds are tight.
\end{abstract}



\section{Introduction}

New technologies inspire new research problems, and wireless networking is a typical example of this.
Nowadays, wireless technologies surround us everywhere.
We use them in devices such as cellular phones, satellite communications, and, in our homes, we use wireless modems to connect to the Internet.

This has triggered, among other things, the development of a new class of algorithms designed specifically to work with wireless networks, such as cellular networks, sensor networks, and \emph{ad-hoc} networks~\cite{Kran2,Urrutia2,Watten}.
The development of GPS, also a byproduct of wireless technologies, has allowed the development of so-called ``local algorithms" for routing problems in cellular and \emph{ad-hoc} networks~\cite{BMSU,Krana,Urrutia2} that allow relayed communication between any two nodes ~and~ of a network, at any time using  only the position of ~and~, as well as the current position of a message while traveling from~~to~.
For more details see~\cite{Urrutia2,Watten}.

In this paper we study what we call the \emph{Modem Illumination Problem}.
This problem stems from our daily use of laptop computers and wireless modems.
Experience shows that when trying to connect a laptop to a wireless modem, there are two factors that have to be considered: the \emph{distance} to the wireless modem and, perhaps most important in many buildings, the \emph{number of walls} separating our laptop from the wireless modem.
From now on, the term \emph{modem} will be used to refer to a wireless modem.
We call a modem \emph{a -modem} if it is strong enough to transmit a stable signal through~ walls along a straight line.
Thus, we say that a point~ in a polygon~ is illuminated by a \mbox{-modem}~ in~ if the line segment joining~ to~ \emph{crosses} at most~ walls (edges) of~.\\

\noindent \emph{\bf{The Modem Illumination Problem}: }
Let~ be a student center (f.k.a. art gallery) modeled by a polygon~ with~ vertices.
How many \mbox{-modems} located at points of~ are always sufficient, and sometimes necessary, to illuminate all points in~?\\

We point out that we allow a modem to be located at a point~ on an edge~ (or even on a vertex~) of~.
In this case, we do not consider~ (or the two edges incident to~) as a barrier for the modem.
That is, the line segments connecting~ and any point of  do not cross~ (or the two edges incident to~).
Thus, if~ is an interior point of~, the line segment connecting~ and~ may cross an odd number of edges of .


For  our problem corresponds to Chv\'atal's Art Gallery Theorem~\cite{Chvatal} which states that  watchmen are always sufficient and sometimes necessary to guard an art gallery with  walls.
Many generalizations of the original Art Gallery problem have been studied, see~\cite{ORourke,Shermer,Urrutia} for comprehensive surveys.
The modem illumination problem was introduced in~\cite{Aich,Ruy}.

A similar problem, posed by Urrutia and solved by Fulek et al.~\cite{Pach}, is the following:
What is the smallest number  such that in any collection of  pairwise disjoint convex sets in the -dimensional Euclidean space, there is a point such that any ray emanating from it meets at most  sets of the collection?
In our language, they proved that one \mbox{-modem} is always sufficient to illuminate the -dimensional Euclidean space in presence of~ convex obstacles.

In~\cite{Ball}, several variations of the Modem Illumination Problem are studied.
For example they present upper and lower bounds on the number of \mbox{-modems} needed to illuminate the plane in the presence of obstacles modeled by line segments (with fixed slopes), or nested polygons.
They also present some bounds on illuminating special classes of simple polygons with \mbox{-modems} in the interior of the polygon.

Polygon illumination with wireless devices has also been studied in a slightly different context,  the so-called sculpture garden problem~\cite{EGS,CHOU}.  
There, each device only broadcasts a signal within a given angle of the polygon and has unbounded range.
The task is to describe the polygon (that is, distinguish it from the exterior) by a combination of the devices, meaning that for each point~ in the interior of the polygon no point outside the polygon receives signals from the same devices as~. 

We remark that the general case of the Modem Illumination Problem for \mbox{-modems} is widely open.
Almost no tight bounds are known so far for any class of polygons for any ; 
see for example the recent Column of O'Rourke~\cite{ORourke2}.
Very recently, there has been some development on these problems in several directions.
Duque and Hidalgo~\cite{DH} have announced
an upper bound of  on the number of  \mbox{-modems}
needed to illuminate the interior of a simple polygon of  sides.
In the case of orthogonal polygons they give a tighter bound of .
For the related question on ``edge-transmitters'' (modems where the signal is emanated from a whole edge instead of only a point; see also~\cite{Shermer}), Cannon et al.~\cite{edgetrans} consider several classes of polygons.
They show lower (i.e., sometimes necessary) and upper (i.e., always sufficient) bounds on the number of needed transmitters for simple (orthogonal) (monotone) polygons.
For example, for monotone polygons they prove  and  as lower and upper bounds, and they give a tight bound of  for monotone orthogonal polygons.
Finally, in~\cite{np}, the same authors study the corresponding algorithmic optimization problems for both, point and edge transmitters.
In particular, they show that it is NP-hard to compute the minimum number of \mbox{-modems} needed to illuminate a given simple polygon. 



In this paper we provide lower and upper bounds for the Modem Illumination Problem for monotone polygons and monotone orthogonal polygons, which are a reasonable model of most real life buildings.
These bounds are asymptotically tight (i.e., the lower and upper bounds match):
 \mbox{-modems} for monotone polygons;
 \mbox{-modems} for monotone orthogonal polygons for~ and even~;
and  \mbox{-modems} for monotone orthogonal polygons for~odd~.
(Note that these bounds improve the bounds from the preliminary version~\cite{Aich}.)

\section{Illumination of (general) monotone polygons with -modems}
\label{sec:gmono}

To keep things as simple as possible we introduce several conventions on monotone polygons.
When we speak of a polygon, we refer to both the boundary and the interior of the polygon.
For technical reasons and without loss of generality, we make the following assumption: for a non-ortho\-gonal monotone polygon, we assume that no two of its edges (on different chains) are parallel. 


For any monotone polygon, we assume that the direction of monotonicity is the \mbox{-axis}, and that no two vertices have the same -coordinate.
We denote a (monotone) polygon~ with~ vertices as (monotone) \mbox{-gon}.
Further, we denote the vertices of~ by , where the labels are given to the vertices with respect to their -sorted order, such that  is the leftmost and  is the rightmost vertex of~.
Note that this implies that, in general, the vertices are not labeled with respect to their order along the boundary of~. 


To validate possible \mbox{-modem} positions we will consider rays from these positions.
For a point  of  let  be the set of all rays starting at .
Observe that at the last edge~ of~ that is intersected by a ray~,  leaves~ and never enters~ again.
In other words, no line segment connecting~ with a point~ on~ is crossing~.
Hence, a \mbox{-modem} does not have to overcome  in the direction of  to fully illuminate .
If a \mbox{-modem} at  illuminates~ (in the direction of a ray ), then we say that~ is a \emph{valid} \mbox{-modem} position for~ (in the direction of~).


\begin{observation}\label{obs:rayk+1}
For any point  of  and each ,  is a valid \mbox{-modem} position for  in the direction of  if and only if  intersects at most  edges of~.

A point~ is a valid \mbox{-modem} position for~ if and only if every  intersects at most  edges of~.
\end{observation}

If  is a point on the boundary~ of~ then we distinguish two subsets of rays.
Let  be the subset of rays that start to the outside of~ and let  be the subset of rays that start to the inside of~.

\begin{observation}\label{obs:rayio}
For any point~ of~, each ray in  intersects an even number of edges of~ and each ray  intersects an odd number of edges of~.
\end{observation}


Recall that, if a possible \mbox{-modem} position~ is on an edge (or vertex) of~, then no ray in  crosses that edge (or the two edges incident to that vertex).
With this in mind we state a few simple observations on the size of polygons that can be fully illuminated by a single \mbox{-modem}.

\begin{proposition}\label{prop:anywhere}
Every -gon~ can be illuminated with a \mbox{-modem} placed anywhere in the interior or on the boundary of~.
\end{proposition}
\begin{proof}
  Let  be the point of~ where the \mbox{-modem} is placed.
  If~ is on the boundary of~, then at least one edge of~ is not crossed by any ray in .
  If~ is in the interior of~, then for each  at least one edge of~ is crossed by the ray starting in the opposite direction of  from .
Therefore, each ray in  crosses at most  edges out of the  edges of~.
By Observation~\ref{obs:rayk+1}, a \mbox{-modem} at~ illuminates the whole polygon. \end{proof}

Note that this proposition will not be used in this paper.
It nevertheless is of interest, as it is also true for general (not necessarily monotone) simple polygons.
The next lemma is true for general simple polygons too, and will be needed later on.

\begin{lemma}\label{lem:vertex}
Every -gon~ can be illuminated with a \mbox{-modem} placed on any vertex of~.
\end{lemma}
\begin{proof}
  Let  be the vertex of~ where the \mbox{-modem} is placed.
  No ray in  crosses any of the two edges incident to~.
Therefore, each ray in  crosses at most  edges out of the  edges of~.
By Observation~\ref{obs:rayk+1}, a \mbox{-modem} at~ illuminates the whole polygon. \end{proof}

For every monotone (-monotone by convention) polygon~, every vertical line~ cuts~ into at most two parts, because  is either one connected component or empty.
Let~ be the (closed) half-plane bounded to the right by~.
Following suit, let~ be the (closed) half-plane bounded to the left by~.
Note that  (or ) is to the left (or right) of~ (including~).

Hence, for a vertical line~ that intersects~, we call  and  the left and right part of~, respectively.
We say that  (or ) contains an edge~ of~ if  (or ) is neither empty nor a single point.
Note that at least one edge of~ is always split by a vertical line that intersects~, because, by convention, no two vertices of~ have the same -coordinate.

Strictly speaking,  and  are not polygons.
Their rightmost and leftmost edge, respectively, is missing.
We could fix this by adding the line segment  to both parts of~ as an (auxiliary) edge.
But this would violate the convention that no two vertices share the same -coordinate.
To maintain this convention we can slightly perturb the end point of~ that is no vertex of~ along its edge of~ without changing the situation.
Furthermore, we say that  (or ) is illuminated if the polygon  (or ) would be illuminated.
The following statement summarizes observations about splitting~. 

\begin{observation}\label{obs:simplesplit}
Let~ be an -monotone -gon and let  be a vertical line through , .
\begin{itemize}\vspace*{-0.5ex}
\item  contains  edges of~.
\item  contains  edges of~.
\item If both  and  are illuminated, then~ is illuminated.
\end{itemize}
\end{observation}


Combining this simple splitting with Lemma~\ref{lem:vertex}, we can state the following:


\begin{proposition}\label{prop:2k+3}
Every -monotone -gon~ can be illuminated with a \mbox{-modem} placed on .
\end{proposition}
\begin{proof}
  Let , let  be the vertical line through~, and let~.
The polygon  contains  edges of~ (Observation~\ref{obs:simplesplit}) plus the (auxiliary) edge~.\footnote{More exactly, the boundary of the polygon  consists of the first  edges of  plus the part of the  edge of  that is to the left of  plus .}
  The polygon  contains  edges of~ (Observation~\ref{obs:simplesplit}) plus the (auxiliary) edge~.
By Lemma~\ref{lem:vertex},  is a valid \mbox{-modem} position for both -gons.
Hence, by Observation~\ref{obs:simplesplit}, a \mbox{-modem} at~ illuminates the whole polygon~. 
\end{proof}

Combining Observation~\ref{obs:simplesplit} and Proposition~\ref{prop:2k+3} we can derive a first bound for the number of \mbox{-modems}.
It is easy to see that we can split each -monotone -gon into a (left) -monotone -gon and a (right) -monotone -gon using a vertical line~ through  (and the auxiliary edge~).
Recursively splitting the (right) -monotone -gon directly leads to a (non-optimal) upper bound of  \mbox{-modems} to illuminate a monotone -gon.

We improve this bound in the remainder of this section, which is divided into three steps.
First we increase the size of a polygon that we can guarantee to illuminate with a single \mbox{-modem}.
Then we introduce an efficient way to split a large polygon into smaller subpolygons.
And in the final step we combine the two previous steps to an upper bound on the Modem Illumination Problem and provide matching lower bound examples.



\subsection{Illuminating monotone polygons with a single -modem}
\label{sec:gmono-single}


Before we can improve Proposition~\ref{prop:2k+3}, we need a solid basis of naming conventions.
Similar as before, let  be a vertex of a monotone -gon~, let  be the vertical line through~, and let  be the edge of~ crossed by~, with~ being the point of this intersection.
Let  and  be the left and right end point of~, respectively.
Let  be the edge sharing  with  and let  be the edge sharing  with .
Let  and  be the left and right neighbor of  on , respectively.
Let  and .
Observe that these naming conventions induce an ``-side'' and an ``-side'' of~.

\begin{figure}[htb]
  \centering
  \includegraphics{denomination}
  \caption{Example to illustrate the naming conventions for a monotone -gon.}
  \label{fig:naming_convention}
\end{figure}

Let  and  be the vertical lines through  and , respectively.
Let  and  be the intersection of  with  and , respectively.
Similarly, let  and  be the intersection of  with~ and~, respectively.
Note that either  or  is  and either~ or~ is~.
Thus, for each of  and , one of the intersections is not a proper crossing with the respective edge, because the line passes through a vertex of~.
Further note that  is either  or , and that  is either  or .

For simplicity and without loss of generality, let~ be above~.
Hence, the ``-side'' of~ is the upper side of~ and the ``-side'' is the lower one.
With upper and lower side of~ we name the respective pieces of the boundary of~ that result from removing the vertices  and~.
See \figurename~\ref{fig:naming_convention} for an illustrating example.


Let~ be an -monotone -gon and let .
We show that at least one out of , , , , , and  is a valid \mbox{-modem} position for~.
We distinguish the two cases of  being even and  being odd, because the proofs differ slightly for these cases.


\begin{lemma}\label{lem:2k+5even}
Let  be even.
Every -monotone -gon~ can be fully illuminated by a \mbox{-modem} placed on at least one position out of , , , and .
\end{lemma}


\begin{lemma}\label{lem:2k+5odd}
Let  be odd.
Every -monotone -gon~ can be fully illuminated by a \mbox{-modem} placed on at least one position out of , , , , , and .
\end{lemma}

We postpone the proofs of both lemmas until we have proven the necessary tools.
The following theorem simply summarizes the two lemmas and realizes the desired improvement over Proposition~\ref{prop:2k+3}.

\begin{theorem}\label{thm:2k+5}
For  and , every -monotone -gon~ can be illuminated by a \mbox{-modem} placed on at least one position out of , , , , , and .
\end{theorem}

In the proof of Proposition~\ref{prop:2k+3} we split the polygon in the middle.
To prove Lemmas~\ref{lem:2k+5even} and~\ref{lem:2k+5odd} (and therefore Theorem~\ref{thm:2k+5}) we consider three possible splitting lines, namely , , and .
We will start by proving always valid \mbox{-modem} positions for the leftmost and rightmost parts of~.

\begin{lemma}\label{lem:2k+5always}
Let  and let~ be an -monotone -gon.
The points  and  are valid \mbox{-modem} positions for , and  and  are valid \mbox{-modem} positions for .
\end{lemma}

\begin{figure}[htb]
  \centering
  \includegraphics{2k+5always_example}
  \caption{Example for the proof of Lemma~\ref{lem:2k+5always}. 
    The gray shaded area indicates , where  is shaded slightly darker.
    The bold dashed edge depicts the artificial edge~.}
  \label{fig:2k+5always}
\end{figure}


\begin{proof}
Recall that there are  edges of~ contained in .
Adding  to  as a rd edge results in a polygon~ that can be illuminated by a \mbox{-modem} placed at any of its vertices, by Lemma~\ref{lem:vertex}.
As  and  are vertices of~, both points are valid \mbox{-modem} positions for .
See \figurename~\ref{fig:2k+5always} for an example.

Similarly, the -gon~, composed of  and , can be illuminated by a \mbox{-modem} placed at any of its vertices, by Lemma~\ref{lem:vertex}.
Hence,  and  are valid \mbox{-modem} positions for .

Recall that we can slightly perturb an end point of both~ and~ to maintain the convention of no two vertices sharing a common -coordinate.
\end{proof}


In the following, we first consider the case of~ being even and then the slightly more involved case of~ being odd.
Before proving Lemma~\ref{lem:2k+5even}, we start with a conditional result for \mbox{-modems} on~.


\begin{lemma}\label{lem:2k+5subeven}
Let  be even and let~ be an -monotone -gon.

The left part  of~ can be illuminated by a \mbox{-modem} placed
at~ if the supporting line of  does not intersect  to the left of  and
at~ if the supporting line of  does not intersect .
Further, at least one of  and  is a valid \mbox{-modem} position for  by these conditions.

The right part  of~ can be illuminated by a \mbox{-modem} placed
at  if the supporting line of  does not intersect  to the right of  and
at  if the supporting line of  does not intersect .
Further, at least one of  and  is a valid \mbox{-modem} position for  by these conditions.
\end{lemma}

\begin{proof}
First we consider .
By Observation~\ref{obs:simplesplit}, each ray in  as well as each ray in  crosses at most  edges.
Further, each ray in  crosses neither~ nor~ and thus crosses at most  edges.
As  is even, by Observation~\ref{obs:rayio}, each ray in  crosses at most  edges.
If the supporting line of  does not intersect  to the left of , then no ray in  crosses~.
Hence, by Observation~\ref{obs:rayk+1},  is a valid \mbox{-modem} position for  in this case.
If the supporting line of  does not intersect  to the right of , then no ray in  crosses~.
Hence, by Observation~\ref{obs:rayk+1},  is a valid \mbox{-modem} position for  in this case.


Now we consider .
By Observation~\ref{obs:simplesplit} and because no ray in  crosses~ (which is an edge contained in both  and ), each ray in  crosses at most  edges.
As  is even, by Observation~\ref{obs:rayio}, each ray in  intersects at most  edges.
If the supporting line of~ does not intersect~, then no ray in  crosses~.
Hence, by Observation~\ref{obs:rayk+1},  is a valid \mbox{-modem} position for  in this case.
If the supporting line of~ does not intersect~, then no ray in  crosses~.
Hence, by Observation~\ref{obs:rayk+1},  is a valid \mbox{-modem} position for  in this case.

Note that the supporting line of~ can intersect~ only if the supporting line of~ does not intersect~ (to the left of ), and vice versa.
Thus, at least one of~ and~ is a valid \mbox{-modem} position for .
Similarly, the supporting line of~ can intersect~ only if the supporting line of~ does not intersect~ (to the right of~), and vice versa.
Thus, at least one of  and  is a valid \mbox{-modem} position for .
\end{proof}

We combine the previous results to prove Lemma~\ref{lem:2k+5even}.


\begin{proof}[Proof of Lemma~\ref{lem:2k+5even}]
For each part  and  of , at least one of  and  is a valid \mbox{-modem} position by Lemma~\ref{lem:2k+5subeven}.
If at least one of these positions is a valid \mbox{-modem} position for both  and , then the lemma is proven.

Thus assume without loss of generality that  is not a valid \mbox{-modem} position for  and that  is not a valid \mbox{-modem} position for .
(See \figurename~\ref{fig:evensplit}.) 
We prove that  is a valid \mbox{-modem} position for  in this case.
Note that this case implies that .
By symmetry,  is a valid \mbox{-modem} position for  in the mirrored case ( is not a valid \mbox{-modem} position for  and  is not a valid \mbox{-modem} position for ).

\begin{figure}[htb]
  \centering
  \includegraphics{evensplit}
  \caption{Example for  even. By assumption,  is not a valid \mbox{-modem} position for  and  is not a valid \mbox{-modem} position for .}
  \label{fig:evensplit}
\end{figure}

By Lemma~\ref{lem:2k+5subeven}, if~ is not a valid \mbox{-modem} position for , then the supporting line of~ intersects~ to the right of~.
Further, if~ is not a valid \mbox{-modem} position for , then the supporting line of~ intersects~.
Observe that this implies that  lies to the right of the ray from  towards (and through) .


Therefore, each ray  that is crossing~ cannot cross~.
Further, if~ is crossing~, then~ is in .
(Note that~ is the only edge in  that can be crossed by~.
No ray~ crosses~.)


The point  is a valid \mbox{-modem} position for  by Lemma~\ref{lem:2k+5always}.
Observe that we split~ at~.
By Observation~\ref{obs:simplesplit}, for  there are in total  edges to be considered (including~ and (a part of)~).
As no ray in  crosses~,  edges remain.

Each ray in  crosses neither~ nor~, leaving at most  edges to cross.
Consider each .
Ray~ crosses at most one of~ and~, leaving at most  edges to cross.
As  is even,  crosses at most  edges by Observation~\ref{obs:rayio}.
Hence, all rays in  cross at most  edges, implying that by Observation~\ref{obs:rayk+1}  is a valid \mbox{-modem} position for~.
\end{proof}


With this proof, ``half of'' Theorem~\ref{thm:2k+5} is proven.
The other ``half'' is shown in the next two proofs, which have of the same structure as the two preceeding ones, but need slightly more involved arguments.


\begin{lemma}\label{lem:2k+5subodd}
Let  be odd and let~ be an -monotone -gon.

The left part  of~ can be illuminated by a \mbox{-modem} placed
at  if  is reflex or  is convex and , and by a \mbox{-modem} placed
at  if  is reflex or  is convex and .
Further, at least one of  and  is a valid \mbox{-modem} position for  by these conditions.
If  is a valid \mbox{-modem} position for  by these conditions and  is to the left of the ray from  towards (and through) , then  is a valid \mbox{-modem} position for .


The right part  of~ can be illuminated by a \mbox{-modem} placed
at  if  is reflex or  is convex and , and by a \mbox{-modem} placed
at  if  is reflex or  is convex and .
Futher, least one of  and  is a valid \mbox{-modem} position for  by these conditions.
If  is a valid \mbox{-modem} position for  by these conditions and  is to the right of the ray from  towards (and through) , then  is a valid \mbox{-modem} position for .
\end{lemma}

\begin{proof}
First we consider .
Each ray in  crosses neither~ nor~ and thus, by Observation~\ref{obs:simplesplit}, crosses at most  edges.
As  is odd, by Observation~\ref{obs:rayio}, each ray in  crosses at most  edges.
If  is reflex, then no ray in  crosses the other edge (besides~) incident to~.
If~ is convex and not the first vertex of~, then each ray in  crosses at most one out of~ and~.
In both cases, by Observation~\ref{obs:rayk+1},  is a valid \mbox{-modem} position for .



Now we consider , , and .
No ray in , , or  crosses~.
Thus, by Observation~\ref{obs:simplesplit}, each ray in  crosses at most  edges, and each ray in  or  crosses at most  edges.
Each ray in  cannot cross~ and likewise, each ray in  cannot cross~.
As  is odd, each ray in , , and  cross at most  edges by Observation~\ref{obs:rayio}.



If  is reflex, then no ray in  and  crosses~.
If, in addition,  is to the left of the ray from  through , then each ray in  that crosses~ does not cross~.
If  is convex and not the first vertex of~, then each ray in  crosses at most one out of the two edges incident to .
If in addition,  is to the left of the ray from  through , then each ray in  either crosses~ but none of the two edges incident to~ or the ray does not cross~ and crosses at most one out of the two edges incident to~.
Summing up, if  is reflex or if  is convex, then each ray in  crosses at most  edges and thus,  is a valid \mbox{-modem} position for  by Observation~\ref{obs:rayk+1}.
If, in addition,  is to the left of the ray from  through , then each ray in  crosses at most  edges and thus,  is a valid \mbox{-modem} position for  by Observation~\ref{obs:rayk+1}.



As  as well as  can either be convex or reflex and not both can be the first vertex of~, at least one out of~ and~ is a valid \mbox{-modem} position for~ by the conditions of the lemma.


\smallskip

Altogether, this proves the first half of the lemma. The second half follows by symmetric arguments.  
\end{proof}

We now prove Lemma~\ref{lem:2k+5odd} which concludes the proof for Theorem~\ref{thm:2k+5}.

\begin{proof}[Proof of Lemma~\ref{lem:2k+5odd}]
For each part  and  of~, at least one of~ and~ is a valid \mbox{-modem} position by Lemma~\ref{lem:2k+5subodd}.
If at least one of these positions is a valid \mbox{-modem} position for both  and , then the lemma is proven.

Thus assume without loss of generality that~ is not a valid \mbox{-modem} position for  and that~ is not a valid \mbox{-modem} position for .
(See \figurename~\ref{fig:oddsplit}.) 
We prove that at least one out of  and  is a valid \mbox{-modem} position for~ in this case.
By symmetry, at least one out of  and  is a valid \mbox{-modem} position for~ in the mirrored case ( is not a valid \mbox{-modem} position for  and  is not a valid \mbox{-modem} position for ).


\begin{figure}[htb]
  \centering
  \includegraphics{oddsplit}
  \caption{Example for  odd. By assumption,  is not a valid \mbox{-modem} position for  and  is not a valid \mbox{-modem} position for .}
  \label{fig:oddsplit}
\end{figure}


If  is to the right of the ray from  through~, then, by Lemma~\ref{lem:2k+5subodd},  is a valid \mbox{-modem} position for , and, by Lemma~\ref{lem:2k+5always},  is a valid \mbox{-modem} position for .

Thus, assume that  is to the left of the ray from  through~.
Observe that this implies that  is reflex.
We prove that  is a valid \mbox{-modem} position for~ in this case.
By Lemma~\ref{lem:2k+5always} this is true for .

For  observe that each ray in  is crossing at most  edges by Observation~\ref{obs:simplesplit} (and because no ray in  crosses~).
We distinguish two cases:
\begin{enumerate}
\item \textbf{ is reflex}:\\
Each ray in  crosses at most one of the two edges incident to~.
As  is odd, each ray in  crosses at most  edges by Observation~\ref{obs:rayio}.
Each ray in  crosses neither of the two edges incident to~ and thus, crosses at most  edges.
\item \textbf{ is convex}:\\
In this case  and  is convex as otherwise, by Lemma~\ref{lem:2k+5subodd},  would be a valid \mbox{-modem} position for , which would contradict our assumption.
Hence, each ray in  crosses at most one out of~ and~, leaving at most  edges to cross.
As  is odd, each ray in  crosses at most  edges by Observation~\ref{obs:rayio}.
Further, no ray in  crosses~.
\end{enumerate}
Therefore,  is a valid \mbox{-modem} position for~ in both cases.
To summarize, we proved that if neither~ nor~ is a valid \mbox{-modem} position for~ and  is not a valid \mbox{-modem} position for , then at least one out of~ and~ is a valid \mbox{-modem} position for~. 
In the symmetric case, if~ is not a valid \mbox{-modem} position for , then symmetric arguments prove that at least one out of~ and~ is a valid \mbox{-modem} position for~.
\end{proof}



\subsection{Improved polygon splitting}
\label{sec:gmono-split}

As mentioned before, we will now provide an ``efficient'' way of splitting a monotone polygon~ into smaller monotone polygons.
Unlike the simple splitting in Observation~\ref{obs:simplesplit}, the resulting parts need to be monotone polygons again, so that a recursive splitting is possible.
Further, the sum over the sizes of the smaller polygons should be as small as possible (hence, ``efficient'' splitting).
And, of course, the splitting has to ensure that if all small monotone polygons are illuminated, then~ is illuminated.




\begin{lemma}[Splitting Lemma]\label{lem:splitting} Let~ be an -monotone -gon and let .
 can be split into two -monotone polygons~ and~, such that:
  \begin{itemize}
  \item  has  vertices,
  \item  has  vertices, and
  \item if  is illuminated by modems placed in  and  is illuminated by modems placed in , then also~ is illuminated by those modems.
  \end{itemize}
\end{lemma}

\begin{figure}[htb]
  \centering
  \includegraphics[page = 1]{splittinglemma}
  \caption{Illustrating the proof of Lemma~\ref{lem:splitting}, with~ to the right of~ and~ lies above~.}
  \label{fig:splittinglemma1}
\end{figure}


\begin{proof}
Let~ be the edge of~ intersected by the vertical line~.
Without loss of generality, we assume that~ lies above~.
Let~ and~ be the left and right end point of~, respectively.
Let~ be the edge of~ having  as its left end point and let~ be the other end point of~.

Observe that the boundary~ of~ can be partitioned at~ and~ into an upper polygonal chain~ and a lower polygonal chain~.
Both chains have~ and~ as end vertices, but are otherwise disjoint (vertex and edge disjoint).

Since we do not have parallel edges, the supporting lines of the edges~ and~ intersect at a point~.


\begin{figure}[htb]
  \centering
  \includegraphics[page = 2]{splittinglemma}
  \caption{Illustrating the proof of Lemma~\ref{lem:splitting}, with~ to the right of~ and~ lies below~.}
  \label{fig:splittinglemma2}
\end{figure}


Assume first that  is to the right of  (see \figurenames~\ref{fig:splittinglemma1},~\ref{fig:splittinglemma2}).
We construct an upper polygonal chain  by joining  with the edge from  to .
Let  be the vertical line through~.
We construct a lower polygonal chain  by joining  with the edge from~ to~.
Apart from the common end vertices~ and~ these two chains are disjoint.
Hence, joining  and  we get the polygonal cycle , which is the boundary of the -monotone polygon~.
To define the -monotone polygon~ we distinguish two cases, depending on whether~ lies above~ or below~:
\begin{enumerate}
\item \textbf{ lies above~} (\figurename~\ref{fig:splittinglemma1}):
    Let  be the intersection of  and~.
    The upper polygonal chain  is the edge from~ to~ joined with .
    Let  be the vertical line through~.
    The lower polygonal chain  is the edge from~ to~ joined with .
\item \textbf{ lies below~} (\figurename~\ref{fig:splittinglemma2}):
    The upper polygonal chain  is .
    The lower polygonal chain  is the edge from~ to~ joined with .
\end{enumerate}

Observe that  and .
Therefore, if  is illuminated by a modem placed in  and  is illuminated by a modem placed in , then~ is illuminated.








It is easy to see that the resulting polygons~ and~ are indeed -monotone in all cases.
Further, note that also the convention that no two vertices of a polygon have the same -coordinate is respected by both~ and~.

Finally,~ is an -gon because it contains  vertices in  plus one vertex to the right of (and excluding) .
Similarly,~ is an -gon because it contains  vertices in  plus one vertex to the left of (and excluding) .

The situation where  is to the left of  is symmetric and hence can be reasoned in essentially the same way.
\end{proof}

 
Note that this splitting breaks an -monotone -gon~ into two smaller -monotone polygons,~ and~, which are not necessarily subpolygons of~.
But, .
Further, the restrictions on the placement for modems together with the restrictions on the splitting position~ ensure, which the Splitting Lemma can be applied recursively on the smaller polygons.


\subsection{Illuminating arbitrarily large monotone polygons}
\label{sec:gmono-main}


We proved that a single \mbox{-modem} can illuminate an -monotone -gon.
In addition we provided an efficient way to break a ``big'' -monotone -gon into smaller -monotone pieces.
We now combine both results to prove one of our main results presented in the following theorem.


\begin{theorem}\label{thm:gmono-generalbound}
Every -monotone -gon can be illuminated with (at most) ~\mbox{-modems}, and there exist -monotone -gons that require at least  \mbox{-modems} to be illuminated.
\end{theorem}


\begin{proof}
For the upper bound let~ be an -monotone -gon.
We iteratively apply the Splitting Lemma (Lemma~\ref{lem:splitting}) to split~ into  -monotone polygons with at most \mbox{} vertices each, as follows: 
Let~ be~.
For , apply Lemma~\ref{lem:splitting} to~ and obtain an -monotone \mbox{-gon}  and an -monotone \mbox{-gon}~.
Let~ be the remaining -monotone polygon , which, by definition of , has at most  vertices.
By Lemma~\ref{lem:splitting},  is illuminated if  is illuminated by modems placed anywhere in it but not to the right of its penultimate point and  is illuminated by modems placed anywhere in it but not to the left of its second point.
Hence, illuminating each of the obtained -monotone -gons  with a modem placed not to the left of its second and not to the right of its penultimate point illuminates~.
This is possible for  and each -monotone -gon by Theorem~\ref{thm:2k+5}.


\begin{figure}[htb]
  \centering
  \includegraphics[page=2]{lowerbound} \\
  \vspace{3mm}
  \includegraphics[page=3]{lowerbound} \\
  \caption{Two -monotone -gons that require  \mbox{-modems} each. 
    (Note that the end points of all edges can be slightly perturbed without changing the number of required modems, to respect the convention of no parallel edges.)
    For each of the points in~ (depicted as little squares) inside~, the gray region indicates from where it can be illuminated with a \mbox{-modem} (top) or with a \mbox{-modem} (bottom).
    Note that for the lower subfigure, part of these regions are single edges.  }
\label{fig:lower}
\end{figure}


For the lower bound we show how to construct -monotone -gons~ for every value of~. 
The construction differs slightly depending on the parity of~.
For each~, the lower chain of~ is a single edge connecting  with .

For  even, the upper chain of  consists of groups of thin triangular spikes that are separated by quadrilateral valleys. 
The first and the last group of spikes contains  spikes each, while all other groups contain  spikes. 
A sketch of the construction for  is shown in \figurename~\ref{fig:lower}~(top).


For  odd, the upper chain of  consists of groups of thin triangular spikes that are separated by quadrilateral towers. 
The first and the last group of spikes contains  spikes each, while all other groups contain~ spikes. 
A sketch of the construction for  is shown in \figurename~\ref{fig:lower}~(bottom).


Observe that each spike needs two vertices, each tower (for  odd) needs three additional vertices, each valley (for  even) needs one additional vertex, and one more vertex is needed to close the polygon. 
Thus, summing up for~ groups of triangular spikes,  has  vertices, independent of whether  is even or odd.



After giving the construction we prove that~ \mbox{-modems} are needed to illuminate~.
We consider a set~ of witness points in the interior of~.
One point of~ is in the leftmost spike of~.
For each of the \mbox{} groups of ~spikes (between two towers or valleys), one point of  is in the middle spike.
One last point of  is placed in the rightmost spike of~.

Note that the areas from which any two of these points can be illuminated with a \mbox{-modem} are disjoint.
(In the sketch in \figurename~\ref{fig:lower} the points of~ are shown as small squares and the areas from which each such witness point can be illuminated is shown shaded.)
Thus, no pair of two points in~ can be illuminated by a single \mbox{-modem} and~ requires at least  \mbox{-modems} to be illuminated. 

With  vertices,  can have  groups of spikes and consequently, needs at least that many \mbox{-modems} to be illuminated. \end{proof}


Observe that this bound (upper and lower) matches the bound of Chv\'atal's Art Gallery Theorem~\cite{Chvatal}.
The ``watchmen'' there correspond to \mbox{-modems}.
Hence, for , the bound  (from the last paragraph of the proof of Theorem~\ref{thm:gmono-generalbound}) matches the bound~ from~\cite{Chvatal}.




\section{Illumination of monotone orthogonal polygons}\label{sec:monoortho}

Very often, orthogonal polygons are a sufficiently realistic scenario for placing modems inside buildings in order to cover the interior of the building with wireless reception.
In this section, we give matching lower and upper bounds on the number of \mbox{-modems} needed and required to illuminate monotone orthogonal polygons.

Let~ be an -monotone orthogonal -gon.
Recall that we label the vertices of~,  with respect to their -sorted order, such that  is the leftmost and  is the rightmost vertex of~. In addition, among two vertices with the same -coordinate, the lower vertex (lower -coordinate) gets the higher label. 
For simplicity, we assume that at most two vertices have the same -coordinate.

For validating possible \mbox{-modem} positions, we adopt the respective definitions and observations from Section~\ref{sec:gmono}.
Observe that an orthogonal polygon has (at least) four 'extremal' edges: a topmost, bottommost, leftmost, and rightmost edge, which we denote by , , , and , respectively.\footnote{Note that  and  are unique. If there are more topmost (or bottommost) edges,  let  be the leftmost  (or let  be the rightmost) among them. }
It is easy to see that for any point  every ray in  crosses at most one out of these four extremal edges.

\begin{observation}\label{obs:extremal}
For every orthogonal polygon~ and every point , every ray in  crosses at most one out of the four extremal edges of~.
\end{observation}

Using this simple observation, we can prove some first results.

\begin{lemma}\label{lem:k+3}
  Every orthogonal polygon~ with at most  vertices can be illuminated by a \mbox{-modem} placed anywhere outside~.
\end{lemma}
\begin{proof}
  Any line segment with one end point outside~ and the other end point inside or on the boundary of~ can cross at most one out of the four extremal edges of~ and hence, crosses at most~ edges of~. \end{proof}


\begin{lemma}\label{lem:k+4}
Every orthogonal polygon~ with at most  vertices can be illuminated by a \mbox{-modem} placed anywhere in the interior or on the boundary of~.
\end{lemma}
\begin{proof}
For any point , by Observation~\ref{obs:extremal}, every ray  crosses at most  edges.
Hence,  is a valid \mbox{-modem} position for~, by Observation~\ref{obs:rayk+1}. \end{proof}


\begin{lemma}\label{lem:k+5}
For every orthogonal polygon~ with at most  vertices there exists a point~ on its leftmost edge~ and a point~ on its rightmost edge~, such that~ can be illuminated by a \mbox{-modem} placed at any of~ and~.
\end{lemma}
\begin{proof}
We prove the case of placing the \mbox{-modem} at~.
The other case, for~, follows analogously.
If  has at most  vertices, then the statement follows by Lemma~\ref{lem:k+4}.
Hence assume that  has  vertices. Note that this implies that  is odd (because an orthogonal polygon always has an even number of vertices).

If  is incident to a horizontal edge that is not an extremal edge of~, then let this horizontal edge be~.
Otherwise,  is incident to the topmost edge~ and the bottommost edge~ of~.
In this case, let~ be any non-extreme horizontal edge.
(Note that, as , there exists at least one such edge.)
For both cases, let  be the straight line supporting~ and let .
By Observation~\ref{obs:extremal}, every ray of  crosses at most  edges.
Further, no ray of  crosses~.
As~ is not an extremal edge of~, every ray of  crosses at most  edges and thus,~ is a valid \mbox{-modem} position for~ by Observation~\ref{obs:rayk+1}.
\end{proof}


\begin{figure}[htb]
  \centering
  \includegraphics[page=2]{stairend}
  \caption{Four examples for stair end polygons:
	  An upper left-sided stair end polygon~(a);
  	  a lower left-sided stair end polygon~(b);
	  an upper right-sided stair end polygon~(c); and 
  	  a lower right-sided stair end polygon~(d).}
  \label{fig:stairend}
\end{figure}

The previous statements are true for general (not necessarily monotone) orthogonal polygons.
The next lemma is only true for a special kind of -monotone orthogonal polygons, which we name upper or lower, left-sided or right-sided \emph{``stair end polygon''}.
An upper left-sided stair end polygon is an -monotone orthogonal -gon that has  as an edge.
Likewise, an upper right-sided stair end polygon is an -monotone orthogonal -gon that has  as an edge.
Further, a lower left- or right-sided stair end polygon is an -monotone orthogonal -gon that has~ or~, respectively, as an edge.
See \figurename~\ref{fig:stairend} for examples.


\begin{lemma}\label{lem:k+7}
Let  be odd and let~ be a stair end polygon with  vertices.
If~ is (upper or lower) left-sided, then there exists a point~ on its leftmost edge~ and if~ is (upper or lower) right-sided then there exists a point~ on its rightmost edge~, such that~ can be illuminated by a \mbox{-modem} placed on~ or~, respectively.
\end{lemma}
\begin{proof}
We prove the case of placing the \mbox{-modem} at~ for an upper right-sided stair end polygon~.
The lower right-sided case and the upper and lower left-sided cases follow analogously.

Let~ be the horizontal edge incident to~, and let~ be the vertical edge that is the left neighbor of~ on the boundary of~.
Further, let , , let~ be the horizontal edge incident to , and let~ be the vertical edge that is the left neighbor of~ on the boundary of~.

Order the horizontal edges of~ by their -coordinates from top to bottom (from  to ).
If two edges have the same -coordinates, let the left one be before the right one.
Let  be the third horizontal edge and let  be the third-to-last horizontal edge in that order.
We distinguish five cases:

\begin{enumerate}
\item \textbf{} (see \figurename~\ref{fig:stairendproof1-3}~(a)):\\
Choose , let  be the supporting line of~, let  be any ray above~, and let  be any ray below~.
Note that in this case,  is not an extremal edge of~. 
By Observation~\ref{obs:extremal},  crosses at most~ edges.
Further,  cannot cross the edges~, , and~, leaving at most  edges to cross.

As , there are only three horizontal edges, , , and~, below (or on)~ and hence, only four vertical edges, , , , and~, (in part) below~. 
Out of this  edges,  cannot cross~.
Further,  can cross at most one edge out of~, , and~, and at most two edges out of~, , and~.
This leaves also at most  edges to cross for~.
Therefore, by Observation~\ref{obs:rayk+1},  is a valid \mbox{-modem} position in this case.

\begin{figure}[htb]
  \centering
  \includegraphics[page=1]{stairendproof}
  \caption{Sketches for the proof of Lemma~\ref{lem:k+7}:
	  Case~ (a).
	  Case~ is between (not including)~ and~ (b).
      Case~ (c).}
  \label{fig:stairendproof1-3}
\end{figure}

\item \textbf{ is between (not including)~ and~} (see \figurename~\ref{fig:stairendproof1-3}~(b)):\\
Let  be the supporting line of~ and choose .
Let  be any ray above~, and let  be any ray below~.

If~, then  either crosses only~ or does not cross~.
Further,  can cross at most one edge out of~, , , and the second horizontal edge.
If~, then  crosses at most~ edges, by Observation~\ref{obs:extremal}, and~ crosses at most one edge out of~ and~.
In both cases,  cannot cross the edges~ and~, leaving at most  edges to cross.

Like before, there are only three horizontal edges, , , and~, below (or on)~ and hence, only four vertical edges, , , , and~, (in part) below~. 
Out of this  edges,  cannot cross~.
Further,  can cross at most one edge out of~, , and~, and at most two edges out of~, , and~.
This leaves also at most  edges to cross for~.
Therefore, by Observation~\ref{obs:rayk+1},  is a valid \mbox{-modem} position in this case.

\item \textbf{} (see \figurename~\ref{fig:stairendproof1-3}~(c)):\\
Let  be the supporting line of~ and choose .
Let  be any ray above~, and let  be any ray below~.

If~, then  either crosses only~ or does not cross~.
Further,  can cross at most one edge out of~, , , and the second horizontal edge.
If~, then  crosses at most~ edges, by Observation~\ref{obs:extremal}, and~ crosses at most one edge out of~ and~.
In both cases,  cannot cross~ and the second-to-last horizontal edge, leaving at most  edges to cross.

By Observation~\ref{obs:extremal},  crosses at most~ edges.
Further,  cannot cross , , and the second horizontal edge, leaving at most  edges to cross.
Therefore, by Observation~\ref{obs:rayk+1},  is a valid \mbox{-modem} position in this case.

\item \textbf{ is below~ and  is above~} (see \figurename~\ref{fig:stairendproof4-5}~(a)):\\
Let  be the supporting line of~ and choose .
Let  be any ray above~, and let  be any ray below~.

By Observation~\ref{obs:extremal},  crosses at most~ edges.
Further,  cannot cross~, the second-to-last horizontal edge, and~, leaving at most  edges to cross.

By Observation~\ref{obs:extremal},  crosses at most~ edges.
Further,  cannot cross~, the second horizontal edge, and~, leaving at most  edges to cross.
Therefore, by Observation~\ref{obs:rayk+1},  is a valid \mbox{-modem} position in this case.

\begin{figure}[htb]
  \centering
  \includegraphics[page=2]{stairendproof}
  \caption{Sketches for the proof of Lemma~\ref{lem:k+7}:
	  Case~ is below~ and~ is above~ (a).
      Case~ is between (and including)~ and~:
	   (b);
	   is between~ and~ (c).
    (The case where~ is not depicted.)}
  \label{fig:stairendproof4-5}
\end{figure}

\item \textbf{ is between (and including)~ and~} (see \figurename~\ref{fig:stairendproof4-5}~(b) and~(c)):\\
Let  be the supporting line of~ and choose .
Let  be any ray above~, and let  be any ray below~.

If~, then  either crosses only~ or does not cross~.
Further,  can cross at most one edge out of~, , , and the second horizontal edge.
If~, then  crosses at most~ edges, by Observation~\ref{obs:extremal}, and~ crosses at most one edge out of~ and~.
Further,  cannot cross  and the second-to-last horizontal edge, leaving at most  edges to cross.

By Observation~\ref{obs:extremal},  crosses at most~ edges.
Further,  cannot cross~, the second horizontal edge, and~, leaving at most  edges to cross.
Therefore, by Observation~\ref{obs:rayk+1},  is a valid \mbox{-modem} position in this last case.
\end{enumerate}

As  cannot be above~ and  cannot be below~, this case analysis is exhaustive and proves the claim.
\end{proof}


Like for (general) -monotone polygons, we split a large -monotone orthogonal -gon~ into smaller pieces.
Let  be even.
We split~ along a vertical line  through  into a left and a right -monotone orthogonal polygon~ and~, respectively.
Note that  is the supporting line of the vertical edge of~ with the end points  and .
Let~ be the horizontal edge of~ that is crossed by~ in the point~.


Recall that with  () we denote the left (right) closed half-plane bounded by a vertical line~.
If the horizontal edge of~ that is incident to~ is completely contained in , then~ is  plus an additional edge~, and~ is , plus an additional edge~.
Otherwise,  is  plus an additional edge~, and~ is , plus an additional edge~.
See \figurename~\ref{fig:orthonaming} for a sketch of the above naming conventions, where the horizontal edge of~ that is incident to~ goes to the right. 

\begin{figure}[htb]
  \centering
  \includegraphics{orthodenom}
  \caption{Example to illustrate the naming conventions for an -monotone orthogonal -gon.
    The vertex labeled~ and~ is either vertex~ or vertex~, depending on whether the vertical edge incident to this point goes up or down.
    Note though that in a different example also the right end point of~ could be either~ or~.}
  \label{fig:orthonaming}
\end{figure}

It is easy to see that the resulting polygons are subpolygons of~, both -monotone and orthogonal, and that they meet the convention that at most two vertices share a common -coordinate.
As both subpolygons are disjoint except for the common part on~, illuminating~ with modems in (or on the boundary of)~ and illuminating~ with  modems in (or on the boundary of)~ illuminates~.
Further,  contains  edges (including~) and~ contains  edges (including~).
We summarize these observations in the following statement.

\begin{observation}\label{obs:orthosplit}
Let~ be an -monotone orthogonal -gon, , and let~ be a vertical line through , ,  is even.
Let~ and~ be subpolygons of~, with~ being the rightmost edge of~ and~ being the leftmost edge of~.
\begin{itemize}\vspace*{-0.5ex}
\item  is an -monotone orthogonal -gon.
\item  is an -monotone orthogonal -gon.
\item If both~ and~ are illuminated by \mbox{-modems} placed to the left of or on~ and to the right of or on~, respectively, then~ is illuminated.
\end{itemize}
\end{observation}



Using the previous results we can prove the following lemma.

\begin{lemma}\label{lem:2k+6}
For every -monotone orthogonal -gon~ there exists a point , such that~ can be illuminated with a \mbox{-modem} placed on~.
\end{lemma}
\begin{proof}
  If~ is even, split~ vertically at  into two -monotone orthogonal -gons by Observation~\ref{obs:orthosplit} and let .
  By Lemma~\ref{lem:k+4}, both -gons are illuminated by a \mbox{-modem} at .


  For odd , split~ vertically at  into one -gon~ and one -gon~, both -monotone and orthogonal, by Observation~\ref{obs:orthosplit}.
  Recall that~ is the leftmost edge of~ and that it is contained in the splitting line. 
By Lemma~\ref{lem:k+5}, there exists a point~ on  where a \mbox{-modem} can be placed to illuminate the -gon~.
  Further, Lemmas~\ref{lem:k+3} and~\ref{lem:k+4} ensure that the -gon~ is also illuminated by a \mbox{-modem} at~.

  Hence, whether~ is even or odd, both subpolygons are illuminated and therefore also~, by Observation~\ref{obs:orthosplit}. \end{proof}

With this lemma we can prove the following first piece of our main result for monotone orthogonal polygons.

\begin{lemma}\label{lem:orthoupperall}
  Every -monotone orthogonal -gon~ can be illuminated with  \mbox{-modems}.
\end{lemma}
\begin{proof}
  By Observation~\ref{obs:orthosplit}, we can split~ into a left -monotone orthogonal -gon~ and a right -monotone orthogonal -gon~.
  Recursing on~, like in the proof of~Theorem~\ref{thm:gmono-generalbound}, results in  -monotone orthogonal subpolygons with at most  vertices each (including~ and the rightmost remaining subpolygon).
By Lemma~\ref{lem:2k+6}, each of these polygons can be illuminated with one \mbox{-modem}.
By Observation~\ref{obs:orthosplit}, the illumination of all subpolygons implies the illumination of~. \end{proof}

Following this upper bound for the number of necessary \mbox{-modems}, we next present a lower bound construction.

\begin{lemma}\label{lem:ortholowerall}
  For even , there exists an -monotone orthogonal -gon requiring  \mbox{-modems} to be illuminated.
For odd , there exists an -monotone orthogonal -gon requiring  \mbox{-modems} to be illuminated.
\end{lemma}


\begin{figure}[htb]
  \centering
  \includegraphics{ortholowerall}
  \caption{Lower bound construction examples for -monotone orthogonal polygons, with illumination regions for~~(left) and~~(right).
  }
 \label{fig:ortholowerall}
\end{figure}


\begin{proof}
  The lower bound construction for an -monotone orthogonal -gon~ is sketched in \figurename~\ref{fig:ortholowerall}, left for even~, right for odd~.
  To analyze the construction we place, for each~, a set of~ witness points into~, such that the regions from which each such witness point can be illuminated by one \mbox{-modem} are pairwise disjoint.
  Thus, the number of \mbox{-modems} needed to illuminate~ is at least~. 

         
  For even , we place a witness point in the middle of every ()-th corridor, starting from the leftmost one.
\figurename~\ref{fig:ortholowerall}~(left) illustrates the set of witness points for~.
The region from which such a witness point can be illuminated with a \mbox{-modem} extends from the corridor containing the point (in both left and right direction) to the next  corridors and a small section of the -th corridor, which ends before the middle point of that corridor. 
This way, the regions from which two witness points can be illuminated are disjoint.
  If~ has~ witness points, then~ has at least  corridors.
  Hence, .
  Thus, in order to illuminate~, the required number of \mbox{-modems} is at least the maximum possible value of~, namely .
  Given the fact that  is odd and  is even,  requires at least  \mbox{-modems}. 


  For odd , we place a witness point in the middle of every ()-th corridor, starting from the leftmost one.
\figurename~\ref{fig:ortholowerall}~(right) illustrates the set of witness points for~.
The region from which such a witness point can be illuminated with a \mbox{-modem} is the same as for the even case , with the addition of one edge of the -th corridor (in both left and right direction).
  Thus, this region ends before the midpoint of the -th corridor.
  This way, the regions from which two witness points can be illuminated are disjoint.
  By a similar analysis as in the even case, the number of \mbox{-modems} needed to illuminate~ is at least . \end{proof}	 


For the special case of~ we can improve the lower bound construction for odd~ to match the bounds of the even case.

\begin{lemma}\label{lem:ortholower1}
  There exists an -monotone orthogonal -gon requiring  \mbox{-modems} to be illuminated.  
\end{lemma}
\begin{proof}
  Consider the -monotone orthogonal -gon~ sketched in \figurename~\ref{fig:ortholower1}~(middle).
  We partition~ into~ subsets~\mbox{} with ``witness boxes''~ (numbered increasingly from top left to bottom right), shown as dotted orthogonal boxes in the figure.
  Further, in each such subset  of~ we place two witness points~~ and~. 
  We show that~ needs  \mbox{-modems} by proving that in order to illuminate all witness points in~, at least one \mbox{-modem} must be placed per subset~.


\begin{figure}[htb]
  \centering
  \includegraphics{ortholower1}
  \caption{An -monotone orthogonal -gon requiring  \mbox{-modems} for illumination (middle).
	  \mbox{-modem} region to illuminate the witness point~ in~ (bottom left).
	  \mbox{-modem} region to illuminate the witness point~ in~ (top right).
}
 \label{fig:ortholower1}
\end{figure}


\figurename~\ref{fig:ortholower1}~(bottom left) and (top right) depict the regions of~ from which each of the witness points~ and~ in one subset~ of~ can be illuminated with a \mbox{-modem}. 
Obviously, it is possible to illuminate all of~ and~ using one \mbox{-modem} placed accordingly in each~.
It is also easy to see that no \mbox{-modem} placed in~ can illuminate any of the points~ and~ in any other subset~ with~ or~.

On one hand, it is possible to illuminate~ or~ by a \mbox{-modem} placed in~.
On the other hand, no \mbox{-modem} placed in~ can simultaneously illuminate~ and~.
	
Further, it is possible to illuminate~ by a \mbox{-modem} placed in~, but no \mbox{-modem} placed in~ can illuminate~.
Finally, a \mbox{-modem} in~ that illuminates~ illuminates neither of the points~, , and~.



Now consider again the whole -monotone orthogonal -gon~ and assume that~ can be illuminated with less than~ \mbox{-modems}. 
Then there exists a minimum , such that the set  of witness points can be illuminated with less than~ \mbox{-modems}, all located in . 
Hence, there must be some subset~, , in which no modem is placed (if a modem is placed in the intersection of two adjacent subsets, then we count it for the left subset).
In the following, we consider the witness points in~ according to the order of the subsets of~ and place \mbox{-modems} as needed.
	
In order to illuminate~ we must place a \mbox{-modem}~ in~, implying .
If~ does not illuminate~, then we need a \mbox{-modem} in~ as well. 
Hence, assume that~ also illuminates~ and therefore, does not illuminate~.
(Note that~ illuminates~ in this case.)

There are two choices for the next \mbox{-modem}.
(1)~We can place a \mbox{-modem}~ in~, such that it illuminates~ and also illuminates~.
(2)~We can avoid placing a \mbox{-modem} in~ by placing a \mbox{-modem}~ in~, such that it also illuminates~.

For choice~(1) observe that, as long as we keep placing one \mbox{-modem} per subset, the situation stays the same as after placing~:
after placing~ \mbox{-modems} for any , all witness points in~ plus (at most)~ are illuminated.
And, in order to illuminate all witness points in~, at least one more \mbox{-modem} is needed.

For choice~(2) assume that each \mbox{-modem}~, , has been placed in~ (according to choice~(1)) and~ is the first subset being skipped, i.e., we place the \mbox{-modem}~ in~.
However, in this case~ illuminates neither~ nor any of the witness points of~.
Thus, we need a \mbox{-modem}~ in~ in order to illuminate~ which, in turn, does not illuminate~.

As long as we now keep placing only one \mbox{-modem} per subset, this situation stays the same as after placing~:
after placing~\mbox{} \mbox{-modems} for any , all witness points in~ plus (at most)~ are illuminated.
In order to illuminate~ and thus, to completely illuminate~, at least one more \mbox{-modem} is needed.

To change this situation, we have to place an additional \mbox{-modem} that illuminates~ either in~ or in~.
In addition to~, with such a \mbox{-modem} at most one out of~ and~ can be illuminated.
And this yields the same situation as after either choice~(1) or choice~(2).
Hence, it is not possible to illuminate~ with less than~ \mbox{-modems}, for any , contradicting the assumption and proving the lemma.
\end{proof}


Comparing the results in Lemmas~\ref{lem:orthoupperall},~\ref{lem:ortholowerall}, and~\ref{lem:ortholower1}, observe that the bounds are tight if~ and if~ is even, but not for odd~.
In fact, we prove in the next two lemmas that the upper bound for odd~ is indeed lower.


\begin{lemma}\label{lem:2k+8}
For odd~ and every -monotone orthogonal -gon~ there exists a point  such that~ can be illuminated with a \mbox{-modem} placed on~.
\end{lemma}

\begin{figure}[htb]
  \centering
  \includegraphics{ortho2k+8_example}
  \caption{Sketch for proof of Lemma~\ref{lem:2k+8}:
	  ``Middle'' part of~ where the splitting takes place (left).
      The resulting upper right-sided stair end -gon~ and the -monotone orthogonal -gon~. The two bold dashed edges depict~ and~ (right).}
 \label{fig:ortho2k+8}
\end{figure}

\begin{proof}
Let~ and let~ be the vertical line through~.
Let~ be the vertical edge~ of~.
Let~ be the horizontal edge of~ that is incident to one end point of~ and has its other end point to the right of~.
Likewise, let~ be the horizontal edge of~ that is incident to one end point of~ and has its other end point to the left of~.
Further, let~ be the horizontal edge of~ that is crossed by~.
See \figurename~\ref{fig:ortho2k+8} for an example. 

We assume that~ has~ as one end point and that~ is below~ (this case is depicted in \figurename~\ref{fig:ortho2k+8}).
We split~ vertically at~ (through~) into one -monotone orthogonal -gon~ and one -monotone orthogonal -gon~ (Observation~\ref{obs:orthosplit}).
Observe that~ is an upper right-sided stair end polygon, with~ as its rightmost edge, which is contained in the splitting line~.
By Lemma~\ref{lem:k+7}, there exists a point~ on  where a \mbox{-modem} can be placed to illuminate~.
Further, Lemmas~\ref{lem:k+3} and~\ref{lem:k+4} ensure that~ is also illuminated by a \mbox{-modem} at~.
Hence, both subpolygons are illuminated and therefore also~, by Observation~\ref{obs:orthosplit}.

If~ has~ as one end point and~ is above~, then splitting~ vertically at~ (through~) results in a lower right-sided stair end -gon~ (and an -monotone orthogonal -gon~).
In the remaining two cases,  has~ as one end point and~ is above~, or~ has~ as one end point and~ is below~, we split~ vertically at~ (through~) into an -monotone orthogonal -gon~ and a lower or upper, respectively, left-sided stair end -gon~.
In all three cases, an analogous argumentation as above proves the lemma. \end{proof}

With this lemma we can prove the last missing piece of our main result for monotone orthogonal polygons.

\begin{lemma}\label{lem:orthoupperodd}
For odd~, every -monotone orthogonal -gon~ can be illuminated with  \mbox{-modems}.
\end{lemma}
\begin{proof}
By Observation~\ref{obs:orthosplit}, we can split~ into a left -monotone orthogonal -gon~ and a right -monotone orthogonal -gon~.
Like in the proof of~Lemma~\ref{lem:orthoupperall}, recursing on~ results in  -monotone orthogonal subpolygons with at most  vertices each (including~ and the rightmost remaining subpolygon).
By Lemma~\ref{lem:2k+8}, each of these polygons can be illuminated with one \mbox{-modem}.
By Observation~\ref{obs:orthosplit}, the illumination of all subpolygons implies the illumination of~. \end{proof}


We summarize the results for monotone orthogonal polygons from Lemmas~\ref{lem:orthoupperall},~\ref{lem:ortholowerall},~\ref{lem:ortholower1}, and~\ref{lem:orthoupperodd} in the following Theorem.


\begin{theorem}\label{thm:orthoall}
Let~ be an -monotone orthogonal -gon.
For  and all even~,  \mbox{-modems} are always sufficient and sometimes necessary to illuminate~.
For odd~,  \mbox{-modems} are always sufficient and sometimes necessary to illuminate~.
\end{theorem}





\section{Conclusion}

Inspired by current wireless networks, we studied a new variant of the classic polygon-illumination problem.
To model the way wireless devices communicate within a building, we allow signals to cross at most a given number~ of walls.

Using as a main tool the Splitting Lemma that allows us to divide a polygon into simpler, overlapping polygons, we gave an upper bound of  on the number of \mbox{-modems} needed to illuminate any given monotone polygon with~ vertices.
We also presented a family of monotone polygons that need at least  \mbox{-modems}, which shows that our upper bound is tight.


Further, we also studied the particular case when the monotone polygons are orthogonal, where we derived similar tight bounds, which differ depending on the parity of~.
For even~,  \mbox{-modems} are always sufficient and sometimes needed to illuminate a monotone orthogonal -gon.
And for odd~\mbox{},  \mbox{-modems} are always sufficient and sometimes needed to illuminate a monotone orthogonal -gon.
Interestingly, the bounds for the number of \mbox{-modems} for~ are the same as for even~.
This is an artifact of the orthogonality and the small constant~.

Let us conclude with the following open problem: 
What is the algorithmic complexity of finding the minimum number of \mbox{-modems} (and their position)
to illuminate a given monotone (orthogonal) polygon?





\section*{Acknowledgments}
We thank Clemens Huemer and Jan P{\"o}schko for helpful discussions.


\begin{thebibliography}{}

\bibitem{Aich} 
	O. Aichholzer, R. Fabila-Monroy, D. Flores-Pe\~naloza, T. Hackl, C. Huemer, J. Urrutia, and B. Vogtenhuber, 
	Modem Illumination of Monotone Polygons. 
	\emph{In: Proc. 25th European Workshop on Computational Geometry EuroCG'09}, 
	pp.~167--170, Brussels, Belgium, 2009.



\bibitem{Kran2} M. Barbeau and E. Kranakis, 
	Principles of Ad Hoc Networking, 
	Wiley, 2007, ISBN:~978-0-470-03290-0. 

\bibitem{Ball}   
	B. Ballinger, N. Benbernou, P. Bose, M. Damian, E.D. Demaine, V. Dujmovic, R. Flatland, F. Hurtado, J. Iacono, A. Lubiw, P. Morin, V. Sacrist\'an, D. Souvaine, and R. Uehara, 
	Coverage with k-transmitters in the presence of obstacles, 
	\emph{Journal of Combinatorial Optimization}, 25(2):208--233, 2013.

\bibitem{BMSU} 
	P. Bose, P. Morin, I. Stojmenovic, and J. Urrutia,
    Routing with guaranteed delivery in ad hoc wireless networks'. 
	\emph{In: Proc. of 3rd ACM Int.\ Workshop on Discrete Algorithms and Methods or Mobile Computing and Communications DIAL M99}, 
	pp.~48--55, Seattle, USA, 1999.
 
\bibitem{np} 
	S. Cannon, T. Fai, J. Iwerks, U. Leopold, and C. Schmidt, 
NP-hardness proofs for point and edge 2-transmitters,
\emph{In: Proc. 24th Fall Workshop on Computational Geometry (FWCG 2014)}, 
Storrs, Canada, 2014.

\bibitem{edgetrans} 
	S. Cannon, T. Fai, J. Iwerks, U. Leopold, and C. Schmidt, 
	Combinatorics of edge 2-transmitter art gallery problems,
	\emph{In: Proc. 31st European Workshop on Computational Geometry (EuroCG 2015)}, 
	pp. 40--43, 
	Ljubljana, Slovenia, 2015.

\bibitem{Ruy}  
R. Fabila-Monroy, A. J. Ruiz-Vargas, and J. Urrutia, 
	On Modem Illumination Problems. 
	\emph{In: Proc. XIII Encuentros de Geometria Computacional}, 
	pp.~9-19, 
	Zaragoza, Espa\~na, 
2009. 



\bibitem{CHOU} 
	D. Christ, M. Hoffmann, Y. Okamoto, T. Uno,
	Improved Bounds for Wireless Localization.
	\emph{In: Proc. 11th Scandinavian Workshop on Algorithm Theory}, 
	pp.~77--89, 
	Gothenburg, Sweden, 2008. 

\bibitem{Chvatal}
	V. Chv\'atal,
	A combinatorial theorem in plane geometry. 
	\emph{Journal of Combinatorial Theory, Series B}, 18:39--41, 1975. 

\bibitem{DH}
         F. Duque, and C. Hidalgo-Toscano.
         An upper bound on the k-modem illumination problem.
         \emph{arXiv preprint arXiv:1410.4099}, 2014.

\bibitem{EGS}
	D. Eppstein, M.T. Goodrich, N. Sitchinava,
	Guard Placement for Efficient Point-in-Polygon Proofs.
	\emph{In: Proc. 23rd Symposium on Computational Geometry}, 
	pp. 27--36, 
	Gyeongju, South Korea,
	2007.

\bibitem{Pach}  
	R. Fulek, A. F. Holmsen, and J. Pach. 
	Intersecting convex sets by rays. 
	\emph{Discrete Comput. Geom.}, 
	42(3):343, 358, 2009. 

\bibitem{Krana} 
	M. Hassinen, J. Kaasinen, E. Kranakis, V. Polishchuk, J. Suomela, A. Wiese, 
	Analyzing Local Algorithms in Location Aware Quasi Unit Disk Graphs. 
	\emph{Discrete Applied Mathematics}, 159(15):1566--1580, 2011.

\bibitem{ORourke}
	J. O'Rourke,
	Art gallery problems and algorithms,
	Oxford University Press,1987.

\bibitem{ORourke2} 
	J. O'Rourke, 
	Computational Geometry Column 52
	\emph{SIGACT News}, 43(1):82--85, 2012. 


\bibitem{Shermer} 
	T. Shermer, 
	Recent Results in Art Galleries, 
	\emph{Proceedings of the IEEE}, 80(9):1384--1399, 1992.

\bibitem{Urrutia}
	J. Urrutia,
	Art Gallery and Illumination Problems.
	\emph{In: J.R. Sack, J. Urrutia (eds.) Handbook of Computational Geometry},
	pp. 973--1027, Elsevier Science Publishers B.V., 2000.

\bibitem{Urrutia2} 
	J. Urrutia, 
	Local solutions for global problems in wireless networks. 
	\emph{Journal of Discrete Algorithms}, 5:395--407, 2007.

\bibitem{Watten} 
D. Wagner, R. Wattenhofer (eds), 
	Algorithms for Sensor and Ad Hoc Networks. 
	Springer 2007.



	



\end{thebibliography}




\end{document}
