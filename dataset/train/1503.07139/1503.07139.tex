

The idea of \SAlA is to exactly mimic the external behavior of $\Qsyse$ in \eqref{equ:Prelim} over finite time intervals of length $l+1$. 
We therefore consider the behavioral system $\Ee=\Tuple{\Nbn,\weS,\BeheQ}$, where $\BeheQ$ is the extension of $\projState{\weS}{\Behf(\Qsyse)}$ to $\Zb$ as discussed in \REFsec{sec:prelim}. 
All finite strings of external symbols of length $l$ which are consistent with the dynamics of $\Qsyse$ are given by
\begin{align}
&\Pi_{l}(\BeheQ):=\bigcup_{k\in\Nbn}\BeheQ\ll{k-l+1,k}.\label{equ:Ds}
\end{align}
Now consider the following gedankenexperiment: assume playing a sophisticated domino game where $\Pi_{l+1}(\BeheQ)$ is the set of dominos. Pick the first domino to be $\BeheQ\ll{-l,0}$ (i.e., a domino with only diamonds except for the last symbol) and append any domino from the set $\Pi_{l+1}(\BeheQ)$ if the last $l$ symbols of the first domino are the same as the first $l$ symbols of the second domino (see Figure~\ref{fig:DominoGame} (left) for an example).
Playing the domino game arbitrarily long and with all possible initial conditions and domino combinations results in the largest, in the sense of set inclusion, behavior $\Behal$ satisfying 
\begin{subequations}\label{equ:BehalW}
\begin{align}
 &\Behal\ll{-l,0}=\BeheQ\ll{-l,0}~\text{and}~\label{equ:BehalW:a}\\
 &\Pi_{l+1}(\Behal)=\Pi_{l+1}(\BeheQ), \label{equ:BehalW:b}
 \end{align}
\end{subequations}
defining the behavioral system ${\Eal=\Tuple{\Nbn,\weS,\Behal}}$.
Observe that the smaller $l$, the less information in the domino game is used, which generates more freedom in constructing signals, implying $\Behal\supseteq\allowbreak\abst{\Beh{}}^{l+1}\supseteq\allowbreak\BeheQ$ for all $l\in\Nbn$. 
This motivates the use of $\Behal$ as an over-approximation of the behavior $\BeheQ$.
Obviously, equality ${\Beha^r=\BeheQ}$ holds for all $r\geq l$ if $\BeheQ$ is itself the largest behavior satisfying \eqref{equ:BehalW}. In \cite{SchmuckRaisch2014_ControlLetters}, a system $\E=\Tuple{\Nbn,\weS,\BeheQ}$ for which the latter is true was called \emph{asynchronously $l$-complete} which inspired the name of \SAlA. Following \cite{SchmuckRaisch2014_ControlLetters}, $\Eal$ constructed in the outlined domino game is the unique \SAlA of $\Ee=\Tuple{\Nbn,\weS,\BeheQ}$.  However, we are usually interested in a state machine \emph{realizing} its step by step evolution.

\begin{figure}
\begin{center}
\begin{tikzpicture}[auto, node distance=0.5cm,scale=1]

  \tikzstyle{myblock} = [draw,rectangle, text centered, minimum height=0.5cm, minimum width=1.4cm,thick]
 \node[myblock] (b1) at (0,0) {$\diamond~~\diamond~~a$};
 \node[myblock,below of=b1,xshift=0.5cm,node distance=0.7cm] (b2) {$\diamond~~a~~b$};
 \node[myblock,below of=b2,xshift=0.5cm,node distance=0.7cm] (b3) {$a~~b~~a$};
  \node[myblock,below of=b3,xshift=0.5cm,node distance=0.7cm] (b4) {$b~~a~~a$};
   \node[below of=b4,xshift=0.4cm,node distance=0.5cm] (b5) {$\hdots$}; 
 \draw [draw,->,ultra thick] (0.6,-3cm) -- ++ (2cm,0);
\foreach \x in {0.6,1.1,1.55,2}
\draw (\x cm,-2.9cm) -- (\x cm,-3.1cm);
\draw (0.6,-3.3cm) node {$ 0 $};
\draw (1.55,-3.3cm) node {$ 2 $};
\draw (2.9,-3.3cm) node {$\Nbn$};


  \tikzstyle{myblock} = [draw,rectangle, text centered, minimum height=0.5cm, minimum width=0.5cm]
   \tikzstyle{myblocks} = [draw,rectangle, text centered, minimum height=0.5cm, minimum width=0.5cm,postaction={pattern color=black!70,pattern=north west lines}] 
   


   \def\n{4} \def\c{1.3cm}
   
  \node[myblock] (b1) at (\n,-\c) {};
  \node[myblocks, right of=b1] (b2)  {};
   \node[myblocks, right of=b2] (b3) {};
   \node[myblocks, right of=b3] (b4) {};
   \node[myblocks, right of=b4] (b5) {}; 
   \node[myblocks,below of=b2,node distance=0.7cm] (a1) {};
  \node[myblocks, right of=a1] (a2) {};
   \node[myblocks, right of=a2] (a3) {};
   \node[myblocks, right of=a3] (a4) {};
   \node[myblock, right of=a4] (a5) {$\w$};   
\node[draw,rectangle, text centered, minimum height=0.5cm, minimum width=2.5cm,ultra thick] () at (b3) {};
    \node[draw,rectangle, text centered, minimum height=0.5cm, minimum width=2.5cm,ultra thick] () at (a3) {};  
\draw [decorate,decoration={brace,amplitude=6pt},thick] (b2.north west) -- (b5.north east) node [black,midway,yshift=0.1cm] (bc){$\xa$};
   \draw [decorate,decoration={brace,amplitude=6pt},thick] (a5.south east) -- (a2.south west) node [black,midway,yshift=-0.1cm] (bc){$\xa'$};
   \draw [draw,-,ultra thick,dotted] (\n-0.4,-1.7cm-\c) -- ++ (\n+0.3cm,0);
 \draw [draw,->,ultra thick] (\n-0.1,-1.7cm-\c) -- ++ (\n+3cm,0);
 \def\d{0.5}
\foreach \i in {0,...,5}
\draw (\n+\i*\d,-1.6cm-\c) -- (\n+\i*\d,-1.8cm-\c);
\draw (\n+0.5,-2cm-\c) node {$t' $};
\draw (\n+2.5,-2cm-\c) node {$ t$};
\draw (\n+3.3,-2cm-\c) node {$\Nbn$};



\end{tikzpicture}   \end{center}
  \caption{Example of a domino game for $l=2$ (left) and an illustration of the usual choice $\xaS^l$ in \REFprop{prop:SforLcomplete} for $t>l=4$ with $t'=t-l$ (right).}\label{fig:DominoGame}
\end{figure}
\begin{definition}\label{def:SAlA_Realization}
 Given \eqref{equ:Prelim} and \eqref{equ:BehalW}, the dynamical system ${\Eal=\Tuple{\Nbn,\weS,\Behal}}$ is the \SAlA of $\Ee=\Tuple{\Nbn,\weS,\BeheQ}$. A state machine $\Qsysa$ is a realization of $\Eal$ 
iff\footnote{As before,  $\Behe(\Qsysa)$ denotes the extension of $\projState{\weS}{\Behf(\Qsysa)}$ to $\Zb$.} $\Behal=\Beh(\Qsysa)$.
\end{definition}
In the work on \SlA and \SAlA the state space $\xaS$ to construct the realization $\Qsysa$ of the abstraction $\Eal$ is usually chosen such that the state represents the \enquote{recent past} of length $l$ of the external signal. Recalling the gedankenexperiment, this choice of $\xaS$ is motivated by the fact that the  next feasible domino of length $l+1$ is determined by the last $l$ symbols of the previous domino (see \REFfig{fig:DominoGame} (right) for an illustration).
Using this state space, the standard state machine realization of \SAlA, denoted by $\Qsysa^l$ in this paper, is defined as follows.

\begin{proposition}[\cite{SchmuckRaisch2014_ControlLetters}, Thm.4]\label{prop:SforLcomplete}
Let $\Eal=\Tuple{\Nbn,\weS,\Behal}$ be the \SAlA of $\Ee$ and define
\begin{subequations}\label{equ:Qsysal_old}
\begin{align}\allowdisplaybreaks
 \xaS^l:=&\Set{\diamond}^l\cup\Pi_l(\Behal),\\
 \xaSo{}^l:=&\Set{\diamond}^l,~\text{and}\\
 \tra^l:=&\SetCompX{\Tuple{\xa,\we,\BR{\xa\sconc\we}\ll{1,l}}}{\xa\sconc\we\in\Pi_{l+1}(\Behal)}.
\end{align}
\end{subequations}
Then $\Eal$ is realized by $\Qsysa^l=\Tuple{\xaS^l,\weS,\tra^l,\xaSo{}^l}$.
\end{proposition}




Summarizing the abstraction procedure outlined above, constructing the finite state abstraction $\Qsysa^l$ in \REFprop{prop:SforLcomplete} using \SAlA only requires knowledge about the set $\Pi_{l+1}(\BeheQ)$.
However, if $\Qsyse$ is available, we can construct $\Qsysa^l$ from $\Qsyse$ directly,
as shown in the following section. 




\subsection{Some State Machine Realizations of \SAlA}\label{sec:SAlA_SM}

Recall from \REFprop{prop:SforLcomplete} that the set of external sequences of length $l$, given by $\Pi_l(\Behal)=\Pi_l(\BeheQ)$ (from \eqref{equ:BehalW:b}), is finite. We now investigate how to use this set as a state space in the construction of different state machine realizations of the \SAlA of a system $\E$. This is be done on the basis of a state machine realization $\Qsyse$ of $\Ee$ satisfying \eqref{equ:Prelim}. For this,
we first investigate how a string $\zeG\in\Pi_l(\BeheQ)$ can correspond to a state $\xe\in\xeS$ of $\Qsyse$. Observe that $\zeG$ is a string of length $l$ and $\xe$ is a state reached at a particular time $k\in\Nbn$. We consider the cases where $\zeG$ is generated by $\Qsyse$ immediately \emph{before}, immediately \emph{after} or \emph{while} $\xe$ was reached. This leads us to a set of intervals 
\begin{equation}\label{equ:Prelim:I}
 \Ilm=[m-l,m-1]\quad\SUCHTHAT~ l,m\in\Nbn,~\text{and}~m\leq l,
\end{equation}


where\footnote{The addition of two intervals is interpreted in the usual sense, i.e., $[a,b]+[c,d]=[a+c,b+d]$.} $[k,k]+\Interval^l_0=[k-l,k-1]$ corresponds to the first,  $[k,k]+\Interval^l_l=[k,k+l-1]$ corresponds to the second, and for all other choices of $m$, $[k,k]+\Ilm$ corresponds to the third case. Based on \eqref{equ:Prelim:I} the sets of compatible states are introduced in \REFdef{def:Xxr} and illustrated in \REFfig{fig:Enabl}.\begin{figure}
\begin{center}
\begin{tikzpicture}[auto,scale=1]
 
 \tikzset{
>=stealth',
help lines/.style={dashed, thick},
axis/.style={<->},
important line/.style={thick},
connection/.style={thick, dotted},
}

\def\y{0} \def\yd{0.5} \def\x{0} \def\xd{1} \def\xmax{10} \def\ymax{2}

\coordinate (zero) at (\x,\y);
\coordinate (top) at (\x,\y+5);
\coordinate (start) at (\x,\y+2);

\foreach \nx/\ny/\a in {1/0/a,2/1/b,3/2/c,4/1.5/c,5/0.5/b,6/0.8/b}{
\draw[help lines] (\x+\nx*\xd,\y) -- (\x+\nx*\xd,\y+\ymax-0.5);
\coordinate (x\nx) at start+(\nx*\xd,\ny*\yd);
\fill [black] (x\nx) node {}  circle [radius=3pt];
\draw (\nx*\xd,\ymax-0.2) node {$ \a $};
}
\draw (\x+0.2,\ymax-0.2) node {$ \hdots $};
\draw (\x-0.5,\ymax-0.2) node {$ \yeG$:};
\draw (6.8*\xd,\ymax-0.2) node {$ \hdots $};
\draw (x4)+(0.3,0) node {$x$};



\foreach \a/\b/\c/\z in {1/3/0/0,2/4/0.2/1,3/5/0.4/2,4/6/0.6/3}{
\draw[important line] (\a*\xd-0.1,\ymax+0.05+\c) -- (\a*\xd-0.1,\ymax+0.15+\c) --(\b*\xd+0.1,\ymax+0.15+\c) --(\b*\xd+0.1,\ymax+0.05+\c);
\draw (\a*\xd+0.4,\ymax+\c+0.35) node {$\zeG_{\z}$};
}



\draw[important line] (x1) to[out=0,in=-140] (x2) to[out=40,in=-180] (x3) to[out=0,in=-210] (x4) to[out=-30,in=-190] (x5) to[out=-10,in=-150] (x6);
\draw[black, dotted,thick] (\x,\y+\yd) to[out=-10,in=-180] (x1);
\draw[black, dotted,thick] (x6) to[out=20,in=-150] (\x+7*\xd,\y+2*\yd);
\draw (\x-0.5,\y+\yd) node {$ \xeG $:};






 


 
\end{tikzpicture}
   \end{center}
 \caption{Illustration of corresponding external sequences $\zeG_m\in\ON{E}^{\Interval^{3}_{m}}(\xe),~m\in\{0,\hdots,3\}$ for state $\xe=\xeG(k)$ where $\weS=\yeS=\Set{a,b,c}$ and $\yeG(k)\in\ON{H}_{\tre}(\xe)$ for some $k\in\Nbn$.}\label{fig:Enabl}
\end{figure}
\begin{definition}\label{def:Xxr}
Given \eqref{equ:Prelim} and \eqref{equ:Prelim:I}, let $\ESe{}=(\Nbn,\weS\times\xeS,\BeheSQ)$ be a dynamical system, where $\BeheSQ$ is the extension of $\projState{\weS\times\xeS}{\Behf(\Qsyse)}$ to $\Zb$ as discussed in \REFsec{sec:prelim}. Then the set of \emph{corresponding external strings w.r.t. $\Ilm$} is defined for every state $\xe\in\xeS$ by
\begin{equation}\label{equ:Xxr}
\EnabWl{}{\Ilm}{\xe}\hspace{-1mm}\deff\SetCompX{\zeG}{
\ExQ{\Tuple{\weG,\xeG}\in\BeheSQ,k\in\Nbn\hspace{-1mm}}{\hspace{-1mm}\begin{propConjA}
                                         \xeG(k)=\xe\\
					\zeG=\weG|_{[k,k]+\Ilm}
                                         \end{propConjA}
}\hspace{-1mm}}\hspace{-1mm}.
\end{equation}
Furthermore, if 
\begin{equation}\label{equ:future_unique}
 \AllQ{\xe\in\xeS,\zeG,\zeG'\in\EnabWl{}{\Ilm}{\xe}}{\zeG\ll{l-m,l-1}=\zeG'\ll{l-m,l-1}}
\end{equation}
$\Qsyse$ is called \emph{future unique} w.r.t. $\Ilm$.
\end{definition}



Observe, that $\zeG,\zeG'\in\EnabWl{}{\Ilm}{\xe}$ in \eqref{equ:future_unique} are obtained from two trajectories $\Tuple{\weG,\xeG},\Tuple{\weG',\xeG'}\in\BeheSQ$ passing $\xe$ at time $k\in\Nbn$ and $k'\in\Nbn$, respectively, (i.e., $\xeG(k)=\xeG'(k')=\xe$) using \eqref{equ:Xxr}. During this restriction of $\weG$ (resp. $\weG'$) to $\zeG$ (resp. $\zeG'$) absolute time information is disregarded (see \REFsec{sec:notation}), implying $\zeG\ll{l-m,l-1}=\weG\ll{k,k+m-1}$ and $\zeG'\ll{l-m,l-1}=\weG\ll{k',k'+m-1}$. Therefore, 
$\Qsyse$ is future unique w.r.t. $\Ilm$ if for all states $\xe\in\xeS$ all trajectories passing $\xe$ have the same $m$-long (non-strict) future of external symbols, i.e. $\weG\ll{k,k+m-1}=\weG'\ll{k',k'+m-1}$. Using this intuition it is easy to see that $\Qsyse$ is always \emph{future unique} w.r.t. $\Interval^l_0=[-l,-1]$, as this interval has no future.\\
We now proceed by constructing $m$ finite state machines using the outlined correspondence between $\xeS$ and $\Pi_{l}(\BeheQ)$.

\begin{definition}\label{def:QsysalW}
Given \eqref{equ:Prelim} and \eqref{equ:Prelim:I}, define 
\begin{subequations}\label{equ:QsysalW}
\begin{align}
\xalS:=&\SetCompX{\zeta}{\ExQ{\xe\in\xeS}{\zeta\in\EnabWl{}{\Ilm}{\xe}}},\label{equ:xalS}\\
\xalSo{}:=&\SetCompX{\zeta}{\ExQ{\xe\in\xeSo{}}{\zeta\in\EnabWl{}{\Ilm}{\xe}}},~\text{and}\label{equ:xalSo}\\
\tral\hspace{-1mm}:=&\SetCompX{\Tuple{\xa,\ue,\ye,\xa'}}{
\begin{propConjA}
\xa'\ll{0,l\mips m\mips1}=\BR{\xa\ll{0,l\mips m\mips1}\sconc\projState{\weS}{\ue,\ye}}\ll{1,l\mips m}\hspace{-1mm}\\[0.1cm]
~\xa\ll{l\mips m,l\mips 1}=\BR{\projState{\weS}{\ue,\ye}\sconc\xa'\ll{l\mips m,l\mips 2}}\ll{0,m-1}\\[0.1cm]
\ExQ{\xe,\xe'\in\xeS}{
\begin{propConjA}
 \xa\in \EnabWl{}{\Ilm}{\xe}\\
 \xa'\in \EnabWl{}{\Ilm}{\xe'}\\
\Tuple{\xe,\ue,\ye,\xe'}\in\tre
\end{propConjA}}
\end{propConjA}
}\hspace{-1mm}.\label{equ:tral}\end{align}
\end{subequations}Then $\Qsysal=\Tuple{\xalS,\ueS,\yeS,\tral,\xalSo{}}$ is called the $\Ilm$-abstract state machine of $\Qsyse$. 
\end{definition}
 


The construction of the abstract state machines in \REFdef{def:QsysalW} can be interpreted as follows.
Using \eqref{equ:xalS} instead of $\xalS=\Set{\diamond}^l\cup\Pi_{l}(\BeheQ)$ ensures that $\Qsysal$ is live and reachable, which is purely cosmetic but allows to simplify subsequent proofs. The last line in the conjunction of \eqref{equ:tral} simply says that we have a transition in $\Qsysal$ from $\xa$ to $\xa'$ if there is a transition in $\Qsyse$ between any two states compatible with $\xa$ and $\xa'$, respectively. However, the first two lines in the conjunction of \eqref{equ:tral} additionally ensure that $\xa$ and $\xa'$ obey the rules of the domino game, i.e.,
\begin{equation*}
 \xa\ll{1,l-1}=\xa'\ll{0,l-2}
\end{equation*}
as depicted in \REFfig{fig:DominoGame} (right) and the current external symbol $\we=\projState{\weS}{\ue,\ye}$ is contained in either $\xa$ or $\xa'$ or both, at the position corresponding to the current time point, i.e.,
\begin{align*}
 \we&=\xa'(l-1)~\text{if}~m=0,\\
 \we&=\xa(l-m)=\xa'(l-1-m)~\text{if}~0<m<l~\text{and}\\
 \we&=\xa(0)~\text{if}~m=l.
\end{align*}
As we are interested in state machine realizations of \SAlA, we show that $\Qsysal$ realizes $\Eal$ for all choices of $l$ and $m$.

\begin{theorem}\label{thm:behequ}
Given \eqref{equ:Prelim} and \eqref{equ:Prelim:I}, let $\Qsysal$ be defined as in \REFdef{def:QsysalW} and let $\Eal=\Tuple{\Nbn,\weS,\Behal}$ be the unique \SAlA of $\Ee=\Tuple{\Nbn,\weS,\BeheQ}$. Then $\Qsysal$ realizes $\Eal$.\end{theorem}
\begin{proof}
See Appendix~\ref{proof:thm:behequ}.
\end{proof}



As an intuitive consequence of \REFthm{thm:behequ}, choosing $m=0$ and the full external symbol set $\weS=\ueS\times\yeS$ when constructing $\Qsysal$ in \REFdef{def:QsysalW} yields the standard realization $\Qsysa^l$ of \SAlA.

 \begin{theorem}\label{thm:Qsysalo_equ}
 Given \eqref{equ:Prelim} and \eqref{equ:Prelim:I} with $\weS=\ueS\times\yeS$, let $\Qsysa^l$ and $\Qsysal$ as in \REFprop{prop:SforLcomplete} and \REFdef{def:QsysalW}, respectively. Then $\Qsysa^l=\Qsysa^{\Interval^l_0}$.
 \end{theorem}
 
 \begin{proof}
 See Appendix~\ref{proof:thm:Qsysalo_equ}.
 \end{proof}
 


\subsection{Ordering $\Qsysal$ based on Simulation Relations}

Before we discuss the ordering between abstract state machines based on changing $l$ and $m$, we show under which conditions the obtained abstraction $\Qsysal$ simulates the original state machine $\Qsyse$ and when both state machines are bisimilar. This investigation is interesting for the comparison to \QBA, as the latter always simulates the original state machine $\Qsyse$. Furthermore, the framework of \QBA allows to construct a bisimilar abstraction whenever the employed repartitioning algorithm terminates. Hence, it is interesting to know if the latter is also true for \SAlA.\\
The investigation of similarity between $\Qsysal$ and $\Qsyse$ requires the construction of a relation between the original state space $\xeS$ and the abstract state space $\xalS$. As $\xalS$ defines a cover for $\xeS$ where each cell is given by all states $\xe$ corresponding to a string $\zeG\in\xalS$ via $\EnabWl{}{\Ilm}{}$, the latter is a natural choice for a relation between $\xeS$ and $\xalS$.\\
Recall from \REFthm{thm:behequ} that the behaviors of $\Qsyse$ and $\Qsysal$ coincide if $\BeheQ$ is asynchronously $l$-complete. Behavioral equivalence is always necessary for a relation $\R$ to be a bisimulation relation but usually not sufficient. We therefore introduce a stronger condition, called \emph{state-based asynchronous $l$-completeness}, to serve the latter purpose.

\begin{definition}\label{def:SBalc}
Given \eqref{equ:Prelim}, $\Qsyse$ is \emph{state-based asynchronously $l$-complete w.r.t. $\Ilm$} if
 \begin{equation}\label{equ:dominolcomplete}
  \AllQ{\xe\in\xeS,\zeG\in\Pi_{l+1}(\BeheQ)}{\propImp{\zeG\ll{0,l-1}\in\EnabWl{}{\Ilm}{\xe}}{\zeG\in\EnabWl{}{[m-l,m]}{\xe}}}.
 \end{equation}
\end{definition}

\begin{remark}\label{rem:SBalc}
Recall from the beginning of this section that the dynamical system $\Ee=\Tuple{\Nbn,\weS,\BeheQ}$ is asynchronously $l$-complete, as  defined in \cite[Def.6]{SchmuckRaisch2014_ControlLetters}, if $\BeheQ$ is the largest behavior satisfying \eqref{equ:BehalW} itself. Intuitively, the latter is true if \emph{for all} $\zeG\in\Pi_{l+1}(\BeheQ)$ there \emph{exists} an $\xe\in\xeS$ s.t. the second part of \eqref{equ:dominolcomplete} holds. Therefore, asynchronous $l$-completeness of $\Ee$ is always implied by \eqref{equ:dominolcomplete}, but not vice-versa. 
\end{remark}

\begin{theorem}\label{thm:SimRel_QsyseQsysal}
Given \eqref{equ:Prelim}, \eqref{equ:Prelim:I} and $\Qsysal$ as in \REFdef{def:QsysalW}, let
\begin{equation}
\R=\SetCompX{\Tuple{\xe,\xa}\in\xeS\times\xalS}{
\xa\in\EnabWl{}{\Ilm}{\xe}
}\label{equ:R0}.\end{equation}
Then it holds that\footnote{Using $\mathfrak{R}_{\ueS\times\yeS}$ instead of $\mathfrak{R}_{\weS}$ in (i) is done on purpose and indicates that this relation holds for $\ueS\times\yeS$ independent from the choice of $\weS$.} 
\begin{compactenum}[(i)]
 \item $\propAequ{\R\in\SR{\ueS\times\yeS}{}{\Qsyse}{\Qsysal}}{\text{$\Qsyse$ is future unique w.r.t. $\Ilm$}}$ and
 \item $\propAequ{{\R^{-1}\in\SR{\weS}{}{\Qsysal}{\Qsyse}}}{\text{$\Qsyse$ is state-based asych. $l$-complete w.r.t. $\Ilm$}}$.
\end{compactenum}
\end{theorem}

\begin{proof}
See Appendix~\ref{proof:thm:SimRel_QsyseQsysal}. \end{proof}

Intuitively, $\Qsysal$ simulates $\Qsyse$ w.r.t. $\weS$ if for every related state pair $\Tuple{\xe,\xa}\in\R$ and every transition $\Tuple{\xe,\ue,\ye,\xe'}\in\tre$ which $\Qsyse$ \enquote{picks}, $\Qsysal$ can \enquote{pick} a transition $\Tuple{\xa,\ue',\ye',\xa'}\in\tral$ s.t. $\we=\projState{\weS}{\ue,\ye}=\projState{\weS}{\ue',\ye'}$. However, if $m>0$, a state $\xa\in\xalS$ has only outgoing transitions s.t. $\we=\xa(l-m)$. Therefore, $\Qsysal$ can only simulate $\Qsyse$ iff in every state $\xe\in\xeS$ all outgoing transitions agree on this $\we$, i.e., $\Qsyse$ is \enquote{output deterministic} w.r.t. $\weS$. For $m>1$ applying this reasoning iteratively gives the (rather restrictive) condition of future uniqueness of $\Qsyse$.
As the outlined problems are absent for $m=0$ (as $\Qsyse$ is always future unique w.r.t. $\Interval^l_0$), $\Qsysa^{\Interval^l_0}$, which we know to coincide with the original realization $\Qsysa^l$ of \SAlA for $\weS=\ueS\times\yeS$, always simulates $\Qsyse$.

\begin{corollary}
 Given \eqref{equ:Prelim}, \eqref{equ:Prelim:I} and $\Qsysa^{\Interval^l_0}$ as in \REFdef{def:QsysalW} it holds that 
$\Qsyse\kgl{\ueS\times\yeS}{}\Qsysa^{\Interval^l_0}$.
\end{corollary}

 \begin{remark}
 In the context of \SlA a state machine $\Qsysa^{l^+}$ was introduced in \cite{Raisch2010} whose state at time $k$ represents the string of external symbols from time $k-l+1$ to time $k$, i.e., from the interval $k+\Interval^l_1$. While the state sets of  $\Qsysa^{l^+}$ and $\Qsysa^{\Interval^l_1}$ coincide, their transition structure slightly differs. This is a consequence of the fact that $\Qsysa^{l^+}$ was intended to serve as a set-valued observer for the states of $\Qsyse$.
\end{remark}
  
Recalling the domino game, we know that using longer dominos (i.e., increasing $l$) gives less freedom in composing them and therefore yields a tighter abstraction. This intuition carries over to the state space realizations of $\Eal$, inducing an ordering in terms of simulation relations. 

\begin{theorem}\label{thm:SimRel_QsysalpmQsysalm}
Given \eqref{equ:Prelim}, \eqref{equ:Prelim:I} and $\Qsysal$ as in \REFdef{def:QsysalW}, let
\begin{equation}\label{equ:R_l}
  \R=\SetCompX{\Tuple{\xa_{l+1},\xa_l}\in\xaS^{{\Ilpm}}\times\xaS^{{\Ilm}}}{
  \xa_{l}=\xa_{l+1}\ll{1,l}}.
 \end{equation}
Then it holds that 
 \begin{compactenum}[(i)]
  \item $\R\in\SR{\weS}{}{\Qsysa^{\Ilpm}}{\Qsysa^{\Ilm}}$ and 
  \item $\propAequ{\R^{-1}\in\SR{\weS}{}{\Qsysa^{\Ilm}}{\Qsysa^{\Ilpm}}}{\Behal=\Beha^{l+1} }$.
 \end{compactenum}
\end{theorem}

\begin{proof}
See Appendix~\ref{proof:thm:SimRel_QsysalpmQsysalm}. 
\end{proof}

\REFthm{thm:SimRel_QsysalpmQsysalm} (ii) implies that the accuracy of the abstraction cannot be increased by increasing $l>r$ if $\BeheQ$ is asynchronously $r$-complete and $m$ is fixed, e.g. $m=0$. Therefore, the standard realization $\Qsysa^l$ for \SAlA might never result in a bisimilar abstraction of $\Qsyse$, no matter how large $l$ is chosen, even if $\Ee=\Tuple{\Nbn,\weS,\BeheQ}$ is asynchronously $r$-complete. This is due to the fact that 
\eqref{equ:dominolcomplete} is not implied by asynchronous $l$-completeness of $\Ee$ (see \REFrem{rem:SBalc}).\\
Interestingly, we will show that increasing $m$, i.e., shifting the interval into the future, results in a tighter abstraction w.r.t. simulation relations, i.e. allows to increase the precision of $\Qsysal$ for $l\geq r$ even if $\Ee$ is $r$-complete. 


\begin{theorem}\label{thm:SimRel_QsysalmpQsysalm}
Given \eqref{equ:Prelim}, \eqref{equ:Prelim:I}, and $\Qsysal$ as in \REFdef{def:QsysalW} with $m<l$, let
\begin{equation}\label{equ:R_m}
 \R=\SetCompX{\Tuple{\xa_{m+1},\xa_m}\in\xaS^{{\Ilmp}}\times\xaS^{{\Ilm}}}{
  \begin{propConjA}
  \xa_{m+1}\ll{0,l-2}=\xa_m\ll{1,l-1}\\
  \ExQ{\xe\in\xeS}{
  \begin{propConjA}
   \xa_{m+1}\in\EnabWl{}{{\Ilmp}}{\xe}\\
   \xa_{m}\in\EnabWl{}{{\Ilm}}{\xe}
\end{propConjA}}
  \end{propConjA}}.
 \end{equation}
Then it holds that
\begin{compactenum}[(i)]
 \item  $\R\in\SR{\weS}{}{\Qsysa^{\Ilmp}}{\Qsysa^{\Ilm}}$ and 
 \item $
  \propAequ{\R^{-1}\in\SR{\weS}{}{\Qsysa^{\Ilm}}{\Qsysa^{\Ilmp}}}{
 \begin{propConjA}
  \text{$\Qsyse$ is future unique w.r.t. $\Ilmp$}\\
  \text{$\Qsyse$ is state-based async. $l$-complete w.r.t. $\Ilm$}
 \end{propConjA}}
 $
\end{compactenum}
\end{theorem}

\begin{proof}
 See Appendix~\ref{proof:thm:SimRel_QsysalmpQsysalm}.
\end{proof}
It is important to note that future uniqueness and state-based asynchronous $l$-completeness are incomparable properties, i.e., none is implied by the other. Therefore, there exist situations where $\Qsysal$ with $m>0$ simulates $\Qsyse$ (i.e., $\Qsyse$ is future unique w.r.t. $\Ilmp$) and $\Qsysal$ is tighter than $\Qsysa^{\Interval^l_0}$ in terms of simulation relations. However, if $\Qsyse$ is both future unique and state-based asynchronously $l$-complete w.r.t. a particular interval $\Interval^r_n$, \REFthm{thm:SimRel_QsysalmpQsysalm} implies that increasing $l>r$ and $m>n$ will not result in a tighter abstraction. Moreover, this is not necessary anyway, as \REFthm{thm:SimRel_QsyseQsysal} implies that in this case $\Qsysa^{\Interval^r_n}$ is bisimilar to $\Qsyse$. 



\subsection{Example}\label{sec:SAlCA_new_exp}
We conclude this section with a detailed example illustrating the construction of $\Ilm$-abstract state machines and the property of future uniqueness and state-based asynchronous $l$-completeness for different choices of $l$ and $m$.
For simplicity, we consider a \emph{finite} state machine 
 \begin{align}
  \Qsyse&=\Tuple{\xeS,\ueS\times\yeS,\tre,\xeSo{}}\quad\SUCHTHAT\quad \weS=\yeS \label{equ:example:Qsyse}\end{align}
 as the original model, whose transition structure is depicted in \REFfig{fig:exp_Qsyse}.\begin{figure}
  \begin{center}
\begin{tikzpicture}[auto]
\def\dh{1.5} \def\dv{1}

\def\h{0} \def\v{0}

\node (name) at (\h-1,\v+0.5) {$\Qsyse:$};

\node[istate] (x1) at (\h,\v) {$\xe_1$};
\node[Sstate] (x2) at (\h+\dh,\v) {$\xe_2$};
\node[Sstate] (x3) at (\h+\dh,\v-\dv) {$\xe_3$};
\node[Sstate] (x4) at (\h+\dh,\v-2*\dv) {$\xe_4$};
\node[istate] (x5) at (\h,\v-2*\dv) {$\xe_5$};
\SFSAutomatEdge{x1}{\EdgeLabelYt{\ue_1}{\ye_1}}{x2}{}{}
\SFSAutomatEdge{x2}{\EdgeLabelYt{\ue_2}{\ye_2}}{x3}{bend left}{}
\SFSAutomatEdge{x3}{\EdgeLabelYt{\ue_3}{\ye_3}}{x2}{bend left}{}
\SFSAutomatEdge{x3}{\EdgeLabelYt{\ue_3}{\ye_3}}{x4}{bend left}{}
\SFSAutomatEdge{x4}{\EdgeLabelYt{\ue_4}{\ye_4}}{x3}{bend left}{}
\SFSAutomatEdge{x5}{\EdgeLabelYt{\ue_1}{\ye_1}}{x4}{}{swap}

\end{tikzpicture}   \end{center}
  \caption{Transition structure of the state machine $\Qsyse$ in \REFfig{fig:exp_Qsyse}.}\label{fig:exp_Qsyse}
 \end{figure}
It can be inferred from \REFfig{fig:exp_Qsyse} that the output behavior of $\Qsyse$ is given by
  \begin{equation*}
   \BeheQ=\Set{y_1y_2((y_3y_2)^*(y_3y_4)^*)^\omega,y_1y_4((y_3y_2)^*(y_3y_4)^*)^\omega}
  \end{equation*}
  where $(\cdot)^*$ and $(\cdot)^\omega$ denote, respectively, the finite and infinite repetition of the respective string. Furthermore, the sets of $1$-long and $2$-long dominos obtained from $\BeheQ$ via \eqref{equ:Ds} are
  \begin{align*}
  \Ds{}{1}&=\yeS\quad\text{and}\quad\\
  \Ds{}{2}&=\Set{\diamond y_1,~y_1y_2,~y_1y_4,~y_2y_3,~y_3y_2,~y_3y_4,~y_4y_3}.\end{align*}
  To play the domino-game for $l=1$, i.e., with dominos from the set $\Ds{}{2}$, we have to pick $\diamond y_1$ as the initial domino and append dominos such that the last element of the first matches the first element of the second domino. It is easy to see that in this example every such combination of dominos yields a sequence contained in $\BeheQ$. Hence, $\Ee=\Tuple{\Nbn,\yeS,\BeheQ}$ is asynchronously $1$-complete and therefore also asynchronously $2$-complete.
  


Using $\Qsyse$ in \REFfig{fig:exp_Qsyse} we can construct the $\Interval^1_0$- and  $\Interval^1_1$-abstract state machines of $\Qsyse$ using \REFdef{def:QsysalW}. Their transition structures are depicted in \REFfig{fig:exp_Qsysa1}. Furthermore, we obtain the following properties of $\Qsyse$ w.r.t $\Interval^1_0$ and $\Interval^1_1$.


 \begin{figure}
  \begin{center}
\begin{tikzpicture}[auto]
\def\dh{1.5} \def\dv{1}



\def\h{0} \def\v{0}

\node (name) at (\h-0.5,\v+0.5) {$\Qsysa^{\Interval^1_0}:$};

\node[istate] (x0) at (\h,\v-\dv) {$\diamond$};
\node[Sstate] (x1) at (\h+0.5*\dh,\v-\dv) {$\StateLabelYo{1}$};
\node[Sstate] (x2) at (\h+1.5*\dh,\v) {$\StateLabelYo{2}$};
\node[Sstate] (x3) at (\h+1.5*\dh,\v-\dv) {$\StateLabelYo{3}$};
\node[Sstate] (x4) at (\h+1.5*\dh,\v-2*\dv) {$\StateLabelYo{4}$};
\SFSAutomatEdge{x0}{\EdgeLabelYt{\ue_1}{\ye_1}}{x1}{}{}
\SFSAutomatEdge{x1}{\EdgeLabelYt{\ue_2}{\ye_2}}{x2}{bend left}{}
\SFSAutomatEdge{x2}{\EdgeLabelYt{\ue_3}{\ye_3}}{x3}{bend left}{}
\SFSAutomatEdge{x3}{\EdgeLabelYt{\ue_2}{\ye_2}}{x2}{bend left}{}
\SFSAutomatEdge{x3}{\EdgeLabelYt{\ue_4}{\ye_4}}{x4}{bend left}{}
\SFSAutomatEdge{x4}{\EdgeLabelYt{\ue_3}{\ye_3}}{x3}{bend left}{}
\SFSAutomatEdge{x1}{\EdgeLabelYt{\ue_4}{\ye_4}}{x4}{bend right}{swap}



\def\h{4.5} \def\v{0}

\node (name) at (\h-0.5,\v+0.5) {$\Qsysa^{\Interval^1_1}:$};

\node[istate] (x1) at (\h,\v-\dv) {$\StateLabelYo{1}$};
\node[Sstate] (x2) at (\h+\dh,\v) {$\StateLabelYo{2}$};
\node[Sstate] (x3) at (\h+\dh,\v-\dv) {$\StateLabelYo{3}$};
\node[Sstate] (x4) at (\h+\dh,\v-2*\dv) {$\StateLabelYo{4}$};
\SFSAutomatEdge{x1}{\EdgeLabelYt{\ue_1}{\ye_1}}{x2}{bend left}{}
\SFSAutomatEdge{x2}{\EdgeLabelYt{\ue_2}{\ye_2}}{x3}{bend left}{}
\SFSAutomatEdge{x3}{\EdgeLabelYt{\ue_3}{\ye_3}}{x2}{bend left}{}
\SFSAutomatEdge{x3}{\EdgeLabelYt{\ue_3}{\ye_3}}{x4}{bend left}{}
\SFSAutomatEdge{x4}{\EdgeLabelYt{\ue_4}{\ye_4}}{x3}{bend left}{}
\SFSAutomatEdge{x1}{\EdgeLabelYt{\ue_1}{\ye_1}}{x4}{bend right}{swap}



\end{tikzpicture}   \end{center}
  \caption{$\Interval^1_0$- and $\Interval^1_1$-abstract state machines of $\Qsyse$ in \REFfig{fig:exp_Qsyse}.}\label{fig:exp_Qsysa1}
 \end{figure}

\begin{enumerate}
 \item[(A1)] $\Qsyse$ is \emph{not} state-based asynch. $1$-complete w.r.t. $\Interval^1_0$:\\
 \eqref{equ:dominolcomplete} does not hold as for $\xe_2$ and $y_1y_4\in\Ds{}{2}$ we have $y_1\in\EnabWl{}{[-1,-1]}{\xe_2}$ but $y_1y_4\notin\EnabWl{}{[-1,0]}{\xe_2}$.
 \item[(A2)] $\Qsyse$ is future unique w.r.t. $\Interval^1_0$ (as this always holds).
\item[(B1)] $\Qsyse$ is \emph{not} state-based asynch. $1$-complete w.r.t. $\Interval^1_1$:\\
\eqref{equ:dominolcomplete} does not hold as for $\xe_1$ and $y_1y_4\in\Ds{}{2}$ we have $y_1\in\EnabWl{}{[0,0]}{\xe_1}$ but $y_1y_4\notin\EnabWl{}{[0,1]}{\xe_1}$.
 \item[(B2)] $\Qsyse$ is future-unique w.r.t. $\Interval^1_1$:\\
 It is easy to see that $\Qsyse$ is output deterministic what immediately implies that $\Qsyse$ is future-unique w.r.t. $\Interval^1_1$ as we chose $\weS=\yeS$.
\end{enumerate}

\noindent Using (A2) and (B2), \REFthm{thm:SimRel_QsyseQsysal} (i) implies that
\begin{subequations}\label{equ:example:Rl1}
 \begin{align}
 \R^{\Interval^1_0}:=&\Set{\Tuple{\xe_1,\diamond}}
 \cup\Set{\Tuple{\xe_2,\ye_1},\Tuple{\xe_2,\ye_3}}
 \cup\Set{\Tuple{\xe_3,\ye_2},\Tuple{\xe_3,\ye_4}}\notag\\
 &\cup\Set{\Tuple{\xe_4,\ye_1},\Tuple{\xe_4,\ye_3}}\cup\Set{\Tuple{\xe_5,\diamond}}~\text{and}\\
   \R^{\Interval^1_1}:=&\Set{\Tuple{\xe_1,\ye_1},\Tuple{\xe_2,\ye_2},\Tuple{\xe_3,\ye_3},\Tuple{\xe_4,\ye_4},\Tuple{\xe_5,\ye_1}}
\end{align}
\end{subequations}
are simulation relations from $\Qsyse$ to $\Qsysa^{\Interval^1_0}$ and $\Qsysa^{\Interval^1_1}$, respectively. It should be noted that every state $\xe_i\in\xeS$ is related via $\R^{\Interval^1_1}$ to its unique output $\Set{\ye_j}=\EnabY{\tre}{\xe_i}$, while $\xe_i\in\xeS$ is related via $\R^{\Interval^1_0}$ to all possible output events $\Qsyse$ might produce immediately \emph{before} reaching $\xe_i$, i.e., the set of $y$-labels of all incoming transitions. 

Using (A1) and (B1) we know from \REFthm{thm:SimRel_QsyseQsysal} (ii), that $\R^{\Interval^1_0}$ (resp.$\R^{\Interval^1_1}$) is not a bisimulation relation between $\Qsyse$ and $\Qsysa^{\Interval^1_0}$ (resp. $\Qsysa^{\Interval^1_1}$). This can be observed from \REFfig{fig:exp_Qsysa1} by choosing $\Tuple{\xe_2,\ye_1}\in\R^{\Interval^1_0}$ and $\Tuple{\ye_1,\Tuple{u_4,y_4},y_4}\in\tra^{\Interval^1_0}$ and observing that $\xe_2$ has no outgoing transition labeled by $\ye_4$. Similarly, we can choose  $\Tuple{\xe_1,\ye_1}\in\R^{\Interval^1_1}$ and $\Tuple{\ye_1,\Tuple{u_1,y_1},y_4}\in\tra^{\Interval^1_1}$ and observe that there actually exists an outgoing transition in $\xe_1$ labeled by $\Tuple{u_1,y_1}$ but this transition reaches state $\xe_2$ which is not related to $\ye_4$ via $\R^{\Interval^1_1}$. 







Increasing $l$ and constructing the $\Interval^2_0$- and  $\Interval^2_2$-abstract state machines of $\Qsyse$ using \REFdef{def:QsysalW} yields the state machines $\Qsysa^{\Interval^2_0}$ and $\Qsysa^{\Interval^2_2}$ whose transition structure is depicted in \REFfig{fig:exp_Qsysa2}. It is interesting to note that using more information from the past, i.e., using $\Interval^2_0=[-2,-1]$ instead of $\Interval^1_0=[-1,-1]$, does not render $\Qsyse$ state-based asynchronously $l$-complete.

 \begin{figure}
  \begin{center}
\begin{tikzpicture}[auto]
\def\dh{1.5} \def\dv{1}






\def\h{0} \def\v{0} \def\dh{1.5}

\node (name) at (\h-1,\v+0.5) {$\Qsysa^{\Interval^2_0}:$};

\node[istate] (x0) at (\h,\v-\dv) {$\StateLabelYt{\diamond}{\diamond}$};
\node[istate] (x1) at (\h+0.8*\dh,\v-\dv) {$\StateLabelYt{\diamond}{1}$};
\node[Sstate] (x2a) at (\h+1.2*\dh,\v) {$\StateLabelYt{1}{2}$};
\node[Sstate] (x2b) at (\h+3*\dh,\v) {$\StateLabelYt{3}{2}$};
\node[Sstate] (x3a) at (\h+2*\dh,\v-\dv) {$\StateLabelYt{2}{3}$};
\node[Sstate] (x3b) at (\h+4*\dh,\v-\dv) {$\StateLabelYt{4}{3}$};
\node[Sstate] (x4a) at (\h+1.2*\dh,\v-2*\dv) {$\StateLabelYt{1}{4}$};
\node[Sstate] (x4b) at (\h+3*\dh,\v-2*\dv) {$\StateLabelYt{3}{4}$};

\SFSAutomatEdge{x0}{\EdgeLabelYt{\ue_1}{\ye_1}}{x1}{}{}
\SFSAutomatEdge{x1}{\EdgeLabelYt{\ue_2}{\ye_2}}{x2a}{bend left}{xshift=0.2cm}
\SFSAutomatEdge{x1}{\EdgeLabelYt{\ue_4}{\ye_4}}{x4a}{bend right}{swap,xshift=0.2cm}

\SFSAutomatEdge{x2a}{\EdgeLabelYt{\ue_3}{\ye_3}}{x3a}{bend right}{pos=0.3,xshift=-0.2cm}
\SFSAutomatEdge{x2b}{\EdgeLabelYt{\ue_3}{\ye_3}}{x3a}{bend left}{pos=0.5,xshift=-0.2cm}
\SFSAutomatEdge{x3a}{\EdgeLabelYt{\ue_2}{\ye_2}}{x2b}{bend left}{pos=0.8,xshift=0.2cm}
\SFSAutomatEdge{x3a}{\EdgeLabelYt{\ue_4}{\ye_4}}{x4b}{bend right}{swap,pos=0.15,xshift=0.2cm}

\SFSAutomatEdge{x4a.300}{\EdgeLabelYt{\ue_3}{\ye_3}}{x3b.340}{bend right=80,in=320,out=280}{pos=0.05,swap,xshift=0.2cm}
\SFSAutomatEdge{x4b}{\EdgeLabelYt{\ue_3}{\ye_3}}{x3b}{bend left}{pos=0.1,xshift=0.2cm}
\SFSAutomatEdge{x3b}{\EdgeLabelYt{\ue_2}{\ye_2}}{x2b}{bend right}{swap,xshift=-0.2cm}
\SFSAutomatEdge{x3b}{\EdgeLabelYt{\ue_4}{\ye_4}}{x4b}{bend left=10}{xshift=-0.2cm,pos=0.1}





\def\h{0} \def\v{-4} \def\dh{2.3}

\node (name) at (\h-1,\v+0.5) {$\Qsysa^{\Interval^2_2}:$};

\node[istate] (x1) at (\h,\v) {$\StateLabelYt{1}{2}$};
\node[Sstate] (x2) at (\h+\dh,\v) {$\StateLabelYt{2}{3}$};
\node[Sstate] (x3a) at (\h+0.5*\dh,\v-\dv) {$\StateLabelYt{3}{2}$};
\node[Sstate] (x3b) at (\h+1.5*\dh,\v-\dv) {$\StateLabelYt{3}{4}$};
\node[Sstate] (x4) at (\h+\dh,\v-2*\dv) {$\StateLabelYt{4}{3}$};
\node[istate] (x5) at (\h,\v-2*\dv) {$\StateLabelYt{1}{4}$};
\SFSAutomatEdge{x1}{\EdgeLabelYt{\ue_1}{\ye_1}}{x2}{}{}
\SFSAutomatEdge{x2}{\EdgeLabelYt{\ue_2}{\ye_2}}{x3a}{bend left}{pos=0.15,xshift=-0.2cm}
\SFSAutomatEdge{x2}{\EdgeLabelYt{\ue_2}{\ye_2}}{x3b}{bend left}{,xshift=-0.2cm}
\SFSAutomatEdge{x3a}{\EdgeLabelYt{\ue_3}{\ye_3}}{x2}{bend left}{pos=0.15,xshift=0.2cm}
\SFSAutomatEdge{x3b}{\EdgeLabelYt{\ue_3}{\ye_3}}{x4}{bend left}{,xshift=-0.2cm}
\SFSAutomatEdge{x4}{\EdgeLabelYt{\ue_4}{\ye_4}}{x3a}{bend left}{pos=0.85,xshift=0.2cm}
\SFSAutomatEdge{x4}{\EdgeLabelYt{\ue_4}{\ye_4}}{x3b}{bend left=10}{pos=0.6,xshift=0.2cm}
\SFSAutomatEdge{x5}{\EdgeLabelYt{\ue_1}{\ye_1}}{x4}{}{swap}
\end{tikzpicture}   \end{center}
  \caption{$\Interval^2_0$- and $\Interval^2_2$-abstract state machines of $\Qsyse$ in \REFfig{fig:exp_Qsyse}, where $\langle\kern-1pt{i}\kern-1pt{j}\kern-1pt\rangle:=\ye_i\ye_j$.}\label{fig:exp_Qsysa2}
 \end{figure}

 \begin{enumerate}[(C1)]
\item $\Qsyse$ is \emph{not} state-based asynch. $2$-complete w.r.t. $\Interval^2_0$:\\
\eqref{equ:dominolcomplete} does not hold as for $\xe_2$ and $\diamond y_1y_4\in\Ds{}{3}$ we have $\diamond y_1\in\EnabWl{}{[-2,-1]}{\xe_2}$ but $\diamond y_1y_4\notin\EnabWl{}{[-2,0]}{\xe_2}$. \item $\Qsyse$ is future unique w.r.t. $\Interval^2_0$ (as this always holds). 
\end{enumerate}

\noindent Contrary, using more information from the future, i.e., using $\Interval^2_2=[0,1]$ instead of $\Interval^1_1=[0,0]$, renders $\Qsyse$ state-based asynchronously $l$-complete. However, in this case the future uniqueness-property is lost.

 \begin{enumerate}[(D1)]
\item $\Qsyse$ is state-based asynchronously $2$-complete w.r.t. $\Interval^2_2$:\\
 Using more future information actually resolves the ambiguity from $\Interval^1_1$. E.g., choosing $\xe_1$ we can only pick $y_1y_2y_3\in\Ds{}{3}$ to obtain $y_1y_2\in\EnabWl{}{[0,1]}{\xe_1}$, obviously implying $y_1y_2y_3\in\EnabWl{}{[0,2]}{\xe_1}$.
\item $\Qsyse$ is \emph{not} future-unique w.r.t. $\Interval^2_2$:\\
\eqref{equ:future_unique} does not hold as for $\xe_2$ we have $y_2y_3y_2,~y_2y_3y_4\in\EnabWl{}{\Interval^2_2}{\xe_1}$ but obviously $y_2y_3y_2\neq y_2y_3y_4$. 
\end{enumerate}

\noindent Using \REFthm{thm:SimRel_QsyseQsysal} we can now construct relations $\R^{\Interval^2_0}$ and $\R^{\Interval^2_2}$ analogously to the ones for $l=1$ in \eqref{equ:example:Rl1}. However, now (C1)-(D2) imply that 
\begin{align*}
 \R^{\Interval^2_0}\in\SR{\weS}{}{\Qsyse}{\Qsysa^{\Interval^2_0}}~&\text{but}~
 \BR{\R^{\Interval^2_0}}^{-1}\notin\SR{\weS}{}{\Qsysa^{\Interval^2_0}}{\Qsyse}~\text{and}\\
 \BR{\R^{\Interval^2_2}}^{-1}\in\SR{\weS}{}{\Qsysa^{\Interval^2_2}}{\Qsyse}~&\text{but}~
 \R^{\Interval^2_2}\notin\SR{\weS}{}{\Qsyse}{\Qsysa^{\Interval^2_2}}.
\end{align*}

\noindent To see that $\R^{\Interval^2_2}$ is not a simulation relation from $\Qsyse$ to $\Qsysa^{\Interval^2_2}$ pick $\Tuple{\xe_3,\ye_3\ye_4}\in\R^{\Interval^2_2}$ and $\Tuple{\xe_3,\ue_3,\ye_3,\xe_2}\in\tre$ and observe that $\ye_3\ye_4$ does not have an outgoing transition labeled by $\Tuple{\ue_3,\ye_3}$.

Recall from (D2) that $\Qsyse$ is not future unique for $\Interval^2_2$. Using \eqref{equ:future_unique} this implies that for any interval $\Ilm$ with $m-1\geq 2$ (i.e., any interval with two or more future values) the property of future uniqueness does not hold. 

In terms of state-based asynchronous $l$-completeness the problem is inverted. If we use $m-1<2$ (implying future uniqueness of $\Qsyse$ w.r.t. $\Ilm$ from (A2) and (C2)) $\Qsyse$ cannot be state-based asynchronously $l$-complete for any $l$ as the ambiguity for attaching dominos cannot be resolved by further knowledge about the past. In this case the counterexamples in (A1) and (C1) can be reused by pre-appending the considered strings by an appropriate number of diamonds. It is rather necessary to look at least two steps into the future, i.e., pick $m-1\geq 2$, to resolve this ambiguity as shown in (D1). 

Concluding the above discussion there obviously exists no $l$ and $m$ s.t. $\Qsyse$ in \REFfig{fig:exp_Qsyse} is both state-based asynchronously $l$-complete and future unique w.r.t. $\Ilm$. Therefore, increasing $l$ and $m$ will never result in a bisimilar abstraction of $\Qsyse$.









