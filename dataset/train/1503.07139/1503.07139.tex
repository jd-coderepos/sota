

The idea of \SAlA is to exactly mimic the external behavior of  in \eqref{equ:Prelim} over finite time intervals of length . 
We therefore consider the behavioral system , where  is the extension of  to  as discussed in \REFsec{sec:prelim}. 
All finite strings of external symbols of length  which are consistent with the dynamics of  are given by

Now consider the following gedankenexperiment: assume playing a sophisticated domino game where  is the set of dominos. Pick the first domino to be  (i.e., a domino with only diamonds except for the last symbol) and append any domino from the set  if the last  symbols of the first domino are the same as the first  symbols of the second domino (see Figure~\ref{fig:DominoGame} (left) for an example).
Playing the domino game arbitrarily long and with all possible initial conditions and domino combinations results in the largest, in the sense of set inclusion, behavior  satisfying 

 &\Behal\ll{-l,0}=\BeheQ\ll{-l,0}~\text{and}~\label{equ:BehalW:a}\\
 &\Pi_{l+1}(\Behal)=\Pi_{l+1}(\BeheQ), \label{equ:BehalW:b}
 
defining the behavioral system .
Observe that the smaller , the less information in the domino game is used, which generates more freedom in constructing signals, implying  for all . 
This motivates the use of  as an over-approximation of the behavior .
Obviously, equality  holds for all  if  is itself the largest behavior satisfying \eqref{equ:BehalW}. In \cite{SchmuckRaisch2014_ControlLetters}, a system  for which the latter is true was called \emph{asynchronously -complete} which inspired the name of \SAlA. Following \cite{SchmuckRaisch2014_ControlLetters},  constructed in the outlined domino game is the unique \SAlA of .  However, we are usually interested in a state machine \emph{realizing} its step by step evolution.

\begin{figure}
\begin{center}
\begin{tikzpicture}[auto, node distance=0.5cm,scale=1]

  \tikzstyle{myblock} = [draw,rectangle, text centered, minimum height=0.5cm, minimum width=1.4cm,thick]
 \node[myblock] (b1) at (0,0) {};
 \node[myblock,below of=b1,xshift=0.5cm,node distance=0.7cm] (b2) {};
 \node[myblock,below of=b2,xshift=0.5cm,node distance=0.7cm] (b3) {};
  \node[myblock,below of=b3,xshift=0.5cm,node distance=0.7cm] (b4) {};
   \node[below of=b4,xshift=0.4cm,node distance=0.5cm] (b5) {}; 
 \draw [draw,->,ultra thick] (0.6,-3cm) -- ++ (2cm,0);
\foreach \x in {0.6,1.1,1.55,2}
\draw (\x cm,-2.9cm) -- (\x cm,-3.1cm);
\draw (0.6,-3.3cm) node {};
\draw (1.55,-3.3cm) node {};
\draw (2.9,-3.3cm) node {};


  \tikzstyle{myblock} = [draw,rectangle, text centered, minimum height=0.5cm, minimum width=0.5cm]
   \tikzstyle{myblocks} = [draw,rectangle, text centered, minimum height=0.5cm, minimum width=0.5cm,postaction={pattern color=black!70,pattern=north west lines}] 
   


   \def\n{4} \def\c{1.3cm}
   
  \node[myblock] (b1) at (\n,-\c) {};
  \node[myblocks, right of=b1] (b2)  {};
   \node[myblocks, right of=b2] (b3) {};
   \node[myblocks, right of=b3] (b4) {};
   \node[myblocks, right of=b4] (b5) {}; 
   \node[myblocks,below of=b2,node distance=0.7cm] (a1) {};
  \node[myblocks, right of=a1] (a2) {};
   \node[myblocks, right of=a2] (a3) {};
   \node[myblocks, right of=a3] (a4) {};
   \node[myblock, right of=a4] (a5) {};   
\node[draw,rectangle, text centered, minimum height=0.5cm, minimum width=2.5cm,ultra thick] () at (b3) {};
    \node[draw,rectangle, text centered, minimum height=0.5cm, minimum width=2.5cm,ultra thick] () at (a3) {};  
\draw [decorate,decoration={brace,amplitude=6pt},thick] (b2.north west) -- (b5.north east) node [black,midway,yshift=0.1cm] (bc){};
   \draw [decorate,decoration={brace,amplitude=6pt},thick] (a5.south east) -- (a2.south west) node [black,midway,yshift=-0.1cm] (bc){};
   \draw [draw,-,ultra thick,dotted] (\n-0.4,-1.7cm-\c) -- ++ (\n+0.3cm,0);
 \draw [draw,->,ultra thick] (\n-0.1,-1.7cm-\c) -- ++ (\n+3cm,0);
 \def\d{0.5}
\foreach \i in {0,...,5}
\draw (\n+\i*\d,-1.6cm-\c) -- (\n+\i*\d,-1.8cm-\c);
\draw (\n+0.5,-2cm-\c) node {};
\draw (\n+2.5,-2cm-\c) node {};
\draw (\n+3.3,-2cm-\c) node {};



\end{tikzpicture}   \end{center}
  \caption{Example of a domino game for  (left) and an illustration of the usual choice  in \REFprop{prop:SforLcomplete} for  with  (right).}\label{fig:DominoGame}
\end{figure}
\begin{definition}\label{def:SAlA_Realization}
 Given \eqref{equ:Prelim} and \eqref{equ:BehalW}, the dynamical system  is the \SAlA of . A state machine  is a realization of  
iff\footnote{As before,   denotes the extension of  to .} .
\end{definition}
In the work on \SlA and \SAlA the state space  to construct the realization  of the abstraction  is usually chosen such that the state represents the \enquote{recent past} of length  of the external signal. Recalling the gedankenexperiment, this choice of  is motivated by the fact that the  next feasible domino of length  is determined by the last  symbols of the previous domino (see \REFfig{fig:DominoGame} (right) for an illustration).
Using this state space, the standard state machine realization of \SAlA, denoted by  in this paper, is defined as follows.

\begin{proposition}[\cite{SchmuckRaisch2014_ControlLetters}, Thm.4]\label{prop:SforLcomplete}
Let  be the \SAlA of  and define
\allowdisplaybreaks
 \xaS^l:=&\Set{\diamond}^l\cup\Pi_l(\Behal),\\
 \xaSo{}^l:=&\Set{\diamond}^l,~\text{and}\\
 \tra^l:=&\SetCompX{\Tuple{\xa,\we,\BR{\xa\sconc\we}\ll{1,l}}}{\xa\sconc\we\in\Pi_{l+1}(\Behal)}.

Then  is realized by .
\end{proposition}




Summarizing the abstraction procedure outlined above, constructing the finite state abstraction  in \REFprop{prop:SforLcomplete} using \SAlA only requires knowledge about the set .
However, if  is available, we can construct  from  directly,
as shown in the following section. 




\subsection{Some State Machine Realizations of \SAlA}\label{sec:SAlA_SM}

Recall from \REFprop{prop:SforLcomplete} that the set of external sequences of length , given by  (from \eqref{equ:BehalW:b}), is finite. We now investigate how to use this set as a state space in the construction of different state machine realizations of the \SAlA of a system . This is be done on the basis of a state machine realization  of  satisfying \eqref{equ:Prelim}. For this,
we first investigate how a string  can correspond to a state  of . Observe that  is a string of length  and  is a state reached at a particular time . We consider the cases where  is generated by  immediately \emph{before}, immediately \emph{after} or \emph{while}  was reached. This leads us to a set of intervals 



where\footnote{The addition of two intervals is interpreted in the usual sense, i.e., .}  corresponds to the first,   corresponds to the second, and for all other choices of ,  corresponds to the third case. Based on \eqref{equ:Prelim:I} the sets of compatible states are introduced in \REFdef{def:Xxr} and illustrated in \REFfig{fig:Enabl}.\begin{figure}
\begin{center}
\begin{tikzpicture}[auto,scale=1]
 
 \tikzset{
>=stealth',
help lines/.style={dashed, thick},
axis/.style={<->},
important line/.style={thick},
connection/.style={thick, dotted},
}

\def\y{0} \def\yd{0.5} \def\x{0} \def\xd{1} \def\xmax{10} \def\ymax{2}

\coordinate (zero) at (\x,\y);
\coordinate (top) at (\x,\y+5);
\coordinate (start) at (\x,\y+2);

\foreach \nx/\ny/\a in {1/0/a,2/1/b,3/2/c,4/1.5/c,5/0.5/b,6/0.8/b}{
\draw[help lines] (\x+\nx*\xd,\y) -- (\x+\nx*\xd,\y+\ymax-0.5);
\coordinate (x\nx) at start+(\nx*\xd,\ny*\yd);
\fill [black] (x\nx) node {}  circle [radius=3pt];
\draw (\nx*\xd,\ymax-0.2) node {};
}
\draw (\x+0.2,\ymax-0.2) node {};
\draw (\x-0.5,\ymax-0.2) node {:};
\draw (6.8*\xd,\ymax-0.2) node {};
\draw (x4)+(0.3,0) node {};



\foreach \a/\b/\c/\z in {1/3/0/0,2/4/0.2/1,3/5/0.4/2,4/6/0.6/3}{
\draw[important line] (\a*\xd-0.1,\ymax+0.05+\c) -- (\a*\xd-0.1,\ymax+0.15+\c) --(\b*\xd+0.1,\ymax+0.15+\c) --(\b*\xd+0.1,\ymax+0.05+\c);
\draw (\a*\xd+0.4,\ymax+\c+0.35) node {};
}



\draw[important line] (x1) to[out=0,in=-140] (x2) to[out=40,in=-180] (x3) to[out=0,in=-210] (x4) to[out=-30,in=-190] (x5) to[out=-10,in=-150] (x6);
\draw[black, dotted,thick] (\x,\y+\yd) to[out=-10,in=-180] (x1);
\draw[black, dotted,thick] (x6) to[out=20,in=-150] (\x+7*\xd,\y+2*\yd);
\draw (\x-0.5,\y+\yd) node {:};






 


 
\end{tikzpicture}
   \end{center}
 \caption{Illustration of corresponding external sequences  for state  where  and  for some .}\label{fig:Enabl}
\end{figure}
\begin{definition}\label{def:Xxr}
Given \eqref{equ:Prelim} and \eqref{equ:Prelim:I}, let  be a dynamical system, where  is the extension of  to  as discussed in \REFsec{sec:prelim}. Then the set of \emph{corresponding external strings w.r.t. } is defined for every state  by

Furthermore, if 

 is called \emph{future unique} w.r.t. .
\end{definition}



Observe, that  in \eqref{equ:future_unique} are obtained from two trajectories  passing  at time  and , respectively, (i.e., ) using \eqref{equ:Xxr}. During this restriction of  (resp. ) to  (resp. ) absolute time information is disregarded (see \REFsec{sec:notation}), implying  and . Therefore, 
 is future unique w.r.t.  if for all states  all trajectories passing  have the same -long (non-strict) future of external symbols, i.e. . Using this intuition it is easy to see that  is always \emph{future unique} w.r.t. , as this interval has no future.\\
We now proceed by constructing  finite state machines using the outlined correspondence between  and .

\begin{definition}\label{def:QsysalW}
Given \eqref{equ:Prelim} and \eqref{equ:Prelim:I}, define 

\xalS:=&\SetCompX{\zeta}{\ExQ{\xe\in\xeS}{\zeta\in\EnabWl{}{\Ilm}{\xe}}},\label{equ:xalS}\\
\xalSo{}:=&\SetCompX{\zeta}{\ExQ{\xe\in\xeSo{}}{\zeta\in\EnabWl{}{\Ilm}{\xe}}},~\text{and}\label{equ:xalSo}\\
\tral\hspace{-1mm}:=&\SetCompX{\Tuple{\xa,\ue,\ye,\xa'}}{
\begin{propConjA}
\xa'\ll{0,l\mips m\mips1}=\BR{\xa\ll{0,l\mips m\mips1}\sconc\projState{\weS}{\ue,\ye}}\ll{1,l\mips m}\hspace{-1mm}\0.1cm]
\ExQ{\xe,\xe'\in\xeS}{
\begin{propConjA}
 \xa\in \EnabWl{}{\Ilm}{\xe}\\
 \xa'\in \EnabWl{}{\Ilm}{\xe'}\\
\Tuple{\xe,\ue,\ye,\xe'}\in\tre
\end{propConjA}}
\end{propConjA}
}\hspace{-1mm}.\label{equ:tral}Then  is called the -abstract state machine of . 
\end{definition}
 


The construction of the abstract state machines in \REFdef{def:QsysalW} can be interpreted as follows.
Using \eqref{equ:xalS} instead of  ensures that  is live and reachable, which is purely cosmetic but allows to simplify subsequent proofs. The last line in the conjunction of \eqref{equ:tral} simply says that we have a transition in  from  to  if there is a transition in  between any two states compatible with  and , respectively. However, the first two lines in the conjunction of \eqref{equ:tral} additionally ensure that  and  obey the rules of the domino game, i.e.,

as depicted in \REFfig{fig:DominoGame} (right) and the current external symbol  is contained in either  or  or both, at the position corresponding to the current time point, i.e.,

As we are interested in state machine realizations of \SAlA, we show that  realizes  for all choices of  and .

\begin{theorem}\label{thm:behequ}
Given \eqref{equ:Prelim} and \eqref{equ:Prelim:I}, let  be defined as in \REFdef{def:QsysalW} and let  be the unique \SAlA of . Then  realizes .\end{theorem}
\begin{proof}
See Appendix~\ref{proof:thm:behequ}.
\end{proof}



As an intuitive consequence of \REFthm{thm:behequ}, choosing  and the full external symbol set  when constructing  in \REFdef{def:QsysalW} yields the standard realization  of \SAlA.

 \begin{theorem}\label{thm:Qsysalo_equ}
 Given \eqref{equ:Prelim} and \eqref{equ:Prelim:I} with , let  and  as in \REFprop{prop:SforLcomplete} and \REFdef{def:QsysalW}, respectively. Then .
 \end{theorem}
 
 \begin{proof}
 See Appendix~\ref{proof:thm:Qsysalo_equ}.
 \end{proof}
 


\subsection{Ordering  based on Simulation Relations}

Before we discuss the ordering between abstract state machines based on changing  and , we show under which conditions the obtained abstraction  simulates the original state machine  and when both state machines are bisimilar. This investigation is interesting for the comparison to \QBA, as the latter always simulates the original state machine . Furthermore, the framework of \QBA allows to construct a bisimilar abstraction whenever the employed repartitioning algorithm terminates. Hence, it is interesting to know if the latter is also true for \SAlA.\\
The investigation of similarity between  and  requires the construction of a relation between the original state space  and the abstract state space . As  defines a cover for  where each cell is given by all states  corresponding to a string  via , the latter is a natural choice for a relation between  and .\\
Recall from \REFthm{thm:behequ} that the behaviors of  and  coincide if  is asynchronously -complete. Behavioral equivalence is always necessary for a relation  to be a bisimulation relation but usually not sufficient. We therefore introduce a stronger condition, called \emph{state-based asynchronous -completeness}, to serve the latter purpose.

\begin{definition}\label{def:SBalc}
Given \eqref{equ:Prelim},  is \emph{state-based asynchronously -complete w.r.t. } if
 
\end{definition}

\begin{remark}\label{rem:SBalc}
Recall from the beginning of this section that the dynamical system  is asynchronously -complete, as  defined in \cite[Def.6]{SchmuckRaisch2014_ControlLetters}, if  is the largest behavior satisfying \eqref{equ:BehalW} itself. Intuitively, the latter is true if \emph{for all}  there \emph{exists} an  s.t. the second part of \eqref{equ:dominolcomplete} holds. Therefore, asynchronous -completeness of  is always implied by \eqref{equ:dominolcomplete}, but not vice-versa. 
\end{remark}

\begin{theorem}\label{thm:SimRel_QsyseQsysal}
Given \eqref{equ:Prelim}, \eqref{equ:Prelim:I} and  as in \REFdef{def:QsysalW}, let

Then it holds that\footnote{Using  instead of  in (i) is done on purpose and indicates that this relation holds for  independent from the choice of .} 
\begin{compactenum}[(i)]
 \item \Qsyse\Ilm and
 \item \Qsysel\Ilm.
\end{compactenum}
\end{theorem}

\begin{proof}
See Appendix~\ref{proof:thm:SimRel_QsyseQsysal}. \end{proof}

Intuitively,  simulates  w.r.t.  if for every related state pair  and every transition  which  \enquote{picks},  can \enquote{pick} a transition  s.t. . However, if , a state  has only outgoing transitions s.t. . Therefore,  can only simulate  iff in every state  all outgoing transitions agree on this , i.e.,  is \enquote{output deterministic} w.r.t. . For  applying this reasoning iteratively gives the (rather restrictive) condition of future uniqueness of .
As the outlined problems are absent for  (as  is always future unique w.r.t. ), , which we know to coincide with the original realization  of \SAlA for , always simulates .

\begin{corollary}
 Given \eqref{equ:Prelim}, \eqref{equ:Prelim:I} and  as in \REFdef{def:QsysalW} it holds that 
.
\end{corollary}

 \begin{remark}
 In the context of \SlA a state machine  was introduced in \cite{Raisch2010} whose state at time  represents the string of external symbols from time  to time , i.e., from the interval . While the state sets of   and  coincide, their transition structure slightly differs. This is a consequence of the fact that  was intended to serve as a set-valued observer for the states of .
\end{remark}
  
Recalling the domino game, we know that using longer dominos (i.e., increasing ) gives less freedom in composing them and therefore yields a tighter abstraction. This intuition carries over to the state space realizations of , inducing an ordering in terms of simulation relations. 

\begin{theorem}\label{thm:SimRel_QsysalpmQsysalm}
Given \eqref{equ:Prelim}, \eqref{equ:Prelim:I} and  as in \REFdef{def:QsysalW}, let

Then it holds that 
 \begin{compactenum}[(i)]
  \item  and 
  \item .
 \end{compactenum}
\end{theorem}

\begin{proof}
See Appendix~\ref{proof:thm:SimRel_QsysalpmQsysalm}. 
\end{proof}

\REFthm{thm:SimRel_QsysalpmQsysalm} (ii) implies that the accuracy of the abstraction cannot be increased by increasing  if  is asynchronously -complete and  is fixed, e.g. . Therefore, the standard realization  for \SAlA might never result in a bisimilar abstraction of , no matter how large  is chosen, even if  is asynchronously -complete. This is due to the fact that 
\eqref{equ:dominolcomplete} is not implied by asynchronous -completeness of  (see \REFrem{rem:SBalc}).\\
Interestingly, we will show that increasing , i.e., shifting the interval into the future, results in a tighter abstraction w.r.t. simulation relations, i.e. allows to increase the precision of  for  even if  is -complete. 


\begin{theorem}\label{thm:SimRel_QsysalmpQsysalm}
Given \eqref{equ:Prelim}, \eqref{equ:Prelim:I}, and  as in \REFdef{def:QsysalW} with , let

Then it holds that
\begin{compactenum}[(i)]
 \item   and 
 \item \Qsyse\Ilmp\Qsysel\Ilm
\end{compactenum}
\end{theorem}

\begin{proof}
 See Appendix~\ref{proof:thm:SimRel_QsysalmpQsysalm}.
\end{proof}
It is important to note that future uniqueness and state-based asynchronous -completeness are incomparable properties, i.e., none is implied by the other. Therefore, there exist situations where  with  simulates  (i.e.,  is future unique w.r.t. ) and  is tighter than  in terms of simulation relations. However, if  is both future unique and state-based asynchronously -complete w.r.t. a particular interval , \REFthm{thm:SimRel_QsysalmpQsysalm} implies that increasing  and  will not result in a tighter abstraction. Moreover, this is not necessary anyway, as \REFthm{thm:SimRel_QsyseQsysal} implies that in this case  is bisimilar to . 



\subsection{Example}\label{sec:SAlCA_new_exp}
We conclude this section with a detailed example illustrating the construction of -abstract state machines and the property of future uniqueness and state-based asynchronous -completeness for different choices of  and .
For simplicity, we consider a \emph{finite} state machine 
 
 as the original model, whose transition structure is depicted in \REFfig{fig:exp_Qsyse}.\begin{figure}
  \begin{center}
\begin{tikzpicture}[auto]
\def\dh{1.5} \def\dv{1}

\def\h{0} \def\v{0}

\node (name) at (\h-1,\v+0.5) {};

\node[istate] (x1) at (\h,\v) {};
\node[Sstate] (x2) at (\h+\dh,\v) {};
\node[Sstate] (x3) at (\h+\dh,\v-\dv) {};
\node[Sstate] (x4) at (\h+\dh,\v-2*\dv) {};
\node[istate] (x5) at (\h,\v-2*\dv) {};
\SFSAutomatEdge{x1}{\EdgeLabelYt{\ue_1}{\ye_1}}{x2}{}{}
\SFSAutomatEdge{x2}{\EdgeLabelYt{\ue_2}{\ye_2}}{x3}{bend left}{}
\SFSAutomatEdge{x3}{\EdgeLabelYt{\ue_3}{\ye_3}}{x2}{bend left}{}
\SFSAutomatEdge{x3}{\EdgeLabelYt{\ue_3}{\ye_3}}{x4}{bend left}{}
\SFSAutomatEdge{x4}{\EdgeLabelYt{\ue_4}{\ye_4}}{x3}{bend left}{}
\SFSAutomatEdge{x5}{\EdgeLabelYt{\ue_1}{\ye_1}}{x4}{}{swap}

\end{tikzpicture}   \end{center}
  \caption{Transition structure of the state machine  in \REFfig{fig:exp_Qsyse}.}\label{fig:exp_Qsyse}
 \end{figure}
It can be inferred from \REFfig{fig:exp_Qsyse} that the output behavior of  is given by
  
  where  and  denote, respectively, the finite and infinite repetition of the respective string. Furthermore, the sets of -long and -long dominos obtained from  via \eqref{equ:Ds} are
  
  To play the domino-game for , i.e., with dominos from the set , we have to pick  as the initial domino and append dominos such that the last element of the first matches the first element of the second domino. It is easy to see that in this example every such combination of dominos yields a sequence contained in . Hence,  is asynchronously -complete and therefore also asynchronously -complete.
  


Using  in \REFfig{fig:exp_Qsyse} we can construct the - and  -abstract state machines of  using \REFdef{def:QsysalW}. Their transition structures are depicted in \REFfig{fig:exp_Qsysa1}. Furthermore, we obtain the following properties of  w.r.t  and .


 \begin{figure}
  \begin{center}
\begin{tikzpicture}[auto]
\def\dh{1.5} \def\dv{1}



\def\h{0} \def\v{0}

\node (name) at (\h-0.5,\v+0.5) {};

\node[istate] (x0) at (\h,\v-\dv) {};
\node[Sstate] (x1) at (\h+0.5*\dh,\v-\dv) {};
\node[Sstate] (x2) at (\h+1.5*\dh,\v) {};
\node[Sstate] (x3) at (\h+1.5*\dh,\v-\dv) {};
\node[Sstate] (x4) at (\h+1.5*\dh,\v-2*\dv) {};
\SFSAutomatEdge{x0}{\EdgeLabelYt{\ue_1}{\ye_1}}{x1}{}{}
\SFSAutomatEdge{x1}{\EdgeLabelYt{\ue_2}{\ye_2}}{x2}{bend left}{}
\SFSAutomatEdge{x2}{\EdgeLabelYt{\ue_3}{\ye_3}}{x3}{bend left}{}
\SFSAutomatEdge{x3}{\EdgeLabelYt{\ue_2}{\ye_2}}{x2}{bend left}{}
\SFSAutomatEdge{x3}{\EdgeLabelYt{\ue_4}{\ye_4}}{x4}{bend left}{}
\SFSAutomatEdge{x4}{\EdgeLabelYt{\ue_3}{\ye_3}}{x3}{bend left}{}
\SFSAutomatEdge{x1}{\EdgeLabelYt{\ue_4}{\ye_4}}{x4}{bend right}{swap}



\def\h{4.5} \def\v{0}

\node (name) at (\h-0.5,\v+0.5) {};

\node[istate] (x1) at (\h,\v-\dv) {};
\node[Sstate] (x2) at (\h+\dh,\v) {};
\node[Sstate] (x3) at (\h+\dh,\v-\dv) {};
\node[Sstate] (x4) at (\h+\dh,\v-2*\dv) {};
\SFSAutomatEdge{x1}{\EdgeLabelYt{\ue_1}{\ye_1}}{x2}{bend left}{}
\SFSAutomatEdge{x2}{\EdgeLabelYt{\ue_2}{\ye_2}}{x3}{bend left}{}
\SFSAutomatEdge{x3}{\EdgeLabelYt{\ue_3}{\ye_3}}{x2}{bend left}{}
\SFSAutomatEdge{x3}{\EdgeLabelYt{\ue_3}{\ye_3}}{x4}{bend left}{}
\SFSAutomatEdge{x4}{\EdgeLabelYt{\ue_4}{\ye_4}}{x3}{bend left}{}
\SFSAutomatEdge{x1}{\EdgeLabelYt{\ue_1}{\ye_1}}{x4}{bend right}{swap}



\end{tikzpicture}   \end{center}
  \caption{- and -abstract state machines of  in \REFfig{fig:exp_Qsyse}.}\label{fig:exp_Qsysa1}
 \end{figure}

\begin{enumerate}
 \item[(A1)]  is \emph{not} state-based asynch. -complete w.r.t. :\\
 \eqref{equ:dominolcomplete} does not hold as for  and  we have  but .
 \item[(A2)]  is future unique w.r.t.  (as this always holds).
\item[(B1)]  is \emph{not} state-based asynch. -complete w.r.t. :\\
\eqref{equ:dominolcomplete} does not hold as for  and  we have  but .
 \item[(B2)]  is future-unique w.r.t. :\\
 It is easy to see that  is output deterministic what immediately implies that  is future-unique w.r.t.  as we chose .
\end{enumerate}

\noindent Using (A2) and (B2), \REFthm{thm:SimRel_QsyseQsysal} (i) implies that

 \R^{\Interval^1_0}:=&\Set{\Tuple{\xe_1,\diamond}}
 \cup\Set{\Tuple{\xe_2,\ye_1},\Tuple{\xe_2,\ye_3}}
 \cup\Set{\Tuple{\xe_3,\ye_2},\Tuple{\xe_3,\ye_4}}\notag\\
 &\cup\Set{\Tuple{\xe_4,\ye_1},\Tuple{\xe_4,\ye_3}}\cup\Set{\Tuple{\xe_5,\diamond}}~\text{and}\\
   \R^{\Interval^1_1}:=&\Set{\Tuple{\xe_1,\ye_1},\Tuple{\xe_2,\ye_2},\Tuple{\xe_3,\ye_3},\Tuple{\xe_4,\ye_4},\Tuple{\xe_5,\ye_1}}

are simulation relations from  to  and , respectively. It should be noted that every state  is related via  to its unique output , while  is related via  to all possible output events  might produce immediately \emph{before} reaching , i.e., the set of -labels of all incoming transitions. 

Using (A1) and (B1) we know from \REFthm{thm:SimRel_QsyseQsysal} (ii), that  (resp.) is not a bisimulation relation between  and  (resp. ). This can be observed from \REFfig{fig:exp_Qsysa1} by choosing  and  and observing that  has no outgoing transition labeled by . Similarly, we can choose   and  and observe that there actually exists an outgoing transition in  labeled by  but this transition reaches state  which is not related to  via . 







Increasing  and constructing the - and  -abstract state machines of  using \REFdef{def:QsysalW} yields the state machines  and  whose transition structure is depicted in \REFfig{fig:exp_Qsysa2}. It is interesting to note that using more information from the past, i.e., using  instead of , does not render  state-based asynchronously -complete.

 \begin{figure}
  \begin{center}
\begin{tikzpicture}[auto]
\def\dh{1.5} \def\dv{1}






\def\h{0} \def\v{0} \def\dh{1.5}

\node (name) at (\h-1,\v+0.5) {};

\node[istate] (x0) at (\h,\v-\dv) {};
\node[istate] (x1) at (\h+0.8*\dh,\v-\dv) {};
\node[Sstate] (x2a) at (\h+1.2*\dh,\v) {};
\node[Sstate] (x2b) at (\h+3*\dh,\v) {};
\node[Sstate] (x3a) at (\h+2*\dh,\v-\dv) {};
\node[Sstate] (x3b) at (\h+4*\dh,\v-\dv) {};
\node[Sstate] (x4a) at (\h+1.2*\dh,\v-2*\dv) {};
\node[Sstate] (x4b) at (\h+3*\dh,\v-2*\dv) {};

\SFSAutomatEdge{x0}{\EdgeLabelYt{\ue_1}{\ye_1}}{x1}{}{}
\SFSAutomatEdge{x1}{\EdgeLabelYt{\ue_2}{\ye_2}}{x2a}{bend left}{xshift=0.2cm}
\SFSAutomatEdge{x1}{\EdgeLabelYt{\ue_4}{\ye_4}}{x4a}{bend right}{swap,xshift=0.2cm}

\SFSAutomatEdge{x2a}{\EdgeLabelYt{\ue_3}{\ye_3}}{x3a}{bend right}{pos=0.3,xshift=-0.2cm}
\SFSAutomatEdge{x2b}{\EdgeLabelYt{\ue_3}{\ye_3}}{x3a}{bend left}{pos=0.5,xshift=-0.2cm}
\SFSAutomatEdge{x3a}{\EdgeLabelYt{\ue_2}{\ye_2}}{x2b}{bend left}{pos=0.8,xshift=0.2cm}
\SFSAutomatEdge{x3a}{\EdgeLabelYt{\ue_4}{\ye_4}}{x4b}{bend right}{swap,pos=0.15,xshift=0.2cm}

\SFSAutomatEdge{x4a.300}{\EdgeLabelYt{\ue_3}{\ye_3}}{x3b.340}{bend right=80,in=320,out=280}{pos=0.05,swap,xshift=0.2cm}
\SFSAutomatEdge{x4b}{\EdgeLabelYt{\ue_3}{\ye_3}}{x3b}{bend left}{pos=0.1,xshift=0.2cm}
\SFSAutomatEdge{x3b}{\EdgeLabelYt{\ue_2}{\ye_2}}{x2b}{bend right}{swap,xshift=-0.2cm}
\SFSAutomatEdge{x3b}{\EdgeLabelYt{\ue_4}{\ye_4}}{x4b}{bend left=10}{xshift=-0.2cm,pos=0.1}





\def\h{0} \def\v{-4} \def\dh{2.3}

\node (name) at (\h-1,\v+0.5) {};

\node[istate] (x1) at (\h,\v) {};
\node[Sstate] (x2) at (\h+\dh,\v) {};
\node[Sstate] (x3a) at (\h+0.5*\dh,\v-\dv) {};
\node[Sstate] (x3b) at (\h+1.5*\dh,\v-\dv) {};
\node[Sstate] (x4) at (\h+\dh,\v-2*\dv) {};
\node[istate] (x5) at (\h,\v-2*\dv) {};
\SFSAutomatEdge{x1}{\EdgeLabelYt{\ue_1}{\ye_1}}{x2}{}{}
\SFSAutomatEdge{x2}{\EdgeLabelYt{\ue_2}{\ye_2}}{x3a}{bend left}{pos=0.15,xshift=-0.2cm}
\SFSAutomatEdge{x2}{\EdgeLabelYt{\ue_2}{\ye_2}}{x3b}{bend left}{,xshift=-0.2cm}
\SFSAutomatEdge{x3a}{\EdgeLabelYt{\ue_3}{\ye_3}}{x2}{bend left}{pos=0.15,xshift=0.2cm}
\SFSAutomatEdge{x3b}{\EdgeLabelYt{\ue_3}{\ye_3}}{x4}{bend left}{,xshift=-0.2cm}
\SFSAutomatEdge{x4}{\EdgeLabelYt{\ue_4}{\ye_4}}{x3a}{bend left}{pos=0.85,xshift=0.2cm}
\SFSAutomatEdge{x4}{\EdgeLabelYt{\ue_4}{\ye_4}}{x3b}{bend left=10}{pos=0.6,xshift=0.2cm}
\SFSAutomatEdge{x5}{\EdgeLabelYt{\ue_1}{\ye_1}}{x4}{}{swap}
\end{tikzpicture}   \end{center}
  \caption{- and -abstract state machines of  in \REFfig{fig:exp_Qsyse}, where .}\label{fig:exp_Qsysa2}
 \end{figure}

 \begin{enumerate}[(C1)]
\item  is \emph{not} state-based asynch. -complete w.r.t. :\\
\eqref{equ:dominolcomplete} does not hold as for  and  we have  but . \item  is future unique w.r.t.  (as this always holds). 
\end{enumerate}

\noindent Contrary, using more information from the future, i.e., using  instead of , renders  state-based asynchronously -complete. However, in this case the future uniqueness-property is lost.

 \begin{enumerate}[(D1)]
\item  is state-based asynchronously -complete w.r.t. :\\
 Using more future information actually resolves the ambiguity from . E.g., choosing  we can only pick  to obtain , obviously implying .
\item  is \emph{not} future-unique w.r.t. :\\
\eqref{equ:future_unique} does not hold as for  we have  but obviously . 
\end{enumerate}

\noindent Using \REFthm{thm:SimRel_QsyseQsysal} we can now construct relations  and  analogously to the ones for  in \eqref{equ:example:Rl1}. However, now (C1)-(D2) imply that 


\noindent To see that  is not a simulation relation from  to  pick  and  and observe that  does not have an outgoing transition labeled by .

Recall from (D2) that  is not future unique for . Using \eqref{equ:future_unique} this implies that for any interval  with  (i.e., any interval with two or more future values) the property of future uniqueness does not hold. 

In terms of state-based asynchronous -completeness the problem is inverted. If we use  (implying future uniqueness of  w.r.t.  from (A2) and (C2))  cannot be state-based asynchronously -complete for any  as the ambiguity for attaching dominos cannot be resolved by further knowledge about the past. In this case the counterexamples in (A1) and (C1) can be reused by pre-appending the considered strings by an appropriate number of diamonds. It is rather necessary to look at least two steps into the future, i.e., pick , to resolve this ambiguity as shown in (D1). 

Concluding the above discussion there obviously exists no  and  s.t.  in \REFfig{fig:exp_Qsyse} is both state-based asynchronously -complete and future unique w.r.t. . Therefore, increasing  and  will never result in a bisimilar abstraction of .









