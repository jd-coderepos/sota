\documentclass[conference]{IEEEtran}


\usepackage{blindtext}
\usepackage[utf8]{inputenc}





\usepackage{pdfpages}
\usepackage{graphicx}
\graphicspath{ {/Users/chedia/Downloads/} }
\usepackage{epstopdf}
\usepackage{amsthm}
\usepackage{amsmath, amssymb}
\usepackage{bbm}


\hyphenation{op-tical net-works semi-conduc-tor IEEEtran pro-duct}


\title{internship report}


\newcommand{\reels}{\mathbb{R}}
\newcommand{\esperance}{\mathbb{E}}

\newtheorem{Pro}{Proposition}
\newtheorem{Exm}{Example}
\newtheorem{Rem}{Remark}
\newtheorem{Lem}{Lemma}
\newtheorem{Cor}{Corollary}
\newtheorem{Fact}{Fact}
\newtheorem{Def}{Definition}
\newtheorem{Quest}{Question}
\newtheorem{Theorem}{Theorem}
\newtheorem*{proof*}{Proof}
\newtheorem{Prob}{Problem}

\abovecaptionskip3pt \belowcaptionskip-12pt

\begin{document}


\title{The Effects of Mobility on the Hit Performance of Cached D2D Networks}

\author{Chedia~Jarray and Anastasios~Giovanidis\
\label{snr}
\mathrm{SNR}(r_i)=\frac{PHr_{i}^{-\alpha}} {N}.

\label{rate}
\mathcal{R}(r_i) & = & W\log_2\left(1+\mathrm{SNR}(r_i)\right).

\sum_{j=1}^{F} a_j = 1.

\label{Pareto}
\bar{F}_{P}(z): = P\left[Z > z\right] = (\beta/z)^{a},

\label{Weibull}
\bar{F}_{W}(z) = \exp(-(z/\mu)^{k}).

\label{lognormalpdf}
f_{\log}(z) = \frac{1}{z\sigma \sqrt{2\pi}}\exp\left(\frac{(\ln z - \mu)^2}{2\sigma^2}\right).

	\label{bj}
    		b_j = Pr(c_j \in \Xi), & & 0\leq b_j \leq 1 , \ \forall {j},
     	
\label{probbjK}
\sum_{j=1}^{F} b_j \leq K,

\label{Aij}
A_{ij}:=\{c_j \in \Xi^{(i)}\},

\label{Bij}
B_{ij}:={\left\{\tau_i W \log_2(1+\mathrm{SNR}(r_i)) \geq z_j\right\}},

   \Psi_{ij}:=\mathbf{1}_{\{A_{ij}\}} \mathbf{1}_{\{B_{ij}\}}=1,

\label{Sj}
S_j := \{\sum_{i=1}^{\infty} \Psi_{ij} \geq 1\}.

\label{Pservj}
P_{srv,j}(b_j;z_j) = Pr(S_j) = Pr\left[\sum_{i=1}^{\infty} \Psi_{ij} \geq 1\right].

\label{Psrv}
 P_{srv}\left(\{b_j\};\left\{a_j\right\},\left\{z_j\right\}\right)= \sum_{j=1}^{F} a_j P_{srv,j}(b_j;z_j).

\label{expPsrv}
\tilde{P}_{srv}(\{b_j\};\{a_j\}) & := & \mathbb{E}\left[P_{srv}\left(\left\{b_j\right\};\left\{a_j\right\},\left\{z_j\right\}\right)\right]\nonumber\\
& \stackrel{(\ref{Psrv}),indep.}{=} & \sum_{j=1}^{F} a_j \mathbb{E}\left[P_{srv,j}(b_j;Z)\right].

\label{Psrvj}
P_{srv,j}(b_j;z_j) = 1 - \exp\left(-\pi \lambda_t b_j  \frac{I_H}{N^{2/ \alpha}} I_{T}(z_j,\bar{\tau}) \right),

\label{Eh}
I_H & := & P\mathbb{E}[H^{2/ \alpha}],\\
\label{Et}
I_{T}(z_j,\bar{\tau}) & := & \mathbb{E}[(\frac {1} {2^{\frac{z_j}{W T}}-1})^{2/ \alpha}].

\label{eqPsrv}
P_{srv,j} & = & Pr\left[\sum_{i = 1}^{\infty}\Psi_{ij}\geq 1\right] =  1-Pr\left[\sum_{i=1}^{\infty} \Psi_{ij}=0\right] \nonumber \\
& = & 1-Pr\left[\bigcap_{i=1}^{\infty} \{\Psi_{ij}=0\}\right]\nonumber \\
& \overset{(a)}{=} & 1-\mathbb{E}\left[\prod_{i=1}^{\infty}\mathbf{1}_{\{\Psi_{ij}=0\}}\right]\nonumber \\
& \overset{(b)}{=} & 1-\mathbb{E}\left[\prod_{i=1}^{\infty}Pr(\Psi_{ij}=0)\right]\nonumber \\
& \overset{(c)}{=}  & 1-\exp\left(-\int\limits_{\reels^{2}}(1-u_{j} (x))\lambda_t dx\right)\nonumber \\
& \overset{(d)}{=}  & 1-\exp\left(-2\pi \lambda_t \int\limits_{0}^{\infty} (1-u_{j} (r))r dr\right).

 \label{eqn:15}
 u_{j} (x_i) & = & Pr(\Psi_{ij}=0) =  Pr(\mathbf{1}_{A_{ij}}\mathbf{1}_{B_{ij}}=0)\nonumber \\
 & = & Pr(\mathbf{1}_{\{A_{ij}\cap B_{ij}\}}=0)\nonumber \\
& = & 1-Pr({A_{ij}}\cap {B_{ij}})\nonumber \\
 & \overset{(e)}{=} & 1-Pr(A_{ij})Pr(B_{ij})\nonumber \\
 & := & 1-f_j(|x_i|).

\label{fj}
f_j(r_i) & := & \underset{(i)}{\underbrace{Pr(c_j \in \Xi_i)}}\underset{(ii)}{\underbrace{Pr(\tau_i W\log_2 (1+\mathrm{SNR}(r_i)) \geq z_j)}}\nonumber\\
 & \overset{\tau_i\sim T}{=}  & b_j Pr\left(\mathrm{SNR}(r_i) \geq 2^{\frac{z_j}{W T}}-1\right).

 P_{srv,j} & = &  1-\exp\left(-2\pi \lambda_t \int\limits_{0}^{\infty} f_j(r)r dr\right).
 \label{eqn:20}

P_{srv,j} & \overset{(k)}{=}  & 1-\exp\left(-2\pi \lambda_t b_j \int\limits_{0}^{\infty} Pr(r^{\alpha} \leq \tilde{R}_j)rdr\right)\nonumber\\
& = & 1-\exp\left(-2\pi \lambda_t b_j \int\limits_{0}^{\infty}\mathbb{E}[\mathbf{1}_{\{r^{\alpha} \leq \tilde{R}_j\}}]rdr\right)\nonumber\\
&  \overset{(\ell)}{=}  & 1-\exp\left(-2\pi \lambda_t b_j \mathbb{E} \left[\int\limits_{0}^{\infty} \mathbf{1}_{\{ r \leq \tilde{R}_j^{1/\alpha} \}} rdr\right]\right)\nonumber\\
& = & 1-\exp\left(-2\pi \lambda_t b_j \mathbb{E} \left[\int_{0}^{\tilde{R}_j^{1/\alpha}}rdr\right]\right)\nonumber\\
& = & 1-\exp\left(-2\pi \lambda_t b_j \mathbb{E} \left[\frac {r^{2}} {2} |_{0}^{\tilde{R}_j^{1/\alpha}}\right]\right)\nonumber\\
& = & 1-\exp\left(-\pi \lambda_t b_j \mathbb{E}[\tilde{R}_j^{2/ \alpha}]\right),\nonumber
 
  P_{srv,j}(b_j;z_j) = 1-\exp(-\pi \lambda_t b_j  \frac{P^{2/\alpha}\Gamma(\frac{2}{\alpha}+1)}{N^{2/\alpha}} I_T(z_j,\bar{\tau})).\nonumber
  
\label{ITfix}
I_T(z_j,\bar{\tau}) \overset{T=\bar{\tau}}{=} \left(2^{\frac{z_j}{W\bar{\tau}}}-1\right)^{-2/\alpha}.

 \label{ITexp}
 I_T(z_j,\bar{\tau}) \overset{Exp}{=} \int_{0}^{\infty} \frac{1}{\bar{\tau}} \exp(-\frac{s}{\bar{\tau}}) \left(2^{\frac{z_j}{Ws}}-1\right)^{-2/ \alpha} ds.

\label{TotCor1}
P_{srv} = 1-\sum_{j=1}^F a_j\exp\left(-\pi\lambda_t b_j \frac{I_H}{N^{2/\alpha}} I_T(z_j,\bar{\tau})\right).

\label{ETotCor1}
\tilde{P}_{srv} = 1-\sum_{j=1}^F a_j\mathbb{E}\left[\exp\left(-\pi\lambda_t b_j \frac{I_H}{N^{2/\alpha}} I_T(Z,\bar{\tau})\right)\right].

\end{Cor}


\section{Numerical Evaluation}
\label{secV}

In this section, the expressions derived for the performance of the D2D model under study are numerically evaluated. Additionally, we have run extended simulations of the system, to validate their correctness. 

Specifically for the simulation environment, we consider the following. \textit{Geometry Parameters:} The simulation is observed within a window of size  [], where transmitter devices are distributed as a Poisson point process (PPP) of density  []. With this transmitter density, the mean distance from  to the closest transmitter is  . Each node has a lifespan that is exponentially distributed with mean value that varies for (a) Audio files  [], and (b) for Video files  []. The receivers form also a PPP but we consider just one receiver per realisation placed at the Cartesian origin , since no resource sharing is assumed. The simulation results are averaged over  iterations (a larger number gives even better fit). \textit{Wireless Parameters:} The transmission is interference free. Each node operates on a bandwidth of  [] and emits with Power  [], whereas noise power is  []. The path-loss exponent is  and fading is Rayleigh (hence exponential distribution). \textit{Content Parameters:} We consider a catalogue of size  objects, and a Zipf distribution for popularity, with exponent . The objects can be (a) Audio files, or (b) Video files. The cache size is  [] and we do not consider the influence of file-sizes when filling in the node inventories.

\textit{Content Placement:} We use the probabilistic block placement policy (PBP) proposed in \cite{BlaGioICC15}, which for each node samples an independent vector of at most  objects, and satisfies the placement probabilities  for every object . The vector  should be given as system input. The choice of entries is critical for the system performance itself and is a design parameter. In the current simulation we take , for  and , otherwise. Then, for , , where  is a sort of normalised popularity , so that .

We investigate different file-size distributions for the random variable . In all cases, the choice of file-size per object  is i.i.d. and , , so that the mean size for Audio is  [] and for Video  []. In the case of a \textit{Uniform} distribution, the range of file-sizes for Audio and Video given in Table \ref{Tab1} satisfy the values of the expectation.


\subsection{Validation}

The correctness of the expression in (\ref{TotCor1}) is validated for the case of Audio and Video files separately. Both sets of file-sizes ( in total in each set) are \textit{sampled} from an exponential distribution with mean values  [] (audio) and  [] (video) respectively. The expression for  is given in (\ref{ITexp}) because lifespan distribution is exponential as well. The comparison between analysis and simulation is illustrated in Fig. \ref{fig:valid} for the Total Service Success Probability over a range of mean lifespan values, which is chosen differently for the two file categories. The plots show an excellent match between analysis and simulations. Interestingly, we observe that for  [], audio files have a success probability , much higher than the success probability of videos for the same value of mean lifespan . This is reasonable due to the difference in mean file-size. Both sub-figures show a diminishing increase of . The probability should converge to some value less than one, because of the placement policy, which leaves  objects definitely uncached. Moving on the -axis in both plots from left to right represents a change in the D2D behaviour, from higher to lower \textit{mobility}.
\begin{figure}[ht!]
\centering
\includegraphics[width=1.7in]{AudioTPsVSlifesW.eps}
\includegraphics[width=1.7in]{VideoTPsVSlifesW.eps}
\caption{Total service probability with respect to mean node lifespan, for cached (a) Audio files and (b) Video files.}
\label{fig:valid}
\end{figure}


\subsection{Evaluation}

\subsubsection{Influence of file-size order related to popularity}
The results in Fig. \ref{fig:valid} are obtained when the file-size of each content is independently sampled from an exponential distribution. There is, hence, no correlation between popularity and file-size, meaning that the most popular file may have any size. We investigate how the service success probability is influenced from a possible correlation between the two file characteristics. More specifically, we simulate (and analytically calculate) two scenarios, one when the file-sizes are in decreasing order in relation to the popularity index (the most popular file is the largest one from the sample set, the second most popular the second largest and so on), and another scenario when the file sizes are in increasing order (most popular file is the smallest one). These two curves over the mean lifespan are produced for Video in Fig. \ref{fig:Corr} (for Audio the behaviour is similar), and can be directly compared to those in Fig. \ref{fig:valid}(b). We observe that when the file-sizes are in increasing order, the  is higher than in the independent case, because more popular files (which are also cached due to the choice of the placement policy) are smaller and thus more likely to be fully transmitted within the lifespan. On the other hand, when the sizes are in decreasing order, the  is lower than in the independent case, for exactly the opposite reason.
\begin{figure}[ht!]
\centering
\includegraphics[width=3.4in]{cVideoTPsVSlifes.eps}
\caption{Increasing/Decreasing file-size with popularity order and how it influences the total service probability, for cached Video files, over the mean node lifespan.}
\label{fig:Corr}
\end{figure}


\subsubsection{Influence of file-size distribution in the Expected/Total Service Success Probability}
We investigate this very important issue with the following methodology. We consider five possible distributions for the file-size of Videos: \textit{A}. \textit{Uniform} within , \textit{B}. \textit{Exponential} with parameter , \textit{C}. \textit{Pareto} with parameters , \textit{D}. \textit{Weibull} with parameters , \textit{E}. \textit{Log-Normal} with parameters . Each one of these has very different characteristics and the system performance depends on the parameter values. 

To keep comparison fair, we choose the parameters so that  [], which is the mean value of the video file-size. Having this in mind, \textit{A}. the Uniform is in agreement with the expected value when using the upper and lower bounds in Table \ref{Tab1}. \textit{B}. For Exponential . \textit{C}. For Pareto it holds that , and  to have finite first moment. Then we choose  and  which not only gives the desired expected value but also guarantees the same lower bound   with the Uniform distribution. \textit{D}. For Weibull, , so that the choice is  and . Finally, \textit{E}. for Log-normal,  and we choose , .

To observe the influence of these file-size distributions on the probability of service, we first make use of the \textit{Expected Service Success Probability} metric, in (\ref{ETotCor1}). To simplify numerical evaluation, we use a fixed lifespan for all nodes  , so that  is given in (\ref{ITfix}). The results of the evaluations, with the parameters of each distribution chosen as above, are given in Fig. \ref{fig:Expdistvar}. From the plots, we observe that the distributions are ordered as: . We note that the three heavy-tailed (or just heavier than the exponential) distributions Weibull, Pareto and Log-Normal, with the parameter set chosen, give the maximum service performance. The reason that the exponential performs poorly is that, although it does not have the tendency to generate very high values, it does however produce a sufficient number of samples large enough to keep the  low. Contrary to this, the heavy-tailed distributions may produce extremely large samples, however not a large number of them, so that small files tend to have smaller size than the ones from the exponential. Different heavy-tailed distributions can generate samples which deviate considerably from the mean towards higher values, but with even smaller low values, in order to keep the same  []. The reason for the poor performance of the uniform is that, although bounded between  [], a considerable amount of samples, due to uniform sampling, will be around the highest value, whereas no samples can be smaller than the lowest bound. The performance in the plots converges to an upper bound , equal to the sum of popularities of the  cached most popular files  for .


\begin{figure}[ht!]
\centering
\includegraphics[width=3.4in]{ExpectedServeSize.eps}
\caption{Comparison of Expected Service Success Probability, with different file-size distributions 
for Video.}
\label{fig:Expdistvar}
\end{figure}

\vspace{+0.5cm}
To further give intuition on the effect of the tail distribution on the performance, we proceed in an alternative way: a larger sample set of size  values is produced from each distribution. The samples of each set are then indexed in decreasing order. Table \ref{Tab3} shows the five highest values per set. Using these, the Total Service Success Probability per mean lifespan is plotted, which shows the same results as the expected case, with the difference in the performance of the Uniform distribution, which is higher due to the file-size upper bound of  []. Both Fig. \ref{fig:Vardec} and Table \ref{Tab3} support the intuition and our explanations given above. 











\begin{table}[ht!]
\normalsize
\caption{Samples of file-sizes [] in descending order.}
\vspace{+0.5cm}
 \centering
\begin{tabular}{|l|c|c|c|c|c|}
 \hline
Distrib./Obj. &  & & & &  \\ [1 ex]
 \hline
A. Uniform & 2.00 & 1.99 & 1.98 & 1.97 & 1.97 \\ [0.5 ex]
 \hline
B. Exponential & 6.21 & 5.64 & 5.33 & 4.22 & 4.01 \\ [0.5 ex]
\hline
C. Pareto & 8.93 & 6.37 & 5.70 & 4.92 & 1.47  \\ [0.5 ex]
\hline
D. Weibull & 24.00 & 18.07 & 1.45 & 0.07 & 0.04  \\ [0.5 ex]
\hline
E. Log-Normal & 7.28 & 7.05 & 2.49 & 1.13 & 1.05  \\ [0.5 ex]
\hline
\end{tabular}
\label{Tab3}
\end{table}



\begin{figure}[ht!]
\centering
\includegraphics[width=3.4in]{CompareTails2theory.eps}
\caption{Comparison of Total Service Success Probability, with different file-size distributions for a Video sample set of  objects in decreasing order related to popularity.}
\label{fig:Vardec}
\end{figure}


\section{Conclusions}
\label{secVI}
The influence of mobility in the performance of cached D2D networks has been investigated. In the proposed model, a link is successful if the object request from the transmitter is cached at the memory of the receiver, and the link quality is sufficient for the entire object to be transferred before the transmitter leaves its place. The change in position is assumed sudden and the link is immediately dropped afterwards. This is a simplified approach to model mobility, and more realistic approaches can be considered in the future where the link quality could evolve more gradually. The work further investigates the influence of different file-size distributions on the network performance, and the analysis concludes that the exponential distribution may underestimate performance compared to heavy-tailed ones. Furthermore, it is shown that caching smaller files is in general beneficial, since these are more probable to be successfully transferred. Further extensions of the work may include the possibility of cooperative transmission \cite{QuekCoopCache16}, as well as an evaluation when interference influences the coverage probability. 

\section{Acknowledgement}
The authors would like to thank Bart{\l}omiej B{\l}aszczyszyn for the discussions and contributions throughout this research.


\bibliography{D2D}
\bibliographystyle{plain}


\end{document}
