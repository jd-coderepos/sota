\documentclass[11pt]{article}

\usepackage{a4wide}
\usepackage{latexsym}
\usepackage{amssymb}
\usepackage{amsmath}
\usepackage{epsfig}
\usepackage{subfigure}
\usepackage{url}
\usepackage{fullpage}
\usepackage[labelsep=none]{caption}

\newtheorem{theorem}{Theorem}\newtheorem{definition}[theorem]{Definition}
\newtheorem{prop}[theorem]{Proposition}
\newtheorem{observation}[theorem]{Observation}
\newtheorem{lemma}[theorem]{Lemma}

\newcommand{\R}{{\mathbb{R}}}
\newcommand{\Z}{{\mathbb{Z}}}
\newcommand{\N}{{\mathbb{N}}}
\newcommand\eps{\varepsilon}
\newcommand{\sgn}{\mathop {\rm sgn}\nolimits}
\newcommand\FF{\mathcal{F}}
\newcommand\GG{\mathcal{G}}
\renewcommand\AA{\mathcal{A}}

\DeclareMathOperator{\bob}{{\sf Bob}}
\DeclareMathOperator{\stairs}{{\sf stairs}}

\newcommand{\qed}{\hspace{\stretch{1}}}
\newenvironment{proof}{\vspace{-.25\baselineskip}\noindent\textbf{Proof.}
}{\qed\par\medskip}

\renewcommand{\dagger}{**}



\begin{document}
\title
{A Doubly Exponentially Crumbled Cake}

\author{
{Tobias Christ \footnote{Institute of Theoretical Computer Science, ETH Z\"urich, 8092 Z\"urich, Switzerland, {\texttt{\{christt, afrancke, gebauerh\}@inf.ethz.ch}} } }\quad
{Andrea Francke } \quad
{Heidi Gebauer } \quad
{Ji\v{r}\'{\i} Matou\v{s}ek \footnote{Dept. of Applied Mathematics and Institute of Theoretical Computer Science, Charles University, Malostransk\'{e} n\'{a}m. 25,
118~00~~Praha~1, Czech Republic, and Institute of Theoretical Computer Science, ETH Z\"urich, 8092 Z\"urich, Switzerland, {\texttt{matousek@kam.mff.cuni.cz}}} } \quad
{Takeaki Uno \footnote{National Institute of Informatics, 2-1-2, Hitotsubashi, Chiyoda-ku,
Tokyo 101-8430, Japan, {\texttt{uno@nii.jp}}}}
}

\maketitle
\begin{abstract}
We consider the following cake cutting game: 
Alice chooses a set~ of ~points in
the square (cake)~, where ; 
Bob cuts out  axis-parallel rectangles with disjoint
interiors, each of them having a point of  as the
lower left corner;  Alice keeps the rest.
It has been conjectured that Bob can always secure at least half
of the cake. This remains unsettled, and it is not even known
whether Bob can get any positive fraction independent of~.
 We prove that \emph{if} Alice can force Bob's share
to tend to zero, \emph{then} she must use very many points; namely,
to prevent Bob from gaining more than  of the cake,
she needs at least  points. 
\end{abstract}
\section{Introduction}

Alice has baked a square cake with raisins for Bob, but
really she would like to keep most of it for herself.
In this, she relies on a peculiar habit of Bob: he eats only
rectangular pieces of the cake, with sides parallel
to the sides of the cake, that contain exactly one raisin each,
and that raisin has to be exactly in the lower left corner
(see Fig.~\ref{f:example}). Alice gets whatever remains
after Bob has cut out all such pieces. In order to give
Bob at least some chance, Alice has to put a raisin
in the lower left corner of the whole cake. 

Mathematically, the cake is the square , the raisins
form an -point set , where
 is required, and Bob's share consists of 
 axis-parallel rectangles with disjoint
interiors, each of them having a point of  as the
lower left corner.

By placing points densely along the main diagonal,
Alice can limit Bob's share to~,
with  arbitrarily small.
A natural question then is, can Bob always obtain
 at least half of the cake?

This question (in a cake-free formulation) appears in
Winkler~\cite{Win07} (``Packing Rectangles'', p.~133),
where he claims it to be at least 10 years old and of
origin unknown to him. The first written reference seems
to be an IBM puzzle webpage~\cite{IBM04}.


\begin{figure}[tbh]
\centering
\subfigure{
\includegraphics[width=0.3\textwidth]{example_points} }
~~~~~~~~~~~~~~~~~~
\subfigure{
\includegraphics[width=0.3\textwidth]{example_rectangles}}
\caption{\label{f:example} Example: Alice's points (left)
and Bob's rectangles (right).}
\end{figure}



We tried to answer the question and could not, probably similar to many 
other people before us. We believe that there are no simple examples
leaving more than  to Alice, but on the other hand,
it seems difficult to prove even that Bob can always secure
 of the cake. We were thus led to seriously considering
the possibility that Alice might be able
to limit Bob's share to less than , for every ,
but that the number of points  she would need 
would grow enormously as a function of~.

Here we prove a doubly exponential lower bound on this function.
First we introduce the following notation. For a finite
,
let  be the largest area Bob can win for , and
let  be the infimum of  over all -point 
as above.\footnote{It is easily checked that, given ,
there are finitely many possible placement of 
Bob's \emph{inclusion-maximal} rectangles, and therefore,
 is attained by some choice of rectangles.
On the other hand, it is not so clear whether
 is attained; we leave this question
aside.} Also, for a real number  let
.

\begin{theorem}\label{t:} There exists a constant  such that for all ,
.
\end{theorem}

The only previous results on this problem we could find
is the Master's thesis of M\"{u}ller-Itten \cite{Mue10}. She conjectured
that  Alice's optimal strategy is placing the  points
on the main diagonal with equal spacing (for which Bob's share
is ). She proved
this conjecture for , and also in the ``grid''
case with , where  is a permutation of .
She also showed that .

The problem considered here can be put into a wider context.
Various problems of fair division of resources, often phrased
as cake-cutting problems, go back at least to Steinhaus, Banach and Knaster;
see, e.g., \cite{RW98}. Even closer to our particular setting
is Winkler's \emph{pizza problem}, recently solved by 
Cibulka et al.~\cite{CKMS10}.  



\section{Preliminaries} 
We call a point~ a \emph{minimum} of a set 
if there is no  for which both 
 and . 
Let  be an enumeration
of the minima of  in the order
of decreasing -coordinate (and increasing -coordinate).
Let  be the union of all the axis-parallel rectangles
with lower left corners at  whose interior avoids ;
see Fig.~\ref{f:sta}(a).


\begin{figure}[tbh]
\centering
\subfigure[ and the subproblems.]
{\includegraphics[width=0.35\textwidth]
{cake-stairs}}
\qquad
\subfigure[Illustration to the proof of Lemma~\ref{l:nobig}.]{
\includegraphics[width=0.41\textwidth]
{cake-gap}}
\caption{\label{f:sta}}
\end{figure}





Furthermore, let  be the area of , and let  be the largest
area of an axis-parallel rectangle contained in .
Let us also define .
For a point  and an axis-parallel rectangle 
 with lower left corner at , we denote by~ be the maximum area
of the cake Bob can gain in~ using only rectangles
with lower left corner in points of .
By re-scaling, we have ,
where  is the area of~ and  denotes the set 
transformed by the affine transform that maps  onto .

We will use the monotonicity of , i.e.,
 for all .
Indeed, Alice can always place an extra point on the right side
of the square, say, which does not influence Bob's share.


\section{The decomposition} We decompose the complement of 
into horizontal rectangles  as indicated
in Fig.~\ref{f:sta}(a), so that  is the lower left
corner of . Let  be the area of ;
we have .

By the above and by an obvious superadditivity, we have

where  (This is a somewhat simple-minded estimate, since it doesn't take
into account any interaction among the ).

The following lemma captures the main properties of this
decomposition.

\begin{lemma}\label{l:nobig}
Let us  assume that , where
 is a suitable (sufficiently large) constant.
Then 
\begin{itemize}\item 
 (the staircase has a small area), and
\item 
 
for every  (none of the subproblems
occupies almost all of the area).
\end{itemize}
\end{lemma}

\begin{proof} First we note that 
since no rectangle with lower left corner 
and upper right corner in  has area bigger than
, the region  lies below the hyperbola
. Thus
. This yields ,
and so 
(for  sufficiently large).


It remains to show that ;
since , it suffices to show 
 for all~.
 
Let  be the -coordinate of  for , and let ;
we have  for . 

First, if , then  by the above, and 
we are done. So we assume .

The area of  can be bounded from above as indicated
in Fig.~\ref{f:sta}(b). Namely, the rectangle  has area 
at most  (since it is contained in ), and
the rectangle  above it also has area no more than  
(using ).  The top right corner of 
lies on the hyperbola  used above, and thus 
has area at most  as well. Finally, the region 
on the right of  and below the hyperbola has area
.





Since , we 
have . Using 
we  obtain  (again using the assumption that  is large).

Finally, we have ,
and the lemma is proved.
\end{proof}

\section{Proof of Theorem~\ref{t:}}
\begin{proof}Let .
We may assume that  is of the form ,
where . In particular,  for all .

We will derive the following recurrence for such an :

Applying it iteratively  times, we find that
 as claimed in the theorem.

We thus start with the derivation of (\ref{e:recur}).
Let us look at the inequality (\ref{e:decompose}) for an -point set 
that attains .\footnote{Or rather,
since we haven't proved that 
is attained, we should choose -point 
with   for all .}
Since  for all , we have
 for all . 

Let  and  be as above.
First we derive .
Indeed, if we had , then the right-hand of 
(\ref{e:decompose}) can be estimated as follows:

which contradicts the inequality~(\ref{e:decompose}).
So  indeed.



Let us set ; this is Bob's ``gain'' over the ratio 
in the th subproblem. From (\ref{e:decompose}) we have

and so


According to Lemma~\ref{l:nobig}, we can partition the index set
 into two subsets  so that
 for . 


Let  be such that , and
similarly for . Then (\ref{e:gains}) gives, for ,

and so .

Let us define  by .
Then we know that at least two of the sets  contain at least
 points each, and hence .
We calculate  (again using ).

So we have derived the desired recurrence (\ref{e:recur}),
and Theorem~\ref{t:} is proved.
\end{proof}
\subsection*{Acknowledgments}
This research was partially done at the \emph{Gremo Workshop on Open Problems} 2010,
and the support of the ETH Z\"urich is gratefully acknowledged.
We would like to thank Michael Hoffmann and Bettina Speckmann
for useful discussion, and the organizers and participants of GWOP~2010 for a beautiful workshop.





\bibliographystyle{abbrv}
\begin{thebibliography}{1}

\bibitem{IBM04}
Ponder this. puzzle for june 2004.
\newblock
  \url{http://domino.research.ibm.com/Comm/wwwr_ponder.nsf/challenges/June2004.html}, June 2004.

\bibitem{CKMS10}
J.~Cibulka, J.~Kyn\v{c}l, V.~M\'{e}sz\'{a}ros, R.~Stola\v{r}, and P.~Valtr.
\newblock Solution of {Peter Winkler's} pizza problem.
\newblock In {\em {Katona, Gyula O. H. (ed.) et al.: Fete of combinatorics and
  computer science}}, pages 63--93. Springer, Berlin; Bolyai Math. Soc.,
  Budapest, 2010.

\bibitem{Mue10}
M.~M{\"{u}}ller-Itten.
\newblock {Packing Rectangles. A Cake Sharing Puzzle.}
\newblock Master's thesis, EPF Lausanne, Switzerland, 2010.

\bibitem{RW98}
J.~Robertson and W.~Webb.
\newblock {\em {Cake-cutting algorithms. Be fair if you can.}}
\newblock {A K Peters Ltd., Natick, MA}, 1998.

\bibitem{Win07}
P.~Winkler.
\newblock {\em {Mathematical mind-benders}}.
\newblock {A K Peters Ltd., Wellesley, MA}, 2007.

\end{thebibliography}

\end{document}
