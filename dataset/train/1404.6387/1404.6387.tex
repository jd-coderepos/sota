\documentclass[letterpaper,compsoc,twoside]{IEEEtran}
\usepackage{fixltx2e} \usepackage{cmap} \usepackage{ifthen}
\usepackage[T1]{fontenc}
\usepackage[utf8]{inputenc}
\usepackage{amsmath}

\usepackage[font={small,it},labelfont=bf]{caption}
\usepackage{float}

\setcounter{secnumdepth}{3}

\pdfoutput=1
\usepackage{scipy}
\makeatletter
\def\PY@reset{\let\PY@it=\relax \let\PY@bf=\relax \let\PY@ul=\relax \let\PY@tc=\relax \let\PY@bc=\relax \let\PY@ff=\relax}
\def\PY@tok#1{\csname PY@tok@#1\endcsname}
\def\PY@toks#1+{\ifx\relax#1\empty\else \PY@tok{#1}\expandafter\PY@toks\fi}
\def\PY@do#1{\PY@bc{\PY@tc{\PY@ul{\PY@it{\PY@bf{\PY@ff{#1}}}}}}}
\def\PY#1#2{\PY@reset\PY@toks#1+\relax+\PY@do{#2}}

\expandafter\def\csname PY@tok@gd\endcsname{\def\PY@tc##1{\textcolor[rgb]{0.63,0.00,0.00}{##1}}}
\expandafter\def\csname PY@tok@gu\endcsname{\let\PY@bf=\textbf\def\PY@tc##1{\textcolor[rgb]{0.50,0.00,0.50}{##1}}}
\expandafter\def\csname PY@tok@gt\endcsname{\def\PY@tc##1{\textcolor[rgb]{0.00,0.27,0.87}{##1}}}
\expandafter\def\csname PY@tok@gs\endcsname{\let\PY@bf=\textbf}
\expandafter\def\csname PY@tok@gr\endcsname{\def\PY@tc##1{\textcolor[rgb]{1.00,0.00,0.00}{##1}}}
\expandafter\def\csname PY@tok@cm\endcsname{\let\PY@it=\textit\def\PY@tc##1{\textcolor[rgb]{0.25,0.50,0.56}{##1}}}
\expandafter\def\csname PY@tok@vg\endcsname{\def\PY@tc##1{\textcolor[rgb]{0.73,0.38,0.84}{##1}}}
\expandafter\def\csname PY@tok@m\endcsname{\def\PY@tc##1{\textcolor[rgb]{0.13,0.50,0.31}{##1}}}
\expandafter\def\csname PY@tok@mh\endcsname{\def\PY@tc##1{\textcolor[rgb]{0.13,0.50,0.31}{##1}}}
\expandafter\def\csname PY@tok@cs\endcsname{\def\PY@tc##1{\textcolor[rgb]{0.25,0.50,0.56}{##1}}\def\PY@bc##1{\setlength{\fboxsep}{0pt}\colorbox[rgb]{1.00,0.94,0.94}{\strut ##1}}}
\expandafter\def\csname PY@tok@ge\endcsname{\let\PY@it=\textit}
\expandafter\def\csname PY@tok@vc\endcsname{\def\PY@tc##1{\textcolor[rgb]{0.73,0.38,0.84}{##1}}}
\expandafter\def\csname PY@tok@il\endcsname{\def\PY@tc##1{\textcolor[rgb]{0.13,0.50,0.31}{##1}}}
\expandafter\def\csname PY@tok@go\endcsname{\def\PY@tc##1{\textcolor[rgb]{0.20,0.20,0.20}{##1}}}
\expandafter\def\csname PY@tok@cp\endcsname{\def\PY@tc##1{\textcolor[rgb]{0.00,0.44,0.13}{##1}}}
\expandafter\def\csname PY@tok@gi\endcsname{\def\PY@tc##1{\textcolor[rgb]{0.00,0.63,0.00}{##1}}}
\expandafter\def\csname PY@tok@gh\endcsname{\let\PY@bf=\textbf\def\PY@tc##1{\textcolor[rgb]{0.00,0.00,0.50}{##1}}}
\expandafter\def\csname PY@tok@ni\endcsname{\let\PY@bf=\textbf\def\PY@tc##1{\textcolor[rgb]{0.84,0.33,0.22}{##1}}}
\expandafter\def\csname PY@tok@nl\endcsname{\let\PY@bf=\textbf\def\PY@tc##1{\textcolor[rgb]{0.00,0.13,0.44}{##1}}}
\expandafter\def\csname PY@tok@nn\endcsname{\let\PY@bf=\textbf\def\PY@tc##1{\textcolor[rgb]{0.05,0.52,0.71}{##1}}}
\expandafter\def\csname PY@tok@no\endcsname{\def\PY@tc##1{\textcolor[rgb]{0.38,0.68,0.84}{##1}}}
\expandafter\def\csname PY@tok@na\endcsname{\def\PY@tc##1{\textcolor[rgb]{0.25,0.44,0.63}{##1}}}
\expandafter\def\csname PY@tok@nb\endcsname{\def\PY@tc##1{\textcolor[rgb]{0.00,0.44,0.13}{##1}}}
\expandafter\def\csname PY@tok@nc\endcsname{\let\PY@bf=\textbf\def\PY@tc##1{\textcolor[rgb]{0.05,0.52,0.71}{##1}}}
\expandafter\def\csname PY@tok@nd\endcsname{\let\PY@bf=\textbf\def\PY@tc##1{\textcolor[rgb]{0.33,0.33,0.33}{##1}}}
\expandafter\def\csname PY@tok@ne\endcsname{\def\PY@tc##1{\textcolor[rgb]{0.00,0.44,0.13}{##1}}}
\expandafter\def\csname PY@tok@nf\endcsname{\def\PY@tc##1{\textcolor[rgb]{0.02,0.16,0.49}{##1}}}
\expandafter\def\csname PY@tok@si\endcsname{\let\PY@it=\textit\def\PY@tc##1{\textcolor[rgb]{0.44,0.63,0.82}{##1}}}
\expandafter\def\csname PY@tok@s2\endcsname{\def\PY@tc##1{\textcolor[rgb]{0.25,0.44,0.63}{##1}}}
\expandafter\def\csname PY@tok@vi\endcsname{\def\PY@tc##1{\textcolor[rgb]{0.73,0.38,0.84}{##1}}}
\expandafter\def\csname PY@tok@nt\endcsname{\let\PY@bf=\textbf\def\PY@tc##1{\textcolor[rgb]{0.02,0.16,0.45}{##1}}}
\expandafter\def\csname PY@tok@nv\endcsname{\def\PY@tc##1{\textcolor[rgb]{0.73,0.38,0.84}{##1}}}
\expandafter\def\csname PY@tok@s1\endcsname{\def\PY@tc##1{\textcolor[rgb]{0.25,0.44,0.63}{##1}}}
\expandafter\def\csname PY@tok@gp\endcsname{\let\PY@bf=\textbf\def\PY@tc##1{\textcolor[rgb]{0.78,0.36,0.04}{##1}}}
\expandafter\def\csname PY@tok@sh\endcsname{\def\PY@tc##1{\textcolor[rgb]{0.25,0.44,0.63}{##1}}}
\expandafter\def\csname PY@tok@ow\endcsname{\let\PY@bf=\textbf\def\PY@tc##1{\textcolor[rgb]{0.00,0.44,0.13}{##1}}}
\expandafter\def\csname PY@tok@sx\endcsname{\def\PY@tc##1{\textcolor[rgb]{0.78,0.36,0.04}{##1}}}
\expandafter\def\csname PY@tok@bp\endcsname{\def\PY@tc##1{\textcolor[rgb]{0.00,0.44,0.13}{##1}}}
\expandafter\def\csname PY@tok@c1\endcsname{\let\PY@it=\textit\def\PY@tc##1{\textcolor[rgb]{0.25,0.50,0.56}{##1}}}
\expandafter\def\csname PY@tok@kc\endcsname{\let\PY@bf=\textbf\def\PY@tc##1{\textcolor[rgb]{0.00,0.44,0.13}{##1}}}
\expandafter\def\csname PY@tok@c\endcsname{\let\PY@it=\textit\def\PY@tc##1{\textcolor[rgb]{0.25,0.50,0.56}{##1}}}
\expandafter\def\csname PY@tok@mf\endcsname{\def\PY@tc##1{\textcolor[rgb]{0.13,0.50,0.31}{##1}}}
\expandafter\def\csname PY@tok@err\endcsname{\def\PY@bc##1{\setlength{\fboxsep}{0pt}\fcolorbox[rgb]{1.00,0.00,0.00}{1,1,1}{\strut ##1}}}
\expandafter\def\csname PY@tok@kd\endcsname{\let\PY@bf=\textbf\def\PY@tc##1{\textcolor[rgb]{0.00,0.44,0.13}{##1}}}
\expandafter\def\csname PY@tok@ss\endcsname{\def\PY@tc##1{\textcolor[rgb]{0.32,0.47,0.09}{##1}}}
\expandafter\def\csname PY@tok@sr\endcsname{\def\PY@tc##1{\textcolor[rgb]{0.14,0.33,0.53}{##1}}}
\expandafter\def\csname PY@tok@mo\endcsname{\def\PY@tc##1{\textcolor[rgb]{0.13,0.50,0.31}{##1}}}
\expandafter\def\csname PY@tok@mi\endcsname{\def\PY@tc##1{\textcolor[rgb]{0.13,0.50,0.31}{##1}}}
\expandafter\def\csname PY@tok@kn\endcsname{\let\PY@bf=\textbf\def\PY@tc##1{\textcolor[rgb]{0.00,0.44,0.13}{##1}}}
\expandafter\def\csname PY@tok@o\endcsname{\def\PY@tc##1{\textcolor[rgb]{0.40,0.40,0.40}{##1}}}
\expandafter\def\csname PY@tok@kr\endcsname{\let\PY@bf=\textbf\def\PY@tc##1{\textcolor[rgb]{0.00,0.44,0.13}{##1}}}
\expandafter\def\csname PY@tok@s\endcsname{\def\PY@tc##1{\textcolor[rgb]{0.25,0.44,0.63}{##1}}}
\expandafter\def\csname PY@tok@kp\endcsname{\def\PY@tc##1{\textcolor[rgb]{0.00,0.44,0.13}{##1}}}
\expandafter\def\csname PY@tok@w\endcsname{\def\PY@tc##1{\textcolor[rgb]{0.73,0.73,0.73}{##1}}}
\expandafter\def\csname PY@tok@kt\endcsname{\def\PY@tc##1{\textcolor[rgb]{0.56,0.13,0.00}{##1}}}
\expandafter\def\csname PY@tok@sc\endcsname{\def\PY@tc##1{\textcolor[rgb]{0.25,0.44,0.63}{##1}}}
\expandafter\def\csname PY@tok@sb\endcsname{\def\PY@tc##1{\textcolor[rgb]{0.25,0.44,0.63}{##1}}}
\expandafter\def\csname PY@tok@k\endcsname{\let\PY@bf=\textbf\def\PY@tc##1{\textcolor[rgb]{0.00,0.44,0.13}{##1}}}
\expandafter\def\csname PY@tok@se\endcsname{\let\PY@bf=\textbf\def\PY@tc##1{\textcolor[rgb]{0.25,0.44,0.63}{##1}}}
\expandafter\def\csname PY@tok@sd\endcsname{\let\PY@it=\textit\def\PY@tc##1{\textcolor[rgb]{0.25,0.44,0.63}{##1}}}

\def\PYZbs{\char`\\}
\def\PYZus{\char`\_}
\def\PYZob{\char`\{}
\def\PYZcb{\char`\}}
\def\PYZca{\char`\^}
\def\PYZam{\char`\&}
\def\PYZlt{\char`\<}
\def\PYZgt{\char`\>}
\def\PYZsh{\char`\#}
\def\PYZpc{\char`\%}
\def\PYZdl{\char`\^{\setcounter{footnotecounter}{1}\fnsymbol{footnotecounter}\setcounter{footnotecounter}{2}\fnsymbol{footnotecounter}}g = f^{-1}(x)f^{-1}x^{2}$. \DUrole{label}{funcbump}}
\end{figure*}

A graph transformation as taught in middle school — translation, scaling,  rotation — is modeled as a function that operates on a \texttt{source} function, producing the transformed function. In Figure \DUrole{ref}{funcbump}, PySTEMM generates a graph plot of the original function, a shifted version, and a “bumped” version of the shifted function. The instances are defined as:
\begin{Verbatim}[commandchars=\\\{\},fontsize=\footnotesize]
\PY{n}{Bump}\PY{p}{(}\PY{n}{source} \PY{o}{=}
        \PY{n}{ShiftX}\PY{p}{(}\PY{n}{source} \PY{o}{=} \PY{n}{RuleFunc}\PY{p}{(}\PY{n}{rule}\PY{o}{=}\PY{n}{square}\PY{p}{)}\PY{p}{,}
               \PY{n}{by}\PY{o}{=}\PY{l+m+mi}{3}\PY{p}{)}\PY{p}{,}
     \PY{n}{start}\PY{o}{=}\PY{l+m+mi}{0}\PY{p}{,} \PY{n}{end}\PY{o}{=}\PY{l+m+mi}{5}\PY{p}{,} \PY{n}{val}\PY{o}{=}\PY{l+m+mi}{100}\PY{p}{)}
\end{Verbatim}
Similarly, the \emph{limit} of a function is a high-order function: it operates on another function and a target point, and evaluates to a single numeric value. Calculus operators, such as \emph{differentiation} and \emph{integration}, can be modeled as high-order functions as well: they operate on a function and produce a new function.





\section{Chemistry: Reaction\label{chemistry-reaction}}
\begin{figure}[]\noindent\makebox[\columnwidth][c]{\includegraphics[width=\columnwidth]{reaction_types.pdf}}
\caption{\texttt{Reaction} concept type. \DUrole{label}{reactiontypes}}
\end{figure}\begin{figure}[]\noindent\makebox[\columnwidth][c]{\includegraphics[width=\columnwidth]{reaction_instance.pdf}}
\caption{An instance of \texttt{Reaction}. \DUrole{label}{reactioninstance}}
\end{figure}\begin{Verbatim}[commandchars=\\\{\},numbers=left,firstnumber=1,stepnumber=1,fontsize=\footnotesize,xleftmargin=2.25mm,numbersep=3pt]
\PY{k}{class} \PY{n+nc}{Element}\PY{p}{(}\PY{n}{Concept}\PY{p}{)}\PY{p}{:}
  \PY{n}{name} \PY{o}{=} \PY{n}{String}

\PY{k}{class} \PY{n+nc}{Molecule}\PY{p}{(}\PY{n}{Concept}\PY{p}{)}\PY{p}{:}
  \PY{n}{formula} \PY{o}{=} \PY{n}{List}\PY{p}{(}\PY{n}{Tuple}\PY{p}{(}\PY{n}{Element}\PY{p}{,} \PY{n}{Int}\PY{p}{)}\PY{p}{)}
  \PY{n}{instance\PYZus{}template} \PY{o}{=} \PY{p}{\PYZob{}}
    \PY{n}{K}\PY{o}{.}\PY{n}{text}\PY{p}{:} \PY{k}{lambda} \PY{n}{m}\PY{p}{:} \PY{n}{computed\PYZus{}label}\PY{p}{(}\PY{n}{m}\PY{p}{)}\PY{p}{\PYZcb{}}

\PY{k}{class} \PY{n+nc}{Reaction}\PY{p}{(}\PY{n}{Concept}\PY{p}{)}\PY{p}{:}
  \PY{n}{products} \PY{o}{=} \PY{n}{List}\PY{p}{(}\PY{n}{Tuple}\PY{p}{(}\PY{n}{Int}\PY{p}{,} \PY{n}{Molecule}\PY{p}{)}\PY{p}{)}
  \PY{n}{reactants} \PY{o}{=} \PY{n}{List}\PY{p}{(}\PY{n}{Tuple}\PY{p}{(}\PY{n}{Int}\PY{p}{,} \PY{n}{Molecule}\PY{p}{)}\PY{p}{)}
\end{Verbatim}
An \texttt{Element} is modeled as just a name, since we ignore electron and nuclear structure. A \texttt{Molecule} has an attribute \texttt{formula} with a list of pairs of element with a number indicating the number of atoms of that element. A \texttt{Reaction} has \texttt{reactants} and \texttt{products}, each some quantity of a certain molecule. This Python model is visualized by PySTEMM in Figure \DUrole{ref}{reactiontypes}.

Note that convenient Python constructs, like \emph{lists} of \emph{tuples}, are visualized in a similarly convenient manner. Also, the \texttt{instance\_template} for molecule (lines 6-7), specifying the visualization properties for a molecule instance, contains a \emph{function} which takes a molecule instance and computes its label. Visualization templates are parameterized by the objects they will be applied to.

Figure \DUrole{ref}{reactioninstance} shows an instance of a reaction, showing reaction structure and computed labels for reactions and molecules, while hiding the \texttt{formula} structure within molecules.

\subsection{Reaction Balancing\label{reaction-balancing}}
\begin{figure}[]\noindent\makebox[\columnwidth][c]{\includegraphics[width=\columnwidth]{reaction_balance.pdf}}
\caption{\texttt{Reaction} balance matrix and solved coefficients. \DUrole{label}{balancing}}
\end{figure}

Our next model computes reaction balancing for reactions. An unbalanced reaction has lists \texttt{ins} and \texttt{outs} of  molecules without coefficients. Figure \DUrole{ref}{balancing} shows how PySTEMM visualizes a reaction with the \texttt{balance} computation, coefficients, and intermediate values, as explained below.





We formulate reaction-balancing as an \emph{integer-linear programming} problem \cite{Sen06}, which we solve for molecule coefficients. The \texttt{formula} of the  molecules constrain the coefficients, since atoms of every element must balance. The function \texttt{elem\_balance\_matrix} computes a matrix of \emph{molecule} vs. \emph{element}, with the number of atoms of each element in each molecule, with \texttt{+} for \texttt{ins} and \texttt{-} for \texttt{outs}. This matrix multiplied by the vector of coefficients must result in all \texttt{0}. All coefficients have to be positive integers (\texttt{diagonal\_matrix}), and the \texttt{objective\_function} seeks the smallest coefficients  satisfying these constraints.

Once we have balanced reactions, we can add attributes and functions to model reaction stoichiometry and thermodynamics. For example:\begin{Verbatim}[commandchars=\\\{\},fontsize=\footnotesize]
\PY{k}{class} \PY{n+nc}{Element}\PY{p}{(}\PY{n}{Concept}\PY{p}{)}\PY{p}{:}
  \PY{n}{name} \PY{o}{=} \PY{n}{String}
  \PY{n}{atomic\PYZus{}mass} \PY{o}{=} \PY{n}{Float}

\PY{k}{class} \PY{n+nc}{Molecule}\PY{p}{(}\PY{n}{Concept}\PY{p}{)}\PY{p}{:}
  \PY{n}{formula} \PY{o}{=} \PY{n}{List}\PY{p}{(}\PY{n}{Tuple}\PY{p}{(}\PY{n}{Element}\PY{p}{,} \PY{n}{Int}\PY{p}{)}\PY{p}{)}
  \PY{n}{molar\PYZus{}mass} \PY{o}{=} \PY{n}{Property}\PY{p}{(}\PY{n}{Float}\PY{p}{)}
  \PY{k}{def} \PY{n+nf}{\PYZus{}get\PYZus{}molar\PYZus{}mass}\PY{p}{(}\PY{n+nb+bp}{self}\PY{p}{)}\PY{p}{:}
    \PY{k}{return} \PY{n+nb}{sum}\PY{p}{(}\PY{p}{[}\PY{n}{n} \PY{o}{*} \PY{n}{el}\PY{o}{.}\PY{n}{atomic\PYZus{}mass}
                  \PY{k}{for} \PY{n}{el}\PY{p}{,} \PY{n}{n} \PY{o+ow}{in} \PY{n+nb+bp}{self}\PY{o}{.}\PY{n}{formula}\PY{p}{]}\PY{p}{)}

\PY{n}{Fe} \PY{o}{=} \PY{n}{Element}\PY{p}{(}\PY{n}{name}\PY{o}{=}\PY{l+s}{\PYZsq{}}\PY{l+s}{Fe}\PY{l+s}{\PYZsq{}}\PY{p}{,} \PY{n}{atomic\PYZus{}mass}\PY{o}{=}\PY{l+m+mi}{56}\PY{p}{)}
\PY{n}{Cl} \PY{o}{=} \PY{n}{Element}\PY{p}{(}\PY{n}{name}\PY{o}{=}\PY{l+s}{\PYZsq{}}\PY{l+s}{Cl}\PY{l+s}{\PYZsq{}}\PY{p}{,} \PY{n}{atomic\PYZus{}mass}\PY{o}{=}\PY{l+m+mf}{35.5}\PY{p}{)}
\PY{n}{FeCl2} \PY{o}{=} \PY{n}{Molecule}\PY{p}{(}\PY{n}{formula}\PY{o}{=}\PY{p}{[}\PY{p}{(}\PY{n}{Fe}\PY{p}{,}\PY{l+m+mi}{1}\PY{p}{)}\PY{p}{,} \PY{p}{(}\PY{n}{Cl}\PY{p}{,}\PY{l+m+mi}{2}\PY{p}{)}\PY{p}{]}\PY{p}{)}

\PY{n}{FeCl2}\PY{o}{.}\PY{n}{molar\PYZus{}mass} \PY{c}{\PYZsh{} = 127}
\end{Verbatim}




\subsection{Reaction Network\label{reaction-network}}
\begin{Verbatim}[commandchars=\\\{\},fontsize=\footnotesize]
\PY{k}{class} \PY{n+nc}{Network}\PY{p}{(}\PY{n}{Concept}\PY{p}{)}\PY{p}{:}
  \PY{n}{reactions} \PY{o}{=} \PY{n}{List}\PY{p}{(}\PY{n}{Reaction}\PY{p}{)}

\PY{n}{R1} \PY{o}{=} \PY{n}{Reaction}\PY{p}{(}\PY{n}{reactants}\PY{o}{=}\PY{p}{[}\PY{p}{(}\PY{l+m+mi}{2}\PY{p}{,} \PY{n}{NO2}\PY{p}{)}\PY{p}{]}\PY{p}{,}
              \PY{n}{products}\PY{o}{=}\PY{p}{[}\PY{p}{(}\PY{l+m+mi}{1}\PY{p}{,} \PY{n}{NO3}\PY{p}{)}\PY{p}{,} \PY{p}{(}\PY{l+m+mi}{1}\PY{p}{,} \PY{n}{NO}\PY{p}{)}\PY{p}{]}\PY{p}{)}

\PY{n}{R2} \PY{o}{=} \PY{n}{Reaction}\PY{p}{(}\PY{n}{reactants}\PY{o}{=}\PY{p}{[}\PY{p}{(}\PY{l+m+mi}{1}\PY{p}{,} \PY{n}{NO3}\PY{p}{)}\PY{p}{,} \PY{p}{(}\PY{l+m+mi}{1}\PY{p}{,} \PY{n}{CO}\PY{p}{)}\PY{p}{]}\PY{p}{,}
              \PY{n}{products}\PY{o}{=}\PY{p}{[}\PY{p}{(}\PY{l+m+mi}{1}\PY{p}{,} \PY{n}{NO2}\PY{p}{)}\PY{p}{,} \PY{p}{(}\PY{l+m+mi}{1}\PY{p}{,} \PY{n}{CO2}\PY{p}{)}\PY{p}{]}\PY{p}{)}

\PY{n}{Net} \PY{o}{=} \PY{n}{Network}\PY{p}{(}\PY{n}{reactions}\PY{o}{=}\PY{p}{[}\PY{n}{R1}\PY{p}{,} \PY{n}{R2}\PY{p}{]}\PY{p}{)}
\end{Verbatim}
\begin{figure}[]\noindent\makebox[\columnwidth][c]{\includegraphics[width=\columnwidth]{reaction_network.pdf}}
\caption{A reaction \texttt{Network} with two reactions. \DUrole{label}{network}}
\end{figure}A \texttt{Network} of coupled chemical reactions has a list of \texttt{reactions}. Given this Python model, and a narrative template for \texttt{Reaction}, PySTEMM generates Figure \DUrole{ref}{network}, including the \emph{instance-level} English narrative. Just as there are element balance constraints on an individual reaction, we could model network-level constraints on the reaction rates and concentrations of chemical species, but have not shown this here.

\subsection{Layered Models\label{layered-models}}
\begin{figure}[]\noindent\makebox[\columnwidth][c]{\includegraphics[scale=0.65]{concept_to_math.pdf}}
\caption{Layered concept models and generated math.}
\end{figure}

The reaction examples illustrate an important advantage of PySTEMM  modeling; instead of directly modeling the mathematics of reaction, we focus on the structure of the concept instances; in this case, what constitutes a molecule, or a reaction?

From this model, we compute the math model. The math version of a molecule is a single column with the number of atoms of each element type in that molecule. The math for a reaction collects this column from each molecule and combines them into an \texttt{element\_balance\_matrix}. Pure functions thus  easily traverse the concept instances to build corresponding math models such as matrices of numbers.

\section{Physics\label{physics}}
\begin{figure*}[]\noindent\makebox[\textwidth][c]{\includegraphics[scale=0.40]{physics_graph_n_animation.pdf}}
\caption{\texttt{Ball} in motion: functions of time as code, graphs, animation \DUrole{label}{phyfig}}
\end{figure*}

Below is a model of the motion of a ball under constant force. The ball has vector-valued attributes for initial position, velocity, and forces (lines 2,3). The functions \texttt{acceleration}, \texttt{velocity}, and \texttt{position} are pure functions of time and use numerical integration. We visualize ball \texttt{b} via \texttt{showGraph} and \texttt{animate} (lines 18-19). Like all visualizations, the animation is specified by a \emph{template} (line 21): a dictionary of visual properties, except that these properties can be \emph{functions} of the \emph{object} being animated, and the \emph{time} at which its attributes values are computed.\begin{Verbatim}[commandchars=\\\{\},numbers=left,firstnumber=1,stepnumber=1,fontsize=\footnotesize,xleftmargin=2.25mm,numbersep=3pt]
\PY{k}{class} \PY{n+nc}{Ball}\PY{p}{(}\PY{n}{Concept}\PY{p}{)}\PY{p}{:}
  \PY{n}{mass}\PY{p}{,} \PY{n}{p0}\PY{p}{,} \PY{n}{v0} \PY{o}{=} \PY{n}{Float}\PY{p}{,} \PY{n}{Instance}\PY{p}{(}\PY{n}{vector}\PY{p}{)}\PY{p}{,} \PY{o}{.}\PY{o}{.}\PY{o}{.}
  \PY{n}{forces} \PY{o}{=} \PY{n}{List}\PY{p}{(}\PY{n}{vector}\PY{p}{)}
  \PY{k}{def} \PY{n+nf}{net\PYZus{}force}\PY{p}{(}\PY{n+nb+bp}{self}\PY{p}{)}\PY{p}{:}
    \PY{k}{return} \PY{n}{v\PYZus{}sum}\PY{p}{(}\PY{n+nb+bp}{self}\PY{o}{.}\PY{n}{forces}\PY{p}{)}
  \PY{k}{def} \PY{n+nf}{acceleration}\PY{p}{(}\PY{n+nb+bp}{self}\PY{p}{,} \PY{n}{time}\PY{p}{)}\PY{p}{:}
    \PY{k}{return} \PY{n+nb+bp}{self}\PY{o}{.}\PY{n}{net\PYZus{}force}\PY{p}{(}\PY{p}{)} \PY{o}{/} \PY{n+nb+bp}{self}\PY{o}{.}\PY{n}{mass}
  \PY{k}{def} \PY{n+nf}{velocity}\PY{p}{(}\PY{n+nb+bp}{self}\PY{p}{,} \PY{n}{time}\PY{p}{)}\PY{p}{:}
    \PY{k}{return} \PY{n+nb+bp}{self}\PY{o}{.}\PY{n}{v0} \PY{o}{+} \PY{n}{v\PYZus{}integrate}\PY{p}{(}\PY{n+nb+bp}{self}\PY{o}{.}\PY{n}{acceleration}\PY{p}{,} \PY{n}{time}\PY{p}{)}
  \PY{k}{def} \PY{n+nf}{position}\PY{p}{(}\PY{n+nb+bp}{self}\PY{p}{,} \PY{n}{time}\PY{p}{)}\PY{p}{:}
    \PY{k}{return} \PY{n+nb+bp}{self}\PY{o}{.}\PY{n}{p0} \PY{o}{+} \PY{n}{v\PYZus{}integrate}\PY{p}{(}\PY{n+nb+bp}{self}\PY{o}{.}\PY{n}{velocity}\PY{p}{,} \PY{n}{time}\PY{p}{)}

  \PY{k}{def} \PY{n+nf}{p\PYZus{}x}\PY{p}{(}\PY{n+nb+bp}{self}\PY{p}{,} \PY{n}{time}\PY{p}{)}\PY{p}{:} \PY{o}{.}\PY{o}{.}\PY{o}{.}\PY{o}{.}
  \PY{k}{def} \PY{n+nf}{p\PYZus{}y}\PY{p}{(}\PY{n+nb+bp}{self}\PY{p}{,} \PY{n}{time}\PY{p}{)}\PY{p}{:} \PY{o}{.}\PY{o}{.}\PY{o}{.}\PY{o}{.}

\PY{n}{b} \PY{o}{=} \PY{n}{Ball}\PY{p}{(}\PY{n}{p0}\PY{o}{=}\PY{o}{.}\PY{o}{.}\PY{o}{.}\PY{p}{,} \PY{n}{v0}\PY{o}{=}\PY{o}{.}\PY{o}{.}\PY{o}{.}\PY{p}{,} \PY{n}{mass}\PY{o}{=}\PY{o}{.}\PY{o}{.}\PY{o}{.}\PY{p}{,} \PY{n}{forces}\PY{o}{=}\PY{o}{.}\PY{o}{.}\PY{o}{.}\PY{p}{)}
\PY{n}{m} \PY{o}{=} \PY{n}{Model}\PY{p}{(}\PY{n}{b}\PY{p}{)}
\PY{n}{m}\PY{o}{.}\PY{n}{showGraph}\PY{p}{(}\PY{n}{b}\PY{p}{,} \PY{p}{(}\PY{l+s}{\PYZsq{}}\PY{l+s}{a\PYZus{}y}\PY{l+s}{\PYZsq{}}\PY{p}{,}\PY{l+s}{\PYZsq{}}\PY{l+s}{v\PYZus{}y}\PY{l+s}{\PYZsq{}}\PY{p}{,}\PY{l+s}{\PYZsq{}}\PY{l+s}{p\PYZus{}y}\PY{l+s}{\PYZsq{}}\PY{p}{)}\PY{p}{,} \PY{p}{(}\PY{l+m+mi}{0}\PY{p}{,}\PY{l+m+mi}{10}\PY{p}{)}\PY{p}{)}
\PY{n}{m}\PY{o}{.}\PY{n}{animate}\PY{p}{(}\PY{n}{b}\PY{p}{,}
    \PY{p}{(}\PY{l+m+mi}{0}\PY{p}{,}\PY{l+m+mi}{10}\PY{p}{)}\PY{p}{,}
    \PY{p}{[}\PY{p}{\PYZob{}}\PY{n}{K}\PY{o}{.}\PY{n}{new}\PY{p}{:} \PY{n}{K}\PY{o}{.}\PY{n}{shape}\PY{p}{,}
      \PY{n}{K}\PY{o}{.}\PY{n}{origin}\PY{p}{:} \PY{k}{lambda} \PY{n}{b}\PY{p}{,}\PY{n}{t}\PY{p}{:} \PY{p}{[}\PY{n}{b}\PY{o}{.}\PY{n}{p\PYZus{}x}\PY{p}{(}\PY{n}{t}\PY{p}{)}\PY{p}{,} \PY{n}{b}\PY{o}{.}\PY{n}{p\PYZus{}y}\PY{p}{(}\PY{n}{t}\PY{p}{)}\PY{p}{]}\PY{p}{]}\PY{p}{\PYZcb{}}\PY{p}{,}
     \PY{p}{\PYZob{}}\PY{n}{K}\PY{o}{.}\PY{n}{new}\PY{p}{:} \PY{n}{K}\PY{o}{.}\PY{n}{line}\PY{p}{,} \PY{n}{point\PYZus{}list}\PY{o}{=}\PY{k}{lambda} \PY{n}{b}\PY{p}{,}\PY{n}{t}\PY{p}{:} \PY{o}{.}\PY{o}{.}\PY{o}{.}\PY{p}{\PYZcb{}}\PY{p}{,}
     \PY{p}{\PYZob{}}\PY{n}{K}\PY{o}{.}\PY{n}{new}\PY{p}{:} \PY{n}{K}\PY{o}{.}\PY{n}{line}\PY{p}{,} \PY{n}{point\PYZus{}list}\PY{o}{=}\PY{k}{lambda} \PY{n}{b}\PY{p}{,}\PY{n}{t}\PY{p}{:} \PY{o}{.}\PY{o}{.}\PY{o}{.}\PY{p}{\PYZcb{}}\PY{p}{]} \PY{p}{)}
\end{Verbatim}
PySTEMM generates graphs of the time-varying functions, and a 2-D animation of the position and velocity vectors of the ball over time (Figure \DUrole{ref}{phyfig}).

\section{Engineering\label{engineering}}
\begin{figure}[]\noindent\makebox[\columnwidth][c]{\includegraphics[scale=0.50]{rov.pdf}}
\caption{\texttt{ROV} made of \texttt{PVCPipes}. \DUrole{label}{rovfig}}
\end{figure}

In Summer 2012 I attended the OEX program at MIT, where we designed and built a marine remote-operated vehicle (ROV) with sensors to monitor water conditions. I later used PySTEMM to recreate models of the ROV, and generate engineering attributes and 3-D visualizations like Figure \DUrole{ref}{rovfig}.

The \texttt{ROV} is built from \texttt{PVCPipes} in a functional style. To create several \texttt{PVCPipes} positioned and sized relative to each other, the model uses pure functions like \texttt{shift} and \texttt{rotate} that take a \texttt{PVCPipe} and some geometry, and produce a transformed \texttt{PVCPipe}. This makes it simple to define parametric models and rapidly try different \texttt{ROV} structures. The model shown excludes motors, micro-controller, and computed drag, net force, and torque.\begin{Verbatim}[commandchars=\\\{\},fontsize=\footnotesize]
\PY{k}{class} \PY{n+nc}{PVCPipe}\PY{p}{(}\PY{n}{Concept}\PY{p}{)}\PY{p}{:}
  \PY{n}{length}\PY{p}{,} \PY{n}{radius}\PY{p}{,} \PY{n}{density} \PY{o}{=} \PY{n}{Float}\PY{p}{,} \PY{n}{Float}\PY{p}{,} \PY{n}{Float}
  \PY{k}{def} \PY{n+nf}{shift}\PY{p}{(}\PY{n+nb+bp}{self}\PY{p}{,} \PY{n}{v}\PY{p}{)}\PY{p}{:}
    \PY{k}{return} \PY{n}{PVCPipe}\PY{p}{(}\PY{n+nb+bp}{self}\PY{o}{.}\PY{n}{p0} \PY{o}{+} \PY{n}{v}\PY{p}{,} \PY{n+nb+bp}{self}\PY{o}{.}\PY{n}{r}\PY{p}{,} \PY{n+nb+bp}{self}\PY{o}{.}\PY{n}{axis}\PY{p}{)}
  \PY{k}{def} \PY{n+nf}{rotate}\PY{p}{(}\PY{n+nb+bp}{self}\PY{p}{,} \PY{n}{a}\PY{p}{)}\PY{p}{:}
    \PY{k}{return} \PY{n}{PVCPipe}\PY{p}{(}\PY{n+nb+bp}{self}\PY{o}{.}\PY{n}{p0}\PY{p}{,} \PY{n+nb+bp}{self}\PY{o}{.}\PY{n}{r}\PY{p}{,} \PY{n+nb+bp}{self}\PY{o}{.}\PY{n}{axis} \PY{o}{+} \PY{n}{a}\PY{p}{)}

\PY{k}{class} \PY{n+nc}{ROV}\PY{p}{(}\PY{n}{Concept}\PY{p}{)}\PY{p}{:}
  \PY{n}{body} \PY{o}{=} \PY{n}{List}\PY{p}{(}\PY{n}{PVCPipe}\PY{p}{)}
  \PY{k}{def} \PY{n+nf}{mass}\PY{p}{(}\PY{n+nb+bp}{self}\PY{p}{)}\PY{p}{:} \PY{o}{.}\PY{o}{.}\PY{o}{.}
  \PY{k}{def} \PY{n+nf}{center\PYZus{}of\PYZus{}mass}\PY{p}{(}\PY{n+nb+bp}{self}\PY{p}{)}\PY{p}{:} \PY{o}{.}\PY{o}{.}\PY{o}{.}
  \PY{k}{def} \PY{n+nf}{moment\PYZus{}of\PYZus{}inertia}\PY{p}{(}\PY{n+nb+bp}{self}\PY{p}{)}\PY{p}{:} \PY{o}{.}\PY{o}{.}\PY{o}{.}

\PY{n}{p1} \PY{o}{=} \PY{n}{PVCPipe}\PY{p}{(}\PY{o}{.}\PY{o}{.}\PY{o}{.}\PY{o}{.}\PY{p}{)}
\PY{n}{p2} \PY{o}{=} \PY{n}{p1}\PY{o}{.}\PY{n}{shift}\PY{p}{(}\PY{p}{(}\PY{l+m+mi}{0}\PY{p}{,}\PY{l+m+mi}{0}\PY{p}{,}\PY{l+m+mi}{3}\PY{p}{)}\PY{p}{,} \PY{o}{.}\PY{o}{.}\PY{o}{.}\PY{p}{)}
\PY{n}{c1}\PY{p}{,} \PY{n}{c2} \PY{o}{=} \PY{n}{p1}\PY{o}{.}\PY{n}{rotate}\PY{p}{(}\PY{p}{(}\PY{l+m+mi}{0}\PY{p}{,}\PY{l+m+mi}{0}\PY{p}{,}\PY{l+m+mi}{90}\PY{p}{)}\PY{p}{)}\PY{o}{.}\PY{o}{.}\PY{o}{.}
\PY{n}{rov} \PY{o}{=} \PY{n}{ROV}\PY{p}{(}\PY{n}{body}\PY{o}{=}\PY{n}{p1}\PY{p}{,} \PY{n}{p2}\PY{p}{,} \PY{n}{c1}\PY{p}{,} \PY{n}{c2}\PY{p}{)}
\end{Verbatim}








\section{Implementation\label{implementation}}


\subsection{Architecture\label{architecture}}
The overall architecture of PySTEMM, illustrated in Figure \DUrole{ref}{archfig}, has two main parts: \emph{Tool} and \emph{Model Library}. The \emph{tool} manipulates \emph{models}, traversing them at the type and instance level and generating visualizations. The \emph{model library} includes the models presented in this paper and any additional models any PySTEMM user would create. The \emph{tool} is implemented with 3 classes:\begin{itemize}

\item 

\texttt{Concept}: a superclass that triggers special handling of the concept type to process attribute-type definitions.
\item 

\texttt{Model}: a collection of concepts classes and concept instances, configured with some visualization.
\item 

\texttt{View}: an interface to a drawing application scripted via AppleScript.
\end{itemize}


Figure \DUrole{ref}{archfig} explains the architecture in more detail, and lists external modules that were used for specific purposes. PySTEMM uses the Enthought \texttt{traits} module \cite{Tra14} to define attribute types for a concept. Traits provides a class \texttt{HasTraits} with a custom meta-class, and pre-defined traits such as \texttt{List}, \texttt{Tuple}, \texttt{String}, and \texttt{Int}. The \texttt{Concept} class derives from \texttt{HasTraits}, which triggers \texttt{traits} to capture concept attribute type definitions and generate constructor and attribute logic for checked attribute assignment.
\begin{figure*}[]\noindent\makebox[\textwidth][c]{\includegraphics[scale=0.40]{architecture.pdf}}
\caption{Architecture of PySTEMM. \DUrole{label}{archfig}}
\end{figure*}

We gain several benefits by building models with immutable objects and pure functions:\begin{itemize}

\item 

The \emph{user models} can be manipulated by the \emph{tool} more easily to provide tool capabilities like animation and graph-plotting, based on evaluating pure functions at different points in time.
\item 

The values of computed attributes and other intermediate values can be visualized as easily and unambiguously as any stored attributes.
\item 

Debugging becomes much less of an issue since values do not change while executing a model, and the definitions parallel the math taught in school science.
\end{itemize}


The source code for PySTEMM is available at \url{https://github.com/kdz/pystemm}.



\subsection{Python\label{python}}


Python provides many advantages to this project:\begin{itemize}

\item 

adequate support for high-order functions and functional programming;
\item 

lightweight and flexible syntax, with convenient modeling constructs like lists, tuples, and dictionaries;
\item 

good facilities to manipulate classes, methods, and source code;
\item 

vast ecosystem of open-source libraries, including excellent ones for scientific computing.
\end{itemize}


\subsection{Templates\label{templates}}


All visualization is defined by \emph{templates} containing visual property values, or functions to compute those values:\begin{Verbatim}[commandchars=\\\{\},fontsize=\footnotesize]
\PY{n}{Concept\PYZus{}Template} \PY{o}{=} \PY{p}{\PYZob{}}
  \PY{n}{K}\PY{o}{.}\PY{n}{text}\PY{p}{:} \PY{k}{lambda} \PY{n}{concept}\PY{p}{:} \PY{n}{computeClassLabel}\PY{p}{(}\PY{n}{concept}\PY{p}{)}\PY{p}{,}
  \PY{n}{K}\PY{o}{.}\PY{n}{name}\PY{p}{:} \PY{l+s}{\PYZsq{}}\PY{l+s}{Rectangle}\PY{l+s}{\PYZsq{}}\PY{p}{,}
  \PY{n}{K}\PY{o}{.}\PY{n}{corner\PYZus{}radius}\PY{p}{:} \PY{l+m+mi}{6}\PY{p}{,}
  \PY{o}{.}\PY{o}{.}\PY{o}{.}
  \PY{n}{K}\PY{o}{.}\PY{n}{gradient\PYZus{}color}\PY{p}{:} \PY{l+s}{\PYZdq{}}\PY{l+s}{Snow}\PY{l+s}{\PYZdq{}}\PY{p}{\PYZcb{}}
\end{Verbatim}
The primary operation on a template is to \emph{apply} it to some modeling object, typically a concept class or instance:\begin{Verbatim}[commandchars=\\\{\},fontsize=\footnotesize]
\PY{k}{def} \PY{n+nf}{apply\PYZus{}template}\PY{p}{(}\PY{n}{t}\PY{p}{,} \PY{n}{obj}\PY{p}{,} \PY{n}{time}\PY{o}{=}\PY{n+nb+bp}{None}\PY{p}{)}\PY{p}{:}
  \PY{c}{\PYZsh{} t.values are drawing\PYZhy{}app values, or functions}
  \PY{c}{\PYZsh{} obj: any object, passed into template functions}
  \PY{c}{\PYZsh{} returns copy of t, F(obj) replaces functions F}
  \PY{k}{if} \PY{n+nb}{isinstance}\PY{p}{(}\PY{n}{t}\PY{p}{,} \PY{n+nb}{dict}\PY{p}{)}\PY{p}{:}
    \PY{k}{return} \PY{p}{\PYZob{}}\PY{n}{k}\PY{p}{:} \PY{n}{apply\PYZus{}template}\PY{p}{(}\PY{n}{v}\PY{p}{,} \PY{n}{obj}\PY{p}{,} \PY{n}{time}\PY{p}{)}
               \PY{k}{for} \PY{n}{k}\PY{p}{,} \PY{n}{v} \PY{o+ow}{in} \PY{n}{t}\PY{o}{.}\PY{n}{items}\PY{p}{(}\PY{p}{)}\PY{p}{\PYZcb{}}
  \PY{k}{if} \PY{n+nb}{isinstance}\PY{p}{(}\PY{n}{t}\PY{p}{,} \PY{n+nb}{list}\PY{p}{)}\PY{p}{:}
    \PY{k}{return} \PY{p}{[}\PY{n}{apply\PYZus{}template}\PY{p}{(}\PY{n}{x}\PY{p}{,} \PY{n}{obj}\PY{p}{,} \PY{n}{time}\PY{p}{)}
               \PY{k}{for} \PY{n}{x} \PY{o+ow}{in} \PY{n}{t}\PY{p}{]}
  \PY{k}{if} \PY{n+nb}{callable}\PY{p}{(}\PY{n}{t}\PY{p}{)}\PY{p}{:}
    \PY{k}{return} \PY{n}{t}\PY{p}{(}\PY{n}{obj}\PY{p}{)} \PY{k}{if} \PY{n}{arity}\PY{p}{(}\PY{n}{t}\PY{p}{)}\PY{o}{==}\PY{l+m+mi}{1} \PY{k}{else} \PY{n}{t}\PY{p}{(}\PY{n}{obj}\PY{p}{,} \PY{n}{time}\PY{p}{)}
  \PY{k}{return} \PY{n}{t}
\end{Verbatim}
Animation templates have special case handling, since their functions take two parameters: the \emph{instance} to be rendered, and the \emph{time} at which to render its attributes.

Templates can also be \emph{merged}. Figure \DUrole{ref}{funcinstances} shows an  instance of \texttt{TableFunction} as a circle in the same color as the \texttt{TableFunction} class, by merging an \texttt{instance\_template} with a \texttt{class\_template}.

\section{Summary\label{summary}}


I have described PySTEMM as a tool, model library, and approach for building executable concept models for a variety of STEM subjects. The PySTEMM approach, using immutable objects and pure functions in Python, can apply to all STEM areas. It supports learning through pictures, narrative, animation, and graph plots, all generated from a single model definition, with minimal incidental complexity and code debugging issues. Such modeling, if given a more central role in K-12 STEM education, could make STEM learning much more deeply engaging.










\begin{thebibliography}{Whi93}
\bibitem[Whi93]{Whi93}{

White, Barbara Y. “ThinkerTools: Causal Models, Conceptual Change, and Science Education”, Cognition and Instruction, Vol. 10, No. 1.}
\bibitem[Orn08]{Orn08}{

Ornek, Funda. “Models in Science Education: Applications of Models in Learning and Teaching Science”, International Journal of Environmental \& Science Education, 2008.}
\bibitem[Edw04]{Edw04}{

Edwards, Jonathan. “Example Centric Programming”, The College of Information Sciences and Technology (Pennsylvania State University: 2004), \url{http://www.subtext-lang.org/OOPSLA04.pdf}}
\bibitem[Fun13]{Fun13}{

\textquotedbl{}9.8. Functools — Higher-order Functions and Operations on Callable Objects.\textquotedbl{},  Python Software Foundation, 2013. \url{http://docs.python.org/2/library/functools.html}.}
\bibitem[Bla07]{Bla07}{

Blais, Martin. “True Lieberman-style Delegation in Python.\textquotedbl{} Active State Software, 2007, \url{http://code.activestate.com/recipes/519639-true-lieberman-style-delegation-in-python/}.}
\bibitem[Sen06]{Sen06}{

Sen, S. K., Hans Agarwal, and Sagar Sen. “Chemical Equation Balancing: An Integer Programming Approach”, Mathematical and Computer Modeling, Vol. 44, No.7-8, 2006.}
\bibitem[Tra14]{Tra14}{

Enthought Traits Library, \url{http://code.enthought.com/projects/traits/}}
\end{thebibliography}

\end{document}
