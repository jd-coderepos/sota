\documentclass[manuscript,nonacm]{acmart} 

\usepackage{soulutf8}
\usepackage{subcaption}
\usepackage{multirow}






\begin{document}

\title{Weighted Multi-Level Feature Factorization for App ads CTR and installation prediction }

\author{Juan Manuel Rodriguez}
\email{juanmanuel.rodriguez@isistan.unicen.edu.ar}
\affiliation{\institution{Aarhus University}
\country{Denmark}
}\affiliation{\institution{ISISTAN, CONICET-UNCPBA}
\country{Argentina}
}


\author{Antonela Tommasel}
\email{antonela.tommasel@isistan.unicen.edu.ar}
\affiliation{\institution{ISISTAN, CONICET-UNCPBA}
\country{Argentina}
}

\begin{abstract}
This paper provides an overview of the approach we used as team ISISTANITOS for the ACM RecSys Challenge 2023. 
The competition was organized by ShareChat, and involved predicting the probability of a user clicking an app ad and/or installing an app, to improve deep funnel optimization and a special focus on user privacy.
Our proposed method inferring the probabilities of clicking and installing as two different, but related tasks. 
Hence, the model engineers a specific set of features for each task and a set of shared features. 
Our model is called Weighted Multi-Level Feature Factorization because it considers the interaction of different order features, where the order is associated to the depth in a neural network. The prediction for a given task is generated by combining the task specific and shared features on the different levels.  
Our submission achieved the 11 rank and overall score of 55 in the competition academia-track final results.
We release our source code at: \url{https://github.com/knife982000/RecSys2023Challenge}

\end{abstract}

\maketitle

\section{Introduction}

Recommender systems have emerged as influential tools across various platforms, aiding users in discovering interesting items and ultimately influencing their consumption patterns. Since the early 2000s, online advertising has developed into a thriving industry worth billions of dollars, playing a significant role in Internet’s growth and offering a common marketing experience to people accessing online services \cite{gharibshah2021user}. The main difference between online and traditional mass advertising is its inherent ability to tailor messages to individual users, democratizing advertising and enabling businesses of all sizes to participate. Accurately predicting clicks, as the first measurable user response, is a crucial step for many digital advertising and recommenders to capture the user likelihood of users taking subsequent actions, such as purchasing a product, subscribing to a service, or, as in this case, installing an app \cite{gharibshah2021user}.

The ACM Recsys Challenge 2023\footnote{\url{http://www.recsyschallenge.com/2023/}} was organized by ShareChat, and involved predicting the probability of a user installing an app, intending to improve deep funnel optimization and a special focus on user privacy. To this end, we proposed a model that generates prediction using different order features~\cite{wdlrs16, deepfm17} and 
 a weighted linear combination, and it is called Weighted Multi-Level Feature Factorization.







\section{Problem formulation}\label{sec:problem}

The ACM RecSys Challenge 2023, organized by ShareChat\footnote{\url{ https://sharechat.com/about}} aims to predict whether a user would install an advertised app on the next day, given user and ad features and data from impressions, clicks and installs from the past 2 weeks. 
The task is proposed in the context of deep funnel optimization with a special interest in preserving users’ privacy. 

\subsubsection*{\textbf{Data collection.}}\hfill

The data collection includes information for approximately 10 million random users who visited the ShareChat and Moj apps for over 3 months. For each user, there are 10 ad impressions available. The collection includes features user, ads, and user-ad interaction features. \textit{User features.} include i) demographic features, like age, gender, and anonymized user location. User sampling guarantees an approximate uniform distribution of these demographic features. ii) content preference embeddings trained based on users’ consumption of the non-ad content of the ShareChat and Moj apps. iii) embeddings trained based on the past apps installed by the user through the platform.
\textit{Ad features} include i) categorical anonymized features representing, for example size and category. ii) emdeddings representing ad's  actual content (video/image). Finally, \textit{historical interactions} include count features representing the user interaction with ads, advertisers and categories of advertisers over different lengths of a time window







\smallskip
Then, each data instance is composed by 82 features:  instance id,  date of the impression,  to  categorical features,  to  binary features,  to  numeric features, and  and  binary labels. The challenge presents two different, yet related tasks, inferring the  and  labels. Regarding class distribution, 21.98\% instances were clicked, while 17.40\% where installed and only 7.11\% were both clicked and installed.
These tasks, complex by themselves, present some additional challenges: i) the lack of semantic information of the individual features; ii) the limited number of interactions known for each user; 
and iii) there is no distinction of which features represent the user and which ones the ad. 

\subsubsection*{\textbf{Evaluation goal.}}\hfill

Predictions are used to generate rankings that measure the expected revenue for the platform and then the winner ad is displayed to the user. Recommendations were evaluated by means of normalized Cross Entropy.

\section{Data engineering}\label{sec:data-eng}
Different pre-processing strategies were applied to the features based on their type. After processing, each instance was represented with two feature vectors, one for the categorical () and one for the numeric () features. Impression date () was ignored as its value is constant in the test set and such value is not present in the training set.

\subsubsection*{\textbf{Categorical features}}\hfill 

Two types of transformations were applied based on the number of categories. For features with over 20 categories, an ordinal encoding process was followed, assigning items IDs from 0 to  to facilitate the feature processing by the neural network. Null and categories not existing in the training set were assigned 
the ID of the most frequent category. Although this strategy might affect the relation of the most frequent categories with the others, less than 0.74\% were affected in the worst case, which might not be enough to cause any negative effect during training or inference.

Categorical features with fewer than 20 categories were transformed into two numeric features, which were defined as the probability of being clicked or being installed given the value of the feature. Probabilities were estimated by computing the percentage of ad clicks and app installs in the training dataset for each value. For example, Eq.~\ref{eq:TC} and Eq.~\ref{eq:TI} show the transformation for . Similar to the previous case, categories in the test set that were not present in the training were replaced by the most frequent category.\vspace{-0.75cm}





\begin{table}[H]
    \centering
   
\begin{tabular}{cc}

 
& 


    \end{tabular}\vspace{-0.5cm}
   
\end{table}

\subsubsection*{\textbf{Numerical features}}\hfill 

Missing numeric features ( to ) were first imputed with their average value. Then, they were standardized according to , where  is the average feature value,  is the standard deviation and  is a hyperparameter set to 3. This value was chosen so, in the case of a normal distribution,  of values would be at most  of , i.e., those values would be between -1 and 1.

Considering that most features contained extreme values and were also highly sparse (for  of numeric features over  of the values were zero), only applying the first transformation caused features to have extreme values that might be difficult to process by ML models. For example, while feature  contained  zeros in the training set with a mean of 0.889 ( 43.27) and  lower than , its highest value was 16,570.97. To handle this case, we transformed the features as Eq. \ref{eq:L2} shows (where  represents ) to obtain a continuous function that it is linear when  is in the range  and logarithmic outside of that range. As an example, Figure \ref{fig:feature-transformation} shows the changes in minimum and maximum values of  after applying the described transformations.\vspace{-0.3cm}



\vspace{-0.95cm}

\begin{figure*}
    \centering

    \subfloat[Original feature distribution\label{fig:original_f60}]{\includegraphics[width=0.30\textwidth]{Original_f_60.pdf}}~
    \subfloat[Transformed feature distribution\label{fig:transformed_f60}]{\includegraphics[width=0.30\textwidth]{Transformed_f_60.pdf}}\vspace{-0.2cm}

    \caption{Kernel Density Estimation of feature  before and after transformation}
    \label{fig:feature-transformation}\vspace{-0.5cm}
\end{figure*}

\section{Method}\label{sec:model}

The overall architecture of the proposed model is schematized in Figure~\ref{fig:model}. Our model uses an operation similar to matrix factorization to compute feature interactions. Instead of directly estimating the matrices involved, sequences of linear layers compute different order features interactions by estimating one feature vector per linear layer. Then, several predictions are generated by computing the dot product between different level vectors and then weighed and added to generate a final prediction. This solves the drawback of matrix factorization requiring specific data input, making it unsuitable for general purpose prediction tasks~\cite{fm2010}. The model takes inspiration on Wide and Deep~\cite{wdlrs16} and DeepFM~\cite{deepfm17} architectures, which recognized the importance of both capturing low-order interaction between features (i.e., Wide or Factorization Machines~(FM) part) and learning high-order interactions (i.e., Deep learning part).

\begin{figure*}
     \centering
    \includegraphics[width=0.98\textwidth]{model}\vspace{-0.2cm}
    \caption{Schematic diagram of the proposed model}
    \label{fig:model}\vspace{-0.54cm}
\end{figure*}

The model is divided into 4 components: \textit{embedded}, \textit{click}, \textit{shared} and \textit{installed} \textit{features}. \textit{Embedded features} is responsible for transforming the categorical categories into a dense representation. After transformations, each categorical feature is embedded, and the resulting embeddings are averaged to obtain the final categorical feature vector. Finally, the obtained vector is concatenated with the numeric features vector to obtain the dense instance representation. This vector is fed to the \textit{click}, \textit{shared} and \textit{installed Features} components. These components share the same architecture and are defined as a sequence of dense and noise layers and an activation function. The noise layer is only applied during the training phase and is defined as Eq.~\ref{eq:noise} where  represents a sampling over a normal distribution with , and a standard deviation . This noise ensures that the mean absolute percentage error (MAPE) between the feature and the noise feature remains constant independently of the scale of the feature. Moreover, it acts as a regularizer, as the noise values affecting features with large absolute values are larger than those for features with small absolute values. Based on the dense instance representation, these components aim at generating the most relevant features for the prediction. Each component returns 3 vectors, one for each linear layer, representing features at different depths, namely  where  is the type of feature and the sub-index is the layer depth that generated the feature. Using \textit{click features} and \textit{shared features}, our model infers the probability of ; while \textit{installed features} and \textit{shared features} are used to infer the probability of . The goal of the \textit{shared features} component serves to represent the similarities between both tasks, and to limit the number of parameters, reducing the variance of the model.\vspace{-0.3cm} 



After generating the features, the model combines the feature vectors of each level using a dot product (similar to a matrix factorization operation), weights predictions with a learned parameter, and scales predictions using a shared parameter for all predictions. Our model defines the \textit{Weighted Multi Level Feature Factorization}~(WMLFF) ad-hoc layer to perform this operation. WMLFF is defined in Eq.~\ref{eq:WMLFF}, where  is a scalar parameter shared by all WMLFF layers and  is a learned weight parameter for that level and the WMLFF layer. Finally, the output of these layers is passed through a sigmoid function to estimate the clicked/installed probability. In brief, Eq.~\ref{eq:is_click} defines the inference for  label, and Eq.~\ref{eq:is_installed} defines the inference for  label.\vspace{-0.3cm}




\begin{table}[H]
    \centering
   
\begin{tabular}{cc}

\multirow{2}{0.5\columnwidth}{   } 

&  \tabularnewline
& 

    \end{tabular}\vspace{-0.5cm}
   
\end{table}

\subsubsection*{\textbf{Model training}}\hfill 

A binary cross-entropy loss function (Eq.~\ref{eq:loss}) reduced by mean is used, where  is the number of predictions in the batch,  and  represent the ground truth and predictions for  and  and  represent the ground truth for .\vspace{-0.3cm}





\subsubsection*{\textbf{Implementation details}}\hfill 

The model was implemented in PyTorch \footnote{\url{https://pytorch.org}}. The optimizer was RAdam\footnote{\url{https://pytorch.org/docs/stable/generated/torch.optim.RAdam.html}} using the default configuration. All embedding and linear layers output sizes were set to 32. Batch size was set to 1024. The learning process ran for 40 epochs.

\section{Experimental evaluation}\label{sec:evaluation}

Table~\ref{tab:results} presents the results obtained for the proposed model. These results are 5.93\% lower than the best submission for the best submission of the academic leaderboard. Aiming at improving the obtained results, we also report different variants of the proposed model that alter its architecture or hyperparameters: 

\noindent\textbullet~\textit{Original with }. As noise acts as a regularizer, the goal of this layer was to reduce over-fitting, however, as noise increases, optimizing the network might be hindered. This variant reduced the standard deviation of the  distribution set for the noise layer from  to . As noise was reduced, the result worsened, which might indicate over-fitting.

\noindent\textbullet~\textit{Original with QHAdam\footnote{\url{https://github.com/jettify/pytorch-optimizer}} \& Original with AdamW}. These variants changed the optimizer function. In both cases, results were similar to those of the original model. As we only have access to the final results, whether differences were statistically significant is unknown. 

\noindent\textbullet~\textit{Original without shared features}. The goal of the \textit{shared features} component was to capture the similarities between the two tasks, thus reducing the number of parameters and variance of the model. Nonetheless, this simplification could negatively affect the results if no actual relation existed between the tasks. Therefore, in this variation, two feature components for the  task and two for  task were included. Results showed that having separated components decreased performance, indicating that there is a correlation between both tasks.

\noindent\textbullet~\textit{WMLFM based on cosine similarity}. This variant modified the WMLFM module to use cosine similarity instead of the dot product. This variant assumed that the likelihood of an ad being clicked, or an app being installed is influenced by the angle between the feature vectors regardless of their length, i.e., vectors whose angles are close to zero are related to the positive label, while angles close to 180 are related with negative labels. As the Table shows, this variant reduced the performance of the model, which could be related to, for example, the fact that the introduced noise has a major effect over the cosine similarity than the dot product or that cosine similarity causes gradients to be closer to zero, resulting in a vanishing gradient problem.

\noindent\textbullet~\textit{Original with k-fold}. Divided the training set into 10-folds sub-training/validating sets, then 10 different models were trained until the results not improved on the validating set. Predictions were computed as the average of all models outputs, i.e., these model act as an ensemble model. The idea is to have several models better fitted to the training set, but without over-fitting it. Results show that this approach might over-fitted the training dataset. This could be because there is a difference between the data in the train set and the test set that was not captured by the splitting or that the ensemble technique was unappropriated for the model.

\noindent\textbullet~\textit{Original with deeper feature components}. This variation defined the \textit{click}, \textit{shared} and \textit{installed feature} components with six linear layers instead of three layers. This version tests whether adding deeper feature layers improved results. Increasing the layer did not achieve better results on the training loss, or in the test results. This point out that such a deep model is hard to train, which might be results a phenomenon like banishing gradient.

\noindent\textbullet~\textit{Original with 64 dimensions}. All embedding and linear layers output sizes were set to 64. Increasing the number of dimensions allows the model to fit the training dataset better, but at the risk of over-fitting it. During training, the model got a lower loss, but results show that the model over-fitted training as it was outperformed on the test data.



\begin{table}
    \centering
    \scalebox{0.99}{
    \begin{tabular}{l c c c} 

Model & Competition Results\tabularnewline	
\toprule
Best submission company & 5.744\tabularnewline	
Best submission academia & 6.054\tabularnewline	
Original & 6.413\tabularnewline		
Original with  & 6.65\tabularnewline		
Original with QHAdam optimizer & 6.414\tabularnewline		
Original with AdamW optimizer & 6.421\tabularnewline		
Original without shared features  & 6.471\tabularnewline		
WMLFM based on cosine similarity & 6.602\tabularnewline		
Original with k-fold & 6.649\tabularnewline		
Original with deeper feature components & 6.489\tabularnewline		
Original with 64 dimensions & 6.494\tabularnewline		


    \bottomrule 
    \end{tabular}
    }
    \caption{Results obtained for the evaluated models for the ShareChat dataset.}
    \label{tab:results}\vspace{-1.0cm}
\end{table}

\section{Experimental evaluation with other data collections}\label{sec:evaluation-other}

To assess the generalizability of the proposed model, this Section introduces additional evaluations performed over other data collections from the literature.

\subsection{Criteo}

This evaluation considers the Criteo\_x1 dataset\footnote{\url{https://github.com/openbenchmark/BARS/tree/main/datasets/Criteo\#criteo_x1}}~\cite{bars_ds}, which is preprocessed and divided into training, validation and test as described in~\cite{afn_criteo}. Training, validation and test consists of 33,003,326, 8,250,124, and 4,587,167 instances, respectively. Each instance consists of 26 categorical features, 13 continuous features, and a binary label. The three partitions have roughly the same proportion of positive labels.Presented in the Criteo Display Advertising Challenge\footnote{\url{https://www.kaggle.com/c/criteo-display-ad-challenge}}, the task was defined as predicting the probability of a user clicking on ad on a given Web site. The model was modified to have only two feature sub-networks, the loss function was set to cross-entropy, and early stopping was applied.

\begin{table}
    \centering
    \scalebox{0.99}{
    \begin{tabular}{l c c c} 

Model & Test AUC & Test Cross-Entropy (LogLoss) & Best Epoch\tabularnewline
    \toprule
Original & 0.796 & 0.454 & 5\tabularnewline
Original with 64 dimensions & 0.799 & 0.451 & 5\tabularnewline
Original  & 0.8 & 0.45 & 5\tabularnewline
Original with 64 dimensions  & 0.802 & 0.448 & 3\tabularnewline
Original  & 0.802 & 0.449 & 4\tabularnewline
Original with 64 dimensions  & 0.803 & 0.448 & 3\tabularnewline
Original No Noise & 0.802 & 0.449 & 4\tabularnewline
Original with 64 dimensions No Noise & 0.804 & 0.447 & 3\tabularnewline
\midrule
DeepFM~\cite{deepfm17} & 0.801 & 0.451 & -\tabularnewline
AFN+~\cite{afn_criteo} & 0.807 & 0.445 & -\tabularnewline
FinalMLP2023~\cite{FinalMLP2023} & 0.815 & - & -\tabularnewline
    \bottomrule 
    \end{tabular}
    }
    \caption{Results obtained for the evaluated models for the Criteo dataset.}
    \label{tab:results-criteo}\vspace{-1.0cm}
\end{table}

Table~\ref{tab:results-criteo} presents the result of the experiment. The version named original uses the same hyper-parameters as the ones used for the task. For the other versions, the number of internal representation dimensions was set to 64, and the noise level was modified. This table also presents the reported results for DeepFM~\cite{deepfm17}, AFN+~\cite{afn_criteo} and FinalMLP2023~\cite{FinalMLP2023}\footnote{State-of-the-art performance according to \url{https://paperswithcode.com/sota/click-through-rate-prediction-on-criteo}. Retrieved: 11th July 2023}. Our results shows that without much tuning our model performs similarly of models from 2017-2019, but it is outperformed by an State-of-the-Art model by less than 2\%. Moreover, increasing the dimensions and removing the noise improves the results pointing out that these hyperparameters effect is tide to the dataset characteristics. 

\subsection{MovieLens 100k}

MovieLens100k dataset~\cite{movielens} is a movie 1 to 5 rating dataset containing a total of 100\,000 ratings given by 943 users on 1682 movies. Given the dataset characteristics, some modifications to the proposed model were introduced. The goal here was to estimate the rating given by a user to a movie. Then, the loss function was changed to MSE (Mean Squared Error). The target feature was transformed from the range 1-5 to 0-1 by subtracting 1 and divided by 4. Later, to compute RMSE. This transformation was not applied for the linear versions where the sigmoid function at the output of the model was removed, and thus the model can return any real value. This was done to fully treat the problem as a regression problem, as when using the sigmoid activation the model restricted the output to values between zero and one. Given the size of the data, the batch size was set to 64. The original u1\footnote{\url{https://grouplens.org/datasets/movielens/100k/}} division into 80,000 training and 20,000 test instances was used.

From the original dataset, we considered categorical and binary features. The included categorical features were: user, occupation, age, and movie, which were ordinal encoded. Binary features represented the movie genres and user gender, which seems to be related to movie rankings~\cite{gender_ml}. User gender is represented as two mutually exclusive binary features Male and Female. Binary features were considered as numeric features as the first dense layer can learn representations for features with a value of 1. Finally, 4 additional features are added to represent the average and the percent standard deviation of the user and item bias. In the case of the non-linear model the obtained average is transformed using the same transformation as the target feature. Percent standard deviation is computed as .

\begin{table}
    \centering
    \scalebox{0.99}{
    \begin{tabular}{l c} 

Model & RMSE\tabularnewline
\toprule
Original & 0.936\tabularnewline
Original - Linear & 0.939\tabularnewline
Original  & 0.963\tabularnewline
Original No Noise & 1.102\tabularnewline
Original + feature user\textsubscript(avg)  + user\textsubscript(\%std) + movie\textsubscript(avg)  + movie\textsubscript(\%std) & 0.928\tabularnewline
Original - Linear + feature user\textsubscript(avg)  + user\textsubscript(\%std) + movie\textsubscript(avg)  + movie\textsubscript(\%std) & 0.928\tabularnewline
\midrule
Factorized EAE~\cite{dmias2018} & 0.920\tabularnewline
GHRS~\cite{GHRS2022} & 0.887\tabularnewline
    \bottomrule 
    \end{tabular}
    }
    \caption{Results obtained for the evaluated models for the MovieLens100k dataset}
    \label{tab:results-movie100k}\vspace{-1.0cm}
\end{table}

Table~\ref{tab:results-movie100k} depicts the results of predicting the rating for movies. We present a variation of the model for regression with a known range of output values (original), and unbounded (linear). Results showed that high noise improves the performance supporting the idea that this hyperparameter effect is tide to the dataset. Moreover, enriching features by using statistics of the rating improves the results suggesting that this information is useful, which is a similar idea to how certain categorical features were encode for the challenge. For the sake of comparison, we included the reported results of Factorized EAE~\cite{dmias2018} and GHRS\footnote{State-of-the-Art according to \url{https://paperswithcode.com/sota/collaborative-filtering-on-movielens-100k}. Retrieved 11th July 2023}~\cite{GHRS2022}. Despite not being designed for this task, our model performs similarly to a 2018 model, and less than 5\% lower than the State-of-the-Art model.  

















\section{Conclusions}\label{sec:conclusions}

This paper presents ISISTANITOS's model for the RecSys Challenge 2023. 
Our model is based on learning sets of multi-level features that capture different order of interactions between the input features.
This is based on the concepts presented in previous works~\cite{wdlrs16, deepfm17}, adapted to the particular tasks of the challenge. Finally, we present a preliminary assessment of the concept of multi-level features applied to other recommender system benchmarks.
These results showed that our model achieves good performance on these benchmarks. Hence, more evaluations are needed to assess whether the multi-level feature concept can obtain State-of-the-Art results in different recommender system benchmarks.

\section*{Acknowledgments}
We gratefully acknowledge support from NVIDIA Corporation through an NVIDIA Academic Hardware Grant.


\bibliographystyle{ACM-Reference-Format}
\bibliography{references}

\end{document}
