\chapter{Microaggregation- and Permutation-Based Anonymisation of Movement Data}
\label{chap:7}

\emph{This chapter describes a novel distance measure between trajectories not
necessarily defined over the same time span.
By using it, two permutation-based trajectory anonymisation algorithms are proposed. Both algorithms preserve the true original locations of trajectories and provide better utility properties than previous algorithms.}

\minitoc


Various technologies such as GPS, RFID, GSM, etc., can sense and track
the whereabouts of objects (cars, parcels, people, etc.).
In addition, the current storage capacities allow
collecting such object movement data in huge spatio-temporal
databases. Analysing this kind of databases containing
the trajectories of objects
can lead to useful and previously unknown knowledge. Therefore, it is
beneficial to share and publish such databases and let the analysts derive
useful knowledge from them ---knowledge that can be applied, for example, to
intelligent transportation, traffic monitoring, urban and road planning,
supply chain management, sightseeing improvement, etc.

However, the privacy of individuals
may be affected by the publication or the outsourcing
of databases of trajectories. Several kinds of privacy threats
exist. Simple de-identification
realised by removing identifying attributes is insufficient to protect the
privacy of individuals. The biggest threat with trajectories is the
``sensitive location disclosure''. In this scenario, knowing the times at
which an individual visited a few locations can help an adversary to
identify the individual's trajectory in the published database, and
therefore learn the individual's other locations at other times.
Privacy preservation in this context means that no sensitive location
ought to be linkable to an individual.


The risk of sensitive location disclosure is also affected by how much the
adversary knows. The adversary may have access to auxiliary
information~\cite{kaplan10}, sometimes called side knowledge,
background knowledge or external knowledge. The adversary can link such
background knowledge obtained from other sources
to information in the published database.
Estimating the amount and extent of auxiliary information available to the
adversary is a challenging task.

There are quite a few differences between spatio-temporal data and
microdata, {\em i.e.} records describing individuals in a standard
database with no movement data.
One real difference becomes apparent when considering
privacy. Unfortunately, the traditional anonymisation and sanitisation
methods for microdata~\cite{fung10} cannot be directly applied to
spatio-temporal data without considerable expense in computation time and
information loss. Hence, there is a need for specific anonymisation methods
to thwart privacy attacks and therefore reduce privacy risks associated with
publishing trajectories.

Trajectories can be modeled and represented in many ways~\cite{forlizzi00}.
Without loss of generality, we
consider a trajectory to be a timestamped path in a plane. By assuming
movements on the surface of the Earth, the altitude of each location
visited by a trajectory stays implicit; it could be explicitly restored
if the need arose. More formally,
let \emph{timestamped location} be a triple  with  being a
timestamp and
 a \emph{location} in . Intuitively, the
timestamped location denotes
that at time  an object is at location .

\begin{definition}[Trajectory] \label{def:traj}
A \emph{trajectory} is an ordered set of timestamped locations

where  for all .
\end{definition}

\begin{definition}[Sub-trajectory]\label{def:subtraj}
A trajectory  is a
\emph{sub-trajectory} of  in Expression~\ref{eq:traj}, denoted
, if there exist
integers  such that  for all .
\end{definition}

Hereinafter, we will use {\em triple} as a synonym for
timestamped location.
When there is no risk of ambiguity,
we also say just ``location'' to denote a timestamped location.

We present two heuristic methods for preserving the
privacy of individuals when releasing
trajectories. Both of them exactly preserve original locations
in the sense that the anonymised trajectories contain no fake, perturbed
or generalised trajectories.
The first heuristic is based on microaggregation~\cite{domingo02} of
trajectories and permutation of locations.
Microaggregation has been successfully used in microdata
anonymisation to achieve -anonymity~\cite{samarati98,sweeney02a,domingo05}.
We use it here for trajectory -anonymity (whereby an adversary
cannot decide which of  anonymised trajectories corresponds
to an original trajectory which she partly knows),
first by grouping the trajectories into clusters of size
at least  based on their similarity and then transforming
via location permutation the
trajectories inside each cluster to preserve privacy.
The second heuristic aims no longer at trajectory -anonymity,
but at location -diversity (whereby knowing a sub-trajectory
 of a certain original trajectory  allows an adversary to discover
a location in  with probability no greater than );
this second heuristic is based on location permutation
and its strong point is that
it takes reachability constraints into account:
movement between locations must follow the edges of an underlying graph
({\em e.g.}, urban pattern) so that not all locations are reachable
from any given location.
Experimental results show that
achieving trajectory -anonymity with reachability constraints
may not be possible without discarding
a substantial fraction of locations, typically those which are rather
isolated. This is the motivation for our second heuristic: it still
considers reachability but it reduces the number of discarded
locations by replacing -anonymity at the trajectory
level by -diversity at the location level.

For clustering purposes, we propose a new distance for trajectories
which naturally considers both spatial and temporal coordinates.
Our distance is able to compare trajectories
that are not defined over the same time span, without resorting
to time generalisation.
Our distance function can compare trajectories
that are timewise overlapping only
partially or not at all. It may seem at first sight that the distance
computation is exponential in terms of all considered trajectories, but
we show that it is in fact computable in polynomial time.

We present empirical results for the two proposed heuristics
using synthetic data and also real-life data.
We theoretically and experimentally compare our first heuristic
with a recent trajectory anonymisation method
called -anonymity~\cite{abul08} also aimed at
trajectory -anonymity without reachability constraints.
Theoretical results show that the privacy
preservation of our first method is the same as that of -anonymity
but dealing with trajectories \emph{not} having the same time span. For the second heuristic involving reachability constraints, no comparable counterparts
seem to exist in the literature.





\section{Trajectory similarity measures}
\label{sec:relworkdistances}

Using microaggregation
for trajectory -anonymisation requires a distance function to measure
the similarity between trajectories.
Such a distance function must consider both space and time.
Although most spatial distances
can be extended into spatio-temporal distances by adding
a time co-ordinate to spatial points,
it is not obvious how to
balance the weight of spatial and temporal
dimensions.
Furthermore, not all similarity measures for trajectories are
suitable for comparing trajectories for anonymisation purposes.
The requirement for anonymisation is not just similarity regarding shape, but
also spatial and temporal closeness. Some typical distances for
trajectories
include the Euclidean distance, the Hausdorff distance~\cite{shonkwiler91},
the Fr\'echet distance~\cite{alt95}, the turning point distance~\cite{arkin91},
and distances based on time series~\cite{liao05}
---{\em e.g.}, dynamic time warping (DTW), short
time series (STS)--- and on edit distance~\cite{chen05}
---{\em e.g}, edit distance
with real penalty (ERP), longest common sub-sequence (LCSS), and the edit
distance on real sequences (EDR) discussed next.

The \emph{edit distance on real sequences} (EDR)~\cite{chen05}
is the number of insert, delete, or replace operations
that are needed to change one sequence into another.
If  and  are two sequences
of  and  triples, respectively, where each triple  has three
attributes -- x-position , y-position  and time
 --
the distance  is defined as

 where  and  are the first elements of a given sequence,
 is a function that returns the input sequence without the
first element, and where  if  and  are ``close'',
that is, they satisfy either
 and  for some
parameter ~\cite{chen05} or  ,
, and  for a triple of
parameters ~\cite{abul10}; otherwise, . This
definition of  means that the cost for one insert, delete, or replace
operation in EDR is 1 if  and  are not ``close''.

EDR has been used for anonymisation in~\cite{abul10}. However, the
edit distance and variations thereof
are not suitable to guide clustering
for anonymisation purposes. Indeed,
Figure~\ref{fig:edrdistances} shows trajectories with
different degrees of ``closeness'' to trajectory A,
but whose EDR distance from A is the same in all cases.
When time-stamps are considered, the situation is even worse.

In Section~\ref{sec:distance}, we define a
distance measure which is better suited for anonymisation
clustering: it can compare trajectories
defined over different time spans and
even trajectories that are time-wise non-overlapping.


\begin{figure}[!ht]
\centering



\includegraphics[width=3in]{trajectory_figures/five_trajectories}

\caption{Trajectories  are placed at varying
``closeness'' from , yet their EDR distance from 
is 3 in all cases. We assume that the first point of 
matches the first point of each of ; also,
second points are assumed to match each other, and the
same for third points.}
\label{fig:edrdistances}
\end{figure}

\section{Utility and privacy requirements} \label{sec:utilpriv}

Every trajectory anonymisation algorithm must combine utility and privacy.
However, utility and privacy are two largely antagonistic concepts. What is
useful in a set of trajectories is application-dependent, so for each
utility feature probably a different anonymisation algorithm is needed.

\subsection{Desirable utility features}

The utility features that are usually considered in
trajectory anonymisation are: (i)
trajectory length preservation,
(ii) trajectory shape preservation, (iii)
trajectory time preservation, and (iv) minimisation of
the number of discarded locations. We include two additional
utility features that are particularly meaningful in urban scenarios:
\begin{itemize}
\item {\em Location preservation}.
This essentially means that no fake or inaccurate locations
are used to replace original locations; otherwise put,
locations in the anonymised trajectories should
be locations visited by the original trajectories, without
any generalisation or accuracy loss.
Preserving original locations helps answering several queries
that may not be responded by generalisation methods~\cite{monreale10}
or some microaggregation methods~\cite{abul08,abul10}:
(i) what is the ranking of original (non-removed) locations, from most visited
to least visited?; (ii) in which original (non-removed)
locations did two or more mobile
objects meet?, etc.
On the other hand, if trajectory anonymisation rests on replacing true
locations with
fake locations, an adversary can distinguish the latter from the former
and discard fake locations.
Hence, location preservation is desirable for both utility and privacy
reasons.
\item {\em Reachability}. In the second proposed heuristic,
easy reachability between two successive locations in each anonymised
trajectory is enforced. This means that
the distance from the -th location
to the -th location on an anonymised location {\em following
the underlying network of streets and/or roads}
should be at most , where  is a preset
parameter. Like location preservation, this is as good for utility
as it is for privacy: if the adversary sees that reaching the -th
location from the -th one takes a long trip across streets and roads,
she will guess that the section between those two locations was not
present in any original trajectory.
\end{itemize}

\subsection{Specific utility measures} \label{subsec:utility}


Basic utility measures are
the number of removed trajectories and the number of
removed locations, whether during pre-processing, clustering or cluster
anonymisation.

The distortion of the trajectory shape is another utility measure,
which can be captured with the
space distortion metric~\cite[Sec.VI.B]{abul08}. This metric
also allows accumulating the total space distortion of all anonymised
trajectories from original ones.

\begin{definition}[Space distortion metric~\cite{abul08}]\label{def:distortion}
The space dis\-tort\-ion of an a\-no\-nym\-ised trajectory
 with respect to its original trajectory  at
time  when  has triple  and  has
possible triple , is

where  is a distance ({\em e.g.} Euclidean), and  a constant that
penalises for removed locations.
The space distortion of an anonymised trajectory
 from its original  is then

where  are all the timestamps where  is defined. In particular, if
 is discarded during anonymisation,  is empty, and so
, where  is the number of locations of .
In this way, the space distortion of a set of trajectories 
from its anonymised set  is easily defined as

where  (which may be empty) corresponds to .
\end{definition}





Another way to measure utility is by comparing the results between queries
performed on both the original data set  and the
anonymised data set . Intuitively, when results
on both data sets are similar for a large and diverse number of queries,
the anonymised data set can be regarded as preserving
the utility of the original data set. The challenge of this utility
measure is the selection of queries, which is usually
application-dependent or even user-dependent, {\em i.e.}
two different users are likely
to perform different queries on the same trajectory data set.

In~\cite{Trajcevski:2004:MUM:1016028.1016030} six types of spatio-temporal
range queries were introduced, aimed at evaluating the relative position
of a moving object with respect to a region  in a time
interval . We have used these queries
in our experimental work, even though they were designed
for use on uncertain trajectories
(see Definition~\ref{def:motion_curve})
rather than synthetic trajectories.

\begin{definition}[Uncertain trajectory] \label{def:motion_curve}
Given a trajectory  and an uncertainty space threshold ,
an \emph{uncertain trajectory}  is defined as
the pair ,
where  if and only if  such that
 and the Euclidean distance between  and 
is not greater than .
\end{definition}

\begin{definition}[Possible motion curve]
A \emph{possible motion curve}  of an uncertain trajectory  is an ordered set of timestamped locations

such that  for all .
\end{definition}

In short, a possible motion curve defines one of the possible
trajectories that an object moving along
an uncertain trajectory could follow.
Unlike in~\cite{Trajcevski:2004:MUM:1016028.1016030},
our anonymised trajectories are not uncertain;
hence, we will only
use the two
spatio-temporal range queries proposed in that paper
that can be adapted to non-uncertain trajectories:

\begin{itemize}
\item \emph{Sometime\_Definitely\_Inside(, , , )}
    is \emph{true} if and only if
    there exists a time  at which
    every possible motion curve  of an uncertain trajectory
     is inside
    region . For a non-uncertain , the previous condition
    can be adapted as: if and only if
    there exists a time  at which
       is inside .
\item \emph{Always\_Definitely\_Inside(, , , )}
    is \emph{true} if and only if at every time , every
    possible motion curve  of an uncertain trajectory 
    is inside region . For a non-uncertain , the previous condition
    becomes: if and only if at every time , trajectory  is
	    inside .
\end{itemize}

\subsection{Adversarial model and target privacy properties}
\label{adversarial}

In our adversarial model, the adversary has access to the
published anonymised set of trajectories
. Furthermore, the adversary also knows that every
location  must be in
the original set of trajectories .
Note that this adversary's knowledge makes an
important difference from
previous adversarial models~\cite{abul08, nergiz09, monreale10, yarovoy09},
because in our model the linkage of some location
with some user reveals the exact location of
this user rather than a generalised or perturbed location.

Further, the method used for
transforming the original set
of trajectories  into  is
assumed known by the adversary. However, this does not
include the method parameters or the seeds for pseudo-random number generators, which are considered secret. Indeed, the two methods we
are proposing rely on random permutations of locations and
random selection of trajectories during the clustering process,
and such randomness is in practice implemented using pseudo-random number
generators.
If an adversary knew the seeds of the generators, she could
easily reconstruct the original trajectories from the anonymised
trajectories.

Finally, the adversary also knows a
sub-trajectory  of some original target trajectory
 () and knows that the
anonymised version of  is in . As in previous works,
we consider that every location in  is sensitive, {\em i.e.}
for any location, learning that a specific user
visited it represents useful knowledge
for the adversary.

Then, we identify two attacks:
\begin{enumerate}
    \item Find a trajectory 
    that is the anonymised version of .
    \item Given a location ,
    determine whether .
\end{enumerate}

If the adversary succeeds in the first attack of
linking a trajectory  with the target , the second
is not trivial, because in general the locations in
 will not be those in , but it is indeed easier.
This means that both attacks
are not independent. However, the second attack
can trivially succeed even if the first attack does not:
if all anonymised trajectories cross the same location 
and , the attacker knows that .
As we show below, both attacks are
related to the two well-known privacy notions of
-anonymity~\cite{samarati98,sweeney02a} and
-diversity~\cite{machanavajjhala06}, respectively.

\begin{definition}[Trajectory -privacy] \label{def:trajectory_private}
Let  denote the probability of the
adversary's correctly linking the anonymised
trajectory  with  given
the adversary's knowledge . Then, \emph{trajectory -privacy}
is met when  for every trajectory  and every subset .
\end{definition}

\begin{definition}[Trajectory -anonymity]\label{def:anonymity}
Trajectory -anonymity is achieved if and
only if trajectory -privacy is met.
\end{definition}

\begin{definition}[Location -privacy] \label{def:location_private}
Let  denote the probability of the
adversary's success in correctly determining
a location ,
given the adversary's knowledge .
Then, \emph{location -privacy} is met
when  for every triple  such
that ,  and .
\end{definition}



\begin{definition}[Location -diversity]\label{def:diversity}
Location -diversity is achieved if and only
if location -privacy is met.
\end{definition}

\subsection{Discussion on privacy models}

Achieving straightforward
trajectory -anonymity, where each anonymised trajectory would
be identical to  other anonymised trajectories, would in general
cause a huge information loss.
This is why some other trajectory -anonymity definitions
under different assumptions have been proposed.

The -anonymity definition~\cite{abul08, abul10}
relies on the uncertainty inherent to trajectory data recorded by
technologies like GPS. However, it may be hardly applied
when accurate data sets of trajectories are needed. Furthermore,
in order to achieve -anonymity, the  identical
anonymised trajectories should be defined roughly in the
same interval of time and they must contain the same number of locations.
Such constraints are indeed hard to meet.

According to our privacy model, trajectory -anonymity
is achieved when there are at least  anonymised trajectories
in  having an anonymised version of 
as a sub-trajectory. Although this definition ignores the
time dimension, it does not require the length of the 
anonymised trajectories to be equal. However, suppose
that the adversary has a trajectory  consisting of only one
location, an individual's home;
whatever the anonymisation method,
the anonymised version of  is likely to be very similar to .
This means that there
will be  anonymised trajectories
containing the single location of .
However, not all of these anonymised
trajectories start at the single location of . Since an individual's home
is likely to be the first location of any individual's original
trajectory, those anonymised trajectories that do not start at the single
location of  (just pass through it) can be filtered out by an adversary
and only the remaining trajectories are considered.
The same filtering process can be performed if the adversary
knows locations where the individual has never been. In this way, using
side knowledge the adversary identifies less than  anonymised
trajectories compatible with the original trajectory .
Hence, this definition may not actually guarantee -anonymity
in the sense of Definition~\ref{def:anonymity}.

In conclusion, different levels of privacy can be provided
according to different assumptions on the
original data, the anonymised data, and the adversary's capabilities.
We defined above trajectory -privacy
(Definition~\ref{def:trajectory_private})
and location -privacy (Definition~\ref{def:location_private})
in order to capture two different privacy notions
when the original locations are preserved.

\section{Distance between trajectories}
\label{sec:distance}

Clustering trajectories requires defining a similarity measure ---a distance
between two trajectories. Because trajectories are distributed over space
and time, a distance that considers both spatial and temporal aspects of
trajectories is needed. Many distance measures have been proposed in the
past for both trajectories of moving objects and for time series but most of them are ill-suited
to compare trajectories for anonymisation purposes.
Therefore we define a new distance which can compare trajectories that are
only partially or not at all timewise overlapping. We believe this is
necessary to cluster trajectories for anonymisation.
We need some preliminary notions.

\subsection{Contemporary and synchronised trajectories}

\begin{definition}[\%-contemporary trajectories]
Two trajectories

and

are said to be \%-contemporary if

with .
\end{definition}

Intuitively, two trajectories are 100\%-contemporary if and only if they
start at the same time and end at the same time; two trajectories are
0\%-contemporary if and only if they occur during
non-overlapping time intervals. Denote the overlap time of two trajectories
 and  as .

\begin{definition}[Synchronised trajectories]
Given two \%-contemporary trajectories  and  for some , both
\emph{trajectories are said to be synchronised} if they have the same number
of locations time-stamped within  and these correspond to the same
time-stamps. A \emph{set of trajectories is said to be synchronised} if all
pairs of \%-contemporary trajectories in it are
synchronised, where  may be different for each pair.
\end{definition}

If we assume that between two locations of a trajectory, the
object is moving along a straight line between the locations at a constant
speed, then interpolating new locations is
straightforward. Trajectories can be then synchronised in the sense that if
one trajectory has a location at time , then other trajectories defined
at that time will also have a (possibly interpolated) location at time .
This transformation
guarantees that the set of new locations interpolated in order to
synchronise trajectories is of minimum cardinality.
Algorithm~\ref{alg:sync} describes
this process.
The time complexity of this algorithm is  where 
is the number of different time-stamps in the data set.

\begin{algorithm}[!ht]
\caption{Trajectory synchronisation} \label{alg:sync}
\begin{algorithmic}[1]
\STATE \textbf{Require:}  a set of trajectories to be
synchronised, where each  is of the form:

\STATE Let  be all time-stamps from all locations of all trajectories;
\FORALL{}
  \FORALL{ with }
    \IF{location having time-stamp  is not in }
      \STATE insert new location in  having the time-stamp  and
      coordinates interpolated from the two timewise-neighboring locations;
    \ENDIF
  \ENDFOR
\ENDFOR
\end{algorithmic}
\end{algorithm}

\subsection{Definition and computation of the distance}
\label{comput}

\begin{definition}[Distance between trajectories] \label{def:distance}
Consider a set of synchronised trajectories  where each trajectory is written as

The \emph{distance between trajectories} is defined as follows.
If  are \%-contemporary with , then

If  are \%-contemporary but there is
at least one subset of 

such that , 
and  and  are \%-contemporary
with  for  to , then

Otherwise  is not defined.
\end{definition}

The computation of the distance between every pair of trajectories is not
exponential as it could seem from the definition. Polynomial-time
computation of a distance graph containing the distances between all pairs
of trajectories can be done as follows.

\begin{definition}[Distance graph] \label{def:distancegraph}
A \emph{distance graph} is a weighted graph where
\begin{enumerate}
\setlength{\itemsep}{-1mm}
\item[(i)] nodes represent trajectories,
\item[(ii)] two nodes  and  are adjacent if the corresponding
trajectories are \%-contemporary for some
, and
\item[(iii)] the weight of the edge  is the distance between
the trajectories  and .
\end{enumerate}
\end{definition}

Now, given the distance graph for ,
the distance  for two trajectories is easily computed as the
minimum cost path between the nodes  and , if such path exists.
The inability to compute the distance for all possible trajectories (the
last case of Definition~\ref{def:distance}) naturally splits the distance
graph into connected components. The connected component that has the majority
of the trajectories must be kept, while the remaining components represent
outlier trajectories that are discarded in order to preserve privacy. Finally,
given the connected component of the distance graph having the majority of
the trajectories of , the distance  for \emph{any
two} trajectories on this connected component is easily computed as the
minimum cost path between the nodes  and .
The minimum cost path between every pair of nodes can
be computed using the Floyd-Warshall algorithm~\cite{Floyd} with computational cost
, {\em i.e.} in polynomial time.

\subsection{Intuition and rationale of the distance}

In order to deal with the time dimension, our distance measure applies
a linear penalty of  to those trajectories that
are -contemporary. This means that, the closer in time are
two trajectories, the shorter is our distance between both.
It should be remarked that we choose a linear penalty because
the Euclidean distance is also linear in terms of the spatial coordinates
and the Euclidean distance is the spatial distance measure we
consider by default. Other distances
and other penalties might be chosen, {\em e.g.} .

A problem appears when considering -contemporary trajectories.
How can two non-overlapping trajectories be penalised?
A well-known strategy is to give a weight to the time dimension
and another weight to the spatial dimension. By doing so,
the time distance and the spatial distance can be computed separately,
and later be merged using their weights. However, determining
proper values for these weights is a challenging task.

Anyway, the following lemma guarantees that, whenever
we consider two trajectories at minimum distance
for clustering, they do have some overlap.

\begin{lemma}
Any two trajectories in data set  at minimum distance are
-contemporary with .
\end{lemma}

\begin{proof} 
Consider a trajectory  and another trajectory
 at minimum distance from .
Assume that  and  are not -contemporary with .
Then, since the distance
between  and  is defined, according to
Definition~\ref{def:distance} a subset of distinct trajectories

must exist such that , 
and  and  are \%-contemporary
with  for  to , and

Then  for all
 from 1 to  (strict inequality holds because
all trajectories in  are distinct). Thus,
we reach the contradiction that  is not minimum.
Hence, the lemma must hold.
\end{proof}




\section{Anonymisation methods} \label{sec:anonymisation}

We present two anonymisation methods,
called SwapLocations and ReachLocations, respectively, which yield anonymised trajectories consisting of true original locations.
The first method is
partially based on the microaggregation~\cite{domingo02}
of trajectories and partially based
on the permutation of locations.
The second method is based on the permutation of locations only.
The main difference between
the SwapTriples method~\cite{domingo10springl}
and the two new methods we propose here is that the latter
effectively guarantees trajectory -anonymity (SwapLocations)
or location -diversity (ReachLocations).
To that end, an original
triple is discarded if it cannot be swapped randomly with
another triple drawn from a set of  other original triples.

Our two methods differ from each other in several aspects.
The first method assumes an unconstrained environment, while
the second one considers an environment with mobility constraints,
like an underlying street or road network.
SwapLocations effectively achieves trajectory -anonymity.
ReachLocations provides higher utility by design, but regarding
privacy, it offers location -diversity instead of
trajectory -anonymity. A common feature of both methods
is that locations in the resulting anonymised trajectories
are true, fully accurate original locations,
{\em i.e.} no fake, generalised or perturbed locations are
given in the anonymised data set of trajectories.

\subsection{The SwapLocations method}
\label{swaploc}

Algorithm \ref{alg:clustering} describes the process
followed by the SwapLocations method in order to anonymise
a set of trajectories. First, the set of trajectories
is partitioned into several clusters.
Then, each cluster is anonymised
using the SwapLocations function in Algorithm \ref{alg:swap_locations}.
We should remark here that we only consider trajectories
for which the distance to other trajectories can be computed
using the distance in Definition~\ref{def:distance}. Otherwise said,
given the distance graph  (Definition~\ref{def:distancegraph}),
our distance
measure can only be used within one of the connected components of ;
obviously, we take the trajectories in the largest connected component
of .
It should also be remarked that Algorithm~\ref{alg:sync} is only used
to compute the distance between trajectories. Once a cluster 
is created, the anonymisation algorithm works over the original
triples of the trajectories in , and not over the triples created
during synchronisation.

We limit ourselves to clustering algorithms which
try to minimise the sum of the intra-cluster distances or approximate
the minimum and such that the cardinality of each cluster
is , with  an input parameter; if the number of trajectories
is not a multiple of , one or more clusters must absorb the
up to  remaining trajectories, hence those clusters will have
cardinalities between  and .
This type of clustering is precisely the one used in
microaggregation~\cite{domingo02}.
The purpose of minimising the sum of the intra-cluster distances
is to obtain clusters as homogeneous as possible, so that the subsequent
independent treatment of clusters does not cause much information loss.
The purpose of setting  as the cluster size
is to fulfill trajectory -anonymity, as shown in
Section~\ref{sec:guaranteesswaploc}.
We might employ any microaggregation heuristic for clustering
purposes (see details in Section~\ref{sec:complex} below).

\begin{algorithm}[!ht]
\caption{Cluster-based trajectory anonymisation()} \label{alg:clustering}
\begin{algorithmic}[1]
\STATE \textbf{Require:} (i)  a set of original
trajectories such
that  is defined for all , (ii)  a
time threshold and  a space threshold, both of them public;
\STATE Use any clustering algorithm to cluster the trajectories of
, while minimising the sum of intra-cluster distances measured
with the distance of Definition~\ref{def:distance} and ensuring that
minimum cluster size is ;
\STATE Let  be the resulting clusters;
\FORALL{clusters }
\STATE  = SwapLocations(); \hfill // Algorithm~\ref{alg:swap_locations}
\ENDFOR
\STATE Let 
be the set of anonymised trajectories.
\end{algorithmic}
\vspace*{1mm}
\end{algorithm}

The SwapLocations function (Algorithm~\ref{alg:swap_locations})
begins with a random trajectory  in .
The function attempts to cluster
each unswapped triple  in 
with another 
unswapped triples belonging to different trajectories
such that: (i) the time-stamps of these triples differ by
no more than a time threshold  from the time-stamp of ;
(ii) the spatial coordinates differ by
no more than a space threshold .
If no  suitable
triples can be found that can be clustered with ,
then  is removed; otherwise,
random swaps of triples
are performed within the formed cluster.
Randomly swapping this cluster of triples guarantees that any of these triples
has the same probability of remaining in its original trajectory or becoming
a new triple in any of the other  trajectories.
Note that Algorithm~\ref{alg:swap_locations} guarantees that every triple  of every trajectory 
will be swapped or removed.




\begin{algorithm}[!ht]
\caption{SwapLocations()} \label{alg:swap_locations}
\begin{algorithmic}[1]
\STATE \textbf{Require:} (i)  a cluster of trajectories to be transformed, (ii)  a time threshold and  a space threshold;
\STATE Mark all triples in trajectories in  as ``unswapped'';
\STATE Let  be a random trajectory in ; \label{line:fewest}
\FORALL {``unswapped'' triples  in } \label{line:for}
    \STATE Let ;
// Initialise  with 
    \FORALL {trajectories  in  with }
        \STATE Look for an ``unswapped''
	triple 
	in  minimising the intra-cluster distance in  and such that:
        
        
        \IF { exists}
            \STATE ;
        \ELSE
        	\STATE Remove  from ;
          \STATE Go to line~\ref{line:for} in order to analyse the next triple
	  ;
        \ENDIF
    \ENDFOR
\STATE Randomly swap all triples in ;
    \STATE Mark all triples in  as ``swapped'';
\ENDFOR
\STATE Remove all ``unswapped'' triples in ;
\STATE \textbf{Return}  
\end{algorithmic}
\end{algorithm}

The SwapLocations function specified by
Algorithm~\ref{alg:swap_locations} swaps entire triples, that is,
time and space coordinates.
The following example illustrates the advantages of swapping
time together with space.

\begin{example}\label{tahrir}{\em
Imagine John attended one day the political protests in Tahrir
Square, Cairo, Egypt, but he would not like his political
views to become broadly known.
Assume John's trajectory is anonymised
and published. Assume further that an adversary knows the precise time
John left his hotel in the morning, say 6:36 AM ({\em e.g.}
because the adversary has bribed the hotel concierge into recording
John's arrival and departure times). Now:
\begin{itemize}
\item If SwapLocations swapped only
spatial coordinates, the adversary could re-identify John's
trajectory as one starting with a triple (6:36 AM, , ).
Furthermore,  must be a location
within a distance  from the hotel coordinates ,
although the adversary does not know the precise value of .
The re-identified trajectory would contain all true timestamps
of John's original trajectory (because they would not have been swapped),
and spatial coordinates within distance  of John's really visited
spatial coordinates.
Hence, it would be easy to check whether John was near Tahrir
Square during that day. Without swapping times,
privacy protection can only be obtained by taking  large enough
so that within distance  of the original locations visited by John there
are several semantically different spatial coordinates.
To explain
what we mean by semantic difference, assume 
is Tahrir Square and the trajectory
anonymiser guarantees that he has taken  large enough
so that  could be swapped
with some spatial coordinates  off Tahrir Square;
even if  turned out to be still within Tahrir Square, John
could claim to have been off Tahrir Square; the adversary
could not disprove such a claim, because in fact
 could be at a distance  from  and hence
outside the Square.
However, a large  means a large total space distortion.
\item If entire triples are swapped, as actually done by SwapLocations,
the adversary can indeed locate an anonymised
trajectory containing
(not necessarily starting with) triple (6:36 AM, , ).
However, there is only a chance  that this triple was not swapped
from another of the  original trajectories with which John's
original trajectory was clustered.
Similarly, the other triples in the anonymised
trajectory containing (6:36 AM, , )
have also most likely ``landed'' in that anonymised trajectory as a result
of a swap with some location in some of the  original trajectories
clustered with John's. Hence, John's trajectory is cloaked with 
other trajectories. We will prove in
Section~\ref{sec:guaranteesswaploc} that
this guarantees trajectory -anonymity in the sense of
Definition~\ref{def:anonymity}. In particular, the triple 
corresponding to John at Tahrir Square will appear in one of the
 anonymised trajectories, unless that triple has been removed
by the SwapLocations function because it was unswappable (the smaller
 and , the more likely it is for the triple to be removed).
\end{itemize}}
\end{example}

\subsection{The ReachLocations method}

The ReachLocations method, described in Algorithm~\ref{alg:reach},
takes reachability constraints into account: from a given
location, only those locations at a distance below a threshold
{\em following a path in an underlying graph} ({\em e.g.},
urban pattern or road network) are considered to be directly
reachable. Enforcing such reachability constraints while requiring
full trajectory -anonymity would result in a lot of original locations
being discarded. To avoid this, trajectory -anonymity is
changed by another useful privacy definition: location -diversity.

Computationally, this means that trajectories are {\em not} microaggregated
into clusters of size . Instead,
each location is -anonymised independently using
the entire set of locations of all trajectories. To do so, a
cluster  of ``unswapped'' locations is created around a
given location , {\em i.e.}
.
The cluster  is constrained as follows:
(i) it must have the lowest intra-cluster distance among those clusters
of  ``unswapped'' locations that contain the location ;
(ii) it must have locations belonging to  different trajectories; and (iii)
it must contain only locations at a path from 
at most  long and with time-stamps differing from
 at most . Then, the spatial coordinates
 are swapped with the
spatial coordinates of some random location in 
and both locations are marked as ``swapped''. If no cluster 
can be found, the location  is removed from the data set
and will not be considered anymore in the subsequent anonymisation.
This process continues until no more ``unswapped'' locations
appear in the data set.

It should be emphasised that, according to Algorithm~\ref{alg:reach},
two successive locations 
and  of
an original trajectory
 may be cloaked with respective sets of  locations belonging
to different sets of  original trajectories; for this reason
we cannot speak of trajectory -anonymity, see the example below.

\begin{example}{\em
Consider  trajectories
within city ,  trajectories within city 
and one trajectory  crossing from  to .
When applying ReachLocations, the initial locations of
 are swapped with locations of trajectories
within , whereas the final locations of 
are swapped with locations of trajectories within .
Imagine that an adversary knows a sub-trajectory 
of  containing
one location  in  and one
location  in .
Assume  and  are not removed by
ReachLocations anonymisation.
Now, the adversary will know that the anonymised trajectory
 corresponding to  is the only anonymised
trajectory crossing from  to . Thus, there is no
trajectory -anonymity, even if the adversary will
be unable to determine the exact locations of ,
because each of them has been swapped within a set of  locations.}
\end{example}

\begin{algorithm}[p]
\caption{ReachLocations()} \label{alg:reach}
\begin{algorithmic}[1]
\fontsize{11}{10}\selectfont
\STATE \textbf{Require:}
(i)  a set of original trajectories,
(ii)  a graph describing the paths between locations,
(iii)  is a time threshold and  is a space threshold, both
of them public;
\STATE Let  contain all locations from all trajectories, where
 and the spatial coordinates
 are called a point;
\STATE Mark all locations in  as ``unswapped'';
\STATE Let  be an empty set of anonymised trajectories;
\WHILE {there exist trajectories in }
    \STATE Let  be a trajectory randomly chosen in ;
\FOR { to }
      \IF { is ``unswapped''}
        \STATE Let  be
	a cluster of locations in  such that: \label{line:bestCluster}
\begin{enumerate}
\item All locations in  are ``unswapped'', with points
different from  and no two equal points;
\item Points in  belong to trajectories
in 
and no two points belong to the same trajectory;
\item For any , it holds that:
\begin{enumerate}
\item 
\item If  there is a path in  between  
and  ;
\item If  there is a path in 
between  and ;
\item The length of each path above is no more than ;
\end{enumerate}
\item The sum of intra-cluster distances
in  is minimum among clusters
of cardinality  meeting the previous conditions;
\end{enumerate}
        \IF {such a cluster  does not exist}
            \STATE Remove  from ;
        \ELSE
	    \STATE Mark  as ``swapped'';
            \STATE With probability :
	    \begin{enumerate}
	    \item Pick a random
	  location  and mark it as ``swapped'';
	    \item Swap the spatial
	    coordinates  of 
	    with the spatial coordinates 
	    of ;
	    \end{enumerate}
        \ENDIF
      \ENDIF
    \ENDFOR
    \STATE ;
    \STATE Remove  from ;
\ENDWHILE
\STATE \textbf{Return} .
\end{algorithmic}
\end{algorithm}


Algorithm~\ref{alg:reach} swaps
only spatial coordinates instead of full triples.
We show in the example below that this is enough
for ReachLocations to achieve location -diversity
(we have shown above that it cannot achieve trajectory
-anonymity anyway). If swapping time coordinates
is not beneficial in terms of privacy guarantees,
they should not be swapped, because the fact that
anonymised trajectories preserve the original sequence
of time-stamps of original trajectories increases their utility.

\begin{example}{\em
Let us resume Example~\ref{tahrir}, but now assume that
ReachLocations is used instead of SwapLocations to anonymise
trajectories. In this case, the adversary will find
an anonymised trajectory starting with (6:36 AM,,).
This anonymised trajectory will contain all true timestamps of John's
original trajectory. However, the spatial coordinates appearing
in any location of this re-identified trajectory are John's
original spatial coordinates with a probability at most .
We will prove in Section~\ref{guaranteesreach} below that this guarantees
location -diversity in the sense of
Definition~\ref{def:diversity}. If we want to prevent the adversary
from making sure that John visited Tahrir Square, we should
take  large enough (the discussion in Example~\ref{tahrir}
about the protection afforded by a large  when time is
not swapped is valid here).
}\end{example}

\subsection{Complexity of SwapLocations and ReachLocations}
\label{sec:complex}

We first give a complexity assessment of SwapLocations and
ReachLocations assuming that the distance graph
mentioned in Section~\ref{comput}
has been precomputed and is available.
This is reasonable, because
the distance graph needs to be computed only once, while the anonymisation
methods may need to be run several times ({\em e.g.} with different parameters).
Regarding SwapLocations, we have:
\begin{itemize}
\item Algorithm~\ref{alg:clustering} can use
any fixed-size microaggregation heuristic for clustering
({\em e.g.} MDAV in~\cite{domingo05}). Most microaggregation heuristics
have quadratic complexity, that is , where  is the number
of trajectories.
\item Algorithm~\ref{alg:clustering} calls the procedure
SwapLocations once for each resulting cluster, that is,
 times.
\item In the worst case, the complexity of procedure SwapLocations
(Algorithm~\ref{alg:swap_locations}) is proportional
to the number of locations of the longest trajectory in ,
say .
For each location, a search of another
location for swapping is performed among
the other  trajectories. The number of candidates
for swapping is . Hence, the complexity
of SwapLocations is .
\item The total complexity of the method is thus

\end{itemize}

Regarding the complexity of ReachLocations, we have
\begin{itemize}
\item Algorithm~\ref{alg:reach} has an external loop
which is called  times, where  is the number
of trajectories in . For each trajectory,
a swap is attempted for each of its unswapped locations.
Hence the algorithm performs  swaps,
where  is the number of locations in the longest
trajectory.
\item Each swap involves forming a cluster which  locations
selected from ,
which takes time
proportional to the total number of locations in ,
that is, .
\item Hence, the total complexity of the method is .
\end{itemize}

By comparing the last expression and Expression~\ref{compswaplocations},
we see that both SwapLocations and ReachLocations are quadratic in
 and quadratic in , but ReachLocations is slower.
Such complexity motivates the following two comments
related to scalability:
\begin{itemize}
\item If the number of trajectories  in the original data set is very large,
quadratic complexity may be very time consuming. In this case, a good
strategy is to use some blocking technique to split the original data set
into several subsets of trajectories, each of which should be anonymised
separately.
\item  being large may be less problematic than  being large,
provided that only a small fraction of trajectories have  or close
to  locations. If a lot of trajectories are very long, a good
strategy would be to split each of these into two or more trajectories and anonymise them independently.
\end{itemize}

Finally, in case we add the time complexity of the computation of the
distance graph mentioned in Section~\ref{comput} (which
is  using the Floyd-Warshall algorithm), the time complexities
of both SwapLocations and ReachLocations become
 and , respectively.

\section{Privacy guarantees}
\label{sec:guarantees}

\subsection{Privacy guarantees of SwapLocations}
\label{sec:guaranteesswaploc}

The main difference between the SwapTriples method
in~\cite{domingo10springl} and the SwapLocations method here
is that, in the latter, no original location
remains unswapped in an anonymised trajectory.

\begin{proposition} \label{prop:prob}
Let  be the adversary's knowledge of a target
original trajectory  and
 be all triples in .
For every trajectory , the probability
that the triple  in  appears in the anonymised version  of  produced by SwapLocations is:

\end{proposition}

\begin{proof} 
By construction of Algorithm~\ref{alg:swap_locations},
if  and  do not lie in the same cluster, there is no
possibility of swapping triples between them. Hence, in this
case, .

Let  be  trajectories that
are anonymised together in the same cluster by the SwapLocations method.
Without loss of generality, let us assume that .
By construction of Algorithm~\ref{alg:swap_locations},
for every , 
is  if  was removed,  otherwise.
Note that a swapping option is to swap a triple with itself,
that is, not to swap it.
Since it does not make sense to consider removed triples in ,
we conclude that  and,
in consequence, . 
\end{proof}

\begin{theorem} \label{theo:anonymity}
The SwapLocations method achieves trajectory -anonymity.
\end{theorem}

\begin{proof} 
By Proposition~\ref{prop:prob},
any sub-trajectory  has the same probability
of being a sub-trajectory of  than of being a sub-trajectory
of any of the  trajectories .
Thus, given , an adversary is not able to link 
with  with probability higher than .
Therefore, SwapLocations satisfies -privacy
according to Definition~\ref{def:trajectory_private}; according
to Definition~\ref{def:anonymity}, it also satisfies
trajectory -anonymity.
\end{proof}

\subsection{Privacy guarantees of ReachLocations}
\label{guaranteesreach}

We show below that ReachLocations
provides location -diversity.

\begin{proposition} \label{prop:reach}
Any triple  in an original
trajectory 
appears in the anonymised trajectory
 corresponding to  obtained with ReachLocations
if and only if  was not removed and was swapped with itself,
which happens with probability at most .
\end{proposition}

\begin{proof}
Let us prove the necessity implication.
By construction of Algorithm~\ref{alg:reach},
any triple
 whose spatial coordinates (point)
cannot be swapped within a cluster
 containing  different points
belonging to  different trajectories
is removed and does not appear in the set of anonymised trajectories.
Further, the only way for a non-removed triple  to remain
unaltered in  is precisely that its point is swapped with itself,
which happens with probability .
Therefore, to remain unaltered in , a triple
in  needs to avoid removal and to have its point swapped with itself,
which happens with probability at most .

Now let us prove the sufficiency implication.
Assume that  appears
in  without having been swapped with itself.
Then, by construction
of ReachLocations,  must have been formed as the result
of swapping a triple  with a triple 
from another original trajectory, where .
Buth then  would contain two triples with the same time-stamp 
and different spatial locations, which is a contradiction. 
\end{proof}

\begin{theorem}
The ReachLocations method achieves location -diversity.
\end{theorem}

\begin{proof}
Assume the adversary knows a sub-trajectory
 of an original trajectory . The sequence of time-stamps
in  allows the adversary to re-identify the anonymised trajectory
 corresponding to  (because the time-stamp
sequence is preserved).
By Proposition~\ref{prop:reach}, any
triple  belongs
to  with probability at most .
Now, consider a triple ,
where  is an anonymised trajectory different
from .
The probability that  came to  from
 is the probability that  was
swapped and swapping did not alter it. This probability is zero,
because swaps preserve time coordinates
but take place only between triples having different
space coordinates.
Hence, in terms of Definition~\ref{def:trajectory_private},
 for every triple
 such that , 
and . 
\end{proof}

Note that the previous proof also implies
that, even if a triple  is shared
by  anonymised
trajectories, the probability of 
remains at most .
What can be inferred by the adversary,
however, is that  original trajectories
(in general not the ones corresponding to the  anonymised
trajectories) visited spatial coordinates 
at possibly different times.
Indeed,  can be obtained by swapping 
and  for any  such that 
and for any  at path distance at most .
If  is the total number of anonymised trajectories, then
the adversary can be sure that original
trajectory  visited spatial coordinates
 at some time  such that .
Such inference by the adversary does
not violate location -diversity: violation
would require guessing {\em both} the spatial {\em and} temporal coordinates
of a triple in .
Of course, the time threshold  must be taken large enough so that
the time coordinate  is sufficiently protected.

\section{Experimental results and evaluation}

We implemented SwapLocations and ReachLocations. SwapLocations
performs clustering of trajectories using the partitioning step
of the MDAV microaggregation heuristic~\cite{domingo05}.
We used two data sets in our experiments:

\begin{itemize}
\item {\em Synthetic data set}.
We used the Brinkhoff's generator~\cite{brinkhoff03} to generate
1,000 synthetic trajectories which altogether visit
45,505 locations in the German city of
Oldenburg.
Synthetic trajectories generated with the Brinkhoff's generator have also
been used in~\cite{abul08,nergiz08,nergiz09,yarovoy09}.
We used this data set mainly for comparing
our methods with -anonymity~\cite{abul08}.
The number of trajectories
being moderate, we were able to run
in reasonable time the methods to be compared with a large
number of different parameter choices. Another advantage
is that the street graph of Oldenburg was available, which
is necessary to run ReachLocations. The downside of this data set having
a moderate number of trajectories is that these are rather sparse,
which causes the relative distortion in the anonymised data set to
be substantial, no matter the method used. Anyway, this is not
a serious problem to compare methods with each other.
\item {\em Real-life data set}.
We also used a real-life data set of cab mobility traces that were
collected in the city of San Francisco \cite{comsnets09piorkowski}.
This data set consists of 536 files, each of them containing
the GPS coordinates of a cab during a period of time.
After a filtering process, we obtained 4582 trajectories and 94
locations per trajectory on average. The advantage
of this data set over the synthetic one is that it
contains a larger number of trajectories and that these are real ones.
Then, we show through a real example how appropriate is our distance metric for trajectory clustering. Also, we present
utility measures on the SwapLocations method for this real-life data set
using different space thresholds. The weakness of this data set
is that it cannot be used for ReachLocations, because it does
not include the underlying street graph of San Francisco.
\end{itemize}


\subsection{Results on synthetic data}

For the sake of reproducibility, we indicate
the parameters we used in Brinkhoff's generator
to generate our Oldenburg synthetic data set: 6
moving object classes and 3 external object classes; 10 moving objects and
1 external object generated per timestamp; 100 timestamps; speed 250;
and ``probability'' 1,000. This resulted in 1,000 trajectories containing
45,405 locations. The maximum trajectory length was 100 points, the
average length was 45.4 locations, and the median length was 44 locations.

\subsubsection{Implementation details of our methods}

We have introduced a new distance measure between trajectories used by
the SwapLocations proposal during the clustering process.
As mentioned in Section~\ref{swaploc} above,
our distance function can only be used
within one of the connected components of the distance graph .
During the construction of
the distance graph for the synthetic data we found  connected
components,  of them of size . Therefore, we removed these 
trajectories in order to obtain a new distance graph with just one connected
component. In this way, we preserved  percent of all
trajectories before the anonymisation process. The
removed trajectories were in fact trajectories of length one, {\em i.e.}
with just one location in each one.

The SwapLocations method has been implemented using the
following simple microaggregation method for trajectories:
first, create clusters of  size  with minimum
intra-cluster distance and then disperse the up to 
unclustered trajectories to existing
clusters while minimising the intra-cluster distance. This algorithm incurs
no additional discarding of trajectories.

On the other hand, the ReachLocations method
does not remove trajectories, unlike the SwapLocations method.
It does, however, remove non-swappable locations, which
causes the removal of any trajectory consisting of
non-swappable locations only.

\subsubsection{Implementing -anonymity for comparison with
our method}

We compared our proposals with -anonymity~\cite{abul08}. Since
-anonymity only works over trajectories having the same
time span, first a pre-processing step to partition the trajectories is
needed. Superimposing the begin and end times of the trajectories
through reduction of the time coordinate modulo a parameter
 does not always yield
at least  trajectories having the same time span; it may also happen that a
trajectory disappears because the new reduced end time lies before the new
reduced begin time.

We have used  which kept the maximum (and so discarded the minimum)
trajectories. From the 1,000 synthetic trajectories, 40 were discarded because
the end time was less than the begin time and 187 were discarded because
there were at most 4 trajectories having the same time span. In total, 227
(22.7\%) trajectories were discarded just in the pre-processing step. The
remaining 773 trajectories were in 32 sets having the same time span, each
set containing a minimum of 15 trajectories and 24 on average.

We performed -anonymisation for , 4, 6, 8, 10, and
15 and , 1000, 2000, 3000, 4000 and 5000. Because of the
pre-processing step, using a higher  was impossible without
causing a
significant number of additional trajectories to be discarded.


\subsubsection{Utility comparison}

The performance of our proposals strongly depends on the values
of the time and space threshold parameters, denoted as 
and , respectively. In practice, these values must be chosen
to maximise utility
while affording sufficient privacy protection. Too large
thresholds reduce utility (large space distortion if  is too
high and large time distortion is  is too high),
but too small thresholds
reduce utility because of removal of many unswappable locations.
As a rule of thumb, as illustrated in Example~\ref{tahrir},
the space threshold  must be sufficiently large so that
within a radius  of any spatial location
there are sufficiently distinct locations ({\em e.g.}
if  lies in Tahrir Square, Cairo, there should
be points outside the Square within a radius  of ).

In order to compute the total space distortion, a value
for  must be chosen and this can be a challenging task.
Note that the value of  is application-dependent
({\em e.g.} for applications where the distortion should
measure the accuracy of trajectories
 should be zero so that only non-removed triples
contribute to , while for
applications that should avoid removing any triple
 should be very high).
For this reason we propose to compare separately the following three utility
properties: (i) total space distortion; (ii) percentage
of removed trajectories; and
(iii) percentage of removed locations. To do so, we set 
when computing the total space distortion. Consequently, the
percentage of removed triples as well as the percentage of removed
trajectories are considered separately from the total space distortion.

It should be remarked that the computation of the total space distortion
of the ReachLocations method is done using the Euclidean distance
between locations rather than the distance defined by the
reachability constraints (distance on the underlying network).
Note that reachability constraints should be considered during
the anonymisation process but not necessarily when computing
the total space distortion.

For successive anonymisations aimed at comparing the
SwapLocations and ReachLocations methods with -anonymity,
we set  and  in a way to obtain roughly the same
total space distortion values as in -anonymity
(cf. Table~\ref{tab:totalsd}) with .
The idea is that, after assuring that the three methods achieve
roughly the same total space distortion,
we will be able to focus on other utility properties
like the percentage of removed trajectories
and the percentage of removed locations. It should be
noted that our comparison is not entirely fair for any of
the three methods because all of them are aimed at achieving
different privacy notions. However, we believe that our
results are indicative of the weaknesses and the strengths of
our proposals.

\begin{table}[!ht]
\centering
\begin{tabular}{|r|r|r|r|r|r|r|}
\hline
   & 2 & 4 & 6 & 8 & 10 & 15 \\
\hline
0 & 48e6 & 93e6 & 120e6 & 143e6 & 165e6 & 199e6 \\
1,000 & 19e6 & 60e6 & 86e6 & 109e6 & 131e6 & 165e6 \\
2,000 & 4e6 & 32e6 & 56e6 & 78e6 & 99e6 & 133e6 \\
3,000 & .9e6 & 14e6 & 32e6 & 52e6 & 71e6 & 104e6 \\
4,000 & .2e6 & 5e6 & 16e6 & 32e6 & 48e6 & 79e6 \\
5,000 & .03e6 & 2e6 & 7e6 & 18e6 & 31e6 & 58e6 \\
\hline
\end{tabular}
\caption{Total space distortion (TotalSD) of -anonymity
for several parameter values (e6 stands for )}
\label{tab:totalsd}
\end{table}

The above principle of equating the space distortions
with -anonymity yields a value for the space threshold
 in each of SwapLocations and ReachLocations; however,
it does not constrain the time threshold, which we set at .
Regarding , we set it to achieve the total space
distortions of -anonymity
for cluster size  and

(parameter values considered in Table~\ref{tab:totalsd}).
In order to find such space thresholds efficiently, we assume that
the total space distortions of our methods define a monotonically
increasing function of the space threshold, {\em i.e.} the higher
the space threshold, the higher the total space distortion.
Under this assumption, we perform a logarithmic search over the set
of space thresholds defined by the interval . The reason behind
defining the maximum value for the space threshold as  is
that it is high enough to achieve low numbers of removed trajectories.
Indeed, as shown in Figure~\ref{fig:both_method}, for both methods there
exists a value  such
that, for every space threshold , neither the total
space distortion nor the percentage of removed locations and
removed trajectories significantly change. Table~\ref{tab:swap_thresholds}
and Table~\ref{tab:reach_thresholds} show the values of space thresholds
used in each configuration of -anonymity for
SwapLocations and ReachLocations, respectively.


\begin{figure}[!ht]
\centering
\includegraphics[width=3in, angle=270]{trajectory_figures/thereshold}
\includegraphics[width=3in, angle=270]{trajectory_figures/complete_distortion}
\caption{Top, percentage of removed trajectories and locations
with ,  and several
values of  for SwapLocations (SL) and ReachLocations (RL). Bottom, total
  space distortion with ,  and several
  values of  for SwapLocations and ReachLocations}
  \label{fig:both_method}
\end{figure}

\begin{table}[!ht]
\centering
\begin{tabular}{|c|c|c|c|c|c|c|}
\hline
   & 2 & 4 & 6 & 8 & 10 & 15 \\
\hline
0 &  &  &  &  &  &  \\
1,000 &  &  &  &  &  &  \\
2,000 & 899 &  &  &  &  &  \\
3,000 & 257 &  &  &  &  &  \\
4,000 & 86 & 1390 &  &  &  &  \\
5,000 & 19 & 681 & 2507 &  &  &  \\
\hline
\end{tabular}
\caption{Space thresholds used in SwapLocations to match the total
space distortion of each
configuration of -anonymity}
\label{tab:swap_thresholds}
\end{table}

\begin{table}[!ht]
\centering
\begin{tabular}{|c|c|c|c|c|c|c|}
\hline
   & 2 & 4 & 6 & 8 & 10 & 15 \\
\hline
0 & 499875 &  &  &  &  &  \\
1,000 & 25090 & 106126 & 270157 &  &  &  \\
2,000 & 4780 & 52468 & 93717 & 151915 & 249999 &  \\
3,000 & 749 & 37124 & 64801 & 95585 & 132857 & 238884 \\
4,000 & 136 & 25540 & 51089 & 73088 & 94465 & 152862 \\
5,000 & 57 & 18059 & 39061 & 58584 & 79101 & 113280 \\
\hline
\end{tabular}
\caption{Space thresholds used in ReachLocations to match
the total space distortion of each configuration of -anonymity}
\label{tab:reach_thresholds}
\end{table}

As it can be seen in Tables~\ref{tab:swap_thresholds}
and~\ref{tab:reach_thresholds}, we use the maximum value () of
the space threshold for several configurations. This is because in those
configurations
the total space distortion caused by the -anonymity
could not be reached by our methods no matter how much we increased
the space threshold.
Figure~\ref{fig:trashed} explains this behaviour by showing
the values of total space distortion SwapLocations
and ReachLocations minus the total space distortion of -anonymity.
With almost every configuration, our methods have a total space
distortion lower than the total space distortion of
-anonymity. In the case of SwapLocations,
the total space distortion is even much lower.

\begin{figure}[!ht]
\centering
\includegraphics[width=3in, angle=270]{trajectory_figures/d_thresholds}
\includegraphics[width=3in, angle=270]{trajectory_figures/d_thresholds_reach}
\caption{Top: total space distortion of SwapLocations minus
total space distortion of -anonymity
for several parameter configurations.
Bottom: total space distortion of ReachLocations minus
total space distortion of -anonymity
for several parameter configurations.
The space thresholds defined in Tables~\ref{tab:swap_thresholds}
  and~\ref{tab:reach_thresholds} have been used, respectively.}
\label{fig:trashed}
\end{figure}

In general, SwapLocations does not reach high values of the
total space distortion because it removes more locations than
ReachLocations in order to achieve trajectory -anonymity. Note that
removing locations does not increase the total space distortion
because we are considering . Tables~\ref{tab:swapLocations}
and~\ref{tab:reachLocations} show in detail the percentage of removed
trajectories and the percentage of removed locations for different
values of  and
, for SwapLocations and
ReachLocations, respectively.

As it can be seen in Table~\ref{tab:swapLocations}, in general SwapLocations
removes less trajectories than -anonymity because
SwapLocations can cluster non-overlapping trajectories. Indeed, with
-anonymity 227 trajectories were discarded
in the pre-processing step alone
because their time span could not match the time
span of other trajectories, and additional outlier trajectories
were discarded during clustering, up to a total  of discarded
trajectories. However, SwapLocations removed up to  of all
locations in the worst cases
and thus, it may not be suitable for applications where
preserving the number of locations really matters. SwapLocations
removes any location whose swapping set  contains less than
 locations, which is a relatively frequent event when
 trajectories with
different lengths are clustered together.
As the cluster size  increases, the length diversity tends
to increase and the removal percentage increases.
A simple way around the location removal problem is to
create clusters that contain trajectories with roughly the same length,
even though this may result in a higher total space distortion;
higher space distortion is a natural consequence of clustering
based on the trajectory length rather than the trajectory distance.


Table~\ref{tab:reachLocations} shows that
ReachLocations removes few trajectories when  is small
and  is large. The reason is that, for those parameterisations,
-anonymity introduces so much total space distortion
that ReachLocations can afford taking the
maximum space threshold  without reaching
that much distortion. Such a high space threshold allows
ReachLocations to easily swap spatial coordinates, so that
very few locations need to be removed.
Furthermore, the trajectories output by ReachLocations
are consistent with the underlying city topology.
As said above, the only drawback of this method is that in general it
does not provide trajectory -anonymity; rather, it provides
location -diversity.


\begin{table}[!ht]
\renewcommand{\arraystretch}{1.0}
\centering

\begin{tabular}{ @{} l c c c c c c c c c c c c @{}}
\toprule \multirow{2}*{   }
					&	\multicolumn{2}{c}{}	&	\multicolumn{2}{c}{}	&
					\multicolumn{2}{c}{}	& \multicolumn{2}{c}{}	& \multicolumn{2}{c}{}	&
                    \multicolumn{2}{c}{}	\\
					&	\textbf{T}	&	\textbf{L}	&	\textbf{T}	&	\textbf{L}	&
					\textbf{T}	&	\textbf{L}	& \textbf{T}	&	\textbf{L}	&
                    \textbf{T}	&	\textbf{L}	& \textbf{T}	&	\textbf{L}\\\cmidrule(l){2-13}

0	&		&		&		&		&		&		&		&		&		&	 	&		 &	 	 \\\cmidrule(l){2-13}

1000	&		&		&		&		&		&		&		&		&		&	 	&		 &	 	\\\cmidrule(l){2-13}

2000	&		&		&		&		&		&		&		&		&		&	 	&	 	 &	 	\\\cmidrule(l){2-13}

3000	&		&		&		&		&		&		&		&		&		&	 	&	 	 &	 	\\\cmidrule(l){2-13}

4000	&		&		&		&		&		&		&		&		&		&	 	&	 	 &	 	\\\cmidrule(l){2-13}
				
5000 &		&		&		&		&		&		&		&		&		&	 	 &	 	 &		\\\bottomrule
				
\end{tabular}
\caption{Percentage of trajectories (columns labeled with \textbf{T})
and locations
(columns labeled \textbf{L}) removed by SwapLocations
when using time threshold 100, 
and space thresholds that match the space distortion
caused by -anonymity with the previous 's and
. Percentages have
been rounded to integers for compactness.
\label{tab:swapLocations}}
\end{table}

\begin{table}[!ht]
\renewcommand{\arraystretch}{1.0}


\centering

\begin{tabular}{ @{} l c c c c c c c c c c c c @{}}
\toprule \multirow{2}*{   }
					&	\multicolumn{2}{c}{}	&	\multicolumn{2}{c}{}	&
					\multicolumn{2}{c}{}	& \multicolumn{2}{c}{}	& \multicolumn{2}{c}{}	&
                    \multicolumn{2}{c}{}	\\
					&	\textbf{T}	&	\textbf{L}	&	\textbf{T}	&	\textbf{L}	&
					\textbf{T}	&	\textbf{L}	& \textbf{T}	&	\textbf{L}	&
                    \textbf{T}	&	\textbf{L}	& \textbf{T}	&	\textbf{L}\\\cmidrule(l){2-13}

0	&		&		&		&		&		&		&		&		&		&		&		&	 	 \\\cmidrule(l){2-13}

1000	&		&		&		&		&		&		&		&		&		&		&		&	 	 \\\cmidrule(l){2-13}

2000	&		&		&		&		&		&		&		&		&		&		 &		&	 	 \\\cmidrule(l){2-13}

3000	&		&		&		&		&		&		&		&		&		&	 	 &	 	&	 	\\\cmidrule(l){2-13}

4000	&		&		&		&		&		&		&		&		&		&	 	 &	 	&	 	\\\cmidrule(l){2-13}

5000 &		&		&		&		&		&		&		&		&		&	 	 &	 	&		\\\bottomrule
				
\end{tabular}
\caption{Percentage of trajectories (columns labeled with \textbf{T})
and locations
(columns labeled \textbf{L}) removed by ReachLocations
when using time threshold 100, 
and space thresholds that match the space distortion
caused by -anonymity with the previous 's and
. Percentages have
been rounded to integers for compactness.
\label{tab:reachLocations}}
\end{table}


\subsubsection{Spatio-temporal range queries}

As stated in Section~\ref{subsec:utility}, a typical use
of trajectory data is to perform spatio-temporal range
queries on them. That is why we report
empirical results when performing the two query types
described and motivated in Section~\ref{subsec:utility}:
\emph{Sometime\_Definitely\_Inside} (SI)
and \emph{Always\_Definitely\_Inside} (AI).
We accumulate the number of trajectories in a set of
trajectories 
that satisfy the SI or AI range queries using
the  SQL style code below.

\begin{itemize}
    \item Query :
    \begin{itemize}
        \item[] \texttt{SELECT COUNT (*) FROM } \texttt{WHERE SI(}\texttt{.traj, R, }
    \end{itemize}
    \item Query :
    \begin{itemize}
        \item[] \texttt{SELECT COUNT (*) FROM } \texttt{WHERE AI(}\texttt{.traj, R, }
    \end{itemize}
\end{itemize}

Then, we define two different \emph{range query distortions}:

\begin{itemize}
    \item SID
where  is a set of SI queries as defined in Section~\ref{subsec:utility}
(definition of SI adapted to non-uncertain trajectories).
    \item AID
where  is a set of AI queries as defined in Section~\ref{subsec:utility}
(definition of AI adapted to non-uncertain trajectories).
\end{itemize}

For our experiments with the synthetic data set, we chose random
time intervals  such that .
Also, we chose random uncertain trajectories with a randomly
chosen radius  as regions .
Actually,  and  are, respectively, roughly a quarter of
the average duration and distance of all trajectories. Note that
we used uncertain trajectories {\em only} as regions ; however,
the methods we are considering in this chapter all
release non-uncertain trajectories.

Armed with these settings, we ran  times
both queries  and  on the original data set
and the anonymised data sets provided by SwapLocations, ReachLocations,
and -anonymity; that is, we took a set  with
.
The ideal range query distortion would be zero,
which means that query  for  yields the same
result for both the original and the anonymised data sets; in practice,
zero distortion is hard to obtain. Therefore, in order to compare
our methods against -anonymity, we use the same parameters
of the previous experiments (Tables~\ref{tab:totalsd},~\ref{tab:swap_thresholds}, and~\ref{tab:reach_thresholds}).
We show in Tables~\ref{tab:range_swapLocations}
and~\ref{tab:range_reachLocations} a comparison of SwapLocations,
respectively ReachLocations, against -anonymity in terms
of SID and AID.

\begin{table}[p]
\renewcommand{\arraystretch}{1.0}

\centering
\begin{tabular}{ @{} l c c c c c c c c c c c c @{}}
\toprule \multirow{2}*{   }
					&	\multicolumn{2}{c}{}	&	\multicolumn{2}{c}{}	&
					\multicolumn{2}{c}{}	& \multicolumn{2}{c}{}	& \multicolumn{2}{c}{}	&
                    \multicolumn{2}{c}{}	\\
					&	\textbf{S}	&	\textbf{A}	&	\textbf{S}	&	\textbf{A}	&
					\textbf{S}	&	\textbf{A}	& \textbf{S}	&	\textbf{A}	&
                    \textbf{S}	&	\textbf{A}	& \textbf{S}	&	\textbf{A}\\\cmidrule(l){2-13}

0	&		&		&		&		&		&		&		&		&		&	 	&		 &	 	 \\\cmidrule(l){2-13}

1000	&		&		&		&		&		&		&		&		&		&	 	 &	 	 &	 	\\\cmidrule(l){2-13}

2000	&		&		&		&		&		&		&		&		&		&	 	&	 	 &	 	\\\cmidrule(l){2-13}

3000	&		&		&		&		&		&		&		&		&		&	 	&	 	 &	 	\\\cmidrule(l){2-13}

4000	&		&		&		&		&		&		&		&		&		&	 	 &	 	 &	 	\\\cmidrule(l){2-13}
				
5000 &		&		&		&		&		&		&		&		&	 	 &	 	 &		 &		\\\bottomrule
				
\end{tabular}
\caption{Range query distortion of SwapLocations compared to
-anonymity for SID
(columns labeled with \textbf{S}) and AID (columns labeled with \textbf{A})
when using 
and space thresholds that match the space distortion
caused by -anonymity with the previous 's and
.
In this table,
a range query distortion  obtained with SwapLocations and a
range query distortion  obtained with -anonymity
are represented as the integer rounding of . Hence,
values in the table are positive if and only if SwapLocations outperforms
-anonymity.}
\label{tab:range_swapLocations}
\end{table}

\begin{table}[p]
\renewcommand{\arraystretch}{1.0}
\centering

\begin{tabular}{ @{} l c c c c c c c c c c c c @{}}
\toprule \multirow{2}*{   }
					&	\multicolumn{2}{c}{}	&	\multicolumn{2}{c}{}	&
					\multicolumn{2}{c}{}	& \multicolumn{2}{c}{}	& \multicolumn{2}{c}{}	&
                    \multicolumn{2}{c}{}	\\
					&	\textbf{S}	&	\textbf{A}	&	\textbf{S}	&	\textbf{A}	&
					\textbf{S}	&	\textbf{A}	& \textbf{S}	&	\textbf{A}	&
                    \textbf{S}	&	\textbf{A}	& \textbf{S}	&	\textbf{A}\\\cmidrule(l){2-13}

0	&		&		&		&		&		&		&		&		&		&		&	 	&	 	 \\\cmidrule(l){2-13}

1000	&		&		&		&		&		&		&		&		&		&		&		 &	 	 \\\cmidrule(l){2-13}

2000	&		&		&		&		&		&		&		&		&		&		 &		 &	 	 \\\cmidrule(l){2-13}

3000	&		&		&		&		&		&		&		&		&		&	 	 &	 	&	 	\\\cmidrule(l){2-13}

4000	&		&		&		&		&		&		&		&		&		&	 	 &	 	&	 	\\\cmidrule(l){2-13}

5000 &		&		&		&		&		&		&		&		&		&	 	 &	 	&		\\\bottomrule
				
\end{tabular}
\caption{Range query distortion of ReachLocations compared to
-anonymity for SID
(columns labeled with \textbf{S}) and AID (columns labeled with \textbf{A})
when using 
and space thresholds that match the space distortion
caused by -anonymity with the previous 's and
.
In this table,
a range query distortion  obtained with ReachLocations and a
range query distortion  obtained with -anonymity
are represented as the integer rounding of . Hence,
values in the table are positive if and only if ReachLocations outperforms
-anonymity.}
\label{tab:range_reachLocations}
\end{table}

It can be seen from Table~\ref{tab:range_swapLocations}
that SwapLocations performs significantly better than -anonymity
for every cluster size and . On
the other hand, Table~\ref{tab:range_reachLocations}
shows that ReachLocations outperforms -anonymity only
for  up to roughly 2000. Not surprisingly,
SwapLocations offers better performance than ReachLocations,
because the latter must deal with reachability constraints.
It is also remarkable that ReachLocations performs much better in
terms of SID than in terms of AID. The explanation is that,
while -anonymity and SwapLocations operate at the trajectory
level, ReachLocations works at the location level.

We conclude that, according to these experiments,
our methods perform better than -anonymity
regarding range query distortion for values of  up to .
The performance for larger values of  is less
and less relevant: indeed,
when , -anonymity means
that no trajectory needs to be anonymised and hence the anonymised
trajectories are the same as the original ones.


\subsection{Results on real-life data}



The San Francisco cab data set \cite{comsnets09piorkowski}
we used consists of several files each of them
containing the GPS information of a specific cab during May 2008.
Each line within a file contains the space coordinates (latitude and longitude)
of the cab at a given time. However,
the mobility trace of a cab during an entire month can hardly be
considered a single trajectory.
We used big time gaps between two consecutive locations
in a cab mobility trace to split that trace into several
trajectories. All trajectory visualisations shown in this Section
were obtained using Google Earth.

For our experiments we considered just one
day of the entire month given in the real-life data set, but
the empirical methodology described below could be extended to several days.
In particular, we chose the day between May 25 at 12:04 hours
and May 26 at 12:04 hours because during this 24-hour period
there was the highest
concentration of locations in the data set. We also defined
the maximum time gap in a trajectory as 3 minutes; above 3 minutes,
we assumed that the current trajectory ended and that the next location
belonged to a different trajectory. This choice was based on
the average time gap between consecutive locations in the data set,
which was 88 seconds; hence, 3 minutes was roughly twice the average.
In this way, we obtained 4582 trajectories and 94 locations
per trajectory on average.

The next step was to filter out trajectories with strange features
(outliers). These outliers could be detected
based on several aspects like velocity, city topology, etc.
We focused on velocity and defined 240 km/h as the maximum speed that could
be reached by a cab. Consequently, the distance between
two consecutive locations could not be greater than 12 km because the maximum
within-trajectory time gap was 3 minutes. This allowed us to
detect and remove trajectories containing obviously erroneous locations;
Figure~\ref{fig:outlier} shows one of
these removed outliers where a cab appeared to have
jumped far into the sea probably due
to some error in recording its GPS coordinates.
Altogether, we removed 45 outlier trajectories and we were left
with a data set of 4547 trajectories with an average of
93 locations per trajectory.
Figure~\ref{fig:ten_trajectories} shows the ten longest
trajectories (in number of locations) in the final data set that we used.


\begin{figure}[!ht]
\begin{minipage}[b]{0.5\linewidth}
\centering
\includegraphics[width=0.80\textwidth, bb = 0 0 1020 922]{trajectory_figures/earth.jpg}
\caption{Example of an outlier trajectory in the original real-life data set}
\label{fig:outlier}
\end{minipage}
\hspace{0.5cm}
\begin{minipage}[b]{0.5\linewidth}
\centering
\includegraphics[width=0.80\textwidth, bb = 0 0 1020 922]{trajectory_figures/10t.jpg}
\caption{Ten longest trajectories in the filtered real-life data set}
\label{fig:ten_trajectories}
\end{minipage}
\end{figure}



\subsubsection{Experiments with the distance metric}

We propose in this chapter
a new distance metric designed specifically for clustering
trajectories. Our distance metric considers both space and time,
dealing even with non-overlapping or partially-overlapping trajectories.
Contrary to the synthetic data where 10
trajectories had to be removed because the distances
to them could not be computed,
in this real-life data set our distance function
could be computed for every pair of trajectories.

Figure~\ref{fig:two_trajectories} shows two trajectories identified by
our distance metric as the two closest ones in the data set.
The two cabs moved around a parking lot and therefore stayed very
close to one another in space. Also in time both trajectories were very close:
one of them was recorded between 12:00:49 hours and 13:50:47
hours, while the other was recorded between 12:00:25 hours and 13:52:30 hours.
Therefore, both trajectories were correctly identified
by our distance metric as being close in time and space; they
could be clustered together with minimum utility loss for anonymisation
purposes.

\begin{figure*}[p]
\centering
\includegraphics[width=0.60\textwidth, bb = 0 0 1020 922]{trajectory_figures/two_trajectories.jpg}
\caption{The two closest trajectories in the real-life data set
according to our distance metric}
\label{fig:two_trajectories}
\end{figure*}

To compare, Figure~\ref{fig:two_trajectories2} shows two trajectories
identified by the Euclidean distance as the two closest ones in the data set.
These trajectories are located in a parking lot inside
San Francisco Airport and,
spatially, they are closer than the two trajectories shown
in Figure~\ref{fig:two_trajectories}. However, one of these trajectories
was recorded between 24:42:55 hours and 24:55:59 hours, while the other
was recorded between 19:05:29 hours and 19:06:15 hours. Hence, they
should not be
in the same cluster, because an adversary with time knowledge can easily
distinguish them.

\begin{figure*}[p]
\centering
\includegraphics[width=0.60\textwidth, bb = 0 0 1020 922]{trajectory_figures/two_trajectories2.jpg}
\caption{The two closest trajectories in the real-life data set
according to the Euclidean distance}
\label{fig:two_trajectories2}
\end{figure*}



\subsubsection{Experiments with the SwapLocations method}

The ReachLocations method cannot be used when the graph of the
city is not provided. Hence, in the experiments with
the San Francisco real data we just considered the SwapLocations method.
As in the experiments with synthetic data, we set 
during the computation of the total space distortion.
Figure~\ref{fig:real_distortion} shows the values of total space
distortion given by the SwapLocations for different space thresholds
and different cluster sizes.

\begin{figure*}[p]
\centering
\includegraphics[width=0.6\textwidth, angle=270]{trajectory_figures/d_real_distortion_threshold}
\caption{Total space distortion (km) for SwapLocations using several
different space thresholds and cluster sizes on the real-life data set}
\label{fig:real_distortion}
\end{figure*}

Two other utility properties we are considering in this work are:
percentage of removed trajectories and percentage of removed locations.
Table~\ref{tab:swapLocations2} shows the values obtained with the
SwapLocations method for both utility properties.

\begin{table}[p]
\renewcommand{\arraystretch}{1.0}
\centering
\begin{tabular}{ @{} l c c c c c c c c c c c c @{}}
\toprule \multirow{2}*{   }
					&	\multicolumn{2}{c}{}	&	\multicolumn{2}{c}{}	&
					\multicolumn{2}{c}{}	& \multicolumn{2}{c}{}	& \multicolumn{2}{c}{}	&
                    \multicolumn{2}{c}{}	\\
					&	\textbf{T}	&	\textbf{L}	&	\textbf{T}	&	\textbf{L}	&
					\textbf{T}	&	\textbf{L}	& \textbf{T}	&	\textbf{L}	&
                    \textbf{T}	&	\textbf{L}	& \textbf{T}	&	\textbf{L}\\\cmidrule(l){2-13}

1	&		&		&		&		&		&		&		&		&		&	 	&		&	 	 \\\cmidrule(l){2-13}

2	&		&		&		&		&		&		&		&		&		&	 	&		&	 	 \\\cmidrule(l){2-13}

4	&		&		&		&		&		&		&		&		&		&	 	 &		&	 	\\\cmidrule(l){2-13}

8	&		&		&		&		&		&		&		&		&		&	 	 &	 	&	 	\\\cmidrule(l){2-13}

16	&		&		&		&		&		&		&		&		&		&	 	 &	 	&	 	\\\cmidrule(l){2-13}
				
32	&		&		&		&		&		&		&		&		&		&	 	 &	 	&	 	\\\cmidrule(l){2-13}
				
64	&		&		&		&		&		&		&		&		&		&	 	 &	 	&	 	\\\cmidrule(l){2-13}
				
 128 &		&		&		&		&		&		&		&		&		&	 	 &	 	&	 	\\\bottomrule
				
\end{tabular}
\caption{Percentage of trajectories (columns labeled with \textbf{T}) and
locations (columns labeled with \textbf{L}) removed by SwapLocations
for several values of  and several space thresholds 
on the real-life data set.
Percentages have been rounded to integers for compactness.
\label{tab:swapLocations2}}
\end{table}

Finally, Table~\ref{tab:range_swapLocations_real} reports
the performance of SwapLocations regarding spatio-temporal range queries.
We picked random time intervals of length at most  minutes. Also,
random uncertain trajectories with uncertainty threshold of
size at most  km were chosen as the regions.
Analogously to the experiments with the synthetic data set,
 and  are roughly a quarter of the average duration and distance
of all trajectories, respectively. It can be seen
that the SwapLocations method provides low range query distortion
for every value of  when the space threshold is small, \emph{i.e.}
when the total space distortion is also small. However, the smaller
the space threshold, the larger the number of removed trajectories and
locations (see Table~\ref{tab:swapLocations2}). This illustrates
the trade-off between the utility properties considered.

\begin{table}[p]
\renewcommand{\arraystretch}{1.0}
\centering
\begin{tabular}{ @{} l c c c c c c c c c c c c @{}}
\toprule \multirow{2}*{   }
					&	\multicolumn{2}{c}{}	&	\multicolumn{2}{c}{}	&
					\multicolumn{2}{c}{}	& \multicolumn{2}{c}{}	& \multicolumn{2}{c}{}	&
                    \multicolumn{2}{c}{}	\\
					&	\textbf{S}	&	\textbf{A}	&	\textbf{S}	&	\textbf{A}	&
					\textbf{S}	&	\textbf{A}	& \textbf{S}	&	\textbf{A}	&
                    \textbf{S}	&	\textbf{A}	& \textbf{S}	&	\textbf{A}\\\cmidrule(l){2-13}

1	&		&		&		&		&		&		&		&		&		&	 	&		 &	 	 \\\cmidrule(l){2-13}

2	&		&		&		&		&		&		&		&		&		&	 	 &	 	 &	 	\\\cmidrule(l){2-13}

4	&		&		&		&		&		&		&		&		&		&	 	&	 	 &	 	\\\cmidrule(l){2-13}

8	&		&		&		&		&		&		&		&		&		&	 	&	 	 &	 	\\\cmidrule(l){2-13}

16	&		&		&		&		&		&		&		&		&		&	 	 &	 	 &	 	\\\cmidrule(l){2-13}
				
32	&		&		&		&		&		&		&		&		&		&	 	 &	 	 &	 	\\\cmidrule(l){2-13}
				
64	&		&		&		&		&		&		&		&		&		&	 	 &	 	 &	 	\\\cmidrule(l){2-13}
				
 128 &		&		&		&		&		&		&		&		&	 	 &	 	 &		 &		\\\bottomrule
				
\end{tabular}
\caption{Range query distortion caused by
SwapLocations on the real-life data set
for SID (columns labeled with \textbf{S}) and AID
(columns labeled with \textbf{A}), for several values of 
and several space thresholds . In this table,
a range query
distortion  is represented as the integer rounding of 
for compactness.
\label{tab:range_swapLocations_real}}
\end{table}



\section{Conclusions}

In this chapter, we have presented two permutation-based heuristic
methods to anonymise trajectories
with the common features that: (i)
places and times in the anonymised trajectories are true original places
and times with full accuracy;
(ii) both methods can deal with trajectories with partial or no time
overlap, thanks to a new distance also introduced in this paper.
The first method aims at trajectory -anonymity while the second method takes reachability constraints into account,
that is, it assumes a territory constrained
by a network of streets or roads; to avoid
removing too many locations, the second method changes its privacy
ambitions from trajectory -anonymity to location -diversity.

Both methods use permutation of locations, and the first method
uses also trajectory microaggregation.
There are few counterparts in the literature comparable to the
first method, and virtually none comparable to the
second method.
Experimental results show that, for most parameter choices
and for similar privacy levels,
our methods offer better utility
than -anonymity.

