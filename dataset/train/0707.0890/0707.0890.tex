\documentclass{LMCS}
\usepackage{hyperref,enumerate,proof}


\def\eqalign#1{\null\,\vcenter{\openup\jot
 \ialign{\strut\hfil&\hfil
     \crcr#1\crcr}}\,}
\def\rbox#1{\mbox{\rm #1}}
\def\:{\mathbin{\,:\,}}
\def\.{\mathbin{\,.\,}}
\def\RIghtarrow{\enspace\Rightarrow\enspace}
\def\aNd{\enspace\&\enspace}

\def\ebox#1{\enspace\hbox{\!}\enspace}

\def\eg{{\em e.g.}}
\def\cf{{\em cf.}}
\newcommand{\Na}{{I\kern-4pt N}}

\def\doi{4 (1:1) 2008}
\lmcsheading {\doi}
{1--19}
{}
{}
{Nov.~21, 2006}
{Jan.~\phantom{0}7, 2008}
{}   

\begin{document}

\title[Hilbert-style Pure Type Systems?]{Are there Hilbert-style Pure 
  Type Systems?}

\author[M.~W.~Bunder]{Martin W.~Bunder\rsuper a}	\address{{\lsuper a}School of Mathematics and Applied Statistics, 
University of Wollongong, Wollongong, NSW 2522, Australia}	 \email{martin\_bunder@uow.edu.au}  \thanks{{\lsuper{a,b}}The authors would like to thank the anonymous 
referees for their useful comments.}	

\author[W.~.J.~M.~Dekkers]{Wil J.~M.~Dekkers\rsuper b}	\address{{\lsuper b}Department of Computer Science, Radboud 
University Nijmegen, Toernooiveld 1, 6525 ED Nij\-megen, The Netherlands}
\email{wil@cs.ru.nl}

\keywords{Hilbert-style logics, pure type systems, type theory, lambda 
calculus, illative combinatory logic}
\subjclass{F.4.1}
\amsclass{03B40, 03B70, 68N18}




\begin{abstract}
  For many a natural deduction style logic there is a Hilbert-style
  logic that is equivalent to it in that it has the same theorems
  (i.e.\ valid judgements  where
  . For intuitionistic implicational logic, the
  axioms of the equivalent Hilbert-style logic can be propositions
  which are also known as the types of the combinators ,
   and .

  Natural deduction versions of illative combinatory logics have
  formulations with axioms that are actual type statements for ,  and . As pure type systems (PTSs) are, in
  a sense, equivalent to systems of illative combinatory logic, it
  might be thought that Hilbert style PTSs (HPTSs) could be based in a
  similar way.

  This paper shows that some PTSs have very trivial equivalent HPTSs,
  with only the axioms as theorems and that for many PTSs no
  equivalent HPTSs can exist.  Most commonly used PTSs belong to these
  two classes.

  For some PTSs however, including  and the PTS at the
  basis of the proof assistant Coq, there is a nontrivial equivalent
  HPTS, with axioms that are type statements for  and\
  .
\end{abstract}

\maketitle

\section*{Introduction}\label{S:one}

  \noindent Most early logical systems (for propositional and
  predicate logic) allowed no hypotheses and so had no rules for
  introducing or cancelling them.  These could be represented by a
  finite set of axiom schemes and rules of inference such as modus
  ponens and generalisation.

  Later natural deduction systems which did allow hypotheses had fewer
  axiom schemes but required introduction and elimination rules for
  hypotheses. Herbrand showed that classical Hilbert style and natural
  deduction style propositional and predicate logics had the same
  theorems (i.e. judgements with empty contexts).

  Pure type systems (PTSs), defined below, have two rules that
  introduce hypotheses and two that cancel them. In this paper we
  answer a question of Fairouz Kamareddine ``Are there Hilbert style
  PTSs?". When we define Hilbert style PTSs (HPTSs) as PTSs with empty
  contexts, with a finite set of extra axiom schemes, with  representing arbitrary sorts, and some extra rules, it is
  obvious that there are HPTSs. We will be interested in whether, for
  PTSs, there are theorem equivalent HPTSs. We will answer this
  question for a number of classes of PTSs which include all the PTSs,
  from the standard literature, that we have examined. The methods we
  use, for proving that a HPTS is equivalent to a PTS, are along the
  lines of those of Herbrand, but rather more complex.

  Just as combinator based programming languages, requiring no free,
  or in fact, no variables, have proved useful in practice, perhaps an
  HPTS, which also requires no (free) variables, that is theorem
  equivalent to a PTS may be useful. Also, perhaps some
  metatheoretical results may be proved more easily for an HPTS than
  for the equivalent PTS.

\section{Pure Type Systems}\label{S:PTS}

  \noindent Each Pure Type System (PTS)  has a set of
  variables , a set of constants , a set of ``sorts"
  . It has a class of pseudoterms
  given by . If
   and  are pseudoterms,  is a statement,  is a
  context if it is a sequence of statements;  is
  then called a judgement. A PTS has a set of axioms 
  each of the form  where  and . Then it has a set  of triples
  , which determine under what
  conditions a term  is in a sort.  Most PTSs are known
  by a ``specification"  (as
  usually ).

  The PTS postulates are as follows:
\smallskip
{\small}\smallskip

  \noindent When there are two judgements as premises in a rule, we call the
  left one the \emph{major premise} and the right one the \emph{minor
    premise}.

  Later we will need the following definition:

\begin{defi}[{Inhabited and Normal Form Inhabited Sorts}]\hfill

   is an \emph{inhabited sort}  if 
  for some .

   is a \emph{normal form inhabited sort}  if
  for some term  in normal form, 
\end{defi}

  The translation  of Bunder and Dekkers [3]
  translates the pseudoterms and statements of PTSs into terms
  of illative combinatory logic (ICL) as follows:


  where  ( is the
  combinator equivalent to . Terms in ICL can be
  represented without any free variables at all using the combinators
   and  (equivalent to ). 
  represents roughly  or .

  ICL, designed as a foundation for logic and mathematics, has a rule
  like (abstraction) which was derived in Bunder [2] from a set of
  axioms. In Section 6 we will see how the methods developed there
  lead to the ones used here. The main difference between PTSs and
  standard ICLs, other than the lack of distinction between terms and
  types, lies in the (abstraction) rule. The direct counterpart to the
  ICL rule would have , for .  This is the most important factor in making it
  difficult to have equivalent Hilbert-style PTSs.

\section{Hilbert-style PTSs}\label{S:HSPTS}

 \noindent We define Hilbert-style PTSs as follows:

\begin{defi}[HPTS]

  Each \emph{Hilbert style Pure Type System} (HPTS) has , , , , statements, contexts and judgements
  as for PTSs, except that the contexts are always empty. A HPTS has a
  set of sorts  and a set of axioms , as for
  PTSs, and an additional finite set  of axiom schemes
  in which ``sort variables" can be replaced by sorts.  Most HPTSs are
  known by a ``specification" 
  (as usually ). A HPTS has the PTS
  (application) and (conversion) rules (with empty contexts) as well
  as:
\bigskip
{\small}
\end{defi}
  Note the latter rules are derivable for all PTSs, for HPTSs neither
  is, even using (conversion).

\begin{defi}[Equivalent  HPTS]

  If  is a PTS with specification , a HPTS , with specification
   will be \emph{equivalent} if

  Here  stands for provability in  and  in .  If the PTS is arbitrary or obvious from
  the context we use  and .  will a
  function of , i.e. it will include axioms such as
   if
  
\end{defi}

  Below are some PTSs that have been studied in the literature
  (particularly Barendregt [1] and Geuvers [4]).

  In , , in
  all other cases  consists of all the
  constants visible in  and .  
  is used as an abbreviation for .
\bigskip
{\small}
  The PTS used in the proof assistant Coq we will call .
  It has as axioms:

  More axioms are generated by

  In early versions  is given by
 for all .
  Coq 8.0 replaces  by .

  We will be able to determine whether or not there are equivalent
  HPTSs for all of the above.

\section{Some PTS Lemmas and Definitions}\label{S:PTSLD}

 \noindent We now state a number of standard lemmas for PTSs.  Most
  proofs can be found in Barendregt [1] or Bunder and Dekkers [3].

\begin{lem}[Free Variable Lemma]\label{L:one}

  If , then
\begin{enumerate}[\em(i)]
\item  are distinct;
\item ;
\item  for .\qed
\end{enumerate}
\end{lem}

\begin{lem}[Substitution Lemma]\label{L:two}

  If ,  and 
  then .\qed
\end{lem}

\begin{lem}[Condensing Lemma]\label{L:three}

  If , where , then .\qed
\end{lem}

\begin{lem}[Generation Lemma]\label{L:four}

  Let .  Then
\begin{enumerate}[\em(i)]
\item
;

\item
;

\item
\hfill\break
;

\item
\hfill\break
;

\item
.
\end{enumerate}
  In each case the derivations, of the judgements of the form
   in \emph{(iii)} to \emph{(v)}, are shorter than
  that of .\qed
\end{lem}

\begin{lem}[Correctness of Types Lemma]\label{L:five}

  If  then   or .\qed
\end{lem}

\begin{lem}[Subject and Type Reduction Lemma]\label{L:six}

  If , then
\begin{enumerate}[\em(i)]
\item  implies
  ,
\item and  implies .\qed
\end{enumerate}
\end{lem}

\begin{lem}[Start Lemma]\label{L:seven}

  If , then
\begin{enumerate}[\em(i)]
\item  implies ,
\item implies that for  there is an  such that .\qed
\end{enumerate}
\end{lem}

\section{PTSs where \texorpdfstring{}{A} is the Set of 
  Theorems}\label{S:PTSST}

 \noindent The following lemma specifies a set of PTSs whose axioms
 are its only theorems.  The equivalent HPTS is then trivially one
 with no extra axioms, i.e. with .

\begin{lem}\label{L:eight} In a PTS satisfying

  we have .
\end{lem}

\proof We show  by induction on the derivation of


  (\ref{EQ:ma}) clearly does not come by (start) or (weakening).

  If (\ref{EQ:ma}) comes by (application) from

  where  and , we have by the induction
  hypothesis , which is impossible.

  If (\ref{EQ:ma}) comes by (abstraction) from
 
  where  and , then by
  the induction hypothesis , which
  is impossible.

  If (\ref{EQ:ma}) comes by (product) from

  where  and , then, by the
  induction hypothesis, , which is impossible
  by (\(M\:B), (A\:s)\in{\mathcal A}ABA\equiv B(M\:A)\in{\mathcal A})} has an equivalent HPTS, with
   but this is trivial in that it has only
  its axioms as theorems.\qed
\end{thm}


\begin{cor}\label{C:ten}
   and P each have an equivalent
  HPTS, but  is the only theorem of both
  systems.\qed
\end{cor}

\section{PTSs with no Equivalent HPTS}\label{S:PTSnoHPTS}

 \noindent In  and P there is no term
   such that  and the only theorem is .

  We can show, by a single (product) rule preceeded by two uses of an
  axiom and a (start) or (weakening) rule, that in the other PTSs,
  given in Section 2, there are theorems that are not axioms. Most of
  these are given below.

\begin{lem}\label{L:eleven}\hfill

\begin{enumerate}[\em(i)]
\item In   we have .
\item In , , , ,
  ,  and , we have
  .
\item In  and
  , we have
  .
\item In ,  and
   we have
  .\qed
\end{enumerate}
\end{lem}

  We now give a condition under which, in a PTS, certain sorts have an
  infinite number of inhabitants of the form  that are
  not substitution instances of each other.  We show later that many
  PTSs with this property cannot be equivalent to HPTSs.

\begin{lem}\label{L:twelve} Assume that in a PTS there is a finite
  sequence  such that:

  then

  for an infinite number of -distinct terms 
  which are not ( for ) substitution instances of each
  other.
\end{lem}

\proof Assume that we have (s_1,\dots,s_ns_1, s_2,\dots \in
 \mathcal Ss_1\in {\mathcal N}A_1i1 < i\leq nB_{i-1}A_i =\Pi x_{i-1}{:}B_{i-1} . A_{i-1}i\), an  such that  and a  such that

  When  we have  above, otherwise we have  by the induction hypothesis. By (weakening) we have

  and by (product) we have (\ref{EQ:two}).  So (\ref{EQ:two}) holds for
   and, as we have ,

  Repeating the above, with  for , we get  and similarly ,\dots

  If  for , then

  and so  and eventually

  But  is a proper part of  and is in normal
  form, which is impossible.  Hence 
  are -distinct inhabitants of  all of the form , which are not substitution instances of each other.\qed
\nobreak

  (s_1,\dots,s_ns_1, s_2,\dots .,s_n\lambda^\tau\lambda^\ast\lambda2\)}. 
\item , ,
  , ,
  ,  and 
  satisfy \emph{(\square,\square\lambda\rbox{AUT-68}\lambda\rbox{AUT-QE}\lambda\rbox{PAL}\)}.
\end{enumerate}
\end{lem}
\vfill\eject

\proof\hfill

\begin{enumerate}[(i)]
\item By Lemma \ref{L:eleven}(i), (ii) with .
\item For , by Lemma \ref{L:eleven}(ii), with .  For the others with .
\item By Lemma \ref{L:eleven}(iv) with .\qed
\end{enumerate}

  \noindent Now we can prove the main result in the section.

\begin{thm}\label{T:fourteen} 
  If, in a PTS , \emph{(s_1,\dots ,s_ns_1,\dots ,s_n\in {\mathcal S}\\lambda X\) holds we have, for
  an infinite number of  - distinct terms , which
  are not substitution instances of each other


  Suppose that there is an equivalent .

  As a HPTS has only a finite set of axioms , at least
  some must be derived, in , by (application) and perhaps
  (conversion), (type reduction) and (subject reduction) from

  where  and .
  By the equivalence of  and  also:

  So by correctness of types (Lemma \ref{L:five}), for some 

  and by the Generation Lemma (Lemma \ref{L:four}(iii)) we have:

  where .

  Now by the substitution lemma (Lemma \ref{L:two}), (\ref{EQ:three})
  and, if needed, subject reduction (Lemma \ref{L:six}) .

  By Lemma \ref{L:four}(i) this contradicts (\), so  has no
  HPTS equivalent.\qed

\begin{thm}\label{T:fifteen}
  , , ,
  , ,
  , ,
  , , ,
   and  have no equivalent HPTSs.
\end{thm}

\proof By Lemma \ref{L:thirteen} and Theorem \ref{T:fourteen}.\qed

  Note that (\) is not satisfied by any sort  in
   and Coq.  For 
  (\) holds but (\ast,s_2,\dots ,\asts_2,\dots \lambda \\ast\\square\) does not for any 
\vfill\eject

\section{How to Prove (abstraction) and (product)}\label{S:AbsProd}

 \noindent In implicational logic the -introduction rule is
  . The
  hypothesis A in  is cancelled in . This rule is proved in a Hilbert-style system by
  induction on the number of steps in a derivation that allows
  hypotheses. We assume that an hypothesis can be cancelled in the
  previous step (or steps) and use this to show it can be cancelled in
  the next.  In intuitionistic and classical implicational logic three
  cases are needed and each requires the Hilbert-style system to have
  a particular axiom or theorem.

  If the hypothesis  is itself the step in the deduction we need

  If the deduction step is an axiom or another hypothesis than  we need

  If the deduction step comes  by modus ponens from

  we need 


  Note that the three theorems we require represent the simple types
  of the combinators ,  and  (when
   is replaced by ).

  In illative combinatory logic, the introduction rule for 
  (restricted generality) is , where  is a constant,  and  is the hypothesis being cancelled. In the
  proof of this rule in a Hilbert-style system,(see Bunder[2]), the
  first two cases are similar to those for the proof of implicational
  introduction. The third is the case where  is
  derived from  and .
  Again, by induction, we assume that the -introduction step can
  be applied to the previous steps.

  The axioms of the Hilbert-style system, when rewritten with  for 
  are:

  where  represent conditions involving  on  and
  .

  These are type assignment statements for ,  and
  .

  It might be thought that this same technique could be employed for
  PTSs, using type assignment statements for ,  and
  , of the form  etc and with hypotheses of the form .  This however may
  not work.

  If we have a PTS with  and can prove  and/or , perhaps with ,
  , it may be that (product) and (abstraction) cannot be
  applied because  for any .

  This does not mean that  can never be cancelled. We may
  obtain:

  where  cannot be cancelled, as, even if we have

   and  may not be in 
  for any . However if

  so that , we can cancel  to give


  This PTS therefore does have theorems not in , but it
  is hard to determine the HPTS corresponding to it.

\section{Supersorted PTSs}\label{S:SupSort}

 \noindent PTSs that have equivalent HPTSs are  and
  Coq (both versions), but these belong to a larger class
  that has the following property{:}

\begin{defi}[Supersorted]\label{D:supers}

  A PTS is said to be \emph{supersorted} if:

\end{defi}

  For supersorted PTSs (abstraction) can be simplified.

\begin{thm}\label{T:sixteen} In every supersorted PTS
  (abstraction) can be replaced by:

\end{thm}

\proof If

  by Lemmas \ref{L:seven}(ii) and \ref{L:five} we have, for some
  :

  If the PTS is supersorted we have, for some ,  in the latter () case, and so the result of the
  former case by Lemma \ref{L:seven}(i).

  Hence by (product) and supersortedness we have, for some ,

  and by (abstraction) we have


  For a supersorted PTS  we define a corresponding HPTS
  , which in Theorem \ref{T:twentyeight} is shown to be
  equivalent to .

\begin{defi}[{Corresponding HPTS}]\label{D:ssHPTS}

  If  is a supersorted PTS with specification  the \emph{corresponding} HPTS
   has specification , with as members of  the following
  theorems of :

\noindent{\bf Axiom \hbox to21 pt{\hfil}} for .\hfill

\noindent{\bf Axiom \hbox to21 pt{\hfil}}.\hfill

\noindent{\bf Axiom \hbox to21 pt{\hfil}}.\hfill

\noindent{\bf Axiom \hbox to21.6 pt{\hfil}}.\hfill

\noindent\phantom{\bf Axiom \hbox to21 pt{\hfil}}\hfill 

\noindent\phantom{\bf Axiom \hbox to21 pt{\hfil}}\hfill

\noindent\phantom{\bf Axiom \hbox to21 pt{\hfil}}.\hfill

\noindent and additional axioms of  generated by (I) and
  (II):
\begin{enumerate}[(I)]
\item If ,  and  is
  obtained from  by replacing any second occurrence of an  in
   by any  not in , then if  for , . Any conditions on  not required in the proof of  are not
  part of the new axiom.
\item If  and  satisfies
  , then

\end{enumerate}
\end{defi}

 \noindent{\bf Note.}\ The  in Axioms I1, K1 and S1 are
  sort variables that can be replaced by arbitrary elements of
  . In the axioms generated by (I) and (II) there are
  restrictions on the sorts that can be substituted for such variables
  based on the PTS provability of the judgements mentioned.

  Given a PTS X, we will assume below that  is
  the corresponding HPTS.

\begin{thm}\label{T:seventeen} If, for a supersorted PTS,  then .
\end{thm}

\proof By induction on the derivation of  .

  If  is one of the axioms of , , I1,
  K1 or S1, or is generated by (I) or (II), we have .

  The (application) and (conversion) cases follow from the induction
  hypothesis.

  The (subject reduction) and (type reduction) cases follow from the
  induction hypothesis and Lemma \ref{L:six}.\qed

\begin{lem}\label{L:eighteen} In a HPTS corresponding to a supersorted PTS,
\begin{enumerate}[\em(i)]
\item If  then there is an  such that .
\item If , there is an 
  such that .
\end{enumerate}
\end{lem}

\proof\hfill

\begin{enumerate}[(i)]
\item If  this follows by supersortedness.

  \noindent If  and , we have
   by Theorem \ref{T:seventeen} and , for
  some , by Lemma \ref{L:five}.

  \noindent Hence  by (I).

\item If , by Theorem \ref{T:seventeen} and Lemma
  \ref{L:four}(iv) and (iii) we have, for some ,

  (abstraction) and (II) then give the result.\qed
\end{enumerate}

 \noindent We also need an extension of  that allows hypotheses.

\begin{defi}[]\label{D:lam}
  If  is a PTS,  has all the postulates of
  , also with nonempty contexts, and the (start) and
  (weakening) rules of .
\end{defi}

\begin{lem}\label{L:nineteen}
  
\end{lem}

\proof Immediate because in a derivation of  no
  (start) or (weakening) rule can be used. No nonempty context can be
  emptied in .\qed

  The extra axioms of  generated by (I) we will need in
  the proof of the Correctness of Types Lemma for  (if
   then  for some ).

  Those generated by (II) we need in the proof of (abstraction) to
  show that, if we have , we also
  have for ,  and ,
  where the derivation of the latter is no longer than the derivation
  of . The ``no longer than" is
  needed for proof by induction to work.

  Many of the axioms are, in a sense, superfluous.  We can for
  example, prove axioms  and  (below) from Axiom
  K1 and Axiom  from Axioms  and .  However,
  using fewer axioms can mean that the derivation of a
   is longer than
  that of .

  To illustrate that the axioms, generated by (I) and (II) above, form
  finite sets, we list all the ones generated by Axiom (
  below is such that ). There are another six
  I axioms, another sixteen K axioms and twentynine more S axioms.

\noindent{}\quad .

\noindent{}\quad .

\noindent{}\quad .

\noindent{}\quad .

\noindent{}\quad .

\noindent{}\quad .

\noindent{}\quad .

\noindent{}\quad .

\noindent{}\quad .

\noindent{}\quad .

  The axiom required by (I) for  and  is ,
  for  and , for ,  and for
   the instance of  where . The axiom required
  by (II) for  is , for  is , for  is
  , for  is , for  is , for  is
   and for  is .

  Each axiom is an axiom scheme in the sense that it is an axiom for
  all  for which it is provable in . Thus most
  axioms (not 6) have some restrictions, other than , for example  in . Some of these restrictions will appear in
  (the proofs of) some of the lemmas for  below.

  We will show later that in , for a suitable
  , (product) and (abstraction) are admissible and that the
  theorems of  are exactly those of  and
  .

\section{The Correctness of Types Lemmas for \texorpdfstring{}{lambda X{h+}}}\label{S:CTL}

 \noindent To state and prove some preliminary lemmas we need some
  definitions.

\begin{defi}[major premise chain]\label{D:mpc}
  A \emph{major premise chain (mpc)} in a derivation is a sequence of
  judgements starting with one formed by a (start) rule or an
  axiom. The remaining judgements of the chain are obtained by
  (weakening), (application) or (conversion), with the previous
  judgement as major premise, or by (subject reduction) or (type
  reduction). The minor premises in (weakening), (application) and
  (conversion) rules for which the major premises are in an mpc, will
  be called the \emph{minor premises attached to the mpc}.
\end{defi}

  The final judgement of an mpc that is not a proper part of a larger
  mpc, must be the final judgement in a derivation, a judgement that
  is the premise for a (start) rule or the minor premise in a
  (weakening), (application) or (conversion) rule.

  Any derivation is therefore made up of linked mpcs.


\begin{defi}\label{D:losh}
   An mpc is said to be \emph{long} if it starts with an axiom of the form

  where ,  is one of  or is formed by
  application from (some of)  and the mpc has at least
   (application) steps and (subject reduction) steps that reduce
  all of the   redexes. An mpc is \emph{short}
  otherwise.

  A derivation is \emph{short} if it has no long mpcs and \emph{long}
  otherwise.
\end{defi}

\begin{defi}[Application Length - alength]\label{D:alength}
  The \emph{application length} or \emph{alength} of a derivation is
  its number of (application) steps, where steps in identical minor
  premises in the derivation, are counted only once.

  A derivation of lesser alength than another will be called
  \emph{ashorter}, one of greater alength \emph{alonger}.
\end{defi}

 \noindent {\bf Note.}\ One derivation of a judgement may be shorter
  (in length) than another without being short.

\begin{lem}\label{L:twenty} If the final mpc in a derivation of

  is long, it starts with an axiom of the form
  \emph{(\ref{EQ:eighteen})} and the (subject reduction) step that
  reduces the  redex comes directly after the th
  (application) step, then that derivation of
  \emph{(\ref{EQ:nineteen})} can be replaced by an ashorter one.
\end{lem}

\proof This has to be proved for each of the axioms of 
  that is of this form. We will prove it for Axiom {S8}, below, the
  proofs for other axioms are similar.
\bigskip

\noindent{\bf{S8}}\quad 


\noindent\phantom{\bf{S8}[}\quad
\hfill

\noindent\phantom{\bf{S8}}\quad


\noindent\phantom{\bf{S8}[}\quad
.
\bigskip

 \noindent Let the minor premises in the six (application) steps
 involving  to  in the long derivation of
 (\ref{EQ:nineteen}) be, for :

  where , , , , ,
   and .

  Then for some ,  and
   and for some , ,  and .

  Note that as contexts can only grow, each  for  is an initial segment of 

  Now by (weakening), (subject reduction), (type reduction) and just
  three (application) steps we get from three of these minor premises:

  which, as , gives
  (\ref{EQ:nineteen}).

  We now have a new derivation of (\ref{EQ:nineteen}), which, given
  that any (application)s in the two uses of 
  are counted only once, has fewer (application)s, and so is ashorter
  than, the old derivation of (\ref{EQ:nineteen}).

\begin{lem}[Shortness Lemma for HPTS]\label{L:twentyone}
  Every valid judgement in a HPTS has a short derivation.
\end{lem}

\proof We prove this by showing that for every long
  derivation there is an ashorter derivation of the same judgement.

  Assume that the following is the part of a long mpc, in a long
  derivation, up to the  reduction, together with the
  minor premises used in the n (application) steps.


{\small
\vspace{-3mm}


\vspace{-2mm}\hspace{3cm}\hbox{\vrule height .4pt width 10cm}

\hspace {6.2cm}\vdots

\vspace{-2mm}\hspace{3cm}\hbox{\vrule height .4pt width 10cm}

\hspace {6.2cm}\vdots

\vspace{-2mm}\hspace{3cm}\hbox{\vrule height .4pt width 10cm}

\vspace{-2mm}\hspace{3cm}


\vspace{-2mm}\hspace{3cm}\hbox{\vrule height .4pt width 10cm}

}

 \noindent Here  is an axiom of
  the form (\ref{EQ:eighteen}) with  made up of (some of)
  ,  and  The
  first, second and th of the  or more (application)s and the
  (subject reduction) contracting the  redex are
  explicitly shown. The steps after the th (application) only alter
   by reducing it, so steps can be permuted so that the  reduction takes place straight after the th (application)
  step as follows:
\medskip

{\small
\hspace {3.6cm}

\vspace{2mm}\hspace{3.6cm}\hbox{\vrule height .4pt width 5.2cm}\vspace{-2mm}

\vspace{-2mm}\hspace{3cm}\hbox{\vrule height .4pt width 10cm}

\vspace{-2mm}\hspace{4cm}\hbox{\vrule height .4pt width 9cm}

\hspace {7cm}\vdots

}

 \noindent This new derivation is no alonger than the original, but
  the part up to 
  is long and can be replaced, by Lemma \ref{L:twenty}, by an ashorter
  derivation, so the whole derivation becomes ashorter. (If the
  derivation had identical mpcs to the above, which were all minor
  premises in the same mpc, all would have to be altered as above to
  ensure that the new derivation is not alonger than the old.)

  In the remaining lemmas and theorems we use a different measure of
  length of a derivation, where ``similar" subderivations are counted
  only once.

\begin{defi}[Similar]\label{D:sim}
  Two derivations are said to be  \emph{similar} if they are identical
  or one, in its final mpc, starts with an axiom of  of
  the form (\ref{EQ:eighteen}), and the other differs only in that its
  final mpc starts with an axiom of  generated from the
  other by one or more applications of (II).
\end{defi}

  We now define:

\begin{defi}[Similarity Length - slength]\label{D:simlen}
  The  \emph{similarity length} (or  \emph{slength}) of a derivation is
  given by:
\begin{enumerate}[(i)]
\item the number of (application) steps,
\item the number of (conversion), (start) and (weakening) steps.
\end{enumerate}
  Similar derivations ending in the two premises of a (weakening)
  step, are counted only once.

  A derivation of lesser slength than another will be said to be
  \emph{sshorter} and one of greater slength as  \emph{slonger}.
\end{defi}

\begin{lem}\label{L:twentytwo} Given, for , a
  short derivation of:

  there is, for some , a derivation, no longer or
  slonger than that of \emph{(\ref{EQ:twenty})}, of

\end{lem}

\proof Consider the first judgement in the final mpc in a short
  derivation of (\ref{EQ:twenty}). This cannot be an axiom of  or be formed by a (start) rule, so it is an axiom of  of the form (\ref{EQ:eighteen}), where 
  and .

  If in this mpc we replace this axiom by the one generated from it by
  (II), then using exactly the same steps and minor premises we obtain
  a derivation of (\ref{EQ:twentyone}) of the same length.

  In this final mpc there are no (weakening) steps in which the
  premises are similar, until perhaps after the last (application)
  step, as, until then, no type can be in . If, at that
  stage, (\ref{EQ:twenty}) is formed by one or more (weakening) steps
  (and perhaps (subject reduction)) from  and similar minor premises such as , these are counted only once each in the
  slength. In the derivation obtained by changing the axiom, the above
  derivations remain similar and so the slength of the derivation
  remains the same.\qed

\begin{lem}[Correctness of Types for HPTS]\label{L:twentythree}
  If X is supersorted and

  then, for some ,  or there is a short
  derivation, of slength no more than that of a short derivation of
  \emph{(\ref{EQ:twentytwo})}, of

\end{lem}

\proof By induction on the number k, of judgements in the final mpc of
  a short derivation of , where .

If k=1 and (\ref{EQ:twentytwo}) comes by a (start) rule from

  where  and ,
  (\ref{EQ:twentythree}) comes from two copies of
  (\ref{EQ:twentyfour}) and (weakening). The two derivations of
  (\ref{EQ:twentyfour}) are counted only once, so this derivation of
  (\ref{EQ:twentythree}) is no slonger than that of
  (\ref{EQ:twentytwo}).

  If (\ref{EQ:twentytwo}) is an axiom we have (\ref{EQ:twentythree})
  by (I) or by supersortedness.

  We now assume k  2.

  If (\ref{EQ:twentytwo}) comes from  and
  , by (weakening), where ,  and , these
  derivations are both counted in the slength of the derivation of
  (\ref{EQ:twentytwo}). We have, by the induction hypothesis, , by a derivation no slonger than that of , for some  and we obtain
  (\ref{EQ:twentythree}) by (weakening), by a derivation that is no
  slonger than that of (\ref{EQ:twentytwo}).

  If (\ref{EQ:twentytwo}) comes from , by
  (subject reduction), we have (\ref{EQ:twentythree}) by a derivation
  no slonger than that of (\ref{EQ:twentytwo}).

  If (\ref{EQ:twentytwo}) comes from , by
  (type reduction), we have  by the induction
  hypothesis and (\ref{EQ:twentythree}) by (subject reduction), by a
  derivation no slonger than that of (\ref{EQ:twentytwo}).

  If (\ref{EQ:twentytwo}) comes from , by
  (conversion), we have (\ref{EQ:twentythree}) by a derivation
  sshorter than that of (\ref{EQ:twentytwo}).

  If (\ref{EQ:twentytwo}) comes from  and , where  and
  , by (application), we have by the induction
  hypothesis,  for some , by a derivation no slonger than that of . Then by Lemma \ref{L:twentytwo} we have
  , for some  by a derivation no slonger than that of . Then by (application) using  we have (\ref{EQ:twentythree}) by a derivation no
  slonger than that of (\ref{EQ:twentytwo}).\qed

\begin{lem}[Start Lemma for HPTS]\label{L:twentyfour}
  If 

  then, for some ,

\end{lem}

\proof By an easy induction on the derivation of (\ref{EQ:twentyfive}).\qed

\section{The Equivalence Results}\label{S:ER}

\begin{lem}\label{L:twentyfive}
  If  is supersorted, (abstraction) is
  admissible in 
\end{lem}

\proof If  is supersorted we prove that if

  then

  by induction on the slength of a short derivation of (\ref{EQ:twentysix}).

\noindent{\bf Case 1.}\ (\ref{EQ:twentysix}) comes by (start) (and (type reduction)) from

  where  and .

  By Axiom I1 and (application) (and (type reduction)) we have (\ref{EQ:twentyseven}).

\noindent{\bf Case 2.} (\ref{EQ:twentysix}) comes by (weakening) (and reduction) from

  then by the Correctness of Types Lemma (Lemma \ref{L:twentythree}) or
  supersortedness, for some .

  and (\ref{EQ:twentyseven}) follows after three (applications) applied to Axiom K1 (and
  reduction).

\noindent{\bf Case 3.}\ (\ref{EQ:twentysix}) comes by (conversion) (and reduction) from

  By the induction hypothesis and (subject reduction) we have:

  and

  where .

  We have by Lemma \ref{L:twentythree} applied to (\ref{EQ:twentysix}), for some ,

  and, by supersortedness  for some
  , so by Axiom  and (subject reduction),

  hence by (\ref{EQ:twentyeight}) and (conversion) we have
  (\ref{EQ:twentyseven}).

\noindent{\bf Case 4}\ (\ref{EQ:twentysix}) comes by (application) (and reduction) from

  and

  where  and .

  By the Correctness of Types lemma we have for some , by a derivation no slonger than that of (\ref{EQ:twentynine}):

  now by Lemma \ref{L:twentytwo} we have for some , by a
  derivation no slonger than that of (\ref{EQ:twentynine}), and so sshorter than that
  of (\ref{EQ:twentynine}):

  Now by the induction hypothesis applied to (\ref{EQ:twentynine}),
  (\ref{EQ:thirty}) and (\ref{EQ:thirtytwo}) we have:



  also by the Correctness of Types Lemma applied to
  (\ref{EQ:thirtyfour}) we have for some 

  and by Lemma \ref{L:twentytwo} for some 

  now by Axiom S1, , (obtained as in Case 3)
  (\ref{EQ:thirtysix}), (\ref{EQ:thirtyfive}), (\ref{EQ:thirtythree}),
  (\ref{EQ:thirtyfour}) and five (application)s, (subject reduction)
  and (type reduction) give (\ref{EQ:twentyseven}). (Note that in
  Axiom S1  and  (here  and ) can be
  arbitrarily chosen in a supersorted PTS).\qed

\begin{lem}\label{L:twentysix}
  If  is supersorted (product) is admissible in .
\end{lem}

\proof If , 
 and , by Lemma \ref{L:twentyfive},

  so by Axiom II1, (application) and (subject reduction) we have


  Lemmas \ref{L:twentyfive} and \ref{L:twentysix} show that a theorem that can be proved in , using hypotheses, (abstraction) and (product), can also be
  proved in . So:

\begin{thm}\label{T:twentyseven}
  If  is supersorted it is equivalent to 
  in that they have the same valid judgements.
\end{thm}

\proof By Theorem \ref{T:seventeen},  is a subsystem of
  . The additional rules of  are rules of
  , so  is a subsystem of . The
  extra rules of  have been shown to be admissible in
   in Lemmas \ref{L:twentyfive} and \ref{L:twentysix},
  so  and  have the same valid
  judgements.\qed

\begin{thm}\label{T:twentyeight}
  If  is supersorted  and  are
  equivalent in that they have the same theorems.
\end{thm}

\proof It follows from Theorem \ref{T:twentyseven} that 
  and  have the same valid judgements with empty
  contexts i.e. theorems and so by Lemma \ref{L:nineteen} that  and  have the same theorems.\qed

\section {Axioms I, K, S and \texorpdfstring{}{Pi} as Types}\label{S:IKSPi}

 \noindent Axioms I1, K1 and S1 can be rewritten in terms of type
  free combinators (allowing -reduction) as:
\medskip

\noindent{\bf Axiom \hbox to21 pt{\hfil}}.
\smallskip

\noindent{\bf Axiom \hbox to21 pt{\hfil}}.
\smallskip

\noindent{\bf Axiom \hbox to21 pt{\hfil}}.

\noindent\phantom{\bf Axiom \hbox to21 pt{\hfil}}.
\medskip

 \noindent These give the standard types of the combinators (writing
   for  when ):
{\small}
  or, as a special case
\small{}
  If , where , were represented as
   (as it is in ), Axiom  represents the type
  for  (or ):


\section{Identifying \texorpdfstring{}{lambda} and 
\texorpdfstring{}{Pi}}\label{S:LamPi}

  \noindent In the de Bruijn AUTOMATH systems  and  are
  usually identified.  Kameraddine has studied the effect of allowing
  -reductions in the (former)  terms in [5].  Doing this
  Axiom {I1} becomes:

  and similarly for the other axioms.  If we write the type in terms
  of combinators we can get (depending on the algorithm)


\section{Conclusion}\label{S:Conc}

  \noindent We have shown that PTSs come in at least three categories:
  those satisfying (s\s\) that have only a trivial
  equivalent HPTS and supersorted PTSs, such as ,
  that have a nontrivial equivalent HPTS. The standard PTSs from the
  literature that we considered all fit into these categories.


\begin{thebibliography}{5}
\bibitem{barendregt} Barendregt, H.P.
{\em Lambda calculi with types}, pp. 117-309 in Handbook of Logic in
Computer Science, vol 2 of Oxford Science Publications, Oxford
University Press, New York, 1992. Theoretical Computer Science,
169:3-21, 1996.\\
\bibitem{bunder-dekkers} Bunder, M.W. A
deduction theorem for restricted generality. {\em Notre Dame Journal of
Formal Logic}, 14:341-346, 1973.\\
\bibitem{bunder} Bunder,
M.W. and Dekkers, W.J.M. Equivalences between pure type systems and
systems of illative combinatory logics. {\em Notre Dame Journal of Formal
Logic}, 46:181-205, 2005.\\
\bibitem{geuvers} Geuvers,
H. {\em Logics and Type Systems}, Thesis University of Nijmegen,
1993.\\
\bibitem{kamareddine} Kamareddine, F. Typed
-calculi with one binder. {\em Journal of Functional
Programming}, 15(5):771-796, 2005.
\end{thebibliography}

\end{document}
