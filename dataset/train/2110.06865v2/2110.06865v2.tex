\pdfoutput=1

\documentclass[11pt]{article}
\PassOptionsToPackage{dvipsnames}{xcolor}

\usepackage[]{acl}

\usepackage{times}
\usepackage{latexsym}

\usepackage[T1]{fontenc}

\usepackage[utf8]{inputenc}

\usepackage{microtype}

\usepackage{algorithm, algorithmic, letltxmacro}
\usepackage{amsmath}
\usepackage{amssymb}
\usepackage{amsfonts}
\usepackage{pgfplots}
\usetikzlibrary[angles, arrows.meta, decorations.pathmorphing, decorations.pathreplacing, backgrounds, intersections, positioning, fit, petri, graphs, math, external, calc, shapes, shapes.geometric, decorations.text, shadings, pgfplots.groupplots]
\usepackage{graphicx}
\usepackage{tikz}
\usepackage{tikz-dependency}
\usepackage{subcaption}
\usepackage{multirow}
\usepackage{booktabs}
\usepackage{mathtools}
\usepackage[dvipsnames]{xcolor}
\usepackage{enumitem}
\usepackage{mathpartir}
\usepackage{mathabx}
\usepackage{tablefootnote}
\usepackage{colortbl}
\usepackage{arydshln}
\usepackage{CJKutf8}
\usepackage{forest}
\usepackage{latexsym}
\usepackage{mathabx}
\usepackage{stmaryrd}
\usepackage{wasysym}
\usepackage{fontawesome}
\forestset{
dg edges/.style={for tree={parent anchor=south, child anchor=north,align=center,base=bottom,where n children=0{tier=word,edge=dotted,calign with current edge}{}}},
}

\setlist{nosep}

\definecolor{brickred}{HTML}{b92622}
\definecolor{midnightblue}{HTML}{005c7f}
\definecolor{salmon}{HTML}{f1958d}
\definecolor{burntorange}{HTML}{f19249}
\definecolor{junglegreen}{HTML}{4dae9d}
\definecolor{forestgreen}{HTML}{499c5e}
\definecolor{pinegreen}{HTML}{3d8a75}
\definecolor{seagreen}{HTML}{6bc1a2}
\definecolor{limegreen}{HTML}{97c65a}
\newcommand{\white}[1]{\textcolor{white}{#1}}
\newcommand{\brickred}[1]{\textcolor{brickred}{#1}}
\newcommand{\midnightblue}[1]{\textcolor{midnightblue}{#1}}
\newcommand{\salmon}[1]{\textcolor{salmon}{#1}}
\newcommand{\junglegreen}[1]{\textcolor{junglegreen}{#1}}
\newcommand{\forestgreen}[1]{\textcolor{forestgreen}{#1}}
\newcommand{\pinegreen}[1]{\textcolor{pinegreen}{#1}}
\newcommand{\seagreen}[1]{\textcolor{seagreen}{#1}}
\renewcommand{\aboverulesep}{0pt}
\renewcommand{\belowrulesep}{0pt}
\newcommand*{\rulefiller}{\arrayrulecolor[gray]{0.95}\specialrule{\heavyrulewidth}{0pt}{-\heavyrulewidth}\arrayrulecolor{black}}
\newcommand{\suda}{\textsuperscript{\faStarO}}
\newcommand{\huawei}{\textsuperscript{\faMoonO}}
\newcommand{\damo}{\textsuperscript{\faSunO}}
\newcommand{\github}{\faGithub}

\DeclareDocumentCommand{\trapezoid}{O{2.0} O{1.0} O{0.5} m m m}{
  \begin{scope}[scale=0.9,thick]
    \draw[anchor=mid] (0, 0) -- (0, -{#2}) node[below=7.5pt,anchor=base] {\footnotesize\ensuremath{#4}} -- ({#1}, -{#2}) node [below=7.5pt,anchor=base] {\ensuremath{#5}} -- ({#1}, -{#3}) -- cycle;
    \draw node[] at () {\footnotesize\ensuremath{#6}};
  \end{scope}
}
\DeclareDocumentCommand{\square}{O{0.5} O{0.5} m m m}{
  \begin{scope}[scale=0.9,thick]
    \draw[anchor=mid] (0, -{#2}) node[below left=7.5pt and 3pt,anchor=base] {\footnotesize\ensuremath{#3}} -- ({#1}, -{#2}) node (square) [below=7.5pt,anchor=base] {\footnotesize\ensuremath{#4}} -- ({#1}, 0) --  (0, 0) -- cycle;
    \draw node[] at () {\footnotesize\ensuremath{#5}};
  \end{scope}
}
\DeclareDocumentCommand{\lefttriangle}{O{0.5} O{0.5} m m m}{
  \begin{scope}[scale=0.9,thick]
    \draw[anchor=mid] (0, -{#2}) node[below left=7.5pt and 3pt,anchor=base] {\footnotesize\ensuremath{#3}} -- ({#1}, -{#2}) node [below=7.5pt,anchor=base] {\footnotesize\ensuremath{#4}} -- ({#1}, 0) -- cycle;
    \draw node[] at () {\footnotesize\ensuremath{#5}};
  \end{scope}
}
\DeclareDocumentCommand{\righttriangle}{O{0.5} O{0.5} m m m}{
  \begin{scope}[scale=0.9,thick]
    \draw[anchor=mid](0, 0) -- (0, -{#2}) node[below=7.5pt,anchor=base] {\footnotesize\ensuremath{#3}} -- ({#1}, -{#2}) node [below right=7.5pt and 3pt,anchor=base] {\footnotesize\ensuremath{#4}} -- cycle;
    \draw node[] at () {\footnotesize\ensuremath{#5}};
  \end{scope}
}


\title{Semantic Role Labeling as Dependency Parsing: Exploring \\
Latent Tree Structures Inside Arguments}


\author{
    \textbf{Yu Zhang}\suda,
    \textbf{Qingrong Xia}\suda\huawei,
    \textbf{Shilin Zhou}\suda,
    \textbf{Yong Jiang}\damo,
    \textbf{Guohong Fu}\suda\thanks{Corresponding author},
    \textbf{Min Zhang}\suda \\
    \suda Institute of Artificial Intelligence, School of Computer Science and Technology, \\
    Soochow University, Suzhou, China\\
    \huawei Huawei Cloud, China \\
    \damo DAMO Academy, Alibaba Group, China \\
    \texttt{\{yzhang.cs,slzhou.cs\}@outlook.com; xiaqingrong@huawei.com} \\
    \texttt{yongjiang.jy@alibaba-inc.com; \{ghfu,minzhang\}@suda.edu.cn}
}

\date{\today}

\begin{document}
\maketitle
\begin{abstract}
    Semantic role labeling (SRL) is a fundamental yet challenging task in the NLP community.
    Recent works of SRL mainly fall into two lines: 1) BIO-based; 2) span-based.
    Despite ubiquity, they share some intrinsic drawbacks of not considering internal argument structures, potentially hindering the model's expressiveness.
    The key challenge is arguments are flat structures, and there are no determined subtree realizations for words inside arguments.
    To remedy this, in this paper, we propose to regard flat argument spans as latent subtrees, accordingly reducing SRL to a tree parsing task.
    In particular, we equip our formulation with a novel span-constrained TreeCRF to make tree structures span-aware and further extend it to the second-order case.
    We conduct extensive experiments on CoNLL05 and CoNLL12 benchmarks.
    Results reveal that our methods perform favorably better than all previous syntax-agnostic works, achieving new state-of-the-art under both \emph{end-to-end} and \emph{w/ gold predicates} settings.
\end{abstract}

\section{Introduction}\label{sec:intro}

Semantic role labeling (SRL) is a fundamental yet challenging task in the NLP community, involving predicate and argument identification, as well as semantic role classification.
As SRL can provide informative linguistic representations, it has been widely adopted in downstream tasks like question answering \cite{berant-etal-2013-semantic,yih-etal-2016-value}, information extraction \cite{christensen-etal-2010-semantic,lin-etal-2017-neural}, and machine translation \cite{liu-gildea-2010-semantic,bazrafshan-gildea-2013-semantic}, etc.

Recent works of SRL mainly fall into two lines: 1) BIO-based; 2) span-based.
The former views SRL as a sequence labeling task \cite{zhou-xu-2015-end,strubell-etal-2018-lisa,shi-etal-2019-simple}.
For each predicate, each token is tagged with a label starting with BIO prefixes indicating if it is at the \textbf{B}eginning, \textbf{I}nside, or \textbf{O}utside of an argument.
The latter \cite{he-etal-2018-jointly,ouchi-etal-2018-span,li-etal-2019-dependency}, in contrast, opts to jointly predict all predicate and argument span pairs using a span-graph formulation.

\begin{figure}
    \centering
    \begin{dependency}
        \begin{deptext}[column sep=0.18cm,font=\small]
            ... \& in \& other \& European \& markets \& ...  \& \emph{\brickred{closed}} \& ... \\
        \end{deptext}
        \depedge[edge vertical padding=0.3ex, hide label, edge height=4ex, thick]{7}{2}{\small NULL}
        \depedge[edge vertical padding=0.3ex, hide label, edge height=2ex, thick]{5}{3}{\small NULL}
        \depedge[edge vertical padding=0.3ex, hide label, edge height=1ex, thick]{5}{4}{\small NULL}
        \depedge[edge vertical padding=0.3ex, hide label, edge height=3ex, thick]{2}{5}{\small NULL}
        \node (a01l) [below left of = \wordref{1}{2}, xshift=2.5ex, yshift=2.5ex] {};
        \node (a01r) [below right of = \wordref{1}{5}, xshift=-0.8ex, yshift=-0.3ex] {};
        \draw [fill=brickred!25, thick, rounded corners=1mm] (a01l) rectangle (a01r);
        \draw [draw=none] (a01l) -- node[] {\small\texttt{AM-LOC}} (a01r);
    \end{dependency}
    \caption{An argument example (below) and its related subtree structure (above) for the predicate ``\brickred{\emph{closed}}''.}
    \label{fig:example}
\end{figure}

Despite ubiquity, there are some drawbacks that limit the expressiveness of the two methods.
First, framing predicate-argument structures as a BIO-tagging scheme is less effective as it lacks explicit modeling of span-level representations, so that long adjacencies of argument phrases can be ignored \cite{cohn-blunsom-2005-semantic,jie-lu-2019-dependency,zhou-etal-2020-latent,xu-etal-2021-better}.
Second, span-based method seeks to pick very few (typically 10\%) positive examples from  candidate predicate-argument pairs, thus suffering from severe class imbalance problem \cite{li-etal-2021-syntax}.
To alleviate this issue, span-based method relies on heavy pruning \cite{he-etal-2018-jointly} to reduce the searching space, potentially impairing the performance.


Meanwhile, both formulations share some common flaws in terms of lacking explicit modeling of internal argument structures, which appear to be beneficial to SRL.
Taking Fig.~\ref{fig:example} as an example, internal dependencies of words (``in other European markets'') inside the span provide strong clues for recognizing it as a locative modifier (``\texttt{AM-LOC}'') of the predicate ``\emph{\brickred{closed}}''.
Besides, the predicate-argument relation can be naturally reflected by the dependency from the predicate to the span headword (``\emph{\brickred{closed}}  in''), and we can properly recognize the argument span boundaries by retrieving all descendants of the subtree.
Such observations have motivated many attempts on utilizing relations inside arguments \cite[\emph{inter alia}]{gildea-hockenmaier-2003-identifying,johansson-nugues-2008-dependency,johansson-nugues-2008-effect,xia-etal-2019-syntax,li-etal-2019-dependency}.
However, stuck on the fact that span-style SRL has no determined internal structure realizations, existing works have to resort to making use of external human-annotated syntax knowledge to bridge the gap \cite{shi-etal-2020-semantic,li-etal-2021-syntax}.



Our main goal in this work is to explicitly take internal argument structures into account meanwhile keeping our framework \emph{end-to-end}.
To this end, we propose to model flat arguments as latent subtrees, thus paving the way for reducing SRL to dependency parsing seamlessly: \emph{we view predicate-argument structures as partially-observed trees where exact subtrees for each argument are not realized yet.}
In this way, we reframe span-style SRL as parsing word-to-word relations by encoding all predicate-argument relations into a unified dependency graph.
Unlike span-based methods \cite{he-etal-2018-jointly}, a dependency graph contains no more than  possible dependencies, so that the class imbalance issue can be side-stepped effortlessly.
Specifically, we make use of TreeCRF \cite{eisner-2000-bilexical,zhang-etal-2020-efficient}, which provides a viable way for probabilistic modeling of tree structures, to learn the partially-observed trees and marginalize the latent structures out during training.
Unlike canonical TreeCRF, which enumerates all possible trees, in our setting, we have to impose many span constraints to reflect the argument boundaries on subtrees correctly.
To accommodate this, we further design a novel span-constrained TreeCRF to adapt it to our learning procedure, which explicitly prohibits invalid edges across different arguments as well as multi-head subtrees \cite{nivre-etal-2014-squibs,zhang-etal-2021-adapting}.

There are further advantages to our reduction.
Conversion to tree structures enables us to easily conduct global optimization \cite{eisner-1996-three,mcdonald-etal-2005-online} in polynomial time, which has already been shown to often lead to improved results and more meaningful predictions \cite{toutanova-etal-2008-global, tackstrom-etal-2015-efficient,fitzgerald-etal-2015-semantic,li-etal-2020-structured} compared to local unconstrained methods.
On the other hand, by drawing on the experience in the parsing literature, we can further extend our method to some well-studied high-order methods \cite{mcdonald-pereira-2006-online} without any obstacle.
We experiment with sibling factors in this work and find significant gains, in line with many parsing works \cite{zhang-etal-2020-efficient,fonseca-martins-2020-revisiting}.
Our contributions can be summarized as follows:\footnote{Our code is publicly available at \url{https://github.com/yzhangcs/crfsrl}.}
\begin{itemize}[leftmargin=11pt]
    \item Aware of the benefits of internal argument structures, we propose to model flat argument spans as latent subtrees, thereby reducing SRL to dependency parsing seamlessly.
    \item We propose a novel span-constrained TreeCRF to learn the converted trees and further extend it to the second-order case.
    \item Experiments on CoNLL05 and CoNLL12 benchmarks reveal that our proposed methods outperform existing works significantly, achieving new state-of-the-art results under the syntax-agnostic setting.
\end{itemize}


















\section{Overview}\label{sec:srl-as-dep}

\begin{figure*}[th!]
    \begin{subfigure}[b]{0.99\columnwidth}
        \centering
        \begin{dependency}
            \begin{deptext}[column sep=0.2cm,font=\small]
                They\scriptsize \& \brickred{\emph{want}}\scriptsize \& to\scriptsize \& do\scriptsize \& more\scriptsize \&.\scriptsize \\
            \end{deptext}
            \node (a00l) [above left of = \wordref{1}{1}, xshift=0.5ex, yshift=-2.5ex] {};
            \node (a00r) [above right of = \wordref{1}{1}, xshift=-1.2ex, yshift=0.5ex] {};
            \draw [fill=brickred!25, thick, rounded corners=1mm] (a00l) rectangle (a00r);
            \draw [draw=none] (a00l) -- node[] {\small\texttt{A0}} (a00r);

            \node (a10l) [above left of = \wordref{1}{3}, xshift=2.3ex, yshift=-2.5ex] {};
            \node (a10r) [above right of = \wordref{1}{5}, xshift=-1.2ex, yshift=0.5ex] {};
            \draw [fill=midnightblue!25, thick, rounded corners=1mm] (a10l) rectangle (a10r);
            \draw [draw=none] (a10l) -- node[] {\small\texttt{A1}} (a10r);


        \end{dependency}
        \caption{Original structure: arguments of the predicate are located in the upper half-plane of the sentence and do not overlap with each other.}
        \label{fig:origin-srl}
    \end{subfigure}
    \hfill
    \begin{subfigure}[b]{0.99\columnwidth}
        \centering
        \begin{dependency}
            \begin{deptext}[column sep=0.2cm,font=\small]
                They\scriptsize \& \brickred{\emph{want}}\scriptsize \& to\scriptsize \& do\scriptsize \& more\scriptsize \&.\scriptsize \\
            \end{deptext}

            \node (a00l) [above left of = \wordref{1}{1}, xshift=0.5ex, yshift=-2.5ex] {};
            \node (a00r) [above right of = \wordref{1}{1}, xshift=-1.2ex, yshift=0.5ex] {};
            \draw [draw=none, fill=brickred!25, thick, rounded corners=1mm] (a00l) rectangle (a00r);

            \node (a10l) [above left of = \wordref{1}{3}, xshift=2.3ex, yshift=-2.5ex] {};
            \node (a10r) [above right of = \wordref{1}{5}, xshift=-1.2ex, yshift=0.5ex] {};
            \draw [draw=none, fill=midnightblue!25, thick, rounded corners=1mm] (a10l) rectangle (a10r);



            \deproot[edge vertical padding=0.6ex, edge height=10ex, label style={fill=black, thick}, edge style={thick}]{2}{\white{\texttt{PRD}}}
            \depedge[edge vertical padding=0.6ex, edge height=4.2ex, label style={fill=brickred!25, thick}, thick, shorten >=2.7ex]{2}{1}{\texttt{A0}}
            \depedge[edge vertical padding=0.6ex, edge height=4.2ex, label style={fill=midnightblue!25, thick}, thick, shorten >=2.7ex]{2}{4}{\texttt{A1}}
            \depedge[edge vertical padding=0.6ex, edge height=6ex, label style={thick}, thick]{2}{6}{}
            \depedge[<->, edge vertical padding=0.6ex, edge start x offset=-1ex, edge height=2ex, edge slant=2pt, hide label, thick, dotted]{3}{5}{}
            \depedge[<->, edge vertical padding=0.6ex, edge start x offset=0ex, edge end x offset=-0.5ex, edge height=1.5ex, edge slant=4pt, hide label, thick, dotted]{3}{4}{}
            \depedge[<->, edge vertical padding=0.6ex, edge start x offset=-0.5ex, edge end x offset=-1ex, edge height=1.5ex, edge slant=4pt, hide label, thick, dotted]{4}{5}{}

        \end{dependency}
        \caption{Training: convert the predicate-argument structure to a dependency tree with (dotted) latent annotations; non-argument spans are assigned ``'' for distinction.}
        \label{fig:srl-latent-trees}
    \end{subfigure}
    \hfill
    \begin{subfigure}[b]{0.99\columnwidth}
        \centering
        \begin{dependency}
            \begin{deptext}[column sep=0.2cm,font=\small]
                They\scriptsize \& \brickred{\emph{want}}\scriptsize \& to\scriptsize \& do\scriptsize \& more\scriptsize \&.\scriptsize \\
            \end{deptext}

            \deproot[edge vertical padding=0.6ex, edge height=10ex, label style={fill=black, thick}, edge style={thick}]{2}{\white{\texttt{PRD}}}
            \depedge[edge vertical padding=0.6ex, edge height=4.2ex, label style={fill=brickred!25, thick}, thick]{2}{1}{\texttt{A0}}
            \depedge[edge vertical padding=0.6ex, edge height=4.2ex, label style={fill=midnightblue!25, thick}, thick]{2}{4}{\texttt{A1}}
            \depedge[edge vertical padding=0.6ex, edge height=6ex, label style={thick}, thick, dashed]{2}{6}{}
            \depedge[edge vertical padding=0.6ex, edge height=2ex, hide label, thick]{4}{3}{}
            \depedge[edge vertical padding=0.6ex, edge height=2ex, hide label, thick]{4}{5}{}

        \end{dependency}
        \caption{Decoding: realize a tree rooted at the predicate with the arc labeled as ``\texttt{PRD}''; (dashed) arcs labeled as ``'' are discarded.}
        \label{fig:dep-trees}
    \end{subfigure}
    \hfill
    \begin{subfigure}[b]{0.99\columnwidth}
        \centering
        \begin{dependency}
            \begin{deptext}[column sep=0.2cm,font=\small]
                They\scriptsize \& \brickred{\emph{want}}\scriptsize \& to\scriptsize \& do\scriptsize \& more\scriptsize \&.\scriptsize \\
            \end{deptext}

            \node (a00l) [above left of = \wordref{1}{1}, xshift=0.5ex, yshift=-2.5ex] {};
            \node (a00r) [above right of = \wordref{1}{1}, xshift=-1.2ex, yshift=0.5ex] {};
            \draw [fill=brickred!25, thick, rounded corners=1mm] (a00l) rectangle (a00r);

            \node (a10l) [above left of = \wordref{1}{3}, xshift=2.3ex, yshift=-2.5ex] {};
            \node (a10r) [above right of = \wordref{1}{5}, xshift=-1.2ex, yshift=0.5ex] {};
            \draw [fill=midnightblue!25, thick, rounded corners=1mm] (a10l) rectangle (a10r);


            \deproot[edge vertical padding=0.6ex, edge height=10ex, label style={fill=black, thick}, edge style={thick}]{2}{\white{\texttt{PRD}}}
            \depedge[edge vertical padding=0.6ex, edge height=4.2ex, label style={fill=brickred!25, thick}, thick, shorten >=2.7ex]{2}{1}{\texttt{A0}}
            \depedge[edge vertical padding=0.6ex, edge height=4.2ex, label style={fill=midnightblue!25, thick}, thick, shorten >=2.7ex]{2}{4}{\texttt{A1}}
            \depedge[edge vertical padding=0.6ex, edge height=2ex, hide label, thick, dashed]{4}{3}{}
            \depedge[edge vertical padding=0.6ex, edge height=2ex, hide label, thick, dashed]{4}{5}{}
        \end{dependency}
        \caption{Recovery: collapse all (dashed) subtrees governed by the predicate into flat argument spans.}
        \label{fig:collation}
    \end{subfigure}
    \caption{
        Illustration of our SRLTree conversion (Fig.~\ref{fig:origin-srl} and Fig.~\ref{fig:srl-latent-trees}), and its inverse TreeSRL process (Fig.~\ref{fig:dep-trees} and Fig.~\ref{fig:collation}).
        We emphasize the predicate ``\brickred{\emph{want}}'' in the figures for clarity.
        The two arguments with roles ``\texttt{A0}'' and ``\texttt{A1}'' are framed by red and blue rectangles, respectively.
    }
    \label{fig:srl-processes}
\end{figure*}

In span-style SRL, an argument of a predicate corresponds to one word or multiple continuous words.
In the latter case, each word in the argument span is treated as equal, and the internal structure of a multi-word argument, i.e., the relationship between words inside the argument, is usually overlooked due to the lack of corresponding annotations.

In this work, we propose to explicitly model internal structures of multi-word arguments and treat arguments as latent subtrees.
Our approach deals with each predicate separately, and assumes each corresponds to a single-root tree.
Consequently, each argument subtree is attached to the predicate.
During the training process, all possible structures are enumerated and accumulated to compose the argument representation.
While decoding, we seek to find a 1-best tree and recover arguments from the subtrees belonging to the resulting structure.
We highlight four key points.
\begin{enumerate}[leftmargin=15pt,label=\roman*]
    \item Our approach is syntax-agnostic.
          The tree structures are modeled and predicted solely to serve the SRL task without referring to any linguistic syntax knowledge.
    \item The predicate identification subtask is handled as a simple classification procedure.
    \item For argument identification, argument boundaries are decided by subtrees attached to the predicate, and edge labels are used for role disambiguation.
    \item We adopt a consistent scoring architecture for the two subtasks and train them jointly.
\end{enumerate}



\subsection{SRL  Tree Conversion}\label{sec:srl-tree}

Formally, given an input sentence , we first seek to obtain tree structures for each predicate , which are taken as materials of training a parser.
We define a directed acyclic dependency tree  by assigning a head  together with a relation label  to each modifier , where a dummy word  is attached before  as the pseudo root node.\footnote{
    In this work, we assume all dependency trees are \emph{projective}, i.e., without any crossing arcs.
    This property allows us to associate the subtree with its continuous argument span \cite{kong-etal-2015-transforming}.
}

For predicate , the first step is to link  to .
To facilitate predicate identification, we assign a special label \texttt{PRD} (resp.  for non-predicate) to the dependency .
Then, we make all corresponding latent argument subtrees descendants of .
As we showcase in Fig.~\ref{fig:origin-srl}, this takes advantage of the non-overlapping constraint for arguments belonging to the same predicate \cite{punyakanok-etal-2004-semantic,li-etal-2019-dependency}.
For an argument with a consecutive word span  and a semantic role , we restrict all possible subtrees are single-rooted at a potential headword  within the span, which is also not realized yet.
The semantic role  is assigned as the label of the dependency pointing from  to the headword.
We adopt a similar strategy for non-argument spans, except that we set the label to   for distinction and remove the single-root restriction.

By enumerating all possible subtrees and combing them together, the resulting tree set  is exponential in size.
During training, we develop a span-constrained Inside algorithm to perform the enumeration (\S~\ref{sec:span-constrained-treecrf}).
Fig.~\ref{fig:srl-latent-trees} gives a brief example of the conversion process.








\subsection{Tree  SRL Recovery}\label{sec:srl-recovery}
Supposing we have trained a parsing model, during the decoding phase, what we need is to recover predicate-argument structures from the outputs of the parser.

We first find all predicates via simple local label classification: a word  is recognized as a predicate if the dependency  is labeled as \texttt{PRD}.
Subsequently, we obtain the highest-scoring tree  (Fig.~\ref{fig:dep-trees}) for  using Eisner algorithm \cite{eisner-2000-bilexical} with complexity :

where  is the tree score, and the tree is restricted to be rooted at .
Arguments for the predicate are then recovered by collapsing subtrees headed by  into flat spans.

Concretely, we take each modifier  of  as the headword of a potential argument.
If the label  of  is not ``'', i.e., non-argument, then an entire argument span comprises  and its descendants and takes  as the semantic role.
The resulting SRL output is the collection of all predicates and corresponding recovered arguments.
A recovery example is demonstrated in Fig.~\ref{fig:collation}.


\section{Methodology}\label{sec:methodology}

Now we elaborate the architecture of our proposed model for training the parser.
Following \citet{dozat-etal-2017-biaffine,zhang-etal-2020-efficient}, our model consists of a contextualized encoder and a (second-order) scoring module.
We further propose a span-aware TreeCRF to compute the probabilities of the converted partially-observed trees.

\subsection{Neural Parameterization}
Given the sentence , we first obtain the hidden representation of each token  via a deep contextualized encoder.

In this work, we experiment with two alternative encoders, i.e., BiLSTMs \cite{yarin-etal-2016-dropout} and pretrained language models (PLMs) \cite{devlin-etal-2019-bert}.
More setting details are available in \S~\ref{sec:impl}.


\paragraph{(Second-order) Tree parameterization}
Following \citet{dozat-etal-2017-biaffine}, we decompose a tree  into two separate  and , where  is a skeletal tree, and  is the related strictly-ordered label sequence.
For each head-modifier pair , we score them using two MLPs followed by a Biaffine layer \cite{cai-etal-2018-full}:

The score of the dependency  with label  is calculated analogously.
We use two extra MLPs and  Biaffine layers to obtain all label scores.

We also make use of adjacent-sibling information \cite{mcdonald-pereira-2006-online} to enhance the first-order biaffine parser further.
Following \citet{wang-etal-2019-second,zhang-etal-2020-efficient}, we employ three extra MLPs as well as a Triaffine layer for second-order subtree scoring,

where  and  are two adjacent modifiers of , and  populates between  and .

Under the first-order factorization \cite{mcdonald-etal-2005-online}, the score of  becomes

For the second-order case \cite{mcdonald-pereira-2006-online}, we further incorporate adjacent-sibling subtree scores into tree scoring:


The probabilities of skeletal tree  and its label sequence  are parameterized as

 is the set of all possible legal unlabeled trees, and  is known as the partition function.
Each label  is independent of tree  and other labels, thus  is locally normalized over all .

Finally, we define the probability of the labeled tree  as the product of the probabilities of its two sub-components.


\subsection{Span-constrained TreeCRF}\label{sec:span-constrained-treecrf}

\paragraph{Training objective}
During training, we seek to maximize the probability of converted trees  for each predicate .
Accordingly, we define the following loss function:

in which  can be further expanded as


\begin{figure}[tb!]
    \renewcommand{\arraystretch}{1}
    \centering
    \small
    \begin{tabular}{ll}
        \textbf{\colorbox{white}{\textsc{R-Comb}}}:\hfill                                                                                                                                                                                                                                                                        & \textbf{\colorbox{white}{\textsc{Comb}}}:    \\\2pt]
                                                                                                                                                                                                                                                                                                                               &
                                                                                                                                                                                                                                                                                                                                                                              \-10pt]
        \begin{tabular}[t]{@{}l@{}}\colorbox{white}{} \\ \colorbox{seagreen!25}{} \end{tabular}                                                                                                                                      &
        \begin{tabular}[t]{@{}l@{}}\colorbox{white}{} \\\colorbox{seagreen!25}{,}\\\colorbox{seagreen!25}{} \end{tabular}                 \
    \mathrm{s}(\boldsymbol{x},\boldsymbol{t})=\mathrm{s}(\boldsymbol{x},\boldsymbol{y})+ \log P(\boldsymbol{r}\mid \boldsymbol{x},\boldsymbol{y})
-10pt]
            \textsc{Crf}                                                      & 83.70                       & 83.18                       & 85.38                     & 84.27          & 70.40                    & 72.97          & 71.66          & 81.03          & \textbf{79.47}         & 82.80          & 81.10          \\
            \textsc{Crf2o}                                                    & \textbf{83.91}              & \textbf{83.26}              & \textbf{86.20}            & \textbf{84.71} & \textbf{70.70}           & \textbf{74.16} & \textbf{72.39} & \textbf{81.16} & 79.27                  & \textbf{83.24} & \textbf{81.21} \\\-10pt]
            \textsc{Crf}                                                      & 84.42                       & 85.38                       & 85.56                     & 85.47          & 75.05                    & 74.05          & 74.55          & 83.22          & 83.21                  & 83.85          & 83.53          \\
            \textsc{Crf2o}                                                    & \textbf{84.65}              & \textbf{85.47}              & \textbf{86.40}            & \textbf{85.93} & 74.92                    & \textbf{75.00} & \textbf{74.96} & \textbf{83.39} & 83.02                  & \textbf{84.31} & \textbf{83.66} \\\-10pt]
            \textsc{Crf}\rlap{}                               & 87.76                       & 88.93                       & 88.58                     & 88.76          & 82.87                    & 81.67          & 82.27          & 87.33          & 87.45                  & 87.56          & 87.51          \\
            \textsc{Crf2o}\rlap{}                             & \textbf{88.05}              & 89.00                       & \textbf{89.03}            & \textbf{89.02} & 82.81                    & \textbf{82.35} & \textbf{82.58} & \textbf{87.52} & 87.52                  & \textbf{87.79} & \textbf{87.66} \\
            \hdashline[1pt/3pt]
            \textsc{Crf}\rlap{}                            & 88.21                       & 89.29                       & 88.99                     & 89.15          & 83.22                    & 82.42          & 82.82          & 87.97          & 87.99                  & 88.22          & 88.11          \\
            \textsc{Crf2o}\rlap{}                          & \textbf{88.49}              & \textbf{89.45}              & \textbf{89.63}            & \textbf{89.54} & \textbf{83.89}           & \textbf{83.39} & \textbf{83.64} & \textbf{88.29} & \textbf{88.11}         & \textbf{88.53} & \textbf{88.32} \\
            \bottomrule
        \end{tabular}
        \caption{
            All results on CoNLL05 and CoNLL12 data, averaged over 4 runs with different random seeds.
        }
        \label{table:main-results}
    \end{small}
\end{table*}

\subsection{Main results}\label{subsec:results}
Table~\ref{table:main-results} gives our main results.
By default, our models work in an \emph{end-to-end} fashion, i.e., predicting all predicates and their associated arguments simultaneously.
However, we note that reporting the results of using gold predicates is a more prevalent practice in the SRL community \cite{he-etal-2018-jointly,shi-etal-2019-simple}.
Therefore, for comprehensive comparisons, in addition to listing most \emph{end-to-end} results of previous works we are aware of, we also conduct experiments with gold predicates, which is achieved by only parsing trees rooted at the pre-specified predicates.\footnote{We eliminate the invalid  simply via setting the dependency score to .}

The two major rows show the results of \emph{end-to-end} and \emph{w/ gold predicates} settings, indicating very consistent trends.
We can clearly see that under the \emph{end-to-end} setting, our LSTM-based \textsc{Crf} models outperform previous works by a large margin on all datasets.
The second-order \textsc{Crf2o} further improves over \textsc{Crf} by 0.2, 0.4 and 0.7 F scores on three CoNLL05 datasets, respectively.
On CoNLL12, \textsc{Crf2o} shows smaller but steady gains.
As revealed in \S~\ref{subsec:analysis}, we attribute the improvements brought by \textsc{Crf} and \textsc{Crf2o} to better performing at global consistency and long-range dependencies.

The results under the \emph{w/ gold predicates} setting are presented in the second major row.
Many PLM-based results comparable to ours are available in this setting.
Among them, the BIO-based parser of \citet{shi-etal-2019-simple} achieves 88.8, 82.0 and 86.5 F scores on CoNLL05 WSJ, Brown and CoNLL12 Test data.
The dependency (word)-based parser of \citet{zhou-etal-2022-fast} achieves 88.78,  82.51 and 87.15 F scores.
Meanwhile, the results of our first-order \textsc{Crf} model with BERT is 88.76, 82.27 and 87.51.
The performance gap between \textsc{Crf} and recent state-of-the-art parsers are negligibly small.
We do not utilize any word/predicate embeddings as well as LSTM layers for simplicity, which may potentially hinder the results.
Despite this fact, our second-order \textsc{Crf2o} achieves 89.02, 82.58, and 87.66, which outperforms the systems of \citet{shi-etal-2019-simple} by 0.2, 0.6 and 1.2 F scores and achieves new state-of-the-art on both CoNLL05 and CoNLL12 datasets.
This implies that imposing stronger structure constraints can still bring remarkable improvements for span-style SRL even when empowered with very expressive encoders.
In the bottom lines, we provide the results of utilizing RoBERTa, we can see that \textsc{Crf} and \textsc{Crf2o} augmented with RoBERTa can obtain further gains on top of BERT.

We highlight that we do not include any syntax-aware work \cite{xia-etal-2019-syntax,zhou-etal-2020-parsing} in Table~\ref{table:main-results}, which has shown to deliver substantial gains for SRL (see \S~\ref{sec:incomparable}).
It is still an open question to be investigated whether the benefits brought by our methods are orthogonal to linguistic syntax knowledge.
We focus on pure syntax-agnostic models in this paper.
So we do not list the results of this line of works in order to make fair comparisons.


\subsection{Analysis}\label{subsec:analysis}

\begin{table}[tb!]
    \renewcommand{\arraystretch}{1.1}
    \setlength{\tabcolsep}{4.5pt}
    \centering
    \begin{small}
        \begin{tabular}{l cccccc}
            \toprule
            \rowcolor[gray]{0.95} & \multicolumn{4}{c}{Dev} & \multicolumn{2}{c}{Test}                                                                     \\
            \rulefiller\cmidrule(lr){2-5} \cmidrule(lr){6-7}
            \rowcolor[gray]{0.95} & P                       & R                        & F          & CM             & F          & CM             \\
            \midrule
            \textsc{Bio}          & 86.80                   & 86.38                    & 86.59          & 69.24          & 88.22          & 71.95          \\
            \textsc{Span}         & 87.68                   & 86.75                    & 87.21          & 68.43          & 88.44          & 70.22          \\\-10pt]
            \multirow{4}{*}{Ours}           & \textsc{Crf}                   & \textbf{242} \\
                                            & \textsc{Crf2o}                 & 214          \\
                                            & \textsc{Crf}   & 136          \\
                                            & \textsc{Crf2o} & 113          \\
            \bottomrule
        \end{tabular}
    \end{small}
    \caption{
        Speed comparison on CoNLL05 Test data.
        We also list the speed of our TreeCRF models using BERT (\textsc{Crf} and \textsc{Crf2o}).
    }
    \label{table:speed}
\end{table}

\section{Related Works}\label{sec:relworks}

\paragraph{Span-style SRL}
Pioneered by \citet{gildea-jurafsky-2002-automatic}, syntax has long been considered indispensable for span-style SRL \cite{punyakanok-etal-2008-importance}.
With the advent of the neural network era, syntax-agnostic models make remarkable progress \cite{zhou-xu-2015-end,tan-etal-2018-deep,cai-etal-2018-full}, mainly owing to powerful model architectures like BiLSTM \cite{yarin-etal-2016-dropout} or Transformer \cite{vaswani-2017-attention}.
Meanwhile, other researchers also pay attention to the utilization of syntax trees, including serving as guidance for argument pruning \cite{he-etal-2018-syntax}, as input features \cite{marcheggiani-titov-2017-encoding,xia-etal-2019-syntax,mohammadshahi-etal-2021-g2g}, or as supervisions for joint learning \cite{swayamdipta-etal-2018-syntactic}.
However, to our best knowledge, very few works have been devoted to mining internal structures of shallow SRL representations.
As exceptions, \citet{he-etal-2018-jointly,zhang-etal-2021-comparing} take into account headwords while recognizing arguments.
Beyond this, this work proposes to model full argument subtree structures rather than merely headwords and find more competitive results.








\paragraph{Parsing with latent variables}
\citet{henderson-etal-2008-latent, henderson-etal-2013-multilingual} design a latent variable model to deliver syntactic and semantic interactions under the setting of joint learning.
In more common situations where gold treebanks may lack, \citet{naradowsky-etal-2012-improving,gormley-etal-2014-low} use LBP for the inference of semantic graphs and treat latent trees as global factors \cite{smith-eisner-2008-dependency} to provide soft beliefs for reasonable predicate-arguments structures.
This work differs in that we make hard constraints on syntax tree structures to conform to the SRL structures, and take only subtrees attached to predicates as latent variables.
The intuition behind latent tree models \cite{marina-michael-2000-mixure,chu-etal-2017-latent,kim-2017-structured} is to utilize tree structures to provide rich structural interactions for problems with prohibitive high complexity.
This idea is also common in many other NLP tasks like text summarization \cite{liu-lapata-2018-learning}, sequence labeling \cite{zhou-etal-2020-latent}, and AMR parsing \cite{zhou-etal-2020-amr}.

\paragraph{Reduction to tree parsing}
Researchers have investigated several ways to recover SRL structures from tree structures, due to their high coupling nature \cite{palmer-etal-2005-propbank}.
Early efforts of \citet{cohn-blunsom-2005-semantic} derive predicate-arguments from pruned phrase structures by using a CKY-style TreeCRF to learn parameters.
\citet{johansson-nugues-2008-dependency} and \citet{choi-palmer-2010-retrieving} investigate retrieving semantic boundaries from dependency outputs.
Their devised heuristics rely heavily on the quality of output trees, leading to inferior results.
Our reduction is also inspired by works on other NLP tasks, including named entity recognition (NER) \cite{yu-etal-2020-named}, nested NER \cite{fu-etal-2021-nested,lou-etal-2022-nested}, semantic parsing \cite{sun-etal-2017-semantic,jiang-etal-2019-hlt}, and EUD parsing \cite{anderson-gomez-rodriguez-2021-splitting}.
As the most relevant work, \citet{shi-etal-2020-semantic} also propose to reduce SRL to syntactic dependency parsing by integrating syntactic-semantic relations into a single dependency tree by means of joint labels.
However, their approach shows non-improved results, possibly due to the label sparsity problem and high back-and-forth conversion loss.
Also, they use gold treebank supervisions, while ours does not rely on any hand-annotated syntax data.

\section{Discussions and Future Works}
The basic idea of this work is to mimic SRL structures with a combination of multiple latent trees.
This new perspective sheds light on some natural extensions of our work to other tightly related semantic parsing tasks, e.g., AMR \cite{zhang-etal-2019-amr} and UCCA \cite{jiang-etal-2019-hlt}.\footnote{We thank an anonymous reviewer for pointing out the connection.}
Tasks fall into this type exhibit very flexible graph representation schemes (e.g., \emph{reentrancy} and \emph{discontinuity}) \cite{zhang-etal-2019-broad}, which are intractable by principled decoding algorithms like dynamic programming.
We believe that employing structured inference in spirit of our approaches can provide considerable help in getting rid of greedy span/dependency selections and finding globally optimal structures.

We prefer to reduce SRL to dependency-based tree parsing rather than another paradigm, i.e., constituency parsing, partly because dependencies provide a more transparent bilexical governor-dependent encoding of predicate-argument relations \cite{hacioglu-2004-semantic}.
We also do not pursue the way of jointly modeling dependencies and phrasal structures with lexicalized trees \cite{eisner-satta-1999-efficient,yang-tu-2022-combining,lou-etal-2022-nested} as our approach enjoys a lower time complexity of .
Nonetheless, we admit potential advantages of this kind of modeling \cite{liu-etal-2022-structured} and leave this as our future work.

There are other interesting perspectives deserve further explorations: given that span-style SRL substantially benefits from our formulation of recovering SRL structures from trees, \emph{can the induced dependency trees learn plausible syntactic structures?} Or in other words, can they agree with linguistic-motivated annotations \cite{marcus-etal-1993-building}?
We conduct thorough analyses in spirit of \citet{gormley-etal-2014-low,li-etal-2021-syntax} and give affirmative answers.
Due to space limitations, we refer readers to \S~\ref{sec:tree-probing} and \S~\ref{sec:induction} for details.


\section{Conclusions}\label{sec:conclusions}

In this paper, we propose to reduce span-style SRL to dependency parsing by viewing flat phrasal arguments as latent subtrees, and design a novel span-constrained TreeCRF to accommodate the span structures.
Taking inspirations from the parsing literature, we also build a second-order extension and find further gains.
Our models are syntax-agnostic and do not rely on any external linguistic syntax knowledge.
Experimental results show that, our proposed methods outperform all previous comparable works, achieving new state-of-the-art on both CoNLL05 and CoNLL12 benchmarks.
Extensive analyses confirm that our approach enjoys some merits of global structural constraints, meanwhile maintaining acceptable time complexity.
Furthermore, we find our modeling of latent subtrees provides effective assistance in terms of long-range dependencies and global consistency.


\section{Acknowledgments}\label{sec:ack}
We would like to thank the anonymous reviewers for their valuable comments, and Prof. Zhenghua Li for very helpful feedbacks, suggestions and idea discussions.
This work was supported by National Natural Science Foundation of China with Grant No. 62176173, 62076173, U1836222 and a project funded by the Priority Academic Program Development (PAPD) of Jiangsu Higher Education Institutions.

\bibliographystyle{acl_natbib}
\bibliography{main}

\appendix

\section{Implementation Details}\label{sec:impl}

In this work, we set up two alternative model architectures, i.e, LSTM-based and PLM-based.
For the LSTM-based model, we directly adopt most settings of \citet{dozat-etal-2017-biaffine} with some adaptions.
The input vector of each token  is the concatenation of three parts,

where  and  are word and lemma embeddings, and  is the outputs of a CharLSTM layer \cite{lample-etal-2016-neural}.
We set the dimension of lemma and CharLSTM representations to 100 in our setting.
We next feed the input embeddings into 3-layer BiLSTMs \cite{yarin-etal-2016-dropout} to get contextualized representations with dimension 800.

Other dimension settings are kept the same as biaffine parser \cite{dozat-etal-2017-biaffine}.
Following \citet{zhang-etal-2020-efficient}, we set the hidden size of Triaffine layer to 100 for \textsc{Crf2o} additionally.
The training process continues at most 1,000 epochs and is early stopped if the performance on Dev data does not increase in 100 consecutive epochs.
In practice, we observe that the training procedure is often stopped within 300 epochs (12 hours), which is efficient enough.

For PLM-based models, we opt to directly finetune the PLM layers without cascading word embedding and LSTM layers for the sake of simplicity.
We use ``\href{https://huggingface.co/bert-large-cased}{\texttt{bert-large-cased}}'' for BERT, and ``\href{https://huggingface.co/roberta-large}{\texttt{roberta-large}}'' for RoBERTa respectively.
We train the model for 20 epochs with roughly 1,000 tokens per batch and use AdamW \cite{kingma-ba-2015-adam,ilya-etal-2018-adamw} with  and  for parameter optimization .
The learning rate is  for PLMs, and  for the rest components.
We adopt the warmup strategy in the first 10\% of the training steps, and then apply a linear decay to the learning rate in the remaining steps.

\section{The Inside Algorithm}\label{sec:inside}

We give the pseudocode of the common second-order Inside algorithm \cite{mcdonald-pereira-2006-online} in Alg.~\ref{alg:inside-2o} as additional explanations to Fig.~\ref{fig:deduction}.
The difference between the common second-order Inside algorithm and our proposed span-constrained one lies in the rule constraints green highlighted in Fig.~\ref{fig:deduction}.
\begin{algorithm}[tb!]
    \begin{algorithmic}[1]
        \newlength{\commentindent}
        \setlength{\commentindent}{.24\textwidth}
        \renewcommand{\algorithmiccomment}[1]{\unskip\hfill\makebox[\commentindent][l]{~#1}\par}
        \LetLtxMacro{\oldalgorithmic}{\algorithmic}
        \renewcommand{\algorithmic}[1][0]{\oldalgorithmic[#1]
            \renewcommand{\ALC@com}[1]{\ifnum\pdfstrcmp{##1}{default}=0\else\algorithmiccomment{##1}\fi}
        }
        \STATE \textbf{Define:} 
        \STATE  \COMMENT{ is batch size}
        \STATE \textbf{Initialize:} \label{alg:init}
        \FOR [span width]{ \TO }
        \STATE \emph{Parallelization on} ; 
        \STATE \label{alg:incomplete-r}
        \STATE \label{alg:incomplete-l}


        \STATE \label{alg:sib}
        \STATE \label{alg:complete-r}
        \STATE \label{alg:complete-l}
        \ENDFOR
        \RETURN 
    \end{algorithmic}
    \caption{The Second-order Inside Algorithm.}
    \label{alg:inside-2o}
\end{algorithm}

In Line~\ref{alg:init},  corresponds to the axiom items  with initial score .
Line~\ref{alg:incomplete-r} corresponds to two merge operations in Fig.~\ref{fig:deduction}.
The incomplete span  () is obtained by summing over either all pairs of complete span  and  (\textbf{\textsc{R-Link}}) or pairs of the incomplete span  and the sibling span  (\textbf{\textsc{R-Link2}}).
In Line~\ref{alg:sib}, the sibling span  () is obtained by summing over all pairs of complete span  and  (\textbf{\textsc{Comb}}).
Line~\ref{alg:complete-r} describes the similar merging operation on all pairs of the incomplete span  and the complete span , resulting a complete span  () (\textbf{\textsc{R-Comb}}).
Line~\ref{alg:incomplete-l} and Line~\ref{alg:complete-l} is the symmetric L-rules, which are omitted in Fig.~\ref{fig:deduction}.

\section{More Comparisons}\label{sec:incomparable}

\begin{table}[tb!]
    \renewcommand{\arraystretch}{1.1}
    \setlength{\tabcolsep}{2.7pt}
    \centering
    \begin{small}
        \begin{tabular}{l ccc ccc}
            \toprule
            \rowcolor[gray]{0.95}                    & \multicolumn{3}{c}{WSJ} & \multicolumn{3}{c}{Brown}                                                                 \\
            \rulefiller	\cmidrule(lr){2-4}		\cmidrule(lr){5-7}
            \rowcolor[gray]{0.95}                    & P                       & R                         & F         & P             & R             & F         \\
            \midrule
            SA                        & 84.17                   & 83.28                     & 83.72         & 72.98         & 70.1\white{0} & 71.51         \\
            SA        & 86.21                   & 85.98                     & 86.09         & 77.1\white{0} & 75.61         & 76.35         \\
            G2G            & 86.40                   & 87.79                     & 87.08         & 78.76         & 80.06         & 79.40         \\
            LIMIT          & 86.62                   & 89.12                     & 87.85         & 79.58         & 83.05         & 81.28         \\
            ParsingAll     & 86.77                   & 88.49                     & 87.62         & 79.06         & 81.67         & 80.34         \\
            ParsingAll    & 87.65                   & 89.66                     & 88.64         & 80.77         & 83.92         & 82.31         \\\-10pt]
            \textsc{Crf}\rlap{}      & 88.93                   & 88.58                     & 88.76         & 82.87         & 81.67         & 82.27         \\
            \textsc{Crf2o}\rlap{}    & 89.00                   & 89.03                     & 89.02         & 82.81         & 82.35         & 82.58         \\
            \textsc{Crf}\rlap{}   & 89.29                   & 88.99                     & 89.15         & 83.22         & 82.42         & 82.82         \\
            \textsc{Crf2o}\rlap{} & 89.45                   & 89.63                     & 89.54         & 83.89         & 83.39         & 83.64         \\
            \bottomrule
        \end{tabular}
        \caption{
            Comparisons with other less comparable works on CoNLL05 WSJ and Brown data.
             means using linguistic syntax knowledge;
             means different evaluation methods.
            SA: \citet{strubell-etal-2018-lisa}; ParsingAll: \citet{zhou-etal-2020-parsing}; LIMIT: \citet{zhou-etal-2020-limit}; G2G: \citet{mohammadshahi-etal-2021-g2g}; TANL: \citet{paolini-etal-2021-structured}.
        }
        \label{table:incomparable}
    \end{small}
\end{table}

In Table~\ref{table:incomparable}, for reference, we list the results of some works with different experimental settings and therefore less comparable.
For example, \citet{paolini-etal-2021-structured} and \citet{strubell-etal-2018-lisa}\footnote{
    Under the \emph{end-to-end} setting, different from the standard pratice \cite{he-etal-2018-jointly}, \citet{strubell-etal-2018-lisa} only ran the evaluation tool once, resulting in slightly higher precision values.
    See discussions in \href{https://github.com/strubell/LISA/issues/9}{their code issue}.
}
adopt different evaluation metrics, resulting in slightly higher F values than official tools.
Nonetheless, we find that our \textsc{Crf2o} with RoBERTa achieves 89.54 F on WSJ data under the \emph{w/ gold predicates} setting, showing very competitive results when compared with T5-based model of \citet{paolini-etal-2021-structured}.
\citet{zhou-etal-2020-parsing} propose a joint-learning framework, integrating both (dependency/constituency) syntactic parse trees and dependency-based SRL resources to enhance their models.
Their ablation studies show that using syntax trees brought an overall improvement of 1.6 F score on CoNLL05 Dev data.
We believe that we could achieve similar or even higher results than their syntax-aware XLNet-based models by incorporating human-annotated syntax knowledge.
However, exploring different ways of injecting syntax is not the core of this paper.
We take this as our future work.

\begin{table}[tb!]
    \renewcommand{\arraystretch}{1.1}
    \centering
    \small
    \begin{tabular}{l ccc}
        \toprule
        \rowcolor[gray]{0.95}                           & P              & R              & F          \\
        \midrule
        \textsc{Crf}                                    & 75.28          & 75.24          & 75.26          \\
        \textsc{Crf}                    & 84.70          & 84.39          & 84.54          \\
        \rowcolor[gray]{0.95} \multicolumn{4}{c}{\emph{w/ gold syntax}}                                    \\
        \citet{johansson-nugues-2008-effect}            & -              & -              & 84.32          \\
        \citet{li-etal-2019-dependency} & -              & -              & 89.20          \\
        \textsc{Crf}                    & \textbf{93.56} & \textbf{93.22} & \textbf{93.39} \\
        \bottomrule
    \end{tabular}
    \caption{Results for dependency-based evaluation on CoNLL09 Test data under \emph{w/o.} and \emph{w/ gold syntax} settings.}
    \label{table:conll09}
\end{table}

\section{Dependency-based evaluation}\label{sec:tree-probing}



Observing that our \textsc{Crf} model can conveniently determine dependencies from predicates to span headwords as by-products of constructing arguments, we therefore conduct dependency-based evaluation on CoNLL09 Test data \cite{hajic-etal-2009-conll} to measure the quality of induced dependencies.
As CoNLL09 data shares the same text content with CoNLL05, we directly make use of the model trained on CoNLL05 to obtain the results of CoNLL09 Test.
Following \citet{johansson-nugues-2008-dependency-based,li-etal-2019-dependency}, we also compare our \textsc{Crf} outputs with the upper bound of utilizing gold syntax tree to determine the headwords of predicted arguments.
Since CoNLL05 contains only verbal predicates, we discard all nominal predicate-argument structures under the guidance of POS tags starting with \texttt{N*}.
Word senses and self-loops are removed as well.

Results are listed in Table~\ref{table:conll09}, from which we can draw some observations:
1) after using BERT, \textsc{Crf} outperforms LSTM-based model (75.26) by a large margin, implying BERT provides fruitful prior knowledge for dependency induction;
2) our \textsc{Crf} with BERT achieves 84.54 F on CoNLL09 Test, exhibiting very promising performance even when compared to models using gold syntax \cite{johansson-nugues-2008-dependency-based,li-etal-2019-dependency}.
This indicates that the dependencies induced by \textsc{Crf} are highly in line with gold dependency-based annotations, illuminating potential extensions of our work on supervised dependency-based SRL.

\section{Grammar Induction}\label{sec:induction}



\begin{table}[tb!]
    \renewcommand{\arraystretch}{1.1}
    \setlength{\tabcolsep}{5pt}
    \centering
    \begin{small}
        \begin{tabular}{lll}
            \toprule
            \rowcolor[gray]{0.95} Rules & Models                                          & WSJ           \\
            \midrule
            \multirow{5}{*}{Stanford}   & NL-PCFGs \cite{zhu-etal-2020-return}            & 40.5          \\
                                        & NBL-PCFGs \cite{yang-etal-2021-neural}          & 39.1          \\
                                        & StructFormer \cite{shen-etal-2021-structformer} & 46.2          \\\-10pt]
                                        & \textsc{Crf}                                    & 48.0          \\
                                        & \textsc{Crf}                    & \textbf{65.4} \\
            \rowcolor[gray]{0.95}\multicolumn{3}{c}{\emph{w/ gold POS tags (for reference)}}              \\
            \multirow{6}{*}{Collins}    & DMV \cite{klein-manning-2004-corpus}            & 39.4          \\
                                        & MaxEnc \cite{le-zuidema-2015-unsupervised}      & 65.8          \\
                                        & NDMV \cite{jiang-etal-2016-unsupervised}        & 57.6          \\
                                        & CRFAE \cite{cai-etal-2017-crf}                  & 55.7          \\
                                        & L-NDMV \cite{han-etal-2017-dependency}          & 59.5          \\
                                        & NDMV2o \cite{yang-etal-2020-second}             & \textbf{67.5} \\
            \bottomrule
        \end{tabular}
        \caption{Grammar induction results of our \textsc{Crf} model under different head-finding rules.}
        \label{table:induction}
    \end{small}
\end{table}

To gain further insights, we make use of the scores defined in Eq.~\ref{eq:tree-score} to extract full dependency tree structures.
Surprisingly, we find they are highly in agreement with expert-designed grammars \cite{marcus-etal-1993-building} when examined on the grammar induction task \cite{klein-manning-2004-corpus}.

We show precise grammar induction results in Table~\ref{table:induction}.
The results are not comparable to typical methods like DMV \cite{klein-manning-2004-corpus} or CRFAE \cite{cai-etal-2017-crf}, as they use gold POS tags as guidance, and we use Stanford Dependencies rather than Collins rules \cite{collins-2003-head}.
Under similar settings, however, our learned task-specific trees perform significantly better than recent works.

Another interesting observation is that the gap between the BERT-based model and the LSTM-based model is much larger than that on SRL results.
This implies LSTMs tend to be more fitted to SRL structures, while BERT is able to provide a strong inductive bias for syntax induction.

\end{document}
