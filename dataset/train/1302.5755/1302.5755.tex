\documentclass{llncs}
\usepackage{llncsdoc}
\usepackage[english]{babel}
\usepackage{amsmath}
\usepackage{amssymb,amsfonts,textcomp}
\usepackage{multicol}
\usepackage{array}
\usepackage{hhline}
\begin{document}

\title{Analysis Of The Girth For Regular Bi-partite Graphs With Degree }

\author{Vivek S Nittoor \and Reiji Suda}

\institute{The University Of Tokyo}

\maketitle
\begin{abstract}
The goal of this paper is to derive the detailed description of the Enumeration Based Search Algorithm from the high level description provided in , analyze the experimental results from our implementation of the Enumeration Based Search Algorithm for finding a regular bi-partite graph of degree , and compare it with known results from the available literature. We show that the values of  for a given girth  for  BTUs are within the known mathematical bounds for regular bi-partitite graphs from the available literature. 
\end{abstract}

\section{Introduction}
The goal of this paper is to develop the detailed description of the Enumeration Based Search Algorithm from the high level description provided in  and analyze the implementation results of the Enumeration Based Search Algorithm for finding a regular bi-partite graph of degree , and compare it with known results from the available literature.  BTU is our notation for a regular bi-partite graph that has been introduced in . The high level description of the Enumeration Based Search Algorithm for searching a girth maximum  BTU has been described in . The theoretical background behind BTUs has been introduced and  explained in detail in  and .

\section {Girth Maximization as a Extremal Graph Theory question}

We consider the problem of searching for a girth maximum  BTU as a question in Extremal Graph Theory by raising two related questions.
\begin{enumerate}
\item Given girth  and   , what is the minimum value of   such that a  BTU has girth   .
\item Given  , what is the maximum attainable girth for a  BTU ?
\end{enumerate}

\subsection{Definitions}
We review definitions from   and   .

\begin{definition}  BTU \\
A  Balanced Tanner Unit (BTU) is a regular bi-partite graph that can be represented by a  square matrix with  non-zero elements in each of its rows and columns. Every  BTU has a bipartite graph representation and an equivalent matrix representation.
\end{definition}

\begin{definition} {Girth maximum  BTU} \\
{A labeled   BTU   is girth maximum if there does not exist another labeled   BTU  with girth greater than that of  .}
\end{definition}

\begin{definition}  where  for    \\
  refers to the family of all labeled  BTUs with compatible permutations   for   that occur in the same order on a complete  symmetric permutation tree ,   where   , such that    is the partition between permutations   and    for all integer values of    given by   .
\end{definition}
\begin{definition} {Optimal partition parameters for girth maximum  BTU}.
 refer to optimal partitions derived in  such that there exists a girth maximum  BTU \ in  ,where  refers to  for , with  obtained as a solution to  such that  is minimized. Thus,    are  ,  , ,  and  respectively.
\end{definition}

\subsection{Search for girth maximum  BTU }
Search for a girth maximum  BTU refers to search for an optimal labelled   BTU in a family of labelled BTUs that we refer to as  
 where  for  . 

\section {Girth Maximization as a Extremal Graph Theory question}

We consider the problem of searching for a girth maximum  BTU as a question in Extremal Graph Theory by raising two related questions.
\begin{enumerate}
\item Given girth  and   , what is the minimum value of   such that a  BTU has girth   .
\item Given  , what is the maximum attainable girth for a  BTU ?
\end{enumerate}

\section{Maximum Attainable Girth}
\subsection {Maximum Attainable Girth for a  BTU}
We denote the maximum Attainable Girth for a  BTU as a function  .

\begin{theorem}
The maximum attainable girth of a    BTU satisfies the inequality    where    is obtained by minimizing    such that   for  .
\end{theorem}
\begin{proof}
From the optimal partition result from  for a  BTU for , we obtain that the maximum possible lenght of the maximum known cycle is , where the optimal paritions are { refers to }  {for }  and  is obtained by minimizing   such that   for . We now need to show that  cannot equal  for . This follows because of micro-partition cycles defined in  and their combinations which do not permit  to equal  for . Hence, the result follows.
\end{proof} 

\section {High Level Description Of Enumeration Based Search from }
\subsection{Enumeration Based Search algorithm for girth maximum  BTU for\ } 
We find  such that  is the smallest integer satisfying ; \\
for(  ++) \{ \\
\   such that  are relatively prime; \\
if(  \ \ == \   \ ) \\
  \\
\ else \{ \\
\ Rearrange the  BTU such that \ \  ; \\
Find  such that it maximizes girth of   BTU is formed by  ; \\
if(i != r -- 1) \\
Scale permutations ; \\
\} \\
\}

\subsection{Enumeration Based Search algorithm for a girth maximum  BTU  where }
We find  such that  is the smallest integer satisfying ; \\
for(  ++) \{ \\
\   such that  are relatively prime; \\
if(  \ \ == \   \ ) \\
  \\
\ else \{ \\
\ Rearrange the  BTU such that \ \  ; \\
Find  such that a girth maximum   BTU is formed by  ; \\
\}

\subsection {Reorganizing the  BTU such that  }
Without loss of generality,  we apply suitable permutations on depth and permutations labels on the  BTU in order to obtain .  Permutations on depth and permutations labels have been explained and defined in  and preserve isomorphism since they correspond to row permutations and column permutations on the matrix representation of the  BTU.

\section {Detailed Description Of Enumeration Based Search for a girth maximum   BTU}

To find permutation a  such that a girth maximum   BTU is formed by   \{  \\
We enumerate all permutations   with node at depth  fixed,
such that partition between    and   is  ; \\
for(each enumerated permutation\   ) \{ \\
We scale up  by   and   where   are relatively prime; \\
We compute the girth of this  BTU; \\
\} \\
We choose permutation  that gives us the best girth;

\section {Detailed Description Of Enumeration Based Search for a girth maximum   BTU where  such that  is the smallest integer satisfying  }

To find permutations  such that a girth maximum   BTU is formed by   \{  \\
We enumerate all permutations   with node at depth  fixed,
such that partition between    and 
 is  ; \\
for(each enumerated permutation\   ) \{ \\
We permute  such that all partitions between any two permutations in the set \   are preserved;  \\
We scale up  by   and   where   are relatively prime; \\
We compute the girth of this  BTU; \\
\} \\
We choose permutation  that gives us the best girth;

\section {Algorithm to Find Permutations Of }
We permute   such that all partitions between any two permutations in the set   are preserved \\
for( ++) \{  \\
  = Label at depth   of  ; \\
Permutations On Depth ; \\
Permutations On Labels ; \\
We calculate the partition between permutations   and   and girth; \\
We accept the change to    if it improves the girth; \\
\} \\
  returns to   after each run of the loop.

\section {Experimental Results for Implementation Of Enumeration Based Search}
Girth obtained for various values of   and  for   has been shown Table  . We find that the values of  for a given value of girth  lie between the lower bound for  and improved lower bound for  from . The execution time is too long for  due to the algorithm being in EXPTIME.

\begin{table}
\caption{Girth obtained for various of  and for  from Implementation}
\begin{tabular}{llllll}
\hline\noalign{\smallskip}
 &  &  &  \\
\noalign{\smallskip}
\hline
\noalign{\smallskip}
5 & 25 & 3 & 8 \\
6 & 36 & 3 & 8  \\
7 & 49 & 3 & 10 \\
8 & 64 & 3 & 10 \\
9 & 81 & 3 & 10  \\
10 & 100 & 3 & 10 \\
\hline
\end{tabular}
\end{table}

\section {Bound from }
For  being a power of a prime  , Lazebnik in   describes explicit construction of a \   {}-regular bipartite graph on   vertices with girth  .

If we consider this as a  BTU, we get   a power of a prime and  , girth  . For  , we obtain   which gives us   and we hence obtain .

\section {Lower bounds from \ }
We quote the main theoem from , "Let  be a bi-partite graph of girth , with  and , the number of vertices on the left and right sides, and  the number of edges. Assume further that all vertex degrees in  are  Then:  and   where  ,   and  is the degree of vertex ."

\subsection {From another form of the bound in }
From another form of the bound in  , \ we obtain  and  .For a  BTU with girth  , we obtain  .

Therefore,   

For even integers  ,  

For odd integers   ,   

Therefore,   

Putting   and   we get  . 
Putting    and   we get  . 
Putting    and   we get   .

\subsection {From Main Theorem in }
Derived from the main theorem, 
From  , \   and   .
For a   BTU with girth   , we obtain,  and   .
Thus, \  .
Therefore,   .
For even integers   ,   .
For odd integers   ,   .
Therefore,   
Putting    and   we get   .
Putting    and   we get   .
Putting    and   we get   .

\section {Other Related Research}
Irregular LDPC codes with girth   in  and Regular LDPC codes of girth at least  from  .

\section {Results from }
We quote Theorem from  for even values of  since our current interest is only in bi-partite graphs. "For  and  put 
 if   is even. Then a graph  with minimal degree  and girth  has at least  vertices."
We use this result to compute  for  and various values of  in Table  by simplifying the equation as 


\begin{table}
\caption{Minimum value of  for different girths  for  from  }
\begin{tabular}{llllll}
\hline\noalign{\smallskip}
 &&   \\
\noalign{\smallskip}
\hline
\noalign{\smallskip}
4 && 6   \\
6 && 14\\
8 && 30  \\
10 && 62  \\
12 && 126  \\
14 && 254 \\
\hline
\end{tabular}
\end{table}

\section {Results from }
We quote theorems from   .
\begin{enumerate}
\item "Given   and   , there exists a ,   with minimal degree of at least  and girth of at least  ".
\item "Lower Bound  if   is odd. 
   if   is even. 
Equality holds for   and   and  ,  " .
\item "If  is odd, \  ".
\item "Upper Bound   if   is odd. 
\   if   is even". 
\item "Let   be an integer. Then there exists a   {}-regular graph of order  and girth of at least  " .
\item "Most significant improvement of the bound for   ,  ".
\end{enumerate}

\section {Bound derived from }
We derive the following bound from  ,  for the minimum order  where  is its girth and  is its degree. By putting  , we obtain a simplified form of the above equation,  which could be further simplified as \\ .
We calculate the bounds for  and the improved upper bound corresponds to  from  in Table . 

\begin{table}
\caption{Lower Bound, Upper Bound and Improved Upper Bound for  for different girths  for  from  }
\begin{tabular}{llllll}
\hline\noalign{\smallskip}
 && Lower Bound  & Upper Bound  & Improved upper bound  \\
\noalign{\smallskip}
\hline
\noalign{\smallskip}
4 && 3  & 15 & 8 \\
6 && 7 & 63  & 32\\
8 && 15 & 255 & 128  \\
10 && 31  & 1023 & 512\\
12 && 63 & 4095 & 2048 \\
14 && 127 & 16383 & 8192  \\
\hline
\end{tabular}
\end{table}

\section {Analysis for   and  }

From , we quote the following result, 
"If the degree is   and girth  , a simple lower bound for number of vertices of a regular graph is given by ." \\ For   we simplify the equation as follows  . While the exponent is similar to the lower bound in  , we cannot apply the result as the girths take odd values and do not directly apply for bi-partite graphs. 

\section {Analysis for  }

We analyze the girths obtained for various size of the matrices from  in Table . However, these matrices have irregular degrees and hence a direct comparison with our obtained results might not be possible.

\begin{table}
\caption{Girth obtained for various size of the matrices in  }
\begin{tabular}{llllll}
\hline\noalign{\smallskip}
Girth & Minimum  \\
\noalign{\smallskip}
\hline
\noalign{\smallskip}
6 & 5\\
8 & 9\\
10 & 39\\
12 & 97\\
\hline
\end{tabular}
\end{table} 

\section {Analysis for   }

We quote from , "Ramanujan graphs  {are }  regular Cayley graphs of the group  if the Legendre symbol  and of \   {if the Legendre symbol }  .  is bi-partite of order   and a bound on the girth is given by the equation, ".

Putting  in order to get degree , we obtain the inequality  which can be simplified as  in order to obtain .

For each value of girth , we calculate the minimum value of  such that  and the Legendre symbol   and then calculate  for  and degree  in Table .

\begin{table}
\caption{Analysis for  }
\begin{tabular}{llllll}
\hline\noalign{\smallskip}
Girth & min , ,   &  & Chosen  & Degree   \\
\noalign{\smallskip}
\hline
\noalign{\smallskip}
6 & 5 & 120 & 2 & 3 \\
8 & 11 & 1320 & 2 & 3  \\
10 & 11 & 1320 & 2 & 3 \\
12 & 13 & 2184 & 2 & 3 \\
\hline
\end{tabular}
\end{table}

\section{Conclusion}
Our implementation for the Enumeration based Search for a girth maximum  BTU finds the maximum attainable girth of a   BTU for   and various values of .  The values of  for a given girth  are within the known mathematical bounds for regular bi-partitite graphs from the available literature. When we compare our results with bounds for more general graphs, or graphs with irregular graphs, a direct comparison may not possible since it is well known that for a given  and average degree, a lower number of vertices can be reached for irregular graphs.  
\begin{thebibliography}{[MT1]}
\bibitem[1]{}
Vivek S Nittoor, Reiji Suda,:
Balanced Tanner Units And Their Properties,
arXiv:1212.6882 [cs.DM].
\bibitem[2]{}
Vivek S Nittoor, Reiji Suda,:
Partition Parameters for Girth Maximum  BTUs,
arXiv:1212.6883 [cs.DM].
\bibitem[3]{}
Vivek S Nittoor, Reiji Suda,:
Parallelizing A Coarse Grain Code Search Problem Based upon LDPC Codes on a Supercomputer,
Proceedings of 6th International Symposium on Parallel Computing in Electrical Engineering (PARELEC 2011), Luton, UK, April 2011.
\bibitem[4]{}
R. M. Tanner,:
A recursive approach to low complexity codes,
IEEE Trans on Information Theory, vol. IT-27, no.5, pp. 533-547, Sep 1981. 
\bibitem[5]{}
C.E. Shannon,:
A Mathematical Theory of Communication,
Bell System Technical Journal, vol. 27, pp 379-423, 623-656, July, October, 1948.
\bibitem[6]{}
D. J. C. MacKay, R. M. Neal,:
Near Shannon limit performance of low density parity check codes,
 Electron. Lett., vol. 32, pp. 1645--1646, Aug. 1996.
\bibitem[7]{}
William E. Ryan, Shu Lin,:
Channel Codes Classical and Modern,
Cambridge University Press, 2009.
\bibitem[8]{}
F. Harary,:
Graph Theory,
Addison-Wesley, 1969.
\bibitem[9]{}
Shlomo Hoory,:
The Size Of Bipartite Graphs with a Given Girth,
Journal Of Combinatorial Theory, Series B 86, 215-220 (2002).
\bibitem[10]{}
Fan Zhang, at al:
High Girth LDPC Code Construction Based on Combinatorial Design,
Vehicular Technology, IEEE Conference - VTC -Spring , vol. 1, pp. 591-594 Vol. 1, 2005.
\bibitem[11]{}
J.M.F. Moura, et al:
Structured Low-Density Parity-Check Codes
IEEE Signal Processing Magazine \ vol. 21, no. 1, pp. 42-55, 2004.
\bibitem[12]{}
F. Lazebnik, V.A. Ustimenko,:
Explicit Construction of Graphs with arbitrary large Girth and of Large Size,
Discrete Applied Mathematics 60 (1995), 275-284.
\bibitem[13]{}
Bollobas,:
Extremal Graph Theory
Academic Press, London, 1978.
\bibitem[14]{}
A. Lubotzky, R. Philips, P. Sarnak,:
Ramanujan Graphs,
Combinatorica 8(3) (1988) 261-277.
\bibitem[15]{}
Bollobas,:
Modern Graph Theory
Springer, London, 1998.
\bibitem[16]{}
Vivek S Nittoor, Reiji Suda,:
Enumeration Based Search Algorithm For Finding A Regular Bi-partite Graph Of Maximum Attainable Girth For Specified Degree And Number Of Vertices,
Available at cs.DM arxiv.
\bibitem[17]{}
Mirka Miller, Jozef Siran,:
Moore graphs and beyond: A survey of the degree/diameter problem,
Electronic Journal Of Combinatorics, Dynamic Survey D, Vol. 14, 2005.
\bibitem[18]{}
N.I. Biggs,:
Girth, valency and excess,
 Linear Algebra Appl. 31 (1980) 55–59.
\bibitem[19]{}
Michael E. O’Sullivan,:
Algebraic Construction of Sparse Matrices With Large Girth,
IEEE Transactions on Information Theory, Vol. 52, No. 2, Feb 2006.
\bibitem[19]{}
S. Ramanujan,:
On certain arithmetical functions,
Trans. Camb. Phil. Soc. 22(1916), 159-184.

\end{thebibliography}

\end{document}
