\documentclass[a4paper, 12pt]{article}
\usepackage{graphicx}
\usepackage{amsmath}
\usepackage{theorem}

\topmargin -1cm
\headheight 0in \headsep 0in \textheight 7.7in \textwidth 6.5in
\oddsidemargin 0in \evensidemargin 0in \headheight 77pt \headsep
0.25in

\title{Characterization Of A Class Of Graphs Related To Pairs Of Disjoint Matchings \thanks{The current work is the main part of the author's Master's thesis defended in May 2007.}}
\author{Anush Tserunyan}
\date{\small{Department of Informatics and Applied Mathematics, Yerevan State
University, \\ Yerevan, 0025, Armenia \\ Email: anush@math.ucla.edu,
anush\_tserunyan@yahoo.com \\} \scriptsize{The work on this paper
was supported by a grant of Armenian National Science and
Educational Fund.}}


\newtheorem{definition}{Definition}[section]
\newtheorem{property}[definition]{Property}
\newtheorem{lemmasection}[definition]{Lemma}
\newtheorem{corollarysection}[definition]{Corollary}
\newtheorem{lemma}{Lemma}[subsection]
\newtheorem{assumption}{Assumption}
\newtheorem{statement}{Statement}
\newtheorem{corollary}[lemma]{Corollary}
\newtheorem{subproperty}[lemma]{Property}

\theorembodyfont{\rmfamily}
\newtheorem{case}{Case}
\newtheorem{remark}{Remark}


\newenvironment{proof}[1][Proof]{\noindent\textbf{#1.} }{\ \rule{0.5em}{0.5em}}

\newenvironment{claimproof}[1][Proof of the Claim.]{\begin{trivlist}
\item[\hskip \labelsep {\bfseries #1}]}{\end{trivlist}}

\newenvironment{necessity}[1][Necessity.]{\begin{trivlist}
\item[\hskip \labelsep {\bfseries #1}]}{\end{trivlist}}

\newenvironment{sufficiency}[1][Sufficiency.]{\begin{trivlist}
\item[\hskip \labelsep {\bfseries #1}]}{\end{trivlist}}

\newenvironment{dedication}[1][]{\begin{trivlist}
\item[\hskip \labelsep {\bfseries #1}]}{\end{trivlist}}



\newenvironment{theorem}[1][Theorem]{\begin{trivlist}
\item[\hskip \labelsep {\bfseries #1}]}{\end{trivlist}}

\newenvironment{acknowledgements}[1][Acknowledgements]{\begin{trivlist}
\item[\hskip \labelsep {\bfseries #1}]}{\end{trivlist}}

\begin{document}
\maketitle

\begin{dedication}

\end{dedication}

\begin{abstract}
For a given graph consider a pair of disjoint matchings the union of
which contains as many edges as possible. Furthermore, consider the
ratio of the cardinalities of a maximum matching and the largest
matching in those pairs. It is known that for any graph
 is the tight upper bound for this ratio. We
characterize the class of graphs for which it is precisely
. Our characterization implies that these graphs
contain a spanning subgraph, every connected component of which is
the minimal graph of this class.
\\ \\ {\bf Keywords}: matching, pair of disjoint matchings, maximum
matching.
\end{abstract}

\section{Introduction}
In this paper we consider finite undirected graphs without multiple
edges, loops, or isolated vertices. Let  and  be the
sets of vertices and edges of a graph , respectively.

We denote by  the cardinality of a maximum matching of
.

Let  be the set of pairs of disjoint matchings of . Set:


Furthermore, let us introduce another parameter:

and define a set:


While working on the problems of constructing a maximum matching 
of a graph  such that  is maximized or
minimized, Kamalian and Mkrtchyan designed polynomial algorithms for
solving these problems for trees \cite{alg}. Unfortunately, the
problems turned out to be NP-hard already for connected bipartite
graphs with maximum degree three \cite{np}, thus there is no hope
for the polynomial time calculation of  even for
bipartite graphs , where 

Note that for any graph   Thus, 
can be efficiently calculated for bipartite graphs  with
 since  can be calculated for that
graphs by using a standard algorithm of finding a maximum flow in a
network. Let us also note that the calculation of  is
NP-hard even for the class of cubic graphs since the chromatic class
of a cubic graph  is three if and only if 
(see \cite{Holyer}).

Being interested in the classification of graphs , for which
, Mkrtchyan in \cite{MPP01} proved a
sufficient condition, which due to \cite{Har, Har-Plum}, can be
formulated as: if  is a matching covered tree then . Note that a graph is said to be matching covered (see
\cite{Perfect}) if its every edge belongs to a maximum matching (not
necessarily a perfect matching as it is usually defined, see e.g.
\cite{Lov-Plum}).

In contrast with the theory of 2-matchings, where every graph 
admits a maximum 2-matching that includes a maximum matching
\cite{Lov-Plum}, there are graphs (even trees) that do not have a
``maximum" pair of disjoint matchings (a pair from ) that
includes a maximum matching.

The following is the best result that can be stated about the ratio
 for any graph  (see \cite{VAV}):


The aim of the paper is the characterization of the class of graphs
, for which the ratio  obtains its
upper bound, i.e. the equality  holds.

\begin{figure}[h]
\begin{center}
\includegraphics{fig1_spanner.eps}\\
\caption{Spanner}\label{fig_spanner}
\end{center}
\end{figure}

Our characterization theorem is formulated in terms of a special
graph called spanner (figure \ref{fig_spanner}), which is the
minimal graph for which  (what is remarkable is
that the equality  also holds
for spanner). This kind of theorems is common in graph theory: see
\cite{Har} for characterization of planar or line graphs. Another
example may be Tutte's Conjecture (now a beautiful theorem thanks to
Robertson, Sanders, Seymour and Tomas) about the chromatic index of
bridgeless cubic graphs, which do not contain Petersen graph as a
minor.

On the other hand, let us note that in contrast with the examples
given above, our theorem does not provide a forbidden/excluded graph
characterization. Quite the contrary, the theorem implies that every
graph satisfying the mentioned equality admits a spanning subgraph
every connected component of which is a spanner.


\section{Main Notations and Definitions}

Let  be a graph and  be the degree of a vertex  of
.

\begin{definition}\label{matching_definition}
A subset of  is called a matching if it does not contain
adjacent edges.
\end{definition}

\begin{definition}\label{maximum_matching_definition}
A matching of  with maximum number of edges is called maximum.
\end{definition}

\begin{definition}\label{covered_missed_by_matching}
A vertex  of  is covered (missed) by a matching  of , if
 contains (does not contain) an edge incident to .
\end{definition}

\begin{definition}\label{trail_definition}
A sequence  is called a trail
in  if , , ,
and  if , for , .
\end{definition}

The number of edges, , is called the length of a trail . Trail is called even (odd) if its
length is even (odd).

Trails  and  are considered equal. Trail  is also
considered as a subgraph of , and thus,  and  are
used to denote the sets of vertices and edges of , respectively.

\begin{definition}\label{cycle_definition}
A trail  is called a cycle if
.
\end{definition}

Similarly, cycles  and  are considered equal for
any .

If  is a trail that is
not a cycle then ,  and ,  are called the
end-vertices and end-edges of , respectively.

\begin{definition}\label{path_definition}
A trail  is called a path if
 for .
\end{definition}

\begin{definition}\label{simple_cycle_definition}
A cycle  is called simple if
 is a path.
\end{definition}

Below we omit -s and write  instead of
 when denoting a trail.

\begin{definition}\label{end_edges_and_vertices}
For a trail  of  and , define
sets , ,  and , ,
,  as follows:



and




\end{definition}

The same notations are used for sets of trails. For example, for a
set of trails ,  denotes the set of end-vertices of all
trails from , that is:


Let  and  be sets of edges of .
\begin{definition}\label{alt_trail_definition}
A trail  is called - alternating if the
edges with odd indices belong to  and others to , or vice-versa.
\end{definition}

If  is an - alternating trail then  () denotes
the graph induced by the set of edges of  that belong to 
().

The set of - alternating trails of  that are not cycles is
denoted by . The subsets of  containing only even
and odd trails are denoted by  and ,
respectively. We use the notation  instead of  do denote the
corresponding sets of - alternating cycles (e.g.  is
the set of - alternating even cycles).

The set of the trails from  starting with an edge from 
() is denoted by  ().

Now, let  and  be matchings of  (not necessarily disjoint).
Note that - alternating trail is either a path, or an even
simple cycle.

\begin{definition}\label{max_alt_path_definition}
An - alternating path  is called maximal if there is no
other - alternating trail (a path or an even simple cycle)
that contains  as a proper subtrail.
\end{definition}

We use the notation  instead of  to denote the corresponding
sets of maximal  - alternating paths (e.g.  is
the subset of  containing only those maximal -
alternating paths whose length is odd and which start (and also end)
with an edge from ).

Terms and concepts that we do not define can be found in \cite{Har,
Lov-Plum, West}.


\section{General Properties and Structural Lemmas}

Let  be a graph, and  and  be (not necessarily disjoint)
matchings of it. The following are properties of - alternating
cycles and maximal paths.

\bigskip

First note that all cycles from  are simple as  and 
are matchings.

\begin{property}\label{path_evencycle_HH'}
If the connected components of  are paths or even simple cycles,
and \\ , then .
\end{property}

\begin{property}\label{altpathequaldegrees}
If  and  then .
\end{property}

\begin{property}\label{edge_lies_on_alt_component}
Every edge  \footnote{ denotes
the symmetric difference of  and , i.e. } lies either on a cycle from
 or on a path from .
\end{property}

\begin{property}\label{AB}\
\renewcommand{\labelenumi}{(\arabic{enumi})}
\begin{enumerate}
\item if  then  and  have the same number of edges that
lie on ,

\item if  then the number of edges from  lying on  is one more than the number of ones from .
\end{enumerate}
\end{property}

These observations imply:

\begin{property}\label{cardinalitydiff}
.
\end{property}

Berge's well-known theorem states that a matching  of a graph 
is maximum if and only if  does not contain an -augmenting
path \cite{Har, Lov-Plum, West}. This theorem immediately implies:

\begin{property}
\label{maxmatchingproperty} If  is a maximum matching and  is
a matching of a graph  then
and therefore, .
\end{property}

The proof of the following property is similar to the one of
property \ref{maxmatchingproperty}:

\begin{property} \label{HH'}
If  then .
\end{property}

\begin{property} \label{lambda=2alpha}
If  and  for which , then  and .
\end{property}
\begin{proof}
Assume that . Denote by  and  the
sets of edges lying on the paths from  with odd and even
indices, respectively (indices start with ). Set

and

Note that  and  (as ). Also note that
 and . Due to property \ref{cardinalitydiff}, , i.e. ,
and therefore , which
contradicts the definition of . Thus, , and, due to property \ref{cardinalitydiff}, , which means that .
\end{proof}



\bigskip

Now let  be a fixed maximum matching of . Over all , consider the pairs  for which  is maximized. Denote the set of those pairs by :


Let  be an arbitrarily chosen pair from .

\begin{lemmasection}\label{MHmain_property}
For every path  from

\renewcommand{\labelenumi}{(\arabic{enumi})}
\begin{enumerate}
\item \label{endedgesH'} ;
\item \label{l>=3} .
\end{enumerate}
\end{lemmasection}
\begin{proof}
Let us show that . If  then , , and  is not adjacent to an edge from 
as  is maximal. Thus,  as otherwise we could enlarge
 by adding  to it which would contradict . Thus, suppose that . Let us show that . If  then define


Clearly,  is a matching, and  which means that . But  which contradicts . Thus . Similarly, it can be shown
that .

Now let us show that . Due to property \ref{HH'},
, thus there is 
, such that , since
, and we have .
\end{proof}

\begin{lemmasection}\label{M_HandH's}
Each vertex lying on a path from  is incident to an
edge from .
\end{lemmasection}
\begin{proof}
Assume the contrary, and let  be a vertex lying on a path 
from , which is not incident to an edge from .
Clearly,  is incident to an edge  lying on .

If  is not incident to an edge from  too, then  and  are disjoint matchings and  which contradicts .

On the other hand, if  is incident to an edge , then
consider the pair , where . Note that  and  are disjoint matchings and
, which means that . But  contradicting .
\end{proof}

For a path , consider one of its end-edges . Due to statement (\ref{endedgesH'}) of lemma
\ref{MHmain_property}, . By maximality of , 
is adjacent to only one edge from , thus it is an end-edge of a
path  from . Moreover,  according to property \ref{HH'}. Define a set  as follows: 

\begin{lemmasection}\label{Y_stuff}\
\begin{enumerate}
\renewcommand{\labelenumi}{(\arabic{enumi})}
\item \label{evendisjoint} The end-edges of paths of 
lie on different paths of ;
\item \label{Yis2(b-a)} .
\item For every , where , , ,
\begin{enumerate}
\renewcommand{\labelenumii}{(\alph{enumii})}
\item \label{povorot}
 and  lie on a path from , but  and
 do not lie on any path from ;
\item \label{H'-H>=4} .
\end{enumerate}
\end{enumerate}
\end{lemmasection}
\begin{proof}
(\ref{evendisjoint}) is true as otherwise we would have a path from
 with both end-edges from  contradicting . Furthermore, (\ref{evendisjoint}) together
with property \ref{maxmatchingproperty} imply (\ref{Yis2(b-a)}).

Now, let us prove (\ref{povorot}).

By the definition of ,  is an end-edge of a path
 from , and therefore  lies on  too.

 does not lie on any path from  as otherwise, due
to lemma \ref{M_HandH's}, both vertices incident to  would be
incident to edges from , which contradicts the maximality of
. Note that  is not incident to an inner vertex (not an
end-vertex) of a path  from  as any such vertex is
incident to an edge from  lying on , and therefore different
from .  is incident neither to an end-vertex of a path
 from  as it would contradict the maximality of
. Thus,  is not adjacent to an edge lying on a path from
, and therefore  does not lie on any path from
. The proof of (\ref{povorot}) is complete.

Statement (\ref{H'-H>=4}) immediately follows from (\ref{povorot}).
\end{proof}

Taking into account that , , and , we get the following result (also
obtained in \cite{VAV}) as a corollary from the statements
(\ref{Yis2(b-a)}) and (\ref{H'-H>=4}) of lemma \ref{Y_stuff}:

\begin{corollarysection}\label{fivefourthinequality}
, i.e. .
\end{corollarysection}


\section{Spanner, S-Forest and S-Graph}

The graph on figure \ref{fig_spanner} is called spanner. A vertex
 of spanner  is called -vertex, , if
. The -vertex closest to a vertex  of spanner is
referred as the base of . The two paths of the spanner of length
four connecting -vertices are called {\em sides}.

For spanner  define sets  and  as follows:




Note that for spanner , and for every , the
edge connecting the -vertices does not belong to ,
hence , , and
 as . The pair  shown on figure \ref{fig_spanner} belongs to
.

It can be implied from the lemma \ref{Y_stuff} that spanner is the
minimal graph for which the parameters  and  are not
equal.

\begin{property} \label{coveredbyboth}
For spanner  and , -vertices and
-vertices of  are covered by both  and .
\end{property}

\begin{property} \label{missedbyone}
For every -vertex  of spanner  there is 
such that  is missed by  ().
\end{property}

\begin{definition}\label{SForest}
-forest is a forest whose connected components are spanners.
\end{definition}

An -vertex of a connected component (spanner) of an -forest
 is referred simply as an -vertex of .

If  are connected components of -forest  then
define sets  and  as follows:



\begin{property} \label{S-forest_lambda=2alpha=8k}
If the number of connected components (spanners) of an -forest
 is , then , and , thus .
\end{property}

\begin{property} \label{S-forest_HH'_is_LU}
If  is an -forest and  then .
\end{property}

\begin{definition}\label{SGraph}
-graph is a graph containing an -forest as a spanning
subgraph (below, we will refer to it as a spanning -forest of an
-graph).
\end{definition}

Note that, spanning -forest of an -graph is not unique in
general.

It is easy to see that spanner, -forests, and -graphs contain
a perfect matching, and for -forest it is unique.

Let  be an -graph with a spanning -forest .

\begin{property} \label{S-graph_beta=5k}
If  has  connected components (spanners) then .
\end{property}

Let us define an - edge of  as an edge connecting an
-vertex to a -vertex of . Also define:




\begin{property}\label{2_and_3_are_equal}
For any - alternating even cycle the numbers of
- and - edges lying on it are equal.
\end{property}
\begin{proof}
Consider an - alternating even cycle  where  .

For a vertex  of the cycle  let  be the frequency
of appearance of the vertex  during the circumference of  (the
number of indices  for which  or ).
As any vertex  lying on  is incident to an edge from 
that lies on  before or after  during the circumference, and
as edges from  are - edges, we get:



On the other hand, denote by  the numbers of
-, -, - edges lying on , respectively. As for
each vertex  lying on ,  is the number of edges
that lie on  and are incident to , implies:


where the left sides of the equalities represent the numbers of
edges lying on  and incident to -vertices and -vertices of
, respectively. Thus, .
\end{proof}

\section{Main Theorem}

\begin{theorem}
\renewcommand{\labelenumi}{(\alph{enumi})}
\textit{ For a graph  ( does not contain isolated vertices),
the equality  holds, if and
only if  is an -graph with a spanning -forest ,
satisfying the following conditions:
\begin{enumerate}
\item \label{one-vertexstuff} -vertices of  are not incident to any edge from ;
\item \label{two-vertexstuff} if a -vertex  of  is incident to an edge from , then
the\footnote{We write ``the'' here as if the condition (a) is
satisfied then there is only one -vertex adjacent to  (the
-vertex connected to  via the edge from ).} -vertex
of  adjacent to  is not incident to any edge from ;
\item \label{LBaltcycles} for every - alternating even cycle  of  containing a - edge,
the graph  is not bipartite.
\end{enumerate}
}
\end{theorem}

The proof of the theorem is long, so it is divided into subsections:
Necessity and Sufficiency, which, in their turn, are split into
numbers of lemmas.


\subsection{Necessity}

In this subsection, we assume that
, and prove that  is an
-graph. Then, on the contrary assumptions we prove consequently
that the conditions (a), (b) and (c) are satisfied for an arbitrary
spanning -forest of . As one can see, we prove a statement
stronger than the Necessity of the theorem.

\bigskip

Let  be a graph,  be a fixed maximum matching of it, and
 be an arbitrarily chosen pair from .

Suppose that for the graph  the equality
 holds.

Due to corollary \ref{fivefourthinequality}, we have:
\renewcommand{\theequation}{\fnsymbol{equation}}
\setcounter{equation}{2}


\begin{lemma}\label{MH5_HH'4}
Each path from  is of length four, each path from
 is of length five, and every edge from  lies on a
path from .
\end{lemma}
\begin{proof}
Due to equality (\ref{fivefourthequality}) and statement
(\ref{Yis2(b-a)}) of lemma \ref{Y_stuff}, we get:

Therefore, as there are at least two edges from  lying on each
path of  (statement (\ref{H'-H>=4}) of the lemma
\ref{Y_stuff}), the length of each path from  is
precisely four, and every edge from  lies on a path from
. Moreover, the length of every path from 
is precisely five (due to statement (\ref{l>=3}) of the lemma
\ref{MHmain_property} it is at least five for any graph), as
otherwise we would have either an edge from  not lying on any
path from , or a path from  with length
greater than four.
\end{proof}

This lemma implies that each path  from  together
with the two paths from  starting from the end-edges of
 form a spanner (figure \ref{fig_alt_path_spanner}).

\begin{figure}[h]
\begin{center}
\includegraphics{fig2_alt_path_spanner.eps}\\
\caption{}\label{fig_alt_path_spanner}
\end{center}
\end{figure}

Since there are  paths in 
(property \ref{maxmatchingproperty}), we get:
\begin{corollary}\label{subgraph_F}
There is a subgraph  of the graph  that is an -forest
containing  spanners as its connected
components.
\end{corollary}

 Now, let  be an -forest arbitrarily chosen among the ones described in the corollary \ref{subgraph_F}.
 Due to property \ref{S-forest_lambda=2alpha=8k}, , therefore due to equality (\ref{fivefourthequality}), . This means that .

\bigskip

Let  be an arbitrarily chosen pair from . Note that the choice of  differs from the one above
(we keep this notation as the reader may have already got used with
a pair from  denoted by ).

\vspace*{0.4cm}

\begin{lemma} \label{mandatory3-vertex}
If , and  or  is a
-vertex of , then  is a -vertex of .
\end{lemma}
\begin{proof}
Due to property \ref{missedbyone}, without loss of generality, we
may assume that  is missed by .

Clearly,  as otherwise  would also be missed by 
and we could ``enlarge''  by ``adding''  to it, which
contradicts .

\begin{figure}[h]
\begin{center}
\includegraphics{fig3_claim1.eps}\\
\caption{}\label{fig_claim_mandatory3vertex}
\end{center}
\end{figure}

Now, let us show that  is neither a -vertex nor a -vertex.
Suppose for contradiction that it is, and let  be the spanner
(connected component) of  containing . Define matchings  as follows (figure \ref{fig_claim_mandatory3vertex}):

where  is the perfect matching of ;


where  is a matching of cardinality three satisfying  (it always exists). Clearly, , and, since ,  and , we have  and . This contradicts , concluding the proof
of the lemma.
\end{proof}

\begin{lemma}\label{cannotbe3-vertex}
If , then  and if  is
a -vertex of  then  is its base.
\end{lemma}
\begin{proof}
Assume the contrary. Let , where 
either belongs to , or is a -vertex whose
base is not . As  satisfies the conditions of the lemma
\ref{mandatory3-vertex}, implies that  is a -vertex of .
Let  be the side of the spanner  (connected component of
) containing . It is easy to notice that  does not belong
to  as otherwise it would be a -vertex of  whose
base is . Due to property \ref{missedbyone}, without loss of
generality, we may assume that  is missed by . Define
matchings  and  as follows (figure
\ref{fig_claim_cannotbe3-vertex}):


where  is an edge from .

\begin{figure}[h]
\begin{center}
\includegraphics{fig4_claim2.eps}\\
\caption{}\label{fig_claim_cannotbe3-vertex}
\end{center}
\end{figure}

Clearly,  and  are disjoint matchings. Moreover, since
,  and , which
contradicts  concluding the proof of the lemma.
\end{proof}

\begin{lemma} \label{S-graph}
 is an -graph with spanning -forest .
\end{lemma}
\begin{proof}
Lemma \ref{cannotbe3-vertex} asserts that there is no edge incident
to a vertex from , i.e. all vertices from
 are isolated. This is a contradiction as we
assume that  has no isolated vertices (see the beginning of
Introduction). Thus,  and  is a
spanning -forest of , which means that  is an -graph.
\end{proof}

Due to this lemma,  (an arbitrarily chosen -forest with
 spanners) is spanning. Obviously, the
converse is also true. So, we may say that  is an arbitrary
spanning -forest of .

\begin{lemma} \label{condition1}
The graph  with its spanning -forest  satisfies the
condition (a) of the theorem.
\end{lemma}
\begin{proof}
Let  be a -vertex of . Lemma \ref{cannotbe3-vertex} asserts
that if  is an edge from  then 
is the base of , thus . This means that the
condition (a) is satisfied.
\end{proof}

\begin{lemma} \label{condition2}
The graph  with its spanning -forest  satisfies the
condition (b) of the theorem.
\end{lemma}
\begin{proof}
Assume that  is a -vertex of  incident to an edge 
from  ( is the base of ), and  is the
-vertex of  adjacent to  (this -vertex is unique as, due
to lemma \ref{condition1},  can be incident only to edges from
 and ). On the opposite assumption  is
incident to an edge  from  (figure
\ref{fig_2_stuff}a).

\begin{figure}[h]
\begin{center}
\includegraphics{fig5_2-stuff.eps}\\
\caption{}\label{fig_2_stuff}
\end{center}
\end{figure}

Let us construct a subgraph  of  by removing  from
 and adding :

Note that  is a spanning -forest of , for which  is a
-vertex (figure \ref{fig_2_stuff}b), and .
Thus, . On the other hand, the graph  with
its spanning -forest  satisfies the condition (a) of the
theorem (lemma \ref{condition1}). Thus,  cannot be incident to an
edge from , and we have a contradiction.
\end{proof}

\begin{lemma} \label{condition3}
The graph  with its spanning -forest  satisfies the
condition (c) of the theorem.
\end{lemma}
\begin{proof}
Suppose for contradiction that there exists an -
alternating even cycle  of the graph  containing a -
edge  and the graph  is bipartite.

The definition of  implies that for each , . Therefore, due to property
\ref{altpathequaldegrees}, , and if
 is a -vertex, then . Hence, the
connected components of the bipartite graph  are paths
or cycles of even length.

Choose . Due to property
\ref{path_evencycle_HH'}, .

Let  be the subgraph of  whose connected components are
those sides of the spanners (connected components)  of , which do not contain any edge from
. Note that none of the edges of  is adjacent to an edge
lying on  since otherwise one of the edges from  would lie on  as well contradicting the definition of
. Let . As the sides of spanners are
paths, again due to property \ref{path_evencycle_HH'}, .

Define the sets  as follows:



Define a pair of disjoint matchings  as follows:

 Note that .

Define a pair of disjoint matchings  as follows:

 Note that .

Finally, define a pair of disjoint matchings  as
follows:

Since  and , we have  Hence,

since . \\
From the equality (\ref{fivefourthequality}) we have that
. Therefore, as , we get:

As  (equality (\ref{fivefourthequality})),
due to property \ref{lambda=2alpha}, there is no maximal
- alternating odd path ().
Let us show that there is one and get a contradiction.

Let  be the edges from  adjacent to . Clearly
 as . The construction of
 and  (or rather  and ) implies that the path
 is an - alternating odd path. Let  be
the -vertex incident to . Lemmas \ref{condition1} and
\ref{condition2} imply that  is not incident to an edge other
than . The same can be shown for . Therefore, the path
 is a maximal - alternating path of odd
length (belongs to ). This contradiction concludes
the proof of the lemma.
\end{proof}

\bigskip
\bigskip

Lemmas \ref{S-graph}, \ref{condition1}, \ref{condition2} and
\ref{condition3} imply the following statement, which is stronger
than the Necessity of the theorem:

\begin{statement}\label{necessity}
If for a graph  the equality
 holds, then  is an
-graph \textbf{any} spanning -forest  of which satisfies
the conditions (a), (b) and (c) of the theorem.
\end{statement}

\subsection{Sufficiency}

The structure of the proof of the Sufficiency is the following: for
an -graph  with a spanning -forest  satisfying the
conditions (a) and (b) of the theorem we show that if
 then the condition (c)
is not satisfied.

The proof of the Sufficiency is more complicated than the one of the
Necessity. Therefore, we first present the idea of the proof
briefly.

Let  be an -graph with a spanning -forest  satisfying
the condition (a) of the theorem. Let us make also an additional
assumption:

\begin{assumption}\label{assumption_no_delta}
There is a pair  such that .
\end{assumption}

Choose an arbitrary pair  from .

Let  be the sets of edges from 
that belong to  respectively. Note that  are pairwise disjoint, and, due to assumption
\ref{assumption_no_delta}, we have:


Define  as the sets of all vertices that are
incident to edges from , , , ,
respectively. Note that  are pairwise disjoint, , and


Using paths from  we construct a set of trails  such that all edges lying on
trails from  that belong to  are from , and
. Note that the trails from  are connected with
edges from  (as shown in figure \ref{fig_LB_alt_cycle}) making
- alternating even cycles, i.e. each edge from 
lies on one such cycle. Also note that for any such cycle , the
graph  does not contain odd cycles as , therefore  is bipartite.

After this, assuming that the condition (b) is also satisfied and
 (it is shown that
these assumptions together are stronger than assumption
\ref{assumption_no_delta}), we prove that  contains a -
edge. Therefore, at least one of the - alternating
even cycles described above contains a - edge contradicting
the condition (c) of the theorem.

The construction of  is a step-by-step process. First, from
 we construct a set of paths  for which . Then,  is transformed to a set of
trails  for which . And finally,  is
transformed to  mentioned above.

\bigskip

Now, let us start the proof.

As mentioned above, assume that  is an -graph with a spanning
-forest  satisfying the condition (a) of the theorem, and the
assumption \ref{assumption_no_delta} holds. Choose  and
 as described above.

In order to characterize the set  define the sets  as follows:





We claim that  is a set of vertices ``potentially" incident to
edges from . Formally:
\begin{lemma}\label{QinR}
.
\end{lemma}
\begin{proof}
Assume that  and  is the edge from 
incident to . If  is not a -vertex then, due to property
\ref{coveredbyboth},  is incident to an edge from , and therefore, belongs to . On the other
hand, if  is a -vertex then, due to condition (a) and
assumption \ref{assumption_no_delta}, . Moreover,  as  belongs to  and does not belong to .
Thus, .
\end{proof}

Further, we show that in fact . We introduce  as its
definition is much easier to work with.

Consider the paths from  (they are the main working tools
throughout the whole proof).

\bigskip

From condition (a) of the theorem and assumption
\ref{assumption_no_delta} we get the following corollaries:
\begin{corollary} \label{one_is_not_incident_to_two}
Any -vertex of  is incident to at most one edge from .
\end{corollary}

\begin{corollary} \label{U_on_paths}\
\renewcommand{\labelenumi}{(\arabic{enumi})}
\begin{enumerate}
\item \label{one_is_end_vertex} If a -vertex (an edge from )
lies on a path from , then it is an end-vertex (end-edge)
of the latter.

\item \label{no_U_lies_on_cycles} No -vertex (edge from ) lies on a cycle
from .
\end{enumerate}
\end{corollary}

\begin{lemma} \label{S_on_U}
If  is an end-edge of a path  and  is an
end-vertex of  incident to , then  and  is a
-vertex of .
\end{lemma}
\begin{proof}
Let  be an end-edge of a path , and  be an end-vertex of  incident to . If 
is not a -vertex then, due to property \ref{coveredbyboth}, 
is incident to an edge . Due to the definition of
alternating path, , therefore the path  belongs to , which contradicts the maximality of
 (see the definition \ref{max_alt_path_definition} of a maximal
path).
\end{proof}

\bigskip

As a corollary from lemma \ref{S_on_U} we get:
\begin{corollary}\label{odd_S_paths_length}
Every path from  is of length at least five.
\end{corollary}

\begin{lemma}\label{cardinality_ScapU}
.
\end{lemma}
\begin{proof}
Due to property \ref{edge_lies_on_alt_component}, every edge from  lies either on a path from  or a cycle from
. Moreover, corollary \ref{U_on_paths} implies that every
edge from  is an end-edge of a path from
, hence . Corollary \ref{odd_S_paths_length} implies that
each path from  has two different end-edges. Thus, due
to lemma \ref{S_on_U}, , which completes the proof of the
lemma.
\end{proof}

\bigskip

The following two lemmas provide us with three inequalities, the
boundary cases (equalities) of which are related to a number of
useful properties. Those inequalities together imply that the
equalities are indeed the case.

\begin{lemma}\label{balance_upper_bound}
, and the equality cases in the
first and second inequalities hold if and only if  and
, respectively.
\end{lemma}
\begin{proof}
By the definition of , . Due to property \ref{cardinalitydiff}, this is
equivalent to

or

Taking into account that  we get:

and the equality holds if and only if .

Furthermore, lemma \ref{QinR} implies that , and we
get
 and the equality holds if and only if , which means that , due to lemma \ref{QinR}.
\end{proof}

\begin{lemma}\label{balance_lower_bound}
, and the equality case
holds if and only if for any path from , if its end-edge  then .
\end{lemma}
\begin{proof}
First note that


Clearly, any vertex from  is incident to one edge from  and vice versa. Furthermore, any edge from
 either belongs to , or belongs to .
Thus, due to lemma \ref{cardinality_ScapU}, we get:


Moreover, property \ref{edge_lies_on_alt_component} together with
corollary \ref{U_on_paths} implies that every vertex  is
an end-vertex of a path from , and the end-edge of that
path incident to  is from . Thus,
\setcounter{equation}{0}

and the equality holds if and only if the vice versa is also true,
i.e. any end-edge  of a path from  is
from .

Hence we have
 or

\end{proof}

\bigskip

Lemmas \ref{balance_upper_bound} and \ref{balance_lower_bound}
together imply the following:
\begin{corollary}\label{balance_corollaries}\
\renewcommand{\labelenumi}{(\arabic{enumi})}
\begin{enumerate}
\item \label{balance_equality}


\item \label{Q=R}


\item \label{lambda_remains_the_same}


\item \label{H_on_U}
If  is an end-edge of a path from 
and  is an end-vertex incident to  then  and 
is a -vertex of ;

\item \label{clear_H'_not_in_U}

\end{enumerate}
\end{corollary}

\bigskip

Statement (\ref{clear_H'_not_in_U}) of corollary
\ref{balance_corollaries} and property \ref{S-forest_HH'_is_LU}
imply:
\begin{corollary}\label{clear_H'_corollaries}\
\renewcommand{\labelenumi}{(\arabic{enumi})}
\begin{enumerate}
\item \label{clear_H' in_L}


\item \label{clear_H'_inc_vertex_in_R}
Vertices incident to edges from  belong
to .
\end{enumerate}
\end{corollary}

\bigskip

Statement (\ref{H_on_U}) of corollary \ref{balance_corollaries} and
lemma \ref{S_on_U} imply the following:
\begin{corollary}\label{paths_corollaries}\
\renewcommand{\labelenumi}{(\arabic{enumi})}
\begin{enumerate}
\item \label{length_at_least_3}
The length of any path from  is at least three;

\item \label{even_length_at_least_4}
The length of any path from  is at least four;

\item \label{every_path_is 1-1}
Every path of  connects -vertices (end-vertices are
-vertices).
\end{enumerate}
\end{corollary}

\bigskip

Now we are able to give a characterization of the set .
\begin{lemma} \label{H_in_U_is_TU}
.
\end{lemma}
\begin{proof}
Let  and  be the -vertex
incident to . The definition of  implies that , and due to statement (\ref{Q=R}) of the corollary
\ref{balance_corollaries}, . As  is a 1-vertex, , thus, . Hence, .

On the other hand, by definition . For any edge  we have
that  as , and  as  and
 (property \ref{S-forest_HH'_is_LU}). Thus, , i.e.  and the proof of the lemma is complete.
\end{proof}

\bigskip

The following lemma describes the placement of edges from  lying
on - alternating trails (maximal paths or simple even cycles):
\begin{lemma} \label{H'on_paths}\
\renewcommand{\labelenumi}{(\arabic{enumi})}
\begin{enumerate}
\item \label{end_and_beforeend} If  lies on a path  then ;
\item \label{no_H'_on_cycles} No edge from  lies on a cycle from
.
\end{enumerate}
\end{lemma}
\begin{proof}
Let  and note that if  lies on a path from  or
on a cycle from  then .

(\ref{end_and_beforeend}). Suppose that . Then,  as otherwise  (property
\ref{S-forest_HH'_is_LU}) and, due to statement
(\ref{one_is_end_vertex}) of corollary \ref{U_on_paths}, . Therefore, due to maximality of ,  is adjacent to two
edges from  lying on , one of which (denote it by
) belongs to . Due to statement (\ref{one_is_end_vertex})
of corollary \ref{U_on_paths}, , which means that .

(\ref{no_H'_on_cycles}). Suppose for contradiction that  lies on
a cycle . Here again,  as otherwise  (property \ref{S-forest_HH'_is_LU}) contradicting
statement (\ref{no_U_lies_on_cycles}) of corollary \ref{U_on_paths}.
Therefore, there are two edges from  adjacent to 
lying on , one of which belongs to , which contradicts
statement (\ref{no_U_lies_on_cycles}) of corollary \ref{U_on_paths}.
\end{proof}

\bigskip

Now let us construct the set of paths  mentioned above:

(the edges from  are considered as paths
of length one).

The following lemma provides the above-mentioned property of ,
for which  is actually constructed.
\begin{lemma} \label{R_on_paths}
.
\end{lemma}
\begin{proof}
Due to statement \ref{Q=R} of corollary \ref{balance_corollaries},
. So, assume that  and  is the
edge from  incident to . If , then
 and we are done. On the other
hand, if , then, as ,  lies on a path  (due to statement (\ref{no_H'_on_cycles}) of lemma
\ref{H'on_paths},  cannot lie on a cycle from ). Hence,
according to statement (\ref{end_and_beforeend}) of lemma
\ref{H'on_paths}, , which means that .

Now let  be a -vertex incident to the edge . Clearly,  lies on a path from 
as it cannot lie on a cycle from  (see statement
(\ref{no_U_lies_on_cycles}) of corollary \ref{U_on_paths}).
Moreover,  lies on  too, and, due to statement
(\ref{one_is_end_vertex}) of corollary \ref{U_on_paths}, is an
end-vertex of , concluding the proof of the lemma.
\end{proof}

\bigskip

Now we intend to transform  to a set of trails , such that  and any
edge from  lying on a trail from  belongs to .

For each path  of  we transform only the edges of sets
 and  (first two and last two edges of the
path). Note that corollary \ref{odd_S_paths_length} and statements
(\ref{length_at_least_3}) and (\ref{even_length_at_least_4}) of
corollary \ref{paths_corollaries} imply that  and
 are sets of cardinality  which do not coincide though
may intersect.

For each path , and for  or ,
where ,  and , do the following:

\begin{case} \label{case_einH_e'inB}
 (figure \ref{fig_trans1_1}a).

Due to statement (\ref{H_on_U}) of corollary
\ref{balance_corollaries}, . Let  be the edge from
 adjacent to  and  (figure \ref{fig_trans1_1}a).

\begin{figure}[h]
\begin{center}
\includegraphics{fig6_trans1_1.eps}\\
\caption{}\label{fig_trans1_1}
\end{center}
\end{figure}

Remove  from  and add (concatenate)  instead (figure
\ref{fig_trans1_1}b).
\end{case}

\begin{case} \label{case_einH_e'notinB}
 (hence ) (figure
\ref{fig_trans1_2}a).

\begin{figure}[h]
\begin{center}
\includegraphics{fig7_trans1_2.eps}\\
\caption{}\label{fig_trans1_2}
\end{center}
\end{figure}

Here again due to statement (\ref{H_on_U}) of corollary
\ref{balance_corollaries}, . Remove  and  from
 (figure \ref{fig_trans1_2}b).
\end{case}

\begin{case} \label{case_einS}
.

As  (lemma \ref{S_on_U}) and  (),  (figure \ref{fig_trans1_3}a).

\begin{figure}[h]
\begin{center}
\includegraphics{fig8_trans1_3.eps}\\
\caption{}\label{fig_trans1_3}
\end{center}
\end{figure}

Remove  from  (figure \ref{fig_trans1_3}b).
\end{case}

Note that the transformation described above is defined correctly
because of the following: for a path ,  and
 may have non-empty intersection only if  and the length of  is three. In this case both
 and  are handled by the case
\ref{case_einH_e'inB}, and the edge 
is treated in the same way.

Define sets of edges  and  as follows:


and let  be the set of trails obtained from the paths of
 by the transformation described above (we do not say that
 is a set of paths as it is not hard to construct an example of
 containing a trail that is not a path using the fact that vertex
 in case \ref{case_einH_e'inB} may also lie on ).

Let  be the following:


\bigskip

Some properties of  are given below.

\begin{lemma} \label{S_cap_H'_is_empty}
No edge from  lies on a trail from .
\end{lemma}
\begin{proof}
First note that no edge from  belongs to . So
we need to examine only . Statement (\ref{end_and_beforeend}) of
lemma \ref{H'on_paths} and the transformation described above imply
that the only edges lying on trails from  that belong to  are
edges denoted by  in the case \ref{case_einH_e'inB} (figure
\ref{fig_trans1_1}). But that edges do not belong to , and the
proof of the lemma is complete.
\end{proof}

\begin{lemma} \label{S_and_B_are_the_same}
If an edge  lies on a trail from   and 
then  \footnote{In other words, the set of
edges from  lying on trails from  and the set of edges
from  lying on trails from  coincide.}.
\end{lemma}
\begin{proof}
Clearly, all edges from  lying on trails from , lie on
trails from  as  (statement
(\ref{clear_H' in_L}) of corollary \ref{clear_H'_corollaries}).
First note that all edges from  lying on paths from
 belong to  as edges from  belong to 
(property \ref{S-forest_HH'_is_LU}). During the transformation of
 to , the only edges we add are edges  in case
\ref{case_einH_e'inB} (figure \ref{fig_trans1_1}), which are from
. Thus, still all edges from  lying on  belong to
, and the proof of the first part of the lemma is complete.

Let edge  lies on a trail . Obviously, .
Due to assumption \ref{assumption_no_delta} . As we do not add any edge from  during
the transformation of  to ,  does not belong to .
Due to lemma \ref{S_cap_H'_is_empty},  does not belong to 
either. Therefore, as  =  (property
\ref{S-forest_HH'_is_LU}), .
\end{proof}

\begin{lemma} \label{LB_alternating_paths}
.
\end{lemma}
\begin{proof}
Due to statement (\ref{clear_H' in_L}) of corollary
\ref{clear_H'_corollaries}, , so we need to show the same for  only.
Corollary \ref{U_on_paths} and the construction of  imply that no
edge from  lies on a trail from , therefore every edge from
 lying on a trail from  belongs to  (property
\ref{S-forest_HH'_is_LU}). On the other hand, due to lemma
\ref{S_and_B_are_the_same}, all edges from  lying on trails from
 are from . The only edges that do not belong to  are edges  in the case \ref{case_einH_e'inB}, which are
from  and are adjacent to edges from  (figure
\ref{fig_trans1_1}b). All these together imply . Moreover, the construction of  implies that the edges
from  belong to . Therefore, .
\end{proof}

\begin{lemma} \label{endvertices_of_A'_in_Q_LcupQ_B}
.
\end{lemma}
\begin{proof}
Let  be an end-vertex of a trail  from . If  then, due to statement
\ref{clear_H'_inc_vertex_in_R} of corollary
\ref{clear_H'_corollaries} and the definition of
, .
Therefore, assume that , and let  be the path from
 corresponding to . Without loss of generality, we may
assume that  is the starting vertex of . Let  be
as it is shown in the corresponding figure describing each case of
the transformation of  to .

Assume that the transformation of  is handled by cases
\ref{case_einH_e'inB} or \ref{case_einH_e'notinB} (figures
\ref{fig_trans1_1}b and figure \ref{fig_trans1_2}b). Lemma
\ref{H_in_U_is_TU} implies that . Thus,  and 
for cases \ref{case_einH_e'inB} and \ref{case_einH_e'notinB},
respectively. Therefore, as  is a -vertex (not a -vertex),
 (statement (\ref{Q=R}) of
corollary \ref{balance_corollaries}). Moreover,  as
 contains only -vertices and -vertices. Hence, .

Now assume that the transformation of  is handled by case
\ref{case_einS} (figure \ref{fig_trans1_3}b). As , we have . Since  is a -vertex (not
a -vertex),  (statement
(\ref{Q=R}) of corollary \ref{balance_corollaries}). On the other
hand, due to lemma \ref{H_in_U_is_TU},  as .
Therefore, . Thus, .
\end{proof}

\begin{lemma} \label{Q_LcupQ_B_in_endvertices_of_A'}
.
\end{lemma}
\begin{proof}
Let . The condition (a) of
the theorem implies that  is not a -vertex. Therefore, as  (statement (\ref{Q=R}) of the corollary
\ref{balance_corollaries}), , hence there
is an edge  such that  and  are
incident. The following cases are possible:

\renewcommand{\labelenumi}{(\arabic{enumi})}
\begin{enumerate}
\item \textbf{}. Then  and , where  is a path
from  (lemma \ref{H'on_paths}). Without loss of generality
we may assume that . Let  be the trail from
 corresponding to . Consider the following two subcases:
\begin{enumerate}
\item \textbf{}. Then  is the starting edge of 
(statement (\ref{one_is_end_vertex}) of corollary \ref{U_on_paths}).
Clearly, the transformation of  is handled by case
\ref{case_einS} (see figure \ref{fig_trans1_2}b,  corresponds
to  in the figure). Hence  is the starting vertex of ,
since  is not the -vertex incident to .

\item \textbf{}. Hence  as  (property \ref{S-forest_HH'_is_LU}).
Due to statement (\ref{one_is_end_vertex}) of corollary
\ref{U_on_paths}, . Thus, the
transformation of  is handled by case
\ref{case_einH_e'notinB}. Let  be as in figure
\ref{fig_trans1_2}b, and note that  corresponds to . Lemma
\ref{H_in_U_is_TU} implies that , hence the -vertex
incident to  belongs to . Since ,  is
the -vertex incident to , and therefore, is the starting
vertex of .
\end{enumerate}


\item \textbf{}. By the definitions of  and , . If , then we are
done. Therefore, assume that . The definition of
 implies that there is an edge  adjacent to
. Due to lemma \ref{H_in_U_is_TU}, , and therefore,  is an end-edge of some path  (property \ref{edge_lies_on_alt_component} and corollary
\ref{U_on_paths}). Without loss of generality, we may assume that
. Note that  as any path from  has length at least three
(corollary \ref{odd_S_paths_length} and statements
(\ref{length_at_least_3}) and (\ref{even_length_at_least_4}) of
corollary \ref{paths_corollaries}). Thus, let . As , . Let  be the
trail from  corresponding to . Since  (see
statement (\ref{clear_H' in_L}) of corollary
\ref{clear_H'_corollaries} and the definition of ) and , we have  and, due to assumption
\ref{assumption_no_delta}, . Hence, the
transformation of  is handled by case
\ref{case_einH_e'inB}, and the edges  in figure
\ref{fig_trans1_1}b correspond to , respectively. As
, the -vertex incident to  belongs to
. Therefore,  is the -vertex incident to . Thus,
 is the starting vertex of .
\end{enumerate}
\end{proof}

\bigskip

From lammas \ref{endvertices_of_A'_in_Q_LcupQ_B} and
\ref{Q_LcupQ_B_in_endvertices_of_A'} we get the following corollary:
\begin{corollary}\label{V_0=Q_LcupQ_B}
.
\end{corollary}

\bigskip

Let us prove one auxiliary lemma, which is used at the end of the
proof of Sufficiency.
\begin{lemma}\label{2-3}
The difference of the numbers of -vertices and -vertices in
 is 
\end{lemma}
\begin{proof}
Let us denote the difference of the numbers of -vertices and
-vertices in a set of vertices  by .

As  (any edge from  is incident to one
-vertex and one -vertex), , and due to corollary \ref{V_0=Q_LcupQ_B}, .

Note that for any trail  and its corresponding path 
\begin{itemize}
\item  starts with a -vertex if and only if  starts
with an edge from  (cases \ref{case_einH_e'inB} and
\ref{case_einH_e'notinB}),
\item  starts with a -vertex if and only if  starts
with an edge from  (case \ref{case_einS}).
\end{itemize}
Thus, there are  -vertices and
 -vertices in . As
 (statement (\ref{clear_H' in_L})
of corollary \ref{clear_H'_corollaries}),
. Therefore,


\end{proof}

\bigskip

Note that lemmas \ref{S_and_B_are_the_same} and
\ref{LB_alternating_paths}, and corollary \ref{V_0=Q_LcupQ_B} are
the main properties of  that were mentioned above while
describing the idea of the proof, and are used further.

\bigskip

Now, we are going to construct a set of trails  mentioned
above, which is the key point of our proof.

First, let us prove two lemmas necessary for the construction of
.
\begin{lemma}\label{TL_in_endedges}
.
\end{lemma}
\begin{proof}
First let us proof that . suppose for
contradiction that there exists an edge , such
that  is not an end-edge of any trail from . Since the set of
end-vertices of all trails from  is  (corollary
\ref{V_0=Q_LcupQ_B}), there are trails  such that
 and  are the end-vertices of  and , respectively
( and  may coincide). As ,  is adjacent to two edges from , which is
impossible because  itself belongs to . Thus,  is an
end-edge of some trail  from .

Since  and
, implies .
\end{proof}

\begin{lemma}\label{no_A'paths_share_the_same_edge}
No edge lies on two different trails from .
\end{lemma}
\begin{proof}
Clearly, the statement of the lemma is true for the following set of
paths:  Let us prove that when
 is transformed to  it still remains true. During the
transformation of  to , the only edges we add are edges
 in the case \ref{case_einH_e'inB}, which belong to  (figure \ref{fig_trans1_1}). Clearly, each such edge
 is attached to only one end of one path from , thus
lies on only one trail from . Furthermore, due to lemma
\ref{H_in_U_is_TU} the edge  shown in the figure
\ref{fig_trans1_1} belongs to . Thus  does not belong to
, i.e. . Moreover,  as the -vertex incident to  belongs to  since
is incident to . Thus, ,
and the proof of the lemma is complete.
\end{proof}

Let us introduce an operation called , using which we ``get
rid of" edges from  ``preserving" the main properties obtained
for . Assume that , and  and  are
trails from  such that ,  is the last vertex of  incident to  and
 is the starting vertex of  incident to . In this case
we say that  is a -adjacent pair of trails
corresponding to  (figure \ref{fig_trans2}a).

\begin{figure}[h]
\begin{center}
\includegraphics{fig9_trans2.eps}\\
\caption{}\label{fig_trans2}
\end{center}
\end{figure}

If  is a -adjacent pair () such that  and  do not coincide
then the  of  is a trail defined as follows
(figure \ref{fig_trans2}b):

As  and  do not have common edges, this definition is
correct, i.e.  is really a trail. Moreover,  is not a cycle,
as  and  do not coincide ( is not the reverse of
). Thus, .

Now assume that  such that no
two trails in it share a common edge. We say that the set  is a
-reduction of  if  is obtained from  by removing
arbitrarily chosen -adjacent trails 
corresponding to an edge  and, if  and 
do not coincide, adding their . The following lemma is
obviously true for :
\begin{lemma}\label{W'_properties}\
\renewcommand{\labelenumi}{(\arabic{enumi})}
\begin{enumerate}
\item \label{W'_in_T_o} ;
\item \label{preserve} no two trails in  share common edge;
\item \label{no_new_edge} no edge is added to , formally: ;
\item \label{end_edges_decrement}  and .
\end{enumerate}
\end{lemma}

Due to lemma \ref{TL_in_endedges}, for each edge from  there is
a trail , . Let  and  is the last vertex of . Similarly, as , there is a trail ,
, such that  is the starting vertex of  and is
incident to  (corollary \ref{V_0=Q_LcupQ_B}). Thus,
 is a -adjacent pair corresponding to , for each
. As no two trails from  share a common edge (lemma
\ref{no_A'paths_share_the_same_edge}),  is applicable for
-reduction operation. Consider a sequence of sets of trails
, such that  is a -reduction of
, for each . Note that  as
-reduction operation can be applied not more than 
times. Furthermore, consider such a sequence 
having maximum length. Due to lemma \ref{TL_in_endedges} and
statement (\ref{end_edges_decrement}) of lemma \ref{W'_properties},
.

Set .
\begin{lemma}\label{A''_properties}\
\renewcommand{\labelenumi}{(\arabic{enumi})}
\begin{enumerate}
\item \label{A''inT_o} ;
\item \label{S_and_B_are_the_same_in_A''} If an edge  lies on a
trail from   and  then ;
\item \label{V_0=Q_B} .
\end{enumerate}
\end{lemma}
\begin{proof}
Statement (\ref{A''inT_o}) immediately follows from statement
(\ref{W'_in_T_o}) of lemma \ref{W'_properties}. Statement
(\ref{S_and_B_are_the_same_in_A''}) is implied from lemma
\ref{S_and_B_are_the_same} and statement (\ref{no_new_edge}) of
lemma \ref{W'_properties}. As , all edges from  are
``remove'' from trails of , i.e. no edge from  lies on a
trail from . Thus no end-vertex of a trail from  belongs
to . As  (lemma \ref{V_0=Q_LcupQ_B}),
we get statement (\ref{V_0=Q_B}).
\end{proof}

The following is the main lemma of the Sufficiency part.
\begin{lemma}\label{LB_alt_cycles}
Any edge from  lies on a cycle  such
that  is bipartite.
\end{lemma}
\begin{proof}
Due to statement (\ref{V_0=Q_B}) of lemma \ref{A''_properties}, each
vertex of  is an end-vertex of a path from . On the other
hand, it is incident to an edge from . Furthermore, every
end-vertex of any path from  belongs to . Hence, every
edge of  lies on a ({\em not necessarily simple}) cycle  (), where  (figure \ref{fig_LB_alt_cycle}).

\begin{figure}[h]
\begin{center}
\includegraphics{fig10_LB_alt_cycle.eps}\\
\caption{}\label{fig_LB_alt_cycle}
\end{center}
\end{figure}

Due to statement (\ref{A''inT_o}) of lemma \ref{A''_properties} 
is an - alternating cycle, i.e. . The construction of  and statement
(\ref{S_and_B_are_the_same_in_A''}) of lemma \ref{A''_properties}
imply that  since 
and , . Therefore,
the graph  cannot contain an odd cycle, i.e. is
bipartite.
\end{proof}

\bigskip

Let us note that all lemmas and corollaries above in this subsection
are proved on the assumption that  is an -graph with spanning
-forest  satisfying the conditions (a) of the theorem and the
assumption \ref{assumption_no_delta} holds.

\bigskip

The following lemma is proved without any of the assumptions made
above.
\begin{lemma} \label{no_delta}
If  is an -graph with spanning -forest  satisfying the
condition (b) of the theorem then there is a pair  such that .
\end{lemma}
\begin{proof}
Let  be a pair from  such that  is minimum. Assume for contradiction that . Let , where  is the -vertex incident to  and
 is its base. Let  be the -vertex adjacent to  and .
The condition (b) of the theorem implies that . Without loss of generality, we may assume that . Let  be one of the two edges incident to , which does not
belong to . Define a matching  as follows:

Note that  and , which
contradicts the choice of .
\end{proof}

\bigskip

Now assume that  is an -graph with spanning -forest 
satisfying the conditions (a) and (b) of the theorem (do not assume
that the assumption \ref{assumption_no_delta} holds). Assume also
that , and therefore
 (property
\ref{S-forest_lambda=2alpha=8k}). Thus, . This, together with lemma \ref{no_delta}, implies that
assumption \ref{assumption_no_delta} holds for . Thus, all of the
lemmas and corollaries proved in this section are true for .

\begin{lemma} \label{MP^S>MP^H}
.
\end{lemma}
\begin{proof}
Corollary \ref{fivefourthinequality} and property
\ref{S-forest_lambda=2alpha=8k} imply that
. As  (property
\ref{S-graph_beta=5k}), , which means that
. Thus, due to property \ref{cardinalitydiff}, we get:

\end{proof}

Lemmas \ref{2-3} and \ref{MP^S>MP^H} together imply
\begin{corollary} \label{at_least_one_2-2}
There is at least one - edge in .
\end{corollary}

Let  be one of - edges from . Due to lemma
\ref{LB_alt_cycles},  lies on a cycle 
such that  is bipartite. Hence,

\begin{corollary} \label{does_not_satisfy_c}
The graph  does not satisfy the condition (c) of the theorem.
\end{corollary}

\bigskip

Let us not that corollary \ref{does_not_satisfy_c} is proven on the
assumption that  is an -graph with spanning -forest 
satisfying the conditions (a) and (b) of the theorem, and
. Clearly, this is
equivalent to the following:

\begin{statement}\label{sufficiency}
If  is an -graph with spanning -forest  satisfying the
conditions (a), (b) and (c) of the theorem then
.
\end{statement}


\subsection{Remarks}

\begin{remark}
Statements (\ref{necessity}) and (\ref{sufficiency}) imply that the
theorem can be reformulated as follows:

\textit{For a graph  the equality
 holds, if and only if 
is an -graph, \textbf{any} spanning -forest of which satisfies
the conditions (a), (b) and (c).}
\end{remark}

\begin{remark}
Due to property \ref{2_and_3_are_equal}, the condition (c) of the
theorem can be changed to the following:

\textit{For every - alternating even cycle  of 
containing a - or - edge, the graph  is not
bipartite.}
\end{remark}

\vspace*{2cm}

\begin{acknowledgements}
\begin{center}
\end{center}
I would like to express my sincere gratitude to my supervisor Dr.
Vahan Mkrtchyan, for stating the problem and giving an opportunity
to perform a research in such an interesting area as Matching theory
is, for teaching me how to write articles and helping me to cope
with the current one, for directing and correcting my chaotic ideas,
for encouraging me, and finally, for his patience.

During the research process I also collaborated with my friend and
colleague Vahe Musoyan, who I want to thank for his great investment
in this work, for helping me to develop the main idea of
 the theorem, and for numerous of counter-examples that corrected my
statements as well as rejected some of my hypotheses.
\end{acknowledgements}


\newpage

\begin{center}

\begin{thebibliography}{0}

\bibitem{Balas} E. Balas, Integer and Fractional Matchings, Studies
on Graphs and Discrete Programming, Ed.: P. Hansen, Ann. Discrete
Math., 11, North-Holland, Amsterdam, 1981, pp.1-13.

\bibitem{Har} F. Harary, Graph Theory, Addison-Wesley, Reading, MA,
1969.

\bibitem{Har-Plum} F. Harary, M. D. Plummer, On the Core of a Graph,
Proc. London Math. Soc. 17 (1967), pp. 305-314.

\bibitem{Holyer} I. Holyer, The NP-Completeness of Edge-Coloring,
SIAM J. Comput. 10(4), 1981, pp. 718-720

\bibitem{alg} R. R. Kamalian, V. V. Mkrtchyan, Two Polynomial
Algorithms for Special Maximum Matching Constructing in Trees,
Discrete Applied Mathematics, 2007 (submitted)

\bibitem{np} R. R. Kamalian, V. V. Mkrtchyan, On Complexity of Special
Maximum Matching Constructing, Discrete Mathematics (to appear)

\bibitem{Lov-Plum} L. Lovasz, M. D. Plummer, Matching Theory, Ann. Discrete Math.
29 (1986).

\bibitem{MPP01} V. V. Mkrtchyan, On Trees with a Maximum Proper
Partial 0-1 Coloring Containing a Maximum Matching, Discrete
Mathematics 306, (2006), pp. 456-459.

\bibitem{Perfect} V. V. Mkrtchyan, A Note on Minimal Matching Covered Graphs, Discrete
Mathematics 306, (2006), pp. 452-455.

\bibitem{VAV} V. V. Mkrtchyan, V. L. Musoyan, A. V. Tserunyan, On
Edge-Disjoint Pairs of Matchings, Discrete Mathematics, (2006),
(Submitted).

\bibitem{West} D. B. West, Introduction to Graph Theory,
Prentice-Hall, Englewood Cliffs, 1996.

\end{thebibliography}
\end{center}


\end{document}
