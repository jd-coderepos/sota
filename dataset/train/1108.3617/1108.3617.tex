\documentclass[submission,copyright,creativecommons]{eptcs}
\providecommand{\event}{WORDS 2011} \usepackage{breakurl}             

\usepackage{amssymb,amsfonts, latexsym}
\usepackage{verbatim}

\newcommand{\N}{\mathbb N}
\newcommand{\s}[1]{\mathcal{#1}}
\newcommand{\ov}[1]{\overline{#1}}
\newcommand{\alp}{{\rm alph}}
\newcommand{\h}{\mathtt H}
\newcommand{\g}{\mathtt G}

\newcounter{dep}
\newenvironment{luet}
{
\setlength{\parskip}{5pt}
\begin{list}
{{\bf \arabic{dep}.}}{\usecounter{dep}\setlength{\topsep}{1pt}\setlength{\rightmargin}{\leftmargin}\setlength{\parsep}{3pt}\setlength{\itemsep}{2pt}}
}
{\end{list}}

\newcounter{dol}
\newenvironment{yks}
{
\setlength{\parskip}{1pt}
\begin{list}
{}{\usecounter{dol}\setlength{\topsep}{0pt}\setlength{\rightmargin}{\leftmargin}\setlength{\parsep}{0pt}\setlength{\itemsep}{0pt}}
}
{\end{list}}

\newtheorem{theorem}{Theorem}
\newtheorem{remark}{Rermark}
\newtheorem{corollary}{Corollary}

\title{Combinatorics on words in information security: Unavoidable regularities
 in the construction of multicollision attacks on iterated hash functions}

\author{Juha Kortelainen
\institute{Department of Information Processing Science, University of Oulu, Finland}
\email{jkortela@tols16.oulu.fi}}

\def\titlerunning{Combinatorics on words in information security}
\def\authorrunning{J. Kortelainen}

\begin{document}
\maketitle

\begin{abstract}
Classically in combinatorics on words one studies unavoidable regularities that appear in sufficiently 
long strings of symbols over a fixed size alphabet. In this paper we take another viewpoint and focus on 
combinatorial properties of long words in which the number of occurrences of any symbol is restritced by 
a fixed constant. We then demonstrate the connection  of these properties to constructing multicollision
attacks on so called generalized iterated hash functions.
\end{abstract}


\section{Introduction}

In combinatorics on words, the theory of 'unavoidable regularities' usually concerns properties of long words over a fixed finite alphabet. Famous classical results in general combinatorics and algebra such as theorems of Ramsey, Shirshov and van der Waerden can then be straightforwardly exploited (\cite{Har}, \cite{dLV}, \cite{Res}, \cite{RR1}, \cite{RR2}). The theory can be applied in the study of finiteness conditions for semigroups and (through the concept of syntactic monoid) also in regular languages and finite automata. 
To give the reader a view of the traditional basic results in unavoidable regularities we list some of its most noteworthy achievements.

Ramsey's Theorem immediately implies 

\begin{theorem}[Repeated Patterns \cite{Har}] For all positive integers  and  there exists a positive integer  satisfying the following. Given an alphabet  and a partition  of  into  sets, if  is any word of length at least , then  is in  for some .
\end{theorem}

Let  be an alphabet totally ordered by . We extend the order  to the \textit{lexiographic order}  of  as follows. For all :  if either  or  and  for some  and  for which .

Given a positive integer , the word  is \textit{-divided} if there exist words  in  such that  and 

for any nontrivial permutation . 

\begin{theorem}[Shirshov \cite{Lot,dLV,RR1}] Let  be an alphabet of  symbols and  and  positive integers with . There then  exists a positive integer  such that any word in  of length at least  either is -divided or contains a th power of a nonempty word of length at most .
\end{theorem}

Let  where  for . A \textit{cadence} of  is any sequence  of integers such that

Here the number  is the \textit{order} of the cadence. The cadence  is \textit{arithmetic} if there exists a positive integer  such that  for .

The celebrated van der Waerden's theorem can be reformulated in words as follows.

\begin{theorem}[van der Waerden \cite{Lot,dLV}] Let  be an alphabet of  symbols and  a positive integer. There then exists a positive integer  such that any word in  of length at least  possesses an arithmetic cadence of order .
\end{theorem}

Combinatorial problems are also encountered in information security, for example, when designing and investigating hash functions, techniques used in message authentication and digital signature schemes. A \textit{hash function of length } where  is a mapping . For computing resource reasons, practical hash functions are often \textit{iterative}, i.e., they are based on some finite compression function and an initial hash value. For more details, see subsection \ref{bc2}.

An ideal hash function  is a \textit{variable input length random oracle}: for
each , the value  is chosen uniformly at random. 

There are three main security properties that usually are required from a hash function : \textit{collision resistance}, \textit{preimage resistance}, and \textit{second preimage resistance}.

\textbf{Collision resistance:} It is computationally infeasible to find , , such that .

\textbf{Preimage resistance:} Given any , it is computationally infeasible to find  such that .

\textbf{Second preimage resistance:} Given any , it is computationally infeasible to find , , such that .

If we want to consider the resistance properties mathematically, the concept 'computationally infeasible' should be rigorously defined. Then the security of  is compared to the security of a random oracle. 

We thus say that  is collision resistant (or possesses the collision resistance property) if to find , , such that  is (approximately) as difficult as to find , , such that  for any random oracle hash function  of length .

The concepts of preimage resistance and second preimage resistance can be defined analogously. 

Given a set  of finite cardinality , we say that  is an \textit{-collision on } if  for all .  Any -collison is also called a collision (on ). 

The sharpened definitions allow us to define a fourth security property, the so called multicollision resistance: The hash function  is \textit{multicollision resistant} if, for each , to find an -collison on  is (approximately) as difficult as to find an -collison on any random oracle hash function  of length .

Our conciderations are connected to multicollison resistance. Given a message   where  are the (equally long) blocks of , the value of a generalized iterated hash function on  is based on the values of a finite compression function on the message blocks . A nonempty word  over the alphabet  may then tell us in which order and how many times each block  is expended by the compression function when producing the value of the respective generalized iterated hash function. Since the length of messages vary, we get to consider sequences of words  in which, for each , the word  is related to messages with  blocks. Practical applications state one more limitation: given a message of any length, a fixed   block is to be consumed by the compression function only a restricted number (, say) of times when computing the generalized iterated hash function value. Thus in the sequence  we assume that for each  and , the number  of occurrences of the symbol  in the word  is at most . 

What can be said about the general combinatorial properties of the word  when  grows? More generally: which kind of unavoidable regularities appear in sufficiently long words in which the number of occurrences of any symbol is bounded by a fixed constant? 

As is easy to imagine, the regularities in the words  weaken the respective generalized iterated hash function against multicollision attacks. This topic was first studied in \cite{HoS}, see also \cite{Jou,NaS,HKK,KHK,KKH,KKV}. We shall present combinatorial results on words which imply that -bounded generalized iterated hash functions are not multicollision resistant.  

We proceed in the following order. In the next section basic concepts are briefly given. In the third section we first introduce the basics of generalized iterated hash functions. The connection to combinatorics on words is then established. The fourth section contains the necessary combinatorial results. Finally, the last section contains conclusions and further research proposals.


\section{Preliminaries}

Let  be the set of all natural numbers and . For each finite set , let  be the \textit{cardinality} of  that is to say, the number of elements in .

Let  be a finite alphabet and . The length of the word  is denoted
by ; for each , let  be the number of occurrences of the letter
 in , and let  denote the set of all letters occurring in  at
least once. The empty word is denoted by . A permutation of  is any word  such that  for each .

Let . Then the \textit{projection morphism} from  into , denoted by   is defined by  if  and  if . We write  instead of  when  is understood. Define the word  as follows:  if  and  if , where , , and  for .

\section{Hash functions and collisions}

In this section we first present a  compact lead-in  to (generalized) iterated hash functions. Later we wish to point out how certain results in  combinatorics on words are interconnected to successful multicollision construction on these type of hash functions. 

\subsection{Introduction to (generalized) iterated hash functions}\label{bc2}

Let  be such that . Then  is the set of \textit{hash values} (of length ) and ) is the set of \textit{message blocks} (of length ). Any  is a \textit{message}. Given a mapping  , call  a \textit{compression function} (of length  and block size ).  

Define the function    inductively as follows. For each ,  and , let  and
. Note that  is nothing but an iterative generalization of the compression function .

Let  and  be a nonemptyword such that . Then , where  and  for . Define the \emph{iterated compression function}  (based on 
and ) by 

for each  and . Note that clearly  only declares how many times and in which order the message blocks  are used when creating the (hash) value  of the message . 

Given  and , a \textit{-collision with  initial value  in the iterated compression function } is a set  such that the following holds:
\begin{luet}
\item[] The cardinality of  is ; 
\item[] For  all  we have ; and
\item[] For any pair of distinct messages  and  in  such that  for , there  exists  for which .
\end{luet}

For each , let now  be such that
. Denote . Define the \emph{generalized iterated hash function} (a  for short)
 (based on  and ) as follows: Given the
initial value  and the message , , let


Thus,  given any message  of  blocks and hash value , to obtain the value , we just pick the word  from the sequence  and compute . For more details, see \cite{KHK} and \cite{HoS}.

\begin{remark}\label{Tra}
A traditional iterated hash function  based on  (with initial value ) can of
course be defined by  for each
. On the other hand  is a generalized iterated
hash function  based on  and 
where  and the initial
value is fixed as . Note that almost all hash functions used nowadays in practise are of this form.  
\end{remark}

Given  and , a \textit{k-collision in the generalized iterated hash function  with initial value } is a set
 of  messages such that for all ,
 and . Now suppose
that  is a -collision in  with
initial value . Let  be such that
, i.e., the length in blocks of each message in
 is . Then, by definition, for each , the equality
 holds. Since
 (and thus each symbol in  occurs in ), the set  is a -collision
in  with initial value . Thus, a -collision in the generalized iterated hash function  necessarily by definition, is a -collision in the iterated compression function  for some .

Now, in our security model, the \textit{attacker} tries to find a -collision in . We assume that the attacker knows how  depends on the respective compression function  (i.e., the attacker knows ), but sees  only
as a black box. She/he does not know anything about the internal
structure of  and can only make \textit{queries} (i.e., pairs ) on
 and get the respective \textit{responses} (values ).

We thus define the \textit{message complexity of a -collision} in  to be the expected number of queries on the compression function  that is needed to create a multicollision of size  in  with any initial value .

According to the (generalized) \textit{birthday paradox}, a -collision for any compression function  of length  can be found 
(with probability approx. ) by hashing   messages \cite{STKT} if 
we assume that there is no memory restrictions. Two remarks can be made immediately: 

\begin{itemize}
\item[] In the case  approximately  hashings (queries on ) are needed; intuitively many of us 
would expect the number to be around .
\item[] For each  in , finding a -collision consumes much more resources than finding a 
-collision.
\end{itemize}

Of course, when attacking, for instance, against an iterated hash function based on a random oracle compression function of length , the attacker needs a lot of computing power when  is large; to create a -collison requires approximately  queries on   and this is resource consuming.   

The paper \cite{Jou} presents a clever way to find a -collision in the traditional iterated hash function  (see Remark \ref{Tra}) for any .   
The attacker starts from the initial value  and searches two distinct message blocks ,  such that  and denotes  
. By the birthday paradox, the expected number of queries on  is  , where  is approximately .
Then, for each , the attacker continues by searching message blocks  and  such that  and 
 and and stating . Now the set 
 is -collision in . The expected number of queries on  is clearly , i.e., the work the attacker is expected to do is only  times greater than the work she or he has to do to find a single -collision. The size of the multicollision grows exponentially while the need of resources increases linearly.  

The question arises whether or not the ideas of Joux can be applied in a more broad setting, i.e., can Joux's approach be used to multicollisions in certain generalized iterated hash functions? 

In the following we shall see that this indeed is possible. Call the sequence 
\textit{-bounded}, , if  for each  and . The   is \textit{-bounded} if  is -bounded. Note that Joux's method is easy to apply to any -bounded generalized iterated hash function. 

Is it possible to extend Joux's method furthermore to be adapted to -bounded s, when ? This question has been investigated first for -bounded s in \cite{NaS} and then for any -bounded  in \cite{HoS} (see also \cite{KHK}). It turned out that it is possible to create -collision in any -bounded  with   queries on , where  is function of  and  which is polynomial with respect to  and  but double exponential with respect to . 

The idea behind the successful construction of the attack is the fact that since
 is -bounded, unavoidable regularities start to appear
in the word  of  when  is increased. More
accurately, choosing  large enough, yet so that  depends only
polynomially on  and  (albeit double exponentially in ), a number  and a set  of cardinality  can be found such that
\begin{yks}
\item[(P1)]  the word 
   is a permutation of  for ; and
\item[(P2)] for any , if
   is a factorization of
   such that  for  and  is a
  factorization of  such that 
  for , then for each 
  , there exists   such
  that .
\end{yks}

The property (P1) allows the attacker construct a -collision  in
 with any initial value  so that the expected number
of queries on  is . The property (P2) ensures that based on the multicollision  guaranteed by (P1),
the attacker can proceed and, for , create a -collision  in  so that the expected number of queries on  is . Thus finally a -collision of complexity  in 
 is generated.

Finally on the basis of the previous attack construction and (the future) Theorem \ref{main2}, the following can be proved  (\cite{KKV}).

\begin{theorem}\label{main3}
Let  and  be positive integers such that  and ,  a compression function, and  a -bounded sequence of words such that  for each
. Then, for each , there exists a -collision attack on the generalized iterated hash function  such that the expected number of queries on  is at most .
\end{theorem}

\begin{remark}
The inequality  (see Theorem \ref{main1}) implies that  
\end{remark}

The results in \cite{STKT} imply that, given a random oracle hash function  of length , the expected number of queries on  to find a -collision is in . 

Call a generalized iterated hash function bounded if it is -bounded for some . 

\begin{corollary}
There does not exist a bounded generalized iterated hash function that is multicollision resistant.
\end{corollary}

\subsection{Essential combinatorial results}

We state a list of combinatorial results that imply Theorem \ref{main3}. The main result in stated is the form of classical combinatorial theorems. For a proof, see \cite{KKV}.  

\begin{theorem} \label{main1}
For all positive integers   and  there exists a (minimal) positive integer  such that if  is a word
for which  and  for
each , there exist 
with , and , as well as words
 such that  and for all , the word  is a permutation of . Moreover, for all  we have .
\end{theorem}


\begin{remark} \label{rem1}
Let . In the case , the previous theorem gives us the boundary value . Let 

be an alphabet of  symbols. Let furthermore 

for  and . It is quite straightforward to see that there does not exist an -letter subalphabet of  such that either\, (i)  is a permutation of  or\  \hbox{(ii) there} exists a factorization  such that  and  are both permutations of . Thus  for . 
\end{remark}

Suppose now that  and  are as in Theorem \ref{main1}, i.e., for all , the word  is a permutation of . To make our multicollision attack succeed, this is not yet sufficient. We need permutations  , , ,  of an sufficiently large alphabet  such that when factoring  into   equal length factors for  where  divides  and the following holds: for each  and  there exists  such that . Only then we can, starting from the first permutation (and the word ) roll on our attack well. Above the permutations  are induced by the words , respectively, when  is long enough (or equivalently, the alphabet  is sufficiently large). That these permutations always can be found, is verified in the following three combinatorial results. 


We wish to further study the mutual structure of permutations in long words guaranteed by Theorem \ref{main1}. By increasing the length of the word  the permutations are forced to possess certain stronger structural properties. The motives are, besides our interest in combinatorics on words, in information security applications. The connection of the results to creating multicollisions on generalized iterated hash functions is more accurately, albeit informally, described in Section 5.

As the first step of our reasoning we need an application of the famous Hall's Matching Theorem. For the proof, see  \cite{KHK} and \cite{HoS}.

\begin{theorem}[Partition Theorem] \label{part}
Let  and  be a finite nonempty set such that  divides
. Furthermore, let  and  be
partitions of  such that  for . Then
for each  such that , there exists a
bijection  for
which  for .
\end{theorem}

The next theorem is also from \cite{KHK}. It is an inductive generalization of Partition Theorem to different size of factorizations. For the proof, see \cite{KHK}.

\begin{theorem}[Factorization Theorem] \label{perint}
Let , where , be positive integers
such that  divides  for ,  an
alphabet of cardinality , and
 permutations of . Then there exists a
subset  of  of cardinality  such that the following
conditions are satisfied.
\begin{enumerate}
\item[\upshape{(1)}] For any , if
   is the factorization of 
   and  is the
  factorization of  into  equal length
   blocks, then for each ,
  there exists  such that
  ; and
\item[\upshape{(2)}] If  is the factorization  into  equal length
   blocks, then    
  is the factorization of 
  into  equal length  blocks.
\end{enumerate}
\end{theorem}

In fact what we need in our considerations is the following 

\begin{corollary}
Let  and  be positive integers such that  divides ,  an
alphabet of cardinality , and
 permutations of . Then there exists a
subset  of  of cardinality  satisfying the following. Let  and  the factorization of  and  the   factorization of  into  equal length  blocks, then for each , there exists  such that .
\end{corollary}

The last result of this section combines the main result of this section (Theorem \ref{main1}) to the previous combinatorial accomplishments. Theorem \ref{main2} is indispensable for the attack constrution in the end of Section \ref{bc2}.

\begin{theorem}\label{main2}
Let  be a word and  , , and  integers such that 
\begin{enumerate}
\item[] ; and 
\item[\rm{(2)}]  for each \ .
\end{enumerate}
Then there
exists ,  and a
factorization  for which
\begin{enumerate}
\item[\rm{(3)}] ; 
\item[\rm{(4)}]  and  is a permutation of  for
  ; and
\item[\rm{(5)}] For any , if
   is the factorization of
  of  into  equal length 
  blocks and  the
  factorization of  into  equal length
   blocks, then for each ,
  there exists   such that
  .
\end{enumerate}
\end{theorem}

\section{Conclusion}

We have considered combinatorics on words from a fresh viewpoint which is induced by applications in information security. Some small steps have already been taken in the new research frame. The results have been promising; they imply more efficient attacks on generalized iterated hash functions and, from their part, confirm the fact that the iterative structure possesses certain generic security weaknesses. 

\medskip

\noindent \textit{Research Problem}.\ \ Consider Theorem \ref{main1}. The exact value of  is known only in the cases ,  and : Trivially  and , furthermore  (see Remark \ref{rem1}). It is probable that in general the number  is significantly smaller than . Moreover, we have not evaluated  from below at all. Find reasonable lower and upper bounds to  for .  

\bibliographystyle{eptcs}
\bibliography{kort}


\end{document}
