
\pdfoutput=1
\documentclass[11pt]{article}
\usepackage{amssymb}
\usepackage{graphicx}
\newtheorem{theorem}{Theorem}
\newtheorem{corollary}[theorem]{Corollary}
\newtheorem{definition}[theorem]{Definition}
\newtheorem{example}[theorem]{Example}
\newtheorem{lemma}[theorem]{Lemma}
\newtheorem{proposition}[theorem]{Proposition}
\newtheorem{remark}[theorem]{Remark}
\newenvironment{proof}[1][Proof]{\textbf{#1.} }{\
\rule{0.5em}{0.5em}}
\topmargin -0.5in
\textheight 23.5cm
\oddsidemargin 0cm
\textwidth 16cm
\parindent 0mm
\parskip \baselineskip
\newcommand{\ds}{\displaystyle}
\newcommand{\R}{\mathbb{R}}
\newcommand{\N}{\mathbb{N}}
\newcommand{\C}{\mathbb{C}}
\newcommand{\Q}{\mathbb{Q}}
\newcommand{\Z}{\mathbb{Z}}
\newcommand{\LvP}{\mathbb{P}}
\newcommand{\ol}{\overline}
\newcommand{\ul}{\underline}
\newcommand{\ra}{\rightarrow}
\newcommand{\Ra}{\Rightarrow}
\newcommand{\LRa}{\Leftrightarrow}
\newcommand{\name}[2]{{#1}{\scriptsize{#2}}}
\renewcommand{\thefootnote}{\arabic{footnote}}
\newcommand{\bmax}{\hbox{\boldmath}}
\newcommand{\bempt}{\hbox{\boldmath}}

\begin{document}
\title{The joy of implications, aka pure Horn formulas: mainly a survey}

\author{Marcel Wild}

\date{}
\maketitle


\begin{quote}
A{\scriptsize BSTRACT}: Pure Horn clauses have also been called (among others) functional dependencies, strong association rules, or simply implications. We survey the mathematical theory of implications with an emphasis on the progress made in the last 30 years.
\end{quote}

{\bf Key words:}

pure Horn functions and their minimization, Boolean logic, association rule, lattice theory, Formal Concept Analysis, closure system, convex geometry, prime implicates, meet-irreducibles, universal algebra.


\section{Extended introduction}

This article is devoted to the mathematics and (to lesser extent) algorithmics of implications; it is mainly a survey of results obtained in the past thirty years but features a few novelties as well. 
The theory of implications mainly developed, often under mutual ignorance, in these five fields:

 Boolean Function Theory, Formal Concept Analysis, Lattice Theory, Relational Database Theory, Learning Theory.

As standard text-books in these fields we recommend [CH], [GW], [Bi]   [G], [MR2]  [M], and [RN, ch.VI]  [FD] respectively. Broadly speaking we collect from each field only those major results that concern (or can be rephrased in terms of) ``abstract implications'', and {\it not} the substance matter of the field itself. There are three minor exceptions to this rule. First, there will be two detours (Subsections 4.1, 4.2) into lattice theory; among the five fields mentioned this is the one the author is most acquainted with. Second, in Subsection 1.1 just below, in order to motivate the theory to come, we glance at three ``real life'' occurencies of implications in these areas:
Relational Databases,
Formal Concept Analysis, and Learning Spaces. The third exception concerns 3.6; more on that later.
The second part (1.2) of our extended introduction gives the detailed section break up of the article.


{\bf 1.1} We shall only give very rudimentary outlines of three areas mentioned above; more detailed accounts of 1.1.1 to 1.1.3 are found in [MR2], [GW], [FD].  The sole purpose here is to convey a feeling for the many meanings that a statement `` implies '' can have. This will contrast with the uniform mathematical treatment that all ``abstract'' implications  obey.

\begin{center}
\includegraphics[scale=0.5]{JoyOfImplicFig1}
\end{center}

{\bf 1.1.1} As to relational database (RDB), imagine this as a large array in which every row (called record) corresponds to a particular object , and in which the columns correspond to the various attributes  
that apply. See Figure 1. Each attribute has a domain which is the set of values that it may assume. Following an example of J. Ullman, take a relational database whose records match the ``teaching events'' occuring at a university in a given semester. The attributes are  course,  teacher,  hour,  room,  student. The domain of  may be algebra, analysis, lattice theory, , the domain of  could be Breuer, Howell, Janelidze, , and so forth. If  are sets of attributes then the validity of   means that {\it any two objects which have identical values for all attributes in , also have identical values for all attributes in .} Examples of implications  (also called {\it functional dependencies}) that likely hold in a well designed database include the following:  (each course has one teacher),  (only one course meets in a room at one time),  (a student can be in only one room at a given time).

{\bf 1.1.2} Let now  and  be any sets and  be a binary relation. In Formal Concept Analysis (FCA) one calls the triple  a {\it context}, and  is interpreted as the {\it object}  having the {\it attribute} . If  then the validity of   has a different\footnote{One may view a context as a RBD all of whose attribute domains are Boolean, thus  or . But depending on viewing it as RBD or context, different implications hold.} ring from before:  {\it Any object that has all attributes in , also has all attributes in } (see also 2.1.2 and 2.2.3). 

Let us focus on particular contexts of type . Thus the objects  become {\it subsets}  of some set  of {\it items}. Saying that  ``has attribute''  now just means . Often the sets  are called {\it transactions}, and the elements  are called {\it items}. 
If  then   is a valid implication iff {\it every transaction  that contains the itemset , also contains the itemset }. For instance, each transaction can contain the items a customer  bought at a supermarket on a particular day. In this scenario a plausible implication e.g. is butter, breadmilk. Notice that  may be a valid implication simply because many transactions do not contain  at all. To exclude this possibility one often strengthens the previous definition of ``valid implication'' by additionally demanding that say 70\% of all transactions must contain the itemset . The terminology ``transaction'' and ``itemset'' is borrowed from Frequent Set Mining (FSM), a paradigm that developed in parallel to FCA for a long time, despite of close ties.  See also 3.6.3.4.

{\bf 1.1.3} As to Learning Spaces [FD], these are mathematical structures applied in mathematical modeling of education. In this framework (closer in spirit to [GW] than to [RN] type learning theory) the validity of an implication  means the following: {\it Every student mastering the (types of) problems in set  also masters the problems in set .} See also Expansion 16.



{\bf 1.2} Some readers may have guessed that this zoo of implications fits the common hat of pure Horn functions, i.e. Boolean functions like  and conjunctions thereof. While this is true the author, like others, has opted for a more {\it stripped down formalism}, using elements and sets rather than literals and truth value assignments, etc. Nevertheless, discarding pure Horn function terminology altogether would be short-sighted; certain aspects can only be treated, in any sensible way, in a framework that provides immediate access to the empire of general Boolean function theory that e.g. houses prime implicates and the consensus algorithm.

Without further mention, all structures considered in this article will be {\bf finite}. Thus we won't point out which concepts extend or can be adapted to the infinite case. A word on [CH, chapter 6, 56 pages] is in order. It is a survey on Horn functions to which the present article (PA) compares as follows. Briefly put, the intersection  is sizeable (though not notation-wise), and so are  (e.g. applications, dualization, special classes), as well as  (e.g. 3.6 and 4.1 to 4.4). We note that 4.1 also features special classes but {\it others}.



Here comes the section break up. Section 2 recalls the basic connections between closure operators  and closure systems  (2.1), and then turns to implications ``lite'' in 2.2. Crucially, each family  of implications  gives rise to a closure operator  and whence to a closure system . Furthermore, {\it each} closure operator  is of type  for suitable . Section 3 is devoted to the finer theory of implications. Centerpieces are the Duquenne-Guigues implicational base (3.2) and the canonical direct base in 3.3. Subsection 3.4 is about mentioned pure Horn functions, 3.5 is about acyclic and related closure operators, and 3.6 surveys the connections between two devices to grasp closure systems . One device is any implicational base, the other is the subset  of meet-irreducible closed sets. 

Section 4 has the title ``Selected topics''.
In 4.1 the attention turns from meet to join-irreducibles, i.e. we show that {\it every} lattice  gives rise to a closure system  on its set  of join irreducibles. Consequently it makes sense to ask about optimum implicational bases  for various types of lattices. We have a closer look at modular, geometric and meet-distributive lattices. The other topics in brief are: an excursion into universal algebra (4.2), {\it ordered} direct implicational bases (4.3), an algorithm for generating  in compact form (4.4), and general (impure) Horn functions in 4.5. According to Theorem 6 implications ``almost'' suffice to capture even impure Horn functions.


In order to have full proofs of some results without interrupting the story line, we store these proofs in little ``boxes'' (called Expansion 1 to Expansion 20) in Section 5. Most of these results are standard; nevertheless we found it worthwile to give proofs fitting our framework. Some Expansions simply contain further material. Due to space limitations the full versions of some Expansions are only available in the preliminary draft [W7].

Recall that this article attempts to survey the {\it mathematical theory} of pure Horn functions ( implications), and apart from mentioned exceptions {\it not} their applications.  Our survey also includes a couple of new results, mainly in 2.2.5, 3.3.2,  3.4.3, 4.1.6, in Expansion 8 and in (33). Further Theorem 3 and 6 are new. In order to stimulate research four {\it Open Problems} are dispersed throughout the text (in 3.6.2, \ Expansion 5, \ Expansion 15). 


\section{The bare essentials of closure systems and implications}

Everything in Section 2 apart from 2.2.5 is standard material.
Because of the sporadic appearance of contexts (1.1.2) a good reference among many is [GW].






\subsection{Closure systems and closure operators}

A {\it closure system}  with universe  is a subset of the powerset  with the property that

(1) \quad  for all .

Here  denotes the intersection of all sets contained in . Its smallest element is  and, crucially, it has a largest element as well. Namely, as a matter of taste, one may either postulate that  belongs to , or one may argue that  implies , and that . Thus  is the largest closure system with universe , and  is the smallest. The members  are called {\it closed} sets, and  is 
{\it meet-irreducible} if there are no strict closed supersets  and  of  with . We write  for the set of meet irreducibles of . It is clear that

(2)  \quad  is closure system )




{\bf 2.1.1} Closure systems are linked to closure operators\footnote{We recommend [BM, sec.6] for a historic account of the origins of these two concepts.}. (The link to lattices is postponed to 4.1.) Namely, {\it closure operators} are maps  which are extensive (), idempotent ( and monotone (. In this situation (see Expansion 1)

(3) \quad  is a closure system.

As to the reverse direction, 
if  is a closure system then

yields a closure operator . 
One can show [GW, Theorem 1] that  and . One calls  a {\it generating set} of  if . On a higher level  is a {\it generating set} of  if  equals . It is easy to see that  is a generating set of  iff . In this case
 can also be calculated as 

(4) \quad .


The first idea that springs to mind to calculate  from  is to keep on calculating  until . Unfortunately the approach is doomed by the frequent recalculation of closed sets, and the need to keep large chunks of  in central memory. A clever idea of C.E. Dowling [FD, p.50] avoids the recalculations, but not the space problem; see also Expansion 4. 



{\bf 2.1.2.} Here comes a frequent source of closure operators. Let  be sets and let  be a binary relation. 
For all  and  put 

Then the pair  yields a {\it Galois connection}. It is easy to see that  iff . Furthermore, it holds [GW, Section 0.4] that  is a closure operator , and  is a closure operator .  
For instance, let  be a context in Formal Concept Analysis as glimpsed in 1.1.2. If  is any set of attributes then  is the set of attributes  enjoyed by every object , i.e. by every object  that has all attributes of .  Put another way,  is a ``valid'' implication in the sense that whenever  has all attributes in , then  has all attributes in . This matches our discussion of ``implications''  in 1.1.2. See [PKID1] for a survey of 1072 papers dedicated to applications of FCA.



\subsection{Implications ``lite''}

A pair of subsets  will be called an {\it implication}. Both  or  are allowed. (See 3.4.2 for the full picture). We shall henceforth write  instead of  and call  the {\it premise} and  the {\it conclusion} of the implication.  Any family

(5) \quad 

of implications gives rise to a closure operator as follows. Putting  for any set  we define

(6) \quad .

By finiteness the chain  stabilizes at some set . This algorithm matches {\it forward chaining} in [CH, 6.2.4]. We call  the -{\it closure} of . It is clear that the function  is a closure operator on . As to speeding up the calculation of  see Expansion 2.
 It is evident that  implies  for all , but say  does not entail .
 By (3) the closure operator  induces a closure system . Hence for all  it holds that
 
 (7) \quad  \ or \ 

Skipping , it is easy to show directly that for any given family  of implications the sets  with , for all  constitute a closure system.

{\bf 2.2.1}  We say that  is {\it equivalent} to  (written ) if the closure operators  and  coincide. There are three obvious (and others in 3.4) notions of ``smallness'' for families  of implications as in (5):
\begin{itemize}
	\item  is {\it nonredundant} if  is not equivalent to  for all .
	\item  is {\it minimum} if  equals .
	\item  is {\it optimum} if   equals\\
	 .
\end{itemize}
For instance,  is {\it redundant} ( not nonredundant) because say  can be dropped. 
Both  and  are equivalent to , and are clearly nonredundant. The latter is minimum, in fact optimum. Generally each minimum family is nonredundant. Less obvious, each optimum family is minimum as proven in Theorem 1. 


{\bf 2.2.2} From  and  ``somehow follows'' , but this notion needs to be formalized. We thus say that  {\it follows} from (or: is a {\it consequence} of) a family  of implications, and write , if  is equivalent to . The following fact is often useful:

(8) \quad  if and only if 

{\it Proof of (8)}. As to , by assumption the two closure operators  and  coincide. Thus in particular . As to , it suffices to show that  which clearly coincides with , is contained in  for . {\it Case 1:} . Then  by the very definition of the closure operator , {\it Case 2:} . Then by assumption , and so again . \quad 

In Expansion 3 we introduce among other things a ``syntactic'' notion  of {\it derivability} and show that  is equivalent to .


{\bf 2.2.3}  Conversely, let  us {\it start out} with any closure operator . Then a family  of implications is called an {\it implicational base} or simply {\it base} of  if  for all . Each closure operator  {\it has} an implicational base, in fact  does the job\footnote{This is slightly less trivial than it first appears.  Clearly , but why not ?}. Unfortunately,  is too large to be useful. How to find smaller ones is the theme of Section 3.  





{\bf 2.2.4} Putting  in (8) we see that  is a consequence of . 
Thus for any closure operator  the implication  is a consequence of any  that happens to be an implicational base of . But implications  often carry a natural meaning ``on their own'', such as  in 2.1.2.


{\bf 2.2.5} Streamlining the proof of [KN, Theorem 20] here comes an example of a visually appealing closure operator , all of whose optimum bases can be determined ``ad hoc'', i.e. without the theory to be developed in Section 3.2. Namely,  arises from an affine point configuration  by setting  where  is the ordinary (infinite) convex hull of . For instance, if  is as in Figure 2, then .


\begin{center}
\includegraphics[scale=0.8]{JoyOfImplicFig2}
\end{center}


From the deliberations below (which generalize to point sets in  without  points in a hyperplane) it will readily follow that  has exactly  optimum bases. 
Let  be the set of all 3-element subsets  with .
Let  be any base of  and let  be arbitrary. From , and the fact that all proper subsets of  are closed, follows that  must contain an implication with premise . Now consider a set  of implications  where  scans  and where  is arbitrary. Obviously,  for all . If we can show that  is a base at all, then it must be optimum by the above. By way of contradication assume that  is no base, and fix a set  with  for which  is minimal. From  follows\footnote{This follows from the well-known fact that convex hulls like  can be obtained by repeatedly taking closures of -element sets.} that  for at least one , and thus . Consider the unique triangulation of  into triangles  all of whose (3-element) vertex sets  contain . Then , and so . Furthermore from  follows , and so


which contradicts .  The mentioned number  arises as  in view of the fact that exactly four  have  (namely , and exactly two  have   (namely ). Here we e.g. wrote 124 instead of . This kind of shorthand will be used frequently.




\section{The finer theory of implications}

In 3.1 we couple to each closure operator  some quasiclosure operator   which will be crucial in the sequel. In [W3] it is shown that certain minimization results independently obtained by Guigues-Duquenne [GD] and Maier [M] are equivalent. By now the formalisation of Guigues-Duquenne has prevailed (mainly due to the beneficial use of closure operators), and also is adopted in Section 3.2. Section 3.3 introduces the canonical direct implication base. Section 3.4 finally introduces pure Horn functions, and 3.5 addresses the acyclic case. 
It seems that the link between implications and the meet-irreducibles of the induced closure system (Section 3.6) must be credited to Mannila and R\"{a}ih\"{a} [MR1]. As indicated in the introduction, in 3.6 we also shed some light on {\it why} it is important to go from  to  and vice versa.


\subsection{Quasiclosed and pseudoclosed sets}

Given any closure operator  and  we put 

(9) \quad .

Because  is finite the chain  will stabilize at some set . It is clear that  is  a closure operator and that  for all . We call  the {\it -quasiclosure}, or simply {\it quasiclosure} operator when  is clear from the context.


\includegraphics[scale=0.6]{JoyOfImplicFig3}


As an example, consider the  grid  in Figure 3 and the closure system  of all contiguous rectangles  (thus  and  are {\it intervals}). Let  be the coupled closure operator. For  (matching the three gray squares on the left in Figure 3) all singleton subsets are closed, and for the 2-element subsets we have

Hence . If  is any set with  then necessarily  (why?), whence . Hence . Jointly with 

follows that . Finally  because e.g. .
We call\footnote{Unfortunately no standard terminology exists. It holds that  iff  {\it directly determines}  (modulo some ``cover of functional dependencies'') in the sense of [M, Def.5.9]. Do not confuse this notion of ``direct'' with the one in Section 3.3.} a subset {\it properly quasiclosed} if we like to emphasize that it is quasiclosed but {\it not} closed.
For instance the set  in Fig.3 is properly quasiclosed. 


\includegraphics[scale=0.5]{JoyOfImplicFig4}


{\bf 3.1.1} As another example take  and let  be the closure system of Figure 4(a) with associated closure operator . For our  at hand the properly quasiclosed generating sets for each closed set are these:



Let  be a closure system. As opposed to (2) one can show that

(10) \quad  is a closure system \  is quasiclosed

See Figure 4(b) where  was added to . One checks that indeed  for all .





\subsection{The canonical Guigues-Duquenne base}


For closure operators  we define the equivalence relation  by

(11) \quad .



For any implicational base  of  and for any  let  be the set of those implications   in  for which . It holds that

(12) \quad  \quad for all \quad ,

where  is the -quasiclosure operator. Being a key ingredient for establishing Theorem 1 below let us repeat and slightly amend the proof of (12) given in [W5, Lemma 4]. For starters we replace   by the equivalent family   of  implications which has each  from  replaced by the {\it full} implication . Because  equals  it suffices to prove that

 \quad  for all .

The inclusion  being obvious  it suffices to show that  implies . Since   and  this further reduces to show that  implies that . But this holds since by construction all implications from  are of type , and thus cannot be used in the generating process of . This proves  and hence (12).


A properly quasiclosed set  is {\it pseudoclosed}\footnote{From an algorithmic point of view this equivalent defintion is more appropriate:  is pseudoclosed iff  and  for all pseudoclosed sets  strictly contained in . Another name for pseudoclosed is {\it critical} (not to be confused with ``critical'' in 4.1.5).} if it is minimal among the properly quasiclosed sets in its -class. (In the set listing of 3.1.1 these are the boldface sets.) Consider now the family of implications

(13) \quad  is pseudoclosed,

where  stands for Guigues-Duquenne. Clearly  for all , and so  will be an implicational base of  if we can show that  for all . Indeed, by (12) applying the implications from  blows up  to . If  then by definition there is a pseudoclosed set  with . Applying the implication  to  shows that .

This establishes part (a) of Theorem 1 below. For the remainder see [W3, Thm.5] which draws on [GD] and again uses (12). Two more concepts are in order.
One calls  {\it essential} if  contains a properly quasiclosed generating set. Thus the essential sets coincide with the closures of the pseudoclosed sets.
The {\it core}  [D] of a closure operator  is

(14) \quad .

\begin{tabular}{|l|} \hline \\
{\bf Theorem 1:} Let  be a closure operator.\\
\\
(a) The family of implications  is an implicational base of .\\
\\
(b) If  is any implicational base then . More specifically, for each pseudoclosed\\
\hspace*{.5cm}  there is some  with  and .\\
\\
(c) If  is a nonredundant implicational base then  equals .\\
\\
(d) If  is a nonredundant implicational base which moreover consists of {\it full} implications\\
\hspace*{.5cm}  then  is minimum.\\
\\
(e) If  is optimum then  is minimum. Furthermore for each of the implications \\
\hspace*{.5cm} defined in (b) the cardinality of  is uniquely determined by  as\\
\hspace*{.5cm} \\ \\ \hline 
\end{tabular}

Because of (b) the Guigues-Duquenne base is often called {\it canonical}\footnote{Some authors as [GW] speak of the {\it stem base} but for us ``stem'' has another meaning (see 3.3).}. 
Those families  of implications that are of type  for some closure operator  were {\it inherently} characterized by Caspard [C]. The whole of Theorem 1 can be raised to the level of semilattice congruencies\footnote{For a glimpse on semilattice congruences in another but related context see 4.2.1.} [D2] but this further abstraction hasn't flourished yet. For practical purposes any minimum base  is as good as .  For instance, a trivial way to shorten  to  is to replace each  in  by . The extra benefit of  is its beauty on a theoretical level as testified by Theorem 1. 



{\bf 3.2.1} To illustrate Theorem 1 we consider  where  is the closure system from 3.1.1. Hence the canonical base of  is


.

It happens that all premises (apart from  which has  and ) contain {\it unique} minimal generating sets of the conclusions, and so by Theorem 1(e) each optimum base of  must be of type

.

It turns out that e.g.



is optimum. To prove it one must (a) show that  is a base at all, and (b) show that the sum  of the sizes of the conclusions is minimum. We omit the argument. See also Problem 4 in Expansion 15.


{\bf 3.2.2} In this section and (only here)  denotes the strong component of , i.e. not . As a less random application of Theorem 1 consider the case where  admits a base  of {\it singleton} premise implications\footnote{We disallow  as premise in order to avoid distracting trivial cases. Further we point to 4.1.2 for the connection to lattice distributivity.}.  Such a situation can be captured by a directed graph. For instance

(15) \quad 

matches the arcs in the directed graph  in Figure 5(a). What, then, do  and the optimal bases  look like? Being singletons, and because of , all premises of implications in  are pseudoclosed (note  is closed), and so Theorem 1(b) implies that these are {\it all} pseudoclosed sets of . From this and Figure 5(a) it follows that



\begin{center}
\includegraphics[scale=0.6]{JoyOfImplicFig5}
\end{center}


The strong components of  are ,  and the resulting factor poset  is depicted in Figure 5(b). We claim that the optimal bases  look like this: The elements in each strong component  are set up, in arbitrary circle formation such as  for . (For  the circle formation reduces to a point.) Furthermore, for any non-minimal  choose any minimal transversal  of the lower covers of  in  and distribute  to the circle formation of  in arbitrary fashion. Thus  admits  and . Choosing  one can e.g. pad up  to  or alternatively  to . Choosing  one can e.g. pad up  to . The latter choice yields an optimum base

To prove the claim, first note that families of type  obviously {\it are} implicational bases. We next show that {\it each} family  equivalent to  in (15) must contain implications that link [2] to both lower covers [6] and [7]. Indeed, suppose each  in  with  has . Then we get the contradiction that  is -closed but not -closed. From this it readily follows that the bases of type  have minimum size . This kind of argument carries over to the optimization of all families  with merely singleton premises.

Calculating  depends in which way  is given. The two most prominent cases are  and .
The first is hard (3.6.3), the second easy (Expansion 11).

\subsection{The canonical direct implicational base}

An implicational base  of  is {\it direct} if  for all  (see (6)).
Analogous to Theorem 1 each closure operator again admits a {\it canonical} direct implicational base . In order to state this in Theorem 2 we need a few definitions.
Let  with . Following [KN] we call  a {\it stem for} , and  a {\it root for} , if  is minimal with the property that . (Other names have been used by other authors.) Further  is a {\it stem} if it is a stem for some , and  is a {\it root} if it is a root for some . If  is a stem, we put

(16) \quad roots,

For instance, if  then roots. Dually, if  is root, we put

(17) \quad stems.

Note that  is {\it not} a root iff  is closed. Vice versa, a subset  does {\it not} contain a stem iff all subsets of  (including  itself) are closed. Such sets  are called\footnote{An equivalent definition occurs in 3.3.1. Note that in [W3] the meaning of ``free''  is ``independent''.} {\it free}.

\begin{tabular}{|l|} \hline \\
{\bf Theorem 2:} Let  be a closure operator. Then\\
\\
\hspace*{3cm} \\
\\
is a direct implicational base of  of minimum cardinality.\\ \\ \hline \end{tabular}


{\it Proof.} Let . We first show that . We may assume that  and pick any . Obviously there is  with . From  it follows that . Thus  is a direct implicational base of .

To show that   for any direct base  of  we fix any stem  (say with root ).  It suffices to show that at least one implication in  has the premise . Consider the -closure

Suppose we had  for all premises  occuring in . Then each  contained in  is a {\it proper} subset of , and so the minimality of  forces , whence , whence . The contradiction  shows that at least one  equals .  \quad 

We stress that ``minimum'' in Theorem 2 concerns only the {\it directness} of ; as will be seen, small subsets of  can remain (non-direct but otherwise appealing) bases of .  The base , has been rediscovered in various guises by various authors; see [BM] for a survey. We may add that in the context of FCA and the terminology of ``proper premises''  seemingly was first introduced in [DHO]. In the relational database world  is called a ``canonical cover'' [M, 5.4] and (according to D. Maier) first appeared in Paredens [P].
We shall relate  to prime implicates of pure Horn functions in 3.4, and to  in 3.6, and we consider {\it ordered} direct bases in 4.3. Other aspects related to  are discussed in Expansions 5 and 6. Furthermore, the following concept will be more closely investigated in the framework of 4.1.5. We define it here because it is of wider interest. Namely, a stem  is {\it closure-minimal} with respect to its root  if  is a minimal member of .


{\bf 3.3.1} If  is a closure operator then  is called {\it independent} if   for all . A closed independent set is {\it free}. Further, a minimal generating set  of  is a {\it minimal key for} , or simply a {\it minimal key} (if  is irrelevant). Recall that a {\it set ideal} is a set system  such that  and  jointly imply . The maximal members of  are its {\it facets}. The following facts  are easy to prove:
\begin{enumerate}
	\item [(a)] A subset is independent iff it is a minimal key.
	\item[(b)] The family Indep of all independent (e.g. free) sets is a set ideal.
	\item[(c)] Each stem is independent  but not conversely.
\end{enumerate}
Since each  contains at least one minimal key for , it follows that .
 Instead of ``minimal key'' other names such as ``minimal generator'' are often used, and ``minimal key'' sometimes means ``minimal key of ''. Generating all minimal keys has many applications and many algorithms have been proposed for the task. See [PKID1, Section 5.1.1] for a survey focusing on FCA applications.

{\bf 3.3.2} Let us indicate an apparently new method to get all minimal keys; details will appear elsewhere. The facets  of Indep can be calculated with the {\it Dualize and Advance} algorithm (google that). It is then clear that the minimal keys of any closed set  are {\it among} the (often few) maximal members of . For special types of closure operators more can be said (see 4.1.4 and 4.1.5).



\subsection{Pure Horn functions, prime implicates, and various concepts of minimization}

We recall some facts about Boolean functions with which we assume a basic familiarity; e.g. consult [CH] as reference. Having dealt with the consensus method and prime implicates on a general level in 3.4.1, we zoom in to pure Horn functions in 3.4.2 and link them to implications. (Impure Horn functions appear in 4.5.) In 3.4.3 we show that the canonical direct base  in effect is the same as the set of all prime implicates. Subsection 3.4.4 is devoted to various ways of measuring the ``size'' of an implicational base, respectively pure Horn function.

{\bf 3.4.1} Recall that a function  is called a {\it Boolean function}. A {\it bitstring}  is called a {\it model} of  if . We write Mod for the set of all models of . For instance,  is a {\it negative} (or {\it antimonotone}) Boolean function if  implies .  
Thus, if we identify  with the powerset  as we henceforth silently do, then Mod is a set ideal in  iff  is a negative Boolean function. Using {\it Boolean variables}  one can represent each Boolean function  (in many ways) by a {\it Boolean formula} . We then say that  {\it induces} . A {\it literal} is either a Boolean variable or its negation; thus  and  are literals. A {\it clause} is a disjunction of literals, such as . A {\it conjunctive normal form} (CNF) is a conjunction of clauses.  The CNF is {\it irredundant} if dropping any clause changes the represented Boolean function. Let  be a Boolean function and let  be a clause. Then  is an {\it implicate} of  if every model of  is a model of . We emphasize that ``implicate'' should not be confused with ``implication'' , but there are connections as we shall see. One calls  a {\it prime implicate} if dropping any literal from  results in a clause which is no longer an implicate of . In Expansion 7 we show how {\it all} prime implicates of  can be generated from an arbitrary CNF of . A {\it prime} CNF is a CNF all of whose clauses are prime implicates.



{\bf 3.4.2} A Boolean function  is a {\it pure Horn function} if Mod is a closure system\footnote{Some authors, e.g. [CH, chapter 6], use a different but dual definition, i.e. that  must be a closure system. Each theorem in one framework immediately translates to the dual one. Do not confuse this kind of duality with the kind of duality in [CH, 6.8].}. 
The induced closure operator  we shall denote by . Conversely, each closure operator  induces the pure Horn function  defined by . Similar to 2.1.1 one has  and . As mentioned in 3.4.1 many distinct formulas  induce any given\footnote{For instance, using concatenation instead of , {\it one} formula  for the Horn function  induced by the closure system in Figure 4(a) is  .} pure Horn function . 
As is common, we shall focus on the most ``handy'' kind of formula , for which the letter  will be reserved. 

In order to define  we first define a {\it pure} (or {\it definite}) {\it Horn clause} as a clause with exactly one positive literal.
 Thus  is a pure Horn clause . Accordingly consider the implication . One checks that the Boolean function induced by formula  is a Horn function  (for any fixed ). In fact . However, this doesn't extrapolate to the implication  which doesn't match ! Rather   is equivalent to  and whence\footnote{This is a good place to address a source of confusion. The formula  {\it also} is the conjunction of two pure Horn clauses; it matches the implication . The formula  is a {\it tautology} which matches the implication . But  matches {\it no implication}. Rather it amounts to the {\it impure} Horn clause , the topic of Section 4.5.} matches the conjunction  of {\it two} pure Horn clauses. Generally, a {\it pure Horn CNF}  is defined as a conjunction of  pure Horn clauses. Thus  matches a family  of unit implications. In particular, this shows that the Boolean function  induced by  really {\it is} a pure Horn function:  equals , which we know to be  closure system (2.2). Conversely, starting with any family  of implications, the {\it unit expansion}  is obtained by replacing each  by the unit implications . By definition  is the pure Horn CNF whose clauses match the members of . Notice that special features of  need not be mirrored in , and vice versa for  and .
 For instance, if  is optimum then the pure Horn clauses in  need not  be prime. See also 3.4.4.1. 
 

{\bf 3.4.3} It is evident from the definitions of stem, root and prime implicate, and from Theorem 2, that each implication in  yields a prime implicate of the pure Horn function  determined by . Do we get {\it all} prime implicates (Horn or not) of  in this way? Yes. The traditional proof is e.g. in [CH, p.271], and a fresh one goes like this.
Suppose   had a prime implicate  which is not a Horn clause, say without loss of generality  is . Then both  and  are no implicates of . Hence there are  such that  but , and such that  but . Thus  but both . Hence  is a model of  but not of , contradicting the assumption that  is an implicate of . \quad 

Thus the members of  are in bijection with the prime implicates of . Any (usually non-direct) base of implications  will henceforth be called a {\it base of prime implicates}. In other words, bases of prime implicates match prime pure Horn CNF's.


{\bf 3.4.4} We now drop pure Horn functions until 3.4.4.1. Apart from  and  introduced in 2.2 there are other ways to measure families of implications. If say

(18) \qquad 

then  and . Further the {\it left hand size} is defined as the sum of the cardinalities of the premises, thus . Similarly the {\it right hand size} is . 
What are the relations between ``usual'' optimality ( as defined in 2.2) and the new kinds of optimality lhs-op and rhs-op? Suppose first  is simultaneously lhs-op and rhs-op. If  is any other base of  then

and so  is optimal. This was observed in [AN1] and likely elsewhere before. Conversely, it follows at once from Theorem 1(e) that op  lhs-op. In [ADS] it is shown (see Figure 6) that also op  rhs-op.  For instance, it is impossible that a closure operator has two optimum bases with implications   and 
 respectively. To summarize:

(19) \qquad op \  \ lhs-op and rhs-op

A slightly less natural parameter is (carhs). According to [ADS] these implications (and their consequences, but no others) take place:

\begin{center}
\includegraphics{JoyOfImplicFig6}
\end{center}

{\bf 3.4.4.1} Let us stick with the measures above and re-enter pure Horn functions to the picture. For starters, when  in (18) is translated in a pure Horn CNF we get

 \qquad 

Notice that  and  is the number of clauses of . Generally, for a fixed pure Horn function  put

Thus  is the minimum number\footnote{Many other acronyms for this measure are dispersed throughout the literature. For instance, [CH, p.297] uses  for . On the side of uniformity, our notation  above matches the one in [CH, p.297].}
of pure Horn clauses needed to represent . Rephrasing the [ADS] result above (which is reproven in [AN1, Thm.10]) one can say: If  is any optimum base of  then  has rhs many clauses. The ``inverse'' operation of unit expansion is {\it aggregation}. Thus if  then . 

If similarly to  we define

then  is not so succinctly expressed in terms of Horn clauses (but is e.g. useful in 4.5.2). Similarly the likewise defined parameters  and  are clumsier than their counterparts  and . Apart from   , the most natural measure for pure Horn functions is the minimum number  of literals appearing in any pure Horn CNF representation of . One calls  the {\it number of literals} measure. Clearly . For instance, if  from  induces , then . Similarly  in view of (18).
Both rhs-optimization and -optimization are NP-hard, and even {\it approximation} remains hard [BG].




\subsection{Acyclic closure operators and generalizations}

To any family  of implications on a set  we can associate its {\it implication-graph}\footnote{The terminology is from [BCKK], while  itself was independently introduced in [W3, p.137] and [HK, p.755].} . It has vertex set  and arcs  whenever there is an implication  in  with  and . What happens when  merely has singleton-premise implications was dealt with in 3.2.2. Another natural question is: If  is acyclic, i.e. has no directed cycles, what does this entail for the closure operator  ?
The first problem is that for equivalent families  and  it may occur that  is acyclic but  isn't. For instance, in the example from [HK, p.755] one checks that  and  are equivalent. While  is acyclic,  is not because it has the cycle . Observe that  is no prime implicate because
it follows from . 

Indeed, the problem evaporates if one restricts attention to the prime implicates. More precisely, call\footnote{In [HK] the authors talk about the acyclicity of pure Horn formulas (or functions). Recall from  3.4.2 the equivalence between closure operators and pure Horn functions.} a closure operator  {\it acyclic} if there is a base  of  which has an acyclic implication-graph . As shown in [HK, Cor.V.3] a  closure operator  is acyclic iff  is acyclic for each base  of prime implicates. Hence (consensus method, Expansion 7) for an arbitrary family  of implications it can be checked in quadratic time whether  is an acyclic closure operator.

{\bf 3.5.1} Let  be any poset and let  be a closure operator with  and such that for all  and  it follows that . Put another way,  is always a {\it subset} of the order ideal  generated by . Following\footnote{This terminology is more telling than ``-geometry'' used in [W3].} [SW] we call such an operator of {\it poset type}.

\begin{tabular}{|l|} \hline \\
{\bf Theorem 3}: A closure operator  is acyclic if and only if it is of poset type.\\ \\ \hline \end{tabular}

{\it Proof.} We shall trim the argument of [W3, Cor.15]. So let  be acyclic and let  be any base of  for which  is acyclic. On  we define a transitive binary relation  by setting  iff there is a directed path from  to  in . By the acyclicity of  this yields a poset . Consider  and  such that . Then  because  is a base of . To fix ideas suppose  where  is as defined in (6), and say that  because  and . Further let  in view of  and . Then  and all of them are  because  has directed paths  and  and . Hence . Thus  is of poset type.

Conversely let  be of poset type with underlying poset . Let  be a base of  whose unit expansion yields a {\it prime} Horn CNF. It suffices to show that  is acyclic. Suppose to the contrary  contains a directed cycle, say . By definition of  there is  with  and , and so . By assumption  where . If we had  then  would be an implicate of , which cannot be since  is a prime implicate. It follows that , whence . By the same token one argues that , and eventually , which is the desired contradiction. 
\hfill
\hfill 

According to [HK, p.756] each acyclic closure operator  admits a unique nonredundant base  of prime implicates. Consequently (why?)  is rhs-optimal and -optimal. Starting out with any family  of unit implications for which  is acylic (and whence  is acyclic), it is easy to calculate .
To fix ideas, one checks that

has  acyclic. Any  in  which is not a prime implicate, can only fail to be one because some  satisfies , and so  is redundant. Here only  isn't a prime implicate (take ). But also prime implicates in  may be redundant. In our case  is a consequence of  and . One checks that  consists of prime implicates and is nonredundant. Hence it must be . Obviously  is not minimum among {\it all} bases of  since  and  can be aggregated to .

{\bf 3.5.2} As to generalizations, two variables  and  of a Boolean formula  are {\it logically equivalent} if they have the same truth value in every model of (the function induced by) . This amounts to say that both  and  are (prime) implicates of . A closure operator  is {\it quasi-acyclic} if there is a base  of prime implicates such that all elements within a strong component of  are logically equivalent. Each acyclic closure operator is quasi-acylic because all components of  are singletons. Also the kind of closure operators  considered in 3.2.2 are evidently quasi-acyclic.


A closure operator  is {\it component-wise quadratic}  if there is a base  of prime implicates such that  has the following property. For each prime implicate  of  and each strong component  of  it follows from  that . Thus for each component  of  the ``traces'' of the prime implicates on  are ``quadratic'' in the sense of having cardinality . Here comes the argument of why quasi-acyclic entails . Suppose  is a prime implicate of  such that  and . Take . Because  is an implicate of  by quasi-acyclicity, we must have  (which implies ).  In a tour de force it is shown in [BCKK] that for each  closure operator an rhs-optimum base (i.e. minimizing the number of clauses) can be calculated in polynomial time; many auxiliary graphs beyond  appear in [BCKK]. The quasi-acyclic case had been dealt with in [HK]. Another way to generalize ``acylcic'' is to forbid so-called -cycles, see Expansion 18.




\subsection{Implications and meet-irreducibles}

First some prerequisites about hypergraphs.
A {\it hypergraph} is an ordered pair  consisting of a {\it vertex set}  and  a set of {\it hyperedges} . The hypergraph is {\it simple} if  for all distinct . (An ordinary simple graph is the special case where  for all ). A {\it transversal} of  is a set  such that  for all . We write  for the set of all transversals. Furthermore, the {\it transversal hypergraph}  consists of all {\it minimal} members of  . It is easy to see that . Arguably the single most important fact about general simple hypergraphs is [S, p.1377] that equality takes place:

(20) \quad 

The {\it transversal hypergraph problem} (or {\it hypergraph dualization}), i.e. the problem of calculating  from  has many applications and has been investigated thoroughly. See [EMG] for a survey and [MU] for a cutting edge implementation of hypergraph dualization.

Let  be a closure system and let  be its set of meet-irreducibles (see 2.1). 
Clearly the set max of all maximal members of  is a subset of . Adopting matroid terminology (4.1.4) we refer to the members of  as {\it hyperplanes}. More generally, for any  let 
\begin{center}
 be the set of all  that are maximal with the property that . 
\end{center}
If  (which we assume to avoid trivial cases) then  for all . In fact each  is meet-irreducible.
Conversely, every  belongs to some . (See Expansion 12.)  
Therefore:

(21) \quad .

It is convenient that the sets  can be retrieved from any generating set  of , i.e. not the whole of  is required:

(22) \quad .  


The proof is given in Expansion 10. 
The smaller , the faster we can calculate the simple hypergraphs

(23) \quad .


\begin{center}
\includegraphics{JoyOfImplicFig7}
\end{center}


The next result is crucial for traveling the right hand side of the triangle in Figure 7.

\begin{tabular}{|l|} \hline \\
{\bf Theorem 4:} For any closure system  with   one has\\
\\
(a) \ stems\\
\\
(b) \ cmax\\
\\
\hline \end{tabular}

{\it Proof.} We draw on [MR2, Lemma 13.3 and Cor.13.1]. We first show that for any fixed  it holds for all  that:

(24) \quad .

{\it Proof of (24)}. Suppose  is such  that ; see (4). Thus from  and  follows . For each  (see (21)) we have , hence , hence , hence . Conversely, let  be such that . Then, because of , there is  with . We may assume that  is {\it maximal} within  with respect to . It then follows from (22) (put ) that . From  it follows that , and so .  This proves (24). 

Let  be fixed. Then the family of minimal 's satisfying  is stems. Likewise the family of minimal 's satisfying  is . By (24) these two set families coincide, which proves (a).
As to (b), it follows from (a) and (20) that
. \hfill 


As was independently done in [BDVG], let us discuss the six directions in the triangle of Figure 7. Notice that matters don't change much if instead of  we substitute any ``small'' (informal notion) generating set  of  in Figure 7, and instead of  we sometimes consider any ``small'' (w.r.t. ) base  of .
Both practical algorithms illustrated by examples, and theoretic complexity will be discussed. As to going from  to a minimum base , the most elegant and only slightly sub-optimal method is the one of Shock [Sh]; see Expansion 11. The way from  to  can be handled by the consensus method (Expansion 7); for another method see [RCEM].  
In Subsections 3.6.1 to 3.6.3 we outline how to travel the remaining four directions, with more details provided in Expansions. 

{\bf 3.6.1} Recall from Theorem 2 that knowing the canonical direct base  means knowing the members of . Likewise, by (21) and (23), knowing  amounts to knowing the set collections cmax. Therefore Theorem 4 says that getting  from  or vice versa is as difficult as calculating all minimal transversals of a hypergraph. To fix ideas let us carry out the way from  to  on a toy example. Suppose that  and  is such that

(25) \quad .

From (21) and (22) we get

(26) \quad  

\hspace*{2.2cm} .

The set union in (26) happens to be disjoint. Generally the union in (21) is disjoint iff  for all . Here  is the unique upper cover of  in . From say  we get , and by Theorem 4(a) we have stems which turns out to be . Dropping  yields stems. Likewise one calculates stems, stems, stems, stems stems. By definition of  in Theorem 2 we conclude that

(27) \quad .

Let us mention a natural enough alternative [W1, Algorithm 3] for . 
By processing the members of  one-by-one it updates a corresponding direct base. The worst case complexity being poor, average behaviour still awaits proper evaluation.

{\bf 3.6.2}  How to get  from an {\it arbitrary} (non-direct) implication base ?  One of the first methods was [MR2, Algorithm 13.2], which was improved in [W1, Sec.9]. 
In brief, in view of (21) both methods proceed as follows. For  let . Then  can be expressed in terms of the set families  and  where  ranges over . Another idea for  in [BMN] features an interesting fixed-parameter-tractability result.
Expansion 8 exhibits a fourth way.

{\bf 3.6.2.1} Unfortunately it is shown in [KKS] that  can be exponential with respect to , and vice versa. Furthermore, according to [K] both transitions  and  are at least as hard as the transversal hypergraph problem. What's more, whatever the complexity of these transitions, they are equivalent under polynomial reductions. 
Along the way a fifth algorithm [K, p.360-361] to get the {\it characteristic models} (i.e. ) from  is offered. (Some of these results extend to the {\it arbitrary} Horn functions in  4.5.)

{\bf Open Problem 1}: Compare on a common platform and in a careful manner akin to [KuO1], mentioned five methods (and possibly others) for calculating  from .


{\bf 3.6.2.2} What is the point of calculating  from ? This problem first arose in the vestige of finding an {\it Armstrong Relation} ( short example database) for a given set of functional dependencies. Albeit an Armstrong Relation is not quite the same as , the number of its records is , see [MR2, Thm.14.4]. Having  enables a ``model-based'' approach to reasoning. For instance, deciding whether  holds, reduces to check whether  entails  for all . This beats the test in (8) when . With the eye on using model-based reasoning in Knowledge Bases article [KR] extends (as good as possible) the concept of characteristic models from Horn functions to arbitrary Boolean functions. Observe that  also occurs in the context of Cayley multiplication tables (4.2.2). Furthermore, many combinatorial problems (e.g. calculating all minimal cutsets of a graph) amount to calculate the subset max from .


{\bf 3.6.3} How can one conversely get a small or minimum base  from  (or from another generating set )?
 This process is nowadays known as {\it Strong Association Rule Mining} (applications follow in 3.6.3.4). For succinctness, suppose we want . Unfortunately, as shown in [KuO2], not only can  be exponential in the input size , but also calculating the {\it number}  is \#-hard.  Despite the exponentiality of  one could imagine (in view of (36)) that  can at least be generated in output-polynomial time, given . As shown in [DS], this problem is at least as hard as generating all minimal transversals. Given , the pseudoclosed sets cannot be enumerated in lexicographic order [DS], or reverse lexicographic order [BK], with polynomial delay unless . Several related results are shown in [BK]. For instance, given  and , it is -complete to decide whether any minimum base  of  (see 2.1.1) contains an implication of type . (Conversely,  can also be ``large'' with respect to , see Expansion 4.)
 
{\bf 3.6.3.1}  A different approach to go from  to a small base  of   was hinted at in [W1, p.118] and developed in [RDB]. It essentially amounts to a detour  and then , but in a clever way that avoids to generate large chunks of . It is argued that even if the resulting base  is considerably larger than , this is more than offset by the short time to obtain . A similar approach is taken in [AN2], but instead of  the -basis of 4.3 (a subset of ) is targeted. Furthermore the likely superior [MU] subroutine for hypergraph dualization is used.

{\bf 3.6.3.2} In another vein, it was recently observed in [R] that for given  one can readily exhibit a set  of implications based on a superset  such that  satisfies . Here  is the {\it projection} of  upon . Furthermore,  and  has a mere  implications. What also is appealing: If  is given by -rows as in 4.4 then  is smoothly calculated by setting to  all components with indices from , and adapting the other components accordingly.

  
{\bf 3.6.3.3} A natural variation of the  theme is as follows. For any  call a family  of implications a {\it Horn approximation} of  if . The intersection of all these  is the smallest closure system  that contains .  Given  and any  there is by [KKS, Thm.15] a randomized polynomial algorithm that calculates a family  of implications which is a Horn approximation of  with probability  and moreover satisfies .
 
{\bf 3.6.3.4} It should be emphasized that current efforts in data mining do however concern ``approximations'' that involve parameters different from  and  above. These approximations are called {\it association rules} and they involve a support-parameter  and a confidence-parameter  taking values in the interval . The association rule  has {\it confidence}  if in 57\% of all situations  one has . Our ordinary implications  coincide with the {\it strong} association rules, i.e. having . Even ordinary implications like butter, breadmilk in 1.1.2 can have a small {\it support} like . Namely, when merely 15\% of all transactions  actually feature {\it both} butter and bread, whereas in the other 85\% the implication ``trivially'' holds. See [B] for an introduction to Association Rule Mining that focuses on the underlying mathematics. See also [PKID2, Section 5.1].




\section{Selected topics}

See the introduction (1.2) for a listing of the five selected topics. More detailed outlooks will be provided at the beginning of each Subsection 4.1 to 4.5.

\subsection{Optimum implicational bases for specific closure operators and lattices}

We first show (4.1.1) that {\it each} lattice  is isomorphic to a closure system  on the set  of its join-irreducibles. It thus makes sense to speak of implicational bases of lattices, and we shall investigate special classes of lattices in this regard. Actually, for some lattices  it is more natural to start out with a suitable closure operator  and turn to  later. For us these 's are distributive (4.1.2), geometric (4.1.4) and meet-distributive (4.1.5) lattices respectively.

 
{\bf 4.1.1} We use a basic familiarity with posets, semilattices and lattices, see e.g. [G]. We denote by  the largest element of a join semilattice, and by  the smallest element of a meet semilattice. Recall that a lattice is a poset  which is both a join and meet semilattice with respect to the ordering . In this case some relevant interplay between the sets  and  of join respectively meet-irreducibles occurs (see Expansion 12).

Each closure system  yields an example of a meet semilattice: The meet of  (i.e. the largest common lower bound) obviously is . The smallest element is , and  has a largest element  as well. Whenever a meet semilattice  happens to have  then it automatically becomes a lattice. The most important instance of this phenomenon concerns closure systems:

(28) \quad Each closure system  is a lattice  with meets and joins given by\\
\hspace*{1cm}  and .


\begin{center}
\includegraphics[scale=0.4]{JoyOfImplicFig8}
\end{center}


Let us show that conversely {\it every} lattice  arises in this way. What's more, the set  can often be chosen much smaller than . Thus for a lattice  and any  we put

We claim that . As to , from  follows . Similarly , and so . As to , take . Then  and  which (by the very definition of ) implies that , and so . If  then . If  then each  minimal with the property that  is easily seen to be join irreducible. Hence  implies . Summarizing we see\footnote{Switching from  to the dually defined  (see 4.1.6) is sometimes more beneficial.} that:

(29) \quad For each lattice  the set system  is a closure system and \\
\hspace*{1cm} is a lattice isomorphism from  onto .

Following [AN1] we call  the {\it standard} closure system coupled to the lattice  (recall ). The standard closure system  of  in Fig.8(a) is shown in Fig.8(b). Now let  be the {\it standard} closure operator coupled to . Explicitely

(30) \quad 

for all subsets .  For instance  in Fig.8(a). 
We emphasize that not every closure operator  is ``isomorphic'' to one of type , see Expansion 14. Each -quasiclosed subset of  clearly is an order ideal of .  This invites to replace each implication  in  by . Along these lines one can associate with each standard closure system  a (generally not unique) {\it -base}  which stays minimum but satisfies . See [AN1, Sec.5]. By definition the {\it binary part} of a family  of implications is . As shown in [AN1, Sec.4], for standard closure spaces the binary parts of implication bases can be ``optimized independently'' to some extent. That relates to Open Problem 3 in Expansion 15.


We now discuss four types of lattices or closure operators for which the structure of the optimum implicational bases is known. These are in turn 
all distributive, all modular, some geometric, and some meet-distributive lattices.

{\bf 4.1.2} A closure operator  is {\it topological} if  for all . For instance, if  consists of {\it singleton-premise} implications as in 3.2.2 then  is easily seen to be topological. Conversely, if  is topological then by iteration , and so  is a
base for .
Furthermore, for  and  in  it follows from (28) that

By (28) always , and so  is a sublattice of the distributive lattice , which thus must be distributive itself. 
In Expansion 15 we show that conversely {\it every} distributive lattice  is isomorphic to a sublattice of , and we determine the unique optimum base  of .


{\bf 4.1.3} A lattice  is {\it modular} if it follows from  that . For instance the lattice of all submodules of an -module is modular. Furthermore, each distributive lattice is modular. 
The -element lattice consisting of  atoms and  will be denoted by .  It is modular but not distributive for . In fact every modular but nondistributive lattice has  as a sublattice. For any lattice  and any  we define  as the meet of all lower covers of .  We call  an {\it -element} if the interval  is isomorphic to  for some .  According to [W2] each optimum base  of a modular lattice is of type  where  is as in Expansion 15, and the implications constituting   are as follows. Coupled to each -element  choose  suitable implications of type . They are not uniquely determined by  but they all satisfy  among other restrictions. To fix ideas, the lattice  in Fig. 8(a) is modular and one possible optimum base is  where  contains the nine implications


It is convenient to think of the  join-irreducibles underlying the  implications coupled to a fixed -element as a {\it line} . These lines have properties akin to the lines occuring in projective geometry (see also 4.1.4). Modular lattices which are freely generated by a poset (in a sense akin to 4.2) are economically computed by combining Theorem 5 with the technique of 4.4. A preliminary version of this work in progress is in [arXiv: 1007.1643.v1].


{\bf 4.1.4} A closure operator  is a {\it matroid} (operator) if it satisfies this {\it exchange axiom} for all  and :

(31) \quad  and 


As a consequence each minimal generating set of  (or ) is maximal independent. Thus for matroids the word ``among'' in 3.3.2 can be replaced by ``exactly''. The edge set  of any graph yields a ``graphic'' matroid  whose circuits in the sense of Expansion 5 coincide with the circuits in the usual graph theoretic sense. As another example, let  be any field and let  be any (finite) subset which need not be a subspace. If for  we define , then the restriction  is an {\it -linear} matroid.  The particular features of   depend on the kind of subset  chosen. For instance, if  is a linearly independent set then  for all . Another extreme case is . Then

is a base of  and  is the {\it complemented} modular lattice\footnote{In fact, for {\it any} matroid  the coupled lattice is complemented but usually only {\it semi}-modular. Such lattices are also called {\it geometric}.} of all subspaces of , thus a special case of 4.1.3. In fact, the -elements of  are the rank two subspaces ( projective lines). The features of  a -linear matroid also depend on the field of scalars . For  one speaks of {\it binary matroids}, in which case the family  of implications , where  ranges over all {\it closed} circuits  and  ranges over , is the unique optimum implication base of , see [W3]. It is well known that each graphic matroid is binary, but not conversely. For the many facets of matroids see [S, Part IV]. We mention in passing that [S] arguably is the most comprehensive, and likely the most readable book on combinatorial optimization around.

{\bf 4.1.5} A closure operator  is a {\it convex geometry} (operator) if it satisfies this {\it anti-exchange axiom}:

(32) \quad If  \ and \  \ and \ .

The kind of operator  in 2.2.5 is the name-giving example of a convex geometry. 
As to another example, it was observed by Bernhard Ganter (around 1990, unpublished) and also follows from [SW, Lemma 7.7] that each closure operator  of poset type (see 3.5.1) is a convex geometry. 

One deduces from (32) that each  contains the {\it unique} minimal generating set  of . In particular  in 3.3.1.  The elements of  are the {\it extreme} points of . If  is closed then so is  for all .
Each circuit  of  (Expansion 5) has a {\it unique} root . If one needs to emphasize , one speaks of the {\it rooted circuit} . Other than for arbitrary closure operators, if  is a stem of  in a convex geometry then  is a rooted circuit. It follows [W3, Cor.13] that the family of all rooted circuits matches the family  of all prime implicates. A rooted circuit  is {\it critical} if  is closed for all . Recall the definition of closure-minimal in 3.3. As we show in Expansion 16, for each rooted circuit  it holds that:

(33) \quad  is critical  is quasiclosed  the stem  of  is closure-minimal

As opposed to the antimatroid side of the coin (Expansion 16), note that the subfamily

of  usually is {\it no} implicational base of . For instance, the set  of prime implicates of the convex geometry  in 2.2.5 is the union of all sets  where  ranges over . If such a rooted circuit  has  then  is quasiclosed. Conversely, assume  contains a point . By considering the triangulation of  induced by  (as in 2.2.5) one sees that , and so  is {\it not} quasiclosed. It follows from (33) that . Hence  is contained in every base of prime implicates but is not itself a base (unless the point configuration in  is rather trivial). 
We mention that closure-minimality of (order-minimal) stems also features in the so-called -basis of [A] and [AN1]. The convex geometries of type 2.2.5 and 3.5.1 can be generalized (Expansion 18) but the results and proofs become quite technical. This is one reason for dualizing (29) in 4.1.6. 



{\bf 4.1.6} For any lattice  and  put . Dually to (29),  is a closure system which is bijective to  under the map . In particular, if  is meet-distributive (thus  ``is'' a convex geometry according to Expansion 16) then a crisp implication base  of  is obtained as follows\footnote{Mutatis mutandis, this is Theorem 2 in [W4]. The acronym JNW means Janssen-Nourine-Wild.}. First  is the dual of  from Expansion 15. Second, each doubleton  which admits a (unique if existing)  with  and , induces two implications. One is , the other . Here  is as in Expansion 12, and say  is the set of upper covers of  in the poset . All these implications make up . In view of  the philosophy in 4.1.6 is similar to 3.6.3.2 which also trades a larger universe for a smaller implication base.





\subsection{Excursion to universal algebra: Finitely presented semilattices and subalgebra lattices}
 
First we show (4.2.1) that finding an implicational base for a lattice  in the sense of 4.1  means finding a presentation for , viewed as  -semilattice, in the sense of universal algebra. Afterwards we show (4.2.2) how subalgebra lattices and homomorphisms between algebras can be calculated by setting up appropriate implications. 

{\bf 4.2.1} For starters imagine a -semilattice that has a set  of (not necessarily distinct) generators  that satisfies this set  of (inequality) relations:

(34) \quad 


 An example of such a semilattice  (with say  replaced by ) is given in Figure 9 on the left. Notice that all relations hold; e.g.  holds because .
It isn't a priori clear whether there is a {\it largest} such semilattice, but universal algebra tells us it must exist.
 It is the so-called {\it relatively free} -semilattice  with set of generators  and subject to the relations in , shown on the right in Figure 9 (discard ). Every other -semilattice satisfying  must be an epimorphic image of ; in our case the definition of the epimorphism  is that  on the right maps to  on the left,  maps to , and so forth.
 
 Each (-semilattice) inequality, like , can be recast as an identity .  Conversely each identity can be replaced by two inequalities. If in turn inequalities  are viewed as implications  then we can state the following.
 
 
\begin{center}
\includegraphics[scale=0.6]{JoyOfImplicFig9}
\end{center}

\begin{tabular}{|l|} \hline \\
{\bf Theorem 5 :} The relatively free -semilattice  is isomorphic to the\\
-semilattice . Here the family  is obtained from  by replacing
each \\ 
inequality in  by the matching implication, and each identity in  by two\\
implications  and .\\
 \\ \\ \hline \end{tabular} 


The proof of Theorem 5 is given in [W5, Thm.5]. The closure system  can be calculated from  in compressed form as explained in 4.4. Specifically for the  matching the inequalities in (34), thus , one gets  as  for certain set systems  to  in Table 1 of 4.4.  We mention that  is also isomorphic to the semilattice  modulo a congruence relation . Here  and  is as in (11) where  is  with  from Theorem 5.
See also Expansion 17.



{\bf 4.2.2} As to subalgebra lattices, we only peak at semigroups but the ideas carry over to general algebraic structures (and what concerns homomorphisms, also to graphs). Suppose we know the multiplication table (Cayley table) of a semigroup  where . Obviously the subsets of  closed with respect to the  implications  are exactly the subsemigroups of . 
The algorithm from 4.4 can thus be invoked to give a compressed representation of all subsemigroups. 

In another vein, sticking again to semigroups  and  for simplicity, the same algorithm also achieves the enumeration of all homomorphisms . Namely, these 's are exactly the functions\footnote{More precisely, imposing these  implications yields the closure system  of all homomorphic {\it relations}  in output-polynomial time.  True, one needs to sieve the functions among them, but this is often feasible. As to the large cardinality  of our family  of implications, instead of calculating  as  one may directly target , see 3.6.2. All of this is work in progress.}   which are closed with respect to all  implications of type . How these ideas compete with other computational tools in algebra (e.g. consult the Magma Handbook) remains to be seen. They will fare the better the fewer structural properties of the algebras at hand can be exploited. Put another way, there are greener pastures for our approach than e.g. the beautiful theory of subgroup lattices of Abelian groups [Bu].  

 
\subsection{Ordered direct implicational bases}

We start by introducing order-minimal prime implicates, thus a third kind besides the closure-minimal ones in 3.3 and the strong ones in Expansion 6. To minimize technicalities we focus on the case of standard closure operators . Then the prime implicates of  are the nonredundant join covers in the lattice  that underlies . Specifically,  in Figure 10 (taken from [ANR]) is a {\it join cover} of 6 since . It is nonredundant since  and . (Generally,  {\it nonredundant} means that no proper subset is  a join cover.) Correspondingly  is a prime implicate of . However  is not {\it order-minimal} since  and still  is a prime implicate. The general definition of ``order minimal'' is the obvious one. The relevance this concept was first observed in [N, p.525]. Notice that  is not closure-minimal since  is a prime implicate with . Conversely a closure-minimal prime implicate need not be order-minimal.

We are now in a position to address the topic in the title.
Recall from 3.3 that the {\it direct} basis  of a closure operator  has the advantage that  as opposed to  (as to , see (6)). However the drawback of  is its usually large cardinality. As a kind of compromise we present {\it ordered direct} implicational bases . The key is a specific {\it ordering} in which the implications of  must be applied exactly once: For given  applying the first implication  of  to  yields . Applying  to  yields . And so forth until applying the last implication  to  yields  which is the correct closure of . Of course such a  is also an implication base in the ordinary sense.

Listing (in any order) all\footnote{In certain circumstances,  one or both ``all'' in this sentence can be weakened (by restricting ``any order'').} {\it binary} prime implicates  (thus ), and then listing (in any order) all order-minimal prime implicates, yields a particular ordered direct implicational base  which is called a {\it -basis}. The ``'' derives from the so-called -relation discussed in Expansion 18.



\begin{center}
\includegraphics[scale=0.6]{JoyOfImplicFig10}
\end{center}


 In our example one possibility is

(35) \quad .

Applying  in this order to say  yields

In contrast, ordinary forward chaining (2.2) needs three runs to find the closure:

Notice that the underlying unordered set of any -basis  coincides with  if  is an antichain: Then there is no binary part, and so each member of  is trivially order-minimal. There is actually no need to stick to bases of prime implicates. Given any basis  of  one can aim for an ordered direct base by suitably ordering , and perhaps repeat some implications. Unfortunately the canonical base  needs not be orderable in this sense [ANR, p.719].



\subsection{Generating  in compact form}

Calculating  amounts to generating the model set Mod of a pure Horn function  given in CNF (see 3.4). As glimpsed this has applications in Formal Concept Analysis, Learning Theory, and Universal Algebra. One could be tempted to calculate  from  with NextClosure (Expansion 4). But this yields the closed sets {\it one-by-one} which is infeasible when  is large.
 In 4.4.1 we thus outline an algorithm for {\it compactly} generating  from . In 4.4.2 we discuss how to get a compact representation of  not from , but from a generating set .

{\bf 4.4.1} A -{\it row} like  is a succinct representation for the interval , which thus has cardinality . Each ``2'' in  is used as a {\it don't care} symbol (other texts use ``'') which indicates that both 0 and 1 can be chosen. For instance, if the clause  (thus ) is viewed as a Boolean function of , then Mod clearly is the disjoint union of these four -rows:

\begin{tabular}{|c|c|c|c|c|c|}
1 & 2 & 3& 4 & 5 & 6\\ \hline \hline
0 & 2 & 2 & 2& 0 & 2 \\ \hline
0 & 2 & 2 & 2 & 1 & 2 \\ \hline
1 & 2 &  2& 2& 0 & 2\\ \hline
1 & 2 & 2 & 1 & 1& 2 \\ \hline \end{tabular}

If we let the  -{\it bubble}  mean ``{\it at least one 0 here}'' then the first three rows can be compressed to the {\it -row}  in Table 1. It thus follows that Mod is the disjoint union of  and  in Table 1. Consider the pure Horn function  given by

In order to calculate Mod we
 need to ``sieve'' from , and then from , those bitstrings which also satisfy . It is evident that this shrinks  to  and does nothing to .  In  the two -bubbles are independent of each other and distinguished by subscripts.

\begin{tabular}{c|c|c|c|c|c|c|}
& 1 & 2 & 3 & 4 & 5 & 6 \\ \hline
& & & & &     &\\ \hline
 &  & 2 & 2 & 2&  & 2\\ \hline
 & 1 & 2 & 2 & 1 & 1& 2\\ \hline
& & & & & & \\ \hline
 &  &  &  & 2 &  & 2\\ \hline
 & 1 &  &  & 2 & 0 & 2 \\ \hline
 & 1 & 2& 2 & 1 & 1& 2 \\ \hline
 & & & & & & \\ \hline
  &  & 2 &  & 2 &  & 2 \\ \hline
  & 0 & 0 &  & 2 & 1 & 2\\ \hline
  & 1 & 2& 2& 1 & 1& 2 \\ \hline
  & & & & & & \\ \hline
 &  & 2 & 0 & 2 &  & 0\\ \hline
 & 0 & 0 & 1 & 2& 1 & 2\\ \hline
 & 1& 2 & 2 & 1& 1&  \\ \hline
 & 1 & 2& 1 & 1 & 1&  \\ \hline   \end{tabular}

Table 1: Using -rows to compress a closure system

Note that forcing the first component of  to 1 in  (due to ) forces the second to 0. Imposing the constraint  (i.e. ) upon  replaces  by , deletes , and leaves  unscathed. Imposing the implication  upon  yields . We were lucky that  didn't clash with , otherwise things  would get uglier. Concerning the deletion of , with some precautions the deletion of rows can be avoided, which is the main reason making the implication -algorithm output-polynomial [W6]. The implication -algorithm easily extends to a {\it Horn -algorithm} which can handle impure Horn functions in the sense of 4.5.  Concerning a speed-up for singleton-premise implications, see Expansion 19. As to connections to  and CNF  DNF conversion, see Expansion 8 and 9 respectively.

{\bf 4.4.2} As to calculating  from a {\it generating} set , the first idea that springs to mind is to use NextClosure or some other algorithm discussed in [KuO1].  However, this as before yields the closed sets one-by-one which is infeasible when  is large. Alternatively, one may calculate a base  of  by either proceeding as in 3.6.3.1 or 3.6.3.2. Feeding  to the implication -algorithm yields a compact representation of . An analysis of the pro's and con's of these ways to enumerate  is  pending.



\subsection{General Horn functions}


We discuss negative functions in 4.5.1 and then use them to define general Horn functions in 4.5.2. Theorem 6 says, in essence, that good old implications suffice to economically capture any impure Horn function; only {\it one} additional impure Horn clause is necessary.

{\bf 4.5.1} For any nonempty  the set ideal {\it generated} by  is . By 3.4.1 a Boolean function  is negative if and only if Mod is a set ideal. Dually one defines {\it set filters}. Consider an arbitrary family  of sets  which we refer to as {\it complications}\footnote{This is handy ad hoc terminology which conveys a link to ``implications''.}.
Call  a {\it noncover} (of ) if it doesn't cover any complication, i.e.  for all . It is evident that the set  of all noncovers is a set ideal . 
Among all families  with  there is smallest one; it obviously is the family  of all minimal members of the set filter . In particular  is an antichain (no two distinct members of  are comparable). Conversely, {\it each} set ideal  admits a unique antichain  of complications  that yields . Put another way, each negative Boolean function  admits a {\it unique} irredundant CNF of {\it negative clauses}. For instance if  and by definition the model set of  is the set ideal, , then  has the unique irredundant CNF  . We see that the ``representation theory'' of negative Boolean functions  via complications ( negative clauses) is much simpler than the representation theory of pure Horn functions  via implications ( pure Horn clauses).

{\bf 4.5.2} This leads us to the definition of a {\it Horn function}  as one that can be represented as a conjunction  of a pure Horn function  with a negative function . One checks that pure Horn functions and negative functions are special cases of Horn functions. It is evident that Mod where  and  are such that  and .  
We call  a {\it base} of . Thus our previous bases  become the special case where . With Mod and Mod also Mod is a subsemilattice\footnote{The only difference between subsemilattices  and closure systems  is that subsemilattices need not contain . The usefulness of meet-irreducible sets, also in the impure case, remains.} of . But Mod can be empty, and so different from 3.4 a general Horn function  need not be satisfiable. The good news is, because  has a {\it smallest} member , it follows that Mod iff  contains some . Since  can be calculated from  as , satisfiability can be tested in linear time. (In plenty texts this simple state of affairs is veiled by clumsy notation.)


Observe that the above representation  is not unique since the subsemilattice  can be written as an intersection  of a closure system  with a set ideal  in many ways. The most obvious way is  where  is the closure system . (The notation  foreshadows the framework (39) in Expansion 20.) 
The parameters defined for pure Horn functions  in 3.4.4.1 carry over to general Horn functions . Here we are only interested in 

Note that  in [CH, p.297], i.e. the minimum number of ``source sides'' possible.


\begin{tabular}{|l|} \hline \\
{\bf Theorem 6:} Let  be any Horn function, and let  be the {\it pure} Horn function\\
defined by Mod. Then .\\ \\ \hline \end{tabular}

{\it Proof.} Since Mod is a subsemilattice, Mod is indeed a closure system. Let  be the induced pure Horn function, and let  be a base of implications for Mod of minimum cardinality . We claim that  is a base of : Indeed, if say  then spelling out the complication  gives . It kills exactly one -closed set, namely . Therefore . 

Conversely, let  be a base of  of cardinality . Putting , it suffices to show that  is a base of ;  then  as claimed. First, each model  of  remains a model of  because  for all . Second, let  be a model of  which is not a model of . Then  for some , and so  in view of . \quad 

Theorem 6 suggests a simple procedure to ``almost minimize'' a given base  of : Take the base  of  and replace it by a minimum base  e.g. by using Shock's algorithm (Expansion 11). Then  is a base of  of cardinality at most . 
In Expansion 20 we indicate that calculating the precise value of  is comparatively tedious.


{\bf 4.5.3} An analogue of the Guigues-Duquenne base (3.2) is introduced in [AB] for general Horn functions . It is shown that a well known query leraning algorithm of Angluin et al. in fact always produces this base, independently of the counterexamples it receives.



\section{Omitted proofs and various expansions}




{\bf Expansion 1}. We note that  as defined in (3) is a closure system even when  is not idempotent. See [W7, Expansion 1] for details.




{\bf Expansion 2}. As to the algorithmic complexity of calculating , let us merely look at the partial problem of calculating  from . If  then it costs time  to check whether or not  for some fixed index . Thus for  as in (5) it costs  to get  from  in the ``naive way'' suggested by definition (6). If we think of the premises  as the rows of a  matrix  with entries  and , then the naive way amounts to process  row-wise. It isn't hard to see [W1, p.114] how a {\it column-wise} processing of  also yields . The theoretic cost is the same, i.e. , but in practise the column-wise way is the better the larger . For instance, it takes more time to process a million sets of cardinality 100 (since they need to be ``fetched'' individually) than to process only 100 sets albeit each of cardinality a million. This trick, known as {\it vertical layout} in the Frequent Set Mining community (also observed in [W1]), often works when many but small sets need to be manipulated. In the Relational Database community the algorithm {\it LinClosure} [MR2] to calculate  has become the standard. Whether LinClosure or vertical layout or something else is best, depends on the shape of  and a smart implementation of vertical layout. 


{\bf Expansion 3.} Recall from Boolean logic (or other logic frameworks) that a formula  is a ``consequence'' of a formula  (written ) if every ``structure'' that satisfies  also satisfies . This is the {\it semantic} level. It contrasts with the {\it syntactic} level where a formula  is ``derivable'' from a formula  (written ) if  can be obtained from  with certain ``inference rules'' in a step-by-step manner. Two pages of details can be found in [W7, Expansion 3].



{\bf Expansion 4}. One algorithm for enumerating all closed sets, called NextClosure, was devised by B. Ganter in 1984 and became a cornerstone of FCA. Its key idea is to generate the closed sets in lexicographic order. See [GW, Thm.5], from which one also readily deduces the following:

(36) \quad Suppose the closure operator  is such that calculating  takes time at\\
\hspace*{.9cm} most  for any . Then NextClosure enumerates all  many closed sets in\\
\hspace*{.9cm} output polynomial time .

One benefit of NextClosure is that it doesn't matter in which way the closure operator  is provided. Thus  could be given as  where  is a -generating set of  (first way), or  where  is an implication base (second way), or any other way. In fact  itself can be a certain selfmap of  more general than a closure operator, see [GR]. 
As to the first way, apart from NextClosure and Dowling's algorithm (2.1.1), many other methods to construct  from  are evaluated in [KuO1]. As to the second way, it usually cannot compete with the compressed calculation of  in Section 4.4. However, the issue (3.6.3) is often how to find an implication base  of  in the first place. Another popular application of NextClosure is {\it attribute exploration} [GW, p.85]. This particular kind of Query Learning strives to compute the canonical base  of some hidden closure system . Unfortunately, as a not always welcome side product, the whole of  gets calculated one by one along the way. Impressive strides to avoid this succeed for the kind of ``modern'' attribute exploration proposed in [RDB] and [AN2]. 



{\bf Expansion 5} A non-independent set is {\it dependent}, and minimal dependent sets are {\it circuits}. This terminology [W3] is motivated by the established use of ``circuit'' for matroids (4.1.4) and convex geometries (4.1.5). Let now  be a circuit of . Since  is dependent there is at least one  with . The minimality of  implies that  is a {\it stem} with root . Thus if

then  and each  induces a {\it root-stem-partition} . Observe that an arbitrary root  with stem  need {\it not} yield a circuit . For instance, let  be the closure operator induced by the implications  and . Then  is a stem for the root  but  is no circuit because it contains the proper dependent subset .

{\bf Open Problem 2}: Develop a theory for those closure operators (e.g. their optimum bases), for which each root-stem-partition  is a circuit.

Most prominently, matroids and convex geometries belong to this class of closure operators. In the first case each circuit  has roots, in the second case .



{\bf Expansion 6} It is easy to see that neither a properly quasiclosed set  needs to contain a -equivalent stem , nor is a stem  necessarily contained in a -equivalent proper quasiclosed set. Nevertheless, those stems  that {\it coincide} with a properly quasiclosed set can be characterized neatly. For starters, since each stem  is independent and a proper subset of an independent set has a strictly smaller closure, we see that:

(37) \quad Each stem which is properly quasiclosed is in fact pseudoclosed.

This raises the problem to grasp the ``pcst-sets'' which by definition are pseudoclosed and a stem (i.e. belong to  {\it and} ). If  is pseudoclosed then one can decide whether  is pcst as follows: For all  check whether  is {\it minimal} with the property that . No better description of the pcst-sets {\it within the family of all pseudoclosed sets} seems to be known. In contrast, the pcst-sets look neat {\it within the family of all stems}:

\begin{tabular}{|l|} \hline \\
{\bf Theorem 7:} For each stem  of a closure operator  the \\
following properties are equivalent:\\
\\
(i) \  is pseudoclosed.\\
\\
(ii)  is inclusion-minimal among all stems of .\\
\\
(iii)  is a {\it strong} stem in the sense that roots.\\ \\ \hline \end{tabular}


{\it Proof of Theorem 7.}
As to (i) \  \ (ii), we show that (i) \ (ii), i.e. that

As to ``'', take . By the definition of  there is a  with . We can shrink  to a stem  of . As to ``'', because  is a stem we can be sure that . If  then , where  is due to the independence of . Thus .

As to (i)  (iii), if  then again  since  (being a stem) is independent. Hence . So for {\it each}  the set  is minimal w.r.t. the property that its closure captures . As to (iii)  (ii), suppose  was a stem, say . Necessarily  since  is independent. But then  by assumption, and so  is impossible. This contradiction shows that  is inclusion-minimal. \quad 

Theorem 7 draws on [KN]. We changed ``prime stem'' in [KN] to ``strong stem'' in order to avoid confusion with the prime implicates in 3.4.3. 



{\bf Expansion 7.} If  is given as a CNF then the well-known {\it consensus method} [CH, 2.7] is applicable to generate all prime implicates of . For instance let  be the conjunction of the four clauses at level  in Table 2 below (where e.g.  abbreviates ). The clauses  and  are such that there is exactly {\it one} literal  which appears in one clause and its negation in the other; namely . In this situation we add (while keeping ) the {\it consensus} clause  which is thus obtained by dropping  and  from the disjunction . All consensi obtained from level  are listed in level . One continues by building consensi between  and , and then between  and . All of these are listed in . The list  is long enough that some of its members get unveiled as redundant; such as  which is implied by . Level  contains the pruned list. Building consensi within  (more precisely between the first and second line of ) yields . Pruning  yields .



Table 2: The consensus  algorithm (simple version)

Now  yields no new consensi. According to a famous 1959 theorem of Quine [Q] the members in  hence constitute {\it all} prime implicates of . We mention that  matches  in (27). See [CH, chapter 6.5] for a consensus method working for all Boolean functions , and running in polynomial incremental time in the case of Horn functions . The consensus method can be viewed as a special case of an algorithm [AACFHS] that generates all maximal bicliques ( complete bipartite subgraphs) of a graph . If  itself is bipartite, say with shores  this problem amounts to generate all closed sets of a Galois connection (2.1.2). 



{\bf Expansion 8}. We present a novel way for the direction . 
Suppose that

(38) \quad 

Observe that  is equivalent to  in Expansion 7 and whence to the family of implications in (27). Hence, if our method is correct, we will wind up with  as in (26). As shown in
Section 4.4 by running the implication -algorithm one can represent  as a disjoint union of eight -rows, i.e. subcubes of , as shown in Table 3. Let us argue that such a representation readily yields  as a side product. 

\begin{tabular}{c|c|c|c|c|c|c|} 
& 1 & 2& 3& 4& 5& 6 \\ \hline
 & 0 & 2& 0 &2 & 2& 0 \\ \hline
 & 1 &2 & 0 & 2 & 0 & 0 \\ \hline
 & 1 & 2& 0 & 1& 1& 0 \\ \hline
 & 0 & 0 & 1 & 2 & 1 & 1 \\ \hline
 & 1 & 0 & 1 & 1 & 1 &1 \\ \hline
 & 1 & 1 & 1& 1 & 1& 2 \\ \hline
 & 0 & 0 & 1 & 2& 1& 0 \\ \hline
 & 1 & 0 & 1 & 1 & 1 &0 \\ \hline \end{tabular}

Table 3: Getting  by column-wise processing a compressed representation of 

By (21) it suffices to show how to get  for any particular . Say .  If  has its fourth component equal to 1 then  cannot contain a member of . This e.g. happens for . If the fourth component of  is 0 or 2 then at most the unique row-maximal set  {\it may} belong to . Hence the collection of all maximal row-maximal sets is . Thus


Likewise the other collections  are obtained, and so we get (matching (26)) that 


Let  be the set of hyperplanes of . Obviously the minimal keys of  are exactly the minimal transverals of , and so any good algorithm for mtr yields them, provided the hyperplanes are known. In particular, the 's can be gleaned from a table like Table 3 since .


{\bf Expansion 9.} Here we present another view of Table 3 in Expansion 8. But first we need to dualize some concepts from 3.4.1. 
Thus a conjunction of literals is called a {\it term}. The model set of a term , viewed as a Boolean function , is an interval in the lattice . (It is also common, although less precise, to speak of ``subcubes'' instead of intervals.) For instance if  is  then Mod. This -{\it row} is a succinct notation for the interval  .  A {\it disjunctive normal form} (DNF) is any disjunction of terms. 


Now back to Table 3. The pure Horn function matching  in (38) is

We aim to convert this CNF into a DNF. Because Mod is represented as the union of the -rows  in Table 3, and because  for obvious terms , one DNF for  is

The above DNF is {\it orthogonal} [CH, chapter 7] in the sense that Mod for . It would be interesting to know how to exploit the orthogonality of a DNF in a (dual) consensus method.



{\bf Expansion 10}. {\it Proof of (22)}. As to , from  follows that  is maximal within  w.r.to . A fortiori  is maximal within  w.r.to , {\it provided}  belongs to  at all. But this follows from (21) and . As to , let  be maximal w.r.to . Then there is  which is maximal w.r.to  and . Hence  by definition of the latter, and so  by (21). By the maximality property of , we have .



{\bf Expansion 11}. As to going from  (or in fact from any base) to a minimum base , we illustrate the method of Shock [Sh], which first demands to replace, for each  in , the conclusion  by  where  is the closure operator induced by . For  in (27) we get an equivalent family of full implications

Recall from (8) that  is redundant iff  is contained in the -closure of . Incidentally , as defined before (12), is  for all, and so the -closure of  is  by (12). Because of  we can thus drop  from . Further  can be dropped because of , and  can be dropped because of . The resulting base

is nonredundant and whence minimum by Theorem 1(d). The kind of minimum base  obtained by Shock can by Theorem 1 easily be ``blown up'' to .


{\bf Expansion 12}.  In [W7, Expansion 13] it is shown how  relates to lattice theory, in particular to the relations  which originated in [D1] and are akin to the ones in [GW, p.31]. Coupled to each lattice  there is an importatn bipartite graph with shores  and .


{\bf Expansion 13}.  In [W7, Expansion 14] we show the well known fact [CM] that the collection  of all closure systems  is itself a closure system, in fact (viewed as a lattice) it is meet-distributive. Furthermore the technical proof of property (39) in Expansion 20 features there.


{\bf Expansion 14.} For any closure operator  consider these properties:

\begin{tabbing}
123456\=\kill
 \> \\
\\
 \>  \ is closed\\
\\
 \> \\ 
\end{tabbing}

The properties  and  are well known ``separation axioms'' from topology. For instance  in Figure 4(a) violates .
The notation  stems from [W5] but the property was previously considered. All three axioms make sense for non-topological operators . It is an exercise to verify ; furthermore  when  holds. In fact, as shown in [W5, Thm.8],  is isomorphic to a standard operator  as in (30) iff  satisfies . It is easy to ``boil down'' any closure operator  on a set  to an operator  of type  on a smaller set , and  to  of type  on a still smaller set , in such a way that the lattices  and  are isomorphic.
See [W5, p.165] or [GW, ch.1.1, 1.2] for details.  
Albeit the lattices  and  are isomorphic, this may be of little help to get a good base of  from one of . For instance it takes some effort to find an optimum base for the closure system  in Figure 4(a). In contrast  is a Boolean lattice and whence has the empty set as an optimum base! (See also Open Problem 4 in Expansion 15.)

{\bf Expansion 15}. Recall from (29) that  is a lattice isomorphism from  onto  and that  but usually . To see that ``'' takes place in the distributive case, fix . Then  and distributivity imply that . Since  is join irreducible this forces  or , whence  or , whence . Hence  is a sublattice of . Consequently the closure operator  from (30) is topological, in fact . 
Therefore  is the lattice  of all order ideals of the poset .
In particular, since  by (29), we have . This is Birkhoff's Theorem, see [Bi, p.59]. 

As to implicational bases, for any lattices  it is natural to consider the set  
of implications

where  is the set of lower covers of  within  and  is the set of all non-minimal members of . It is clear that  is the collection of all order ideals of . Hence  is a base of  iff  is distributive.  Actually  is the unique {\it optimum} base for each distributive lattice . That follows immediately from 3.2.2 (all circle formations are points here). Note that  when  is Boolean. For nondistributive lattices  may constitute a relevant part of larger bases. Most prominently, according to 4.1.3 each optimum base of a modular lattice includes .  On the downside,  needs {\it not} be part of every optimum base of a lattice. For instance the lattice  in Figure 11 has  whereas one optimum base of  is .

{\bf Open Problem 3}: Determine the class  lattices  (among which all modular ones) for which  in Expansion 15 is part of every optimum base of .



\begin{center}
\includegraphics[scale=0.7]{JoyOfImplicFig11}
\end{center}

As seen above, for topological operators  the lattice  is a sublattice of , and whence distributive. However as seen in 3.1,  can be distributive without being a sublattice of .

{\bf Open Problem 4:} Let  have a distributive lattice  which is {\it not} a sublattice of . Can one find an optimum base of  in polynomial time?



{\bf Expansion 16.} We start by proving (33) in 4.1.5. So let  be critical, i.e.  is closed for all . In order to show that  is quasiclosed\footnote{Notice that when  is quasiclosed then it is properly quasiclosed since .} we take (in view of (9))  with  and aim to show that . There must be an  with  (otherwise  yields the contradiction ). But then , and so by assumption .

Next, assuming  is quasiclosed, we show that  is a closure-minimal stem of  in the sense of Expansion 6. Suppose to the contrary there was a stem  of  with . From  and  follows (since ) that . This is impossible since  (by the definition of stem).

 Finally, letting  be closure-minimal, assume by way of contradiction that  is not closed for some . First,  is closed because . Hence , and so there is a stem  of . It satisfies , and thus  is not closure-minimal.  This proves (33). \quad 
 
 Yet another (equivalent) definition of ``critical'' is given in [W3, p.136]. Furthermore  is called {\it extra-critical} in [W3] if the quasiclosed set  in 
 (33) coincides with .
  
  If  is a convex geometry, then the set system  is a so called {\it antimatroid}. One can define antimatroids independent of  as union-closed set systems  which are hereditary in the sense that for each nonempty  there is some  with . What we defined as a rooted circuit  in 4.1.5 relates as follows to : Whenever  then ; and  is minimal with this property. In fact this is the {\it original} definition of a rooted circuit [KLS, p.28]. Apart from rooted circuits our definition of a critical circuit  in 4.1.5 similarly matches the definition given in [KLS, p.31]. Each antimatroid  can (apart from the set system view) equivalently be rendered as a certain {\it hereditary language}. From this perspective the critical circuits provide an optimal representation of , see [KLS, Thm.3.11]. This contrasts with the fact that  usually is {\it no} implicational base (see 4.1.5). Antimatroids and convex geometries arise in many contexts, often related to combinatorial optimization, see [KLS, III.2].
  

A lattice  is {\it meet-distributive} if the interval  is Boolean for all . (Many equivalent characterizations exist.) Every convex geometry  has a meet-distributive lattice . Conversely, if  is meet-distributive then  is a convex geometry. 
 The dual concept of meet-distributivity is {\it join-distributivity}, i.e. when 
 is a Boolean interval for all .  A lattice which is both meet and join-distributive must be distributive, and conversely.



 
 
 
A lattice  is {\it join-semidistributive}  if for all  it follows from  that . 
In such a lattice . See also [W7, Expansion 13].  Notice that ``meet-distributive  join-semidistributive''. In fact, the  lattices  of length  are exactly the meet-distributive lattices. If  is  then by [AN1, prop.49] every essential set  of  has a {\it unique} quasi-closed generating set  (which equals  in the meet distributive case). Conversely such a {\it unique-criticals} lattice need not be . See Figure 12.   Further topics in [AN1] include the uniqueness of the -basis (see 4.1.1) for  standard closure systems , and the fact that such  generally don't belong to the class  in Open Problem 3 of Expansion 15. Dually to  one defines {\it meet-semidistributivity} .  It comes as no surprise that ``join-distributive  
meet-semidistributive''. Results about bases of -lattices are given in [JN], and exploited in [W4].  See also Expansion 18.



{\bf Expansion 17.} As a variation of Theorem 5, -semilattices (in particular lattices ) can also be described as systems of {\it restricted} order ideals of a poset. This generalizes the representation of  distributive lattices, for which {\it all} order ideals are used (Expansion 15). The restriction imposed on the order ideals is governed by core where  is as in (30) and core as in (14). We mention that in [D] core is determined for many types of lattices . Notice that  and that from core alone one cannot obtain . See [W7, Expansion 18] for more details.

 
{\bf Expansion 18.} The -relation, which is of importance in the study of free lattices, is defined as follows. For  put  if  appears in some order-minimal join cover  of . A {\it -cycle} is a configuration of type . For instance the convex geometry in 2.2.5 has the -cycle  because 146 is a minimal join cover of 8 and 238 is a minimal join cover of 6.
Each -cycle induces a cycle in  for each base  of , but not conversely. Hence closure operators without -cycles are strictly more general than acyclic operators.
Indeed, the former have  closure systems by [FJN], the latter meet-distributive ones by Theorem 3 (see also Expansion 16). While the presence of -cycles can be decided from  in polynomial time [AN1, Thm.43], this is unknown for checking .

Likewise the {\it affine} convex geometries (as 2.2.5 but in , not just ) can be generalized, i.e. to convex geometries satisfying  the so-called -{\it Carousel Property}. This property was crucial in article [AW] that dealt with the realizability (in ) of convex geometries. Implication bases of convex geometries with the -Carousel Property can  be optimized in polynomial time [A, Thm.12], but the arguments get 	``uglier'' than the deliberations in 2.2.5. Notice that checking the -Carousel property ( fixed), as opposed to checking realizability, is ``straightforward'' albeit tedious. Furthermore, optimization of implication bases of order-convex\footnote{By definition the closed sets of an {\it order-convex} geometry are all intervals of some poset.} geometries is polynomial-time [A, sec.6].

 

\begin{center}
\includegraphics{JoyOfImplicFig12}
\end{center}
 
{\bf Expansion 19}. It is easy to replace a -row by a couple of disjoint -rows. For instance  is the same as . 
Sometimes -rows are easier to handle, if only for pedagogical reasons as in Table 3 of Expansion 8. Conversely, a random collection of -rows usually cannot be compressed to fewer -rows. As seen in 4.4 the -algorithm produces its rows ``from scratch'' without an intermediate state of -rows.
Further fine-tuning is possible. For instance, instead of replacing  by  in Table 1 we could have replaced it by the single row  where generally the wildcard  signifies that either  or  must take place. 
The author exploited this idea in the special case where all  are of type  in the first place; this essentially amounts to enumerating all order ideals of a poset. 
In a similar manner all anticliques ( independent vertex sets) of a graph can be enumerated in a compact manner (work in progress). 
 
 

{\bf Expansion 20.} Let us sketch how to (a) get a -minimum base of a Horn function , and (b) how to merely calculate .

As to (a), it relates to [W7, Expansion 14] where we showed that for any -subsemilattice  the collection

(39) \quad 

is a sublattice of the lattice  of all closure systems on . Clearly  is the smallest element of . Let  be the pure Horn function matching the {\it largest} element  of . Albeit  as a subset of  cannot be described as simply as , it is shown in [HK, Lemma 4.2] that  must be the conjunction of all {\it pure} prime implicates of . Once calculated (consensus method), this {\it pure Horn part}  of  can be used as follows to minimize . Compute all {\it negative} prime implicates ( complications)  of . Take them as the vertices of a graph  which has an arc from  to  iff  is a consequence of . Let  be the strong components of  that have in-degree  when viewed as elements of the induced factor poset. Now let  be any transversal of  and let  be any minimum base of . Then  is a minimum base of  [HK,Theorem 6.2]. 

As to (b), up to duality in [CH, 6.7.3] one associates with an impure Horn function  in  variables a {\it pure} Horn function  in  variables as follows. Take any base  of  and let  be the function induced by  where . According to [CH, Lemma 6.8, Thm.6.15] this is well-defined, i.e. independent of the chosen base  of . Furthermore . The intricasies of proving  are not mirrored on the algorithmic side: Switching from  to  is trivial, and minimizing  to  works in quadratic time (Expansion 11) and yields .



 
 
 
 
\section*{Acknowledgement:} I am grateful for comments from  Kira Adaricheva, Roni Khardon, Sergei Kuznetsov, Jos\'{e} Balc\'{a}zar, Ron Fagin,  Gert Stumme,  Sergei Obiedkov, Sebastian Rudolph, Hiroshi Hirai, Giorgio Ausiello, Bernard Monjardet.

\section*{References}
\begin{enumerate}
\item[{[A]}] K. Adaricheva, Optimum basis of finite convex geometry, to appear in Disc. Appl. Mathematics.
	\item[{[AN1]}] K. Adaricheva, J.B. Nation, On implicational basis of closure systems with unique critical sets, Appl. Math. 162 (2014) 51-69.
\item[{[AN2]}] K. Adarichva,  J.B. Nation, Discovery of the -basis in binary tables based on hypergraph dualization, arXiv:1504.02875v2.
	\item[{[ANR]}] K. Adaricheva, J.B. Nation, R. Rand, Ordered direct implicational basis of a finite closure system, Discrete Appl. Math. 161 (2013) 707-723.
\item[{[AACFHS]}] G. Alexe, S. Alexe, Y. Crama, S. Foldes, P.L. Hammer, B. Simeone, Consensus algorithms for the generation of all maximal bicliques, Disc. Appl. Math. 145 (2004) 11-21.
	\item [{[AB]}] M. Arias, J.L. Balcazar, Canonical Horn representations and Query Learning, Lecture Notes in Computer Science 5809 (2009) 156-170.
	\item[{[ADS]}] G. Ausiello, A. D'Atri, D. Sacca, Minimal representation of directed hypergraphs, SIAM J. Comput. 15 (1986) 418-431.
\item[{[AW]}] K. Adaricheva, M. Wild, Realization of abstract convex geometries by point configurations, Europ. J. Comb. 31 (2010) 379-400.
\item[{[B]}] J. L. Balc\'{a}zar, Redundancy, deduction schemes and minimum-size bases for association rules, Logical Methods in Computer Science 6 (2010) 1 - 33.
\item[{[Bi]}] G. Birkhoff, Lattice Theory, AMS 1984.
\item[{[BCKK]}] E. Boros, O. Cepek, A. Kogan, P. Kucera, A subclass of Horn CNFs optimally compressible in polynomial time, Annals Math. Artif. Intelligence (2009) 249-291.
\item[{[BDVG]}] K. Bertet, C. Demko, J.F. Viaud, C. Gu\'{e}rin, Lattices, closure systems and implication bases: a survey of structural aspects and algorithms, arXiv.
	\item[{[BG]}] E. Boros, A. Gruber, Hardness results for approximate pure Horn CNF Formulae minimization, Ann. Math. Artif. Intell. 71 (2014) 327-363.
\item[{[BK]}] M.A. Babin, S.O. Kuznetsov, Computing premises of a minimal cover of functional dependencies is intractable, Disc. Applied Math. 161 (2013) 742-749.
	\item[{[BM]}] K. Bertet, B. Monjardet, The multiple facets of the canonical direct unit implicational basis, Theoretical Computer Science 411 (2010) 2155-2166.
\item[{[BMN]}] L. Beaudou, A. Mary, L. Nourine, Algorithms for -meet semidistributive lattices, arXiv.
\item[{[Bu]}] L.M. Butler, Subgroup lattices and symmetric functions, Memoirs AMS 539 (1994).
\item[{[C]}] N. Caspard, A characterization theorem for the canonical basis of a closure operator, Order 16 (1999) 227-230.
\item[{[CH]}] Y. Crama, P.L. Hammer, Boolean Functions, Encyc. of Math. and Appl. 142, Cambridge Univ. Press 2011.
\item[{[CM]}] N. Caspard, B. Monjardet, The lattices of closure systems, closure operators, and implicational systems on a finite set: a survey. Discrete Applied Mathematics 127 (2003) 241-269.
\item[{[D]}] V. Duquenne, The core of finite lattices, Discrete Mathematics 88 (1991) 133-147.
\item[{[D1]}] A. Day, Characterization of finite lattices that are bounded-homomorphic images or sublattices of free lattices, Can. J. Math. 31 (1979) 69-78.
\item[{[D2]}] A. Day, The lattice theory of functional dependencies and normal decompositions, International Journal of Algebra and Computation 2 (1992) 409-431.
\item[{[DHO]}] P.O. Degens, H.J. Hermes, O. Opitz (eds), Die Klassifikation und ihr Umfeld, Indeks Verlag, Frankfurt 1986.
\item[{[DS]}] F. Distel, B. Sertkaya, On the complexity of enumerating pseudo-intents, Disc. Appl. Math. 159 (2011) 450-466.
\item[{[EMG]}] T. Eiter, K. Makino, G. Gottlob, Computational aspects of monotone dualization: A brief survey, Discrete Appl. Math. 156 (2008) 2035-2049.
\item[{[F]}] R. Fagin, Functional dependencies in a relational data-base and propositional logic, IBM. J. Res. Develop. 21 (1977) 534-544. (Cited in [W7].)
\item[{[FV]}] R. Fagin, M.Y. Vardi, The theory of database dependencies - a survey. Mathematics of Information Processing, Proceedings of Symposia in Applied Mathematics 34 (1986) 19-71. (Cited in [W7].)
\item[{[FD]}] J.C. Falmagne, J.P. Doignon, Learning Spaces, Springer-Verlag Berlin Heidelberg 2011.
\item[{[FJN]}] R. Freese, J. Jezek, J.B. Nation, Free lattices, Math. Surveys and Monographs 42, Amer. Math. Soc. 1995.
\item[{[G]}] G. Gr\"{a}tzer, Lattice Theory: Foundation, Birkh\"{a}user 2011.
\item[{[GD]}] J.L. Guigues, V. Duquenne, Familles minimales d'implications informatives r\'{e}sultant d'une table de donn\'{e}es binaires, Math. Sci. Hum. 95 (1986) 5-18.
\item[{[GW]}] B. Ganter, R. Wille, Formal Concept Analysis, Springer 1999.
	\item[{[GR]}] B. Ganter, K. Reuter, Finding all closed sets: A general approach, Order 8 (1991) 283-290.
\item[{[HK]}] P.L. Hammer, A. Kogan, Quasi-acyclic propositional Horn knowledge bases: Optimal compression, IEEE Trans. on knowledge and data engineering 7 (1995) 751-762.
\item[{[JN]}] P. Jansen, L. Nourine, Minimum implicational bases for -semidistributive lattices, Inf. Proc. Letters 99 (2006) 199-202.
\item[{[KKS]}] H. Kautz, M. Kearns, B. Selman, Horn approximations of empirical data, Artificial Intelligence 74 (1995) 129-145.
\item[{[K]}] R. Khardon, Translating between Horn Representations and their characteristic models, Journal of Artificial Intelligence Research 3 (1995) 349-372.
\item[{[KLS]}] B. Korte, L. Lova\'{a}sz, R. Schrader, Greedoids, Springer-Verlag 1991.
\item[{[KN]}] K. Kashiwabara, M. Nakamura, The prime stems of rooted circuits of closure spaces, The electronic journal of combinatorics 20 (2013), Paper 22, 13 pages.
\item[{[KR]}] R. Khardon, D. Roth, Reasoning with models, Artificial Intelligence 87 (1996) 187-213.
\item[{[KuO1]}] S.O. Kuznetsov, S. Obiedkov, Comparing performance of algorithms for generating concept lattices, J. Expt. Theor. Art. Intelligence 14 (2002) 189-216.
\item[{[KuO2]}] S.O. Kuznetsov, S.A. Obiedkov, Some Decision and Counting Problems of the Duquenne-Guigues Basis of Implications. Discrete Applied Mathematics 156 (2008) 1994-2003.
\item[{[M]}] D. Maier, The Theory of Relational Databases, Computer Science Press 1983.
\item[{[Ma]}] D. Marker, Model Theory: An Introduction, Springer Verlag 2002. (Cited in [W7].)
\item[{[MR1]}] H. Mannila, K-J. R\"{a}ih\"{a}, Design by example: An application of Armstrong Relations, Journal of Computer and System Sciences 33 (1986) 126-141.	
		\item[{[MR2]}] H. Mannila, K.J. R\"{a}ih\"{a}, The design of relational databases, Addison-Wesley 1992.
\item[{[MU]}] K.Murakami, T. Uno, Efficient algorithms for dualizing large scale hypergraphs, Disc. Appl. Math. 170 (2014) 83-94.
\item[{[N]}] J.B. Nation, An approach to lattice varieties of finite height, Algebra Universalis 27 (1990) 521-543.
\item[{[P]}] J. Paredaens, About functional dependencies in a database structure and their coverings, Philips MBLE Lab. Report 342, Brussels 1977.
\item[{[PKID1]}] J. Poelmans, S.O. Kuznetsov, D.I. Ignatov, G. Dedene, Formal Concept Analysis in Knowledge Processing: A survey on models and techniques.
\item[{[PKID2]}] J. Poelmans, S.O. Kuznetsov, D.I. Ignatov, G. Dedene, Formal Concept Analysis in Knowledge Processing: A survey on applications, Expert Systems with Applications 40 (2013) 6538-6560.
\item[{[Q]}] WV. Quine, On cores and prime implicants of truth functions, Amer. Math. Monthly 66 (1959) 755-760.
\item[{[R]}] S. Rudolph, Succinctness and tractability of closure operator representations, arXiv.
\item[{[RCEM]}] E. Rodriguez-Lorenzo, P. Cordero, M. Enciso, A. Mora, A logical approach for direct-optimal basis of implications, Bull. Eur. Assoc. Theor. Comp. Sci. 116 (2015) 204-211.
\item[{[RDB]}] U. Ryssel, F. Distel, D. Borchmann, Fast algorithms for implication bases and attribute exploration using proper premises, Ann Math Artif Intell 70 (2014) 25-53.
\item[{[RN]}] S. Russell, P. Norvig, Artificial Intelligence: A modern approach, Prentice Hall 2003.
	\item[{[S]}] A. Schrijver, Combinatorial Optimization (three volumes), Springer 2003.
	\item[{[Sh]}] R.C. Shock, Computing the minimum cover of functional dependencies, Inf. Proc. Letters 22 (1986) 157-159.
	\item[{[SW]}] L. Santocanale, F. Wehrung, Lattices of regular closed subsets of closure spaces, Internat. J. Algebra Comput. 24 (2014) 969-1030.
	\item[{[W1]}] M. Wild, Computations with finite closure systems and implications, Lecture Notes in Computer Science 959 (1995) 111-120. (An extended version, available as pdf, is the Tech. Hochschule Darmstadt Preprint Nr. 1708 from 1994.)
	\item[{[W2]}] M. Wild, Optimal implicational bases for finite modular lattices, Quaestiones Mathematicae 23 (2000) 153-161.
	\item[{[W3]}] M. Wild, A theory of finite closure spaces based on implications, Advances in Mathematics 108 (1994) 118-139.
\item[{[W4]}] M. Wild, Compressed representation of Learning Spaces. To appear in the Journal of Mathematical Psychology.
\item[{[W5]}] M. Wild, Implicational bases for finite closure systems, {\it Arbeitstagung, Begriffsanalyse und K\"{u}nstliche Intelligenz}, Informatik-Bericht 89/3 (1989), pp.147-169, Institut f\"{u}r Informatik, Clausthal. (The article is downloadable from the ResearchGate.)
\item[{[W6]}] M. Wild, Compactly generating all satisfying truth assignments of a Horn formula, Journal on Satisfiability, Boolean Modeling and Computation 8 (2012) 63-82.
\item[{[W7]}] M. Wild, The joy of implications, aka pure Horn formulas: mainly a survey. This is a preliminary version (arXiv: 1411.6432v2) of the present article. It features the full versions of Expansions 1, 3, 12, 13, 17.
\item[{[Wi]}] R. Wille, Subdirect decomposition of concept lattices, Algebra Universalis 17 (1983) 275-287. (Cited in [W7].)
\end{enumerate}


\end{document}
