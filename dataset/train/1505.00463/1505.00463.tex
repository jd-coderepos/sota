\documentclass[conference]{IEEEtran}




\usepackage{cite}
\usepackage{graphicx}
\usepackage{caption}
\usepackage{subcaption}

\usepackage[cmex10]{amsmath}
\usepackage{algorithmic}
\usepackage{array}
\usepackage[font=footnotesize]{subfig}
\usepackage{url}
\usepackage{fontenc}
\usepackage[utf8]{inputenc}
\usepackage{mathtools}
\usepackage{tabularx}
\usepackage{algorithmic}
\hyphenation{op-tical net-works semi-conduc-tor}

\newsavebox{\ieeealgbox}
\newenvironment{boxedalgorithmic}
  {\begin{lrbox}{\ieeealgbox}
   \begin{minipage}{\dimexpr\columnwidth-2\fboxsep-2\fboxrule}
   \begin{algorithmic}}
  {\end{algorithmic}
   \end{minipage}
   \end{lrbox}\noindent\fbox{\usebox{\ieeealgbox}}}
\newcolumntype{M}[1]{>{\arraybackslash}m{#1}}
\newcolumntype{N}{@{}m{0pt}@{}}
\begin{document}
\title{Opportunistic Spectrum Allocation for Max-Min Rate in NC-OFDMA}


\author{\IEEEauthorblockN{Ratnesh Kumbhkar, Tejashri Kuber, Narayan B. Mandayam, Ivan Seskar} \IEEEauthorblockA{WINLAB, Rutgers
University\\ 671 Route 1, North Brunswick NJ 08902\\ Email:
\{ratnesh, tkuber, narayan, seskar\}@winlab.rutgers.edu}}
\IEEEoverridecommandlockouts
\maketitle

\begin{abstract}
We envision a scenario of opportunistic spectrum access among multiple links when the available spectrum is not contiguous due to the presence of external interference sources. Non-contiguous Orthogonal Frequency Division Multiplexing (NC-OFDM) is a promising technique to utilize such disjoint frequency bands in an efficient manner. In this paper we study the problem of fair spectrum allocation across multiple NC-OFDM-enabled point-to-point cognitive radio links under certain practical considerations that arise from such non-contiguous access. When using NC-OFDMA, the channels allocated to a cognitive link are spread across several disjoint frequency bands leading to a large \textit{spectral span} for that link. Increased spectral span requires higher sampling rates, leading to increased power consumption in the ADC/DAC of the transmit/receive nodes. 
In this context, this paper proposes a spectrum allocation that maximizes the minimum rate achieved by the cognitive radio links, under a constraint on the maximum permissible spectral span. Under constant transmit powers and orthogonal spectrum allocation, such an optimization is a mixed-integer linear program and can be solved efficiently. 
There exists a clear trade-off between the max-min rate achieved and the maximum permissible spectral span. The spectral allocation obtained from the proposed optimization framework is shown to be close to the trade-off boundary, thus showing the effectiveness of the proposed technique. We find that it is possible to limit the spectrum span without incurring a significant penalty on the max-min rate under different interference environments. We also discuss an experimental evaluation of the techniques developed here using the Universal Software Radio Peripheral (USRP) enabled ORBIT radio network testbed. 
\end{abstract}


\section{Introduction}
As the number of devices using wireless spectrum has increased, availability of usable spectrum for the licensed devices is a concern. Cognitive radio (CR) plays an important role in addressing this problem with dynamic spectrum access. In the past few years there has been a large amount of research on addressing different aspects of cognitive radios (e.g. \cite{ mitola99cognet, pricognet, haykin05cognetbrain, thomas05cognet,  ileri08dsamodel, liang08sense, srini07cognetdsa, Cesana11routing, cheng07joint, NazmulWiOpt}). Orthogonal frequency division multiplexing (OFDM)  has been suggested as one of the candidates for dynamic spectrum access in CRs due to its flexible and efficient use of the spectrum \cite{weiss04}. Non-contiguous OFDM (NC-OFDM) is a method of transmission where some of the subcarriers in OFDM are nulled and  only the remaining subcarriers are used for transmission \cite{rajbanshi06ncofdmrx, UCSB1, UCSB2, NCOFDM_Implementation}. Since available unused spectrum is generally non-contiguous,  usage of  NC-OFDM  results in better spectrum utilization. Further, since NC-OFDM allows the CRs to access the unused spectrum without interfering with the licensed users, it also complies with the broader objective  that primary users of the spectrum need not consider the presence of CRs and can be completely oblivious to them.
Techniques for efficient implementation of the DFT operation for NC-OFDM when multiple subcarriers are nulled are also available. \cite{rajbanshi06ncofdmrx}. However, one main drawback of NC-OFDM is that it suffers from high out-of-band radiation due to the high sidelobes of its modulated subcarriers, which can potentially affect the performance of licensed users, or other CRs in the unlicensed band. Several techniques to address this issue have been proposed and we briefly touch upon these issues in the latter part of this paper. 

A significant concern when using NC-OFDMA is that the cognitive links are allocated disjoint frequency bands that lead to an increased spectral span of a cognitive link. The spectral span is defined as the difference between the frequencies of the extreme channels allocated to a cognitive link. Increase in the spectral span leads to higher sampling rates that in turn lead to an increase in the power consumption at the transmit/receive nodes.
Usually, the transmit power requirements of a transceiver system have dominated the total power consumption. However, the ADC/DAC power consumption can become comparable or even significantly larger than the transmit power consumption when the sampling rates become very large \cite{Nazmul_NCOFDM1}. Therefore, it is important to impose a reasonable limit on the spectrum span.



In this paper, we consider the problem of  spectrum allocation across multiple point-to-point cognitive links between  NC-OFDM-enabled transceivers in the presence of interference from the primary users. The main goal is to achieve a fair spectrum allocation that maximizes the minimum data rate across these cognitive links while limiting the spectral span. Towards this goal, we propose an optimization framework to maximize the minimum rate, subject to the constraint that spectrum span is not too large. Under constant transmit powers and orthogonal spectrum allocation, such an optimization is a mixed-integer linear program and can be solved efficiently using readily available solvers. Simulation results show a trade-off between the max-min rate and spectrum span. In our simulations, we also show improvement in data rate based on spectrum allocation obtained from solving the optimization problem in presence of interference. We also implement the NC-OFDM system using USRP \cite{ettus} radios with GNU Radio software platform on ORBIT testbed \cite{ray06cognet}. GNU Radio is a free and open-source software development toolkit that provides signal processing blocks to implement software radios \cite{gnuradio}. 
 


The remainder of the paper is organized as follows. The related work is presented in section \ref{sec:relwork}. In section \ref{sec:sys} we present  our system model with various channel and allocation constraints and as well as the  problem formulation. In section \ref{sec:sim} we present our simulation setup and  simulation results. The experimental setup on the  ORBIT testbed and corresponding results are shown in section \ref{sec:exp} and we  conclude in section \ref{sec:conclude}.
\section{Related work}
\label{sec:relwork}
While optimizing communication links for total transmit power is a well studied area, considerations for total system power consumption is an area of active research. The authors of \cite{Li11energysurvey} consider the effect of system power for energy efficient wireless communications. Modulation schemes optimized for system power consumption are studied in \cite{shug05energyopt}, while the authors in \cite{grover11syspower} present a communication-theoretic view of system power consumption. System power constraints specifically related to NC-OFDM are studied in in \cite{jia11cap, cao10dsa}, where it is shown that the maximum spectral span is limited by ADC/DAC \cite{jia11cap} and that the requirement of a guard band affects the overall system throughput. Nazmul et. al \cite{Nazmul_NCOFDM1} characterize the trade-off between the system power and spectrum span from a cross-layer perspective in a multi-hop network. The authors in \cite{zhang10grcol} provide a graph coloring method for  spectrum allocation with the goal of providing equal rates to each user. To the best of our knowledge, previous works have not considered fair spectrum allocation with system power considerations for NC-OFDM. Our work focuses on the opportunistic spectrum allocation to maximize the max-min rate while limiting the spectral span of the NC-OFDM-enabled cognitive radio links. 


\section{System Model}
\label{sec:sys}
We consider  a network of  point-to-point links  that use NC-OFDM for communication. The set of  links in this model is represented by . These links have access to  channels, represented by the  set , with each channel having a bandwidth of  Hz. We assume that each channel supports  OFDM subcarriers.  Transceivers in these links can be dynamically programmed to use different sets of channels.  The distance between the transmitter and the receiver in link  is denoted as . The channel gain  for link  on the th channel is represented as . The link gain encompasses antenna gain, coding gain and fading. The received power at the receiver of link  on the channel  is given by .  We assume that each channel experiences flat fading and that there is no correlation between any two channels. 
We denote the  channel allocation matrix by  . Elements of matrix  can either be 1 or 0. The th row of  represents the channel allocation vector for the th link. Elements of  are defined  as follows\\

\begin{table}[b]
\caption{List of notations}
\centering
\begin{tabular}{|M{0.7cm}|M{0.39\textwidth}|N}
  \hline
   & Set of links &\3pt]
  \hline
   & Set of total available channels &\3pt]
  \hline
   & Resource allocation matrix of size  &\3pt]
  \hline
   & Interference matrix of size  &\3pt]
  \hline
   & Channel gain for link  using channel &\3pt]
  \hline
   & Data rate for link  using channel &\3pt]
  \hline
   & Distance between transmitter and receiver for link  &\3pt]
  \hline
   & Noise spectral density&\
  \sum_{l=1}^N a_{lm} &\leq 1,\;\;\; \forall\; m \in \mathcal{M}. 
  B_l=&\Big(\max_{m \in \mathcal{M}}(m\cdot a_{lm})\nonumber\\ 
  &- \min_{m \in \mathcal{M}}(m\cdot a_{lm}+M(1-a_{lm}))+1\Big)\cdot W\;\;\;  
  B_l&\leq b\cdot W\;\;\; \forall\; l \in \mathcal{N}\, 
    \left\lceil \frac{M}{N}\right\rceil &\leq b \leq  M.
  s_{l}^m &= \frac{p_{l}^m g_{l}^m}{N_0 W+u_{lm}}\;\;\; \forall\; l \in \mathcal{N},\; m \in \mathcal{M}\, 
  c_{l}^m &= W \log_2 (1+s_{l}^m) \;\;\; \forall\; l \in \mathcal{N},\; m \in \mathcal{M}. 
  r_{l}^m &\leq c_{l}^m a_{lm}. 
  r_l &= \sum_{m=1}^M r_{l}^m\;\;\; \forall\; l \in \mathcal{N}.\, 
\begin{aligned}
&{\text{maximize}}\;\;\; \min_{l\in \mathcal{N}}{r_l} \\
&\text{subject to :}\\
& B_l\leq b\cdot W\;\;\;\;&\forall \; l\in \mathcal{N},\\
&r_l^m \leq c_l^m . a_{lm} \;\;\;\;&\forall \; l\in \mathcal{N},\;\forall\; m \in \mathcal{M},\\
&r_l = \sum_{m=1}^M r_{l}^m \;\;\; &\forall\; l \in \mathcal{N},\\
&\sum_{l=1}^N a_{lm} \leq 1,\;\;\; &\forall\; m \in \mathcal{M},\\
&r_l^m \geq 0,\ a_{lm}\in \{0,1\}\;\;\;\;&\forall \; l\in \mathcal{N},\;\forall\; m \in \mathcal{M}.
\end{aligned}

  \mathcal{N} &= \{L_1,L_2,L_3,L_4\},\, \nonumber\\
  \mathcal{M} &= \{1,2,3,4,5,6,7,8,9,10,11,12\}, \nonumber\\
  W&=100\mbox{KHz}, \nonumber\\
  N_0&=kT, \;\; k=\mbox{Boltzman constant}.\nonumber

  \mathcal{N} &= \{L_1,L_2\},\, \nonumber\\
  \mathcal{M} &= \{1,2,\ldots,27,28\}. \nonumber



\begin{figure}[!t]
  \centering
    \includegraphics[width=0.43\textwidth]{gnurblock.pdf}
    \caption{Block diagram for implementation of NC-OFDM with GNU Radio.}
  \label{fig:gnurblock}
\end{figure}



\begin{table}[t!]
\centering
\caption{Parameters used in experiment}
\begin{tabular}{|r|p{0.19\textwidth}|}
  \hline
  FFT Length & 128 \\
  \hline
  Size of data packet & 1500 bytes\\
  \hline
  Center frequency & 1.5GHz \\
  \hline
  Total available bandwidth & 1 MHz\\
  \hline
  Modulation type & BPSK \\
  \hline
\end{tabular}
\label{table:expparam}
\end{table}


One of the most important challenges faced during our experiments were related to synchronization and interference to a link from the sidelobe power of another links which are using the adjacent channels. Different methods of handling the issue of sidelobe power have been proposed such as, usage of guardband or sidelobe suppression \cite{dong2009,ghassemi2010,yuan2010}.  We note that the problem associated with synchronization  can also be addressed by using a filter with sharp cut-offs corresponding to the allocation vector at the transmitter and receiver. However designing these filters which are also reconfigurable is difficult and providing very sharp cut-off is very resource consuming. Instead,  to mitigate the  problem of synchronization, we implement an out-of-band synchronization method in our NC-OFDM implementation using PN-sequence preambles \cite{tufve99sync}.
\begin{figure}
\begin{boxedalgorithmic}[1]
 \renewcommand{\algorithmicrequire}{\textbf{Input:}}
 \renewcommand{\algorithmicensure}{\textbf{Output:}}
 \REQUIRE \textbf{A}, , 
 \ENSURE  
\STATE  = link using channel 
  \STATE  =  
\FOR { to }
  \STATE  = link using channel 
  \IF {( and  NULL))}
  \IF {()}
  \STATE ,
  \STATE ,
  \ELSE
  \STATE ,
  \STATE ,
  \ENDIF
  \ENDIF
  \STATE  = 
  \STATE  = 
  \ENDFOR
 \RETURN  
 \end{boxedalgorithmic} 
 \caption{Algorithm for creating guardband.}
 \label{fig:algogb}
\end{figure}
Fig. \ref{fig:gnurblock} shows the block diagram of our implementation of NC-OFDM using GNU Radio. We modify the synchronization block to separate the  process of synchronization from data path. To address the problem of interference from sidelobe power, we provide a guardband between two adjacent channel allocations. 


The channel allocation is decided by an outside node which behaves as a controller, and it allocates the channels under  different interference environment implementing the proposed  max-min rate optimization algorithm. In our experiments, the distance between USRPs is very small, therefore we approximate the channel between them using a line-of-sight path loss model to fit in our optimization problem. In our experiments, we find that a guardband of 4 subcarriers  was sufficient for successful communication. Once we get the allocation vector, we create these guardbands in such a way that it has minimal effect on the max-min rate obtained after optimization. We use  the algorithm presented in Fig. \ref{fig:algogb} to create the guardband for our experiments.  Fig. \ref{fig:orbresult} shows the changes in data rate for both links due to the  interference from node A averaged over 10 iterations. We find that even when we did not have the perfect knowledge about the channel gain, the trend in the variation of data rate is similar to that observed in our simulation results.


\begin{figure}[!t]
  \centering
    \includegraphics[width=0.37\textwidth]{orb_result.pdf}
    \caption{Data rate obtained in the ORBIT testbed.}
  \label{fig:orbresult}
\end{figure}


\section{Conclusion} 
\label{sec:conclude}
In this paper we considered the problem of spectrum allocation  that maximizes the minimum rate for NC-OFDM-enabled  point-to-point cognitive radio links under the spectrum span constraint. Under the constraint of constant transmit powers and orthogonal spectrum allocation, we formulated a mixed-integer linear program that can be solved efficiently using readily available solvers.  We showed that there exists a clear trade-off between the spectrum span of the spectrum allocation and max-min rate. We found that spectrum span can be kept to small value with very little penalty on the max-min rate under different interference scenarios.The max-min rate obtained from the spectrum allocation that is calculated using our optimization formulation showed improvement under different interference conditions. We also presented an experimental evaluation of the techniques developed in this paper using  USRP enabled ORBIT radio network testbed.  We found similar trends in the variation of  of data rate in our experiments to that observed in our simulation results.

\section*{Acknowledgment}
The authors thank Dr. Gokul Sridharan for several valuable discussions and insights that improved the presentation of the results in this paper.



\bibliographystyle{IEEEtran}
\bibliography{comsnets} \end{document}