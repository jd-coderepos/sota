\documentclass{article} \usepackage{iclr2020_conference,times}



\usepackage{amsmath,amsfonts,bm}

\newcommand{\figleft}{{\em (Left)}}
\newcommand{\figcenter}{{\em (Center)}}
\newcommand{\figright}{{\em (Right)}}
\newcommand{\figtop}{{\em (Top)}}
\newcommand{\figbottom}{{\em (Bottom)}}
\newcommand{\captiona}{{\em (a)}}
\newcommand{\captionb}{{\em (b)}}
\newcommand{\captionc}{{\em (c)}}
\newcommand{\captiond}{{\em (d)}}

\newcommand{\newterm}[1]{{\bf #1}}


\def\figref#1{figure~\ref{#1}}
\def\Figref#1{Figure~\ref{#1}}
\def\twofigref#1#2{figures \ref{#1} and \ref{#2}}
\def\quadfigref#1#2#3#4{figures \ref{#1}, \ref{#2}, \ref{#3} and \ref{#4}}
\def\secref#1{section~\ref{#1}}
\def\Secref#1{Section~\ref{#1}}
\def\twosecrefs#1#2{sections \ref{#1} and \ref{#2}}
\def\secrefs#1#2#3{sections \ref{#1}, \ref{#2} and \ref{#3}}
\def\eqref#1{equation~\ref{#1}}
\def\Eqref#1{Equation~\ref{#1}}
\def\plaineqref#1{\ref{#1}}
\def\chapref#1{chapter~\ref{#1}}
\def\Chapref#1{Chapter~\ref{#1}}
\def\rangechapref#1#2{chapters\ref{#1}--\ref{#2}}
\def\algref#1{algorithm~\ref{#1}}
\def\Algref#1{Algorithm~\ref{#1}}
\def\twoalgref#1#2{algorithms \ref{#1} and \ref{#2}}
\def\Twoalgref#1#2{Algorithms \ref{#1} and \ref{#2}}
\def\partref#1{part~\ref{#1}}
\def\Partref#1{Part~\ref{#1}}
\def\twopartref#1#2{parts \ref{#1} and \ref{#2}}

\def\ceil#1{\lceil #1 \rceil}
\def\floor#1{\lfloor #1 \rfloor}
\def\1{\bm{1}}
\newcommand{\train}{\mathcal{D}}
\newcommand{\valid}{\mathcal{D_{\mathrm{valid}}}}
\newcommand{\test}{\mathcal{D_{\mathrm{test}}}}

\def\eps{{\epsilon}}


\def\reta{{\textnormal{}}}
\def\ra{{\textnormal{a}}}
\def\rb{{\textnormal{b}}}
\def\rc{{\textnormal{c}}}
\def\rd{{\textnormal{d}}}
\def\re{{\textnormal{e}}}
\def\rf{{\textnormal{f}}}
\def\rg{{\textnormal{g}}}
\def\rh{{\textnormal{h}}}
\def\ri{{\textnormal{i}}}
\def\rj{{\textnormal{j}}}
\def\rk{{\textnormal{k}}}
\def\rl{{\textnormal{l}}}
\def\rn{{\textnormal{n}}}
\def\ro{{\textnormal{o}}}
\def\rp{{\textnormal{p}}}
\def\rq{{\textnormal{q}}}
\def\rr{{\textnormal{r}}}
\def\rs{{\textnormal{s}}}
\def\rt{{\textnormal{t}}}
\def\ru{{\textnormal{u}}}
\def\rv{{\textnormal{v}}}
\def\rw{{\textnormal{w}}}
\def\rx{{\textnormal{x}}}
\def\ry{{\textnormal{y}}}
\def\rz{{\textnormal{z}}}

\def\rvepsilon{{\mathbf{\epsilon}}}
\def\rvtheta{{\mathbf{\theta}}}
\def\rva{{\mathbf{a}}}
\def\rvb{{\mathbf{b}}}
\def\rvc{{\mathbf{c}}}
\def\rvd{{\mathbf{d}}}
\def\rve{{\mathbf{e}}}
\def\rvf{{\mathbf{f}}}
\def\rvg{{\mathbf{g}}}
\def\rvh{{\mathbf{h}}}
\def\rvu{{\mathbf{i}}}
\def\rvj{{\mathbf{j}}}
\def\rvk{{\mathbf{k}}}
\def\rvl{{\mathbf{l}}}
\def\rvm{{\mathbf{m}}}
\def\rvn{{\mathbf{n}}}
\def\rvo{{\mathbf{o}}}
\def\rvp{{\mathbf{p}}}
\def\rvq{{\mathbf{q}}}
\def\rvr{{\mathbf{r}}}
\def\rvs{{\mathbf{s}}}
\def\rvt{{\mathbf{t}}}
\def\rvu{{\mathbf{u}}}
\def\rvv{{\mathbf{v}}}
\def\rvw{{\mathbf{w}}}
\def\rvx{{\mathbf{x}}}
\def\rvy{{\mathbf{y}}}
\def\rvz{{\mathbf{z}}}

\def\erva{{\textnormal{a}}}
\def\ervb{{\textnormal{b}}}
\def\ervc{{\textnormal{c}}}
\def\ervd{{\textnormal{d}}}
\def\erve{{\textnormal{e}}}
\def\ervf{{\textnormal{f}}}
\def\ervg{{\textnormal{g}}}
\def\ervh{{\textnormal{h}}}
\def\ervi{{\textnormal{i}}}
\def\ervj{{\textnormal{j}}}
\def\ervk{{\textnormal{k}}}
\def\ervl{{\textnormal{l}}}
\def\ervm{{\textnormal{m}}}
\def\ervn{{\textnormal{n}}}
\def\ervo{{\textnormal{o}}}
\def\ervp{{\textnormal{p}}}
\def\ervq{{\textnormal{q}}}
\def\ervr{{\textnormal{r}}}
\def\ervs{{\textnormal{s}}}
\def\ervt{{\textnormal{t}}}
\def\ervu{{\textnormal{u}}}
\def\ervv{{\textnormal{v}}}
\def\ervw{{\textnormal{w}}}
\def\ervx{{\textnormal{x}}}
\def\ervy{{\textnormal{y}}}
\def\ervz{{\textnormal{z}}}

\def\rmA{{\mathbf{A}}}
\def\rmB{{\mathbf{B}}}
\def\rmC{{\mathbf{C}}}
\def\rmD{{\mathbf{D}}}
\def\rmE{{\mathbf{E}}}
\def\rmF{{\mathbf{F}}}
\def\rmG{{\mathbf{G}}}
\def\rmH{{\mathbf{H}}}
\def\rmI{{\mathbf{I}}}
\def\rmJ{{\mathbf{J}}}
\def\rmK{{\mathbf{K}}}
\def\rmL{{\mathbf{L}}}
\def\rmM{{\mathbf{M}}}
\def\rmN{{\mathbf{N}}}
\def\rmO{{\mathbf{O}}}
\def\rmP{{\mathbf{P}}}
\def\rmQ{{\mathbf{Q}}}
\def\rmR{{\mathbf{R}}}
\def\rmS{{\mathbf{S}}}
\def\rmT{{\mathbf{T}}}
\def\rmU{{\mathbf{U}}}
\def\rmV{{\mathbf{V}}}
\def\rmW{{\mathbf{W}}}
\def\rmX{{\mathbf{X}}}
\def\rmY{{\mathbf{Y}}}
\def\rmZ{{\mathbf{Z}}}

\def\ermA{{\textnormal{A}}}
\def\ermB{{\textnormal{B}}}
\def\ermC{{\textnormal{C}}}
\def\ermD{{\textnormal{D}}}
\def\ermE{{\textnormal{E}}}
\def\ermF{{\textnormal{F}}}
\def\ermG{{\textnormal{G}}}
\def\ermH{{\textnormal{H}}}
\def\ermI{{\textnormal{I}}}
\def\ermJ{{\textnormal{J}}}
\def\ermK{{\textnormal{K}}}
\def\ermL{{\textnormal{L}}}
\def\ermM{{\textnormal{M}}}
\def\ermN{{\textnormal{N}}}
\def\ermO{{\textnormal{O}}}
\def\ermP{{\textnormal{P}}}
\def\ermQ{{\textnormal{Q}}}
\def\ermR{{\textnormal{R}}}
\def\ermS{{\textnormal{S}}}
\def\ermT{{\textnormal{T}}}
\def\ermU{{\textnormal{U}}}
\def\ermV{{\textnormal{V}}}
\def\ermW{{\textnormal{W}}}
\def\ermX{{\textnormal{X}}}
\def\ermY{{\textnormal{Y}}}
\def\ermZ{{\textnormal{Z}}}

\def\vzero{{\bm{0}}}
\def\vone{{\bm{1}}}
\def\vmu{{\bm{\mu}}}
\def\vtheta{{\bm{\theta}}}
\def\va{{\bm{a}}}
\def\vb{{\bm{b}}}
\def\vc{{\bm{c}}}
\def\vd{{\bm{d}}}
\def\ve{{\bm{e}}}
\def\vf{{\bm{f}}}
\def\vg{{\bm{g}}}
\def\vh{{\bm{h}}}
\def\vi{{\bm{i}}}
\def\vj{{\bm{j}}}
\def\vk{{\bm{k}}}
\def\vl{{\bm{l}}}
\def\vm{{\bm{m}}}
\def\vn{{\bm{n}}}
\def\vo{{\bm{o}}}
\def\vp{{\bm{p}}}
\def\vq{{\bm{q}}}
\def\vr{{\bm{r}}}
\def\vs{{\bm{s}}}
\def\vt{{\bm{t}}}
\def\vu{{\bm{u}}}
\def\vv{{\bm{v}}}
\def\vw{{\bm{w}}}
\def\vx{{\bm{x}}}
\def\vy{{\bm{y}}}
\def\vz{{\bm{z}}}

\def\evalpha{{\alpha}}
\def\evbeta{{\beta}}
\def\evepsilon{{\epsilon}}
\def\evlambda{{\lambda}}
\def\evomega{{\omega}}
\def\evmu{{\mu}}
\def\evpsi{{\psi}}
\def\evsigma{{\sigma}}
\def\evtheta{{\theta}}
\def\eva{{a}}
\def\evb{{b}}
\def\evc{{c}}
\def\evd{{d}}
\def\eve{{e}}
\def\evf{{f}}
\def\evg{{g}}
\def\evh{{h}}
\def\evi{{i}}
\def\evj{{j}}
\def\evk{{k}}
\def\evl{{l}}
\def\evm{{m}}
\def\evn{{n}}
\def\evo{{o}}
\def\evp{{p}}
\def\evq{{q}}
\def\evr{{r}}
\def\evs{{s}}
\def\evt{{t}}
\def\evu{{u}}
\def\evv{{v}}
\def\evw{{w}}
\def\evx{{x}}
\def\evy{{y}}
\def\evz{{z}}

\def\mA{{\bm{A}}}
\def\mB{{\bm{B}}}
\def\mC{{\bm{C}}}
\def\mD{{\bm{D}}}
\def\mE{{\bm{E}}}
\def\mF{{\bm{F}}}
\def\mG{{\bm{G}}}
\def\mH{{\bm{H}}}
\def\mI{{\bm{I}}}
\def\mJ{{\bm{J}}}
\def\mK{{\bm{K}}}
\def\mL{{\bm{L}}}
\def\mM{{\bm{M}}}
\def\mN{{\bm{N}}}
\def\mO{{\bm{O}}}
\def\mP{{\bm{P}}}
\def\mQ{{\bm{Q}}}
\def\mR{{\bm{R}}}
\def\mS{{\bm{S}}}
\def\mT{{\bm{T}}}
\def\mU{{\bm{U}}}
\def\mV{{\bm{V}}}
\def\mW{{\bm{W}}}
\def\mX{{\bm{X}}}
\def\mY{{\bm{Y}}}
\def\mZ{{\bm{Z}}}
\def\mBeta{{\bm{\beta}}}
\def\mPhi{{\bm{\Phi}}}
\def\mLambda{{\bm{\Lambda}}}
\def\mSigma{{\bm{\Sigma}}}

\DeclareMathAlphabet{\mathsfit}{\encodingdefault}{\sfdefault}{m}{sl}
\SetMathAlphabet{\mathsfit}{bold}{\encodingdefault}{\sfdefault}{bx}{n}
\newcommand{\tens}[1]{\bm{\mathsfit{#1}}}
\def\tA{{\tens{A}}}
\def\tB{{\tens{B}}}
\def\tC{{\tens{C}}}
\def\tD{{\tens{D}}}
\def\tE{{\tens{E}}}
\def\tF{{\tens{F}}}
\def\tG{{\tens{G}}}
\def\tH{{\tens{H}}}
\def\tI{{\tens{I}}}
\def\tJ{{\tens{J}}}
\def\tK{{\tens{K}}}
\def\tL{{\tens{L}}}
\def\tM{{\tens{M}}}
\def\tN{{\tens{N}}}
\def\tO{{\tens{O}}}
\def\tP{{\tens{P}}}
\def\tQ{{\tens{Q}}}
\def\tR{{\tens{R}}}
\def\tS{{\tens{S}}}
\def\tT{{\tens{T}}}
\def\tU{{\tens{U}}}
\def\tV{{\tens{V}}}
\def\tW{{\tens{W}}}
\def\tX{{\tens{X}}}
\def\tY{{\tens{Y}}}
\def\tZ{{\tens{Z}}}


\def\gA{{\mathcal{A}}}
\def\gB{{\mathcal{B}}}
\def\gC{{\mathcal{C}}}
\def\gD{{\mathcal{D}}}
\def\gE{{\mathcal{E}}}
\def\gF{{\mathcal{F}}}
\def\gG{{\mathcal{G}}}
\def\gH{{\mathcal{H}}}
\def\gI{{\mathcal{I}}}
\def\gJ{{\mathcal{J}}}
\def\gK{{\mathcal{K}}}
\def\gL{{\mathcal{L}}}
\def\gM{{\mathcal{M}}}
\def\gN{{\mathcal{N}}}
\def\gO{{\mathcal{O}}}
\def\gP{{\mathcal{P}}}
\def\gQ{{\mathcal{Q}}}
\def\gR{{\mathcal{R}}}
\def\gS{{\mathcal{S}}}
\def\gT{{\mathcal{T}}}
\def\gU{{\mathcal{U}}}
\def\gV{{\mathcal{V}}}
\def\gW{{\mathcal{W}}}
\def\gX{{\mathcal{X}}}
\def\gY{{\mathcal{Y}}}
\def\gZ{{\mathcal{Z}}}

\def\sA{{\mathbb{A}}}
\def\sB{{\mathbb{B}}}
\def\sC{{\mathbb{C}}}
\def\sD{{\mathbb{D}}}
\def\sF{{\mathbb{F}}}
\def\sG{{\mathbb{G}}}
\def\sH{{\mathbb{H}}}
\def\sI{{\mathbb{I}}}
\def\sJ{{\mathbb{J}}}
\def\sK{{\mathbb{K}}}
\def\sL{{\mathbb{L}}}
\def\sM{{\mathbb{M}}}
\def\sN{{\mathbb{N}}}
\def\sO{{\mathbb{O}}}
\def\sP{{\mathbb{P}}}
\def\sQ{{\mathbb{Q}}}
\def\sR{{\mathbb{R}}}
\def\sS{{\mathbb{S}}}
\def\sT{{\mathbb{T}}}
\def\sU{{\mathbb{U}}}
\def\sV{{\mathbb{V}}}
\def\sW{{\mathbb{W}}}
\def\sX{{\mathbb{X}}}
\def\sY{{\mathbb{Y}}}
\def\sZ{{\mathbb{Z}}}

\def\emLambda{{\Lambda}}
\def\emA{{A}}
\def\emB{{B}}
\def\emC{{C}}
\def\emD{{D}}
\def\emE{{E}}
\def\emF{{F}}
\def\emG{{G}}
\def\emH{{H}}
\def\emI{{I}}
\def\emJ{{J}}
\def\emK{{K}}
\def\emL{{L}}
\def\emM{{M}}
\def\emN{{N}}
\def\emO{{O}}
\def\emP{{P}}
\def\emQ{{Q}}
\def\emR{{R}}
\def\emS{{S}}
\def\emT{{T}}
\def\emU{{U}}
\def\emV{{V}}
\def\emW{{W}}
\def\emX{{X}}
\def\emY{{Y}}
\def\emZ{{Z}}
\def\emSigma{{\Sigma}}

\newcommand{\etens}[1]{\mathsfit{#1}}
\def\etLambda{{\etens{\Lambda}}}
\def\etA{{\etens{A}}}
\def\etB{{\etens{B}}}
\def\etC{{\etens{C}}}
\def\etD{{\etens{D}}}
\def\etE{{\etens{E}}}
\def\etF{{\etens{F}}}
\def\etG{{\etens{G}}}
\def\etH{{\etens{H}}}
\def\etI{{\etens{I}}}
\def\etJ{{\etens{J}}}
\def\etK{{\etens{K}}}
\def\etL{{\etens{L}}}
\def\etM{{\etens{M}}}
\def\etN{{\etens{N}}}
\def\etO{{\etens{O}}}
\def\etP{{\etens{P}}}
\def\etQ{{\etens{Q}}}
\def\etR{{\etens{R}}}
\def\etS{{\etens{S}}}
\def\etT{{\etens{T}}}
\def\etU{{\etens{U}}}
\def\etV{{\etens{V}}}
\def\etW{{\etens{W}}}
\def\etX{{\etens{X}}}
\def\etY{{\etens{Y}}}
\def\etZ{{\etens{Z}}}

\newcommand{\pdata}{p_{\rm{data}}}
\newcommand{\ptrain}{\hat{p}_{\rm{data}}}
\newcommand{\Ptrain}{\hat{P}_{\rm{data}}}
\newcommand{\pmodel}{p_{\rm{model}}}
\newcommand{\Pmodel}{P_{\rm{model}}}
\newcommand{\ptildemodel}{\tilde{p}_{\rm{model}}}
\newcommand{\pencode}{p_{\rm{encoder}}}
\newcommand{\pdecode}{p_{\rm{decoder}}}
\newcommand{\precons}{p_{\rm{reconstruct}}}

\newcommand{\laplace}{\mathrm{Laplace}} 

\newcommand{\E}{\mathbb{E}}
\newcommand{\Ls}{\mathcal{L}}
\newcommand{\R}{\mathbb{R}}
\newcommand{\emp}{\tilde{p}}
\newcommand{\lr}{\alpha}
\newcommand{\reg}{\lambda}
\newcommand{\rect}{\mathrm{rectifier}}
\newcommand{\softmax}{\mathrm{softmax}}
\newcommand{\sigmoid}{\sigma}
\newcommand{\softplus}{\zeta}
\newcommand{\KL}{D_{\mathrm{KL}}}
\newcommand{\Var}{\mathrm{Var}}
\newcommand{\standarderror}{\mathrm{SE}}
\newcommand{\Cov}{\mathrm{Cov}}
\newcommand{\normlzero}{L^0}
\newcommand{\normlone}{L^1}
\newcommand{\normltwo}{L^2}
\newcommand{\normlp}{L^p}
\newcommand{\normmax}{L^\infty}

\newcommand{\parents}{Pa} 

\DeclareMathOperator*{\argmax}{arg\,max}
\DeclareMathOperator*{\argmin}{arg\,min}

\DeclareMathOperator{\sign}{sign}
\DeclareMathOperator{\Tr}{Tr}
\let\ab\allowbreak
 
\usepackage{hyperref}
\usepackage{url}
\usepackage[pdftex]{graphicx}
\usepackage{wrapfig}
\usepackage{diagbox}
\newcommand{\todo}[1]{\textcolor{red}{#1}}

\renewcommand{\paragraph}[1]{\textbf{#1}}
\def\pair{\mathcal{P}}
\def\extpair{\widetilde{\mathcal{P}}}


\title{Reformer: The Efficient Transformer}

\author{Nikita Kitaev\thanks{Equal Contribution}  \\
U.C. Berkeley \& Google Research\\
\texttt{kitaev@cs.berkeley.edu} \\
\And 
\L{}ukasz Kaiser \\
Google Research\\
\texttt{\{lukaszkaiser,levskaya\rlap{\}@google.com}} \\
\And 
Anselm Levskaya \\
Google Research\\
}

\iclrfinalcopy 

\begin{document}


\maketitle

\begin{abstract}
Large Transformer models routinely achieve state-of-the-art results on
a number of tasks but training these models can be prohibitively costly,
especially on long sequences. We introduce two techniques to improve
the efficiency of Transformers. For one, we replace dot-product attention
by one that uses locality-sensitive hashing, changing its complexity
from O() to O(), where  is the length of the sequence.
Furthermore, we use reversible residual layers instead of the standard
residuals, which allows storing activations only once in the training
process instead of  times, where  is the number of layers.
The resulting model, the Reformer, performs on par with Transformer models
while being much more memory-efficient and much faster on long sequences.
\end{abstract}

\section{Introduction}

The Transformer architecture \citep{transformer} is widely used in natural language processing
and yields state-of-the-art results on a number of tasks. To obtain these results,
researchers have resorted to training ever larger Transformer models. The number of parameters exceeds
0.5B per layer in the largest configuration reported in \citep{meshtf} while the number
of layers goes up to 64 in \citep{chartransformer}. Transformer models are also used on
increasingly long sequences. Up to 11 thousand tokens of text in a single example were
processed in \citep{wikipedia} and when processing other modalities, like music \citep{huang2018music}
and images \citep{parmar2018imagetransformer}, even longer sequences are commonplace.
These large-scale long-sequence models yield great results but strain resources to
the point where some argue that this trend is breaking NLP
research\footnote{\url{https://hackingsemantics.xyz/2019/leaderboards/}}.
Many large Transformer models can only realistically be trained in large
industrial research laboratories and such models trained with model parallelism
cannot even be fine-tuned on a single GPU as their memory requirements demand a 
multi-accelerator hardware setup even for a single training step.

Do large Transformer models fundamentally require such huge resources or are
they simply inefficient? Consider the following calculation: the 0.5B parameters used in the largest reported Transformer layer account for 2GB of memory. Activations for 64K tokens with embedding size 1024 and batch size 8 account for B floats, requiring another 2GB of memory. If our memory use was only per-layer, then we should fairly easily fit a large Transformer even on sequences of length 64K on a single accelerator. Further, the whole corpus used to train BERT only requires 17GB to store. Why is it then that we cannot even fine-tune these models on single machines?

The above estimate includes only per-layer memory and input activations cost and does 
not take into account the following major sources of memory use in the Transformer.
\begin{itemize}
\item Memory in a model with  layers is -times larger than in a single-layer model due to
  the fact that activations need to be stored for back-propagation.
\item Since the depth  of intermediate feed-forward layers is often much larger than
  the depth  of attention activations, it accounts for a large fraction of memory use.
\item Attention on sequences of length  is O() in both computational and memory complexity,
  so even for a single sequence of K tokens can exhaust accelerator memory.
\end{itemize}

We introduce the Reformer model which solves these problems using the following techniques:
\begin{itemize}
\item Reversible layers, first introduced in \cite{gomez2017reversible}, enable storing only a single copy of activations in the whole model, so the  factor disappears.
\item Splitting activations inside feed-forward layers and processing them in chunks removes the  factor and saves memory inside feed-forward layers.
\item Approximate attention computation based on locality-sensitive hashing replaces the O() factor in attention layers with O() and so allows operating on long sequences.
\end{itemize}

We study these techniques and show that they have negligible impact on the training process compared to the standard Transformer. 
Splitting activations in fact only affects the implementation; it is numerically identical to the layers used in the Transformer. 
Applying reversible residuals instead of the standard ones does change the model but has a negligible effect on training in all 
configurations we experimented with.  Finally, locality-sensitive hashing in attention is a more major change that can influence
the training dynamics, depending on the number of concurrent hashes used.  We study this
parameter and find a value which is both efficient to use and yields results very close to full attention.

We experiment on a synthetic task, a text task (enwik8) with sequences of length 64K and an image generation task (imagenet-64 generation)
with sequences of length 12K. In both cases we show that Reformer matches the results obtained with full Transformer
but runs much faster, especially on the text task, and with orders of magnitude better memory efficiency.

\section{Locality-sensitive Hashing Attention}

\paragraph{Dot-product attention.}
The standard attention used in the Transformer is the scaled dot-product attention \citep{transformer}. 
The input consists of queries and keys of dimension , and values of dimension .  
The dot products of the query with all keys are computed, scaled by , and a softmax function is 
applied to obtain the weights on the values. In practice, the attention function on a set of queries is computed 
simultaneously, packed together into a matrix .  Assuming the keys and values are also packed together into 
matrices  and , the matrix of outputs is defined as:



\paragraph{Multi-head attention.}
In the Transformer, instead of performing a single attention function with -dimensional keys, 
values and queries, one linearly projects the queries, keys and values  times with different, 
learned linear projections to ,  and  dimensions, respectively.  Attention is applied to 
each of these projected versions of queries, keys and values in parallel, yielding -dimensional 
output values. These are concatenated and once again projected, resulting in the final values. This 
mechanism is known as multi-head attention.

\paragraph{Memory-efficient attention.}
To calculate the memory use of the attention mechanism, let us focus on
the attention computation from Equation~\ref{eq:attn}.
Let us assume that Q, K and V all have the shape
. The main issue is
the term , which has the shape .
In the experimental section we train a model on sequences of length
 -- in this case, even at batch-size of 1, this is a 
matrix, which in 32-bit floats would take 16GB of memory.
This is impractical and has hindered the use of the Transformer for long
sequences. But it is important to note that the  matrix does not
need to be fully materialized in memory. The attention can indeed be computed
for each query  separately, only calculating  
once in memory, and then re-computing it on the backward pass when needed for gradients.
This way of computing attention may be less efficient but it only uses
memory proportional to . We use this memory-efficient implementation
of attention to run the full-attention baselines presented in the experimental
section.

\paragraph{Where do Q, K, V come from?}
The multi-head attention described above operates on keys, queries and
values, but usually we are only given a single tensor of activations A
of the shape  -- e.g., coming from
embedding the tokens in a sentence into vectors. To build Q, K and V
from A, the Transformer uses 3 different linear layers projecting A
into Q, K and V with different parameters. For models with LSH attention,
we want queries and keys (Q and K) to be identical. This is easily achieved
by using the same linear layer to go from A to Q and K, and a separate one
for V. We call a model that behaves like this a shared-QK Transformer.
It turns out that sharing QK does not affect the performance of Transformer,
even if we additionally normalize the length of the keys K,
as we show in the experimental Section~\ref{sec:exp_qk}.

\paragraph{Hashing attention.}
For the LSH attention, we start with two tensors, Q=K and V
of the shape .
We keep the multi-head mechanism intact and focus on the attention
computation from Equation~\ref{eq:attn}. As already mentioned,
the main issue is the term , which has the shape
. But note that we are actually only
interested in . Since softmax is dominated
by the largest elements, for each query  we only need to focus
on the keys in K that are closest to . For example, if K is of
length 64K, for each  we could only consider a small subset of,
say, the  or  closest keys. That is much more efficient, but how
can we find the nearest neighbors among the keys?

\begin{figure}
    \centering
    \includegraphics[width=0.8\textwidth]{figures/angular-LSH.png}
    \caption{An angular locality sensitive hash uses random rotations of spherically projected 
    points to establish buckets by an argmax over signed axes projections.  In this highly 
    simplified 2D depiction, two points  and  are unlikely to share the same hash buckets 
    (above) for the three different angular hashes unless their spherical projections are close 
    to one another (below).}
    \label{fig:lsh}
\end{figure}

\paragraph{Locality sensitive hashing.}
The problem of finding nearest neighbors quickly in high-dimensional spaces
can be solved by locality-sensitive hashing (LSH). A hashing scheme that assigns each
vector  to a hash  is called locality-sensitive if nearby
vectors get the same hash with high probability and distant ones do not.
In our case, we actually only require that nearby vectors get the same
hash with high probability and that hash-buckets are of similar size with
high probability.

We achieve this by employing random projections as follows (see Figure \ref{fig:lsh}).
To get  hashes, we first fix a random matrix  of size .
We then define  where  denotes the concatenation of two vectors.
This method is a known LSH scheme \citep{andoni2015angularLSH} and is easy to implement
and apply to batches of vectors.

\begin{figure}
        \centering
        \includegraphics[width=0.9\textwidth]{figures/combined-attention-figs.png}
        \caption{Simplified depiction of LSH Attention showing the hash-bucketing, sorting, 
                  and chunking steps and the resulting causal attentions. (a-d) Attention matrices for these varieties of attention.}
        \label{fig:attnpattern}
\end{figure}

\paragraph{LSH attention.}
Knowing our LSH scheme and the general idea of hashing attention, we will now formalize the LSH attention we use in this paper. We first rewrite the equation for normal attention, (\ref{eq:attn}), for a single query position  at a time:


We introduce the notation  to represent the set that the query at position  attends to, and  to denote the partition function (i.e.\ the normalizing term in the softmax). For clarity, we also omit scaling by .

For batching purposes we typically perform attention over a larger set  while masking out elements not in :


Now we turn to LSH attention, which we can think of in terms of restricting the set  of target items a query position  can attend to, by only allowing attention within a single hash bucket.


Figure~\ref{fig:attnpattern}(a-b) shows a schematic comparison of full-attention with a hashed variant. Part (a) depicts that the attention matrix for full attention is typically sparse, but the computation does not take advantage of this sparsity. In (b), the queries and keys have been sorted according to their hash bucket. Since similar items fall in the same bucket with high probability, the full attention pattern can be approximated by only allowing attention within each bucket.

Hash buckets in this formulation tend to be uneven in size, which makes it difficult to batch across buckets. Moreover, the number of queries and the number of keys within a bucket may be unequal -- in fact, it is possible for a bucket to contain many queries but no keys. To alleviate these issues, we first ensure that  by setting . Next, we sort the queries by bucket number and, within each bucket, by sequence position; this defines a permutation where  after sorting. In the sorted attention matrix, pairs from the same bucket will cluster near the diagonal (as depicted in Figure~\ref{fig:attnpattern}c). We can follow a batching approach where chunks of  consecutive queries (after sorting) attend to each other, and one chunk back (Figure~\ref{fig:attnpattern}d). Following our earlier notation, this corresponds to setting:

If , then . In practice we set  (where  is the sequence length). The average bucket size is , and we assume that the probability of a bucket growing to twice that size is sufficiently low.
The overall process of LSH attention is summarized in Figure~\ref{fig:attnpattern}.

\paragraph{Multi-round LSH attention.}
With hashing, there is always a small probability that similar items nevertheless fall in different buckets. This probability can be reduced by doing multiple rounds of hashing with  distinct hash functions , such that:

The multi-round case essentially involves performing LSH attention  times in parallel; the details of the procedure are described in in Appendix~\ref{sec:multi-round-detail}.

\paragraph{Causal masking for shared-QK attention.}
In a Transformer decoder, masking (denoted by  in Equation~\ref{eq:attn-masked}) is used to prevent positions from attending into the future. To implement masking in LSH attention, we associate every query/key vector with a position index, re-order the position indices using the same permutations used to sort the query/key vectors, and then use a comparison operation to compute the mask.

While attention to the future is not allowed, typical implementations of the Transformer \emph{do} allow a position to attend to \emph{itself}. Such behavior is undesirable in a shared-QK formulation because the dot-product of a query vector with itself will almost always be greater than the dot product of a query vector with a vector at another position. We therefore modify the masking to forbid a token from attending to itself, except in situations where a token has no other valid attention targets (e.g.\ the first token in a sequence).

\begin{table}
\caption{Memory and time complexity of attention variants.
  We write  for length,  for batch size,  for the number of heads,
   for the number of LSH chunks,  for the number of hash repetitions.}
\label{tab:complexity}
\begin{center}
\begin{tabular}{lcc}
Attention Type & Memory Complexity & Time Complexity  \\
\hline
Scaled Dot-Product &  &  \\
Memory-Efficient &  &  \\
LSH Attention &  &  \\
\end{tabular}
\end{center}
\end{table}


\subsection{Analysis on a synthetic task}

To verify the performance of LSH attention and study its behavior,
we start with the following synthetic task: duplicate a sequence
of symbols. In this task, each training and testing example has
the form  where  is a sequence of
symbols ranging from  to  (we use  in our experiments).
An example with the word  of length  is given below.

\begin{center}
\vspace{0.5em}
\begin{tabular}{|c|c|c|c|c|c|c|c|c|c|}
\hline
{\bf Example:} & 0 & 19 & 113 & 72 & 0  & 19  & 113  & 72  \\ \hline
\end{tabular}
\vspace{0.5em}
\end{center}

To study LSH attention, we train a language model on examples of
the above form where each  is of length  (so the whole input
 is of length ). As this is a language modeling task,
we always predict the next symbol given all the previous ones,
but we mask the loss and accuracy to only consider positions in the
second half of the input, i.e., those that can actually be predicted.

The above task can be solved perfectly (to accuracy 100\% and loss 0) by
a 1-layer Transformer model. Note though, that it requires non-local
attention lookups, so it cannot be solved by any model relying on sparse
attention with a limited span.
To make it easy and fast to train but similar
to models used in NLP, we use a 1-layer Transformer with
 and  heads. We train it for 150K steps
in  different settings: with full attention, LSH attention with
,  and .


\begin{table}
\caption{Accuracies on the duplication task of a 1-layer Transformer model
  with full attention and with locality-sensitive hashing attention using
  different number of parallel hashes.}
\label{tab:dupres}
\begin{center}
\begin{tabular}{l|c|c|c|c|c}
\diagbox{Train}{Eval} & Full Attention & LSH- & LSH- & LSH- & LSH- \\
\hline
Full Attention & 100\% & 94.8\% & 92.5\% & 76.9\% & 52.5\% \\
LSH-        & 0.8\% & 100\%  & 99.9\% & 99.4\% & 91.9\% \\
LSH-        & 0.8\% & 100\%  & 99.9\% & 98.1\% & 86.8\% \\
LSH-        & 0.8\% & 99.9\% & 99.6\% & 94.8\% & 77.9\% \\ \end{tabular}
\end{center}
\end{table}

From the results summarized in Table~\ref{tab:dupres} we see that
a model trained with full attention can be immediately used with LSH
attention, but at some loss of accuracy. When trained from scratch
with LSH attention, the model trained with 4 hashes achieves almost
perfect accuracy as well. Interestingly, the accuracy becomes perfect
when evaluated with 8 hashes. It goes down when evaluated with 2 or 1
hashes. Models trained with less hashes show worse results but even
the model trained with just 1 hash performs almost perfectly when
evaluated with 8 hashes.


\section{Reversible Transformer} \label{sec:reversible}

As the above section shows, the complexity of attention can be reduced
from square in length to linear, provided an approximation is acceptable.
But it is clear from Table~\ref{tab:complexity} that each field starts
with a  term: the , 
or alternatively  cost
cannot be avoided. Indeed, the activations before each layer are already
of the size , so the memory use of the whole model with 
layers is at least . Even worse: inside the feed-forward layers of 
Transformer this goes up to . In a big Transformer
it is usual to set  and  so with  this again
would use an impractical  of memory

In this section, we show how to reduce this cost by first dealing with the
 part of the term using reversible layers and then showing how chunking
can allow us to handle the  problem.
The effects of each of these approaches on memory and time complexity are summarized in Table~\ref{tab:tcomplexity}.


\paragraph{RevNets.}
Reversible residual networks were introduced by \citet{gomez2017reversible} where it was shown
that they can replace ResNets for image classification.
The main idea is to allow the activations at any given layer to be recovered from the activations at the following layer, using only the model parameters. Rather than having to checkpoint intermediate values for use in the backward pass, layers can be reversed one-by-one as back-propagation proceeds from the output of the network to its input. Whereas a normal residual layer performs a function  that operates on a single input and produces a single output and has the form , a reversible layer works on pairs of inputs/outputs: , and follows the equations:


A layer can be reversed by subtracting (rather than adding) the residuals:



\paragraph{Reversible Transformer.}
We apply the RevNet idea to the Transformer by combining the attention and
feed-forward layers inside the revnet block. In the notation above,
F becomes an attention layer while G becomes the feed-forward layer. Note that Layer Normalization \citep{layernorm2016} is moved inside the residual blocks.


The reversible Transformer does not need to store activations in each layer
and so gets rid of the  term.  In Section~\ref{sec:exp_rev} we show that it performs the same as the normal Transformer when using  the same number of parameters; we achieve this by having both  and  have size .

\paragraph{Chunking.}
While reversibility covers the  term, the thicker layers can still use
a lot of memory. The feed-forward layer in particular can use intermediate vectors 
of dimensionality  or higher. However, computations in feed-forward layers 
are completely independent across positions in a sequence, so the computation can be split 
into  chunks:


This layer is typically batched by performing operations for all positions in parallel, but operating on one chunk at a time can reduce memory. The reverse computation in (\ref{eq:reverse-ff}) and the backward pass are also chunked.
In addition to the feed-forward layers, for models with large vocabulary (more than  word types) we also chunk the log-probabilities at the output and calculate the loss for sections of the sequence at a time.

\paragraph{Chunking, large batches and parameter reuse.}
With chunking and reversible layers the memory we use for activations
in the whole network is independent of the number of layers. The same is
not true for parameters though as their number grows with the number of layers. This problem is remedied though because we can swap layer
parameters to and from CPU memory when this layer is not computing.
In a standard Transformer this would be inefficient because memory
transfer to CPU is slow. The batch size multiplied by length in Reformer
is much larger though and therefore the amount of compute done with the parameters amortizes the cost of their transfer.

\begin{table}
\caption{Memory and time complexity of Transformer variants.
  We write  and  for model depth and assume ;  stands for batch size,  for length,  for the number of layers.
  We assume  so  and we write .}
\label{tab:tcomplexity}
\begin{center}
\begin{tabular}{lcc}
Model Type & Memory Complexity & Time Complexity  \\
\hline
Transformer &  &  \\
Reversible Transformer &  &  \\
Chunked Reversible Transformer &  &  \\
LSH Transformer &  &  \\
Reformer &  &  \\

\end{tabular}
\end{center}
\end{table}

\section{Related Work} \label{sec:relwork}

The Transformer model introduced in \citep{transformer} has been used
widely in natural language tasks and further extended to model diverse data such as music scores \citep{huang2018music}, 
and images \citep{parmar2018imagetransformer, ramachandran2019attentionimage}.  Most notably, this model class 
has been applied successfully in the self-supervised training of extremely large language models
\citep{devlin2018BERT, radford2019GPT2}.

Given the enormous computational requirements of state of the art sequence models, 
there has been increasing interest in finding methods to reduce the memory footprint and computational 
requirements of Transformer models.  In addition to standard methods such as precision reduction and gradient checkpointing \citep{sohoni2019lowmemory}, more efficient versions of the Transformer model's self-attention mechanism \citep{sukhbaatar2019adaptiveattn, sukhbaatar2019persistentmemory} have also recently been explored.

In particular, leveraging sparsity in the attention layers has proved fruitful. OpenAI introduced
the sparse Transformer \citep{child2019sparsetransformer} which exploits a factorized sparse 
representation of attention.  Using product-key attention to increase the key space has also been used to reduce memory requirements in the feed-forward layers with no loss in performance 
\citep{lample2019productkeys}.

Locality-sensitive hashing (LSH) has, to our knowledge, not been directly applied to 
Transformer attention layers before. But previous work using external memory with neural networks has
dealt with memories of large sizes.  The original implementation of memory networks~\citep{mem_nets} and later work on scaling it \citep{large_mem_nets, hier_mem_nets} used memory with size in the millions.  The cost of doing so is that the memory must be fixed prior to training. Moreover, since during the beginning of training the model is unlikely to query the memory correctly, strong supervision is used to encourage the model to query memory locations that are useful. These hints are either given as additional supervising information by the  task or determined heuristically as in \citet{goldilocks}.
The requirement that the memory be fixed before has been removed in
\citet{santoro16} at the cost of memory size and later alleviated by \citet{jack_rae}. The last paper considered memory lookups with approximate nearest neighbors including both LSH and random kd-trees, but only for lookups in external memory.



\section{Experiments} \label{sec:exp}

In this section we present experimental results demonstrating the techniques
described above. We analyze the techniques one-by-one to make clear which
combinations have impact on performance. We start by showing that reversible
layers and shared query-key spaces do not impact performance, then proceed 
to analyze hashing attention and finally the full Reformer model.

We ran our experiments on the imagenet64 and enwik8-64K tasks, where the latter is a variant of enwik8 that is chunked into subsequences of  tokens. We use 3-layer models for our ablations so as to make it tractable to compare with the regular Transformer, which has high memory usage and performs full  attention. All experiments have , , , and a total batch size of 8 sequences. We used the Adafactor optimizer \citep{adafactor} for training these models. We also evaluate on the WMT 2014 English-to-German translation task, following the hyperparameters of \citet{transformer}. Training for all experiments was parallelized across 8 devices (8 GPUs or 8 TPU v3 cores). Code for training our models is made publicly available.\footnote{\url{https://github.com/google/trax/tree/master/trax/models/reformer}}

\paragraph{Effect of sharing QK.} \label{sec:exp_qk}
We first consider the effect of shared-QK attention on a regular Transformer model. Shared-QK attention sets  and prevents tokens from attending to themselves (except when no other context is available). In the left part of Figure~\ref{fig:share-qk-rev}, we plot perplexity curves for both regular and shared-QK attention. A shared query-key space does not perform worse than regular attention; in fact, for enwik8 it appears to train slightly faster. In other words, we are not sacrificing accuracy by switching to shared-QK attention. 

\begin{figure}[t]
    \centering
    \includegraphics[width=0.9\textwidth]{figures/joint_qk_rev_plot.png}
    \caption{Effect of shared query-key space (left) and reversibility (right) on 
    performance on enwik8 and imagenet64 training. The curves show bits per dim on held-out data.}
    \label{fig:share-qk-rev}
\end{figure}

\begin{table}
\caption{BLEU scores on newstest2014 for WMT English-German (En–De). We additionally report detokenized BLEU scores as computed by sacreBLEU~\citep{sacrebleu}.}
\label{tab:wmt}
\begin{center}
\begin{tabular}{lccc}
&& \multicolumn{2}{c}{\textit{sacreBLEU}} \\
Model & BLEU & \textit{Uncased}\footnotemark[3] & \textit{Cased}\footnotemark[4] \\
\hline
\citet{transformer}, base model & 27.3 \\
\citet{transformer}, big & 28.4 \\
\citet{ott2018scaling}, big & 29.3 \\
\hline
Reversible Transformer (base, 100K steps) & 27.6 & \textit{27.4} & \textit{26.9} \\
Reversible Transformer (base, 500K steps, no weight sharing) & 28.0 & \textit{27.9} & \textit{27.4} \\
Reversible Transformer (big, 300K steps, no weight sharing) & 29.1 & \textit{28.9} & \textit{28.4}
\end{tabular}
\end{center}
\end{table}

\footnotetext[3]{\scriptsize \texttt{BLEU+case.lc+lang.en-de+numrefs.1+smooth.exp+test.wmt14/full+tok.intl+version.1.4.3}} 
\footnotetext[4]{\scriptsize \texttt{BLEU+case.mixed+lang.en-de+numrefs.1+smooth.exp+test.wmt14/full+tok.intl+version.1.4.3}} 

\paragraph{Effect of reversible layers.} \label{sec:exp_rev}
In the two plots on the right in Figure~\ref{fig:share-qk-rev}, we compare a regular Transformer per \citet{transformer} with the reversible one describe in Section~\ref{sec:reversible}. The two models have identical parameter counts, and the learning curves likewise appear to be nearly the same. These results show that the memory savings in the reversible Transformer do not come at the expense of accuracy.

\paragraph{Reversible layers in machine translation.}
We also evaluate reversible layers in the context of an encoder-decoder Transformer model for machine translation from English to German. We start by making both the encoder and the decoder fully reversible in the Transformer-base architecture, and see that the resulting model performs comparably to \citet{transformer} when trained for 100K steps. We also evaluate training for a greater number of steps and with a larger model. Reformer models are very memory-efficient, so for the latter two experiments we do not need to save memory by sharing embedding and output projection weight matrices throughout the model. Results are shown in Table~\ref{tab:wmt}. We do not apply LSH attention in this setting because examples are single sentences, and sentences tend to be relatively short. Our typical LSH attention configuration uses chunks of 128 tokens after hashing and sorting, whereas the examples in the WMT14 test set are all shorter than 128 tokens.

\paragraph{LSH attention in Transformer.} \label{sec:exp_lsh}
LSH attention is an approximation for full attention that, as evidenced in Figure~\ref{fig:hash-curves}, becomes more accurate as the number of hashes increases. At , it already almost matches full attention. The computational cost of a model grows with the number of hashes, so this hyperparameter can be adjusted depending on the available compute budget. Additionally, as in Table 2, the number of hashes can be increased at evaluation time to produce more accurate results.
On the right half of Figure~\ref{fig:layer-curves}, we plot the speed of different attention types vs. the sequence length, while holding the total number of tokens fixed. We see that while regular attention becomes slower at longer sequence length, LSH attention speed remains flat.


\begin{figure}[!t]
    \centering
    \includegraphics[width=1.0\textwidth]{figures/hash_plot.png}
    \caption{LSH attention performance as a function of hashing rounds on imagenet64.}
    \label{fig:hash-curves}
\end{figure}

\begin{figure}[!t]
    \centering
    \includegraphics[width=1.0\textwidth]{figures/merge_depth_speed.png}
    \caption{Left: LSH attention performance as a function of number of layers on enwik8. 
    Right: Speed of attention evaluation as a function of input length for full- and LSH- attention.}
    \label{fig:layer-curves}
\end{figure}


\paragraph{Large Reformer models.} \label{sec:exp_large}
To verify that the Reformer can indeed fit large models on a single core
and train fast on long sequences, we train up to 20-layer big Reformers on enwik8 and imagenet64.
As can be seen in Figure~\ref{fig:layer-curves}, these models
fit into memory and train. We were not able to train Transformer baselines
in this case as they are too slow and memory-hungry, but we see clear improvement
with the number of layers. A 12-layer model on enwik8 trained for 20K steps with a dropout rate of 0.1 achieves 1.19 bits/dim on the test set. We also trained a 12-layer Reformer model for longer with
further tuning and improvements and we reached 1.05 bits/dim on the enwiki8 test
set.


\section{Conclusion}

Reformer combines the modeling capacity of a Transformer with an architecture that can be executed efficiently on long sequences and with small memory
use even for models with a large number of layers.
We believe that this will help large, richly-parameterized Transformer models become more widespread and accessible. 
Also, the ability to handle long sequences opens the way for the use of
the Reformer on many generative tasks. In addition to generating very long
coherent text, the Reformer can bring the power of Transformer models to other
domains like time-series forecasting, music, image and video generation.

\bibliography{references}
\bibliographystyle{iclr2020_conference}

\newpage
\appendix
\section{Multi-round LSH Attention} \label{sec:multi-round-detail}

In this section we describe in more detail the multi-hash version of our LSH attention mechanism. We first repeat Equation (\ref{eq:attn-masked}) from the main text, which describes a general formulation of attention with sparsity:


In the multi-round case, a query position  can attend to key positions  as defined in (\ref{eq:union-pairs}), which we also repeat here:


For batching purposes, attention is performed on chunks of sorted queries/keys:


Combining (\ref{eq:attn-masked}) and (\ref{eq:union-pairs}) gives:


Each round of LSH attention produces a vector  that can be computed independently from other rounds, except for the inclusion of a term  to avoid double-counting elements when constructing the union of  sets. In our implementation we fold the  factor into the masking term .

We also modify  to introduce a special case for .
This case is added because causal masking in a standard Transformer allows position  to attend to itself, which is not desirable in a shared-QK formulation. We set the mask to a large but finite value to disallow attention-in-place, except in the situation where a token has no other valid attention targets. For example, the first token in a sequence attends only to itself, because no prior context is available.

\end{document}
