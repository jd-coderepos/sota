

\documentclass[runningheads]{llncs}
\usepackage{graphicx}
\usepackage{comment}
\usepackage{amsmath,amssymb} \usepackage{color}

\usepackage{ruler}
\usepackage[width=122mm,left=12mm,paperwidth=146mm,height=193mm,top=12mm,paperheight=217mm]{geometry}



\begin{document}
\pagestyle{headings}
\mainmatter
\def\ECCVSubNumber{100}  

\title{Author Guidelines for ECCV Submission} 

\titlerunning{ECCV-20 submission ID \ECCVSubNumber} 
\authorrunning{ECCV-20 submission ID \ECCVSubNumber} 
\author{Anonymous ECCV submission}
\institute{Paper ID \ECCVSubNumber}


\begin{comment}
\titlerunning{Abbreviated paper title}
\author{First Author\inst{1}\orcidID{0000-1111-2222-3333} \and
Second Author\inst{2,3}\orcidID{1111-2222-3333-4444} \and
Third Author\inst{3}\orcidID{2222--3333-4444-5555}}
\authorrunning{F. Author et al.}
\institute{Princeton University, Princeton NJ 08544, USA \and
Springer Heidelberg, Tiergartenstr. 17, 69121 Heidelberg, Germany
\email{lncs@springer.com}\\
\url{http://www.springer.com/gp/computer-science/lncs} \and
ABC Institute, Rupert-Karls-University Heidelberg, Heidelberg, Germany\\
\email{\{abc,lncs\}@uni-heidelberg.de}}
\end{comment}
\maketitle

\begin{abstract}
The abstract should summarize the contents of the paper. LNCS guidelines
indicate it should be at least 70 and at most 150 words. It should be set in 9-point
font size and should be inset 1.0~cm from the right and left margins.
\dots
\keywords{We would like to encourage you to list your keywords within
the abstract section}
\end{abstract}


\section{Introduction}


This document serves as an example submission. It illustrates the format
we expect authors to follow when submitting a paper to ECCV. 
At the same time, it gives details on various aspects of paper submission,
including preservation of anonymity and how to deal with dual submissions,
so we advise authors to read this document carefully.

\section{Initial Submission}

\subsection{Language}

All manuscripts must be in English.

\subsection{Paper length}
Papers submitted for review should be complete. 
The length should match that intended for final publication. 
Papers accepted for the conference will be allocated 14 pages (plus references) in the proceedings. 
Note that the allocated 14 pages do not include the references. The reason for this policy
is that we do not want authors to omit references for sake of space limitations.

Papers with more than 14 pages (excluding references) will be rejected without review.
This includes papers where the margins and
formatting are deemed to have been significantly altered from those
laid down by this style guide.  The reason such papers will not be reviewed is that there is no provision for supervised revisions of manuscripts. The reviewing process cannot determine the suitability of the paper for presentation in 14 pages if it is reviewed in 16.

\subsection{Paper ID}

It is imperative that the paper ID is mentioned on each page of the manuscript.
The paper ID is a number automatically assigned to your submission when 
registering your paper submission on the submission site.



All lines should be numbered in the initial submission, as in this example document. This makes reviewing more efficient, because reviewers can refer to a line on a page. Line numbering is removed in the camera-ready.


\subsection{Mathematics}

Please number all of your sections and displayed equations.  Again,
this makes reviewing more efficient, because reviewers can refer to a
line on a page.  Also, it is important for readers to be able to refer
to any particular equation.  Just because you didn't refer to it in
the text doesn't mean some future reader might not need to refer to
it.  It is cumbersome to have to use circumlocutions like ``the
equation second from the top of page 3 column 1''.  (Note that the
line numbering will not be present in the final copy, so is not an
alternative to equation numbers).  Some authors might benefit from
reading Mermin's description of how to write mathematics:
\url{www.pamitc.org/documents/mermin.pdf}.
\section{Policies}
\subsection{Review Process}
By submitting a paper to ECCV, the authors agree to the review process and understand that papers are processed by the Toronto system to match each manuscript to the best possible chairs and reviewers.
\subsection{Confidentiality}
The review process of ECCV is confidential. Reviewers are volunteers not part of the ECCV organisation and their efforts are greatly appreciated. The standard practice of keeping all information confidential during the review is part of the standard communication to all reviewers. Misuse of confidential information is a severe professional failure and  appropriate measures will be taken when brought to the attention of ECCV organizers. It should be noted, however, that the organisation of ECCV is not and cannot be held responsible for the consequences when reviewers break confidentiality.

Accepted papers will be published by Springer (with appropriate copyrights) electronically up to three weeks prior to the main conference. Please make sure to discuss this issue with your legal advisors as it pertains to public disclosure of the contents of the papers submitted.
\subsection{Dual and Double Submissions}
By submitting a manuscript to ECCV 2020, authors acknowledge that it has not been previously published or accepted for publication in substantially similar form in any peer-reviewed venue including journal, conference, or workshop. Furthermore, no paper substantially similar in content has been or will be submitted to a journal, another conference or workshop during the review period (March 05, 2020 – July 3, 2020). The authors also attest that they did not submit substantially similar submissions to ECCV 2020. Violation of any of these conditions will lead to rejection and the violation will be reported to the other venue or journal, which will typically lead to rejection there as well. 

The goals of the dual submission policy are (i) to have exciting new work be published for the first time at ECCV 2020, and (ii) to avoid duplicating the efforts of the reviewers.
Therefore, all papers under review are checked for dual submissions and this is not allowed, independent of the page size of submissions. 

For already published papers, our policy is based upon the following particular definition of ``publication''. A publication, for the purposes of the dual submission policy, is defined to be a written work longer than four pages that was submitted for review by peers for either acceptance or rejection, and, after review, was accepted. In particular, this definition of publication does not depend upon whether such an accepted written work appears in a formal proceedings or whether the organizers declare that such work ``counts as a publication''. 

An arXiv.org paper does not count as a publication because it was not peer-reviewed for acceptance. The same is true for university technical reports. However, this definition of publication does include peer-reviewed workshop papers, even if they do not appear in a proceedings, if their length is more than 4 pages including citations. Given this definition, any submission to ECCV 2020 should not have substantial overlap with prior publications or other concurrent submissions. As a rule of thumb, the ECCV 2020 submission should contain no more than 20 percent of material from previous publications. 

\subsection{Requirements for publication}
Publication of the paper in the ECCV 2020 proceedings of Springer requires that at least one of the authors registers for the conference and present the paper there. It also requires that a camera-ready version that satisfies all formatting requirements is submitted before the camera-ready deadline. 
\subsection{Double blind review}
\label{sec:blind}
ECCV reviewing is double blind, in that authors do not know the names of the area chair/reviewers of their papers, and the area chairs/reviewers cannot, beyond reasonable doubt, infer the names of the authors from the submission and the additional material. Avoid providing links to websites that identify the authors. Violation of any of these guidelines may lead to rejection without review. If you need to cite a different paper of yours that is being submitted concurrently to ECCV, the authors should (1) cite these papers, (2) argue in the body of your paper why your ECCV paper is non trivially different from these concurrent submissions, and (3) include anonymized versions of those papers in the supplemental material.

Many authors misunderstand the concept of anonymizing for blind
review. Blind review does not mean that one must remove
citations to one's own work. In fact it is often impossible to
review a paper unless the previous citations are known and
available.

Blind review means that you do not use the words ``my'' or ``our''
when citing previous work.  That is all.  (But see below for
technical reports).

Saying ``this builds on the work of Lucy Smith [1]'' does not say
that you are Lucy Smith, it says that you are building on her
work.  If you are Smith and Jones, do not say ``as we show in
[7]'', say ``as Smith and Jones show in [7]'' and at the end of the
paper, include reference 7 as you would any other cited work.

An example of a bad paper:
\begin{quote}
\begin{center}
    An analysis of the frobnicatable foo filter.
\end{center}

   In this paper we present a performance analysis of our
   previous paper [1], and show it to be inferior to all
   previously known methods.  Why the previous paper was
   accepted without this analysis is beyond me.

   [1] Removed for blind review
\end{quote}


An example of an excellent paper:

\begin{quote}
\begin{center}
     An analysis of the frobnicatable foo filter.
\end{center}

   In this paper we present a performance analysis of the
   paper of Smith [1], and show it to be inferior to
   all previously known methods.  Why the previous paper
   was accepted without this analysis is beyond me.

   [1] Smith, L. and Jones, C. ``The frobnicatable foo
   filter, a fundamental contribution to human knowledge''.
   Nature 381(12), 1-213.
\end{quote}

If you are making a submission to another conference at the same
time, which covers similar or overlapping material, you may need
to refer to that submission in order to explain the differences,
just as you would if you had previously published related work. In
such cases, include the anonymized parallel
submission~\cite{Authors14} as additional material and cite it as
\begin{quote}
1. Authors. ``The frobnicatable foo filter'', BMVC 2014 Submission
ID 324, Supplied as additional material {\tt bmvc14.pdf}.
\end{quote}

Finally, you may feel you need to tell the reader that more
details can be found elsewhere, and refer them to a technical
report.  For conference submissions, the paper must stand on its
own, and not {\em require} the reviewer to go to a techreport for
further details.  Thus, you may say in the body of the paper
``further details may be found in~\cite{Authors14b}''.  Then
submit the techreport as additional material. Again, you may not
assume the reviewers will read this material.

Sometimes your paper is about a problem which you tested using a tool which
is widely known to be restricted to a single institution.  For example,
let's say it's 1969, you have solved a key problem on the Apollo lander,
and you believe that the ECCV audience would like to hear about your
solution.  The work is a development of your celebrated 1968 paper entitled
``Zero-g frobnication: How being the only people in the world with access to
the Apollo lander source code makes us a wow at parties'', by Zeus.

You can handle this paper like any other.  Don't write ``We show how to
improve our previous work [Anonymous, 1968].  This time we tested the
algorithm on a lunar lander [name of lander removed for blind review]''.
That would be silly, and would immediately identify the authors. Instead
write the following:
\begin{quotation}
\noindent
   We describe a system for zero-g frobnication.  This
   system is new because it handles the following cases:
   A, B.  Previous systems [Zeus et al. 1968] didn't
   handle case B properly.  Ours handles it by including
   a foo term in the bar integral.

   ...

   The proposed system was integrated with the Apollo
   lunar lander, and went all the way to the moon, don't
   you know.  It displayed the following behaviours
   which show how well we solved cases A and B: ...
\end{quotation}
As you can see, the above text follows standard scientific convention,
reads better than the first version, and does not explicitly name you as
the authors.  A reviewer might think it likely that the new paper was
written by Zeus, but cannot make any decision based on that guess.
He or she would have to be sure that no other authors could have been
contracted to solve problem B. \\

For sake of anonymity, it's recommended to omit acknowledgements
in your review copy. They can be added later when you prepare the final copy.

\section{Manuscript Preparation}

This is an edited version of Springer LNCS instructions adapted
for ECCV 2020 first paper submission.
You are strongly encouraged to use \LaTeX2 for the
preparation of your
camera-ready manuscript together with the corresponding Springer
class file \verb+llncs.cls+.

We would like to stress that the class/style files and the template
should not be manipulated and that the guidelines regarding font sizes
and format should be adhered to. This is to ensure that the end product
is as homogeneous as possible.

\subsection{Printing Area}
The printing area is .
The text should be justified to occupy the full line width,
so that the right margin is not ragged, with words hyphenated as
appropriate. Please fill pages so that the length of the text
is no less than 180~mm.

\subsection{Layout, Typeface, Font Sizes, and Numbering}
Use 10-point type for the name(s) of the author(s) and 9-point type for
the address(es) and the abstract. For the main text, please use 10-point
type and single-line spacing.
We recommend using Computer Modern Roman (CM) fonts, Times, or one
of the similar typefaces widely used in photo-typesetting.
(In these typefaces the letters have serifs, i.e., short endstrokes at
the head and the foot of letters.)
Italic type may be used to emphasize words in running text. Bold
type and underlining should be avoided.
With these sizes, the interline distance should be set so that some 45
lines occur on a full-text page.

\subsubsection{Headings.}

Headings should be capitalized
(i.e., nouns, verbs, and all other words
except articles, prepositions, and conjunctions should be set with an
initial capital) and should,
with the exception of the title, be aligned to the left.
Words joined by a hyphen are subject to a special rule. If the first
word can stand alone, the second word should be capitalized.
The font sizes
are given in Table~\ref{table:headings}.
\setlength{\tabcolsep}{4pt}
\begin{table}
\begin{center}
\caption{Font sizes of headings. Table captions should always be
positioned {\it above} the tables. The final sentence of a table
caption should end without a full stop}
\label{table:headings}
\begin{tabular}{lll}
\hline\noalign{\smallskip}
Heading level & Example & Font size and style\\
\noalign{\smallskip}
\hline
\noalign{\smallskip}
Title (centered)  & {\Large \bf Lecture Notes \dots} & 14 point, bold\\
1st-level heading & {\large \bf 1 Introduction} & 12 point, bold\\
2nd-level heading & {\bf 2.1 Printing Area} & 10 point, bold\\
3rd-level heading & {\bf Headings.} Text follows \dots & 10 point, bold
\\
4th-level heading & {\it Remark.} Text follows \dots & 10 point,
italic\\
\hline
\end{tabular}
\end{center}
\end{table}
\setlength{\tabcolsep}{1.4pt}

Here are some examples of headings: ``Criteria to Disprove Context-Freeness of
Collage Languages'', ``On Correcting the Intrusion of Tracing
Non-deterministic Programs by Software'', ``A User-Friendly and
Extendable Data Distribution System'', ``Multi-flip Networks:
Parallelizing GenSAT'', ``Self-determinations of Man''.

\subsubsection{Lemmas, Propositions, and Theorems.}

The numbers accorded to lemmas, propositions, and theorems etc. should
appear in consecutive order, starting with the number 1, and not, for
example, with the number 11.

\subsection{Figures and Photographs}
\label{sect:figures}

Please produce your figures electronically and integrate
them into your text file. For \LaTeX\ users we recommend using package
\verb+graphicx+ or the style files \verb+psfig+ or \verb+epsf+.

Check that in line drawings, lines are not
interrupted and have constant width. Grids and details within the
figures must be clearly readable and may not be written one on top of
the other. Line drawings should have a resolution of at least 800 dpi
(preferably 1200 dpi).
For digital halftones 300 dpi is usually sufficient.
The lettering in figures should have a height of 2~mm (10-point type).
Figures should be scaled up or down accordingly.
Please do not use any absolute coordinates in figures.

Figures should be numbered and should have a caption which should
always be positioned {\it under} the figures, in contrast to the caption
belonging to a table, which should always appear {\it above} the table.
Please center the captions between the margins and set them in
9-point type
(Fig.~\ref{fig:example} shows an example).
The distance between text and figure should be about 8~mm, the
distance between figure and caption about 5~mm.
\begin{figure}
\centering
\includegraphics[height=6.5cm]{eijkel2}
\caption{One kernel at  ({\it dotted kernel}) or two kernels at
 and  ({\it left and right}) lead to the same summed estimate
at . This shows a figure consisting of different types of
lines. Elements of the figure described in the caption should be set in
italics,
in parentheses, as shown in this sample caption. The last
sentence of a figure caption should generally end without a full stop}
\label{fig:example}
\end{figure}

If possible (e.g. if you use \LaTeX) please define figures as floating
objects. \LaTeX\ users, please avoid using the location
parameter ``h'' for ``here''. If you have to insert a pagebreak before a
figure, please ensure that the previous page is completely filled.


\subsection{Formulas}

Displayed equations or formulas are centered and set on a separate
line (with an extra line or halfline space above and below). Displayed
expressions should be numbered for reference. The numbers should be
consecutive within the contribution,
with numbers enclosed in parentheses and set on the right margin.
For example,


Please punctuate a displayed equation in the same way as ordinary
text but with a small space before the end punctuation.

\subsection{Footnotes}

The superscript numeral used to refer to a footnote appears in the text
either directly after the word to be discussed or, in relation to a
phrase or a sentence, following the punctuation sign (comma,
semicolon, or full stop). Footnotes should appear at the bottom of
the
normal text area, with a line of about 2~cm in \TeX\ and about 5~cm in
Word set
immediately above them.\footnote{The footnote numeral is set flush left
and the text follows with the usual word spacing. Second and subsequent
lines are indented. Footnotes should end with a full stop.}


\subsection{Program Code}

Program listings or program commands in the text are normally set in
typewriter font, e.g., CMTT10 or Courier.

\noindent
{\it Example of a Computer Program}
\begin{verbatim}
program Inflation (Output)
  {Assuming annual inflation rates of 7years};
   const
     MaxYears = 10;
   var
     Year: 0..MaxYears;
     Factor1, Factor2, Factor3: Real;
   begin
     Year := 0;
     Factor1 := 1.0; Factor2 := 1.0; Factor3 := 1.0;
     WriteLn('Year  7repeat
       Year := Year + 1;
       Factor1 := Factor1 * 1.07;
       Factor2 := Factor2 * 1.08;
       Factor3 := Factor3 * 1.10;
       WriteLn(Year:5,Factor1:7:3,Factor2:7:3,Factor3:7:3)
     until Year = MaxYears
end.
\end{verbatim}
\noindent
{\small (Example from Jensen K., Wirth N. (1991) Pascal user manual and
report. Springer, New York)}



\subsection{Citations}

The list of references is headed ``References" and is not assigned a
number
in the decimal system of headings. The list should be set in small print
and placed at the end of your contribution, in front of the appendix,
if one exists.
Please do not insert a pagebreak before the list of references if the
page is not completely filled.
An example is given at the
end of this information sheet. For citations in the text please use
square brackets and consecutive numbers: \cite{Alpher02},
\cite{Alpher03}, \cite{Alpher04} \dots

\section{Submitting a Camera-Ready for an Accepted Paper}
\subsection{Converting Initial Submission to Camera-Ready}
To convert a submission file into a camera-ready for an accepted paper:
\begin{enumerate}
    \item  First comment out \begin{verbatim}
        \usepackage{ruler}
    \end{verbatim} and the line that follows it.
    \item  The anonymous title part should be removed or commented out, and a proper author block should be inserted, for which a skeleton is provided in a commented-out version. These are marked in the source file as \begin{verbatim}
\end{verbatim} and \begin{verbatim}
\end{verbatim}
    \item Please write out author names in full in the paper, i.e. full given and family names. If any authors have names that can be parsed into FirstName LastName in multiple ways, please include the correct parsing in a comment to the editors, below the \begin{verbatim}\author{}\end{verbatim} field.
    \item Make sure you have inserted the proper Acknowledgments.
  \end{enumerate}  
 
\subsection{Preparing the Submission Package}
We need all the source files (LaTeX files, style files, special fonts, figures, bib-files) that are required to compile papers, as well as the camera ready PDF. For each paper, one ZIP-file called XXXX.ZIP (where XXXX is the zero-padded, four-digit paper ID) has to be prepared and submitted via the ECCV 2020 Submission Website, using the password you received with your initial registration on that site. The size of the ZIP-file may not exceed the limit of 60 MByte. The ZIP-file has to contain the following:
  \begin{enumerate}
 \item  All source files, e.g. LaTeX2e files for the text, PS/EPS or PDF/JPG files for all figures.
 \item PDF file named ``XXXX.pdf" that has been produced by the submitted source, where XXXX is the four-digit paper ID (zero-padded if necessary). For example, if your paper ID is 24, the filename must be 0024.pdf. This PDF will be used as a reference and has to exactly match the output of the compilation.
 \item PDF file named ``XXXX-copyright.PDF": a scanned version of the signed copyright form (see ECCV 2020 Website, Camera Ready Guidelines for the correct form to use). 
 \item If you wish to provide supplementary material, the file name must be in the form XXXX-supp.pdf or XXXX-supp.zip, where XXXX is the zero-padded, four-digit paper ID as used in the previous step. Upload your supplemental file on the ``File Upload" page as a single PDF or ZIP file of 100 MB in size or less. Only PDF and ZIP files are allowed for supplementary material. You can put anything in this file – movies, code, additional results, accompanying technical reports–anything that may make your paper more useful to readers.  If your supplementary material includes video or image data, you are advised to use common codecs and file formats.  This will make the material viewable by the largest number of readers (a desirable outcome). ECCV encourages authors to submit videos using an MP4 codec such as DivX contained in an AVI. Also, please submit a README text file with each video specifying the exact codec used and a URL where the codec can be downloaded. Authors should refer to the contents of the supplementary material appropriately in the paper.
 \end{enumerate}

Check that the upload of your file (or files) was successful either by matching the file length to that on your computer, or by using the download options that will appear after you have uploaded. Please ensure that you upload the correct camera-ready PDF–renamed to XXXX.pdf as described in the previous step as your camera-ready submission. Every year there is at least one author who accidentally submits the wrong PDF as their camera-ready submission.

Further considerations for preparing the camera-ready package:
  \begin{enumerate}
    \item Make sure to include any further style files and fonts you may have used.
    \item References are to be supplied as BBL files to avoid omission of data while conversion from BIB to BBL.
    \item Please do not send any older versions of papers. There should be one set of source files and one XXXX.pdf file per paper. Our typesetters require the author-created pdfs in order to check the proper representation of symbols, figures, etc.
    \item  Please remove unnecessary files (such as eijkel2.pdf and eijkel2.eps) from the source folder. 
    \item  You may use sub-directories.
    \item  Make sure to use relative paths for referencing files.
    \item  Make sure the source you submit compiles.
\end{enumerate}

Springer is the first publisher to implement the ORCID identifier for proceedings, ultimately providing authors with a digital identifier that distinguishes them from every other researcher. ORCID (Open Researcher and Contributor ID) hosts a registry of unique researcher identifiers and a transparent method of linking research activities to these identifiers. This is achieved through embedding ORCID identifiers in key workflows, such as research profile maintenance, manuscript submissions, grant applications and patent applications.
\subsection{Most Frequently Encountered Issues}
Please kindly use the checklist below to deal with some of the most frequently encountered issues in ECCV submissions.

{\bf FILES:}
\begin{itemize}
    \item My submission package contains ONE compiled pdf file for the camera-ready version to go on Springerlink.
\item I have ensured that the submission package has all the additional files necessary for compiling the pdf on a standard LaTeX distribution.
\item I have used the correct copyright form (with editor names pre-printed), and a signed pdf is included in the zip file with the correct file name.
\end{itemize}

{\bf CONTENT:}
\begin{itemize}
\item I have removed all \verb \vspace  and \verb \hspace  commands from my paper.
\item I have not used \verb \thanks  or \verb \footnote  commands and symbols for corresponding authors in the title (which is processed with scripts) and (optionally) used an Acknowledgement section for all the acknowledgments, at the end of the paper.
\item I have not used \verb \cite  command in the abstract.
\item I have read the Springer author guidelines, and complied with them, including the point on providing full information on editors and publishers for each reference in the paper (Author Guidelines – Section 2.8).
\item I have entered a correct \verb \titlerunning{}  command and selected a meaningful short name for the paper.
\item I have entered \verb \index{Lastname,Firstname}  commands for names that are longer than two words.
\item I have used the same name spelling in all my papers accepted to ECCV and ECCV Workshops.
\item I have inserted the ORCID identifiers of the authors in the paper header (see http://bit.ly/2H5xBpN for more information).
\item I have not decreased the font size of any part of the paper (except tables) to fit into 14 pages, I understand Springer editors will remove such commands.
\end{itemize}
{\bf SUBMISSION:}
\begin{itemize}
\item All author names, titles, and contact author information are correctly entered in the submission site.
\item The corresponding author e-mail is given.
\item At least one author has registered by the camera ready deadline.
\end{itemize}


\section{Conclusions}

The paper ends with a conclusion. 


\clearpage\mbox{}Page \thepage\ of the manuscript.
\clearpage\mbox{}Page \thepage\ of the manuscript.

This is the last page of the manuscript.
\par\vfill\par
Now we have reached the maximum size of the ECCV 2020 submission (excluding references).
References should start immediately after the main text, but can continue on p.15 if needed.

\clearpage
\bibliographystyle{splncs04}
\bibliography{egbib}
\end{document}
