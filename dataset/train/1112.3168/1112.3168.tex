\documentclass[a4paper,11pt]{article}
\usepackage{latexsym}
\usepackage{amssymb}
\usepackage{amsfonts}
\usepackage{amsmath}
\usepackage{indentfirst}
\usepackage[dvips]{graphicx}
\usepackage{epsfig}
\sloppy

\newcommand{\nkinN}{n,k\in \mathbf{N}}
\newcommand{\ninN}{n\in \mathbf{N}}
\newcommand{\nkinNstar}{n,k\in \mathbf{N^{*}}}
\newcommand{\ninNstar}{n\in \mathbf{N^{*}}}
\newcommand{\kinNstar}{k\in \mathbf{N^{*}}}
\newcommand{\kinN}{k\in \mathbf{N}}
\newcommand{\cvd}{\hfill \bigskip}
\newcommand{\Rx}{\mathbf{R}[x]}
\newtheorem{definition}{Definition}[section]
\newtheorem{proposition}{Proposition}[section]
\newtheorem{example}{Example}[section]
\newtheorem{theorem}{Theorem}[section]
\newtheorem{conjecture}{Conjecture}[section]

\def \theequation{\thesection.\arabic{equation}}
\def \thefigure{\thesection.\arabic{figure}}
\def \thetable{\thesection.\arabic{table}}
\newcommand{\qed}{\hfill\square\medskip}
\textwidth 16cm \textheight 22.9cm \topmargin -1cm \oddsidemargin
0cm \setlength{\parindent}{20pt}
\date{}
\author{S. Bilotta\thanks{Dipartimento di Sistemi e Informatica, Universit\`a degli Studi di Firenze, Viale
 G.B. Morgagni 65, 50134 Firenze, Italy. {\newline
 \tt \ bilotta@dsi.unifi.it,\quad elisa@dsi.unifi.it,\quad pinzani@dsi.unifi.it}} \and E. Pergola\and R. Pinzani}

\title{A new approach to cross-bifix-free sets}

\begin{document}

\maketitle

\begin{abstract}
Cross-bifix-free sets are sets of words such that no prefix of any
word is a suffix of any other word. In this paper, we introduce a
general constructive method for the sets of cross-bifix-free
binary words of fixed length. It enables us to determine a
cross-bifix-free words subset which has the property to be
non-expandable.
\end{abstract}

\section{Introduction}

In digital communication systems, synchronization is an essential
requirement to establish and maintain a connection between a
transmitter and a receiver.


Analytical approaches to the synchronization acquisition process
and methods for the construction of sequences with the best
aperiodic autocorrelation properties have been the subject of
numerous analyses in the digital transmission.

The historical engineering approach started with the introduction
of bifix. It denotes a subsequence that is both a prefix and
suffix of a longer observed sequence. Rather than to the bifix,
much attention has been devoted to a bifix-indicator, an indicator
function implying the existence of the bifix \cite{10}. Such
indicators were shown to be without equal in performing various
statistical analysis, mainly concerning the search process
\cite{3,10}

However, an analytical study of simultaneous search for a set of
sequences urged the invention of cross-bifix indicators \cite{1,2}
and, accordingly, turned attention to the sets of sequences which
avoid cross-bifixes, called cross-bifix-free sets.

In \cite{1}, the author analyzes some properties of binary words
that form a cross-bifix-free set, in particular, a general
constructing method called the kernel method is presented. This
approach leads to sets  of cross-bifix-free binary words, of
fixed length , having cardinality
 for
 respectively.

This sequence forms a Fibonacci progression and satisfies the
recurrence relation  with  and
.

The problem of determining cross-bifix-free sets is also related
to several other scientific applications, for instance in
multiaccess systems, pattern matching and automata theory.

The aim of this paper is to introduce a method for the generation
of sets of cross-bifix-free binary words of fixed length based
upon the study of lattice paths on the Cartesian plane. This
approach enables us to obtain cross-bifix-free sets having greater
cardinality than the ones presented in \cite{1}.

The paper is organized as follows. In Section 2 we give some basic
definitions and notation related to the notions of bifix-free word
and cross-bifix-free set. In Section 3 we propose a method to
construct particular sets of cross-bifix-free binary words of
fixed length  related to the parity of . We are not able to
say if such cross-bifix-free sets have maximal cardinality on the
set of bifix-free binary words of fixed length  or not.

\section{Basic definitions and notations}
Let  be a finite, non-empty set called \emph{alphabet}. The
elements of  are called \emph{letters}. A (finite) sequence of
letters in  is called (finite) \emph{word}. Let  denote
the monoid of all finite words over  where 
denotes the \emph{empty word} and . Let  be a word in , then 
indicates the length of  and  denotes the
number of occurrences of  in , being . Let
 then  is called \emph{prefix} of  and 
is called \emph{suffix} of . A \emph{bifix} of  is
a subsequence of  that is both its prefix and suffix.


A word  of  is said to be \emph{bifix-free} or
\emph{unbordered} \cite{7,11} if and only if no strict prefix of
 is also a suffix of . Therefore,  is
bifix-free if and only if , being  any
necessarily non-empty word and  any word. Obviously, a
necessary condition for  to be bifix-free is that the
first and the last letters of  must be different.
\begin{example}
In the monoid , the word  of length  is
bifix-free, while the word  contains two bifixes, 
and .
\end{example}


Let  denote the set of all bifix-free words of length 
over an alphabet of fixed size . The following formula for the
cardinality of , denoted by , is well-known
\cite{11}.



The number sequences related to this recurrence can be found in
Sloane's database of integer sequences \cite{12}: sequences
A003000 (), A019308 () and A019309 ().

Table \ref{bfree} lists the set , , the
last row reports the cardinality of each set.

\begin{table}[htb]
\begin{center}
\begin{tabular}{|c|c|c|c|c|c|}
\hline \hline n=2 & n=3 & n=4 & n=5 & n=6\\
\hline

10 \ 01 & 100 \ 001 & 1000 \ 0001 & 10000 \ 00001 & 100000 \ 000001 \\
   & 110 \ 011 & 1100 \ 0011 & 10100 \ 00101 & 101000 \ 000101\\
   &     & 1110 \ 0111 & 11000 \ 00011 & 101100 \ 001101\\
   &     &      & 11100 \ 00111 & 110000 \ 000011\\
   &     &      & 11010 \ 01011 & 110100 \ 001011\\
   &     &      & 11110 \ 01111 & 111000 \ 000111\\
   &     &      &       & 111100 \ 001111\\
   &     &      &       & 110010 \ 010011\\
   &     &      &       & 111010 \ 010111\\
   &     &      &       & 111110 \ 011111\\
\hline
 2 & 4 & 6 & 12 & 20\\
\hline


\end{tabular}
\end{center}
\caption{\label{bfree}The set , }
\end{table}

Let  and  be fixed. Two distinct words  are said to be \emph{cross-bifix-free} if and only if
no strict prefix of  is also a suffix of  and
vice-versa.

\begin{example}\label{esemp}
The binary words  and  in  are
cross-bifix-free, while the binary words  and
 in  have the cross-bifix .
\end{example}

A subset of  is said to be \emph{cross-bifix-free set} if
and only if for each , with , in this set,  and  are
cross-bifix-free. This set is said to be \emph{non-expandable} on
 if and only if the set obtained by adding any other word
is not a cross-bifix-free set. A non-expandable cross-bifix-free
set on  having maximal cardinality is called
\emph{maximal cross-bifix-free set} on .

Each word  can be naturally represented as a
lattice path on the Cartesian plane, by associating a \emph{rise
step}, defined by  and denoted by , to each 1's in
, and a \emph{fall step}, defined by  and denoted
by , to each 0's in , running from 
to , .

From now on, we will refer interchangeably to words or their
graphical representations on the Cartesian plane, that is paths.

The definition of bifix-free and cross-bifix-free can be easily
extended to paths. Figure \ref{exmpcrossbf} shows the two paths
corresponding to the cross-bifix-free words of Example
\ref{esemp}.

\begin{figure}[!htb]
\begin{center}
\epsfig{file=excbf01.eps,width=3.3in,clip=} \caption{\small{Two
paths in  which are cross-bifix-free}
\label{exmpcrossbf}}\vspace{-15pt}
\end{center}
\end{figure}

A lattice path on the Cartesian plane using the steps  and
 and running from  to , with , is
said to be \emph{Grand-Dyck} or \emph{Binomial} path (see \cite{5}
for further details). A \emph{Dyck} path is a sequence of rise
step and fall steps running from  to  and remaining
weakly above the -axis (see Figure \ref{dick}). The number of
-length Dyck paths is the th Catalan number , see \cite{13} for further details.

\begin{figure}[!htb]
\begin{center}
\epsfig{file=Dyck.eps,width=5in,clip=} \caption{\small{The
-length Dyck paths, }
\label{dick}}\vspace{-15pt}
\end{center}
\end{figure}

In this paper, we are interested in investigating a possible
non-expandable cross-bifix-free set, that is the set 
of cross-bifix-free words of fixed length  on the monoid
. In order to do so, we focus on the set
 of bifix-free binary words of fixed length 
having 1 as the first letter and 0 as last letter or equivalently
the set of bifix-free lattice paths on the Cartesian plane using
the steps  and , running from  to ,
, beginning with a rise step and ending with a fall step.
Of course  is
obtained by switching rise and fall steps.

Let  denote the set of the paths in
 having  as the ordinate of their endpoint, .

\section{On the non-expandability of }
In order to prove that  is a non-expandable
cross-bifix-free set on  we have to distinguish the
following two cases depending on the parity of .

\subsection{Non-expandable } Let  that is the set of paths beginning
with a rise step linked to a -length Dyck path (see Figure
\ref{rappdis}). Note that  is a subset of
, .
\begin{figure}[!htb]
\begin{center}
\epsfig{file=rapdisp.eps,width=3.5in,clip=} \caption{\small{A
graphical representation of , with }
\label{rappdis}}\vspace{-15pt}
\end{center}
\end{figure}

Of course , being  the th
Catalan number, .\\
Figure \ref{rap7} shows the set , with
.
\begin{figure}[!htb]
\begin{center}
\epsfig{file=rapp7bis.eps,width=6in,clip=} \caption{\small{A
graphical representation of }
\label{rap7}}\vspace{-15pt}
\end{center}
\end{figure}

\begin{proposition}\label{dispariCB}
The set  is a cross-bifix-free set on ,
.
\end{proposition}
\emph{Proof.}\quad The proof consists of two distinguished steps.
The first one proves that each  is
bifix-free and the second one proves that  is a
cross-bifix-free set. Each  can be
written as , being  any necessarily non-empty
word while  can also be an empty word. For each prefix  of
 we have  and for each suffix  of  we have . Therefore ,  so
 is bifix-free.

The proof that, for each  then
 and  are cross-bifix-free, follows the logical
steps described above. \cvd

\begin{proposition}\label{dispariNE}
The set  is a non-expandable cross-bifix-free set on
, .
\end{proposition}
\emph{Proof.}\quad It is sufficient to prove that the set
 is a non-expandable cross-bifix-free set on
, as each  and  match on the last letter of  and
the first one of  at least.

Let  be fixed, we can prove that  is a
non-expandable cross-bifix-free set on  by
distinguishing  from .
\begin{itemize}
\item[  :] a path  in  can be written as  (see Figure \ref{classep}, where
), being  a Grand-Dyck path beginning with a rise
step,  a rise step,  Dyck paths, ,
and  a necessarily non-empty Dyck path. Therefore, we
can find paths in  having a prefix which matches
with a suffix of . It is sufficient to consider the path
, being  a Dyck path of
appropriate length.

\begin{figure}[!htb]
\begin{center}
\epsfig{file=classppos.eps,width=3.7in,clip=} \caption{\small{A
graphical representation of a path  in } \label{classep}}\vspace{-15pt}
\end{center}
\end{figure}

\item[  :] a path  in 
can be written as  (see
Figure \ref{classepnegativa}, where ), being  a
necessarily non-empty Dyck path,  a fall step,
 Dyck paths, , and  a
Grand-Dyck path. Therefore, we can find paths in 
having a suffix which matches with a prefix of . It is
sufficient to consider the path ,
being  a Dyck path of appropriate length.

\begin{figure}[!htb]
\begin{center}
\epsfig{file=classpneg.eps,width=3.7in,clip=} \caption{\small{A
graphical representation of a path  in } \label{classepnegativa}}\vspace{-12pt}
\end{center}
\end{figure}
\end{itemize}
\cvd

\subsection{Non-expandable }
In this case we have to distinguish
two further subcases depending on the parity of .\\

If  is an even number then , that is the set of paths consisting
of the following consecutive sub-paths: a -length Dyck path, a
rise step, a -length Dyck path, a fall step, where  (see Figure \ref{paripari}). Note that
 is a subset of , for any even
number .

\begin{figure}[!htb]
\begin{center}
\epsfig{file=rapparpar22.eps,width=5.3in,clip=} \caption{\small{A
graphical representation of , for any even number
} \label{paripari}}\vspace{-15pt}
\end{center}
\end{figure}

Of course ,  is
the th Catalan Number, for any even number . Figure
\ref{exmpari} shows the set , with
.

\begin{figure}[!htb]
\begin{center}
\epsfig{file=rapp10_01.eps,width=6.2in,clip=} \caption{\small{A
graphical representation of }
\label{exmpari}}\vspace{-15pt}
\end{center}
\end{figure}

\begin{proposition}\label{paripariCB}
The set  is a cross-bifix-free set on ,
for any even number .
\end{proposition}
\emph{Proof.}\quad The proof consists of two distinguished steps.
The first one proves that each  is
bifix-free and the second one proves that  is a
cross-bifix-free set. Each  can be
written as , being  any necessarily non-empty
word while  can also be an empty word. Let  be fixed, we
have to take into consideration two different cases: in the first
one  and in the second one .

If  then , and for each prefix  of  we have  and for each suffix  of  we have . Therefore ,  and
 is bifix-free.

Otherwise, , then for each prefix  of  we have  and for each suffix  of  we have . If  then ,  and therefore  is bifix-free. Let  be fixed, if
 then the path  is a -length Dyck path, . In this case, both , where
 is any suffix of , and ,
where  is any suffix of . If  then , therefore ,  and therefore  is bifix-free. If  then  does not match with , therefore ,  and
therefore  is bifix-free.

The proof that, for each  then
 and  are cross-bifix-free, follows the logical
steps described above. \cvd

\begin{proposition}\label{paripariNE}
The set  is a non-expandable cross-bifix-free set on
, for any even number .
\end{proposition}
\emph{Proof.}\quad It is sufficient to prove that the set
 is a non-expandable cross-bifix-free set on
, as each  and  match on the last letter of  and
the first one of  at least.

Let  be fixed, we have to take into consideration three
different cases: in the first one we prove that  is
a non-expandable cross-bifix-free set on ,
, in the second one we prove that  is a
non-expandable cross-bifix-free set on ,
, and in the last one we prove that  is a
non-expandable cross-bifix-free set on .

\begin{itemize}
\item[  :] a path  in 
can be written as  (see Figure \ref{classep}, where ), being 
a Grand-Dyck path beginning with a rise step,  a rise step,
 Dyck paths, , and  a
necessarily non-empty Dyck path. Therefore, we can find paths in
 having a prefix which matches with a suffix of
. It is sufficient to consider the path , being  a Dyck path of
appropriate length.

\item[  :] a path  in 
can be written as  (see
Figure \ref{classepnegativa}, where ), being  a
necessarily non-empty Dyck path,  a fall step,
 Dyck paths, , and  a
Grand-Dyck path. Therefore, we can find paths in 
having a suffix which matches with a prefix of . It is
sufficient to consider the path , being  a Dyck path of appropriate length.

\item[  :] a path  in  either never falls below the -axis or
crosses the -axis. In the first case, it can be written as
, where  is a
necessarily non-empty -length Dyck path and  is a
-length Dyck path, with ,
see Figure \ref{parparNE} a). Therefore, we can find paths in
 having a prefix which matches with a suffix of
. It is sufficient to consider the path , since  being .

If a path  in 
crosses the -axis then it can be written as  where  is a necessarily non-empty -length Dyck
path, , and  is a necessarily non-empty
Grand-Dyck beginning with a fall step, see Figure \ref{parparNE}
b). Therefore, we can find paths in  having a suffix
which matches with a prefix of . It is sufficient to
consider the path ,
being  a Dyck path of appropriate length.

\begin{figure}[!htb]
\begin{center}
\epsfig{file=bfo.eps,width=4.5in,clip=} \caption{\small{The two
possible configurations for a path  in , for any even number }
\label{parparNE}}\vspace{-12pt}
\end{center}
\end{figure}
\end{itemize}
\cvd

If  is an odd number then , that is the set of paths consisting of the following
consecutive sub-paths: a -length Dyck path, a rise step, a
-length Dych path, a fall step, where , and excluding those consisting of the following
consecutive sub-paths: a rise step, a -length Dyck path, a
fall step followed by a rise step, a -length Dyck path, a
fall step (see Figure \ref{paridispari}). In other words, the
paths which result from the concatenation of two elevated Dyck
paths of the same length must be excluded.

In particular, if  then the excluded paths are not
bifix-free, otherwise if  then the excluded
paths match with the paths  in
. Note that  is a subset of
, for any odd number .

\begin{figure}[!htb]
\begin{center}
\epsfig{file=rappardispari.eps,width=6.2in,clip=}
\caption{\small{A graphical representation of , for
any odd number } \label{paridispari}}\vspace{-15pt}
\end{center}
\end{figure}

Of course ,  is the th Catalan Number, for any
odd number . Figure \ref{exmparidis} shows the set
, with .

\begin{figure}[!htb]
\begin{center}
\epsfig{file=rapp8.eps,width=6.2in,clip=} \caption{\small{A
graphical representation of the set }
\label{exmparidis}}\vspace{-15pt}
\end{center}
\end{figure}

\begin{proposition}\label{paridispariCB}
The set  is a cross-bifix-free set on ,
for any odd number .
\end{proposition}

\begin{proposition}\label{padispariNE}
The set  is a non-expandable cross-bifix-free set on
, for any odd number .
\end{proposition}

The proof of Proposition \ref{paridispariCB} follows the logical
steps as far Proposition \ref{paripariCB} and the proof of
Proposition \ref{padispariNE} follows the logical steps as far
Proposition \ref{paripariNE}.

Therefore, the presented constructing method gives sets
 of cross-bifix-free binary words, of fixed length ,
having cardinality  for
 respectively.

\section{Conclusions and further developments}
In this paper, we introduce a general constructing method for the
sets of cross-bifix-free binary words of fixed length  based
upon the study of lattice paths on the Cartesian plane. This
approach enables us to obtain the cross-bifix-free set 
having greater cardinality than the ones presented in \cite{1}
based upon the kernel method.

Moreover, we prove that  is a non-expandable
cross-bifix-free set on , i.e.  is
not a cross-bifix-free set on , for any . The non-expandable property is
obviously a necessary condition to obtain a maximal
cross-bifix-free set on , anyway we are not able to find
and prove a sufficient condition.

Further studies to prove that could investigate both the
nontrivial subsets of  in which  is a maximal
cross-bifix-free set, and the study of other non-expandable
cross-bifix-free sets on .

Another approach to reach the goal could be to find a different
characterization of bifix-free words which could be obtained
through bijective methods between particular bifix-free subsets
and other well-known discrete structures.

Successive studies should take into consideration the general
study of cross-bifix-free sets on , where  is grater
than 2.

\begin{thebibliography}{99}

\bibitem[1]{1} D. Bajic. On Construction of Cross-Bifix-Free Kernel Sets.
2nd MCM COST 2100, TD(07)237, February 2007, Lisbon, Portugal.

\bibitem[2]{2} D. Bajic, D. Drajic. Duration of search for a fixed
pattern in random data: Distribution function and variance.
\emph{Electronics letters}, Vol. 31, No. 8, 631-632, 1995.

\bibitem[3]{3} D. Bajic, J. Stojanovic. Distributed Sequences and
Search Process. \emph{IEEE International Conference on
Communications ICC2004}, Paris, 514-518, June 2004.

\bibitem[4]{4} R. H. Barker. Group synchronizing of binary digital
systems. \emph{Communication theory}, W. jackson, Ed. London,
U.K.: Butterworth, 273-287, 1953.

\bibitem[5]{5} L. Comtet. Advanced Combinatorics: The Art of Finite
and Infinite Expansions, D. Reidel Publishing Company, 1974.

\bibitem[6]{6} E. N. Gilbert. Synchronization of Binary Messages.
\emph{IRE Trans. Inform. Theory}, vol. IT-6, 470-477, 1960.

\bibitem[7]{7} T. Harju, D. Nowotka. Counting bordered and
primitive words with a fixed weight. \emph{Theoretical Computer
Science}: 340 (2005) 273-279.

\bibitem[8]{8} M. Lothaire. Combinatorics on Words. Encyclopedia of
Mathematics and its Applications, Vol. 17, Addison-Wesley
Publishing Co., Reading, MA, 1983.

\bibitem[9]{9} J. L. Massey. Optimun frame synchronization.
\emph{IEEE Transactions on Commununications}, vol. COM-20,
115-119, February 1972.

\bibitem[10]{10} P. T. Nielsen. On the Expected Duration of a Search
for a Fixed Pattern in Random Data. \emph{IEEE Trans. Inform.
Theory}, vol. IT-29, 702-704, September 1973.

\bibitem[11]{11} P. T. Nielsen. A Note on Bifix-Free Sequences.
\emph{IEEE Trans. Inform. Theory}, vol. IT-29, 704-706, September
1973.

\bibitem[12]{12} N. J. A. Sloane. On-line encyclopedia of integer
sequences, http://oeis.org/.

\bibitem[13]{13} R. P. Stanley. Enumerative Combinatorics, volume 2.
Cambridge University Press, Cambridge, 1999.

\bibitem[14]{14} A. J. de Lind van Wijngaarden, T. J. Willink. Frame
Synchronization Using Distributed Sequences. \emph{IEEE
Transactions on Commununications}, vol. 48, No.12, 2127-2138,
2000.

\end{thebibliography}

\end{document}
