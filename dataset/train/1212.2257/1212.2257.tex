
\documentclass{elsarticle}
\usepackage{amsmath,amsfonts,amsthm}
\usepackage{amssymb,latexsym}

\theoremstyle{plain}
\newtheorem{theorem}{Theorem}[section]
\newtheorem{lemma}[theorem]{Lemma}
\newtheorem{proposition}[theorem]{Proposition}
\newtheorem{corollary}[theorem]{Corollary}

\theoremstyle{definition}
\newtheorem{mydefn}[theorem]{Definition}
\newtheorem{example}[theorem]{Example}
\newtheorem{observation}[theorem]{Observation}
\newtheorem{convention}[theorem]{Convention}
\newtheorem{rmk}{Remark}

\usepackage{mathrsfs}
\usepackage{amsfonts}
\usepackage{graphicx}
\usepackage{slashed}




\begin{document}

\author[nuaa]{Yan Zhang}

\author[nuaa]{Zhaohui Zhu\corref{cor}}

\author[nanshen]{Jinjin Zhang}

\author[nuaa]{Yong Zhou}

\cortext[cor]{Corresponding author. Email: zhaohui.nuaa@gmail.com, bnj4892856@jlonline.com (Zhaohui Zhu).}

\address[nuaa]{College of Computer Science, Nanjing University of Aeronautics and Astronautics, Nanjing, P.R. China, 210016}
\address[nanshen]{College of Information Science, Nanjing Audit University, \\Nanjing, P.R. China, 211815}


\title{A Process Calculus with Logical Operators\tnoteref{t1}}
\tnotetext[t1]{This work received financial support of the National Natural Science of China(No. 60973045) and Fok Ying-Tung Education Foundation.}
\date{\today}
\begin{abstract}

In order to combine operational and logical styles of specifications in one unified framework, the notion of logic labelled transition systems (Logic LTS, for short) has been presented and explored by L\"{u}ttgen and Vogler in [TCS 373(1-2):19-40; Inform. \& Comput. 208:845-867].
In contrast with usual LTS, two logical constructors  and  over Logic LTSs are introduced to describe logical combinations of specifications.
Hitherto such framework has been dealt with in considerable depth, however, process algebraic style way has not yet been involved and the axiomatization of constructors over Logic LTSs is absent.
This paper tries to develop L\"{u}ttgen and Vogler's work along this direction.
We will present a process calculus for Logic LTSs (CLL, for short).
The language CLL is explored in detail from two different but equivalent views.
Based on behavioral view, the notion of ready simulation is adopted to formalize the refinement relation, and the behavioral theory is developed.
Based on proof-theoretic view, a sound and ground-complete axiomatic system for CLL is provided, which captures operators in CLL through (in)equational laws.
\end{abstract}

\begin{keyword}
  Logic operators \sep Process calculus \sep Ready simulation \sep Axiomatic system \sep Logic labelled transition system
\end{keyword}


\maketitle


\section{Introduction}

    Over the past three decades, a lot of approaches have been proposed to formally specify and reason about reactive systems. Process algebra \cite{Handbook} and temporal logic \cite{Pnueli} are two popular paradigms of them.

    In process-algebraic paradigm, a system specification and its implementation usually are formulated in the same notation, and the underlying semantics are often given operationally.
    The notion of refinement is adopted to capture the correctness of implementation.

    In contrast, the temporal-logic paradigm adopts the language of temporal logics to formulate specifications abstractly, and implementations are described in terms of operational notations \cite{Pnueli}.
    Usually, model checking technique is used to establish that the system satisfies its specification \cite{ModelChecking}.

    Process-algebraic paradigm supports compositional reasoning (i.e., refinement of one part of a system does not depend on others), which is one of the most significant advantages of it.
    The merit of temporal-logic paradigm lies in its support for  abstract specifications, where relevant operational details may not be concerned.
    Traditionally, process-algebraic and temporal-logic formalisms are not mixed.

    In order to take advantage of both paradigms when designing systems, some theories for heterogeneous specifications have been proposed, e.g., \cite{Cleaveland00, Cleaveland02, Graf86,Kurshan94,Olderog}, which uniformly integrate refinement-based and temporal-logic specification styles.
    Among them,  Cleaveland and L{\"u}ttgen present a semantic framework for heterogenous system design based on B{\"u}chi automata and labelled transition systems augmented with an unimplementability predicate, and adopt Nicola and Hennessy's must-testing preorder \cite{Nicola83} to describe refinement relation \cite{Cleaveland00, Cleaveland02}.
    However, such refinement relation is not a precongruence \cite{Cleaveland00}.
    Hence, it does not support compositional reasoning.
    Moreover, in such framework, the operator conjunction lacks the desired property that  is an implementation of the specification  if and only if  implements both  and  \cite{Cleaveland02}.

    Recently, L{\"u}ttgen and Vogler propose the notion of Logic LTS, which combines operational and logical styles of specification in one unified framework \cite{Luttgen07,Luttgen10}.
    Roughly speaking, a Logic LTS is a labelled transition system with an inconsistency predicate on states.
    A few of constructors over Logic LTSs are introduced, which include operational constructors, such as CSP-style parallel composition and hiding, and logic constructors conjunction and disjunction.
    This framework allows one to freely mix operational operators and logic operators, while most early heterogenous specifications couple them loosely and do not allow for mixed specification \cite{Boudol92,Dam}.
    Moreover, the drawbacks in \cite{Cleaveland00,Cleaveland02} mentioned above have been remedied by using ready-tree semantics \cite{Luttgen07}.
    In order to support compositional reasoning when introducing the parallel constructor over Logic LTSs, a kind of modified ready simulation is adopted to describe the refinement relation \cite{Luttgen10}.
    Some standard temporal logic operators, such as \textquotedblleft always\textquotedblright and \textquotedblleft unless\textquotedblright, are also integrated into this framework \cite{Luttgen11}.

    \begin{figure}
    \begin{center}
        \includegraphics[scale=.6]{conjunction9}
    \end{center}
    \caption{Conjunction}\label{F:CONJUNCTION}
    \end{figure}

    Conjunction is a distinctive constructor in the framework of Logic LTSs, we give a simple example to illustrate how it works in heterogenous system design. In Figure~\ref{F:CONJUNCTION},  represents a specification which involves actions  and , and  represents an additional constraint which requires that the action  cannot occur in the implementation.
    Then, the dashed line in the Logic LTS  represents a specification which refines  and satisfies the constraint .
    Briefly speaking, if a state in  has -derivative, then the constructor conjunction turns this state into inconsistent state, i.e., unimplementable state.
    Thus, such state cannot be reached at run-time.
    The reader may refer to \cite{Luttgen10} for the concrete construction of .

    Up to now, Logic LTSs have been explored deeply, however, term-based way has not yet been involved, and the axiomatization of constructors is absent.
    This paper intends to develop L\"{u}ttgen and Vogler's work along the direction suggested by them in \cite{Luttgen10}.
    We present a process calculus for Logic LTSs (CLL, for short). In addition to prefix , external choice  and parallel operator , CLL contains logical operators  and  over process terms, which correspond to the constructors conjunction and disjunction over Logic LTSs respectively.
    Follows \cite{Luttgen10}, a variant of the usual notion of ready simulation is adopted to formalize the refinement relation.
    Moreover, an axiomatic system  is provided to characterize the operators in CLL in terms of (in)equational laws, and the soundness and ground-completeness w.r.t ready simulation are established.


    The remainder of this paper is organized as follows. The next section recalls some related notions.
    Section~3 introduces SOS rules of CLL.
    In section~4, based on reducing technique, the existence and uniqueness of stable transition model for CLL is demonstrated, and a few of  basic properties of the LTS associated with CLL are explored.
    Section~5  develops a behavioral theory of CLL.
    Section~6 provides an axiomatic system for CLL, and the soundness and ground-completeness are showed.
    Finally, a brief conclusion and discussion are given in Section~7.

\section{Preliminaries}

\subsection{Logic LTS}

In this subsection, we introduce some useful notations and recall the definition of Logic LTS.

Let  be the set of visible action names ranged over by letters , , etc., and let  denote  ranged over by  and , where  represents invisible actions.
A labelled transition system with inconsistency predicate is a quadruple , where  is a set of states,  is the transition relation and  is the set of all inconsistent states.

As usual, we write  if .
 is called an -derivative of  if .
We write  (or, ) if  (, respectively).
 stands for the ready set  of .
A state  is said to be stable if it cannot engage in any -transition, i.e., .
Some useful decorated transition relations are listed below.

 iff  and .

 iff , where  is the transitive reflexive closure of .

 iff .

 (or, ) iff   (, respectively) and all states along the sequence, including  and , are not in .

 (or, ) iff  (, respectively) and  is stable.

\begin{mydefn}[Logic LTS \cite{Luttgen10}]\label{D:LLTS}
  An LTS  is said to be a Logic LTS if, for each ,

\noindent\textbf{(LTS1) } if ,

\noindent\textbf{(LTS2)}  if .
\end{mydefn}

The condition (LTS1) formalizes the backward propagation of inconsistencies, and (LTS2) captures the intuition that divergence (i.e., infinite sequences of -transitions) should be viewed as catastrophic.
For more motivation behind (LTS1) and (LTS2), the reader may refer to \cite{Luttgen10}.

\begin{mydefn}[-pure  \cite{Luttgen10}]
An LTS  is said to be {}-pure if, for each ,  implies .
\end{mydefn}

Hence, for any state  in a -pure LTS, either  or .
Following \cite{Luttgen10}, this paper will focus on -pure Logic LTSs.


\subsection{Transition System Specification}
Structural Operational Semantics (SOS) is proposed by G.~Plotkin in \cite{Plotkin81}, which is a logical method of giving operational semantics.
The basic idea behind SOS is to describe the behavior of a program in terms of the behavior of its components.
Thus, SOS provides a syntax oriented view on operational semantics.
Transition System Specifications (TSS's), as presented by Groote and Vaandrager in \cite{Groote92}, are formalizations of SOS.
This subsection recalls basic concepts related to TSS.
Further information on this issue may be found in \cite{Aceto01,Bol96,Groote92}.

 Let  be an infinite set of variables and   a signature. The set of -terms over , denoted by , is the least set such that (I)  and (II) if  and , then , where  is the arity of .  is used to abbreviate ,  is abbreviated by , elements in  are called closed or ground terms.

    A substitution  is a mapping from  to .  As usual, a substitution  is lifted to a mapping  by  for n-arity  and . A substitution is said to be closed if it maps all variables to ground terms.


    A TSS is a quadruple , where  is a signature,  is a set of labels,  is a set of predicate symbols and  is a set of rules.
    Positive literals are all expressions of the form  or , while negative literals are all expressions of the form  or , where ,  and .
A literal is a positive or negative literal, and , ,  are used to range over literals.
    A literal is said to be ground or closed if all terms occurring in it are ground.
    A rule  has the form like , where , the premises of the rule , denoted as , is a set of literals, and , the conclusion of the rule , denoted as , is a positive literal. Furthermore, we write  for the set of positive premises of  and  for the set of negative premises of . A rule  is said to be positive if . A TSS is said to be  positive if it has only positive rules. A rule is said to be an axiom if its set of premises is empty. An axiom  is often written as .
    Given a substitution  and a rule ,  is the rule obtained from  by replacing each variable in  by its -image, that is, .
    Moreover, if  is closed then  is said to be a ground instance of .



\begin{mydefn}[Proof in Positive TSS]
    Let  be a positive TSS. A proof of a closed positive literal  from  is a well-founded, upwardly branching tree, whose nodes are labelled by closed literals, such that\\
    --- the root is labelled with ,\\
    --- if  is the label of a node  and  is the set of labels of the nodes directly above , then there is a rule  in  and a closed substitution  such that  and  for each .

    If a proof of  from  exists, then  is said to be provable from , in symbols .
\end{mydefn}


\section{Syntax and SOS Rules of CLL}


The process terms in CLL are defined by BNF below:
 where  and  .
We denote  as the set of all process terms. We shall always use  to mean that the expressions  and  are syntactically identical.

As usual, 0 is a process that can do nothing.
The prefix  has a single capability, expressed by ; the process  cannot proceed until  has been exercised.
 is an external choice operator.
 is a CSP-style parallel operator,  represents a process that behaves as  in parallel with  under the synchronization set .
 represents an inconsistent process with empty behavior.
 and  are logical operators, which are intended for describing logical combinations of processes.

We now provide SOS rules to specify the behavior of process terms formally.
These rules reflect L\"{u}ttgen and Vogler's constructions in \cite{Luttgen10} in process algebraic style.
Unless noted, let ,  and .
All SOS rules are divided into two parts: Table~\ref{Ta:OPERATIONAL_RULES} consists of operational rules , and Table~\ref{Ta:PREDICATIVE_RULES} contains predicate rules .

\begin{table}[ht]
\begin{center}
\fbox{}
\caption{Operational Rules\label{Ta:OPERATIONAL_RULES}}
\end{center}
\end{table}

Negative premises in rules , ,  and  give -transition precedence over visible transitions, which guarantees that the transition model of CLL is -pure (see, Section~4).
Rules  and  illustrate that the operational aspect of  is same as internal choice in usual process calculus.
The rule  reflects that conjunction operator is a synchronous product for visible transitions.
The rules about other operators are usual.

\begin{table}[ht]
\begin{center}
\fbox{}
\caption{Predicate Rules}\label{Ta:PREDICATIVE_RULES}
\end{center}
\end{table}

Predicate rules in Table~\ref{Ta:PREDICATIVE_RULES} specify the properties of the predicate .
In CLL, although both  and  have empty behavior, they represent different processes.
The rule  says that  is inconsistent.
Thus, the process  cannot be implemented.
While  is consistent(see Lemma~\ref{L:F_NORMAL}(5)), which is an implementable process.
The rule  reflects that if both two disjunctive parts are inconsistent then so is the disjunction.
Rules  describe the system design strategy that if one part is inconsistent, then so is the whole composition.
The rules  and  reveal that the conjunction is inconsistent if its conjuncts have different ready sets.
The rule below formalizes (LTS1) in Def.~\ref{D:LLTS} for process terms of the form .

However, the universal quantifier occurs in it explicitly.
We adopt the method presented in \cite{Verhoef95} to eliminate universal quantifier.
For this purpose, we introduce auxiliary predicate symbol  for each  and rules  and .
Intuitively, by the rule ,  states that  has a consistent -derivative.
 and  together say that a conjunction is inconsistent if it can engage in -transition and all its -derivatives are inconsistent. A formal result concerning this will be given in Section~4 (see, Lemma~\ref{L:LLTS_I}).

Summarily, the TSS for CLL is , where ---,\\
---, and \\
---.



\section{Operational Semantics of CLL}
A natural and simple method of describing the operational nature of processes is in terms of LTSs. Given a TSS, an important problem is how to associate LTSs with process terms. For positive TSS, the answer is straightforward.
However, this problem is non-trivial for TSS containing negative premises. This section will consider the well-definedness of the TSS  (i.e., existence and uniqueness of the LTS induced by ), and explore basic properties of the induced transition model.
\subsection{Basic Notations}

Let  be a set of labels and    a set of predicate symbols. A transition model  is a subset of , where  and , elements  and  in  are written as  and  respectively.


     A positive closed literal  holds in  or  is valid in , in symbols , if . A negative closed literal   (or, ) holds in , in symbols  (, respectively), if there is no  such that (, respectively). For a set of closed literals , we write  iff  for each .

\begin{mydefn}
Let  be a TSS and  a transition model.
     is said to be a model of  if, for each  and  such that , we have .
     is said to be supported by  if, for each , there exists  and  such that  and .
     is said to be a supported model of  if  is supported by  and  is a model of .
\end{mydefn}

It is well known that every positive TSS  has a least transition model, which exactly consists of provable transitions of  \cite{Bol96}.
However, since it is not immediately clear what can be considered a \textquotedblleft proof\textquotedblright for a negative formula, it is much less trivial to associate an LTS with a TSS containing negative premises \cite{Groote93}.
The first generic answer to this question is formulated in \cite{Bloom95,Groote93}, in which the above notion of supported model is introduced.
However, this notion doesn't always work well. Several alternatives have been proposed, and a good overview on this issue is provided in \cite{Glabbeek04}.
In the following, we recall the notions of stratification and stable transition model, which play an important role in this field.

\begin{mydefn}[Stratification \cite{Aceto01,Bol96}]\label{D:STRATIFICATION}
     Let  be a TSS and  an ordinal number. A function  is said to be a stratification of  if, for every rule  and every substitution , the following conditions hold.\\
    (1)  for each ,\\
    (2)  for each , and\\
    (3)  for each  and .

    A TSS is said to be stratified iff there exists a stratification function for it.
\end{mydefn}

In the following, given a set  of rules, we denote  the set of all ground instances of rules in . Similarly, given a rule ,  is the set of all ground instances of it.

\begin{mydefn}[Stable Transition Model  \cite{Bol96,Gelfond88}]\label{D:STABLE}
    Let  be a TSS and  a transition model.  is said to be a stable transition model for  if
    
    where
     is the TSS  with
    
    and  is the least transition model of the positive TSS .
\end{mydefn}

The above notion provides a reasonable semantics for TSS's with negative premises.
TSS's that do not have a unique stable model should be ruled out and considered pathological \cite{Bol96}.
As is well known, stable models are supported models, and each stratified TSS  has a unique stable model \cite{Bol96}, moreover, such stable model does not depend on particular stratification function \cite{Groote93}. Unfortunately, the TSS  in Section~3 cannot be stratified.

\begin{observation}
  cannot be stratified.
\end{observation}
\begin{proof}
    Assume that  is a stratification of . Let . By Def.~\ref{D:STRATIFICATION}, the rule  entails . However, the rule  requires that  holds. Hence, the stratification of  does not exist.
\end{proof}

In \cite{Bol96}, the notion of positive after reduction (also called complete in \cite{Glabbeek04}) is introduced as a criterion for well-definedness of the semantics and is shown to be more general than stratification.
We shall adopt such reducing technique to obtain a stable transition model of .
Related concepts and results are recalled below.

\begin{mydefn}\label{D:REDUCTION}
    Let  be a TSS and ,  be transition models. The reduction TSS of  w.r.t  and  is defined as
    
    where  is the set of all rules  such that there exists  satisfying
    \begin{enumerate}
      \item ,
      \item , and
      \item .
    \end{enumerate}
    In such case, we will say that  originates from . Clearly,
     is ground, that is, its all rules are ground.
\end{mydefn}


\begin{mydefn}
    Let  be a ground TSS.\\
    ---, where .\\
    ---, where
    
\end{mydefn}

\begin{mydefn}
    Let  be a TSS. For every ordinal number , the -reduction of , in symbols , is recursively defined below\\
    ---, \\
    ---.
\end{mydefn}
In the above,  (or, ) is the least transition
    model of the positive TSS  (, respectively).
A useful result about reducing technique is cited below.

\begin{theorem}\label{C:stable}
    Let  be a TSS. Suppose that  is a stratification of  for some ordinal  and . Then, the stable transition model of  is the unique stable transition model of .
\end{theorem}
\begin{proof}
  See Theorem 6.1 and Corollary 6.5 in \cite{Bol96}.
\end{proof}

\subsection{Stable Transition Model of }


    This subsection will adopt the reducing technique recalled above to show that  has a unique stable model. It turns out that 1-reduction is enough to realize our aim.


    Firstly, we reduce . By the definition of -reduction, we have
    
    and
    
    where

\noindent    \textbf{(1)}  and  are the least models of the positive TSS's   and  below respectively,

where  and , and

where  and .

\noindent  \textbf{(2)}  is the set of all rules  such that there exists  satisfying
    \begin{enumerate}
      \item [2.1.] ,
      \item [2.2.] , and
      \item [2.3.] .
    \end{enumerate}

Next, we will show that  is stratified.
For this purpose, a few preliminary definitions are needed.
As usual, the degree of process terms is defined below.
\begin{enumerate}
  \item ,
  \item  for each ,
  \item  for each .
\end{enumerate}

Further, the function  from   to  is defined as
\begin{enumerate}
  \item ,
  \item , and
  \item  for each .
\end{enumerate}

In order to show that  is a stratification of , we shall first give the result below.

\begin{lemma}\label{L:POS}
    If  then .
\end{lemma}
\begin{proof}
Let .
 We proceed by induction on the depth of the proof tree of .

For the induction basis, the only rule applied in the proof tree is an axiom.
Clearly, such axiom originates from ,  or .
It is a simple matter to check one by one.
For instance, if  is inferred by the ground rule in , then ,  and . Obviously, .

The induction step proceeds by distinguishing fifteen cases based on the format of the last rule applied in the proof tree.
It is routine, and we deal with two cases as examples and leave the remainder to the reader.\\

\noindent Case 1 .

Hence, we get , , ,  and . By induction hypothesis (IH, for short),  and .
Then, it immediately follows that .\\

\noindent Case 2 .

Clearly, this rule originates from one in , where .
Thus, we have , ,  and .
By IH, we get .
So, .
\end{proof}

\begin{lemma}\label{L:Stratification}
     is a stratification of .
\end{lemma}
\begin{proof}


    We want to prove that, for each rule in ,  satisfies the items (1), (2) and (3) in Def.~\ref{D:STRATIFICATION}.
    Let .
    We distinguish cases based on the origin of . Here, we only consider three cases as illustrations, the remainder is similar and omitted.\\

\noindent Case 1  originates from a rule in .


    Then,  and  for some  and .
    Clearly, it suffices to consider the case where .
    It immediately follows from the definition of  that  and, for any , , as desired.\\

\noindent Case 2  originates from a rule in .

    Similar to Case 1, it is enough to deal with the case where  has the form  with .
By Lemma~\ref{L:POS}, .
    Then, by the definition of , we have 
    and .   \\

\noindent Case 3  originates from a rule in .

    Similarly, we treat the rule with the form .
It immediately follows that  and , as desired.
\end{proof}

We now arrive at the main result of this subsection.

\begin{theorem}
     has a unique stable transition model.
\end{theorem}
\begin{proof}
Follows from Theorem~\ref{C:stable} and Lemma~\ref{L:Stratification}.
\end{proof}

\noindent\textbf{Notation}
Henceforward the unique stable transition model of  is denoted by .  \\

The LTS associated with CLL is defined below.

\begin{mydefn}
    The LTS associated with CLL, in symbols , is the quadruple
    , where \\
    ---  iff ,\\
    ---  iff .
\end{mydefn}

Since  is a stable transition model, which exactly consists of provable transitions of the positive TSS , the result below follows.

\begin{theorem}
    Let  and .
  \begin{enumerate}
     \item The following are equivalent:

            1.1. ,

            1.2. ,

            1.3. ,

            1.4. .

     \item The following are equivalent:

            2.1. ,

            2.2. ,

            2.3. ,

            2.4. .

   \end{enumerate}
\end{theorem}
\begin{proof}
  Straightforward.
\end{proof}

The above theorem is trivial but useful. It provides a way to prove the properties of  and . That is, we can demonstrate some conclusions by proceeding induction on the depth of proof trees in the positive TSS .  In the remainder of this paper, we will apply this theorem without any reference.

\subsection{Basic Properties of }

This subsection will provide a number of simple properties of . In particular, we will show that  is a Logic LTS and it is -pure.
Some useful notations are listed below.

 iff  with .

 iff .

 iff .

 (or, ) iff   (, respectively) and all process terms  occurring in this sequence, including  and , are not in .

 (or, ) iff  (, respectively) and  is stable.

A few simple properties of transition relation  are listed in the next three lemmas, which will be frequently used in subsequent sections.

\begin{lemma}\label{L:Basic_I}
Let  and . Then
    \begin{enumerate}
      \item  iff  and . Hence,  for any .
      \item  iff , and either  or .
      \item  and .
      \item If  then . Hence, there is no infinite transition sequence in .
    \end{enumerate}
\end{lemma}
\begin{proof}
  \noindent \textbf{(1)} (Left implies Right) Assume . Then, 
  Further, since the axiom  is the only rule whose conclusion has the head , we get
    and .\\

  \noindent (Right implies Left) Since  is an axiom in , we immediately get . Hence, . \\

  \noindent \textbf{(2)} Similar to (1). \\

  \noindent \textbf{(3)} Assume that  for some  and .
  Since  is a supported transition model, there exists a rule  such that  and . However, there is no such ground rule, a contradiction.
  Similarly,  holds for each .\\

  \noindent \textbf{(4)} By Lemma~\ref{L:POS}, it follows from the fact that
   .
\end{proof}

The properties in the above lemma hold for any kind of transitions (visible or invisible). The next lemma contains some simple properties which hold only for visible transitions.

\begin{lemma}\label{L:Basic_II}
Let  and . Then
    \begin{enumerate}
      \item   iff  either  and , or  and .
          In particular,  iff  either  and , or  and .
      \item  iff ,  and  for some .
    \end{enumerate}
\end{lemma}
\begin{proof}
  \noindent \textbf{(1)} (Left implies Right) Suppose .
  Clearly, the last rule applied in the proof tree of   is of the form
  
  Then, we get either ,  and , or ,  and .\\

  \noindent (Right implies Left) W.l.o.g, suppose  and .
  Since , we get .
  So,  immediately follows from .\\

  \noindent \textbf{(2)} We shall prove that the left implies the right, the converse is trivial and omitted.
   Suppose .
   Since , the last rule applied in the proof tree of  is of the form  for some , .
   Thus, it follows that ,  and .
\end{proof}

The lemma below shows that the operators ,  and  are static combinators w.r.t -transitions, in other words, the structure that they represent is undisturbed by -transitions.


\begin{lemma}\label{L:STABILIZATION}
Let  and .
Then,  iff there exists  such that either  and , or  and .
Hence,  and  are stable iff  is stable.
\end{lemma}
\begin{proof}
(Left implies Right) Proceed by distinguishing cases based on the form of the last rule applied in the proof tree of .\\

\noindent (Right implies Left) It immediately follows from the fact that both  and  are rules in .
\end{proof}

The next lemma provides some basic properties of .

\begin{lemma}\label{L:F_NORMAL}
  Let . Then
  \begin{enumerate}
    \item   iff .
    \item  iff .
    \item Either  or  iff   for each .
    \item Either  or  implies .
    \item  and .
  \end{enumerate}
\end{lemma}
\begin{proof}
We prove items (1) and (5), the others are similar and omitted.

\noindent \textbf{(1)} Assume that . So, both  and  are in . Further, it follows from  that .

Conversely, assume that . Then, the last rule applied in the proof tree of  is . So, we get  and .\\

\noindent \textbf{(5)}  immediately follows from . Next, we prove . Assume that . Since  is a supported transition model, there exists a rule  such that  and . However, there is no rule which has the conclusion , a contradiction.
\end{proof}

An immediate consequence of the above lemma is that the operators , ,  and  preserve consistency. More precisely,  and  are not in  if , where  is any binary operator except . Similar to conjunction in usual logic systems, the operator  doesn't preserve consistency. That is, the converse of (4) in the above lemma fails. For instance, consider the process terms  and . Clearly, by (2) and (5) in the above lemma,  and .
However, by Lemma~\ref{L:Basic_I}(1) and \ref{L:STABILIZATION}, we have , hence .
Then, it is easy to see that .
In fact, the above lemma doesn't completely capture the nature of , and further properties will be revealed in the subsequent.

\begin{lemma}\label{L:CON_LLTS}
    Let  and . If  then
     for each  such that .
\end{lemma}
\begin{proof}
   Assume  that  with .
   So, , which implies . Further,  comes from , which contradicts .
\end{proof}

As mentioned in Section~2, the notion of -pure is introduced in \cite{Luttgen07, Luttgen10}, which is a technique constraint for Logic LTSs. Follows \cite{Luttgen07, Luttgen10}, this paper insists such constraint. SOS rules in Table \ref{Ta:OPERATIONAL_RULES} have reflected it, while the result below will formally show that the LTS associated with CLL is indeed -pure.

\begin{theorem}\label{L:TAU_PURE}
     is -pure.
\end{theorem}
\begin{proof}
It suffices to prove that for each ,
  
We prove it by induction on the structure of .


\noindent   or .

        By Lemma~\ref{L:Basic_I}(3), it holds trivially.

  \noindent  .

          Assume that  and  for some .
         So, by Lemma~\ref{L:Basic_I}(1), we have , a contradiction.

  \noindent  .

        Immediately follows from Lemma~\ref{L:Basic_I}(2).

  \noindent  .

        Assume that .
        Then, by Lemma~\ref{L:STABILIZATION}, we get  for some . Further, by IH and Lemma~\ref{L:Basic_II}(1), it follows that  for each .

  \noindent  .

        Similar to , but using Lemma~\ref{L:Basic_II}(2) instead of Lemma~\ref{L:Basic_II}(1).

  \noindent  .

        Assume that .
         Then, by Lemma~\ref{L:STABILIZATION}, we get  for some .
         By IH, we have  for each .
         Assume that  for some  and .
         Then, the last rule applied in the proof tree of  has one of formats below.
         \begin{enumerate}
           \item  with  and ,
           \item  with  and ,
           \item  with .
         \end{enumerate}
        It is trivial to check that each case  above leads to a contradiction, as desired.
\end{proof}

In the following, we shall prove that  is a Logic LTS. We proceed by proving that both (LTS1) and (LTS2) hold in .

\begin{lemma}\label{L:LLTS_I}
    satisfies (LTS1).
\end{lemma}
\begin{proof}
 It is enough to show that for each 
    
    We prove it by induction on the structure of .


  \noindent  , ,  or .

        Immediately follows from Lemma~\ref{L:F_NORMAL}(1)(2) and Lemma~\ref{L:Basic_I}(1)(2)(3).
        Notice that, for  or , it holds trivially due to .

  \noindent  .

        Assume that  for some . Since , we get  for some .
        Consider two cases below.\\

       \noindent Case 1 .

        By Lemma~\ref{L:Basic_II}(1), we have
        
        W.l.o.g, we consider the first alternative.
        In such case, it follows from Lemma~\ref{L:Basic_II}(1) that
        
        Thus, by the assumption, it holds that
        
        So, by IH, .
        Then,  immediately follows from Lemma~\ref{L:F_NORMAL}(3).\\

       \noindent Case 2 .

        So, by Lemma~\ref{L:STABILIZATION}, it follows that either  or .
        W.l.o.g, we consider the former.
        If  then  comes from Lemma~\ref{L:F_NORMAL}(3) at once.
        In the following, we deal with the case .
        Firstly, it follows from Lemma~\ref{L:STABILIZATION} that
        
        Then, by the assumption, we get
        
        Further, by  and Lemma~\ref{L:F_NORMAL}(3), it holds that
         
        So, by IH, . Therefore, by Lemma~\ref{L:F_NORMAL}(3), it follows that , as desired.


  \noindent  .

        Assume that  and  for each  such that .
        Thus,  for some .
        If (i.e., ) then , which, with the helping of , implies . Therefore, in order to complete the proof, it is enough to show that .

        Suppose that . Clearly, the last rule applied in the proof tree of   has the form below
        
        So, . Then, by the assumption, we get , which contradicts .

  \noindent  .

        Assume that  for some .
        Since , we have  for some .
         If  then the proof is similar to one of , we omit it.
         In the following, we consider the case where . In such case, the last rule applied in the proof tree of  has one of the following three formats.\\

        \noindent Case 1  with .

        Then,  and .
        If  then, by Lemma~\ref{L:F_NORMAL}(3), , as desired.
        Next, we consider another case where .
        Since  and , we get
        
        Then, it follows that
        
        Moreover, by the assumption, we have
         
         Further, since , by Lemma~\ref{L:F_NORMAL}(3), it holds that
         
         So, by IH, .
         Then, by Lemma~\ref{L:F_NORMAL}(3),  immediately follows.\\


       \noindent Case 2  with .

        Similar to Case 1.\\

       \noindent Case 3 .

        In such case, we get  and . Thus,  and . In order to complete the proof, it is enough to show that either  or .
        Assume that  and .
        Then, by IH, we get  with  and  with  for some  and .
        Since , it follows that 
         Hence, .
         Further, by assumption, we have .
         By Lemma~\ref{L:F_NORMAL}(3), this contradicts  and .
\end{proof}

In fact, (LTS1) can be strengthened so as to provide a complete character of non-stable processes in the set  in terms of -transitions.
Formally, we have the result below.

\begin{lemma}\label{L:F_TAU_I}
 If  then
  
\end{lemma}
\begin{proof}
It immediately follows from Lemma~\ref{L:LLTS_I} that the right implies the left.
Another implication can be proved by induction on the structure of , we leave it to the reader.
\end{proof}

As an immediate consequence, we also have, for each ,

Although this result characterizes all processes in , it isn't more interesting than Lemma~\ref{L:F_TAU_I}.
In fact, for stable process terms, this result is trivial.
For non-stable process terms, since there is no infinite -transition sequence in , it isn't difficult to see that this result is implied by Lemma~\ref{L:F_TAU_I}.

\begin{lemma}\label{L:LLTS_II}
     satisfies (LTS2).
\end{lemma}
\begin{proof}
    It suffices to show that, for each , if  then  for some .
   We prove it by induction on the degree of .
   Assume that it holds for all  with .

   If  is stable, then  follows from .
   Otherwise, since  and , by Lemma~\ref{L:LLTS_I}, we have  for some .
   By Lemma~\ref{L:Basic_I}(4), we get . Then, by IH, we obtain   for some . Hence, .
\end{proof}

It is now a short step to

\begin{theorem}\label{L:LLTS}
     is a Logic LTS.
\end{theorem}
\begin{proof}
    Immediately follows from Lemma~\ref{L:LLTS_I} and \ref{L:LLTS_II}.
\end{proof}

In the next section, the relation  will play an important role in developing the behavioral theory of CLL. The remainder of this section is devoted to basic properties of it.

Lemma~\ref{L:STABILIZATION} asserts that, for operators ,  and , the structure that they represent is preserved under -transitions.
A possible conjecture is that such property may be generalized to the circumstance where the stability and consistency of process terms are involved, that is, these operators are also static combinators w.r.t the transition relation  .
It turns out that this conjecture almost holds except that we need to add a moderate condition when considering the operator . Formally, we have the result below.

\begin{lemma}\label{L:TAU_I}
  Let  and .
    \begin{enumerate}
      \item If  and , then   for ,
           moreover,  if .
      \item If  then ,  and  for some .
    \end{enumerate}
\end{lemma}
\begin{proof}
\noindent\textbf{(1)} We consider the case where , the others can be treated similarly and omitted.

Suppose ,  and .
So, we have  for some  with , and  for some  with .
    By Lemma~\ref{L:STABILIZATION}, it is easy to see that
    
    If  then both  and  are stable and  for .
    So,  holds trivially.
    Next, we consider the case either  or .
    Then, by Lemma~\ref{L:STABILIZATION},  for each process term  except  along the sequence (\ref{L:TAU_I}.1).
    Further, since , by Lemma~\ref{L:F_TAU_I}, it is easy to know that   for each  occurring in (\ref{L:TAU_I}.1).
    Moreover, since  and  are stable, by Lemma~\ref{L:STABILIZATION}, so is .
    Therefore, .\\

\noindent \textbf{(2)}
    We shall treat the case where , the remaining proofs are similar and omitted.

    Suppose . Then, for some  and , we have
    
     In the following, we consider the non-trivial case . In such case, by Lemma~\ref{L:STABILIZATION}, there exist  such that
     \begin{enumerate}
       \item   for each  ,
       \item either  and , or   and   for each .
     \end{enumerate}

     For each , since , by Lemma~\ref{L:F_NORMAL}(4), we have . Moreover, by Lemma~\ref{L:STABILIZATION}, since  is stable, so are  and . Finally, we can obtain a -transition sequence from  to  by removing duplicate process terms in the sequence  \footnote{Notice that this -transition sequence is empty if , i.e.,  is stable.}.
     So, .
     Similarly,  holds.
\end{proof}

By the way, if the consistency is ignored, similar to the result above, it is not hard to show that, for each ,

\begin{enumerate}
  \item   If  and , then .
  \item  If  then ,  and  for some .
\end{enumerate}

We conclude this section with a useful result, which will be used in Section~5 when we deal with distributive laws.

\begin{lemma}\label{L:DIS}
    Let . If   then there is  and a sequence  () such that ,  and  for some  and .
\end{lemma}
\begin{proof}
   We consider the case where , the others are handled in a similar way.
   Let .
   Since , we get .
   So, . We proceed by induction on .


   For the inductive basis , since  is stable, by Lemma~\ref{L:STABILIZATION} and \ref{L:Basic_I}(2), the last rule applied in the proof tree of  is
   
   W.l.o.g, we consider the first alternative.
   In such case,  and we get the sequence .
   Hence, by Lemma~\ref{L:STABILIZATION},  is stable.
   Moreover, since , by Lemma~\ref{L:F_NORMAL}(4), we  have .
   So, .  Then,  and  are what we need.

   For the inductive step, assume .
   So,  for some .
   We distinguish two cases based on the form of the last rule applied in the proof tree of .\\

   \noindent Case 1 .

    So,  and .
    By IH, there exists a sequence
   
    and for some  and , we have
    \begin{enumerate}
      \item  ,
      \item  and
      \item either  or .
    \end{enumerate}
   Moreover, by Lemma~\ref{L:F_NORMAL}(4),  and  follow from  and , respectively.
   Then, we have the sequence below
   
   and ,  and either  or , as desired.\\

   \noindent  Case 2 .

    So,  and .
    Moreover, by Lemma~\ref{L:Basic_I}(2), either  or . Clearly, we have the sequence , and,  and  are what we need.
\end{proof}


\section{Behavioral Theory of CLL}

This section will develop the behavioral theory of CLL. Follows \cite{Luttgen10}, the notion of ready simulation below is adopted to formalize the refinement relation among process terms, which is a modified version of the usual notion of ready simulation (see, e.g., \cite{Glabbeek01}).
We will only care about the properties of ready simulation which are needed in establishing the soundness of the axiomatic system  in the next section.

\begin{mydefn}[Ready simulation \cite{Luttgen10}]\label{D:READYSIMULATION_TERMS}
A relation  is a stable ready simulation relation, if for any  and \\
\textbf{(RS1)} ,  stable;\\
\textbf{(RS2)}  implies ;\\
\textbf{(RS3)}  implies ;\\
\textbf{(RS4)}  implies .
\end{mydefn}

 We say that  is stable ready simulated by , in symbols , if there exists a stable ready simulation relation  with .
 Further,  is said to be ready simulated by , written , if 
 It is easy to see that both  and  are pre-order (i.e., reflexive and transitive). The equivalence relations induced by them are denoted by  and , respectively.
 That is,  The identity relation over stable process terms is denoted by .
 It is obvious that both  and  are stable ready simulation relations.


 The remainder of this section is taken up with proving (in)equational laws concerning  (, respectively).
 These laws capture inherent properties of composition operators, e.g., commutativity, associativity and zero element.
 A number of laws obtained by L\"{u}ttgen and Vogler in \cite{Luttgen10}, which are needed in the next section, will also be rephrased in process-algebraic style.
 In particular, we will show that the ready simulation is a precongruence, that is, it is preserved by all algebraic contexts.

We begin with the laws about  and  occurring in prefix construction.

\begin{proposition}[Prefix]\label{S:PREFIX}\hfill
    \begin{enumerate}
      \item .
      \item .
    \end{enumerate}
\end{proposition}
\begin{proof}
\noindent \textbf{(1) }  Since , by Lemma~\ref{L:F_NORMAL}(2), we have . So,  holds trivially.\\

\noindent \textbf{(2)} Firstly, we prove . Suppose . So, .
Then, we have  and .\\

Secondly, we prove .
Suppose .
Then, .
Moreover, by Lemma~\ref{L:F_NORMAL}(2),  follows from  .
Thus,  and , as desired.
\end{proof}

The second set of (in)equations focuses on the properties of the combinator .

\begin{proposition}[Disjunction]\label{S:DISJUNCTION}\hfill
    \begin{enumerate}
      \item .
      \item .
      \item .
      \item .
      \item .
    \end{enumerate}
\end{proposition}
\begin{proof}
\noindent \textbf{(1)} It is enough to prove .
    Suppose .
    It follows from Lemma~\ref{L:Basic_I}(2) that  for some .
    W.l.o.g, assume that .
    Since , by Lemma~\ref{L:F_NORMAL}(1), we get .
    Further, it follows from  that  and . Thus,  .\\

\noindent \textbf{(2)}, \textbf{(3)} Similar to (1).\\

\noindent \textbf{(4)} Firstly, we prove .
Suppose .
By Lemma~\ref{L:Basic_I}(2), we have  with either  or . Further, since  and , we get .
    Thus,  and .

Secondly, we prove .
    Suppose  .
    By Lemma~\ref{L:F_NORMAL}(1),  follows from .
    Further, since ,
    we obtain  and .\\

\noindent \textbf{(5)} Straightforward.
\end{proof}

In the following, we will consider (in)equational laws about the operators ,  and .
For convenience, we adopt the convention below:

\begin{convention}\label{C:STABLE_TERM}
  When treating (in)equations  (or, ) in Prop. \ref{L:EC2}, \ref{L:CON_ID_I}, \ref{S:CONJUNCTION} and \ref{S:PARALLEL}, we assume that  ranges over stable process terms for .
\end{convention}

The next group of equations is concerned with the operator .

\begin{proposition}[External Choice]\hfill\label{L:EC2}
    \begin{enumerate}
      \item .
      \item .
      \item .
      \item .
      \item .
      \item .
      \item .
      \item .
      \item .
    \end{enumerate}
\end{proposition}
\begin{proof}
    We will prove (1), (2) and (7) one by one, the other parts of the proposition are handled in a similar way and omitted.

   \noindent \textbf{(1)}
    It is enough to prove . Put
    

    We want to show that the relation  is a stable ready simulation relation. It suffices to prove that each pair in  satisfies (RS1)-(RS4). It is trivial for pairs in . In the following, we consider the pair . It immediately follows from Convention \ref{C:STABLE_TERM}, Lemma~\ref{L:STABILIZATION}, \ref{L:F_NORMAL}(3) and \ref{L:Basic_II}(1) that this pair satisfies (RS1), (RS2) and (RS4).

    \textbf{(RS3)} Suppose .
    Since  is stable,  for some .
    Then, by Lemma~\ref{L:Basic_II}(1),  we get .
    Moreover, by Lemma~\ref{L:F_NORMAL}(3),  comes from .
    So,  and .\\

    \noindent\textbf{(2)} We shall prove .
    Suppose .
    It follows from Lemma~\ref{L:TAU_I}(2) that
    
    So, by Lemma~\ref{L:TAU_I}(1), we get . Further, by item (1) in this lemma, we obtain .\\

    \noindent \textbf{(7)} Since , by Lemma~\ref{L:F_NORMAL}(3), we have   for each . Then,  holds trivially.
\end{proof}

In order to treat (in)equations concerning the operator , the next two lemmas are needed.

\begin{lemma}\label{L:RS_CON}
If ,  and , then .
\end{lemma}
\begin{proof}
    We prove it by induction on the degree of .
    Suppose that it holds for all  such that .
    Assume that ,  and .
    Thus, it follows at once that ,  and .
    Assume that .
    We distinguish cases based on the form of the last rule applied in the proof tree of .\\

\noindent Case 1 .

            Then, , which contradicts .\\

\noindent Case 2 .

            Analogous to Case 1.\\

\noindent Case 3  with .

            So,  and , which contradicts .\\

\noindent Case 4   with .

          Similar to Case 3.\\

\noindent Case 5  with .

        Thus, .
        Since both  and  are stable, by Lemma~\ref{L:STABILIZATION}, so is .
        Hence, .
        Then, by Lemma~\ref{L:Basic_II}(2),  due to .
        Further, by Lemma~\ref{L:LLTS_I}, it follows from  that  for some .
        Then, by Lemma~\ref{L:LLTS_II}, we have  for some .
        So, .
        Hence, it immediately follows from    and  that
        
        
        Since  and  are stable, it follows from (\ref{L:RS_CON}.1) and (\ref{L:RS_CON}.2) that, for ,  for some .
        So, by Lemma~\ref{L:Basic_II}(2), .
        Further, by Lemma~\ref{L:CON_LLTS} and , we have
        
        On the other hand, since  and , by IH, (\ref{L:RS_CON}.1) and (\ref{L:RS_CON}.2), we obtain .
        Further,  by Lemma~\ref{L:TAU_I}(1), it follows from   and   that , which contradicts (\ref{L:RS_CON}.3).
\end{proof}

\begin{lemma}\label{L:CON_ID_I}\hfill
     \begin{enumerate}
      \item  for .
      \item  for .
      \item If  and , then .
      \item If  and , then .
    \end{enumerate}
\end{lemma}
\begin{proof}
\textbf{(1)} We will prove . The proof of  is similar and omitted.
    Put
     
     By Convention~\ref{C:STABLE_TERM}, it is enough to prove that  is a stable ready simulation relation. Let .
    By Lemma~\ref{L:STABILIZATION} and \ref{L:F_NORMAL}(4), it immediately follows that the pair  satisfies (RS1) and (RS2). We will prove (RS3) and (RS4) below.

    \textbf{(RS3)} Suppose . Since  is stable, we get  for some .
    So, it follows from Lemma~\ref{L:F_NORMAL}(4) and \ref{L:Basic_II}(2) that
    
    Since , by Lemma~\ref{L:TAU_I}(2), we obtain ,  and   for some . Hence,  and .

    \textbf{(RS4)} Suppose . We want to prove .
    It is straightforward that .
    Next, we show .
    Let .
    Thus,  for some .
    Assume that . So,  comes from .
    Then,  immediately follows, a contradiction. \\

\noindent \textbf{(2)}
  Suppose . By Lemma~\ref{L:TAU_I}(2), there exist  such that  ,  and .
  By item (1) in this lemma, we have . Hence,  holds. Similarly,  holds.\\


\noindent \textbf{(3)} Put
        
We intend to show that  is a stable ready simulation relation. Let .
    By Lemma~\ref{L:STABILIZATION} and \ref{L:RS_CON}, it is easy to see that  satisfies both (RS1) and (RS2).

   \textbf{(RS3)} Suppose . Since  is stable,  for some .
   It follows from  that  and  for some .
    Similarly,  and  for some .
     Further, by Lemma~\ref{L:RS_CON}, we get  because of .
   Moreover, since  and  are stable,  and  for some .
   Then, it follows from Lemma~\ref{L:TAU_I}(1) that 
   Since   and , by Lemma~\ref{L:Basic_II}(2), we get 
    By Lemma~\ref{L:RS_CON},  due to .
    Then, by (\ref{L:CON_ID_I}.3.1) and (\ref{L:CON_ID_I}.3.2), we obtain  and  .

   \textbf{(RS4)} Suppose .
    It follows from  and  that .
    Further, since ,  and  are stable, we get  for .
    Hence, by Lemma~\ref{L:Basic_II}(2), we have  .\\

\noindent \textbf{(4)} Suppose .
    It is enough to find  such that  and .
    Clearly, it follows from  that  and  for some . Similarly,  and  for some .
    We will check that  is exactly what we need.
    By item~(3) in this lemma, we get .
    Further,  comes from . Then,  by Lemma~\ref{L:TAU_I}(1), we obtain , as desired.
\end{proof}

As an immediate consequence of items (2) and (4) in the above lemma, the property below is given, which has been obtained in \cite{Luttgen10}.


As pointed out by L\"{u}ttgen and Vogler in \cite{Luttgen07,Luttgen10}, this is a fundamental property of ready simulation in the presence of logic operators. Intuitively, it says that  is an implementation  of the specification  if and only if  implements both  and .

We now are ready to prove some basic properties of the operator .

\begin{proposition}[Conjunction]\label{S:CONJUNCTION}\hfill
    \begin{enumerate}
      \item .
      \item .
      \item .
      \item .
      \item .
      \item .
      \item .
    \end{enumerate}
\end{proposition}
\begin{proof}
  \noindent \textbf{(1)} By Lemma~\ref{L:CON_ID_I}(1),  and .
  Then, by Lemma~\ref{L:CON_ID_I}(3), . Similarly, .\\

\noindent \textbf{(2)} Similar to (1), it follows from Lemma~\ref{L:CON_ID_I}(2) and \ref{L:CON_ID_I}(4).\\

\noindent \textbf{(3)} By Lemma~\ref{L:CON_ID_I}(1),  for .
Then, it immediately follows from Lemma~\ref{L:CON_ID_I}(3) that .
Similarly, . \\

\noindent \textbf{(4)}
Similar to (3), but using Lemma~\ref{L:CON_ID_I}(2)(4) instead of Lemma~\ref{L:CON_ID_I}(1)(3).\\

\noindent \textbf{(5)} Immediately follows from Lemma~\ref{L:CON_ID_I}(1)(3).\\

\noindent \textbf{(6)} Immediate consequence of Lemma~\ref{L:CON_ID_I}(2)(4).\\

\noindent \textbf{(7)} By Lemma~\ref{L:F_NORMAL}(4),  follows from . Then, it holds trivially that .
\end{proof}

The next lemma records some simple properties of the operator .

\begin{proposition}[Parallel]\label{S:PARALLEL}\hfill
    \begin{enumerate}
      \item .
      \item .
      \item .
      \item .
\end{enumerate}
\end{proposition}
\begin{proof}
  Analogous to Prop.~\ref{L:EC2}.
\end{proof}

In the following, we will show that  is a precongruence.
We first prove that the operators ,  and  are monotonic w.r.t .

\begin{lemma}[Monotonic w.r.t ]\label{L:CONGRUENCE_STABLE}
If  and  is stable, then
\begin{enumerate}
  \item ,
  \item ,
  \item .
\end{enumerate}
\end{lemma}
\begin{proof}
\textbf{(1)} Assume that  and  is stable. Put

It suffices to show that  is a stable ready simulation relation.
Clearly, by Lemma~\ref{L:STABILIZATION}, both  and  are stable, that is, (RS1) holds. In the following, we show that the pair  satisfies (RS2)-(RS4).

\textbf{(RS2)} Suppose  .
    By Lemma~\ref{L:F_NORMAL}(3),  or .
    For the first alternative, it follows from  that . So, .
    For the second alternative, by Lemma~\ref{L:F_NORMAL}(3),  immediately follows.

\textbf{(RS3)} Suppose .
    Since  is stable, it follows that
    
    Since , by (RS2), we get
    
     The argument now splits into two cases depending on  the form of the last rule applied in the proof tree of .\\

\noindent  Case 1  with .

        So,  and .
        Then, by Lemma~\ref{L:F_NORMAL}(3) and (\ref{L:CONGRUENCE_STABLE}.1.1), we have .
        It follows from  that  and  for some .
        Moreover, since  is stable, we get
        
        Further, it follows from  that .
        Thus,
        
        Hence, by (\ref{L:CONGRUENCE_STABLE}.1.2), (\ref{L:CONGRUENCE_STABLE}.1.3) and (\ref{L:CONGRUENCE_STABLE}.1.4), we have   and .\\

\noindent  Case 2  with .

    Hence,  and .
   Since  is stable, it follows that .
   So, .
   Then, by (\ref{L:CONGRUENCE_STABLE}.1.1) and  (\ref{L:CONGRUENCE_STABLE}.1.2), we have  and ,  as desired.

\textbf{(RS4)} Suppose .
 So, by Lemma~\ref{L:F_NORMAL}(3), .
 Hence,  due to .
Further, by Lemma~\ref{L:Basic_II}(1), we get .\\

\noindent \textbf{(2)}
    

    

    \\

\noindent \textbf{(3)}
        Put 
We want to show that  is a stable ready simulation relation. Let .

\textbf{(RS1)} Since ,  and  are stable, by Lemma~\ref{L:STABILIZATION}, so are  and .

\textbf{(RS2)} Suppose .
    So, by Lemma~\ref{L:F_NORMAL}(3),  or .
    Then, by   and Lemma~\ref{L:F_NORMAL}(3), it immediately follows that .

\textbf{(RS3)} Suppose .
    Since  is stable,  for some .
    We consider three cases based on the form of the last rule applied in the proof tree of .\\

 \noindent   Case 1  with  and .

    So,  and .
    By Lemma~\ref{L:TAU_I}(2), it follows from  that     ,  and  for some .
    Since  is stable,  .
    It is easy to see that .
    So, .
    Further, since , we obtain ,    and  for some .
    Since  is stable, there exists  such that ,
    further, by Lemma~\ref{L:TAU_I}(1) and , it follows that
    
    Moreover, since  is stable and , we obtain
    
    Hence
    
    Further, since  and , by Lemma~\ref{L:F_NORMAL}(3), we get .
    Thus, it follows from (\ref{L:CONGRUENCE_STABLE}.3.1) and (\ref{L:CONGRUENCE_STABLE}.3.2) that
    .
    Moreover,  due to .\\

\noindent Case 2  with  and .

 Similar to Case 1.\\

\noindent Case 3  with .

We invite the reader to check it.\\

\textbf{(RS4)}  Suppose .
    We will prove .
    Assume that . Then  for some . In the following, we consider three cases based on the form of the last rule applied in the proof tree of  .\\

\noindent    Case 1  with  and .

    So,  and . Then, .
    Since , by Lemma~\ref{L:F_NORMAL}(3), we have .
    So,  comes from .
    Then,  for some .
    Since  and , we get .
    So, .
    Hence, .\\

\noindent    Case 2  with  and .

Similar to Case 1.\\

\noindent    Case 3  with .

    Thus,  and .
    Similar to Case 1, we get  for some .
    Further, since ,  follows from . Thus, .\\

    Similarly, we can show .
\end{proof}

\begin{lemma}[Monotonic w.r.t ]\label{L:CONGRUENCE_PRE}
For any , if  then  for each .
\end{lemma}
\begin{proof}
  For , it immediately follows from Lemma~\ref{L:CON_ID_I}(2)(4).
  Next, we consider , the remaining parts raise no significantly different issues and omitted.
    Suppose  .
    It is enough to find  such that  and .
    By Lemma~\ref{L:TAU_I}(2), we have ,  and  for some .
     Moreover, it follows from  that  and  for some .
    We will check that  is exactly what we need.
    Clearly, by Lemma~\ref{L:CONGRUENCE_STABLE}(3), .
    And, by Lemma~\ref{L:TAU_I}(1), we get , as desired.
\end{proof}

We can now state an important result about ready simulation relation, which reveals that  is substitutive under all our combinators.

\begin{proposition}[Precongruence]\label{L:CONGRUENCE}
  If  and , then
  \begin{enumerate}
    \item    for each  and
    \item  for each .
  \end{enumerate}
\end{proposition}
\begin{proof}
\noindent \textbf{(1)} The argument splits into two cases depending on the kind of  (visible or invisible). The proof is straightforward and omitted.
\\

\noindent \textbf{(2)} By Prop.~\ref{S:DISJUNCTION}(1), \ref{L:EC2}(2), \ref{S:CONJUNCTION}(2)  and \ref{S:PARALLEL}(2),  satisfies commutative law for each .
Then, it immediately follows from the transitivity of  and Lemma~\ref{L:CONGRUENCE_PRE}.
\end{proof}

Hitherto we have only considered (in)equational laws involving one operator alone.
Next, we shall deal with a few of laws which refer to different operators in one (in)equation.
More such laws will be treated in the next section when establishing the soundness of .

\begin{proposition}[Distributive]\label{S:DISTRIBUTIVE}
        for each  .
\end{proposition}
\begin{proof}
    We consider the case where , the others are similar.

    Firstly, we prove .
    Suppose .
    Then, by Lemma~\ref{L:DIS}, there exists  and a sequence  such that ,  and either  or  for some .
    W.l.o.g, assume that .
    So, by Lemma~\ref{L:STABILIZATION} and \ref{L:Basic_I}(2), .
    Further, by Lemma~\ref{L:F_NORMAL}(1)(3), .
    Then, it follows from  that .

    Secondly, we prove .
    By Prop.~\ref{S:DISJUNCTION}(1)(5), we have   and .
    Further, by Prop.~\ref{L:CONGRUENCE}(2), we get  and . Then,  comes from Prop.~\ref{L:CONGRUENCE}(2) and \ref{S:DISJUNCTION}(3).
\end{proof}


\begin{proposition}\label{S:SPECIAL_I}
 for each .
\end{proposition}
\begin{proof}
     \qquad \;\;\;\; (by Prop.~\ref{S:DISJUNCTION}(1)(5))\\
   (by Prop.~\ref{L:CONGRUENCE}(1))\\
  \qquad \qquad \qquad \qquad(by Prop.~\ref{L:CONGRUENCE}(2) and \ref{L:EC2}(6)).
\end{proof}

A natural problem arises at this point, that is, whether the inequation below holds

The answer is negative.
For instance, consider  and .
By Lemma~\ref{L:F_NORMAL}, we have  and .
Thus, it doesn't hold that .

We conclude this section with the next proposition, which establishes a necessary and sufficient condition for the inequation (DS) with  to be true.
To this end, we introduce the notion below.

\begin{mydefn}[Uniform w.r.t ]
  Two process terms  and  are said to be uniform w.r.t  if  iff .
\end{mydefn}

\begin{proposition}\label{S:SPECIAL}
For each ,
  iff
  and  are uniform w.r.t .
\end{proposition}
\begin{proof}
\noindent (Left implies Right) Suppose  and  are not uniform w.r.t . W.l.o.g, assume that  and . By Lemma~\ref{L:F_NORMAL}, we get  and . Hence, .\\

\noindent (Right implies Left)
Since  , by Lemma~\ref{L:Basic_I}(1) and \ref{L:STABILIZATION}, both  and  are stable. So, it is enough to prove  .
Put 
We need to show that  is a stable ready simulation relation.
It is trivial to check that (RS1)-(RS4) hold for each pair in .
In the following, we treat the pair .
By Lemma~\ref{L:Basic_I}(1), \ref{L:Basic_II}(1) and \ref{L:STABILIZATION}, this pair satisfies (RS1) and (RS4).

\textbf{(RS2)} Suppose . By Lemma~\ref{L:F_NORMAL}, it follows that  for some . Further, since  and  are uniform w.r.t , we obtain  and . Hence, by Lemma~\ref{L:F_NORMAL} again, .

\textbf{(RS3)} Suppose . Since , it suffices to prove that .
Since  is stable,  for some .
So, by Lemma~\ref{L:Basic_I}(1), we get  and .
Further, by Lemma~\ref{L:Basic_I}(2), either  or .
W.l.o.g, we consider the first alternative.
Then, it immediately follows that .
Moreover, since , by (RS2), we get .
Hence, , as desired.
\end{proof}

Notice that the situation is different if .
In such case, the inequation (DS) does not always hold even if  and  are  uniform w.r.t .
As a simple example, consider  and  with .
Clearly, they are uniform w.r.t .
Moreover, ,  and  is the unique process term such that .
Hence, it follows from  and  that .
Thus, .

\section{Axiomatic System }
Section~5 has developed the behavioral theory of CLL on the level of semantics. This section will provide an algebraic and axiomatic approach to reason about behavior.
We will propose an axiomatic system to capture and investigate the operators in CLL through (in)equational laws, and establish its soundness and ground-completeness w.r.t ready simulation.

\subsection{}
In order to introduce the axiomatic system , a few preliminary definitions are given below.


\begin{mydefn}[Basic Process Term] \label{D:BPT}
The basic process terms are defined by BNF below

where  and . We denote  as the set of all basic process terms.
\end{mydefn}

\begin{mydefn}\label{D:GEC}
Let  be a finite sequence of process terms with . We define the general external choice  by recursion:
\begin{enumerate}
  \item ,
  \item ,
  \item  for .
\end{enumerate}
\end{mydefn}

Moreover, given a finite sequence  and , the general external choice  is defined as , where the sequence  is the restriction of  to .\\

In fact, up to  (or, =, see below), the order and grouping of terms in  may be ignored by virtue of Prop.~\ref{L:EC2}(2)(4) (axioms  and  below, respectively).

\begin{mydefn}[Injective in Prefixes]
  A process term  is said to be injective in prefixes if  for any .
\end{mydefn}

We now can present the axiomatic system . As usual,  has two parts: axioms and inference rules.\\

\noindent \textbf{Axioms}

 Unless otherwise stated, we shall assume variables in axioms below to be in range of . As usual,  means  and .\\

\noindent  
  
  
  
  
  
  
  
  
  
  
  \noindent 
   

   \noindent 
      

\noindent  
        

\noindent 
    
     
      
    .\\

\noindent \textbf{Inference rules}
    

Given the axioms and rules of inference, we assume that the resulting notions of proof, length of proof and theorem are already familiar to the reader. Following standard usage,  means that  is a theorem of .

\subsection{Soundness}
This subsection will establish the soundness of . To this end, a number of properties of general external choice  are needed. First, a simpler result is given below:

\begin{lemma}\label{L:BIG_SQUARE}
Let  and .
\begin{enumerate}
  \item  iff  for some .
  \item   for each .
  \item If  then  and  for some .
\end{enumerate}
\end{lemma}
\begin{proof}
  Proceed by induction on , omitted.
\end{proof}

\begin{proposition}\label{L:MULTIPLE_VI}
Let  for each  and .
  \begin{enumerate}
    \item If  then .
    \item If  then
     .
    \item .
  \end{enumerate}
\end{proposition}
\begin{proof}
  \noindent \textbf{(1)} Since , w.l.o.g, we may assume that  for some .
  Since  for each  and , by Lemma~\ref{L:BIG_SQUARE}(2)(3), we have , ,  and .
  Further, by Lemma~\ref{L:STABILIZATION}, we have .
    So, .
    Hence,  we have .
Then,  holds trivially.\\

\noindent \textbf{(2)}
If  then  and it is trivial to check .
In the following, we consider the case where .
For each , by Lemma~\ref{L:CON_ID_I}(2) and Prop.~\ref{L:CONGRUENCE}(1), we have

Therefore, by Prop.~\ref{L:EC2}(2)(4)(6) and \ref{L:CONGRUENCE}(2), it follows that

Similarly, we have

Since , by Lemma~\ref{L:EC2}(2)(4), we have

Thus, by Lemma~\ref{L:CON_ID_I}(4), it follows from (\ref{L:MULTIPLE_VI}.1), (\ref{L:MULTIPLE_VI}.2) and (\ref{L:MULTIPLE_VI}.3) that
 

\noindent \textbf{(3)} Immediately follows from (2).
\end{proof}

In the following, we provide an example to illustrate that it does not always hold that .

\begin{example}
Consider process terms
, ,  and  where .
Then, ,

and .
Assume that .
Since these process terms are stable, we have .
It follows from  and  that .
So,  follows from .
Further, by Lemma~\ref{L:F_NORMAL}, we get .
Thus, it follows from  that .
Since ,  and , the last rule applied in the proof tree of  is of the form 
Then, by Lemma~\ref{L:CON_LLTS},  for each  such that .
But, it is easy to see that  and , a contradiction.

\end{example}

However, for any general external choice  with distinct prefixes, we have

\begin{proposition}\label{L:MULTIPLE_II}
Let  for each .
  If  is injective in prefixes then .
\end{proposition}
\begin{proof}
We consider the non-trivial case where .
Since  for each , by Lemma~\ref{L:BIG_SQUARE}(3) and \ref{L:STABILIZATION}, it is easy to see that  and  are stable.
Thus, it suffices to prove . Put

We need to show that the pair  satisfies (RS1)-(RS4).
It is easy to check that (RS1) and (RS4) hold. We deal with (RS2) and (RS3) as follows.

\textbf{(RS2)} Suppose .
By Lemma~\ref{L:BIG_SQUARE}(1) and \ref{L:F_NORMAL}(2), there exists  such that .
On the other hand, by Lemma~\ref{L:BIG_SQUARE}(2), we obtain  and .
Further, it follows from Lemma~\ref{L:Basic_II}(2) that .
Moreover, since both  and  are injective in prefixes, by Lemma~\ref{L:Basic_II}(2) and \ref{L:BIG_SQUARE}(3), it follows that  is the unique -derivative of . Therefore, by Lemma~\ref{L:LLTS_I},  comes from , as desired.

\textbf{(RS3)} Suppose .
Since  is stable, we have  for some .
Since  and  are injective in prefixes, by Lemma~\ref{L:Basic_II}(2) and \ref{L:BIG_SQUARE}(3), we get , ,  and  for some .
On the other hand, by Lemma~\ref{L:BIG_SQUARE}(2), we get .
Moreover, since , by (RS2), .
Hence, we obtain  and .
\end{proof}

The next two propositions state the properties of the interaction of general external choice and parallel operator, which are analogous to the expansion law in usual process calculus.

\begin{proposition}\label{L:MULTIPLE_I}
Let , ,  and
 for each  and .Then

where
,
 and
.
\end{proposition}
\begin{proof}
  Set  and .
  Since  for each  and , by Lemma~\ref{L:BIG_SQUARE}(3) and \ref{L:STABILIZATION}, both  and  are stable.
  It is enough to prove . Put
  
  It suffices to show that  is a stable ready simulation relation. We will check that the pair  satisfies (RS2), (RS3) and (RS4) one by one.

  \textbf{(RS2)} Suppose .
  Then, by Lemma~\ref{L:BIG_SQUARE}(1) and \ref{L:F_NORMAL}(3), we have  for some .
  We shall consider the case where , the others may be treated similarly and omitted.
  In such case, we may assume that  with  and .
So, by Lemma~\ref{L:F_NORMAL}(2)(3), either  or .
  Then, by Lemma~\ref{L:BIG_SQUARE}(1) and \ref{L:F_NORMAL}(2)(3), it is easy to see that each of them implies   , as desired.

  \textbf{(RS3)} Suppose .
  So, , further, by (RS2), we get .
  Since  is stable,  for some .
  We consider three cases based on the form of the last rule applied in the proof tree of  .\\

\noindent  Case 1  with  and .

  So,  and .
  By Lemma~\ref{L:BIG_SQUARE}(3), we have  and  for some .
  Since , .
  So, by Lemma~\ref{L:BIG_SQUARE}(2), .
  Further, by Lemma~\ref{L:BIG_SQUARE}(3), it follows from  that
  
  Then, by Lemma~\ref{L:Basic_II}(1), we get  .
  Hence,  and .\\

\noindent  Case 2  with  and .

  Similar to Case 1.\\


\noindent  Case 3  with .

  So, ,  and .
  By Lemma~\ref{L:BIG_SQUARE}(3), we have ,  for some  and ,  for some .
  Since , .
  So, by Lemma~\ref{L:BIG_SQUARE}(2), .
  Moreover, by Lemma~\ref{L:BIG_SQUARE}(3), it follows from  that  and . Then, by Lemma~\ref{L:Basic_II}(1), we get  .
  Hence,  and .

  \textbf{(RS4)} We just prove , the proof of  is similar and omitted.
   Assume that .
   So,  for some .
   By Lemma~\ref{L:Basic_II}(1),  for some .
     We shall consider the case where , the others are similar and omitted.
     In such case, we may assume  with  and .
     Then, we get  and .
   Since , by Lemma~\ref{L:BIG_SQUARE}(3), .
   Further, it follows from  that
   
   On the other hand, by Lemma~\ref{L:BIG_SQUARE}(2), we have .
    Therefore, it follows from (\ref{L:MULTIPLE_I}.1.1) that .
    So, , as desired.
\end{proof}

Compare to usual expansion law in process calculus, e.g., Prop. 3.3.5 in \cite{Milner89}, we expect that the inequation below holds, where  () is same as ones in Prop.~\ref{L:MULTIPLE_I}.

Unfortunately, it isn't valid.
For instance, consider the process terms ,  and  with ,  and .
Let .
Clearly, the set  corresponding to ones in the above proposition are:  and .
Then, 
By Lemma~\ref{L:F_NORMAL},  and  .
Then, it is easy to see that .

However, the inequation (EXP) holds for the process terms satisfying a moderate condition. Formally, we have the result below.

\begin{proposition}\label{L:MULTIPLE_IV}
Let , , and let  and
 for each  and .
Assume that ,
 then 
where  () is same as ones in Prop.~\ref{L:MULTIPLE_I}.
\end{proposition}
\begin{proof}
  Set  and .
  Similar to Prop.~\ref{L:MULTIPLE_I}, we shall prove . Put
  
  It suffices to show that  is a stable ready simulation relation. We will check that the pair  satisfies (RS2), the remainder is similar to Prop.~\ref{L:MULTIPLE_I}.

  \textbf{(RS2)} Suppose . It follows from Lemma~\ref{L:F_NORMAL} and \ref{L:BIG_SQUARE}(1) that
  
  W.l.o.g, we consider the first alternative.
  Then, by the assumption, we get either  or  for some .
  Consequently, either  or .
  So, by Lemma~\ref{L:F_NORMAL} and \ref{L:BIG_SQUARE}(1),  follows, as desired.
\end{proof}



By Lemma~\ref{L:F_NORMAL}, it is easy to see that the operators , ,  and  preserve consistency. Thus, an immediate consequence of Lemma~\ref{L:F_NORMAL} is

\begin{lemma}\label{L:BPT}
  .
\end{lemma}
\begin{proof}
  Induction on the structure of basic process terms (see, Def.~\ref{D:BPT}).
\end{proof}

Therefore, as a corollary of Prop.~\ref{S:SPECIAL} and \ref{L:MULTIPLE_IV}, we have

\begin{proposition}\label{L:MULTIPLE_V}
Let  and  for each  and . Then
  \begin{enumerate}
    \item  for each .
    \item 
        where  is same as ones in Prop.~\ref{L:MULTIPLE_I}.
  \end{enumerate}
\end{proposition}
\begin{proof}
    Immediately follows from Lemma~\ref{L:BPT} and Prop.~\ref{S:SPECIAL} and \ref{L:MULTIPLE_IV}.
\end{proof}

We now have all of the properties we require to prove the soundness of the axiomatic system .

\begin{mydefn}
  For any , the inequation  is said to be valid in , in symbols , if and only if .
\end{mydefn}

\begin{theorem}[Soundness]\label{T:SOUNDNESS}
If  then  for any .
\end{theorem}
\begin{proof}
As usual, it is enough to show that
\begin{enumerate}
  \item all ground instances of axioms are valid in , and
  \item all inference rules preserve validity.
\end{enumerate}
Item (1) is implied by Prop.~\ref{S:PREFIX}, \ref{S:DISJUNCTION}, \ref{L:EC2}, \ref{S:CONJUNCTION}, \ref{S:PARALLEL}, \ref{S:DISTRIBUTIVE}, \ref{L:MULTIPLE_VI}, \ref{L:MULTIPLE_II}, \ref{L:MULTIPLE_I} and \ref{L:MULTIPLE_V}.
Item (2) is implied by Prop.~\ref{L:CONGRUENCE} and the fact that  is reflexive and transitive.
\end{proof}

\subsection{Normal Form and Ground-Completeness}
This subsection will establish the ground-completeness of . We begin by giving two useful notations.\\

\noindent \textbf{Notation}
\begin{enumerate}
  \item .
  \item Let  be a finite sequence of process terms with . We define the general disjunction  by recursion:
    \begin{enumerate}
    \item ,
    \item  for .
    \end{enumerate}
\end{enumerate}

Similar to general external choice, up to =, the order and grouping of terms in   may be ignored by virtue of axioms  and .

To prove the ground-completeness of , we use a general technique involving normal forms.
The idea is to isolate a particular subclass of terms, called normal forms, such that the proof of the completeness is straightforward for it.
The completeness for arbitrary terms will follow if we can show that each term can be reduced to normal form using the equations in .
We define normal form as follows.

\begin{mydefn}[Normal Form]\label{D:NORMAL_FORM}
    The set  is the least subset of  such that  if  and for each ,  has the format  with  such that

        \noindent (N)\;\;\;   for each ,

        \noindent (D) \;\;  is injective in prefixes, and

        \noindent (N-)   for each .

We put 
Each process term in  is said to be  in normal form.
\end{mydefn}

Notice that , and  by taking  and  in .
In the following, we will show that each process term can be transformed using the equations into a normal form, which is a crucial step in establishing the ground-completeness of .
To this end, the next five lemmas are firstly proved.

\begin{lemma}\label{L:DIS_INEQUATION}
   for each .
\end{lemma}
\begin{proof}
   and   \qquad\qquad\qquad(by ,  and TRANS)

\noindent    and  (by CONTEXT and REF)

\noindent   \qquad \qquad (by , CONTEXT and TRANS)
\end{proof}

\begin{lemma}\label{L:COMP_CONJ}
  If  then  for some .
\end{lemma}
\begin{proof}
We prove it by induction on the number .
  Since , we may assume that  and .
   By , , ,  and Lemma~\ref{L:DIS_INEQUATION}, we get
   
    Let  and . We will show that  for some . Clearly, we may assume that  and  satisfying (N), (D) and (N-) in Def.~\ref{D:NORMAL_FORM}. We consider two cases below.\\


    \noindent Case 1 .

          By  , we have .\\

    \noindent Case 2  .

        Thus, by the item (D) in Def.~\ref{D:NORMAL_FORM}, we have .
        If  then, by the definition of general external choice, we get .
        Moreover,  follows from .
        In the following, we consider the nontrivial case where .
        By , ,  and , it follows that
        
         For each pair   with ,
         since  and , by IH, we have  for some  .
         Set 
        Consequently, by  CONTEXT and TRANS, we have
            
        Clearly, if  for each pair  with  , then .
        Otherwise, we have  for some , then it follows from  that 
        Further, by , CONTEXT and TRANS, we get .

     In summary, it follows from the discussion above that, for each  and ,
     
     Then, by ,  and (\ref{L:COMP_CONJ}.1), we get either  for some  or .
\end{proof}


In the above proof, we do not  explicitly show the proof for the induction basis where , as it is an instance of the proof of the induction step.

\begin{lemma}\label{L:SP3}
      .
\end{lemma}
\begin{proof}
 and    \qquad\;\;\;\;\; (by   ,   and TRANS )

\noindent  and   \;\qquad\qquad\qquad\;  (by CONTEXT)

\noindent  \qquad\qquad\qquad (by CONTEXT,   and TRANS)
\end{proof}

\begin{lemma}\label{L:BIG_SQUARE_EC}
  If   and , then  for some .
\end{lemma}
\begin{proof}
If  or  then it immediately follows from ,  and Def.~\ref{D:GEC}.
In the following, we consider the non-trivial case where  and .
We distinguish two cases below.\\

\noindent Case 1 .

          Set
            
        Then, it is trivial to check that  satisfies (N), (D) and (N-) in Def.~\ref{D:NORMAL_FORM}, that is, .
        Moreover, by  and TRANS, it immediately follows that .\\

\noindent Case 2 .

  Let  and  with , since , by Lemma~\ref{L:SP3} and , we get .
  Further, by Def.~\ref{D:NORMAL_FORM}, , , CONTEXT and TRANS, it follows from  that
  
  Thus, for each  and  with , we can fix a process term  such that
  
  Put
    \begin{enumerate}
      \item  ,
      \item  
      \item 
    \end{enumerate}
    Then, by , , TRANS and CONTEXT, we obtain .
    Clearly, both  and  are in .
    Moreover, since  and  are injective in prefixes, so is .
    Hence,  is also in .
    Further, since  for , similar to Case 1, we have
     for some .
\end{proof}

\begin{lemma}\label{L:COMP_PARALLEL}
  If  then  for some .
\end{lemma}
\begin{proof}
We prove it by induction on the number .
Since , we may assume that  and .
 By axioms , , ,  and Lemma~\ref{L:DIS_INEQUATION}, we get
    
    We shall show that for each  and ,
    
     Let  and .
     We may assume that  and  satisfying (N), (D) and (N-) in Def.~\ref{D:NORMAL_FORM}.
     By  and , we have
            
    We consider two cases.\\

\noindent    Case 1  or .

    W.l.o.g, assume that . Then, by (\ref{L:COMP_PARALLEL}.2), , , CONTEXT and TRANS, we get
    
    If  then .
    Next, we consider the case where .
    For each  with , we have , moreover, .
    Then, by IH, we get  for some .
    Therefore, by CONTEXT, TRANS and (\ref{L:COMP_PARALLEL}.3), it is easy to see that  for some .\\

\noindent    Case 2  and  .

    In such case, for each  and , we have ,  and .
      Moreover, .
      Then, by IH, there exist  such that ,   and .
            Set
    \begin{enumerate}
      \item  ,
      \item  ,
      \item .
    \end{enumerate}
    Clearly,  and .
    Further, by Lemma~\ref{L:BIG_SQUARE_EC}, we get  for some , as desired.

     In summary, by the discussion above, we conclude that, for each  and ,  for some .
     Then, by Def.~\ref{D:NORMAL_FORM} and (\ref{L:COMP_PARALLEL}.1), it immediately follows that  for some , as desired.
\end{proof}

Now, we can prove that each process term is normalizable. That is,  we have the result below.

\begin{theorem}[Normal Form Theorem]\label{T:NORMALFORM}
  For each ,    for some .
\end{theorem}
\begin{proof}
  We prove it by induction on the structure of  .

\noindent   or .

    Trivially.

\noindent  .

        By IH and CONTEXT, we get  for some .
        If  and , then .
        If , by  ,  and TRANS, we obtain .
        If , by  and TRANS, we have  .

\noindent   with .

        For , by IH, we have  for some . We distinguish four cases based on .\\

\noindent Case 1 .

        If    and   (i.e., ), then it immediately follows from , , CONTEXT and TRANS that  for some .
        Otherwise, w.l.o.g, assume that .
        Then, by ,  and TRANS, we get .\\

\noindent Case 2 .

        If either  or ,  then it follows from  and  that .
        In the following, we consider the case where  and .
        In such case, we get .
        So, we may assume that  and  with  for each  and .
        Thus, by , , CONTEXT, TRANS,  and Lemma~\ref{L:DIS_INEQUATION},  we obtain
        
        Further, by Lemma~\ref{L:BIG_SQUARE_EC} and Def.~\ref{D:NORMAL_FORM}, it immediately follows that  for some .\\

\noindent Case 3 .

         If  for  then, by Lemma~\ref{L:COMP_CONJ}, we have  for some , otherwise, by  and , we get .\\

\noindent Case 4 .

        If either  or  then, by  and , we get .
        Otherwise,  we have , so, by Lemma~\ref{L:COMP_PARALLEL}, we obtain  for some .
\end{proof}

We now turn our attention to the ground-completeness of .
First, we state a trivial result about general disjunction.

\begin{lemma}\label{L:SUBSTRUCTURE}
  Let  and  be stable for each .
  \begin{enumerate}
    \item If  then  for each .
    \item If  then  for some .
  \end{enumerate}
\end{lemma}
\begin{proof}
Proceed by induction on , omitted.
\end{proof}

An important step in proving the ground-completeness is:

\begin{lemma}\label{L:COMPLETENESS}
  If  and , then 
\end{lemma}
\begin{proof}
    We prove it by induction on the degree of .
    Since , both  and  are stable. Further, since , we get, for 

    

    Therefore, the argument splits into three cases below.\\

\noindent Case 1 .

        Then, by , ,  and TRANS, we have .\\

\noindent Case 2 .

       By Lemma~\ref{L:F_NORMAL}(5) and \ref{L:Basic_I}(3),  and . Further, since , we get  and . Thus, by (\ref{L:COMPLETENESS}.1), we have . Then,  follows from REF.\\

\noindent Case 3  with .

        Since , by Lemma~\ref{L:BPT}, we have .
        Hence, by ,  we get  and .
        Further, it follows from (\ref{L:COMPLETENESS}.1) and the condition (D) in Def.~\ref{D:NORMAL_FORM} that
        there exist  and () such that
         
       By CONTEXT, it is easy to know that, in order to complete the proof, it is enough to show that
       
    Let . We have  for some .
    Since , by Def.~\ref{D:NORMAL_FORM}, there exist ,  and  such that
    \begin{enumerate}
      \item  and ,
      \item  and  are stable for each  and ,
      \item  for each  and .
    \end{enumerate}
    In the following, we want to show that  for each .
     Let .
     Since , by Lemma~\ref{L:BPT} and \ref{L:SUBSTRUCTURE}(1), it immediately follows that .
    Thus, .
    Then, it follows from   that
    
    Further, since  is injective in prefixes and  is stable, we get .
    Then, by Lemma~\ref{L:SUBSTRUCTURE}(2), we obtain
    
    Since , by (\ref{L:COMPLETENESS}.2), (\ref{L:COMPLETENESS}.3) and IH, we get .
    Further, by , ,  and TRANS, we have , as desired.

    So far, we have obtained
    
    Then, by , , , CONTEXT and TRANS, we get , that is, .
    So, by CONTEXT, it follows that .
\end{proof}

We are now ready to prove the following, the main result of this section.

\begin{theorem}[Ground-Completeness]
  For any ,  implies .
\end{theorem}
\begin{proof}
  Assume that . Thus, .
  By Theorem~\ref{T:NORMALFORM},  and  for some .
  It suffices to prove that .
  By Theorem~\ref{T:SOUNDNESS}, we have  and . So, .

   If  then it follows from , ,  and TRANS that .
   Next, we consider the case . Then, .
   We may assume  with  and for each ,  with .
    In order to complete the proof, it is enough to show that
    
    Let .
    Since , by Lemma~\ref{L:BPT} and \ref{L:SUBSTRUCTURE}(1), we have .
    Then, it follows from  that  and  for some .
    So, , that is, .
    Thus,  and we may assume that   with  and for each ,  for some .
    Thus,  is stable for each .
    Then, by Lemma~\ref{L:SUBSTRUCTURE}(2), it follows from  that  for some .
    Further, by Lemma~\ref{L:COMPLETENESS},  follows from .
    Finally, by , ,  and TRANS, we obtain , as desired.
\end{proof}

\section{Conclusions and Further Work}

Inspired by L{\"u}ttgen and Vogler's work in \cite{Luttgen10}, this paper considers a process calculus with logical operators. Two different views of the language CLL are explored in detail:

\noindent --- a behavioral view, ready simulation,

\noindent --- a proof-theoretic view, the axiomatic system .

The soundness and completeness of  reveal that the above two views are equivalent, that is,


CLL is designed as a process calculus for Logic LTSs, which rephrases L\"{u}ttgen and Vogler's setting in a process-algebraic style.
The constructors prefix, conjunction, disjunction, external choice and parallel over Logic LTSs are captured by corresponding operators in CLL.
Moreover, a number of properties concerning these constructors are re-established in behavioral theory of CLL by very different method.
Similar to usual process algebras, this paper develops behavioral theory based on the SOS rules which specify the behavior of process terms, while L{\"u}ttgen and Vogler establish these properties depending on the constructions of Logic LTSs, and do not refer to any syntactical element.
Compared with their work, the main contribution of this paper is to present a sound and ground-complete axiomatic system of ready simulation in the presence of logic operators.

It is well known that, in addition to behavior and proof-theoretic views, the language of process algebra may be interpreted in denotational view (see, e.g. \cite{Hennessy88}).
The denotational method aims at defining a denotational function which associates semantic objects to process terms.
Such function is often given recursively by induction on the structure of process terms.
It is easy to see that constructors explored by L{\"u}ttgen and Vogler in \cite{Luttgen10} is useful when considering denotational semantics of CLL.
In fact, we can show the result below\\

\noindent \textbf{Observation}  for  and .\\

Here,  denotes the sub-LTS of  generated by the process term ,  is the constructor over Logic LTSs corresponding to , and  is the sub-LTS of  generated by the state \footnote{Since  and  are states in  and  respectively, according to the construction in \cite{Luttgen10},  contains the state labelled by .}.
This result suggests to us that, based on the constructors in \cite{Luttgen10}, it seems not difficult to provide a denotational semantics for CLL, which is fully abstract with respect to operational semantics presented in this paper.
We leave it as further work.

This paper adopts the predicate  to describe unimplementable processes.
Follows \cite{Luttgen10}, this predicate is involved in the notion of ready simulation.
In this sense, the value of  is regarded as an observable signal of processes.
The process algebra with observations of propositional formulae have been considered by Baeten and Bergstra in \cite{Baeten97}.
The motivation behind their work lies in providing a framework to deal with conditional process expression (e.g., ) based on the view that the visible part of the state of a process is a proposition.
Clearly, this is another method to incorporate logical components with process algebras.
An interesting research is to compare it with the framework adopted in \cite{Luttgen10} and this paper, in which logical operators over processes are introduced directly.

This paper focuses on exploring logical constructors of Logic LTSs in process algebraic style, some useful operators occurring in usual process algebras, such as hiding, recursion et.al, are not involved in  CLL.
Extending CLL by incorporating these operators is worth further investigation.\\

\noindent \textbf{References}
\begin{thebibliography}{30}

\bibitem{Aceto01}
  \bibinfo{author}{L.~Aceto},
  \bibinfo{author}{W.J.~Fokkink},
  \bibinfo{author}{C.~Verhoef},
\newblock
  \bibinfo{title}{Structural operational semantics},
\newblock
  in: \bibinfo{editor}{J.A.~Bergstra},
  \bibinfo{editor}{A.~Ponse},
  \bibinfo{editor}{S.A.~Smolka}, (Eds.),
  \bibinfo{booktitle}{Handbook of Process Algebra, Chapter 3},
  \bibinfo{publisher}{Elsevier Science},
  \bibinfo{year}{2001},
  pp. \bibinfo{pages}{197-292}.

 \bibitem{Baeten97}
  \bibinfo{author}{J.C.M.~Baeten},
  \bibinfo{author}{J.A.~Bergstra},
\newblock
  \bibinfo{title}{Process algebra with propositional signals},
\newblock
  \bibinfo{journal}{Theoretical Computer Science}
  \bibinfo{volume}{177}
  (\bibinfo{year}{1997})
  \bibinfo{pages}{381-405}.

  \bibitem{Handbook}
  \bibinfo{author}{J.~Bergstra},
  \bibinfo{author}{A.~Ponse},
  \bibinfo{author}{S.~Smolka},
  \bibinfo{title}{Handbook of Process Algebra},
  \bibinfo{publisher}{Elsevier Science},
  \bibinfo{year}{2001}.

   \bibitem{Bloom95}
  \bibinfo{author}{B.~Bloom},
  \bibinfo{author}{S.~Istrail},
  \bibinfo{author}{A.~Meyer},
\newblock
  \bibinfo{title}{Bisimulation can't be traced},
\newblock
  \bibinfo{journal}{Journal of the ACM}
  \bibinfo{volume}{42}
  (\bibinfo{year}{1995})
  \bibinfo{pages}{232-268}.

 \bibitem{Bol96}
  \bibinfo{author}{R.~Bol},
  \bibinfo{author}{J.F.~Groote},
\newblock
  \bibinfo{title}{The meaning of negative premises in transition system specifications},
\newblock
  \bibinfo{journal}{Journal of the ACM}
  \bibinfo{volume}{43}
  (\bibinfo{year}{1996})
  \bibinfo{pages}{863-914}.

 \bibitem{Boudol92}
  \bibinfo{author}{G.~Boudol},
  \bibinfo{author}{K.~Larsen},
\newblock
  \bibinfo{title}{Graphical versus logical specifications},
\newblock
  \bibinfo{journal}{Theoretical Computer Science}
  \bibinfo{volume}{106}
  (\bibinfo{year}{1992})
  \bibinfo{pages}{3-20}.

  \bibitem{ModelChecking}
  \bibinfo{author}{E.M.~Clarke},
  \bibinfo{author}{O.~Grumberg},
  \bibinfo{author}{D.A.~Peled},
  \bibinfo{title}{Model Checking},
  \bibinfo{publisher}{The MIT Press},
  \bibinfo{year}{2000}.

\bibitem{Cleaveland00}
  \bibinfo{author}{R.~Cleaveland},
  \bibinfo{author}{G.~L\"{u}ttgen},
\newblock
  \bibinfo{title}{A semantic theory for heterogeneous system design},
\newblock
  in: \bibinfo{booktitle}{FSTTCS 2000},
  in: \bibinfo{booktitle}{LNCS, vol. 1974},
  \bibinfo{publisher}{Springer-Verlag},
  \bibinfo{year}{2000},
  pp. \bibinfo{pages}{312-324}.




   \bibitem{Cleaveland02}
  \bibinfo{author}{R.~Cleaveland},
  \bibinfo{author}{G.~L\"{u}ttgen},
\newblock
  \bibinfo{title}{A logical process calculus},
\newblock
  in: \bibinfo{booktitle}{EXPRESS 2002},
  in: \bibinfo{booktitle}{ENTCS},
  \bibinfo{volume}{68, 2},
  \bibinfo{publisher}{Elsevier Science},
  \bibinfo{year}{2002}.



 \bibitem{Dam}
  \bibinfo{author}{M.~Dam},
\newblock
  \bibinfo{title}{Process-algebraic interpretation of positive linear and relevant logics},
\newblock
  \bibinfo{journal}{Journal of Logic and Computation}
  \bibinfo{volume}{4}
  (\bibinfo{year}{1994})
  \bibinfo{pages}{939-973}.


    \bibitem{Gelfond88}
  \bibinfo{author}{M.~Gelfond},
  \bibinfo{author}{V.~Lifchitz},
\newblock
  \bibinfo{title}{The stable model semantics for logic programming},
\newblock
  in: \bibinfo{editor}{R.~kowalski},
  \bibinfo{editor}{K.~Bowen},(Eds.),
  \bibinfo{booktitle}{Proceedings of the 5th International Conference on Logic Programming},
  \bibinfo{publisher}{MIT Press},
  \bibinfo{year}{1988},
  pp. \bibinfo{pages}{1070-1080}.





\bibitem{Glabbeek01}
  \bibinfo{author}{R.J.~van Glabbeek},
\newblock
  \bibinfo{title}{The linear time - branching time spectrum I},
\newblock
  in: \bibinfo{editor}{J.A.~Bergstra},
  \bibinfo{editor}{A.~Ponse},
  \bibinfo{editor}{S.A.~Smolka}, (Eds.),
  \bibinfo{booktitle}{Handbook of Process Algebra, Chapter 1},
  \bibinfo{publisher}{Elsevier Science},
  \bibinfo{year}{2001},
  pp. \bibinfo{pages}{3-100}.

 \bibitem{Glabbeek04}
  \bibinfo{author}{R.J.~van Glabbeek},
\newblock
  \bibinfo{title}{The meaning of negative premises in transition system specification II},
\newblock
  \bibinfo{journal}{Journal of Logic and Algebraic Programming}
  \bibinfo{volume}{60-61}
  (\bibinfo{year}{2004})
  \bibinfo{pages}{229-258}.



   \bibitem{Graf86}
  \bibinfo{author}{S.~Graf},
  \bibinfo{author}{J.~Sifakis},
\newblock
  \bibinfo{title}{A logic for the description of non-deterministic programs and their properties},
\newblock
  \bibinfo{journal}{Information Control}
  \bibinfo{volume}{68}
  (\bibinfo{year}{1986})
  \bibinfo{pages}{254-270}.

 \bibitem{Groote92}
  \bibinfo{author}{J.F.~Groote},
  \bibinfo{author}{F. Vaandrager},
\newblock
  \bibinfo{title}{Structured operational semantics and bisimulation as a congruence},
\newblock
  \bibinfo{journal}{Information and Compuation}
  \bibinfo{volume}{100}
  (\bibinfo{year}{1992})
  \bibinfo{pages}{202-260}.


 \bibitem{Groote93}
  \bibinfo{author}{J.F.~Groote},
\newblock
  \bibinfo{title}{Trsnsition system specifications with negative premises},
\newblock
  \bibinfo{journal}{Theoretical Computer Science}
  \bibinfo{volume}{118}
  (\bibinfo{year}{1993})
  \bibinfo{pages}{263-299}.



  \bibitem{Hennessy88}
  \bibinfo{author}{M. Hennessy},
  \bibinfo{title}{Algebraic Theory of Processes},
  \bibinfo{publisher}{The MIT Press},
  \bibinfo{year}{1988}.

\bibitem{Kurshan94}
  \bibinfo{author}{R.~Kurshan},
  \bibinfo{title}{Computer-Aided Verification of Coordinating Processes: The Automata-Theoretic Approach},
  \bibinfo{publisher}{Princeton Univ. Press},
  \bibinfo{year}{1994}.

 \bibitem{Luttgen07}
  \bibinfo{author}{G.~L\"{u}ttgen},
  \bibinfo{author}{W.~Vogler},
\newblock
  \bibinfo{title}{Conjunction on processes: full-abstraction via ready-tree semantics},
\newblock
  \bibinfo{journal}{Theoretical Computer Science}
  \bibinfo{volume}{373}
  (\bibinfo{number}{1-2})
  (\bibinfo{year}{2007})
  \bibinfo{pages}{19-40}.

 \bibitem{Luttgen10}
  \bibinfo{author}{G.~L\"{u}ttgen},
  \bibinfo{author}{W.~Vogler},
\newblock
  \bibinfo{title}{Ready simulation for concurrency: it's logical},
\newblock
  \bibinfo{journal}{Information and computation}
  \bibinfo{volume}{208}
  (\bibinfo{year}{2010})
  \bibinfo{pages}{845-867}.


 \bibitem{Luttgen11}
  \bibinfo{author}{G.~L\"{u}ttgen},
  \bibinfo{author}{W.~Vogler},
\newblock
  \bibinfo{title}{Safe reasoning with Logic LTS},
\newblock
  \bibinfo{journal}{Theoretical Computer Science}
  \bibinfo{volume}{412}
  (\bibinfo{year}{2011})
  \bibinfo{pages}{3337-3357}.

\bibitem{Milner89}
  \bibinfo{author}{R.~Milner},
  \bibinfo{title}{Communication and Concurrency},
  \bibinfo{publisher}{Prentice Hall},
  \bibinfo{year}{1989}.

 \bibitem{Nicola83}
  \bibinfo{author}{R.~De Nicola},
  \bibinfo{author}{M.~Hennessy},
\newblock
  \bibinfo{title}{Testing equivalences for processes},
\newblock
  \bibinfo{journal}{Theoretical Computer Science}
  \bibinfo{volume}{34}
  (\bibinfo{year}{1983})
  \bibinfo{pages}{83-133}.


\bibitem{Olderog}
  \bibinfo{author}{E.~Olderog},
  \bibinfo{title}{Nets, Terms and Formulas in: Cambridge Tracts in Theoretical Computer Science, vol. 23},
  \bibinfo{publisher}{Cambridge Univ. Press},
  \bibinfo{year}{1991}.

\bibitem{Plotkin81}
  \bibinfo{author}{G. Plotkin},
\newblock
  \bibinfo{title}{A structural approach to operational semantics},
\newblock
  \bibinfo{booktitle}{Report DAIMI FN-19},
  \bibinfo{publisher}{Computer Science Department, Aarhus University}
  (\bibinfo{year}{1983}).
  Also in,
  \bibinfo{journal}{Journal of Logic and Algebraic Programming}
  \bibinfo{volume}{60}
  (\bibinfo{year}{2004})
  \bibinfo{pages}{17-139}.

\bibitem{Pnueli}
  \bibinfo{author}{A.~Pnueli},
\newblock
  \bibinfo{title}{The temporal logic of programs},
\newblock
  in: \bibinfo{booktitle}{FOCS'77},
  \bibinfo{publisher}{IEEE Computer Socitey Press},
  \bibinfo{year}{1977},
  pp. \bibinfo{pages}{46-57}.

 \bibitem{Verhoef95}
  \bibinfo{author}{C. Verhoef},
\newblock
  \bibinfo{title}{A congruence theorem for structured operational semantics with predicates and negative premisess},
\newblock
  \bibinfo{journal}{Nordic Journal of Computing}
  \bibinfo{volume}{2(2)}
  (\bibinfo{year}{1995})
  \bibinfo{pages}{274-302}.

\end{thebibliography}
\end{document} 