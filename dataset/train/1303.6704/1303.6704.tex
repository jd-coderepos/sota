\documentclass[runningheads]{llncs}

\pagestyle{headings}
\usepackage{latexsym}
\usepackage{amssymb}  
\usepackage{amsmath}
\usepackage{stmaryrd}
\usepackage{xy}
\xyoption{all}

\newcommand{\valpha}{\boldsymbol{\alpha}}
\newcommand{\veta}{\boldsymbol{\eta}}

\begin{document}

\title{Revisiting the Equivalence Problem for Finite Multitape Automata}
\author{ James
  Worrell\thanks{Supported by EPSRC grant EP/G069727/1.}}
\institute{Department of Computer Science, University of Oxford, UK}

\maketitle

\begin{abstract}
  The decidability of determining equivalence of deterministic
  multitape automata (or transducers) was a longstanding open problem
  until it was resolved by Harju and Karhum\"{a}ki in the early 1990s.
  Their proof of decidability yields a \textbf{co-NP} upper bound, but
  apparently not much more is known about the complexity of the
  problem.  In this paper we give an alternative proof of
  decidability, which follows the basic strategy of Harju and
  Karhum\"{a}ki but replaces their use of group theory with results on
  matrix algebras.  From our proof we obtain a simple randomised
  algorithm for deciding language equivalence of deterministic
  multitape automata and, more generally, multiplicity equivalence of
  nondeterministic multitape automata.  The algorithm involves only
  matrix exponentiation and runs in polynomial time for each fixed
  number of tapes.  If the two input automata are inequivalent then
  the algorithm outputs a word on which they differ.
\end{abstract}

\section{Introduction}
One-way multitape finite automata were introduced in the seminal 1959
paper of Rabin and Scott~\cite{RabinS59}.  Such automata (under
various restrictions) are also commonly known as transducers---see
Elgot and Mezei~\cite{ElgotM63} for an early reference.  A multitape
automaton with  tapes accepts a -ary relation on words.  The
class of relations recognised by deterministic automata coincides with
the class of -ary rational relations~\cite{ElgotM63}.

Two multitape automata are said to be equivalent if they accept the
same relation.  Undecidability of equivalence of non-deterministic
automata is relatively straightforward~\cite{Griffiths68}.  However
the deterministic case remained open for many years, until it was
shown decidable by Harju and Karhum\"{a}ki~\cite{HarjuK91}.  Their
solution made crucial use of results about ordered
groups---specifically that a free group can be endowed with a
compatible order~\cite{Neumann49b} and that the ring of formal power
series over an ordered group with coefficients in a division ring and
with well-ordered support is itself a division ring (due independently
to Malcev~\cite{Malcev48} and Neumann~\cite{Neumann49}).  Using these
results~\cite{HarjuK91} established the decidability of
\emph{multiplicity equivalence} of non-deterministic multitape
automata, i.e., whether two non-deterministic multitape automata have
the same number of accepting computations on each input.  Decidability
in the deterministic case (and, more generally, the unambiguous case)
follows immediately.  We refer the reader to~\cite{Sakarovich03} for a
self-contained account of the proof, including the underlying group
theory.

Harju and Karhum\"{a}ki did not address questions of complexity
in~\cite{HarjuK91}.  However the existence of a \textbf{co-NP}
\emph{guess-and-check} procedure for deciding equivalence of
deterministic multitape automata follows directly from~\cite[Theorem
  8]{HarjuK91}.  This theorem states that two inequivalent automata
are guaranteed to differ on a tuple of words whose total length is at
most the total number of states of the two automata. Such a tuple can
be guessed, and it can be checked in polynomial time whether the tuple
is accepted by one automaton and rejected by the other.  In the
special case of two-tape deterministic automata, a polynomial-time
algorithm was given in~\cite{FriedmanG82}, before decidability was
shown in the general case.

A \textbf{co-NP} upper bound also holds for multiplicity equivalence
of -tape automata for each fixed .  However, as we observe
below, if the number of tapes is not fixed, computing the number of
accepting computations of a given non-deterministic multitape automata
on a tuple of input words is \#\textbf{P}-hard.  Thus the
guess-and-check method does not yield a \textbf{co-NP} procedure for
multiplicity equivalence in general.

It is well-known that the equivalence problem for single-tape weighted
automata with rational transition weights is solvable in polynomial
time~\cite{Schutzenberger61,Tzeng}.  Now the decision procedure
in~\cite{HarjuK91} reduces multiplicity equivalence of multitape
automata to equivalence of single-tape automata with transition
weights in a division ring of power series over an ordered group.
However the complexity of arithmetic in this ring seems to preclude an
application of the polynomial-time procedures
of~\cite{Schutzenberger61,Tzeng}.  Leaving aside issues of
representing infinite power series, even the operation of multiplying
a family of polynomials in two non-commuting variables yields a result
with exponentially many monomials in the length of its input.

In this paper we give an alternative proof that multiplicity
equivalence of multitape automata is decidable, which also yields new
complexity bounds on the problem.  We use the same basic idea
as~\cite{HarjuK91}---reduce to the single-tape case by enriching the
set of transition weights.  However we replace their use of power
series on ordered groups with results about matrix algebras and
Polynomial Identity rings (see Remark~\ref{rem:difference} for a more
technical comparison).  In particular, we use the Amitsur-Levitzki
theorem concerning polynomial identities in matrix algebras.  Our use
of the latter is inspired by the work of~\cite{BogdanovW05} on
non-commutative polynomial identity testing, and our starting point is
a simple generalisation of the approach of~\cite{BogdanovW05} to what
we call \emph{partially commutative} polynomial identity testing.

Our construction for establishing decidability immediately yields a
simple randomised algorithm for checking multiplicity equivalence of
multitape automata (and hence also equivalence of deterministic
automata).  The algorithm involves only matrix exponentiation, and
runs in polynomial time for each fixed number of tapes. 





\section{Partially Commutative Polynomial Identities}
\label{sec:semi}

\subsection{Matrix Algebras and Polynomial Identities}

Let  be an infinite field.  Recall that an -algebra is a vector
space over  equipped with an associative bilinear product that has
an identity .  Write  for the free -algebra
over a set .  The elements of  can be viewed as
polynomials over a set of non-commuting variables  with
coefficients in .  Each such polynomial is an -linear
combination of monomials, where each monomial is an element of .
The degree of a polynomial is the maximum of the lengths of its
monomials.

Let  be an -algebra and .  If 
evaluates to  for all valuations of its variables in  then we
say that  satisfies the \emph{polynomial identity} .  For
example, an algebra satisfies the polynomial identity  if and
only if it is commutative.  Note that since the variables  and
 do not commute, the polynomial  is not identically zero.

We denote by  the -algebra of  matrices with
coefficients in .  The Amitsur-Levitzki theorem~\cite{AL50,Cohn03}
is a fundamental result about polynomial identities in matrix
algebras.
\begin{theorem}[Amitsur-Levitzki]
The algebra  satisfies the polynomial identity

where the sum is over the  elements of the symmetric group .
Moreover  satisfies no identity of degree less than .
\label{thm:AL}
\end{theorem}


Given a finite set  of non-commuting variables, the \emph{generic
   matrix algebra}  is defined as
follows.  For each variable  we introduce a family of
commuting indeterminates  and
define  to be the -algebra of 
matrices generated by the matrices  for each .
Then  has the following universal property: any
homomorphism from  to a matrix algebra ,
with  an -algebra, factors uniquely through the map  given by
.  

Related to the map  we also define an -algebra homomorphism

by 
 
where the matrix on the right has zero entries everywhere but along
the superdiagonal.

\subsection{Partially Commutative Polynomial Identities}
In this section we introduce a notion of \emph{partially commutative
  polynomial identity}.  We first establish notation and recall some
relevant facts about tensor products of algebras.  


Write  for the tensor product of -algebras  and ,
and write  for the -fold tensor power of .  If
 is an algebra of  matrices and  an algebra of
 matrices, then we identify the tensor product 
with the algebra of  matrices spanned by the
matrices ,  and
, where 

In particular we have .

A \emph{partially commuting set of variables} is a tuple
, where the  are disjoint sets.
Write  for the tensor product
.  We
think of  as a set of polynomials in
partially commuting variables.  Intuitively two variables  do not commute, whereas  commutes with  if
.  Note that if each  is a singleton  then  is the familiar ring of polynomials in
commuting variables .  At the other extreme, if 
then we recover the non-commutative case.

An arbitrary element  can be written
uniquely as a finite sum of distinct \emph{monomials}, where each
monomial is a tensor product of elements of , ,\ldots,
and .  Formally, we can write

where  and  for each  and
.  Thus we can identify  with the free -algebra over the product monoid .

Define the \emph{degree} of a monomial  to be the total length  of its constituent
words.  The degree of a polynomial is the maximum of the degrees of
its constituent monomials.

Let  be a -tuple of -algebras.
A \emph{valuation} of  in
 is a tuple of functions
, where .
Each  extends uniquely to an -algebra homomorphism
, and we
define the map  by
.  Often we will abuse terminology slightly and speak
of a valuation of  in .  Given ,
we say that  satisfies the partially commutative
identity  if  for all valuations
.

Next we introduce two valuations that will play an important role in
the subsequent development.  Recall that given a set of non-commuting
variables , we have a map  from the free -algebra to the
generic -dimensional matrix algebra.  We now define a valuation

by .    Likewise we
define

by .
We will usually elide the superscript  from 
 and   when it is clear from the 
context.

The following result generalises (part of) the Amitsur-Levitzki
theorem, by giving a lower bound on the degrees of partially
polynomial identities holding in tensor products of matrix algebras.

\begin{proposition}
Let  and let  be a field
extending .  Then the following are equivalent: (i)~The partially
commutative identity  holds in ; (ii)~.  Moreover, if  has degree strictly
less than  then (i) and (ii) are both equivalent to
(iii)~; and (iv)~ is identically  in .
\label{prop:sc-AL}
\end{proposition}
\begin{proof}
  The implication (ii)  (i) follows from the fact that
  any valuation from  to  factors through .  To
  see that (i)  (ii), observe that  is an
   matrix in which each entry is a polynomial in the
  commuting variables .  Condition (i) implies in particular
  that each such polynomial evaluates to  for all valuations of its
  variables in .  Since  is an infinite field, it must be that
  each such polynomial is identically zero, i.e., (ii) holds.

  The implications (ii)  (iii) and (iv)  (i)
  are both straightforward, even without the degree restriction on .

Finally we show that (iii)  (iv).  Let  be a monomial in , where  has length
.  Then  is an  matrix whose first row has a single non-zero entry, which
is the monomial

at index .

It follows that  maps the set of monomials in  of degree less than  injectively into a 
linearly independent set of matrices.  Condition (iv)
immediately follows.  \qed
\end{proof}
The hypothesis that  have degree less than  in
Proposition~\ref{prop:sc-AL} can be weakened somewhat, but is
sufficient for our purposes.

\subsection{Division Rings and Ore Domains}
A ring  with no zero divisors is a \emph{domain}.  If moreover each
non-zero element of  has a two-sided multiplicative inverse, then we
say that  is a \emph{division ring} (also called a
\emph{skew field}).  A domain  is a \emph{(right) Ore domain}
if for all , .  The
significance of this notion is that an Ore domain can be embedded in a
division ring of fractions~\cite[Corollary 7.1.6]{Cohn03}, something
that need not hold for an arbitrary domain.  If the Ore condition
fails then it can easily be shown that the subalgebra of  generated
by  and  is free on  and .  It follows that a domain 
that satisfies some polynomial identity is an Ore
domain~\cite[Corollary 7.5.2]{Cohn03}.

\begin{proposition}
  The tensor product of generic matrix algebras
  
  is an Ore domain for each .
\label{prop:ore}
\end{proposition}
\begin{proof}[sketch]
  We give a proof sketch here, deferring the details to
  Appendix~\ref{sec:appendix}.

By the Amitsur-Levitzki theorem,  satisfies a polynomial
identity.  Thus it suffices to show that  is a domain for each
.  Now it is shown in~\cite[Proposition 7.7.2]{Cohn03} that
 is a domain for each  and set of variables
.  While the tensor product of domains need not be a domain (e.g.,
), the proof in~\cite{Cohn03} can be adapted \emph{mutatis
  mutandis} to show that  is also a domain.

To prove the latter, it suffices to find central simple -algebras
, each of degree , such that the -fold tensor
product  is a domain.  Such an example
can be found, e.g., in~\cite[Proposition 1.1]{Saltman}.  Then, using
the fact that  for any algebraically
closed extension field of , one can infer that  is also a
domain.  \qed
\end{proof}

\section{Multitape Automata}
\label{sec:multitape}
Let  be a tuple of finite
alphabets.  We denote by  the product monoid .  Define the length of  to be  and write  for the set
of elements of  of length .  A \emph{multitape automaton} is a
tuple , where  is a set of \emph{states},  is a set of \emph{edges},  is a set of \emph{initial states}, and  is
a set of \emph{final states}.  A \emph{run} of  from state  to
state  is a finite sequence of edges 
such that .  The \emph{label} of  is
the product .  Define the \emph{multiplicity}
 of an input  to be the number of runs with label 
such that  and .  An automaton is
\emph{deterministic} if each state reads letters from a single tape
and has a single transition for every input letter.  Thus a
deterministic automaton has a single run on each input .

\subsection{Multiplicity Equivalence}
We say that two automata  and  over the same alphabet are
\emph{multiplicity equivalent} if  for all .  The
following result implies that multiplicity equivalence of multitape
automata is decidable.


\begin{theorem}[Harju and Karhum\"{a}ki]
Given automata  and  with  states in total, 
and  are equivalent if and only if  for all  of
length at most .
\label{thm:harju}
\end{theorem}

Theorem~\ref{thm:harju} immediately yields a \textbf{co-NP} bound for
checking language equivalence of deterministic multitape automata.
Given two inequivalent automata  and , a distinguishing input
 can be guessed, and it can be verified in polynomial time that
only one of  and  accepts .  A similar idea also gives a
\textbf{co-NP} bound for multiplicity equivalence in case the number
of tapes is fixed.  In general we note however that the
\emph{evaluation problem}---given an automaton  and input ,
compute ---is -complete.  Thus it is not clear
that the \textbf{co-NP} upper bound applies to the multiplicity
equivalence problem without bounding the number of tapes.

\begin{proposition}
The evaluation problem for multitape automata is -complete.
\end{proposition}
\begin{proof} 
  Membership in  follows from the observation that
  a non-deterministic polynomial-time algorithm can enumerate all
  possible runs of an automaton  on an input .

  The proof of -hardness is by reduction from \#SAT, the
  problem of counting the number of satisfying assignments of a
  propositional formula.  Consider such a formula  with 
  variables, each with fewer than  occurrences.  We define a
  -tape automaton , with each tape having alphabet ,
  and consider as input the -tuple .  The
  automaton  is constructed such that its runs on input  are in
  one-to-one correspondence with satisfying assignments of .
  Each run starts with the automaton reading the symbol  from a
  non-deterministically chosen subset of its tapes, corresponding to
  the set of false variables.  Thereafter it evaluates the formula
   by repeatedly guessing truth values of the propositional
  variables.  If the -th variable is guessed to be true then the
  automaton reads  from the -th tape; otherwise it reads 
  from the -th tape.  The final step is to read the symbol  from
  a non-deterministically chosen subset of the input tapes---again
  corresponding to the set of false variables.  The consistency of the
  guesses is ensured by the requirement that the automaton have read
   by the end of the computation.\qed
\end{proof}

\subsection{Decidability}
We start by recalling from~\cite{HarjuK91} an equivalence-respecting
transformation from multitape automata to single-tape weighted
automata.  

Recall that a single-tape automaton on a unary alphabet with
transition weights in a ring  consists of a set of \emph{states}
, \emph{initial states} ,
\emph{final states} , and \emph{transition matrix} .  Given such an automaton, define the
\emph{initial-state vector}  and
\emph{final-state vector}  respectively by

Then  is the weight of the (unique) input word of
length .

Consider a -tape automaton ,
where , and write
.  Recall the ring of
polynomials  as
defined in Section~\ref{sec:semi}.  Recall also that we can identify
the monoid  with the set of monomials in , where  corresponds to
---indeed  is the free -algebra on .

We derive from  an -weighted
automaton  (with a single tape and unary input
alphabet) that has the same sets of states, initial states, and final
states as .  We define the transition matrix  of  by
combining the different transitions of  into a single matrix with
entries in .  To this
end, suppose that the set of states of  is .
Define the matrix 
by  for .

Let  and  be the respective initial- and final-state
vectors of .  Then the following proposition is
straightforward.  Intuitively it says that the weight of the unary
word of length  in  represents the language of all
length- tuples accepted by .
\begin{proposition}
For all  we have
.
\label{prop:matrix}
\end{proposition}

Now consider two -tape automata  and .  Let the weighted
single-tape automata derived from  and  have respective
transition matrices  and , initial-state vectors 
and , and final-state vectors  and .  We
combine the latter into a single weighted automaton with transition
matrix , initial-state vector , and final-state vector
, respectively defined by:

\begin{proposition}
  Automata  and  are multiplicity equivalent if and only if
   for , where  is the
  total number of states of the two automata.
\label{prop:main}
\end{proposition}
\begin{proof}
Since  is a linearly independent subset of , from Proposition~\ref{prop:matrix} it
follows that  and  are multiplicity equivalent just in case
 for all .  The latter is clearly equivalent to  for all .  It remains to show that we can check
equivalence by looking only at exponents  in the range
.

Suppose that  for .  We show
that  for an arbitrary .

Consider the map , as
defined in (\ref{eq:psi}).  Observe that  is a
polynomial expression in  of degree at
most .  Therefore by Proposition~\ref{prop:sc-AL} ((ii)
 (iv)), to show that  it
suffices to show that

Let us write  for the pointwise application of 
to the matrix , so that 
is an  matrix, each of whose entries is an 
matrix belonging to
.  Since  is a homomorphism
and  and  are integer vectors,
(\ref{eq:goal4}) is equivalent to:


Recall from Proposition~\ref{prop:ore} that the tensor product of
generic matrix algebras  is an Ore domain and hence can be
embedded in a division ring.  Now a standard result about single-tape
weighted automata with transition weights in a division ring is that
such an automaton with  states is equivalent to the zero automaton
if and only if it assigns zero weight to all words of length 
(see~\cite[pp143--145]{Eilenberg74} and~\cite{Schutzenberger61}).
Applying this result to the unary weighted automaton defined by
, , and , we see that (\ref{eq:goal2}) is implied
by

But, since  is a homomorphism, (\ref{eq:goal3}) is implied by 

This concludes the proof.\qed
\end{proof}
Theorem~\ref{thm:harju} immediately follows from
Proposition~\ref{prop:main}.  

\begin{remark}
  The difference between our proof of Theorem~\ref{thm:harju} and the
  proof in~\cite{HarjuK91} is that we consider a family of
  homomorphisms of  into Ore
  domains of matrices---the maps ---rather than a single
  ``global'' embedding of  into
  a division ring of power series over a product of free groups.  None
  of the maps  is an embedding, but it suffices to use the
  lower bound on the degrees of polynomial identities in
  Proposition~\ref{prop:sc-AL} in lieu of injectivity.  On the other
  hand, the fact that  satisfies a polynomial identity
  makes it relatively straightforward to exhibit an embedding of the
  latter into a division ring.  As we now show, this approach leads
  directly to a very simple randomised polynomial-time algorithm for
  solving the equivalence problem.
\label{rem:difference}
\end{remark}

\subsection{Randomised Algorithm}

Proposition~\ref{prop:main} reduces the problem of checking
multiplicity equivalence of multitape automata  and  to checking
the partially commutative identities ,  in .  Since
each identity has degree less than , applying
Proposition~\ref{prop:sc-AL} ((iii)  (iv)) we see
that  and  are equivalent if and only if


Each equation  in (\ref{eq:pit})
asserts the zeroness of an  matrix of polynomials in
the commuting variables , with each polynomial having degree
less than .  Suppose that 
for some ---say the matrix entry with index
 contains a monomial with non-zero
coefficient.  By (\ref{eq:psi2}) such a monomial determines a term  with non-zero
coefficient in , and by
Proposition~\ref{prop:matrix} we have .

We can verify each polynomial identity in (\ref{eq:pit}), outputting a
monomial of any non-zero polynomial, using a classical identity testing
procedure based on the \emph{isolation lemma} of~\cite{MVV87}.

\begin{lemma}[\cite{MVV87}]
  There is a randomised polynomial-time algorithm that inputs a
  multilinear polynomial , represented as an
  algebraic circuit, and either outputs a monomial of  or that 
  is zero.  Moreover the algorithm is always correct if  is the
  zero polynomial and is correct with probability at least  if
   is non-zero.
\label{lem:pit}
\end{lemma}

The idea behind the algorithm described in Lemma~\ref{lem:pit} is to
choose a weight  for each variable  of
 independently and uniformly at random.  Defining the weight of a
monomial  to be , then
with probability at least  there is a unique minimum-weight
monomial.  The existence of a minimum-weight monomial can be detected
by computing the polynomial , since a
monomial with weight  in  yields a monomial of degree  in
.  Using similar ideas one can moreover determine the composition of
a minimum-weight monomial in .

Applying Lemma~\ref{lem:pit} we obtain our main result:
\begin{theorem}
Let  be fixed.  Then multiplicity equivalence of -tape automata can
be decided in randomised polynomial time.  Moreover there is a
randomised polynomial algorithm for the function problem of computing
a distinguishing input given two inequivalent automata.
\label{thm:main}
\end{theorem}

The reason for the requirement that  be fixed is because the
dimension of the entries of the transition matrix , and thus the
number of polynomials to be checked for equality, depends
exponentially on .

The above use of the isolation technique
generalises~\cite{KieferMOWW13}, where it is used to generate
counterexample words of weighted single-tape automata.  A very similar
application in~\cite{ArvindM08} occurs in the context of identity
testing for non-commutative algebraic branching programs.

\section{Conclusion}
We have given a simple randomised algorithm for deciding language
equivalence of deterministic multitape automata and multiplicity
equivalence of nondeterministic automata.  The algorithm arises
directly from algebraic constructions used to establish decidability
of the problem, and runs in polynomial time for each fixed number of
tapes.  We leave open the question of whether there is a deterministic
polynomial-time algorithm for deciding the equivalence of
deterministic and weighted multitape automata with a fixed number of
tapes.  (Recall that the 2-tape case is already known to be in
polynomial time~\cite{FriedmanG82}.)  We also leave open whether there
is a deterministic or randomised polynomial time algorithm for solving
the problem in case the number of tapes is not fixed.






\appendix
\section{Proof of Proposition~\ref{prop:ore}}
\label{sec:appendix}
We first recall a construction of a \emph{crossed product division
  algebra} from~\cite[Proposition 1.1]{Saltman}.  Let 
be commuting indeterminates and write
 for the field of rational functions
obtained by adjoining  to .
Furthermore, let  be a field extension whose Galois group is
generated by commuting automorphisms , each
of order , which has fixed field .  (Such an extension can easily
be constructed by adjoining extra indeterminates to , and having
the  be suitable permutations of the new indeterminates.)
For each , , write  for the subfield of 
that is fixed by each  for ; then define  to
be the -algebra generated by  and  such that  for all .  Then each  is a simple
algebra of dimension  over its centre .  It is shown
in~\cite[Proposition 1.1]{Saltman} that the tensor product  can be characterised as the
localisation of an iterated skew polynomial ring---and is therefore a
domain.

The following two propositions are straightforward adaptations
of~\cite[Proposition 7.5.5.]{Cohn03} and~\cite[Proposition
  7.7.2]{Cohn03} to partially commutative identities.
\begin{proposition}
Let .  If the partially commutative identity  holds in
 then it also holds in  for any
extension field  of .
\label{prop:extend}
\end{proposition}
\begin{proof}
Noting that the  are all isomorphic as -algebras, let
 be a basis of each  over its centre .
For each variable  appearing in , introduce commuting
indeterminates , , and write
.  Then we can express  in the form

where .

By assumption, each  evaluates to  for all values of the
 in .  Since  is an infinite field it follows that each
 must be identically zero.  Now we can also regard
 as a basis for  over .
Then by (\ref{eq:formula}),  also on .  \qed
\end{proof}

\begin{proposition}

  is a domain.
\label{prop:domain}
\end{proposition}
\begin{proof}
Recall that if  is an algebraically closed field extension of ,
then we have  for each .  By
Proposition~\ref{prop:extend} it follows that an identity  holds
in  if and only if it holds in .  But by
Proposition~\ref{prop:sc-AL} the latter holds if and only if
 is identically zero.

To prove the proposition it will suffice to show that the image of
 contains no zero divisors, since the latter is a surjective map.
Now given  with , we have that  satisfies the identity .  Since  is a domain, it follows that it satisfies the identity
 for any  in .  But now  satisfies the identity  for any .  Since  can
take the value of an arbitrary matrix (in particular, any matrix unit)
it follows that  satisfies
either the identity  or , and so, by
Proposition~\ref{prop:sc-AL} again, either  or
.  \qed
\end{proof}

\subsubsection{Acknowledgments}
The author is grateful to Louis Rowen for helpful pointers in the
proof of Proposition~\ref{prop:ore}.

\bibliographystyle{plain} 

\begin{thebibliography}{10}

\bibitem{AL50}
S.A.~Amitsur and J.~Levitzki.
\newblock Minimal identities for algebras.
\newblock {\em Proceedings of the American Mathematical Society}, 1:449--463,
  1950.

\bibitem{ArvindM08}
V.~Arvind and P.~Mukhopadhyay.
\newblock Derandomizing the isolation lemma and lower bounds for circuit size.
\newblock In {\em APPROX-RANDOM}, volume 5171 of {\em Lecture Notes in Computer
  Science}, pages 276--289. Springer, 2008.

\bibitem{BogdanovW05}
A.~Bogdanov and H.~Wee.
\newblock More on noncommutative polynomial identity testing.
\newblock In {\em IEEE Conference on Computational Complexity}, pages 92--99.
  IEEE Computer Society, 2005.

\bibitem{Cohn03}
P.M.~Cohn.
\newblock {\em Further Algebra and Applications}.
\newblock Springer-Verlag, 2003.

\bibitem{Eilenberg74}
S.~Eilenberg.
\newblock {\em Automata, Languages, and Machines, Vol.A}.
\newblock Academic Press, 1974.

\bibitem{ElgotM63}
C.C.~Elgot and J.E.~Mezei.
\newblock Two-sided finite-state transductions (abbreviated version).
\newblock In {\em SWCT (FOCS)}, pages 17--22. IEEE Computer Society, 1963.

\bibitem{FriedmanG82}
E.~P. Friedman and S.~A. Greibach.
\newblock A polynomial time algorithm for deciding the equivalence problem for
  2-tape deterministic finite state acceptors.
\newblock {\em SIAM J. Comput.}, 11(1):166--183, 1982.

\bibitem{Griffiths68}
T.V.~ Griffiths.
\newblock The unsolvability of the equivalence problem for -free
  nondeterministic generalized machines.
\newblock {\em J. ACM}, 15(3):409--413, July 1968.

\bibitem{HarjuK91}
T.~Harju and J.~Karhum{\"a}ki.
\newblock The equivalence problem of multitape finite automata.
\newblock {\em Theor. Comput. Sci.}, 78(2):347--355, 1991.

\bibitem{KieferMOWW13}
S.~Kiefer, A.~Murawski, J.~Ouaknine, B.~Wachter, and J.~Worrell.
\newblock On the complexity of equivalence and minimisation for {Q}-weighted
  automata.
\newblock {\em Logical Methods in Computer Science}, 9, 2013.

\bibitem{Malcev48}
A.I.~ Malcev.
\newblock On the embedding of group algebras in division algebras.
\newblock {\em Dokl. Akad. Nauk}, 60:1409--1501, 1948.

\bibitem{MVV87}
K.~Mulmuley, U.V.~Vazirani, and V.V.~Vazirani.
\newblock Matching is as easy as matrix inversion.
\newblock In {\em STOC}, pages 345--354, 1987.

\bibitem{Neumann49b}
B.H.~Neumann.
\newblock On ordered groups.
\newblock {\em Amer. J. Math.}, 71:1--18, 1949.

\bibitem{Neumann49}
B.H.~Neumann.
\newblock On ordered division rings.
\newblock {\em Trans. Amer. Math. Soc.}, 66:202--252, 1949.

\bibitem{RabinS59}
M.~Rabin and D.~Scott.
\newblock Finite automata and their decision problems.
\newblock {\em IBM Journal of Research and Development}, 3(2):114--125, 1959.

\bibitem{Sakarovich03}
J.~Sakarovich.
\newblock {\em Elements of Automata Theory}.
\newblock Cambridge University Press, 2003.

\bibitem{Saltman}
D.~Saltman.
\newblock {\em Lectures on Division Algebras}.
\newblock American Math. Soc., 1999.

\bibitem{Schutzenberger61}
M.-P.~Sch\"{u}tzenberger.
\newblock On the definition of a family of automata.
\newblock {\em Inf.\ and Control}, 4:245--270, 1961.

\bibitem{Tzeng}
W.~Tzeng.
\newblock A polynomial-time algorithm for the equivalence of probabilistic
  automata.
\newblock {\em SIAM Journal on Computing}, 21(2):216--227, 1992.

\end{thebibliography}

\end{document}
