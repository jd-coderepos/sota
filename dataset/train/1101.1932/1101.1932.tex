\documentclass[12pt]{article}
\textheight=9in
\textwidth=6.5in
\headheight=0in
\headsep=0in
\topmargin=0in
\oddsidemargin=0in

\usepackage{amsfonts}
\usepackage{epsfig}
\usepackage{epstopdf}

\def\ddc{{\delta C}}
\def\ddp{{\delta P}}
\def\bbz{{\mathbb Z}}
\def\dk{{\cal B}_K}
\begin{document}
\pagestyle{plain}
\setcounter{page}{1}

\baselineskip16pt

\begin{titlepage}

\begin{flushright}
MIT-CTP-4202
\end{flushright}
\vspace{8 mm}

\begin{center}

{\Large \bf Efficient tilings of de Bruijn and Kautz graphs\\}
\vspace{3mm}

\end{center}

\vspace{7 mm}


\begin{center}

Washington Taylor, Jud Leonard, 
and Lawrence C.\ Stewart

\vspace{3mm}
{\small \sl Center for Theoretical Physics\\
MIT\\
Cambridge, MA 02139, USA} \0.15in]
{\small \sl 
Serissa Research\\
7 Erwin Road\\
Wayland, MA 01778, USA}\\

\vspace{3mm}

{\small \tt wati} at {\tt mit.edu, jud.leonard} at {\tt gmail.com, stewart} at {\tt serissa.com} \\

\end{center}

\vspace{8 mm}

\begin{abstract}
Kautz and de Bruijn graphs have
a high degree of connectivity which makes them ideal candidates for
massively parallel computer network topologies.  In order to realize a
practical computer architecture based on these graphs, it is useful to
have a means of constructing a large-scale system from smaller,
simpler modules.  In this paper we consider the mathematical problem
of uniformly tiling a de Bruijn or Kautz graph.  This can be viewed as
a generalization of the graph bisection problem.  We focus on the
problem of graph tilings by a set of identical subgraphs.  Tiles
should contain a maximal number of internal edges so as to minimize
the number of edges connecting distinct tiles.  We find necessary and
sufficient conditions for the construction of tilings.  We derive a
simple lower bound on the number of edges which must leave each tile,
and construct a class of tilings whose number of edges leaving each
tile agrees asymptotically in form with the lower bound to within a
constant factor.  These tilings make possible the construction of
large-scale computing systems based on de Bruijn and Kautz graph
topologies.
\end{abstract}
\vspace*{0.2in}
Keywords: Parallel architectures (C.1.4), Graph theory (G.2.2)

\vspace{1cm}
\begin{flushleft}
\today
\end{flushleft}
\end{titlepage}
\newpage


\section{Introduction}

The family of graphs known as de Bruijn graphs
\cite{db},
Kautz graphs \cite{Kautz}, and generalized de Bruijn and Kautz graphs
\cite{ii1, ii2, rpk, dch}, have closely
related mathematical structure.  These graphs exhibit a high degree of
connectivity, which makes them natural candidates for massively
parallel computer network topologies \cite{bp}.  In particular, for
graphs of degree  and diameter , de Bruijn and Kautz graphs
achieve the largest possible asymptotic number of vertices  \cite{pz}.

The efficient connectivity of de Bruijn and Kautz graphs makes it
difficult to partition these graphs into collections of subgraphs in
such a way as to minimize the number of edges connecting the subgraphs
to one another.  Precisely such a partition is desirable, however, in
order to physically realize a computer network topology based on a
graph of this type.  The problem of constructing a computer based on a
de Bruijn or Kautz graph from a set of isomorphic subgraphs (circuit
boards) connected together by a minimal number of edges motivates the
problem we consider in this paper of tiling de Bruijn and Kautz graphs
with isomorphic subgraph tiles having a maximal number of internal
edges.

In this paper, we define the tiling problem for de Bruijn and Kautz
graphs.  This problem can be thought of as a generalization of the
graph bisection problem applied to these graphs, where instead of
splitting the graph into two parts we wish to split the graph into a
larger number of equal-size components.  We derive a lower bound on
the number of edges which must connect the tiles of a tiling.  Based
on the criterion that a tile decomposition should be scalable to
construct graphs of arbitrary size, we give necessary and sufficient
conditions for the construction of a uniform tiling.  We construct a
family of scalable tilings which asymptotically realize the lower
bound on connecting edges to within a constant factor.  We also
provide examples of optimal tilings for small tiles.  Finally, we show
that not only de Bruijn and Kautz graphs but also their
generalizations can be tiled in a similar way using the basic
mathematical structure underlying de Bruijn graphs.


The structure of this paper is as follows.  In section 2, we fix
notation and state the fundamental problem addressed in the paper.  In
section 3, we derive a simple upper bound on the number of edges which
can be incorporated into a tile of given size, equivalent to a lower
bound on the number of connecting edges.  In section 4, we prove
necessary and sufficient conditions for the construction of a tiling
of a de Bruijn or Kautz graph, and describe a general approach for the
construction of tilings.  Section 5 describes the generalization of
the results of section 4 to generalized de Bruijn and Kautz graphs.
Section 6 contains some examples of tilings constructed using the
method of section 4.  In section 7 we give an explicit construction of
a family of tiles which asymptotically realize the lower bound of
section 3 to within a constant factor, as the tile size becomes large.
Section 8 contains a brief discussion of the application of the tiling
approach developed in this paper to supercomputer architectures,
including a mathematical argument for the optimality of degree 3
Kautz/de Bruijn graphs in certain contexts.  Section 9 contains some
brief concluding remarks.

A related formulation of this problem was given in \cite{Viterbi},
along with some explicit decompositions of degree 2 de Bruijn graphs
into isomorphic subgraph tiles similar to those described in this
paper.  We focus in this paper on the directed graph (digraph) form of
de Bruijn and Kautz graphs.  Similar considerations could be used for
tiling the undirected forms of the graphs.

\section{Definitions and problem statement}

We begin by defining some notation.

First we define a (directed) de Bruijn graph.  The de Bruijn graph 
 of degree
 and diameter  
is a digraph with  vertices.  The vertices can be labeled
by strings of  integers base 

so .  The  de Bruijn graph 

has edges

for all values of .
There is thus a natural one-to-one map between edges of 
and vertices of .  This leads naturally to an
inductive construction of  as a family of iterated
line digraphs.

A (directed)  Kautz graph  of degree  and diameter 
can be defined similarly as a digraph with 
vertices labeled by strings

where the integers 
are taken base , but subject to the condition that
adjacent integers must
be distinct.  The edges of  are given in a similar
fashion to (\ref{eq:dbe}) by


Some simple examples of de Bruijn and Kautz graphs for small values of
the degree  and diameter are depicted in
Figures~\ref{f:dk22}-\ref{f:dk32}
\begin{figure}
\begin{center}
\begin{picture}(200,100)(- 100,- 50)
\put(10,10){\makebox(0,0){\epsfig{file=dk22.eps,width=9cm}}}
\put(-70,-50){\makebox(0,0){(a)}}
\put(70,-50){\makebox(0,0){(b)}}
\end{picture}
\end{center}
\caption[x]{\footnotesize  The de Bruijn graph  (a)
and Kautz graph   (b) of degree  and diameter 2.}
\label{f:dk22}
\end{figure}
\begin{figure}
\begin{center}
\begin{picture}(200,100)(- 100,- 45)
\put(5,15){\makebox(0,0){\epsfig{file=dk23.eps,width=10cm}}}
\put(-77,-50){\makebox(0,0){(a)}}
\put(77,-50){\makebox(0,0){(b)}}
\end{picture}
\end{center}
\caption[x]{\footnotesize  The de Bruijn graph  (a)
and Kautz graph   (b) of degree  and diameter 3.}
\label{f:dk23}
\end{figure}


Kautz graphs and de Bruijn graphs both have uniform in-degree and
out-degree .
Kautz graphs and de Bruijn graphs are closely related.  Indeed, if we
take a Kautz graph  with vertices , we can map the vertices to the vertices of the de Bruijn graph
 through

This map is a (many to one) graph homomorphism, since if
 is an edge of
 it follows immediately that the associated
 is an edge of
.  We can thus label every vertex in the Kautz graph
by , where  and
 is a vertex in a de Bruijn graph .  The edges of the Kautz graph in this notation are

where  is taken modulo .  
We will use this relationship between
Kautz and de Bruijn graphs in the following section.
\vspace*{0.1in}

\noindent
{\bf Notation}: if a directed graph has an edge  we say that  is
a {\it child} of  and  is a {\it parent} of .  We denote the
set of children of  by  and the set of parents of  by .
\vspace*{0.1in}

We are interested in finding tilings of de Bruijn and Kautz graphs with certain
properties, where a tiling is defined as follows:
\vspace*{0.05in}

\noindent
{\bf Definition}: A {\it tiling} of a graph  is a one-to-one (on
vertices) graph embedding  taking the
product of an index set  and the vertices  of a smaller graph
 (the {\it tile}) into the vertices of the graph  such that if
 is an edge of  then  is an edge of
, .
\begin{figure}
\begin{center}
\begin{picture}(200,140)(- 100,- 47)
\put(5,17){\makebox(0,0){\epsfig{file=dk32.eps,width=10cm}}}
\put(- 82,-55){\makebox(0,0){(a)}}
\put(82,-55){\makebox(0,0){(b)}}
\end{picture}
\end{center}
\caption[x]{\footnotesize  The de Bruijn graph  (a)
and Kautz graph   (b) of degree  and diameter 2.}
\label{f:dk32}
\end{figure}
\vspace*{0.1in}

We wish to find tilings of the de Bruijn and Kautz graphs  and  such that the number of edges of
the tile  is maximized.  This problem is motivated from the problem
of designing a computer with a de Bruijn or Kautz topology, where we
wish to construct the machine using a set of identical boards (the
tiles), minimizing the number of wires connecting the boards by
maximizing the number of on-board connections corresponding to edges
of .
\vspace*{0.1in}
\noindent
{\bf Problem 1} For fixed , and tile size 
such that ,
find a  tiling of  which maximizes , the number
of edges in .
\vspace*{0.1in}

This problem has previously been formulated in the case , and
some example tilings described in \cite{Viterbi}.

\vspace*{0.1in}
\noindent
{\bf Problem 2} For fixed , and tile size ,
find a tiling of  which maximizes .
\vspace*{0.1in}


\vspace*{0.1in}
\noindent
{\bf Definition}: a tiling is {\it scalable} if the associated tile
 can be used to tile a de Bruijn or Kautz graph of any size or  (for  fixed ).
\vspace*{0.1in}

This property is useful practically because machines of different
sizes can be built with the same modular component.
The condition that the tiling be scalable also enables us to make more
general statements about the set of allowed tilings.
\vspace*{0.1in}

\noindent
{\bf Problem 3}: For a fixed degree  and tile size ,
find a tile  with  and
maximum 
which leads to a scalable tiling
for either de Bruijn or Kautz graphs.
\vspace*{0.1in}

When  is prime, any tile must have size
 for some .  For non-prime , it may be possible to find
tiles of other sizes giving scalable tilings.  We will focus in this
paper on tiles with sizes which are powers of .  Such tiles can be
constructed for all .

\vspace*{0.1in}
\noindent
{\bf Definition:} We define the maximum number of edges  which
can be realized in any tile which has  vertices and 
which gives a scalable tiling for de Bruijn and Kautz graphs of degree
 to be
.

\vspace*{0.1in}
\noindent
{\bf Definition:} We define the minimum possible number of ``broken'' edges in
any tile with  vertices to be
.
\vspace*{0.1in}



In this paper we construct a class of tilings which turn out to have
the following useful property 

\vspace*{0.1in}
\noindent
{\bf Definition}: a tiling has the {\it parallel routing} property if for
each vertex , the children of   
are mapped
under  to vertices in  with the same set of 
values  (for each ).
\vspace*{0.1in}
In other words, defining the projection  of a vertex in  to
the tile  through

the parallel routing property states that

for any  in the index set .
\vspace*{0.1in}

This property is useful practically because it simplifies the wiring
of a machine based on a tiling with this property.  

\section{Upper bound on internal tile edges}
\label{sec:asymptotic}

In this section we derive an upper bound on the number of edges in the
tiles of a tiling, equivalent to a lower bound on the number of edges
which must leave each tile.  We assume here that tiles are of size ; a
similar bound can be found for tiles with sizes not of this form for
composite  by the same analysis.  The lower bound found here takes a simple
asymptotic form as  for fixed .  We focus on
de Bruijn tilings, but analysis of Kautz tilings gives the same form
for the bound.


\vspace*{0.1in}
\noindent
{\bf Theorem 1}: The number of edges achievable in a tile which
leads to scalable tilings has an upper bound


\vspace*{0.1in}
\noindent
{\bf Proof}:

We find this upper bound by considering a set of paths which leave the
tile; we bound the number of these paths which cross any given edge,
which gives us a lower bound on the number of edges which must leave
the tile.  This basic method, described in \cite{Leighton}, is used in
\cite{rttv} to construct a lower bound on the edge bisection width of
de Bruijn and Kautz graphs.


Consider a tiling of the de Bruijn graph  by tiles of
size .  For a given tile , consider the  paths
of length  which go from a node in  to a node in the de Bruijn
graph outside tile .  Each of these must cross at least one edge
which begins at a node in  and ends at a node outside .  We
refer to such at edge as an edge ``leaving'' the tile .  Each edge
leaving the tile is traversed by at most  of the paths
under consideration (since the edge fixes  of the  's
defining the path, but may appear at any of  places in the path).
Thus, the number of edges leaving the tile () is at
least

for any choice of paths, and for any tile .
Thus we have shown

for any ,
which proves the theorem.
\vspace*{0.1in}

\noindent
{\bf Corollary}
For fixed  and large  the lower bound on
the number of edges leaving the tile goes asymptotically as


\vspace*{0.1in}
\noindent
{\bf Proof}:

By taking , we have

In Section \ref{sec:efficient} we explicitly construct a class of
tilings which realize this asymptotic bound to within a factor of 2.

The asymptotic lower bound we have found here for the number of broken
edges is similar in form to that found in \cite{rttv} for  bisection of
de Bruijn and Kautz graphs.  Those authors found that the edge
bisection width  satisfies the asymptotic inequalities

Note that the asymptotic form of the lower bounds agrees for , where the tiling and bisection problems are equivalent.



\section{Structure  of  tiles}

In this section we prove a necessary condition which a tile of size
 must satisfy to lead to a scalable tiling.  We then prove a
related but slightly stronger sufficient condition for scalable
tilings, by giving a general construction of a class of tilings based
on any tile which satisfies the sufficient condition.

We begin with a few definitions

\vspace*{0.1in}
\noindent
{\bf Definition}: A  {\it stratification} of a directed
graph  is a map
 from the vertices of  into the
integers such that for every edge  in , we have



We refer to  as the {\it level} of the vertex  in the
stratified graph .  By convention, we take the smallest level in a
stratification to be 0 by shifting all levels equally.
We refer to a graph which admits a stratification as {\it stratifiable}.
\vspace*{0.1in}

\noindent
{\bf Definition}: A {\it loop} is a directed graph
which is topologically equivalent to a circle.
\vspace*{0.1in}

Note that the edges in a loop need not have compatible orientations.
It is straightforward to show that a loop is stratifiable if and only
if it contains an equal number of edges of each orientation.
For example, the loop defined by the graph containing 3 vertices  and edges  is unstratifiable since a
stratification would give
 from the first two
edges and  from the third edge.


\vspace*{0.1in}
\noindent
{\bf  Lemma 1}: A graph  is stratifiable if and only if it contains
no unstratifiable loops as subgraphs.


\vspace*{0.1in}
\noindent
{\bf Proof}: If  contains an unstratifiable loop  then  is
clearly not itself stratifiable since any stratification of  would
provide a stratification of .  If  contains no unstratifiable
loops, then we can explicitly construct a stratification of  by
choosing a single reference vertex in each connected component of 
and assigning the level of each other vertex  in  by
implementing the rule (\ref{eq:stratification-rule}) on each of the
edges needed to reach  on any path from a reference vertex,
allowing the path to traverse edges in either direction.  In the
absence of unstratifiable loops, this assignment is path independent
and provides a stratification of .
\vspace*{0.1in}

We can now prove the following
\vspace*{0.1in}

\noindent
{\bf Theorem 2}: A necessary condition for a tile  of size 
to give scalable tilings of  and 
for all  is that  must be stratifiable.


\vspace*{0.1in} \noindent {\bf Proof}:
We begin by assuming that  is unstratifiable and therefore
contains an unstratifiable loop.  We will show that this assumption
leads to the conclusion that  cannot be used to construct
scalable tilings, thus proving the theorem by contradiction.

Assuming that a given tile  contains an unstratifiable loop  of
length , we write  in terms of the sequence of vertices in 
it traverses.

where  denote the sets of indices of vertices associated
with edges 
which are traversed in a forward/backward direction in the loop
, so that .
Since  is unstratifiable, we must have


Now, imagine that the tile   gives a scalable
tiling of de Bruijn graphs  for  but nonetheless
contains the unstratifiable loop  given in  (\ref{eq:loop-l}).  In a
tiling of the graph  , there will be  nodes
 which project to each node  through .
In particular, there will be  nodes  with
.  Choose any such node  which
can be written as  for some fixed
.  Then we can define a loop in  through

whose vertices must satisfy

Since , when we follow the loop around, the string
of characters defining  has been shifted to the left  times.  But we have returned to  after going around the
full loop; this implies that the bulk of the word  is invariant under shifting left by .
Specifically,

But only  nodes have this property (such nodes are
completely defined by giving ), so we cannot have 
 nodes with  when .  Thus, the
assumption of an unstratifiable loop is incompatible with the
assumption of a scalable tiling for , so we conclude that if
 can be used to construct a scalable tiling it cannot contain an
unstratifiable loop, and therefore must be stratifiable.  The theorem
is thus proven.
\vspace*{0.2in}

The condition of stratifiability is not by itself sufficient to
guarantee that a tile gives rise to a scalable tiling.   In some
cases, stratifiable loops can form obstructions to the construction of
a tiling.  We now will proceed to prove that scalable tilings can be
constructed on stratifiable tiles with certain types of loops.  First,
however, we need a few more definitions and a lemma.

\vspace*{0.1in}


\noindent
{\bf Lemma 2}: A tile of size  giving scalable tilings of
 and   for  must be a
stratifiable subgraph of

\vspace*{0.1in}

\noindent
{\bf Proof}: The tile must be stratifiable by Theorem 2.  It must be a
subgraph of  since it must tile  with .
Note that since , the subgraph contains all vertices of
, but will only contain a subset of the edges.
\vspace*{0.1in}

We can thus associate with any scalable tiling of a de Bruijn or Kautz graph
 or   a map
.  For the tilings which we will
construct, this map extends to a complete graph homomorphism
.  
We define several properties for
maps of this type
\vspace*{0.1in}

\noindent
{\bf Definition}:  A map  has  the
{\it parent distribution} property when

\vspace*{0.1in}

\noindent
{\bf Definition}:  A map  has  the
{\it child distribution} property when


Note in particular that if  has both
the parent and child distribution properties, then for any vertex  with , there is a unique parent  of  with  and a unique child
 of  with  for each .
\vspace*{0.1in}

We now prove by example that there exist maps 
 with
 or  which have both the parent and child
distribution properties, for any 


For a de Bruijn graph , we begin by defining the th
{\it discrete 
differentials}.  For each vertex  we define

Let us fix   and take .
Then the discrete 
differentials define a map

which is a graph homomorphism

since if  is an edge of  then  is an
  edge of .
It is straightforward to check that this graph homomorphism has both
  the child and parent distribution properties.   
Note that using this homomorphism we can relabel the vertices of the graph
 
  using the first  's and the 's


Combining the discrete differentials on the de Bruijn graph
 with the homomorphism
(\ref{eq:kd-homomorphism}) 
gives us a map 

defined through , with  defined as in
(\ref{eq:kd-homomorphism}).  
Again, it is easy to
verify that this graph homomorphism has the child and parent
distribution properties.
Using this homomorphism, we
can label the nodes in a Kautz graph
 through

\vspace*{0.1in}

\noindent
{\bf Definition}: We define the {\it height} of a stratifiable loop
 to be the difference 
between the largest and smallest levels in a loop.
\vspace*{0.1in}

We can now prove the following
\vspace*{0.1in}

\noindent
{\bf Theorem 3} Given a stratified tile  of size  which is a
subgraph of  (containing all vertices but not all
edges
of  ), tilings of  and  can be constructed for all ,
as long as  contains no loops of height .
Such a tiling can be constructed explicitly from any map
 with the child and parent distribution properties.


\vspace*{0.1in}
\noindent
{\bf Proof}: We showed above by explicit construction that a map
 with the child and parent distribution
properties exists for any , for  a de Bruijn or Kautz graph of
degree  with .  Let us take any particular such .
From  we can define the tiling constructively as follows: choose
a vertex  of the tile .  Associated with
this vertex of  there is a set of vertices in our de Bruijn or
Kautz graph  which map to  under .  We can arbitrarily
associate this set of vertices with the elements  in the index set
 to define  for all .  For example, using the
discrete differential map described above, given a specific  we can define the index set by the leading indices before
the  on this set of vertices in the notation of
(\ref{eq:labeling-d}, \ref{eq:labeling-k}) ({\it i.e.},  for de Bruijn,  for Kautz).  

We have now defined  for all  and for a specific 
in .
Now, if
there are any edges in  containing  ({\it i.e.} directed edges
 or  beginning or
ending on ), we can extend the
definition.  Say  contains the edge .  For each , using the child distribution property, there is a unique vertex
 in  which is a child of  and which has .  We can thus define  for each .  These
vertices  are distinct for , by the parent
distribution property---if  then  and
 would both be parents of the same vertex with the same
value of , which is impossible given the parent distribution
property.  Instead of extending along an edge where  is the
parent, we could equivalently extend along an edge where  is the
child, using an equivalent argument where the roles of parents and
children are exchanged.  We continue in this way for further edges
containing either  or .  Each time we include a new edge we
define the map  for a new value of .  If the graph  is
disconnected, we run out of edges before the map  is completely
constructed.  In this case we choose a new vertex  where  is
not yet defined and proceed as above with  in place of .
In this fashion, we can construct the complete tiling.

It remains to be shown that the construction of the tiling we have
just presented is well-defined in that any loops existing in the tile
do not lead to incompatible definitions for the tiling by following
different paths to extend the definition of the tiling.
This is straightforward to demonstrate for any loop whose height is
less or equal to . Given a loop  of height 
we need to show that
the loop can be consistently lifted to a loop in the graph  by
following the edges and using the child and parent distribution
properties as in the above construction.  Let us start at a vertex
 such that .
We lift  to a vertex  with
.  Labeling the vertices of the loop in order as
 where  is the length of the
loop we can use the 
child and parent distribution properties to lift each  to a
vertex  as above, following the edges of the loop one after
another.  We need to show that when we return to  the final value
of  is the same as .  Assume that ---a
similar argument proceeds for Kautz graphs.  We can write
.  From the form of the de Bruijn graph edges,
and since  has the lowest level of any vertex in the loop , we
see that all vertices  contain the coordinates , shifted to the right by  places.
But then, .  Since by the child distribution property
there is a unique th child of 
with , we must have  when , so
the loop of height  does not obstruct the construction of
the tiling.  This completes the proof of the theorem.
\vspace*{0.1in}

\noindent
{\bf Corollary}: The tilings constructed by the method of Theorem 3
have the parallel routing property.
\vspace*{0.1in}

\noindent
{\bf Proof}: This follows from the use of  with parent and child
distribution properties, since

is independent of  so (\ref{eq:parallel}) holds for all .
The parallel routing property can also be seen to follow immediately
from the fact that  is a graph homomorphism.
\vspace*{0.2in}

We have now provided results which make it possible to systematically
search for optimal solutions to Problem 3.  We know that for tile size
, scalable tilings are given by stratifiable subgraphs of
.  We can thus consider the set of stratifiable subgraphs of
, ordered by their number of edges.  We know that all
stratifiable subgraphs with no loops of height  can give
scalable tilings.  Thus, if a stratifiable subgraph of  can be
realized with the maximal number of edges and , this
provides a solution to Problem 3.  In some cases, the stratifiable
subgraph of  with the maximal number of edges may have a loop
of height .  In such a case, a further check is needed to
verify that such a subgraph gives a scalable tiling; generally this
will not be possible, but it may be possible in special cases.  In any
case, finding a subgraph of  with the maximal number of edges
subject to the condition that there are no loops of height 
will give rise to a good set of scalable tilings.

We conclude this section with some further comments.

First, we note that the discrete differential construction above is
not the only way to realize a map  
with the parent and child distribution properties.  Indeed,
consider any function 

with the following properties:
\begin{itemize}
\item  for each   gives a one-to-one
  map from  ({\it i.e.}, for fixed
  ,  takes different values for each )
\item  for each   gives a one-to-one
  map from  ({\it i.e.}, for fixed
  ,  takes different values for each )
\end{itemize}

If we replace  in (\ref{eq:dcc})
with any other function  with these
properties, we see that the associated map  still has the child and parent distribution properties when
 is either a de Bruijn or Kautz graph.  Using such a more general
map in the construction of Theorem 3 provides a more general class of
constructions of scalable tilings with the parallel routing property.


Note however that not every graph homomorphism of the form  has the parent and child distribution
properties.  For example, while the map  defined by taking a contiguous subset  of  characters in the string 
is a homomorphism, it does not have these properties, and indeed
cannot be used to define a tiling.  To see this clearly in a specific
simple case, consider the case .  We cannot tile the
27-node de Bruijn graph  using tiles with vertices
addressed by .  For example, the 3 parents of the vertex
 have coordinates .  These vertices
all have the same value of , so the graph homomorphism
taking  does not have the parent
distribution property.  Assume  now for example that our tile includes the
edge .  The vertex , which should be associated with
the vertex  on the tile has no parent with , so the
edge  cannot be contained in the tile.  Similar problems
arise with any other edge, so we cannot tile  with tiles
having vertices addressed by  and more than 0 edges.

\section{Tilings of generalized de Bruijn and Kautz graphs}
\label{sec:generalized}

In this section we describe tilings of generalized de Bruijn and
Kautz graphs.
So far, we have discussed only  de Bruijn  and Kautz graphs
of sizes  and  respectively.
Generalizations of the de Bruijn and Kautz graphs to other sizes are
described in \cite{ii1, ii2, rpk, dch}.
We show here that the construction of the previous section gives
tilings by tiles of size   for generalized de Bruijn and Kautz
graphs with  vertices for any integer .  

\subsection{Generalized de Bruijn graphs}

A generalized de Bruijn graph of degree  having  vertices can be
defined by taking the directed graph on vertices  with
edges

If , it is easy to see that this definition agrees with the
one in (\ref{eq:dbe}) by simply taking  to be the base
 representation of the integer  for each vertex .
Multiplication by  simply shifts the base  representation left
by one digit, adding  shifts in an arbitrary new digit 
and modding by  truncates to  digits base .
More generally, we can write any   in the form

where  and  and  are relatively prime, {\it i.e.} .  We
can then represent any vertex  in mixed-base form as

As before, the transformation rule associated with edges (\ref{eq:gdb})
shifts the 's left.  This transformation rule
takes  where

Since  and  are relatively prime, multiplication by  is
invertable mod , so that
for any fixed   the map defined in (\ref{eq:fmap}) from  is a one-to-one map.  In particular, given 
there exists a unique  such that (\ref{eq:fmap}) is satisfied.
This demonstrates that the map from the generalized de Bruijn graph
with  vertices to the regular de Bruijn with  vertices given
by dropping the first digit  in (\ref{eq:db-mixed}) has the parent
distribution property (\ref{eq:pdp}).  This map also clearly has the
child distribution property (\ref{eq:cdp}), which follows from the
fact that the children of any node with mixed-base representation
 are just  with  given by
(\ref{eq:fmap}) and .  Since this map has both the
parent and child distribution properties, so does any  map to
a smaller de Bruijn formed by the  composition of this map with a map
 also having the parent
and child distribution properties.  This shows that the tiles which
can be used to construct tilings of standard de Bruijn and Kautz
graphs in Theorem 3 of the previous section can also be used to tile
generalized de Bruijn graphs.  We give
some examples of such tilings in Section \ref{sec:examples}

\subsection{Generalized Kautz graphs}

Similar to the generalized de Bruijn graphs defined through
(\ref{eq:gdb}), a Kautz graph of degree  having  vertices can be
defined by taking the directed graph on vertices  with
edges

To relate this to the standard Kautz graph in the case , we again introduce a mixed mode representation for each vertex
, writing  with .  We denote the digits in
this mixed mode representation by

where the last digit is  if  is odd, and  if  is
even.
The barred digits are defined to be , giving a
shift register representation of the vertices.  From (\ref{eq:gk}) we
see that the graph edges are given by

where

In particular, if we choose , we have , so , which is precisely
the transformation rule on edges in (\ref{eq:relabeling}) if we
 identify .  Thus, when  we have the standard
 Kautz graph, while for other  we have a generalization.

As in the previous subsection, the invertibility of multiplication by
 mod  guarantees that the map to a  node de Bruijn given by
dropping the first digit  in the mixed mode representation
(\ref{eq:gk-mixed}) has the parent and child distribution properties,
and therefore the generalized Kautz graphs can also be tiled using the
methods of the previous section.  Examples are given in the following
section.

\section{Examples}
\label{sec:examples}

In this section we give some explicit examples of tilings constructed
using the discrete differential method of Theorem 3.


Let us first consider tilings of
de Bruijn graphs with  by tiles of size 4.  The de Bruijn graph
 is shown in  Figure~\ref{f:dk22}.  Up to graph
isomorphisms there are  4 kinds of loops in this graph: 

where the second and third
loops differ only in the orientation of the link connecting
 and 10.
All these loops are non-stratifiable.  Thus, there are no stratifiable
tiles given by subgraphs of  with more  edges than the
tile with 3 edges:

This tile therefore gives a scalable tiling which solves Problem 3 for
.  As an example of a tiling using this tile,
consider the tiling of  using the map  described
using discrete differentials (\ref{eq:dd-map}).  This tiling is
depicted in Figure~\ref{f:d23}.  In this figure the numbers in
brackets are the addresses  for each node as in
(\ref{eq:labeling-d}).  The colored/bold links are those realized on
a tile  described by (\ref{eq:tile-22}).
\begin{figure}
\begin{center}
\epsfig{file=d23.eps,width=13cm}
\end{center}
\caption[x]{\footnotesize Optimal scalable
tiling of the de Bruijn graph
 by tiles of size 4}
\label{f:d23}
\end{figure}

Let us now consider simple examples of a Kautz graph tiling with .  The optimal size 4 () tile for  again has no loops
and has only 3 edges.  We can, for example choose the stratified
subgraph of  containing edges , with .  We can use the discrete differential map
(\ref{eq:dd-Kautz}) 
to label the vertices of the Kautz graph  by .  The tiling associated with this map is depicted in
Figure~\ref{f:k23-4}, where nodes are labeled by  (the
nodes are, however, ordered according to the index  from
(\ref{eq:gk})).  In this figure, again colored/bold links are realized
on a tile .  The direction of each link is indicated by using
dotted lines on the outgoing part of the link and solid lines on the
incoming part of the link.
\begin{figure}
\begin{center}
\epsfig{file=k23-4.eps,width=12cm}
\end{center}
\caption[x]{\footnotesize Optimal scalable tiling of the Kautz graph
   by 3 tiles of size 4.  Direction of link goes from
  dotted end to solid end.}
\label{f:k23-4}
\end{figure}

For , up to relabeling there is only one function of the form
(\ref{eq:f}) with the desired properties, which is the one used in the
definition (\ref{eq:dcc}) of discrete differentials.
For ,   it is  easy to check that up to relabeling
of the integers, which is a symmetry of both the de Bruijn and Kautz
graphs, there are precisely two distinct functions with the desired properties:

and

Either of these functions can be used to construct tilings with .

Now, for , consider the tile with .  The de Bruijn graph
here is just the 3 vertices  with edges going from each
vertex to each other vertex.  The largest subgraph without directed
loops has two edges, such as .  This
defines the best tile for a scalable tiling.  Now consider .  A
systematic analysis of possible stratifiable subgraphs of  shows that the maximum number of edges compatible with
stratifiability is 11, which can be realized for example by the tile
depicted in Figure~\ref{f:tile-32}.  This tile has 4 loops of length 4
(of which 3 are homotopically independent), all of which are of height
1.
\begin{figure}
\begin{center}
\epsfig{file=tile-32a.eps,width=12cm}
\end{center}
\caption[x]{\footnotesize Optimal tile for , .  Tile has
11 internal edges  (shown) and 7 external edges (not shown)}
\label{f:tile-32}
\end{figure}
Tilings of the generalized de Bruijn and Kautz graphs with 18 nodes by
two copies of this tile are shown in Figures~\ref{f:d9-18}
and~\ref{f:k9-18}.  In  these figures the node numbers are the
numbers  used in the description of generalized de Bruijn and Kautz
graphs in Section \ref{sec:generalized}.
\begin{figure}
\begin{center}
\epsfig{file=d9-18.eps,width=12cm}
\end{center}
\caption[x]{\footnotesize Optimal scalable tiling of the generalized
  de Bruijn graph with  by tiles of size 9}
\label{f:d9-18}
\end{figure}
\begin{figure}
\begin{center}
\epsfig{file=k9-18.eps,width=12cm}
\end{center}
\caption[x]{\footnotesize Optimal scalable tiling of the generalized
Kautz graph with  by tiles of size 9}
\label{f:k9-18}
\end{figure}


We have carried out a systematic analysis of all tiles for small
values of .  We have computed the optimal tiles by performing a
complete search over all possibilities in all cases where .
Our results are
tabulated here.

\begin{center}
\begin{tabular}{| c | c | c | | c |}
\hline
  &  &   & max \\
\hline
2 & 2 & 4 & 3\\
2 & 3 & 8 & 8\\
3 & 2 & 9 & 11\\
2 & 4 & 16 & 19\\
4 & 2 & 16 & 27\\
\hline
\end{tabular}
\end{center}

Using a simple greedy-first algorithm we have found some good, but not
necessarily optimal tiles for larger values of ; as examples, for
 there is a tile with 44 internal edges, and with  there is a tile with  150 internal edges.  In the next section we
discuss tilings for arbitrarily large values of 


\section{Asymptotically efficient tilings}
\label{sec:efficient}

In this section we provide an explicit construction of an efficient
class of tilings.  This class of tilings realizes the asymptotic bound
(\ref{eq:lower-bound}) on edges leaving the tile to within a factor of
2.  As we discuss below, we believe that this class of tilings may in
fact be asymptotically optimal, and that the theoretical bound may be
improvable by a factor of 2.  We cannot, however, demonstrate this
conclusively at this time.  Note that for the graph bisection problem
there is a similar gap (in that case by a factor of 4) between upper
and lower bounds on edge bisection bandwidth
(\ref{eq:bisection-bounds}) for general  \cite{rttv}.


\subsection{Local score-based tiles}
\label{sec:score-tiles}

We will now define some tiles explicitly by choosing a stratification
of the nodes in a de Bruijn graph and applying Theorem 3.
The basic idea we will use to construct efficient tilings is the
identification of local patterns in the  (discrete differential)
address space to choose the levels in the stratification of the nodes
of the tile of size .

For each node of the de Bruijn graph
 labeled by , we
assign to each position in the node address a score between 0 and 1
according to how well the node address in the vicinity of 
matches some desired pattern.  We will use the convention that a
smaller score indicates a closer match to the desired pattern.  The
idea is that if this score is very low at some particular position
 for a node , then the score will generally be
low for position  in the node addresses of all the children of
.  Thus, if we choose the level of the node  to be 
where  is the position of lowest score, then generically children
 of  will have lowest score at position , and will be
assigned level , so that the edge  can
be included in the tile.

There are many ways in which we can define such a local pattern-based
score.  To give a concrete example we define such a score in a way
which leads to efficient tilings.

\vspace*{0.1in}

\noindent
{\bf Definition:} We define a {\it -ary expansion
score} by interpreting the sequence
of digits beginning with  as a base  expansion of a  real number
between 0 and 1 in the following way

where we define   and  for
.  Thus for example, the node  for 
would have .  The idea
is that the position in the node address describing the smallest
real value will determine the level of the node.
The leading digit is complemented base  so that the starting point of this
lowest value is clearly marked; without this complementation the
resulting tile would have multiple broken edges whenever  a very small
number (long sequence of 0's) initiates the node address sequence .  The boundary conditions are similarly chosen to minimize
broken edges.

With this score function, we then define a tile by defining a
stratification of nodes through

There are (limited)
circumstances in which two positions 
can have the same value of  for a fixed , leading to
``ties'' where multiple values of  satisfy (\ref{eq:score-level}).
Such ties only occur for nodes in which the last  digits are all
equal to  and all previous digits are .  In this case we define

With this assignment of levels, each of these nodes has 
children at level  (those with ) and one child at
level  
giving a broken edge.

\vspace*{0.1in}

\noindent
{\bf Example}: In the tile with , consider the node .  The smallest score is found at the position
where the smallest number appears base 2, with the first bit in the
sequence beginning at that point complemented.  This is position , with .  So
this node is placed at level .  Similarly, each
of the child nodes 100100 and 100101 have the smallest  at  so , so edges (010010,
100100) and (010010, 100101) are both included in the tile.  Note that
for the node 100100, , while .

We can use this method based on the -ary expansion
score to construct the full tile for any .   For each node in
the tile the level of the stratification is defined through
(\ref{eq:score-level}), with (\ref{eq:ties}) used to break ties.
\vspace*{0.1in}

\noindent
{\bf Example}: For the case of  the -ary expansion
score stratification gives the tile shown in Figure~\ref{f:tile-24}.
\begin{figure}
\begin{center}
\epsfig{file=tile-24.eps,width=12cm}
\end{center}
\caption[x]{\footnotesize A tile with  based on the
2-ary expansion score stratification.  Nodes with the smallest string
beginning at position  (preceded by a 1) are placed on level ,
with boundary conditions and tie-break conditions as discussed in
text.
Tile is optimal, having 19 internal edges (shown) and 13 external edges (not shown).}
\label{f:tile-24}
\end{figure}
This tile has 19 internal edges and 13 broken edges.  Thus, this
algorithm for tile construction yields an optimal tile in this case.
\vspace*{0.1in}

It is straightforward to automate the construction of tiles according
to this system.  The numbers of internal (broken) edges for tiles with
small values of  are listed in the following table.

\begin{center}
\begin{tabular}{|| c  || c | c | c | c |c  ||}
\hline
\hline
 & 2 & 3 & 4 & 5 & 6\\
\hline
\hline
2 &  (5) &  (8) &  (13) & 42 (22) &  (38)\\
3 &  (17) &  (40) &  (97) & 485 (244) & 1559 (628)\\
4 & 23 (41) & 129 (127) & 615 (409) & 2729 (1367) & 11697 (4687)\\
5 & 44 (81) & 314 (311) & 1876 (1249) & 10414 (5211) & 55794 (22331)\\
\hline
\hline
\end{tabular}
\end{center}

In this table we denote with an asterisk (*) cases where the tile is
known to be optimal.  The  tile with 42 internal edges may also
be optimal; we have not found a better tile using heuristic search
methods.  For most of the other small tiles, we have found slightly
better tiles than those constructed using this method with brute force
or greedy algorithm searches, or by slight modifications of the
score-based stratification algorithm described above; for example as
mentioned above for the  tile we have found a tile with 150
internal edges.  For larger tiles, we do not know of any general
method for generating tiles with significantly more internal edges.  As we now
discuss, we expect that the asymptotic behavior of these tiles may be
the optimal achievable.  Note, however, that for larger values of 
the approach to the asymptotic form is slower and the tiles produced
by this method are somewhat sub-optimal.  In particular, when  is
large and  is small, many nodes have no digits , which
leads to extra broken edges.  This might be improved by some heuristic
method combining such nodes into sub-tiles which are then attached to
increase total edge numbers.  But, as , for any
fixed  the fraction of such nodes goes to 0 exponentially as , so this issue does not affect the asymptotics.


Note that in \cite{Viterbi} a somewhat similar approach was taken to
tiling degree 2 de Bruijn graphs .  In the language of
score-based tiles used here, the tilings described in that paper can
be defined by assigning a score of  when  and  otherwise, with ties broken by choosing  to be
the largest  with .  As described in \cite{Viterbi},
this gives a tile for  with  internal edges (compared
to 42 from the table above for the system defined here).  The tiles
described in \cite{Viterbi} also do not have good asymptotic behavior,
although as mentioned in that paper a generalization of their approach
to match longer sequences can improve behavior in a fixed range of .

\subsection{Asymptotics of  tiles for local pattern-based scores}

We now analyze the asymptotic behavior of the number of broken edges
in the -ary expansion score-based tiles for fixed  as .  We find that these tiles have the same
asymptotic form for the number of broken edges as the lower bound
(\ref{eq:lower-bound}), multiplied by an overall factor of .
This asymptotic behavior can be understood from a simple idealized
model for the score-based tiles.
We give a simple proof of the asymptotics of the number of broken
edges in this idealized model, and then describe how the tiles just
defined deviate from the idealized model and the consequences of
these deviations for the asymptotic form of the number of broken
edges.  In Section \ref{sec:numerical-asymptotics} we numerically
analyze the score-based tiles and compare to the theoretical asymptotics.

\subsubsection{Idealized model}

Consider an infinite random walk on the de Bruijn graph 
underlying a given tile.  Such a random walk will on average traverse
each edge an equal number of times, so the fraction of edges traversed
which are broken will be equal to the fraction of broken edges on the
tile.  In an idealized model of the score-based tile system described
in the previous subsection, we assume that each node of size  has
scores  which are uniformly and independently distributed
random numbers between 0 and 1.  We assume that the scores are
completely local, so that a node with the score sequence  has children with score sequences  for various values of .  In the idealized
model we can associate the random walk on the de Bruijn graph
underlying the tile with an infinite sequence of numbers , each chosen from a uniform random distribution on the set
.  The values of  correspond to the sequence of scores
 on the nodes encountered in the random walk (while for
   we used the convention that the smallest  determined
  the level of a node, we reverse this convention for the 's to
  simplify the computations below).  In this
idealization we ignore correlations between the values of  at
different points , so each  is chosen independently from the
uniform distribution.  We associate each ``node'' in the random walk
with a subsequence of  numbers  .  In this model, there are no ties.  An edge (, ) is broken only if either  or
 is greater than all of .  Heuristically,
the chances of either occurrence are , giving a probability of
broken edges of .

To make this computation more precise, for each consecutive
subsequence of  numbers   representing a node we assign a level  where  is the
largest value in the subsequence.  We thus have a sequence of values
 defining the positions of the largest 's in each
subsequence .  If a value  is the largest  in
subsequence , and also is the largest  in , then .  We say that the sequence has a ``broken edge''
at position  when .  We can now show
\vspace*{0.1in}

\noindent
{\bf Theorem 4} In this idealized model, as  the
probability of a broken edge asymptotically approaches .

\vspace*{0.1in}

\noindent
{\bf Proof}

Consider the set of ``local maximum'' 's which are maximum values
in some subsequence  in which they are contained.  There is a
broken edge precisely when the maximum  in one subsequence  is
replaced by a new maximum  in the subsequence .  Thus,
the frequency of broken edges is the same as the frequency of local
maxima.  For a given number  we can determine the probability that
it is the largest  in a window of size precisely .  For
example,  is the largest in a window of size precisely 1
when .  This occurs with probability
.   is the largest in a window of size precisely 2 when either
 or the symmetric condition
with  but .  The probability
that  is largest in a window of size 2 is therefore .
A similar set of conditions give  independent cases with
probability  where  is the maximum in a window of
size .  Integrating over all possible , the probability that a
given  will be largest in a window of precisely size  is

the frequency of local maxima is then 

This proves that in the idealized model the frequency of broken edges
is .

\vspace*{0.1in}

This result in the idealized model suggests that the smallest fraction
of broken edges we can achieve in a tile based on local patterns in
the node address is asymptotically
.  
Now let us consider the relevant differences between the idealized
model and the -ary score-based tiles defined in the previous
subsection.

\subsubsection{Deviations from idealized model}

There are  three primary ways in which the score-based tiles we have
defined deviate from the idealized model.  First, there are some
situations in which different positions have the same score  for a fixed node.  Second, the scores are not completely
independent.  Third, the scores  depend on multiple local
symbols , and therefore in particular the score associated with a
point  in a node does not necessarily stay invariant in the
children of that node--- we  have  along an edge  whenever the final symbol in 
is not .

Let us treat these deviations from the idealized model in order.
First, consider the situation of ties.  Generally, ties will increase
the frequency of broken edges.  For example, consider the variation on
the above model where  is chosen randomly from a uniform
distribution on the discrete set , and ties
are broken by choosing  to be the smallest  with the maximum
value of  in the set .  Then when , the
frequency of broken edges will be .  (The approach taken
in \cite{Viterbi} is an extreme example of this; their rule for
constructing tiles can be formulated in the language of scores as
described above by taking a score which is either  if  or  otherwise.  This gives a frequency of
broken edges of 1/4 for large .)

As noted above,  for the -ary score based tiles defined in 
\ref{sec:score-tiles},
ties
only occur for nodes with .  The total number of nodes where such ties occur is

The fraction of nodes where ties happen therefore decreases
exponentially in  as  when .
The consequences of ties are therefore negligible in the asymptotic
behavior of the number of broken edges.

Next, let us consider the other two issues: the lack of independence
between nearby scores and the non-invariance of scores when edges are
followed.  Each of these effects arises from dependence of the score
 on all the symbols  for any .  From the
form of (\ref{eq:phi-score}) we see that the effect of  on
the score  is suppressed exponentially as .  For a node
in a tile of size , the average spacing between the scores of the
 different positions  is .  Thus, the impact of 
becomes negligible compared to this spacing when .
Another way to see this is to note that the smallest  will arise
from the longest sequence of 0's in the node which is preceded by the
symbol .  As , the average number of
sequences of  zeros in a random node address goes as .
Thus, we expect  sequences of  0's but 
 sequences of  times this many 0's.  So again, the
number of relevant digits in determining the minimum  is of order
.  The relevant correlation distance between 's is
thus also .  Similarly, when a new digit sequence giving the
lowest  is being shifted in from the right, asymptotically of
order  digits must be shifted in to realize the smaller
value of .

The upshot of this analysis is that there are corrections to the
asymptotic form of the idealized model of order  when we
consider the score-based tiles described above.  The most important of
these effects is the delay by  digits in shifting in a new
score .  When ,  will be higher than
the appropriately shifted value in any of the descendants of a given
node due to the boundary condition  for  described
below (\ref{eq:phi-score}).  This means that the effective size of the
window associated with the tile is really ,
which contributes a correction term of order  to the
asymptotic form of the fraction of broken edges  for the
idealized model demonstrated in Theorem 4.

Because all of these effects are either exponentially suppressed in
 or suppressed by a factor of  relative to the
leading , none of these effects will modify the asymptotic form
of the number of broken edges for the score-based tiles described
above.  We now verify this analysis with numerical computation.

\subsubsection{Numerical verification of asymptotics}
\label{sec:numerical-asymptotics}

To verify the asymptotic analysis performed in the previous
subsection, and the validity of the approximations made in going from
the score-based tiling system to the idealized model, we have done a
numerical analysis of the number of broken edges for tiles with small
 and reasonably large .  This data corresponds closely with the
theoretical analysis just performed.

\begin{figure}
\begin{center}
\epsfig{file=numbers-2.eps,width=12cm}
\end{center}
\caption[x]{\footnotesize (Logarithm of) number of broken edges for
  2-ary expansion score based tile, compared to theoretical asymptotic
  fraction  and asymptotic lower bound .}
\label{f:numbers-2}
\end{figure}

In Figure~\ref{f:numbers-2} we have graphed the logarithm of the
number of broken edges in the 2-ary expansion score-based tiles for .  The number of broken edges in these tiles matches
very closely to the theoretically estimated fraction of , and
lies above the asymptotic form of the minimum fraction .
Note that for , the number of broken edges exceeds the
fraction  of the total number of edges.  This is compatible, however, with the expected 
 
form of the corrections to the asymptotic form computed above.  
For  the computed number of broken edges lies between the
asymptotic form  and the log-corrected asymptotic form
.

For tiles with  we have a similar close match to the asymptotic
fraction  of broken edges, as depicted in Figure~\ref{f:numbers-3}.
\begin{figure}
\begin{center}
\epsfig{file=numbers-3.eps,width=12cm}
\end{center}
\caption[x]{\footnotesize (Logarithm of) number of broken edges for
  3-ary expansion score based tile, compared to theoretical asymptotic
  fraction  and asymptotic lower bound .}
\label{f:numbers-3}
\end{figure}



To summarize, from our analysis of an idealized model, we expect that
a fraction of  broken edges is asymptotically the optimum which
can be achieved based on a local pattern for level determination.  We
have constructed a family of tiles which realize this asymptotic form,
and are thus asymptotically optimal for local pattern-based tile
structures.  This asymptotic result differs from the lower bound
computed in Section
\ref{sec:asymptotic} by a factor of 2.  We suspect that it is not possible to
find tilings with an asymptotic fraction of less than  broken
edges, based on our asymptotic analysis and brute force and greedy
algorithm analysis of small tiles, but we have not proven this
conclusively.  Any mechanism for constructing tiles with an asymptotic
fraction of  broken edges with  would require a means of
determining the level of each node based on global properties of the
 node address, like some kind of global hashing function, rather
than a determination of level based on local patterns such as we have
considered here.

\section{Application to supercomputers}

The results of this paper give not only a theoretical understanding of
how de Bruijn and Kautz graphs can be decomposed into isomorphic
subgraph tiles, but also a concrete approach to constructing such
tilings.  Computer systems containing thousands of individual
processing elements need efficient communication networks to minimize
overhead in passing data between the processors.  Because of their
high degree of connectivity, de Bruijn and Kautz graphs are very well
suited to such large scale processing networks.  The practical problem
of wiring together many processors in such a network is substantially
simplified by the approach of combining multiple processing units into
tiles with isomorphic wiring, and then combining the tiles as we have
described in this paper.

In principle, the methods described in this paper can be used to
design wiring systems for computing systems at a range of scales.  For
any given size of system, the tradeoff between communication and
processing power will affect the choice of degree , and practical
design considerations will affect the choice of tile size .  The
results described here should be useful in determining the complexity
of wiring needed for any such system design.  In particular, the lower
bound on edges connecting tiles derived in Section
\ref{sec:asymptotic} gives an absolute lower limit on the complexity
of wiring necessary for such a system.  The explicit tile
constructions given here give concrete examples of wiring patterns
which can be used for such systems.

The feasibility of a practical implementation of the tiling methods
developed in this paper is demonstrated by the Sicortex, Inc.\ family
of cluster computer systems \cite{sc-reference}.  The internal
communication network in these systems is based on a degree 
Kautz digraph.  Circuit boards contain processor notes connected in a
stratified subgraph of a degree 3, diameter 3 de Bruijn digraph, so
that a full Kautz graph of any desired size can be wired by connecting
identical boards as described in Theorem 3, taking  
advantage of the parallel routing property.

\section{Conclusions}

In this paper we have developed a systematic approach to partitioning
de Bruijn and Kautz graphs into isomorphic subgraph  ``tiles''
connected by a minimal number of additional edges.  These results
utilize the common mathematical structure underlying de Bruijn and
Kautz graphs, and shed light on the structure of these graphs and
their generalizations.  The tilings we have constructed here have
practical application to the construction of massively parallel
computer systems.

We have given necessary and sufficient conditions for
constructing efficient tilings, we have characterized optimal
tiles in terms of de Bruijn graphs of the size of the desired tile,
and we have constructed an asymptotically optimal class of tilings.
We
have not, however, found a general method for explicitly constructing provably
optimal tiles of arbitrary size, nor have we found a general formula
for the number of internal edges achievable by an optimal tile of
arbitrary size.  We leave these open problems for future work.

\newpage


\bibliographystyle{plain}

\begin{thebibliography}{10}


\bibitem{db}
N.\  G.\ de Bruijn, A combinatorial problem,
{\it Nederl.\ Akad.\ Wetensh.\ Proc.\ Ser. } {\bf A}
49 (1946) 758-764.

\bibitem{Kautz}
W.\ H.\ Kautz, Bounds on directed  graphs,
{\it Theory of cellular logic networks and machines, AFCRL-68-0668
Final report}, (1968) 20-28.

\bibitem{ii1}
M.\ Imase and M.\ Itoh, Design to minimize a diameter on building
block network, {\it IEEE Trans.\ on Computers}, C-30 (1981) 439-443.

\bibitem{ii2}
M.\ Imase and M.\ Itoh, A design for directed graph with minimum
diameter, {\it IEEE Trans.\ on Computers}, C-32 (1983) 782-784.

\bibitem{rpk}
S.\ M.\ Reddy, D.\ K.\ Pradhan and J.\ G.\ Kuhl, Directed graphs with
minimal diameter and maximal connectivity,
{\it School of Engineering Oakland Univ.\ Tech.\ Rep.,} 1980
 
\bibitem{dch} 
D.-Z.\ Du, F.\ Cao and D.\ F.\ Hsu, de Bruijn digraphs, Kautz
digraphs, and their generalizations,
{\it in} D.-Z.\ Du and D.\ F.\ Hsu, eds., ``Combinatorial Network
Theory'', Kluwer Academic (1996) pp.\ 65-105

\bibitem{bp} 
J.-C.\ Bermond \& C.\  Peyrat, De Bruijn and Kautz networks: A
competitor for the hypercube?, {\it in} F.\ Andr\'e, J.\ P.\ Verjus,
eds., ``Hypercube and Distributed Computers'', North Holland,  (1989) 279-293.

\bibitem{pz}
J.\ Plesnik and S.\ Znam, Strongly geodetic directed graphs, {\it in}
Recent Advances in Graph Theory, Proc.\ Symp., Prague, Academia
  Prague (1975) 455-465.

\bibitem{Viterbi} O.\ Collins, F.\ Pollara, S.\ Dolinar and J.\
Statman, Wiring Viterbi Decoders (Splitting de Bruijn Graphs), 
{\it TDA
Progress Report 42-95}, Jet. Propulsion Laboratory, Pasadena,
California, 1988.

\bibitem{rttv}
J.\ Rolim, P.\ Tvrd\'ik, J.\ Trdlicka and I.\ Vrt'o, bisecting de
Bruijn and Kautz graphs, 
{\it Discrete Applied Mathematics}, 87-97, 1998.

\bibitem{Leighton}
F.\ T.\ Leighton, {\it Introduction to Parallel Algorithms and
Architectures: Arrays, Trees, Hypercubes}, Morgan Kaufmann, San Mateo
CA (1992).

\bibitem{sc-reference}
N.\  Godiwala, J.\  Leonard, M.\  Reilly, A Network Fabric for  
Scalable Multiprocessor Systems, {\it in} Proceedings of
16th IEEE Symposium on High  
Performance Interconnects, (2008) 137-144.



\end{thebibliography}


\end{document}
