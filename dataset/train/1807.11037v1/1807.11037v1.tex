\section{Experiment}

\Hsu{In this section, we describe the dataset that used for the video segmentation in Sec.~\ref{sec.dataset} and our implementation details in Sec.~\ref{sec.impl}. Next, we compare our TA-MC dropout and RTA-MC dropout with MC dropout in several different aspects. First, in Sec.~\ref{sec.acc}, we compare the performance of video segmentation and the inference time. Second, in Sec.~\ref{sec.unct},  we show that our methods can obtain comparable uncertainty estimation.}

\subsection{Dataset}
\label{sec.dataset}
\Hsu{CamVid \cite{brostow2009semantic} is a road scene segmentation dataset which contains four 30Hz videos. The frames are labeled every 1 second, and each pixel is labeled into 11 classes such as sky, building, road, car, etc. There are total 701 labeled frames split into 367 training frames, 101 validation frames and 233 test frames.  All frames are resized to 360x480 pixels in our experiments.}


\subsection{Implementation Details}
\label{sec.impl}

\New{We apply the Bayesian SegNet model \cite{kendall2015bayesian} and Tiramisu model \cite{jegou2017one} to demonstrate our TA and RTA method. We train both models by the same setting in the original paper.} We set $\alpha$ in Eq.~\ref{eq.mean} for TA-MC dropout to 0.2. For RTA-MC dropout, we use threshold $\lambda$ as 10 to determine whether the flow estimation is wrong. Note that the reconstruction error of flow is in the range of $[0,255]$ and the average error is about 2. Hence, we choose the threshold $\lambda$ slightly higher than the average error. $\alpha_{acc}$ and $\alpha_{err}$ in Eq.~\ref{eq.spatial_alpha} are set to 0.2 and 0.7, respectively. We use FlowNet 2.0 \cite{ilg2017flownet} for our optical flow estimation. We implement all methods in Pytorch \cite{paszke2017automatic} framework and experiment on GTX 1080 for time measurement.


\begin{table}[t]
\centering
\setlength{\extrarowheight}{0.05mm}
\addtolength{\tabcolsep}{0.5mm}
\fontsize{6pt}{10pt}\selectfont
\begin{tabular}{ccccccccccccccccc}
Method & \rotatebox{90}{Building} & \rotatebox{90}{Tree} & \rotatebox{90}{Sky} & \rotatebox{90}{Car}    & \rotatebox{90}{Sign-Symbol} & \rotatebox{90}{Road}  & \rotatebox{90}{Pedestrian} & \rotatebox{90}{Fence} & \rotatebox{90}{Column-Pole} & \rotatebox{90}{Side-walk} & \rotatebox{90}{Bicyclist} & \rotatebox{90}{Class avg.} & \rotatebox{90}{Global avg.} & \rotatebox{90}{Mean IoU} & \rotatebox{90}{Inference Time (s)}& \rotatebox{90}{Speed Up Ratio}\\ 
\hline\hline


\multicolumn{1}{c|}{\begin{tabular}[c]{@{}c@{}}SegNet \\ MC (N=50) \end{tabular}} & \multicolumn{1}{c|}{88.9}   & \multicolumn{1}{c|}{88.7}     & \multicolumn{1}{c|}{95.0}     & \multicolumn{1}{c|}{89.0}   & \multicolumn{1}{c|}{49.4}     & \multicolumn{1}{c|}{95.8}     & \multicolumn{1}{c|}{73.5}     & \multicolumn{1}{c|}{51.4}     & \multicolumn{1}{c|}{43.3}     & \multicolumn{1}{c|}{93.2}     & \multicolumn{1}{c|}{57.0}     &     75.0       &    90.6         &    \multicolumn{1}{c|}{63.7} &2.00&1      \\ \hline
\multicolumn{1}{c|}{\begin{tabular}[c]{@{}c@{}}SegNet \\ TA-MC\end{tabular}}      & \multicolumn{1}{c|}{88.6}   & \multicolumn{1}{c|}{89.7}     & \multicolumn{1}{c|}{94.5}     & \multicolumn{1}{c|}{88.2}   & \multicolumn{1}{c|}{48.1}     & \multicolumn{1}{c|}{95.3}     & \multicolumn{1}{c|}{70.7}     & \multicolumn{1}{c|}{44.7}     & \multicolumn{1}{c|}{36.4}     & \multicolumn{1}{c|}{94.1}     & \multicolumn{1}{c|}{52.7}     &        73.0    &    90.3         &     \multicolumn{1}{c|}{62.1}  &0.18&10.97 \\ \hline
\multicolumn{1}{c|}{\begin{tabular}[c]{@{}c@{}}SegNet \\ RTA-MC\end{tabular}}     & \multicolumn{1}{c|}{88.4}   & \multicolumn{1}{c|}{89.3}     & \multicolumn{1}{c|}{94.9}     & \multicolumn{1}{c|}{88.9}   & \multicolumn{1}{c|}{48.7}     & \multicolumn{1}{c|}{95.4}     & \multicolumn{1}{c|}{73.0}     & \multicolumn{1}{c|}{45.6}     & \multicolumn{1}{c|}{41.4}     & \multicolumn{1}{c|}{94.0}     & \multicolumn{1}{c|}{51.6}     &     73.7       &     90.4        &    \multicolumn{1}{c|}{62.5}    &0.18&10.97   \\ \hline\hline
\multicolumn{1}{c|}{\begin{tabular}[c]{@{}c@{}}Tiramisu \\ MC (N=50) \end{tabular}} & \multicolumn{1}{c|}{89.7}   & \multicolumn{1}{c|}{87.2}     & \multicolumn{1}{c|}{95.6}     & \multicolumn{1}{c|}{84.9}   & \multicolumn{1}{c|}{58.4}     & \multicolumn{1}{c|}{95.1}     & \multicolumn{1}{c|}{82.5}     & \multicolumn{1}{c|}{54.1}     & \multicolumn{1}{c|}{49.6}     & \multicolumn{1}{c|}{84.6}     & \multicolumn{1}{c|}{52.3}     &     75.8       &    89.8         &    \multicolumn{1}{c|}{64.0}    &11.72&1  \\ \hline
\multicolumn{1}{c|}{\begin{tabular}[c]{@{}c@{}}Tiramisu \\ MC (N=5) \end{tabular}} & \multicolumn{1}{c|}{88.7}   & \multicolumn{1}{c|}{86.6}     & \multicolumn{1}{c|}{95.4}     & \multicolumn{1}{c|}{83.7}   & \multicolumn{1}{c|}{58.4}     & \multicolumn{1}{c|}{94.6}     & \multicolumn{1}{c|}{80.6}     & \multicolumn{1}{c|}{52.0}     & \multicolumn{1}{c|}{49.2}     & \multicolumn{1}{c|}{84.0}     & \multicolumn{1}{c|}{55.0}     &     75.3       &    89.2         &    \multicolumn{1}{c|}{62.4}     &1.17&10.00 \\ \hline
\multicolumn{1}{c|}{\begin{tabular}[c]{@{}c@{}}Tiramisu \\ TA-MC\end{tabular}}      & \multicolumn{1}{c|}{90.3}   & \multicolumn{1}{c|}{87.4}     & \multicolumn{1}{c|}{94.8}     & \multicolumn{1}{c|}{84.2}   & \multicolumn{1}{c|}{55.8}     & \multicolumn{1}{c|}{94.5}     & \multicolumn{1}{c|}{79.2}     & \multicolumn{1}{c|}{51.3}     & \multicolumn{1}{c|}{40.6}     & \multicolumn{1}{c|}{85.6}     & \multicolumn{1}{c|}{46.7}     &        73.8    &    89.6         &     \multicolumn{1}{c|}{62.3} &0.37&31.58  \\ \hline
\multicolumn{1}{c|}{\begin{tabular}[c]{@{}c@{}}Tiramisu \\ RTA-MC\end{tabular}}     & \multicolumn{1}{c|}{90.1}   & \multicolumn{1}{c|}{87.1}     & \multicolumn{1}{c|}{94.9}     & \multicolumn{1}{c|}{84.1}   & \multicolumn{1}{c|}{56.7}     & \multicolumn{1}{c|}{94.7}     & \multicolumn{1}{c|}{79.2}     & \multicolumn{1}{c|}{48.1}     & \multicolumn{1}{c|}{42.2}     & \multicolumn{1}{c|}{85.4}     & \multicolumn{1}{c|}{49.8}     &     73.9       &     89.5        &    \multicolumn{1}{c|}{62.4}    &0.37&31.58   \\ \hline
\end{tabular}
\small \caption{\small \New{Performance test on CamVid dataset. Upper three rows are comparisons of SegNet backbone; Lower four rows are comparisons of Tiramisu backbone. Both comparisons show that our methods can speed up more than 10x with only 1-2 percentage drop. For fairly comparison, we reduce the Tiramisu MC sample time to N=5 to get the same accuracy as our methods. In this situation, our methods are still 10x faster.}}
\label{table.acc}
\end{table}


\subsection{Results of Video Segmentation}
\label{sec.acc}
\New{We compare MC dropout with our TA-MC dropout and RTA-MC dropout on CamVid dataset by two models.} We first show the performance of the video segmentation on several different metrics: (1) pixel-wise classification accuracy on every class, (2) class average accuracy (class avg.), (3) overall pixel-wise classification accuracy (global avg.), (4) mean intersection over union (mean IoU) and (5) inference time. The results are shown in Table~\ref{table.acc}. For SegNet, Our TA-MC dropout can reach comparable accuracy, and the RTA-MC dropout further improves the performance with only 1.2\% drop on mean IoU metric. \New{For Tiramisu, our methods also can reach comparable accuracy.}


\New{In the case of comparable accuracy, our method can further speed up the inference time. Since our methods only need to forward one time for each frame; while MC dropout needs to sample $N$ times. For SegNet, we can obtain almost 11 times speed up. Note that our TA-MC dropout and RTA-MC dropout can perform the same speed as the only difference between them is the multiplying factor ($\alpha$ in Eq.~\ref{eq.mean}) which doesn't affect the speed. The inference time of the RTA-MC dropout mainly contains the inference time of the Bayesian SegNet model and the FlowNet 2.0 model which are 0.04 seconds and 0.13 seconds, respectively. FlowNet 2.0 model takes 70\% of the whole inference time. If we use the bigger segmentation model, we can get a better improvement in the speed. Therefore, we use Tiramisu model which is the state-of-the-art model in CamVid but 6x slower than SegNet to show better speed up ratio. For Tiramisu, Our method can achieve 31x faster than MC dropout sample 50 times. To fairly compare inference time, we reduce the MC dropout sample time to 5 times. The accuracy becomes the same as our methods. In this case, our methods are still 10x faster than MC dropout. This table shows that in the same accuracy level, our methods can speed up inference time 10x.}







\begin{figure}[t]
\begin{center}   
\includegraphics[width=0.9\linewidth]{figures/PR_Curve_explain.pdf}
\small \caption{\small \New{Explanation of the Pixel-level metric Precision-Recall curve. First, calculate the uncertainty map of all test data. Then rank all pixels by uncertainty value. The horExplanizontal axis is the percentage of recall pixel which means we keep how many percentages of most certain pixels to calculate precision. The vertical axis is the mIoU.}}
\label{fig.pr_explan}
\end{center}
\end{figure}











\subsection{Results of Uncertainty Estimation}



We evaluate the uncertainty estimation in pixel-level and frame-level metrics.\\
\noindent\textbf{Pixel-level Metric.}  The pixel-level evaluation is inspired by Precision-Recall Curve metric in \cite{kendall2017uncertainties}. This curve shows the accuracy of the remained pixels as removing pixels with uncertainty larger than different percentile thresholds. Detail explanation is in Fig.~\ref{fig.pr_explan}. A reliable uncertainty estimation should let the PR-Curve monotonically decrease. We compare MC dropout and our TA and RTA method with different uncertainty function in Fig.~\ref{fig.pr_curve}. Left four figures are the results of SegNet backbone. Right four figures are the results of Tiramisu backbone. All results show that as the recall percentage drop from 1 to 0.5, the mean IoU of all methods monotonically increase which means the uncertain pixels is correlated to misclassified pixels. Although MC dropout has the highest accuracy almost at all percentage, our TA-based methods are still comparable to MC dropout. TA-MC and RTA-MC have similar results in PR-Curve, but at the frame-level metric, RTA-MC will outperform TA-MC.\\






\begin{figure}[t]
\centering
\setlength\belowcaptionskip{-20pt}
\begin{minipage}{.5\textwidth}
  \centering
  \includegraphics[width=\linewidth]{figures/pixel_pr_curves.pdf}
\end{minipage}\begin{minipage}{.5\textwidth}
  \centering
  \includegraphics[width=\linewidth]{figures/tiramisu_PR_curve.pdf}
\end{minipage}
\small \caption{\small \New{Pixel-level precision-recall curves. Left four figures are the results of SegNet backbone. Right four figures are the results of Tiramisu backbone. We use mean IoU as the precision metric. We show the comparison of MC dropout, TA-MC dropout and RTA-MC dropout on four different uncertainty estimation methods. Our methods achieve comparable results especially when using \textit{Entropy} and \textit{Variation Ratio} as the uncertainty estimation functions.}}
\label{fig.pr_curve}
\end{figure}






\noindent\textbf{Frame-level Metric.} Pixel-level metric is good to evaluate the uncertainty estimation. However, pixel-wise uncertainty estimation is hard to leverage in real applications. For example, active learning system wants to find which frame is valuable to be labeled rather than decides which pixel should be labeled. Here, we propose frame-level uncertainty metrics to show that our uncertainty estimation can work well and faster in real applications. The procedure is shown in Fig.~\ref{fig.frame_level_explain}. First, frames are ranked by the error of prediction as the ground truth ranking sequence. Then, we rank frames by the uncertainty estimation and evaluate the uncertainty ranking sequence by two metrics: \textbf{Kendall tau} \cite{kendall1938new} and \textbf{Ranking IoU}. Kendall tau is a well-known ranking metric measures how the ranking sequence is similar to the ground truth sequence. The value is bounded in 1 (fully identical sequence) and -1 (fully different sequence). Table~\ref{table.frame_rank} shows the comparison of Kendall tau by SegNet and Tiramisu backbone.Though Table~\ref{table.frame_rank} shows that the highest scores appears in \textit{Mean STD} and \textit{BALD}, RTA-MC outperforms MC in \textit{Entropy} and \textit{Variation Ratio}. It also shows that RTA-MC improves frame-level ranking compare to TA-MC. It is attributed to the decision function $A(E)$ which reduces the uncertainty value of pixels with wrong flow estimation which harm the frame-level uncertainty. Kendall tau compares the whole sequence similarity, but real applications pay more attention on higher ranking similarity than whole ranking. Therefore, we define a novel frame-level ranking metric called Ranking IoU. Given a percentage of frame $P_{f}$ to retrieve, we retrieve frames depend on error $G\left(P_{f}\right) = \left \{ g_{1}, g_{2}, ..., g_m, ..., g_{M} \right \}$ and uncertainty $U\left(P_{f}\right) = \left \{ u_{1}, u_{2}, ..., u_m, ..., u_{M} \right \}$. The ranking IoU is:
\begin{align}
\text{Ranking IoU} = \frac{G\left ( P_{f}  \right )\cap U\left ( P_{f}  \right )}{G\left ( P_{f}  \right )\cup  U\left ( P_{f}  \right )}
\end{align} 
Larger Ranking IoU means that those frames we choose are hard to predict, so they are valuable to be labeled. Left of the Table~\ref{table.rankingIOU} shows the ranking IoU performance of SegNet backbone between different methods and uncertainty functions. We show performance in different $P_{f}$. In column $P_{f}=10\%$, TA in \textit{Variation Ratio} has $52.2\%$ which is larger than RTA's $47.8\%$ about $4.4\%$; however, $10\%$ of test data only contains 23 frames so that RTA and TA actually only have 1 frame difference. For $P_{f}=30\%$ which is a practical percentage for real applications, RTA outperforms other methods in all uncertainty functions. The best score of RTA $69.6\%$ is larger than MC dropout's best score $66.7\%$ about $3\%$, which means our uncertainty method can more retrieve $3\%$ of hardest frames. For \textit{Entropy} and \textit{Variation Ratio}, RTA outperforms other methods in almost all percentage. The right of the Table~\ref{table.rankingIOU} shows the Ranking IOU of tiramisu backbone. The results are similar to SegNet backbone that for \textit{Entropy} and \textit{Variation Ratio}, RTA outperforms other methods. Table~\ref{table.frame_rank} and Table~\ref{table.rankingIOU} indicate that RTA can generate high-quality uncertainty. Fig.~\ref{fig.examples} shows the visualization of prediction, uncertainty, and error. It shows that RTA's uncertainty quality is comparable to MC dropout and the large uncertainty pixels are correlated to the misclassified pixels. 
\begin{figure}[t]
\begin{center}   
\includegraphics[width=0.9\linewidth]{figures/frame_level_explain.pdf}
\small \caption{\small \New{Explanation of frame-level metric. First, calculate the error rate of each test data frame by looking at the ground truth to get the error ranking sequence. Second, calculate the frame uncertainty of all test data to get the uncertainty ranking sequence. Then measure the similarity between two sequences by Kendall tau and Ranking IOU.}}

\label{fig.frame_level_explain}
\end{center}
\end{figure}



\begin{table}[t]
\centering
\addtolength{\tabcolsep}{2mm}
\begin{tabular}{lcccc}
\hline
Method & Entropy & Variation Ratio & Mean STD & BALD  \\ 
\hline\hline
SegNet MC   &  0.647  &  0.668  &\textbf{0.677} & \textbf{0.671}\\ 
SegNet TA-MC   &  0.626  &  0.631 & 0.540 & 0.527\\ 
SegNet RTA-MC & \textbf{0.662} & \textbf{0.674} & 0.627 & 0.620\\
\hline\hline
Tiramisu MC   &  0.636  &  0.653  & 0.659 & \textbf{0.647}\\ 
Tiramisu TA-MC   &  0.660 &  0.674 & \textbf{0.663} & 0.626\\ 
Tiramisu RTA-MC & \textbf{0.664} & \textbf{0.678} & 0.635 & 0.612\\
\hline
\end{tabular}
\small \caption{\small \New{Comparison of Kendall tau. For both backbone, the result shows that RTA-MC has the highest value in \textit{Entropy} and \textit{Variation Ratio}. For overall performance, RTA-MC is comparable to MC dropout. }}
\label{table.frame_rank}
\end{table}






\label{sec.unct}























\begin{table}[t]
\centering
\scriptsize
\setlength\belowcaptionskip{-20pt}
\begin{minipage}{.5\textwidth}
\begin{tabular}{ccllll}
\hline
\multirow{2}{*}{Metric}          & \multirow{2}{*}{Method} & \multicolumn{4}{c}{Percentage}                                                                            \\
                                 &                         & \multicolumn{1}{c}{10\%} & \multicolumn{1}{c}{30\%} & \multicolumn{1}{c}{50\%} & \multicolumn{1}{c}{70\%} \\ \hline\hline
\multirow{3}{*}{Entropy}         & MC                      & 47.8                     & 59.4                     & 72.4                     & 85.2                     \\
                                 & TA                      & 47.8                     & 65.2                     & 69.8                     & 85.8                     \\
                                 & RTA                     & 47.8                     & \textbf{66.7}            & \textbf{73.3}            & \textbf{88.9}            \\ \hline
\multirow{3}{*}{Variation Ratio} & MC                      & 47.8                     & 62.3                     & 74.1                     & 85.2                     \\
                                 & TA                      & \textbf{52.2}            & 63.8                     & 70.7                     & 85.2                     \\
                                 & RTA                     & 47.8                     & \textbf{65.2}            & \textbf{76.7}            & \textbf{88.3}            \\ \hline
\multirow{3}{*}{Mean STD}        & MC                      & \textbf{52.2}            & 66.7                     & \textbf{71.6}            & \textbf{86.4}            \\
                                 & TA                      & 43.5                     & 65.2                     & 65.5                     & 74.1                     \\
                                 & RTA                     & 43.5                     & \textbf{69.6}            & 69.8                     & 82.7                     \\ \hline
\multirow{3}{*}{BALD}            & MC                      & \textbf{47.8}            & 66.7                     & \textbf{71.6}            & \textbf{87.7}            \\
                                 & TA                      & 43.5                     & 62.3                     & 64.7                     & 74.1                     \\
                                 & RTA                     & 43.5                     & \textbf{69.6}            & 67.2                     & 82.1                      \\ \hline
\end{tabular}
  
  \centering
\label{table.rankingIOU_segnet}
\end{minipage}\begin{minipage}{.5\textwidth}
 \begin{tabular}{ccllll}
\hline
\multirow{2}{*}{Metric}          & \multirow{2}{*}{Method} & \multicolumn{4}{c}{Percentage}                                                                            \\
                                 &                         & \multicolumn{1}{c}{10\%} & \multicolumn{1}{c}{30\%} & \multicolumn{1}{c}{50\%} & \multicolumn{1}{c}{70\%} \\ \hline\hline
\multirow{3}{*}{Entropy}         & MC                      & 34.8                     & 60.9                     & 70.7                     & \textbf{86.4}                     \\
                                 & TA                      & \textbf{47.8}                     & \textbf{63.8}                     & 71.6                     & 84.6                     \\
                                 & RTA                     & \textbf{47.8}                     & \textbf{63.8}            & \textbf{74.1}            & \textbf{86.4}            \\ \hline
\multirow{3}{*}{Variation Ratio} & MC                      & 34.8                     & 60.9                     & 74.1                     & 86.4                     \\
                                 & TA                      & 47.8            & \textbf{65.2}                     & 72.4                     & 85.8                     \\
                                 & RTA                     & \textbf{52.1}                     & \textbf{65.2}            & \textbf{75.9}            & \textbf{87.7}            \\ \hline
\multirow{3}{*}{Mean STD}        & MC                      & 30.4            & 63.8                     & \textbf{76.7}            & \textbf{86.4}            \\
                                 & TA                      & \textbf{47.8}                     & \textbf{75.4}                     & 74.1                     & 82.0                     \\
                                 & RTA                     & 43.4                     & 68.1            & 73.3                     & 82.1                     \\ \hline
\multirow{3}{*}{BALD}            & MC                      & 30.4            & 62.3                     & 72.4            & \textbf{86.4}            \\
                                 & TA                      & 43.5                     & \textbf{71.0}                     & \textbf{71.6}
                                & 80.2                     \\
                                 & RTA                     & \textbf{47.8}                     & 66.7            & \textbf{71.6}                     & 80.9                      \\ \hline
\end{tabular}
  
  \centering
\label{table.rankingIOU_tiramisu}
\end{minipage}
\small \caption{\small Ranking IoU. Left table is the result of SegNet backbone. Right table is the result of Tiramisu backbone. For retrieving 30\%, 50\% and 70\% of frames RTA-MC have the highest score by using Variation ratio.}
\label{table.rankingIOU}
\end{table}


































