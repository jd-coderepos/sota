\documentclass{article}
\pdfoutput=1
\usepackage{amsmath,amssymb}
\usepackage{graphicx}
\usepackage{color}
\usepackage{mathtools}
\usepackage[table]{xcolor}
\usepackage{graphicx} 
\usepackage{subcaption}

\usepackage{array}
\renewcommand{\baselinestretch}{1}
\renewcommand{\arraystretch}{1.3}
\newcommand\Myperm[2][n]{\prescript{#1\mkern-2.5mu}{}P_{#2}}
\newcommand\Mycomb[2][n]{\prescript{#1\mkern-0.5mu}{}C_{#2}}




\setlength{\topmargin}{-0.1in}
\setlength{\textheight}{8.3in}
\setlength{\oddsidemargin}{0.1 in}
\setlength{\textwidth}{6.2 in}






\newtheorem{fact}{Fact}
\newtheorem{algorithm}{Algorithm}
\newtheorem{theorem}{Theorem}
\newtheorem{lemma}{Lemma}
\newtheorem{corollary}{Corollary}
\newtheorem{property}{Property}
\newtheorem{definition}{Definition}
\newtheorem{proposition}{Proposition}
\newtheorem{remark}{Remark}
\newtheorem{conjecture}{Conjecture}

\newcommand{\notsim}{{\, \not \sim \,}}

\newcommand{\F}{\ensuremath{\mathbb F}}
\newcommand{\Z}{\ensuremath{\mathbb Z}}
\newcommand{\N}{\ensuremath{\mathbb N}}
\newcommand{\Q}{\ensuremath{\mathbb Q}}
\newcommand{\R}{\ensuremath{\mathbb R}}
\newcommand{\C}{\ensuremath{\mathbb C}}

\newcommand{\done}{\hfill  }
\newcommand{\rmap}{\stackrel{\rho}{\leftrightarrow}}
\newcommand{\mys}{\vspace{0.15in}}

\newcommand{\ebu}{{\bf{e}}}
\newcommand{\Abu}{{\bf{A}}}
\newcommand{\Bbu}{{\bf{B}}}

\newcommand{\abu}{{\bf{a}}}
\newcommand{\abuj}{{\bf{a_j}}}
\newcommand{\bbu}{{\bf{b}}}
\newcommand{\vbu}{{\bf{v}}}
\newcommand{\ubu}{{\bf{u}}}
\newcommand{\dbu}{{\bf{d}}}
\newcommand{\tbu}{{\bf{t}}}
\newcommand{\cbu}{{\bf{c}}}
\newcommand{\kbu}{{\bf{k}}}
\newcommand{\hbu}{{\bf{h}}}
\newcommand{\sbu}{{\bf{s}}}
\newcommand{\wbu}{{\bf{w}}}
\newcommand{\xbu}{{\bf{x}}}
\newcommand{\ybu}{{\bf{y}}}
\newcommand{\zbu}{{\bf{z}}}


\newcommand{\corres}{\leftrightarrow}

\newcommand{\zero}{{\bf{0}}}
\newcommand{\one}{{\bf{1}}}
\newcommand{\nodiv}{{\, \not| \,}}
\newcommand{\notequiv}{{\,\not\equiv\, }}




\def\comb#1#2{{#1 \choose #2}}
\newcommand{\ls}[1]
    {\dimen0=\fontdimen6\the\font\lineskip=#1\dimen0
     \advance\lineskip.5\fontdimen5\the\font
     \advance\lineskip-\dimen0
     \lineskiplimit=0.9\lineskip
     \baselineskip=\lineskip
     \advance\baselineskip\dimen0
     \normallineskip\lineskip\normallineskiplimit\lineskiplimit
     \normalbaselineskip\baselineskip
     \ignorespaces}

\def\stir#1#2{\left\{#1 \atop #2 \right\}}
\def\dsum#1#2{#1 \atop #2 }
\def\defn{\stackrel{\triangle}{=}}

\begin{document}

\bibliographystyle{abbrv}

\title{Analysis of Boolean Functions based on Interaction Graphs and their influence in System Biology}
\author{Jayanta Kumar Das ,Ranjeet Kumar Rout, Pabitra Pal Choudhury\\
\small  Applied Statistics Unit, Indian Statistical Institute, Kolkata-700108, India.\\
\small Email:dasjayantakumar89@gmail.com,
\small ranjeetkumarrout@gmail.com ,\\
\small pabitrapalchoudhury@gmail.com
}

\date{}
\maketitle

\thispagestyle{plain}
\setcounter{page}{1}
\ls{1.5}


\begin{abstract}
Interaction graphs provide an important qualitative modeling approach for System Biology. This paper presents a novel approach for construction of interaction graph with the help of Boolean function decomposition. Each decomposition part (Consisting of 2-bits) of the Boolean functions has some important significance. In the dynamics of a biological system, each variable or node is nothing but gene or protein. Their regulation has been explored in terms of interaction graphs which are generated by Boolean functions. In this paper, different classes of Boolean functions with regards to Interaction Graph with biologically significant properties have been  adumbrated.
{\bf Keyword.} Boolean Function;  Decomposition Method;Interaction Graph

\end{abstract}

\section{Introduction}
Biological components (such as genes, proteins etc.) are continuously interacting through paths and their interaction regulates  the system into complex global dynamic behavior  and Biologists are currently wasting a lot of time and effort in searching for all of the available information form biological regulatory networks of biological components. Dynamics of the network can be described by recurrence of synchronous iteration of Boolean function which can be used to form Boolean Network. Again On the other hand topology of the network can be described by a sign directed graph. . An interaction graph talks about the positive and negative influences between components. A signed directed graph having one vertexes which considered to be components, indicates the static abstraction of Biological Network .

Boolean functions have huge application in the theory of computer science, cellular Networks etc.  and Boolean Networks in System Biology have been elaborately discussed in . Boolean networks (BNs) are extensively used to model biological regulatory networks  i.e. to study the interactions between Biological components such as genes, proteins etc. Each Boolean Network has some Biological Components which are independently represented by local logical Boolean functions and associates with a Boolean value for each component in Boolean Networks. All Boolean functions are not accurately reflecting the behaviors of Biological systems and it is imperative to recognize classes of Boolean functions with biologically relevant properties. A subset of Boolean functions having noble characteristics of dynamics of Boolean networks is constructed. These functions have significance for determining their potential in a model. One such notable class and their biological properties have been introduced by Kauffman . To identify Boolean network, which are biologically relevant is a major problem as the number of Boolean functions and the size of the state space of Boolean networks are growing exponentially  with the increase of components. Different technique such as classical analysis, model checking may be intractable with large complex systems. A number of operations can be carried out on Interaction graph to make biologically relevant predictions about a regulatory system and Interaction Graph can also be used for predicting qualitative aspects of system biology. Fundamental issues in the analysis of Interaction graphs are the enumeration of paths and cycles (feedback loops) and calculation of shortest positive or negative paths .  Some static analysis of Boolean Networks through Interaction graph has been studied in .

In this paper, analysis of the Boolean functions through interaction graph have been discussed by partitioning -variable Boolean function into  fixed bits. Here we present a slightly different approach from [1] with regards to the definition of Interaction graph.  Partitioning of a Boolean function into  bits helps us to identify an edge or arc and cycle on interaction graph. Arcs and Cycles on Interaction Graph are basically responsible for static analysis of Boolean Network. First we will give a formal definition of Interaction Graph based on partitioning method and then classify the Boolean function based on Interaction Graph. In section , decomposition technique is discussed and thereby Interaction Graph and their matrix representation are given. Section , Boolean functions have been analyzed with regards to Interaction Graph and section  deals with concluding remarks emphasizing the key factors of the entire analysis.

\section{Definition and Notations}
Given any Boolean function ( ,,,...., ) of -variable is a mapping from  which are having the string of bit length  bits. Decomposition of a Boolean function of -variable is the segmentation of the function into  functions with respect to inputs for all possible combination of fixed variables . Output of each segmented function is two bits string which are fixed (). Decomposition technique of any Boolean function with a single and  variable has been given in section 2.1.

\subsection{ - Decomposition}
\noindent
 -Decomposition of any Boolean function  in input  is the segmentation of  into two functions  and  which are defined by all possible inputs  where 
  
 \begin{center}
   \resizebox{10cm}{!}
   {
    \centering
    
   }
 \end{center}
 
 The bit string representations of  and  are called decomposition fragments of the  -Decomposition having the length of bits string for each decomposition fragment is .
 To decompose a -variable Boolean function from  to position having  number of variables for each segment is defined as follow, 
  \begin{center}
     \resizebox{8cm}{!}
     {
      \centering
      
     }
   \end{center}
 Where 
 
 \textbf{Example 1.} Let consider a -variable Boolean function  with the bits string . We have taken 2(n-1) variable at a time to decompose  as it is 3-variable Boolean function. So there are  decomposition fragments of the function   and they are shown below; 
 \begin{center}
   \resizebox{10cm}{!}
   {
   \centering
      
      
      
 }
 \end{center}     
 Here  indicates decomposition of the function  with regards to variable  and  and so on. The definition of Interaction Graphs with regards to decomposition technique and the analysis of Interaction Graph can be detected with the help decomposition fragments of any Boolean function.
   
 \subsection{Interaction Graph (I.G) of f}
 \noindent
 The Interaction graph of , denoted by , is the sign directed graph on vertexes set  corresponds to nodes and edges set , an arc (positive or negative) between nodes. For all , there exist an arc  if and only if there exist at least one  or  in decomposition fragments for positive and negative arc respectively.
 
 \textbf{Example } Let consider three -variable Boolean functions  with the bits string ,  and  respectively. The Decomposition fragments of the three functions  and  are shown below;
 
 \begin{center}
    \resizebox{12cm}{!}
    {
 
     
    } \end{center}
    
     \begin{center}
            
    \resizebox{6cm}{!}
            {


         }
      \end{center}
    
  Here three Boolean functions for -variable and there are  decomposition segments for each function. So there are total  decomposition segments. Output for each decomposition segments (first segment for function , second segment for function  and third segment for  and so on are shown . To represent edges connectivity of these three functions (three functions represents corresponding three nodes ,  and  respectively) of the Interaction Graph of running Example  is shown in Fig .
 \begin{table}[ht]
        \centering
        \resizebox{7.5cm}{!}
        {
        \begin{tabular}{c c c}
        \includegraphics [scale=1]{fig_1.jpg} \\
        
\end{tabular}
        }
        \begin{center}
        \textbf{Fig. 1.} I.G for functions  and 
        \end{center}
 \end{table}
   
\subsection{Matrix Representation of Interaction Graph}
\noindent
Since a graph is completely determined by specifying either its adjacency structure or its incidence structure, these specifications provide far more efficient ways of representing a large or complicated graph than a pictorial representation. As computers are more adept at manipulating numbers than at recognizing pictures, it is standard practice to communicate the specification of a graph to a computer in matrix form. We can represent node to node (vertex to vertex) connectivity of Interaction Graph by the matrices. For  nodes size of the matrix will be  i.e. a square matrix   whose both the  rows and  columns correspond to the  vertices shown in TABLE  such that \\
\begin{center}
 if ith node is connected to jth node by positive edge  if ith node is connected to jth node by negative edge  otherwise  
\end{center}
As Interaction Graph is signed directed Graph and direction of edges will be ith row to jth column () as each column is considered an individual Boolean function from node  to node , then each row from  to  represents the output (1 for 01, -1 for 10 and 0 for 11 or 00) of decomposition segment 1, segment  \ldots segment  respectively and vice versa . So the value of each cell will be  or  or .

\begin{table}[th]
\centering
\caption{Representation of  Matrix}
{\begin{tabular}{|c|c|c|c|c|c|c|c|c|c|c|}
\hline
\textbf{} & \textbf{1} & \textbf{2} & \textbf{3} & \textbf{4} & . & . & .& \textbf{n-2} & \textbf{n-1} &\textbf{n}\\
\hline 
\textbf{1} &  &   &  & & & & &  & &  \\
\hline 
\textbf{2} &  &   &  &  & & & &  & & \\
\hline 
\textbf{3} &  &   &  & & & & &  & &  \\
\hline 
\textbf{4} &  &   &  & & & & &  & &  \\
\hline 
.&  &   &  & & & & &  & &  \\
\hline
.&  &   &  & & & & &  & &  \\
\hline
.&  &   &  & & & & &  & &  \\
\hline
\textbf{n-2} &  &   &  &  & & & &  & &  \\
\hline
\textbf{n-1} &  &   &  &  & & & &  & &  \\
\hline
\textbf{n} &  &   &  &  & & & &  & &  \\
\hline
\end{tabular} }
\end{table} 
We represent two separate Matrixes i.e. Positive Matrix  and Negative Matrix  for positive edges and negative edges connectivity among nodes respectively for running Example  shown below in Table .
\begin{table}[th]
\centering
 \caption{Representation of  both Positive Matrix (M+) and Negative Matrix (M-)}
 {
  \begin{tabular}{|c|c|c|c|c|c|c|c|}
  \hline
  \multicolumn{4}{|c}{M+} & \multicolumn{4}{|c|}{M-} \\
  \hline
    & \textbf{1} & \textbf{2} & \textbf{3} &  & \textbf{1} & \textbf{2} & \textbf{3} \\ \hline
   \textbf{1} & 1 & 1 & 0 & \textbf{1} & 0 & 0 & 0\\ \hline
   \textbf{2} & 1 & 1 & 0 & \textbf{2} & 0 & 0 & -1\\ \hline
   \textbf{3} & 1 & 1 & 0 & \textbf{3} & 0 & 0 & -1\\ \hline
  \end{tabular} }
  \end{table} 
 From , we can see that there exist  paths from node  to node  (self-loop), node  to node  and node  to node  which are represented in column  and so on. And from  there exist paths node  to node  and node  to node .

\section{Analysis of Boolean Function}
This section provides analysis of Boolean functions with regards to their symmetrical Interaction graphs. For each case we classify the two sets of Boolean functions i.e. Positive Boolean Functions (PBF) and Negative Boolean Function (NBF) both binary and decimal value (DV) having similar Interaction graphs separately with positive edges and negative edges respectively.

\subsection{Only Positive or Negative Edges/Cycles in I.G}
\noindent
The Interaction graphs  have either only positive edges and positive cycles if  or only negative edge and negative cycle if  for all . Thus the Graph  using this type of functions may not always have a path  and thereby may not always cycles of any length. List of functions (for n=2, 3 and 4 variable) which are satisfied this condition are shown in Table  separately for positive and negative functions.
For  there are total  functions, for  there are total  functions and for  there are total  functions.
   
  
\textbf{Example 3:} Fig. 2(a) shown Interaction Graph of  Boolean functions  and  having positive edges only. 
  
\textbf{Example 4:} Fig. 2(b) shown Interaction Graph of  Boolean functions  and  having negative edges only.
  \begin{table}[ht]
     \centering
     \resizebox{10cm}{!}
     {
     \begin{tabular}{c c c}
     \includegraphics [scale=1]{3_3_1_1.jpg} & \includegraphics [scale=1]{3_3_1_2.jpg}\\
     
     {\fontsize{1cm}{1cm}\selectfont (a)}&
     {\fontsize{1cm}{1cm}\selectfont (b)}\\
      \end{tabular}
     }
     \begin{center}
     \textbf{Fig. 2.}(a) I.G for functions  and , (b) I.G for functions  and .
     \end{center}
 \end{table}
 
 \subsection{All Positive or All Negative Edges/Cycles in I.G (complete graph)}
  \noindent
  The Interaction graphs  (Complete I.G) have either all positive edge and positive cycle iff  or all negative edge and negative cycle iff  for all . Thus the Graph  using this type of functions always have a path  and thereby cycles of any length. List of functions (for n=2, 3 and 4 variable) which are satisfied this condition are shown in Table  separately for positive and negative functions. For  there are total  functions, for  there are total  functions and for  there are total  functions.
   
    
  \textbf{Example 5:} Fig. 3(a) shown Interaction Graph of  Boolean functions  and  having positive edges only. 
  
  \textbf{Example 6:} Fig. 3(b) shown Interaction Graph of  Boolean functions  and  having negative edges only.
  \begin{table}[ht]
       \centering
       \resizebox{10cm}{!}
       {
       \begin{tabular}{c c c}
       \includegraphics [scale=1]{3_3_2_1.jpg} & \includegraphics [scale=1]{3_3_2_2.jpg}\\
       
       {\fontsize{1cm}{1cm}\selectfont (a)}&
       {\fontsize{1cm}{1cm}\selectfont (b)}\\
        \end{tabular}
       }
       \begin{center}
       \textbf{Fig. 3.}(a) I.G for functions  and , (b) I.G for functions  and .
       \end{center}
  \end{table}
  
\subsection{Nested Canalizing Functions (NCFs) with I.G}
\noindent
Not all Boolean functions reflect the behavior of biological systems and it is imperative to recognize the biologically relevant Boolean functions. One such class of Boolean functions is nested canalyzing function having small limit cycles and small average height in state space graph. In order to reduce the chaotic behavior and to attain stability in the gene regulatory network, nested Canalizing Functions (NCFs) are best suited. NCFs and its variants have a wide range of applications in systems biology .  So identification of all -variable NCFs will be helpful for studying Boolean networks and hence biological networks.

If the Interaction graph  has no cycle, then iteraction graph  has a unique fixed point. Nested canalizing functions carry special characteristics of an Interaction Graph. NCFs are connected to all components with self-loop in I.G.  That’s why all the nested canalizing Boolean functions can be used to generate graph with cycle having both positive and negative edges simultaneously. Nested Canalizing functions  which are satisfied these conditions are shown in Table . For -variable there are total  functions, for -variable there are total  functions.

 \textbf{Example 7:} Fig. 4. shown Interaction Graph of  Nested Canalizing functions  and  having three positive edges and six negative edges.\\
\begin{table}[ht]
       \centering
       \resizebox{7.5cm}{!}
       {
       \begin{tabular}{c c c}
       \includegraphics [scale=1]{3_3_3_1.jpg} \\
       
\end{tabular}
       }
       \begin{center}
       \textbf{Fig. 4.} I.G for functions  and 
       \end{center}
  \end{table}
  
 \section{Conclusion}
 In this paper, an attempt has been made for designing interaction graphs using Boolean function decomposition and various classes of Boolean functions are obtained to model a biological system with the help of interaction graph. In this method, parallel edges are not counted between two consecutive nodes for an Interaction Graph. Further study can be extended for counting the number of Boolean functions for variable and their applications towards static analysis of biologically regulated network. By knowing the functions, which are used to represent the genes/proteins, we can predict the characteristics of these functions and thereby help to the understanding of different biological networks through the pathways.


\begin{table}[th]
\centering 
\caption{Functions List For Section 3.1}
{
\centering
       \resizebox{6cm}{!}
       {
\begin{tabular}{|c|c|c|c|c|}
\hline
\textbf{VARIABLE} & \textbf{PBF} & \textbf{DV} & \textbf{NBF} & \textbf{DV}\\
\hline 
& 1000 & 8  & 0111 & 7  \\
n=2 & 1010 & 10 & 0101 & 5 \\
& 1100 & 12 & 0011 & 3 \\
& 1110 & 14 & 0001 & 1 \\
\hline 
& 10000000 & 128 & 01111111 & 127 \\
& 10001000 & 136 & 01110111 & 119 \\
& 10100000 & 160 & 01011111 & 95  \\
& 10101000 & 168 & 01010111 & 87 \\ 
& 10101010 & 170 & 01010101 & 85 \\  
& 11000000 & 192 & 00111111 & 63 \\
& 11001000 & 200 & 00110111 & 55 \\
n=3 & 11001100 & 204 & 00110011 & 51 \\
& 11100000 & 224 & 00011111 & 31 \\
& 11101000 & 232 & 00010111 & 23 \\
& 11101010 & 234 & 00010101 & 21 \\ 
& 11101100 & 236 & 00010011 & 19 \\
& 11101110 & 238 & 00010001 & 17 \\ 
& 11110000 & 240 & 00001111 & 15 \\
& 11111000 & 248 & 00000111 & 7 \\
& 11111010 & 250 & 00000101 & 5 \\
& 11111100 & 252 & 00000011 & 3 \\
& 11111110 & 254 & 00000001 & 1 \\
\hline
& 1000000000000000 & 32768 & 0111111111111111 & 32767 \\
& 1000000010000000 & 32896 & 0111111101111111 & 32639 \\
& 1000100000000000 & 34816 & 0111011111111111 & 30719 \\
&.&.&.&. \\
n=4 &.&.&.&. \\
&.&.&.&. \\
& 1110111011001000 & 61128 & 0001000100110111 & 4407 \\
& 1110111011001100 & 61132 & 0001000100110011 & 4403 \\
& 1110111011100000 & 61152 & 0001000100011111 & 4383 \\

\hline
\end{tabular}
}}
\end{table}

\begin{table}[th]
\centering
\caption{Functions List For Section 3.2}
{
       \resizebox{5cm}{!}
       {
\begin{tabular}{|c|c|c|c|c|}
\hline
\textbf{VARIABLE} & \textbf{PBF} & \textbf{DV} & \textbf{NBF} & \textbf{DV}\\
\hline 
n=2 & 1000 & 8 & 0111 & 7 \\
 & 1110 & 14 & 0001 & 1 \\
\hline
& 10000000 & 128 & 01111111 & 127 \\
& 10101000 & 168 & 01010111 & 87 \\
& 11001000 & 200 & 00110111 & 55 \\
& 11100000 & 224 & 00011111 & 31 \\
& 11101000 & 232 & 00010111 & 23 \\
n=3 & 11101010 & 234 & 00010101 & 21 \\
& 11101100 & 236 & 00010011 & 19 \\
& 11111000 & 248 & 00000111 & 7 \\
& 11111110 & 254 & 00000001 & 1 \\
\hline
& 1000000000000000 & 32768 & 0111111111111111 & 32767 \\
& 1000100010000000 & 34944 & 0111011101111111 & 30591 \\
& 1010000010000000 & 41088 & 0101111101111111 & 24447 \\
&.&.&.&. \\ 
n=4 &.&.&.&. \\
&.&.&.&. \\
& 1111100011100000 & 63712 & 0000011100011111 & 1823 \\
& 1111100011101000 & 63720 & 0000011100010111 & 1815 \\
& 1111100011110000 & 63728 & 0000011100001111 & 1807 \\
\hline
\end{tabular}}
}
\end{table}

\begin{table}[th]
\centering
\caption{Functions List For Section 3.3}
{
       \resizebox{9cm}{!}
       {
\begin{tabular}{|c|c|c|c|c|c|c|c|c|}
\hline
\textbf{VARIABLE} & \textbf{DV} & \textbf{BF} & \textbf{DV} & \textbf{BF} & \textbf{DV} & \textbf{BF} & \textbf{DV} & \textbf{BF}\\
\hline 
& 1 & 0001 & 8 & 1000 
& 2 & 0010 & 11 & 1011 \\
n=2 & 4 & 0100 & 13 & 1101 
& 7 & 0111 & 14 & 1110 \\
\hline
& 1 & 00000001 & 2 & 00000010 
& 4 & 00000100 & 7 & 00000111 \\
& 8 & 00001000 & 11 & 00001011 
& 13 & 00001101 & 14 & 00001110 \\
& 16 & 00010000 & 19 & 00010011 
& 21 & 00010101 & 31 & 00011111 \\
& 32 & 00100000 & 35 & 00100011 
& 42 & 00101010 & 47 & 00101111 \\
& 49 & 00110001 & 50 & 00110010 
& 55 & 00110111 & 59 & 00111011 \\
& 64 & 01000000 & 69 & 01000101  
& 76 & 01001100 & 79 & 01001111 \\
& 81 & 01010001 & 84 & 01010100 
& 87 & 01010111 & 93 & 01011101 \\
n=3 & 112 & 01110000 & 115 & 01110011 
& 117 & 01110101 & 127 & 01111111 \\
& 128 & 10000000 & 138 & 10001010 
& 140 & 10001100 & 143 & 10001111 \\
& 162 & 10100010 & 168 & 10101000 
& 171 & 10101011 & 174 & 10101110 \\
& 176 & 10110000 & 179 & 10110011 
& 186 & 10111010 & 191 & 10111111 \\
& 196 & 11000100 & 200 & 11001000 
& 208 & 11010000 & 213 & 11010101 \\
& 220 & 11011100 & 223 & 11011111 
& 224 & 11100000 & 234 & 11101010 \\
& 236 & 11101100 & 239 & 11101111 
& 241 & 11110001 & 242 & 11110010 \\
& 244 & 11110100 & 247 & 11110111 
& 248 & 11111000 & 251 & 11111011 \\
& 253 & 11111101 & 254 & 11111110 
& 205 & 11001101 & 206 & 11001110 \\
\hline
\end{tabular}}
}
\end{table}

\newpage
\begin{thebibliography}{1}
 \bibitem{Pauleve} Pauleve L, Richard A, Stastic Analysis of Boolean Networks Based on Interaction graph: A survaey, {\it Springer, Electronics notes Theoretical Computer scinece} {\bf 284}:93-104, 2012.
    
 \bibitem{Huber} Huber W, Carry VJ, Long L, Falcon S, Gentleman R, Graphs in molecular biology,{\it BMC Bioinformatics} {\bf 8(Suppl 6)}:S8, 2007.
 
 \bibitem{Richard} Richard A, Positive circuits and maximal number of fixed points in discrete dynamical systems, {\it Discrete Applied Mathematics} {\bf 157} pp. 3281 – 3288, 2009.
 
 \bibitem {Golomb} Golomb S W, On the classification of Boolean functions \it{IRE transactions on circuit theory} {\bf Vol. 06}, Issue. 05, pp.176- 186, 1959.
  
  \bibitem{Slepian} Slepian D, On the number of symmetry types of Boolean functions of n variables, {\it Society for industrial and Mathematics} {\bf Vol. 5}, No. 2, pp. 185-193, 1954.
  
  \bibitem {Rout} Rout R K, Choudhuey P P and Sahoo S, Classification of Boolean Functions Wrere Affine Functions Are Uniformly Distributed, {\it Hindawi Publishing Corporation
  Journal of Discrete Mathematics} {\bf Volume 2013}, Article ID 270424, 12 pages http://dx.doi.org/10.1155/2013/270424, 2013.
  
  \bibitem{Kauffman} Kauffman S A, Metabolic stability and epigenesis in randomly connected nets,"{\it Journal of Theoretical Biology} {\bf 22} pp. 437–467, 1969
  .
  \bibitem{Kauffman} Kauffman S A, Origins of Order Self-Organization and Selection in Evolution,{\it  Oxford University Press}, 1993.
  
  \bibitem{Thomas}Thomas R, Boolean formalization of genetic control circuits, {\it Journal of Theoretical Biology} {\bf 42}, pp. 563 – 585, 1973.
  
  \bibitem{Thomas}Thomas R and d’Ari R, Biological Feedback,{\it  CRC Press}, 1990.
  
  \bibitem{Thieffry} Thieffry D, Dynamical roles of biological regulatory circuits,{\it  Brief Bioinform}, {\bf 8}:220-225, 2007.
    
  
  \bibitem{Steffen} Klamt S,and Kamp A von, Computing paths and cycles in biological interaction graphs,{\it BMC Bioinformatics} {\bf 10}:181 doi:10.1186/1471-2105-10-181, 2009.
  
  \bibitem{Cinquin} Cinquin O, Demongeot J, Positive and negative feedback: striking balance between necessary antagonists, {\it J theor Biol} {\bf 216}:229-241, 2012.
  
  \bibitem{Remy} Remy E, Ruet P and Thieffry D, Graphic requirements for multistability and attractive cycles in a boolean dynamical framework, {\it Advances in Applied Mathematics} {\bf 41}, pp. 335 – 350, 2008.
  
  \bibitem {Hinkelman} Hinkelmann F and Jarrah A S, Inferring Biologically Relevant Models: Nested Canalyzing Functions, {\it International Scholarly Research Network ISRN Biomathematics}{\bf  Volume 2012}, Article ID 613174, 7 pages doi:10.5402/2012/613174, 2012.
  
  \bibitem {Shmulevich} Murrugarra D and Laubenbacher R, The Number of Multistate Nested Canalyzing Functions,{\it  Physica D: Nonlinear Phenomena} {\bf Volume 241}, Issue 10, Pages 929-938, 2012.
  
  \bibitem{Jarrah}	Jarrah A S, Raposa B and Laubenbacher R, Nested Canalyzing Unate Cascade, and Polynomial functions, {\it Physica D} {\bf 233(2)}: 167-174, 2007.
  
  \bibitem{Ray} Ray C, Das J K and Choudhury P P, On Analysis and Generation of Some Biologically Important Boolean Functions,{\it  Conference Presented in International Symposium on CDSA -2014} http://arxiv.org/abs/1405.2271.
  
\end{thebibliography}
\end{document}
