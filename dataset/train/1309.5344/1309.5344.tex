\documentclass[conference]{IEEEtran}


\usepackage{graphicx}
\usepackage[sort&compress,numbers]{natbib}
\usepackage{amssymb}
\usepackage{amsmath}
\usepackage{amsfonts}
\usepackage{colortbl}
\usepackage{hhline}
\usepackage{color}
\usepackage[bf,small]{caption}
\usepackage{indentfirst}
\usepackage{url}
\usepackage{subfigure}
\usepackage{tikz}
\usepackage{bbding}
\usepackage[all]{xy}
\usepackage{array}
\usepackage{multirow}
\usepackage{multicol}
\usepackage{balance}
\usepackage{etex}
\usepackage{grffile}
\usepackage{paralist}
\usepackage[ruled,noend]{algorithm2e}

\newcommand{\descr}[1]{\vspace{0.25cm} \noindent \textbf{#1}}
\usepackage{url}
\usepackage{color}
\definecolor{darkblue}{RGB}{0,0,120}
\usepackage[bookmarks=false, colorlinks=true, plainpages=false,  linkcolor=darkblue,   citecolor=darkblue, urlcolor=darkblue, filecolor=darkblue]{hyperref}
\usepackage{breakurl}

\pagenumbering{arabic}
\pagestyle{plain}








\newcommand{\edc}[1]{\ding{43}\textcolor{red}{\bf #1}}
\newcommand{\julien}[1]{\textcolor{blue}{Julien:  \bf #1}}


\hyphenation{ded-i-cat-ed}




\begin{document}



\title{A Comparative Usability Study of\\ Two-Factor Authentication\thanks{A preliminary version of this paper appears in USEC 2014.}}


\author{\IEEEauthorblockN{Emiliano De Cristofaro}
\IEEEauthorblockA{University College London\\
e.decristofaro@ucl.ac.uk}
\and
\IEEEauthorblockN{Honglu Du}
\IEEEauthorblockA{PARC\\
honglu.du@parc.com}\\
\and
\IEEEauthorblockN{Julien Freudiger}
\IEEEauthorblockA{PARC\\
julien.freudiger@parc.com}\\
\and
\IEEEauthorblockN{Greg Norcie}
\IEEEauthorblockA{Indiana University\\
gnorcie@indiana.edu}
}




\maketitle







\begin{abstract}
Two-factor authentication (2F) aims to enhance resilience of password-based authentication by requiring users to provide an additional authentication factor, e.g., a code generated by a security token. However, it also introduces non-negligible costs for service providers and requires users to carry out additional actions during the authentication process.
In this paper,  we present an exploratory comparative study of the usability of 2F technologies. First, we conduct a pre-study interview to identify popular technologies as well as contexts and motivations in which they are used. We then present the results of a quantitative study based on a survey completed by 219 Mechanical Turk users, aiming to measure the usability of three popular 2F solutions: codes generated by security tokens, one-time PINs received via email or SMS, and dedicated smartphone apps (e.g., Google Authenticator). We record contexts and motivations, and study their impact on perceived usability.

We find that 2F technologies are overall perceived as usable, regardless of motivation and/or context of use. We also present an exploratory factor analysis, highlighting that three metrics -- ease-of-use, required cognitive efforts, and trustworthiness -- are enough to capture key factors affecting 2F usability.
\end{abstract}







\renewcommand{\thefootnote}{\fnsymbol{footnote}}
\footnotetext{Work done while authors were at PARC.}



\section{Introduction}

Despite their well-known security issues, passwords are still the most popular method of end-user authentication. Guessing and offline dictionary attacks on user-generated passwords are often possible due to their limited entropy. According to Ashlee Vance~\cite{hackme}, 20\% of passwords are covered by a list of only 5,000 passwords. Therefore, aiming to increase security, complex password policies are often enforced, by requiring, e.g., a minimum number of characters,  inclusion of non alpha-numeric symbols, or frequent password expiration. This often creates an undesirable conflict between security and usability -- as highlighted in the context of password selection~\cite{egelman2013does}, management~\cite{karole2011comparative,weiss2008passshapes} and composition~\cite{komanduri2011passwords,von2013survival} -- and drives users to find the easiest password that is policy-compliant~\cite{adams1999users}.



Multi-factor authentication has emerged as an alternative way to improve security by requiring the user to provide more than one authentication {\em factor}, as opposed to only a password. Authentication factors are usually of three kinds:
\begin{enumerate}
\item {\em Knowledge} -- something the user knows, e.g., a password;
\item {\em Possession} -- something the user has, e.g., a security token (also known as hardware token);
\item {\em Inherence} -- something the user is, e.g., a biometric characteristic. 
\end{enumerate}

\renewcommand{\thefootnote}{\arabic{footnote}}

In this paper, we concentrate on the most common instantiation of multi-factor authentication, i.e., the one based on two factors, which we denote as 2F.
Historically, 2F has been deployed mostly in enterprise, government, and financial sectors, where sensitivity of information and services has driven institutions to accept increased implementation/maintenance costs, and/or to impose additional actions on authenticating users. In 2005, the United States' Federal Financial Institutions Examination Council officially recommended the use of multi-factor authentication~\cite{council2005}, thus pressuring most institutions to adopt some forms of 2F authentication for online banking. Similarly, government agencies and enterprises often require employees to use 2F for, e.g., VPN authentication or B2B transactions. More recently, an increasing number of service providers, such as Google, Facebook, Dropbox, Twitter, GitHub, have also begun to provide their users with the option of enabling 2F, arguably, motivated by the increasing number of password databases hacked. Recent highly publicized incidents affected, among others, Dropbox, Twitter, Linkedin, Rockyou.


Alas, security of 2F also suffers from a few limitations: 2F technologies, including recently proposed ones based on fingerprints~\cite{iphone}, are often vulnerable to man-in-the-middle, forgery, or Trojan-based attacks, and are not completely effective against phishing~\cite{schneier2005two}. Furthermore, 2F systems introduce non-negligible costs for service providers and require users to carry out additional actions in order to authenticate, e.g., entering a one-time code and/or carrying an additional device with them. A common assumption in the IT sector, partially supported by prior work~\cite{bauer2007lessons,bonneau2012quest,braz2006security,gunson2011user,sabzevar2008universal,strouble2009productivity}, is that 2F technologies have low(er) usability compared to authentication based only on passwords, and this likely hinders larger adoption. 
In fact, a few start-up companies (e.g., PassBan, Duo Security, Authy, Encap) aim to innovate the 2F landscape by introducing more usable solutions to the market.
However, little research actually studied the usability of different 2F technologies. 

We begin to address this gap by presenting an exploratory comparative usability study. First, we conduct a pre-study interview, involving 9 participants, to identify popular 2F technologies as well as the contexts and motivations in which they are used. Then, we present the design and the results of a quantitative study (based on a survey involving 219 Mechanical Turk users), that aims to
measure the usability of a few second-factor solutions: one-time codes generated by security tokens, one-time pins received via SMS or email, and dedicated smartphone apps (such as, Google Authenticator).
Note that all our participants make use of 2F (i.e., had been forced to or had chosen to), and thus might already have a reasonable mental model of how 2F works.  




Our comparative analysis of 2F usability yields some interesting findings. We show how users' perception of 2F usability is often correlated with their individual characteristics (such as, age, gender, background), rather than with the actual technology or the context in which it is used.
We find that, overall, 2F technologies are perceived as highly usable, with little difference among them, not even when they are used for different motivations and different contexts. 
We also present an exploratory {\em factor analysis}, which demonstrates that three metrics -- ease-of-use, required cognitive efforts, and trustworthiness -- are enough to capture key factors affecting the usability of 2F technologies.
Finally, we pave the way for future qualitative studies, based on our factor analysis, to further analyze our findings and confirm their generalizability.


\section{Related Work}\label{sec:rw}
In this section, we review prior work on the usability of single- and multi-factor authentication technologies.

\subsection{Usability of Single Factor Technologies}
Adams and Sasse~\cite{adams1999users} showed that, for users, security is not a primary task, thus users feel under attack by ``capricious'' password polices. Password policies often mandate the use of long (and hard-to-remember) passwords, frequent password changes, and using different passwords across different services. This ultimately drives the user to find the simplest password that barely complies with requirements~\cite{adams1999users}. Inglesant and Sasse~\cite{inglesant2010true} analyzed ``password diaries'', i.e., they asked users to record the times they authenticated via passwords, and found that frequent password changes are a burden, users do not change passwords unless forced to, and that it is difficult for them to create memorable, secure passwords adhering to the policy. They also concluded that context of use has a significant impact on the ability of users to become familiar with complex passwords and, essentially, on their usability.

Bardram et al.~\cite{bardram2005trouble} discussed burdens on nursing staff created by hard-to-remember passwords in conjunction with frequent logouts required by healthcare security standards, such as the Health Insurance Portability and Accountability Act (HIPAA).


The impact on usability and security of password composition policies has also been studied. For instance, Komanduri et al.~\cite{komanduri2011passwords} showed that complex password policies can actually \textit{decrease} average password entropy, and that a 16-character with no additional requirements provided the highest average entropy per password. 
Egelman et al.~\cite{egelman2013does} found that for
``important'' accounts, a password meter (i.e., a visual clue on password's strength)  successfully helps increase entropy.  

Another line of work has focused on {\em password managers}.
Chiasson et al.~\cite{chiasson2006usability} compared the usability of two password managers (PwdHash and Password Multiplier), pointing to a few usability issues in both implementations and showing that users were often uncomfortable ``relinquishing control'' to password managers. 
Karole et al.~\cite{karole2011comparative} studied the usability of three password managers (LastPass, KeePassMobile, and Roboform2Go), with a focus on mobile phone users. They concluded that users preferred portable, stand-alone managers over cloud-based ones, despite the better usability of the latter, as they were not comfortable giving control of their passwords to an online entity. 

Finally, Bonneau et al.~\cite{bonneau2012quest} evaluated, without conducting any user study, authentication schemes including: plain passwords, OpenID~\cite{recordon2006openid}, security tokens, phone-based tokens, etc. They used a set of 25 subjective factors: 8 measuring usability, 6 measuring deployability, and 11 measuring security. Although they did not conduct any user study, authors concluded that: (i) no existing authentication scheme does best in all metrics, and (ii) technologies that one could classify as 2F do better than passwords in security but worse in usability.

Although not directly related to our 2F study, we will use in our factor analysis some metrics introduced in the context of password replacements~\cite{bonneau2012quest} and password managers~\cite{karole2011comparative}. 

\subsection{Usability of Multi-Factor Authentication Technologies}



Previous work has suggested that security via 2F decreases usability of end-user authentication. For instance, Braz et al.~\cite{braz2006security} showed that 2F increases ``redundancy'', thus augmenting security but decreasing usability. Along similar lines, Strouble et al.~\cite{strouble2009productivity} analyzed the effects of implementing 2F on productivity, focusing on the ``Common Access Card'' (CaC), a combined smart card/photo ID card used (at that time) by US Department of Defense (DoD) employees. They reported that users stopped checking emails at home (due to the unavailability of card readers) and that many employees accidentally left their card in the reader. Authors also estimated that the DoD spent about \6222301413010 Amazon Gift Card.

We started the interviews by reading from a list of 2F technologies, asking participants if they had used them: 
\begin{compactitem}
\item PIN from a paper/card (one-time PIN)
\item A digital certificate
\item An RSA token code
\item A Verisign token code
\item A Paypal token code
\item Google Authenticator
\item A PIN received by SMS/email
\item A USB token
\item A smartcard
\end{compactitem}

To assess users' understanding and familiarity with 2F, we let them provide a brief 
description of two-factor authentication, and explain the difference with password-based authentication. (Obviously, we did not provide users with a 2F definition prior to this question, nor mention that the study was about 2F).

Then, we asked participants {\em why} they used 2F and why they thought other people would; this helped us understand the motivation and the context in which they used 2F.
Users were also asked to recall the last time they had used any 2F technology and report any encountered issues and whether or not they wanted to change the technology (and, if so, how). If users had used multiple technologies, we also asked to compare them, and this helped us understand how participants use and perceive 2F technologies.


\subsection{Findings}
We found that most used 2F technologies included: codes generated by a \emph{security token}, received via \emph{SMS or email}, and codes generated by a dedicated \emph{smartphone app}, entered along with username and password.


Participants used 2F technology in three contexts: \emph{work} (e.g., to log into their company's VPN), \emph{personal} (e.g., to protect a social networking account), or \emph{financial} (e.g., to gain access to online banking).

Study participants used 2F because they either were \emph{forced}, \emph{wanted to}, or \emph{had an incentive}. Most users adopted security tokens because an employer or bank had forced them. Some were unhappy about this: A participant mentioned 2F was not ``worth spending 5 minutes for \2.00 for no more than 30 minutes of survey taking. 

\descr{On MTurk studies.} Previous research showed that MTurk users are a valid alternative to traditional human subject pools in social sciences. For example, Jakobsson~\cite{jakobsson2009experimenting} compared the results of a study conducted on MTurk with results conducted by an independent survey company, and found that both results were statistically indistinguishable. Furthermore, MTurk users are often more diverse in terms of age, income, education level, and geographic location than the traditional pool for social science experiments~\cite{henrich2010weirdest}.
However, research has also highlighted that MTurkers are often younger and more computer savvy~\cite{Christenson:2013}. As we will discuss in Sec.~\ref{sec:discussion}, our work is intended to serve as a preliminary study, which should guide and inform the design of a qualitative study. 


\descr{Data sanitization.} Kittur et al.~\cite{kittur2008crowdsourcing} point out that MTurk users often try to cheat at tasks. Therefore, we designed the survey to include several sanity-check questions, such as simple math questions (in the form of Likert questions) in order to verify that participants were paying attention (and were not answering randomly). We also introduced some contrasting Likert questions (e.g., ``I enjoyed using the technology'' and ``I did not enjoy using the technology'') and verified that answers were consistent. Users who did not answer correctly all sanity checks were to be discarded from the analysis (but still compensated). Actually, all users answered the sanity-checks correctly. 
Also note that analysis of the time spent by each survey participant showed completion times in line with those of test runs done by experimenters (~15--30 minutes, depending on number of 2F tech used). 

\descr{Recruitment.} We screened potential participants by asking whether they had used 2F, and presented a list of examples: security tokens, codes received via SMS/email, and dedicated smartphone apps. Users who reported to have never used any of these technologies were told that they were not eligible to participate in our survey, and blocked from proceeding further or going back to change their answer. Also note the MTurk task announcement did not state that users were required to have used 2F and merely presented it alongside other basic demographics such as age and gender. 



\descr{Demographics.} The demographics of the 219 study participants are reported in Table~\ref{demo}. Our population included 135 (61.6\%) males and 84 females (38.4\%). 50/219 (22.8\%) users reported a background in computer science, and 12/219 (5.4\%) users reported a background in computer security. Education levels ranged from high school diploma to PhD degrees. Ages ranged from 18 to 66, with an average age of 32 and a standard deviation of 10.2.

\begin{table}[ttt]
\centering
\begin{tabular}{|l r|}
\hline
\textbf{Gender} & \\
\hline
Male & 61.6\% \\
Female & 38.4\% \\
\hline \hline
\textbf{Age}& \\
\hline
18--24 & 22.4 \%\\
25--34 & 48.4 \%\\
35--44 & 17.8 \%\\
45--54 & 5.4 \%\\
55--65 & 5.4 \%\\
65+ & 0.5 \%\\
\hline \hline
\textbf{Income}& \\
\hline
Less than \10,000 -- \20,001 -- \35,001 -- \50,0001-- \75,001 -- \90,0001 -- \120,001 -- \1043\%37\%20\%21989.95\%197/21945.20\%99/21924.20\%53/21969.42\%20.39\%10.19\%54.48\%29.75\%15.77\%45.36\%39.18\%15.46\%\chi^2\chi^2(4, N=582)=65.18p<0.000119.73\%11.65\%9.25\%44.90\%53.18\%\chi^2\chi^2(4, N=775)=14.68p<0.00160.84\%27.97\%51.26\%34.73\%45.45\%42.91\%\chi^2\chi^2(4, N=775)=29.76p<0.0001\chi^2\chi^2(1, N=219)=29.76p<0.0515< 0.461\%132\%15\%14\%38\%80\%r=0.87135\alpha = 0.92\alpha = 0.74\alpha = 0.81V= 0.07, F(3,124)=3.04, p=0.020.01670.05/3Md= 5.88Md = 4.88U = 1792p = 0.001V= 0.13, F(3,63)=3.12, p=0.030.01670.05/333; N = 12Md = 3N=17Md = 2, U = 873, p = 0.003V= 0.13, F(3,63)=3.12, p=0.030.01670.05/3Md = 5.5Md=6.0, U=2755, p =0. 007V= 0.12, F(3,168) = 7.44, p = 0.00010.01670.05/3Md = 2.75, N = 12Md = 2.00, U= 4124, p = 0.001, N = 29$).






\subsection{Analysis of Open-Ended Questions}
For each 2F technology, we asked users to answer a few open-ended questions about the services/websites where the used 2F and they issues they encountered. 
As mentioned earlier, security tokens tend to be used for work, finance, and personal websites. Interestingly, users often rely on tokens to protect their online gaming accounts: the fear of losing their gaming profile is high enough for users to adopt 2F. Users complain that the authentication process is often prone to failure (``The authentication to the server was down.''), is time sensitive (``Sometimes, during the code rollover, you'd end up with a mismatch and have to start the whole process over''), and that problem resolution is complicated (``If I made three mistakes entering my code, I had to call the state help desk to have my PIN reset''). 

Email/SMS have, overall, a high variety of use cases, but were frequently used with banks as well as with Facebook, Google, and Paypal. People complained about specific issues with codes expiring or failing to be received, especially while traveling abroad. For instance, a number of users complain about SMS not working abroad (``Sometimes it wouldn't send'', ``My husband changed his phone number when moving to the US and had a lot of problems getting things.'', ``Sometimes I am unable to receive a code if I am overseas. In that case, I have to call a toll free number or e-mail customer support to receive the code via e-mail instead of text.''), and again regarding the difficult problem resolution (``The passcode they sent me didn't work and I had to call them to get a new one. It was very frustrating.''). 

Finally, we noticed that enterprises rely on (mostly proprietary) security tokens (e.g., RSA/Verisign tokens) for authentication to corporate networks in the workplace. Also, smartphone apps (e.g., Google Authenticator) are mostly used by customers who opt-in to 2F with online services providers, such as, Google, Dropbox, or Facebook.



\section{Discussion}\label{sec:discussion}



We now discuss findings drawn by our exploratory study, and highlight items for future work.

\descr{Adoption.} 2F technologies are adopted at different rates, depending on {\em contexts} and {\em motivations}. Specifically, in the work environment, codes generated by security tokens constitute the most used second factor of authentication. Codes received via email or SMS are most popular in the financial and personal contexts. Also, few users receive incentives to adopt 2F, while many utilize security tokens because they are forced to, or decide to opt-in to use dedicated smartphone apps. 

\descr{Usability.} User perceptions of the usability of 2F is often correlated with their individual characteristics (e.g., age, gender, background), rather than with the actual technology or the context/motivation in which it is used. We find that, overall, 2F technologies are perceived as highly usable, with little difference among them, not even when they are used for different motivations and different contexts. This seems to  contrast with prior work on password policies~\cite{inglesant2010true}
which showed that context of use impacts the ability of users to become 
familiar with complex passwords and, ultimately, affects their usability.

One possible explanation, supported by participants' responses to open-ended questions, is that 
most 2F users do not need to provide the second authentication factor very often. For instance, some financial institutions (e.g., Chase and Bank of America) or services provider (such as Google and Facebook) only require the second factor to be entered if a user is authenticating from an unrecognized device (e.g., from a new location or after clearing cookies).

\descr{Trustworthiness.} Another relevant finding is that users' perception of trustworthiness is not negatively correlated with ease of use and required cognitive efforts, somewhat in contrast to prior work~\cite{braz2006security,gunson2011user}. We find that 2F technologies perceived as more trustworthy are not necessarily less usable. One possible explanation is that prior work mostly compared 2F with passwords.


\descr{Impact.} We argue that our comparative analysis is essential to begin assessing attitudes and perceptions of 2F users, identifying causes of friction, driving user-centered design of usable 2F technologies, as well as informing future usable security research.
Note that, in many cases, authentication based on passwords only is actually not an option (e.g., for corporate VPN access, or for some financial services), and thus more usable 2F technologies in that context should be favored 
to avoid friction~\cite{adams1999users}, negative impact on productivity~\cite{strouble2009productivity,inglesant2010true}, as well as driving users to circumvent authentication policies they perceived as unnecessarily stringent~\cite{inglesant2010true}.
Similarly, when users have the choice to opt-in, adoption rates will likely depend on 2F usability. 



\descr{Limitations and future work.} We acknowledge that our work presents some limitations and leaves a few items to future work. First, it is based on a survey of 219 MTurk users, who, arguably, might be more computer savvy and might adopt 2F more than the general population. Second, some of the points raised by our analysis -- such as, non-correlation of usability and context/motivation of use as well as the usability metrics derived by our factor analysis -- should be validated by open-ended interviews and qualitative studies. Indeed, our current and future work includes the design of a real-world user study building on the experience and the findings from this work.


\section{Conclusions}
This paper presented an exploratory comparative study of two-factor authentication (2F) technologies. First, we reported on a pre-study interview involving 9 participants, intended to identify popular 2F technologies as well as how they are used, when, where, and why. Next, we designed and administered an online survey to 219 Mechanical Turk users, aiming to measure the usability of a few popular 2F technologies: one-time codes generated by security tokens, one-time PINs received via SMS or email, and dedicated smartphone apps. We also recorded contexts and motivations, and study their impact on perceived usability of different 2F technologies. We considered participants that used specific 2F technologies, either because they were forced to, or because they wanted to.

We presented an exploratory factor analysis to evaluate a series of parameters, including some suggested by previous work, to evaluate the usability of 2F, and show that ease of use, trustworthiness, and required cognitive effort are the three key aspects defining 2F usability. Finally, we showed that differences among the usage of 2F depend on individual characteristics of people, more than the actual technologies or contexts of use. We considered a few characteristics, such as age, gender and computer science background, and obtained a few insights into user preferences. 

Our preliminary study is essential to guide and inform the design of follow-up qualitative studies, which we plan to conduct as part of future work.





\balance
\bibliographystyle{abbrv}
\bibliography{bibfile}








\end{document}
