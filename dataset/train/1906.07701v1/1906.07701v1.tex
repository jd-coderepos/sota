

\documentclass[11pt,a4paper]{article}
\usepackage[hyperref]{acl2019}
\usepackage{times}
\usepackage{latexsym}
\usepackage{subcaption}
\usepackage{multirow,bigdelim}
\usepackage{amsmath}
\usepackage{rotating}
\usepackage{booktabs,xcolor}
\usepackage{xspace}
\usepackage{booktabs}
\usepackage{multirow}
\usepackage{todonotes}
\usepackage{url}

\newcommand{\bleu}{\texttt{BLEU}\xspace}
\newcommand{\meteor}{\texttt{METEOR}\xspace}
\newcommand{\base}{\texttt{base}\xspace}
\newcommand{\basesum}{\texttt{base+sum}\xspace}
\newcommand{\baseatt}{\texttt{base+att}\xspace}
\newcommand{\baseattobj}{\texttt{base+obj}\xspace}
\newcommand{\delib}{\texttt{del}\xspace}
\newcommand{\delibsum}{\texttt{del+sum}\xspace}
\newcommand{\delibatt}{\texttt{del+att}\xspace}
\newcommand{\delibattobj}{\texttt{del+obj}\xspace}
\newcommand{\rnd}{\texttt{RND}\xspace}
\newcommand{\amb}{\texttt{AMB}\xspace}
\newcommand{\pers}{\texttt{PERS}\xspace}
\newcommand{\comment}[1]{}

\aclfinalcopy 



\newcommand\BibTeX{B\textsc{ib}\TeX}

 
\title{Distilling Translations with Visual Awareness}

\author{Julia Ive, Pranava Madhyastha \and Lucia Specia\\
  DCS, University of Sheffield, UK\\
  Department of Computing, Imperial College London, UK \\
  {\tt j.ive@sheffield.ac.uk}\\
{\tt \{pranava,l.specia\}@imperial.ac.uk} \\}


\date{}

\begin{document}
\maketitle
\begin{abstract}
Previous work on multimodal machine translation has shown that visual information is only needed in very specific cases, for example in the presence of ambiguous words where the textual context is not sufficient. As a consequence, models tend to learn to ignore this information. 
We propose a translate-and-refine approach to this problem where images are only used by a second stage decoder. This approach is trained jointly to generate a good first draft translation and to improve over this draft by (i) making better use of the target language textual context (both left and right-side contexts) and (ii) making use of visual context. This approach leads to the state of the art results. Additionally, we show that it has the ability to recover from erroneous or missing words in the source language.
\end{abstract}

\section{Introduction}\label{sec:intro}

Multimodal machine translation (MMT) is an area of research that addresses the task of translating texts using context from an additional modality, generally static images. The assumption is that the visual context can help ground the meaning of the text and, as a consequence, generate more adequate translations. 
Current work has focused on datasets of images paired with their descriptions, which are crowdsourced in English and then translated into different languages, namely the Multi30K dataset \cite{elliott-etall_VL:2016}.

Results from the most recent evaluation campaigns in the area \cite{elliott-EtAl:2017:WMT,BarraultEtAl:2018} have shown that visual information can be helpful, as humans generally prefer translations generated by multimodal models than by their text-only counterparts. However, previous work has also shown that images are only needed in very specific cases \cite{lala-EtAl:2018:WMT}. This is also the case for humans. \newcite{frank_elliott_specia_NLE:2018} (see Figure \ref{fig:examples}) concluded that visual information is needed by humans in the presence of the following: {\bf incorrect or ambiguous} source words and {\bf gender-neutral words} that need to be marked for gender in the target language. In an experiment where human translators were asked to first translate descriptions based on their textual context only and then revise their translation based on a corresponding image, they report that these three cases accounted for 62-77\% of the revisions in the translations in two subsets of Multi30K.  

\begin{figure*}[t]
\small{
  \begin{subfigure}[c]{\textwidth}
  \vspace{1em}
    \begin{tabular}{c p{0.3cm}p{11cm}}
      \multirow{3}[15]{*}{\includegraphics[width=0.22\textwidth]{figures/5094295894.jpg}} & EN: & Three children in \underline{football} uniforms are playing \underline{football}. \1ex] 
      & PE: & Drei Kinder in \underline{Footballtrikots}  spielen \underline{Football}. \1ex]
      & DE: & \underline{Ein Baseballspieler} in einem schwarzen Shirt f\"{a}ngt \underline{einen Spieler} in einem wei{\ss}en Shirt.\1ex]
  \end{tabular}
  \vspace{3em}
\caption{\textbf{Gender-neutral} word {\em player} translated as male player ({\em Spieler})}
  \end{subfigure}
\begin{subfigure}[c]{\textwidth}
  \vspace{1em}
    \begin{tabular}{c p{0.3cm}p{11cm}}
      \multirow{3}{*}{\includegraphics[width=0.22\textwidth]{figures/581630745.jpg}} & EN: & A woman wearing a white \underline{shirt} works out on an elliptical machine.\1ex]
      & PE: & Eine Frau in einem wei{\ss}en \underline{Pullover} trainiert auf einem Crosstrainer.\1ex]
      & \baseatt: & Zwei M{\"a}nner arbeiten unter der Motorhaube eines wei{\ss}en \underline{Rennens}.\1ex]
      & \delibattobj: & Zwei M{\"a}nner arbeiten unter der Motorhaube eines wei{\ss}en \underline{Rennwagen}.\1ex]
  \end{tabular}
  \end{small}
  \caption{ \baseatt translates \textit{race car} with \textit{Rennen} (race), \delib with \textit{Auto} (car) and \delibattobj with \textit{Rennwagen} (race car). \\ \textbf{Objects}: land, vehicle, car, wheel}
  \end{subfigure}
  \begin{subfigure}[c]{\textwidth}
  \vspace{1em}
    \begin{small}
    \begin{tabular}{c p{1.7cm}p{9.6cm}}
      \multirow{3}[15]{*}{\includegraphics[width=0.22\textwidth]{figures/2725508159.jpg}} & EN: & A young child holding an oar \underline{paddling} a blue kayak \underline{in a body of water}. \1ex]
      & \delib: &  Un jeune enfant tenant une rame dans un kayak bleu \underline{sur un plan d'eau}. \1ex]
      & FR: & Un jeune enfant avec une rame \underline{pagayant} dans un kayak bleu \underline{sur un plan d'eau}. \1ex]
      & \base: & Drei Bauern ernten sich mit einem \underline{Reisfeld}.  \1ex]
      & \delibattobj: & Drei Bauern ernten sich mit einem \underline{Reishut} auf. \1ex]
  \end{tabular}
  \caption{Example of a blank resolved by the textual context for \amb: \textit{field} translated as \textit{Reisfeld} (rice field) by \base. \delibattobj incorrectly translated the blank into \textit{Reishut} (rice hat) due to detected objects. \textbf{Objects}: person, clothing, mammal}
  \end{subfigure}
  \begin{subfigure}[c]{\textwidth}
  \vspace{1em}
    \begin{tabular}{c p{1.5cm}p{9cm}}
      \multirow{3}[15]{*}{\includegraphics[width=0.22\textwidth]{figures/3350002347.jpg}} & EN: & The \underline{boy} is outside enjoying a summer day.\1ex]
      & \delib: & \underline{La femme} profite d'une journ\'{e}e d'\'{e}t\'{e}. \1ex]
      & FR: & \underline{Le gar\c{c}on} est dehors, profitant d'une journ\'{e}e d'\'{e}t\'{e}. \1ex]
      & \base: & \underline{Gel\"{a}ndemotorradfahrer} macht in einem Wald eine Kurve.\1ex]
      & \delibattobj: & \underline{Gel\"{a}ndemotorradfahrer} macht in einem Herbst eine Kurve. \1ex]
  \end{tabular}
  \caption{Example of a blank resolved by the textual context for \pers. \textit{biker} correctly translated into the Masc. form \textit{Gel\"{a}ndemotorradfahrer} (dirt biker) by \base. \textbf{Objects}: person, tree, bike, helmet}
  \end{subfigure}}
\caption{\label{table:gapped_ex} Examples of resolved blanks for test set 2016. Underlined text denotes blanked words and their translations. Object field indicates the detected objects.}
\end{figure*}

\subsection{Source degradation setup}\label{ssec:gaps_results}

Results of our source degradation experiments are shown in Table~\ref{table:gapped_res}. A first observation is that -- as with the standard setup -- the performance of our deliberation models is overall better than that of the base models. The results of the multimodal models differ for German and French. For German, \delibattobj is the most successful configuration and shows statistically significant improvements over \base for all setups. Moreover, for \rnd and \amb, it shows statistically significant improvements over \delib. However, especially for \rnd and \amb, \delib and \delibsum are either the same or slightly worse than \base. 

For French, all the deliberation models show statistically significant improvements over \base (average , ), but the image information added to \delib only improve scores significantly for test 2018 \rnd. 

This difference in performances for French and German is potentially related to the need of more significant restructurings while translating from English into German.\footnote{English and French are both languages with the subject--verb--object (SVO) sentence structure. German, on the other hand, can have subject--object--verb (SOV) constructions. For example, a German sentence \textit{Gestern bin ich in London gewesen} ({\em Yesterday have I to London been}) would need to be restructured to \textit{Yesterday I have been to London} in English.} This is where a more complex \delibattobj architecture is more helpful. This is especially true for \rnd and \amb setups where blanked words could also be verbs, the part-of-speech most influenced by word order differences between English and German (see the decreasing complexity of translations for \delib and \delibattobj for the example (c) in Figure~\ref{table:gapped_ex}). 

To get an insight into the contribution of different contexts to the resolution of blanks, we performed manual analysis of examples coming from the English-German \base, \delib and \delibattobj setups (50 random examples per setup), where we count correctly translated blanks per system.

\begin{table}[!h]\begin{center}
\scalebox{0.73}{
\begin{tabular}{c c c c c}
\toprule
setup & \base & \delib & \delibattobj & gold \\ \midrule
\rnd & 22 & 23 & \bf 24 & 79 \\
\amb & 29 & 25 & \bf 33 & 88 \\
\pers & 43 & 46 & \bf 51 & 84\\
\bottomrule
\end{tabular}}
\end{center}
\caption{\label{table:gapped_human_res} Results of human annotation of blanked translations (English-German). We report counts of blanks resolved by each system, as well as total source blank count for each selection (50 sentences selected randomly).}
\end{table}

The results are shown in Table~\ref{table:gapped_human_res}. As expected, they show that the \rnd and \amb blanks are more difficult to resolve (at most 40\% resolved as compared to 61\% for \pers). Translations of the majority of those blanks tend to be guessed by the textual context alone (especially for verbs). Image information is more helpful for \pers: we observe an increase of 10\% in resolved blanks for \delibattobj as compared to \delib. However, for \pers the textual context is still enough in the majority of the cases: models tend to associate men with sports or women with cooking and are usually right (see Figure~\ref{table:gapped_ex} example (c)).

The cases where image helps seem to be those with rather generic contexts: see Figure~\ref{table:gapped_ex} (b) where \textit{enjoying a summer day} is not associated with any particular gender and make other models choose \textit{homme} (man) or \textit{femme} (woman), and only \baseattobj chooses \textit{enfant} (child) (the option closest to the reference).  

In some cases detected objects are inaccurate or not precise enough to be helpful (e.g., when an object Person is detected) and can even harm correct translations. 

\section{Conclusions} \label{sec:concl}
We have proposed a novel approach to multimodal machine translation which makes better use of context, both textual and visual. 
Our results show that further exploring textual context through deliberation networks already leads to better results than the previous state of the art. Adding visual information, and in particular structural representations of this information, proved beneficial when input text contains noise and the language pair requires substantial restructuring from source to target. Our findings suggest that the combination of a deliberation approach and information from additional modalities is a promising direction for machine translation that is robust to noisy input. Our code and pre-processing scripts are available at \url{https://github.com/ImperialNLP/MMT-Delib}.
\section*{Acknowledgments}
The authors thank the anonymous reviewers for their useful feedback. This work was supported by the MultiMT (H2020 ERC Starting Grant No. 678017) and MMVC (Newton
Fund Institutional Links Grant, ID 352343575)
projects. We also thank the annotators for their valuable help.

\bibliography{acl2019}
\bibliographystyle{acl_natbib}
\newpage
\appendix
\section{Appendices}
\label{sec:appendix}
\begin{figure*}[ht!]
\small{
  \begin{subfigure}[c]{\textwidth}
    \begin{tabular}{c p{1.5cm}p{9cm}}
      \multirow{3}[15]{*}{\includegraphics[width=0.22\textwidth]{figures/3052436578.jpg}} & EN: & A \underline{bride} and \underline{groom} kiss under the \underline{bride}'s veil.\1ex]
      & \delib: & Ein \underline{Mann} und eine \underline{Frau} k\"{u}ssen sich unter dem \underline{Brautschleier}.\1ex]
      & DE: & Eine \underline{Braut} und \underline{Br\"{a}utigam} k\"{u}ssen sich unter dem \underline{Brautschleier} . \1ex]
      & \base: & Ein brauner Hund \underline{l\"{a}uft} an einem sandigen Strand.\1ex]
      & \delibattobj: & Ein brauner Hund \underline{l\"{a}uft} an einem sandigen Strand hinunter. \1ex]
  \end{tabular}
  \caption{\amb example: \textit{runs} is correctly translated by \base into \textit{l\"{a}uft}. {\bf Objects}: dog}
  \end{subfigure}
   }
\caption{\label{table:gapped_ex1} Examples of blanks for test set 2016 that were correctly resolved by the textual context. The underlined words denote blanked words and their translations.}
\end{figure*}
\newpage
\begin{figure*}[ht!]
\small{
  \begin{subfigure}[c]{\textwidth}
  \vspace{1em}
    \begin{tabular}{c p{1.5cm}p{9cm}}
      \multirow{3}[15]{*}{\includegraphics[width=0.22\textwidth]{figures/4953536921.jpg}} & EN: & A woman and a dog sit on a white \underline{bench} near a beach.\1ex]
      & \delib: & Eine Frau und ein Hund sitzen auf einem wei{\ss}en \underline{Sofa} in der n\"{a}he eines Strands. \1ex]
      & DE: & Eine Frau und eine Hund sitzen auf einer wei{\ss}en \underline{Bank} in der n\"{a}he eines Strandes.\1ex]
      & \base: & Deux \underline{femmes} v\^{e}tues de vert pr\'{e}parent des aliments dans un restaurant. \1ex]
      & \delibattobj: & Deux \underline{asiatiques} en vert pr\'{e}parent de la nourriture dans un restaurant. \1ex]
  \end{tabular}
  \caption{\pers example. \textit{men} correctly translated into \textit{asiatiques} (asians) by \delibattobj. \textbf{Objects}: person, clothing, man, food, cake}
  \end{subfigure}}
\caption{\label{table:gapped_ex2} Examples of blanks for test set 2016 that were correctly resolved by the multimodal context. The underlined words denote blanked words and their translations.}
\end{figure*}

\begin{figure*}[ht]
\small{
  \begin{subfigure}[c]{\textwidth}
  \vspace{1em}
    \begin{tabular}{c p{1.5cm}p{9cm}}
      \multirow{3}[15]{*}{\includegraphics[width=0.2\textwidth]{figures/72008434.jpg}} & EN: & A \underline{guy} give a kiss to a \underline{guy} also.\1ex]
      & \delib: & Ein \underline{Mann}, der sich vor, um eine \underline{Frau} zu k\"{u}ssen.  \1ex]
      & DE: & Ein \underline{Typ} k\"{u}sst einen anderen \underline{Typ} .\1ex]
      & \base: & Eine Gruppe von Studenten sitzt und \underline{schaut} nach \underline{rechts} . \1ex]
      & \delibattobj: & Eine Gruppe Sch\"{u}ler sitzt und \underline{schaut} zu rechts auf das \underline{Wasser}. \1ex]
  \end{tabular}
  \caption{\amb example. The blanks \textit{listen} and \textit{speaker} are consistently translated into \textit{schaut} (look) and \textit{rechts} (right) or \textit{Wasser} (water). \textbf{Objects}: person, clothing, man, food, cake}
  \end{subfigure}}
\caption{\label{table:gapped_ex3} Examples of unresolved blanks. The underlined words denote blanked words and their translations.}
\end{figure*}
\end{document}
