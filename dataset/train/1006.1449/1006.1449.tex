\subsection{Control-flow patterns}
\label{controlflowpatterns}

The control-flow patterns describe tasks and how their execution is controlled via connecting constructors like sequence, choice, parallel execution and join. From the perspective of de-centralisation, interesting control-flow issues relate to aspects of concurrency management, and co-ordination between the participants. We first discuss the \emph{Sequence} pattern, which provides the foundation for linking tasks into processes. The \emph{Multiple instances without a priori runtime knowledge} show how different participants can co-operate in ensuring workflow structure, and the \emph{Interleaved Parallel Routing} provides an example of a task sequence ordering between parties that are executing concurrent threads of a flow.

\subsubsection{Pattern WCP-1 (Sequence)}

The \emph{Sequence} pattern provides the most fundamental building block for processes by enabling the sequential execution of tasks one after the other. Traditionally a CWMS tracks the progress of processes and manages the allocation of the \emph{task instances} (work items) to resources (see Section \ref{resourcepatterns}), provides status information, and handles error situations. In a decentralised environment the participants need to co-operate in achieving the same functionality, for example, by using a group broadcast protocol for sending updates on process status.

All work item life cycle stages \emph{(offered, allocated, withdrawn, started, completed, failed)} and transitions between these stages need consideration in the decentralised model. Who will maintain oversight of the work item orchestration? In our proposed SOE the instance that initiated the process for the service can perform this function, and as services are built on other services, the orchestration is a hierarchical responsibility. If a truly distributed orchestration service is required, the participating members can share the work item information via secure broadcast protocols and use existing scheduling algorithms to share the management load.




\subsubsection{Pattern WCP-15 Multiple instances without a priori runtime knowledge}

\emph{"Within a given process instance, multiple instances of a task can be created. The required number of instances may depend on a number of runtime factors, including state data, resource availability and inter-process communications and is not known until the final instance has completed. Once initiated, these instances are independent of each other and run concurrently. At any time, whilst instances are running, it is possible for additional instances to be initiated. It is necessary to synchronize the instances at completion before any subsequent tasks can be triggered."} \cite{Russell2007}

\begin{figure}[htbp]
\begin{center}
\includegraphics[scale=.35]{WCP15.png}
\caption{Pattern \emph{WCP-15 Multiple instances without a priori runtime knowledge} together with its YAWL-model counterpart, adapted from a CPN-model in \cite{Russell2007}}
\label{fig:wcp15}
\end{center}
\end{figure}

Task C in figure \ref{fig:wcp15} is the multiple instance task. The two dashed areas (pre and post) show the areas of the CPN petrinet model that implement the concurrent instance creation and accounting, and the termination and merging chores respectively. While in a centrally managed workflow these chores would be performed centrally; in a decentralised setup the activity must be distributed. This chore can be split in a number of ways. For example, the pre-processing activity could be performed by a node that was allocated it by a workload balancing algorithm, and the post-processing activity could be performed by the node that is allocated with the next task. Whether this is an acceptable arrangement would depend on whether the next task can be trusted with the task of merging the parallel threads, as they would be in a position to omit individual results from selected threads. Should this be an acceptable arrangement, the pre-processor and the post-processor are required to establish a secure communication channel, for example, by using a conference key agreement protocol, so that the pre-processor can safely pass the information from \emph{p1}, \emph{p2} and \emph{p5} to the post-processor.


\subsubsection{Pattern WCP-17 Interleaved Parallel Routing}
\emph{"The Interleaved Parallel Routing pattern offers the possibility of relaxing the strict ordering that a process usually imposes over a set of tasks. Note that Interleaved Parallel Routing is related to mutual exclusion, i.e. a semaphore makes sure that tasks are not executed at the same time without enforcing a particular order."} \cite{Russell2007} In figure \ref{fig:wcp17} the place p3 implements a mutex variable in the CPN-model of the pattern. In a de-centralised environment an equivalent implementation can be achieved using the distributed mutual exclusion protocol outlined in section \ref{otherprotocols}.

\begin{figure}[htbp]
\begin{center}
\includegraphics[scale=.35]{WCP17.png}
\caption{Pattern WCP-17 Interleaved Parallel Routing as depicted in \cite{Russell2007}}
\label{fig:wcp17}
\end{center}
\end{figure}











