These appendices are not intended for publication and references to
them will be removed in the final version.

\section{Proof of Lemma \ref{lemma:partial-order}}\label{proof:lemma:partial-order}

\begin{proof}

{\bf Reflexivity} .

\noindent To show that  let us take . If  then  as required. Otherwise
if  and either  is a defined relation or , then
form  we get . The last case is when  and  is a
constrained relation. Then from  we get
. Thus we get the required .

\noindent {\bf Transitivity} .

\noindent Let us assume that . From  we have  such that conditions
(\ref{itm:rank-less})--(\ref{itm:rank-s}) are fulfilled for
. Let us take  to be the minimum of  and . Now we
need to verify that conditions
(\ref{itm:rank-less})--(\ref{itm:rank-s}) hold for . If 
we have  and . It follows that , hence
(\ref{itm:rank-less}) holds. Now let us assume that  and
either  is a defined relation or . We have  and  and
from transitivity of  we get , which gives (\ref{itm:rank-eq-def}). Alternatively
 and  is a constrained relation. We have  and  and
from transitivity of  we get , thus (\ref{itm:rank-eq-con}) holds. Let us now assume
that , hence  for some
 and . Without loss of generality let us assume that
. In case  is a defined relation we
have  and , hence . Similarly in case  is a constrained relation we have
 and . Hence , and
(\ref{itm:rank-s}) holds.

\noindent {\bf Anti-symmetry} .

\noindent Let us assume  and
. Let  be minimal such that
 and  for some . If  or  is a defined relation, then we have  and .
Hence  which is a
contradiction. Similarly if  is a constrained relation we have
 and . It follows that , which
again is a contradiction. Thus it must be the case that  for all .
\qed
\end{proof}

\section{Proof of Lemma \ref{lemma:complete-lattice}}\label{proof:lemma:complete-lattice}

\begin{proof}
  First we prove that  is a lower bound of ; that is
   for all . Let 
  be maximum such that ; since  and  clearly such  exists. From definition of 
  it follows that  for all  with
  ; hence (\ref{itm:rank-less}) holds.

  \noindent If  and either  is a defined relation or
   we have  showing that
  (\ref{itm:rank-eq-def}) holds.

  \noindent Similarly, if  is a constrained relation with
   we have  showing that
  (\ref{itm:rank-eq-con}) holds.

  \noindent Finally let us assume that ; we need to show
  that there is some  with  such that . Since we know that  is maximum such that
  , it follows that , hence
  there is a relation  with  such that ; thus (\ref{itm:rank-s}) holds.

  \noindent Now we need to show that  is the greatest
  lower bound. Let us assume that  for
  all , and let us show that . If  the result holds vacuously,
  hence let us assume . Then there exists a
  minimal  such that  for some 
  with . Let us first consider  such that
  . By our choice of  we have  hence (\ref{itm:rank-less}) holds.

  \noindent Next assume that  and either  is a defined
  relation of . Then  for all
  . It follows that  for all . Thus we have . Since
  ,
  we have  which proves
  (\ref{itm:rank-eq-def}).

  \noindent Now assume  and  is a constrained
  relation. We have that  for all
  . Since  is a constrained relation it follows
  that  for all . Thus we have . Since , we have  which proves (\ref{itm:rank-eq-con}).

  \noindent Finally since we assumed that  for some  with , it follows that
  (\ref{itm:rank-s}) holds. Thus we proved that .
  \qed

\end{proof}

\section{Proof of Proposition \ref{prop:moore-family}}\label{proof:prop:moore-family}

In order to prove Proposition \ref{prop:moore-family} we first state
and prove two auxiliary lemmas.

\begin{definition}\label{def:order-j}
\noindent We introduce  an ordering  defined by  if and only if
\begin{itemize}
\item 
\item 
\end{itemize}
\end{definition}

\begin{lemma}\label{lemma:cond}
  Assume a condition  occurs in , and let  be
  a valuation of free variables in . If  and 
  then .
\end{lemma}
\begin{proof}
  We proceed by induction on  and in each case perform a structural
  induction on the form of the condition  occurring in
  .\newline \textbf{Case: } \newline Assume
   and

From Table \ref{LFPSemantics} it follows that

Depending of the rank of  we have two sub-cases.\newline (1) Let
, then from Definition \ref{def:order-j} we know that
 and hence

Which according to Table \ref{LFPSemantics} is equivalent to

(2) Let us now assume , then from Definition
\ref{def:order-j} we know that  and hence

which is equivalent to

and finishes the case. \newline \textbf{Case: }
\newline Assume  and

From Table \ref{LFPSemantics} it follows that

Since , then from Definition \ref{def:order-j} we have
 and hence

Which according to Table \ref{LFPSemantics} is equivalent to


\noindent \textbf{Case: }

\noindent Assume  and

From Table \ref{LFPSemantics} it follows that

The induction hypothesis gives

Hence we have 


\noindent \textbf{Case: }

\noindent Assume  and

From Table \ref{LFPSemantics} it follows that

The induction hypothesis gives

Hence we have 


\noindent \textbf{Case: }

\noindent Assume  and

From Table \ref{LFPSemantics} it follows that

The induction hypothesis gives

Hence from Table \ref{LFPSemantics} we have


\noindent \textbf{Case: }

\noindent Assume  and

From Table \ref{LFPSemantics} it follows that

The induction hypothesis gives

Hence from Table \ref{LFPSemantics} we have

\qed
\end{proof}

\begin{lemma}\label{lemma:cl}
  If  and  for all  then .
\end{lemma}
\begin{proof}
We proceed by induction on  and in each case perform a structural
induction on the form of the clause  occurring in .

\noindent\textbf{Case: }

\noindent Assume

Let us also assume

Since  we know that

Let  occur in . We have two possibilities; either
 and  is a defined relation, then from
\eqref{eq:glb-property} if follows that . Alternatively  and from
\eqref{eq:glb-property} it follows that . Hence from Definition \ref{def:order-j} we have that
. Thus from Lemma \ref{lemma:cond} it
follows that

Hence from \eqref{eq:assumption-def-imply} we have

Which from Table \ref{LFPSemantics} is equivalent to

It follows that

Which from Table \ref{LFPSemantics} is equivalent to

and finishes the case.

\noindent\textbf{Case: }

\noindent Assume

From Table \ref{LFPSemantics} we have that for all 

The induction hypothesis gives

Hence from Table \ref{LFPSemantics} we have


\noindent\textbf{Case: }

\noindent Assume

From Table \ref{LFPSemantics} we have that

Thus

The induction hypothesis gives

Hence from Table \ref{LFPSemantics} we have


\noindent\textbf{Case: }

\noindent Assume

Let us also assume

From Table \ref{LFPSemantics} it follows that

Thus there is some  such that

From \eqref{eq:assumption-con-imply} it follows that

Since  we know that

Let  occur in . We have two possibilities; either
 and  is a constrained relation, then from
\eqref{eq:glb-property2} if follows that . Alternatively  and from
\eqref{eq:glb-property2} it follows that . Hence from Definition \ref{def:order-j} we have that
. Thus from Lemma \ref{lemma:cond} it
follows that

which finishes the case.

\noindent\textbf{Case: }

\noindent Assume

From Table \ref{LFPSemantics} we have that for all 

The induction hypothesis gives

Hence from Table \ref{LFPSemantics} we have


\noindent\textbf{Case: }

\noindent Assume

From Table \ref{LFPSemantics} we have that

Thus

The induction hypothesis gives

Hence from Table \ref{LFPSemantics} we have

\qed
\end{proof}

\noindent{\bf Proposition \ref{prop:moore-family}}: Assume  is a
stratified LFP formula,  and  are
interpretations of the free variables and function symbols in ,
respectively. Furthermore,  is an interpretation of all
relations of rank 0. Then  is a Moore family.
\begin{proof}
The result follows from Lemma \ref{lemma:cl}.
\qed
\end{proof}

\section{Proof of Proposition
  \ref{prop:complexity}}\label{proof:prop:complexity}
{\bf Proposition \ref{prop:complexity}}:
For a finite universe , the best solution  such that
   of a LFP formula  (w.r.t. an interpretation of the constant symbols) can be
  computed in time

where  is the maximal nesting depth of quantifiers in the 
and  is the sum of cardinalities of predicates
 of rank . We also assume unit time hash table
operations (as in \cite{bib:complex}).
\begin{proof}
  Let  be a clause corresponding to the i-th layer. Since 
  can be either a define clause, or a constrain clause, we have two
  cases.

  Let us first assume that ; the proof proceed in
  three phases. First we transform  to  by replacing
  every universal quantification  by the conjunction
  of all  possible instantiations of , every existential
  quantification  by the disjunction of all 
  possible instantiations of  and every universal quantification
   by the conjunction of all  possible
  instantiations of . The resulting clause 
  is logically equivalent to  and has size
  
  where  is the maximal nesting depth of quantifiers in .
  Furthermore,  is {\it boolean}, which means that there are
  no variables or quantifiers and all literals are viewed as nullary
  predicates.



  In the second phase we transform the formula , being the
  result of the first phase, into a sequence of formulas
   as follows. We first replace
  all top-level conjunctions in  with ",". Then we
  successively replace each formula by a sequence of simpler ones
  using the following rewrite rule
 
where
   is a fresh nullary predicate that is generated for each
  application of the rule. The transformation is completed as soon as no
  replacement can be done. The conjunction of the resulting define
  clauses is logically equivalent to .

  To show that this process terminates and that the size of  is
  at most a constant times the size of the input formula  , we
  assign a cost to the formulae.  Let us define the cost of a sequence
  of clauses as the sum of costs of all occurrences of predicate
  symbols and operators (excluding ","). In general, the cost of
  a symbol or operator is 1 except disjunction that counts 6. Then the
  above rule decreases the cost from  to , for suitable
  value of k. Since the cost of the initial sequence is at most 6
  times the size of , only a linear number of rewrite steps can
  be performed. Since each step increases the size at most by a
  constant, we conclude that the  has increased just by a
  constant factor. Consequently, when applying this transformation to
  , we obtain a boolean formula without sharing of size as in
  \eqref{eq:def-size}.  

The third phase solves the system that is a
  result of phase two, which can be done in linear time by the
  classical techniques of e.g. \cite{bib:hornsat}.

Let us now assume that the . We begin by
transforming  into a logically equivalent (modulo fresh
predicates) {\it define} clause. The transformation is done by
function  defined as

where  is a new predicate corresponding to the
complement of . The size of the formula increases by a number of
constraint predicates; hence the size of the input formula is increased
by a constant factor. Then the proof proceeds as in case of {\it
  define} clause.

The three phases of the transformation result in the sequence of
define clauses of size

which can then be solved in linear time. We also need to take into
account the size of the initial knowledge i.e.~the cardinality of all
predicates of rank ; thus the overall worst case complexity is

\qed
\end{proof}
