These appendices are not intended for publication and references to
them will be removed in the final version.

\section{Proof of Lemma \ref{lemma:partial-order}}\label{proof:lemma:partial-order}

\begin{proof}

{\bf Reflexivity} $\forall \varrho \in \Delta: \varrho \sqsubseteq \varrho$.

\noindent To show that $\varrho \sqsubseteq \varrho$ let us take $j =
s$. If $rank(R)<j$ then $\varrho(R)=\varrho(R)$ as required. Otherwise
if $rank(R)=j$ and either $R$ is a defined relation or $j=0$, then
form $\varrho(R)=\varrho(R)$ we get $\varrho(R) \subseteq
\varrho(R)$. The last case is when $rank(R)=j$ and $R$ is a
constrained relation. Then from $\varrho(R) = \varrho(R)$ we get
$\varrho(R) \supseteq \varrho(R)$. Thus we get the required $\varrho
\sqsubseteq \varrho$.

\noindent {\bf Transitivity} $\forall \varrho_1, \varrho_2, \varrho_3
\in \Delta: \varrho_1 \sqsubseteq \varrho_2 \wedge \varrho_2
\sqsubseteq \varrho_3 \Rightarrow \varrho_1 \sqsubseteq \varrho_3$.

\noindent Let us assume that $\varrho_1 \sqsubseteq \varrho_2 \wedge
\varrho_2 \sqsubseteq \varrho_3$. From $\varrho_i \sqsubseteq
\varrho_{i+1}$ we have $j_i$ such that conditions
(\ref{itm:rank-less})--(\ref{itm:rank-s}) are fulfilled for
$i=1,2$. Let us take $j$ to be the minimum of $j_1$ and $j_2$. Now we
need to verify that conditions
(\ref{itm:rank-less})--(\ref{itm:rank-s}) hold for $j$. If $rank(R)<j$
we have $\varrho_1(R) = \varrho_2(R)$ and $\varrho_2(R) =
\varrho_3(R)$. It follows that $\varrho_1(R) = \varrho_3(R)$, hence
(\ref{itm:rank-less}) holds. Now let us assume that $rank(R)=j$ and
either $R$ is a defined relation or $j=0$. We have $\varrho_1(R)
\subseteq \varrho_2(R)$ and $\varrho_2(R) \subseteq \varrho_3(R)$ and
from transitivity of $\subseteq$ we get $\varrho_1(R) \subseteq
\varrho_3(R)$, which gives (\ref{itm:rank-eq-def}). Alternatively
$rank(R)=j$ and $R$ is a constrained relation. We have $\varrho_1(R)
\supseteq \varrho_2(R)$ and $\varrho_2(R) \supseteq \varrho_3(R)$ and
from transitivity of $\supseteq$ we get $\varrho_1(R) \supseteq
\varrho_3(R)$, thus (\ref{itm:rank-eq-con}) holds. Let us now assume
that $j \neq s$, hence $\varrho_i(R) \neq \varrho_{i+1}(R)$ for some
$R \in \R$ and $i=1,2$. Without loss of generality let us assume that
$\varrho_1(R) \neq \varrho_2(R)$. In case $R$ is a defined relation we
have $\varrho_1(R) \subsetneq \varrho_2(R)$ and $\varrho_2(R)
\subseteq \varrho_3(R)$, hence $\varrho_1(R) \neq
\varrho_3(R)$. Similarly in case $R$ is a constrained relation we have
$\varrho_1(R) \supsetneq \varrho_2(R)$ and $\varrho_2(R) \supseteq
\varrho_3(R)$. Hence $\varrho_1(R) \neq \varrho_3(R)$, and
(\ref{itm:rank-s}) holds.

\noindent {\bf Anti-symmetry} $\forall \varrho_1, \varrho_2 \in \Delta: \varrho_1
\sqsubseteq \varrho_2 \wedge \varrho_2 \sqsubseteq \varrho_1
\Rightarrow \varrho_1 = \varrho_2$.

\noindent Let us assume $\varrho_1 \sqsubseteq \varrho_2$ and
$\varrho_2 \sqsubseteq \varrho_1$. Let $j$ be minimal such that
$rank(R)=j$ and $\varrho_1(R) \neq \varrho_2(R)$ for some $R \in
\R$. If $j=0$ or $R$ is a defined relation, then we have $\varrho_1(R)
\subseteq \varrho_2(R)$ and $\varrho_2(R) \subseteq \varrho_1(R)$.
Hence $\varrho_1(R) = \varrho_2(R)$ which is a
contradiction. Similarly if $R$ is a constrained relation we have
$\varrho_1(R) \supseteq \varrho_2(R)$ and $\varrho_2(R) \supseteq
\varrho_1(R)$. It follows that $\varrho_1(R) = \varrho_2(R)$, which
again is a contradiction. Thus it must be the case that $\varrho_1(R)
= \varrho_2(R)$ for all $R \in \R$.
\qed
\end{proof}

\section{Proof of Lemma \ref{lemma:complete-lattice}}\label{proof:lemma:complete-lattice}

\begin{proof}
  First we prove that $\bigsqcap M$ is a lower bound of $M$; that is
  $\bigsqcap M \sqsubseteq \varrho$ for all $\varrho \in M$. Let $j$
  be maximum such that $\varrho \in M_j$; since $M=M_0$ and $M_j
  \supseteq M_{j+1}$ clearly such $j$ exists. From definition of $M_j$
  it follows that $(\bigsqcap M)(R)=\varrho(R)$ for all $R$ with
  $rank(R)<j$; hence (\ref{itm:rank-less}) holds.

  \noindent If $rank(R)=j$ and either $R$ is a defined relation or
  $j=0$ we have $(\bigsqcap M)(R)= \bigcap \{ \varrho'(R) \mid
  \varrho' \in M_j \} \subseteq \varrho(R)$ showing that
  (\ref{itm:rank-eq-def}) holds.

  \noindent Similarly, if $R$ is a constrained relation with
  $rank(R)=j$ we have $(\bigsqcap M)(R)= \bigcup \{ \varrho'(R) \mid
  \varrho' \in M_j \} \supseteq \varrho(R)$ showing that
  (\ref{itm:rank-eq-con}) holds.

  \noindent Finally let us assume that $j \neq s$; we need to show
  that there is some $R$ with $rank(R)=j$ such that $(\bigsqcap
  M)(R)\neq\varrho(R)$. Since we know that $j$ is maximum such that
  $\varrho \in M_j$, it follows that $\varrho \notin M_{j+1}$, hence
  there is a relation $R$ with $rank(R)=j$ such that $(\bigsqcap
  M)(R)\neq\varrho(R)$; thus (\ref{itm:rank-s}) holds.

  \noindent Now we need to show that $\bigsqcap M$ is the greatest
  lower bound. Let us assume that $\varrho' \sqsubseteq \varrho$ for
  all $\varrho \in M$, and let us show that $\varrho' \sqsubseteq
  \bigsqcap M$. If $\varrho' = \bigsqcap M$ the result holds vacuously,
  hence let us assume $\varrho' \neq \bigsqcap M$. Then there exists a
  minimal $j$ such that $(\bigsqcap M)(R)\neq\varrho'(R)$ for some $R$
  with $rank(R)=j$. Let us first consider $R$ such that
  $rank(R)<j$. By our choice of $j$ we have $(\bigsqcap M)(R) =
  \varrho'(R)$ hence (\ref{itm:rank-less}) holds.

  \noindent Next assume that $rank(R)=j$ and either $R$ is a defined
  relation of $j=0$. Then $\varrho' \sqsubseteq \varrho$ for all
  $\varrho \in M_j$. It follows that $\varrho'(R) \subseteq
  \varrho(R)$ for all $\varrho \in M_j$. Thus we have $\varrho'(R)
  \subseteq \bigcap \{ \varrho(R) \mid \varrho \in M_j \}$. Since
  $(\bigsqcap M)(R) = \bigcap \{ \varrho(R) \mid \varrho \in M_j \}$,
  we have $\varrho'(R) \subseteq (\bigsqcap M)(R)$ which proves
  (\ref{itm:rank-eq-def}).

  \noindent Now assume $rank(R)=j$ and $R$ is a constrained
  relation. We have that $\varrho' \sqsubseteq \varrho$ for all
  $\varrho \in M_j$. Since $R$ is a constrained relation it follows
  that $\varrho'(R) \supseteq \varrho(R)$ for all $\varrho \in
  M_j$. Thus we have $\varrho'(R) \supseteq \bigcup \{ \varrho(R) \mid
  \varrho \in M_j \}$. Since $(\bigsqcap M)(R) = \bigcup \{ \varrho(R)
  \mid \varrho \in M_j \}$, we have $\varrho'(R) \supseteq (\bigsqcap
  M)(R)$ which proves (\ref{itm:rank-eq-con}).

  \noindent Finally since we assumed that $(\bigsqcap
  M)(R)\neq\varrho'(R)$ for some $R$ with $rank(R)=j$, it follows that
  (\ref{itm:rank-s}) holds. Thus we proved that $\varrho' \sqsubseteq
  \bigsqcap M$.
  \qed

\end{proof}

\section{Proof of Proposition \ref{prop:moore-family}}\label{proof:prop:moore-family}

In order to prove Proposition \ref{prop:moore-family} we first state
and prove two auxiliary lemmas.

\begin{definition}\label{def:order-j}
\noindent We introduce  an ordering $\subseteq_{/j}$ defined by $\varrho_1
\subseteq_{/j} \varrho_2$ if and only if
\begin{itemize}
\item $\forall R: rank(R) < j \Rightarrow \varrho_1(R) = \varrho_2(R)$
\item $\forall R: rank(R) = j \Rightarrow \varrho_1(R) \subseteq
  \varrho_2(R)$
\end{itemize}
\end{definition}

\begin{lemma}\label{lemma:cond}
  Assume a condition $\Cond$ occurs in $cl_j$, and let $\varsigma$ be
  a valuation of free variables in $\Cond$. If $\varrho_1
  \subseteq_{/j} \varrho_2$ and $(\varrho_1,\varsigma) \models \Cond$
  then $(\varrho_2,\varsigma) \models \Cond$.
\end{lemma}
\begin{proof}
  We proceed by induction on $j$ and in each case perform a structural
  induction on the form of the condition $\Cond$ occurring in
  $cl_j$.\newline \textbf{Case: }$\Cond=R(\vec x)$ \newline Assume
  $\varrho_1 \subseteq_{/j} \varrho_2$ and
$$
(\varrho_1,\varsigma) \models R(\vec x)
$$
From Table \ref{LFPSemantics} it follows that
$$
\sem{\vec x}([\,], \varsigma) \in \varrho_1(R)
$$
Depending of the rank of $R$ we have two sub-cases.\newline (1) Let
$rank(R)<j$, then from Definition \ref{def:order-j} we know that
$\varrho_1(R) = \varrho_2(R)$ and hence
$$
\sem{\vec x}([\,], \varsigma) \in \varrho_2(R)
$$
Which according to Table \ref{LFPSemantics} is equivalent to
$$
(\varrho_2,\varsigma) \models R(\vec x)
$$
(2) Let us now assume $rank(R)=j$, then from Definition
\ref{def:order-j} we know that $\varrho_1(R) \subseteq \varrho_2(R)$ and hence
$$
\sem{\vec x}([\,], \varsigma) \in \varrho_2(R)
$$
which is equivalent to
$$
(\varrho_2,\varsigma) \models R(\vec x)
$$
and finishes the case. \newline \textbf{Case: }$\Cond=\neg R(\vec x)$
\newline Assume $\varrho_1 \subseteq_{/j} \varrho_2$ and
$$
(\varrho_1,\varsigma) \models \neg R(\vec x)
$$
From Table \ref{LFPSemantics} it follows that
$$
\sem{\vec x}([\,], \varsigma) \notin \varrho_1(R)
$$
Since $rank(R)<j$, then from Definition \ref{def:order-j} we have
$\varrho_1(R) = \varrho_2(R)$ and hence
$$
\sem{\vec x}([\,], \varsigma) \notin \varrho_2(R)
$$
Which according to Table \ref{LFPSemantics} is equivalent to
$$
(\varrho_2,\varsigma) \models \neg R(\vec x)
$$

\noindent \textbf{Case: }$\Cond=\Cond_1 \wedge \Cond_2$

\noindent Assume $\varrho_1 \subseteq_{/j} \varrho_2$ and
$$
(\varrho_1,\varsigma) \models \Cond_1 \wedge \Cond_2
$$
From Table \ref{LFPSemantics} it follows that
$$
(\varrho_1,\varsigma) \models \Cond_1 \text{ and } (\varrho_1,\varsigma) \models \Cond_2
$$
The induction hypothesis gives
$$
(\varrho_2,\varsigma) \models \Cond_1 \text{ and } (\varrho_2,\varsigma) \models \Cond_2
$$
Hence we have 
$$
(\varrho_2,\varsigma) \models \Cond_1 \wedge \Cond_2
$$

\noindent \textbf{Case: }$\Cond=\Cond_1 \vee \Cond_2$

\noindent Assume $\varrho_1 \subseteq_{/j} \varrho_2$ and
$$
(\varrho_1,\varsigma) \models \Cond_1 \vee \Cond_2
$$
From Table \ref{LFPSemantics} it follows that
$$
(\varrho_1,\varsigma) \models \Cond_1 \text{ or } (\varrho_1,\varsigma) \models \Cond_2
$$
The induction hypothesis gives
$$
(\varrho_2,\varsigma) \models \Cond_1 \text{ or } (\varrho_2,\varsigma) \models \Cond_2
$$
Hence we have 
$$
(\varrho_2,\varsigma) \models \Cond_1 \vee \Cond_2
$$

\noindent \textbf{Case: }$\Cond=\exists x: \CondPr$

\noindent Assume $\varrho_1 \subseteq_{/j} \varrho_2$ and
$$
(\varrho_1,\varsigma) \models \exists x: \CondPr
$$
From Table \ref{LFPSemantics} it follows that
$$
\exists a \in \U: (\varrho_1,\varsigma[x \mapsto a]) \models \CondPr
$$
The induction hypothesis gives
$$
\exists a \in \U: (\varrho_2,\varsigma[x \mapsto a]) \models \CondPr
$$
Hence from Table \ref{LFPSemantics} we have
$$
(\varrho_2,\varsigma) \models \exists x: \CondPr
$$

\noindent \textbf{Case: }$\Cond=\forall x: \CondPr$

\noindent Assume $\varrho_1 \subseteq_{/j} \varrho_2$ and
$$
(\varrho_1,\varsigma) \models \forall x: \CondPr
$$
From Table \ref{LFPSemantics} it follows that
$$
\forall a \in \U: (\varrho_1,\varsigma[x \mapsto a]) \models \CondPr
$$
The induction hypothesis gives
$$
\forall a \in \U: (\varrho_2,\varsigma[x \mapsto a]) \models \CondPr
$$
Hence from Table \ref{LFPSemantics} we have
$$
(\varrho_2,\varsigma) \models \forall x: \CondPr
$$
\qed
\end{proof}

\begin{lemma}\label{lemma:cl}
  If $\varrho=\bigsqcap M$ and $(\varrho', \zeta, \varsigma) \models
  cl_j$ for all $\varrho' \in M$ then $(\varrho,\zeta, \varsigma)
  \models cl_j$.
\end{lemma}
\begin{proof}
We proceed by induction on $j$ and in each case perform a structural
induction on the form of the clause $cl$ occurring in $cl_j$.

\noindent\textbf{Case: }$cl_j=\define(\Cond \Rightarrow R(\vec u))$

\noindent Assume
\begin{equation}
  \forall \varrho' \in M: (\varrho',\zeta,\varsigma) \models \Cond
  \Rightarrow R(\vec u) \label{eq:assumption-def-imply}
\end{equation}
Let us also assume
$$
(\varrho,\varsigma) \models \Cond
$$
Since $\varrho=\bigsqcap M$ we know that
\begin{equation}
\forall \varrho' \in M: \varrho \sqsubseteq \varrho' \label{eq:glb-property}
\end{equation}
Let $R'$ occur in $\Cond$. We have two possibilities; either
$rank(R')=j$ and $R'$ is a defined relation, then from
\eqref{eq:glb-property} if follows that $\varrho(R') \subseteq
\varrho'(R')$. Alternatively $rank(R')<j$ and from
\eqref{eq:glb-property} it follows that $\varrho(R') =
\varrho'(R')$. Hence from Definition \ref{def:order-j} we have that
$\varrho \subseteq_{/j} \varrho'$. Thus from Lemma \ref{lemma:cond} it
follows that
$$
\forall \varrho' \in M: (\varrho',\varsigma) \models \Cond
$$
Hence from \eqref{eq:assumption-def-imply} we have
$$
\forall \varrho' \in M: (\varrho',\zeta,\varsigma) \models R(\vec u)
$$
Which from Table \ref{LFPSemantics} is equivalent to
$$
\forall \varrho' \in M: \sem{\vec u}(\zeta, \varsigma) \in \varrho'(R)
$$
It follows that
$$
\sem{\vec u}(\zeta, \varsigma) \in \bigcup \{ \varrho'(R) \mid \varrho' \in M \} = \varrho(R)
$$
Which from Table \ref{LFPSemantics} is equivalent to
$$
(\varrho,\zeta,\varsigma) \models R(\vec u)
$$
and finishes the case.

\noindent\textbf{Case: }$cl_j=\define(\Def_1 \wedge \Def_2)$

\noindent Assume
$$
  \forall \varrho' \in M: (\varrho',\zeta,\varsigma) \models \Def_1 \wedge \Def_2
$$
From Table \ref{LFPSemantics} we have that for all $\varrho' \in M$
$$
(\varrho',\zeta,\varsigma) \models \Def_1 \textit{ and } (\varrho',\zeta,\varsigma)
\models \Def_2
$$
The induction hypothesis gives
$$
(\varrho,\zeta,\varsigma) \models \Def_1 \textit{ and } (\varrho,\zeta,\varsigma)
\models \Def_2
$$
Hence from Table \ref{LFPSemantics} we have
$$
(\varrho,\zeta,\varsigma) \models \Def_1 \wedge \Def_2
$$

\noindent\textbf{Case: }$cl_j=\define(\forall x: \Def)$

\noindent Assume
\begin{equation}
  \forall \varrho' \in M: (\varrho',\zeta,\varsigma) \models \forall x: \Def
\end{equation}
From Table \ref{LFPSemantics} we have that
$$
\varrho' \in M: \forall a \in \mathcal{U}: (\varrho',\zeta,\varsigma[x
\mapsto a]) \models \Def
$$
Thus
$$
\forall a \in \mathcal{U}: \varrho' \in M: (\varrho',\zeta,\varsigma[x
\mapsto a]) \models \Def
$$
The induction hypothesis gives
$$
\forall a \in \mathcal{U}: (\varrho,\zeta,\varsigma[x
\mapsto a]) \models \Def
$$
Hence from Table \ref{LFPSemantics} we have
$$
 (\varrho,\zeta,\varsigma) \models \forall x: \Def
$$

\noindent\textbf{Case: }$cl_j=\constrain(R(\vec u) \Rightarrow \Cond)$

\noindent Assume
\begin{equation}
  \forall \varrho' \in M: (\varrho',\zeta,\varsigma) \models R(\vec u) 
  \Rightarrow \Cond \label{eq:assumption-con-imply}
\end{equation}
Let us also assume
$$
(\varrho,\zeta,\varsigma) \models R(\vec u)
$$
From Table \ref{LFPSemantics} it follows that
$$
\sem{\vec u}(\zeta, \varsigma) \in \bigcup \{ \varrho'(R) \mid \varrho' \in M \}
$$
Thus there is some $\varrho' \in M$ such that
$$
\sem{\vec u}(\zeta, \varsigma) \in \varrho'(R)
$$
From \eqref{eq:assumption-con-imply} it follows that
$$
(\varrho',\varsigma) \models \Cond
$$
Since $\varrho=\bigsqcap M$ we know that
\begin{equation}
\forall \varrho' \in M: \varrho \sqsubseteq \varrho' \label{eq:glb-property2}
\end{equation}
Let $R'$ occur in $\Cond$. We have two possibilities; either
$rank(R')=j$ and $R'$ is a constrained relation, then from
\eqref{eq:glb-property2} if follows that $\varrho(R') \supseteq
\varrho'(R')$. Alternatively $rank(R')<j$ and from
\eqref{eq:glb-property2} it follows that $\varrho(R') =
\varrho'(R')$. Hence from Definition \ref{def:order-j} we have that
$\varrho' \subseteq_{/j} \varrho$. Thus from Lemma \ref{lemma:cond} it
follows that
$$
(\varrho,\varsigma) \models \Cond
$$
which finishes the case.

\noindent\textbf{Case: }$cl_j=\constrain(\Con_1 \wedge \Con_2)$

\noindent Assume
$$
  \forall \varrho' \in M: (\varrho',\zeta,\varsigma) \models \Con_1 \wedge \Con_2
$$
From Table \ref{LFPSemantics} we have that for all $\varrho' \in M$
$$
(\varrho',\zeta,\varsigma) \models \Con_1 \textit{ and } (\varrho',\zeta,\varsigma)
\models \Con_2
$$
The induction hypothesis gives
$$
(\varrho,\zeta,\varsigma) \models \Con_1 \textit{ and } (\varrho,\zeta,\varsigma)
\models \Con_2
$$
Hence from Table \ref{LFPSemantics} we have
$$
(\varrho,\zeta,\varsigma) \models \Con_1 \wedge \Con_2
$$

\noindent\textbf{Case: }$cl_j=\constrain(\forall x: \Con)$

\noindent Assume
\begin{equation}
  \forall \varrho' \in M: (\varrho',\zeta,\varsigma) \models \forall x: \Con
\end{equation}
From Table \ref{LFPSemantics} we have that
$$
\varrho' \in M: \forall a \in \mathcal{U}: (\varrho',\zeta,\varsigma[x
\mapsto a]) \models \Con
$$
Thus
$$
\forall a \in \mathcal{U}: \varrho' \in M: (\varrho',\zeta,\varsigma[x
\mapsto a]) \models \Con
$$
The induction hypothesis gives
$$
\forall a \in \mathcal{U}: (\varrho,\zeta,\varsigma[x
\mapsto a]) \models \Con
$$
Hence from Table \ref{LFPSemantics} we have
$$
 (\varrho,\zeta,\varsigma) \models \forall x: \Con
$$
\qed
\end{proof}

\noindent{\bf Proposition \ref{prop:moore-family}}: Assume $cls$ is a
stratified LFP formula, $\varsigma_0$ and $\zeta_0$ are
interpretations of the free variables and function symbols in $cls$,
respectively. Furthermore, $\varrho_0$ is an interpretation of all
relations of rank 0. Then $\{ \varrho \mid (\varrho, \zeta_0,
\varsigma_0) \models cls \wedge \forall R: rank(R) = 0 \Rightarrow
\varrho(R) \supseteq \varrho_0(R) \}$ is a Moore family.
\begin{proof}
The result follows from Lemma \ref{lemma:cl}.
\qed
\end{proof}

\section{Proof of Proposition
  \ref{prop:complexity}}\label{proof:prop:complexity}
{\bf Proposition \ref{prop:complexity}}:
For a finite universe $\U$, the best solution $\varrho$ such that
  $\varrho_0 \sqsubseteq \varrho$ of a LFP formula $cl_1, \ldots,
  cl_s$ (w.r.t. an interpretation of the constant symbols) can be
  computed in time
\[
\mathcal{O}(|\varrho_0| + \sum_{1\leq i \leq s} |cl_i||\mathcal{U}|^{k_i})
\]
where $k_i$ is the maximal nesting depth of quantifiers in the $cl_i$
and $|\varrho_0|$ is the sum of cardinalities of predicates
$\varrho_0(R)$ of rank $0$. We also assume unit time hash table
operations (as in \cite{bib:complex}).
\begin{proof}
  Let $cl_i$ be a clause corresponding to the i-th layer. Since $cl_i$
  can be either a define clause, or a constrain clause, we have two
  cases.

  Let us first assume that $cl_i=\define(\Def)$; the proof proceed in
  three phases. First we transform $\Def$ to $\DefPr$ by replacing
  every universal quantification $\forall x: \Def_{\it cl}$ by the conjunction
  of all $|\U|$ possible instantiations of $\Def_{\it cl}$, every existential
  quantification $\exists x: \Cond$ by the disjunction of all $|\U|$
  possible instantiations of $\Cond$ and every universal quantification
  $\forall x: \Cond$ by the conjunction of all $|\U|$ possible
  instantiations of $\Cond$. The resulting clause $\DefPr$
  is logically equivalent to $\Def$ and has size
  \begin{equation}
    \O(|\U|^{k}|\Def|) \label{eq:def-size}
  \end{equation}
  where $k$ is the maximal nesting depth of quantifiers in $\Def$.
  Furthermore, $\DefPr$ is {\it boolean}, which means that there are
  no variables or quantifiers and all literals are viewed as nullary
  predicates.



  In the second phase we transform the formula $\DefPr$, being the
  result of the first phase, into a sequence of formulas
  $\DefDPr=\DefPr_1, \ldots, \DefPr_l$ as follows. We first replace
  all top-level conjunctions in $\DefPr$ with ",". Then we
  successively replace each formula by a sequence of simpler ones
  using the following rewrite rule
\[\Cond_1 \vee \Cond_2
  \Rightarrow R(\vec u) \mapsto \Cond_1 \Rightarrow Q_{new},
  \Cond_2 \Rightarrow Q_{new}, Q_{new} \Rightarrow R(\vec u)
\] 
where
  $Q_{new}$ is a fresh nullary predicate that is generated for each
  application of the rule. The transformation is completed as soon as no
  replacement can be done. The conjunction of the resulting define
  clauses is logically equivalent to $\DefPr$.

  To show that this process terminates and that the size of $\DefDPr$ is
  at most a constant times the size of the input formula $\DefPr$ , we
  assign a cost to the formulae.  Let us define the cost of a sequence
  of clauses as the sum of costs of all occurrences of predicate
  symbols and operators (excluding ","). In general, the cost of
  a symbol or operator is 1 except disjunction that counts 6. Then the
  above rule decreases the cost from $k + 7$ to $k + 6$, for suitable
  value of k. Since the cost of the initial sequence is at most 6
  times the size of $\Def$, only a linear number of rewrite steps can
  be performed. Since each step increases the size at most by a
  constant, we conclude that the $\DefDPr$ has increased just by a
  constant factor. Consequently, when applying this transformation to
  $\DefPr$, we obtain a boolean formula without sharing of size as in
  \eqref{eq:def-size}.  

The third phase solves the system that is a
  result of phase two, which can be done in linear time by the
  classical techniques of e.g. \cite{bib:hornsat}.

Let us now assume that the $cl_i=\constrain(\Con)$. We begin by
transforming $\Con$ into a logically equivalent (modulo fresh
predicates) {\it define} clause. The transformation is done by
function $f_i$ defined as
$$
\begin{array}{lll}
  f_i(\constrain(\Con)) & = & \define(g(\Con)),\define(h_i(\Con)) \\ \\
  g (\forall x: \Con) & = & \forall x: g(\Con) \\
  g(\Con_1 \wedge \Con_2) & = & g(\Con_1) \wedge g(\Con_2) \\
  g (R(\vec u) \Rightarrow \Cond) & = & (\neg \Cond[R^\complement(\vec
  u) / \neg R(\vec u)] \Rightarrow R^\complement(\vec u)) \\ \\

  h_i(\forall x: \Con) & = & \forall x: h_i(\Con) \\
  h_i(\Con_1 \wedge \Con_2) & = & h_i(\Con_1) \wedge h_i(\Con_2) \\
  h_i (R(\vec u) \Rightarrow \Cond) & = & let\ \CondPr = \Cond[true / (R'(\vec v) \mid rank(R')=i)]\ in \\
  & & \CondPr \wedge \neg R^\complement(\vec u) \Rightarrow R(\vec u) \\
\end{array}
$$
where $R^\complement$ is a new predicate corresponding to the
complement of $R$. The size of the formula increases by a number of
constraint predicates; hence the size of the input formula is increased
by a constant factor. Then the proof proceeds as in case of {\it
  define} clause.

The three phases of the transformation result in the sequence of
define clauses of size
\[
\O(\sum_{1\leq i \leq s} |cl_i||\mathcal{U}|^{k_i})
\]
which can then be solved in linear time. We also need to take into
account the size of the initial knowledge i.e.~the cardinality of all
predicates of rank $0$; thus the overall worst case complexity is
\[
\O(|\varrho_0| + \sum_{1\leq i \leq s} |cl_i||\mathcal{U}|^{k_i})
\]
\qed
\end{proof}
