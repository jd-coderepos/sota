\documentclass[proceedings]{stacs}
\stacsheading{2009}{397--408}{Freiburg}
\firstpageno{397}

\usepackage{color}

\newcommand{\AND}{\textbf{and }}
\newcommand{\BEGIN}{\textbf{begin }}
\newcommand{\FOR}{\textbf{for }}
\newcommand{\TO}{\textbf{to }}
\newcommand{\DO}{\textbf{do }}
\newcommand{\ELSE}{\textbf{else }}
\newcommand{\END}{\textbf{end}}
\newcommand{\IF}{\textbf{if }}
\newcommand{\INPUT}{\textbf{Input: }}
\newcommand{\OUTPUT}{\textbf{Output: }}
\newcommand{\RETURN}{\textbf{return }}
\newcommand{\THEN}{\textbf{then }}
\newcommand{\WHILE}{\textbf{while }}

\newlength{\ai}
\settowidth{\ai}{{\bf b}}
\newcommand{\ei}{\hspace*{\ai}}

\begin{document}
\title[Computing Graph Roots Without Short Cycles]{Computing Graph Roots Without Short Cycles}

\author[ref1]{B. Farzad}{Babak Farzad}
\address[ref1]{Department of Mathematics, Brock University, Canada.}
\email{bfarzad@brocku.ca}

\author[ref2]{L.C. Lau}{Lap Chi Lau}
\address[ref2]{Department of Computer Science and Engineering, The Chinese University of Hong Kong.}
\email{chi@cse.cuhk.edu.hk}

\author[ref3]{V.B. Le}{Van Bang Le}
\address[ref3]{Universit\"at Rostock, Institut f\"ur Informatik, Germany.}
\email{{le,nn024}@informatik.uni-rostock.de}

\author[ref3,ref4]{N.N. Tuy}{Nguyen Ngoc Tuy}
\address[ref4]{Hong Duc University, Vietnam.}
\email{nntuy@yahoo.com}
\thanks{\dag ~This author is supported by the Ministry of Education and Training, Vietnam, Grant No. 3766/QD-BGD~\&~DT}

\keywords{Graph roots, Graph powers, Recognition algorithms, NP-completeness.}
\thanks{\color{white}{.}}

\begin{abstract}
\noindent
Graph  is the square of graph  if two vertices  have an edge in
   if and only if  are of distance at most two in .
Given  it is easy to compute its square ,
  however Motwani and Sudan proved that it is NP-complete
  to determine if a given graph  is the square of some graph  (of girth ).
In this paper we consider the characterization and recognition
  problems of graphs that are squares of graphs of small girth,
  i.e. to determine if  for some graph  of small girth.
The main results are the following.
\begin{itemize}
\item There is a graph theoretical characterization for graphs that are squares of
  some graph of girth at least .
A corollary is that if a graph  has a square root  of girth at least  then  is unique up to isomorphism.
\item There is a polynomial time algorithm to recognize if  for some graph  of girth at least .
\item It is NP-complete to recognize if  for some graph  of girth .
\end{itemize}
These results almost provide a dichotomy theorem for the complexity
  of the recognition problem in terms of girth of the square roots.
The algorithmic and graph theoretical results generalize previous
  results on tree square roots, and provide polynomial time algorithms
  to compute a graph square root of small girth if it exists.
Some open questions and conjectures will also be discussed.
\end{abstract}

\maketitle

\section{Introduction}


{\em Root} and {\em root finding} are concepts familiar to most
branches of mathematics.
In graph theory,  is a {\em square root} of 
and  is the {\em square} of  if
two vertices  have an edge in  if and only if
 are of distance at most two in .
Graph square is a basic operation
with a number of results about its properties in the literature. In this paper we are interested in the characterization
and recognition problems of graph squares.
Ross and Harary \cite{RosHar1960} characterized squares of trees and showed
that tree square roots, when they exist, are unique up to isomorphism.
Mukhopadhyay \cite{Muk1967} provided a characterization of graphs which have a
square root, but this is not a good characterization in the sense
that it does not give a short certificate
when a graph does not have a square root.
In fact, such a good characterization may not exist
as Motwani and Sudan proved that it is NP-complete
to determine if a given graph has a square root \cite{MotSud1994}.
On the other hand, there are polynomial time algorithms
to compute the tree square root
\cite{LinSki1995,KeaCor1998,Lau2006,BraLeSri2006,ChaKoLu2006},
a bipartite graph square root \cite{Lau2006},
and a proper interval graph square root \cite{LauCor2004}.

The algorithms for computing tree square roots
and bipartite graph square roots are based on the fact that
the square roots have no cycles and no odd cycles respectively.
Since computing the graph square uses only local information
from the first and the second neighborhood,
it is plausible that there are polynomial time algorithms
to compute square roots that have no short cycles
(locally tree-like), and more generally to compute square roots that
have no short odd cycles (locally bipartite).
The {\em girth} of a graph is the length of a shortest cycle.
In this paper we consider the characterization and recognition
problems of graphs that are squares of graphs of small girth,
i.e. to determine if  for some graph  of small girth.

The main results of this paper are the following.
In Section~\ref{sec:girthseven} we will provide a good characterization
for graphs that are squares of some graph of girth at least .
This characterization not only leads to a simple algorithm to compute
a square root of girth at least  but also shows such a square root, if it exists, is unique
up to isomorphism.
Then, in Section~\ref{sec:girthsix}, we will present a polynomial time
algorithm to compute a square root of girth at least , or report that none exists.
In Section~\ref{sec:girth3&4} we will show that it is NP-complete to determine if a graph  has
a square root of girth .
Finally, we discuss some open questions and conjectures.

These results almost provide a dichotomy theorem for the complexity
of the recognition problem in terms of girth of the square roots.
The algorithmic and graph theoretical results considerably
generalize previous results on tree square roots.
We believe that our algorithms can be extended to
compute square roots with no short odd cycles (locally bipartite),
and in fact one part of the algorithm for computing square
roots of girth at least  uses only the assumption that
the square roots have no  cycles or  cycles.
Coloring properties of squares in terms of girth of the roots
have been considered in the literature \cite{AloMoh,CraKim,Havet};
our algorithms would allow those results to apply even
though a square root was not known apriori.

{\bf Definitions and notation:}
All graphs considered are finite, undirected and simple. Let  be a graph.
We often write  for . Following \cite{MotSud1994,LauCor2004}, we sometimes
also write  for the adjacency of  and  in the graph in question; this
is particularly the case when we describe reductions in NP-completeness proofs.

The \emph{neighborhood}  in 
of a vertex  is the set all vertices in  adjacent to  and the
\emph{closed neighborhood} of  in  is . Set ,
the \emph{degree} of  in . We call vertices of degree one in  \emph{end-vertices}
of . A \emph{center vertex} of  is one that is adjacent to all other vertices.

Let  be the length, i.e., number of edges, of a shortest path in  between
 and . Let  with  if and only if  denote
the {\em -th power of }. If  then  is the -th power of the graph  and
 is a \emph{-th root} of . Since the power of a graph  is the union of the powers
of the connected components of , we may assume that all graphs considered are connected.

A set of vertices  is called a \emph{clique} in  if every two
distinct vertices in  are adjacent; a \emph{maximal clique} is a clique that is not
properly contained in another clique. A \emph{stable set} is a set of pairwise non-adjacent
vertices. Given a set of vertices , the subgraph induced by  is written
 and  stands for . If , we write
 for . Also, we often identify a subset of vertices with the
subgraph induced by that subset, and vice versa.

The \emph{girth} of , , is the smallest length of a cycle in ; in
case  has no cycles, we set . In other words,  has girth
 if and only if  contains a cycle of length  but does not contain any (induced)
cycle of length .
Note that the girth of a graph can be computed
in  time, where  and  are the number of vertices, respectively, edges of the input
graph~\cite{ItaRod}.

A complete graph is one in which every two distinct vertices are adjacent; a complete graph on 
  vertices is also denoted by .
A \emph{star} is a graph with \emph{at least two} vertices that has a center vertex and
  the other vertices are pairwise non-adjacent.
Note that a star contains at least one edge and at least one center vertex;
  the center vertex is unique whenever the star has more than two vertices.


\section{Squares of graphs with girth at least seven}\label{sec:girthseven}
In this section, we give a good characterization of graphs that are squares
of a graph of girth at least seven.
Our characterization leads to a simple polynomial-time
recognition for such graphs.
\begin{prop}\label{prop:maxclique}
Let  be a connected, non-complete graph such that  for some graph . \\
(i)  If girth and  is a vertex with  then  is a maximal clique in~;\\
(ii) If girth and  is a maximal clique in  then  for some vertex  where .
\end{prop}

\proof
(i) Let  be a vertex with . Clearly,  is a clique in .
Consider an arbitrary vertex  outside ; in particular,  is non-adjacent in  to .
If  is non-adjacent in  to all vertices in , then . If  is adjacent
in  to a vertex , let . Then 
(otherwise  would contain a cycle of length at most five), hence . Thus, in
any case,  cannot be adjacent, in , to all vertices in , and so  is a maximal
clique in .

(ii) Let  be a maximal clique in  and  be a vertex that maximizes .
We prove that .
It can be seen that by the maximality of , .
Now, we show that if  and , then :
As , this is clear in case . So, let  and
assume to the contrary that . Then, by the choice of , there exists a vertex
, . Note that  because  has no
. As , there exists a vertex  with
. But then  contains a  or . Contradiction.

Finally, we show that , and so, by the maximality of , :
Assume otherwise and let .
As , there exists a vertex  such that , and so,
. By the maximality of ,  must be non-adjacent (in ) to a vertex
. In fact,  as  is adjacent in  to every vertex in .
Since , there exists a vertex  such that
; note that . Now, if  then 
contains a cycle of length at most five. If , let  be a vertex such that
; possibly . Then  contains a cycle of length at most
six. In any case we have a contradiction, hence .\qed

The -cycle  and the -cycle  show that (i), respectively, (ii) in
Proposition~\ref{prop:maxclique} is best possible with respect to the girth condition of
the root. More generally, the maximal cliques in the square of the subdivision of any complete graph
on  vertices do not satisfy Condition (ii).


\begin{definition}\label{defi:forced}
Let  be an arbitrary graph. An edge of  is called \emph{forced} if it is contained in (at least)
two distinct maximal cliques in .
\end{definition}
\begin{proposition}\label{prop:forced}
Let  be a connected, non-complete graph such that  for some graph  with girth at least
seven, and let  be the subgraph of  consisting of all forced edges of . Then\\
(i)  is obtained from  by deleting all end-vertices in ;\\
(ii) for every maximal clique  in ,  is a star; and\\
(iii) every vertex in  belongs to exactly one maximal clique in .
\end{proposition}
\proof First we observe that  is a forced edge in  iff  is an edge in 
  with  and .
Now, (i) follows directly from the above observations. For (ii),
consider a maximal clique  in . By Proposition~\ref{prop:maxclique},  for
some vertex  with . Let  be the set of all neighbors of  in  that are
end-vertices in  and . Since  is not complete, . By (i),
, hence  which implies (ii).
For (iii), consider a vertex  and a maximal clique  containing .
Then,  cannot belong to  and therefore  is the only maximal clique containing . \qed

We now are able to characterize squares of graphs with girth at least seven as follows.
\begin{theorem}\label{thm:girthseven}
Let  be a connected, non-complete graph. Let  be the subgraph of  consisting of all forced
edges in . Then  is the square of a graph with girth at least seven if and only if the following
conditions hold.\\
(i)   Every vertex in  belongs to exactly one maximal clique in . \\
(ii)  Every edge in  belongs to exactly two distinct maximal cliques in .\\
(iii) Every two non-disjoint edges in  belong to a common maximal clique in .\\
(iv)  For each maximal clique  of ,  is a star.\\
(v)    is connected and has girth at least seven.
\end{theorem}
\proof For the only if-part, (ii) and (iii) follow easily from Proposition~\ref{prop:maxclique},
and (i), (iv) and (v) follow directly from Proposition~\ref{prop:forced}.

For the if-part, let  be a connected graph satisfying (i) -- (v). We will construct a spanning
subgraph  of  with girth at least seven such that  as follows. For each edge  in
 let, by (ii) and (iv),  be \emph{the} two maximal cliques in  with .
Let, without loss of generality, .
Assuming  is a center vertex of the star , then  is a center vertex of the star :
Otherwise, by (iv),  is the center vertex of the star  and
  there exists some  such that ; note that  (by (iv)).
As , there is an edge  in . By (iii), .
Now, as  is maximal, the maximal clique  containing  is different from .
But then , i.e., , hence  contains a triangle ,
contradicting (v).

Thus, assuming  is a center vertex of the star ,  is a center vertex of the
star . Then put the edges , , and , , into .

By construction,  and by (i),


Furthermore, as every maximal clique in  contains a forced edge (by (iv)),  is a spanning
subgraph of . Moreover,  is an induced subgraph of : Consider an edge  with
. By construction of ,  or  is a center vertex of the star 
for some maximal clique  in . Since ,  must be an edge of this star, i.e.,
. Thus,  is an induced subgraph of . In particular, by (\ref{eqn:2}) and (v),
 is connected and .

Now, we complete the proof of Theorem~\ref{thm:girthseven} by showing that .
Let  and let  be a maximal clique in  containing
. By (iv),  contains a forced edge  and  or  is a center vertex of the star
. By construction of ,  and , or else  and  are edges of ,
hence . This proves .
Now, let . Then there exists a vertex  such that
. By (\ref{eqn:2}), , and by (\ref{eqn:3}), .
This proves . \qed

\begin{corollary}\label{coro:girthseven}
Given a graph , it can be recognized in  time if  is the
square of a graph  with girth at least seven. Moreover, such a square root, if any, can be
computed in the same time.
\end{corollary}
\proof Note that by Proposition~\ref{prop:maxclique}, any square of an -vertex graph with girth
at least seven has at most  maximal cliques. Now, to avoid triviality, assume  is connected
and non-complete. We first use the algorithm in \cite{TIAS} to list the maximal cliques in  in
time . If there are more than  maximal cliques,  is not the square of any graph
with girth at least seven. Otherwise, compute the forced edges of  to form the subgraph  of
. This can be done in time  in an obvious way.
Conditions (i) -- (v) in Theorem~\ref{thm:girthseven} then can
be tested within the same time bound, as well as the square root , in case all conditions are
satisfied, according to the proof of Theorem~\ref{thm:girthseven}. \qed

\begin{corollary}\label{coro:unique}
The square roots with girth at least seven of squares of graphs with girth at least seven are
unique, up to isomorphism.
\end{corollary}
\proof Let  be the square of some graph  with girth . If  is complete, clearly,
every square root with girth  of  must be isomorphic to the star  where
 is the vertex number of .

Thus, let  be non-complete, and let  be the subgraph of  formed by the forced edges. If
 has only one edge,  clearly consists of exactly two maximal cliques, , , say,
and  is the only forced edge of . Then, it is easily seen that every square root
with girth  of  must be isomorphic to the double star  having center edge 
and .

So, assume  has at least two edges. Then for each two maximal cliques
 in  with ,  or  is the unique center vertex of the star
 or . Hence, for any end-vertex  of , i.e., ,
the neighbor of  in  is unique. Since  is the graph resulting from  by deleting
all end-vertices,  is therefore unique. \qed


\subsection{Further Considerations}\label{sec:girth-k}

Squares of bipartite graphs can be recognized in  time in \cite{Lau2006},
where  is the maximum degree of the -vertex input graph  and  is
the time needed to perform the multiplication of two -matrices. However, no good
characterization is known so far. As bipartite graphs with girth at least seven are exactly the
-free bipartite graphs, we immediately have:

\begin{corollary}\label{coro:C4C6freebip}
Let  be a connected, non-complete graph. Let  be the subgraph of  consisting of all forced
edges in . Then  is the square of a -free bipartite graph if and only if the following
conditions hold.\\
(i)  Every vertex in  belongs to exactly one maximal clique in .\\
(ii)  Every edge in  belongs to exactly two distinct maximal cliques in .\\
(iii) Every two non-disjoint edges in  belong to the same maximal clique in .\\
(iv)  For each maximal clique  of ,  is a star.\\
(v)    is a connected -free bipartite graph.\\
Moreover, squares of -free bipartite graphs can be recognized in  time, and the
-free square bipartite roots of such squares are unique, up to isomorphism.
\end{corollary}

Using the results in this section, we obtain a new characterization
for tree squares that allow us to derive the known results on tree
square roots easily.



It was shown in \cite{LinSki1995} that \textsc{clique} and \textsc{stable set} remain NP-complete on
squares of graphs (of girth three). Another consequence of our results is.

\begin{corollary}\label{coro:CLIQUE}
The weighted version of \textsc{clique} can be solved in  time on squares of graphs
with girth at least 7, where  and  are the number of vertices, respectively, edges of the
input graph.
\end{corollary}
\proof Let  be the square of some graph with girth at least seven.
By Proposition~\ref{prop:maxclique},  has  maximal cliques. By \cite{TIAS}, all maximal
cliques in  then can be listed in time .\qed

In \cite{HorKil}, it was shown that \textsc{stable set} is even NP-complete on squares of the subdivision
of some graph (i.e. the squares of the total graph of some graph). As the subdivision of a graph has girth
at least six, \textsc{stable set} therefore is NP-complete on squares of graphs with girth at least six.


\section{Squares of graphs with girth at least six}\label{sec:girthsix}
In this section we will show that squares of graphs with girth at least six can be recognized
efficiently. Formally, we will show that the following problem


\textsc{square of graph with girth at least six}\1ex]
\begin{tabular}{l l}
{\em Instance:}& A graph ,  and . \\
{\em Question:}& Does there exist a -free graph  such that  and ?\\
\end{tabular}

An efficient recognition algorithm for \textsc{-free square root with a specified neighborhood}
relies on the following fact.

\begin{lemma}\label{lem:c3c5-free}
Let  for some -free graph . Then, for all vertices  and
all vertices , .
\end{lemma}
\proof First, consider an arbitrary vertex . Clearly, , as well
 . Also, since  is -free, . Thus
 .

 Conversely, let  be an arbitrary vertex in .
 Assuming , then  and there exist vertices  and  such that
  and . As  is -free, , ,
 and . But then  and  induce a  in , a contradiction.
 Thus .\qed

Recall that  stands for the time needed to perform a matrix multiplication of two  matrices;
currently, .

\begin{theorem}\label{thm:c3c5-free}
\textsc{-free square root with a specified neighborhood} has at most one solution.
The unique solution, if any, can be constructed in time .
\end{theorem}
\proof Given ,  and , assume  is a -free
square root of  such that . Then, by Lemma~\ref{lem:c3c5-free},
the neighborhood in  of each vertex  is uniquely determined by
. By repeatedly applying Lemma~\ref{lem:c3c5-free}
for each  and  and noting that all considered graphs are
connected, we can conclude that  is unique.

Lemma~\ref{lem:c3c5-free} also suggests the following BFS-like procedure, Algorithm~1
below, for constructing the -free square root  of  with ,
if any.

It can be seen, by construction, that  is -free, and thus the
correctness of Algorithm~1 follows from Lemma~\ref{lem:c3c5-free}. Moreover,
since every vertex is enqueued at most once, lines 1--13 take  steps,
. Checking if  (line~14) takes  steps, .\qed

\begin{center}
    ALGORITHM 1 \1ex]
\begin{tabular}{l l}
{\em Instance:}& A graph  and an edge .\\
{\em Question:}& Does there exist a graph  with girth at least six such that \\
               & and ?\\
\end{tabular}

The question is easy if .
So, for the rest of this section, assume that .
Then, we will reduce this problem to
\textsc{-free square root with a specified neighborhood}.
Given a graph  and an edge  of , write ,
i.e.,  is the set of common neighbors of  and  in .

\begin{lemma}\label{lem:fix-edge1}
Suppose  is of girth at least ,  and . Then
 has at most two connected components. Moreover, if  and  are the connected components of 
one of them maybe empty then (i)  and , or (ii)  and .
\end{lemma}



By Lemma~\ref{lem:fix-edge1}, we can solve \textsc{girth  root graph with one specified edge}
as follows: Compute . If  has more than two connected components, there is no solution.
If  is connected, solve
\textsc{-free square root with a specified neighborhood} for inputs  and . If, for  or , Algorithm~1 outputs 
and  is -free, then  is a solution. In other cases there is no solution. If 
has two connected components,  and , solve
\textsc{-free square root with a specified neighborhood} for inputs , , , , and make a decision
similar as before. In this way, checking if a graph is -free is the most expensive step,
and we obtain

\begin{theorem}\label{thm:fix-edge}
\textsc{girth  root graph with one specified edge} can be solved in time .
\end{theorem}


Let  denote the minimum vertex degree in . Now we can state the main result of this
section as follows.
\begin{theorem}\label{thm:girth6}
\textsc{square of graph with girth at least six} can be solved in time .
\end{theorem}
\proof Given , let  be a vertex of minimum degree in . For each vertex 
check if the instance  for \textsc{girth  root graph with one specified edge}
has a solution.\qed




\section{Squares of graphs with girth four}\label{sec:girth3&4}
Note that the reductions for proving the NP-completeness results by Motwani and Sudan~\cite{MotSud1994} show that recognizing squares of graphs with
girth three is NP-complete.
In this section we show that the following problem is NP-complete.

\textsc{square of graph with girth four}\1ex]
\begin{tabular}{l l}
{\em Instance:}& Collection  of subsets of a finite set .\\
{\em Question:}& Is there a partition of  into two disjoint subsets  and  such that\\
               & each subset in  intersects both  and ?\\
\end{tabular}

Our reduction is a modification of the reductions
for proving the NP-completeness of \textsc{square of chordal graph} \cite[Theorem 3.5]{LauCor2004}
and for \textsc{cube of bipartite graph} \cite[Theorem 7.6]{Lau2006}.
We also apply the tail structure of a vertex , first described in~\cite{MotSud1994},
to ensure that  has the same neighbors in any square root  of .

\begin{lemma}[\cite{MotSud1994}]\label{lem:tail}
Let  be vertices of a graph  such that (i) the only neighbors of  are  and ,
(ii)~the only neighbors of  are , and , and
(iii)~ and  are adjacent.
Then the neighbors, in , of  in any square root of  are the same as the
neighbors, in , of  in ; see Figure~.
\end{lemma}
\begin{figure}[H]
  \begin{center}
    \fbox{\quad
    \begin{picture}(0,0)\includegraphics{tail.pstex}\end{picture}\setlength{\unitlength}{3947sp}\begingroup\makeatletter\ifx\SetFigFont\undefined \gdef\SetFigFont#1#2#3#4#5{\reset@font\fontsize{#1}{#2pt}\fontfamily{#3}\fontseries{#4}\fontshape{#5}\selectfont}\fi\endgroup \begin{picture}(4805,1236)(109,-980)
\put(120,-512){\makebox(0,0)[lb]{\smash{\SetFigFont{7}{8.4}{\rmdefault}{\mddefault}{\updefault}{\color[rgb]{0,0,0}}}}}
\put(573,-514){\makebox(0,0)[lb]{\smash{\SetFigFont{7}{8.4}{\rmdefault}{\mddefault}{\updefault}{\color[rgb]{0,0,0}}}}}
\put(1020,-514){\makebox(0,0)[lb]{\smash{\SetFigFont{7}{8.4}{\rmdefault}{\mddefault}{\updefault}{\color[rgb]{0,0,0}}}}}
\put(1473,-507){\makebox(0,0)[lb]{\smash{\SetFigFont{7}{8.4}{\rmdefault}{\mddefault}{\updefault}{\color[rgb]{0,0,0}}}}}
\put(2775,-514){\makebox(0,0)[lb]{\smash{\SetFigFont{7}{8.4}{\rmdefault}{\mddefault}{\updefault}{\color[rgb]{0,0,0}}}}}
\put(3228,-516){\makebox(0,0)[lb]{\smash{\SetFigFont{7}{8.4}{\rmdefault}{\mddefault}{\updefault}{\color[rgb]{0,0,0}}}}}
\put(3675,-516){\makebox(0,0)[lb]{\smash{\SetFigFont{7}{8.4}{\rmdefault}{\mddefault}{\updefault}{\color[rgb]{0,0,0}}}}}
\put(4128,-509){\makebox(0,0)[lb]{\smash{\SetFigFont{7}{8.4}{\rmdefault}{\mddefault}{\updefault}{\color[rgb]{0,0,0}}}}}
\end{picture}
     \quad}
    \caption{Tail in  (left) and in  (right)}
    \label{tail}
  \end{center}
\end{figure}

We now are going to describe the reduction.
Let ,  where , ,
be an instance of \textsc{set splitting}. We construct an instance  for
\textsc{square of graph with girth four} as follows.

The vertex set of graph  consists of:\\
\textbf{(I)} , . Each `element vertex'  corresponds to the element  in .\\
\textbf{(II)} , . Each `subset vertex'  corresponds to the subset  in .\\
\textbf{(III)} , . Each three `tail vertices'  of
       the subset vertex  correspond to the subset  in .\\
\textbf{(IV)} , four `partition vertices'.\\
\textbf{(V)} , a `connection vertex'.

The edge set of graph  consists of:\\
\textbf{(I)} Edges of tail vertices of subset vertices:\\
       For all : , , , ,
       , and for all ,  whenever .\\
\textbf{(II)} Edges of subset vertices:\\
       For all : , , , ,
       ,  for all , and  for all  with .\\
\textbf{(III)} Edges of element vertices:\\
       For all : , , , , ,
       and  for all .\\
\textbf{(IV)} Edges of partition vertices:\\
       , , ,
       , , ,
       , .

Clearly,  can be constructed from  in polynomial time. For an illustration, given
 and  with ,
, , and , the graph  is depicted in
Figure~\ref{sq-girth4}. In the figure, the two dotted lines from a vertex to the clique
 mean that the vertex is adjacent to all vertices in that clique.

Note that, apart from the three vertices , and  (or, symmetrically, , and
), our construction is the same as those in \cite[\S 3.1.1]{LauCor2004}. While  and 
will represent a partition of the ground set  (Lemma~\ref{lem:girth4-ss}), the vertices ,
and  allow us to make a square root of  being -free (Lemma~\ref{lem:ss-girth4}).
\begin{figure}[ht]\begin{center}
    \fbox{\quad
    \begin{picture}(0,0)\includegraphics{sq-girth4.pstex}\end{picture}\setlength{\unitlength}{3947sp}\begingroup\makeatletter\ifx\SetFigFont\undefined \gdef\SetFigFont#1#2#3#4#5{\reset@font\fontsize{#1}{#2pt}\fontfamily{#3}\fontseries{#4}\fontshape{#5}\selectfont}\fi\endgroup \begin{picture}(4991,3415)(168,-2711)
\put(4777,-1430){\makebox(0,0)[lb]{\smash{\SetFigFont{7}{8.4}{\rmdefault}{\mddefault}{\updefault}{\color[rgb]{0,0,0}}}}}
\put(3732,-1433){\makebox(0,0)[lb]{\smash{\SetFigFont{7}{8.4}{\rmdefault}{\mddefault}{\updefault}{\color[rgb]{0,0,0}}}}}
\put(1628,-1436){\makebox(0,0)[lb]{\smash{\SetFigFont{7}{8.4}{\rmdefault}{\mddefault}{\updefault}{\color[rgb]{0,0,0}}}}}
\put(3282,-2684){\makebox(0,0)[lb]{\smash{\SetFigFont{7}{8.4}{\rmdefault}{\mddefault}{\updefault}{\color[rgb]{0,0,0}}}}}
\put(2144,-1436){\makebox(0,0)[lb]{\smash{\SetFigFont{7}{8.4}{\rmdefault}{\mddefault}{\updefault}{\color[rgb]{0,0,0}}}}}
\put(2633,-1443){\makebox(0,0)[lb]{\smash{\SetFigFont{7}{8.4}{\rmdefault}{\mddefault}{\updefault}{\color[rgb]{0,0,0}}}}}
\put(566,168){\makebox(0,0)[lb]{\smash{\SetFigFont{7}{8.4}{\rmdefault}{\mddefault}{\updefault}{\color[rgb]{0,0,0}}}}}
\put(837,166){\makebox(0,0)[lb]{\smash{\SetFigFont{7}{8.4}{\rmdefault}{\mddefault}{\updefault}{\color[rgb]{0,0,0}}}}}
\put(1307,169){\makebox(0,0)[lb]{\smash{\SetFigFont{7}{8.4}{\rmdefault}{\mddefault}{\updefault}{\color[rgb]{0,0,0}}}}}
\put(1853,167){\makebox(0,0)[lb]{\smash{\SetFigFont{7}{8.4}{\rmdefault}{\mddefault}{\updefault}{\color[rgb]{0,0,0}}}}}
\put(2106,168){\makebox(0,0)[lb]{\smash{\SetFigFont{7}{8.4}{\rmdefault}{\mddefault}{\updefault}{\color[rgb]{0,0,0}}}}}
\put(2367,167){\makebox(0,0)[lb]{\smash{\SetFigFont{7}{8.4}{\rmdefault}{\mddefault}{\updefault}{\color[rgb]{0,0,0}}}}}
\put(2719,169){\makebox(0,0)[lb]{\smash{\SetFigFont{7}{8.4}{\rmdefault}{\mddefault}{\updefault}{\color[rgb]{0,0,0}}}}}
\put(2997,169){\makebox(0,0)[lb]{\smash{\SetFigFont{7}{8.4}{\rmdefault}{\mddefault}{\updefault}{\color[rgb]{0,0,0}}}}}
\put(3937,168){\makebox(0,0)[lb]{\smash{\SetFigFont{7}{8.4}{\rmdefault}{\mddefault}{\updefault}{\color[rgb]{0,0,0}}}}}
\put(4214,166){\makebox(0,0)[lb]{\smash{\SetFigFont{7}{8.4}{\rmdefault}{\mddefault}{\updefault}{\color[rgb]{0,0,0}}}}}
\put(4530,166){\makebox(0,0)[lb]{\smash{\SetFigFont{7}{8.4}{\rmdefault}{\mddefault}{\updefault}{\color[rgb]{0,0,0}}}}}
\put(5159,169){\makebox(0,0)[lb]{\smash{\SetFigFont{7}{8.4}{\rmdefault}{\mddefault}{\updefault}{\color[rgb]{0,0,0}}}}}
\put(3504,168){\makebox(0,0)[lb]{\smash{\SetFigFont{7}{8.4}{\rmdefault}{\mddefault}{\updefault}{\color[rgb]{0,0,0}}}}}
\put(3285,168){\makebox(0,0)[lb]{\smash{\SetFigFont{7}{8.4}{\rmdefault}{\mddefault}{\updefault}{\color[rgb]{0,0,0}}}}}
\put(1102,166){\makebox(0,0)[lb]{\smash{\SetFigFont{7}{8.4}{\rmdefault}{\mddefault}{\updefault}{\color[rgb]{0,0,0}}}}}
\put(4885,164){\makebox(0,0)[lb]{\smash{\SetFigFont{7}{8.4}{\rmdefault}{\mddefault}{\updefault}{\color[rgb]{0,0,0}}}}}
\put(168,-2064){\makebox(0,0)[lb]{\smash{\SetFigFont{7}{8.4}{\rmdefault}{\mddefault}{\updefault}{\color[rgb]{0,0,0}Clique}}}}
\put(879,-1439){\makebox(0,0)[lb]{\smash{\SetFigFont{7}{8.4}{\rmdefault}{\mddefault}{\updefault}{\color[rgb]{0,0,0}}}}}
\put(1452,-2519){\makebox(0,0)[lb]{\smash{\SetFigFont{7}{8.4}{\rmdefault}{\mddefault}{\updefault}{\color[rgb]{0,0,0}}}}}
\put(3875,-2514){\makebox(0,0)[lb]{\smash{\SetFigFont{7}{8.4}{\rmdefault}{\mddefault}{\updefault}{\color[rgb]{0,0,0}}}}}
\put(2078,-2679){\makebox(0,0)[lb]{\smash{\SetFigFont{7}{8.4}{\rmdefault}{\mddefault}{\updefault}{\color[rgb]{0,0,0}}}}}
\end{picture}
     \quad}
    \caption{An example of }
    \label{sq-girth4}
  \end{center}
\end{figure}



\begin{lemma}\label{lem:ss-girth4}
If there exists a partition of  into two disjoint subsets  and  such that
each subset in  intersects both  and , then there exists a graph 
with girth four such that .
\end{lemma}

In the above example,  and  is a possible legal partition
of . The corresponding graph  constructed in the proof of Lemma~\ref{lem:ss-girth4} is depicted
in Figure~\ref{root-girth4}.

\begin{lemma}\label{lem:girth4-ss}
If  is a square root of , then there exists a partition of
 into two disjoint subsets  and  such that each subset in  intersects
both  and .
\end{lemma}

Note that in Lemma~\ref{lem:girth4-ss} above we did not require that  has girth four.
Thus, any square root of --particularly, any square root with girth four--will tell us how to
do set splitting. Together with Lemma~\ref{lem:ss-girth4} we conclude:
\begin{theorem}
\textsc{square of graph with girth four} is NP-complete.
\end{theorem}


\begin{figure}[H]
  \begin{center}
    \fbox{\quad
    \begin{picture}(0,0)\includegraphics{root-girth4.pstex}\end{picture}\setlength{\unitlength}{3947sp}\begingroup\makeatletter\ifx\SetFigFont\undefined \gdef\SetFigFont#1#2#3#4#5{\reset@font\fontsize{#1}{#2pt}\fontfamily{#3}\fontseries{#4}\fontshape{#5}\selectfont}\fi\endgroup \begin{picture}(2240,2390)(1271,-1814)
\put(1561,494){\makebox(0,0)[lb]{\smash{\SetFigFont{7}{8.4}{\rmdefault}{\mddefault}{\updefault}{\color[rgb]{0,0,0}}}}}
\put(1561,199){\makebox(0,0)[lb]{\smash{\SetFigFont{7}{8.4}{\rmdefault}{\mddefault}{\updefault}{\color[rgb]{0,0,0}}}}}
\put(2011,494){\makebox(0,0)[lb]{\smash{\SetFigFont{7}{8.4}{\rmdefault}{\mddefault}{\updefault}{\color[rgb]{0,0,0}}}}}
\put(1561,-101){\makebox(0,0)[lb]{\smash{\SetFigFont{7}{8.4}{\rmdefault}{\mddefault}{\updefault}{\color[rgb]{0,0,0}}}}}
\put(1561,-361){\makebox(0,0)[lb]{\smash{\SetFigFont{7}{8.4}{\rmdefault}{\mddefault}{\updefault}{\color[rgb]{0,0,0}}}}}
\put(2011,-361){\makebox(0,0)[lb]{\smash{\SetFigFont{7}{8.4}{\rmdefault}{\mddefault}{\updefault}{\color[rgb]{0,0,0}}}}}
\put(2011,-106){\makebox(0,0)[lb]{\smash{\SetFigFont{7}{8.4}{\rmdefault}{\mddefault}{\updefault}{\color[rgb]{0,0,0}}}}}
\put(2011,199){\makebox(0,0)[lb]{\smash{\SetFigFont{7}{8.4}{\rmdefault}{\mddefault}{\updefault}{\color[rgb]{0,0,0}}}}}
\put(2461,499){\makebox(0,0)[lb]{\smash{\SetFigFont{7}{8.4}{\rmdefault}{\mddefault}{\updefault}{\color[rgb]{0,0,0}}}}}
\put(2456,199){\makebox(0,0)[lb]{\smash{\SetFigFont{7}{8.4}{\rmdefault}{\mddefault}{\updefault}{\color[rgb]{0,0,0}}}}}
\put(2461,-101){\makebox(0,0)[lb]{\smash{\SetFigFont{7}{8.4}{\rmdefault}{\mddefault}{\updefault}{\color[rgb]{0,0,0}}}}}
\put(2461,-361){\makebox(0,0)[lb]{\smash{\SetFigFont{7}{8.4}{\rmdefault}{\mddefault}{\updefault}{\color[rgb]{0,0,0}}}}}
\put(2911,-101){\makebox(0,0)[lb]{\smash{\SetFigFont{7}{8.4}{\rmdefault}{\mddefault}{\updefault}{\color[rgb]{0,0,0}}}}}
\put(2911,199){\makebox(0,0)[lb]{\smash{\SetFigFont{7}{8.4}{\rmdefault}{\mddefault}{\updefault}{\color[rgb]{0,0,0}}}}}
\put(2911,499){\makebox(0,0)[lb]{\smash{\SetFigFont{7}{8.4}{\rmdefault}{\mddefault}{\updefault}{\color[rgb]{0,0,0}}}}}
\put(1271,-821){\makebox(0,0)[lb]{\smash{\SetFigFont{7}{8.4}{\rmdefault}{\mddefault}{\updefault}{\color[rgb]{0,0,0}}}}}
\put(2031,-821){\makebox(0,0)[lb]{\smash{\SetFigFont{7}{8.4}{\rmdefault}{\mddefault}{\updefault}{\color[rgb]{0,0,0}}}}}
\put(2536,-821){\makebox(0,0)[lb]{\smash{\SetFigFont{7}{8.4}{\rmdefault}{\mddefault}{\updefault}{\color[rgb]{0,0,0}}}}}
\put(3051,-821){\makebox(0,0)[lb]{\smash{\SetFigFont{7}{8.4}{\rmdefault}{\mddefault}{\updefault}{\color[rgb]{0,0,0}}}}}
\put(2296,-1296){\makebox(0,0)[lb]{\smash{\SetFigFont{7}{8.4}{\rmdefault}{\mddefault}{\updefault}{\color[rgb]{0,0,0}}}}}
\put(1756,-1777){\makebox(0,0)[lb]{\smash{\SetFigFont{7}{8.4}{\rmdefault}{\mddefault}{\updefault}{\color[rgb]{0,0,0}}}}}
\put(2751,-1780){\makebox(0,0)[lb]{\smash{\SetFigFont{7}{8.4}{\rmdefault}{\mddefault}{\updefault}{\color[rgb]{0,0,0}}}}}
\put(2258,-1780){\makebox(0,0)[lb]{\smash{\SetFigFont{7}{8.4}{\rmdefault}{\mddefault}{\updefault}{\color[rgb]{0,0,0}}}}}
\put(3511,-826){\makebox(0,0)[lb]{\smash{\SetFigFont{7}{8.4}{\rmdefault}{\mddefault}{\updefault}{\color[rgb]{0,0,0}}}}}
\put(2912,-350){\makebox(0,0)[lb]{\smash{\SetFigFont{7}{8.4}{\rmdefault}{\mddefault}{\updefault}{\color[rgb]{0,0,0}}}}}
\put(3261,-1787){\makebox(0,0)[lb]{\smash{\SetFigFont{7}{8.4}{\rmdefault}{\mddefault}{\updefault}{\color[rgb]{0,0,0}}}}}
\end{picture}
     \quad}
    \caption{An example of root  with girth }
    \label{root-girth4}
  \end{center}
\end{figure}

\section{Conclusion and open problems}
We have shown that squares of graphs with girth at least six can be recognized in polynomial time.
We have found a good characterization for squares of graphs with girth at least seven that
gives a faster recognition algorithm in this case.
For squares of graphs with girth at most four we have shown that recognizing the squares of
such graphs is NP-complete.

The complexity status of computing square root with girth (exactly) five is not yet determined.
However, we believe that this problem should be efficiently solvable.
Also, we believe that the algorithm to compute a square root of girth 
can be extended to compute a square root with no  or .
More generally, let  be a positive integer and consider the following problem.

\textsc{-power of graph with girth }\1ex]
\begin{tabular}{l l}
{\em Instance:}& A graph .\\
{\em Question:}& Does there exist a graph  with girth  such that ?\\
\end{tabular}

\begin{conjecture}\label{conj}
\textsc{-power of graph with girth } is polynomially solvable.
\end{conjecture}

The truth of the above conjecture together with the results in this paper would imply a
complete dichotomy theorem: \textsc{squares of graphs of girth } is polynomial if
 and NP-complete otherwise.


\begin{thebibliography}{99}
\bibitem{AgnGreHal}
  Geir Agnarsson, Raymond Greenlaw, Magn\'us M. Halld\'orsson,
  On powers of chordal graphs and their colorings,
  {\sl Congressus Numer.} 144 (2000) 41--65.

\bibitem{AloMoh}
  Noga Alon, Bojan Mohar,
  The chromatic number of graph powers,
  {\sl Combinatorics, Probability and Computing} 11 (2002) 1--10.

\bibitem{BraLeSri2006}
  Andreas Brandst\"adt, Van Bang Le, and R. Sritharan,
  Structure and linear time recognition of -leaf powers,
  {\sl ACM Transactions on Algorithms}, to appear.

\bibitem{ChaKoLu2006}
  Maw-Shang Chang, Ming-Tat Ko, and Hsueh-I Lu,
  Linear time algorithms for tree root problems,
  {\sl Lecture Notes in Computer Science}, 4059 (2006) 411--422.

\bibitem{CraKim}
  Daniel W. Cranston, Seog-Jin Kim,
  List-coloring the square of a subcubic graph,
  {\sl J. Graph Theory} 57 (2007) 65--87.

\bibitem{DahDuc1987}
  Elias Dahlhaus, P.~Duchet,
  On strongly chordal graphs,
  {\sl Ars Combin.} 24 B (1987) 23--30.

\bibitem{EscMonRoj1974}
  F. Escalante, L. Montejano, and T. Rojano,
  Characterization of -path graphs and of graphs having th root,
  {\sl J. Combin. Theory B} 16 (1974) 282--289.

\bibitem{GarJoh}
  Michael R. Garey, David S. Johnson,
  {\sl Computers and Intractability--A Guide to the Theory of NP-Completeness},
  Freeman, New York (1979), twenty-third printing 2002.

\bibitem{Golumbic}
  Martin C. Golumbic,
  {\sl Algorithmic Graph Theory and Perfect Graphs},
  Academic Press, New York (1980).

\bibitem{HarKarTut1967}
  Frank Harary, R.M. Karp, and W.T. Tutte,
  A criterion for planarity of the square of a graph,
  {\sl J. Combin. Theory} 2 (1967) 395--405.

\bibitem{Havet}
  Fr\'ed\'eric Havet,
  Choosability of the square of planar subcubic graphs with large girth,
  {\sl Discrete Math.}, to appear.

\bibitem{HorKil}
  J.D. Horton, K. Kilakos,
  Minimum edge dominating sets,
  {\sl SIAM J. Discrete Math.} 6 (1993) 375--387.

\bibitem{ItaRod}
  Alon Itai, Michael Rodeh,
  Finding a minimum circuit in a graph,
  {\sl SIAM J. Computing} 7 (1978) 413--423.

\bibitem{KeaCor1998}
  Paul E.~Kearney, Derek G.~Corneil,
  Tree powers,
  {\sl J. Algorithms} 29 (1998) 111--131.

\bibitem{Lau2006}
  Lap Chi Lau,
  Bipartite roots of graphs,
  {\sl ACM Transactions on Algorithms} 2 (2006) 178--208.

\bibitem{LauCor2004}
  Lap Chi Lau, Derek G.~Corneil,
  Recognizing powers of proper interval, split and chordal graphs,
  {\sl SIAM J. Discrete Math.} 18 (2004) 83--102.

\bibitem{LinSki1995}
  Yaw.-Ling Lin, Steven S.~Skiena,
  Algorithms for square roots of graphs,
  {\sl SIAM J. Discrete Math.} 8 (1995) 99--118.

\bibitem{Lubiw1982}
  A. Lubiw,
  -free matrices,
  {\sl Master Thesis}, Dept. of Combinatorics and Optimization, University of Waterloo, Canada, 1982.

\bibitem{MotSud1994}
  Rajeev Motwani, Madhu Sudan,
  Computing roots of graphs is hard,
  {\sl Discrete Appl. Math.} 54 (1994) 81-88.

\bibitem{Muk1967}
  A. Mukhopadhyay,
  The square root of a graph,
  {\sl J. Combin. Theory} 2 (1967) 290-295.

\bibitem{Raych1992}
  A. Raychaudhuri,
  On powers of strongly chordal and circular arc graphs,
  {\sl Ars Combin.} 34 (1992) 147--160.

\bibitem{RosHar1960}
  I.C. Ross, Frank Harary,
  The square of a tree,
  {\sl Bell System Tech. J.} 39 (1960) 641--647.

\bibitem{TIAS}
  Shuji Tsukiyama, Mikio Ide, Hiromu Ariyoshi, and Isao Shirakawa,
  A new algorithm for generating all the maximal independent sets,
  {\sl SIAM J. Computing} 6 (1977) 505--517.

\end{thebibliography}

\end{document}
