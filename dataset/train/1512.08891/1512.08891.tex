\documentclass[letterpaper]{sig-alternate-pages}

\newfont{\mycrnotice}{ptmr8t at 7pt}
\newfont{\myconfname}{ptmri8t at 7pt}
\let\crnotice\mycrnotice \let\confname\myconfname 

\permission{}
\conferenceinfo{MSWiM'13,}{November 3--8, 2013, Barcelona, Spain.}
\copyrightetc{doi:\href{http://dx.doi.org/10.1145/2507924.2507943}{10.1145/2507924.2507943}}
\crdata{978-1-4503-2353-6/13/11\ ...\{}_{\!\!\;\MakeUppercase{#2}\!\!\:\shortrightarrow\!\MakeUppercase{#3}}#11\dval{dip}ZX_iZX_iZZZ100(\textit{destination},\textit{sequence number},\textit{validity},\textit{hop count},\textit{next hop})250(D,3,\inval,1,D)3\bullet\!\bullet\!\bullet\!\bullet\!\bullet\!X(\dexp{exp}_1,\ldots,\dexp{exp}_n)P+Q\cond{\varphi}PP\varphi\assignment{\keyw{var}:=\dexp{exp}}PP\broadcastP{\dexp{ms}}.P P\groupcastP{\dexp{dests}}{\dexp{ms}}.P\unicast{\dexp{dest}}{\dexp{ms}}.P \prio Q\dexp{ms}\dexp{dest}PQ\deliver{\dexp{data}}.P\receive{\dexp{msg}}.PP\|Q$		&parallel composition of nodes\\
\hline
\end{tabular}}
\label{tb:awn}
\end{table}







Based on a formal specification one can now perform a careful analysis of the model to see if the
model is consistent and if it satisfies
properties such as loop freedom. Our analysis uses ``classical'' paper-and-pen verification
techniques, and, in this end, 
guarantees the loop freedom of some interpretations of the AODV specification. 
\pagebreak[4]

\subsection{Using Formal Methods to Augment RFCs}
Though we have not shown our proofs in the present paper---the purpose was to show 
that sequence numbers do not a priori guarantee loop freedom and that formal methods are needed---our analysis is based 
on a rigorous, formal and mathematical approach. 
As mentioned before, each interpretation of the RFC has been formalised in an unambiguous way.

We strongly believe that a ``good'' specification should consist of both a formal specification such as the one given in 
\cite{MSWIM12,TR11} \emph{and\/} an English description. 
The English text then serves the purpose of informing the reader about the main behaviour of the protocol and explains design decisions; 
the formal specification gives all the details, without allowing any ambiguities.

The IETF argues for the value of formal methods for specifying, analysing and verifying 
protocols. 
\begin{quote}
``{\sf Formal languages are useful tools for specifying parts of protocols. 
However, as of today, there exists no well-known language that is 
able to capture the full syntax and semantics of reasonably rich IETF protocols.\/}''\mbox{}\hfill [IETF Web page\raisebox{3.8pt}{\footnotescriptsize{\ref{foot}}}]
\end{quote}

\noindent 
The quote is dated 1 Oct 2001; we 
believe that since then formal methods have advanced to such a state that they are now able to
capture the full syntax and the full semantics of protocols and should be used for protocol specification and analysis. 

\section{Related Work}\label{sec:related}
Analysing and verifying routing protocols has a long tradition.
Merlin and Segall \cite{MS79} were amongst the first to use sequence numbers to guarantee loop freedom of a routing protocol.
As we have pointed out, at least two proofs for AODV's loop freedom
have been proposed~\cite{AODV99,ZYZW09}. 
These proof attempts are discussed in Section~\ref{ssec:sqns}.
Besides,
other researchers
have used formal specification and analysis techniques to investigate the correctness of AODV; we point only at some examples.

A preliminary draft of AODV has been shown to be not loop free by Bhargavan et al.\
in~\cite{BOG02}. Since then, AODV has changed to such a degree that
the analysis of \cite{BOG02} has no bearing on the current version \cite{rfc3561}.
Furthermore, their loop had to do with timing issues (as is also the case in \cite{Garcia04, Rangarajan05}), whereas ours is time-independent.
In the same paper they show the use  of model checking on a draft of AODV, demonstrating the feasibility and value of automated verification of routing protocols. Using the model checker SPIN they were able to find/verify the routing loop in the preliminary draft. Musuvathi et al.~\cite{MPCED02} introduced the CMC model checker primarily to search for coding errors in implementations and used AODV as an example. Chiyangwa and Kwiatkowska~\cite{CK05} use the timing features of the model checker {\uppaal} to study the relationship between the timing parameters and the performance of route discovery. 
A ``time-free'' model of AODV is exhaustively analysed by Fehnker et
al.\ in~\cite{TACAS12}, where variants of AODV, which
yield performance improvements, are  proposed and analysed. The presented model is based on
the process algebra AWN introduced by the same authors in~\cite{ESOP12}; the complete model of AODV can be found in~\cite{TR11}.

Next to the analysis of AODV with various formal methods, 
formal languages are also used for the specification and verification of other reasonably rich protocols.
In \cite{GS05}, Griffin and Sobrinho use path algebras and  algebraic routing to 
define a metarouting language that allows the definition of routing protocols. Their main interest lies in the combination 
of different algebras with applications in interdomain routing protocols such as BGP\@. Zave et al.\ provide abstractions for implementing the 
SIP protocol; the major elements of the language are presented in~\cite{ZaveEtAl09}.


\section{Discussion \& Conclusion}\label{sec:conclusion}
We have shown that, in contrast to common belief, sequence numbers do not 
guarantee loop freedom, even if they are increased monotonically over time and 
incremented\linebreak whenever a new route request is generated. This was mainly achieved by 
analysing the Ad hoc On-demand Distance Vector (AODV) routing protocol. 
We have shown that AODV can yield routing loops.

Of course, one could argue that the given loop example does not occur in practice very often, since the 
sequence in which different requests have to be initiated and afterwards handled by different nodes
might be really rare. However, it has been claimed that routing loops are avoided in {\em all possible scenarios} by the use of 
sequence numbers---this has been proven to be incorrect. 
Here, it does {\em not} matter how often this scenario occurs in real life. The fact that the example exists and can occur, makes the protocol flawed and disproves the  fact that monotonically increasing sequence numbers are sufficient to guarantee loop freedom.

Next to that we have analysed
several different interpretations of the AODV RFC\@. 
It turned out that
several interpretations can yield unwanted behaviour such as routing loops. 
We also found that implementations of 
AODV behave differently in crucial aspects of protocol behaviour, although they all follow the lines of the RFC\@. 
This is often caused by ambiguities, contradictions or unspecified behaviour in the RFC\@.
Of course a specification ``{\sf needs to be reasonably implementation independent\/}''\footnote{\label{foot}\url{http://www.ietf.org/iesg/statement/pseudocode-guidelines.html}}
and can leave some decisions to the software engineer; however it is our belief that any specification should be clear and 
unambiguous enough to guarantee the same behaviour when given to different developers. 
As demonstrated, this is not the case for AODV, and likely
not for many other RFCs provided by the IETF.

Our work confirms that RFCs written merely in a natural
language contain ambiguities and contradictions. As a consequence, the
various implementations
depart in various ways from the RFC\@. Moreover, semi-informal
reasoning is inadequate to ensure critical safety-properties like
loop freedom.  We believe that formal specification languages and
analysis techniques---offering rigorous verification and ana\-lysis techniques---are
now able to capture the full syntax and semantics of reasonably rich
IETF protocols. These are an indispensable augmentation to natural
language, both for specifying protocols such as AODV, AODVv2 and HWMP,
and for verifying their essential properties.

\subsubsection*{Acknowledgement.}
NICTA is funded by the Australian Government as represented by the Department of Broadband,
Communications and the Digital Economy and the Australian Research Council through the ICT Centre of
Excellence program.

\begin{thebibliography}{10}

\bibitem{AODVNIST}
{Kernel AODV (ver. 2.2.2), NIST}.
\newblock \url{http://www.antd.nist.gov/wctg/aodv_kernel/} (accessed \today).

\bibitem{AODVUU}
{AODV-UU}: An implementation of the {AODV} routing protocol {(IETF RFC 3561)}.
\newblock \url{http://sourceforge.net/projects/aodvuu/} (accessed \today).

\bibitem{BOG02}
K.~Bhargavan, D.~Obradovic, and C.~Gunter.
\newblock Formal verification of standards for distance vector routing
  protocols.
\newblock {\em J. ACM}, 49(4):538--576, 2002.

\bibitem{CB04}
I.~Chakeres and E.~Belding-Royer.
\newblock {AODV} routing protocol implementation design.
\newblock In {\em 24th International Conference on Distributed Computing
  Systems Work- shops (WWAN'04)}, pages 698--703. IEEE Press, 2004.
\bibitem{CK05}
S.~Chiyangwa and M.~Kwiatkowska.
\newblock A timing analysis of {AODV}.
\newblock In {\em Formal Methods for Open Object-based Distributed Systems
  (FMOODS'05)}, volume 3535 of {\em LNCS}, pages 306--322. Springer, 2005.

\bibitem{TACAS12}
A.~Fehnker, R.~J. van Glabbeek, P.~H{\"o}fner, A.~McIver, M.~Portmann, and
  W.~L. Tan.
\newblock Automated analysis of {AODV} using {UPPAAL}.
\newblock In C.~Flanagan and B.~K{\"onig}, editors, {\em Tools and Algorithms
  for the Construction and Analysis of Systems (TACAS'12)}, volume 7214 of {\em
  LNCS}, pages 173--187. Springer, 2012.

\bibitem{ESOP12}
A.~Fehnker, R.~J. van Glabbeek, P.~H{\"o}fner, A.~McIver, M.~Portmann, and
  W.~L. Tan.
\newblock A process algebra for wireless mesh networks.
\newblock In H.~Seidl, editor, {\em European Symposium on Programming
  (ESOP'12)}, volume 7211 of {\em LNCS}, pages 295--315. Springer, 2012.

\bibitem{TR11}
A.~Fehnker, R.~J. van Glabbeek, P.~H{\"o}fner, A.~McIver, M.~Portmann, and
  W.~L. Tan.
\newblock A process algebra for wireless mesh networks used for modelling,
  verifying and analysing {AODV}.
\newblock Technical Report 5513, NICTA, 2013.
\newblock \url{http://www.nicta.com.au/pub?doc=5513}.

\bibitem{Garcia-Luna-Aceves89}
J.~J. Garcia-Luna-Aceves.
\newblock A unified approach to loop-free routing using distance vectors or
  link states.
\newblock In {\em Symposium Proceedings on Communications, Architectures \&
  Protocols (SIGCOMM '89)}, volume 19(4) of {\em ACM SIGCOMM Computer
  Communication Review}, pages 212--223. ACM Press, 1989.

\bibitem{Garcia04}
J.~J. Garcia-Luna-Aceves and H.~Rangarajan.
\newblock A new framework for loop-free on-demand routing using destination
  sequence numbers.
\newblock In {\em IEEE International Conference on Mobile Ad-hoc and Sensor
  Systems (MASS)}, pages 426--435, 2004.

\bibitem{GS05}
T.~Griffin and J.~Sobrinho.
\newblock Metarouting.
\newblock In {\em Proceedings of the 2005 Conference on Applications,
  Technologies, Architectures, and Protocols for Computer Communications
  (SIGCOMM '05)}, volume 35(4) of {\em ACM SIGCOMM Computer Communication
  Review}, pages 1--12. ACM Press, 2005.

\bibitem{MSWIM12}
P.~H{\"o}fner, R.~J. van Glabbeek, W.~L. Tan, M.~Portmann, A.~McIver, and
  A.~Fehnker.
\newblock A rigorous analysis of {AODV} and its variants.
\newblock In {\em Modeling, Analysis and Simulation of Wireless and Mobile
  Systems (MSWIM'12)}. ACM Press, 2012.

\bibitem{rfc1768}
C.~Huitema, J.~Postel, and S.~Crocker.
\newblock Not all {RFC}s are standards.
\newblock RFC 1768 (Informational), 1995.
 \vfill\eject \bibitem{HWMP}
{IEEE P802.11s}.
\newblock {IEEE} draft standard for information technology---telecommunications
  and information exchange between systems---local and metropolitan area
  networks---specific requirements---part 11: Wireless {LAN} {M}edium {A}ccess
  {C}ontrol {(MAC)} and physical layer {(PHY)} specifications-amendment 10:
  Mesh networking, July 2010.

\bibitem{Kawadia03}
V.~Kawadia, Y.~Zhang, and B.~Gupta.
\newblock System services for ad-hoc routing: Architecture, implementation and
  experiences.
\newblock In {\em Proceedings of the 1st international conference on Mobile
  systems, applications and services}, MobiSys '03, pages 99--112. ACM, 2003.

\bibitem{MS79}
P.~M. Merlin and A.~Segall.
\newblock A failsafe distributed routing protocol.
\newblock {\em IEEE Transactions on Communications}, com-27(9), 1979.

\bibitem{MPCED02}
M.~Musuvathi, D.~Y.~W. Park, A.~Chou, D.~R. Engler, and D.~L. Dill.
\newblock {CMC:} a pragmatic approach to model checking real code.
\newblock In {\em Operating Systems Design and Implementation (OSDI'02)}, 2002.

\bibitem{NipkowPaulsonWenzel02}
T.~Nipkow, L.~C. Paulson, and M.~Wenzel.
\newblock {\em {Isabelle/HOL} --- A Proof Assistant for Higher-Order Logic},
  volume 2283 of {\em LNCS}.
\newblock Springer, 2002.

\bibitem{rfc3561}
C.~Perkins, E.~Belding-Royer, and S.~Das.
\newblock Ad hoc on-demand distance vector {(AODV)} routing.
\newblock RFC 3561 (Experimental), 2003.
\newblock \url{http://www.ietf.org/rfc/rfc3561.txt}.

\bibitem{DSDV94}
C.~Perkins and P.~Bhagwat.
\newblock Highly dynamic destination-sequenced distance-vector routing {(DSDV)}
  for mobile computers.
\newblock In {\em Conference on Communications, Architectures, Protocols \&
  Applications (SIGCOMM '94)}, volume 24(4) of {\em ACM SIGCOMM Computer
  Communication Review}, pages 234--244. ACM Press, 1994.

\bibitem{DYMO25}
C.~Perkins and I.~Chakeres.
\newblock Dynamic {MANET} on-demand {(AODVv2)} routing.
\newblock Internet Draft (Standards Track), 2013.
\newblock \url{http://tools.ietf.org/html/draft-ietf-manet-dymo-26}.

\bibitem{AODV99}
C.~Perkins and E.~Royer.
\newblock {Ad-hoc On-Demand Distance Vector Routing}.
\newblock In {\em 2nd {IEEE} Workshop on Mobile Computing Systems and
  Applications}, pages 90--100, 1999.

\bibitem{Rangarajan05}
H.~Rangarajan and J.~J. Garcia-Luna-Aceves.
\newblock Making on-demand routing protocols based on destination sequence
  numbers robust.
\newblock In {\em IEEE International Conference on Communications (ICC)},
  volume~5, pages 3068--3072, 2005.

\bibitem{Zave12}
P.~Zave.
\newblock Using lightweight modeling to understand {CHORD}.
\newblock {\em SIGCOMM Comput. Commun. Rev.}, 42(2):49--57, 2012.

\bibitem{ZaveEtAl09}
P.~Zave, E.~Cheung, G.~W. Bond, and T.~M. Smith.
\newblock Abstractions for programming {SIP} back-to-back user agents.
\newblock In {\em International Conference on Principles, Systems and
  Applications of IP Telecommunications}, IPTComm '09, pages 11:1--11:12. ACM
  Press, 2009.

\bibitem{ZYZW09}
M.~Zhou, H.~Yang, X.~Zhang, and J.~Wang.
\newblock The proof of {AODV} loop freedom.
\newblock In {\em Wireless Communications \& Signal Processing (WCSP09)}, pages
  1--5. IEEE Press, 2009.

\end{thebibliography}

\label{lastpage}
\end{document}
