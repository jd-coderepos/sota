





\documentclass[conference]{IEEEtran}



















\ifCLASSINFOpdf
\else
\fi















































\hyphenation{op-tical net-works semi-conduc-tor}

\usepackage{amssymb}
\setcounter{tocdepth}{3}
\usepackage{graphicx}
\usepackage{graphics}
\usepackage[cmex10]{amsmath}
\usepackage{amssymb, amsfonts}
\usepackage{subfigure}
\usepackage{slashbox}
\usepackage{caption}
\usepackage[T1]{fontenc}
\usepackage[utf8]{inputenc}
\usepackage{graphics}                 \usepackage{graphicx}
\usepackage[cmex10]{amsmath}
\usepackage{amssymb, amsfonts}
\usepackage{algorithmic}
\usepackage{algorithm}
\usepackage{verbatim}
\usepackage{url}
\usepackage{subfigure}
\usepackage{slashbox}
\usepackage{setspace}
\usepackage{multirow}
\usepackage{array}
\usepackage{hhline}

\graphicspath{{figures/}}

\usepackage{url}

\newcolumntype{P}[1]{>{\centering\arraybackslash}p{#1\textwidth}}

\begin{document}
\title{Tracking Topology Dynamicity for Link Prediction in Intermittently Connected Wireless Networks}


\author{\IEEEauthorblockN{Mohamed-Haykel Zayani, Vincent Gauthier, Ines Slama and Djamal Zeghlache}
\IEEEauthorblockA{Lab. CNRS SAMOVAR UMR 5157\\ Institut Mines-Telecom, Telecom SudParis\\
Evry, France\\\{mohamed-haykel.zayani, vincent.gauthier, ines.slama,
djamal.zeghlache\}@telecom-sudparis.eu} }






\IEEEoverridecommandlockouts
\IEEEpubid{\makebox[\columnwidth]{978-1-4577-1379-8/12/\T(T+1)^{th}\boldsymbol{\mathcal{Z}}p\mathbf{S}a\bf{a}\mathbf{A}r^{th}\mathbf{A}\bf{a_r}\boldsymbol{\mathcal{T}}n^{th}\boldsymbol{\mathcal{T}}\mathbf{T_n}i^{th}\bf{a}\bf{a}(i)(i,j)\mathbf{A}\mathbf{A}(i,j)(i, j, k)\boldsymbol{\mathcal{T}}\mathbf{T_{i}}(j, k)\boldsymbol{\mathcal{Z}}\mathbf{X}\mathbf{S}\mathbf{X}\boldsymbol{\mathcal{Z}}\mathbf{Z_{p}}(i, j)ij[(p-1) \cdot t,p \cdot t[p\mathbf{Z}_{p}(i, j)\mathbf{Z_{1}}\mathbf{Z_{T}}\mathbf{X}\boldsymbol{\mathcal{Z}}\theta\mathbf{S}ij\beta\beta^{\ell}\ellP_{\left \langle \ell  \right \rangle}(i,j)\ellij\mathbf{S}\mathbf{I}\mathbf{X}\theta\betaTTtTtTTLLTttttttttTttttttttTPFPFNTNTTTTTLL\frac{TP+TN}{TP+FP+TN+FN}\frac{TP}{TP+FP}\frac{TP}{TP+FN}2.\frac{precision.recall}{precision+recall}TTTTTTTTTTTT$=8)}
 & Thakur's Metric                & 55.70\% & 99.08\% & 63.51\% & 41.85\% & 0.5045 \\  \cline{2-7}
 & Spatial Cosine Sim.            & 57.57\% & 99.14\% & 72.82\% & 37.22\% & 0.4926 \\  \cline{2-7}
 & Co-Location Rate               & 61.71\% & 99.24\% & 77.89\% & 45.10\% & 0.5712 \\  \cline{2-7}
 \hhline{|~|------}
 \hhline{|~|------}
 \hhline{|~|------}
 \hhline{|~|------}
 & Adamic-Adar Meas.              & 59.13\% & 99.14\% & 57.52\% & 65.48\% & 0.6124 \\  \cline{2-7}
 & Jaccard's Coeff.               & 58.95\% & 99.22\% & 71.27\% & 45.73\% & 0.5571 \\  \cline{2-7}
 & Katz Measure                   & 61.00\% & 99.28\% & 74.90\% & 50.09\% & 0.6003 \\  \cline{2-7}
\hline
\end{tabular}}
\end{table}

The results obtained enable us to attest that the use of the Katz
measure has been one of the best choices to perform prediction
through the tensor-based technique. Using this metric achieves
better performance than those of the other link prediction metrics
proposed in the literature. Hence, the Katz measure is the best
metric that we can use to perform link prediction.

The prediction made through the Katz measure achieves better
performance than those of mobility homophily metrics and Thakur et
al.'s similarity. Indeed, our framework quantifies the similarity of
nodes based on their encounters and geographical closeness. In other
words, the proposed prediction method cares about contacts (or
closenesses) at (around) the same location and at the same time.
Meanwhile, the mobility homophily metrics and Thakur et al.'s
similarity are defined as an association metric. Hence, they measure
the degree of similarity of behaviors of two mobile nodes without
necessarily seeking if they are in the same location at the same
time. Regarding the comparison with the other network proximity
metrics, the Katz measure quantifies better the behavior similarity
of two nodes as it takes into consideration only the paths that
separate them. Meanwhile, the Adamic-Adar metric and the Jaccard's
coefficient are dependent respectively on the degree of common
neighbors between two nodes and the size of the intersection of the
neighbors of two nodes. These latter metrics express similarity
based on common neighbors of two nodes but don't seek if a link is
occurring between them. This criterion highly influences the value
of Katz measure and make it more precise.

\section{Conclusion}
Human mobility patterns are mostly driven by social intentions and
correlations appear in the behavior of people forming the network.
These similarities highly govern the mobility of people and then
directly influence the structure of the network. The knowledge about
the behavior of nodes greatly helps in improving the design of
communication protocols. Intuitively, two nodes that follow the same
social intentions over time promote the occurrence of a link in the
immediate future.

In this paper, we presented a link prediction technique inspired by
data mining and exploit it in the context of human-centered wireless
networks. Through the link prediction evaluation, we have obtained
relevant results that attest the efficiency of our contribution and
agree
with some findings referred in the literature. 
Good link prediction offers the possibility to further improve
opportunistic packet forwarding strategies by making better
decisions in order to enhance the delivery rate or limiting latency.
Therefore, it will be relevant to supply some routing protocols with
prediction information and to assess the contribution of our
approach in enhancing the performance of the network especially as
we propose an efficient distributed version of the prediction
method. The proposed technique also motivates us to inquire into
future enhancements as a more precise tracking of the behavior of
nodes and a more efficient similarity computation.


\bibliographystyle{ieeetran}
\bibliography{Biblio}



\end{document}
