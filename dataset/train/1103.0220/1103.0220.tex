
In Section~\ref{sec:motiv} we reduced the  problem of protocol  insecurity in presence of several intruders 
to solving  a system of deducibility constraints. 
In this section we present a decision procedure for a constraint system where  
Dolev-Yao deduction system is extended by an associative-commutative-idempotent symbol (DY+ACI).
We consider operators for pairing, symmetric and asymmetric encryptions, decryption, signature 
and an  ACI operator that will be used as a set constructor.


As for the proof structure, after  introducing the formal notations, the main steps to show the decidability are as follows:
\begin{enumerate}
 \item We present an algorithm for solving a ground derivability in DY+ACI model.
 \item We prove, that the normalization does not change satisfiability: either we normalize a model or a constraint system.
 \item We show existence of a conservative solution of satisfiable constraint system: a substitution $\sigma$ that sends a variable to an ACI-set of quasi-subterms of the constraint system instantiated with $\sigma$ 
  together with $\oppriv$-ed atoms of the constraint system;
 \item We give a bound on size of a conservative solution, and, as consequence, we obtain decidability.
\end{enumerate}

\subsection{Formal introduction to the problem}

\subsubsection{Terms and notions}\label{subs:def2}



\begin{df}\label{def:term}
\emph{Terms}  are defined according to the following grammar:
\LONG{
 \begin{align*} 
& term & ::= & \, variable \, | \, atom \,|\, \pair{term}{term}  \,|\, \\
& & &  \enc{term}{term} \,|\,  \opaci(tlist) \,|\,  \priv{Keys}  \,|\, \\
& &  &  \aenc{term}{Keys} \,|\, \sig{term}{\priv{Keys}} \\
& Keys  & ::= & \, variable \, | \, atom \\
& tlist & ::= & \,  \ term \, | \, term,\, tlist \, 
\end{align*}
}\SHORT{
 \begin{align*} 
\SHORT{&} term \SHORT{&} ::= \SHORT{&} \, variable \, | \, atom \,|\, \pair{term}{term}  \,|\, \SHORT{\\}
\SHORT{&} \SHORT{&} \SHORT{&}  \enc{term}{term} \,|\,  \opaci(tlist) \,|\,  \priv{Keys}  \,|\, \\
\SHORT{&} \SHORT{&}  \SHORT{&}  \aenc{term}{Keys} \,|\, \sig{term}{\priv{Keys}} \\
\SHORT{&} Keys  \SHORT{&} ::= \SHORT{&} \, variable \, | \, atom \\
\SHORT{&} tlist \SHORT{&} ::= \SHORT{&} \, \{ \ term \ \llbracket, \  term\rrbracket* \, \}
\end{align*}
}where
$atom \in \UniAt{A}$ and
$variable \in \UniVar{X}$.
 We denote $\Universe(\UniAt{A},\UniVar{X})$ the set of all terms over 
a set of atoms $\UniAt{A}$ and a set of variables $\UniVar{X}$. 
For short, we write $\Universe$ instead of $\Universe(\UniAt{A},\UniVar{X})$.

\end{df}

By $\sig{p}{\priv{a}}$ we mean a signature of message $p$ with private key $\priv{a}$ 
We do not assume that one can retrieve the  message itself from the signature.

Note that we do allow complex keys for symmetric encryption only.
As a consequence, we have to introduce a condition on substitution applications: 
substitution $\sigma$ cannot be applied to the term $t$, if after replacing the resulting entity is not a term 
(for example, we cannot apply $\sigma=\set{x\mapsto \pair{a}{b}}$ to the term $\aenc{a}{x}$).




We denote a term on $i$-th position of a list $L$ as $L[i]$. 
Then $t\in L$ is a shortcut for $\exists i: t = L[i]$. 
We also define two binary relations $\subseteq$ and $\approx$ on lists as follows:
$L_1\subseteq L_2$ if and only if  any $t\in L_1$ implies  $t\in L_2$;
$L_1 \approx L_2$ if and only if $L_1 \subseteq L_2$ and $L_2 \subseteq L_1$,
and naturally extend them if $L_1$ or $L_2$ is a set. 

\begin{df}
We consider symbol $\opaci$ to be 
	associative,
	commutative,
	idempotent
(shortly, $ACI$).
\end{df}

We will use $\opbin$ throughout the paper as a generalization of all binary operators: $\opbin\in\set{\openc,\opaenc,\oppair,\opsig}$.

\begin{df}

 For every term $t\in \Universe$ we define its root symbol by 
\[
 \rut{t} = \left\{ 
 \begin{array}{rl} 
 \opbin, & \mbox{ if } t= \bin{p}{q} \\\opaci, &\mbox{ if } t= \aci{L},\\
 \oppriv, &\mbox{ if } t= \priv{p},\\
 t, &\mbox{ if } t \in \UniVar{X}\cup \UniAt{A},\\
\end{array}
\right. 
\]

\end{df}






\begin{df}\label{def:elems}
	For any term $t\in\Universe$ we define its \emph{set of elements} by:\[
\elems{t} =
 \begin{cases}
 	\bigcup_{p\in L} \elems{p}\, & \mbox{if }  t=\aci{L};\\
  	\set{t} , & \mbox{otherwise. }\\
 \end{cases}
\]
We extend $\elems{}$ to sets of terms or lists of terms $T$ by $\elems{T} = \bigcup_{t\in T} \elems{t}$. 
\end{df}
	
	
\begin{example}\label{ex:term}
 For term $t=\aci{\lst{a, \aci{\lst{b,a,\pair{a}{b}}},\pair{\aci{\lst{b,b}}}{a}}}$
set of its elements is $\elems{t}=\set{a,b,\pair{\aci{\lst{b,b}}}{a},\pair{a}{b}}$.
\end{example}





\begin{df}
Let $\prec$ be a strict total order 
on $\Universe$,
such that comparing can be done in polynomial time. 

\end{df}

\begin{df}
The cardinality of a  set  $P$ is denoted by $\card{P}$.
\end{df}


\begin{df} \label{def:norm}
The \emph{normal form} of a term $t$ (denoted by $\norm{t}$) is recursively defined by:
\begin{itemize}
	\item $\norm{t}=t$, if $t\in \UniVar{X}\cup\UniAt{A}$
	\item $\norm{\bin{t_1}{t_2}} = \bin{\norm{t_1}}{\norm{t_2}} $\item $\norm{\priv{t}}=\priv{\norm{t}}$
	\item 
\SHORT{
	$\norm{\aci{L}}= 
	     \begin{cases}
		\aci{L'},& \mbox{if } \card{\norm{\elems{{L}}}}>1 \SHORT{\\ &} \mbox{ and }  L'\approx  \norm{\elems{{L}}}\\
		  & \mbox{ and for all } i<j,\  L'[i]\prec L'[j];\\
		t',& \mbox{if } \norm{\elems{{L}}}=\set{t'}
	      
	     \end{cases},
	$
}\LONG{
	$\norm{\aci{L}}= 
	     \begin{cases}
		\aci{L'},& \mbox{if } \card{\norm{\elems{{L}}}}>1  \mbox{ and }  L'\approx  \norm{\elems{{L}}}\\
		  & \mbox{ and for all } i<j,\  L'[i]\prec L'[j];\\
		t',& \mbox{if } \norm{\elems{{L}}}=\set{t'}
	     \end{cases},
	$
}

\end{itemize}
where for set of terms $T$, $\norm{T} = \set{\norm{t}:t\in T}$.
	
\end{df}

We can show easily that two terms  are congruent modulo the ACI properties of $''.''$ iff they have the same normal form. 
Other properties are stated in Lemma~\ref{lemma:normprop}. 


\begin{example}
 Referring to Example~\ref{ex:term} for the value of term $t$, we have
$\norm{t}=\aci{\set{a,b,\pair{a}{b},\pair{b}{a}}}$.
\end{example}






\begin{df}
 Let $t$ be a term. We define a set of quasi-subterms $\sub{t}$ as follows:
\[
\sub{t} =
 \begin{cases}
 	\{t\}, & \mbox{if } t \in \UniVar{X} \cup \UniAt{A};\\
  	\{t\} \cup \sub{t_1}, & \mbox{if } t = \priv{t_1};\\
 	\{t\} \cup \sub{t_1} \cup \sub{t_2}, & \mbox{if } t = \bin{t_1}{t_2} \\\{t\} \cup \bigcup_{p\in\elems{L}}\sub{p}, & \mbox{if } t = \aci{L}\\	
 \end{cases}
\]
If $T$ --- set of terms, then $\sub{T} = \bigcup_{t \in T}{\sub{t}}$. 
If $\ConstrSys{S}=\set{E_i\rhd t_i}_{i=1,\dots, n}$ is a constraint system, we define $\sub{\ConstrSys{S}} = \bigcup_{t \in \bigcup_{i=1}^n E_i\cup\set{t_i}}{\sub{t}}$.
\end{df}

\begin{example}
 Referring to Example~\ref{ex:term}, we have
\begin{align*}
\sub{t}=\{
\aci{\lst{a, \aci{\lst{b,a,\pair{a}{b}}},\pair{\aci{\lst{b,b}}}{a}}}, \\
a,b,\pair{a}{b},\pair{\aci{\lst{b,b}}}{a}, \aci{\lst{b,b}}
\}.
\end{align*}
\end{example}




\begin{df}\label{df:vars}
 Let $t$ be a term. We define $\vars{t}$ as set of all the variables in $t$:
\[
\vars{t} = \UniVar{X} \cap \subII{t}
\]
\end{df}










We define $\osubtermsII(t)$ as the set of subterms of $t$ 
and the  DAG-size of a term, as the number of its different subterms. 
The DAG-size gives the size of a natural representation of a term in the 
considered ACI theory. 

\begin{df}\label{df:subdag}
 Let $t$ be a term. We define $\subII{t}$ as follows:
\[
\subII{t} =
 \begin{cases}
    \{t\}, & \mbox{if } t \in \UniVar{X} \cup \UniAt{A};\\
    \{t\} \cup \subII{t_1}, & \mbox{if } t = \priv{t_1};\\
    \{t\} \cup \subII{t_1} \cup \subII{t_2}, & \mbox{if } t = \bin{t_1}{t_2}\\
\{t\} \cup \bigcup_{p\in L}\subII{p}, & \mbox{if } t = \aci{L}.
 \end{cases}
\]
If $T$ is a set of terms, then $\subII{T} = \bigcup_{t \in T}{\subII{t}}$.
If $\ConstrSys{S}=\set{E_i\rhd t_i}_{i=1,\dots, n}$ is a constraint system, we define $\subII{\ConstrSys{S}} = \bigcup_{t \in \bigcup_{i=1}^n E_i\cup\set{t_i}}{\subII{t}}$.
\end{df}

\begin{example}
 Referring to Example~\ref{ex:term}, we have
\begin{align*}
\subII{t}=\{
\aci{\lst{a, \aci{\lst{b,a,\pair{a}{b}}},\pair{\aci{\lst{b,b}}}{a}}}, \\
\aci{\lst{b,a,\pair{a}{b}}}, \pair{\aci{\lst{b,b}}}{a}, \LONG{\\}
a,b,\pair{a}{b}, \aci{\lst{b,b}}
\}.
\end{align*}
\end{example}



\begin{df}\label{df:sizedag}
    We define a DAG-size $\oDAGsize$ of a term $t$ as
\( \DAGsize{t} = \card{\subII{t}}\),
for set of terms $T$,
\( \DAGsize{T} = \card{\subII{T}}\)
and for constraint system $\ConstrSys{S}$ as
\( \DAGsize{\ConstrSys{S}} = \card{\subII{\ConstrSys{S}}}\).
\end{df}
Remark, that for a constraint system such a definition does not polynomially approximate a number of bits needed to write it down\LONG{ (cf. Def.~\ref{def:DAGSys})}.








We define a Dolev-Yao deduction system modulo ACI equational theory  (denoted DY+ACI). 
It consists of composition rules and decomposition rules, depicted in Table~\ref{tab:DYACI} 
where $t_1, t_2, \dots, t_m \in \Universe$.

\begin{table}[ht]
\centering
\begin{tabular}{|l|l|}
\hline
Composition rules & Decomposition rules \\
\hline
 $ {t_1, t_2} \rightarrow \norm{\enc{t_1}{t_2}}$ & ${\enc{t_1}{t_2},  \norm{t_2}} \rightarrow \norm{t_1}$ \\
$ {t_1, t_2} \rightarrow \norm{\aenc{t_1}{t_2}}$ &  ${\aenc{t_1}{t_2},  \norm{\priv{t_2}}} \rightarrow \norm{t_1}$\\
$ {t_1, t_2} \rightarrow \norm{\pair{t_1}{t_2}}$ & $ {\pair{t_1}{t_2}} \rightarrow \norm{t_1}$\\
$ {t_1, \priv{t_2}} \rightarrow \norm{\sig{t_1}{\priv{t_2}}}$ & ${\pair{t_1}{t_2}} \rightarrow \norm{t_2}$\\ 
$ {t_1, \dots, t_m} \rightarrow \norm{\aci{{t_1, \dots, t_m}}}$ & $\  \aci{{t_1, \dots, t_m}} \rightarrow \norm{t_i}$ for all $i$ \\
\hline
\end{tabular}
\caption{DY+ACI deduction system rules}\label{tab:DYACI}
\end{table}






We suppose, hereinafter,  that for a constraint system $\ConstrSys{S}$,  $\sub{\ConstrSys{S}} \cap \UniAt{A} \neq \emptyset$. Otherwise, we can add one constraint $\set{a} \rhd a$ to $\ConstrSys{S}$ which will be satisfied by any substitution.
We denote $\set{\priv{t}:t\in T}$ for set of terms $T$ as $\priv{T}$.
We define $\vars{\ConstrSys{S}} = \bigcup_{i=1}^n \vars{E_i} \cup \vars{t_i}$.
We say that $\ConstrSys{S}$ is normalized, iff for all $t\in \sub{\ConstrSys{S}}$, $t$ is normalized. 

\begin{example}\label{ex:constrsys}
 We give a sample of general constraint system and its solution within DY+ACI deduction system.
\[
   \ConstrSys{S}=\left\{ 
 \begin{array}{l l l}
	    { \enc{x}{a},\pair{c}{a}} & \rhd &  b \\
	    { \aci{\lst{x,c}}} & \rhd & a 
 \end{array}
  \right\},
\]  
where $a,b,c \in \UniAt{A}$ and $x \in \UniVar{X}$.
 One of the eventual models within DY+ACI is $\sigma = \set{x\mapsto \enc{\pair{a}{b}}{c}}$.
\end{example}


\begin{df}\label{def:pairing}
 Let $T=\set{t_1,\dots,t_k}$ be a non-empty set of terms. Then we define $\pairing{T}$ as follows:
 \[	
  \pairing{T} = \norm{\aci{{t_1, \dots{}, t_k}}}
 \]
Remark: $\pairing{\set{t}}=\norm{t}$.
 \end{df}

\begin{df}
We denote $\sub{\ConstrSys{S}}\setminus \UniVar{X}$ as $\subo{S,\UniVar{X}}$ or, for shorter notation, $\subo{S}$.
\end{df}


We introduce a transformation $\pairing{\ahreal{\cdot}}$ on ground terms that replaces recursively all binary root symbols such that they are different from
all the non-variable quasi-subterms of the constraint system instantiated with its model $\sigma$,  with ACI symbol $\cdot$. 
Later, we will show, that $\pahs$ is also a model of $\ConstrSys{S}$.

\begin{df}\label{def:ah}

 Let us have a constraint system $\ConstrSys{S}$ which is satisfiable with model $\sigma$.
 Let us fix some $\alpha \in ( \UniAt{A} \cap \sub{\ConstrSys{S}})$.
 For given $\ConstrSys{S}$ and $\sigma$ we define a function $\ahreal{\cdot}: \UniverseG \rightarrow \bool{\UniverseG}$  as follows:


\[
 \ahreal{t} = \left\{ 
 \begin{array}{rl} 
   \set{\alpha}, & \mbox{if } {t}  \in (\UniAt{A} \setminus \sub{\ConstrSys{S}});\\
   \set{a}, & \mbox{if } {t} = a \in (\UniAt{A}\cap \sub{\ConstrSys{S}});\\
   
	\set{\priv{\pairing{\ahreal{t_1}}}} , & \mbox{if } t = \priv{t_1};\\
    

   \set{\bin{\pairing{\ahreal{t_1}}}{\pairing{\ahreal{t_2}}}} , & \mbox{if } {t} = \bin{t_1}{t_2}\\
&  \norm{t} \in \norm{\subo{S}\sigma}\\

   \ahreal{t_1} \cup\ahreal{t_2} , & \mbox{if } {t} = \bin{t_1}{t_2}   \\
				&  \wedge\  \norm{t} \notin \norm{\subo{S}\sigma} \\
\bigcup_{p\in L}\ahreal{p},  & \mbox{if } {t}=\aci{L}.\\
  
 \end{array}
 \right. 
\]

\end{df}

Henceforward, we will omit parameters and write $\ah{\cdot}$ instead of $\ahreal{\cdot}$ for shorter notation.


\begin{df}
 We define the superposition of $\pairing{\cdot}$ and $\ah{\cdot}$ on a set of terms $T=\set{t_1,\dots,t_k}$ as follows: $\pah{T} = \set{\pah{t}\,|\ t\in T}$.
\end{df}

\begin{df}
 Let $\theta =\set{x_1 \mapsto t_1, \dots, x_k \mapsto t_k }$  be a substitution. We  define $\pairing{\ah{\theta}}$ the  substitution $\set{x_1 \mapsto \pairing{\ah{t_1}}, \dots, x_k \mapsto \pairing{\ah{t_k}}}$.

\end{df}
Note, that $\dom{\pairing{\ah{\theta}}} = \dom{\theta}$.

\begin{example}\label{ex:constrsysmodel}
 We refer to Example~\ref{ex:constrsys} and show, that $\pah{\sigma}$ is also a model of $\ConstrSys{S}$.
$\pah{\enc{\pair{a}{b}}{c}} = \pairing{\ah{\pair{a}{b}}\cup \set{c}} = \pairing{\set{a}\cup\set{b} \cup \set{c}} = \aci{\lst{a,b,c}}$ (we suppose that $a\prec b \prec c$).
One can see, that $\pahs=\set{x\mapsto \aci{\lst{a,b,c}}}$ is also a model of $\ConstrSys{S}$ within DY+ACI.
\end{example}



\subsubsection{General properties used in proof}

The two following lemmas state simple properties of derivability.

\begin{lemma}\label{lemma:dertrans}
 Let $A,B,C \subseteq \UniverseG$. Then if $A\subseteq \der{B}$ and $B\subseteq \der{C}$ then $A\subseteq \der{C}$.

\end{lemma}

\begin{lemma}\label{lemma:derext}
 Let $A,B,C,D \subseteq \UniverseG$. Then if $A\subseteq \der{B}$ and $C\subseteq \der{D}$ then $A\cup C \subseteq \der{B\cup D}$.

\end{lemma}

In Lemma~\ref{lemma:normprop} we list some auxiliary properties that will be used in main proof.



\begin{lemma}\label{lemma:normprop}
	The following statements are true:
	\begin{enumerate}
\item	\label{pACI}         For terms $t, t_1,t_2$, we have $\norm{\aci{t, t}} = \norm{t}$, $\norm{\aci{t_1, t_2}}=\norm{\aci{t_2, t_1}}$, $\norm{\aci{\aci{t_1, t_2},  t_3}} = \norm{\aci{t_1, \aci{t_2, t_3}}} = \norm{\aci{t_1, t_2, t_3}}$
\item 	\label{pNormNorm}      if $t$ and $t\sigma$ are terms, then  $\norm{t\sigma}=\norm{\norm{t\sigma}}=\norm{\norm{t}\sigma}=\norm{t\norm{\sigma}}=\norm{\norm{t}\norm{\sigma}}$

		\item	\label{pSubNorm}     $ s\in\sub{\norm{t}} \implies s=\norm{s}$	



		\item	\label{pSubNormHasProimage}           $ \forall s\in\subII{\norm{t}} \exists s' \in \subII{t} \,:\, s=\norm{s'}$	



		\item \label{pNormElems}       $\norm{\elems{t}} = \elems{\norm{t}}$ 
		
		\item \label{pNormDotNorm}        $\norm{\aci{\norm{t_1}, \dots{}, \norm{t_m}}}=\norm{\aci{t_1, \dots, t_m}}$; $\pairing{T} = \pairing{\norm{T}}$



		\item \label{pDotNormElems}       $\elems{\norm{\aci{\norm{t_1}, \dots, \norm{t_m}}}}={\elems{\aci{\norm{t_1}, \dots, \norm{t_m}}}} =$  \\ $\bigcup_{i=1,\dots,m}\elems{\norm{t_i}}$ 


		\item \label{pAhList}$\ah{t} = \bigcup_{p\in\elems{t}}\ah{p}$, \item \label{pAhNorm}                     ${\ah{t}} = \ah{\norm{t}}$

		\item \label{pPahNorm}                $\pah{t}=\pah{\norm{t}}=\norm{\pah{t}}=\norm{\pah{\norm{t}}}$
		\item \label{pPairingOfPairing}          $\pairing{T_1\cup\dots\cup T_m} = \pairing{\set{\pairing{T_1},\dots,\pairing{T_m}}}$
\item \label{pSubSubSub}            $\sub{\sub{t}} = \sub{t}$           
		\item \label{pNormSub}	$\sub{\norm{t}}\subseteq \norm{\sub{t}}$
 		\item	\label{pSepVar}$\sub{t\sigma}\subseteq \sub{t}\sigma\cup\sub{\vars{t}\sigma}$
		\item	\label{pSepVarII}          $\subII{t\sigma} = \subII{t}\sigma\cup\subII{\vars{t}\sigma}$

		\item	\label{pNormCard}       $\card{\norm{T}}\leq \card{T}$, $\card{T\sigma}\leq \card{T}$ \item	\label{pSizesComparision}         $\elems{t}\subseteq \sub{t} \subseteq \subII{t}$
\item	\label{pNormSize}              For term $t$, $\DAGsize{\norm{t}}\leq \DAGsize{t}$; \\ for set of terms $T$, $\DAGsize{\norm{T}}\leq \DAGsize{T}$; \\
			for constraint system $\ConstrSys{S}$, $\DAGsize{\norm{\ConstrSys{S}}}\leq \DAGsize{\ConstrSys{S}}$

		\item   \label{pSubACI}         $\sub{\aci{\lst{t_1, \dots, t_l}}}\subseteq \set{\aci{\lst{t_1, \dots,t_l}}}\cup \sub{t_1}\cdots\cup \sub{t_l}$
\item	\label{pNormSubSize} $\forall s\in\subII{t} \ \DAGsize{\norm{t\sigma}} \geq \DAGsize{\norm{s\sigma}}$.
\end{enumerate}
\LONG{
\begin{proof}
We will give proofs of several statements. 
Some other technical proofs are given in \ref{app:lemma|normprop} \hfil \par
 \begin{description}
  \item [Statement~\ref{pNormElems}:]
      This statement is trivial, if $t\neq \aci{L}$.
      Otherwise, let $t=\aci{t_1,\dots,t_n}$. 
      \begin{itemize}
	\item if $\norm{\elems{t}}=\set{p}$, where $p\neq \aci{L_p}$. Then $\norm{t}=p$ and then $\elems{\norm{t}}=\elems{p} = \set{p} = \norm{\elems{t}}$.
	\item if $\norm{\elems{t}}=\set{p_1,\dots,p_k}$, $k>1$, where $p_i\neq \aci{L_i}$ for all $i$. Then $\norm{t} = \aci{L}$, where $L \approx \set{p_1,\dots,p_k}$. 
	      That means, that $\elems{\norm{t}} = \bigcup_{p\in\set{p_1,\dots,p_k}}\elems{p} = \set{p_1,\dots,p_k}$.
      \end{itemize}
  \item [Statement~\ref{pNormDotNorm}:]
      The first part follows from the definition of normal form and Statement~\ref{pNormElems}. The second one directly follows from the first.
  \item [Statement~\ref{pAhNorm}:]    
	By induction on $\DAGsize{t}$:
	  \begin{itemize}
	  \item $\DAGsize{t}=1$ is possible in the only case: $t = a \in\UniAt{A}$ and as $a = \norm{a}$, the equality is trivial.
	  \item Suppose, that for any $t: \DAGsize{t} < k$ ($k>1$), ${\ah{t}}=\ah{\norm{t}}$ holds.
	  \item Given a term  $t: \DAGsize{t}=k$, $k>1$. We need to prove that ${\ah{t}}=\ah{\norm{t}}$.
		  \begin{itemize}
		  \item if $t=\priv{t_1}$, then ${\ah{t}}=\set{\priv{\pah{t_1}}}=$ (by induction supposition) 
		  $=\set{\priv{\pah{\norm{t_1}}}}=\ah{\priv{\norm{t_1}}}= \ah{\norm{t}}$.
		  \item if $t=\bin{p}{q}$ and $\norm{t}\in \norm{\subo{\ConstrSys{S}}\sigma}$.
			Then $\ah{\norm{t}} = \ah{\bin{\norm{p}}{\norm{q}}}=\set{\bin{\pah{\norm{p}}}{\pah{\norm{q}}}} = $
			(by induction supposition) $=\set{\bin{\pairing{\norm{\ah{p}}}}{\pairing{\norm{\ah{q}}}}}=$ (by Statement~\ref{pNormDotNorm}) 
			$=\set{\bin{\pah{p}}{\pah{q}}}={\ah{\bin{p}{q}}}$.
		  \item if $t=\bin{p}{q}$ and $\norm{t}\notin \norm{\subo{\ConstrSys{S}}\sigma}$.
			Then ${\ah{t}} = {\ah{p}}\cup{\ah{q}} = $ (by induction) $=\ah{\norm{p}}\cup\ah{\norm{q}} = $
			(as $\norm{\bin{\norm{p}}{\norm{q}}}=\norm{t}\notin \norm{\subo{\ConstrSys{S}}\sigma}$) \\ $=\ah{\bin{\norm{p}}{\norm{q}}}=\ah{\norm{t}}$ 
		  \item if $t=\aci{L}$, where $L=\lst{t_1,\dots,t_m}$.
			Note first, that as $t=\aci{L}$, we have for all$s \in \elems{t}$, $\DAGsize{s} < \DAGsize{t}$.
			Then, by Statement~\ref{pAhList}, $\ah{t} = \bigcup_{p\in\elems{t}}\ah{p} = $ (by induction supposition) $=\bigcup_{p\in\elems{t}}\ah{\norm{p}}$.
			On the other part, $\ah{\norm{t}} = \bigcup_{p\in\elems{\norm{t}}}\ah{p} =$  (by Statement~\ref{pNormElems}) $=\bigcup_{p\in\norm{\elems{t}}}\ah{p} =
			\bigcup_{p\in\elems{t}}\ah{\norm{p}}=\ah{t}$.
\end{itemize}

	  \end{itemize} 
  \item [Statement~\ref{pPairingOfPairing}:] 
	From definition of  $\opairing$ and Statement~\ref{pNormElems}, we obtain that \\ $\elems{\pairing{T_i}} = \norm{\elems{T_i}}$. 
	Next $\pairing{\set{\pairing{T_1},\dots,\pairing{T_m}}}=\norm{\aci{L}}$ (here we use $\norm{\aci{L}}$ to capture two cases from definition of normalization at once), where $L\approx \norm{\elems{\set{\pairing{T_1},\dots,\pairing{T_m}}}} = $ \\
	$\norm{\bigcup_{i=1,\dots,m}\norm{\elems{T_i}}} =\norm{\bigcup_{i=1,\dots,m}{\elems{T_i}}}$, \\
	while  $\pairing{T_1\cup\dots\cup T_m} = \norm{\aci{L'}}$, where $L' \approx \norm{\bigcup_{i=1,\dots,m}\elems{T_i}}$. 
  \item [Statement~\ref{pNormSub}:]
      By induction on $\DAGsize{t}$.
      \begin{itemize}
	\item $\DAGsize{t}=1$. 
	      Then $t\in\UniAt{A}\cup  \UniVar{X}$. As $\sub{t}=\set{t}$ and $t=\norm{t}$, the statement holds.
	\item Suppose, that for any $t: \DAGsize{t} < k$ ($k>1$), the statement is true.
	\item Given a term  $t: \DAGsize{t}=k$, $k>1$. Let us consider all possible cases:
	      \begin{itemize}
		\item $t=\bin{t_1}{t_2}$.On the one hand, $\sub{t}=\set{t}\cup\sub{t_1}\cup\sub{t_2}$.  
		      On the other hand, $\norm{t}=\bin{\norm{t_1}}{\norm{t_2}}$ and then,
		      $\sub{\norm{t}} = \set{\norm{t}}\cup\sub{\norm{t_1}}\cup\sub{\norm{t_2}}$.
		      Then, as $\sub{\norm{t_i}}\subseteq \norm{\sub{t_i}}$, we have that 
		      $\sub{\norm{t}} \subseteq \norm{\sub{t}}$.
		\item $t=\priv{t_1}$. Proof is similar to one for the case above.
		\item $t=\aci{L}$. We have $\sub{t} = \set{t}\cup\bigcup_{p\in\elems{{L}}}\sub{p}$.
		      From Statement~\ref{pNormElems} we have $\elems{\norm{\aci{L}}}  = \norm{\elems{\aci{L}}}$,
		      and then, $\sub{\norm{\aci{L}}} = $ \br$\set{\norm{\aci{L}}} \cup \bigcup_{p\in\elems{\norm{\aci{L}}}}\sub{p} = $ \br$
		      \norm{\set{\aci{L}}} \cup \bigcup_{p\in{\elems{\aci{L}}}}\sub{\norm{p}} \subseteq$ (by supposition) \\
		      $\subseteq \norm{\set{\aci{L}}} \cup \bigcup_{p\in{\elems{\aci{L}}}}\norm{\sub{p}} = $ \br$
		      \norm{{\set{\aci{L}}} \cup \bigcup_{p\in{\elems{\aci{L}}}}{\sub{p}}} = \norm{\sub{t}}$.
	      \end{itemize} 
      \end{itemize}      
  \item [Statement~\ref{pSepVar}:]
      By induction on $\DAGsize{t}$
      \begin{itemize}
	\item  $\DAGsize{t}=1$. 
	      \begin{itemize}
	      \item $t\in\as$. As $t\sigma = t$ and $\vars{t} = \emptyset$, the statement becomes trivial.
	      \item $t\in \UniVar{X}$. Then $\sub{t}\sigma = t\sigma$, $\vars{t} = \set{t}$; 
		    We have $\sub{t\sigma} \subseteq \set{t\sigma}\cup\sub{t\sigma}$.
	      \end{itemize}
	\item Suppose, that for any $t: \DAGsize{t} < k$ ($k\geq 1$), the statement is true.
	\item Given a term  $t: \DAGsize{t}=k$, $k>1$. Let us consider all possible cases:
	      \begin{itemize}
		\item $t=\bin{t_1}{t_2}$. Then 
		      $t\sigma =  \bin{t_1\sigma}{t_2\sigma}$ and $\vars{t} = \vars{t_1}\cup\vars{t_2}$.
		      $\sub{t\sigma} = \set{t\sigma} \cup \sub{t_1\sigma} \cup $ \br $ \sub{t_2\sigma} \subseteq $ (as $\DAGsize{t_i} < k$)
		      $\subseteq \set{t\sigma} \cup \sub{t_1}\sigma\cup\sub{\vars{t_1}\sigma} \cup \sub{t_2}\sigma\cup\sub{\vars{t_2}\sigma}
		      = \set{t\sigma} \cup \sub{t_1}\sigma  \cup \sub{t_2}\sigma \cup \sub{(\vars{t_1}\cup \vars{t_2})\sigma}
		      = \sub{t}\sigma \cup \sub{\vars{t}\sigma}$.
		\item $t=\priv{t_1}$. Proof is similar to one for the case above.
		\item $t=\aci{\lst{t_1,\dots,t_m}}$. We have $t\sigma =  \aci{\lst{t_1\sigma,\dots,t_m\sigma}}$ and $\vars{t} = \bigcup_{i=1,\dots,m}\vars{t_i}$.
		      Then we have $\sub{t\sigma} = \set{t\sigma} \cup \bigcup_{p\in\elems{\lst{t_1\sigma,\dots,t_m\sigma}}}\sub{p} \subseteq $ (using Statement~\ref{pSizesComparision})
		      $\subseteq \set{t\sigma} \cup \bigcup_{p\in\bigcup_{i=1}^m\sub{t_i\sigma}}\sub{p} = $
		      (as $\sub{\sub{p}}=\sub{p}$)
		      $= \set{t\sigma} \cup \bigcup_{i=1,\dots,m}\sub{t_i\sigma} \subseteq $
		      (as $\DAGsize{t_i} < k$) \\
		      $\subseteq  \set{t\sigma} \cup \bigcup_{i=1,\dots,m}\left(\sub{t_i}\sigma\cup\sub{\vars{t_i}\sigma} \right )
		      = \set{t\sigma} \cup \bigcup_{i=1,\dots,m}\sub{t_i}\sigma  \cup \sub{\left( \bigcup_{i=1,\dots,m}\vars{t_i}\right)\sigma}  $ \br $ 
		      = \sub{t}\sigma \cup \subII{\vars{t}\sigma}$. 
	      \end{itemize}
      \end{itemize}
 \end{description}
\end{proof}
} \end{lemma}

\begin{lemma}\label{lemma:normalsigma}
Given a  constraint system $\ConstrSys{S}$ and its model $\sigma$. Then 
substitution $\pahs$ is normalized 
\LONG{
\begin{proof}
 For any $x\in\dom{\pahs}$, $x\pahs = \pah{x\sigma} = \norm{\pah{x\sigma}}$ (by Lemma~\ref{lemma:normprop}).
\end{proof}
}\end{lemma}



\begin{lemma}\label{lemma:DAGvsQuasi}
 For any normalized term $t$, $\sub{t} = \subII{t}$.
\LONG{
\begin{proof} By induction on $\DAGsize{t}$.
 \begin{itemize}
  \item  $\DAGsize{t}=1$. Then  $t \in\UniVar{X} \cup \UniAt{A}$, and thus, $\sub{t} = \subII{t} = \set{t}$.
  \item Suppose, that for any $t: \DAGsize{t} < k$ ($k>1$), $\sub{t} = \subII{t}$.
  \item Given a term  $t: \DAGsize{t}=k$, $k>1$. We need to show that $\sub{t} = \subII{t}$.
	\begin{itemize}
	  \item $t=\bin{t_1}{t_2}$. Then 
		$\sub{\bin{t_1}{t_2}} = \set{t} \cup \sub{t_1} \cup \sub{t_2} = $ (as $\DAGsize{t_i} < k$)
		$= \set{t} \cup \subII{t_1} \cup \subII{t_2} = \subII{t}$
	  \item $t=\priv{t_1}$. Then $\sub{\priv{t_1}} = \set{t}\cup\sub{t_1} = 
		\set{t}\cup\subII{t_1} = \subII{t}$
	  \item $t=\aci{L}$. As $t$ is normalized, we have that for all $p\in L$, $p\neq \aci{L_p}$. Then $\elems{L} \approx L$. Thus, we have $\sub{t} = \set{t}\cup \bigcup_{p\in\elems{L}}\sub{p}
		= \set{t}\cup \bigcup_{p\in L}\sub{p} = \set{t}\cup \bigcup_{p\in L}\subII{p}=\subII{t}$.
	\end{itemize}

 \end{itemize}

\end{proof}
}\end{lemma}


In Proposition~\ref{prop:pairing} we remark, that ACI-set 
of normalized terms has the same deductive expressiveness as that set of normalized terms itself.

\begin{prop}\label{prop:pairing}
Let $T$ be a set of terms $T=\set{t_1,\dots, t_k}$. Then $\pairing{T}\in \der{\norm{T}}$ and  $\norm{T} \subseteq \der{\set{\pairing{T}}}$.


\end{prop}




In Proposition \ref{prop:normalSys}  we state that a  constraint system and its normal form have the same models. 
In Proposition \ref{prop:normalSigma} we show the equivalence, for a constraint system,  
between the existence of a model and the existence of a  normalized model. 
As a consequence we will need  only to consider  normalized constraints and models in the sequel.

\begin{prop}\label{prop:normalSys}
	The substitution $\sigma$ is a model of constraint system $\ConstrSys{S}$ if and only if $\sigma$ is a model of $\norm{\ConstrSys{S}}$.
\begin{proof}
	By definition, 
	$\sigma$ is a model of $\ConstrSys{S}=\set{E_i \rhd t_i}_{i=1, \dots, n}$, 
	iff $\forall i\in \set{1,\dots,n}$, $\norm{t_i\sigma} \in \der{\norm{E_i \sigma}}$. 
	But by Lemma~\ref{lemma:normprop} we have that  $\norm{t_i\sigma}=\norm{\norm{t_i}\sigma}$ and $\norm{E_i \sigma}=\norm{\norm{E_i} \sigma}$. 
	Thus, $\sigma$ is a model of ${\ConstrSys{S}}$ if and only if $\sigma$ is a model of $\norm{\ConstrSys{S}}$.
\end{proof}

\end{prop}



\begin{prop}\label{prop:normalSigma}
	The substitution $\sigma$ is a model of constraint system $\ConstrSys{S}$ if and only if $\norm{\sigma}$ is a model of $\ConstrSys{S}$.
\begin{proof}
	Proof is similar to one of Proposition~\ref{prop:normalSys}.
\end{proof}

\end{prop}






\subsection{Ground case of DY+ACI}\label{subs:groundcase}
In Algorithm~\ref{alg:solving} we need to check whether a ground substitution $\sigma$ satisfies a constraint system $\ConstrSys{S}$. For this, we have to check 
the derivability of a ground term from a set of ground terms. In this subsection we  present such an algorithm.

First, for the ground case we consider an  equivalent to DY+ACI deduction system DY+ACI' obtained from the first by replacing a set of rules 
 $$\forall i  \  \aci{{t_1, \dots, t_m}} \rightarrow \norm{t_i}$$
with   
 $$ \forall s\in\elems{t} \ {t} \rightarrow \norm{s}, \mbox{ if }  t=\aci{L}. $$


Now, we show an equivalence of the two deduction systems.
 
\begin{lemma}
	$t\in\xder{DY+ACI}{E} \iff t\in\xder{DY+ACI'}{E}$
\begin{proof}[Proof sketch] 

We show that every   rule of one deduction system can be simulated by a combination 
of rules from the other. It is sufficient to show it for non common rules. 

The DY+ACI' rules
	$\forall s\in\elems{t} \ {t} \rightarrow \norm{s}, \mbox{ if }  t=\aci{L}$ 
	are modeled by successive application of rules
	$\forall i  \  \aci{{t_1, \dots, t_m}} \rightarrow \norm{t_i}$.
	The converse simulation of $ \aci{{t_1, \dots, t_m}} \rightarrow \norm{t_i}$ by DY+ACI'
is based on getting all the normalized elements of $t_i$ 
	and, if $\card{\norm{\elems{t_i}}} \geq 2$ then reconstructing $\norm{t_i}$ by rule ${p_1,\dots, p_l} \rightarrow \norm{\aci{p_1, \dots, p_l}}$,
	where $p_1, \dots, p_l$ are $\norm{\elems{t_i}}$.
\end{proof}

\end{lemma}

\begin{algorithm}[H]
  \caption{Verifying derivability of term}
  \label{alg:ground}
\KwIn{A normalized ground constraint $E \rhd t$}
  \KwOut{$t\in\xder{DY+ACI}{E}$}
  \BlankLine
Let  $S:=\sub{E}\cup\sub{t}\setminus E$\;
Let $D:=E$\;
\While{true}{ 
	\eIf{ exists DY rule $l\rightarrow r$, such that $l \subseteq D$ and $r\in S$}{  \nllabel{step:DY}
		$S:=S\setminus \set{r}$\;
		$D:=D\cup\set{r}$\;
}{
	\eIf{ exists $s\in S : \elems{s} \subseteq D$}{  \nllabel{step:setcompos}
		$S:=S\setminus\set{s}$\;
		$D:=D\cup\set{s}$\;
}
{
	\eIf{ exists $s\in D :  \elems{s} \nsubseteq D$}{ \nllabel{step:setdecompos}
		$S:=S\setminus \elems{s}$\;
		$D:=D\cup\elems{s}$\;
}{
		\Return{ $t\in D$}\; \nllabel{step:ret}
}}}
}
\end{algorithm}


\begin{lemma}\label{lemma:propalgo}
	For Algorithm~\ref{alg:ground} the following statements are true:
	\begin{itemize}
		\item for any step\footnote{Consider two sequential assignments as one step}, $D\cup S =\sub{E\cup\set{t}}$ and $D\cap S=\emptyset$;
		\item it terminates;
		\item for any step, $D\subseteq \xder{DY+ACI}{E}$.
	\end{itemize}

\end{lemma}

The following lemmas will be used to prove correctness of the algorithm. 

\begin{lemma}\label{lemma:strokerules} \
\begin{itemize}
	\item For any decomposition rule $l\rightarrow r$ of DY+ACI', if  $l$ is normalized, then $r$ is a quasi-subterm of $l$.
	\item For any composition rule $l\rightarrow r$ of DY+ACI' except $\set{t_1, \dots, t_m} \rightarrow \norm{\aci{t_1, \dots, t_m}}$, if $l$ is normalized, then $l\subseteq \sub{r}$.
\end{itemize}
\end{lemma}




\begin{lemma}\label{lemma:walkout}
	After the execution of Step~\ref{step:ret} of Algorithm~\ref{alg:ground}, if $l\rightarrow r$ is a DY+ACI' rule, such that $l\subseteq D$ and $r\notin D$, then $l\rightarrow r$ is a composition rule and $r\notin \sub{E\cup\set{t}}$.
\begin{proof}
	Suppose, $l\rightarrow r$ is a decomposition.   By Lemma~\ref{lemma:strokerules} we have that $r\in\sub{l}$ and thus, $r\in\sub{D}\subseteq D \cup S$. Then  $r\notin D$ implies $r\in S$, and then, Step~\ref{step:ret} must be skipped, as branch \ref{step:DY} or \ref{step:setdecompos} should have been visited. 

	Thus, $l\rightarrow r$ is  a composition.
	 As algorithm reached  Step~\ref{step:ret}, that means $r\notin S$ (otherwise one of three branches must be visited and this step would be skipped). As $r\notin S$ and $r\notin D$, we have $r\notin S\cup D = \sub{E\cup\set{t}}$.
\end{proof}

\end{lemma}

\begin{lemma}\label{lemma:composelems}
	Given a set of normalized terms $S$ such that for any $s\in S$, $\elems{s}\subseteq S$.
	Then for any DY+ACI' composition rule $l\rightarrow r$ such that $l\subseteq S$ we have $\elems{r}\subseteq S\cup\set{r}$.
\begin{proof}
	All cases of composition rules except $ {t_1, \dots, t_m} \rightarrow \norm{\aci{t_1, \dots, t_m}}$ are trivial, as for them $\elems{r}=\set{r}$.
	For this case, as $\elems{t_i}\subseteq S$ for all $i$, then (by Lemma~\ref{lemma:normprop}, Statement \ref{pDotNormElems}) $\elems{\norm{\aci{t_1, \dots, t_m}}} =  $ \\ $ \elems{\aci{t_1, \dots, t_m}} =\bigcup_{i=1}^m\elems{t_i}\subseteq S$.
\end{proof}

\end{lemma}


\begin{prop}
 Algorithm~\ref{alg:ground} is correct.
\begin{proof}
	If algorithm returns true, then, by Lemma~\ref{lemma:propalgo}, $t\in\xder{DY+ACI'}{E}$.
	
	Show, that output is correct, if algorithm returns false. Note, that we consider values of $D$ and $S$ that they have after finishing the algorithm. Suppose that output is false ($t\notin D$), but $t\in \xder{DY+ACI'}{E}$. Then there exists minimal by length derivation $\set{E_i}_{i=0,\dots,n}$ where $n\geq 1$, $D=E_0$ (as $D\subseteq \xder{DY+ACI'}{E}$ and $t\notin D$) and $t\in E_n$ and $E_{i+1}\setminus E_i \neq \emptyset$ and $E_i \rightarrow_{l_i\rightarrow r_i} E_{i+1}$ for all $i=0,\dots,n-1$.
	Then, applying Lemma~\ref{lemma:walkout} we have $l_0 \rightarrow r_0$ is a composition, and $r_0 \notin \sub{E\cup\set{t}}$.
	
	Let $m$ be the smallest index such that there exists $s \in S=\sub{E\cup\set{t}}\setminus D$ and $s\in E_m$.
	
	Let $k$ be the minimal integer, such that $l_k\rightarrow r_k$ is a decomposition.
	
	Show, $k\leq m$. Suppose the opposite, then $s$ is built by a chain of composition rules from $D$. If $l_{m-1}\rightarrow r_{m-1}$ (where $r_{m-1} =s $) is
	\begin{itemize}
	 \item a rule in form of  $ \set{t_1, \dots, t_c} \rightarrow \norm{\aci{t_1, \dots, t_c}}$, then $\elems{s} \neq \set{s}$ (otherwise it contradicts to minimality of the derivation) and from Lemma~\ref{lemma:composelems}, $\elems{s}\subseteq E_{m-1}$ ($m\neq 1$, otherwise this step would be executed in the algorithm). As $s\in S$, then $\elems{s}\subseteq \sub{s}\subseteq \sub{E\cup\set{t}}$.
	If $\elems{s}\subseteq D$ then we got contradiction with with the fact, that this step would be executed in the algorithm. If there exists $e\in\elems{s}$ and $e\notin D$ (that means, $e\in S$), then we get a contradiction with the minimality of $m$, as $e\in S$ was deduced before.
	 \item any other composition rule, then by Lemma~\ref{lemma:strokerules}, $l_{m-1} \subseteq \sub{s}$, and thus, $l_{m-1} \subseteq D\cup S$. Similarly to the previous case, $m\neq 1$ and we get a contradiction with either minimality of $m$, or with the fact, that the algorithm would have to add $s$ into $D$.\end{itemize}
      
	Note, that this also shows, that decomposition rule is present in derivation.
	
	Show, $l_k\nsubseteq D$. Suppose the opposite. Then by Lemma~\ref{lemma:strokerules}, we have $r_k\subseteq D$ what contradicts to $E_{k+1}\setminus E_k \neq \emptyset$.
	Thus, at least one element from $l_k$ is not from $D$.
	Let us consider all possible decomposition rules $l_k\rightarrow r_k$:
	\begin{itemize}
		\item $\set{\pair{t_1}{t_2}} \rightarrow \norm{t_1}$. We know, that $\pair{t_1}{t_2}$ is not in $D$, thus, it was built by composition. As $E_i$ are normalized, the only possible way to build by composition $\pair{t_1}{t_2}$ from normalized terms is $\set{t_1,t_2}\rightarrow \pair{t_1}{t_2}$ (other ways, like $\pair{t_1}{t_2}, \pair{t_1}{t_2} \rightarrow \norm{\aci{\lst{\pair{t_1}{t_2}, \pair{t_1}{t_2}}}}$ would contradict the minimality of the derivation). 
		Thus, $t_1$ was derived before (or was in $D$), i.e. $t_1 \in E_k$. 
		That contradicts to $E_{k+1}\setminus E_k \neq \emptyset$.
		
		\item  $\set{\pair{t_1}{t_2}} \rightarrow \norm{t_2}$. Similar case.
		
		\item $\set{\enc{t_1}{t_2},  \norm{t_2}} \rightarrow \norm{t_1}$. The case where $\enc{t_1}{t_2}\notin D$ has similar explanations as two cases above. Thus, $\enc{t_1}{t_2}\in D$. That means, $t_2\in\sub{E\cup\set{t}}$ and 
		 $t_2\notin D$, i.e. $t_2\in S$. 
This means, $t_2$ was derived before and $t_2\in S$, what contradicts to  $k\leq m$.
		
\item $\set{\aenc{t_1}{t_2},  \norm{\priv{t_2}}} \rightarrow \norm{t_1}$ is a similar case to previous one. Note, that if $\priv{t_2}$ is not in $D$, that it must be obtained by decomposition.
		
		\item ${t} \rightarrow \norm{s}$, where $s\in\elems{t}$ and $t=\aci{L}$. By Lemma~\ref{lemma:composelems}, $\elems{t}\subseteq E_k$, that contradicts minimality of derivation ($E_{k+1}\setminus E_k \neq \emptyset$).
	\end{itemize}

	
\end{proof}
\end{prop}



\subsection{Existence of conservative solutions}

In this subsection we will show that for any satisfiable constraint system, there exist a model in special form (so called conservative solution). 
Roughly speaking, a model in this form can be defined per each variable by set of quasi-subterms of the constraint system and set of atoms (also from the constraint system) that must be prived.
This will bound a search space for the model (see \S~\ref{subs:bounds}).

First, we show, that on quasi-subterms of constraint system instantiated with its model, the transformation $\pah{\cdot}$ will be a homomorphism modulo normalization. \begin{prop}\label{prop:subt}
Given a normalized constraint system $\ConstrSys{S}$ and its  normalized model $\sigma$.
For all $t \in \sub{\ConstrSys{S}}$, $\norm{t\pahs} = \norm{\pairing{\ah{t\sigma}}} $.
 \begin{proof}
  We will prove it by induction on $\card{\subII{t}}$, where $t$ is normalized. 
\begin{itemize}
 \item Let $\card{\subII{t}} = 1$. Then:
	\begin{itemize}
 		\item either $t \in \UniAt{A}$.  In this case $t \in (\UniAt{A} \cap \sub{\ConstrSys{S}})$, and as $t \mu = t $ for any substitution $\mu$, then  $\pairing{\ah{t\sigma}} = \pairing{\ah{t}}  = \pairing{\set{t}}= t$ and $t\pahs = t$. Thus, $t\pahs = \pairing{\ah{t\sigma}} $.

		\item or $t \in \UniVar{X}$. As $\sigma$ is a model and $t\in \sub{\ConstrSys{S}}$, we have $t \in \dom{\sigma}$, and, by definition, $t \in \dom{\pahs}$. Then, by definition of $\pahs$, $t\pahs = \pairing{\ah{t\sigma}}$.
	\end{itemize}

 \item Assume that for some $k \geq 1$ if $\card{\subII{t}} \leq k$, then $\norm{t\pahs} = \norm{\pairing{\ah{t\sigma}}} $.
 \item Show, that for any $t$ such that $\card{\subII{t}} \geq k+1$, where $t=\bin{p}{q}$ 
or $t=\priv{q}$ 
 or $t=\aci{t_1, \dots, t_m}$,
but $\card{\subII{p}} \leq k$, $\card{\subII{q}} \leq k$ and $\card{\subII{t_i}}\leq k$, for all $i\in\set{1,\dots,m}$,
 statement  $\norm{t\pahs} = \norm{\pairing{\ah{t\sigma}}} $ is still true. 
 We have: 
	\begin{itemize}
 		\item either $t = \bin{p}{q}$. As $t = \bin{p}{q} \in \sub{\ConstrSys{S}} \Rightarrow p \in \sub{\ConstrSys{S}}$ and $q \in \sub{\ConstrSys{S}}$. As $\card{\subII{p}} < \card{\subII{t}}$  and from the induction assumption, we have $\norm{p\pahs} = \norm{\pairing{\ah{p\sigma}}}$. The same holds for $q$.

		Again, since $\bin{p}{q}\sigma \in \subo{S}\sigma$ (as $\bin{p}{q} \notin \UniVar{X}$ and $t\in \sub{\ConstrSys{S}}$) we have that 
		$\norm{\pah{\bin{p}{q}\sigma}} =  $ \br $ 
		\norm{\pah{\bin{p\sigma}{q\sigma}}}   =  \norm{\pah{\norm{\bin{p\sigma}{q\sigma}}}}  =   $ \br $ 
		\norm{\pairing{\ah{\bin{\norm{p\sigma}}{\norm{q\sigma}}}}}  =   $ \br $ 
		\norm{\pairing{\set{\bin{\pairing{\ah{\norm{p\sigma}}}}{\pairing{\ah{\norm{q\sigma}}}}}}} =  $ \br $ 
		\norm{\pairing{\set{\bin{\norm{\pairing{\ah{p\sigma}}}}{\norm{\pairing{\ah{q\sigma}}}}}}} =  $ \br $ 
		\norm{  \bin{\norm{\pairing{\ah{p\sigma}}}}{\norm{\pairing{\ah{q\sigma}}}}} = $ \br $ 
		\norm{\bin{\norm{p\pahs}}{\norm{q\pahs}}} =  $ \br $ 
		\norm{\bin{p\pahs}{q\pahs}} =  $ \br $ 
		\norm{\bin{p }{q }\pahs}=
		\norm{t\pahs}$.


		\item or $t=\aci{t_1, \dots, t_m}$. As $t$ is normalized, 
		it implies that for all $i\in\set{1,\dots,m}$, $t_i$ are not in form of $\aci{L_{i}}$ and then $t_i \in \sub{\ConstrSys{S}}$, and thus, we have $t_i\in\sub{\ConstrSys{S}} \wedge \norm{\pah{t_i\sigma}} = \norm{t_i\pahs}$. 
		 $\pah{t\sigma}=
		 \pah{\aci{t_1\sigma,\dots,t_m\sigma}}=
		 \pairing{\ah{t_1\sigma} \cup\dots\cup \ah{t_m\sigma}}= $
		 (by Statement~\ref{pPairingOfPairing} of Lemma~\ref{lemma:normprop}) \br
		 $=
\pairing{\set{\pah{t_1\sigma},\dots,\pah{t_m\sigma}}}=  $ \br $ 
		 \pairing{\set{\norm{t_1\pahs},\dots,\norm{t_m\pahs}}}= $ \br $ 
		 \norm{\aci{\norm{t_1\pahs}, \dots, \norm{t_m\pahs}}}= $ \br $ 
		 \norm{\aci{t_1\pahs, \dots, t_m\pahs}}=  $ \br $ 
		 \norm{(\aci{t_1, \dots,t_m})\pahs} = \norm{t\pahs} $


		\item or $t=\priv{q}$. Then $q\in\sub{\ConstrSys{S}}$.
		
		$\pah{t\sigma}=\pairing{\set{\priv{\pah{q\sigma}}}} = \norm{\priv{\pah{q\sigma}}}= $ \br $ \norm{\priv{q\pahs}}=\norm{\priv{q}\pahs}=\norm{t\pahs}$.
		
	\end{itemize}
\end{itemize}

Thus, the proposition is proven.

 \end{proof}

\end{prop}












Now we show, that relation of derivability between a term and a set of terms is stable with regard to transformation $\pah{\cdot}$. 

\begin{lemma}\label{lemma:validstep}
Given a normalized constraint system $\ConstrSys{S}$ and its normalized model $\sigma$.
For any DY+ACI rule \ ${l_1,\dots,l_k}\rightarrow r$, \br $\pairing{\ah{r}} \in \der{\set{\pairing{\ah{l_1}}, \dots, \pairing{\ah{l_k}}} }$.


\begin{proof}[Proof idea] 
We proceed by considering all possible deduction rules. To give an idea, we show a proof for only one rule (see full proof in \ref{app:lemma|validstep}):
  	${\aenc{t_1}{t_2},\norm{\priv{t_2}}} \rightarrow \norm{t_1}$. Here we have to show that $\pah{\norm{t_1}}$ is derivable from $\set{\pah{\aenc{t_1}{t_2}}, \pah{\norm{\priv{t_2}}}}$.
	Consider two cases:
  	\begin{itemize}
		\item $\exists u\in \subo{S}$ such that $\norm{\aenc{t_1}{t_2}}=\norm{u\sigma}$. Then \br $\pah{\aenc{t_1}{t_2}}  = \aenc{\pah{t_1}}{\pah{t_2}}$, \br and then 
		$\pah{\norm{t_1}}=\pah{t_1} \in   $ \br $ \der{\set{\aenc{\pah{t_1}}{\pah{t_2}},\norm{\priv{\pah{t_2}}}}}$. \br
		On the other hand, $\pah{\norm{\priv{t_2}}} = \pah{\priv{t_2}} =  $ \br $ \pairing{\set{\priv{\pah{t_2}}}}=\norm{\priv{\pah{t_2}}}$.
		
		\item $\nexists u\in \sub{S}$ such that $\norm{\aenc{t_1}{t_2}}=\norm{u\sigma}$. 
		 Then \br $\pah{\aenc{t_1}{t_2}} = \pairing{\ah{t_1}\cup \ah{t_2}}$. Using Proposition~\ref{prop:pairing}, we have $\norm{\ah{t_1}\cup \ah{t_2}} \subseteq \der{\set{\pah{\aenc{t_1}{t_2}}}}$, thus (by Lemma~\ref{lemma:normprop}) $\norm{\ah{t_1}}\subseteq \der{\set{\pah{\aenc{t_1}{t_2}}}}$. And then, by Proposition~\ref{prop:pairing} we have that $\pah{t_1} \in \der{\norm{\ah{t_1}}}$. Therefore, by Lemma~\ref{lemma:dertrans},   $\pah{\norm{t_1}}=\pah{t_1} \in \der{\pah{\aenc{t_1}{t_2}}}$.
	\end{itemize}
\end{proof}

\end{lemma}

Using Proposition~\ref{prop:subt} and Lemma~\ref{lemma:validstep} we will show, that transformation $\pah{\cdot}$ preserves the property of substitution to be a model. 

\begin{theorem}\label{theorem:solution}
Given a normalized constraint system $\ConstrSys{S}$ and its normalized model $\sigma$.
Then substitution $\pahs$ also satisfies $\ConstrSys{S}$.
 \begin{proof}
 Suppose, without loss of generality, $\ConstrSys{S}= \set{E_i \rhd t_i}_{i=1, \dots, n}$.
  Let us take any constraint $(E \rhd t) \in \ConstrSys{S}$. As $\sigma$ is a model of $\ConstrSys{S}$,  there exists a derivation $D=\set{A_0, \dots, A_{k}}$  such that $A_0=\norm{E\sigma}$ and $\norm{t\sigma} \in A_{k}$. 

By Lemma~\ref{lemma:validstep} and Lemma~\ref{lemma:derext} we can easily prove that if $k>0$, $\pah{A_{j}} \subseteq \der{\pah{A_{j-1}}}, \  j=1,\dots,k$. 
Then, applying transitivity of $\der{\cdot}$ (Lemma~\ref{lemma:dertrans}) $k$ times, we have that $\pah{A_k} \subseteq \der{\pah{A_0}}$. In the case where $k=0$, the statement $\pah{A_k} \subseteq \der{\pah{A_0}}$ is also true.


 Using Proposition~\ref{prop:subt} we get $\pah{A_0} =\pah{E\sigma} = \norm{E\pahs}$, as $E \subseteq \sub{\ConstrSys{S}}$. The same for $t$: $\pah{t\sigma} = \norm{t\pahs}$, and as $\norm{t\sigma} \in A_k$, we have $\norm{t\pahs} \in \pah{A_k}$. 
Thus, we have that $\norm{t\pahs} \in \pah{A_k} \subseteq \der{\pah{A_0}} = \der{\norm{E\pahs}} $, that means $\pahs$ satisfies any constraint of $\ConstrSys{S}$. 

 \end{proof}
\end{theorem}

From now till the end of subsection we will study a very useful property of $\pahs$.
Proposition~\ref{prop:mainprop} and its corollary show, 
that if constraint system has a normalized model ($\sigma$) which sends different variables to different values, 
then there exists another normalized model ($\pahs$)  that  sends any variable of its domain
to an ACI-set of some non-variable quasi-subterms of constraint system instantiated by itself
and some private keys built with atoms of  the constraint system.

\begin{lemma}\label{lemma:goodSub}
	If $\norm{u\sigma}=\enc{p}{q}$, $\sigma$ is normalized, $u=\norm{u}$,  $u\notin \UniVar{X}$ and $x\sigma\neq y\sigma, x\neq y$,
	then there exists $s\in\sub{u}$ such that $s=\enc{p'}{q'}$ and $\norm{s\sigma}=\enc{p}{q}$. 
	The similar is true in the case of  $\norm{u\sigma}=\pair{p}{q}$, $\norm{u\sigma}=\aenc{p}{q}$, $\norm{u\sigma}=\sig{p}{q}$ and for $\norm{u\sigma}=\priv{p}$.
\LONG{
\begin{proof}
	As $u=\norm{u}$ and $\norm{u\sigma}=\enc{p}{q}$, we have:
\begin{itemize}
		\item $u$ not in form of $\aci{L}$. Then, as $u\notin\UniVar{X}$ and $\norm{u\sigma}=\enc{p}{q}$, we have $u=\enc{p'}{q'}$ (where $\norm{p'\sigma}=p$ and $\norm{q'\sigma}=q$). 
		Then we can choose $s=\enc{p'}{q'}=u\in\sub{u}$.


		\item $u=\aci{t_1,\dots,t_m}$, $m>1$, as $u=\norm{u}$. Then, for all $i$,  $t_i$ is either a variable, or $\enc{p'_i}{q'_i}$. 
		But, as  $x\sigma\neq y\sigma, x\neq y$  and as $\sigma$ is normalized, we can claim, that $\set{t_1,\dots,t_m}$ contains at most one variable. 
		Then, as $m>1$, there exists $i$ such that $t_i = \enc{p'_i}{q'_i}$. Then by definition of normalization function, and from $\norm{u\sigma}=\enc{p}{q}$ we have, that 
		$\norm{\elems{u\sigma}} = \set{\enc{p}{q}}$ and as $t_i\sigma$ is an element of $u\sigma$, we have $\norm{\enc{p'_i}{q'_i}\sigma} = \enc{p}{q}$. 
		Thus, we can choose $s=t_i$, as $t_i \in \sub{u}$ and $t_i= \enc{p'_i}{q'_i}$.
	\end{itemize}
	
	The other cases  ($\oppair$, $\oppriv$, etc...) can be proved similarly.

\end{proof}
}
\end{lemma}



\begin{prop}\label{prop:mainprop}
 Given a normalized constraint system $\ConstrSys{S}$ 
 and its normalized model $\sigma$ such that for all $x,y\in\dom{\sigma}$, $x\neq y \implies x\sigma\neq y\sigma$. 
 Then  for all $x\in \dom{\pahs}$ there exist $k \in \mathbb{N}$ and $s_1,\dots, s_k \in \subo{S}\cup \priv{\sub{\ConstrSys{S}}\cap\UniAt{A}}$  such that $\rut{s_i} \neq \cdot$ 
and \br $ x\pahs = \pairing{\set{s_1\pahs, \dots, s_k\pahs}}$.

\begin{proof}
\raggedright
 By definition, $x\pahs = \pah{x\sigma}$. Let us take any $s\in \ah{x\sigma}$ (note, that $s$ is a ground term). Then, by definition of $\ah{\cdot}$ we have:
\begin{itemize}
 \item either $s \in \UniAt{A}$. Then, by definition of $\ah{\cdot}$, $s\in (\UniAt{A} \cap \sub{\ConstrSys{S}})$. Thus, $s\pahs = s$, $s \in  \subo{S}$, $s\neq \aci{L}$;
 
 \item or $s = \bin{\pah{t_1}}{\pah{t_2}}$ and there exists $u \in \subo{S}$ such that $\norm{u\sigma}=\norm{\bin{t_1}{t_2}}=\bin{\norm{t_1}}{\norm{t_2}}$. 
  As all conditions of Lemma~\ref{lemma:goodSub} are satisfied, then there exists $v\in\sub{u}$ such that $\norm{v\sigma}=\bin{\norm{t_1}}{\norm{t_2}}$ and 
$v=\bin{p}{q}$ 
  and as $u\in\subo{S}$ then $v\in\subo{S}$.
  By Proposition~\ref{prop:subt}, $\norm{v\pahs}=\pah{v\sigma}=\pah{\norm{v\sigma}}=\pah{\bin{t_1}{t_2}}=\pairing{\set{\bin{\pah{t_1}}{\pah{t_2}}}} = \bin{\pah{t_1}}{\pah{t_2}} = s$. 
  That means that there exists  $v \in \subo{S}$ such that $v\neq \aci{L}$ and $s=\norm{v\pahs}$.
 


 \item or $s=\priv{\pah{t_1}}$. In this case, as $s$ is ground, $\pah{t_1}$ must be an atom, moreover, by definition of $\ah{\cdot}$, this atom is from $(\UniAt{A} \cap \sub{\ConstrSys{S}})$. 
	Therefore, $s=\priv{a}$, where $a\in \UniAt{A} \cap \sub{\ConstrSys{S}}$ (and of course, $s\neq  \aci{L}$).
\end{itemize}

Thus,  for all $s\in \ah{x\sigma}$, there exists $v \in   (\sub{\ConstrSys{S}})\cup \priv{\sub{\ConstrSys{S}}\cap\UniAt{A}} \setminus \UniVar{X}\,|\ s=\norm{v\pahs}$. 
Therefore, as  $x\pahs=\pah{x\sigma}$, we have that $x\pahs = \pairing{\set{\norm{s_1\pahs},\dots, \norm{s_k\pahs}}} =  \pairing{\set{s_1\pahs,\dots, s_k\pahs}}$, where   $s_1,\dots, s_k \in \subo{\ConstrSys{S}} \cup \priv{\sub{\ConstrSys{S}}\cap\UniAt{A}}$ and $s_i\neq \aci{L}, \forall 1\leq i \leq k$.  That proves the proposition.
\end{proof}

\end{prop}



\begin{cor}\label{cor:existsgood}
	 Given normalized constraint system $\ConstrSys{S}$ and $\sigma'$ --- its normalized model,   such that $x\neq y \implies x\sigma' \neq y\sigma'$. Then there exists a normalized model $\sigma$ of  $\ConstrSys{S}$ such that for all $x\in \dom{\sigma}$ there exist $k \in \mathbb{N}$ and $s_1,\dots, s_k \in \subo{S}\cup \priv{\sub{\ConstrSys{S}}\cap\UniAt{A}}$ such that $x\sigma = \pairing{\set{s_1\sigma, \dots, s_k\sigma}}$ and $s_i \neq s_j$, if $i\neq j$; $s_i\neq\aci{L}, \forall i$.


\end{cor}
 Any normalized model with property shown in Corollary \ref{cor:existsgood} we will call \emph{conservative}.

\subsection{Bounds on conservative solutions}
\label{subs:bounds}

To get a decidability result, 
we first show an upper bound on size of conservative model 
and then, 
by reducing any satisfiable constraint system to one that have 
conservative model and showing that reduced one is smaller
(by size) than original one,
we obtain an existence of a model with bounded size for any 
satisfiable constraint system.


\begin{lemma}\label{lemma:subsigma}
	Given a normalized constraint system $\ConstrSys{S}$ and its conservative model $\sigma$. Then for all $x \in \vars{\ConstrSys{S}}$ we have $\sub{x\sigma}\subseteq \norm{\sub{\ConstrSys{S}}\sigma}\cup \priv{\sub{\ConstrSys{S}}\cap\UniAt{A}}$.
\begin{proof}
	Given a ground substitution $\sigma$, let us define a  strict total order on variables: $x\sqsubset y \iff (\DAGsize{x\sigma}<\DAGsize{y\sigma}) \vee (\DAGsize{x\sigma}=\DAGsize{y\sigma} \wedge x\prec y)$.
	
	By Proposition~\ref{prop:mainprop} for all $x$ $x\sigma = \pairing{\set{s^x_1\sigma, \dots, s^x_{k^x}\sigma}}$, where  $s^x_i \in (\sub{\ConstrSys{S}} \setminus \UniVar{X})\cup \priv{\sub{\ConstrSys{S}}\cap\UniAt{A}}$  and $s^x_i\neq \aci{L}$. 

Let us show that if $y\in\vars{s^x_i}$ for some $i$, then $y\sqsubset x$.
	Suppose, that $y\in\vars{s^x_i}$ and $x\sqsubset y$. 
	Then $\DAGsize{x\sigma}  = \DAGsize{\pairing{\set{s^x_1\sigma, \dots, s^x_{k^x}\sigma}}} = 
	\DAGsize{\norm{\aci{s^x_1\sigma,\dots, s^x_{k^x}\sigma}}} \geq$ 
	(by Lemma~\ref{lemma:normprop}) 
	$ \geq \DAGsize{\norm{s^x_i\sigma}}> $ \br $ \DAGsize{\norm{y\sigma}}$,  
	because we know that $s^x_i=\bin{p}{q}$ or $s^x_i=\priv{p}$ and $y\in\vars{s^x_i}$ 
	(for example, in first case, $\DAGsize{\norm{s^x_i\sigma}} =  $ \br $ \DAGsize{\bin{\norm{p\sigma}}{\norm{q\sigma}}}=
	  1+\DAGsize{\set{\norm{p\sigma},\norm{q\sigma}}}$ and   \br  since $y\in\vars{\set{p,q}}$, 
	  using Statement~\ref{pNormSubSize} of Lemma~\ref{lemma:normprop}, 
	  we get   \br   $\DAGsize{\norm{s^x_i\sigma}} \geq 1+ \DAGsize{\norm{y\sigma}}$)
	And as $\DAGsize{\norm{y\sigma}} = \DAGsize{y\sigma}$ 
	That means, $y\sqsubset x$. Contradiction.
	
	Now we show by induction main property of this lemma.
	\begin{itemize}
		\item let $x=\min_{\sqsubset}(\vars{\ConstrSys{S}})$. \br 
		Then $ x\sigma = \pairing{\set{s^x_1\sigma, \dots, s^x_{k^x}\sigma}}= \norm{\aci{s^x_1\sigma, \dots, s^x_{k^x}\sigma}}$ and 
		 all $s^x_i$ \br are ground  (as there does not exists $y \sqsubset x$). 
		Then $x\sigma=\norm{\aci{s^x_1, \dots{}, s^x_{k^x}}}$.
 We have that $\sub{x\sigma}=\set{\norm{\aci{s^x_1, \dots, s^x_{k^x}}}}\cup\sub{s^x_1}\cup\dots\cup\sub{s^x_{k^x}} \subseteq \norm{\sub{\ConstrSys{S}}\sigma}\cup\priv{\UniAt{A}\cap\sub{\ConstrSys{S}}}$, 
as for any $s\in\sub{s^x_i}$, $s\in\UniverseG$ and $s\in\sub{\ConstrSys{S}}$ or $s=\priv{a}$ or $s=a$, where $a\in \sub{\ConstrSys{S}}\cap \UniAt{A}$, therefore  $s=\norm{s}=s\sigma\in\norm{\sub{\ConstrSys{S}}\sigma}\cup \priv{\sub{\ConstrSys{S}}\cap \UniAt{A}}$ 
and 
$\norm{\aci{s^x_1, \dots, s^x_{k^x}}} = x \sigma \in \norm{\sub{\ConstrSys{S}}\sigma}$.

		\item
			Suppose, that for all $z\sqsubset y$ we have \br  $\sub{z\sigma} \subseteq \norm{\sub{\ConstrSys{S}\sigma}}\cup \priv{\sub{\ConstrSys{S}}\cap\UniAt{A}}$.
		
		\item
			Show, that $\sub{y\sigma} \subseteq \sub{\ConstrSys{S}\sigma}\cup \priv{\sub{\ConstrSys{S}}\cap\UniAt{A}}$.
			We know that $ y\sigma = \pairing{\set{s^y_1\sigma, \dots, s^y_{k^y}\sigma}}= \norm{\aci{s^y_1\sigma,\dots, s^y_{k^y}\sigma}}$ 
			and for any  $z\in \vars{s^y_i}$, $z\sqsubset y$.
			Then we have $\sub{y\sigma} = $ \br $ \set{y\sigma}\cup\sub{\norm{s^y_1\sigma}}\cup\dots\cup\sub{\norm{s^y_{k^y}\sigma}}$. We know that $y\sigma\in\norm{\sub{\ConstrSys{S}}\sigma}$. Let us show that $\sub{\norm{s^y_i\sigma}}\subseteq\norm{\sub{\ConstrSys{S}}\sigma}\cup\priv{\sub{\ConstrSys{S}}\cap\UniAt{A}}$.
			By Lemma~\ref{lemma:normprop} 
we have $\sub{\norm{s^y_i\sigma}}\subseteq  $ \br $ \norm{\sub{s^y_i\sigma}}\subseteq \norm{\sub{s^y_i}\sigma\cup\sub{\vars{s^y_i}\sigma}}=  $ \br $ 
			\norm{\sub{s^y_i}\sigma}\cup\sub{\vars{s^y_i}\sigma}$. We can see that $\norm{\sub{s^y_i}\sigma}\subseteq\norm{\sub{\ConstrSys{S}}\sigma}\cup\priv{\sub{\ConstrSys{S}}\cap\UniAt{A}}$  (as $s^y_i\in\sub{\ConstrSys{S}}\cup  $ \br $ \priv{\sub{\ConstrSys{S}}\cap\UniAt{A}}$); and by induction supposition and by statement proved above we have 
			$\sub{\vars{s^y_i}\sigma}\subseteq \norm{\sub{\ConstrSys{S}}\sigma}\cup $ \br $ \priv{\sub{\ConstrSys{S}}\cap\UniAt{A}}$. \br
			Thus, $\sub{y\sigma} \subseteq \norm{\sub{\ConstrSys{S}}\sigma}\cup\priv{\sub{\ConstrSys{S}}\cap\UniAt{A}}$.

	\end{itemize}

\end{proof}

\end{lemma}



\begin{prop}\label{prop:limit}
	For normalized constraint system $\ConstrSys{S}$ that have conservative  model $\sigma$, 
	for any $x\in \vars{\ConstrSys{S}}$ we have $\DAGsize{x\sigma}\leq 2\times\DAGsize{\ConstrSys{S}}$.
\begin{proof}
	As $\card{\norm{\subII{\ConstrSys{S}}\sigma}} \leq \card{\subII{\ConstrSys{S}}\sigma} \leq \card{\subII{\ConstrSys{S}}}=\DAGsize{\ConstrSys{S}}$, we have (using the fact that $\sigma$ is normalized and  Lemma~\ref{lemma:subsigma}) 
	that  $\card{\subII{x\sigma}} = \card{\sub{x\sigma}}\leq \card{\norm{\sub{\ConstrSys{S}}\sigma}\cup\priv{\UniAt{A}\cap\sub{\ConstrSys{S}}}} \leq \card{\norm{\sub{\ConstrSys{S}}\sigma}}+\card{\priv{\UniAt{A}\cap\sub{\ConstrSys{S}}}} \leq  \DAGsize{\ConstrSys{S}} + \card{\UniAt{A}\cap\sub{\ConstrSys{S}}} \leq 2 \times \DAGsize{\ConstrSys{S}}$;
thus, $\DAGsize{x\sigma}\leq 2 \times\DAGsize{\ConstrSys{S}}$.
\end{proof}

\end{prop}

From this proposition and Corollary~\ref{cor:existsgood} we obtain an existence of bounded model 
for a normalized constraint system 
that have a model sending different variables to different values.
We will reduce an arbitrary constraint system to already studied case. 
The target properties are stated in Proposition~\ref{prop:generalLimit} and Corollary~\ref{cor:polynom}.

\begin{lemma}\label{lemma:sizesubst}
Given any constraint system $\ConstrSys{S}$ and any substitution $\theta$ such that $\dom{\theta}=\vars{\ConstrSys{S}}$ and $\dom{\theta}\theta \subseteq \dom{\theta}$.	Then $\DAGsize{\ConstrSys{S\theta}}\leq\DAGsize{\ConstrSys{S}}$.
\LONG{
\begin{proof}
	From Lemma~\ref{lemma:normprop} we obtain $\DAGsize{\ConstrSys{S}\theta}= \card{\subII{\ConstrSys{S}\theta}} 
	= \card{\subII{\ConstrSys{S}}\theta\cup\subII{\vars{\ConstrSys{S}}\theta}}$, but $\vars{\ConstrSys{S}}\theta \subseteq  \dom{\theta}=\vars{\ConstrSys{S}}$ ($\vars{\ConstrSys{S}\sigma}$ consists only of variables), and then  $\subII{\vars{\ConstrSys{S}}\theta} = \vars{\ConstrSys{S}}\theta$. As  $\vars{\ConstrSys{S}} \subseteq\subII{ \ConstrSys{S}}$, we have  $\subII{\ConstrSys{S}}\theta\cup\subII{\vars{\ConstrSys{S}}\theta}=\subII{\ConstrSys{S}}\theta$. \br
	Thus, $\DAGsize{\ConstrSys{S}\theta}= \card{\subII{\ConstrSys{S}}\theta}\leq \card{\subII{\ConstrSys{S}}} = \DAGsize{\ConstrSys{S}}$.
\end{proof}
}\end{lemma}



\begin{df}
	Let $\sigma$ and $\delta$ be substitutions. Then $\sigma[\delta]$ is a substitution such that $\dom{\sigma[\delta]} = \dom{\delta}$ and for all $x\in\dom{\sigma[\delta]}$, $x\sigma[\delta] = (x\delta)\sigma$.
\end{df}


\begin{lemma}\label{lemma:substcomp}
	Let $\theta$ and $\sigma$ be substitutions such that $\dom{\theta}\theta=\dom{\sigma}$, $\dom{\sigma}\subseteq \dom{\theta}$ and $\sigma$ is ground. Then, for any term $t$, $(t\theta)\sigma= t\sigma[\theta]$.
\LONG{
\begin{proof}


When apply $\theta$ to $t$, every variable $x$ of $t$ such that $x\in\dom{\theta}$ is replaced by $x\theta$; then we apply $\sigma$ to $t\theta$:  every variable $y$ of $t\theta$ is replaced by $y\sigma$, thus, every variable $x$ from $\dom{\theta}$ will be replaced to $(x\theta)\sigma$ (as $\dom{\theta}\theta=\dom{\sigma}$); and no other variables will be replaced (as $\dom{\sigma}\subseteq \dom{\theta}$). Thus, we can see that it is the same as in definition of $\sigma[\theta]$.
\end{proof}
}\end{lemma}



\begin{prop} \label{prop:generalLimit}
  Given any satisfiable constraint system $\ConstrSys{S}$. Then there exists a model $\sigma$ of $\ConstrSys{S}$ such that for any $x\in\dom{\sigma}$, $\DAGsize{x\sigma}\leq 2\times\DAGsize{\ConstrSys{\norm{S}}}$ \begin{proof}[Proof idea]
Given a normalized model $\sigma'$ of $\ConstrSys{S}$ we build a substitution $\theta$ that maps different variables whose $\sigma'$-instnatces are the same to one.
In this way we obtain a new constraint system and its normalized model on which we can apply Corollary~\ref{cor:existsgood} and get its conservative model $\sigma''$,
and by applying Proposition~\ref{prop:limit} we get a bound on size for this model. 
On the other part, we use Lemma~\ref{lemma:substcomp} to show that $\sigma''[\theta]$ is a model of $\norm{\ConstrSys{S}}$.
And then, using obtained bound and Lemma~\ref{lemma:sizesubst} show existence of a model with stated property.
The detailed proof is given in \ref{app:prop|generalLimit}

\end{proof}

\end{prop}


\begin{cor}\label{cor:polynom}
Constraint system  $\ConstrSys{S}$ is satisfiable if and only if there exists  a normalized model of $\ConstrSys{S}$ defined on $\vars{\ConstrSys{S}}$ 
which maps a variable to a ground term in $\Universe(\UniVar{A}\cap \sub{\ConstrSys{\norm{S}}}, \emptyset)$  with size not greater than double $\DAGsize{\ConstrSys{S}}$.


\end{cor}

Using this result, we propose an algorithm of satisfiability of constraint system (Algorithm~\ref{alg:solving}).


\begin{algorithm}[H]
  \caption{Solving constraint system}
  \label{alg:solving}
  \SetKw{Guess}{Guess}
  \SetKw{Normalize}{Normalize}
\KwIn{A constraint system  $\ConstrSys{S} = \set{E_i\rhd t_i}_{i=1,\dots,n}$}
  \KwOut{Model $\sigma$, if exists; otherwise $\bot$}
  \BlankLine
\Guess for every variable of $\ConstrSys{S}$ a value of ground normalized substitution $\sigma$ with size not greater than $2\times\DAGsize{\ConstrSys{{S}}}$\;
\eIf{ $\sigma$ satisfies $E_i\rhd t_i$ for all $i=1,\dots,n$}{
		\Return{ $\sigma$} }{\Return{ $\bot$ }}
\end{algorithm}






\begin{prop}
	Algorithm~\ref{alg:solving} is correct.
\begin{proof}
Let $\sigma$ be an output of Algorithm~\ref{alg:solving}. Then $\sigma$ is a ground substitution and $\sigma$ satisfies all constraints from $\ConstrSys{S'}$ and therefore, satisfies all constraints from $\ConstrSys{S}$ . This means, $\sigma$ is a model of $\ConstrSys{S}$.
\end{proof}

\end{prop}

\begin{prop}
	Algorithm~\ref{alg:solving} is complete.
\begin{proof}
Suppose, $\ConstrSys{S}$ is satisfiable.  Then, by Corollary~\ref{cor:polynom}, there exists a guess of value of ground substitution on every element of $\vars{\ConstrSys{S}}$  with size not greater than  $2\times\DAGsize{\ConstrSys{S}}$ which represents a model $\sigma$ of $\ConstrSys{S}$. Thus, algorithm~\ref{alg:solving} will return this $\sigma$.

\end{proof}

\end{prop}
