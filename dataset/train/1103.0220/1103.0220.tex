
In Section~\ref{sec:motiv} we reduced the  problem of protocol  insecurity in presence of several intruders 
to solving  a system of deducibility constraints. 
In this section we present a decision procedure for a constraint system where  
Dolev-Yao deduction system is extended by an associative-commutative-idempotent symbol (DY+ACI).
We consider operators for pairing, symmetric and asymmetric encryptions, decryption, signature 
and an  ACI operator that will be used as a set constructor.


As for the proof structure, after  introducing the formal notations, the main steps to show the decidability are as follows:
\begin{enumerate}
 \item We present an algorithm for solving a ground derivability in DY+ACI model.
 \item We prove, that the normalization does not change satisfiability: either we normalize a model or a constraint system.
 \item We show existence of a conservative solution of satisfiable constraint system: a substitution  that sends a variable to an ACI-set of quasi-subterms of the constraint system instantiated with  
  together with -ed atoms of the constraint system;
 \item We give a bound on size of a conservative solution, and, as consequence, we obtain decidability.
\end{enumerate}

\subsection{Formal introduction to the problem}

\subsubsection{Terms and notions}\label{subs:def2}



\begin{df}\label{def:term}
\emph{Terms}  are defined according to the following grammar:
\LONG{
 
}\SHORT{
 
}where
 and
.
 We denote  the set of all terms over 
a set of atoms  and a set of variables . 
For short, we write  instead of .

\end{df}

By  we mean a signature of message  with private key  
We do not assume that one can retrieve the  message itself from the signature.

Note that we do allow complex keys for symmetric encryption only.
As a consequence, we have to introduce a condition on substitution applications: 
substitution  cannot be applied to the term , if after replacing the resulting entity is not a term 
(for example, we cannot apply  to the term ).




We denote a term on -th position of a list  as . 
Then  is a shortcut for . 
We also define two binary relations  and  on lists as follows:
 if and only if  any  implies  ;
 if and only if  and ,
and naturally extend them if  or  is a set. 

\begin{df}
We consider symbol  to be 
	associative,
	commutative,
	idempotent
(shortly, ).
\end{df}

We will use  throughout the paper as a generalization of all binary operators: .

\begin{df}

 For every term  we define its root symbol by 


\end{df}






\begin{df}\label{def:elems}
	For any term  we define its \emph{set of elements} by:
We extend  to sets of terms or lists of terms  by . 
\end{df}
	
	
\begin{example}\label{ex:term}
 For term 
set of its elements is .
\end{example}





\begin{df}
Let  be a strict total order 
on ,
such that comparing can be done in polynomial time. 

\end{df}

\begin{df}
The cardinality of a  set   is denoted by .
\end{df}


\begin{df} \label{def:norm}
The \emph{normal form} of a term  (denoted by ) is recursively defined by:
\begin{itemize}
	\item , if 
	\item \item 
	\item 
\SHORT{
	
}\LONG{
	
}

\end{itemize}
where for set of terms , .
	
\end{df}

We can show easily that two terms  are congruent modulo the ACI properties of  iff they have the same normal form. 
Other properties are stated in Lemma~\ref{lemma:normprop}. 


\begin{example}
 Referring to Example~\ref{ex:term} for the value of term , we have
.
\end{example}






\begin{df}
 Let  be a term. We define a set of quasi-subterms  as follows:

If  --- set of terms, then . 
If  is a constraint system, we define .
\end{df}

\begin{example}
 Referring to Example~\ref{ex:term}, we have

\end{example}




\begin{df}\label{df:vars}
 Let  be a term. We define  as set of all the variables in :

\end{df}










We define  as the set of subterms of  
and the  DAG-size of a term, as the number of its different subterms. 
The DAG-size gives the size of a natural representation of a term in the 
considered ACI theory. 

\begin{df}\label{df:subdag}
 Let  be a term. We define  as follows:

If  is a set of terms, then .
If  is a constraint system, we define .
\end{df}

\begin{example}
 Referring to Example~\ref{ex:term}, we have

\end{example}



\begin{df}\label{df:sizedag}
    We define a DAG-size  of a term  as
,
for set of terms ,

and for constraint system  as
.
\end{df}
Remark, that for a constraint system such a definition does not polynomially approximate a number of bits needed to write it down\LONG{ (cf. Def.~\ref{def:DAGSys})}.








We define a Dolev-Yao deduction system modulo ACI equational theory  (denoted DY+ACI). 
It consists of composition rules and decomposition rules, depicted in Table~\ref{tab:DYACI} 
where .

\begin{table}[ht]
\centering
\begin{tabular}{|l|l|}
\hline
Composition rules & Decomposition rules \\
\hline
  &  \\
 &  \\
 & \\
 & \\ 
 &  for all  \\
\hline
\end{tabular}
\caption{DY+ACI deduction system rules}\label{tab:DYACI}
\end{table}






We suppose, hereinafter,  that for a constraint system ,  . Otherwise, we can add one constraint  to  which will be satisfied by any substitution.
We denote  for set of terms  as .
We define .
We say that  is normalized, iff for all ,  is normalized. 

\begin{example}\label{ex:constrsys}
 We give a sample of general constraint system and its solution within DY+ACI deduction system.
  
where  and .
 One of the eventual models within DY+ACI is .
\end{example}


\begin{df}\label{def:pairing}
 Let  be a non-empty set of terms. Then we define  as follows:
 
Remark: .
 \end{df}

\begin{df}
We denote  as  or, for shorter notation, .
\end{df}


We introduce a transformation  on ground terms that replaces recursively all binary root symbols such that they are different from
all the non-variable quasi-subterms of the constraint system instantiated with its model ,  with ACI symbol . 
Later, we will show, that  is also a model of .

\begin{df}\label{def:ah}

 Let us have a constraint system  which is satisfiable with model .
 Let us fix some .
 For given  and  we define a function   as follows:




\end{df}

Henceforward, we will omit parameters and write  instead of  for shorter notation.


\begin{df}
 We define the superposition of  and  on a set of terms  as follows: .
\end{df}

\begin{df}
 Let   be a substitution. We  define  the  substitution .

\end{df}
Note, that .

\begin{example}\label{ex:constrsysmodel}
 We refer to Example~\ref{ex:constrsys} and show, that  is also a model of .
 (we suppose that ).
One can see, that  is also a model of  within DY+ACI.
\end{example}



\subsubsection{General properties used in proof}

The two following lemmas state simple properties of derivability.

\begin{lemma}\label{lemma:dertrans}
 Let . Then if  and  then .

\end{lemma}

\begin{lemma}\label{lemma:derext}
 Let . Then if  and  then .

\end{lemma}

In Lemma~\ref{lemma:normprop} we list some auxiliary properties that will be used in main proof.



\begin{lemma}\label{lemma:normprop}
	The following statements are true:
	\begin{enumerate}
\item	\label{pACI}         For terms , we have , , 
\item 	\label{pNormNorm}      if  and  are terms, then  

		\item	\label{pSubNorm}     	



		\item	\label{pSubNormHasProimage}           	



		\item \label{pNormElems}        
		
		\item \label{pNormDotNorm}        ; 



		\item \label{pDotNormElems}         \\  


		\item \label{pAhList}, \item \label{pAhNorm}                     

		\item \label{pPahNorm}                
		\item \label{pPairingOfPairing}          
\item \label{pSubSubSub}                       
		\item \label{pNormSub}	
 		\item	\label{pSepVar}
		\item	\label{pSepVarII}          

		\item	\label{pNormCard}       ,  \item	\label{pSizesComparision}         
\item	\label{pNormSize}              For term , ; \\ for set of terms , ; \\
			for constraint system , 

		\item   \label{pSubACI}         
\item	\label{pNormSubSize} .
\end{enumerate}
\LONG{
\begin{proof}
We will give proofs of several statements. 
Some other technical proofs are given in \ref{app:lemma|normprop} \hfil \par
 \begin{description}
  \item [Statement~\ref{pNormElems}:]
      This statement is trivial, if .
      Otherwise, let . 
      \begin{itemize}
	\item if , where . Then  and then .
	\item if , , where  for all . Then , where . 
	      That means, that .
      \end{itemize}
  \item [Statement~\ref{pNormDotNorm}:]
      The first part follows from the definition of normal form and Statement~\ref{pNormElems}. The second one directly follows from the first.
  \item [Statement~\ref{pAhNorm}:]    
	By induction on :
	  \begin{itemize}
	  \item  is possible in the only case:  and as , the equality is trivial.
	  \item Suppose, that for any  (),  holds.
	  \item Given a term  , . We need to prove that .
		  \begin{itemize}
		  \item if , then  (by induction supposition) 
		  .
		  \item if  and .
			Then 
			(by induction supposition)  (by Statement~\ref{pNormDotNorm}) 
			.
		  \item if  and .
			Then  (by induction) 
			(as ) \\  
		  \item if , where .
			Note first, that as , we have for all, .
			Then, by Statement~\ref{pAhList},  (by induction supposition) .
			On the other part,   (by Statement~\ref{pNormElems}) .
\end{itemize}

	  \end{itemize} 
  \item [Statement~\ref{pPairingOfPairing}:] 
	From definition of   and Statement~\ref{pNormElems}, we obtain that \\ . 
	Next  (here we use  to capture two cases from definition of normalization at once), where  \\
	, \\
	while  , where . 
  \item [Statement~\ref{pNormSub}:]
      By induction on .
      \begin{itemize}
	\item . 
	      Then . As  and , the statement holds.
	\item Suppose, that for any  (), the statement is true.
	\item Given a term  , . Let us consider all possible cases:
	      \begin{itemize}
		\item .On the one hand, .  
		      On the other hand,  and then,
		      .
		      Then, as , we have that 
		      .
		\item . Proof is similar to one for the case above.
		\item . We have .
		      From Statement~\ref{pNormElems} we have ,
		      and then,  \br \br (by supposition) \\
		       \br.
	      \end{itemize} 
      \end{itemize}      
  \item [Statement~\ref{pSepVar}:]
      By induction on 
      \begin{itemize}
	\item  . 
	      \begin{itemize}
	      \item . As  and , the statement becomes trivial.
	      \item . Then , ; 
		    We have .
	      \end{itemize}
	\item Suppose, that for any  (), the statement is true.
	\item Given a term  , . Let us consider all possible cases:
	      \begin{itemize}
		\item . Then 
		       and .
		       \br  (as )
		      .
		\item . Proof is similar to one for the case above.
		\item . We have  and .
		      Then we have  (using Statement~\ref{pSizesComparision})
		      
		      (as )
		      
		      (as ) \\
		       \br . 
	      \end{itemize}
      \end{itemize}
 \end{description}
\end{proof}
} \end{lemma}

\begin{lemma}\label{lemma:normalsigma}
Given a  constraint system  and its model . Then 
substitution  is normalized 
\LONG{
\begin{proof}
 For any ,  (by Lemma~\ref{lemma:normprop}).
\end{proof}
}\end{lemma}



\begin{lemma}\label{lemma:DAGvsQuasi}
 For any normalized term , .
\LONG{
\begin{proof} By induction on .
 \begin{itemize}
  \item  . Then  , and thus, .
  \item Suppose, that for any  (), .
  \item Given a term  , . We need to show that .
	\begin{itemize}
	  \item . Then 
		 (as )
		
	  \item . Then 
	  \item . As  is normalized, we have that for all , . Then . Thus, we have .
	\end{itemize}

 \end{itemize}

\end{proof}
}\end{lemma}


In Proposition~\ref{prop:pairing} we remark, that ACI-set 
of normalized terms has the same deductive expressiveness as that set of normalized terms itself.

\begin{prop}\label{prop:pairing}
Let  be a set of terms . Then  and  .


\end{prop}




In Proposition \ref{prop:normalSys}  we state that a  constraint system and its normal form have the same models. 
In Proposition \ref{prop:normalSigma} we show the equivalence, for a constraint system,  
between the existence of a model and the existence of a  normalized model. 
As a consequence we will need  only to consider  normalized constraints and models in the sequel.

\begin{prop}\label{prop:normalSys}
	The substitution  is a model of constraint system  if and only if  is a model of .
\begin{proof}
	By definition, 
	 is a model of , 
	iff , . 
	But by Lemma~\ref{lemma:normprop} we have that   and . 
	Thus,  is a model of  if and only if  is a model of .
\end{proof}

\end{prop}



\begin{prop}\label{prop:normalSigma}
	The substitution  is a model of constraint system  if and only if  is a model of .
\begin{proof}
	Proof is similar to one of Proposition~\ref{prop:normalSys}.
\end{proof}

\end{prop}






\subsection{Ground case of DY+ACI}\label{subs:groundcase}
In Algorithm~\ref{alg:solving} we need to check whether a ground substitution  satisfies a constraint system . For this, we have to check 
the derivability of a ground term from a set of ground terms. In this subsection we  present such an algorithm.

First, for the ground case we consider an  equivalent to DY+ACI deduction system DY+ACI' obtained from the first by replacing a set of rules 
 
with   
 


Now, we show an equivalence of the two deduction systems.
 
\begin{lemma}
	
\begin{proof}[Proof sketch] 

We show that every   rule of one deduction system can be simulated by a combination 
of rules from the other. It is sufficient to show it for non common rules. 

The DY+ACI' rules
	 
	are modeled by successive application of rules
	.
	The converse simulation of  by DY+ACI'
is based on getting all the normalized elements of  
	and, if  then reconstructing  by rule ,
	where  are .
\end{proof}

\end{lemma}

\begin{algorithm}[H]
  \caption{Verifying derivability of term}
  \label{alg:ground}
\KwIn{A normalized ground constraint }
  \KwOut{}
  \BlankLine
Let  \;
Let \;
\While{true}{ 
	\eIf{ exists DY rule , such that  and }{  \nllabel{step:DY}
		\;
		\;
}{
	\eIf{ exists }{  \nllabel{step:setcompos}
		\;
		\;
}
{
	\eIf{ exists }{ \nllabel{step:setdecompos}
		\;
		\;
}{
		\Return{ }\; \nllabel{step:ret}
}}}
}
\end{algorithm}


\begin{lemma}\label{lemma:propalgo}
	For Algorithm~\ref{alg:ground} the following statements are true:
	\begin{itemize}
		\item for any step\footnote{Consider two sequential assignments as one step},  and ;
		\item it terminates;
		\item for any step, .
	\end{itemize}

\end{lemma}

The following lemmas will be used to prove correctness of the algorithm. 

\begin{lemma}\label{lemma:strokerules} \
\begin{itemize}
	\item For any decomposition rule  of DY+ACI', if   is normalized, then  is a quasi-subterm of .
	\item For any composition rule  of DY+ACI' except , if  is normalized, then .
\end{itemize}
\end{lemma}




\begin{lemma}\label{lemma:walkout}
	After the execution of Step~\ref{step:ret} of Algorithm~\ref{alg:ground}, if  is a DY+ACI' rule, such that  and , then  is a composition rule and .
\begin{proof}
	Suppose,  is a decomposition.   By Lemma~\ref{lemma:strokerules} we have that  and thus, . Then   implies , and then, Step~\ref{step:ret} must be skipped, as branch \ref{step:DY} or \ref{step:setdecompos} should have been visited. 

	Thus,  is  a composition.
	 As algorithm reached  Step~\ref{step:ret}, that means  (otherwise one of three branches must be visited and this step would be skipped). As  and , we have .
\end{proof}

\end{lemma}

\begin{lemma}\label{lemma:composelems}
	Given a set of normalized terms  such that for any , .
	Then for any DY+ACI' composition rule  such that  we have .
\begin{proof}
	All cases of composition rules except  are trivial, as for them .
	For this case, as  for all , then (by Lemma~\ref{lemma:normprop}, Statement \ref{pDotNormElems})  \\ .
\end{proof}

\end{lemma}


\begin{prop}
 Algorithm~\ref{alg:ground} is correct.
\begin{proof}
	If algorithm returns true, then, by Lemma~\ref{lemma:propalgo}, .
	
	Show, that output is correct, if algorithm returns false. Note, that we consider values of  and  that they have after finishing the algorithm. Suppose that output is false (), but . Then there exists minimal by length derivation  where ,  (as  and ) and  and  and  for all .
	Then, applying Lemma~\ref{lemma:walkout} we have  is a composition, and .
	
	Let  be the smallest index such that there exists  and .
	
	Let  be the minimal integer, such that  is a decomposition.
	
	Show, . Suppose the opposite, then  is built by a chain of composition rules from . If  (where ) is
	\begin{itemize}
	 \item a rule in form of  , then  (otherwise it contradicts to minimality of the derivation) and from Lemma~\ref{lemma:composelems},  (, otherwise this step would be executed in the algorithm). As , then .
	If  then we got contradiction with with the fact, that this step would be executed in the algorithm. If there exists  and  (that means, ), then we get a contradiction with the minimality of , as  was deduced before.
	 \item any other composition rule, then by Lemma~\ref{lemma:strokerules}, , and thus, . Similarly to the previous case,  and we get a contradiction with either minimality of , or with the fact, that the algorithm would have to add  into .\end{itemize}
      
	Note, that this also shows, that decomposition rule is present in derivation.
	
	Show, . Suppose the opposite. Then by Lemma~\ref{lemma:strokerules}, we have  what contradicts to .
	Thus, at least one element from  is not from .
	Let us consider all possible decomposition rules :
	\begin{itemize}
		\item . We know, that  is not in , thus, it was built by composition. As  are normalized, the only possible way to build by composition  from normalized terms is  (other ways, like  would contradict the minimality of the derivation). 
		Thus,  was derived before (or was in ), i.e. . 
		That contradicts to .
		
		\item  . Similar case.
		
		\item . The case where  has similar explanations as two cases above. Thus, . That means,  and 
		 , i.e. . 
This means,  was derived before and , what contradicts to  .
		
\item  is a similar case to previous one. Note, that if  is not in , that it must be obtained by decomposition.
		
		\item , where  and . By Lemma~\ref{lemma:composelems}, , that contradicts minimality of derivation ().
	\end{itemize}

	
\end{proof}
\end{prop}



\subsection{Existence of conservative solutions}

In this subsection we will show that for any satisfiable constraint system, there exist a model in special form (so called conservative solution). 
Roughly speaking, a model in this form can be defined per each variable by set of quasi-subterms of the constraint system and set of atoms (also from the constraint system) that must be prived.
This will bound a search space for the model (see \S~\ref{subs:bounds}).

First, we show, that on quasi-subterms of constraint system instantiated with its model, the transformation  will be a homomorphism modulo normalization. \begin{prop}\label{prop:subt}
Given a normalized constraint system  and its  normalized model .
For all , .
 \begin{proof}
  We will prove it by induction on , where  is normalized. 
\begin{itemize}
 \item Let . Then:
	\begin{itemize}
 		\item either .  In this case , and as  for any substitution , then   and . Thus, .

		\item or . As  is a model and , we have , and, by definition, . Then, by definition of , .
	\end{itemize}

 \item Assume that for some  if , then .
 \item Show, that for any  such that , where  
or  
 or ,
but ,  and , for all ,
 statement   is still true. 
 We have: 
	\begin{itemize}
 		\item either . As  and . As   and from the induction assumption, we have . The same holds for .

		Again, since  (as  and ) we have that 
		 \br  \br  \br  \br  \br  \br  \br  \br .


		\item or . As  is normalized, 
		it implies that for all ,  are not in form of  and then , and thus, we have . 
		 
		 (by Statement~\ref{pPairingOfPairing} of Lemma~\ref{lemma:normprop}) \br
		  \br  \br  \br  \br 


		\item or . Then .
		
		 \br .
		
	\end{itemize}
\end{itemize}

Thus, the proposition is proven.

 \end{proof}

\end{prop}












Now we show, that relation of derivability between a term and a set of terms is stable with regard to transformation . 

\begin{lemma}\label{lemma:validstep}
Given a normalized constraint system  and its normalized model .
For any DY+ACI rule \ , \br .


\begin{proof}[Proof idea] 
We proceed by considering all possible deduction rules. To give an idea, we show a proof for only one rule (see full proof in \ref{app:lemma|validstep}):
  	. Here we have to show that  is derivable from .
	Consider two cases:
  	\begin{itemize}
		\item  such that . Then \br , \br and then 
		 \br . \br
		On the other hand,  \br .
		
		\item  such that . 
		 Then \br . Using Proposition~\ref{prop:pairing}, we have , thus (by Lemma~\ref{lemma:normprop}) . And then, by Proposition~\ref{prop:pairing} we have that . Therefore, by Lemma~\ref{lemma:dertrans},   .
	\end{itemize}
\end{proof}

\end{lemma}

Using Proposition~\ref{prop:subt} and Lemma~\ref{lemma:validstep} we will show, that transformation  preserves the property of substitution to be a model. 

\begin{theorem}\label{theorem:solution}
Given a normalized constraint system  and its normalized model .
Then substitution  also satisfies .
 \begin{proof}
 Suppose, without loss of generality, .
  Let us take any constraint . As  is a model of ,  there exists a derivation   such that  and . 

By Lemma~\ref{lemma:validstep} and Lemma~\ref{lemma:derext} we can easily prove that if , . 
Then, applying transitivity of  (Lemma~\ref{lemma:dertrans})  times, we have that . In the case where , the statement  is also true.


 Using Proposition~\ref{prop:subt} we get , as . The same for : , and as , we have . 
Thus, we have that , that means  satisfies any constraint of . 

 \end{proof}
\end{theorem}

From now till the end of subsection we will study a very useful property of .
Proposition~\ref{prop:mainprop} and its corollary show, 
that if constraint system has a normalized model () which sends different variables to different values, 
then there exists another normalized model ()  that  sends any variable of its domain
to an ACI-set of some non-variable quasi-subterms of constraint system instantiated by itself
and some private keys built with atoms of  the constraint system.

\begin{lemma}\label{lemma:goodSub}
	If ,  is normalized, ,   and ,
	then there exists  such that  and . 
	The similar is true in the case of  , ,  and for .
\LONG{
\begin{proof}
	As  and , we have:
\begin{itemize}
		\item  not in form of . Then, as  and , we have  (where  and ). 
		Then we can choose .


		\item , , as . Then, for all ,   is either a variable, or . 
		But, as    and as  is normalized, we can claim, that  contains at most one variable. 
		Then, as , there exists  such that . Then by definition of normalization function, and from  we have, that 
		 and as  is an element of , we have . 
		Thus, we can choose , as  and .
	\end{itemize}
	
	The other cases  (, , etc...) can be proved similarly.

\end{proof}
}
\end{lemma}



\begin{prop}\label{prop:mainprop}
 Given a normalized constraint system  
 and its normalized model  such that for all , . 
 Then  for all  there exist  and   such that  
and \br .

\begin{proof}
\raggedright
 By definition, . Let us take any  (note, that  is a ground term). Then, by definition of  we have:
\begin{itemize}
 \item either . Then, by definition of , . Thus, , , ;
 
 \item or  and there exists  such that . 
  As all conditions of Lemma~\ref{lemma:goodSub} are satisfied, then there exists  such that  and 
 
  and as  then .
  By Proposition~\ref{prop:subt}, . 
  That means that there exists   such that  and .
 


 \item or . In this case, as  is ground,  must be an atom, moreover, by definition of , this atom is from . 
	Therefore, , where  (and of course, ).
\end{itemize}

Thus,  for all , there exists . 
Therefore, as  , we have that , where    and .  That proves the proposition.
\end{proof}

\end{prop}



\begin{cor}\label{cor:existsgood}
	 Given normalized constraint system  and  --- its normalized model,   such that . Then there exists a normalized model  of   such that for all  there exist  and  such that  and , if ; .


\end{cor}
 Any normalized model with property shown in Corollary \ref{cor:existsgood} we will call \emph{conservative}.

\subsection{Bounds on conservative solutions}
\label{subs:bounds}

To get a decidability result, 
we first show an upper bound on size of conservative model 
and then, 
by reducing any satisfiable constraint system to one that have 
conservative model and showing that reduced one is smaller
(by size) than original one,
we obtain an existence of a model with bounded size for any 
satisfiable constraint system.


\begin{lemma}\label{lemma:subsigma}
	Given a normalized constraint system  and its conservative model . Then for all  we have .
\begin{proof}
	Given a ground substitution , let us define a  strict total order on variables: .
	
	By Proposition~\ref{prop:mainprop} for all  , where    and . 

Let us show that if  for some , then .
	Suppose, that  and . 
	Then  
	(by Lemma~\ref{lemma:normprop}) 
	 \br ,  
	because we know that  or  and  
	(for example, in first case,  \br  and   \br  since , 
	  using Statement~\ref{pNormSubSize} of Lemma~\ref{lemma:normprop}, 
	  we get   \br   )
	And as  
	That means, . Contradiction.
	
	Now we show by induction main property of this lemma.
	\begin{itemize}
		\item let . \br 
		Then  and 
		 all  \br are ground  (as there does not exists ). 
		Then .
 We have that , 
as for any ,  and  or  or , where , therefore   
and 
.

		\item
			Suppose, that for all  we have \br  .
		
		\item
			Show, that .
			We know that  
			and for any  , .
			Then we have  \br . We know that . Let us show that .
			By Lemma~\ref{lemma:normprop} 
we have  \br  \br . We can see that   (as  \br ); and by induction supposition and by statement proved above we have 
			 \br . \br
			Thus, .

	\end{itemize}

\end{proof}

\end{lemma}



\begin{prop}\label{prop:limit}
	For normalized constraint system  that have conservative  model , 
	for any  we have .
\begin{proof}
	As , we have (using the fact that  is normalized and  Lemma~\ref{lemma:subsigma}) 
	that  ;
thus, .
\end{proof}

\end{prop}

From this proposition and Corollary~\ref{cor:existsgood} we obtain an existence of bounded model 
for a normalized constraint system 
that have a model sending different variables to different values.
We will reduce an arbitrary constraint system to already studied case. 
The target properties are stated in Proposition~\ref{prop:generalLimit} and Corollary~\ref{cor:polynom}.

\begin{lemma}\label{lemma:sizesubst}
Given any constraint system  and any substitution  such that  and .	Then .
\LONG{
\begin{proof}
	From Lemma~\ref{lemma:normprop} we obtain , but  ( consists only of variables), and then  . As  , we have  . \br
	Thus, .
\end{proof}
}\end{lemma}



\begin{df}
	Let  and  be substitutions. Then  is a substitution such that  and for all , .
\end{df}


\begin{lemma}\label{lemma:substcomp}
	Let  and  be substitutions such that ,  and  is ground. Then, for any term , .
\LONG{
\begin{proof}


When apply  to , every variable  of  such that  is replaced by ; then we apply  to :  every variable  of  is replaced by , thus, every variable  from  will be replaced to  (as ); and no other variables will be replaced (as ). Thus, we can see that it is the same as in definition of .
\end{proof}
}\end{lemma}



\begin{prop} \label{prop:generalLimit}
  Given any satisfiable constraint system . Then there exists a model  of  such that for any ,  \begin{proof}[Proof idea]
Given a normalized model  of  we build a substitution  that maps different variables whose -instnatces are the same to one.
In this way we obtain a new constraint system and its normalized model on which we can apply Corollary~\ref{cor:existsgood} and get its conservative model ,
and by applying Proposition~\ref{prop:limit} we get a bound on size for this model. 
On the other part, we use Lemma~\ref{lemma:substcomp} to show that  is a model of .
And then, using obtained bound and Lemma~\ref{lemma:sizesubst} show existence of a model with stated property.
The detailed proof is given in \ref{app:prop|generalLimit}

\end{proof}

\end{prop}


\begin{cor}\label{cor:polynom}
Constraint system   is satisfiable if and only if there exists  a normalized model of  defined on  
which maps a variable to a ground term in   with size not greater than double .


\end{cor}

Using this result, we propose an algorithm of satisfiability of constraint system (Algorithm~\ref{alg:solving}).


\begin{algorithm}[H]
  \caption{Solving constraint system}
  \label{alg:solving}
  \SetKw{Guess}{Guess}
  \SetKw{Normalize}{Normalize}
\KwIn{A constraint system  }
  \KwOut{Model , if exists; otherwise }
  \BlankLine
\Guess for every variable of  a value of ground normalized substitution  with size not greater than \;
\eIf{  satisfies  for all }{
		\Return{ } }{\Return{  }}
\end{algorithm}






\begin{prop}
	Algorithm~\ref{alg:solving} is correct.
\begin{proof}
Let  be an output of Algorithm~\ref{alg:solving}. Then  is a ground substitution and  satisfies all constraints from  and therefore, satisfies all constraints from  . This means,  is a model of .
\end{proof}

\end{prop}

\begin{prop}
	Algorithm~\ref{alg:solving} is complete.
\begin{proof}
Suppose,  is satisfiable.  Then, by Corollary~\ref{cor:polynom}, there exists a guess of value of ground substitution on every element of   with size not greater than   which represents a model  of . Thus, algorithm~\ref{alg:solving} will return this .

\end{proof}

\end{prop}
