


\subsection{Function specifications}
The \textsc{searchNode}() function will return \textbf{true} whenever value 
has been inserted in one of the Tree () leaf node and that node's 
property is currently set to \textbf{false}. Or if  is placed on one of the 
Node's buffer located at the lowest level of . It returns \textbf{false} whenever
it couldn't find a leaf node with , or  couldn't be found in 
the last level . 

\textsc{insertNode}() will insert value  and returns \textbf{true} 
if there is no leaf node with , or
there is a leaf node  which satisfy  but with , 
or  is not found in the last 's .
In the other hand, \textsc{insertNode} returns \textbf{false} if there is a leaf node with 
 and , or  is found in .

For \textsc{deleteNode}(), a value of \textbf{true} is returned if there
is a leaf node with  and , or  is found in the
last 's . The value  will be then deleted.
In the other hand, \textsc{deleteNode} returns \textbf{false} if there is a leaf
node with  and , or  is not found in
.


\subsection{Synchronisation calls}
For synchronisation between update and maintenance operations, we define
\textsc{flagup}() that is doing atomic increment of  and also
a function that do atomic decrement of  as
\textsc{flagdown}().

\begin{figure}[!t] \centering 
\begin{algorithmic}[1] 
\small
\Function{waitandcheck}{}
    \Do
        \State \textsc{spinwait}() \label{lst:line:spinwait}
        \State \textsc{flagup}() 
        \State 
        \If{}
            \State \textsc{flagdown}() 
            \State 
        \EndIf
    \doWhile{}
\EndFunction
\end{algorithmic}
\caption{Wait and check algorithm}
\label{lst:waitandcheck}
\end{figure}

Also there is \textsc{spinwait}() that basically instruct a thread to
spin waiting while  value is \textbf{true}. Only \textsc{Merge} and 
\textsc{Rebalance} that will have to privilege to set  as \textbf{true}.
Lastly there is
\textsc{waitandcheck}() function (Figure
\ref{lst:waitandcheck}) that is preventing updates in getting mixed-up with maintenance
operations. The mechanism of \textsc{waitandcheck}()
will instruct a thread to wait at the tip of a current Node
whenever another thread has obtained a lock on that Node
for the purpose of doing any maintenance operations.

\begin{figure}[!t] \centering \begin{algorithmic}[1]
\small
\Start{\textbf{node} }
	\START{member fields:}
\State , if  indicates the node is root of a \par
\hskip\algorithmicindent Node with an id of  () 

\State , the node value, default is \textbf{empty} 

\State , a value of  indicates a logically \par \hskip\algorithmicindent deleted
node \State , left / right child pointers \State , indicates whether the \par \hskip\algorithmicindent node is a
leaf of a Node, default is  	\label{lst:line:leafdefault}

\END \End

\Statex
\Start{\textbf{Node} }
	\START{member fields:}
\State , a group of   amount of\par
\hskip\algorithmicindent pre-allocated node .
\State , an array of value with a length \par
\hskip\algorithmicindent  of the current number of threads \State
, an array of value with a length \par \hskip\algorithmicindent 
of the current number of threads \State , indicates whether a Node is
locked \State , semaphore for active update operations \State ,
pointer the root node of the Node \State , pointer to root node of
the Node's \par \hskip\algorithmicindent mirror \END \End

\Statex
\Start{\textbf{universe} }
	\START{member fields:}
\State , pointer to the  of the topmost Node \par
\hskip\algorithmicindent () \END \End
\end{algorithmic}
\caption{Cache friendly binary search tree structure} \label{lst:datastruct}
\end{figure}

\begin{figure}[!t] \centering \begin{algorithmic}[1]
\small
\Function{searchNode}{}
\State 
\While{ not end of tree  TRUE}	\label{lst:line:searchifleaf}
    \State 	\label{lst:line:lasnode-p}
        \If{}		\label{lst:line:searchless}
            \State 
        \Else					\label{lst:line:searchelse}
            \State 
        \EndIf
\EndWhile
\If{} \label{lst:line:linsearch3}
	\If{ FALSE}				\label{lst:line:linsearch1}
		\State \Return TRUE 
	\Else
		\State \Return FALSE
	\EndIf
\Else \State Search () for 	\label{lst:line:searchbuffer}
	\If{ is found}							\label{lst:line:linsearch2}
\State\Return TRUE \Else \State\Return FALSE \EndIf \EndIf \EndFunction
\end{algorithmic}
\caption{A wait-free searching algorithm of Tree}\label{lst:nodeSearch}
\end{figure}

\begin{figure}
\centering
\begin{algorithmic}[1]
\small

\Function{insertNode}{} \Comment{\parbox[t]{.5\linewidth}{Inserting an new item  into Tree }}
\State 
\State \Return \textsc{insertHelper}()
\EndFunction
\State

\Function{deleteNode}{} \Comment{\parbox[t]{.5\linewidth}{Deleting an item  from Tree }}
\State 
\State \Return \textsc{deleteHelper}()
\EndFunction
\State

\Function{deleteHelper}{}						
\State  TRUE
\If{Entering new Node }             \Comment{\parbox[t]{.5\linewidth}{Observed by examining  value change}}    			                                                 
   \State   \textsc{getParentNode}()
   \State  \textsc{flagdown}() 	\Comment{\parbox[t]{.5\linewidth}{Flagging down operation count on the previous/parent triangle}}			
   \State  \textsc{waitandcheck}(, )
   \State  \textsc{flagup}()
\EndIf

\If{( TRUE)} 			\Comment{Are we at leaf?}
        \If{}
            \If{\textsc{CAS}(, FALSE, TRUE) != FALSE)}              	\label{lst:line:markdel}       \Comment{Mark it delete}
                	\State  FALSE                                           		 \Comment{Unable to mark, already deleted}
            \Else	
             	\If{(=\textbf{empty}\&=\textbf{empty})} 	\label{lst:line:markdel-check} 
			\State 	
                		\State \textsc{mergeNode}(\textsc{parentOf}( TRUE      \Comment{Delete succeed, invoke merging}                 
		\Else		 
			\State \textsc{deleteHelper}()	 		\Comment{\parbox[t]{.5\linewidth}{Not leaf, re-try delete from }}
		\EndIf
            \EndIf
        \Else												
        	\State Search () for 
	   	\If{ is found in }
                 		\State 	\label{lst:line:bufdel} 
				\State 
        			\State 
            	\Else	
			\State \textsc{flagdown}()
            		\State  FALSE					\Comment{\parbox[t]{.5\linewidth}{Value not found}}
	    	\EndIf                                             
    	\EndIf
	\State \textsc{flagdown}()
\Else \If{}
        \State \textsc{deleteHelper}()
\Else
        \State \textsc{deleteHelper}()
\EndIf
\EndIf
\State \Return 
\EndFunction

\algstore{myalg}
\end{algorithmic}
\end{figure}

\begin{figure}
\centering
\begin{algorithmic} [1]                   \algrestore{myalg}
\small

\Function{insertHelper}{}
\State  TRUE
\If{Entering new Node }       				\Comment{\parbox[t]{.5\linewidth}{Observed by examining  change}}                                     	                      
   \State   \textsc{getParentNode}()	 
   \State  \textsc{flagdown}() 				\Comment{\parbox[t]{.5\linewidth}{Flagging down operation count on the previous/parent triangle}}
   \State  \textsc{waitandcheck}(, )
	\State  \textsc{flagup}()
\EndIf
  
 

   
\If{}                          \Comment{\parbox[t]{.5\linewidth}{At the lowest level of a Tree?}}                          
        \If{}                                                                              
            \If{( TRUE)}                                 
                \If{CAS(, \textbf{empty}, ) = \textbf{empty}}       \label{lst:line:ins-lp1} 	\label{lst:line:growins1} 
                	\State                                                        
                    \State 
                    \State  FALSE	
                    \State  \textsc{flagdown}()
                \Else
                    \State \textsc{insertHelper}(, )     \Comment{\parbox[t]{.5\linewidth}{Else try again to insert starting with the same node}}                     
                \EndIf
            \Else
                \State \textsc{insertHelper}(, )             \Comment{Not a leaf, proceed further to find the leaf}        
            \EndIf
        \ElsIf {}                                                                      
            \If{( TRUE)}                                  
                \If{CAS(, , ) =
                \textbf{empty}}  \label{lst:line:ins-lp2}  \label{lst:line:growins2}
                	\State                                                         
                    \State 
                    \State \textsc{atomic} 					\label{lst:line:atomic-isleaf}
                    	
                    	\State  FALSE
                    	
                  \State  \textsc{flagdown}()
                \Else
                    \State \textsc{insertHelper}(, )                      \Comment{\parbox[t]{.5\linewidth}{Else try again to insert starting with the same node}}     
                \EndIf
            \Else
                \State \textsc{insertHelper}(, )                	   \Comment{Not a leaf, proceed further to find the leaf}
            \EndIf
        \ElsIf {}                                                                                            
            \If{( TRUE)}                                   
                \If{ FALSE}				
                    \State  FALSE            				\Comment{\parbox[t]{.5\linewidth}{Duplicate Found}}
                   \State  \textsc{flagdown}()                                   
                \Else
                	   \State Goto \ref{lst:line:growins2} 	
                \EndIf
            \Else
                \State \textsc{insertHelper}()                    	 \Comment{Not a leaf, proceed further to find the leaf}      
            \EndIf
        \EndIf
\Else			
	\If{}                                                                            
            \If{}
                \State                                                
            \EndIf
        \Else 					\Comment{\parbox[t]{.5\linewidth}{All's failed, need to rebalance or expand the triangle }}
\algstore{myalg}
\end{algorithmic}
\end{figure}

\begin{figure}[!t]
\centering
\begin{algorithmic} [1]                   \algrestore{myalg}
\small
		\If{ already in }
			   
		\Else
\State \textit{put  inside } \label{lst:line:insertbuffer}
			\State 
        			\State 
\EndIf            
          	\If{\textsc{TAS}()}                                            			\Comment{\parbox[t]{.5\linewidth}{All threads try to lock }}

                
                
                		\State \textsc{flagdown}()                                      \Comment{\parbox[t]{.5\linewidth}{Make sure no flag is still raised}}
                		\State \textsc{spinwait}()                                           \Comment{\parbox[t]{.5\linewidth}{Now wait all insert/delete operations to finish}}
                
                		\State 
                		
                		\If {}                             \Comment{Expanding needed, }
                    		\State \ldots \textit{Create(a new triangle) AND attach it on the to the parent of } \ldots                   
                		\Else
                    		\If{ don't have triangle child(s)}
                        			\State  \textsc{rebalance}(, )
                        			\State \textsc{switchtree}(, )
                        			\State 
                    		\Else
                        			\If{}
                            			\State Fill  with all value in  	\Comment{\parbox[t]{.2\linewidth}{Do non-blocking insert here}}
                            			\State 
                        			\EndIf                             
				\EndIf
                		\EndIf	
			\State \textsc{spinunlock}()   
		\Else
			\State \textsc{flagdown}() 
		\EndIf    
	\EndIf                                  
\EndIf
\State \Return 
\EndFunction
\end{algorithmic}
\caption{Update algorithms and their helpers functions}\label{lst:pseudo-ops}
\end{figure}

\begin{figure}[!t]
\begin{algorithmic}[1]
\small
\Procedure{balanceTree}{}
\State Array  Traverse() \Comment{\parbox[t]{.5\linewidth}{Traverse all the non-empty node into  array}}
\State RePopulate() \Comment{\parbox[t]{.5\linewidth}{Re-populate the tree  with all the value from  recursively. RePopulate will resulting a balanced tree }}
\EndProcedure


\Procedure{mergeTree}{}
\State  \textsc{parentOf}()

\If{}
	\State   
\Else
	\State 
\EndIf

\State  \textsc{triangleOf}()			\Comment{\parbox[t]{.5\linewidth}{Get the  of  node}}
\State  \textsc{triangleOf}()			\Comment{\parbox[t]{.5\linewidth}{Get the  of }}
\State  \textsc{triangleOf}()			\Comment{\parbox[t]{.5\linewidth}{Get the  of }}

\If{\textsc{spintrylock}()}                \Comment{\parbox[t]{.5\linewidth}{Try to lock the current triangle}}
      	\State \textsc{spinlock}()	 \Comment{\parbox[t]{.5\linewidth}{lock the sibling triangles}}
	\State \textsc{flagdown}() 
	\State \textsc{spinwait}()	 \Comment{\parbox[t]{.3\linewidth}{Wait for all insert/delete operations to finish}}

	\State 

	\If{( don't have children) }
		\State MERGE  into  
		\If{}				\Comment{\parbox[t]{.5\linewidth}{Now re-do the pointer}}
			\State   	\Comment{\parbox[t]{.5\linewidth}{Merge Left}}
		\Else
			\State 	\Comment{\parbox[t]{.5\linewidth}{Merge Right}}
		\EndIf
	\EndIf
   	\State \textsc{spinunlock}()
\Else
	\State \textsc{flagdown}() 
\EndIf

\EndProcedure
\end{algorithmic}
\caption{Merge and Balance algorithm}\label{mergeTree}
\end{figure}


\subsection{Wait-free and Linearisability of search}

\begin{lemma}
Tree search operation is wait-free.
\end{lemma}

\begin{proof}(Sketch) In the searching algorithm (cf. Figure
\ref{lst:nodeSearch}), the Tree will be traversed from the root node
using iterative steps. When at , the value to search  is compared to
. If , the left side of the tree will be traversed
by setting  (line \ref{lst:line:searchless}), in contrary
 will cause the right side of the tree to be traversed
further (line \ref{lst:line:searchelse}). The procedure will repeat until a leaf
has been found () in line \ref{lst:line:searchifleaf}.

If the value  couldn't be found and search has reached the end of 
Tree, a buffer search will be conducted
(line \ref{lst:line:searchbuffer}).
This search is done by simply searching the buffer array from left-to-right
to find , therefore no waiting will happen in this phase.

The \textsc{deleteNode} and \textsc{insertNode} algorithms (Figure
\ref{lst:pseudo-ops}) are non-intrusive to the structure of a tree, thus they
won't interfere with an ongoing search. A \textsc{deleteNode} operation, if
succeeded, is only going to mark a node by setting a  variable as
\textbf{true} (line \ref{lst:line:markdel} in Figure \ref{lst:pseudo-ops}).
The  is retained so that a search will be able to proceed further.
For \textsc{insertNode}, it can \textit{"grow"} the current leaf node as it
needs to lays down two new leaves (lines  \ref{lst:line:growins1} and
\ref{lst:line:growins2} in Figure \ref{lst:pseudo-ops}), however the operation
never changes the internal pointer structure of a Node, since
Node internal tree structure is pre-allocated beforehand, allowing a
search to keep moving forward.
As depicted in Figure \ref{fig:treetransform}(a), after an insertion of 
grows the node, the old node (now ) still contains the same value as 
(assuming ), thus a search still can be directed to find either  or
.
 The \textsc{rebalance/Merge} operation is also not an obstacle for searching
 since its operating on a mirror Node.
\end{proof}

We have designed the searching to be linearisable in various concurrent
operation scenarios (Lemma \ref{lem:linear-search}). This applies as well to the
update operations.

\begin{lemma} 
For a value that resides on the leaf node of a Node,
\textsc{searchNode} operation (Figure \ref{lst:nodeSearch}) has the linearisation 
point to \textsc{deleteNode} at line \ref{lst:line:linsearch1} and the
linearisation point to \textsc{insertNode} at line \ref{lst:line:linsearch3}. 
For a value that stays in the buffer of a Node, \textsc{searchNode} 
operation has the \textit{linearisation point} at line \ref{lst:line:linsearch2}.
\label{lem:linear-search}
\end{lemma}
\begin{proof}(Sketch) It is trivial to demonstrate this in relation to
deletion algorithm in Figure \ref{lst:pseudo-ops} since only an atomic operation
is responsible for altering the  property of a node (line
\ref{lst:line:markdel}).
Therefore \textsc{deleteNode} has the linearisation point to \textsc{searchNode} 
at line \ref{lst:line:markdel}. 


For \textsc{searchNode} interaction with an insertion that grows new subtree, we
rely on the facts that: 1) a snapshot of the current node  is recorded on
 as a first step of searching iteration (Figure \ref{lst:nodeSearch},
line \ref{lst:line:lasnode-p}); 2)  change, if needed, is not done
until the last step of the insertion routine for insertion of 
and will be done in one atomic step with  change (Figure
\ref{lst:pseudo-ops}, line \ref{lst:line:atomic-isleaf}); and 3) 
property of all internal nodes are by default \textbf{true} (Figure
\ref{lst:datastruct}, line \ref{lst:line:leafdefault}) to guarantee that values
that are inserted are always found, even when the leaf-growing (both
left-and-right) are happening concurrently. Therefore
\textsc{insertNode} has the linearisation point to \textsc{searchNode} at line 
\ref{lst:line:ins-lp1} when inserting a value  smaller than the leaf node's 
value, or at line \ref{lst:line:ins-lp2} otherwise. 


A search procedure is also able to cope well with a "buffered" insert, that is
if an insert thread loses a competition in locking a Node for expanding
or rebalancing and had to dump its carried value inside a buffer (Figure
\ref{lst:pseudo-ops}, line \ref{lst:line:insertbuffer}). Any value inserted to
the buffer is guaranteed to be found. This is because after a leaf 
has been located, the search is going to evaluate whether the 
is equal to . Failed comparison will cause the search to look further inside
a buffer () located in a Node where the leaf resides
(Figure \ref{lst:nodeSearch}, line \ref{lst:line:searchbuffer}).
By assuring  that the switching of a root Node with its mirror includes
switching  with , we can show that any new
values that might be placed inside a buffer are guaranteed to be found
immediately after the completion of their respective insert procedures.
The "buffered" insert has the linearisation point to \textsc{searchNode} at
line \ref{lst:line:insertbuffer}.

Similarly, deleting a value from a buffer is as trivial as exchanging that value
inside a buffer with an \textbf{empty} value. The search operation will failed
to find that value when doing searching inside a buffer of Node. This
type of delete has the linearisation point to \textsc{searchNode} at the same
line it's emptying a value inside the buffer (line \ref{lst:line:bufdel}).
\end{proof}

\subsection{Non-blocking Update Operations} 

\begin{lemma}
Tree Insert and Delete operations are
non-blocking to each other in the absence of maintenance operations.
\end{lemma}

\begin{proof} (Sketch)
Non-blocking update operations supported by Tree are possible by
assuming that any of the updates are not invoking \textsc{Rebalance} and
\textsc{Merge} operations. 
In a case of concurrent insert operations (Figure \ref{lst:pseudo-ops}) 
at the
same leaf node , assuming all insert threads need to "grow" the node 
(for illustration, cf. Figure \ref{fig:treetransform}), they
will have to do \textsc{CAS}() 
(line  \ref{lst:line:growins1} and \ref{lst:line:growins2}) as their
first step. This CAS is the only thing needed since the whole Node
structure is pre-allocated and the CAS is an atomic operation. Therefore, only
one thread will succeed in changing  and proceed populating the
 node. Other threads will fail the CAS operation and they are going to
try restart the insert procedure all over again, starting from the node .

To assure that the marking delete (line \ref{lst:line:markdel}) behaves nicely
with the "grow" insert operations, \textsc{deleteNode}() that has found
the leaf node  with a value equal to , will need to check again whether
the node is still a leaf (line \ref{lst:line:markdel-check}) after completing
\textsc{CAS}().
The thread needs to restart the delete process from  if it has found that 
is no longer a leaf node.

The absence of maintenance operations means that a Node  is 
never set to \textbf{true}, thus either insert/delete operations are never blocked
at the execution of line number \ref{lst:line:growins2} in Figure \ref{lst:waitandcheck}.
\end{proof}

\begin{lemma} 
In Figure \ref{lst:pseudo-ops}, \textsc{insertNode} operation has the linearisation 
point against \textsc{deleteNode} at line \ref{lst:line:growins1} and line \ref{lst:line:growins2}. 
Whereas \textsc{deleteNode} has a linearisation point at line \ref{lst:line:markdel-check}
against an \textsc{insertNode} operation. 
For inserting and deleting into a buffer of a Node, an \textsc{insertNode} 
operation has the \textit{linearisation point} at line \ref{lst:line:insertbuffer}. While
\textsc{deleteNode} has its linearisation point at line \ref{lst:line:bufdel}.
\label{lem:linear-ins-del}
\end{lemma}
\begin{proof}(Sketch) An \textsc{insertNode} operation will do a \textsc{CAS} on the left node
as its first step after finding a suitable  for growing a subtree. If value 
is lower than  , the correspondent operation is the line \ref{lst:line:growins1}.
Line \ref{lst:line:growins2} is executed in other conditions. A \textsc{deleteNode} will always
check a  is still a leaf by ensuring  as \textbf{empty}
(line \ref{lst:line:markdel-check}). This is done after it tries to mark that . If the comparison
on line \ref{lst:line:markdel-check} returns \textbf{true}, the operation finishes successfully.
A \textbf{false} value will instruct the \textsc{insertNode} to retry again, starting from
the current node.

A buffered insert and delete are operating on the same buffer. When a value  is put inside
a buffer it will always available for delete. And that goes the opposite for the deletion case.  
\end{proof}


\subsection{Memory Transfer and Time Complexities} 
 
In this subsection, we will show that Tree is relaxed cache oblivious
and the overhead of maintenance operations (e.g. rebalancing, expanding and
merging) is negligible for big trees. The memory transfer analysis is based on
the ideal-cache model \cite{Frigo:1999:CA:795665.796479}. Namely, re-accessing
data in cache due to re-trying in non-blocking approaches incurs no memory
transfer.

For the following analysis, we assume that values to be searched, inserted or
deleted are randomly chosen. As Tree is a binary search tree (BST),
which is embedded in the dynamic vEB layout, the expected height of a randomly
built Tree of size  is  \cite{CormenSRL01}.

\begin{lemma}
A search in a randomly built Tree needs  expected memory
transfers, where  and  is the tree size and the {\em unknown} memory
block size in the ideal cache model \cite{Frigo:1999:CA:795665.796479}, 
respectively. 
\label{lem:DeltaTree_search}
\end{lemma}
\begin{proof}(Sketch)
Similar to the proof of Lemma \ref{lem:dynamic_vEB_search}, let  be the
coarsest levels of detail such that every recursive subtree contains at most
 nodes or  nodes, respectively. Since , . There are
at most  subtrees along to the search path in a Node and no
subtree of depth  is split due to the boundary of Nodes (cf. Figure
\ref{fig:search_complexity}). Since every subtree of depth  fits in a
Node of size , the subtree is stored in at most 2 memory blocks of
size .

Since a subtree of height  contains more than  nodes, 
, or . 

Since a randomly built Tree has an expected height of , there
are  subtrees of depth  are traversed in a search and
thereby at most   memory
blocks are transferred.

As , expected memory transfers in a search are .
\end{proof}

\begin{lemma}
Insert and Delete operations within the Tree are having a similar 
amortised time complexity of , where  is the size of Tree,
and  is the maximum size of element stored in Node.
\label{lem:amortised_cost}
\end{lemma}

\begin{proof} (Sketch) An insertion operation at Tree is tightly coupled
with the rebalancing and expanding algorithm. 

We assume that Tree was built using random values, therefore the 
expected height is . 
Thus, an insertion on a Tree costs . 
Rebalancing after insertion only happens
at single Node, and it has an upper bound cost of ,
because it has to read all the stored elements, sort it out and re-insert it in a balanced fashion.  
In the worst possible case for Tree, there will be an  insertion 
that cost  and there is at most  rebalancing operations with a cost of
 each. 

Using aggregate analysis, we let total cost for insertion 
as . Therefore the amortised time complexity 
for insert is . If we have defined  such as 
, the amortised time complexity for inserting a value into Tree 
is now becoming .

For the expanding scenarios, an insertion will trigger \textsc{expand}()
whenever an insertion of  in a Node  is resulting on  and
. An expanding will require a memory allocation
of a -sized Node, cost merely , together with two pointer
alterations that cost  each. In conclusion, we have shown that
the total amortised cost for insertion, that is incorporating both rebalancing
and expanding procedures as .

In the deletion case, right after a deletion on a particular Node 
will trigger a merging of that Node with its sibling in a condition 
of at least one of the Nodes is filled less than half
of its maximum capacity () and all values from both 
Nodes can fit into a single Node.

Similar to insertion, a deletion in Tree costs . However merging that combines
2 Nodes costs  at maximum. Using aggregate analysis, 
the total cost of deletion could be formulated as
. The amortised time complexity is 
therefore  or , if .

\end{proof}



