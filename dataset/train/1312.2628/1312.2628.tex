


\subsection{Function specifications}
The \textsc{searchNode}($v, U$) function will return \textbf{true} whenever value $v$
has been inserted in one of the $\Delta$Tree ($U$) leaf node and that node's $mark$
property is currently set to \textbf{false}. Or if $v$ is placed on one of the 
$\Delta$Node's buffer located at the lowest level of $U$. It returns \textbf{false} whenever
it couldn't find a leaf node with $value=v$, or $v$ couldn't be found in 
the last level $T_{tid}.rootbuffer$. 

\textsc{insertNode}($v, U$) will insert value $v$ and returns \textbf{true} 
if there is no leaf node with $value=v$, or
there is a leaf node $x$ which satisfy $x.value=v$ but with $x.mark=\textbf{true}$, 
or $v$ is not found in the last $T_{tid}$'s $rootbuffer$.
In the other hand, \textsc{insertNode} returns \textbf{false} if there is a leaf node with 
$value=v$ and $mark=\textbf{false}$, or $v$ is found in $T_{tid}.rootbuffer$.

For \textsc{deleteNode}($v, U$), a value of \textbf{true} is returned if there
is a leaf node with $value=v$ and $mark=\textbf{false}$, or $v$ is found in the
last $T_{tid}$'s $rootbuffer$. The value $v$ will be then deleted.
In the other hand, \textsc{deleteNode} returns \textbf{false} if there is a leaf
node with $value=v$ and $mark=\textbf{true}$, or $v$ is not found in
$T_{tid}.rootbuffer$.


\subsection{Synchronisation calls}
For synchronisation between update and maintenance operations, we define
\textsc{flagup}($opcount$) that is doing atomic increment of $opcount$ and also
a function that do atomic decrement of $opcount$ as
\textsc{flagdown}($opcount$).

\begin{figure}[!t] \centering 
\begin{algorithmic}[1] 
\small
\Function{waitandcheck}{$lock, opcount$}
    \Do
        \State \textsc{spinwait}($lock$) \label{lst:line:spinwait}
        \State \textsc{flagup}($opcount$) 
        \State $repeat \gets \textbf{false}$
        \If{$lock=true$}
            \State \textsc{flagdown}($opcount$) 
            \State $repeat \gets \textbf{true}$
        \EndIf
    \doWhile{$repeat=\textsc{true}$}
\EndFunction
\end{algorithmic}
\caption{Wait and check algorithm}
\label{lst:waitandcheck}
\end{figure}

Also there is \textsc{spinwait}($lock$) that basically instruct a thread to
spin waiting while $lock$ value is \textbf{true}. Only \textsc{Merge} and 
\textsc{Rebalance} that will have to privilege to set $T_x.lock$ as \textbf{true}.
Lastly there is
\textsc{waitandcheck}($lock, opcount$) function (Figure
\ref{lst:waitandcheck}) that is preventing updates in getting mixed-up with maintenance
operations. The mechanism of \textsc{waitandcheck}($lock, opcount$)
will instruct a thread to wait at the tip of a current $\Delta$Node
whenever another thread has obtained a lock on that $\Delta$Node
for the purpose of doing any maintenance operations.

\begin{figure}[!t] \centering \begin{algorithmic}[1]
\small
\Start{\textbf{node} $n$}
	\START{member fields:}
\State $tid \in \mathbb{N}$, if $> 0$ indicates the node is root of a \par
\hskip\algorithmicindent $\Delta$Node with an id of $tid$ ($T_{tid}$) 

\State $value \in
\mathbb{N}$, the node value, default is \textbf{empty} 

\State $mark \in
\{true,false\}$, a value of $true$ indicates a logically \par \hskip\algorithmicindent deleted
node \State $left, right \in \mathbb{N}$, left / right child pointers \State $isleaf
\in {true,false}$, indicates whether the \par \hskip\algorithmicindent node is a
leaf of a $\Delta$Node, default is $true$ 	\label{lst:line:leafdefault}

\END \End

\Statex
\Start{\textbf{$\Delta$Node} $T$}
	\START{member fields:}
\State $nodes$, a group of  $(|T| \times 2)$ amount of\par
\hskip\algorithmicindent pre-allocated node $n$.
\State $rootbuffer$, an array of value with a length \par
\hskip\algorithmicindent  of the current number of threads \State
$mirrorbuffer$, an array of value with a length \par \hskip\algorithmicindent 
of the current number of threads \State $lock$, indicates whether a $\Delta$Node is
locked \State $flag$, semaphore for active update operations \State $root$,
pointer the root node of the $\Delta$Node \State $mirror$, pointer to root node of
the $\Delta$Node's \par \hskip\algorithmicindent mirror \END \End

\Statex
\Start{\textbf{universe} $U$}
	\START{member fields:}
\State $root$, pointer to the $root$ of the topmost $\Delta$Node \par
\hskip\algorithmicindent ($T_1.root$) \END \End
\end{algorithmic}
\caption{Cache friendly binary search tree structure} \label{lst:datastruct}
\end{figure}

\begin{figure}[!t] \centering \begin{algorithmic}[1]
\small
\Function{searchNode}{$v, U$}
\State $lastnode, p \gets U.root$
\While{$p \neq$ not end of tree $\And p.isleaf \neq$ TRUE}	\label{lst:line:searchifleaf}
    \State $lastnode \gets p$	\label{lst:line:lasnode-p}
        \If{$p.value < v$}		\label{lst:line:searchless}
            \State $p \gets p.left$
        \Else					\label{lst:line:searchelse}
            \State $p \gets p.right$
        \EndIf
\EndWhile
\If{$lastnode.value = v$} \label{lst:line:linsearch3}
	\If{$lastnode.mark =$ FALSE}				\label{lst:line:linsearch1}
		\State \Return TRUE 
	\Else
		\State \Return FALSE
	\EndIf
\Else \State Search ($T_{tid}.rootbuffer$) for $v$	\label{lst:line:searchbuffer}
	\If{$v$ is found}							\label{lst:line:linsearch2}
\State\Return TRUE \Else \State\Return FALSE \EndIf \EndIf \EndFunction
\end{algorithmic}
\caption{A wait-free searching algorithm of $\Delta$Tree}\label{lst:nodeSearch}
\end{figure}

\begin{figure}
\centering
\begin{algorithmic}[1]
\small

\Function{insertNode}{$v,U$} \Comment{\parbox[t]{.5\linewidth}{Inserting an new item $v$ into $\Delta$Tree $U$}}
\State $t \gets U.root$
\State \Return \textsc{insertHelper}($v,t$)
\EndFunction
\State

\Function{deleteNode}{$v,T$} \Comment{\parbox[t]{.5\linewidth}{Deleting an item $v$ from $\Delta$Tree $U$}}
\State $t \gets U.root$
\State \Return \textsc{deleteHelper}($v,t$)
\EndFunction
\State

\Function{deleteHelper}{$v, node$}						
\State $success \gets$ TRUE
\If{Entering new $\Delta$Node $T_x$}             \Comment{\parbox[t]{.5\linewidth}{Observed by examining $x \gets node.tid$ value change}}    			                                                 
   \State  $T'_x \gets$ \textsc{getParent$\Delta$Node}($T_x$)
   \State  \textsc{flagdown}($T'_x.opcount$) 	\Comment{\parbox[t]{.5\linewidth}{Flagging down operation count on the previous/parent triangle}}			
   \State  \textsc{waitandcheck}($T_x.lock$, $T_x.opcount$)
   \State  \textsc{flagup}($T_x.opcount$)
\EndIf

\If{($node.isleaf =$ TRUE)} 			\Comment{Are we at leaf?}
        \If{$node.value = v$}
            \If{\textsc{CAS}($node.mark$, FALSE, TRUE) != FALSE)}              	\label{lst:line:markdel}       \Comment{Mark it delete}
                	\State $success \gets$ FALSE                                           		 \Comment{Unable to mark, already deleted}
            \Else	
             	\If{($node.left.value$=\textbf{empty}\&$node.right.value$=\textbf{empty})} 	\label{lst:line:markdel-check} 
			\State $T_x.bcount \gets T_x.bcount - 1 $	
                		\State \textsc{mergeNode}(\textsc{parentOf}($T_x)) \gets$ TRUE      \Comment{Delete succeed, invoke merging}                 
		\Else		 
			\State \textsc{deleteHelper}($v, node$)	 		\Comment{\parbox[t]{.5\linewidth}{Not leaf, re-try delete from $node$}}
		\EndIf
            \EndIf
        \Else												
        	\State Search ($T_{x}.rootbuffer$) for $v$
	   	\If{$v$ is found in $T_{x}.rootbuffer.idx$}
                 		\State $T_{x}.rootbuffer.idx \gets \textbf{empty}$	\label{lst:line:bufdel} 
				\State $T_x.bcount\gets T_x.bcount - 1$
        			\State $T_x.countnode\gets T_x.countnode - 1$
            	\Else	
			\State \textsc{flagdown}($T_x.opcount$)
            		\State $success \gets$ FALSE					\Comment{\parbox[t]{.5\linewidth}{Value not found}}
	    	\EndIf                                             
    	\EndIf
	\State \textsc{flagdown}($T_x.opcount$)
\Else \If{$v < node.value$}
        \State \textsc{deleteHelper}($v, node.left$)
\Else
        \State \textsc{deleteHelper}($v, node.right$)
\EndIf
\EndIf
\State \Return $success$
\EndFunction

\algstore{myalg}
\end{algorithmic}
\end{figure}

\begin{figure}
\centering
\begin{algorithmic} [1]                   \algrestore{myalg}
\small

\Function{insertHelper}{$v, node$}
\State $success \gets$ TRUE
\If{Entering new $\Delta$Node $T_x$}       				\Comment{\parbox[t]{.5\linewidth}{Observed by examining $x \gets node.tid$ change}}                                     	                      
   \State  $T'_x \gets$ \textsc{getParent$\Delta$Node}($T_x$)	 
   \State  \textsc{flagdown}($T'_x.opcount$) 				\Comment{\parbox[t]{.5\linewidth}{Flagging down operation count on the previous/parent triangle}}
   \State  \textsc{waitandcheck}($T_x.lock$, $T_x.opcount$)
	\State  \textsc{flagup}($T_x.opcount$)
\EndIf
  
 

   
\If{$node.left \And node.right$}                          \Comment{\parbox[t]{.5\linewidth}{At the lowest level of a $\Delta$Tree?}}                          
        \If{$v < node.value$}                                                                              
            \If{($node.isleaf =$ TRUE)}                                 
                \If{CAS($node.left.value$, \textbf{empty}, $v$) = \textbf{empty}}       \label{lst:line:ins-lp1} 	\label{lst:line:growins1} 
                	\State $node.right.value \gets node.value$                                                       
                    \State $node.right.mark \gets node.mark$
                    \State $node.isleaf \gets$ FALSE	
                    \State  \textsc{flagdown}($T_x.opcount$)
                \Else
                    \State \textsc{insertHelper}($v$, $node$)     \Comment{\parbox[t]{.5\linewidth}{Else try again to insert starting with the same node}}                     
                \EndIf
            \Else
                \State \textsc{insertHelper}($v$, $node.left$)             \Comment{Not a leaf, proceed further to find the leaf}        
            \EndIf
        \ElsIf {$v > node.value$}                                                                      
            \If{($node.isleaf =$ TRUE)}                                  
                \If{CAS($node.left.value$, $\textbf{empty}$, $v$) =
                \textbf{empty}}  \label{lst:line:ins-lp2}  \label{lst:line:growins2}
                	\State $node.right.value \gets v$                                                        
                    \State $node.left.mark \gets node.mark$
                    \State \textsc{atomic} $\{$					\label{lst:line:atomic-isleaf}
                    	$node.value \gets v$
                    	\State $node.isleaf \gets$ FALSE
                    $\}$	
                  \State  \textsc{flagdown}($T_x.opcount$)
                \Else
                    \State \textsc{insertHelper}($v$, $node$)                      \Comment{\parbox[t]{.5\linewidth}{Else try again to insert starting with the same node}}     
                \EndIf
            \Else
                \State \textsc{insertHelper}($v$, $node.right$)                	   \Comment{Not a leaf, proceed further to find the leaf}
            \EndIf
        \ElsIf {$v = node.value$}                                                                                            
            \If{($node.isleaf =$ TRUE)}                                   
                \If{$node.mark =$ FALSE}				
                    \State $success \gets$ FALSE            				\Comment{\parbox[t]{.5\linewidth}{Duplicate Found}}
                   \State  \textsc{flagdown}($T_x.opcount$)                                   
                \Else
                	   \State Goto \ref{lst:line:growins2} 	
                \EndIf
            \Else
                \State \textsc{insertHelper}($v, node.right$)                    	 \Comment{Not a leaf, proceed further to find the leaf}      
            \EndIf
        \EndIf
\Else			
	\If{$val = node.value$}                                                                            
            \If{$node.mark = 1$}
                \State $success \gets FALSE$                                               
            \EndIf
        \Else 					\Comment{\parbox[t]{.5\linewidth}{All's failed, need to rebalance or expand the triangle $T_x$}}
\algstore{myalg}
\end{algorithmic}
\end{figure}

\begin{figure}[!t]
\centering
\begin{algorithmic} [1]                   \algrestore{myalg}
\small
		\If{$v$ already in $T_x.rootbuffer$}
			$success \gets FALSE$   
		\Else
\State \textit{put $v$ inside $T_x.rootbuffer$} \label{lst:line:insertbuffer}
			\State $T_x.bcount\gets T_x.bcount + 1$
        			\State $T_x.countnode\gets T_x.countnode + 1$
\EndIf            
          	\If{\textsc{TAS}($T_x.lock$)}                                            			\Comment{\parbox[t]{.5\linewidth}{All threads try to lock $T_x$}}

                
                
                		\State \textsc{flagdown}($T_x.opcount$)                                      \Comment{\parbox[t]{.5\linewidth}{Make sure no flag is still raised}}
                		\State \textsc{spinwait}($T_x.opcount$)                                           \Comment{\parbox[t]{.5\linewidth}{Now wait all insert/delete operations to finish}}
                
                		\State $total \gets T_x.countnode + T_x.bcount$
                		
                		\If {$total * 4 \geqslant U.maxnode + 1$}                             \Comment{Expanding needed, $density > 0.5$}
                    		\State \ldots \textit{Create(a new triangle) AND attach it on the to the parent of $node$} \ldots                   
                		\Else
                    		\If{$T_x$ don't have triangle child(s)}
                        			\State $T_x.mirror \gets$ \textsc{rebalance}($T_x.root$, $T_x.rootbuffer$)
                        			\State \textsc{switchtree}($T_x.root$, $T_x.mirror$)
                        			\State $T_x.b_count \gets 0$
                    		\Else
                        			\If{$T_x.bcount > 0$}
                            			\State Fill $childA$ with all value in $T_x.rootbuffer$ 	\Comment{\parbox[t]{.2\linewidth}{Do non-blocking insert here}}
                            			\State $T_x.bcount \gets 0$
                        			\EndIf                             
				\EndIf
                		\EndIf	
			\State \textsc{spinunlock}($T_x.lock$)   
		\Else
			\State \textsc{flagdown}($T_x.opcount$) 
		\EndIf    
	\EndIf                                  
\EndIf
\State \Return $success$
\EndFunction
\end{algorithmic}
\caption{Update algorithms and their helpers functions}\label{lst:pseudo-ops}
\end{figure}

\begin{figure}[!t]
\begin{algorithmic}[1]
\small
\Procedure{balanceTree}{$T$}
\State Array $temp[|H|]\gets$ Traverse($T$) \Comment{\parbox[t]{.5\linewidth}{Traverse all the non-empty node into $temp$ array}}
\State RePopulate($T, temp$) \Comment{\parbox[t]{.5\linewidth}{Re-populate the tree $T$ with all the value from $temp$ recursively. RePopulate will resulting a balanced tree $T$}}
\EndProcedure


\Procedure{mergeTree}{$root$}
\State $parent \gets$ \textsc{parentOf}($root$)

\If{$parent.left = root$}
	\State $sibling \gets parent.right$  
\Else
	\State $sibling \gets parent.left$
\EndIf

\State $T_r \gets$ \textsc{triangleOf}($root$)			\Comment{\parbox[t]{.5\linewidth}{Get the $T$ of $root$ node}}
\State $T_p \gets$ \textsc{triangleOf}($parent$)			\Comment{\parbox[t]{.5\linewidth}{Get the $T$ of $parent$}}
\State $T_s \gets$ \textsc{triangleOf}($sibling$)			\Comment{\parbox[t]{.5\linewidth}{Get the $T$ of $sibling$}}

\If{\textsc{spintrylock}($T_r.lock$)}                \Comment{\parbox[t]{.5\linewidth}{Try to lock the current triangle}}
      	\State \textsc{spinlock}($T_s.lock, T_p.lock$)	 \Comment{\parbox[t]{.5\linewidth}{lock the sibling triangles}}
	\State \textsc{flagdown}($T_r.opcount$) 
	\State \textsc{spinwait}($T_r.opcount, T_s.opcount, T_p.opcount$)	 \Comment{\parbox[t]{.3\linewidth}{Wait for all insert/delete operations to finish}}

	\State $total \gets T_s.nodecount + T_s.bcount + T_r.nodecount + T_r.bcount$

	\If{($T_s \And T_r$ don't have children) $\And (T_p \geqslant U.maxnode + 1) / 2) \| T_s \geqslant U.maxnode + 1) / 2)) \And total \leqslant (U.maxnode + 1) / 2$}
		\State MERGE $T_r.root , T_r.rootbuffer, T_s.rootbuffer$ into $T.s$ 
		\If{$parent.left = root$}				\Comment{\parbox[t]{.5\linewidth}{Now re-do the pointer}}
			\State $parent.left \gets root.left$  	\Comment{\parbox[t]{.5\linewidth}{Merge Left}}
		\Else
			\State $parent.right \gets root.right$	\Comment{\parbox[t]{.5\linewidth}{Merge Right}}
		\EndIf
	\EndIf
   	\State \textsc{spinunlock}($T_r.lock, T_s.lock, T_p.lock$)
\Else
	\State \textsc{flagdown}($T_r.opcount$) 
\EndIf

\EndProcedure
\end{algorithmic}
\caption{Merge and Balance algorithm}\label{mergeTree}
\end{figure}


\subsection{Wait-free and Linearisability of search}

\begin{lemma}
$\Delta$Tree search operation is wait-free.
\end{lemma}

\begin{proof}(Sketch) In the searching algorithm (cf. Figure
\ref{lst:nodeSearch}), the $\Delta$Tree will be traversed from the root node
using iterative steps. When at $root$, the value to search $v$ is compared to
$root.value$. If $v < root.value$, the left side of the tree will be traversed
by setting $root \gets root.left$ (line \ref{lst:line:searchless}), in contrary
$v \geqslant root.value$ will cause the right side of the tree to be traversed
further (line \ref{lst:line:searchelse}). The procedure will repeat until a leaf
has been found ($v.isleaf = \textbf{true}$) in line \ref{lst:line:searchifleaf}.

If the value $v$ couldn't be found and search has reached the end of 
$\Delta$Tree, a buffer search will be conducted
(line \ref{lst:line:searchbuffer}).
This search is done by simply searching the buffer array from left-to-right
to find $v$, therefore no waiting will happen in this phase.

The \textsc{deleteNode} and \textsc{insertNode} algorithms (Figure
\ref{lst:pseudo-ops}) are non-intrusive to the structure of a tree, thus they
won't interfere with an ongoing search. A \textsc{deleteNode} operation, if
succeeded, is only going to mark a node by setting a $v.mark$ variable as
\textbf{true} (line \ref{lst:line:markdel} in Figure \ref{lst:pseudo-ops}).
The $v.value$ is retained so that a search will be able to proceed further.
For \textsc{insertNode}, it can \textit{"grow"} the current leaf node as it
needs to lays down two new leaves (lines  \ref{lst:line:growins1} and
\ref{lst:line:growins2} in Figure \ref{lst:pseudo-ops}), however the operation
never changes the internal pointer structure of a $\Delta$Node, since
$\Delta$Node internal tree structure is pre-allocated beforehand, allowing a
search to keep moving forward.
As depicted in Figure \ref{fig:treetransform}(a), after an insertion of $v$
grows the node, the old node (now $x'$) still contains the same value as $x$
(assuming $v < x$), thus a search still can be directed to find either $v$ or
$x$.
 The \textsc{rebalance/Merge} operation is also not an obstacle for searching
 since its operating on a mirror $\Delta$Node.
\end{proof}

We have designed the searching to be linearisable in various concurrent
operation scenarios (Lemma \ref{lem:linear-search}). This applies as well to the
update operations.

\begin{lemma} 
For a value that resides on the leaf node of a $\Delta$Node,
\textsc{searchNode} operation (Figure \ref{lst:nodeSearch}) has the linearisation 
point to \textsc{deleteNode} at line \ref{lst:line:linsearch1} and the
linearisation point to \textsc{insertNode} at line \ref{lst:line:linsearch3}. 
For a value that stays in the buffer of a $\Delta$Node, \textsc{searchNode} 
operation has the \textit{linearisation point} at line \ref{lst:line:linsearch2}.
\label{lem:linear-search}
\end{lemma}
\begin{proof}(Sketch) It is trivial to demonstrate this in relation to
deletion algorithm in Figure \ref{lst:pseudo-ops} since only an atomic operation
is responsible for altering the $mark$ property of a node (line
\ref{lst:line:markdel}).
Therefore \textsc{deleteNode} has the linearisation point to \textsc{searchNode} 
at line \ref{lst:line:markdel}. 


For \textsc{searchNode} interaction with an insertion that grows new subtree, we
rely on the facts that: 1) a snapshot of the current node $p$ is recorded on
$lastnode$ as a first step of searching iteration (Figure \ref{lst:nodeSearch},
line \ref{lst:line:lasnode-p}); 2) $v.value$ change, if needed, is not done
until the last step of the insertion routine for insertion of $v > node.value$
and will be done in one atomic step with $node.isleaf$ change (Figure
\ref{lst:pseudo-ops}, line \ref{lst:line:atomic-isleaf}); and 3) $isleaf$
property of all internal nodes are by default \textbf{true} (Figure
\ref{lst:datastruct}, line \ref{lst:line:leafdefault}) to guarantee that values
that are inserted are always found, even when the leaf-growing (both
left-and-right) are happening concurrently. Therefore
\textsc{insertNode} has the linearisation point to \textsc{searchNode} at line 
\ref{lst:line:ins-lp1} when inserting a value $v$ smaller than the leaf node's 
value, or at line \ref{lst:line:ins-lp2} otherwise. 


A search procedure is also able to cope well with a "buffered" insert, that is
if an insert thread loses a competition in locking a $\Delta$Node for expanding
or rebalancing and had to dump its carried value inside a buffer (Figure
\ref{lst:pseudo-ops}, line \ref{lst:line:insertbuffer}). Any value inserted to
the buffer is guaranteed to be found. This is because after a leaf $lastnode$
has been located, the search is going to evaluate whether the $lastnode.value$
is equal to $v$. Failed comparison will cause the search to look further inside
a buffer ($T_x.rootbuffer$) located in a $\Delta$Node where the leaf resides
(Figure \ref{lst:nodeSearch}, line \ref{lst:line:searchbuffer}).
By assuring  that the switching of a root $\Delta$Node with its mirror includes
switching $T_x.rootbuffer$ with $T_x.mirrorbuffer$, we can show that any new
values that might be placed inside a buffer are guaranteed to be found
immediately after the completion of their respective insert procedures.
The "buffered" insert has the linearisation point to \textsc{searchNode} at
line \ref{lst:line:insertbuffer}.

Similarly, deleting a value from a buffer is as trivial as exchanging that value
inside a buffer with an \textbf{empty} value. The search operation will failed
to find that value when doing searching inside a buffer of $\Delta$Node. This
type of delete has the linearisation point to \textsc{searchNode} at the same
line it's emptying a value inside the buffer (line \ref{lst:line:bufdel}).
\end{proof}

\subsection{Non-blocking Update Operations} 

\begin{lemma}
$\Delta$Tree Insert and Delete operations are
non-blocking to each other in the absence of maintenance operations.
\end{lemma}

\begin{proof} (Sketch)
Non-blocking update operations supported by $\Delta$Tree are possible by
assuming that any of the updates are not invoking \textsc{Rebalance} and
\textsc{Merge} operations. 
In a case of concurrent insert operations (Figure \ref{lst:pseudo-ops}) 
at the
same leaf node $x$, assuming all insert threads need to "grow" the node 
(for illustration, cf. Figure \ref{fig:treetransform}), they
will have to do \textsc{CAS}($x.left, \textbf{empty},\ldots$) 
(line  \ref{lst:line:growins1} and \ref{lst:line:growins2}) as their
first step. This CAS is the only thing needed since the whole $\Delta$Node
structure is pre-allocated and the CAS is an atomic operation. Therefore, only
one thread will succeed in changing $x.left$ and proceed populating the
$x.right$ node. Other threads will fail the CAS operation and they are going to
try restart the insert procedure all over again, starting from the node $x$.

To assure that the marking delete (line \ref{lst:line:markdel}) behaves nicely
with the "grow" insert operations, \textsc{deleteNode}($v, U$) that has found
the leaf node $x$ with a value equal to $v$, will need to check again whether
the node is still a leaf (line \ref{lst:line:markdel-check}) after completing
\textsc{CAS}($x.mark, FALSE, TRUE$).
The thread needs to restart the delete process from $x$ if it has found that $x$
is no longer a leaf node.

The absence of maintenance operations means that a $\Delta$Node $lock$ is 
never set to \textbf{true}, thus either insert/delete operations are never blocked
at the execution of line number \ref{lst:line:growins2} in Figure \ref{lst:waitandcheck}.
\end{proof}

\begin{lemma} 
In Figure \ref{lst:pseudo-ops}, \textsc{insertNode} operation has the linearisation 
point against \textsc{deleteNode} at line \ref{lst:line:growins1} and line \ref{lst:line:growins2}. 
Whereas \textsc{deleteNode} has a linearisation point at line \ref{lst:line:markdel-check}
against an \textsc{insertNode} operation. 
For inserting and deleting into a buffer of a $\Delta$Node, an \textsc{insertNode} 
operation has the \textit{linearisation point} at line \ref{lst:line:insertbuffer}. While
\textsc{deleteNode} has its linearisation point at line \ref{lst:line:bufdel}.
\label{lem:linear-ins-del}
\end{lemma}
\begin{proof}(Sketch) An \textsc{insertNode} operation will do a \textsc{CAS} on the left node
as its first step after finding a suitable $node$ for growing a subtree. If value $v$
is lower than  $node.value$, the correspondent operation is the line \ref{lst:line:growins1}.
Line \ref{lst:line:growins2} is executed in other conditions. A \textsc{deleteNode} will always
check a $node$ is still a leaf by ensuring $node.left.value$ as \textbf{empty}
(line \ref{lst:line:markdel-check}). This is done after it tries to mark that $node$. If the comparison
on line \ref{lst:line:markdel-check} returns \textbf{true}, the operation finishes successfully.
A \textbf{false} value will instruct the \textsc{insertNode} to retry again, starting from
the current node.

A buffered insert and delete are operating on the same buffer. When a value $v$ is put inside
a buffer it will always available for delete. And that goes the opposite for the deletion case.  
\end{proof}


\subsection{Memory Transfer and Time Complexities} 
 
In this subsection, we will show that $\Delta$Tree is relaxed cache oblivious
and the overhead of maintenance operations (e.g. rebalancing, expanding and
merging) is negligible for big trees. The memory transfer analysis is based on
the ideal-cache model \cite{Frigo:1999:CA:795665.796479}. Namely, re-accessing
data in cache due to re-trying in non-blocking approaches incurs no memory
transfer.

For the following analysis, we assume that values to be searched, inserted or
deleted are randomly chosen. As $\Delta$Tree is a binary search tree (BST),
which is embedded in the dynamic vEB layout, the expected height of a randomly
built $\Delta$Tree of size $N$ is $O(\log N)$ \cite{CormenSRL01}.

\begin{lemma}
A search in a randomly built $\Delta$Tree needs $O(\log_B N)$ expected memory
transfers, where $N$ and $B$ is the tree size and the {\em unknown} memory
block size in the ideal cache model \cite{Frigo:1999:CA:795665.796479}, 
respectively. 
\label{lem:DeltaTree_search}
\end{lemma}
\begin{proof}(Sketch)
Similar to the proof of Lemma \ref{lem:dynamic_vEB_search}, let $k, L$ be the
coarsest levels of detail such that every recursive subtree contains at most
$B$ nodes or $UB$ nodes, respectively. Since $B \leq UB$, $k \leq L$. There are
at most $2^{L-k}$ subtrees along to the search path in a $\Delta$Node and no
subtree of depth $2^k$ is split due to the boundary of $\Delta$Nodes (cf. Figure
\ref{fig:search_complexity}). Since every subtree of depth $2^k$ fits in a
$\Delta$Node of size $UB$, the subtree is stored in at most 2 memory blocks of
size $B$.

Since a subtree of height $2^{k+1}$ contains more than $B$ nodes, 
$2^{k+1} \geq \log_2 (B + 1)$, or $2^{k} \geq \frac{1}{2} \log_2 (B+ 1)$. 

Since a randomly built $\Delta$Tree has an expected height of $O(\log N)$, there
are $\frac{O(\log N)}{2^k}$ subtrees of depth $2^k$ are traversed in a search and
thereby at most  $2  \frac{O(\log N)}{2^k} = O(\frac{\log N}{2^k})$ memory
blocks are transferred.

As $\frac{\log N}{2^k} \leq 2\frac{\log N}{\log (B+ 1)} = 2 \log_{B+1} N \leq 2
log_B N$, expected memory transfers in a search are $O(\log_B N)$.
\end{proof}

\begin{lemma}
Insert and Delete operations within the $\Delta$Tree are having a similar 
amortised time complexity of $O(\log n + UB)$, where $n$ is the size of $\Delta$Tree,
and $UB$ is the maximum size of element stored in $\Delta$Node.
\label{lem:amortised_cost}
\end{lemma}

\begin{proof} (Sketch) An insertion operation at $\Delta$Tree is tightly coupled
with the rebalancing and expanding algorithm. 

We assume that $\Delta$Tree was built using random values, therefore the 
expected height is $\mathcal{O}(\log n)$. 
Thus, an insertion on a $\Delta$Tree costs $\mathcal{O}(\log n)$. 
Rebalancing after insertion only happens
at single $\Delta$Node, and it has an upper bound cost of $\mathcal{O}(UB + UB \log UB)$,
because it has to read all the stored elements, sort it out and re-insert it in a balanced fashion.  
In the worst possible case for $\Delta$Tree, there will be an $n$ insertion 
that cost $\log n$ and there is at most $n$ rebalancing operations with a cost of
$\mathcal{O}(UB + UB \log UB)$ each. 

Using aggregate analysis, we let total cost for insertion 
as $\displaystyle\sum_{k=1}^{n} c_{i} \leqslant n\log n + \displaystyle\sum_{k=1}^{n} UB + UB \log UB
\approx n \log n + n \cdot (UB + UB \log UB)$. Therefore the amortised time complexity 
for insert is $\mathcal{O}(\log n + UB + UB \log UB)$. If we have defined $UB$ such as 
$UB<<n$, the amortised time complexity for inserting a value into $\Delta$Tree 
is now becoming $\mathcal{O}(\log n)$.

For the expanding scenarios, an insertion will trigger \textsc{expand}($v$)
whenever an insertion of $v$ in a $\Delta$Node $T_j$ is resulting on $depth(v) = H(T_j)$ and
$|T_j|\geqslant (2^{H(T_j)-1})-1$. An expanding will require a memory allocation
of a $UB$-sized $\Delta$Node, cost merely $\mathcal{O}(1)$, together with two pointer
alterations that cost $\mathcal{O}(1)$ each. In conclusion, we have shown that
the total amortised cost for insertion, that is incorporating both rebalancing
and expanding procedures as $\mathcal{O}(\log n)$.

In the deletion case, right after a deletion on a particular $\Delta$Node 
will trigger a merging of that $\Delta$Node with its sibling in a condition 
of at least one of the $\Delta$Nodes is filled less than half
of its maximum capacity ($density(v) < 0.5$) and all values from both 
$\Delta$Nodes can fit into a single $\Delta$Node.

Similar to insertion, a deletion in $\Delta$Tree costs $\log n$. However merging that combines
2 $\Delta$Nodes costs $2UB$ at maximum. Using aggregate analysis, 
the total cost of deletion could be formulated as
$\displaystyle\sum_{k=1}^{n} c_{i} \leqslant n\log n + \displaystyle\sum_{k=1}^{n} 2\cdot UB
\approx n \log n + 2n \cdot UB$. The amortised time complexity is 
therefore $\mathcal{O}(\log n + UB)$ or $\mathcal{O}(\log n)$, if $UB<<n$.

\end{proof}



