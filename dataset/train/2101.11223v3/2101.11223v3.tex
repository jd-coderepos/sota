\documentclass[10pt,twocolumn,letterpaper]{article}
\usepackage{authblk}
\usepackage{datetime2}
\usepackage{iccv_style_rebuttal}
\usepackage{times}
\usepackage{epsfig}
\usepackage{graphicx}
\usepackage{amsmath}
\usepackage{amssymb}
\usepackage{bm}
\usepackage{color}          \usepackage{microtype}      \usepackage{dcolumn}        \usepackage[pagebackref=true,breaklinks=true,letterpaper=true,colorlinks,bookmarks=false]{hyperref}
\usepackage{url}            \usepackage{booktabs}       \usepackage{amsfonts}       \usepackage{nicefrac}       

\usepackage{enumitem}
\usepackage{multirow}
\usepackage{float}
\usepackage{algorithm}
\usepackage[noend]{algpseudocode}
\usepackage{amsthm}
\usepackage{subfiles}
\usepackage{adjustbox}
\usepackage{subcaption}
\usepackage{makecell}
\usepackage[normalem]{ulem}
\usepackage{dirtytalk}
\usepackage{caption}
\usepackage[multiple]{footmisc}
\usepackage[dvipsnames]{xcolor}

\newcommand{\rawal}[1]{\textcolor{red}{#1}}
\DeclareMathOperator*{\argmax}{argmax}
\DeclareMathOperator*{\argmin}{argmin}
\newtheorem{theorem}{Theorem}
\newtheorem{corollary}{Corollary}[theorem]
\newtheorem{lemma}[theorem]{Lemma}
\newtheorem{definition}{Definition}[theorem]


\usepackage[pagebackref=true,breaklinks=true,letterpaper=true,colorlinks,bookmarks=false]{hyperref}

\def\iccvPaperID{6754} \def\httilde{\mbox{\tt\raisebox{-.5ex}{\symbol{126}}}}

\definecolor{cadmiumgreen}{rgb}{0.1, 0.32, 0.1}
\newcommand{\Rone}{\textcolor{red}{R1}}
\newcommand{\Rtwo}{\textcolor{green}{R2}}
\newcommand{\Rthree}{\textcolor{cyan}{R3}}





\ifcvprfinal\pagestyle{empty}\fi
\begin{document}



\title{Multi-Instance Pose Networks: Rethinking Top-Down Pose Estimation}

\maketitle
\thispagestyle{empty}


\noindent
We sincerely thank all the reviewers (\textbf{\Rone, \Rtwo, \Rthree}) for their insightful suggestions and positive feedback. We are very encouraged by the unanimous pre-rebuttal acceptance. We address each of the reviewer's concerns below. 

\vspace{1.5mm}\noindent
\textbf{[\Rone] Ablation study for MIMB.} We indeed performed an ablative analysis of the MIMB module. Please refer to Sec. 6 of the supplemental, where we show that when only using  with the  operation and disabling  and  operations leads to sub-optimal results on both COCO \texttt{val} (0.3 AP drop) and OCHuman \texttt{val} (3.6 AP drop). Empirically, we found that the \textit{squeeze} and \textit{excite} operations allow for a better feature separation of different instances. Certainly, we will move this important discussion to the main paper.

\vspace{1.5mm}\noindent
\textbf{[\Rtwo] Same model evaluated for COCO and OCHuman.} Yes. We train the models on the \texttt{train} set of COCO and test on the \texttt{val} and \texttt{test} sets of COCO and OCHuman.

\begin{table}[b]
\centering
\captionsetup{font=small}
  \vspace*{-0.2in}
    \setlength{\tabcolsep}{1.9pt}
    \small
    \renewcommand{\arraystretch}{1.0} \begin{tabular}{@{}l|c|c c c|c c c@{}}
    \hline
   \multirow{2}{*}{Method} & \multirow{2}{*}{\#Params} & \multicolumn{3}{c|}{COCO} & \multicolumn{3}{c}{OCHuman}\\
    & &  &  &  &  &  &    \\
    \hline
    HRNet & 28.5M & 76.5 & 93.5 & 83.7 & 63.1 & 79.4 & 69.0 \\
    Two-Heads (\textit{light}) & 28.6M & 76.7 & 93.4 & 84.0 & 64.0 & 78.7 & 71.2 \\
    Two-Heads (\textit{heavy}) & 48.9M & 77.1 & 94.1 & \textbf{85.5} & 69.8 & 84.5 & 74.9 \\
    MIPNet & 28.6M & \textbf{77.6} & \textbf{94.4} & 85.3 & \textbf{72.5} & \textbf{89.2} & \textbf{79.4} \\
    \hline
    \end{tabular}

    \vspace*{-0.1in}
    \caption{Comparison of MIPNet with the Two-Heads baseline (\textit{light}, \textit{heavy}) and HRNet on the \texttt{val} sets using HRNet-W32 backbone with  input resolution and ground-truth bounding boxes.}
    \label{tab:two_head_baseline}
\end{table} 






\vspace{1.5mm}\noindent
\textbf{[\Rtwo] Two-Heads baseline.} As suggested, we compare MIPNet against the discussed Two-Heads baseline from the main paper (L121-128) in Table \ref{tab:two_head_baseline}. The Two-Heads baseline network has a primary head ( and a secondary head (. To analyze the effect of head capacity in multi-instance prediction, we create two baselines: Two-Heads (\textit{light}), and Two-Heads (\textit{heavy}) by splitting HRNet at stage  (ref. Fig 2 of supplemental) and, after stage  respectively. We found that both the heads in the Two-Heads (\textit{light}) baseline predict the same pose for the input due to the lack of head capacity getting negligible improvement in performance on both COCO and OCHuman. On the other hand, Two-Heads (\textit{heavy}) which is x the number of parameters in MIPNet achieves comparable performance on COCO but trails behind MIPNet on OCHuman. In summary, MIPNet outperforms Two-Heads (\textit{light}) by  AP with a similar number of parameters and outperforms Two-Heads (\textit{heavy}) by  AP on OCHuman. Note, in contrast to a multi-headed baseline, the parameter overhead of MIPNet does not linearly increase with the number of predicted instances  (ref. Sec. 3 of the supplemental). We will add this comparison to the main paper.


\vspace{1.5mm}\noindent
\textbf{[\Rthree] Modulation of excitation weights instead of input feature.} The input feature  is modulated using both  and , outputs of the \textit{excite} and \textit{embed} operation respectively (ref. eq. 5 of the main paper). This is equivalent to first  modulating  to  and then using  to modulate the input feature . Inspired by the feedback, we compare three instance modulation variants for   s.t. , a)  (Ours), b) , c)  where  denotes concatentation and  denotes elementwise multiplication. We found that concatentating -dim  to -dim  resulted in no modulation, infact it hurts the performance of single instance pose estimation. Whereas, the variant  performs instance modulation  but is sub-optimal compared to our MIMB design with equal parameters.


\begin{table}
\centering
\captionsetup{font=small}
\setlength{\tabcolsep}{1.9pt}
    \small
    \renewcommand{\arraystretch}{1.0} \begin{tabular}{@{}l|c c c|c c c@{}}
    \hline
   \multirow{2}{*}{Modulation Variant ()} & \multicolumn{3}{c|}{COCO} & \multicolumn{3}{c}{OCHuman}\\
    &  &  &  &  &  &    \\
    \hline
      & 74.7 & 91.8 & 80.2 & 61.8 & 79.8 & 68.4 \\
     & 77.2 & 94.4 & 84.7 & 71.1 & 87.5 & 77.8 \\
      (Ours) & \textbf{77.6} & \textbf{94.4} & \textbf{85.3} & \textbf{72.5} & \textbf{89.2} & \textbf{79.4} \\
    \hline
    \end{tabular}

    \vspace*{-0.1in}
    \caption{Comparison of MIMB (Ours) with different instance modulation variants using HRNet-W32 backbone with  input resolution and ground-truth bounding boxes on the \texttt{val} sets.}
    \label{tab:modulation}
    \vspace*{-0.25in}
\end{table} 
\vspace{1.5mm}\noindent
\textbf{[\Rthree] Multiple averaged runs.} As suggested, we report the mean and standard deviation of MIPNet on COCO \texttt{val} set using ground-truth bounding boxes over five independent randomly seeded runs in Table \ref{tab:averaged_runs}. MIPNet consistently improves over HRNet on the COCO dataset. Further, we are waiting on the averaged results of the MIMB ablation (Table 10 of supplemental) which will be added to the main paper.



\begin{table}[b]
\centering
\captionsetup{font=small}
  \vspace*{-0.2in}
    \resizebox{3.3in}{!}{
    \small
    \renewcommand{\arraystretch}{1.0} \begin{tabular}{@{}l|l|l|l|l@{}}
    \hline
Method  & Arch &  &  &    \\
    \hline
    
    HRNet  & H-32 &  &  &  \\
    MIPNet & H-32 &  &  &   \\
    
    \hline
    
    HRNet & H-48 & 77.1 & 93.6 & 84.7\\
    MIPNet & H-48 &  &  &  \\
    
    \hline

    \end{tabular}
    }
    \vspace*{-0.1in}
    \caption{We report mean  std-dev of MIPNet over five runs on the COCO \texttt{val} set with ground-truth bounding boxes using  input resolution. H-@ stands for HRNet-W@ backbone.}
    \label{tab:averaged_runs}
    \vspace*{-0.2in}
\end{table}

















%
 


\vspace{1.5mm}\noindent
\textbf{[\Rthree] Restructure main paper and supplemental.} Thanks for these detailed suggestions. We will move Table 1, 8, 10, and Section 5 from the supplemental to the main paper and modify the introduction as asked. As suggested, we will modify Table 2 and move Table 6 from the main paper to the supplemental for the camera-ready version.



\vspace{1.5mm}\noindent
\textbf{[\Rthree] Other changes.} \textit{Related work updates}: Thank you for the additional references. We will update the related works with them and also replace the arxiv versions with the conference versions. \textit{Figure updates}: We will modify Fig. 6 and Fig. 5 in the supplemental to be placed side-by-side and makes the suggested changes to Fig. 4 and Fig. 7 in the main paper. \textit{ interpolation for }: We will add this visualization to the supplemental. \textit{Typo}: We will fix L685.



\end{document}