

\documentclass[11pt,a4paper]{article}
\usepackage[hyperref]{acl2019}
\usepackage{times}
\usepackage{latexsym}
\usepackage{url}
\usepackage{multirow}
\usepackage{adjustbox}
\usepackage[ruled]{algorithm2e} \usepackage{amssymb}\usepackage{pifont}\usepackage{url}
\usepackage{mwe}
\usepackage{amsmath}
\usepackage{amssymb}
\usepackage{wasysym}
\usepackage{bbm}
\usepackage{todonotes} \usepackage{graphicx}
\usepackage{caption}
\usepackage[caption=false]{subfig}
\newcommand{\cmark}{\ding{51}}\newcommand{\xmark}{\ding{55}}\usepackage{comment}
\usepackage{multirow}
\usepackage{multicol}
\usepackage{array}
\newcommand{\head}[2]{\multicolumn{1}{>{\centering\arraybackslash}p{#1}}{#2}}

\aclfinalcopy 



\newcommand\BibTeX{B{\sc ib}\TeX}
\DeclareGraphicsExtensions{.pdf,.png,.jpg}

\newcommand{\JG}[1]{{\color{blue}{[{\bf JG}: #1]}}}
\newcommand{\WZ}[1]{{\color{green}{[{\bf WZ}: #1]}}}
\newcommand{\PC}[1]{{\color{red}{[{\bf PC}: #1]}}}
\newcommand{\XD}[1]{{\color{yellow}{[{\bf XD}: #1]}}}
\newcommand{\xiaodl}[2][]{\todo[color=yellow,size=\scriptsize,fancyline,caption={},#1]{Xiaodong:#2}} 
\newcommand{\xnet}{\textit{XNet}}
\newcommand{\lmt}{\textit{LMT}}
\newcommand{\smt}{\textit{SMT}}
\newcommand{\mmt}{\textit{M-MTL}}
\newcommand{\nmodel}{HNN}
\newcommand{\wsc}{WSC}

\title{A Hybrid Neural Network Model for Commonsense Reasoning}

\author{Pengcheng He, Xiaodong Liu, Weizhu Chen, Jianfeng Gao\\
   Microsoft Dynamics 365 AI~
   Microsoft Research~~~~~~~~\\
  {\tt \{penhe,xiaodl,wzchen,jfgao\}@microsoft.com}}


\begin{document}
\maketitle
\begin{abstract}
This paper proposes a hybrid neural network (HNN) model for commonsense reasoning. 
An HNN consists of two component models, a masked language model and a semantic similarity model, which share a BERT-based contextual encoder but use different model-specific input and output layers. 
HNN obtains new state-of-the-art results on three classic commonsense reasoning tasks, pushing the WNLI benchmark to 89\%, the Winograd Schema Challenge (WSC) benchmark to 75.1\%, and the PDP60 benchmark to 90.0\%.
An ablation study shows that language models and semantic similarity models are complementary approaches to commonsense reasoning, and HNN effectively combines the strengths of both.
The code and pre-trained models will be publicly available at \url{https://github.com/namisan/mt-dnn}.
\end{abstract}
\section{Introduction}
\label{sec:introduction}


Commonsense reasoning is fundamental to natural language understanding (NLU). As shown in the examples in Table \ref{tab:example}, in order to infer what the pronoun ``they'' refers to in the first two statements, one has to leverage the commonsense knowledge that ``demonstrators usually cause violence and city councilmen usually fear violence.'' Similarly, it is obvious to humans what the pronoun ``it'' refers to in the third and fourth statements due to the commonsense knowledge that ``An object cannot fit in a container because either the object (trophy) is too big or the container (suitcase) is too small.''

\begin{table}[!ht]
\begin{enumerate} 

\item \textit{The city councilmen refused the demonstrators a permit because \textbf{they} feared violence.} 
Who feared violence? \\
A. \textbf{The city councilmen}\hspace{0.02\textwidth} B. The demonstrators

\item \textit{The city councilmen refused the demonstrators a permit because \textbf{they} advocated violence.} 
Who advocated violence? \\
A. The city councilmen\hspace{0.02\textwidth} B. \textbf{The demonstrators}

\item \textit{The trophy doesn't fit in the brown suitcase because \textbf{it} is too big.}
What is too big? \\
A. \textbf{The trophy}\hspace{0.02\textwidth} B. The suitcase

\item \textit{The trophy doesn't fit in the brown suitcase because \textbf{it} is too small.}
What is too small? \\
A. The trophy\hspace{0.02\textwidth} B. \textbf{The suitcase}

\end{enumerate}
\caption{Examples from Winograd Schema Challenge (WSC). The task is to identify the reference of the pronoun in bold.}
\label{tab:example}
\end{table}

In this paper, we study two classic commonsense reasoning tasks: the Winograd Schema Challenge (WSC) and Pronoun Disambiguation Problem (PDP) \cite{levesque2011winograd,davis2015commonsense}. 
Both tasks are formulated as an anaphora resolution problem, which is a form of co-reference resolution, where a machine (AI agent) must identify the antecedent of an ambiguous pronoun in a statement. 
WSC and PDP differ from other co-reference resolution tasks \cite{soon2001machine,ng2002improving,peng2016event} in that commonsense knowledge, which cannot be explicitly decoded from the given text, is needed to solve the problem, as illustrated in the examples in Table \ref{tab:example}.

Comparing with other commonsense reasoning tasks, such as 
COPA \cite{roemmele2011choice}, 
Story Cloze Test \cite{mostafazadeh-EtAl:2016:N16-1},
Event2Mind \cite{rashkin2018event2mind},
SWAG \cite{zellers2018swag}, 
ReCoRD \cite{zhang2018record}, and so on, 
WSC and PDP better approximate real human reasoning, can be easily solved by native English-speaker \cite{levesque2011winograd}, and yet are challenging for machines. For example, the WNLI task, which is derived from WSC, is considered the most challenging NLU task in the General Language Understanding Evaluation (GLUE) benchmark \cite{wang2018glue}. Most machine learning models can hardly outperform the naive baseline of majority voting (scored at 65.1) \footnote{See the GLUE leaderboard at \url{https://gluebenchmark.com/leaderboard}}, including BERT \cite{devlin2018bert} and Distilled MT-DNN \cite{liu2019mt-dnn-kd}.

While traditional methods of commonsense reasoning rely heavily on human-crafted features and knowledge bases \cite{D12-1071,Sharma:2015:TAW:2832415.2832433,KR147958,SSS1510295,liu2016combing}, we explore in this study machine learning approaches using deep neural networks (DNN). Our method is inspired by two categories of  DNN models proposed recently. 

The first are neural language models trained on large amounts of text data. \citet{trinh2018simple} proposed to use a neural language model trained on raw text from books and news to calculate the probabilities of the natural language sentences which are constructed from a statement by replacing the to-be-resolved pronoun in the statement with each of its candidate references (antecedent), and then pick the candidate with the highest probability as the answer. \citet{kocijan2019surprisingly} showed that a significant improvement can be achieved by fine-tuning a pre-trained masked language model (BERT in their case) on a small amount of WSC labeled data.

The second category of models are semantic similarity models. \citet{wang-etal-2019-unsupervised} formulated WSC and PDP as a semantic matching problem, and proposed to use two variations of the Deep Structured Similarity Model (DSSM) \cite{huang2013dssm} to compute the semantic similarity score between each candidate antecedent and the pronoun by (1) mapping the candidate and the pronoun and their context into two vectors, respectively, in a hidden space using deep neural networks, and (2) computing cosine similarity between the two vectors.  The candidate with the highest score is selected as the result.

The two categories of models use different inductive biases when predicting outputs given inputs, and thus capture different views of the data. While language models measure the semantic coherence and wholeness of a statement where the pronoun to be resolved is replaced with its candidate antecedent, DSSMs measure the semantic relatedness of the pronoun and its candidate in their context. 

Therefore, inspired by multi-task learning \cite{caruana1997multitask,liu2015mtl,liu2019multi}, we propose a hybrid neural network (HNN) model that combines the strengths of both neural language models and a semantic similarity model. 
As shown in Figure \ref{fig:hybrid-model}, HNN consists of two component models, a masked language model and a deep semantic similarity model. The two component models share the same text encoder (BERT), but use different model-specific input and output layers. The final output score is the combination of the two model scores. 
The architecture of HNN bears a strong resemblance to that of Multi-Task Deep Neural Network (MT-DNN) \cite{liu2019multi}, which consists of a BERT-based text encoder that is shared across all tasks (models) and a set of task (model) specific output layers.  
Following \cite{liu2019multi,kocijan2019surprisingly}, the training procedure of HNN consists of two steps: (1) pretraining the text encoder on raw text \footnote {In this study we use the pre-trained BERT large models released by the authors.}, and (2) multi-task learning of HNN on WSCR which is the most popular WSC dataset, as suggested by \citet{kocijan2019surprisingly}.

HNN obtains new state-of-the-art results with significant improvements on three classic commonsense reasoning tasks, pushing the WNLI benchmark in GLUE to 89\%, 
the {\wsc} benchmark \footnote{\url{https://cs.nyu.edu/faculty/davise/papers/WinogradSchemas/WS.html}} 
\cite{levesque2011winograd} to 75.1\%, 
and the PDP-60 benchmark \footnote{\url{https://cs.nyu.edu/faculty/davise/papers/WinogradSchemas/PDPChallenge2016.xml}} 
to 90.0\%.
We also conduct an ablation study which shows that language models and semantic similarity models provide complementary approaches to commonsense reasoning, and HNN effectively combines the strengths of both.

 




\section{The Proposed HNN Model}
\label{sec:method}
\begin{figure*}
	\centering
	\vspace{-1mm}
{
 	\includegraphics[width=0.75\textwidth]{fig/hnn-model.png}


    }
\caption{Architecture of the hybrid model for commonsense reasoning. The model consists of two component models, a masked language model (MLM) and a semantic similarity model (SSM). The input includes the sentence , which contains a pronoun to be resolve, and a candidate antecedent . The two component models share the BERT-based contextual encoder, but use different model-specific input and output layers. The final output score is the combination of the two component model scores.}
\label{fig:hybrid-model}
\end{figure*}

The architecture of the proposed hybrid model is shown in Figure \ref{fig:hybrid-model}. The input includes a sentence , which contains the pronoun to be resolved, and a candidate antecedent . The two component models, masked language model (MLM) and semantic similarity model (SSM), share the BERT-based contextual encoder, but use different model-specific input and output layers. The final output score, which indicates whether  is the correct candidate of the pronoun in , is the combination of the two component model scores.

\subsection{Masked Language Model (MLM)}
\label{sec:mlm}

This component model follows \citet{kocijan2019surprisingly}.
In the input layer, a masked sentence is constructed using  by replacing the to-be-resolved pronoun in  with a sequence of  \texttt{[MASK]} tokens, where  is the number of tokens in candidate .

In the output layer, the likelihood of  being referred to by the pronoun in  is scored using the BERT-based masked language model . If  consists of multiple tokens,  is computed as the average of log-probabilities of each composing token:


\subsection{Semantic Similarity Model (SSM)}
\label{sec:ssm}

In the input layer, we treat sentence  and candidate  as a pair  that is packed together as a word sequence, where we add the \texttt{[CLS]} token as the first token and the \texttt{[SEP]} token between  and .

After applying the shared embedding layers, we obtain the semantic representations of  and , denoted as  and , respectively. We use the contextual embedding of \texttt{[CLS]} as . Suppose  consists of  tokens, whose contextual embeddings are , respectively. The semantic representation of the candidate , , is computed via attention as follows:


where  is a learnable parameter matrix, and  is the attention score. 

We use the contextual embedding of the first token of the pronoun in  as the semantic representation of the pronoun, denoted as . 
In the output layer, the semantic similarity between the pronoun and the context is computed using a bilinear model:

where  is a learnable parameter matrix.
Then, SSM predicts whether  is a correct candidate (i.e.,  is a positive pair, labeled as ) using the logistic function:


The final output score of pair  is a linear combination of the MLM score of Eqn. \ref{eqn:mlm} and the SSM score of Eqn. \ref{eqn:ssm}:

\subsection{The Training Procedure}

We train our model of Figure \ref{fig:hybrid-model} on the WSCR dataset, which consists of 1886 sentences, each being paired with a positive candidate antecedent and a negative candidate. 

The shared BERT encoder is initialized using the published BERT uncased large model \cite{devlin2018bert}. We then finetune the model on the WSCR dataset by optimizing the combined objectives:

where  is the negative log-likelihood based on the masked language model of Eqn.~\ref{eqn:mlm}, and  is the cross-entropy loss based on semantic similarity model of Eqn.~\ref{eqn:ssm}. 

 is the pair-wise rank loss. 
Consider a sentence  which contains a pronoun to be resolved, and two candidates  and , where  is correct and  is not. We want to maximize , where  is defined by Eqn. \ref{eqn:score}. We achieve this via optimizing a smoothed rank loss:

where  is the smoothing factor and  the margin hyperparameter. In our experiments, the default setting is , and .

\iffalse
We construct two tasks to solve the problem, one is language model task(\lmt) as in \cite{kocijan2019surprisingly}, the other is semantic matching task(\smt). 
While \lmt~ is robust to solve this problem, it lacks good explanation to the results. \smt~ is designed to produce meaningful semantic representations for the antecedent  and pronoun  with regarding to the context of the sentence .

\smt~ uses contextualized attentive pooling (\textit{CAP}) to produce the representation for antecedent, denoted as , and uses first token pooling (\textit{FTP}) to produce the representation of the pronoun, denoted as  . We will demonstrate that \smt~ performs well to capture semantic meaning of the antecedent and pronoun which will help a lot in WNLI task.



Our final model is trained via modified multi-task learning(\mmt) which combines the benefits of \lmt~ and \smt. 

\subsection{Language Model Task}
\label{sec:lmt}
Following \cite{kocijan2019surprisingly} we use the same setting to construct the input. Each input  is separated into three sub-sentences according to the pronoun, that is, the sub-sentence at the left of the pronoun, , the pronoun , the sub-sentence at the right of the pronoun, , as shown in \eqref{input:s}. 
Given a candidate , it will be tokenized into  sub-words . \lmt~first construct a new sentence  by replace  with  \textbf{[MASK]} tokens, as shown in \eqref{input:s1}. 





Then it will produce a language model probability of  conditioned on , as shown in \eqref{p:lm}



Instead of using hinge loss as the training objective function, we use a smoothed ranking loss as the objective function, as shown in \eqref{loss:lmt}



Where  are hyper-parameters chosen by grid search and we usally use  as the default value. With the modification, the final loss and gradient will be smoother which makes the model easier to be trained. We will show in the experiment that it will improve the performance of \lmt.

\subsection{Semantic Matching Task}
\label{sec:smt}






\smt~first produces the semantic representations for the given antecedent and pronoun. Then it evaluates the matching score via a semantic matching function. Finally it determines if a candidate is the antecedent of the pronoun by compare the score with a threshold, e.g. binary classification, or with the score of other candidates, e.g. ranking.
\smt~contains five major parts: input encoding, antecedent encoding, pronoun encoding, semantic matching and objective function.

\subsubsection{Input Encoding}
\label{sec:smt-input}
Given a sentence  and antecedent candidate , we will treat  as a separate sentence and pair it with  together as sentence pairs. Following \cite{devlin2018bert}, these two sentences are packed together with a special separator token \textbf{[SEP]}. \todo{Add Example} The type id of the first sentence  is assigned as , while the type id of the second sentence  is assigned as .




The token embedding of the th token  in  is calculated via  \eqref{ebd:s} while the token embedding of th token  in  is calculated via \eqref{ebd:a}. Where ,  correspond to position bias embedding of  and , , correspond to sub-word embedding of   and , while ,  correspond to the sentence type embedding of type  and type .

\subsubsection{Antecedent Encoding}
We denote BERT\cite{devlin2018bert} encoder's last layer output of token \textbf{[CLS]} and  as , correspondingly.
\smt~ uses  as a contextualized representation for the whole sentence. It is used to do contextualized attentive pooling(\textit{CAP}) over . Bi-linear attention is used to calculate the embedding of antecedent via \eqref{att:score} and \eqref{att:pooling}. 





Where  is a learnable projection parameter.  is the attention score of antecedent token .  the the final embedding of the antecedent .

\subsubsection{Pronoun Encoding}
For the given pronoun  in , it is tokenized into  sub-word tokens  where  is the index of the sub-word token. We denote BERT\cite{devlin2018bert} encoder's last layer output of tokens  as . We just simply use first token pooling(\textit{FTP}) to produce the representation of . e.g. 


\subsubsection{Semantic Matching}
To evaluate the semantic matching score of the two representations, we apply two matching function. One is bi-linear model \eqref{bisim}, where  is a learnable parameter. 





The final output of \smt~is is normalized to make the output in the same range as the output of \eqref{p:lm} in \lmt, e.g. .
When bi-linear is applied, \eqref{p:bi} is used to normalize the output. 



For WNLI task, we also scan a binary threshold on the score using WNLI training data, so that we can do binary  classification and don't need to extract other candidates from the premise sentence as in \cite{kocijan2019surprisingly}.

\subsubsection{Objective Function}
We use \eqref{loss:bi} to calculate the loss.


Where   are corresponding to the matching score of positive example and negative example calculated via \eqref{bisim}, ,  is the hyper-parameter chosen via grid search. We usally use ,  as the default value of   and 

When using cosine as the similarity function, the objective function choosing is crucial for the model to converge well. Hence we design a new loss as \eqref{loss:cos} when cosine similarity is used.


Where   are corresponding to the matching score of positive example and negative example calculated via \eqref{cos}.

We will show in experiment that, while bi-linear matching performs well generally, cosine performance better for WNLI task.

\subsection{Modified Multi-Task Learning}
\label{sec:mmt}

\begin{figure*}[ht!]
	\centering
{
	\includegraphics[width=1.0\textwidth]{fig/Multi-Task-Arch.png}
    }
\caption{Multi-Task Architecture}
\label{fig:MMT}
\end{figure*}

While \lmt~and \smt~can solve the problem from different views of the same problem, we introduce \mmt~to combine the benefits of the two.
The structure of \mmt is shown in figure \ref{fig:MMT}. Typical multitask learning usually tries to solve different problems with different objects in the same pass and use one output as the final output of each problem. During training, the tasks are scheduled randomly at mini-batch level or instance level. Our setting is different from typical multitask leaning in several ways,
\begin{itemize}
    \item First, the two tasks in our setting try to solve the same problem using the same training data from different views. 
    \item Second, the two tasks are grouped together for each input-label pair instead of shuffled randomly during training. 
    \item Finally, during inference we will combine both of the two tasks' output to produce the final output instead of just use one of the outputs.
\end{itemize}
\todo{Add picture to show how the multitask learning works}
\subsubsection{Input}
Suppose each training instance is composed of sentence , a pair of candidates  corresponding to positive candidate and negative candidate. In \lmt, each instance will generate a pair of sentences  as in \ref{sec:lmt}. In \smt, each instance will generate a pair of sentences  as in \ref{sec:smt-input}. In \mmt, those two pairs will be grouped together as the input, e.g. . This is different from typical multi-task learning which often shuffles the instances of different tasks randomly.

\subsubsection{Output}
In \mmt, the output of \lmt~and \smt~are added together as the ensemble output of \mmt, , as shown in \eqref{p:en}

Where  is the output of \lmt, calculated via \eqref{p:lm},  is the output of \smt, calculated via \eqref{p:bi}.

\subsubsection{Objective Function}
A new ranking loss  is constructed as \eqref{loss:en}.

Where  are hyper-parameters chosen via grid search. We usually use ,  as their default values.

As we already have ranking loss in \eqref{loss:en}, we remove the ranking loss part in \eqref{loss:lmt}, which simplifies it as 


The final loss  is a sum of these three \eqref{loss:mt}.




\fi 


\section{Experiments}
\label{sec:exp}
We evaluate the proposed {\nmodel} on three commonsense benchmarks: {\wsc} \cite{winograd2012}, PDP60\footnote{\url{https://cs.nyu.edu/faculty/davise/papers/WinogradSchemas/PDPChallenge2016.xml}} and WNLI. 
WNLI is derived from WSC, and is considered the most challenging NLU task in the GLUE benchmark \cite{wang2018glue}.



\subsection{Datasets}
\label{subsec:data}
\begin{table}[htb!]
	\begin{center}
		\begin{tabular}{l|c|c|c}
			\hline \bf Corpus & \#Train & \#Dev & \#Test\\ \hline \hline
            WNLI& - & 634 + 71& 146  \\ \hline
            PDP60 &-& -& 60  \\ \hline
            {\wsc} &-& -& 285  \\ \hline
            WSCR &1322&564 &-   \\ \hline
		\end{tabular}
	\end{center}
\caption{Summary of the three benchmark datasets: {\wsc}, PDP60 and WNLI, and the additional dataset WSCR. Note that we only use WSCR for training. For WNLI, we merge its official training set containing 634 instances and dev set containing 71 instances as its final dev set.
	}
	\label{tab:datasets}
\end{table}
Table~\ref{tab:datasets} summarizes the datasets which are used in our experiments. 
Since the {\wsc} and PDP60 datasets do not contain any training instances, following \cite{kocijan2019surprisingly}, we adopt the WSCR dataset \cite{rahman2012resolving} for model training and selection. 
WSCR contains 1886 instances (1322 for training and the rest as dev set). 
Each instance is presented using the same structure as that in WSC. 


For the WNLI instances, we convert them to the format of {\wsc} as illustrated in Table~\ref{tab:convert}: we first detect pronouns in the premise using spaCy\footnote{\url{https://spacy.io}}; then given the detected pronoun, we search its left of the premise in hypothesis to find the longest common substring (LCS) ignoring character case. Similarly, we search its right part to the LCS; by comparing the indexes of the extracted LSCs, we extract the candidate. A detailed example of the conversion process is provided in Table~\ref{tab:convert}.
\begin{table}[!ht]
\begin{enumerate} 

\item \textbf{Premise:} The cookstove was warming the kitchen, and \textit{\textcolor{brown}{the lamplight made} \textbf{it} \textcolor{violet}{seem even warmer.}} \\
\textbf{Hypothesis:} \textit{\textcolor{brown}{The lamplight made} \textbf{the cookstove} \textcolor{violet}{seem even warmer.}} \\
\textbf{Index of LCS in the hypothesis:} left[0, 2], right[5, 7] \\
\textbf{Candidate:} [3, 4] (the cookstove)

\item \textbf{Premise:} The cookstove was warming the kitchen, and \textit{\textcolor{brown}{the lamplight made} \textbf{it} \textcolor{violet}{seem even warmer.}} \\
\textbf{Hypothesis:} \textit{\textcolor{brown}{The lamplight made} \textbf{the kitchen} \textcolor{violet}{seem even warmer.}} \\
\textbf{Index of LCS in the hypothesis:} left[0, 2], right[5, 7] \\
\textbf{Candidate:} [3, 4] (the kitchen)

\item \textbf{Premise:} The cookstove was warming the kitchen, and \textit{\textcolor{brown}{the lamplight made} \textbf{it} \textcolor{violet}{seem even warmer.}} \\
\textbf{Hypothesis:} \textit{\textcolor{brown}{The lamplight made} \textbf{the lamplight} \textcolor{brown}{seem even warmer.}}\\
\textbf{Index of LCS in the hypothesis:} left[0, 2], right[5, 7] \\
\textbf{Candidate:} [3, 4] (the lamplight)

\item \textbf{\textcolor{blue}{Converted:}} The cookstove was warming the kitchen, and \textit{the lamplight made \textbf{it} seem even warmer.} 
 \\
A. the cookstove\hspace{0.01\textwidth} B. \textbf{the kitchen} \hspace{0.01\textwidth} C. the lamplight

\end{enumerate}
\caption{Examples of transforming WNLI to {\wsc} format. Note that the text highlighted by \textcolor{brown}{brown} is the longest common substring from the left part of pronoun \textit{it}, and the text highlighted by \textcolor{violet}{violet} is the longest common substring from its right.}
\label{tab:convert}
\end{table}

\subsection{Implementation Detail}
\label{exp:imp}
Our implementation of {\nmodel} is based on the PyTorch implementation of BERT\footnote{\url{https://github.com/huggingface/pytorch-pretrained-BERT}}. All the models are trained with hyper-parameters depicted as follows unless stated otherwise.
The shared layer is initialized by the BERT uncased large model. Adam \cite{kingma2014adam} is used as our optimizer with a learning rate of 1e-5 and a batch size of 32 or 16. The learning rate is linearly decayed during training with 100 warm up steps. We select models based on the dev set by greedily searching epochs between 8 and 10. The trainable parameters, e.g.,  and , are initialized by a truncated normal distribution with a mean of  and a standard deviation of .
The margin hyperparameter,  in Eqn.~\ref{eqn:maxm}, is set to 0.6 for MLM and 0.5 for SSM, and  is set to 10 for all tasks.  
We also apply SWA \cite{izmailov2018averaging} to improve the generalization of models. All the texts are tokenized using WordPieces, and are chopped to spans containing 512 tokens at most.




\subsection{Results}
\label{exp:results}

We compare our {\nmodel} with a list of state-of-the-art models in the literature, including BERT \cite{bert2018}, GPT-2 \cite{radford2019language} and DSSM \cite{wang-etal-2019-unsupervised}. The brief description of each baseline is introduced as follows.
\begin{enumerate}
    \item BERT\textsubscript{LARGE-LM} \cite{bert2018}: This is the large BERT model, and we use MLM to predict a score for each candidate following Eq~\ref{eqn:mlm}.
    \item GPT-2 \cite{radford2019language}: During prediction, We first replace the pronoun in a given sentence with its candidates one by one.  We use the GPT-2 model to compute a score for each new sentence after the replacement, and select the candidate with the highest score as the final prediction. 
   \item BERT\textsubscript{Wiki-WSCR} and BERT\textsubscript{WSCR} \cite{kocijan2019surprisingly}: These two models use the same approach as BERT\textsubscript{LARGE-LM}, but are trained with different additional training data. For example, BERT\textsubscript{Wiki-WSCR} is firstly fine-tuned on the constructed Wikipedia data and then on WSCR. 
   BERT\textsubscript{WSCR} is directly fine-tuned on WSCR.
  \item{DSSM} \cite{wang-etal-2019-unsupervised}: It is the unsupervised semantic matching model trained on the dataset generated with heuristic rules.
  \item {\nmodel}: It is the proposed hybrid neural network model.
\end{enumerate}
\begin{table*}[htb!]
	\begin{center}
		\begin{tabular}{l |c c c} \hline
			 &WNLI & {\wsc}  & PDP60 \\ \hline \hline
			DSSM \cite{wang-etal-2019-unsupervised}    &- &63.0 & 75.0 \\  \hline \hline
            
			BERT \cite{devlin2018bert}& 65.1 &62.0  &78.3\\ \hline
			GPT-2 \cite{radford2019language}   &- &70.7 & - \\ \hline
            BERT \cite{kocijan2019surprisingly} &71.9 &72.2 & - \\ \hline	   
            BERT \cite{kocijan2019surprisingly} &70.5 &70.3 & - \\	    \hline \hline 
            {\nmodel} &\bf 83.6 & \bf 75.1 &  \bf 90.0\\\hline 
            {\nmodel} &\bf 89.0 & - &  - \\\hline
\end{tabular}
	\end{center}
    \caption{Test results}
	\label{tab:test}
\end{table*}

\begin{figure*}[ht!]
	\centering

    {
	\includegraphics[width=1.0\textwidth]{fig/sample.png}
    }
	\caption{Comparison with SSM and MLM on WNLI examples.}
	\label{fig:sample}
\end{figure*}



\begin{table}[htb!]
	\begin{center}
		\begin{tabular}{@{\hskip1pt}l |c c |c c } \hline
			 &WNLI &WSCR & {\wsc} & PDP60 \\ \hline \hline
            {\nmodel} &{\bf 77.1} & \textbf{85.6} &\bf 75.1 &  \bf 90.0\\\hline
            -SSM & 74.5 & 82.4& 72.6 &  86.7\\\hline
            -MLM &75.1 & 83.7& 72.3 & 88.3\\\hline
		\end{tabular}
	\end{center}
    \caption{Ablation study of the two component model in {\nmodel}. Note that WNLI and WSCR are reported on dev sets while WSC and PDP60 are reported on test sets.}
	\label{tab:mlm_ssm}
\end{table}












The main results are reported in Table~\ref{tab:test}. Compared with all the baselines, {\nmodel} obtains much better performance across three benchmarks. 
This clearly demonstrates the advantage of the {\nmodel} over existing models. 
For example, {\nmodel} outperforms the previous state-of-the-art BERT model with a 11.7\% absolute improvement (83.6\% vs 71.9\%) on WNLI and a 2.8\% absolute improvement (75.1\% vs 72.2\%) on {\wsc} in terms of accuracy.
Meanwhile, it achieves a 11.7\% absolute improvement over the previous state-of-the-art BERT model on PDP60 in accuracy. 
Note that both BERT and BERT are using language model-based approaches to solve the pronoun resolution problem. 
On the other hand, We observe that DSSM without pre-training is comparable to BERT which is pre-trained on the large scale text corpus (63.0\% vs 62.0\% on WSC and 75.0\% vs 78.3\% on PDP60). Our results show that {\nmodel}, combining the strengths of both DSSM and BERT, has consistently achieved new state-of-the-art results on all three tasks.


To further boost the WNLI accuracy on the GLUE benchmark leaderboard, we record the model prediction at each epoch, and then produce the final prediction based on the majority voting from the last six model predictions. 
We refer to the ensemble of six models as {\nmodel} in Table~\ref{tab:test}. {\nmodel} brings a 5.4\% absolute improvement (89.0\% vs 83.6\%) on WNLI in terms of accuracy. 







\subsection{Ablation study}

In this section, we study the importance of each component in {\nmodel} by answering following questions: 

\noindent \textbf{How important are the two component models: MLM and SSM?}

To answer this question, we first remove each component model, either SSM or MLM, and then report the performance impact of these component models. 
Table~\ref{tab:mlm_ssm} summarizes the experimental results. It is expected that the removal of either component model results in a significant performance drop. 
For example, with the removal of SSM, the performance of {\nmodel} is downgraded from 77.1\% to 74.5\% on WNLI. Similarly, with the removal of MLM, {\nmodel} only obtains 75.1\%, which amounts to a 2\% drop. All these observations clearly demonstrate that SSM and MLM are complementary to each other and the {\nmodel} model benefits from the combination of both. 

Figure~\ref{fig:sample} gives several examples showing how SSM and MLM complement each other on WNLI. We see that in the first pair of examples, MLM correctly predicts the label while SSM does not. This is due to the fact that ``the roof repaired'' appears more frequently  than ``the tree repaired'' in the text corpora used for model pre-training. 
However, in the second pair, since both ``the demonstrators'' and ``the city councilment'' could advocate violence and neither occurs significantly more often than the other,  SSM is more effective in distinguishing the difference based on their context. 
The proposed HNN, which combines the strengths of these two models, can obtain the correct results in both cases.









\noindent \textbf{Does the additional ranking loss help?}

As in Eqn.~\ref{eqn:obj}, the training objective of {\nmodel} model contains three losses. The first two are based on the two component models, respectively, and the third one, as defined in Eqn.~\ref{eqn:maxm}, is a ranking loss based on the score function in Eqn.~\ref{eqn:score}. At first glance, the ranking loss seems redundant. 
Thus, we compare two versions of HNN trained with and without the ranking loss.
Experimental results are shown in Table~\ref{tab:ranking_loss}. We see that without the ranking loss, the performance of {\nmodel} drops on three datasets: WNLI, WSCR and WSC. On the PDP60 dataset, without the ranking loss, the model performs slightly better. However, since the test set of PDP60 includes only 60 samples, the difference is not statistically significant. 
Thus, we decide to always include the ranking loss in the training objective of HNN.


\begin{table}[htb]
	\begin{center}
		\begin{tabular}{@{\hskip1pt}l |c c |c c } \hline
			 &WNLI &WSCR & {\wsc} & PDP60 \\ \hline \hline
            {\nmodel} &{\bf 77.1} & \bf 85.6 &\bf 75.1 &  90.0\\\hline
            {\nmodel}-Rank & 74.8 &85.1& 71.9 &  \bf 91.7\\\hline
		\end{tabular}
	\end{center}
    \caption{Ablation study of the ranking loss. Note that WNLI and WSCR are reported on dev sets while WSC and PDP60 are reported on test sets.}
	\label{tab:ranking_loss}
\end{table}


\noindent \textbf{Is the WNLI task a ranking or classification task?}








\begin{figure}[htb]
	\centering
	\adjustbox{trim={0.08\width} {0.0\height} {0.09\width} {0.01\height},clip}
    {
	\includegraphics[width=0.58\textwidth]{fig/ranking-vs-cls.png}
    }
	\caption{Comparison of different task formulation on WNLI.}
\label{fig:rank_cls}
\end{figure}

The WNLI task can be formulated as either a ranking task or a classification task. To study the difference in problem formulation, we conduct experiments to compare the performance of a model used as a classifier or a ranker. For example, given a trained HNN, when it is used as a classifier we set a threshold to decide label (0/1) for each input. When it is used as a ranker, we simply pick the top-ranked candidate as the correct answer.
We run the comparison using all three models HNN, MLM and SSM. 
As shown in Figure~\ref{fig:rank_cls}, the ranking formulation is consistently better than the classification formulation for this task. 
For example, the difference in the HNN model is about absolute 2.5\% (74.6\% vs 77.1\%) in terms of accuracy. 
















 
\section{Conclusion}
\label{sec:con}
We propose a hybrid neural network (HNN) model for commonsense reasoning.
HNN consists of two component models, a masked language model and a deep semantic similarity model, which share a BERT-based contextual encoder but use different model-specific input and output layers.

HNN obtains new state-of-the-art results on three classic commonsense reasoning tasks, pushing the WNLI benchmark to 89\%, the {\wsc} benchmark to 75.1\%, and the PDP60 benchmark to 90.0\%. We also justify the design of HNN via a series of ablation experiments.

In future work, we plan to extend HNN to more sophisticated reasoning tasks, especially those where large-scale language models like BERT and GPT do not perform well, as discussed in \cite{gao2019neural, niven2019probing}.  


\section*{Acknowledgments}
We would like to thank Michael Patterson from Microsoft for his help on the paper.  


\bibliography{acl_snli}
\bibliographystyle{acl_natbib}
\end{document}
