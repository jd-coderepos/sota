\documentclass{llncs}

\usepackage{graphicx}
\usepackage{pifont}
\usepackage{verbatim}
\usepackage{amsmath,amsfonts}
\usepackage{enumerate,url}
\usepackage{paralist}
\usepackage{subfig}
\usepackage{wrapfig}
\usepackage{sidecap}

\begin{document}

\let\doendproof\endproof
\renewcommand\endproof{~\hfill\qed\doendproof}



\spnewtheorem{myremark}{Remark}{\bfseries}{\itshape}
\spnewtheorem{obs}{Observation}{\bf}{\it}
\newcommand{\down}[1]{\raisebox{-.4ex}{#1}}



\makeatletter
    \renewcommand*{\@fnsymbol}[1]{\ensuremath{\ifcase#1\or *\or \dagger\or \S\or
       \mathsection\or \mathparagraph\or \|\or **\or \dagger\dagger
       \or \ddagger\ddagger \else\@ctrerr\fi}}
\makeatother

\newcommand*\samethanks[1][\value{footnote}]{\footnotemark[#1]}

\title{A \boldmath-Approximation Algorithm for Guarding Orthogonal Art
Galleries with Sliding Cameras}

\author{
Stephane Durocher\thanks{{Work of the author is supported in part by the Natural Sciences and Engineering Research Council of Canada
(NSERC).}} \and 	
Omrit Filtser\and
Robert Fraser\and\\
Ali Mehrabi\thanks{{Work of the author is supported by the Netherlands' Organization for Scientific Research (NWO)}}\and
Saeed Mehrabi\thanks{{Work of the author is supported in part by a University of Manitoba Graduate Fellowship (UMGF).}}
}

\institute{
Department of Computer Science, University of Manitoba, Canada.\\
Department of Computer Science, Ben-Gurion University of the Negev, Israel.\\
Department of Mathematics and Computer Science, Eindhoven University of Technology, Netherlands.\\
\email{durocher@cs.umanitoba.ca, omritna@cs.bgu.ac.il,  fraser@cs.umanitoba.ca, amehrabi@win.tue.nl, mehrabi@cs.umanitoba.ca}
}



\newcommand{\keywords}[1]{\par\addvspace\baselineskip
\noindent\keywordname\enspace\ignorespaces#1}
\newtheorem{observation}{Observation}{\bfseries}{\itshape}

\pagenumbering{arabic}
\pagestyle{plain}

\maketitle


\begin{abstract}
Consider a sliding camera that travels back and forth along
an orthogonal line segment  inside an orthogonal polygon  with  vertices. The camera can see a point  inside
 if and only if there exists a line segment containing  that crosses  at a right angle and 
is completely contained in . In the minimum sliding cameras (MSC)
problem, the objective is to guard  with the minimum number of sliding cameras.
In this paper, we give an -time -approximation algorithm to the
MSC problem on any simple orthogonal polygon with  vertices,
answering a question posed by Katz and Morgenstern (2011).
To the best of our knowledge, this is the first constant-factor approximation algorithm for this problem.
\end{abstract}

\section{Introduction}
\label{sec:introduction}
In the classical art gallery problem, we are given
a polygon and the objective is to cover the polygon with the union of visibility
regions of a set of points guards while minimizing the number of guards. The problem was introduced by Klee
in 1973~\cite{orourke1987}. Two years later, Chv\'{a}tal~\cite{chvatal1975} showed that 
point guards are always sufficient and sometimes necessary to guard the polygon.
Since then, the problem and its many variants have been studied extensively for
different types of polygons (e.g., orthogonal polygons~\cite{dietmar1995} and polyominoes~\cite{biedl2012}), different types of
guards (e.g., points and line segments) and different visibility types.

Recently, Katz and Morgenstern~\cite{katz2011} introduced a variant of the art gallery problem
in which \emph{sliding cameras} are used to guard an
orthogonal polygon. Let  be an orthogonal polygon with  vertices. A sliding camera travels back and forth along an
orthogonal line segment  inside . The camera can
see a point  if the there is a point  such that
 is a line segment normal to  that is completely inside . In the minimum
sliding cameras ({\bf MSC}) problem, the objective is to guard  using a
the minimum number of sliding cameras.

In this paper, we give an -time -approximation algorithm to the minimum sliding cameras (MSC)
problem on any simple orthogonal polygon.
To do this, we introduce the \emph{minimum guarded sliding cameras} ({\bf MGSC})
\emph{problem}. In the MGSC problem, the objective is to guard  using a set of minimum
cardinality of guarded sliding cameras.
A sliding camera  is guarded by a sliding camera  if every point on  is seen by some point on . Note that  and  could be perpendicular in which case  and  mutually guard each other if and only if they intersect. If  and  mutually guard each other and have the same orientation (e.g., both are horizontal), then the visibility region of  is identical to that of . Consequently, when minimizing the number of sliding cameras in the MGSC problem, it suffices to consider solutions in which each horizontal sliding camera is guarded by a vertical sliding camera and vice-versa.
We first establish a connection between the MGSC problem and a related guarding problem on grids.

A grid  is a connected union of vertical and horizontal line segments;
each maximal line segment in the grid is called a \emph{grid segment}. A \emph{point guard} 
in grid  is a point that sees a point  in the grid if the
line segment . Moreover, a \emph{sliding camera}
 is a point guard that moves along a grid
segment . The camera  can see a point  on the grid if and
only if there exists a point  such that the line segment
; that is, point  is seen by camera  if
either  is located on  or  belongs to a grid segment that
intersects . Note that sliding cameras are called \emph{mobile guards} in
grid guarding problems~\cite{kosowski2004,adrian2006}. A \emph{guarded} set of point guards and a guarded set of sliding cameras on grids
are defined analogously to a guarded set of sliding cameras in polygons. A simple grid is defined as follows:

\begin{definition}[Kosowski et al.~\cite{adrian2006}]
A grid is \emph{simple} if \begin{inparaenum}[(i)]\item the
endpoints of all of its segments lie on the outer face of the planar
subdivision induced by the grid, and
\item there exists an  such that every grid segment can
be extended by  in both directions such that its new
endpoints are still on the outer face. \end{inparaenum}
\end{definition}

Throughout the paper, we denote a simple orthogonal polygon by ; note that the
polygon  is a closed region.
The rest of the paper is organized as follows. Section~\ref{sec:relatedWork} presents
related work. In Section~\ref{sec:3approxAlgorithm}, we give our -approximation
algorithm to the MSC problem and we conclude the paper in Section~\ref{sec:conclusion}.

\section{Related Work and Definitions}
\label{sec:relatedWork}
The minimum sliding cameras problem was introduced by Katz and Morgenstern~\cite{katz2011}.
They first considered a restricted version of the problem in which only vertical cameras are allowed; by reducing
the problem to the minimum clique cover problem on chordal graphs, they solved the problem exactly
in polynomial time. For the generalized case, where both vertical and horizontal cameras are allowed,
they gave a -approximation algorithm for the MSC problem under the assumption that the polygon 
is -monotone. 
Durocher and Mehrabi~\cite{durocher2013} showed that the MSC
problem is \textsc{NP}-hard when the polygon  is allowed to have holes. They also gave an exact algorithm
that solves in polynomial time a variant of the MSC problem in which the objective is to minimize the sum of the lengths of line
segments along which cameras travel.


The guard problem on grids was first formulated by Ntafos~\cite{ntafos1986}.
He proved that a set of (stationary) point guards of minimum cardinality covering a grid of
 grid segments has  guards, where  is the size of the maximum
matching in the intersection graph of the grid that can be found in 
time. Malafijeski and Zylinski~\cite{malafijeski2005} showed that the problem of
finding a minimum-cardinality set of guarded point guards for a grid is \textsc{NP}-hard.
Katz et al.~\cite{katz2005} showed that the problem of finding a minimum number
of sliding cameras covering a grid is \textsc{NP}-hard. Moreover,
Kosowski et al.~\cite{kosowski2004} proved that the problem of finding
the minimum number of guarded sliding cameras covering a grid (we call
this problem the {\bf MMGG} problem) is \textsc{NP}-hard.
Due to these hardness results, Kosowski et al.~\cite{adrian2006} studied
the MMGG problem on some restricted classes of grids. In particular, they show the following result
on simple grids:
\begin{theorem}[Kosowski et al.~\cite{adrian2006}.]
\label{thm:nSquarForGridProblem}
There exists an -time algorithm for solving the MMGG problem on
simple grids, where  is the number of grid segments.
\end{theorem}



Throughout the paper, we denote optimal solutions for the MSC problem and the MGSC problem on  by
 and , respectively.
We denote the set of reflex vertices of 
by  and let  and  be the maximum-length
horizontal and vertical line segments, respectively, inside  through a vertex
. Let .
Let  and  be two
orthogonal line segments (with respect to  not necessarily each other) inside ; the \emph{visibility region} of  is the union
of the points in  that are seen by the sliding camera that travels along . Moreover,
we say that  dominates  if the visibility
region of  is a subset of that of .

\section{A \boldmath-Approximation Algorithm for the MSC Problem}
\label{sec:3approxAlgorithm}In this section, we present an -time -approximation algorithm for the MSC problem.



\subsection{Relating the MGSC and MMGG problems}
Consider an optimal solution  for the MSC problem and let
 be the set of line segments obtained by taking two instances every line segment in . We
observe that  is a feasible solution for the MGSC problem and, therefore, we have
the following observation.
\begin{observation}
\label{obs:mgscAndMSC}
An optimal solution for the MGSC problem on  is a -approximation to an
optimal solution for the MSC problem on .
\end{observation}

We first consider how to apply a solution for the MGSC problem to the MSC problem.
For the MGSC problem, the idea is to reduce the MGSC problem to the MMGG problem.
Given any simple orthogonal polygon , 
we construct a grid  associated with  as follows: initially, let  be the
set of all line segments in . Now, for any pair of reflex vertices
 and  where  dominates  (resp.,  dominates ) in , we remove
 (resp., ) from ; if two segments mutually dominate each other, remove
one of the two arbitrarily. Let  be the set of remaining grid segments in . Observe that 
can be constructed in  time, where  is the number of vertices of . We first show the following result:

\begin{lemma}
\label{lem:gIsSimple}Grid  is a simple and connected grid.
\end{lemma}
\begin{proof}
It is straightforward from the construction of  that both endpoints of each grid
segment in  lie on the boundary of ; this means that the endpoints of every grid
segment in  lie on the outer face of  and, therefore,  is simple. To show that 
is connected, we first observe that the grid induced by the line segments in  is connected.
We now need to show that the grid remains connected after removing the set of grid
segments that are dominated by other grid segments. Let  be a grid segment
that is removed from  (i.e., ). It is straightforward to see that the set of
grid segments that are intersected by  are also intersected by , where 
is the grid segment that dominates . Therefore, grid  is connected.
\end{proof}
\begin{figure}[t]
\centering \includegraphics[width=0.85\textwidth]{example}
\caption{(a) A simple orthogonal polygon .
(b) Grid  with set  of grid segments shown in red. (c) The set
 (represented by solid red line segments) is an optimal solution for the MMGG problem
on , but  and  cannot guard  entirely; in particular, the hatched regions of  are not guarded.}
\label{fig:notEquivalent}\end{figure}

The objective is to solve the MMGG problem on  exactly
and to use the solution , the set of guarded grid segments computed, as the solution to the MGSC problem.
However,  is not always a feasible solution to the MGSC problem since some regions in 
might remain unguarded; see Figure~\ref{fig:notEquivalent}
for an example. In the following, we characterize the regions of  that may remain unguarded by the line segments in ;
we call these the \emph{critical regions} of .

Consider  and choose any unguarded point  inside . Let  be a maximal axis-aligned
rectangle contained in  that
covers  and is also not guarded by the line segments in . We observe that \begin{inparaenum}[(i)]\item some line segments
in  can guard ,
and that \item no such line segments are in  since  is unguarded. \end{inparaenum} 
Consider the maximal regions in  that lie immediately above, below, left, and right of ;
any sliding camera that sees any part of  must intersect one of these regions.
See Figure~\ref{fig:growingRp}(a) for an example; note that the hatched region
cannot contain any line segment in  since  is unguarded. Moreover, the hatched region must contain at least one line segment
of  in both horizontal and vertical directions and without loss of generality we can assume that the length of these
line segments is maximal (i.e., both endpoints are on the boundary of ).

The rectangle  defines a partition of  into three parts: the \emph{vertical slab} through , (i.e., the slab
whose sides are aligned with the vertical sides of ), the subpolygon of  to the left of the vertical slab and the subpolygon of  to the
right of the vertical slab. Similarly, another partition of  can be obtained by considering the \emph{horizontal slab} through ; see
Figure~\ref{fig:growingRp}(b) for an example. We know that the union of the visibility regions of the line segments in  is a
connected subregion of the plane. Therefore, the set  can only be found on one side of each of the partitions of  and so  must be
in one \emph{corner} of the partitioned polygon. Without loss of generality, assume that  is on the bottom left corner of the partitioned polygon
(see Figure~\ref{fig:growingRp}(b)).

Let  be the set of line segments that can see . Note that 
is non-empty since  or  sees  entirely where  denotes the nearest reflex vertex to  in the horizontal or vertical slabs. Therefore, polygon  is partitioned into three subpolygons (see Figure~\ref{fig:growingRp}(c)): the lower-left corner that is the location of  denoted by , the lower-right and upper-left corners that correspond to line segments in  denoted by , and the upper-right corner of the polygon that is unguarded denoted by .
Each line segment in  intersects at least one line segment in 
since the line segments in  are not in  and  is feasible solution for the MMGG problem (see Figure~\ref{fig:growingRp}(c)).

\begin{figure}[t]
\centering \includegraphics[width=0.95\textwidth]{growRp}
\caption{(a) Point  inside the polygon  with rectangle  hatched in gold. The purple hatched region
indicates the subregion of  covered by growing  orthogonally towards the boundaries of ; the boundary
of  is shown in blue.
(b) The horizontal rectangle and the vertical histogram shown in red indicate, respectively, the horizontal and vertical slabs of the partitions induced
by rectangle . The hatched region of  on the bottom left corner of the partition indicates the location of the set . (c) The partition of  into three subpolygons ,  and . The line segments in 
(i.e., the set of line segments that can see ) are shown in red; observe that each line segment in  intersects at least one line
segment in . The line segment  illustrates Lemma~\ref{lem:opposite} and vertex  is in support of Lemma~\ref{lem:contradiction}.}
\label{fig:growingRp}\end{figure}

\begin{lemma}
\label{lem:opposite}
No line segment in  that is orthogonal to a line segment in  can intersect .
\end{lemma}
\begin{proof}
To derive a contradiction, suppose without loss of generality that there is one such vertical line segment  intersecting  
(as shown in Figure~\ref{fig:growingRp}(c)).
Since  is connected and is a feasible solution for the MMGG problem, there must be a line segment in  that
guards . So,  must contain a line segment in , which is a contradiction.
\end{proof}

By Lemma~\ref{lem:opposite}, we conclude that  is guarded by the line
segments in ,
but not by any line segment in , and, furthermore,  is restricted to a corner as described above (i.e., the subpolygon ). We now show that each of the regions of  that are not guarded by
the line segments in  must be a staircase with the reflex vertices oriented towards .

To derive a contradiction, suppose that there exists a reflex vertex  (as shown in Figure~\ref{fig:growingRp}(c)) in the unguarded subpolygon  of . However, no 
line segment in  can intersect  by Lemma \ref{lem:opposite}. This
contradicts the existence of . Therefore, any unguarded region of  by  must be bounded by the line segments 
in \footnote{If the bounding line segment is not in  then a dominating line segment must be in . 
See Figure \ref{fig:ex_poly} for an example.} on adjacent horizontal and
vertical sides, and by a staircase of  on the other sides; we call these regions the \emph{critical regions} of , and denote  to be the set of critical regions
of . We now have the following lemma.



\begin{lemma}
\label{lem:contradiction}
Every point of  that is not inside a critical region of  is visible to at least one
line segment in . Moreover, each critical region of  is a staircase.
\end{lemma}

Let  denote an optimal solution for the MMGG problem on . We first prove
that .

\begin{lemma}
\label{lem:polygonAndGP}For any feasible solution  for the MGSC problem on , there exists a feasible solution  for the MMGG problem on  such that .
\end{lemma}
\begin{proof}
Let  be a feasible solution to the MGSC problem on ; that is,  is a guarded set of orthogonal line segments
inside  that collectively guard . We construct a feasible solution  for the MMGG problem such that
. To compute , for each horizontal line segment  (resp., vertical
line segment ), move  up or down (resp., to the left or to the right) until it is collinear with a line segment
. If , then add  to ; otherwise, add  to , where  is the line segment that dominates . Note that there exists at least one such line segment  because otherwise the line segment  would have not been removed from . It is straightforward to see that the union of visibility regions of line segments in  is a subset of the visibility regions of line segments in . Since the camera travelling along each line segment in  is seen by at least one other camera (see the definition of a guarded set of sliding cameras) and the grid 
is entirely contained in , we conclude that  is a feasible solution for the MMGG problem on . The inequality
 follows from the fact that each line segment in  corresponds to at most one line
segment in . This completes the proof of the lemma.
\end{proof}

Next we need to find a set of minimum cardinality of orthogonal line
segments inside  that collectively guard the critical regions of .





\subsection{Guarding Critical Regions: A \boldmath-Approximation Algorithm}
\label{subsec:exactAlgorithmForCriticals}
In this section, we give an approximation algorithm for the problem of guarding the critical regions of
. The algorithm relies on reducing the problem to the minimum edge cover problem in graphs. The
minimum edge cover problem in graphs is solvable in  time, where  is
the number of graph vertices~\cite{vazirani1980}. We first need the following result:

\begin{lemma}
\label{lem:atLeastOneSegment}
Every critical region in  is guarded entirely by some orthogonal line segment in .
\end{lemma}
\begin{proof}
Observe that if  is a rectangle, then the MSC problem is trivial to solve. 
Suppose that  is not a rectangle
and so it has at least one reflex vertex.
Furthermore, suppose that some regions of  are not guarded by  (the set of segments returned by
solving the MMGG problem on ), i.e., the set  of critical regions of  is non-empty.
Let  be a critical region of . The lemma is implied by the fact that there exists at least one
reflex vertex on the boundary of ; this is because  is not a rectangle and the set of line segments in
 do not guard . It is now
straightforward to see that one of the orthogonal line segments in  that passes through either the lowest or
the highest reflex vertex of  can see the critical region  entirely.
\end{proof}

Recall , the set of critical regions of . We construct a graph  associated with 
as follows: for each critical region , we add a vertex  to . Two vertices
 and  are adjacent in  if and only if there exists an orthogonal line segment inside
 that can guard both critical regions  and  entirely. Finally, we add a self-loop edge for every
isolated vertex of .



\begin{lemma}
\label{lem:atMostTwoCriticals}
Any orthogonal line segment inside  can guard at most two critical regions of  entirely.
\end{lemma}
\begin{proof}
Let  be an orthogonal line segment inside . Observe that the only way for  to guard a critical region  entirely is that at least
one of its endpoints lies on the boundary of , covering one entire edge of ; see Figure~\ref{fig:atMostTwoCriticals} for an example.
Therefore,  can guard at most two critical regions of .
\end{proof}



\begin{figure}[t]
\centering \includegraphics[width=0.3\textwidth]{atMostTwoCriticals}
\caption{Any orthogonal line segment inside  can guard at most two critical regions
of  entirely.}
\label{fig:atMostTwoCriticals}\end{figure}


\begin{lemma}
\label{lem:criticalEdgeCoverEquivalence}
The problem of guarding the critical regions of  using only those line segments that may individually guard a critical region reduces to the minimum edge cover problem on .
\end{lemma}
\begin{proof}
We prove that \begin{inparaenum}[(i)]\item for any solution  to the minimum edge cover
problem on , there exists a solution  for guarding the critical regions of  such
that , and that \item for any solution  to the problem of guarding
the critical regions of , there exists a solution  to the minimum edge cover problem on
 such that . \end{inparaenum}\\

\noindent{\bf Part 1.} Choose any edge cover  of . We construct a solution  for guarding the critical regions of  as follows. For each edge
 let  be the line segment in  that can see both critical
regions  and  of ; we add  to . It is straightforward to see that
the line segments in  collectively guard all critical regions of .\\

\noindent{\bf Part 2.} Choose any solution  for guarding the critical regions of . We
now construct a solution  for the minimum edge cover problem on . By Lemma~\ref{lem:atMostTwoCriticals},
we know that every line segment in  can see at most two critical regions of .
First, for each line segment in  that can see exactly one critical region  of , we add the
self-loop edge of  that corresponds to  in . Next, for each line segment  that
can see two critical regions of , we add to  the edge in  that corresponds to .
Since any line segment in  can see at most two critical regions of , we conclude that every
vertex of  is incident to at least one edge in  and, therefore,  is a feasible solution for
the minimum edge cover problem on .
\end{proof}

In general, it is possible for the solution  to be non-optimal. Only those edges which may individually guard a critical region were considered, while an optimal guarding solution may use two line segments to collectively guard a critical region, as shown in Figure \ref{fig:ex_poly}. By Lemma \ref{lem:atLeastOneSegment},  requires at most one edge for each critical region and,
therefore, the number of guards returned y our algorithm is at most equal to the number of critical regions.
If an optimal solution uses two segments to collectively guard one critical region, 
these two edges suffice to guard three critical regions, while our solution uses three segments to guard the same three
critical regions. This results in an approximation factor of  in the number of segments used to guard 
the set of critical regions.

\begin{figure}[t!]
\centering \includegraphics[width=0.9\textwidth]{ex_poly1}
\caption{An example of a polygon for which two line segments may collectively guard a critical region.  (a) The polygon has unguarded critical regions after finding an optimal MMGG solution. The thick red line segments indicate guarding lines, light gray line segments are unused, and the shaded regions are unguarded.
(b) A possible solution (solid red line segments) for guarding the critical regions using the edge guarding approach. (c) An optimal critical region guarding solution for the polygon.}
\label{fig:ex_poly}\end{figure}

We now examine the running time of the algorithm. Let  denote the number of vertices of .
To compute the critical regions of , we first compute the set of staircases of  and, for each of them, we check to see whether
they are guarded by the set of line segments in . Each critical region of  can be found
in  time and so the set of unguarded critical regions of  is easily computed in  time.
Moreover, the graph  can be constructed
in  time by checking whether there is an edge between every pair of vertices of the graph. Therefore,
by Lemma~\ref{lem:criticalEdgeCoverEquivalence} and the fact that the minimum edge cover
problem is solvable on a graph with  vertices in  time, we have the following
lemma.
\begin{lemma}
\label{lem:exactAlgForCriticals}
There exists a -approximation algorithm for solving the problem of guarding the critical regions of
a simple orthogonal polygon  with  vertices in  time.
\end{lemma}

Given any simple orthogonal polygon , we find a set of sliding cameras that guards  
by first solving the instance  of the MMGG problem determined by . This may leave a set of critical regions within 
that remain unguarded. We then add a second set of sliding cameras to guard these critical regions.
Recall the set , an optimal solution to the MMGG problem on , found in  time,
where  is the number of vertices of .
By Lemma~\ref{lem:exactAlgForCriticals}, we approximate the problem of guarding the critical
regions of  in  time;
let  be the solution returned by the algorithm. 
Since the union of the critical regions of  is
a subset of , any feasible solution to the MSC problem also guards the critical regions of . 
Therefore, .
Moreover, by Lemma~\ref{lem:polygonAndGP} we know  and since
 we have that .
Therefore, by combining  and  we obtain a feasible solution
to the MSC problem whose cardinality is at most 7/2 times ; that is,
. This gives the main result:



\begin{theorem}
\label{thm:msc3Approx}
There exists an -time -approximation algorithm for the MSC problem on any simple orthogonal polygon  with  vertices.
\end{theorem}

As a consequence of our main result, we note that  and again by Lemma~\ref{lem:polygonAndGP} we know . Therefore, . To show that the set  is a feasible solution for the MGSC problem, we first observe that every line segment in  is guarded by at least one other line segment in . Moreover, every grid segment that is not in  is guarded by some line segment in  because  is a feasible solution for the MMGG problem. This means that every line segment in  is guarded by at least one line segment in . Therefore, we have the following result.
\begin{corollary}
Given a simple orthogonal polygon  with  vertices, there exists an -time -approximation algorithm for the MGSC problem on .
\end{corollary}

\section{Conclusion}
\label{sec:conclusion}
In this paper, we studied a variant of the art gallery problem, introduced by Katz and Morgenstern~\cite{katz2011},
where sliding cameras are used to guard an orthogonal polygon and the objective is to guard the polygon with
minimum number of sliding cameras. We gave an -time -approximation algorithm to this problem
by deriving a connection between a guarded variant of this problem (i.e., the MGSC problem) and the problem of guarding simple grids with sliding cameras. The complexity of the
problem remains open for simple orthogonal polygons. Giving an -approximation algorithm, for any
, is another direction for future work. Finally, studying the MGSC problem (the complexity of the problem or improved approximation results) might be of independent interest.\\


\subsection*{Acknowledgements} The authors thank Mark de Berg and Matya Katz for
insightful discussions of the sliding cameras problem.

\bibliographystyle{plain}
\bibliography{ref}


\end{document}
