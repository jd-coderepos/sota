
\section{Experiment}

\begin{table*}[!]
\begin{center}
\resizebox{\linewidth}{!}{
  \begin{tabular}{lcccccccl}
  \toprule[1pt]
Aug.                       & AdaFit &  \makecell[c]{DeepFit \\~\cite{ben2020deepfit}} & \makecell[c]{Denoising+\\DeepFit~\cite{ben2020deepfit}} & \makecell[c]{Lenssen \\\etal~\cite{lenssen2020deep}} & \makecell[c]{Nesti-Net\\~\cite{ben2019nesti}} & \makecell[c]{PCPNet\\~\cite{guerrero2018pcpnet}}& PCA~\cite{hoppe1992surface} & Jet~\cite{cazals2005estimating}   \\ \hline
No noise& \textbf{5.19} & 6.51  & 8.48 & 6.72           & 6.99      & 9.66   & 12.29 & 12.23 \\
Noise () & \textbf{9.05}& 9.21 & 10.38    & 9.95           & 10.11     & 11.46  & 12.87& 12.84 \\
Noise ()   & \textbf{16.44}& 16.72 & 16.79  & 17.18& 17.63     & 18.26  & 18.38 & 18.33 \\
Noise ()   & \textbf{21.94}& 23.12 & 22.18 & 21.96& 22.28& 22.8& 27.5& 27.68 \\
Varing Density(Strips)& \textbf{6.01}& 7.92 & 9.62 & 7.73& 8.47& 11.74  & 13.66& 13.39 \\
Varing Density(gradients)  & \textbf{5.90}& 7.31 & 9.37   & 7.51& 9.00      & 13.42  & 12.81& 13.13 \\ \hline
Average& \textbf{10.76}& 11.8  & 12.8  & 11.84          & 12.41     & 14.56  & 16.25                   & 16.29 \\ \bottomrule[1pt]
\end{tabular}
}
\vspace{-0.4cm}
\end{center}
\caption{Normal RMSE of AdaFit and baseline methods on the PCPNet dataset.}
\label{table-RMSE} 
 \vspace{-0.15cm}
\end{table*}



\subsection{PCPNet dataset}

We follow exactly the same experimental setup as~\cite{guerrero2018pcpnet} including train-test split, training data augmentation and adding noise or density variations on testing data. 
We use the Adam~\cite{kingma2015adam} optimizer and a learning rate of  for the training with a batch size of 256. AdaFit is trained with 600 epochs on a 2080Ti GPU.

\textbf{Baselines}. We consider three types of baseline methods: 1) the traditional normal estimation methods PCA~\cite{hoppe1992surface} and jet~\cite{cazals2005estimating}; 2) the learning-based surface fitting methods Lenssen \etal~\cite{lenssen2020deep}, DeepFit~\cite{ben2020deepfit}; 3) the learning-based normal regression methods PCPNet~\cite{guerrero2018pcpnet} and Nesti-Net~\cite{ben2019nesti}.

\textbf{Metrics}. We use the angle RMSE between the predicted normal and the ground truth as our main metrics for evaluation. We also draw the AUC curve of normal errors to show the error distribution.

\textbf{Quantitative results}. The RMSE of AdaFit and baseline methods are shown in Table~\ref{table-RMSE} and the AUC of all methods are shown in Fig.~\ref{fig:PCG}. The results show that AdaFit outperforms both the traditional methods and learning-based methods in all settings, which demonstrates the effectiveness of using offsets to adjust the point set. Especially on the point clouds with density variations, baseline methods may fail to find enough points on sparse regions for the surface fitting while AdaFit use the offset to project points to the neighboring regions for more robust surface fitting. In addition, we adding the results of DeepFit~\cite{ben2020deepfit} with denoising pre-processing~\cite{rakotosaona2020pointcleannet}. Though denoising reduces the noise and finds smooth surfaces, it does not offset the points to the ground truth positions on the surfaces, which still results in a larger RMSE.


\textbf{Qualitative results}. Fig.~\ref{fig:normal-map} shows the normals estimated by AdaFit and Fig.~\ref{fig:normal-error} visualizes the angle errors of AdaFit and baseline methods on different shapes. It can be seen that Lenssen \etal~\cite{lenssen2020deep} perform better on the flattened regions while DeepFit~\cite{ben2020deepfit} can better handle curved regions but is not accurate for regions with sharp curves. In contrast, the proposed AdaFit is relatively more robust on all regions and all settings (noise or density variations). Additional detailed analysis on normal estimation of sharp edges can be found in the supplementary materials.

\begin{figure}[]
\begin{center}
\includegraphics[width=0.9\linewidth]{./images/PCG_v2.pdf}\vspace{-0.5cm}
\end{center}
   \caption{Normal error AUC curves of AdaFit and baseline methods. X-axis shows the threshold in degree and Y-axis shows the ratio of correct estimated normals under a given threshold.}
\label{fig:PCG}
\vspace{-0.4cm}
\end{figure}

\begin{figure}[]
\begin{center}
\includegraphics[width=0.9\linewidth]{./images/normal_mapping_bigger.pdf}\vspace{-0.3cm}

\end{center}
   \caption{Normals estimated by AdaFit on the PCPNet dataset.}
\label{fig:normal-map}
\vspace{-0.2cm}

\end{figure}

\begin{figure}[]
\begin{center}
\includegraphics[width=0.9\linewidth]{./images/PCPNet_error_map_v1.pdf}\vspace{-0.5cm}
\end{center}
   \caption{Errors of normal estimation on the PCPNet dataset.}
   
\label{fig:normal-error}
\end{figure}


\subsection {Results on real datasets}
\label{Transfer-new}

To test the generalization capability of AdaFit, we directly evaluate AdaFit on real datasets including the indoor SceneNN~\cite{hua2016scenenn} dataset and the outdoor Semantic3D~\cite{hackel2017isprs} dataset. Note that all the methods are only trained on the PCPNet dataset and directly evaluated on these datasets.

\textbf{The SceneNN dataset}. 
The SceneNN dataset contains more than 100 indoor scenes collected by a depth camera with provided ground-truth reconstructed meshes. We obtain the point clouds by sampling on the reconstructed meshes and compute the ground-truth normal from the meshes. The RMSE of AdaFit and baselines are shown in Table~\ref{RMSE_error_indoor} and the angle error visualizations are shown in Fig.~\ref{indoor}. The results show that AdaFit can predict more accurate normals than baseline methods.

\textbf{The Semantic3D dataset}. The Semantic3D dataset contains point clouds collected by laser scanners. Since there is no ground truth for normal estimation, we only show qualitative results in Fig.~\ref{Semantic3D}. The results show that AdaFit can find sharp edges on point clouds while other methods oversmooth these normals.



\begin{figure}[]
   \begin{center}
      \includegraphics[width=0.9\linewidth]{./images/indoor_v2.pdf}\vspace{-0.4cm}
   \end{center}
   \caption{Error maps of estimated normals on the SceneNN dataset.
   }
   \label{indoor}
\end{figure}  



\begin{table}[]
\begin{center}
    \begin{tabular}{cccc}
\toprule[1pt]
& \multicolumn{1}{l}{AdaFit} &\multicolumn{1}{l}{DeepFit~\cite{ben2020deepfit}}&\multicolumn{1}{l}{Lenssn \etal~\cite{lenssen2020deep}} \\ \hline
RMSE & \textbf{16.25}& 18.33& 18.54\\
\bottomrule[1pt]
\end{tabular}
\vspace{-0.2cm}
\end{center}
\caption{Normal RMSE of AdaFit, DeepFit~\cite{ben2020deepfit} and Lenssen \etal~\cite{lenssen2020deep} on the SceneNN dataset.}
\label{RMSE_error_indoor}
\end{table}


\begin{figure}[]
   \begin{center}
      \includegraphics[width=0.9\linewidth]{./images/Semantic3D_v5.pdf}\vspace{-0.5cm}
   \end{center}
   \caption{Estimated normals on the semantic 3D dataset.}
   \label{Semantic3D}
\end{figure}

\subsection {Ablation study}

\begin{table}[]
\begin{center}
    \setlength{\tabcolsep}{1.5mm}{
        \begin{tabular}{l|cc|cc|cc}
        
        \toprule[1pt]
\textbf{\textbf{  n}} & \multicolumn{2}{c|}{1} & \multicolumn{2}{c|}{2} & \multicolumn{2}{c}{3} \\ \hline
\textbf{Weight} & \checkmark & \checkmark & \checkmark & \checkmark & \checkmark & \checkmark \\
\textbf{Offset} &  & \checkmark &  & \checkmark &  & \checkmark \\
No Noise & 8.09 & 5.96 & 10.07 & 6.06 & 7.91 & 6.00 \\
Low Nose & 10.49 & 9.34 & 11.30 & 9.28 & 9.82 & 9.27 \\
Med Noise & 16.59 & 16.56 & 16.59 & 16.48 & 16.36 & 16.48 \\
High Noise & 21.80 & 21.82 & 21.61 & 21.77 & 21.48 & 21.76 \\
Stripes & 9.72 & 7.10 & 12.00 & 7.18 & 9.80 & 7.09 \\
Gradients & 8.54 & 6.67 & 10.83 & 6.80 & 8.59 & 6.72 \\ \hline
Average & 12.54 & 11.24 & 13.73 & 11.26 & 12.33 & 11.22 \\ 
\bottomrule[1pt]

        \end{tabular}
    }
\vspace{-0.4cm}
\end{center}
\caption{Normal RMSE of models with or without offset prediction on the PCPNet dataset.}
\label{Ablation_ADA}

\end{table}


\begin{table}[]
\begin{center}
\setlength{\tabcolsep}{4mm}{
    \begin{tabular}{l|ccc|c}
\toprule[1pt]
\textbf{Scale} & 256 & 500 & 700 & 700 \\
\hline
\textbf{Weight} & \multicolumn{1}{c}{\checkmark} & \checkmark & \multicolumn{1}{c|}{\checkmark} & \multicolumn{1}{c}{\checkmark} \\
\textbf{Offset} & \multicolumn{1}{c}{\checkmark} & \checkmark & \multicolumn{1}{c|}{\checkmark} & \multicolumn{1}{c}{\checkmark} \\
\textbf{CSA} & \multicolumn{1}{l}{} &  & \multicolumn{1}{c|}{} & \multicolumn{1}{c}{\checkmark} \\  
No Noise & 5.17 & 5.79 & 6.00 & 5.19 \\
Low Noise & 9.17 & 9.17 & 9.27 & 9.05 \\
Med Noise & 16.71 & 16.47 & 16.48 & 16.44 \\
High Noise & 23.02 & 22.12 & 21.76 & 21.94 \\
Stripes & 6.03 & 6.64 & 7.09 & 6.01 \\
Gradients & 6.00 & 6.30 & 6.72 & 5.90 \\
\hline
Average & 11.02 & 11.08 & 11.22 & 10.76 \\ 
\bottomrule[1pt]
\end{tabular}}
\vspace{-0.4cm}
\end{center}
\caption{Normal RMSE of models with or without CSA layers on the PCPNet dataset.}
\label{Ablation_CSA} 
 \vspace{-0.1cm}
\end{table}


\begin{table}[]
\begin{center}
\setlength{\tabcolsep}{1.5mm}{
    \begin{tabular}{lcccccc}
    
    \toprule[1pt]
          \textbf{Threshold}       & \textbf{0.0}  & \textbf{0.05}   & \textbf{0.10}   & \textbf{0.30}   & \textbf{0.50}   &  \textbf{Offset} \\
          \hline
    No Noise   & 7.91  & 7.35  & 7.41  & 7.94  & 8.41  & 6.00   \\
    Low Noise  & 9.82  & 9.53  & 9.57  & 10.07 & 10.49 & 9.27   \\
    Med Noise  & 16.36 & 16.31 & 16.37 & 16.98 & 18.06 & 16.48  \\
    High Noise & 21.48 & 21.43 & 21.44 & 22.22 & 23.42 & 21.76  \\
    Stripes  & 9.80  & 8.97  & 8.84  & 8.94  & 9.61  & 7.09   \\
    Gradients    & 8.59  & 8.09  & 8.11  & 8.52  & 9.11  & 6.72   \\
    \hline
    Avarage    & 12.33 & 11.95 & 11.96 & 12.44 & 13.18 & 11.22  \\
    
    \bottomrule[1pt]
    \end{tabular}}
    \vspace{-0.4cm}
\end{center}

\caption{Normal RMSE of models using thresholds to truncate points and the model with offset prediction on the PCPNet dataset. The first row shows the truncation thresholds.}
\vspace{-0.4cm}
\label{fig:truncate}
\end{table}


\textbf{Offset prediction}. To demonstrate the effectiveness of the proposed offset prediction, we conduct experiments on the PCPNet dataset using models with or without offset prediction. The backbone network in this experiment is the same as DeepFit without any CSA layer and uses the neighborhood size of 700 points. The results are shown in Table~\ref{Ablation_ADA}, which demonstrates that the offset prediction can effectively improve the accuracy of the predicted normals. Meanwhile, we notice that the performance of the model without offset prediction is sensitive to the polynomial order  while the model with offset prediction has similar performance on different orders.

\textbf{CSA layer}. In Table~\ref{Ablation_CSA}, we conduct experiments on the PCPNet dataset using models with or without CSA layers. The results show that using CSA layers can bring improvements on the normal estimation and reduce the necessity to select a specific neighborhood size (scale).

\textbf{Comparison to truncating weights}. Since outliers may have small predicted weights, a more simple way to prevent these outliers from affecting the estimated normals is to truncate points with a predefined threshold. In Table~\ref{fig:truncate}, we compare the model using proposed offset predictions with the model that truncates points with small weights. All models take 700 neighboring points as inputs. The results show that truncating indeed slightly improves the results but is still inferior to the model with offset prediction.