






\documentclass[conference]{IEEEtran}


\pagestyle{plain}



\usepackage{epsfig,endnotes}

\usepackage{epsfig}     \usepackage{epstopdf}


\usepackage{graphicx}
\usepackage{caption}
\usepackage{subcaption}
\usepackage{xspace}

\usepackage{multirow}
\usepackage{array}
\usepackage{color}
\usepackage{times}
\usepackage{tablefootnote}
\usepackage{tabularx}
\usepackage{balance}
\usepackage{url}

\newcommand{\zebra}{ZEBRA\xspace}
\newcommand{\KBactivity}{keyboard-only\xspace}
\newcommand{\allactivity}{all-activity\xspace}
\newcommand{\attacker}{\xspace}
\newcommand{\victim}{\xspace}
\newcommand{\attackedterminal}{\xspace}
\newcommand{\victimdevice}{\xspace}




\newfont{\eaddfnt}{phvr8t at 10pt}
\def\email#1{{{\eaddfnt{\par #1}}}}       

\newif\ifabridged
\newif\ifanonymous
\newif\ifcomments
\newif\ifllncs
\newif\ifapps

\ifdefined\isapps
\appstrue
\fi

\ifdefined\isllncs
\llncstrue
\fi

\ifdefined\isabridged
\abridgedtrue
\fi

\ifdefined\isanonymous
\anonymoustrue
\fi

\ifdefined\iscomments
\commentstrue
\fi





\newcommand{\fixme}[1]{\textcolor{red}{\textbf{#1}}}

\ifcomments
\newcommand\otto[1]{\textcolor{red}{Otto: #1}}
\newcommand\asokan[1]{\textcolor{green}{Asokan: #1}}
\newcommand\mika[1]{\textcolor{blue}{Mika: #1}}
\newcommand\changeOtto[1]{\textcolor{red}{#1}}
\newcommand\changeMika[1]{\textcolor{red}{#1}}
\newcommand\changeAsokan[1]{\textcolor{red}{#1}}
\else

\newcommand\otto[1]{}
\newcommand\asokan[1]{}
\newcommand\mika[1]{}
\newcommand\added[1]{{#1}}
\newcommand\replaced[1]{{#1}}
\newcommand\deleted[1]{{#1}}
\newcommand\changeOtto[1]{{#1}}
\newcommand\changeMika[1]{{#1}}
\newcommand\changeAsokan[1]{{#1}}
\fi

\newcommand{\figwidth}{\columnwidth}
\newcommand{\figwidthappendix}{0.5\columnwidth}

\begin{document}

\title{Pitfalls in Designing Zero-Effort Deauthentication: Opportunistic Human Observation Attacks}



\ifanonymous
\else
\author{\IEEEauthorblockN{Otto Huhta\IEEEauthorrefmark{1}, Prakash Shrestha\IEEEauthorrefmark{2}, Swapnil Udar\IEEEauthorrefmark{1}, Mika Juuti\IEEEauthorrefmark{1},
Nitesh Saxena\IEEEauthorrefmark{2} and N. Asokan\IEEEauthorrefmark{3}}
\IEEEauthorblockA{\IEEEauthorrefmark{1}Aalto University}
\IEEEauthorblockA{\IEEEauthorrefmark{2}University of Alabama at Birmingham}
\IEEEauthorblockA{\IEEEauthorrefmark{3}Aalto University and University of Helsinki}
\IEEEauthorblockA{\{otto.huhta, swapnil.udar, mika.juuti\}@aalto.fi, \{saxena, prakashs\}@uab.edu, asokan@acm.org}}

\fi

\IEEEoverridecommandlockouts
\makeatletter\def\@IEEEpubidpullup{9\baselineskip}\makeatother
\IEEEpubid{\parbox{\columnwidth}{Permission to freely reproduce all or part
    of this paper for noncommercial purposes is granted provided that
    copies bear this notice and the full citation on the first
    page. Reproduction for commercial purposes is strictly prohibited
    without the prior written consent of the Internet Society, the
    first-named author (for reproduction of an entire paper only), and
    the author's employer if the paper was prepared within the scope
    of employment.  \\
    NDSS '16, 21-24 February 2016, San Diego, CA, USA\\
    Copyright 2016 Internet Society, ISBN TBD\\
    http://dx.doi.org/10.14722/ndss.2016.23199
}
\hspace{\columnsep}\makebox[\columnwidth]{}}

\maketitle








\begin{abstract} 
	
	
	Deauthentication is an important component of any authentication
	system. The widespread use of computing devices in daily life has
	underscored the need for \textit{zero-effort} deauthentication schemes.
	However, the quest for eliminating user effort may lead to hidden
	security flaws \ifabridged \else in the authentication schemes\fi. 

As a case in point, we investigate a prominent zero-effort deauthentication
scheme, called \zebra, which provides an interesting and a useful solution to a
difficult problem as demonstrated in the original paper. We identify a subtle
incorrect assumption in its adversary model that leads to a fundamental design flaw. We
exploit this to break the scheme with a class of attacks that are much easier
for a human to perform in a realistic adversary
model, compared to the na\"{i}ve attacks studied in the \zebra paper. For
example, one of our main attacks, where the human attacker has to opportunistically
mimic only the victim's keyboard typing activity at a nearby terminal, is
significantly more successful compared to the na\"{i}ve attack
that requires mimicking keyboard and mouse activities as well as keyboard-mouse movements.
Further, by understanding the design flaws in \zebra as cases of \textit{tainted
  input}, we show that we can draw on well-understood design principles to
  improve \zebra's security. 
  
  \end{abstract}


 
\section{Introduction}
\label{sec:intro}

User authentication is critical to many on-line and off-line services. 
\ifabridged
\else
Computing devices of all types and sizes,
ranging from mobile phones through personal computers to remote
servers rely on user authentication.  
\fi
\textit{Deauthentication} --
promptly recognizing when to terminate a previously authenticated user
session -- is an essential component of an authentication system.
\ifabridged
\else


\fi
The pervasive use of computing in people's daily lives underscores the
need to design effective, yet intuitive and easy-to-use deauthentication
mechanisms. However, this remains an important unsolved problem in
information security. A promising approach to improving usability
of (de)authentication mechanisms is to make them \textit{transparent} to
users by reducing, if not eliminating, the cognitive effort required
from users.  Although such \textit{zero-effort} authentication schemes
are compelling, designing them correctly is difficult. The need
to minimize additional user interactions required by the
scheme is a severe constraint that can lead to design
decisions which might affect the security of the scheme.

One prominent approach for improving usability of security mechanisms
involves comparing information observed from two different sources. Such
a \textit{bilateral} approach has been proposed as part of solutions
for a variety of security problems such as deauthentication of
users~\cite{mare2014zebra}, determining if two or more devices are
co-present in the same place~\cite{TruongPerCom14}, establishing
security associations among nearby devices
(``pairing'')~\cite{VanDyken+14,DBLP:journals/csur/ChongMG14} and
authorizing transactions between co-present devices~\cite{bump}.
Bilateral authentication schemes are attractive because they can avoid
imposing any cognitive load on users (thus making them
``zero-effort''), or the need to store sensitive or user-specific information on 
devices~\cite{mare2014zebra}.  However, an adversary \changeAsokan{capable of influencing} one or
both sources of information being compared in a bilateral 
scheme may compromise security.

In this paper, we illustrate the problem of subtle flaws in the design
of zero-effort bilateral schemes by examining an interesting class of schemes \changeAsokan{represented by \zebra,} a zero-effort bilateral deauthentication scheme, proposed
recently in a premier security research venue~\cite{mare2014zebra}.
\zebra is intended for scenarios where users authenticate to ``terminals'' (such as \changeAsokan{desktop computers}). In such scenarios, users typically have
to either manually deauthenticate themselves by logging out or locking the
terminal, or the terminal can deauthenticate a user automatically after
a sufficiently long period of inactivity. The former requires user
effort while the latter sacrifices promptness. \zebra attempts to make
the process of deauthentication \textit{both prompt and transparent}: once a
user is authenticated to a terminal (using say a password), \changeAsokan{it} continuously,
yet transparently \textit{re-authenticates} the user so that prompt
deauthentication is possible without explicit user action. A user is required to wear a bracelet equipped with sensors on his
\changeMika{mouse-holding} hand. The bracelet is wirelessly connected to
the terminal, which compares the sequence of events it observes
(e.g., keyboard/mouse interactions) with the sequence of events
inferred using measurements from the bracelet
sensors. \changeAsokan{The logged-in user is deauthenticated} when the two sequences no
longer match. 

\zebra is particularly compelling because of its
simplicity of design. \changeAsokan{However, the simplicity hides a design assumption that an adversary can exploit to defeat the scheme. We show how a more realistic adversary can circumvent ZEBRA. Since no implementation of ZEBRA was available, we built an end-to-end implementation and use it in our attack. We also implemented changes needed to make ZEBRA work in real-time.}

\begin{figure*}[!bht]
\centering
\includegraphics[width=.8\textwidth]{pics/benign.eps}
\caption{Normal operation of \zebra}
\label{fig:benign}
\end{figure*}

Our primary contributions can be summarized as follows:
\begin{enumerate}
\itemsep0em
\item 
We highlight fundamental pitfalls in designing zero-effort bilateral
security schemes by studying \zebra, a notable prior scheme. We 
identify a hidden design choice in \zebra that allows us to develop an \textbf{effective attack strategy}: a human attacker observing a victim at a nearby terminal and \textit{opportunistically} mimicking only a subset of the victim's activities (e.g., keyboard events) at the authentication terminal (Section~\ref{sec:attack_overview}).
\itemsep0.5em
\item  We build a \textbf{end-to-end
    implementation}\footnote{Unlike \cite{mare2014zebra} which only
    described the implementation of individual components \changeAsokan{and off-line classification}.} of \zebra
  (Section~\ref{sec:system_setup}), and demonstrate via \textbf{experiments in
  realistic adversarial settings} that \zebra as designed can be
  defeated by our opportunistic attacker with a \changeMika{(statistically)} significantly higher probability compared to a \changeAsokan{na\"{i}ve} attacker, \changeMika{also} considered in \cite{mare2014zebra} (one who attempts to mimic all, keyboard and mouse, activities) (Section~\ref{sec:results}).
\itemsep0.5em
\item We cast \zebra's design flaw as a case of
  \textbf{tainted input}, and thus draw from well-understood principles
  of secure system design that may help improve the security of \zebra (Section~\ref{subsec:strengthening}).
\end{enumerate}
 
\section{Background}
\label{sec:background}

\ifabridged
\else
Since we use \zebra~\cite{mare2014zebra} as our exemplary bilateral
zero-effort deauthentication scheme, we now describe it in more
detail. 
\fi
\changeAsokan{It} is intended for multi-terminal environments where users frequently move between terminals. Mare et al.~\cite{mare2014zebra} present a hospital environment as their motivating scenario. Hospital staff members often use shared terminals. However, a user must not, intentionally or unintentionally, access hospital systems from terminals where other users have logged in. Users may leave terminals without logging out, but may still remain in the vicinity. Proximity-based zero-effort deauthentication schemes such as ZIA~\cite{CN02} or BlueProximity~\cite{Blueproximity} cannot be used because these methods are not accurate enough for short distances. Although the motivating scenario is an environment with shared terminals, zero-effort deauthentication schemes \ifabridged
\else
like \zebra
\fi
are broadly applicable to any scenario where users may leave their terminals unattended.

\vspace{2mm}
\noindent\textbf{Adversary Model}:
\zebra \changeMika{\cite{mare2014zebra}} considers two types of adversaries: ``innocent'' and ``malicious''. An innocent adversary is a legitimate user who starts using an unattended terminal inadvertently without realizing that another user (``victim'') is logged into that terminal. In contrast, a malicious adversary deliberately uses an unattended terminal of the victim with the intent of performing some action impersonating the victim. A malicious adversary may observe the behavior and actions of the victim (such as imitating the victim's hand movements \changeAsokan{made} while interacting with another terminal). The goal of \zebra is to quickly detect if a previously authenticated session on a terminal is being used by anyone other than the user who originally authenticated, and promptly deauthenticate the session. Naturally, decisions made by \zebra should minimize false positives
(incorrectly recognizing an adversary as the original authenticated user, thereby failing to deauthenticate him as well as false negatives (incorrectly concluding that current user is not the original user, thereby deauthenticating him.

\vspace{2mm}
\noindent\textbf{System Architecture}:
Figure~\ref{fig:benign} depicts the normal (benign) operation of \zebra.
It correlates a user's activities on a terminal with measurements of user activity relayed from a wrist-worn device (we call it a bracelet for simplicity, but it can be a general-purpose smartwatch as in our implementation and analysis). 
\ifabridged
\else
The goal is to continuously verify that the logged in user is the one using the terminal and to quickly deauthenticate any unintended users. 
\fi
\zebra assumes terminals with keyboard/mouse and a personal bracelet for each user of the system. The bracelet has accelerometer and gyroscope sensors to record wrist movements. Terminals and bracelets securely communicate using ``paired'' wireless channels like Bluetooth. In addition, a terminal knows the identity of the bracelet associated with each authorized user. Users initially authenticate themselves to terminals using some mechanism external to \zebra (such as using a username/password). Once a user has been authenticated, the terminal connects to \changeAsokan{that} user's bracelet and starts receiving sensor measurements from it.

The basic principle of operation is to compare the sequence of user activity seen at the terminal with that inferred from data sent by the bracelet. \zebra's system architecture is shown in Figure \ref{fig:architecture}. An \textit{Interaction Extractor} on the terminal identifies the \textit{actual interaction sequence} based on input events observed by the terminal peripherals. \changeAsokan{It} defines three different types of such interactions: typing, scrolling, and hand movements between the mouse and keyboard (referred to as ``MKKM'')\footnote{\added{\zebra neither cares about which key was pressed nor about what direction the mouse was scrolled. It actually cares about whether the interaction is typing, scrolling or movement between mouse and keyboard.}}.  Interaction Extractor records the timestamps of each event in the actual interaction sequence.
A \textit{Segmenter} on the terminal receives measurement data sent by the bracelet
and segments this data according to the timestamps it receives from Interaction Extractor.
Segmenter ignores all measurements that fall outside these time slots. From the segments, a \textit{Feature Extractor} extracts salient features and feeds them to an \textit{Interaction Classifier} that has been trained to identify the type of interaction from bracelet measurement data.
The classifier outputs a \textit{predicted interaction sequence}. Finally, an \textit{Authenticator} compares the two interaction sequences and determines whether the current user at the terminal is the ``same'' as, or ``different'' from, the originally authenticated user.

Authenticator can be tuned by a number of parameters. It compares sequences of length \textit{w (window size)} at a time. In each window, if the fraction of matching interactions exceeds a threshold \textit{m (matching threshold)}, it records 1 for that window; otherwise it records 0. If the record is 0 for \textit{g (grace period)} successive windows, the authenticator outputs ``different'' causing \zebra to deauthenticate the session. Successive windows may overlap, as determined by \textit{f (overlap fraction)}, with 0 signifying no overlap.

Segmenter ignores readings from the bracelet when Interaction Extractor detects no activity on the terminal. \changeMika{The choice was motivated in} \changeAsokan{\cite{mare2014zebra} by privacy considerations: the user's activities are} \changeMika{not monitored when nobody is using the terminal.} \changeAsokan{At first glance, it} is a natural and reasonable design decision: if there is no terminal activity, there is reason to deauthenticate the session (thus reducing the chances of false negative decisions). However, as we shall see, an adversary can exploit this subtle aspect of the design.



\begin{figure*}[!htb]
\centering
\vspace{-8mm}
\includegraphics[width=0.9\textwidth]{pics/architecture.eps}
\vspace{-20mm}
\caption{\zebra system architecture}
\label{fig:architecture}
\vspace{-5mm}
\end{figure*}

\vspace{1mm}
\noindent\changeMika{\textbf{Validation}: Mare et al.~\cite{mare2014zebra} validated usability of their deauthentication scheme by calculating false negative rates for normal usage scenarios with different parameter settings. They validated the security by considering three separate scenarios.} \changeAsokan{The first two scenarios model the ``innocent adversary'':} \changeMika{the logged in user (victim) is either walking or writing nearby while the} \changeAsokan{attacker accesses the victim's terminal. The last scenario models the ``malicious'' adversary: the victim uses} \changeMika{\emph{another} terminal, while the} \changeAsokan{attacker uses the victim's original terminal. The activity conducted by both victims and attackers is filling forms. These scenarios were chosen as representative of} \changeMika{multi-user environments such as hospitals, where physicians enter form-type data about their patients and routinely forget to log out of their terminals. 
It is reasonable to assume there are multiple terminals that} \changeAsokan{users access and use.}
\changeMika{Similar usage scenarios are plausible in other contexts as well, such as in factory floors or control rooms. In \cite{mare2014zebra}, the} \changeAsokan{malicious adversary is required} \changeMika{to mimic \emph{all} mouse-hand movements of the victim. Ordinary non-expert users act as the attackers in their analysis.} \changeAsokan{Because Mare et al. \cite{mare2014zebra}} \changeMika{``realize that a real adversary can be motivated and skilled enough to mimic user very well, compared to our adversaries'',} \changeAsokan{they tried to make the scenario advantageous to the attacker by (a) providing the attacker with a clear view of the victim's screen and (b) have the victim give verbal cues to indicate what the victim was doing during the experiments (e.g., answering question 2 in the form).} 
\changeMika{They concluded that their system was able to deauthenticate such attackers in reasonable time, while keeping false negative rates low. 
}









 

\section{Our Attack}
\label{sec:attack_overview}



There are a number of attributes that make \zebra attractive. In
particular, rather than trying to \textit{recognize} the user,
\zebra's bilateral approach simply \textit{compares} two sequences
that characterize user interaction. Consequently, its decisions neither limit how a user interacts with the terminal nor require storing any information about the user or his style of interaction. Such simplicity makes \zebra robust but also vulnerable. In this section, we revisit the security analysis in \cite{mare2014zebra}, point out a design flaw, and explain how it can be used to attack \zebra.

\subsection{Revisiting \zebra Security Analysis}
\label{sec:revisiting_zebra}

Recall from Section~\ref{sec:background} that Segmenter ignores all measurement data from the bracelet during periods when Interaction Extractor does not record any activity \changeAsokan{on the terminal involving the three types of interactions recognized by \zebra}. However, the attacked terminal is under the control of the adversary \changeMika{and thus she} can effectively choose which parts of the bracelet measurement data will be used by \zebra\changeMika{ to re-authenticate the user}. 
Mimicking all interactions is not the best \changeAsokan{attack strategy}. 
A smart adversary can selectively choose only a subset of the victim's interactions to mimic since it can ensure that the rest of the victim's interactions will be ignored by \changeAsokan{Authenticator}. \changeOtto{Furthermore,} \changeMika{to validate security,} 
\changeAsokan{we need to use a realistic adversary model which allows attackers to be skilled and experienced in mimicking how people interact with terminals. It is unreasonable to use inexperienced test participants to model the adversary.} 
Thus, the role of the attacker in this paper was played by two members of our research group that were knowledgeable of the \zebra system and experienced at mimicking attacks.







 
\subsection{Attack Scenarios and Strategies}
\label{subsec:scenarios}






In our attack scenarios, we model a malicious adversary against \zebra
as discussed in Section~\ref{sec:background}.  We assume that the
adversary \attacker accesses the attacked terminal \attackedterminal when the victim \victim steps away from
it without logging out. We also assume that \victim is using
another computing device (the ``victim device'', \victimdevice) elsewhere (e.g., a nearby terminal). \changeAsokan{Figure~\ref{fig:rta-scenario} illustrates the attack setting.}


\vspace{1mm}
\noindent\textbf{Strategy}: The goal of \attacker is to remain logged in on \attackedterminal for as long as possible, while interacting with the terminal. To this end, \attacker needs to consistently produce a sufficiently large fraction of interactions that will match \victim's interactions on \victimdevice. Since \attackedterminal is under the control of \attacker, it can choose when \attackedterminal's Interaction Extractor triggers Authenticator to compare the predicted and actual interaction sequences. If \attacker adopts an \textit{opportunistic} strategy, it can \textit{selectively} choose only a subset of \victim's interactions to mimic so as to maximize the fraction of matching interactions.
We conjecture that such an opportunistic adversary will be more
successful than the na\"{i}ve adversary that was considered
in~\cite{mare2014zebra}.

First, we consider a
\textit{\KBactivity} attack where \attacker mimics only the typing
interactions while ignoring all others. Typing sequences are typically longer and less prone to delays in mimicking.
The opportunistic strategy is for \attacker to start typing only after \victim starts
typing and attempt to stop as soon as 
\victim stops. A sophisticated \KBactivity attacker may estimate
the expected length of \victim's typing session and
attempt to stop before \victim does. If \attacker makes just a few key presses each time \victim begins typing, he can be confident that the actual interaction sequence he produces will match the predicted interaction sequence. These \KBactivity attacks are powerful because in all modern personal computer operating systems a wide range of actions can be performed using only the keyboard.

Second, we consider an \textit{\allactivity} attack, where 
\attacker mimics
all types of interactions (typing, scrolling and MKKM) but
opportunistically chooses a subset of the set of interactions.
\ifllncs
This opportunistic strategy is described in detail in Appendix~\ref{app:extra_attackers}.
\else
As before, the \attacker's selection
criterion is the likelihood of correctly mimicking \victim. In
particular, \attacker will use the following strategy:
\begin{itemize}
\itemsep0em
\item Once \attacker successfully mimics a keyboard to mouse interaction, he
  is free to carry out any interaction involving the mouse (scroll,
  drag, move) at will because the bracelet measurements for all
  \changeAsokan{interactions} involving the mouse are likely to be similar.
\item If \attacker fails to quickly mimic a keyboard to mouse (or vice
  versa) interaction, he
  does nothing until the next opportunity for an MKKM
  interaction arises (foregoing all interactions until after the MKKM
  is completed).
\end{itemize}
\fi


\zebra concatenates continual typing events into up-to 1 second long interactions: as such the typing speed of \attacker is not particularly relevant. Instead, \attacker may divert more of his attention to observing \victim.


\vspace{1mm}
\noindent\textbf{Observation Channels}: By default, and similar to \cite{mare2014zebra}, we consider an adversary \attacker who has a
clear view of \victim's \changeAsokan{interactions} (Figure~\ref{fig:rta-scenario}). This models two cases: where \attacker has direct visual access to \victim and where \attacker has access to a \textit{video aid} such as a surveillance camera aimed at \victimdevice. During our attacks that use visual information of the victim's behavior, victim's new device \victimdevice was placed next to the victim terminal \attackedterminal. We also consider the case where \attacker has no visual access to but can still hear sounds resulting from \victim's activities. Again, this models two cases: where both \victim and \attacker are in the same physical space separated by a visual barrier (e.g., adjacent cubicles) and where \attacker has planted an \textit{audio aid} (e.g., a small hidden bug or a microphone) close to \victimdevice. 



\begin{figure*}[!htbp]
\centering
\vspace{-5mm}
\includegraphics[width=.95\textwidth]{pics/rta.eps}
\vspace{-3mm}
\caption{Basic attack setting}
\label{fig:rta-scenario}
\end{figure*}








\vspace{1mm}
\noindent\textbf{Scenarios}: The combination of attack strategy and
type of observation channel leads to several different attack scenarios. We consider four of the most significant ones:

\begin{itemize}
\itemsep0em
\item In \textbf{\changeMika{na\"ive} \allactivity} attack, \attacker is able to both see and hear \victim. \attacker attempts to mimic \textit{all interactions} of \victim. This is the attack scenario proposed and studied in \cite{mare2014zebra}.
\itemsep0.35em
\item In \textbf{opportunistic \KBactivity} attack, \attacker is able to both see and hear \victim. \attacker selectively mimics only a \textit{subset of \victim's typing interactions}.
\itemsep0.35em
\item In \textbf{opportunistic \allactivity} attack, \attacker is able to both see and hear \victim. \attacker selectively mimics a \textit{subset of all types of interactions} of \victim following the guidelines mentioned 
\ifllncs
in Appendix~\ref{app:extra_attackers}.
\else
above.
\fi
\itemsep0.35em
\item In \textbf{audio-only opportunistic \KBactivity} attack, \attacker is able to hear, but not see, \victim's \changeAsokan{interactions}. \attacker listens for keyboard activity and attempts to mimic \textit{a subset of \victim's typing interactions}.


\end{itemize}
	
	While one can imagine other attack combinations, we consider these four to be representative of different choices available to \attacker. For example, we leave out an audio-only \allactivity attack because it is unlikely to succeed. Although our experiments are ``unaided'' (i.e., no audio or video recording), the results generalize to aided scenarios, if data transmission between the aid and the attacker does not introduce excessive delays.



























 
\section{\zebra End-to-End System}
\label{sec:system_setup}

Mare et al~\cite{mare2014zebra} \changeMika{describe} a framework for \zebra and implemented \changeMika{some} individual pieces. \changeMika{However, this was not} a complete system. Therefore, we needed to build an end-to-end system from scratch to evaluate our conjecture about opportunistic attacks. Our goal was to make this system as close to the one in \cite{mare2014zebra} as possible. We now describe our system and how we evaluated its performance.


\subsection{Design and Implementation}
\label{sec:system_design}


\noindent\textbf{Software and Hardware}: We followed the ZEBRA system architecture as described in Figure~\ref{fig:architecture}. Our system consists of two applications: the bracelet runs an Android Wear application and the terminal runs a Java application. Interaction Classifier is implemented in Matlab. Communicator modules in both applications orchestrate communication over Bluetooth to synchronize clocks between them and to transfer bracelet measurements to the terminal. The rest of the terminal software consists of the ``\zebra Engine'' (shaded rectangle) with the functionality described in Section~\ref{sec:background}. The bracelet and terminal synchronize their clocks during connection setup. For our experiments, we used a widely available smartwatch (4GB LG G Watch R with a 1.2 GHz CPU and 512MB RAM) with accelerometer/gyroscope as the bracelet and standard PCs as terminals. 










\vspace{1mm}
\noindent\textbf{Parameter Choices}: Mare et al~\cite{mare2014zebra} do not fully describe the parameters used in their implementation of \zebra components. Wherever available, we used the exact parameters provided in \cite{mare2014zebra} \cite{mare}. For the rest, we strived to choose reasonable values. A full list of parameters and rationales for choosing their values appears in Appendix~\ref{app:parameteres}.


\vspace{1mm}
\noindent\textbf{Classifier}: We use the Random
Forest~\cite{breiman2001random} classifier. Again, as \cite{mare2014zebra} did not include all details on how their classifier was trained and tuned, we made parameter choices that gave the best results. Our forest consisted of 100 weak-learners. Each split in a tree considered  features, where  was the total number of features, and the trees were allowed to fully grow. In addition, classes were weighted to account for any imbalances in the training dataset (described below in Section~\ref{sec:victim-data}).
We adopt the same set of features used in \cite{mare2014zebra}, and extract them for both accelerometer and gyroscope segments. A full list appears in Appendix~\ref{app:features}.


\noindent\textbf{Differences}: Despite our efforts to keep our system similar to that in~\cite{mare2014zebra}, there are some differences. First, we wanted to use commercially widely available general-purpose smartwatches as bracelets. They tend to be less well-equipped compared to the high-end Shimmer Research bracelet used in~\cite{mare2014zebra}. Our smartwatch has a maximum sampling rate of around 200 Hz, whereas the Shimmer bracelet had a sampling frequency of 500 Hz. We discuss the implications of this difference in \changeMika{Section~\ref{subsec:implementation_differences}}.

In addition, \cite{mare2014zebra} mentions a rate of 21 interactions in a 6s period (3.5 interactions per second). However, in our measurements, users filling standard web forms averaged around 1.5 interactions per second. Their typing interactions were slightly less than 1s long on average and MKKM interactions typically spanned 1-1.5s. With our chosen parameters we could produce a rate of 3.5 interactions per second only in sessions involving hectic activity -- switching extremely rapidly between a few key presses and mouse scrolls. Such a high rate could not be sustained in realistic PC usage.




\subsection{Data Collection}
\label{sec:victim-data}

In our study, we recruited 20 participants to serve as users (victims) of the system.
They were mostly students recruited by word of mouth (ages 20--35, 15 males; 5
females, \changeMika{all right-handed)}. 
Participation was voluntary, based on explicit consent. 
The study included both dexterous typists and less-experienced ones.
Initially, we
told the participants that the purpose of the study was to collect information
on how they typically use a PC. At the end of the study, we explained the actual nature of the experiment. The members of our research groups played the role of the adversary \attacker, \changeOtto{compared to the untrained users in~\cite{mare2014zebra}}.
No feedback was given to \attacker whether a given attack attempt was successful or not.

Experiments were conducted in a realistic office setting (with several
other people working at other nearby desks). During a session, a participant did four
10-minute tasks filling a web form, \changeMika{in a similar setting} as in
\cite{mare2014zebra}. From each task, two sets of user data were collected
simultaneously: accelerometer and gyroscope measurements from the user's
bracelet and the actual interaction sequence extracted by Interaction Extractor
on the terminal.  An attacker \attacker assigned to a participant \victim
conducted each of the four types of attack scenarios from
\ref{subsec:scenarios} in turn. In the first three scenarios, \attacker had direct visual
access to \victim. In the fourth scenario, we placed a narrow shoulder-high partition
between \victim and \attacker so that \attacker can hear but not see \victim.
The 20 sessions thus resulted in a total of 80 samples, with each sample
consisting of three traces: bracelet data of the user, actual interaction
sequence of the user, and the actual interaction sequence of the attacker. All
traces within a sample were synchronized. No other information (e.g., the
content of what the participant typed in) was recorded. Participants were told
what data was collected.


The data collection and the study followed IRB procedures at our institutions.
The data we collected has very little personal information. It is
conceivable that the interaction sequences or bracelet data could potentially
be used to link a participant in our study to similar data from the same
participant elsewhere. For this reason, we cannot \changeAsokan{make our datasets 
 public,} but will make them available to other researchers for research use.








 








\subsection{Performance Evaluation}
\label{sec:performance_evaluation}

\textbf{Usability:} To evaluate {usability}, we follow the same approach as in \cite{mare2014zebra} to compute the false negative rate (FNR) as the fraction of windows in which Authenticator comparing the actual and predicted interaction sequences from the same user incorrectly outputs ``different user.''
We 
employ the leave-one-user-out cross-validation approach: for each session, 
we train a random forest classifier using the 76 samples of bracelet data from all the other 19 sessions. We then use the four samples from the current session to test the classifier. We thus train 20 different classifiers, and report results aggregating classification of 80 samples in all.

\ifllncs
\begin{figure}
\centering
\begin{subfigure}{.5\textwidth}
  \centering
  \includegraphics[width=1\textwidth]{pics/200HZ_victim_all_fnr.eps}
\caption{}
  \label{fig:victim_fnr}
\end{subfigure}\begin{subfigure}{.5\textwidth}
  \centering
  \includegraphics[width=1\textwidth]{pics/200HZ_victim_all_time.eps}
\caption{}
  \label{fig:victim_time}
\end{subfigure}
\caption{Performance for legitimate users: \textbf{(a)} Average FNR for different threshold ()values. \textbf{(b)} Fraction of users remaining logged in after () authentication windows (with , for different grace periods ().}
\end{figure}

\else

\begin{figure}[bh!]
	\centering
	\begin{subfigure}[b]{1\linewidth}
	\centering
		\includegraphics[width=0.7\linewidth]{pics/200HZ_victim_all_fnr.eps}
		\caption{Average FNR vs. window size () for different threshold () values. \changeMika{Fraction of windows that are incorrectly classified as mismatching.}}
		\label{fig:victim_fnr}
	\end{subfigure}

	\begin{subfigure}[b]{1\linewidth}
	\centering
		\includegraphics[width=0.7\linewidth]{pics/200HZ_victim_all_time.eps}
		\caption{Fraction of users remaining logged in after () authentication windows (with , for different grace periods ().}
		\label{fig:victim_time}
	\end{subfigure}
	\caption{Performance for legitimate users}
\end{figure}

\fi



Figure~\ref{fig:victim_fnr} shows how different window size () and matching threshold () values affect average FNR. 
As can be seen, FNR is very low for our system. The original \zebra paper \cite{mare2014zebra} reports FNRs in the range of 0-16\% whereas in our system the FNRs are 0-6\%, and below 1\% for window sizes above 10. 

We also estimated the length of time (in terms of the number of windows) for which a legitimate user remained logged in. For this, we fix  and  as in \cite{mare2014zebra}. On average, a window was 13 seconds long. The low FNRs result in no legitimate users getting logged out in any of the 10 minute samples. Figure~\ref{fig:victim_time} depicts this by plotting the fraction of users still logged in after a given number of authentication windows. The situation is the same when allowing one additional failed authentication window before logging a user out (), or when directly logging the user out after the first failed window (). This also seems in line with the results reported in \cite{mare2014zebra}, where one legitimate user was logged out when using a stricter grace period (). 



Table~\ref{fig:cm} presents the confusion matrix for the classification performance of our Interaction Classifier. It combines data for all 80 (20 x 4) classifications. 
\changeMika{It shows that our system is very} \changeAsokan{good at} \changeMika{recognizing events accurately.}
For example, for the \textit{typing} events, we obtain a precision of 96.9\% () and a recall of 96.5\% ().

\begin{table}[h]
	\centering
	\begin{center}
	\caption{Confusion matrix for 80 legitimate user samples.}
	\begin{tabular}
	{|c| >{\bfseries}r @{\hspace{0.7em}}|c @{\hspace{0.7em}}|c @{\hspace{0.7em}}|c|}
	\hline
	\multirow{7}{*}{\rotatebox{90}{\parbox{1.1cm}{\bfseries\centering Actual}}} & 
	\multicolumn{3}{c}{\bfseries Predicted} & \0.5em]
	\cline{2-5}
	& Typing & 15753 & 354 & 225 \0.5em]
	\cline{2-5}
	& MKKM & 228 & 71 & 15378 \6pt]
	\hline
	Min. duration\footnote{For scrolling, also a minimum of 5 recorded events.} & 25 ms & \cite{mare}\\
	\hline
	Max. duration\footnote{\label{fn:two}For MKKM, a max. duration and idle threshold of 5s.\cite{mare} } & 1 s & \cite{mare2014zebra}\\
	\hline
	Idle threshold\textsuperscript{\ref{fn:two}} & 1 s & \cite{mare}\\	
	\hline
	Window size () & 5-30 & \cite{mare2014zebra}\\
	\hline
	Match threshold () & 50-70\% & \cite{mare2014zebra}\\
	\hline
	Overlap fraction () & 0 & Estimated\footnote{Estimate based on reported \cite{mare2014zebra} times \& authentication windows needed for logging out users.}\\
	\hline
	Grace period () & 1-2 & \cite{mare2014zebra}\\
	\hline
	\end{tabularx}
\end{minipage}
	\label{tab:parameters}
\end{table}


\section{Features}
\label{app:features}
We consider the same features as in \cite{mare2014zebra}, listed in Table \ref{tab:features}. These are extracted from segments of sensor readings and used to classify interactions.
\begin{table}[t!]
\centering
	\caption{Features used in this paper.}
	\begin{tabular}{|l|l|}
	\hline
	\textbf{Feature} & \textbf{Description} \6pt]
	\hline
	Mean & mean value of signal\\
	\hline
	Median & median value of signal\\
	\hline
	Variance & variance of signal\\	
	\hline
	Standard Deviation & standard deviation of signal\\
	\hline
	MAD & median absolute \changeMika{deviation}\\
	\hline
	IQR & inter-quartile range\\
	\hline
	Power & power of signal\\
	\hline
	Energy & energy of signal\\
	\hline
	Peak-to-peak & peak-to-peak amplitude\\
	\hline
	Autocorrelation & similarity of signal\\
	\hline
	Kurtosis & peakedness of signal\\
	\hline
	Skewness & asymmetry of signal\\
	\hline
	\end{tabular}
	\label{tab:features}
\end{table}

\ifllncs
\section{Deauthentication Time}
\label{app:deauthentication_time}

Figure~\ref{fig:kb_smart_time_sec} represents the fraction of logged
in \KBactivity attackers as a function of time (in minutes). 
Figure~\ref{fig:kb_smart_time_sec} corresponds to Figure~\ref{fig:kb_smart_time} in Section~\ref{sec:results}.

\begin{figure}[h]\centering
  \includegraphics[width=\textwidth]{pics/200HZ_attacker_kb_smart_seconds.eps}
\caption{\textbf{Opportunistic \KBactivity} attacker: 
Fraction of attackers remaining logged in after () minutes
  (with , for different grace periods ().}
  \label{fig:kb_smart_time_sec}
\end{figure}



\vspace{-5mm}
\section{Opportunistic all-activity and audio-only attackers}
\label{app:extra_attackers}
In opportunistic \allactivity attack, \attacker's selection
criterion is the likelihood of correctly mimicking \victim. In
particular, \attacker will use the following guidelines:
\begin{itemize}
\itemsep0em
\item Once \attacker successfully mimics a keyboard to mouse interaction, he
  is free to carry out any interaction involving the mouse (scroll,
  drag, move) at will because the bracelet measurements for all
  activities involving the mouse are likely to be similar.
\item If \attacker fails to quickly mimic a keyboard to mouse (or vice
  versa) interaction, he
  does nothing until the next opportunity for an MKKM
  interaction arises (foregoing all interactions until after the MKKM
  is completed).
\end{itemize}


\begin{figure}[htbp]
\centering
\begin{subfigure}{.5\textwidth}
  \centering
  \includegraphics[width=1\textwidth]{pics/200HZ_attacker_all_smart_fpr.eps}
\caption{}
  \label{fig:all_smart_fpr}
\end{subfigure}\begin{subfigure}{.5\textwidth}
  \centering
  \includegraphics[width=1\textwidth]{pics/200HZ_attacker_all_smart_time.eps}
\caption{}
  \label{fig:all_smart_time}
\end{subfigure}
\caption{\textbf{Opportunistic \allactivity} attacker: \textbf{(a)}
  Average FPR for different threshold () values. \textbf{(b)}
  Fraction of attackers remaining logged in after () authentication
  windows (with , for different grace periods ().}
\centering
\begin{subfigure}{.5\textwidth}
  \centering
  \includegraphics[width=1\textwidth]{pics/200HZ_attacker_kb_audioonly_fpr.eps}
\caption{}
  \label{fig:kb_audioonly_fpr}
\end{subfigure}\begin{subfigure}{.5\textwidth}
  \centering
  \includegraphics[width=1\textwidth]{pics/200HZ_attacker_kb_audioonly_time.eps}
\caption{}
  \label{fig:kb_audioonly_time}
\end{subfigure}
\caption{\textbf{Audio-only opportunistic \KBactivity} attacker:
  \textbf{(a)} Average FPR () for different threshold ()
  values. \textbf{(b)} Fraction of attackers remaining logged in after
  () authentication windows (with , for different grace periods ().}
\end{figure}
\fi
 


\end{document}
