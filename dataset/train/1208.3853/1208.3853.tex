\label{sec:algorithm}

In this section we deal with monitoring of an STL* formula, which means to determine the satisfaction of a formula over a finite length signal. 
The monitoring procedure introduced in this paper is inspired by constrained LTL model-checking~\cite{Calzone} and monitoring of STL formulae~\cite{Maler_STL}, but needed to be extended into 2D space due to freeze operator . It is because the task is solved simultaneously for two time instants, actual time and frozen time.

The idea of the monitoring procedure is to construct a parse tree of the formula and check the satisfaction in a bottom-up manner. First step in checking a formula  over a signal  is to construct a set of time points in which the formula  is satisfied.

\begin{definition}
	Let  be a STL* formula and  be a real-valued signal. A~set  is called the \emph{satisfaction set} of formula  over signal . We use short-form notation  whenever the signal  is obvious from the context.
\end{definition}

Now we inductively construct satisfaction sets for nodes on higher levels of the parse tree. The last step of the procedure is to decide if  from satisfaction set for the formula . According to Definition \ref{defSatisfactionSignal}, it is equivalent to checking whether the satisfaction set contains the point .

Constructing the satisfaction set for a general signal can be a very difficult task. For this reason we will consider the monitoring algorithm only for piecewise linear signals. This is a reasonable requirement because in most cases we deal with time series produced by numerical simulations of modelled systems or by some measurements. These series can be interpreted as piecewise linear signals considering the values changing linearly between two adjacent points. If required, other points can be generated in between the existing points to make the signal more precise. 


\begin{definition}
	A real-valued signal  of order  is called \emph{piecewise linear signal} iff all the projections  are piecewise linear functions defined on a finite set of intervals  where  and .
\end{definition}

\begin{theorem}[Inductive construction of the satisfaction set]
	\label{monitoring_STL*}
	Let  be a piecewise linear signal and  a formula in STL*. The satisfaction set  of the formula  over the signal  can be constructed inductively with respect to the structure of the formula :       
	

	\begin{proof}
		Each of the equations follows directly from the definition of semantics of STL* \eqref{STL*_sematics}.
	\end{proof}
\end{theorem}

The assumption of piecewise linear signals in combination with linearity of atomic predicates (Definitions \ref{defLinearPredicate}, \ref{defPredicate}) enables us to construct satisfaction sets for atomic predicates in polynomial time and to easily compute sets belonging to higher formulae. The formalism we use for the representation of satisfaction sets are convex polytopes.

\begin{definition}
	\label{defConvexPolytope}
	\emph{Convex polytope}  is a set  where  is a matrix and  is a  column. A convex polytope in  is called a \emph{line segment}, in  a \emph{convex polygon}.
\end{definition}

Each inequality  in Definition \ref{defConvexPolytope} identifies a subspace of . The convex polytope is obtained as intersection of these subspaces.

\begin{theorem}
	\label{lemma}
	Assume  a piecewise linear signal and  the set of intervals on which  is defined. Let  be an atomic predicate over the signal . The predicate  is a linear function over the signal  in variables  and  according to Definition \ref{defPredicate}. Denote  where  the set of time instants at which the atomic predicate  is satisfied over signal  on rectangular area  and  its complement. For  there can arrive two situations:

\begin{enumerate}
	\item if , then the set  makes a line segment (possibly degenerated to a single point or an empty set); 
	\item if , then the space  is divided by the line  into two subspaces  and  (Figure \ref{figCut}). (Again, there might be a degenerated case when the line  does not cross the rectangle . In this case  and  or vice versa).
\end{enumerate}

\begin{proof}
		Described properties result from the linear form of atomic predicates \ref{defPredicate} and from the fact that signal  behaves linearly on each interval .
	\end{proof}

\end{theorem}

\begin{figure}[h]
\vspace*{-3mm}
	\begin{center}
    \includegraphics[scale=0.4]{figures/cut.png}
     \caption{An illustration of the set  for a formula  on the subspace .}
    \label{figCut}
    \end{center}
\vspace*{-3mm}
\end{figure}


In both cases, the set  makes a convex set easily describable by set operations over convex polytopes. In the first case, a line segment itself is a convex polytope. In the second case, either the set  is a convex polytope, or (when ) it is a convex polytope after subtraction of one bordering line segment, which is a polytope.

Using this algorithm we can express the satisfaction set  as , where  are the sets described above constructed by set operations on polytopes. The set  can be further simplified by joining adjacent polytopes. Having satisfaction sets for atomic predicates, we could inductively construct satisfaction sets for composite formulae from Theorem \ref{monitoring_STL*} again as operations on polytopes, which could be achieved in polynomial time~\cite{GeometricAlgorithms}. 

However, working with precise satisfaction sets as defined in Theorem \ref{monitoring_STL*} can be quite a difficult task. We have to work with polytopes of different dimensions (polygons, lines and points) and operations over sets of these objects can be unnecessarily demanding. In real-world applications, we could end up performing an expensive monitoring over noisy signals measured on imperfect devices or computed on computers with finite numerical precision. Hence think about less precise, yet faster monitoring algorithm.



We can imagine a \emph{noisy signal} as a signal measured or computed with finite precision. The actual values of the variables can differ from the values presented by the signal up to some error . This inaccuracy in the domain of values implies also inaccuracy in the time domain. When investigating the time instant in which some event occurred (e.g., ) we cannot be sure when exactly it happened. That is because the critical value  is burdened by an error and it is possible that , but also that  or  in that particular time instant.





Imagine we are in the situation from Theorem \ref{lemma}, but working with a noisy signal. We cannot be sure about the border of the set  due to the error . The condition  defining points on the border of  could be easily broken by changing the values of the signal  by arbitrary small values from . For this reason, we will ignore these border points and count them as they were in  or in  depending on what would be computationally easier. As a consequence, in the first case in Theorem \ref{lemma}, , and in the second case, it does not matter if  or .

As a result of this simplification, the set of allowed operators in atomic predicates is restricted to . The remaining three operators lost their sense. With signals burden with some error  the satisfaction of atomic predicates can be meaningfully determined only for time instants inside the satisfaction set. 
Hence the satisfaction of a formula using the operator , e.g.,  or , cannot be determined. Every reference to a precise value has to be replaced by a sufficiently large interval, e.g., formula  can be replaced by  for some .

Based on these assumptions, monitoring can be solved approximately. For every pair of intervals , the satisfaction set on this area can be described by a single convex polygon . For the whole satisfaction set  we get . The problem of construction of satisfaction sets for composite formulae from Theorem \ref{monitoring_STL*} is reduced to operations with sets of convex polytopes in plane for which efficient algorithms exist~\cite{GeometricAlgorithms}. 
 
\begin{algorithm}[Approximative monitoring algorithm for piecewise linear signals]
\label{algorithm1}

\end{algorithm}
\smallskip
\textbf{Input:} A piecewise linear signal  and an STL* formula .\\
\textbf{Output:} Answer to the question .\\

Algorithm inductively constructs the satisfaction set :\\

All the computations are performed on parts of the signal  of sufficient length. This can be computed according to \eqref{sufficientSignalLength}. If the signal does not have the sufficient length, the answer returned by the algorithm might be wrong. 


\begin{enumerate}


	\item . The computation of the satisfaction sets  for individual intervals is performed according to Theorem \ref{lemma}, but without determining the satisfaction of the borders (as was justified in this section).

	\item . The result can be computed as a simple Boolean operation on polygons. It is necessary to ensure that the resulting set consists only of convex polygons, which could be achieved for example by triangulation~\cite{GeometricAlgorithms}. 
	
	\item . Again a Boolean polygonal operation. 

	\item . The satisfaction set for the freeze operation can be computed by: (1) finding an intersection between the line  and the satisfaction set  (black line segments in Figure~\ref{figFreezeSS-a}), (2) substituting values in the second component by the whole axis , i.e., making a projection to the first component  and then a Cartesian product with the axis  (Figure \ref{figFreezeSS-b}).  
	
	
	
\begin{figure}[h]
   \centering  
	\subfloat[][]{\includegraphics[scale=0.4]{figures/ss1.png}
			\label{figFreezeSS-a}}
    \subfloat[][]{\includegraphics[scale=0.4]{figures/ss2.png}
			\label{figFreezeSS-b}}
    	
      	\caption{Example of computing satisfaction set for the formula . a) The satisfaction set represented as a set of (overlapping) convex polygons (different colors are used only for depicting the fact that the set consists of convex polygons) and the line . The regions of intersection of the line and the satisfaction set are depicted in black color. b) Resulting set  obtained by projecting the intervals of intersection to the axis  and making a Cartesian product with the axis .}
	\label{figFreezeSS}    
    
\end{figure}



	\item \label{step} . To be more illustrative, we explain this part of the algorithm from the geometrical point of view, identifying values  with the horizontal axis and values  with the vertical axis.
	
	
	
\begin{figure}[h]
   \centering  
	\subfloat[][]{\includegraphics[scale=0.4]{figures/alg1.png}
			\label{figAlgorithm1-a}}
    \subfloat[][]{\includegraphics[scale=0.4]{figures/alg2.png}
			\label{figAlgorithm1-b}}
    \subfloat[][]{\includegraphics[scale=0.4]{figures/alg3.png}
			\label{figAlgorithm1-c}}
      	\caption{First step of computation of  is to find the intersection .}
	\label{figAlgorithm1}    
    
\end{figure}	

\begin{figure}[h]
   \centering  
	\subfloat[][]{\includegraphics[scale=0.4]{figures/alg4.png}
			\label{figAlgorithm2-a}}
    \subfloat[][ordered vertices]{\includegraphics[scale=0.4]{figures/alg5.png}
			\label{figAlgorithm2-b}}
    \subfloat[][rectangular polygons (stripes) \\]{\includegraphics[scale=0.4]{figures/alg6.png}
			\label{figAlgorithm2-c}}
      	\caption{Next, the space is divided into horizontal stripes. The task is solved in each stripe separately.}
	\label{figAlgorithm2}    
    
\end{figure}
	
	For each convex polygon  (Figure~\ref{figAlgorithm1-c}) the following procedure is performed:
		 \begin{enumerate}
			\item Vertices of polygons  are increasingly ordered by the second component  (duplicates can be removed). Denote the ordered set  (Figures~\ref{figAlgorithm2-a},~\ref{figAlgorithm2-b});
			\item For  every neighbouring pair of vertices  specifies a rectangular polygon (stripe):  (Figure~\ref{figAlgorithm2-c}). 
			 in a form of line segment is considered empty because we do not care about the border points.


			\item The task is solved in every stripe  separately. For every nonempty  denote  and . Due to the way of construction of , all polygons in  and the single polygon  have all their vertices on the upper () or the lower () border of the stripe  (Figure~\ref{figAlgorithm3-a}). In fact, these polygons can take only the shape of a triangle or a trapezoid (Figure \ref{figStripe}).\\
			
\begin{figure}[h]
   \centering  
	\subfloat[][the stripe ]{\includegraphics[scale=0.4]{figures/alg7.png}
			\label{figAlgorithm3-a}}
    \subfloat[][points on the left side of the line ]{\includegraphics[scale=0.4]{figures/alg8.png}
			\label{figAlgorithm3-b}}
    \subfloat[][ the rightmost polygon in ]{\includegraphics[scale=0.4]{figures/alg9.png}
			\label{figAlgorithm3-c}}
      	\caption{The potential polygons are identified. (The solution cannot be placed further in time than the property . Moreover, the property  must continuously hold at all time points until  is satisfied.)}
	\label{figAlgorithm3}    
    
\end{figure}				
\enlargethispage*{5mm}			
Now we get rid of the polygons lying on the right side of  because their points cannot be in the solution (they represent the time past the event ). To do so, we isolate the upper and lower rightmost vertices of the polygon . Denote them as  and  (Figure~\ref{figAlgorithm3-b}). We get a new area:



 The solution can lie only in this area, so we can restrict  to . 
			
\begin{figure}[h]
	\begin{center}
    	\includegraphics[scale=0.35]{figures/stripe.png}
      	\caption{An example of a stripe and all possible shapes of polygons inside a stripe.}
      	\label{figStripe}
    \end{center}
\end{figure}
			
			
			\item All the adjoining polygons (sharing an edge) in  are connected to form maximal seamless convex polygons (e.g., the green and orange polygons in Figure~\ref{figAlgorithm3-b}).  
			
			\item Let  be the rightmost polygon in  (Figure~\ref{figAlgorithm3-c}) (Even after connecting some polygons in  their shape can still be only triangular or trapezoidal (Figure \ref{figStripe}) so there is only one rightmost polygon in ). Then  is the solution for the stripe  (Figure~\ref{figAlgorithm4}).\\
The operation  is defined as  which is equal to , where  is the Minkowski sum.
			
			\item The final satisfaction set is given by .   
			
\begin{figure}[h]
   \centering  
	\subfloat[][ before application of the operator ]{\includegraphics[scale=0.4]{figures/alg10.png}
			\label{figAlgorithm4-a}}
    \subfloat[][the shifted polygon ]{\includegraphics[scale=0.4]{figures/alg11.png}
			\label{figAlgorithm4-b}}
    \subfloat[][, the final solution in the stripe ]{\includegraphics[scale=0.4]{figures/alg12.png}
			\label{figAlgorithm4-c}}
      	\caption{Last step is to consider the interval  bounding the operator .}
	\label{figAlgorithm4}    
    
\end{figure}			
			
 \end{enumerate}

	\item The result  which is equivalent to  is returned.

\end{enumerate}	


\begin{remark}

The satisfaction set for temporal operator future  (and hence ) can be computed more easily than according to the step \ref{step}. It can be obtained directly as .


\end{remark}



\subsection{Time and space complexity}
\enlargethispage{8mm}
Time and space complexity of all steps of the Algorithm \ref{algorithm1} is proportional to the number of polygons they are working with. Number of polygons generated in step 1 is at most  for each atomic predicate, where  is the number of intervals on which the signal is defined.

The number of polygons does not asymptotically change during the computation of operations described in other steps of the algorithm and all these functions do not work slower than  (see~\cite{GeometricAlgorithms} for more details). The output polygons also keep small number of vertices during the process.

The upper bound on the total computation time is therefore , where  is the number of intervals on which the signal is defined and  is the size of the investigated formula.  

