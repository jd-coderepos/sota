






\documentclass[journal]{IEEEtran}

\usepackage[dvips]{graphics}
\usepackage{epsfig}
\usepackage{amsmath}

















\ifCLASSINFOpdf
\else
\fi

















































\hyphenation{op-tical net-works semi-conduc-tor}


\begin{document}
\title{A Kalman Filtering approach of improved precision
for fault diagnosis in distributed parameter systems}

\author{Gerasimos G. Rigatos,~\IEEEmembership{Member,~IEEE}
        \thanks{G. Rigatos is with the Unit of Industrial Automation, Industrial Systems Institute, 26504, Rion Patras, Greece, Email: grigat@ieee.org.}}



\markboth{}{Shell \MakeLowercase{\textit{et al.}}: Bare Demo of IEEEtran.cls for Journals}











\maketitle


\begin{abstract}
The Derivative-free nonlinear Kalman Filter is proposed for state estimation and fault diagnosis in distributed parameter systems and particularly in dynamical systems described by partial differential equations of the nonlinear wave type. At a first stage, a nonlinear filtering approach for estimating the dynamics of a 1D nonlinear wave equation, from measurements provided from a small number of sensors is developed. It is shown that the numerical solution of the associated partial differential equation results into a set of nonlinear ordinary differential equations. With the application of diffeomorphism that is based on differential flatness theory it is shown that an equivalent description of the system is obtained in the linear canonical (Brunovsky) form. This transformation enables to obtain local estimates about the state vector of the system through the application of the standard Kalman Filter recursion. At a second stage, the local statistical approach to fault diagnosis is used to perform fault diagnosis for the distributed parameters system by processing with elaborated statistical tools the differences (residuals) between the output of the Kalman Filter and  the measurements obtained from the distributed parameter system. Optimal selection of the fault threshold is succeeded by using the local statistical approach to fault diagnosis. The efficiency of the proposed filtering approach for performing fault diagnosis in distributed parameters systems is confirmed through simulation experiments.
\end{abstract}

\begin{IEEEkeywords}
distributed parameter systems, differential flatness theory, derivative-free nonlinear Kalman Filtering, nonlinear wave equations, local statistical approach to fault diagnosis, fault detection, fault isolation.
\end{IEEEkeywords}






\IEEEpeerreviewmaketitle

\section{Introduction} \label{section 1: Introduction}

\noindent Fault diagnosis in distributed parameter systems is a complicated problem that has been little explored up to now. A reason for this is that state estimation methods used for residual generation in distributed parameter systems and in infinite dimensional systems described by partial differential equations are much more complicated than state estimation methods for lumped parameter systems [\ref{WoiRud12}-\ref{HaiRam10}]. Of particular interest is state estimation of wave-type nonlinear phenomena, appearing in several engineering applications [\ref{HidBabSchuNun11}-\ref{Chau11}]. The paper treats the problem of estimation and fault diagnosis for 1D nonlinear infinite dimensional systems, which are described by Partial Differential Equations (PDEs) of the wave type.

\noindent At a first stage the dynamics of the PDE model is computed through a state estimator that processes measurements from a small number of sensors. To this end the following steps are followed. Using the method for numerical solution of the PDE through discretization the initial partial differential equation is decomposed into a set of nonlinear ordinary differential equations with respect to time [\ref{Pin91}]. Next, each one of the local models associated with the ordinary differential equations is transformed into a model of the linear canonical (Brunovsky) form through a change of coordinates (diffeomorphism) which is based on differential flatness theory. This transformation provides an extended model of the nonlinear PDE for which state estimation is possible by application of the standard Kalman Filter recursion [\ref{MouRud01}-\ref{MarTom92}]. Unlike other nonlinear estimation methods (e.g. Extended Kalman Filter) the application of the standard Kalman Filter recursion to the linearized equivalent of the nonlinear PDE system does need the computation of Jacobian matrices and partial derivatives [\ref{Rig11}-\ref{Rig12a}].



\noindent At a second stage, the paper proposes the \textit{local statistical approach to fault diagnosis} for detecting changes and faults in the distributed parameter system  [\ref{ZhaBasBen98}-\ref{BasBenZha96}]. Residuals are generated by comparing the outputs measured from the distributed parameter system against the outputs obtained from the derivative-free nonlinear Kalman Filter. The processing of these differences through the local statistical approach to fault diagnosis provides clear indications about the existence of incipient changes in the model of the monitored PDE. Fault diagnosis with the Local Statistical Approach has two significant advantages: i) it provides a credible criterion ( test) to detect if faults have taken place in the distributed parameters system. This criterion is more efficient than the normalized square error and mean error tests since it employs the modeling error derivative and records the tendency for change. Thus early change detection for the filter's parameters becomes possible ii) it recognizes the parameters of the PDE model which are responsible for the deviation of the filter's estimates from the real output of the monitored dynamical system.

\noindent The structure of the paper is as follows: In Section \ref{section 2: Filtering_using_differential_flatness_theory_and_canonical_forms} nonlinear filtering using differential flatness theory and transformation of the system's dynamics into canonical forms is analyzed. A new filtering method, under the name Derivative-free distributed nonlinear Kalman Filtering, is proposed for distributed state estimation. In Section \ref{Section 3: Estimation_of_nonlinear_wave_equations} it is shown how the Derivative-free distributed nonlinear Kalman Filtering can be used for estimating the dynamics of distributed parameter systems as the ones described by 1D nonlinear PDEs of the wave type. In Section \ref{section 4 : Equivalence_between_Kalman_Filters_and_regressor_models} the equivalence between Kalman Filtering and regressor models is analyzed. In Section \ref{section 5 : Change_detection_using_the_local_statistical_approach} the local statistical approach is introduced as a systematic method for performing fault detection and isolation in dynamical systems. The method is proposed also for diagnosing faults in distributed parameter systems described by PDEs. In Section  \ref{section 6 : Simulation_tests} simulation tests are presented about the performance of the Derivative-free distributed nonlinear Kalman Filter in the problem of state estimation of the wave-type type partial differential estimation and about detecting and isolating parametric changes in such systems. Finally, in Section \ref{Section 7: Conclusions} concluding remarks are stated.
























































































\section{Filtering using differential flatness theory and canonical forms} \label{section 2: Filtering_using_differential_flatness_theory_and_canonical_forms}





\subsection{Conditions for applying the differential flatness theory} \label{subsection 3.3: diffeomorphism_canonical_form}

\noindent Next, a new filter will be developed, in accordance to the differential flatness theory. It will be shown that the filter can be efficiently used in the problem of state estimation in distributed parameter systems. First, the  generic class of nonlinear systems  (including MIMO systems) is considered. Such systems can be transformed to the form of an affine in-the-input system by adding an integrator to each input [\ref{Lev10}],[\ref{BouBouZheBarKra11}]



\noindent The following definitions are now used [\ref{Rig12a}]:\\

\noindent (i) Lie derivative:   stands for the Lie derivative  and the repeated Lie derivatives are recursively defined as , . \\

\noindent (ii) Lie Bracket:  stands for a Lie Bracket which is defined recursively as  with  and .  \\

\noindent If the system of Eq. (\ref{affine_in_the_input_system}) can be linearized by a diffeomorphism  and a static state feedback  into the following form



\noindent with , then  for  are the 0-flat outputs which can be written as functions of only the elements of the state vector . To define conditions for transforming the system of Eq. (\ref{affine_in_the_input_system}) into the canonical form described in Eq. (\ref{system_transformed_in_canonical_form}) the following theorem holds [\ref{BouBouZheBarKra11}] \\

\noindent \textit{Theorem}: For nonlinear systems described by Eq. (\ref{affine_in_the_input_system}) the following variables are defined: (i) , (ii) , 
(k) .
Then, the linearization problem for the system of Eq. (\ref{affine_in_the_input_system})
can be solved if and only if: (1). The dimension of  is constant for  and for , (2). The dimension of  if of order , (3). The distribution  is involutive for each .

\subsection{Transformation of MIMO systems into canonical forms} \label{subsection 3.4: transformation into the Brunovksy form}

\noindent It is assumed now that after defining the flat outputs of the initial MIMO nonlinear system and after expressing the system state variables and control inputs as functions of the flat output and of the associated derivatives, the system can be transformed in the Brunovsky canonical form:

 \\
 \\
 \\
 \\
\\


\noindent where  is the state vector of the transformed system (according to the differential flatness formulation),  is the set of control inputs,  is the output vector,  are the drift functions and  are smooth functions corresponding to the control input gains, while  is a variable associated to external disturbances. It holds that . Having written the initial nonlinear system into the canonical (Brunovsky) form it holds



\noindent Next the following vectors and matrices can be defined:
, , with  , , , where matrix  has the MIMO canonical form, i.e. with block-diagonal elements



\noindent Thus, Eq. (\ref{compact_Brunovsky_form_dynamics}) can be written in state-space form



\noindent where the control input is written as . The system of Eq. (\ref{matrix_A_B_C_MIMO_canonical_form}) and Eq. (\ref{Brunovsky_form_MIMO_system}) is
in controller and observer canonical form.

\subsection{Derivative-free nonlinear Kalman Filtering} \label{subsection : flatness_based_control_2}

\noindent As mentioned above, for the system of Eq. (\ref{Brunovsky_form_MIMO_system}), state estimation is possible by applying the standard Kalman Filter. The system is first turned into discrete-time form using common discretization methods and then the recursion of the linear Kalman Filter described in Eq. (\ref{KF_meas_update}) and Eq. (\ref{KF_time_update}) is applied. \\

\noindent If the derivative-free Kalman Filter is used in place of the Extended Kalman Filter then in the EKF equations the following matrix substitutions should be performed: , , where  matrices  and  are the discrete-time equivalents of matrices  and  which have been defined Eq. (\ref{Brunovsky_form_MIMO_system}) and which appear also in the measurement and time update of the standard Kalman Filter recursion. Matrices  and  can be computed using established discretization methods. Moreover, the covariance matrices  and  are the ones obtained from the linear Kalman Filter update equations given in Section \ref{Section 3: Estimation_of_nonlinear_wave_equations}.\\

\section{Estimation of nonlinear wave dynamics} \label{Section 3: Estimation_of_nonlinear_wave_equations}

\noindent The following nonlinear wave equation is considered



\noindent Using the approximation for the partial derivative



\noindent and considering spatial measurements of variable  along axis  at points  one has



\noindent By considering the associated samples of  given by 
one has



\noindent By defining the following state vector



\noindent one obtains the following state-space description



\noindent Next, the following state variables are defined



\noindent and the state-space description of the system becomes as follows



\noindent The dynamical system described in Eq. (\ref{state_space_description_v2}) is a differentially flat one with flat output defined as the vector . Indeed all state variables can be written as functions of the flat output and its derivatives. \\

\noindent Moreover, by defining the new control inputs



\noindent the following state-space description is obtained

\\

\end{tabular}
 \label{measurement_equation}
\begin{tabular}{c}

\end{tabular}
 \label{state_space_description_matrix_form}
\begin{tabular}{c}
 \\

\end{tabular}
  \label{state_space_description_canonical_form_v2}
\begin{tabular}{c}
\\


\noindent The associated control inputs are defined as



\noindent By selecting measurements from a subset of points , the associated observation (measurement) equation remains as in Eq. (\ref{measurement_equation}), i.e.




\noindent For the linear description of the system in the form of Eq. (\ref{state_space_description_matrix_form}) one can perform estimation using the standard Kalman Filter recursion. The discrete-time Kalman filter can be decomposed into two parts: i) time update (prediction stage), and ii) measurement update (correction stage).\\

\noindent \textit{measurement update}:



\noindent \textit{time update}:



\noindent Therefore, by taking measurements of  at time instant  at a small number of measuring points  it is possible to estimate the complete state vector, i.e. to get values of  in a mesh of points that covers efficiently the variations of . By processing a sequence of output measurements of the system, one can obtain local estimates of the state vector . The measuring points (active sensors) can vary in time provided that the observability criterion for the state-space model of the PDE holds.\\

\noindent \textit{Remark}: The proposed derivative-free nonlinear Kalman Filter is of improved precision because unlike other nonlinear filtering schemes, e.g. the Extended Kalman Filter it does not introduce cumulative numerical errors due to approximative linearization of the system's dynamics. Besides it is computationally more efficient (faster) because it does not require to calculate Jacobian matrices and partial derivatives.


\begin{figure}[htb]
\begin{center}
\rotatebox{-90}{\epsfig{file=measurement_grid.eps, width=60mm,
height=70mm}}
\end{center}
\caption{Grid points for measuring } \label{figure : Grid_points}
\end{figure}

\section{Equivalence between Kalman filters and regressor models}  \label{section 4 : Equivalence_between_Kalman_Filters_and_regressor_models}

\subsection{Equivalence between the standard Kalman Filter and linear regressor models}

\noindent For fault diagnosis purposes it is convenient to turn the Kalman Filter model of distributed parameter systems into equivalent ARMAX (autoregressive moving average model with auxiliary input) models. An ARMAX model is an input-output model of the form



\noindent , ,  are polynomial matrices in the backwards shift operator :



\noindent such that  has non-singular constant term  and where  is a white noise sequence with covariance matrix . A state-space model and particularly the Kalman Filter estimator can be written in the form of an ARMAX model. For linear systems, the Kalman Filter (for the single-input case) can be written in the form [\ref{BasNik93}]

\\

\end{tabular}
 \label{Kalman_Filter2}
\begin{tabular}{c}

\end{tabular}
 \label{ARMAX_model_Kalman_Filter_1}
A(z){Y_k}=C(z){U_k}+B(k,z){\epsilon_k}
 \label{ARMAX_model_Kalman_Filter_2}
\begin{tabular}{c}
 \\
 \\
\\

\end{tabular}
 \label{ARMAX_model_Kalman_Filter_3}
\begin{tabular}{c}

\end{tabular}
 \label{partial_derivative_residual}
\begin{tabular}{c}

\end{tabular}
 \label{partial_derivative_residual}
\begin{tabular}{c}

\end{tabular}
 \label{derivative_output_weight}
{{{\partial}{\hat{y}}} \over {{\partial}w_i}}={{x_i}}
 \label{Jacobian}
{J(\theta_0,\hat{y}_k)}={\partial {\hat{y}_k(\theta)} \over {\partial{\theta}} } {\Bigg\vert}_{\theta={\theta_0}}
 \label{Gaussian_variable}
X={1 \over {\sqrt{N}}}{\sum_{i=1}^N}{e_k}{{{\partial}{\hat{y}_k}} \over {{\partial}{\theta}}}{\sim}\emph{N}(\mu,\sigma^2)
\label{hypothesis_test}
\begin{tabular}{c}
 \\

\end{tabular}

& H_0 : X \sim \emph{N}(0,S)  \\
& H_1 : X \sim \emph{N}(M{\delta}{\theta},S)
 \label{sensitivity_matrix}
\begin{tabular}{c}

\end{tabular}
 \label{covariance_matrix}
\begin{tabular}{l}
\\
\\

\end{tabular}
 \label{chi2_test}
\begin{tabular}{c}

\end{tabular}
 \label{alarm_threshold}
\begin{tabular}{c}
,
\end{tabular}

\mu=MA{\phi},  \ \
A=[0, I]^T
 \label{chi2_test_sensitivity}
\begin{tabular}{c}

\end{tabular}

{M^T}{S^{-1}}M=
\begin{pmatrix}
I_{\varphi\varphi} & I_{\varphi\psi} \\
I_{\psi\varphi}    & I_{\psi\psi}    \\
\end{pmatrix}

\gamma=
\begin{pmatrix}
\varphi \\ \psi
\end{pmatrix}^T \ \cdot
\begin{pmatrix}
I_{\varphi\varphi} & I_{\varphi\psi} \\
I_{\psi\varphi}    & I_{\psi\psi}
\end{pmatrix}  \cdot
\begin{pmatrix}
\varphi \\ \psi
\end{pmatrix}

\psi^{*}=\arg\min\limits_{\psi}{\gamma}={\varphi^T}(I_{\varphi\varphi}-I_{\varphi\psi}
{I_{\psi\psi}^{-1}}I_{\psi\varphi})\varphi

\begin{tabular}{l}
\\

\end{tabular}

\begin{tabular}{l}
\\

\end{tabular}

X_{\phi}^{*}={[\it{I},-I_{\varphi\psi}{I_{\psi\psi}^{-1}}]}{M^T}{\Sigma^{-1}}X

{\mu_{\varphi}^{*}}={I_{\varphi}^{*}}\varphi

{I_{\varphi}^{*}}=I_{\varphi\varphi}-I_{\varphi\psi}{I_{\psi\psi}^{-1}}I_{\psi\varphi}

& H_0^{*} : \mu^{*}=0 \\
& H_1^{*} : \mu^{*}=I_{\varphi}^{*}{\varphi}
 \label{chi2_test_min_max}
\tau_{\varphi}^{*}={{X_{\varphi}^{*}}^T}{{I_{\varphi}^{*}}^{-1}}{X_{\varphi}^{*}}
 \label{nonlinear_1D_wave_PDE}
\begin{tabular}{c}

\end{tabular}

\begin{tabular}{c}
 \\

\end{tabular}
  \label{discrete_state_space_model1}
\begin{tabular}{c}

\end{tabular}
 \label{discrete_state_space_model2}
\begin{tabular}{c}

\end{tabular}
 \label{ARMA_model1}
\begin{tabular}{c}

\end{tabular}
 \label{discrete_state_space_model1_KF}
\begin{tabular}{c}
\\

\end{tabular}
 \label{discrete_state_space_model2_KF}
\begin{tabular}{c}

\end{tabular}
\label{ARMA_model_KF}
\begin{tabular}{c}
\\

\end{tabular}
\label{ARMA_model_KF_v2}
\begin{tabular}{c}

\end{tabular}
\label{weights_vector}
\begin{tabular}{l}
\\

\end{tabular}
 \label{regressor_vector}
\begin{tabular}{l}
\\

\end{tabular}


\noindent Indicative results about the performance of the global  in detection of incipient changes of parameter  of the nonlinear wave PDE are depicted in Fig. \ref{fig: global_chi2_test}. The fault threshold is set equal to the number of parameters in the associated ARMAX model, i.e. . It can be observed that the proposed FDI method was capable of detecting changes in parameter  which were less than  of the coefficient's nominal value. For small deviations from the parameter's nominal value the  test obtained a value that was several times larger than the fault threshold.

\begin{figure} [htb]
\begin{center}

\end{center}
\caption{Detection of incipient changes in the coefficient  of the nonlinear wave equation using the  test (a) nominal value   (b)  nominal value }
\label{fig: global_chi2_test}
\end{figure}


\section{Conclusions} \label{Section 7: Conclusions}

\noindent The paper has proposed state estimation and fault diagnosis in distributed parameter systems with a new nonlinear filtering approach, the so-called Derivative-free nonlinear Kalman Filter. The method is based into decomposition of the nonlinear Partial Differential equation that describes the dynamics of the distributed parameter system, into a set of nonlinear ordinary differential equations. Next, with the application of a change of coordinates (diffeomorphism) that is based on differential flatness theory, the local nonlinear differential equations are turned into linear ones. This enables to describe the PDE dynamics with a state-space equation that is in the linear canonical (Brunovsky) form. For the linearized equivalent of the PDE system it is possible to perform state estimation with the use of the standard Kalman Filter recursion. Unlike other nonlinear filtering methods, such as the Extended Kalman Filter, the proposed approach does not require the computation of partial derivatives and Jacobian matrices. Moreover, it avoids the cumulative numerical errors which appear in distributed Extended Kalman Filtering and which are due to truncation of higher order terms in the Taylor expansion of the system's dynamical model. Thus, the proposed filtering method succeeds improved accuracy in the estimates of the dynamics of the distributed parameters system.

\noindent Next, the \textit{local statistical approach to fault diagnosis} has been proposed for performing fault diagnosis of distributed parameter systems. To this end, statistical processing of the differences (residuals) between the Kalman filter's outputs and measurements from the distributed parameters system have undergone statistical processing. Fault diagnosis with the Local Statistical Approach has two significant advantages: (i) it provides a credible criterion ( test) to detect changes in the parameters of the PDE system. This criterion is more efficient than the normalized square error and mean error tests since it employs the modeling error derivative and records the tendency for change. Thus early change detection for distributed parameters system becomes possible (ii) it recognizes the parameters of the PDE system that have undergone a change. Thus fault isolation becomes possible as well. The efficiency of the Derivative free nonlinear Kalman Filter in fault diagnosis  has been confirmed through simulation experiments in the case of distributed parameter systems described by 1D-wave equations.


\begin{thebibliography}{100}

\bibitem{1} \label{WoiRud12}
F. Woittennek and J. Rudolph, Controller canonical forms and flatness-based state feedback for 1D hyperbolic systems,
7th Vienna International Conference on Mathematical Modelling, MATHMOD 2012.

\bibitem{2} \label{BerChoFerGerMoi12}
C. Bertoglio, D. Chapelle, M.A. Fernandez, J.F. Gerbeau and P. Moireau, State observers of a vascular fluid-structure interaction model through measurements in the solid, INRIA research report no 8177, Dec. 2012.

\bibitem{3} \label{SalMayOxl10}
S.A. Salberg, P.S. Maybeck and M.E. Oxley, Infinite-dimensional sampled-data Kalman Filtering and stochastic
heat equation, 49th IEEE Conference on Decision and Control, Atlanta, Georgia, USA, Dec. 2010.

\bibitem{4} \label{YuCha12}
D. Yu and S. Chakravotry, A randomly perturbed iterative proper orthogonal decomposition technique for filtering distributed parameter systems, American Control Conference, Montreal, Canada, June 2012.

\bibitem{5} \label{WuWanLi12}
H.N. Wu, J.W. Wang and H.K. Li, Design of distributed  fuzzy controllers with constraint for nonlinear hyperbolic PDE systems, Automatica, Elsevier, vbl. 48, pp. 2535-2543, 2012.

\bibitem{6} \label{HaiRam10}
G. Haine, Observateurs en dimension infinie. Application \`{a} l \'{e}tude de quelques probl\`{e}mes inverses, Th\`{e}se de doctorat, Institut Elie Cartan Nancy, 2012.

\bibitem{7} \label{HidBabSchuNun11}
Z. Hidayat, R. Babuska, B. de Schutter and A. Nunez, Decentralized Kalman Filter comparison for distributed parameter systems: a case study for a 1D heat conduction process, Proc. of the 16th IEEE Intl. Conference on Emerging Technologies and Factory Automatio, ETFA 2011, Toulouse, France, Sep. 2011.

\bibitem{8} \label{Dem10}
M.A. Demetriou, Design of consensus and adaptive consensus filters for distributed parameter systems, Automatica, Elsevier, vol. 46, pp. 300-311, 2010.

\bibitem{9} \label{GuoXuHam12}
B.Z. Guo, C.Z. Xu and H. Hammouri, Output feedback stabilization of a one-dimensional wave equation with an arbitrary
time-delay in boundary observation, ESAIM: Control, Optimization and Calculus of Variations, vol. 18, pp. 22-25, 2012.

\bibitem{10} \label{Chau11}
J. Chauvin, Observer design for a class of wave equations driven by an unknown periodic input, 18th World Congress, Milano, Italy, Sep. 2011.

\bibitem{11} \label{Pin91}
M. Pinsky, Partial differential equations and boundary value problems, Prentice-Hall, 1991.


\bibitem{12} \label{MouRud01}
H. Mounier and J. Rudolph, Trajectory tracking for -flat nonlinear dealy systems with a motor example, In: A.
Isidori, F. Lamnabhi-Lagarrigue and W. Respondek, editors, Nonlinear control in the year 2000, vol.1, Lecture
Notes in Control and Inform. Sci.,vol. 258, pp. 339-352, Springer, 2001.


\bibitem{13} \label{Rud03}
J. Rudolph, Flatness Based Control of Distributed Parameter Systems, Steuerungs- und Regelungstechnik,
\textit{Shaker Verlag}, Aachen, 2003.

\bibitem{14} \label{Lev10}
J. L\'{e}vine, On necessary and sufficient conditions for differential flatness, Applicable Algebra in Engineering, Communications and Computing,  Springer, vol. 22, no. 1, pp. 47-90, 2011.

\bibitem{15} \label{FliMou99}
M. Fliess and H. Mounier, Tracking control and -freeness of infinite dimensional linear systems, In: G. Picci and D.S. Gilliam Eds.,Dynamical Systems, Control, Coding and Computer Vision, vol. 258, pp. 41-68, \textit{Birkha\"{u}ser}, 1999.

\bibitem{16} \label{BouBouZheBarKra11}
S. Bououden, D. Boutat, G. Zheng, J.P. Barbot and F. Kratz, A triangular canonical form for a class of 0-flat nonlinear systems, International Journal of Control, Taylor and Francis, vol. 84, no. 2, pp. 261-269, 2011.

\bibitem{17} \label{MarTom92}
R. Marino and P. Tomei, Global asymptotic observers for nonlinear systems via filtered transformations, \textit{IEEE Transactions on Automatic Control}, vol. 37, no. 8, pp. 1239-1245, 1992.

\bibitem{18} \label{Rig11}
G.G. Rigatos, Modelling and control for intelligent industrial systems: adaptive algorithms in robotics and industrial engineering, Springer, 2011.

\bibitem{19} \label{RigTza07}
G.G. Rigatos, and S.G. Tzafestas, Extended Kalman Filtering for Fuzzy Modeling and Multi-Sensor Fusion, Mathematical and Computer Modeling of Dynamical Systems, vol. 13, no. 3, \textit{Taylor and Francis}, 2007.

\bibitem{20} \label{Rig12a}
G.G. Rigatos, A derivative-free Kalman Filtering approach to state estimation-based control of nonlinear dynamical systems, IEEE Transactions on Industrial Electronics, vol. 59, no. 10, pp. 3987-3997, 2012.

\bibitem{21} \label{ZhaBasBen98}
Q. Zhang, M. Basseville and A. Benveniste, Fault Detection and Isolation in Nonlinear Dynamic
Systems : A Combined Input-Output and Local Approach, Automatica, Elsevier, vol.34, no. 11, pp. 1359-1373,
1998.

\bibitem{22} \label{BasNik93}
M. Basseville and I. Nikiforov, Detection of Abrupt changes, Prentice Hall, 1993.

\bibitem{23} \label{RigZha09}
G. Rigatos and Q. Zhang, Fuzzy model validation using the local statistical approach, Fuzzy
Sets and Systems, Elsevier, vol. 60, no.7, pp. 882-904, 2009.

\bibitem{24} \label{BeBaMou87}
A. Benveniste, M. Basseville and G. Moustakides, The asymptotic local approach to change detection
and model validation, IEEE Transactions on Automatic Control, vol. 32, no. 7, pp. 583-592, 1987.

\bibitem{25} \label{RigSiaPic09}
G.G. Rigatos, P. Siano and A. Piccolo, A neural network-based approach for early detection
of cascading events in electric power systems, IET Journal on Generation Transmission and Distribution, vol.3, no. 7, pp. 650-665, 2009.

\bibitem{26} \label{BasBenZha96}
M. Basseville, A. Benveniste and Q. Zhang, Surveilliance d' installations industrielles : d\'{e}marche
g\'{e}n\'{e}rale et conception de l' algorithmique, IRISA Publication Interne No 1010, 1996.

\bibitem{27} \label{SaaLev13}
A. Saadatpour and M. Levi, Travelling waves in chains of pendula, Physica D, vol. 244, pp. 68-73, 2013.

\bibitem{28} \label{GuoBil07}
L. Guo and S.A. Billings, State-Space Reconstruction and Spatio-Temporal Prediction of Lattice Dynamical Systems, IEEE Transactions on Automatic Control, vol. 52, no.4, pp. 622-632, 2007














\end{thebibliography}



\end{document}
