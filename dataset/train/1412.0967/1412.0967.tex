\documentclass[12pt]{article}
\usepackage[left=1.25in,top=1in]{geometry}
\usepackage{amsthm}
\usepackage{amsmath}
\usepackage{graphicx,url}
\usepackage{booktabs}
\usepackage[small,bf]{caption}

\setlength{\textwidth}{6in}
\setlength{\textheight}{9in}

\newcommand{\Oh}[1]
  {\ensuremath{\mathcal{O}\!\left({#1}\right)}}
\newcommand{\access}
  {\ensuremath{\mathrm{access}}}
\newcommand{\rank}
  {\ensuremath{\mathrm{rank}}}
\newcommand{\select}
  {\ensuremath{\mathrm{select}}}

\newtheorem{obs}{Observation}
\newtheorem{lemm}[obs]{Lemma}
\newtheorem{thrm}[obs]{Theorem}
\newtheorem{cor}[obs]{Corollary}

\begin{document}

\title{\vspace{-4ex}Queries on LZ-Bounded Encodings\thanks{This work is supported by the Academy of Finland.}}


\author{\normalsize Djamal Belazzougui, Travis Gagie, Pawe\l\ Gawrychowski,\\ 
\normalsize Juha K\"arkk\"ainen, Alberto Ord\'o\~{n}ez, Simon J.\ Puglisi, and Yasuo Tabei\-0.8ex]
\footnotesize Department of Computer Science, University of Helsinki, Finland\-0.8ex]
\footnotesize  Database Lab, University of A Coru\~{n}a, Spain\
  \access(i,j) &= \text{return the substring } \\
  \rank_a(i) &= \text{return the number of occurrences of the character }\\
  &\ \ \ \ \text{among the first  symbols of } \\
  \select_a(j) &= \text{return the position of the th occurrence of .} 
S [1..n]r \leq n\Oh{\log_r \left( \frac{n\log\sigma}{z\log n} \right) \cdot \left( \frac{m \log \sigma}{\log n} +1 \right)}n < rS [1..n]n \geq rS_1, \ldots, S_r|S_1| = \cdots = |S_{n \bmod r}| = \lceil n / r \rceil|S_{(n \bmod r) + 1}| = \cdots = |S_r| = \lfloor n / r \rfloor1 \leq i < rS_i S_{i + 1}S_i S_{i + 1}|S_i| \leq rr z\log_r zS [i]S [i]S [i]S [i]S [i']S [i'] = S [i]S [i']S [i]S [37]S [37]S [35..38]S [15..18]S [37] = S [17]S [17..20]m > \log_\sigma n\log_\sigma n\Oh{\log_r \left( \frac{n\log\sigma}{z\log n} \right) \cdot \left( \frac{m \log \sigma}{\log n} +1 \right)}\,.\Oh{z r \log (\sigma) / \log n} = \Oh{z r}B_1 B_2B_1.\rank_a (g - 1)B_u.\rank_a (d)i < dB_u.\rank_a (i) = B_1.\rank_a (g + i) - B_1.\rank_a (g - 1)i = di > dB_u.\rank_a (i) = B_u.\rank_a (d) + B_2.\rank_a (i - d)B_1.\rank_a (g - 1)B_u.\rank_a (d)B_1 B_2j \leq B_u.\rank_a (d)B_u.\select_a (j) = B_1.\select (j + B_1.\rank_a (g - 1)) - gj > B_u.\rank_a (d)B_u.\select_a (j) = B_2.\select (j - B_u.\rank_a (d)) + dS [1..n]\Oh{zr \log (n) \log_r \frac{n\log\sigma}{z\log n}}r \leq n\Oh{\log_r \left( \frac{n\log\sigma}{z\log n} \right) \cdot \left( \frac{m \log \sigma}{\log n} +1 \right)}S [i..k]S [i] + \cdots + S [j]S [i..k]\log nS [i..k]S [i - 1]\Oh{ (n/(B\log_\sigma n))\cdot \mathtt{max}\{\log_{M/B}(n/(B\log_\sigma n)),\log_r(n/(z\log_\sigma n))\}}
I/Os; however, due to lack of space, we defer the details to the full article. 

\section{Experiments}
\label{sec:experiments}

We have implemented our data structure and in this section we report on its
practical performance in comparison to other state-of-the art solutions. Due 
to lack of space, in this extended abstract we only provide experimental results 
for rank, select, and access operations, leaving treatment of range minimum 
and previous smaller value to the full article. Our implementation of select
diverges somewhat from the description in Section~\ref{subsec:select} and is 
closer to a binary search using rank queries.

We refer to the implementation of our data 
structure as BG, for {\em block graph}. We compared space and time performance 
of BG with other relevant data structures supporting rank, select, and access
operations. These included: GCC, a grammar-compressed structure by
Navarro and Ord\'o\~{n}ez~\cite{NO14}; CM, an efficient implementation of the classical 
succinct (but not compressed) solution by Clarke and Munro~\cite{Mun96}; and RRR, the 
widely used -compressed data structure of Raman et al.~\cite{RRR07}. 

All test were run on an Intel(R) Xeon(R) E5620 at GHz with GB of RAM. 
The OS was Ubuntu 10.04 with kernel 2.6.32-33-server.x86\_64. All implementations 
were written in {\tt C++}. The compiler was \verb|g++| version , with \verb|-O9| 
optimization.

For test data, we built suffix trees for two different collections: {\sf einstein},
a collection of Wikipedia files with full version history; and {\sf influenza} a 
collection of hundreds of individual genomes of influenza viruses. The suffix tree topologies
were represented as sequences of balanced parentheses (and are thus binary strings). For {\sf influenza}, the 
length of this string was 603,704,964 and parsed into 1,133,015 LZ factors. 
The string for {\sf einstein} was 367,324,468 bits long and parsed into just 50,541
LZ factors.

\begin{figure}[htb]
\minipage{0.49\textwidth}
  \includegraphics[width=\linewidth]{accessEinsteinST.pdf}
\endminipage\hfill
\minipage{0.49\textwidth}
  \includegraphics[width=\linewidth]{accessInfluenzaST.pdf}
\endminipage
\vspace{1ex}
\newline
\minipage{0.49\textwidth}
  \includegraphics[width=\linewidth]{rankEinsteinST.pdf}
\endminipage\hfill
\minipage{0.49\textwidth}
  \includegraphics[width=\linewidth]{rankInfluenzaST.pdf}
\endminipage
\vspace{1ex}
\newline
\minipage{0.49\textwidth}
  \includegraphics[width=\linewidth]{selectEinsteinST.pdf}
\endminipage\hfill
\minipage{0.49\textwidth}
  \includegraphics[width=\linewidth]{selectInfluenzaST.pdf}
\endminipage

\caption{Space-time trade-offs for access (top), rank (middle) and select (bottom) queries
of various methods when applied to two suffix tree topologies.
Note the log scale on the y-axes and that not all x-axes start at 0.
}
\label{figure:bitvectors}
\end{figure}

Results are shown in Figure~\ref{figure:bitvectors}. 
BG is more than an order of magnitude faster than the grammar compressed data structure GCC
on all operations. GCC is capable of greater compression, but in many applications the 
tradeoff achieved by the block graph is much more useful and spans the long gap between
GCC and CM. The trade-off is achieved via parameter , the arity of the block graph. 

The uncompressed solution CM is consistently fastest, 
as expected. Notably, because the repetition in these binary strings is non-local, the 
RRR method is unable to achieve any compression and is actually larger than the 
uncompressed CM structure, due to the dominance of lower-order terms.

\begin{thebibliography}{10}

\bibitem{BPT14}
Djamal Belazzougui, Simon~J. Puglisi, and Yasuo Tabei.
\newblock Rank, select and access in grammar-compressed strings.
\newblock Technical Report 1408.3093, CoRR, 2014.
 
\bibitem{Gag??}
Travis Gagie.
\newblock Rank and select operations on sequences.
\newblock In Ming{-}Yang Kao, editor, {\em Encyclopedia of Algorithms}.
  Springer, 2nd edition, to appear.
 
\bibitem{GGP11}
Travis Gagie, Pawel Gawrychowski, and Simon~J. Puglisi.
\newblock Faster approximate pattern matching in compressed repetitive texts.
\newblock In {\em Proc.\ ISAAC}, pages 653--662, 2011.
 
\bibitem{GHP14}
Travis Gagie, Christopher Hoobin, and Simon~J. Puglisi.
\newblock Block graphs in practice.
\newblock In {\em Proc.\ ICABD}, pages 30--36, 2014.
 
\bibitem{KMSTSA03}
Takuya Kida, Tetsuya Matsumoto, Yusuke Shibata, Masayuki Takeda, Ayumi
  Shinohara, and Setsuo Arikawa.
\newblock Collage system: a unifying framework for compressed pattern matching.
\newblock {\em Theoretical Computer Science}, 1(298):253--272, 2003.
 
\bibitem{LMM13}
Markus Lohrey, Sebastian Maneth, and Roy Mennicke.
\newblock {XML} tree structure compression using repair.
\newblock {\em Information Systems}, 38(8):1150--1167, 2013.
 
\bibitem{Mun96}
I.~Munro.
\newblock Tables.
\newblock In {\em Proc.\ FSTTCS}, LNCS 1180, pages 37--42, 1996.
 
\bibitem{NO14}
Gonzalo Navarro and Alberto Ord{\'o}{\~n}ez.
\newblock Grammar compressed sequences with rank/select support.
\newblock In {\em Proc.\ SPIRE}, pages 31--44, 2014.
 
\bibitem{NOsea14.1}
Gonzalo Navarro and Alberto Ord{\'o}{\~n}ez.
\newblock Faster compressed suffix trees for repetitive text collections.
\newblock In {\em Proc.\ SEA}, pages 424--435, 2014.
 
\bibitem{NPV11}
Gonzalo Navarro, Simon~J. Puglisi, and Daniel Valenzuela.
\newblock Practical compressed document retrieval.
\newblock In {\em Proc.\ SEA}, pages 193--205, 2011.
 
\bibitem{RR08}
Naila Rahman and Rajeev Raman.
\newblock Rank and select operations on binary strings.
\newblock In Ming{-}Yang Kao, editor, {\em Encyclopedia of Algorithms}.
  Springer, 1st edition, 2008.
 
\bibitem{RRR07}
R.~Raman, V.~Raman, and S.~Srinivasa Rao.
\newblock Succinct indexable dictionaries with applications to encoding {\it
  k}-ary trees, prefix sums and multisets.
\newblock {\em ACM Transactions on Algorithms}, 3(4):art.~43, 2007.

\end{thebibliography}

\end{document}
