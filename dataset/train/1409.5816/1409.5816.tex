




\documentclass{article}
\usepackage{amsmath}
\usepackage{amsthm}
\usepackage{amsfonts}
\usepackage{graphicx}
\usepackage[usenames,dvipsnames]{color}
\usepackage{authblk}

\sloppy

\renewcommand{\int}[1]{}


\newcounter{thecase} \setcounter{thecase}{0}
	
\newcommand{\case}[1]{\noindent
            \refstepcounter{thecase}\paragraph*{\textnormal{\textit{Case \arabic{thecase}: #1}}}
}

\newcommand{\add}[1]{
\textcolor{red}{#1}
}
\newcommand{\exclude}[1]{{\color{Gray}
\flushleft{{\textbf{Skip this for now\\}}}
#1}}
\newcommand{\bb}[1]{\ensuremath{\mathbf{#1}}}

\newtheorem{theorem}{Theorem}[section]
\newtheorem{lemma}[theorem]{Lemma}
\newtheorem{definition}[theorem]{Definition}
\newtheorem{corollary}[theorem]{Corollary}
\newtheorem{observation}[theorem]{Observation}

\begin{document}




\title{1-String CZ-Representation of Planar Graphs}

\author{T.~Biedl\thanks{\texttt{biedl@uwaterloo.ca}, Research supported by NSERC.}~}
\author{M.~Derka\thanks{\texttt{mderka@uwaterloo.ca}, The second author was supported by the NSERC Vanier CGS.}}


\affil{\small{David R.~Cheriton School of Computer Science, University of Waterloo}}

\renewcommand\Authands{~and~}





\maketitle

\begin{abstract}
In this paper, we prove that every planar 4-connected graph has a CZ-representation---a
string representation using paths in a rectangular grid that contain at most one vertical segment. 
Furthermore, two paths representing vertices  intersect precisely once whenever there
is an edge between  and . The required size of the grid is .
\end{abstract}




\section{Preliminaries}

A possible way of representing graphs is to assign to every vertex a curve so
that two curves cross if and only if there is an edge between the respective vertices. 
Here, two curves  \emph{cross} means that
they share a point  internal to both of them and 
the boundary of a sufficiently small closed disk around  
is crossed by  (in this order).
The representation of graphs using crossing curves is referred to as a \emph{string representation},
and graphs that can be represented in this way are called \emph{string graphs}.

In 1976, Ehrlich, Even and Tarjan showed that every planar graph has a string representation~\cite{cit:tarjan}.
It is only natural to ask if this result holds if one is restricted to using
only some ``nice'' types of curves. In 1984, Scheinerman conjectured that all planar graphs can
be represented as intersection graphs of line segments~\cite{cit:scheinerman}.
This was proved first for bipartite graphs~\cite{cit:arroyo, cit:pach} with the strengthening
that every segment is vertical or horizontal. The result was extended to triangle-free
graphs, which can be represented by line segments with at most three distinct slopes~\cite{cit:castro}.


Since Scheinerman's conjecture seemed difficult to prove for all planar
graphs, interest arose in possible relaxations.
Note that any two line segments intersect at most once.
Define 1-STRING to be the class of graphs that are intersection graphs
of curves (of arbitrary shape) that intersect at most once.
The original construction of string representation for planar graphs 
given in~\cite{cit:tarjan} requires curves to cross multiple times. 
In 2007, Chalopin, Gon\c{c}alves and Ochem showed that every
planar graph is in 1-STRING~\cite{cit:chalopin-gonclaves-ochem, cit:chalopin-string}.  With respect to Scheinerman's
conjecture, while the argument of~\cite{cit:chalopin-gonclaves-ochem, cit:chalopin-string} shows that the prescribed number
of intersections can be achieved, it provides no idea on the complexity of curves that is required.   

Another way of restricting curves in string representations is to require them
to be \emph{orthogonal}, i.e., to be paths in a grid.  Call a graph a
{\em VPG-graph} (as in ``Vertex-intersection graph of Paths in a Grid'')
if it has a string representation with orthogonal curves.
It is easy to see that all planar graphs are VPG-graphs (e.g.~by generalizing
the construction of Ehrlich, Even and Tarjan).  For bipartite planar graphs,
curves can even be required to have no bends \cite{cit:arroyo, cit:pach}.
For arbitrary planar graphs bends are required in orthogonal curves, and
recently Chaplick and Ueckerdt showed that 2 bends per curve always suffice
\cite{cit:chaplick}.  Let {\em -VPG} be the graphs that have
a string representation where curves are
orthogonal and have at most 2 bends; the result in
\cite{cit:chaplick} then states that planar graphs are in -VPG.
Unfortunately, in Chaplick and Ueckerdt's construction, curves may cross 
each other repeatedly, and so it
does not prove that planar graphs are in 1-STRING.

The conjecture of Scheinerman remained open until 2009 when it was proved true by Chalopin and Gon\c{c}alves \cite{cit:chalopin-gonclaves-ochem}
who extended the technique used to prove their 1-STRING 
result~\cite{cit:chalopin-seg}. 

\paragraph*{Our results:} 
In this paper, we show that every planar 4-connected graph has a 
string representation that simultaneously satisfies the requirements for
1-STRING (any two curves cross at most once) and the requirements
for -VPG (any curve is orthogonal and has at most two bends).
Our result hence re-proves, in one construction, the results
by Chalopin et al.~\cite{cit:chalopin-seg} and the result by
Chaplick and Ueckerdt \cite{cit:chaplick}, albeit only for 4-connected 
planar graphs. (We briefly discuss extensions in Section~\ref{sec:outlook}.)

In our construction all curves have one of four
possible shapes: C-shape, Z-shape, or their mirror images. We call such a 
representation a \emph{CZ-representation} (see Section~\ref{sec:def} for
the formal definition).

\begin{theorem}
\label{thm:main-claim}
Every -connected planar graph has a 1-string CZ-rep\-re\-sen\-ta\-tion. 
\end{theorem}



Since our construction has at most  vertical and  horizontal line segments, and
since any orthogonal grid can be deformed to be on integer coordinates 
without empty rows
or columns, our construction can be embedded to a rectangular grid of size .
Note that none of the previous results provided an intuition of the required size of the grid.

Our approach is inspired by the construction of 1-string representations from 
2007~\cite{cit:chalopin-gonclaves-ochem, cit:chalopin-string}. 
The authors proved the result in two steps. First,
they showed that triangulations without separating triangles 
admit 1-string representations. By induction on the number of 
separating triangles, they then showed that 1-string representation
exists for any planar triangulation, and consequently for any 
planar graph. 

In order to show that triangulations without separating triangles
have 1-string representation, Chalopin et at.~\cite{cit:chalopin-string} used
a method inspired by Whitney's proof that 4-connected planar graphs
are Hamiltonian~\cite{cit:whitney}. Asano, Saito and Kikuchi later improved
Whitney's technique and simplified his proof~\cite{cit:ham-cycle}. 
Our paper uses the same approach as~\cite{cit:chalopin-string}, but borrows ideas from~\cite{cit:ham-cycle}
and develops them further to reduce the number of cases and hence
simplify the proof. 


\iffalse
A~CZ-representation uses two bends for every curve, thus our result
implies that triangulations with no separating triangles as 2-VPG. 
Our construction uses precisely one intersection per edge. Thus, it is 
stronger than the result of~\cite{cit:chaplick} for this specific subclass
of planar graphs.

Furthermore, the result presented here immediately implies that planar triangulations
with no separating triangles have 1-string representations, which is an intermediate
step of~\cite{cit:chalopin-string} for showing that this is true for all planar graphs.
Compared to~\cite{cit:chalopin-string}, our approach significantly reduces the number of analysed cases.
\fi






\section{CZ-Representation of -Connected Planar Graphs}
\label{sec:def}

Let us begin with a formal definition of a \emph{CZ-representation}.

\begin{definition}[CZ-representation]
\label{def:repre}
 A planar graph  has a \emph{1-string CZ-representation} if every vertex  of  can be
 represented by a curve  such that: \begin{enumerate}
    \item Curve  is \emph{orthogonal}, i.e., it consists of horizontal and vertical segments.
    \item Curve  has at most two bends and at most one vertical segment (see Figure~\ref{fig:cz-curve}).
    \item Curves  and  intersect at most once, and  intersects  if and
	only if  is an edge of .
 \end{enumerate}
A {\em partial 1-string CZ-representation} is a 1-string CZ-representation of a subgraph of .
 \end{definition}

\begin{figure}
\centering
    \includegraphics[width=.7\textwidth]{cz}
    \caption{Every curve in a CZ-representation has one of the depicted shapes.}
    \label{fig:cz-curve}
\end{figure}

For brevity, we use ``CZ-representation'' to mean ``partial 1-string 
CZ-repre\-sentation''.
Our technique for constructing a CZ-representation of a graph uses an intermediate step
referred to as ``an \emph{\int{F}-CZ-representation}\footnote{Here, \emph{int} is used as an abbreviation of \emph{interior}.} of a
W-triangulation that satisfies the chord condition with respect to
three chosen corners''.
We define these terms first. 

A \emph{triangulated disk} is a 2-connected planar graph  such that every 
interior face is a triangle. 
A {\em separating triangle} is a cycle of length  which contains vertices
both inside and outside.   Following the notation of
\cite{cit:chalopin-string}, a \emph{W-triangulation} is a triangulated disk
which does not contain a separating triangle.
A {\em chord} of a triangulated disk is an
interior edge for which both endpoints are on the outer face.

For two vertices  on the outer face of a connected planar graph,
define  to be the counter-clockwise (ccw) path on the outer face from  to  ( and  inclusive).
We will often study triangulated disks with
three specified distinct vertices  called the {\em corners}
which must appear on the outer face in ccw order. 
We denote ,  
and , where ,  
and . 

\begin{definition}[Chord condition]
\label{def:chord-condition}
A W-triangulation  satisfies the \emph{chord condition} with respect
to the corners  if  has no chord within  or ,
i.e., no interior edge of  has 
both ends on , or both ends of , or
both ends on .\footnote{For readers familiar with \cite{cit:chalopin-string}
or \cite{cit:ham-cycle}:
A W-triangulation that satisfies the chord condition with respect
to corners  is called a \emph{W-triangulation 
with 3-boundary } 
in \cite{cit:chalopin-string},
and the chord condition is 
the same as \emph{Condition (W2b)} in~\cite{cit:ham-cycle}.}\end{definition}


\begin{definition}[\int{F}-CZ-representation]
Let  be a connected planar graph with corners . 
Let  be a set of outer face edges incident to .
An \emph{\int{F}-CZ-representation}\footnote{An \int{F}-CZ-representation corresponds roughly to what Chalopin
et al.~\cite{cit:chalopin-string} call Property 1, except that they do not
restrict the shape of the curves, and they fix  to be edge .}
of  is a CZ-representation of  for which curves  cross if and
only if  is an interior edge of  or . 
Furthermore, the CZ-representation must satisfy that:
\begin{enumerate}
    \item There exists a rectangle  containing all intersections 
    of curves so that the top of  is intersected, from right to 
	left in order, by the curves of the vertices of , 
and the bottom
    of  is intersected, from left to right in order, 
by the curves of the vertices of . 
    \item The curve  of an outer face vertex  has at most one
    bend. (By (1), this implies that  and  have no bends.)
\end{enumerate}
\end{definition}


See Figure~\ref{fig:partial-cz-ex} for examples of an \int{F}-CZ-representation.
In all our constructions, we have , i.e.,  consists of
at most one edge that is on the outer face and incident to .
If , then  is called the \emph{special edge}.
We sometimes write \int{e}-CZ-representation rather than 
\int{\{e\}}-CZ-representation, and
int-CZ-representation rather than
\int{\emptyset}-CZ-representation.
Note that the roles of corners  and  in an \int{F}-CZ-representation 
are symmetric: we can exchange  and , as long as we also reverse
all cyclic orders of edges around each vertex (to preserve the sense
of counter-clockwise) and flip the resulting representation horizontally
(to undo the reversal.)
Corner , on the other hand, is distinct from the other two, for 
example because the special edge must be incident to . 
Our key result is the following:

\begin{figure}
    \centering
    \includegraphics[width=\textwidth]{cz-rep}
    \caption{An int-CZ-representation (left) 
and an \int{(C,c_2)}-CZ-representation (right).}
    \label{fig:partial-cz-ex}
\end{figure}

\begin{lemma}
	\label{lem:representation}
    Let  be a W-triangulation that 
    satisfies the chord condition with respect to corners . 
	Then  has an \int{F}-CZ-representation
    for any set  of at most one outer face edge incident to .
\end{lemma}

The proof of Lemma~\ref{lem:representation} will be given in Section~\ref{sec:proof}. Here we show how it implies our main result.

\begin{proof}[Proof of Theorem~\ref{thm:main-claim}]
First assume that  is a triangulation, which by 4-con\-nec\-tivity means that
it has no separating triangles. Let  be the vertices on the outer face
in ccw order.
As the outer face is a triangle,  clearly satisfies the chord condition with respect to .
Thus, by Lemma~\ref{lem:representation}, it has an
\int{(B,C)}-CZ-representation contained in a rectangular box .  This
CZ-representation has an intersection for 
every edge except for  and . The  ends of curves  and 
outside of  can be used to create intersections for these edges
as follows.
Bend and stretch the upper end of  rightwards and the upper end of  upwards so that
both the curves cross. 
Bend and stretch the lower end of curve  leftwards and stretch  downwards so that the two curves cross.
Recall that  and  initially did not have any bends, so they have each one bend in the
constructed 1-string 
-representation of . See Figure~\ref{fig:completion} for an illustration.

Now assume that  is a -connected planar graph. Then \emph{stellate} the graph, i.e.,
insert a vertex into each non-triangulated face and connect it to all vertices on that face.
The resulting graph is triangulated and has no separating triangles, so it has a 1-string CZ-representation
by the above. Deleting the curves of added vertices produces the result. 
\end{proof}

\begin{figure}
    \centering
    \includegraphics[width=.25\textwidth]{completion}
    \caption{Completing the \int{F}-CZ-representation of a triangulation .}
    \label{fig:completion}
\end{figure}


\section{\int{F}-CZ-representations}
\label{sec:proof}

In this section, we provide the proof of Lemma~\ref{lem:representation}.
We proceed by induction on the number of edges. In the base case, , so
 is a triangle, and the three corners  must be the three vertices 
of this triangle.  The \int{F}-CZ-representations 
for 
are depicted in Figure~\ref{fig:base-case}.

\begin{figure}
	\centering
	\includegraphics[width=.8\textwidth]{base-case}
	\caption{\int{F}-CZ-representations of a triangle.}
	\label{fig:base-case}
\end{figure}

The induction step for  is divided into five cases.

\case{ has a chord incident to .}  
\label{case:special}
By the chord condition, this chord has the form  for some .
The graph  can be split along the chord  into two graphs  and
.  Both  and  are bounded by simple cycles, hence triangulated
disks.  No edges were added, so neither  nor 
contains a separating triangle.   We select  as corners for 
and  as corners for  and can easily verify that with this
 and  satisfy the chord condition: 
\begin{itemize}
\item  has no chords on  or  as they would violate the chord condition in . 
There is no chord on  as it is a single edge.
\item  has no chords on  or  as they would violate the chord condition in . 
There is no chord on  as it is a single edge.
\end{itemize}

So, by induction, Lemma~\ref{lem:representation} holds for both  and .




\begin{figure}
\centering
\includegraphics[width=.35\textwidth]{case0-graph}\hspace{4em}
\includegraphics[width=.2\textwidth]{case0-no-special}\hspace{4em}
\includegraphics[width=.2\textwidth]{case0done}
\caption{Case~\ref{case:special}: Constructing an 
\int{F}-CZ-representation for  (left) and  
(right) when  is incident with a chord.}
\label{fig:case0}
\end{figure}

For  or ,
an \int{F}-CZ-representation of  
can be constructed as follows. Inductively, construct an 
\int{F}-CZ-representation of  and an 
\int{(C,a_i)}-CZ-representation of ; note that the special edge
of each indeed attaches at the required corner. Rotate
the CZ-representation of  by 180, and translate it
so that it is below the CZ-representation of  with the two
copies of  in the same column.  Stretch one of the CZ-representations 
horizontally as needed until the two copies of  are also in
the same column; then  and  can each be unified without
adding bends by adding vertical segments.    Stretch the CZ-representation
of  further so that everything to the left of  in 
appears to the left of the entire CZ-representation of ; the curves
of outer face vertices of  then cross (after suitable lengthening)
a bounding box in the required order.  
See also Figure~\ref{fig:case0}.
The construction does not create any new bends. Since  has no bends in the CZ-representation of  and  has
no bends in the CZ-representation of , the number of bends on each curve on the outer face does not exceed .

This finishes the construction for  or .  
An \int{(C,b_{s-1})}-CZ-representation can be obtained using the
same construction after exchanging the roles of vertices  and 
as described earlier.

\case{ has a chord in the form , , .}
\label{case:case1}
We may assume that  (otherwise we are in Case~\ref{case:special}), 
and  and  (otherwise the chord condition is violated).


\begin{figure}
\centering
\includegraphics[width=.5\textwidth]{case1-graph}\hspace{5em}
\includegraphics[width=.3\textwidth]{cc}
\caption{Case~\ref{case:case1}: Constructing an int-CZ-representation of  with a chord of 
the form , , .}
\label{fig:case1-detail}
\end{figure}

Let  be the chord that maximizes  (i.e., it is the furthest chord from vertex ). 
Note that possibly , i.e., . 
Split the graph  along this chord into graphs   (which contains )
and  (which contains  and the special edge, if any).
Select  as corners for  and  as corners for .
As before both  and  are W-triangulations, and we can verify
that they satisfy the chord condition:
\begin{itemize}
\item  has no chords on path  and  as they would contradict the chord condition
in . The remaining side  is a single edge, and so does not have any chords either.
\item  has no chords on path  and  as they would contradict
the chord condition for . Furthermore,  has no chord on path  due to the selection
of  and since  has no chords.
\end{itemize}

In order to construct an \int{F}-CZ-representation of ,
apply induction to get an \int{(a_i,c_j)}-CZ-representations of 
 and an \int{F}-CZ-representation of 
. Similarly as before, stretch the representations so that we can 
align and join the two 
curves  and  as shown in Figure~\ref{fig:case1-detail}. Since  
has no bends in the CZ-representation of  and  has no bends in the CZ-representation of , the number of bends of those curves in the constructed representation is at most~1. The
number of bends of any other curve representing an outer face vertex does not change, so it is also at most . Thus, the constructed representation is a valid
\int{F}-CZ-representation.
Figure~\ref{fig:case1-detail} shows the construction 
(with ) when  and . 
Figure~\ref{fig:case1-special} shows the construction 
when  or .


\begin{figure}
\begin{center}
\includegraphics[width=.2\textwidth]{c1-special1}\hspace{4em}
\includegraphics[width=.2\textwidth]{c1-special2}\hspace{4em}
\includegraphics[width=.2\textwidth]{c1-special3}\\\medskip
\includegraphics[width=.2\textwidth]{c1-special1-split}\hspace{4em}
\includegraphics[width=.2\textwidth]{c1-special2-split}\hspace{4em}
\includegraphics[width=.2\textwidth]{c1-special3-split}\\\medskip\medskip
\includegraphics[width=.2\textwidth]{c1-special1-constr}\hspace{4em}
\includegraphics[width=.2\textwidth]{c1-special2-constr}\hspace{4em}
\includegraphics[width=.2\textwidth]{c1-special3-constr}
\end{center}
\caption{The construction for Case~\ref{case:case1} also covers borderline cases:  (Left) The chord  is incident with . (Middle) The chord
is incident to the end  of the special edge.  (Right) Both the incidencies are present.}
\label{fig:case1-special}
\end{figure}


\case{ has a chord in the form , , .}
\label{case:case1a}
By interchanging the roles of corners  and , this case can be transformed into Case~\ref{case:case1}.  


\case{ has a chord in the form , , .}
\label{case:case2}
Note that we may assume  and  as all other cases either violate the chord condition
or were already covered.
Let  be the chord maximizing  (i.e., furthest from ). 

In order to construct an int-CZ-representation of , split the graph 
along  into two W-triangulations  (which includes 
and the special edge, if any) and
 (which includes ).  Set
  as corners for  and  as corners for 
and verify the chord condition:

\begin{itemize}
    \item  has no chords on either  or 
        as they would contradict the chord condition in .
        The third side is a single edge  and so it does not have any chords either. 
    \item  has no chords on either  or  as they
    would violate the chord condition in . It does not have any chords on the path  due to 
    the selection of the chord . 
\end{itemize}

Thus, by induction,  has an \int{F}-CZ-representation and  has
an \int{(b_j,c_k)}-CZ-representation. 
After horizontal deformation, the CZ-representations can be aligned 
so that the ends of  and  in  can be connected to the upper ends of 
and  in . 
As  and  have no bends in the CZ-representation of , the
construction does not increase the number of bends on any curve and produces 
an \int{F}-CZ-representation of . 
Figure~\ref{fig:case2-construction} shows the construction. 


\begin{figure}
\begin{center}
\includegraphics[width=.25\textwidth]{case2-graph}\hspace{2em}
\includegraphics[width=.25\textwidth]{case2-graph-split}\hspace{3em}
\includegraphics[width=.25\textwidth]{case2-representation}
\end{center}
\caption{Case~\ref{case:case2}: Construction of an \int{F}-CZ-representation of 
    with a chord .}
\label{fig:case2-construction}
\end{figure}


\case{ has no chords.}
\label{case:chordless}
Assume after possible exchange of  and  that the special edge, if it exists, is .
Let  be the neighbours of vertex  in clockwise order, starting
with  and ending with .  We know that , for
otherwise the neighbours of  would have a chord between them since  is a 
triangulated disk.  Therefore,  has at least one neighbour
.  We also know that  are not on the outer 
face, since  is not incident to a chord.

Let  be a neighbour of  that has at least one other neighbour on 
, and among all those, choose  to be minimal. Such a
 exists and   because  is triangulated and 
therefore  is adjacent to both  and .   
We also know that , sicne otherwise there would be a chord from
 to some vertex on .

Let the {\em terminals} be the neighbours of 
on ; we denote these by  in the order in which
they appear on . 
Separate  into two graphs  (the \emph{top} graph) and  (the \emph{bottom} graph) as follows:  is bounded 
by ;
and  is bounded by . See also Figure~\ref{fig:top-bottom}.


\begin{figure}
	\centering
	\includegraphics[width=.6\textwidth]{chain-better}
	\caption{Decomposition into the top graph and bottom graph.}
        \label{fig:top-bottom}
\end{figure}

\begin{observation}
        The top graph  is a W-triangulation that satisfies the chord condition with respect to corners
	 and . 
\end{observation}
\begin{proof}
Since  are interior vertices of , the outer face of 
is a simple cycle, and so  is a -triangulation.

    Since  satisfies the chord condition, graph  does not have any 
chords with both ends on  or .
If there were any chords with both ends on , then by 
the chord would either
connect two neighbours of  (hence give a separating triangle of ),
or connect some  for  to  (contradicting minimality of ),
or connect  to some other vertex on  (contradicting that
 is the last terminal), or have both ends on  (contradicting the chord condition
for ).    Hence no such chord can exist either.
\end{proof}

So, we can apply induction on  and obtain an
\int{(u_1,u_2)}-CZ-rep\-re\-sen\-ta\-tion of .  Similarly as in previous cases
the plan is to combine this with a representation of the rest.
Define ; we call this graph the {\em chain graph}.\footnote{Comparing to the terminology of \cite{cit:ham-cycle}, our chain
graph is similar to the graph  that is defined by the -chain and
satisfies Property (A).}
Unfortunately  is not
necessarily 2-connected, and so we cannot apply induction to it directly,
but we can obtain a CZ-representation for it by splitting it into smaller subgraphs.

For , let  be the graph bounded by , 
and let the {\em  block}
be the graph .  See also Figure~\ref{fig:chain_blocks}.

\begin{figure}
	\centering
	\includegraphics[width=.45\textwidth]{chain}
	\caption{The chain graph. Blocks  and  are isolated
	edges, graphs  and  are W-triangulations.  We illustrate
	the chosen corners for .}
	\label{fig:chain_blocks}
\end{figure}



\begin{observation}
\label{obs:chain-block}
The  block  is either a single edge, or a W-triangulation
that satisfies the chord condition with respect to corners
, and  the successor of  on .
\end{observation}
\begin{proof}
Assume  is not a single edge.  First note that no vertex can appear twice
on the outer face boundary of , otherwise (since  was a triangulated disk)
there would be a double edge to , or another terminal on .
So  is bounded by a simple cycle and hence a W-triangulation.

Before we can argue the chord condition, we must see that the corners are
distinct.  Clearly  by the definition of terminals.
Also  by definition of  as a neighbour of .  
Finally, , for otherwise  would be an
edge and  hence a separating triangle of  (since
 is not a single edge).  Now we can verify the chord condition:
\begin{itemize}
\item  has no chord on , since all vertices on  are 
	neighbours of  and  has no separating triangle.  
\item  has no chord on  or , since both of
	these are sub-paths of  and  satisfies the chord
	condition.
\end{itemize}
\end{proof}

Set  if  is not a single edge and  otherwise.
Set  for .  By induction, any 
that is not an edge has an \int{F_i}-CZ-representation.
If  is a single edge , then we can represent it with two
vertical segments for  and .  We now merge these representations
of  as illustrated in Figure~\ref{fig:chain-representation},
by merging for  the two vertical segments .
The result satisfies all conditions for an 
\int{F_1}-CZ-representation 
of  (for corners  and ).


\begin{figure}
    \centering
\raisebox{-0.5\height}{\includegraphics[width=.4\textwidth]{chain}}
\raisebox{-0.5\height}{~~~~~~}
    \raisebox{-0.5\height}{\includegraphics[width=.35\textwidth]{path-rep}}
    \caption{Merging CZ-representations of the blocks to get a CZ-representation
of .}
    \label{fig:chain-representation}
\end{figure}

Now we merge this \int{F_1}-CZ-representation of  with the 
\int{(u_1,u_2)}-CZ-representation of  as illustrated in
Figure~\ref{fig:case5extended}.
The neighbours of  in  are vertices . 
The CZ-representation of  can be horizontally deformed and aligned so that 
it is below  and the order of vertical segments is as in 
Figure~\ref{fig:case3-complete},
i.e., from left to right it is , 
\{\bb{v} for  in 's boundary path
 in reverse  order\},
\{\bb{c} for  in 's boundary path
 in order\}.

To create the required intersections, we first
extend  upward and (still below ) rightward until it crosses .
Note that as a result,  receives its first bend.

The CZ-representation of  includes an intersection for the special
edge , but does not include intersections for edges 
.
Since these are internal to , such intersections need to be created. Do this for 
by extending  vertically below its crossing with  and  and then horizontally
until it crosses  (which is adjacent due to the properties of a CZ-representation).
We assume for this that  has been extended vertically suitably. 
This increases the number of bends of  to at most 2, which is allowed
since  is an internal vertex of .

Curve  already extends below its crossing with  and
. Now, extend
it rightward so that it crosses all vertical segments of up to (and including) .
All these segments belong to vertices in 's boundary path ,
which are indeed neighbours  of  since  is a triangulated disk.
Afterwards,  has up to two bends, which is allowed since
it is not on the outer face of .

It remains to create intersections for some of the edges in 's boundary
path .
We can create these intersections similarly as for 
by extending each curve upward and rightward until it hits the next one.
This adds one bend in each curve, which is acceptable since the curve 
remains on the outer face only if it was a terminal before, hence had no bends 
previously.   Note that an edge in 
possible does not need an intersection (see e.g.~edge  in 
Figure~\ref{fig:case3-complete}), namely, if the  block consists of
a single edge  and this edge is not in .  In this case,
simply stop  before it intersects .

With this, all interior edges of  receive exactly one intersection of
their corresponding curves.  If  is non-empty, then the special edge
 received its intersection either via the \int{F_1}-CZ-representation
of  (if  is a triangulated disk), or  and 
the intersection was created when handling .
 
This ends the description of constructing an 
\int{F}-CZ-representation in Case~\ref{case:chordless},
and hence proves Lemma~\ref{lem:representation} and
Theorem~\ref{thm:main-claim}.
    
    

\begin{figure}
	\centering
	\includegraphics[width=.7\textwidth]{case3-composition-empty-new-done}
	\caption{Obtaining an \int{F}-CZ-representation in Case~\ref{case:chordless}.}
	\label{fig:case3-complete}
\end{figure}


\begin{figure}
	\centering
        \raisebox{-0.5\height}{\includegraphics[height=.45\textwidth]{case3-composition-empty-new-special-1}}
~~~~~~~~~~~
        \raisebox{-0.5\height}{\includegraphics[height=.46\textwidth,trim=260 0 0 0,clip]{case3-composition-ink-extended1}}
	\caption{If , then the intersection of  and  is added during the merge if  is a single edge (left), or occurs within
the CZ-representation of  otherwise (right).}
        \label{fig:case5extended}
\end{figure}

\section{Conclusions and Outlook}
\label{sec:conclusions}
\label{sec:outlook}

We showed that every 4-connected planar graph has a 1-string CZ-representation. A natural 
question is to extend this result to all planar graphs. We believe that this is possible,
but currently cannot achieve fewer than  bends.


\iffalse
\begin{observation}[Chalopin, Gon\c{c}alves, Ochem~{\cite[Lemma~1]{cit:chalopin-string}}]
\label{obs:triangulation-only}
    Every planar graph  is an induced subgraph of a~-connected 
    planar triangulation. 
\end{observation}
\fi
    
By again stellating the graph (possibly repeatedly if it was not 3-connected),
it suffices to show that every -connected
planar triangulation has a -string -VPG representation. This statement can be proved by induction
on the number of separating triangles by a technique used in~\cite{cit:chalopin-string}
(and re-discovered in~\cite{cit:mfcs}). With every triangular face,
create a ``face region'' (called ``private region'' in \cite{cit:mfcs})
that intersects the curves of vertices of the face
in a predefined way and does not intersect anything else. This is easy for 
W-triangulations by inspecting the constructions in Cases~\ref{case:special}--\ref{case:chordless}.

In the inductive step, find the smallest separating triangle  in .
By induction, the graph  obtained by removing the inside of
 has a 1-string -VPG representation. 
The graph  strictly inside  is either a single vertex or (as one
can show) it is a W-triangulation that satisfies the chord condition for
some suitably chosen corners, and hence has an int-CZ-representation.
Place it inside the face region for ,
create the intersections needed for edges on the outer face of  and
edges between  and , and
identify face regions for newly created faces.

If one aims for a 1-string -VPG representation, two bend per
can be added to each curve of outer face vertices of , and hence 
such a merge is easy (details are omitted).

If one aims for a 1-string -VPG representation of planar
graphs, only one bend can be added to each such curve, 
which seems impossible with our current 
geometric restrictions of \int{F}-CZ-representation. 
However, this might be feasible
if we allow the outer face vertices 
to use rays in three directions. This is our ongoing research.


\medskip
As for other future work,
the CZ-representation constructed in this paper uses curves of four possible shapes.
Is it possible to use fewer shapes or to restrict them further?  
Felsner et al.~\cite{cit:mfcs} asked the question whether every planar
graph is the intersection graph of only two shapes, namely .
(This would also provide a different proof of Scheinerman's conjecture.)
Somewhat inbetween: is every planar graph the intersection graph of
-monotone orthogonal curves, preferably in the 1-string model?


\bibliography{arxive}{}
\bibliographystyle{plain}

\end{document}
