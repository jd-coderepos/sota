

\documentclass[11pt]{article}

\usepackage{times} 
\usepackage{epsfig}
\usepackage{amsmath}
\usepackage{amssymb}
\usepackage{amsthm}
\usepackage{graphics}
\usepackage{psfrag}

\setlength{\oddsidemargin}{0in}
\setlength{\evensidemargin}{0in}
\setlength{\topmargin}{0in}
\setlength{\headheight}{0in}
\setlength{\headsep}{0in}
\setlength{\textwidth}{6.5in}
\setlength{\textheight}{9in}

\newtheorem{theorem}{Theorem}
\newtheorem{lemma}{Lemma}
\newtheorem{corollary}{Corollary}
\newtheorem{proposition}{Proposition}
\newtheorem{observation}{Observation}

\newtheorem{conjecture}{Conjecture}
\newtheorem{question}{Question}

\newcommand{\old}[1]{{}}
\newcommand{\later}[1]{{}}

\newcommand{\etal}{et~al.}

\newcommand{\eps}{\varepsilon}
\newcommand{\NN}{\mathbb{N}}
\newcommand{\ZZ}{\mathbb{Z}}
\newcommand{\RR}{\mathbb{R}}

\def\B{\mathcal B}
\def\C{\mathcal C}
\def\F{\mathcal F}
\def\H{\mathcal H}
\def\I{\mathcal I}
\def\X{\mathcal X}

\newcommand{\per}{{\rm per}}
\newcommand{\conv}{{\rm conv}}
\newcommand{\area}{{\rm area}}
\newcommand{\vol}{{\rm vol}}


\def\marrow{{\marginpar[\hfill]{}}}
\def\adrian#1{{\sc Adrian says: }{\marrow\sf #1}}
\def\minghui#1{{\sc Minghui says: }{\marrow\sf #1}}


\title{On the largest empty axis-parallel box amidst  points}


\author{Adrian Dumitrescu\thanks{Department of Computer Science,
University of Wisconsin--Milwaukee,
WI 53201-0784, USA\@. Email: \texttt{ad@cs.uwm.edu}.
Supported in part by NSF CAREER grant CCF-0444188.
Part of the research by this author was done at the
Ecole Polytechnique F\'ed\'erale de Lausanne.}
\and
Minghui Jiang\thanks{Department of Computer Science, 
Utah State University, Logan, UT 84322-4205, USA\@.
Email: \texttt{mjiang@cc.usu.edu}.
Supported in part by NSF grant DBI-0743670.}}


\begin{document}

\maketitle

\thispagestyle{empty}

\begin{abstract}
We give the first nontrivial upper and lower bounds on the maximum volume
of an empty axis-parallel box inside an axis-parallel unit hypercube
in  containing  points. 
For a fixed , we show that the maximum volume is of the order
.  We then  use the fact that the maximum volume is
 in our design of the first efficient
-approximation algorithm for the following problem: 
Given an axis-parallel -dimensional box  in 
containing  points, compute a maximum-volume empty axis-parallel
-dimensional box contained in . The running time of our
algorithm is nearly linear in , for small , and increases only
by an  factor when one goes up one dimension. 
No previous efficient exact or approximation algorithms were known for this
problem for . As the problem has been recently shown to be
NP-hard in arbitrary high dimensions (i.e., when  is part of the
input), the existence of efficient exact algorithms is unlikely.

We also obtain tight estimates on the maximum volume of an empty
axis-parallel hypercube inside an axis-parallel unit hypercube
in  containing  points. For a fixed , 
this maximum volume is of the same order order .  
A faster -approximation algorithm, with a  milder dependence
on  in the running time, is obtained in this case.  
\end{abstract}

\medskip
\hspace{0.15in}
\textbf{\small Keywords}:
Largest empty box, largest empty hypercube, approximation algorithm.


\newpage
\setcounter{page}{1}
\setcounter{footnote}{0}

\section{Introduction}

Given a set  of  points in the unit square ,
let  be the maximum area of an empty axis-parallel rectangle
contained in , and let  be the minimum value
of  over all sets  of  points in .
For any dimension , given a set  of  points in the unit 
hypercube , let  as the maximum volume of an
empty axis-parallel hyperrectangle (-dimensional axis-parallel box)
contained in , and let  be the minimum value of 
over all sets  of  points in .
For simplicity we sometimes omit the subscript  in the planar case (). 

In this paper we give the first nontrivial upper and lower bounds on .
For any dimension , our estimates are within a multiplicative
constant (depending on ) from each other. For a fixed , we show
that the maximum volume is of the order .   
While the algorithmic problem of finding an empty axis-parallel box
of maximum volume has been previously studied for  (see below),
estimating the maximum volume of such a box as a function of  and
 seems to have not been previously considered. 

We first introduce some notations and definitions. Throughout this paper,
a \emph{box} is an \emph{open} axis-parallel hyperrectangle
contained in the unit hypercube , .
Given a set  of points in ,
a box  is \emph{empty} if it contains no points in ,
i.e., .
If  is a box, we also refer to the side length of  in the th
coordinate as the extent in the th coordinate of . 
Throughout this paper,  and  are the logarithms of  in
base  and , respectively.

Given an axis-parallel rectangle  in the plane containing  points,
the problem of computing a maximum-area empty axis-parallel
sub-rectangle contained in  is one of the oldest problems studied 
in computational geometry. For instance, this problem arises 
when a rectangular shaped facility is to be located within a
similar region which has a number of forbidden areas, or in cutting
out a rectangular piece from a large similarly shaped metal sheet with
some defective spots to be avoided~\cite{NLH84}. In higher dimensions,
finding the largest empty axis-parallel box has applications
in data mining, in finding large gaps in a multi-dimensional data
set~\cite{EGLM03}. 

Several algorithms have been proposed for the planar problem
over the years \cite{AS87,AF86,AK89,CDL86,D92,MOS85,NLH84,O90}. 
For instance, an early algorithm by Chazelle, Drysdale and Lee
\cite{CDL86} runs in  time and  space.
The fastest known algorithm,
proposed by Aggarwal and Suri in 1987 \cite{AS87},
runs in  time and  space. 
A lower bound of  in the algebraic decision tree
model for this problem has been shown by Mckenna et~al.\ \cite{MOS85}.

For any dimension , there is an obvious brute-force algorithm
running in  time and  space. 
No significantly faster algorithms, i.e., with a fixed degree
polynomial running time in , where known.
Confirming this state of affairs,
Backer and Keil~\cite{BK09a,BK09b} recently proved that the problem is
NP-hard in arbitrary high dimensions (i.e., when  is part of the input).
They also gave an exact algorithm running in 
time, for any .  
In particular, the running time of their exact algorithm for  is
. Previously, Datta and Soundaralakshmi~\cite{DS00}
had reported an  time exact algorithm for the  case, but
their analysis for the running time seems incomplete. Specifically,
the  running time depends on an  upper bound on the
number of maximal empty boxes (see discussions in the next paragraph),
but they only gave an  lower bound.
Here we present the first efficient -approximation algorithm 
for finding an axis-parallel empty box of maximum volume, 
whose running time is nearly linear for small , and increases only
by an  factor when one goes up one dimension. 

An empty box of maximum volume must be maximal with respect to inclusion. 
Following the terminology in \cite{NLH84},
a {\em maximal empty} box is called {\em restricted}. 
Thus the maximum-volume empty box in  is restricted.
Naamad et~al.\ \cite{NLH84} have shown that in the plane, the number of 
restricted rectangles is , and that this bound is tight.
It was conjectured by Datta and Soundaralakshmi~\cite{DS00} that the
maximum number of restricted boxes is  for each (fixed) .  
The conjecture has been recently confirmed by Backer and
Keil~\cite{BK09a,BK09b} (for ). 
Here we extend (Theorem~\ref{thm:restricted}, Appendix~\ref{sec:restricted})
the constructions with  restricted boxes for 
in \cite{NLH84} and  in \cite{DS00} for arbitrary . 
Independently and simultaneously, Backer and Keil have also obtained
this result~\cite{Ba09,BK09a,BK09b}. Hence the maximum number of restricted
boxes is  for each fixed .   
This means that any algorithm for computing a maximum-volume empty box 
based on enumerating restricted boxes is inefficient in the worst case.
On the other hand, at the expense of giving an -approximation, 
our algorithm does not enumerate all restricted boxes, and
achieves efficiency by enumerating all canonical boxes (to be defined)
instead.  


\paragraph{Our results are:}\quad
\begin{description}
\item[(I)] In Section~\ref{sec:rectangle} we show 
that  for . More precisely: 
, and
. 
From the other direction we have , and
 for any
. Here  is the th prime. 

\item[(II)] In Section~\ref{sec:approx1} we exploit the fact that
the maximum volume is  in our design
of the first efficient 
-approximation algorithm for finding the largest empty box:
Given an axis-parallel -dimensional box  in  containing
 points, there is a -approximation algorithm, running in  \linebreak
 time, 
for computing a maximum-volume empty axis-parallel box
contained in . 

\item[(III)] In Appendix~\ref{sec:square} we show that the
  estimate also holds for the maximum volume
(or area) of an axis-aligned hypercube (or square) amidst  point in . 
In Appendix~\ref{sec:approx2} we present a faster
-approximation algorithm for finding the largest empty
hypercube: Given an axis-parallel -dimensional hypercube  in  containing
 points, there is a -approximation algorithm, running in  
 time, 
for computing a maximum-volume empty axis-parallel hypercube
contained in . 

\item[(IV)] In Appendix~\ref{sec:restricted} we derive
an  lower bound on the number of restricted boxes in
-space, for fixed . This matches the recent  upper bound
of Backer and Keil~\cite{BK09a,BK09b}. Following their idea, we further
narrow the gap between the bounds (in the dependence of ) 
based on a finer estimation. 

\end{description}



\section{Empty rectangles and boxes}\label{sec:rectangle}


\subsection{Empty rectangles in the plane}\label{subsec:plane}


\paragraph{The lower bound.}
We start with a very simple-minded lower bound; however, as it turns
out, it is very close to optimal.
One can immediately see that , by
partitioning the unit square with vertical lines through each point:
out of at most  resulting empty rectangles, the largest rectangle
has area at least . Thus we have:

\begin{proposition} \label{P1}

\end{proposition}


The following observation is immediate from invariance under scaling 
with respect to any of the coordinate axes. 

\begin{observation} \label{O1}
Assume that  holds for some  and .
Then, given  points in an axis-aligned rectangle , there is an empty
rectangle contained in  of area at least . 
\end{observation}



Using the next two lemmas we will slightly improve the trivial lower bound
 in  our next Theorem~\ref{T2}. 
Let   be the solution in  of the
quadratic equation . 


\begin{lemma} \label{L1}
Given  points in the unit square, there exists an empty rectangle
with area at least . 
This bound is tight, i.e., .
\end{lemma}
\begin{proof}
Let , and assume without loss of generality
that , and  . Write
, and . 
Consider the three empty rectangles ,
, and
.
Their areas are , , and , respectively. 
If  or ,
we are done, as one of the first two rectangles has area at least
. So we can assume that  and .
Then it follows that

so the third rectangle has area at least , as required.

To see that this bound is tight, take ,
, and check that the largest empty rectangle has area . 
\end{proof}

The proof of the next lemma appears in Appendix \ref{app:L2}.

\begin{lemma} \label{L2}
Given  points in the unit square, there exists an empty rectangle
with area at least . This bound is tight, i.e.,
.  
\end{lemma}





\begin{theorem} \label{T2}
Given  points in the unit square, there exists an empty rectangle
with area at least . 
That is, . 
\end{theorem}
\begin{proof}
Write , for some  and . 
Partition  into  rectangles of equal width. There exists
at least one rectangle  with at most  points in its interior. 
By Lemma~\ref{L2} and Observation~\ref{O1}, 
 contains an empty rectangle of area at least

as claimed.
\end{proof}


The lower bound derived in the proof, , is
better than  for all . For ,
the resulting bound is . 
An alternative partition, yielding the same bound in Theorem~\ref{T2},
can be obtained by dividing  into rectangles with vertical lines
through every th point of the set. Slightly better lower
bounds, particularly 
for small values of  can be obtained by constructing different
partitions tailored for specific values of  (with a number of
points other than  in a few of the rectangles), and using estimates
on , , etc.  For instance, from Lemma~\ref{L2} we can derive that 
. 
Incidentally, we remark that a suitable -point construction gives
from the other direction that . 


\paragraph{The upper bound.}
Let  be the van der Corput set of  points~\cite{C35a,C35b},
with coordinates , , constructed as
follows~\cite{Ch00,Ma99}: Let .
If  is the binary representation of ,
where , then .
Observe that all points in  lie in the unit square .


\begin{theorem} \label{T1}
For the van der Corput set of  points, , the area of
the largest empty axis-parallel rectangle is less than .
\end{theorem}
\begin{proof}
Let  be any open empty axis-parallel rectangle inside the unit square.
We next show\footnote{The argument we use here is similar to
that used for bounding the geometric discrepancy of the van der Corput
set of points.} that the area of  is less than . 
Following the presentation in \cite[p.~39]{Ma99},
a \emph{canonical} interval is an interval of the form

for some positive integer  and an integer .

Let 
be the longest canonical interval
contained in the projection of the empty rectangle  onto the -axis
(recall that  is open, so this projection is an open interval).
Then the side length of  along  must be less than
 because otherwise the projection would
contain a longer canonical interval of length .

Let  be the binary representation of
an integer , .
In the van der Corput construction,
a point in  with -coordinate 
has its -coordinate in the canonical interval  if and only if
,
which happens exactly when
.
In this case,  is a constant
.
It then follows that the side length of  along 
is at most .
Therefore the area of  is less than
, as required.
\end{proof}

\begin{corollary} \label{C1}
.
\end{corollary} 


\subsection{Empty boxes in higher dimensions}\label{subsec:higher}

As in the planar case,  is immediate,
by partitioning the hypercube  with  axis-parallel hyperplanes,
one through each of the  points. 
By projecting the  points on one of the faces of ,
and proceeding by induction on , it follows that the lower bound in
Theorem \ref{T1} carries over here too. 
Thus we have:

\begin{proposition} \label{P2}
. Moreover,
. 
\end{proposition}


\old{
By projecting the  points on one of the faces of ,
and proceeding by induction on , it follows that the lower bound in
Theorem \ref{T1} carries over here too. 
That is, . 
} 

We next show that, modulo a constant factor depending on , this estimate is also best possible. 
Let  be the Halton-Hammersely set of 
points~\cite{Hal60,Ham60}, with coordinates 
, ,
constructed as follows \cite{Ch00,Ma99}:
Let  be the th prime number.
Each integer  has a unique base- representation
, where .
Let ,
and let , .
Then all points in  are inside the unit hypercube .

\begin{theorem} \label{T3}
For the  Halton-Hammersely set of  points, , 
the volume of the largest empty axis-parallel box is less than 
, where  is the th prime.
\end{theorem}
\begin{proof}
Let  be any open empty box inside the unit hypercube.
We next show that the volume of 
is less than .
Generalizing the planar case,
a \emph{canonical} interval of the axis , ,
is an interval of the form

for some positive integer  and an integer .
Note that .

First consider each axis , .
Let 
be a longest canonical interval (there could be more than one for
)
contained in the projection of the empty box  onto the axis .
Then the side length of  along  must be less than
 because otherwise the projection would
contain a longer canonical interval of length .

Next consider the axis .
Let  be the base- representation of
an integer ,  and .
In the Halton-Hammersely construction,
a point in  with -coordinate 
has its -coordinate in the canonical interval  if and only if
,
which happens exactly when
.
In this case, 
is a constant .

Note that the  integers , ,
are relatively prime.
By the Chinese remainder theorem,
it follows that a point in  with -coordinate 
has its -coordinate in the canonical interval  for all

if and only if
 for some integer
.
Therefore the side length of  along 
is at most .
Consequently, the volume of  is less than
.
\end{proof}

\begin{corollary} \label{C2}
.
\end{corollary} 

It is known  that  as  \cite{R97}. 


\section{A -approximation algorithm for finding the largest
empty box}\label{sec:approx1} 

Let  be an axis-parallel -dimensional box in  
containing  points. In this section, we present an efficient
-approximation algorithm for computing a maximum-volume
empty axis-parallel box contained in~. 

\begin{theorem} \label{T4}
Given an axis-parallel -dimensional box  in  containing
 points, there is a -approximation algorithm, running in  

time, for computing a maximum-volume empty axis-parallel box
contained in . 
\end{theorem}


We first set a few parameters. 


\paragraph {Parameters.}

We assume that , and , 
which cover all cases of interest. To somewhat simplify our
calculations we also assume that . Let us choose parameters
 
Let  be the unique positive integer such that
 


We next derive some inequalities that follow from this setting.
By assumptions  and ,
we have ,
and .  Then a simple calculation shows
that 
 

It is also clear that .
So  satisfies
 
Since  and , it follows from the
second inequality in (3) that . 
We now derive an upper bound on  as a function of ,  and
. First observe that 

We also have 

From~\eqref{E28} we deduce the following sequence of inequalities: 
 


From~\eqref{E5}, a straightforward calculation 
(where we use  and ) gives
 


\paragraph{Overview of the algorithm.}
By a direct generalization of Observation~\ref{O1}, we can assume w.l.o.g.\  
that . Let  be the set of  points contained in .
The algorithm generates a finite set  of {\em canonical boxes};
to be precise, only a subset of {\em large} canonical boxes.
For each large canonical box , a corresponding {\em
canonical grid} is considered, and  is placed with its lowest
corner at each such grid position and tested for emptiness and containment in . 
The one with the largest volume amongst these is returned in the end. 

\paragraph{Canonical boxes and their associated grids.}
Consider the set  of {\em canonical boxes}, whose 
all side lengths are elements of
 
For a given canonical box , with sides 
,
consider the {\em canonical grid associated with } 
with points of coordinates 

 
contained in . 

Let  be a maximum-volume empty box in , with
. By the trivial inequality  of Proposition~\ref{P2}, we have
. This lower bound is crucial
in the design of our approximation algorithm, as it enables us to
bound from above the number of large canonical boxes (canonical boxes
of smaller volume can be safely ignored).  

Consider the following set  of  intervals 
 
Observe that for each , the extent in the th
coordinate of  is at least ,
since otherwise we would have 
, a contradiction.
Let  be the extent in the th coordinate of , for
. By the above observation, for each ,
 belongs to one of the last  intervals in the set . That
is, there exists an integer , such that 
 

The next lemma shows that  contains an (empty) canonical box
with side lengths 
 
at some position in the canonical grid associated with it. 
We call such a canonical box contained in a maximum-volume empty box,
a  {\em large canonical box}. Two key properties of large
canonical boxes are proved in Lemma~\ref{L9} and Lemma~\ref{L8}. 


\begin{lemma} \label{L7}
If for each , 
the extent in the th coordinate of  belongs to the interval
as in \eqref{E1}, then  contains an (empty) large canonical box
 with side lengths as in \eqref{E2} at some position in the
canonical grid associated with it.    
\end{lemma}
\begin{proof}
It is enough to prove the containment for each coordinate axis . 
Let  and  be the corresponding intervals of 
and , respectively.
Assume for contradiction that the placement of  with its left end
point at the first canonical grid position in  is not contained in 
. But then we have, by taking into account the grid cell lengths:

and consequently,

We reached a contradiction to the 2nd inequality in \eqref{E13}, and
the proof is complete. 
\end{proof}

We now show that the (empty) large canonical box  from the
previous lemma yields a -approximation for the empty box 
of maximum volume. 


\begin{lemma} \label{L9}

\end{lemma}
\begin{proof}
By \eqref{E2} and \eqref{E1}, 

It remains to be shown that 

But this follows from our choice of  and from Bernoulli's
inequality: 

Indeed,

and the proof of Lemma~\ref{L9} is complete.
\end{proof}


Observe that the number of canonical boxes in  is exactly ,
and by~\eqref{E20} is bounded from above as follows:

We can prove however a better upper bound on the number of large canonical
boxes. 

\begin{lemma} \label{L8}
The number of large canonical boxes in  is at most

\end{lemma}
\begin{proof}
Recall that  satisfies

for some integers . 
It follows immediately that
 
and we want an upper bound on the number of solutions of  \eqref{E6}.
Make the substitution , for . 
So , for . 
The above inequalities are equivalent to
 


Let  be a nonnegative integer. It is well-known (see for instance 
\cite{T95}) that the number of
nonnegative integer solutions of the equation  
equals , that is, when we ignore the upper bound
on each . Summing over all values of ,
and using a well-known binomial identity (see for instance 
\cite[p. 217]{T95}) 
yields that the number of solutions of \eqref{E29}, hence also of \eqref{E6},
is no more than

A well-known upper bound approximation for binomial coefficients

for positive integers  and  with , further yields that
 


We now check that 

Recall inequality \eqref{E5}. A straightforward calculation 
(where we use , , and ), gives
 
as claimed. Substituting this upper bound into \eqref{E7} yields
 
as required. This expression is an upper bound on the number of 
solutions of \eqref{E6}, hence also on the number of large canonical
boxes in .  
\end{proof}

Given a grid with cell lengths , we superimpose it so that the
origin of  is a grid point of the above grid. Denote the
corresponding grid cells by index tuples , 
where . 
Note that some of the grid cells on the boundary of  may be
smaller. Given a grid  superimposed on , let  be the
number of cells (with nonempty interior) into which  is partitioned. 

Consider a (fixed) canonical box, say ,  with side lengths as in
\eqref{E2}. The associated canonical grid, say , has side lengths
 times smaller in each coordinate. 
We now derive an upper bound on the number of canonical
grid positions where a canonical box is placed and tested for emptiness,
according to \eqref{E10}. 

\begin{lemma} \label{L10}
The number of canonical grid positions for placing  in  is
bounded as follows: 

\end{lemma}
\begin{proof}
We have

Observe that

By substituting this bound in the product we get that
 
For the last inequality above we used \eqref{E6}.
We now bound from above each of the three factors in \eqref{E11}.
For bounding the second and the third factors we use inequalities
\eqref{E12} and \eqref{E28}, respectively.



Substituting these upper bounds in \eqref{E11} gives the desired bound:


\vspace{-2\baselineskip}
\end{proof}


\paragraph{Testing canonical boxes for emptiness.}

Given a grid with cell lengths , denote the corresponding 
grid {\em cell counts} or {\em cell numbers}
(i.e., the number of points) in cell
 by .
For simplicity, we can assume w.l.o.g.\ that in all the grids that are
generated by the algorithm, no point of  lies on a grid cell
boundary. Indeed the points of  on the boundary of  can be
safely ignored, and the above condition holds with probability 
if instead of the given , the algorithm uses a value chosen
uniformly at random from the interval ;
see also the setting of the parameters in \eqref{E14}. 
Given a grid , and dimensions (array sizes) , 
a floating box at some position aligned with it, that is,
whose lower left corner is a grid point, and with the specified 
dimensions is called a {\em grid box}. All the canonical boxes
generated by our algorithm are in fact grid boxes. 

The next four lemmas (\ref{L3},~\ref{L4},~\ref{L5},~\ref{L6})
outline the method we use for efficiently computing the 
number of points in  in a rectangular box, over a sequence 
of boxes. In particular these boxes can be tested for emptiness within
the same specified time. 

\begin{lemma} \label{L3}
Let  be a grid with cell lengths , 
superimposed on , with  cells.
Then the number of points of  lying in each cell, over all cells,
can be computed in  time. 
\end{lemma}
\begin{proof}
The number of points in each cell of  is initialized to . 
For each point , its cell index tuple (label) is computed
in  time using the floor function for each coordinate, 
and the corresponding cell count is updated. 
\end{proof}

\noindent{\bf Remark.} If the floor function is not an option,
the number of points in each cell can be computed using binary search 
for each coordinate. The resulting time complexity is
. 

\medskip
Denote by  the number of points in  in the
box with lower left cell , and upper right cell
; we refer to the numbers  
as {\em corner box} numbers. 


\begin{lemma} \label{L4}
Given a grid  with cell lengths  placed at the origin, 
with  cells, and grid cell counts ,  over all
cells, the corner box numbers , over all cells,
can be computed in  time.  
\end{lemma}
\begin{proof}
Define , if  for some . 
Let  be a binary vector.
Let the {\em parity} of  be 
. 
By the inclusion-exclusion principle, the corner box numbers
are given by the following formula with at most  operations:

Since  has  cells, the bound follows.
\end{proof}


\begin{lemma} \label{L5}
Given is a grid  with cell lengths  placed at the origin, 
with  cells, and corner box numbers ,
over all cells. Let  be a (canonical) grid box with dimensions
(array sizes) , and lower left cell
. Then the number of points of  in ,
denoted , can be computed in  time. 
\end{lemma}
\begin{proof}
Let  be the upper right cell of
. By the inclusion-exclusion principle, the corner box number
 can be computed as follows:

Hence  can be extracted from the above formula with at most 
operations. 
\end{proof}


Let  be the number of points in  in the
canonical box of dimensions (array sizes) , 
and lower left cell .

\begin{lemma} \label{L6}
Given is a grid  with cell lengths  placed at the origin, 
with  cells, and corner box numbers ,
over all cells. Then the numbers (counts)  ,
over all cells, can be computed in  time. 
\end{lemma}
\begin{proof}
There are  cells determined by  in , and for each, apply
the bound of Lemma~\ref{L5}.
\end{proof}


\paragraph{The last step in the proof of Theorem~\ref{T4}.}

For each canonical box, say , there is a unique associated
canonical grid, say . The time taken to test  for emptiness
and containment in  when placed at all grid positions in , is
obtained by adding the running times in lemmas~\ref{L3}, 
\ref{L4}, and~\ref{L6}:
 
where we have used the upper bound on  in Lemma~\ref{L10}.
By multiplying this with the upper bound on the number of large canonical
boxes in Lemma~\ref{L8}, we get that the total running time of the 
approximation algorithm is 
 
The proof of Theorem~\ref{T4} is now complete.


\later{
\section{Concluding remarks}

Reducing the gap between the lower and upper bounds on ,
particularly in higher dimensions remains an interesting problem. 
Other questions can be asked regarding the computational complexity
of computing a maximum-volume empty box. 
We list some specific questions and directions for further study:

\begin{itemize}
\itemsep 0.01in
\item [(1)] Is the dependence on  necessary in the upper bound on
   as given by our Theorem~\ref{T3}, or is , where  is an absolute constant? As a preliminary
  question: Given  points in the unit hypercube , is there
  always an empty box of volume , where  is an absolute
  constant, or does  tend to zero with the dimension? 
\item [(2)] Most likely the dependence on  of the running time of
  our approximation algorithms, for boxes and hypercubes, is close to
optimal. However, reducing the dependence on  and  in the
running time may extend the range of dimensions for which the
algorithm is practical.  
\end{itemize}
} 


\begin{thebibliography}{9}
\itemsep 3pt

\bibitem{AS87}
A. Aggarwal and S. Suri,
Fast algorithms for computing the largest empty rectangle,
in: \emph{Proceedings of the 3rd Annual Symposium on Computational Geometry},
1987, pp.~278--290.

\bibitem{AF86}
M. Atallah and G. Fredrickson,
A note on finding the maximum empty rectangle, 
\emph{Discrete Applied Mathematics},
\textbf{13} (1986), 87--91.

\bibitem{AK89}
M. Atallah and S.R. Kosaraju,
An efficient algorithm for maxdominance, with applications,
\emph{Algorithmica},
\textbf{4} (1989), 221--236.

\bibitem{Ba09}
J. Backer, personal communication, August 2009.

\bibitem{BK09a}
J. Backer and M. Keil,
The bichromatic rectangle problem in high dimensions,
in: \emph{Proceedings of the 21st Canadian Conference on Computational Geometry},
2009, pp.~157--160.

\bibitem{BK09b}
J. Backer and M. Keil,
The mono- and bichromatic empty rectangle and square problems in all dimensions,
manuscript submitted for publication, 2009.

\bibitem{Ch00}
B. Chazelle,
\emph{The Discrepancy Method: Randomness and Complexity},
Cambridge University Press, 2000.

\bibitem{CDL86}
B. Chazelle, R. Drysdale and D.T. Lee,
Computing the largest empty rectangle, 
\emph{SIAM J.\ Comput.}, 
\textbf{15} (1986), 300--315.

\bibitem{C35a}
J.G. van der Corput,
Verteilungsfunktionen I.,
\emph{Proc.\ Nederl.\ Akad.\ Wetensch.}, 
\textbf{38} (1935), 813--821.

\bibitem{C35b}
J.G. van der Corput,
Verteilungsfunktionen II.,
\emph{Proc.\ Nederl.\ Akad.\ Wetensch.}, 
\textbf{38} (1935), 1058--1066.

\bibitem{D92}
A. Datta,
Efficient algorithms for the largest empty rectangle problem, 
\emph{Inform.\ Sci.},
\textbf{64} (1992), 121--141.

\bibitem{DS00}
A. Datta and S. Soundaralakshmi,
An efficient algorithm for computing the
maximum empty rectangle in three dimensions,
\emph{Inform.\ Sci.},
\textbf{128} (2000), 43--65.

\bibitem{EGLM03}
J. Edmonds, J. Gryz, D. Liang, and R. Miller,
Mining for empty spaces in large data sets,
\emph{Theoretical Computer Science}, 
\textbf{296(3)} (2003), 435--452.

\bibitem{Hal60}
J.H. Halton,
On the efficiency of certain quasi-random sequences of points in
evaluating multi-dimensional integrals, 
\emph{Numer.\ Math.}, 
\textbf{2} (1960), 84--90.

\bibitem{Ham60}
J.M. Hammersley,
Monte Carlo methods for solving multivariable problems,
\emph{Ann.\ New York Acad.\ Sci.}, 
\textbf{86} (1960), 844--874.

\bibitem{Ma99}
J. Matou\v{s}ek,
\emph{Geometric Discrepancy: An Illustrated Guide},
Springer, 1999.

\bibitem{MOS85}
M. Mckenna, J. O'Rourke, and S. Suri,
Finding the largest rectangle in an orthogonal polygon,
in: \emph{Proceedings of the 23rd Annual Allerton Conference on Communication,
  Control and Computing}, Urbana-Champaign, Illinois, October 1985. 

\bibitem{NLH84}
A. Namaad, D.T. Lee, and  W.-L. Hsu,
On the maximum empty rectangle problem,
\emph{Discrete Applied Mathematics}
\textbf{8} (1984), 267--277.

\bibitem{O90}
M. Orlowski,
A new algorithm for the largest empty rectangle problem,
\emph{Algorithmica},
\textbf{5} (1990), 65--73.

\bibitem{R97}
S.M. Ruiz,
A result on prime numbers,
\emph{Mathematical Gazette},
\textbf{81} (1997), 269--270.

\bibitem{T95}
A. Tucker,
\emph{Applied Combinatorics}, 3rd edition,
Wiley, 1995.

\end{thebibliography}




\appendix


\section{Proof of Lemma~\ref{L2}} \label{app:L2}

To see that , consider the  points
, 
,
,
,
and check that the largest empty rectangle has area  . 
Next we prove the lower bound. 
Let  be a set of  points, and assume without
loss of generality that they are in lexicographic order: 
,  
and if  for , then . 
We can also assume that . Encode each possible such
-point configuration by a permutation  of  as follows:
for ,  if and only if . 
For example  encodes the configuration shown in
Fig.~\ref{f1}(right). 
\begin{figure} [htb]
\centerline{\epsfxsize=4in \epsffile{f1.eps}}
\caption{\small Left:  is special.
Right:  is non-special;  is the right side of , 
is the lower left corner of , and  is the top side of .} 
\label{f1}
\end{figure}


By our assumption , there are only 
permutations (types) out of the total of  to consider, those with
. 
Two of these permutations, namely  and , are
called {\em special}: the  points are in convex position and there
is an empty rectangle , with one of these points on each side of
. All the remaining  permutations are called {\em non-special}.
We distinguish two cases: 

\medskip
{\em Case 1:}  is encoded by a special permutation.
For each of the four sides  of , let  be the largest
empty rectangle containing .
See Fig.~\ref{f1}(left) for an example. 
We can assume that the area of each rectangle 
 is smaller than , since else we are done. But then it
follows that each of the four sides of  is longer than 
, so the area
of  is larger than ,
so this case is settled.

\medskip
{\em Case 2:}  is encoded by a non-special permutation.
For each of the four vertices  of , let  be the largest
empty rectangle having  as a vertex. A routine verification shows that for
each of the  non-special permutations there is a side  of 
and a vertex  of  such that (i)  and  have a common
boundary segment, and (ii)  is an endpoint of the side opposite to .
More precisely,
if  is one of six permutations
, , , 
, , , 
then  is the left side, and  is the lower-right corner;
if  is one of four permutations
, , , , 
then  is the right side, and  is the lower-left corner. 
See Fig.~\ref{f1}(right) for an example. 

As in Case 1, we can assume that
the area of  is smaller than , thus its shorter side
is smaller than . By the same token, one of the sides of  
is longer than , hence the other side must
be shorter than , 
since otherwise the area of  would exceed .
Let  be the side of  adjacent to  and disjoint from .
Consequently, the rectangle  with side  and adjacent to 
has the other side longer than . Observe
that  has at most two points in its interior. By Lemma~\ref{L1}
and Observation~\ref{O1},  contains an empty rectangle of area at
least 

as required. This concludes the analysis of the second case.

Thus in both cases, there is an empty rectangle of area at least .
\qed



\section{Empty squares and hypercubes}\label{sec:square}

Define  as the volume of the largest empty axis-parallel
hypercube (over all -element point sets in in ), analogous
to  for the largest empty axis-parallel box.  
For simplicity we sometimes omit the subscript  in the planar case
().  That is,  denotes the area of the largest empty 
axis-parallel square. Then for any fixed dimension , our next
theorem shows that , too: 

\begin{theorem} \label{T5}
For a fixed , . 
More precisely,

\end{theorem}
\begin{proof}
We will prove the bounds for the planar case :

The proof can be easily generalized for .

We first prove the lower bound. Let  be a set of  points in the unit
square . Let  be a positive number to be determined. Let  be
an axis-parallel square of side  that is concentric with . For
each point , place an axis-parallel (open) square of side 
centered at . If there is a point  that is not covered by the
union of the  squares, then the axis-parallel square of side 
centered at  is an empty square contained in .

The area of  is . The total area of  squares of side
 is . Let  be the solution to the following equation

The solution is . For this value of , 
either the  small squares cover  with no interior overlap among
themselves, or there is interior overlap and they don't cover . In
either case, there exists an open axis-parallel square of side length
, centered at a point in , and empty of points in . Consequently,  



We next prove the upper bound. Let . Note
that . Partition the unit square  into a 
square grid of cell length . Place a point at each of the 
grid vertices in the interior of . Then any axis-parallel square
contained in  whose side is longer than , must be
non-empty. Consequently, 



It remains to show that \eqref{E31} implies that for a fixed , 
we have .
The following inequalities are straightforward:


Putting them together yields

as claimed.
\end{proof}


\section{A -approximation algorithm for finding the largest
empty \\ hypercube}\label{sec:approx2}  

Let  be an axis-parallel -dimensional hypercube in  
containing  points. In this section, we present an efficient
-approximation algorithm for computing a maximum-volume
empty axis-parallel hypercube contained in . 
\later{
Recall that 
with exact algorithms, a largest empty hypercube can be found faster
than a largest empty box~\cite{BK09a,BK09b}, as mentioned in the
introduction. In our case, with approximation algorithms, 
the situation is analogous, and we are able to obtain a faster
algorithm for finding the largest hypercube:
} 

\begin{theorem} \label{T6}
Given an axis-parallel -dimensional hypercube  in  containing
 points, there is a -approximation algorithm, running in  

time, for computing a maximum-volume empty axis-parallel hypercube 
contained in . 
\end{theorem}
\begin{proof} The overall structure of the algorithm is
similar to that for finding the largest empty box. 
We can assume w.l.o.g. that , , and . 
Recall that, by Theorem~\ref{T5}, the volume of a largest empty
hypercube in  is at least .  
We set the parameters ,  and  as in equation \eqref{E14}. 
Inequalities \eqref{E12} and \eqref{E13} also follow.
Let now  be the unique positive integer such that
 
Thus 

Since  and  we have 

It follows that 
 


Consider the set  of  {\em canonical hypercubes} whose sides
are elements of  (as in \eqref{E18}):
 


For a given canonical hypercube , with side ,
consider the {\em canonical grid associated with } 
with points of coordinates 
 
contained in . 

Consider the set  of  intervals (as in \eqref{E17}): 
 


Let  be a maximum-volume empty hypercube in , with side
length  and . Observe that : 
indeed,  would imply that 

in contradiction to the lower bound in Theorem~\ref{T5}. 
This means that  belongs to one of the last  intervals in the
set . That is, there exists an integer , such that 
 

Analogous to Lemma~\ref{L7}, we conclude that  contains a 
{\em large canonical hypercube}, say , whose side is
 
at some position in the canonical grid associated with it. 
Analogous to Lemma~\ref{L9}, we show that 
:
By \eqref{E23} and \eqref{E22}, 

since the setting of  is the same as before. 
Analogous to Lemma~\ref{L8}, now \eqref{E27} is the upper bound we
need on the number of canonical hypercubes. 
The bound in Lemma~\ref{L10} needs to be adjusted because  is
chosen differently, and we have a different upper bound on the third
factor in the product, . From the definition of  in
\eqref{E30} and from \eqref{E13} we deduce 

The resulting bound analogous to that in Lemma~\ref{L10} is now
 


The time taken to test  for emptiness and containment in  when
placed at all relevant grid positions is now
 


By multiplying this with the upper bound in~\eqref{E27}, 
on the number of canonical hypercubes, we get that the total
running time of the approximation algorithm is 

The proof of Theorem~\ref{T6} is now complete.
\end{proof}


\section{An asymptotically tight bound on the number of restricted
boxes}\label{sec:restricted} 

In this section we prove the following theorem:

\begin{theorem}\label{thm:restricted}
Let  be the unit hypercube .
For any , there exist  points in 
such that the number of restricted boxes in 
is at least
.
On the other hand,
the number of restricted boxes determined by any set of  points in 
is at most
.
\end{theorem}


We prove the lower bound in Theorem~\ref{thm:restricted} by construction.
We will use the following lemma:

\begin{lemma}\label{lem:construction}
Let , where , .
Then there exist  points in 
such that the number of maximal empty axis-parallel boxes in 
is at least .
\end{lemma}
\begin{proof}
Let 
be the positive and negative unit vectors along the  axes of .
Partition these  vectors into  groups of orthogonal vectors,

with one positive vector and one negative vector in each group.
Then, for each group of two orthogonal vectors, say ,
place a sequence of  points in  as

where each pair of consecutive points in the sequence,
say

corresponds to a pair of open half-spaces

\old{
The intersection of the two half-spaces
contains all points in the other sequences but no points in this sequence,
and is bounded by the two points
in the two directions  and .
There are  combinations of
 pairs of consecutive points,
one pair from each sequence.
For each combination,
the intersection of the corresponding  pairs of half-spaces
is a unique maximal empty axis-parallel box.
}

Consider the pair of open half-spaces  and 
corresponding to the pair of consecutive points in the sequence for
the group .
Since the points in the sequence have monotonic  and  coordinates,
we have property (i) that
the intersection of the two half-spaces contains no points in the sequence,
and property (ii) that
each of the two points is on the boundary of one half-space
and is in the interior of the other half-space.
Moreover,
since the  and  coordinates of the points in the other sequences
are either zero or different in sign from the points in this sequence,
we have (iii) that
each of the two half-spaces contains all points in the other sequences.
There are  combinations of  pairs of consecutive
points, one pair from each sequence.
Consider the intersection  of the  pairs of half-spaces
corresponding to any of these combinations.
By (i), the intersection  must be empty.
By (ii) and (iii),
there is a point in the interior of each bounding face,
thus the intersection box  must be maximal.
Hence for each combination,
the intersection of the corresponding  pairs of half-spaces
is a unique maximal empty axis-parallel box.
\begin{figure}[htb]
\psfrag{x}{}
\psfrag{y}{}
\centering\includegraphics[width=0.33\linewidth]{construction.eps}
\caption{An example of the construction.}
\label{fig:construction}
\end{figure}

We refer to Fig.~\ref{fig:construction} for an example of the planar case.
For , , and ,
the four unit vectors  and  are grouped into
 and .
The corresponding two sequences of points have the following
-coordinates:  

Then the following two pairs of consecutive points

correspond to the following two pairs of half-planes:

whose intersection is the maximal empty box .
\end{proof}

By scaling and translation, the  points in Lemma~\ref{lem:construction}
can be placed in the unit hypercube  such that
the number of restricted boxes inside  is at least
,
where the change from  to  in the product
accounts for the two bounding faces of the unit hypercube
perpendicular to each axis.
This proves the lower bound. The same lower bound was obtained
independently and simultaneously by Backer and Keil~\cite{Ba09,BK09a,BK09b}.

To prove the upper bound in Theorem~\ref{thm:restricted},
we borrow the deflation-inflation idea of Backer and Keil~\cite{BK09a,BK09b}.
Assume for simplicity that
the points have distinct coordinates along each axis
(it is possible to perturb the points symbolically so this condition holds).
Let  be an arbitrary restricted box.
Consider the  faces of the box in any fixed order.
If a face contains a point in its interior,
deflate the box by pushing the face toward its opposite face
until it contains a point on its boundary.
After  such deflations,
we obtain an empty box  that is the smallest box containing
exactly  points on its boundary.
To recover the original box  from ,
it suffices to inflate the box at the  faces in reverse order,
by pushing each face away from its opposite face until
it contains a point in its interior.
Therefore the number of restricted boxes  is at most
the number of deflated boxes  times the number of subsets of 
deflated faces, that is, .
Since 

and
,
we have

By Stirling's formula, ,
hence 

Thus



Our upper bound is sharper (with respect to the dependence on )
than the upper bound of  by Backer and 
Keil~\cite{BK09a,BK09b}. The ratio of our upper bound to the lower bound is

In comparison, the ratio of their upper bound to the same lower bound is 



\end{document}
