\documentclass[submission,copyright,creativecommons]{eptcs}
\providecommand{\event}{EXPRESS/SOS 2012} 

\usepackage{svn-multi}
\usepackage{cite}
\usepackage{graphicx}
\usepackage{amsmath}
\usepackage{amssymb}
\usepackage{amsthm}
\usepackage{latexsym}
\usepackage{array}
\usepackage{paralist}
\usepackage{color}
\usepackage{textcomp}
\usepackage{mathrsfs}
\usepackage{txfonts}
\usepackage{datetime}
\usepackage{etoolbox}
\usepackage{fancybox}
\usepackage{tikz}
\input xy
\xyoption{all}
\usepackage{ulem}
\normalem
\usepackage{color}

\newtheorem{example}{Example}
\newtheorem{definition}{Definition}
\newtheorem{theorem}{Theorem}
\newtheorem{lemma}{Lemma}
\newtheorem{corollary}{Corollary}

\svnid{}
\settimeformat{hhmmsstime}
\newcommand{\builddate}{\ifnumcomp{\year}{<}{10}{0}{}\the\year-\ifnumcomp{\month}{<}{10}{0}{}\the\month-\ifnumcomp{\day}{<}{10}{0}{}\the\day\ \currenttime}
\ifnumcomp{\svnhour}{=}{23}{\def\svnhour{01}}{}
\ifnumcomp{\svnhour}{=}{22}{\def\svnhour{00}}{}
\ifnumcomp{\svnhour}{=}{21}{\def\svnhour{23}}{}
\ifnumcomp{\svnhour}{=}{20}{\def\svnhour{22}}{}
\ifnumcomp{\svnhour}{=}{19}{\def\svnhour{21}}{}
\ifnumcomp{\svnhour}{=}{18}{\def\svnhour{20}}{}
\ifnumcomp{\svnhour}{=}{17}{\def\svnhour{19}}{}
\ifnumcomp{\svnhour}{=}{16}{\def\svnhour{18}}{}
\ifnumcomp{\svnhour}{=}{15}{\def\svnhour{17}}{}
\ifnumcomp{\svnhour}{=}{14}{\def\svnhour{16}}{}
\ifnumcomp{\svnhour}{=}{13}{\def\svnhour{15}}{}
\ifnumcomp{\svnhour}{=}{12}{\def\svnhour{14}}{}
\ifnumcomp{\svnhour}{=}{11}{\def\svnhour{13}}{}
\ifnumcomp{\svnhour}{=}{10}{\def\svnhour{12}}{}
\ifnumcomp{\svnhour}{=}{09}{\def\svnhour{11}}{}
\ifnumcomp{\svnhour}{=}{08}{\def\svnhour{10}}{}
\ifnumcomp{\svnhour}{=}{07}{\def\svnhour{09}}{}
\ifnumcomp{\svnhour}{=}{06}{\def\svnhour{08}}{}
\ifnumcomp{\svnhour}{=}{05}{\def\svnhour{07}}{}
\ifnumcomp{\svnhour}{=}{04}{\def\svnhour{06}}{}
\ifnumcomp{\svnhour}{=}{03}{\def\svnhour{05}}{}
\ifnumcomp{\svnhour}{=}{02}{\def\svnhour{04}}{}
\ifnumcomp{\svnhour}{=}{01}{\def\svnhour{03}}{}
\ifnumcomp{\svnhour}{=}{00}{\def\svnhour{02}}{}
\newcommand{\wipinfo}{
  \ \ Last change: \svnyear-\svnmonth-\svnday\ \svnhour:\svnminute:\svnsecond \\
  \qquad\quad Build: \builddate \\
  \ \qquad\, Revision: \svnfilerev \qquad\qquad\qquad\qquad\ \ 
}

\newcommand{\N}{\mathbb{N}} \newcommand{\Z}{\mathbb{Z}} \newcommand{\Q}{\mathbb{Q}} \newcommand{\R}{\mathbb{R}} 

\newcommand{\powerset}{\mathscr{P}}

\newcommand{\cardof}[1]{|{#1}|}

\newcommand{\ddedrule}[2]{\frac{\displaystyle #1}{\displaystyle #2}}
\newcommand{\dedrule}[2]{\frac{#1}{#2}}
\newcommand{\trans}[1][]{\xrightarrow{\, {#1} \, }}
\newcommand{\ntrans}[1][]{\mathrel{{\trans[#1]}\makebox[0em][r]{\hspace{2ex}}}{\!}}
\newcommand{\gtgeq}{\trianglerighteq}

\newcommand{\strans}[1][]{\stackrel{#1}{\longrightarrow}}


\newcommand{\tuple}[1]{\langle{#1}\rangle}

\newcommand{\rank}{\mathop{\sf r}}
\newcommand{\openT}{\mathbb{T}}
\newcommand{\openTerms}{\openT(\Sigma)}
\newcommand{\closedTerms}{T(\Sigma)}
\newcommand{\openDT}{\mathbb{DT}}
\newcommand{\openDTerms}{\openDT(\Sigma)}
\newcommand{\closedDTerms}{\DT(\Sigma)}
\newcommand{\Var}{\mathop{\textit{Var}}}
\newcommand{\PTrn}{\textit{PTr}}
\newcommand{\PTr}{\PTrn(\Sigma, A)}
\newcommand{\degree}[1]{\textrm{degree}(#1)}
\newcommand{\pprem}[1]{\textrm{pprem}(#1)}
\newcommand{\nprem}[1]{\textrm{nprem}(#1)}
\newcommand{\qprem}[1]{\textrm{qprem}(#1)}
\newcommand{\prem}[1]{\textrm{prem}(#1)}
\newcommand{\conc}[1]{\textrm{conc}(#1)}
\newcommand{\reduce}[1]{\textrm{Reduce}(#1)}
\newcommand{\Red}[2]{\textrm{Red}}
\newcommand{\nvdg}{n_{\mathrm{VDG}}}


\newcommand{\TVar}{\mathcal{V}}
\newcommand{\PVar}{\mathcal{M}}
\newcommand{\DVar}{\mathcal{D}}

\newcommand{\yes}{\textit{yes}}
\newcommand{\no}{\textit{no}}
\newcommand{\FPr}{\textsf{Pr}}
\newcommand{\FB}{\textsf{B}}
\newcommand{\Ttr}{\textsf{\itshape T}}

\newcommand{\Tr}{\textsf{Tr}}


\newcommand{\support}{\mathsf{support}}

\newcommand{\FC}{\textsf{C}}


\newcommand{\V}{\mathcal{V}}
\newcommand{\M}{\mathcal{M}}

\newcommand{\degPTSS}{\textsf{D}}

\newcommand{\Lab}{\mathcal{L}}
\newcommand{\nullproc}{\mathbf{0}}
\newcommand{\ppop}[4]{{{#1}.\sum_{i=1}^{#2}[{#3}_i]{#4}_i}}
\newcommand{\ppopt}[4]{{{#1}.\textstyle\sum_{i=1}^{#2}[{#3}_i]{#4}_i}}
\newcommand{\parop}[1]{\mathop{||_{#1}}}
\newcommand{\unreach}{\textsf{U}}
\newcommand{\dk}{\textsf{dk}}
\newcommand{\Act}{A}
\newcommand{\tick}{{\surd}}

\newcommand{\set}{\mathop{set}}

\newcommand{\tyft}{\ensuremath{\textit{tyft}}}
\newcommand{\tyxt}{\ensuremath{\textit{tyxt}}}
\newcommand{\tyfxt}{\ensuremath{\tyft\textit{/}\tyxt}}

\newcommand{\ntmufnu}{\ensuremath{\textit{nt}\mu\textit{f}\nu}}
\newcommand{\ntmuxnu}{\ensuremath{\textit{nt}\mu\textit{x}\nu}}
\newcommand{\ntmufxnu}{\ensuremath{\ntmufnu\textit{/}\ntmuxnu}}
\newcommand{\bntmufnu}{\ensuremath{\textit{\bfseries nt}\boldmath\mu\textit{\bfseries f}\boldmath\nu}}
\newcommand{\bntmuxnu}{\ensuremath{\textit{\bfseries nt}\boldmath\mu\textit{\bfseries x}\boldmath\nu}}
\newcommand{\bntmufxnu}{\ensuremath{\bntmufnu/\bntmuxnu}}

\newcommand{\tmuft}{\ensuremath{\textit{t}\mu\textit{f}\theta}}
\newcommand{\tmuxt}{\ensuremath{\textit{t}\mu\textit{x}\theta}}
\newcommand{\tmufxt}{\ensuremath{\tmuft\textit{/}\tmuxt}}
\newcommand{\ntmuft}{\ensuremath{\textit{nt}\mu\textit{f}\theta}}
\newcommand{\ntmuxt}{\ensuremath{\textit{nt}\mu\textit{x}\theta}}
\newcommand{\ntmufxt}{\ensuremath{\ntmuft\textit{/}\ntmuxt}}
\newcommand{\bntmuft}{\ensuremath{\textit{\bfseries nt}\boldmath\mu\textit{\bfseries f}\boldmath\theta}}
\newcommand{\bntmuxt}{\ensuremath{\textit{\bfseries nt}\boldmath\mu\textit{\bfseries x}\boldmath\theta}}
\newcommand{\bntmufxt}{\ensuremath{\bntmuft/\bntmuxt}}


\newcommand{\xmuft}{\ensuremath{\textit{x}\mu\textit{f}\theta}}
\newcommand{\xmuxt}{\ensuremath{\textit{x}\mu\textit{x}\theta}}
\newcommand{\xmufxt}{\ensuremath{\tmuft\textit{/}\xmuxt}}
\newcommand{\nxmuft}{\ensuremath{\textit{nx}\mu\textit{f}\theta}}
\newcommand{\nxmuxt}{\ensuremath{\textit{nx}\mu\textit{x}\theta}}
\newcommand{\nxmufxt}{\ensuremath{\ntmuft\textit{/}\nxmuxt}}

\newcommand{\nxmutt}{\ensuremath{\textit{nx}\mu\textit{t}\theta}}


\newcommand{\ntyft}{\ensuremath{\textit{ntyft}}}
\newcommand{\ntyxt}{\ensuremath{\textit{ntyxt}}}
\newcommand{\ntyfxt}{\ensuremath{\ntyft\textit{/}\ntyxt}}

\newcommand{\Diag}{\mathop{\textsf{Diag}}}
\newcommand{\diag}[1]{\textsf{Diag}\{#1\}}
\newcommand{\family}[3][]{{\left\{{#2}\right\}_{#3}^{#1}}}

\newcommand{\partZ}{\Xi}

\newcommand{\limp}{\Rightarrow}






\definecolor{lightblue}{RGB}{224,224,255}
\definecolor{lightred}{RGB}{255,224,224}
\definecolor{lightgreen}{RGB}{224,255,224}
\definecolor{lightyellow}{RGB}{255,255,224}
\definecolor{lightpurple}{RGB}{255,224,255}
\definecolor{darkerred}{RGB}{64,0,0}
\definecolor{darkred}{RGB}{128,0,0}
\definecolor{darkblue}{RGB}{0,0,128}
\definecolor{darkgreen}{RGB}{0,128,0}
\definecolor{darkpurple}{RGB}{128,0,128}

\newcommand{\colorpar}[3]{\colorbox{#1}{\parbox{#2}{#3}}}

\newcommand{\marginremark}[3]{\marginpar{\colorpar{#2}{\linewidth}{\color{#1}#3}}}

\makeatletter
\def\THICKhrulefill{\leavevmode \leaders \hrule height 5pt\hfill \kern \z@}
\makeatother
\newcommand{\highlightedremark}[4]{\begin{center}\fcolorbox{#1}{#2}{\begin{minipage}{.98\linewidth}\color{#1}\textbf{\THICKhrulefill[ #3 ]\THICKhrulefill}\par\noindent#4\end{minipage}}\end{center}}

\newcommand{\remarkPRD}[1]{\marginremark{darkred}{lightred}{\tiny{[PRD]~ #1}}}
\newcommand{\remarkMDL}[1]{\marginremark{darkpurple}{lightpurple}{\tiny{[MDL]~ #1}}}
\newcommand{\remarkDG}[1]{\marginremark{darkgreen}{lightyellow}{\tiny{[DG]~ #1}}}

\newcommand{\hrmkPRD}[1]{\highlightedremark{darkred}{lightred}{PRD}{#1}}
\newcommand{\hrmkMDL}[1]{\highlightedremark{darkpurple}{lightpurple}{MDL}{#1}}



\newcommand{\bisim}{\sim}
\newcommand{\traceCong}{\equiv^T_P}
\newcommand{\boundedBisim}{\simeq}
\newcommand{\relR}{\mathrel{\textsf{R}}}

\newcommand{\closed}[1]{{#1}\text{-closed}}



\newcommand{\liff}{\Leftrightarrow}
\renewcommand{\implies}{\Rightarrow}

\newcommand{\proves}{\vdash}
\newcommand{\rtop}{\text{top}}
\newcommand{\qtop}{\text{qtop}}


\newcommand{\DT}{\textsf{DT}}

\newcommand{\wsproves}{\vdash_{\textit{ws}}}
 

\newif\ifdraft
\draftfalse

\ifdraft
\else
\renewcommand{\remarkDG}[1]{}
\renewcommand{\remarkMDL}[1]{}
\renewcommand{\remarkPRD}[1]{}
\fi


\title{Tree rules in probabilistic transition system specifications with negative and quantitative premises\thanks{Supported by Project ANPCYT PAE-PICT 02272, SeCyT-UNC, 
                 Eramus Mundus Action 2 Lot 13A EU Mobility Programme 2010-2401/001-001-EMA2 and EU 7FP grant agreement 295261 (MEALS).}}
\author{
Matias David Lee  \qquad\qquad
Daniel Gebler  \qquad\qquad
Pedro R. D'Argenio
\and
\institute{FaMAF -- CONICET \\Universidad Nacional de C\'ordoba\\Ciudad Universitaria, X5000HUA C\'ordoba \\ Argentina}
\email{\{lee,dargenio\}@famaf.unc.edu.ar}
\and
\institute{Department of Computer Science\\VU University Amsterdam \\ De Boelelaan 1081a, 1081HV Amsterdam \\ The Netherlands}
\email{e.d.gebler@vu.nl}
\ifdraft
\and\institute{\wipinfo}
\fi
}

\def\titlerunning{Tree rules in probabilistic transition system specifications}
\def\authorrunning{M. D. Lee, D. Gebler \& P. R. D'Argenio}

\begin{document}
\maketitle

\begin{abstract}
Probabilistic transition system specifications (PTSSs) in the
\ntmufxnu\ format provide structural operational semantics for
Segala-type systems that exhibit both probabilistic and
nondeterministic behavior and guarantee that bisimilarity is a
congruence.
Similar to the nondeterministic case of the rule format {\it tyft/tyxt}, we
show that the well-foundedness requirement is unnecessary in the
probabilistic setting.
To achieve this, we first define a generalized version of the
\ntmufxnu\ format in which quantitative premises and conclusions
include nested convex combinations of distributions.  
Also this format guarantees that bisimilarity
is a congruence.
Then, for a given (possibly non-well-founded) PTSS in the new format,
we construct an equivalent well-founded PTSS consisting
of only rules of the simpler (well-founded) probabilistic ntree
format.
Furthermore, we develop a proof-theoretic
notion for these PTSSs that coincides with the existing stratification-based meaning
in case the PTSS is stratifiable. This
continues the line of research lifting structural operational semantic
results from the nondeterministic setting to systems with both
probabilistic and nondeterministic behavior.
\end{abstract}

\section{Introduction}


Plotkin's structural operational semantics~\cite{Plotkin81} is a
popular method to provide 
a rigorous interpretation to specification and programming languages. The interpretation 
is given in terms of transition systems.
The method has been formalized with an algebraic flavor as \emph{transition
  systems specifications
  (TSS)}~\cite[etc.]{GrooteVaandrager92,BloomIM95:jacm,Groote93,BolGroote96}.
Basically, a TSS contains a signature, a set of labels, and a set of
rules. The signature defines the terms in the language.  Labels
represent actions performed by a process (i.e., a term over the
signature) in one step of the execution (i.e., one transition).
Rules define how a process should behave (i.e., produce a transition)
in terms of the behavior of its subprocesses. 
That is, rules define
compositionally the transition system associated to each term of the
language.
This technique has been widely studied mainly on the realm of
languages and process algebras describing only non-deterministic
behavior (see~\cite{MousaviEtAl07} for an overview).

The introduction of probabilistic process
algebras~\cite[etc.]{DBLP:journals/iandc/BaetenBS95,DBLP:journals/iandc/GlabbeekSS95}
motivated the need for a theory of structural operational semantics to
define \emph{probabilistic} transition systems.
A few results have appeared in this direction,
notably~\cite{DBLP:journals/entcs/Bartels02,Bartels2004,DBLP:journals/tocl/LanotteT09,klin2008structural,DL-fossacs12}.
All these works introduced rule formats that ensures that bisimulation
equivalence is a congruence for operators whose semantics is defined
within such format.
The most general of those formats is the \ntmufxnu\ format ~\cite{DL-fossacs12} that provides semantics in terms of Segala's probabilistic automata~\cite{Segala95}.

\pagebreak[4]

The \ntmufxnu\ format is the probabilistic relative to the
\ntyfxt\ format~\cite{Groote93} extending it in two ways.  First,
it is designed to deal with probabilistic transitions of the form
, where  is a term in the appropriate signature, and
 is a distribution on terms. Second, it includes quantitative
premises that allow for probabilistic testing of the form
, that is, it allows to verify if the
probability that the system moves to one state (i.e.\ term) in
 according to  is greater than .
The congruence theorem for the
\ntmufxnu\ format~\cite[Thm.~12]{DL-fossacs12} states that if a
probabilistic transition system specification (PTSS)  has all its
rules in \ntmufxnu\ format, then bisimulation equivalence is a
congruence for all operators in .
Unfortunately, \cite{DL-fossacs12}~missed an important condition:
rules have to be well-founded (basically, there should not be a cyclic
dependency on the terms appearing in the premises of the rule).
This paper will correct this mistake.

The well-foundedness condition has also appeared from the very beginning
in the non-deterministic setting.  Most of the formats have it
implicit as they did not allowed lookahead.  Congruence theorems for
formats with lookahead such as
\textit{tyft/tyxt}~\cite{GrooteVaandrager92} or
\ntyfxt~\cite{Groote93} explicitly demanded TSS to be well-founded.
It remained unknown for a while whether such condition was actually
required until Fokkink and van Glabbeek proved it
unnecessary~\cite{FokkinkvanGlabbeek96}.
The proof proceeds by reducing a TSS in \textit{tyft/tyxt}~format (not
necessarily well-founded) to an equivalent TSS containing only so
called \emph{tree rules} (i.e., well-founded rules in \textit{tyft}
format with premises containing only variables instead of
arbitrary open terms).  Similarly, they showed that a TSS in
\ntyfxt\ format can be translated into an equivalent TSS containing
only ntree rules (tree rules with negative premises which are not
necessarily restricted to single variables).

In this paper, we also show that the restriction to well-founded PTSSs
is not necessary to guarantee congruence. We also proceed by
reducing a PTSS in \ntmufxnu\ format to an equivalent PTSS containing
only \textit{pntree rules}.
However, a pntree rule cannot simply be defined as an \ntmufnu\ rule
where positive premises are restricted to the form ,
with  and  being term and distribution variables, respectively.
It turns out that quantitative premises in \ntmufxnu\ rules are
too limited.  The \ntmufxnu\ format only allows for
quantitative premises of the form  with  being a
distribution variable,  an infinite set of term variables,
, and .
Instead, the pntree format requires premises of the form
 where  is a nested convex combinations of
products of distribution variables.  We call these objects
\emph{distribution terms}.
So, we extend the \ntmufxnu\ format to deal with distribution terms, and prove, more generally, that a
PTSS in the new format ---\,called \ntmufxt\,--- can be translated into an equivalent PTSS
with only pntree rules (hence, well-founded).
Just like for the case of the \ntyfxt\ format, full negative
premises are required, i.e., negative premises in pntree rules cannot
be limited to the form , with  being a term
variable.

\medskip

Summarizing, the following results are introduced in this paper:
\begin{itemize}
\item We define the \ntmufxt\ format, which extends the \ntmufxnu\ format
  to deal with distribution terms in quantitative premises.
\item We prove that if a PTSS is in \ntmufxt\ format and it is
  well-founded, then bisimulation equivalence is a congruence for all
  its operators.  This also corrects the mistake in the proof of
  Theorem~12 in~\cite{DL-fossacs12} which omitted to consider the
  well-foundedness hypothesis.
\item We show that for all PTSS in \ntmufxt\ format (not necessarily
  well-founded) there is a PTSS with only pntree rules that defines
  exactly the same probabilistic transition relation (by ``defines'' we
  mean ``has as a \emph{supported model}'')
\item We dropped the well-foundedness hypothesis from the congruence theorem:
  since every pntree rule is also a well-founded \ntmuft\ rule,
  the previous results imply that bisimulation equivalence is a
  congruence for all operators of a (not necessarily
  well-founded) PTSS in \ntmufxt\ format.
\remarkDG{Does statement 4 not states explicitly that `if a PTSS is in \ntmufxt\ format (but not necessarily well-founded), bisimulation equivalence is a congruence for all its operators' - thus make well-foundedness a not necessary precondition? What exactly is in this case the correction?}
\remarkPRD{Actually this is a good point \texttt{:-S}.  Check if the new text is better}
\remarkDG{I believe statements 2 and 4 are still conflicting. Is it not the case that: 1) The proof of bisimilarity is a congruence assumes and works well with the well-foundedness condition. 2) Every PTSS in non well-founded \ntmufxt\  format can be translated to a transition equivalence well-founded PTSS in pntree format 3) Same equiv + pntree are well-founded \ntmufxt\ rules concludes that congruence property is given also for non-well founded PTSS. Do you mean in the beginning of the statement \ntmufxnu\ format which cannot be reduced to a well-founded A\ntmufxnu\ format?}
\remarkPRD{Well, I remove the explanation. There should not be any conflict, now}
\item Besides, in the process, we also redefined important concepts for PTSS
  originally defined for TSS, in particular, the concept of
  ``well supported proof''.
\end{itemize}









\section{Preliminaries}\label{sec:preliminaries}






We assume the presence of an infinite set of (term) variables  and
we let   range over .
A \emph{signature} is a structure , where 
\begin{inparaenum}[(i)]
\item  is a set of \emph{function names} disjoint with , and
\item  is a \emph{rank function} which gives the arity 
  of a function name; if  and  then  is called
  a \emph{constant name}.
\end{inparaenum}
Let  be a set of variables. The set of -terms 
over , notation  is the least set satisfying: 
\begin{inparaenum}[(i)]
\item , and
\item if  and , then
  .
\end{inparaenum}
 is abbreviated as ; the elements
of  are called \emph{closed terms}.  is
abbreviated as ; the elements of  are called
\emph{open terms}.  is the set of variables in
the open term t. 



In order to deal with languages that describe probabilistic behavior we need expressions denoting probability distributions.
Let  denote the set of all (discrete) probability
distributions on .
We let  range over
.
As usual, for  and , we define .
For , let  denote the Dirac
distribution, i.e.,  and   if .
Moreover, the product measure  is defined by
. In
particular, if ,  is
the distribution that assigns probability 1 to the empty tuple.
Let  and recall that 
.  Then
 is a well defined probability
distribution on closed terms.
In particular, if  and , 
then .


For a term  we let  be an
\emph{instantiable Dirac distribution}. That is,  is a
symbol that takes value  when variables in 
are substituted so that  becomes a closed term .
Let  be the set of
instantiable Dirac distributions.
A \emph{distribution variable} is a variable that takes values on
.  Let  be an infinite set of distribution
variables. Let  range over 
and  range over .
Let  be a set of distribution variables and  be a set of term variables.  
The set of \emph{distribution terms} over  and , notation  is the least set satisfying: 
\begin{inparaenum}[(i)]
\item , and
\item  where
     with ,
each  is a
function s.t.\ ,
  and
.
\end{inparaenum}
Intuitively,  decomposes term  into its sub-terms  and probability  of term  is calculated as the convex combination of the product probability of its sub-terms .
 is abbreviated as ; the elements
of  are actual distributions on terms.  is
abbreviated as .
 is the set of  
(distribution and term) variables appearing in .


A substitution is a mapping that assigns terms to variables.  In our
case we need to extend this notion to distribution terms and
instantiable Dirac distributions.
A \emph{substitution}  is a mapping in  such that  whenever , and  whenever .
A substitution  extends to open terms and sets of terms as usual,
to instantiable Dirac distributions by  
and to distribution terms by 
. Notice that the construction of distribution terms ensures that closed substitution 
instances of distribution terms denote indeed probability distribution.


\section{Probabilistic Transition System Specifications}\label{sec:ptss}


A (probabilistic) transition relation describes the behavior of a
process by prescribing the possible actions it can perform at each
state.
Each action is described with a label on the relation and the
evolution to the next state is given by a probability distribution on
terms.
We will follow the probabilistic automata style of~\cite{Segala95} which generalize the so
called reactive model~\cite{LarsenSkou91}.  Let  be a
signature and  be a set of labels.  A \emph{transition relation}
is a set , where .  We denote 
by .

Transition relations are usually defined by means of structured
operational semantics in Plotkin's style~\cite{Plotkin81}. 
We follow the approach
of~\cite{GrooteVaandrager92,Groote93,BolGroote96} which provides an
algebraic characterization for transition system specifications.


\begin{definition}\label{def:ptss}A \emph{probabilistic transition system specification} (PTSS) is a
  triple  where  is a signature,
   is a set of labels, and  is a set of rules of the form:

where
 are index sets, 
, , 
  , ,
, 
and 
\end{definition}


An expression of the form , (resp. , ) is a \emph{positive literal} (resp. \emph{negative literal, quantitative literal})
where , , ,  and .
For any rule , literals above the line are called
\emph{premises}, notation ; the literal below the line is
called \emph{conclusion}, notation .
We denote with  (, ) the set of
positive (negative, quantitative) literals of the rule .
A rule  is called \emph{positive} if .  A PTSS is called positive if
it has only positive rules. A rule  without premises is called an
\emph{axiom}.
In general, we allow the sets of positive, negative, and quantitative premises to be infinite.


Substitutions provide instances to the rules of a PTSS that, together
with some appropriate machinery, allows us to define probabilistic
transition relations.  Given a substitution , it extends to
literals as follows:
, \
, and
.
Then, the notion of substitution extends to rules as expected.  We say
that  is a (closed) instance of a rule  if there is a
(closed) substitution  so that .


 We say that  is a \emph{proper substitution of } if for all
 quantitative premises  of  it holds that
  for all .  Thus, if  is proper, all
 terms in  are in the support of .  Proper
 substitutions avoid the introduction of spurious terms.  This is of
 particular importance for the conservative extension theorem of
 \cite[Theorem~14]{DL-fossacs12}.
We use only this kind of substitution in the paper. 



As has already been argued many times (e.g.~\cite{Groote93,BolGroote96,vanGlabbeek04}), transition system
specifications with negative premises do not uniquely define a
transition relation and different reasonable techniques may lead to
incomparable models.
In any case, we expect that a transition relation associated to a PTSS

\begin{inparaenum}[(i)]
\item respects the rules of , that is, whenever the premises of a
  closed instance of a rule of  belong to the transition relation,
  so does its conclusion; and
\item it does not include more transitions than those explicitly
  justified, i.e., a transition is defined only if it is the conclusion of a
  closed rule whose premises are in the transition relation.
\end{inparaenum}
The first notion corresponds to that of model, and the second one to
that of supported transition.

Before formally defining these notions we introduce some notation.
Given a transition relation , a positive
literal  \emph{holds in} , notation
, if .
A negative literal   \emph{holds in} ,
notation ,
if there is no  s.t.\ .
A quantitative literal  \emph{holds in} ,
notation  precisely when .  Notice that the satisfaction of a quantitative literal
does not
depend on the transition relation. We nonetheless
use this last notation as it turns out to be convenient.
Given a set of literals , we write 
if .

\begin{definition}\label{def:supmodel}Let  be a PTSS. Let  be
  a probabilistic transition system (PTS). Then  is \emph{a
   supported model of}  if
it satisfies that:  iff there is a rule
   and a proper substitution 
  s.t.\  and .
For  to be a \emph{model} of  we only require that
   the ``if'' holds, and for  to be
   \emph{supported by}  we only require that the ``only if'' holds.
\end{definition}

We have already pointed out that PTSSs with negative premises do not
uniquely define a transition relation.  In fact, a PTSS may have more
than one supported model.  For instance, the PTSS with the single
constant , set of labels  and the two rules
 and
,
has two supported models:  and
.
We will not dwell on this problem which has been studied at length
in~\cite{BolGroote96} and~\cite{vanGlabbeek04} in a non-probabilistic
setting.
Instead we present two different approaches to resolve this problem:
stratification and well supported proofs. 



\subsection{Stratification}\label{sec:stratification}
A stratification defines an order on closed positive literals that
ensures that the validity of a transition does
not depend on the negation of the same transition.

\begin{definition}\label{def:stratification}Let  be a PTSS.  A function , where  is an ordinal, is called a
  \emph{stratification} of  (and  is said to be
  \emph{stratified}) if for every rule

and proper substitution

it holds that:
\begin{inparaenum}[(i)]
  \item for all , , and
  \item for all  and , .
  \end{inparaenum}
Each set , with
  , is called a \emph{stratum}.
If for all , ,
  then the stratification is said to be \emph{strict}.
\end{definition}

A transition relation is constructed stratum by stratum in an
increasing manner by transfinite recursion.  
If it has been decided whether a
transition in a stratum , with , is valid or
not, we already know the validity of the negative premise occurring in
the premises of a transition  in stratum  (since all
positive instances of the negative premises are in strictly lesser
strata) and hence we can determine the validity of .
Notice that a stratification does not take quantitative premises into account because 
their satisfaction does not depend on the transition relation.


\begin{definition}\label{def:assoc_with}Let  be a PTSS with a stratification
   for some ordinal .
For all rules , let  be the smallest regular cardinal
  such that , 
  and let  be the smallest
  regular cardinal such that  for all
  .
The transition relation  \emph{associated with} 
  (and based on ) is defined by
,
where each  and each  is defined
  by
-.5ex]
  & \qquad \textstyle (\bigcup_{ \gamma < \beta} {\trans_{P_\gamma}}) \cup (\bigcup_{ j' < j} {\trans_{P_{\beta,j'}}}) \models
       {\qprem{\rho(r)} \cup \pprem{\rho(r)}}  \text{ and } \
\end{definition}
\noindent
A PTSS  with rules 
can be stratified by  and .  This stratification induces the
transition relation . Because 
(non-strict) stratifications allow that positive premises are in the same
stratum as the conclusion, the validity of a premise may depend on a rule with a conclusion 
literal of the same stratum.
In this case, the construction of  requires
to iterate up to   times, denoted by ,
to decide the the validity of all literals of this stratum.

The  existence of a stratification guarantees the existence of a
supported model. In fact, such model is the one in
Def.~\ref{def:assoc_with} (Theorem~\ref{th:existence:supmodel}). Furthermore,
all stratification define the same supported model (Theorem~\ref{th:weakunicity:supmodel}) which allows to omit the stratification
symbol in  and use  instead. Moreover, 
strict stratification ensures uniqueness of the supported model 
(Theorem~\ref{th:strongunicity:supmodel}). The proofs follow closely
their non-probabilistic counterparts in~\cite{Groote93} (Theorem
2.15, Lemma 2.16 and Theorem 2.18, resp.).
The only actual difference lies on the quantitative premises, which do
not pose any particular problem since their validity depends only on
the substitution.


\begin{theorem}\label{th:existence:supmodel}Let  be a PTSS with stratification . Then  is a supported model of . 
\end{theorem}

\begin{theorem}\label{th:weakunicity:supmodel}Let  be a PTSS. For all stratifications ,  of  it holds .
\end{theorem}

\begin{theorem}\label{th:strongunicity:supmodel}Let  be a PTSS with a strict stratification . Then  is the only supported model of .
\end{theorem}


\subsection{Proof structures}\label{sec:proofStructure}


 In this section we introduce the notion of \emph{provable rules from a PTSS}.
To define this notion we use \emph{proof structures} \cite{FokkinkvanGlabbeek96}.
A proof structure is like a derivation tree where the rules do not share variable names. 
The connection between the conclusion of a rule  
 and a premise  in other rule is represented by a mapping  
 from rules to literals, i.e. .
A substitution \emph{matches} with a proof structure if both 
 the conclusion and the premise related by  are mapped to the same literal. 
Thus, matching substitutions translate a proof structure into an actual derivation tree.  As a consequence, a matching substitution applied to a proof structure defines a \emph{provable rule} in which the premises
 are the leaves of the derivation tree and the conclusion is the root.
The absence of shared variables allows to 
 define substitution on proof structures avoiding name clashes.
Provable rules will be used in the following way through the paper:  
 given a PTSS  we take the set of provable rules from  
 with a particular format, these rules will be used to define a 
 a new PTSS , then we show that  and  derive the same 
 PTS.

 A PTSS is \emph{small} if for each of its rules the cardinality of its collection of premises does not exceed the cardinality of the set of variables . Small PTSS ensure that there are enough variables to construct the proof structures.

\begin{definition}
 \label{def:proofStructure}
 A \emph{proof structure} is a tuple  such that 
\begin{itemize}
 \item  and  is a set of transition rules which do not have any variables in common,
 \item  is an injective mapping from  to the collection of positive premises in , 
          such that each chain  in , with  is a premise of ,
          is a finite chain. 
\end{itemize}
Let  be the set of all premises of rules in  that are outside the image of . Let   be the set of all quantitative premises in  .
\end{definition}

We introduce a partial well-order  on proof structures to allow inductive reasoning. Define the partial order  by
 iff 
  , 
   is  restricted to ,
  ,
and there is a chain  with 
  , ,  and  is a premise of .

A substitution  \emph{matches} with the proof structure  if  for every .

\begin{definition}\label{def:provable}
 Let  a set of literals s.t. , (resp.  and ) is a set of positive (resp. negative and open quantitative) literals.
A rule  is provable from a small PTSS , 
 notation , if  or there is a proof structure  such that
 each rule in  is in  modulo -conversion and there is a substitution  that matches
 with   such that:
\begin{itemize}
 \item , 
 \item if  is a closed quantitative premise then  holds,
          otherwise  and 
 \item .          
\end{itemize}
\end{definition}

 Note that closed quantitative literals do not need to be included in the premise of a provable rule because their validity can be decided without further instantiation. Notice additionally that all negative literals of premises of rules in  are included in  and thus no negative literals can be derived.


  \begin{figure}
\begin{minipage}{0.59 \linewidth}
 \centering\small
\begin{tikzpicture}[node distance=1.8cm]
\node (r1) at (0.7,0){}; 
\node (r2) at (0,-1.5){}; 
\node (r3) at (4.8, -1.5){};
\node (r4) at (2.2, -3.5){};
 \node (r5) at (2, -5.3){};
\draw [->] (0.55,-0.3) -- (0.55,-0.8); 
\draw [->] (0.1,-2.1) -- (1.1,-2.9); 
\draw [->] (4.8,-2.1) -- (3.3,-3); 
\draw [->] (2,-4.1) -- (2,-4.7); 
\end{tikzpicture}
\end{minipage}
\hfill \begin{minipage}{0.33\linewidth}
 
\end{minipage}
\caption{An example of proof structure. (See Example~\ref{ex:proofStructure})}
\label{fig:proofStructure}
\end{figure}

\begin{example} \label{ex:proofStructure}
 Let  be a PTSS with
 ,
  and all rules in
 Fig.~\ref{fig:proofStructure} appear in .
Let  the proof structure of
 Figure~\ref{fig:proofStructure} where mapping  is represented
 by the arrows.
Let  be the substitution defined in
 Fig.~\ref{fig:proofStructure}, with  for any
 other (term or distribution) variable not specified in the figure.
Then the following rule is provable from :

Both in Fig.~\ref{fig:proofStructure} and in the above rule we used
 shorthand notations for the different distribution terms.  We write
  and  instead of
  and , with ,
 respectively (trivial summations are omitted).

 Since  for all  and , then  is closed, and moreover, it holds.  As a consequence, it does not
 appear as a premise of rule~(\ref{eq:ex-provable-rule}).
Also notice that  was substituted by .  This is why we needed to upgrade the format
 of~\cite{DL-fossacs12} to consider the more complex distribution
 terms on the quantitative premises instead of only distribution
 variables.
\end{example}


The set of all provable rules from a PTSS can be alternatively defined
in a recursive manner without using the notion of proof structure
(Def.~\ref{def:provClosure}). We prove that both definitions are
equivalent in Lemma~\ref{lemma:closure}.



\begin{definition}\label{def:provClosure}
 The \emph{provable closure} of a PTSS  
  is the smallest set  of rules such that
\begin{itemize}
 \item if  then ,
 \item if  and there is a substitution  such that 
 \begin{itemize}
    \item for all  it holds   and
    \item for all  if  is not a closed literal then , otherwise  holds
 \end{itemize}   
  then  .
\end{itemize}
\end{definition}

\begin{lemma}\label{lemma:closure}
 A rule  is provable from a small PTSS  iff
 .
\end{lemma}

The following lemma is an immediate consequence of Def.~\ref{def:provClosure}.
\begin{lemma}\label{lemma:provability}
 Let  and  be two PTSS such that all rules in  are provable from . Then all rules provable from  are also provable from .
\end{lemma}


\subsection{Well-supported proofs}


In the following we adapt the notion of \emph{well-supported proof}~\cite{vanGlabbeek04} to PTSS.
In the following, we say that literals  and  \emph{deny each other}.

\begin{definition}
 \label{def:ws-proof}
 A \emph{well-supported proof} of a closed literal  from a PTSS
  is a well-founded, upwardly branching tree of which the nodes
 are labeled by positive or negative literals, such that
\begin{itemize}
 \item the root is labeled by , and 
 \item if  is the label of the node  and  is the set of labels of 
  the nodes directly above , then:
\begin{itemize}
 \item if  is a positive literal then there is a rule  and a closed proper substitution  such that 
          , the quantitative premises  are valid and  
          ,
\item if  is a negative premise then for all  with  a closed literal denying , a 
       literal in  denies a literal in .
\end{itemize}
\end{itemize}
A literal  is \emph{ws-provable}, notation , if there is a well-supported proof of
 from . A literal  is \emph{ws-refutable} if there is a literal  ws-provable from  and  denies .
\end{definition}

Notice that nodes in the proof tree of Def.~\ref{def:ws-proof} are not
quantitative literals.  This is due to the fact that the validity of
closed quantitative literals is already known.  In fact, the
definition requires that all quantitative literal introduced by a rule
 should become valid after substitution.



We say that a PTSS  is \emph{complete} if for all closed literal ,  for some distribution  or .
In addition,  is \emph{consistent} if there are no pair of literals
derived from  that deny each other.
We will focus only on complete PTSSs. 
\remarkDG{Main theorem does not require completeness, so for which part and why is completeness required?}
\remarkPRD{You are right and I thought of it, but since we are defining in this paper the concept of a model based on well-supported proof maybe it is ok to keep it.  Anyway I am not really aiming to push this.  If space is needed, this would be a candidate to remove}
The transition relation based on well-supported proofs associated to 
a (complete) PTSS  (denoted by ) is the set of ws-provable transitions
of .

\begin{lemma}\label{lemma:completeThenCons}
 Let  be a PTSS. If  is complete then it is also consistent.
\end{lemma}

 Lemma~\ref{lemma:completeThenCons} allows us to show that, for any stratifiable PTSS, the model obtained using well-supported proofs coincides with the model obtained through stratification. 
Notice that this does not imply that the methods are equivalent: it could be the case
 that a PTSS is complete but not stratifiable (see \cite[Prop. 27]{vanGlabbeek04}).


\begin{lemma}\label{lemma:WSPsubsumesStra}
 Let  be a PTSS with stratification  and  a positive or negative literal, 
 then   iff .
\end{lemma}

The proof of this lemma follows the same structure of its non-probabilistic counterpart (see \cite[Prop. 25]{vanGlabbeek04}).


The next lemma states that it suffices to show that the same rules
having only negative premises are provable in two different PTSSs to
state that these PTSSs define the same set of ws-provable transitions.


\begin{lemma}\label{lemma:sameNegRules}
 Let  and  be two PTSSs over the same signature such that
  iff  for all closed rule  with  containing only negative premises.  Then  
  iff  for all closed literal . 
\end{lemma}
  

\section{The \bntmufxt\  format}


In this section we revise the \ntmufxnu\ format of~\cite{DL-fossacs12}
adapting it to the richer quantitative premises introduced before.
Furthermore we correct some mistakes of~\cite{DL-fossacs12}.

Before, we recall the notion of bisimulation on PTSs~\cite{LarsenSkou91}. 
Given a relation ,
a set  is  if for all  and ,  implies 
(i.e. ).
If a set  is  we write .
It is easy to verify that if two relation  are such that
, then for all set ,  implies .

\begin{definition}
  A relation  is a
  bisimulation if  is symmetric and for all , , ,
\begin{quote}
     and  imply that there exists
     s.t.\  and
    ,
  \end{quote}
where  if and only if .
We define bisimilarity  as the smallest relation that
  includes all other bisimulations.
It is well-known that  is itself a bisimulation and an
  equivalence relation.
\end{definition}

Let  be a family of sets of term variables with the same cardinality.
The -th element of a tuple   is denoted by .  For a set of
tuples  we denote the -th projection by 
.
Fix a set  such that:
\begin{enumerate}[(i)]
\item for all , ; and
\item for all , .
\end{enumerate}
Property (ii) ensures that different  differ in all positions and by property (i) every variable of every  is used in one .
 stands for ``diagonal'', following the intuition that each
 represents a coordinate in the space ,
then  can be seen as the line that traverses
the main diagonal of the space.
Notice that, letting  be a natural number,  for  a possible
definition for  is .

\begin{definition} \label{def:ntmufxnu}
  Let  be a PTSS.
A rule  is in \emph{\ntmuft\ format} 
  if it has the following form 
with 
  for all  and , and it satisfies the following conditions:
\begin{enumerate}
  \item\label{item:conditions_on_cardinality}Each set  should be at least countably infinite, for all
    , and the cardinality of  should be strictly smaller
    than that of the 's.
  \item\label{item:condition_on_Z}, with 
    .
\item\label{item:condition_mus_are_different}All variables , with  and
    , are different.
\remarkPRD{Notice that there is one item less thanks to the removal of the 's in the 's and a simplification of items~\ref{item:conditions_on_nonrepeating_z}, \ref{item:conditions_on_variables_of_theta}, and \ref{item:conditions_on_distribution_terms}.}
\item\label{item:conditions_on_nonrepeating_z}For all , , if 
    then .
  \item\label{item:conditions_on_Y}For all ,
    , and  for all , .
\item\label{item:conditions_on_z}All variables  are different.
\item\label{item:conditions_on_variables_of_theta}For all ,
    .
\item\label{item:conditions_on_terms} and for all  and , .  In all cases, if  and ,  is the same
    term as  where each occurrence of variable  (if it appears
    in ) has been replaced by variable , for .
\item\label{item:conditions_on_distribution_terms} for all .
\end{enumerate}
A rule  is in \emph{\ntmuxt{} format} if its form is like above
  but has a conclusion of the form

  and, in addition,
  it satisfies the same conditions as above only that whenever we write , we should write .
A rule  is in \emph{\nxmuft{} format} if it is in 
  \ntmuft{} format and the sources of its positive premises are term variables.
 is in \ntmuft{} (resp.\ \emph{\ntmuxt{}},\ \emph{\nxmuft{}}) \emph{format}
  if all its rules are in \ntmuft{} (resp.\ \ntmuxt{},\ \nxmuft{}) format.
 is in \emph{\ntmufxt{} format} if each of its rules is either in
  \ntmuft{} format or \ntmuxt{} format.
\end{definition}
The rationale behind each of the restrictions are discussed
in~\cite{DL-fossacs12} in depth.
In the following we briefly summarize it.
Term variables  appearing in the source of the
conclusion are binding.  Variables in  and those
appearing in instantiable Dirac distributions are also binding when
appearing in quantitative premises.  Therefore they need to be all
different.  This is stated in conditions
\ref{item:condition_mus_are_different}, \ref{item:conditions_on_Y},
and \ref{item:conditions_on_variables_of_theta}.
Distribution variables in 

are also binding when appearing on the target of a positive
premise. Hence they also need to be different, which is stated in
condition \ref{item:conditions_on_z}.
If  is finite, quantitative premises will allow to count the
minimum number of terms that gather certain probabilities.  This goes
against the spirit of bisimulation that measures equivalence classes of
terms regardless of the size of them.  Therefore  needs to be
infinite (condition~\ref{item:conditions_on_cardinality}).
Condition~\ref{item:conditions_on_nonrepeating_z} is more subtle;
together with each set of premises

it ensures a symmetric behaviour of terms  for every
possible instantiation of variables .  A clear example that
shows the need for this symmetry is provided in~\cite{DL-fossacs12}.
The need for the source of the conclusion and targets of positive
premises to have a particular shape is the same as in the
\emph{tyft/tyxt} format~\cite{GrooteVaandrager92}.
Conditions~\ref{item:condition_on_Z},
\ref{item:conditions_on_terms},
and~\ref{item:conditions_on_distribution_terms} are actually notations
and definitions.





The definition provided here corrects some mistakes inadvertently
introduced in the \ntmufxnu\ format in~\cite{DL-fossacs12}, more
precisely on the quantitative premises and condition~4 in Def.~11
(which corresponds to our
condition~\ref{item:conditions_on_nonrepeating_z}).
Another mistake in~\cite{DL-fossacs12} was omitting to require that
PTSS are well-founded as hypothesis for the congruence theorem.  This
is corrected in the following, where we extend the congruence theorem
to the \ntmufxt\ format.
\footnote{Both issues are explained in detail in the corrigendum of \cite{DL-fossacs12}:
http://cs.famaf.unc.edu.ar/~lee/publications/corrigendum-Fossacs2012.pdf}



\begin{definition}
 Let  be a set of positive and quantitative premises.
 The \emph{dependency directed graph} of  is given by  with
  and
.
We say that  is \emph{well-founded} if any backward chain of edges in   is finite.
Define for each , ,
 where .   
A rule is called \emph{well-founded} if its set of positive and quantitative premises is well-founded. 
 A PTSS is called \emph{well-founded} if all its rules are well-founded. 
\end{definition}



\begin{theorem}\label{th:congruence}
 Let  be a well-founded stratifiable PTSS in \ format. Then  is a congruence relation for all operators defined in .
\end{theorem}




\section{\bntmufxt\ format reduces to pntree}\label{sec:reduction_pntree}


The reduction procedure requires results from unification theory over infinite domains.  
Instead using the result presented in \cite{Fokkink1997183}, we use the 
variation presented in \cite[Lemma 3.2]{FokkinkvanGlabbeek96} that proves some extra
properties needed to prove our main result.


\begin{definition}
 A substitution  is a \emph{unifier} for a substitution  if .
 In this case, we say that  is \emph{unifiable}.
\end{definition}

\begin{lemma}\label{lemma:unification}
 If a substitution  is unifiable, then there is a unifier  for   such that:
\begin{inparaenum}[(i)]
  \item each unifier  for  is also a unifier for 
  \item if  then , for all , and
  \item if  is a variable for all  then  is a variable.
 \end{inparaenum}
We call  the most general unifier.
\end{lemma}

The main theorem~\ref{th:ntumufxtPNTree} showing that every PTSS in \ntmufxt-format 
can be reduced to a transition equivalent PTSS in pntree format is 
developed incrementally. 
First of all, we show that every \ntmuxt-rule can be expressed by a set of \ntmuft-rules
by replacing the source variable of the conclusion with an appropriate context 
 (Lemma~\ref{lemma:ntmuft}).
Secondly, we show that for all PTSS  in \ntmuft\ format 
there is a PTSS  in \nxmuft\ format such that 
 iff 
for all rules  in \nxmuft\ format (Lemma~\ref{lemma:nxmuft}).
Notice that this result implies that 
 iff 
for all rule  with  a set of closed negative premises, then
by Lemma~\ref{lemma:sameNegRules},  and  are equivalent. 
Finally, we prove that for all PTSS  in \nxmuft\ format 
there is a PTSS  in  pntree format
(a PTSS in well-founded \nxmuft\ format without free variables),
such that for every closed transition rule   
with only negative premises,
 iff  (Lemma~\ref{lemma:pntree}).
Again, by Lemma~\ref{lemma:sameNegRules},  and  are
equivalent. 
This series of lemmas leads to the main theorem stating that every PTSS consisting of rules in the \ntmufxt\ format can be reduced to a transition equivalent PTSS in the more restrictive pntree format. Furthermore, this shows also that the rules of a PTSS in \ntmufxt\ format do not have to be well-founded in order to guarantee that the bisimilarity of the induced PTS is a congruence.

The reduction of proof structures follows the logic of \cite{FokkinkvanGlabbeek96}. In the probabilistic setting we need to treat additionally quantitative premises as follows: While substitutions replace distribution variables by distribution terms the substitution  leads to a well-defined quantitative literal ( is defined as  for all ). Because by construction  unifies  we have that whenever  then also . This shows the satisfaction of the quantitative premises.

\begin{lemma}\label{lemma:ntmuft}
  Let  be a stratifiable PTSS in \ntmufxt\ format.
  Then there is a stratifiable PTSS  in
  \ntmuft\ format that is transition equivalent to .
\end{lemma}

\begin{lemma}\label{lemma:nxmuft}
  Let  be a PTSS in \ntmuft\ format.
  Then there is a PTSS  in
  \nxmuft\ format such that 
   iff 
  for all rule  in \nxmutt\ format.
(A rule is in \nxmutt\ format if the source of every positive
  premise is a term variable and its target is a distribution
  variable.)
\end{lemma}

\begin{proof}
 Define  such that  iff 
  is a provable rule from  in \nxmuft\ format.
The right to left implication follows straightforward from Lemma~\ref{lemma:provability}.

 For the left to right implication we proceed by induction on the partial order over proof structures.
 Suppose , with a rule  in \nxmutt\ format, 
 and let  be a proof structure for   over .
Then by Def.~\ref{def:provable} there is substitution  s.t.
 \begin{inparaenum}[(a)]
 \item ,
 \item closed quantitative premise in  hold,
 \item open quantitative premise in  belong to ,
 and
 \item .  
 \end{inparaenum}

 From  we construct recursively a substructure  
 which is a proof structure for a rule , i.e.  is in \nxmuft\  format, 
 such that   and 
 for each premise  of  the rule  is provable from 
i.e.   or  is a valid closed quantitative literal.
 Then, by Lemma~\ref{lemma:closure},  is provable from .
Furthermore, we construct a partial substitution  which is unified by , 
  i.e. if  is defined then .
  In this construction  is defined as the identity function.
We proceed with the definitions of the transition rules  and the substitution : 
\begin{enumerate}[(i)]
  \item . 
  \item \label{cond:B&rho:ii}
  If  and  is a premise 
  of a rule in  s.t there is  with:
 \begin{enumerate}
  \item  is defined for  
  \item  are variables for  \item  has the form  with 
 \end{enumerate}
 then . 
 Notice that the conditions can be satisfied only if  
  is a variable for . 
 Moreover  is a variable. 
 In addition, this variable belongs to  .

\item \label{cond:B&rho:iii}Since  matches with , . 
 Because the rule format restricts the form of the conclusion , 
 then we can rewrite the last equality by:
  
In addition,  unifies the partial substitution ,
 then if  is a variable it holds:
 

 Because  has the form  
 it holds  for  and 
 .
Define  for 
 (here we define the left side of a conclusion of a rule in ).   
Besides, define .
Notice that this extension of  is unified by  and, 
 by Def.~\ref{def:provable}, the variables  and  
 appear only in this rule, then we are not redefining substitution .
\item Define  for all variable 
 if  is not defined for . 
 Substitution  unifies this extension of .
\item Finally,  is the restriction of  to .  
(Notice that the substitution  is defined for the the right side 
of a positive premise in the image of  in item (\ref{cond:B&rho:iii}).)


\end{enumerate}

Substitution  unifies substitution , by Lemma~\ref{lemma:unification},
there is a substitution  which unifies  and:
\begin{enumerate}[(i)]
 \item \label{unification:i} 
 .
 \item \label{unification:ii}
 If  then , with   a term or distribution variable.
\item \label{unification:iii} 
 If  is a variable for  then  is a variable. 
\end{enumerate}
 
The proof structure   and the substitution  are completely defined,
now we can prove that  matches with . 
Let  a rule used to construct  and consider the substitution .
Recall that the conclusion of  has the form
 and
 is such that 
 
by (\ref{cond:B&rho:ii}) and the definition of  for  in 
(\ref{cond:B&rho:iii}). 
Since  unifies  then 

Then the substitution  matches with the proof structure .

To show that the rule 

is provable (Def.~\ref{def:provable}), it remains to show that if a quantitative premise 
in  is closed then it is also valid.
Let  be a quantitative premise. 
 Then if   is closed, since  unifies , it holds that
,
which implies that also  is a closed literal. 
 Because the rule 
 
 is provable we have that  holds and therefore also  holds.


Finally we prove that the rule  is in \nxmuft\ format. 
From the construction by  we know that if  is
s.t.  then  satisfies one of the following conditions:

\begin{enumerate}
 \item  appears in the left-hand side of a conclusion of a rule in ,
 \item  appears in the right-hand side of a positive premise in the image of . 
\end{enumerate}

 Then if  is the conclusion of , 
  for  and, 
 hence  because of (\ref{unification:ii}).
On the other hand, if  is a variable that appears in the right-hand 
 side of a positive premise in the image of , i.e.  is a distribution variable, 
 we have  and then 
 . 
 Therefore the conclusion  of  
 has the form  as the \nxmuft\ format demands. 


 We continue with the premises of .   
 Let  be a positive premise in  
 then  is a positive premise of a rule in  which does not belong
 to the image of . Then
  is such that  and this implies .
To prove that  is a variable there are 2 cases to investigate:

\begin{itemize}
 \item . 
Then  and because  is in  \nxmutt\ format, 
 then  is a variable.
 Therefore  and then  is a variable.
\item .
Then there is a rule  s.t. .
  Since   does not belong to the image of  we have that 
  . By  and the construction of 
  we have that  is a variable for all . 
  Then (\ref{unification:iii}) ensures that  is a variable.  
\end{itemize}

This shows that the positive premises also fulfill the requirements of the \nxmuft\ format.

We proceed with the quantitative premises. 
Let  with .
By the same reasoning as applied for the target of the conclusion we get . 
In addition,  for all  because they do not appear in the left-hand side of a conclusion, and hence .
Thus,  has the proper form.

Syntactical restriction for positive and quantitative premises  and conclusion are satisfied.
Besides, there is no restriction for negative premises, therefore  is in \nxmuft\ format 
and then .


 For all positive premises  the rule 
 is in  and it is provable in  by a proof sub-structure smaller 
 than . Thus, by induction we get that these rules are provable in .
 Applying Lemma~\ref{lemma:closure} on these rules and  shows that  is provable in .
\end{proof}


\begin{definition}
We say that a variable  occurs \emph{free} in a rule  if it
occurs in  but not in the source of the conclusion nor in 
 with . 
\remarkDG{Does a instantiable dirac distribution  as part of  not also bind a variable and this should be included in the def?}
We say that a distribution variable  occurs \emph{free} in a rule  if it occurs in  
but not in the target of a positive premise. \end{definition}

\begin{definition}
 A PTSS  is in \emph{pntree format} if
 all rules in  are well-founded \nxmuft\ rules without free variables. 
\end{definition}

\begin{lemma}\label{lemma:pntree}
  Let  be a PTSS in \nxmuft\ format.
  Then there is a PTSS  in
  pntree format such that for every closed transition rule 
  with only negative premises,
   iff 
\end{lemma}
\begin{proof}
 Let  such that  is the set of provable rules 
 from  in pntree format. 
 By Lemma~\ref{lemma:provability}, the right to left implication holds. 

 For the left to right implication we proceed by induction.
Let  be closed with  containing negative literals
 only. Let  be provable from ,
 i.e.\ .  Then either ,  is a
 valid closed quantitative literal, or there is a rule  and a
 substitution  such that  and, for all
 premises , .
 Then  either trivially or by
 induction.

 Because  is \nxmuft\ format,  has the form

where each  is a variable in .


 Let  be the variable dependency graph associated to
 .
From , we construct a rule  as follows.
Let  be the target of a positive premise such that
 there is no backward path in  from a vertex  to
 some vertex , with .  Notice
 that, by the symmetry requirements in Def.~\ref{def:ntmufxnu}, this
 happens for all  with .
We first obtain a rule  by
 \begin{inparaenum}[(i)]
 \item replacing variables  and  by
    and , respectively,
   and
 \item replacing every free variable  in  and  by
   .
 \end{inparaenum}
The resulting rule  does not have free variables and it is a
 substitution instance of , so  is provable from .
To obtain , replace each closed positive premise
  by .  Since,
  is a positive premise of ,
 .
 Then  is also provable from .

 Notice that the resulting rule  is in \nxmuft\ format without
 free variables.  Morever,  is well-founded since any dependency
 backward chain ends in a vertex .  Hence  is a pntree rule
 and therefore .

 Let . Then either  (and hence  is closed)
 or . In any case, 
 (if , it follows by induction). Therefore
 .  Since
 , .
\end{proof}









\begin{theorem}\label{th:ntumufxtPNTree}
 Let  be a PTSS in \ntmufxt\ format. There is a 
 PTSS  in pntree format that is transition equivalent to .
\end{theorem}

The proof of Theorem~\ref{th:ntumufxtPNTree} follows by applying
Lemmas~\ref{lemma:ntmuft}, \ref{lemma:nxmuft}, \ref{lemma:pntree},
and~\ref{lemma:sameNegRules}, in that order.

Let  be a stratifiable PTSS in \ntmufxt\ format and let  be its
stratification.
If  is a provable rule from , conditions (i) and (ii) in
Def.~\ref{def:stratification} also hold for stratification  in rule
.  (This can be shown by induction.)
Then,  is also a stratification for the PTSS  in pntree format
obtained as in Theorem~\ref{th:ntumufxtPNTree}.
Since pntree rules are well-founded \ntmuft\ rules, from
Theorems~\ref{th:congruence} and~\ref{th:ntumufxtPNTree}, we have the
following corollary.

\begin{corollary}  
 If  is a stratifiable PTSS in \ format,  is a congruence for all operators in .
\end{corollary}
\remarkDG{Even if it is a straight forward conclusion from the earlier propositions should we really just use corollary for this main theorem/main result of the paper (beside the technical machinery developed)? Maybe some of the earlier theorems should become only propositions?}

To conclude the section, we remark that negative premises cannot be
reduced to variables.
Following the nomenclature of~\cite{FokkinkvanGlabbeek96}, we say that
a rule is in \emph{simple pntree format} if it is in pntree format and
all its negative premises have the form .
It turns out that the pntree format (and hence also the
\ntmufxt\ format) is strictly more expressive than simple pntree
format.
We will not dwell on this since example and rationale of the
difference of expressiveness in the non-probabilistic case applies
mutatis mutandi to our case (see~\cite{FokkinkvanGlabbeek96}).




\section{Concluding remarks}


We introduced the rule format \ntmufxt\ which enriches
\ntmufxnu~\cite{DL-fossacs12} by allowing distribution terms to appear
in quantitative premises and conclusions of rules.
We showed that it ensures that bisimulation equivalence is a
congruence for operators of well-founded PTSSs.  On proving this, we
corrected a mistake introduced in~\cite{DL-fossacs12}.
The richer syntactic structure of the quantitative premises and the
conclusion of the rules allows us to define a reduction of
\ntmufxt\ PTSSs to a transition equivalent PTSS consisting of only
pntree rules.
This construction confirms that the well-foundedness requirement in
\ntmufxt\ is not necessary to guarantee that bisimilarity is a
congruence.



We already know that the \ntmufxt\ format is equally expressive if
restricted to quantitative premises of the form  with
.  
\remarkDG{Do you want to say that without loss of expressivity one can restrict  to only  and express  by  ?}\remarkPRD{Not only that, but also that it suffices that  is a rational number (instead of real). BTW, I corrected  by .}\remarkDG{Not required for the paper but just for my understanding: What is the reason that  suffices? Is this following Dedikind-cut argumentation?}
\remarkPRD{Yes. For the case of , we need to introduce an (denumerably) infinite number of rules. (This remark is also for me to remember how is the encoding.)}
However, we do not know whether distribution
terms are really needed.  We actually suspect that they are, and
hence, that the \ntmufxt\ format is strictly more expressive than the
\ntmufxnu\ format. 


Pntree rules are nearly ruloids \cite{BloomIM95:jacm} except that
negative premises may still contain non-variable terms. The decomposition 
method of \cite{Bloom:2004:PFD:963927.963929,Gebler:2012:phml_lt_sos} to develop
modular compositional proof systems can be adapted to pntree
rules by applying the negation-as-failure semantics for the logical characterization of
negative premises of pntree rules.
This will allow us to derive expressive congruence formats for probabilistic behavioral equivalences from their logical characterization in a structured way, following the approach of \cite{Bloom:2004:PFD:963927.963929}.


Both~\cite{DL-fossacs12} and this work have opened a new way of
thinking about probabilistic transition system specifications.  One of the
nicest things is that the \ntmufxt\ follows quite closely the
structure of non-probabilistic formats (particularly, \ntyfxt).
Hence, many ideas for further work can be borrowed from the
non-probabilistic setting.






\bibliographystyle{eptcs}
\bibliography{pntree}

\end{document}
