\documentclass[runningheads]{llncs}
\usepackage{amsmath}
\usepackage{amsfonts}
\usepackage{amssymb}
\usepackage{graphicx}

\widowpenalty=10000
\clubpenalty =10000
\interlinepenalty=10

\newcommand{\eps}{\varepsilon}
\renewcommand{\epsilon}{\varepsilon}

\pagestyle{plain}

\begin{document}
\title{The state complexity of star-complement-star}
\titlerunning{The state complexity of star-complement-star}
\author{
Galina Jir\'{a}skov\'{a}\,\inst{1,}\thanks{Research supported by  VEGA grant  2/0183/11.}
  \and 
   Jeffrey Shallit\,\inst{2}}
\institute{
Mathematical Institute, Slovak Academy of Sciences\\
Gre{\v s}{\' a}kova 6, 040 01 Ko\v{s}ice, Slovakia\\
\email{jiraskov@saske.sk}
  \and 
School of Computer Science, University of Waterloo \\
Waterloo, ON  N2L 3G1 Canada \\ 
\email{shallit@cs.uwaterloo.ca}
}

\maketitle

\begin{abstract}
\label{***abstract}
We resolve an open question by determining matching (asymptotic) 
upper and lower bounds on the state complexity of the
operation that sends a language  to 
.
\end{abstract}
  
\section{Introduction}
\label{***intro}

Let  be a finite nonempty alphabet, let  be
a language, let  denote the complement of
, and let  (resp., ) denote the Kleene closure
(resp., positive closure) of the language .
If  is a regular language, its {\it state complexity\/}
is defined to be the number of states in the minimal deterministic
finite automaton accepting  \cite{yzs94}.
In this paper we resolve an open question by determining matching
(asymptotic) upper and
lower bounds on the deterministic state complexity of the operations


To simplify the exposition, we will write everything using an exponent notation,
using  to represent complement,
as follows:

and similarly for  and .

Note that


It follows that the state complexity of  and
 differ by at most .  In what follows, we will work
only with .

\section{Upper Bound}
\label{***upper}

Consider a deterministic finite automaton
(DFA) 
accepting a language , where .
As an example, consider the three-state DFA over 
shown in Fig.~\ref{fig:d_n1} (left).
To get a nondeterministic finite automaton
(NFA)  for the language  from the DFA ,
we add an -transition 
from every non-initial final state to the state .
In our example, we add an -transition from state  to state ;
see Fig.~\ref{fig:d_n1} (right).
After applying the subset construction to the NFA 
we get a DFA  for the language .
The state set of  consists 
of subsets of 
see Fig.~\ref{fig:d1d2} (left). Here the sets in the labels of states
are written without commas and brackets;
thus, for example  stands for the set .
Next, we interchange the roles of the
final and non-final states of the DFA ,
and get a DFA  for the language ; see Fig.~\ref{fig:d1d2} (right).



To get an NFA  for  from the DFA ,
we add an -transition from each non-initial final state of 
to the state , see Fig.~\ref{fig:n2d3d3min} (top).
Applying the subset construction to the NFA 
results in a DFA  for the language 
with its state set
consisting of some sets of subsets of ;
see Fig.~\ref{fig:n2d3d3min} (middle). Here, for example, the label 
corresponds to the set .
This gives an  upper bound of 
on the state complexity of the operation plus-complement-plus.

Our first result shows that 
in the minimal DFA for 
we do not have
any state ,
in which a set  is a subset of some other set ;
see Fig.~\ref{fig:n2d3d3min} (bottom).
This reduces the upper bound 
to the  number of antichains of subsets
of an -element set known as the Dedekind number  with
\cite{KM75}


\vskip-10pt
\begin{figure}\label{-----fig1}
\centering
\includegraphics[scale=0.35]{automat7.eps}
\caption{DFA  for a language  and NFA  for the language .}
\label{fig:d_n1}
\end{figure}

\vskip-20pt
\begin{figure}[h!]\label{-----fig2}
\centering
\includegraphics[scale=0.35]{automat6.eps}
\caption{DFA  for  language  and DFA  for the language .}
\label{fig:d1d2}
\end{figure}


\begin{figure}[h!]\label{-----fig3}
\centering
\includegraphics[scale=0.40]{automat5.eps}
\caption{NFA , DFA , and the minimal DFA 
          for the language .}
\label{fig:n2d3d3min}
\end{figure}

\begin{lemma}\label{-----le1}
 If  and  are subsets of 
 such that ,
 then the states  and  of
 the DFA  for the language 
 are equivalent.
\end{lemma}

\begin{proof}
 Let  and  be subsets of 
 such that .
 We only need to show that
 if a string  is accepted by the NFA  starting from the state ,
 then it also is accepted by  from the state .

 Assume   is accepted  by  from .
 Then in the NFA , an accepting computation on  from state 
 looks like this:
 
 where , and
 state  goes to an accepting state  on  
 without using any -transitions,
 then  goes to  on ,
 and then  goes to an accepting state 
 on ; it also may happen that , in which case
 the computation ends in .
 Let us show that  goes to an accepting state of the NFA  on .

 Since  goes to an accepting state  on  in the NFA 
 without using any -transition,
 state  goes to the accepting state  in the DFA ,
 and therefore to the rejecting state  of the DFA .
 Thus, every state  in  goes to  rejecting states in the NFA .
 Since , every state in  goes to
 rejecting states in the NFA ,
 and therefore  goes to a rejecting state  in the DFA ,
 thus to the accepting state  in the DFA  .
 Hence  is accepted from  in the NFA 
 by computation
 
\qed
\end{proof}


Hence whenever a state  of the DFA 
contains two subsets  and  with  and ,
then it is equivalet to state .
Using this property, we get the following result.

\begin{lemma}\label{-----le2}\label{le:express}
 Let  be a DFA for a language  with state set ,
 and  be the minimal DFA for  as described above.
 Then every  state of 
 can be expressed in the form

where
 \begin{itemize}
 \item  ;
\item there exist subsets 
; and
\item there exist , pairwise distinct states of 
            not in ; such that
\item   for .
\end{itemize}
\end{lemma}

\begin{proof}
 Let . 

 For a state  in  and a symbol  in , 
 let  denote the state in , to~which  goes on ,
 that is, .
 For a subset  of  let  denote
 the set of states to which states in  go by ,
 that is,
 
 Consider transitions on a symbol 
 in automata ; 
 Fig.~\ref{fig:transitions} illustrates these transitions.
 In the NFA ,
 each state  goes to a state in  if  is a final state of ,
 and to state   if  is non-final.
 It follows that in the DFA  for ,
 each state  (a subset of )
 goes on  to  final state  if  contains a final state of ,
 and to non-final state  if all states in  are non-final in .
 Hence in the DFA  for ,
 each state  goes on  
 to non-final state  if  contains a final state of ,
 and to the final state  if all states in  are non-final in .

 Therefore, in the NFA  for ,
 each state  goes on  to a state in
   if all states in  are non-final in ,
 and to state   if  contains a final state of .

 To prove the lemma for each state,
 we use induction on the length of the shortest
 path from the initial state to the state of  in question.
 The base case is a path of length .  In this case,
 the initial state is , 
 which is in the required form (\ref{eq1}) with
  and .

 \begin{figure}[h]\label{-----fig4}
 \centering
 \includegraphics[scale=0.35]{transitions_a.eps}
 \caption{Transitions under symbol  in automata .}
 \label{fig:transitions}
 \end{figure}


 For the induction step, let
  
 where , and

   ,

    are pairwise distinct states of 
            that are not in  and

    for .

\bigskip
 We now prove the result for all states reachable from 
 on a symbol .

 First, consider the case that each  goes on 
 to a non-final state  in the NFA~.
 It follows that  goes on 
 to ,
 where
 
 Write  and .
 Then we have .

 If  for some  with ,
 then ,
 and therefore  can be removed from state  
 in the minimal DFA .
 After  several such removals, we arrive at an equivalent state
 
 where ,
 
 and the states 
 are pairwise distinct.

 If  for some  with ,
 then ; thus  can be removed.
 After all such removals, we get an equivalent set
  
 where ,
 
 and the states 
 are pairwise distinct and  are not in .
 If , then the state   
 is in the required form (\ref{eq1}).
 Otherwise, if  is a proper subset of ,
 then there is a state  in ,
 and then we can take :
 since  are not in ,
 they are distinct from ,
 and moreover .

 If , then ,
 and therefore  can be removed from .
 After all these removals we
 either reach some  that is a proper subset of ,
 and then pick a state  in  in the same way as above,
 or we only get a single set ,
 which is in the required form .

 This proves that if  each  in  goes on 
 to a non-final state  in the NFA~,
 then  goes on  in the DFA  
 to a set that is in the required form (\ref{eq1}).

 \bigskip
 Now consider the case that 
 at least one   in 
 goes to a final state  in the NFA 
 It follows that 
  goes to a final state
 
 where  and if ,
 then  or 
 We now can remove all  that contain state ,
 and arrive at an equivalent state 
 
 where , and
 , and
 ,
 and each  is distinct from .

 Now in the same way as above we arrive at an equivalent state 
 
 where ,
 all the  are pairwise distinct and different from ,
 and moreover, the states  are not in .
 If  is not in , then we are done.
 Otherwise, we remove all sets with .
 We either arrive at a proper subset  of ,
 and may pick a state  in  to play the role of new ,
 or we arrive at ,
 which is in the required form .
 This completes the proof of the lemma.
\qed
\end{proof}

\begin{corollary}[Star-Complement-Star: Upper Bound]\label{-----co1}
 If a language  is accepted by a DFA of  states,
 then the language  is accepted by a DFA of  states.
\end{corollary}

\begin{proof}
 Lemma~\ref{le:express} gives the following upper bound
 
 since 
 we first choose any permutation of  distinct elements ,
 and then
 represent each set  as disjoint union of sets 
 given by a function  from  to 
 as follows:
 
 while states with  will be outside each ;
 here  denotes a disjoint union.
 Next, we  have
 
 and the upper bound follows.
\qed
\end{proof}

\begin{remark}
The summation  differs by
one from Sloane's sequence A072597 \cite{Sloane}.  These numbers are the
coefficients of the exponential generating function of
.  It follows, by standard techniques, that
these numbers are asymptotically given by
,
where
 is the
Lambert W-function evaluated at , equal to the positive real
solution of the equation , and  is a constant,
approximately 

The convergence
is quite fast; this gives a somewhat more explicit version of the
upper bound.
\end{remark}

\section{Lower Bound}
\label{***lower}

We now turn to the matching lower bound on the state complexity
of plus-complement-plus.  The basic idea is to create one DFA
where the DFA for  has many reachable states, and another
where the DFA for  has many distinguishable states.
Then we ``join'' them together in Corollary~\ref{-----co2}.

The following lemma uses a four-letter alphabet 
to prove the reachability of some specific states
of the DFA  for plus-complement-plus.

\begin{lemma}\label{-----le3}\label{le:reach}
 There exists an -state
 DFA  
 such that in the  DFA  for the language 
 every state of the form
 
 is reachable, 
 where , 
  are subsets of  with
 , and the
  are pairwise distinct states in 
 that are not in .
\end{lemma}

\begin{proof}
 Consider the DFA  over 
 shown in  Fig.~\ref{fig:max_reach}.
 Let  be the language accepted by the DFA .
 
 \begin{figure}[t]\label{-----fig5}
 \centering
 \includegraphics[scale=0.40]{automat8.eps}
 \caption{DFA  over  with many reachable states in 
         DFA  for .}
 \label{fig:max_reach}
 \end{figure}

 Construct the NFA  for the language  from the DFA 
 by adding loops on  and  in the initial state .
 In the subset automaton corresponding to the NFA ,
 every subset of  containing state 
 is reachable from the initial state  on a string over 
 since each subset  of size ,
 where  and ,
 is reached from the set  of size 
 on the string .
 Moreover, after reading every symbol of string ,
 the subset automaton is always in a set that contains state .
 All such states are rejecting in the DFA  for the language ,
 and therefore, in the NFA  for ,
 the initial state  only goes to the rejecting state
  on .

 Hence in the DFA , for every subset   of 
 containing  ,
 the initial state  goes to the state 
 on a string  over .

 Now notice that transitions on symbols  and 
 perform the cyclic permutation of states  in .
 For every state  in  and an integer ,
 let 
 
 denote the state in 
 that goes to the state  on string ,
 and, in fact, on every string over  of length .
 Next, for a subset  of  let
 
 Thus  is a shift of ,
 and if , then .

 The proof of the lemma now proceeds by induction on .
 To prove the base case, let  be a subset of 
 and  be a state in  with .
 In the NFA , the initial state  goes to the state 
 on a string  over .
 Next, state  is  in ,
 and it is reached from state  on a string ,
 while state  goes to itself on .
 In the DFA  we thus have
 
 which proves the base case.

 Now assume that every set of size  satisfying the lemma
 is reachable in the DFA .
 Let 
 
 be a set of size  satisfying the lemma.
 Let  be a string, on which 
 goes to ,
 and let  be an integer such that   goes to  on .
 Let
 
 where the operation  is understood to have left-associativity.
 Then  is reachable by induction. 
 On ,
 every set 
 goes  to the accepting state 
 
 in the NFA , 
 and therefore also to the initial state .
 Then, on , every state 
  
 goes to the rejecting state 
 ,
 while  goes to .
 Hence, in the DFA  we have 
 
 It follows that  is reachable in the DFA .
 This  concludes the proof.
\qed
\end{proof}

The next lemma 
shows that some  rejecting states of the DFA , 
in which no set is a subset of some other set,
may be pairwise distinguishable.
To prove the result
it uses four symbols,
one of which is the symbol  from the proof of the previuos lemma.

\begin{lemma}\label{-----le4}\label{le:equiv}
 Let .
 There exists an -state DFA 
 over a four-letter  alphabet 
 such that all the states of the DFA  for the language 
 of the form
 
 in which no set is a subset of some other set and 
 each ,
 are~pairwise distinguishable.
\end{lemma} 

\begin{proof}
 To prove the lemma, we reuse the symbol 
 from the proof of Lemma~\ref{le:reach},
 and define three new symbols 
 as shown in Fig.~\ref{fig:befg}.

 \begin{figure}[t]\label{-----fig6}
 \centering
 \includegraphics[scale=0.40]{befg.eps}
 \caption{DFA  over  with many distinguishable states in 
         DFA .}
 \label{fig:befg}
 \end{figure}
 
 Notice that on states ,
 the symbol  performs a big permutation,
 while  performs a trasposition,
 and   a contraction.
 It follows that every transformation of states 
 can be performed by strings over .
 In particular, for each subset  of ,
 there is a string  over 
 such that in  ,
 each state in  goes to state  on ,
 while each state in  
 goes to state  on .
 Moreover, state  remains in itself while reading the string~.
 Next, the symbol  sends state  to state ,
 state  to state ,
 and state  to itself.

 It follows that in the NFA ,
 the  state ,
 as well as each state  with ,
 goes to the accepting state  on .
 However,
 every other state  with 
 is  in a state containig ,
 thus in a rejecting state of ,
 while reading , 
 and it is in the rejecting state 
 after reading .
 Then  goes to the rejecting state  on reading .
 
 Hence the string  is accepted by the NFA 
 from each state  with ,
 but rejected from any other state 
 with .
 
 Now consider  two different states of the DFA 
 
 in which no set is a subset of some other set and where
 each  and each  is a subset of .
 Then, without loss of generality,
 there is a set  in  that is not in .
 If no set  with  is in ,
 then the string 
 is accepted from  but not from .
 If there is a subset  of 
 such that  is  in ,
    then for each suset  of 
    the set   cannot be in ,
    and then the string  
 is accepted from  but not from .
\qed
\end{proof}

\begin{corollary}[Star-Complement-Star: Lower Bound]\label{-----co2}
 There exists a language  
 accepted by an -state DFA  over a seven-letter input alphabet,
 such that any DFA for the language  
 has   states.
\end{corollary}

\begin{proof}
 Let 
  and
  be the language 
 accepted by -state DFA ,
 where transitions on symbols  
 are defined as in the proof of Lemma~\ref{le:reach},
 and on symbols  as in the proof of Lemma~\ref{le:equiv}.

 Let .
 By Lemma~\ref{le:reach},
 the following states
 are reachable in the DFA     
 for :
 
 where 
 .
 The number of such subsets  is given by , and we have
 
 By Lemma~\ref{le:equiv}, all these states are pairwise distinguishable,
 and the lower bound follows. 
\qed
\end{proof}

Hence we have an asymptotically tight bound on the state complexity of
star-complement-star operation that is significantly smaller than
.

\begin{theorem}\label{-----thm1}
 The state complexity of star-complement-star is .
\qed
\end{theorem}

\section{Applications}
\label{***applications}

We conclude with an application.

\begin{corollary}
Let  be a regular language, accepted by a DFA with  states.  Then
any language that can be expressed in terms of  and the operations
of positive closure, Kleene closure, and complement has state complexity
bounded by .
\end{corollary}

\begin{proof}
As shown in \cite{BGS}, every such language
can be expressed, up to inclusion of~,
as one of the following  languages and their complements:

If the state complexity of  is , then
clearly the state complexity of  is also~.  Furthermore,
we know that the state complexity of  is bounded by
 (a more exact bound can be found in \cite{yzs94}); this
also handles .
The remaining languages can be handled with Theorem~\ref{-----thm1}.
\qed
\end{proof}


\begin{thebibliography}{00}
\label{***biblio}

\bibitem{BGS} Brzozowski, J., Grant, E., and Shallit, J.:  
Closures in formal languages and Kuratowski's theorem,
Int. J. Found. Comput. Sci. 22, 301--321 (2011)

\bibitem{KM75} Kleitman, D. and Markowsky, G.:
On Dedekind's problem: the number of isotone Boolean functions. II,
Trans. Amer. Math. Soc. 213, 373--390 (1975)

\bibitem{rs59} \mbox{Rabin, M.,  Scott, D.:}
    Finite automata and their decision problems.
    IBM Res. Develop. 3,  114--129 (1959)
\bibitem{si97} \mbox{Sipser, M.:}
    Introduction to the theory of computation.
    PWS Publishing Company, Boston (1997)
\bibitem{Sloane} Sloane, N. J. A.:
	Online Encyclopedia of Integer Sequences,
	{\tt http://oeis.org}
\bibitem{yu97} \mbox{Yu, S.:}
    Chapter~2: Regular languages.
    In: \mbox{Rozenberg, G.,} \mbox{Salomaa, A.} (eds.)
    Handbook of Formal Languages - Vol. I, pp. 41--110.
    Springer,  Heidelberg (1997)  
\bibitem{yzs94} \mbox{Yu, S.,} \mbox{Zhuang, Q.,}   \mbox{Salomaa, K.:}
    The state complexity of some basic operations on regular languages.
    Theoret. Comput. Sci. 125, 315--328 (1994)

\end{thebibliography}

\end{document}