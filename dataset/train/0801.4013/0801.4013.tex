

\documentclass[pdftex,leqno,fleqn,12pt]{article}



\usepackage[latin1]{inputenc}   \usepackage{xspace}

\usepackage{graphicx}           

\usepackage{latexsym}           \usepackage{amsmath}
\usepackage{amssymb}
\usepackage{amsfonts}
\usepackage{subfigure}

\usepackage[sort&compress,numbers]{natbib}

\usepackage{algorithmic}
\usepackage{algorithm}
\renewcommand{\algorithmicrequire}{\textbf{Input:}}
\renewcommand{\algorithmicensure}{\textbf{Output:}}
\newcommand{\algorithmiclocal}{\textbf{Local Variables: }}
\newcommand{\LOCAL}{\item[\algorithmiclocal]}
\newcommand{\algorithmicsend}{\textbf{send }}
\newcommand{\algorithmicreceive}{\textbf{receive }}
\newcommand{\SEND}{\algorithmicsend}
\newcommand{\RECEIVE}{\algorithmicreceive}
\newcommand{\true}{\textbf{true}}
\newcommand{\false}{\textbf{false}}

\newcommand{\MIS}{{\rm{MIS}}}
\newcommand{\DEL}{{\rm{Del}}}
\newcommand{\VOR}{{\rm{Vor}}}
\newcommand{\YAO}{{\rm{Yao}}}
\newcommand{\sdsr}{{\rm{SP-DT}}}




\newcommand{\mnote}[1]{\setlength{\marginparsep}{0.1in}
    \marginpar[\flushright\tiny\fbox{\parbox{0.5in}{\raggedright#1}}]
    {\flushleft\tiny\fbox{\parbox{0.5in}{\raggedright#1}}}}

\newcommand{\tabul}{\mbox{\ \ \ \ }}


\newtheorem{theorem}{Theorem}[section]
\newtheorem{corollary}[theorem]{Corollary}
\newtheorem{lemma}[theorem]{Lemma}
\newtheorem{claim}[theorem]{Claim}
\newtheorem{axiom}[theorem]{Axiome}
\newtheorem{conj}[theorem]{Conjecture}
\newtheorem{fact}[theorem]{Fact}
\newtheorem{hypo}[theorem]{Hypothesis}
\newtheorem{assum}[theorem]{Assumption}
\newtheorem{proposition}[theorem]{Proposition}
\newtheorem{crit}[theorem]{Criterion}
\newtheorem{definition}[theorem]{Definition}
\newtheorem{notn}[theorem]{Notation}
\newtheorem{exmp}[theorem]{Example}
\newtheorem{remark}[theorem]{Remark}
\newtheorem{problem}[theorem]{Problem}
\newtheorem{prin}[theorem]{Principle}
\newtheorem{alg}{Algorithm}

\newcommand{\dem}{\noindent\textit{Proof :}}
\newcommand{\RANK}{\textsc{rank}}
\newcommand{\COLOR}{\textsc{color}}


\newcommand{\JOIN}{\noindent{\textsc{join}}}
\newcommand{\GIVEUP}{\noindent{\textsc{give-up}}}

\newenvironment{proof}{{\textit Proof:} \rm}{\hfill  \medskip\\}
\newcommand{\old}[1]{{}}
\newcommand{\etal}{\textit{et al.}}

\begin{document}

\title{Spanners of Additively Weighted Point Sets\thanks{Research partially supported by NSERC, MRI, CFI, and MITACS.}}
\author{Prosenjit Bose \and Paz Carmi \and Mathieu Couture\\ {\small Scool of Computer Science, Carleton University, Ottawa, Canada}}




\maketitle

\begin{abstract} We study the problem of computing geometric spanners for
(additively) weighted point sets. A weighted point set is a set of pairs  where  is a
point in the plane and  is a real number. The distance between two points  and
 is defined as . We show that in the case where all  are positive
numbers and  for all  (in which case the points can be seen as
non-intersecting disks in the plane), a variant of the Yao graph is a -spanner that
has a linear number of edges. We also show that the Additively Weighted Delaunay graph (the
face-dual of the Additively Weighted Voronoi diagram) has constant spanning ratio. The straight
line embedding of the Additively Weighted Delaunay graph may not be a plane graph. 
We show how to compute a plane embedding that also has a
constant spanning ratio.  \end{abstract}

\newcommand{\UDG}{\rm UDG}
\newcommand{\bis}{\rm bis}

\newpage

\section{Introduction}

Let  be a complete weighted graph where edges have positive weight. Given
two vertices  of , we denote by  the length of a
shortest path in  between  and . A spanning subgraph  of  is
a \emph{-spanner} of  if  for all
pair of vertices  and . The smallest  having this property is
called the \emph{spanning ratio} of the graph  with respect to . Thus,
a graph with spanning ratio  approximates the  distances
between the vertices of  within a factor of . Let  be a set of 
points in the plane. A \emph{geometric graph} with vertex set  is an
undirected graph whose edges are line segments that are weighted by their
length.  The problem of constructing -spanners of geometric graphs with
 edges for any given point set has been studied extensively; see the
book by Narasimhan and Smid~\cite{smid07} for an overview.

In this paper, we address the problem of computing geometric spanners with
additive constraints on the points. More precisely, we define a weighted
point set as a set of pairs  where  is a point in the plane and
 is a real number. The distance between two points  and
 is defined as . The problem we address is to
compute a spanner of a complete graph on a weighted point set. To the best
of our knowledge, the problem of constructing a geometric spanner in this
context has not been previously addressed. We show how the Yao graph can be
adapted to compute a -spanner in the case where all  are
positive real numbers and  for all  (in which
case the points can be seen as non-intersecting disks in the plane). In the
same case, we also how the Additively Weighted Delaunay graph (the face-dual
of the Additively Weighted Voronoi diagram) provides a plane spanner that
has the same spanning ratio as the Delaunay graph of a set of points.

\subsection{Motivations}

It has been claimed (see ~\cite{alzoubi03,schindelhauer04,schindelhauer07})
that geometric spanners can be used to address the link selection problem in
wireless networks. In most cases, however, two assumptions are made:
\begin{enumerate}
\item nodes can be represented as points in the plane and
\item the cost of routing a message is a function of the length of the links
that are successively used.
\end{enumerate}
However, these assumptions do
not always hold. For example, the first assumption does not hold in the case
of wide area mesh networks, where nodes are vast areas such as
villages~\cite{raman04}.  The second assumption does not take into account
the fact that some nodes may have higher energy resources or introduce more
delay than others. In such cases, an additional cost must be taken into
account for each node. The study of spanners of additively weighted
point sets is a first step in addressing some of these issues.

\subsection{Paper Organization}

The rest of the paper is divided as follows: In Section~\ref{section-disk-del-related}, we review
related work. In Section~\ref{section-disk-del-def}, we give a formal definition of our problem and
show that it is not solved by a straightforward extension of the Yao graph. However, in
Section~\ref{section-yao}, we show that a minor adjustment to the Yao graph allows to compute a
-spanner. In Section~\ref{section-disk-del-quotient}, we develop some tools used in
Section~\ref{section-disk-del-spanning-ratio} to show that the Additively Weighted Delaunay graph
has a constant spanning ratio. We conclude in Section~\ref{section-disk-del-conclusion}.

\section{Related Work}\label{section-disk-del-related}

Well known examples of geometric -spanners include the Yao
graph~\cite{yao82}, the -graph~\cite{ruppert91}, the Delaunay
graph~\cite{keil92}, and the Well-Separated Pair Decomposition
(WSPD)~\cite{callahan95}. Let  be an angle such that
, where  is an integer. The Yao graph with angle 
is defined as follows. For every point , partition the plane into 
cones  of angle  and apex .  Then, there
is an oriented edge from  to  if and only if  is the closest point
to  in some cone . The Yao graph is sometimes confused with the
-graph, although they are different graphs. The first phase of the
construction of the -graph using  cones with angle  and
apex  is identical to the construction of the Yao graph. This may be the
root of the confusion. However, there is an oriented edge from  to  in
the -graph if and only if  has the shortest projection on the
bisector of the cone containing . For Yao graphs~\cite{yao82}, the
spanning ratio is at most  provided that
, and for -graphs, the spanning ratio is at
most~ provided that
~\cite{ruppert91}.

Given a set of points in the plane, there is an edge between  and  in the
Delaunay graph if and only if there is an empty circle with  and  on its
boundary~\cite{keil92}.  If no four points are cocircular, then the Delaunay
graph is a uniquely defined near-triangulation. Otherwise, four or more
co-circular points may create crossings. In that case, removing edges that
cause crossings leads to \emph{a} Delaunay triangulation. Since our results
hold for any Delaunay triangulation, when we refer to \emph{the} Delaunay
triangulation in the case of co-circular points, we mean \emph{any} Delaunay
triangulation. \citet{dobkin90} showed that the Delaunay triangulation has a
spanning ratio of at most . This result
was improved by~\citet{keil92}, who showed that the spanning ratio of the
Delaunay triangulation is at most .  Later,
\citet{bose04} showed that the Delaunay triangulation is also a strong
-spanner for the same constant .  Although the exact
spanning ratio of the Delaunay triangulation is unknown, it is conjectured that
the spanning ratio is . For the remainder of this paper, we will refer
to the spanning ratio of the Delaunay triangulation as the spanning ratio of
the {\em standard} Delaunay triangulation and denote it as \sdsr.

The \emph{Voronoi diagram}~\cite{deberg97} of a finite set of points  is
a partition of the plane into  regions such that each region contains
exactly those points having the same nearest neighbor in . The points in
 are also called \emph{sites}.  It is well known that the Voronoi diagram
of a set of points is the face dual of the Delaunay graph of that set of
points~\cite{deberg97}, i.e. two points have adjacent Voronoi regions if and
only if they share an edge in the Delaunay graph (see
Figure~\ref{fig-vor-del}).

\begin{figure}
\centering
\includegraphics{del-vor2.pdf}
\caption{The Delaunay graph and its dual: the Voronoi diagram.}\label{fig-vor-del}
\end{figure}

Let  be a real number. Two set of points  and  in  are
\emph{well-separated with respect to } if there exists two -dimensional
balls  and  of same radius  respectively containing the bounding
boxes of  and  such that the distance between  and  is greater
than or equal to . The distance between  and  is defined
as the distance between their centers minus . A \emph{Well-Separated
Pair Decomposition with separation ratio  of a set of points
}~\cite{callahan95,smid07} is a set of unordered pairs
 of subsets of  that are well-separated
with respect to  with the additional property that for every two points
 there is exactly one pair  such that  and
. \citet{callahan95} showed that for , every point set admits a
WSPD with separation ratio  of  size that can be computed in  time. Choosing one edge per pair allows to construct a -spanner that has
 size with .

Our work falls in the context of computing spanners for geometric graphs other than the complete
Euclidean graph. Typically, variations of the spanner problem arise by either changing the distance
function or removing edges from the complete graph. For example, for a set  of points in the
plane and a set  of non-intersecting line segments whose endpoints are in , the
\emph{visibility graph} of  with respect to  is the geometric graph with vertex set  and
there is an edge  if and only if the segment  is in  or it does not cross
any segment in  (in that case,  and  are said to be \emph{visible}). A spanner of the
visibility graph should then approximate Euclidean distances for every pair of points that are
visible from each other. The constrained Delaunay triangulation (a variation of the Delaunay
triangulation) is a -spanner of the visibility graph~\cite{ioannis01, klein06, bose06}.

Unit disk graphs~\cite{hale80,johnson90} received a lot of attention from the wireless community. A
\emph{unit disk graph} is a graph whose nodes are points in the plane and edges join two points
whose distance is at most one unit. It is well-known that intersecting a unit disk graph with the
Delaunay or the Yao graph of the points provides a -spanner of the unit disk
graph~\cite{bose04}, where the constant  is the same as the one of the original graph. However,
this simple strategy does not work with all spanners. In particular, it does not work with the
-graph~\cite{couture07c}. Unit disk graphs can be seen as intersection graphs of disks of
same radius in the plane. The general problem of computing spanners for geometric intersection
graphs has been studied by~\citet{furer07}.

Another graph that has been looked at is the \emph{complete -partite Euclidean graph}. In that
case, points are assigned a unique color (which may be thought of as a positive integer) between 1
and , and there is an edge between two points if and only if they are assigned different colors.
Bose \etal~\cite{couture07bispanReport} showed that the WSPD can be adapted to compute a
-spanner of that graph that has  edges for arbitrary values of  strictly greater than
5.

For spanners of arbitrary geometric graphs, much less is known. Alth{\"o}fer \emph{et
al.}~\cite{addjs-sswg-93} have shown that for any , every weighted graph  with  vertices
contains a subgraph with  edges, which is a -spanner of . Observe that this
result holds for any weighted graph; in particular, it is valid for any geometric graph. For
geometric graphs, a lower bound was given by Gudmundsson and Smid~\cite{gs-osogg-06}: They proved
that for every real number  with , there exists a geometric graph
 with  vertices, such that every -spanner of  contains  edges.
Thus, if we are looking for spanners with  edges of arbitrary geometric graphs, then the best
spanning ratio we can obtain is .

In the literature, spanners that use a distance other than the Euclidean distance have also been
proposed. For example, in a \emph{power}
spanner~\cite{aurenhammer87,li01,grunewald02,schindelhauer04}, the distance used to measure the
length of an edge is the square of the Euclidean distance between its two end points. This models
the fact that in wireless networks, the amount of energy needed to send a packet is proportional to
a power (not necessarily the square, however) of the Euclidean distance between the sender and
receiver~\cite{pahlavan95}. When reducing the latency is more important than reducing the amount of
energy being used, a \emph{hop} spanner~\cite{alzoubi03}, which gives an equal weight to every
edge, can be used.

In this paper, the Additively Weighted Voronoi diagram (AW-Voronoi diagram) is
of particular interest.
Its definition is similar to that of the (standard) Voronoi diagram, except
that each site  is assigned a weight which is a real number .
Weights are used to define a weighted distance. More detail about how the
weighted distance is used to define the AW-Voronoi diagram is given in
Section~\ref{section-disk-del-spanning-ratio}. The Additively Weighted Delaunay
graph (AW-Delaunay graph) is defined as the face-dual of the AW-Voronoi
diagram. Properties of the AW-Voronoi diagram and its dual have been studied by
Lee and Drysdale~\cite{drysdale81}, who showed how to compute it in  time. Later on, Fortune~\cite{fortune87} showed how to compute it in
 time. The AW-Voronoi diagram may have empty cells. For this
reason, one would hope that it is possible to design an algorithm whose running
time gets better as the number of empty cells increases. Karavelas and
Yvinec~\cite{karavelas02} provided an  time algorithm to
compute the AW-Voronoi diagram where  is the number of non-empty cells and
 is the time to locate the nearest neighbor of a query point within a set
of  points.  Experimental results suggested an  behavior. In 3D,
the complexity of the (Additively Weighted) Voronoi diagram is
~\cite{klee80}. Aurenhammer~\cite{aurenhammer87} showed how to
compute it in time  using Power Voronoi diagrams. Will~\cite{will98}
gave an  time algorithm with experimental results suggesting an
 time behavior in the expected case. Kim \etal~\cite{kim05}
showed how to obtain a running time of , where  is the number of
edges.



\section{Definitions and Notation}\label{section-disk-del-def}

\begin{definition}
A set  of ordered pairs, where each  is a point in the plane
and each  is a real number, is called a \emph{weighted point set}. The notation 
means that there exists an ordered pair  such that . The \emph{additive
distance} from a point  in the plane to a point , noted , is
defined as , where  is the Euclidean distance from  to . The \emph{additive distance}
between two points , noted , is defined as , where 
is the Euclidean distance from  to .
\end{definition}
The problem we address in this paper is the following:
\begin{problem} Let  be a weighted point set and let  be
the complete weighted graph with vertex set  and edges weighted
by the additive distance between their endpoints. Compute a
-spanner with  edges of  for a fixed constant .
\end{problem}

Notice that in the case where all  are positive numbers, the pairs
 can be viewed as disks  in the plane. If, for all  we
also have , then the disks are disjoint. In that case, the
distance  is also equal to
 and , where the notation 
means .  To compute a spanner of an additively weighted point
set is then equivalent to computing a spanner of a set of disks in the plane.
\textbf{From now to the end of this paper, it is assumed that all  are
positive numbers and  for all .} If  is a
set of disks in the plane, then a \emph{spanner} of  is a spanner
of the complete graph whose vertex set is  and whose edges
 are given weights equal to .

\begin{figure}
\centering\includegraphics{nometric.pdf}\caption{The additively weighted distance is not a
metric.}\label{fig-nometric}
\end{figure}

Notice also that the additive distance may not be a metric since the triangle inequality does not
necessarily hold (see Figure~\ref{fig-nometric}). Although this may seem counter-intuitive, this
makes sense in some networks, since a direct communication is not always easier than routing
through a common neighbor. For example, in wireless networks, the amount of energy that is needed
to transmit a message is a power of the Euclidean distance between the sender and the receiver.
Therefore, using several small hops can be more energy efficient that a direct communication over
one long-distance link.

\begin{figure}[htb]
\begin{center}
\includegraphics[scale=0.95]{yao-disks.pdf} \caption{A straightforward generalization of the Yao graph.}\label{fig-yao-disks}
\end{center}
\end{figure}

\begin{figure}[htb]
\begin{center}
\includegraphics[scale = 0.85]{yao-disks-not-spanner.pdf}\caption{The straightforward generalization of the Yao graph
does not have constant spanning ratio.}\label{fig-yao-disks-not-spanner}
\end{center}
\end{figure}

Figure~\ref{fig-yao-disks} shows how the Yao graph can be generalized using the
additive distance: every node keeps an outgoing edge with the closest disk that
intersects each cone. However, this graph is not a spanner.
Figure~\ref{fig-yao-disks-not-spanner} shows how to construct an example with
four disks that has an arbitrarily large spanning ratio. Nonetheless, in
Section~\ref{section-yao}, we see that a minor adjustment to the Yao graph can
be made in order to compute a -spanner of a set of disjoint disks
that has  edges.


The Delaunay graph in the additively weighted setting is computable in time
~\cite{fortune87}. To the best of our knowledge, its spanning
properties have not been previously studied. In the two next sections, we show
that it is a spanner and that its spanning ratio is \sdsr\ (i.e the same as
that of the standard Delaunay graph). Finally, we show that although the
additively weighted Delaunay graph is not necessarily plane,  it contains a
plane subgraph that is a spanner with the same spanning ratio.
\section{The Additively Weighted Yao Graph}\label{section-yao}
\begin{figure} \centering\includegraphics{yao-lemma}\caption{Illustration of
the proof of Lemma~\ref{lemma-yao}.}\label{fig-lemma-yao} \end{figure}
As we saw in the previous section, a straightforward generalization of the
Yao graph fails to provide a graph with bounded spanning ratio. In this section,
we show how a few subtle modifications to the construction, provide
an approach to build a -spanner. We define the modified
Yao construction below.
\begin{definition} Let  be a finite set of disjoint disks and
 be an angle such that , where  is an
integer. The  graph is defined as follows. For every
disk , partition the plane into  cones  of
angle  and apex . A disk \emph{blocks} a cone  provided
that the disk intersects both rays of . Let  be a
disk different from  with center in . Add an edge from 
to  in  if and only if one of the two following
conditions is met: \begin{enumerate} \item among all blocking disks that have
their center in ,  is the one that is the closest to ; \item
among all disks that have their center in  and are at a distance of at
least  to ,  is the one that is the closest to .  \end{enumerate}
\end{definition}
Notice that there are two main changes. Within each cone, we now add
potentially two edges as opposed to only one edge in the case of unweighted
points. Next, in the second condition to add an edge, we do not add an edge to
the closest disk within a cone but to the closest disk whose distance is at
least  from the disk centered at the apex with radius . We now prove that
these two modifications imply that the resulting graph is a -spanner.
\begin{lemma}\label{lemma-yao} Let  such that the angle  and . Then
.  \end{lemma}
\begin{proof} Let  be the projection of  on the line through 
and  (see Figure~\ref{fig-lemma-yao}). Then  \end{proof}



\begin{theorem}\label{thm-add-yao} Let  be a finite set of
disjoint disks and .  Then  is a
-spanner of , where .  \end{theorem} \begin{proof} We proceed by induction
on the rank of the weighted distances between the pairs of disks  and
.

\textbf{Base case:} The disks  and  form a closest pair. In that case, the edge
 is in .

\textbf{Induction case:} Let  and . Without loss
of generality, . If the edge  is in
, then there is nothing to prove.  Otherwise, there
are two cases to consider depending on whether or not the shortest path from
 to  in the complete graph on  is the edge .
If the shortest path is not the edge , then all edges on the
shortest path must have length less than . By applying the
induction hypothesis on each of those edges, we conclude that the distance from
 to  in  is at most  times the length of
the shortest path  to  in the complete graph on , as
required.

\begin{figure}
\centering\includegraphics{no-block.pdf}\caption{If  blocks the cone but the edge  is not in , then there exists  such that .}\label{fig-no-block}
\end{figure}

We now consider the case when the edge :
\begin{enumerate}
\item is not in  and
\item is the shortest path from  to  in the complete graph.
\end{enumerate}

Observe that the conjunction of those two facts imply that the disk  does
not block the cone whose apex is  and contains : If  was
blocking the cone, then since  is not an edge in
, there must be a disk  that is also blocking
the cone and is closer to  than . However, this implies that the
shortest path from  to  in the complete graph is not the edge
 (see Figure~\ref{fig-no-block}).

The conjunction of the three following facts:
\begin{enumerate}
\item ;
\item  and
\item  does not block the cone,
\end{enumerate}
imply that . Since  is not an edge, there another
disk whose distance is at least  that is closer to . Let
 be the closest disk to  such that  is in the same
-cone with apex at  as  and . By
definition, the edge  is in .  Observe
that . To see this, let . We have
that 
\begin{figure} \centering\includegraphics{add-yao-thm.pdf}\caption{Illustration
of the proof of Theorem~\ref{thm-add-yao}.}\label{fig-add-yao} \end{figure}
Let  be the point of  that is the closest to ,  be the
point of  that is the closest to ,  be the point of  that
is the closest to , and  be the point of  that is the closest
to  (see Figure~\ref{fig-add-yao}). Notice that  and that since , then the angle
 is at most .  Therefore, we can apply
Lemma~\ref{lemma-yao} to conclude that  which implies that  Also, since
, we have


Finally, since , the induction hypothesis tells us that
 contains a path from  to  whose length is at most
. This means that the distance from  to  in  is at
most

The value 0.228 is an upper bound on the values of  such that .
\end{proof}



\begin{corollary} For any  and any set  of  disjoint disks, it is
possible to compute a -spanner of  that has  edges.
\end{corollary}
\begin{proof} The bound on the number of edges comes from the fact that each cone contains at most
two edges, and the stretch factor of  comes from the fact that
.
\end{proof}

\section{Quotient Graphs and Quotient Spanners}\label{section-disk-del-quotient}

The main idea in the remainder of this paper is the following: we show how to compute a set of points
from each  such that the (standard) Delaunay graph of those points is \emph{equivalent} to the
Additively Weighted Delaunay graph. By choosing the appropriate equivalence relation as well as the
appropriate point set, we can then show that the spanning ratio of the Additively Weighted Delaunay
graph is bounded by the spanning ratio of the standard Delaunay graph. The reduction of one graph
to another is done by means of a quotient:


\begin{definition} Let  and  be non-empty sets of points in the plane.
The \emph{distance} between  and , denoted by , is defined as the minimum 
over all pairs of points such that  and .
\end{definition}

\begin{definition}Let  be a geometric graph
and  be a partition of . The \emph{quotient graph} of  by , denoted
, is the graph having  as vertices and there is an edge  (where
 and  are in ) if and only if there exists an edge  with 
and . The weight of the edge  is equal to .
\end{definition}

If  is a (non-weighted) point set and  is a partition of , then the notation
 designates the quotient of the complete Euclidean graph on  by . If
 is a set of pairwise disjoint sets of points in the plane such that , then the notation  designates the quotient of the complete
Euclidean graph on  by the partition of  induced by .

\begin{figure}
\centering
\includegraphics{quotient.pdf}
\caption{Illustration of Lemma~\ref{thm-quotient}.}
\end{figure}

\begin{lemma}\label{thm-quotient}Let 
be a complete geometric graph,  be a partition of  and  be a -spanner of .
Then  is a -spanner of .
\end{lemma}
\old{
\begin{proof} Let  be an edge of  and 
be an edge of  such that . Since  is complete, the edge  is in , and
since  is a -spanner of , there is a path  in  such that
 and the length of  is at most . By definition,  contains
a path  such that  and . The length of  is at most

which means that  is a -spanning path for  in .
\end{proof}
}
\begin{proof} Let  be an edge of  and 
be an edge of  such that . Since  is complete, the edge  is in , and
since  is a -spanner of , there is a path  in  such that
 and the length of  is at most .  For each  of , let
 be such that . Notice that it is possible that  for
some . Let  be the subsequence of  that consists in those  such
that  and . By definition, the sequence  is a path in 
and it consists of at most  nodes. The length of  is at most

which means that  is a -spanning path for  in .
\end{proof}


\section{The Additively Weighted Delaunay Graph}\label{section-disk-del-spanning-ratio}


\citet{drysdale81} studied a variant of the Voronoi diagram called the Additively Weighted Voronoi
diagram, which is defined as follows: Let  be a weighted point set. The \emph{Additively
Weighted Voronoi diagram} of  is a partition of the plane into  regions such that each
region contains exactly the points in the plane having the same closest neighbor in  according
to the additive distance. In other words, the Voronoi cell of a pair  contains the
points  such that  is minimum over all other pairs in . The \emph{Additively
Weighted Delaunay graph} (AW-Delaunay graph) is defined as the face-dual of the Additively Weighted
Voronoi diagram.

Alternatively, if all  are positive and for all , we have , then
the pairs  can be seen as disks  of radius  centered at  and 
is the minimum  over all . For a set  of disks in the plane, we
denote the AW-Delaunay graph computed from  as . When no two disks
intersect, the AW-Delaunay graph is a natural generalization of the Delaunay graph of a set of
points. We say that two disks  and  \emph{properly} intersect if .


\begin{proposition}\label{prop-dual} Let  be a set of disjoint disks in the
plane, and . The edge  is in  if and only if there is
a disk  that is tangent to both  and  and does not properly intersect any other disk in
.
\end{proposition}
\begin{proof}Suppose  is in , and let 
be a point on the boundary of the Voronoi cells of  and  and  be the distance from  to
. Since  is equidistant from  and , it is also at distance  from . This means that
the disk  centered at  is tangent to both  and . This disk cannot properly intersect
any other disk of , since this would contradict the fact that  is in the Voronoi
cells of  and . Similarly, if there is a disk that is tangent to both  and  but does
not properly intersect any other disk of , then  and  are Voronoi neighbors.
\end{proof}

\begin{figure}
\centering
\includegraphics{del-disks.pdf}
\caption{The Additively Weighted Delaunay graph compared with the Delaunay graph of the disks
centers.}\label{fig-del-disks}
\end{figure}

Note that the Additively Weighted Delaunay graph is not necessarily isomorphic
to the Delaunay graph of the centers of the disks (see
Figure~\ref{fig-del-disks}). When all radii are equal, however, the two graphs
coincide. We now show that if  is a set of disks in the plane,
then  is a spanner of . The intuition behind
the proof is the following: we show the existence of a finite set of points 
such that  (where  is the complete graph with vertex
set ) is isomorphic to the complete graph on  and
 is a subgraph of . Then, we use
Lemma~\ref{thm-quotient} to prove that  is a spanner of
, which implies that  is a spanner of
.

\begin{definition}\label{def-repr} Let  be disjoint disks and  a set of points such
that  and . A set of points 
\emph{represents}  with respect to  and  if for every disk  that is tangent to both  and
, we have . If  is a set of
disjoint disks, then a set of points  \emph{represents}  if for all
, there is a subset of  that represents  with respect to 
and .
\end{definition}
From here to the end of the proof of Lemma~\ref{lemma-repr}, unless stated otherwise, let
\begin{enumerate}
\item  be two disjoint disks in the plane having their
center on the -axis;
\item  be the disk that is tangent to both  and  and whose center has
-coordinate equal to ;
\item  be the -coordinate of the center of a disk ;
\item  be the two lines that are outer-tangent to both  and  (respectively, from below and above);
\item  be such that  and ;
\item  be the line through the intersection points of the boundaries of  and 
(if  and  are tangent, then  is the unique line that is tangent to both
 and );
\item  denote the region below , above  and between
 and ; and
\item  () be the closed half-plane above (below) a non-vertical line .
\end{enumerate}
Throughout this section, it is implicitly assumed that  and .
\begin{figure}
\centering\includegraphics{lunes.pdf}\caption{Illustration of the proof of
Lemma~\ref{lemma-lunes}.}\label{fig-lemma-lunes}
\end{figure}



\begin{lemma}\label{lemma-lunes}  Given  and , we have  and
 (see Figure~\ref{fig-lemma-lunes}).
\end{lemma}
\begin{proof}
Notice that either  or . Therefore, all we need to show is that  is not empty. Let  be the respective centers of  and ,
and  be the intersection point of the infinite ray from  through  with the boundary of
.

We show by contradiction that  is not in . If that was the case, then  would be
completely contained in . The reason for this is that there is no point of  that is
farther from  than . Let  be a point of . Then . But the fact that  is completely contained in  contradicts
the fact that they are both tangent to  and .

Therefore, since , we have ,
which imply that . Similarly, .
\end{proof}

\begin{figure}
\centering
\includegraphics{lemma-tunnel.pdf}\caption{Illustration of the proof of Lemma~\ref{lemma-tunnel}.}\label{fig-lemma-tunnel}
\end{figure}

\begin{lemma}\label{lemma-tunnel}
Let  be the intersection points of the boundaries of  and  (if 
and  are tangent, then ). Then  and  are in  and in 
(see Figure~\ref{fig-lemma-tunnel}).
\end{lemma}
\begin{proof}
Let  be the tangency points of  with  and  and  be the tangency
points of  with  and . By Lemma~\ref{lemma-lunes},  are below  and
 are above . Since  is above  and , which are in turn above
, it follows that  and  are above . By a symmetric argument,  and
 are below .
\end{proof}

\begin{lemma}\label{lemma-containment} The following are true:
\begin{enumerate}
\item For all , there exists a line  such that for all disk  that is tangent to both  and
, if the center of  is above  then .
\item For all , there exists a line  such that for all disk  that is tangent to both  and
, if the center of  is below  then .
\item For all  in , there exists two lines  and  such that for all disk  that is tangent to both  and
,  if and only if the center of  is between  and .
\end{enumerate}
\end{lemma}
\begin{proof}
For (1), the existence of  is guaranteed by the fact that . Now, let  be such that  and . Let  and  be
the lunes respectively defined by the intersection of  and  with the half-plane
above . By Lemma~\ref{lemma-tunnel}, the two points where the boundaries of  and
 intersect are below . Therefore, we have either  or
. But since , by Lemma~\ref{lemma-lunes} we have 
and therefore . The proof of (2) is symmetric.

\begin{figure}
\centering\includegraphics{case3.pdf}\caption{Illustration of the proof of
Lemma~\ref{lemma-containment} (3) (first part).} \label{fig-lemma-cont-tunnelb}
\end{figure}

\begin{figure}
\centering\includegraphics{cont-tunnel.pdf}\caption{Illustration of the proof of
Lemma~\ref{lemma-containment} (3) (second part).} \label{fig-lemma-cont-tunnel}
\end{figure}

For (3), the existence is easy to show. Without loss of generality, assume
.  Let  be the disk centered at  that is tangent to
 and let  be the tangency point of  and  see
Figure~\ref{fig-lemma-cont-tunnelb}. Since , there exists  such
that . Since , there exists a disk that is tangent
to both  and  and contains .

We now show that  implies  (see
Figure~\ref{fig-lemma-cont-tunnel}). Let  be the line through the intersection points of
the boundaries of  and  and let  be the line through the intersection
points of the boundaries of  and . Let . Since  is
above  in ,  is either above , below  or both. If
, then since , by Lemma~\ref{lemma-lunes} we have that
 and . Similarly, if
, then since , by Lemma~\ref{lemma-lunes} we have that
 and . In either case,
, which completes the proof.
\end{proof}
\begin{figure}
\centering\includegraphics{cases-cap.pdf}\caption{The five regions for
Lemma~\ref{lemma-repr}.}\label{fig-lemma-repr}
\end{figure}
\begin{figure}
\centering\includegraphics{case41.pdf}\caption{Case  of the proof of
Lemma~\ref{lemma-repr}.}\label{fig-lemma-repr4}
\end{figure}
\begin{lemma}\label{lemma-repr} Let  be a disk that is disjoint of both  and . There exists a
set of at most six points that represents  with respect to  and .
\end{lemma}
\begin{proof} Let

These five regions partition the disk  (see Figure~\ref{fig-lemma-repr}). We show that for each
region, there is a finite set of points that represents it. The cardinality of the union of the
sets is no more than six.

If , then let  be the minimum  such that  intersects . Let
. By definition of , for any disk  that is tangent to both  and 
and intersects , we have , and by Lemma~\ref{lemma-containment}, we have .

Similarly, if , then let  be the maximum  such that  intersects
. Let . By definition of , for any disk  that is tangent to both 
and  and intersects , we have , and by Lemma~\ref{lemma-containment}, we have
.

If , then let  be the minimum  such that  intersects  and
 as the maximum  such that  intersects . Let  and
. By definition of , for any disk  with  that is tangent to both
 and  and intersects , we have , and by Lemma~\ref{lemma-containment}, we
have . The same reasoning applies to  when .

If , then let  be the minimum  such that  intersects  and
 as the maximum  such that  intersects . Let  and . Let  be such that  (see Figure~\ref{fig-lemma-repr4}).
Let  be a disk that is tangent to both  and
 and intersects . We show that  (and similarly, ). It is sufficient to show that . Let . By Lemma~\ref{lemma-containment},  such
that  disk  tangent to both  and , we have . Therefore, the following hold:

But since , we have , which imply that .

Finally, since  for any disk  that is tangent to both  and , there is no
need to select representative points for .
\end{proof}

Careful analysis of the proof of Lemma~\ref{lemma-repr} allows us to observe that in fact, only two
points are necessary to represent a disk  with respect to two other disks  and . First,
note that  and .
This reduces to four the number of points that are necessary. Also, if  and
, then  is on  and is not required since any disk that contains it
also intersects  and therefore contains . Similarly, if  and
, then  is not required since any disk that contains it also intersects
 and therefore contains . Therefore, if , then the number of points
that are necessary is at most two. A similar argument applies to the case where .
Finally, if both  and  are empty, then only  and  may be required. Therefore,
we have the following corollary:

\begin{corollary}\label{cor-finite-rep} Let  be a set of  disjoint disks.
There exists a set of at most  points that represents .
\end{corollary}

\begin{figure}
\centering
\includegraphics{witness-constr.pdf}\caption{Proof of Lemma~\ref{lemma-witness-constr}.}\label{fig-witness-constr}
\end{figure}

\begin{lemma}\label{lemma-witness-constr} Let  and  be two disjoint disks and  be a
disk intersecting both of them. Then there exists a disk  inside  that is tangent to both 
and .
\end{lemma}
\begin{proof} We show how to construct . Let  and  respectively be the
centers and radii of  and . Without loss of generality, assume . Let
 be the disk centered at  and having radius  (see
Figure~\ref{fig-witness-constr}). The disk  is tangent to . If  is also tangent to ,
then let  and we are done. Otherwise,  is properly intersecting . In that case, let 
be the tangency point of  and ,  be the line through  and , and  be the disk
through  having its center on  and tangent to . The result follows from the fact that 
is tangent to  and inside .
\end{proof}


\begin{figure}
\centering
\includegraphics{dist-points.pdf}\caption{The distance points of  and
.}\label{fig-dist-points}
\end{figure}

\begin{definition} Let  and  be two disks in the plane. The
\emph{distance points} of  and  are the two ends of the shortest line segment between  and
 (see Figure~\ref{fig-dist-points}). If  is a set of disjoint disks, then the set
of \emph{distance points} of  is the set containing the distance points of every pair
of disks in .
\end{definition}

\begin{figure}
\centering
\includegraphics{del-disk-thm.pdf}
\caption{Illustration of the proof of Theorem~\ref{thm-del-disks}.}
\end{figure}

\begin{theorem}\label{thm-del-disks} Let  be a set of  disjoint disks.
Then  is a -spanner of , where  is the spanning ratio of the
Delaunay triangulation of a set of points.
\end{theorem}
\begin{proof}
By Corollary~\ref{cor-finite-rep}, let  be a set of size at most  that represents
, let  be the set of distance points of , and let . Since
 is a -spanner of , by Lemma~\ref{thm-quotient}, we have  is a
-spanner of , where  is the complete graph with vertex set .
Since  contains the distance points of ,
 is isomorphic to the complete graph defined on . We show that each
edge  of  is in . Let  be an edge of
. This means that in , there are two points  and  with  such that there is an empty circle  through  and . By Lemma~\ref{lemma-witness-constr},
 contains a disk  that is tangent to both  and . The disk  is a witness of the
presence of the edge  in . If that was not the case, this would mean that
there exists a disk  such that . By definition of , this
implies that  and thus , which contradicts the fact
that  is an empty circle. Therefore, the edge  is in .
Since  is a -spanner of  and a subgraph of
, we conclude that  is a -spanner of .
\end{proof}



\begin{figure}
\centering\includegraphics{not-plane.pdf}\caption{Even if the embedding of the AW-Delaunay graph
that consists of straight line segments between the centers of the disks is not necessarily a plane
graph, it is planar.}\label{fig-not-plane}
\end{figure}


Note that the embedding of the AW-Delaunay graph that consists of straight line segments between
the centers of the disks is not necessarily a plane graph (see Figure~\ref{fig-not-plane}).
However, the Voronoi diagram of a set of disks , denoted , is
planar~\cite{okabe00}. Since  is the face-dual of , it is
also planar. An important characteristic of the Delaunay graph of a set of points regarded as a
spanner is that it is a plane graph. Therefore, a natural question is whether 
has a plane embedding that is also a spanner.

The proof of Theorem~\ref{thm-del-disks} suggests the existence of an algorithm allowing to compute
such an embedding: compute the Delaunay triangulation of the set  that contains the distance
points and the representative of . The graph  can be regarded as a multigraph
whose vertex set is . Then, for each pair of disks that share one or more edges, just
keep the shortest of those edges. This simple algorithm allows to compute a plane embedding of
 that is also a spanner of . However, its running time is
. Whether or not it is possible to compute a plane embedding of 
that is also a spanner of  in a better running time remains a open question.



\old{

In this section, we show that  has a plane embedding that has the same spanning
ratio as the Delaunay graph of a set of points and that given  (which can be
computed in time ), this embedding can be computed in time .

\begin{definition} Let  be a finite set of disjoint disks, and 
such that the edge  is in . The points  are \emph{witnesses} of
the edge  if there is a disk  such that
\begin{enumerate}
\item  and  and
\item  does not intersect any other disk of .
\end{enumerate}
They are also \emph{minimum witnesses} if the distance between them is minimum. Notice that in that
case, the disk  has minimum radius and is tangent to  and . A set of points  is a
\emph{(minimum) witness} of  if it contains (minimum) witnesses for all edges in
.
\end{definition}

\begin{lemma} Let  be a finite set of disjoint disks. Given ,
a minimum witness set of  can be computed in time .
\end{lemma}
\begin{proof}
We show that given , minimum witnesses can be found for each edge of
 in amortized constant time. Let  such that the edge 
is in . Notice that any disk that is tangent to both  and  has its center
on the bisector  of  and . Let  be the disk that is tangent to both  and  whose
center  is the intersection of  with the line through the centers of  and . If  is
on the common boundary of the Voronoi regions of  and , then since  has minimum radius
among all disks that are tangent to both  and , the tangency points of  with  and 
are minimum witnesses of the edge . Otherwise, let  be the vertices defining the common
boundaries of the Voronoi regions of  and  (note that all points of  are on ). The
center of the disk defining the minimum witnesses of  is in . Since 
has linear complexity~\cite{fortune87},  has constant size on average. The result follows from
the fact that since  is planar, it has a linear number of edges.
\end{proof}

\begin{theorem}\label{reverse-prop} Let  be a finite set of disjoint disks.
Then  has a planar embedding that is a -spanner of , where 
is the spanning ratio of the Delaunay triangulation of a set of points. Moreover, given
, this embedding can be computed in time .
\end{theorem}
\begin{proof} The algorithm proceeds as follows: for every edge  of ,
draw an edge  where  form a pair of minimum witnesses of the edge .
Let  be the set of all minimum witnesses and  the resulting graph. The only difference
between  and  is the length of the edges.

To show that  is a -spanner of , consider the set  that is the union of a
finite set that represents  and the distance points of . Let
 and  be the distance points of  and . In ,
there is a -spanning path  from  to . The existence of this path implies the
existence of a -spanning path in  from  to : Let  be an edge of . We show
that  contains an edge whose adjacent vertices are on the same disks as  and whose length is
at most . The only case that we need to consider is when the two endpoints of  belong to
two different disks  and . In that case, from the proof of Theorem~\ref{thm-del-disks},
the edge  is in  and the endpoints of  are witnesses of that edge.
By definition of , it contains an edge  where  and  are minimum witnesses
of the edge . Since it is minimum, its length is at most , which completes the
proof.
\end{proof}

}

\section{Conclusion}\label{section-disk-del-conclusion}

In this paper, we showed how, given a weighted point set where weights are positive and
 for all , it is possible to compute a -spanner of
that point set that has a linear number of edges. We also showed that the Additively Weighted
Delaunay graph is a -spanner of an additively weighted point set in the same case. The constant
 is the same as for the Delaunay triangulation of a point set (the best current value is
2.42~\cite{keil92}). We could not see how the Well-Separated Pair Decomposition (WSPD) can be
adapted to solve that problem. The first difficulty resides in the fact that it is not even clear
that, given a weighted point set, a WSPD of that point set always exists. Other obvious open
questions are whether our results still hold when some weights are negative or 
for some . Also, we did not verify whether our variant of the Yao graph can be computed in
time . Finally, another problem that could be explored is whether it is possible to
compute -spanners for multiplicatively weighted point sets.

\bibliographystyle{myBibliographyStyle}
\bibliography{add-weighted-spanners}

\end{document}
