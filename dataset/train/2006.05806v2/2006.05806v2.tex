\documentclass{article}

\PassOptionsToPackage{numbers, compress}{natbib}




\usepackage[preprint]{neurips_2020}





\usepackage[utf8]{inputenc} \usepackage[T1]{fontenc}    \usepackage{hyperref}       \usepackage{url}            \usepackage{booktabs}       \usepackage{amsfonts}       \usepackage{nicefrac}       \usepackage{microtype}      

\usepackage{lipsum}		\usepackage{amsmath}
\usepackage{amsthm}
\usepackage{amssymb}
\usepackage{algpseudocode}
\usepackage{algorithm}
\usepackage{amsmath}
\usepackage{nicefrac}
\usepackage[toc,page]{appendix}
\newtheorem{theorem}{Theorem}
\newtheorem{lemma}{Lemma}
\newtheorem{property}{Property}
\newtheorem{proposition}{Proposition}
\newcommand{\norm}[1]{\lVert#1\rVert_2}
\usepackage{graphicx}
\usepackage{bm}
\graphicspath{ {images/} }
\usepackage{epstopdf}
\usepackage{comment}
\usepackage{color}


\newcommand{\Le}[1]{{\color{red}{\bf\sf [Le: #1]}}}

\title{\Large Bandit Samplers for Training Graph Neural Networks}




\newcommand*\samethanks[1][\value{footnote}]{\footnotemark[#1]}
\author{Ziqi Liu\thanks{Equal Contribution.} \\
  Ant Financial Services Group \\
  \texttt{ziqiliu@antfin.com} \\
  \And
  Zhengwei Wu\samethanks \\
  Ant Financial Services Group \\
  \texttt{zejun.wzw@antfin.com} \\
  \AND
  Zhiqiang Zhang \\
  Ant Financial Services Group \\
  \texttt{lingyao.zzq@antfin.com} \\
  \And
  Jun Zhou \\
  Ant Financial Services Group \\
  \texttt{jun.zhoujun@antfin.com} \\
  \And
  Shuang Yang \\
  Ant Financial Services Group \\
  \texttt{shuang.yang@antfin.com} \\
  \And
  Le Song \\
  Ant Financial Services Group \\
  Georgia Institute of Technology \\
  \texttt{lsong@cc.gatech.edu} \\
  \And
  Yuan Qi \\
  Ant Financial Services Group \\
  \texttt{yuan.qi@antfin.com} 
}

\begin{document}

\maketitle

\begin{abstract}
Several sampling algorithms with variance reduction have been 
proposed for accelerating the training of Graph Convolution Networks (GCNs). 
However, due to the intractable computation of optimal sampling distribution,
these sampling algorithms are suboptimal for GCNs and are not
applicable to more general graph neural networks (GNNs) where 
the message aggregator contains learned weights rather than
fixed weights, such as Graph Attention Networks (GAT). 
The fundamental reason is that the embeddings of the neighbors or learned weights involved in the optimal sampling distribution
are \emph{changing} during the training and \emph{not known a priori}, 
but only \emph{partially observed} when sampled, thus 
making the derivation of an optimal variance reduced samplers non-trivial. 
In this paper, we formulate the optimization of the sampling 
variance as an adversary bandit problem, where the rewards are related to the node embeddings and learned weights, 
and can vary constantly. Thus a good sampler needs to acquire 
variance information about more neighbors (exploration) while at the same time optimizing the immediate sampling variance (exploit). We theoretically
show that our algorithm asymptotically approaches the optimal variance within a factor of 3. We show the efficiency and effectiveness of our approach on multiple
datasets.
\end{abstract}

\section{Introduction}
Graph neural networks~\cite{kipf2016semi,hamilton2017inductive} have emerged as a powerful tool for 
representation learning of 
graph data in irregular or non-euclidean 
domains~\citep{battaglia2018relational,wu2019comprehensive}. 
For instance, graph neural networks have demonstrated 
state-of-the-art performance on learning tasks
such as node classification, link and graph property 
prediction, with applications ranging from drug 
design~\cite{dai2016discriminative}, 
social networks~\cite{hamilton2017inductive}, 
transaction networks~\cite{liu2018heterogeneous}, 
gene expression networks~\cite{fout2017protein}, and 
knowledge graphs~\cite{schlichtkrull2018modeling}.

One major challenge of training GNNs comes from
the requirements of heavy floating point operations and 
large memory footprints, due to the recursive
expansions over the neighborhoods.
For a minibatch with a single vertex , 
to compute its embedding  at 
the -th layer, we have to expand
its neighborhood from the -th layer to
the -th layer, i.e. -hops neighbors. That will
soon cover a large portion of the
graph if particularly the graph is dense.
One basic idea of alleviating such ``neighbor explosion'' problem
was to sample neighbors in a top-down
manner, i.e. sample neighbors in the 
-th layer given the nodes in the 
-th layer recursively.

Several layer sampling 
approaches~\cite{hamilton2017inductive,chen2018fastgcn,huang2018adaptive,zou2019layer} 
have been proposed to alleviate above 
``neighbor explosion'' problem and improve the 
convergence of training GCNs, e.g. with importance
sampling. 
However, the optimal sampler~\cite{huang2018adaptive}, 
 
for vertex ,
to minimize the variance of the estimator  
involves all its neighbors' hidden 
embeddings, i.e. , 
which is infeasible to be computed because we can only observe
them partially while doing sampling. Existing 
approaches~\cite{chen2018fastgcn,huang2018adaptive,zou2019layer}
typically compromise the optimal sampling distribution via approximations,
which may impede the convergence. Moreover, such approaches are 
not applicable to more general cases
where the weights or kernels 's are not known a priori,
but are learned weights parameterized by 
attention functions~\cite{velivckovic2017graph}. 
That is, both the hidden embeddings and learned weights
involved in the optimal sampler constantly
\emph{vary} during the training process, and only \emph{part} of 
the unnormalized attention values or hidden embeddings
can be observed while do sampling.

{\bfseries Present work}. We derive novel variance reduced 
samplers for training of GCNs and attentive GNNs with a 
fundamentally different perspective. That is, different 
with existing approaches that need to compute the 
immediate sampling distribution, 
we maintain nonparametric estimates
of the sampler instead, and update the 
sampler towards optimal variance after we 
acquire partial knowledges 
about neighbors being sampled, as the algorithm
iterates.

To fulfil this purpose, we formulate the optimization
of the samplers as a bandit problem, where
the regret is the gap between expected
loss (negative reward) under current policy (sampler) and expected
loss with optimal policy. We define the reward 
with respect to each action, i.e. the choice 
of a set of neighbors with sample size ,
as the derivatives of the sampling variance, and show
the variance of our samplers asymptotically approaches
the optimal variance within a factor of .
Under this problem formulation, we propose two bandit algorithms.
The first algorithm based on multi-armed bandit (MAB) chooses 
 arms (neighbors) repeatedly.
Our second algorithm based on MAB with multiple plays
chooses a combinatorial set of neighbors with size  only once.


To summarize,
(\textbf{1}) We recast the sampler for GNNs as a bandit 
problem from a fundamentally different perspective. It works
for GCNs and attentive GNNs while existing approaches apply
only to GCNs.
(\textbf{2}) We theoretically show that the regret with 
respect to the variance of our estimators asymptotically 
approximates the optimal sampler within a factor of 3 while
no existing approaches optimize the sampler. 
(\textbf{3}) We empirically show that our approachs are way competitive in
terms of convergence and sample variance, compared with 
state-of-the-art approaches on multiple public datasets.

\section{Problem Setting}

Let  denote the graph with 
 nodes , and edges 
. Let the adjacency matrix 
denote as . 
Assuming the feature matrix  
with  denoting the -dimensional feature 
of node . We focus on the following simple 
but general form of GNNs:

where  is the hidden embedding of node  at the -th layer,
 is a kernel or weight matrix,  
 is the 
transform parameter on the -th layer, and  
is the activation function. The weight 
, or  for simplicity, 
is non-zero only if  is in the -hop 
neighborhood  of . It varies with 
the aggregation 
functions~\cite{battaglia2018relational,wu2019comprehensive}. 
For example, 
\textbf{(1)} GCNs~\cite{dai2016discriminative,kipf2016semi}~define fixed weights as  or  respectively, where , and  is the diagonal node degree matrix of .
    \textbf{(2)} The attentive GNNs~\cite{velivckovic2017graph,liu2019geniepath} define a learned weight  by attention functions:
, where the unnormalized
attentions
    ,
are parameterized by . Different from GCNs, the learned weights  can be evaluated only given all the unnormalized weights in the neighborhood.


The basic idea of layer sampling 
approaches~\cite{hamilton2017inductive,chen2018fastgcn,huang2018adaptive,zou2019layer}
was to recast the evaluation of Eq.~\eqref{eq:gnn} as

where , and 
. Hence
we can evaluate each node  at the -th layer,
using a Monte Carlo
estimator with sampled neighbors at the -th layer.
Without loss of generality, we assume 
and 
that meet the setting of attentive GNNs
in the rest of this paper.
To further reduce the variance, let us
consider the following importance sampling

where we use  to include transform parameter  into
the function  for conciseness.
As such, one can find an alternative sampling distribution 
 
to reduce the variance of an estimator,
e.g. a Monte Carlo estimator ,
where . 

Take expectation over , we define the variance of 

at step  and -th layer to be:

Note that  and  that are
inferred during training may
vary over steps 's. 
We will explicitly include step  and layer  only 
when it is necessary.
By expanding Eq.~\eqref{eqn:pseudo-variance} one can write 
 as the difference of two terms. The first is a 
function of , which we refer to as the 
\textit{effective variance}:

while the second does not depend on , and we denote it by 
.
The optimal sampling distribution~\cite{chen2018fastgcn,huang2018adaptive} at -th layer for vertex  that minimizes the variance is:

However, evaluating this sampling distribution is
infeasible because we cannot have all the knowledges
of neighbors' embeddings in the denominator of Eq.~\eqref{eq:optimal_q}. Moreover, the 's
in attentive GNNs could also vary during the training
procedure. Existing layer sampling approaches based
on importance sampling just ignore the effects of
norm of embeddings and assume the 's are fixed
during training. As a result, the sampling distribution is 
suboptimal and only applicable to GCNs where the weights
are fixed. Note that our derivation above follows the setting
of node-wise sampling approaches~\cite{hamilton2017inductive}, 
but the claim remains to hold for layer-wise 
sampling approaches~\cite{chen2018fastgcn,huang2018adaptive,zou2019layer}.

\begin{comment}
``Layer sampling'' approaches~\cite{chen2018fastgcn,huang2018adaptive} aim to sample neighbors
in a top-down manner, however, it is intractable 
to compute the above optimal sampling distribution 
because it requires evaluating the whole neighbors that are not known a priori,
and the neighbors' embedding  
or even weight  in attentive GNNs vary constantly
during the training procedure. FastGCN~\cite{chen2018fastgcn} calculates their
sampling distribution by simply ignoring the norm of neighbors'
embeddings, and AS-GCN~\cite{huang2018adaptive} approximates the neighbors' embeddings
via a linear mapping to the oracle.
\end{comment}


\section{Related Works}
We summarize three types of works for training graph neural networks.

First, several ``\textit{layer sampling}'' 
approaches~\cite{hamilton2017inductive,chen2018fastgcn,huang2018adaptive,zou2019layer} have been 
proposed to alleviate the ``neighbor explosion'' problems. 
Given a minibatch of labeled vertices at each 
iteration, such approaches sample neighbors layer by 
layer in a top-down manner.
Particularly, node-wise samplers~\cite{hamilton2017inductive} randomly sample
neighbors in the lower layer given each node in the upper layer, 
while layer-wise samplers~\cite{chen2018fastgcn,huang2018adaptive,zou2019layer} 
leverage importance sampling to sample neighbors in the lower
layer given all the nodes in upper layer with sample sizes
of each layer be independent of each other. 
Empirically, the layer-wise samplers work even 
worse~\cite{chen2017stochastic}
compared with node-wise samplers, and
one can set an appropriate sample size for each
layer to alleviate the growth issue of node-wise samplers.
In this paper, we focus on optimizing the variance
in the vein of layer sampling approaches.
Though the derivation of our bandit samplers follows 
the node-wise samplers, it can be extended
to layer-wise. We leave this extension 
as a future work.

Second, Chen et al.~\cite{chen2017stochastic} proposed 
a variance reduced estimator by maintaining 
historical embeddings of each vertices,
based on the assumption that the embeddings of 
a single vertex would be
close to its history. This estimator uses a simple 
random sampler and works efficient in practice at
the expense of requiring an extra storage that
is linear with number of nodes.

Third, two ``\textit{graph sampling}'' 
approaches~\cite{chiang2019cluster,zeng2019graphsaint} 
first cut the graph into partitions~\cite{chiang2019cluster} or 
sample into subgraphs~\cite{zeng2019graphsaint}, then they train 
models on those partitions or subgraphs in a batch 
mode~\cite{kipf2016semi}.
They show that the training time of each epoch may be much faster
compared with ``layer sampling'' approaches.
We summarize the drawbacks as follows. 
First, the partition of the original graph
could be sensitive to the training problem. Second, these
approaches assume that all the vertices in the graph
have labels, however, in practice 
only partial vertices may have labels~\cite{hu2019cash,liu2018heterogeneous}.



{\bfseries GNNs Architecture.} 
For readers who are interested in the works related to the architecture of GNNs, please refer to
the comprehensive survey~\cite{wu2019comprehensive}.
Existing sampling approaches works only on GCNs, but
not on more advanced architectures like GAT~\cite{velivckovic2017graph}.

\section{Variance Reduced Samplers as Bandit Problems}\label{sec:problem}
We formulate the optimization of
sampling variance as a 
bandit problem. Our basic idea is that instead of 
explicitly calculating the intractable
optimal sampling distribution in Eq.~\eqref{eq:optimal_q} 
at each iteration, 
we aim to optimize a sampler or \textbf{policy}  for each vertex
 over the horizontal steps , and make the
variance of the estimator following this sampler
asymptotically approach the optimum 
,
such that 
 
for some constant . 
Each \textbf{action} of policy  is a choice of any 
subset of neighbors 
 where .
We denote  as the probability 
of the action that  chooses  at .
The gap to be minimized between
effective variance and the oracle is

Note that the function 
 is convex w.r.t
, hence for  and  we have the 
upper bound derived on right hand of Eq.~\eqref{eq:bandit_obj}.
We define this upper bound as \textbf{regret} at ,
which means the expected loss (negative
reward) with policy  minus the expected loss with
optimal policy . 
Hence the \textbf{reward} 
w.r.t choosing  at  is the negative 
derivative of the effective variance
.

In the following, we adapt this bandit problem in the adversary bandit 
setting~\cite{auer2002nonstochastic} because the rewards
vary as the training proceeds and do not follow a priori
fixed distribution~\cite{burtini2015survey}. We leave the
studies of other bandits as a future work.
We show in section~\ref{sec:regret_analysis} that 
with this regret the variances of our estimators 
asymptotically approach the optimal 
variance within a factor of .

Our samplers sample -element subset of neighbors  times  
or a -element subset of neighbors once
from the alternative sampling distribution 

for each vertex .
We instantiate above framework under two bandit settings.
\textbf{(1)} In the adversary
MAB setting~\cite{auer2002nonstochastic}, 
we define the sampler  as , that 
samples exact an \textbf{arm} (neighbor) 
 from . 
In this case the set  is
the 1-element subset . To have a sample size of 
neighbors, we repeat this action  times.
After we collected  rewards
 
we update  by 
\textbf{EXP3}~\cite{auer2002nonstochastic}.
\textbf{(2)} In the adversary MAB with multiple plays 
setting~\cite{uchiya2010algorithms},
it uses an efficient -combination sampler 
(\textbf{DepRound}~\cite{gandhi2006dependent})
 to sample any -element subset 
 that satisfies
,
where  corresponds to the alternative probability
of sampling . As such, it allows us
to select from a set of 
distinct \textbf{subsets of arms}
from  arms at once.
The selection can be done in . 
After we collected the reward 
,
we update  by \textbf{EXP3.M}~\cite{uchiya2010algorithms}.

{\bfseries Discussions.} We have to select a sample
size of  neighbors in GNNs. Note that in the 
rigorous bandit setting, exact one action should 
be made and followed by updating
the policy. In adversary MAB, we do the selection  times and 
update the policy, hence strictly speaking applying MAB to our
problem is not rigorous. Applying MAB with multiple plays 
to our problem is rigorous because it allows
us to sample  neighbors at once and update
the rewards together. For readers who are interested in 
EXP3, EXP3.M and DepRound, please find them in 
Appendix~\ref{appendix:alg}.


\begin{algorithm}
\caption{Bandit Samplers for Training GNNs.}
\label{alg:train_gnn}
\begin{algorithmic}[1]
\Require step , sample size , number of layers , node features , adjacency matrix .
\State \textbf{Initialize:} , 
. 
\For { to }
\State Read a minibatch of labeled vertices at layer .
\State Use sampler  or \textbf{DepRound} to sample neighbors top-down with sample size .
\State Forward GNN model via estimators defined in Eq.~\eqref{eq:estimator1} or Proposition~\ref{proposition:estimator2}.
\State Backpropagation and update GNN model.
\For{each  in the -st layer} 
\State Collect 's  sampled neighbors
, and rewards . 
\State Update  and  by \textbf{EXP3} 
or \textbf{EXP3.M}.
\EndFor
\EndFor
\State \Return GNN model.
\end{algorithmic}
\end{algorithm}

\section{Algorithms}
The framework of our algorithm is: \textbf{(1)} 
use a sampler  to sample  arms from the 
alternative sampling distribution  for any vertex ,
\textbf{(2)} establish the \emph{unbiased 
estimator}, \textbf{(3)} do feedforward and 
backpropagation, and finally \textbf{(4)}
\emph{calculate the rewards} and \emph{update the alternative
sampling distribution}
with a proper bandit algorithm. We show this framework
in Algorithm~\ref{alg:train_gnn}. Note that
the variance w.r.t  in Eq.~\eqref{eqn:pseudo-variance} 
is defined only at the -th layer, hence
we should maintain multiple 's at each layer.
In practice, we find that maintain a single  
and update it only using rewards from the -st 
layer works well enough. The time complexity of our algorithm
is same with any node-wise 
approaches~\cite{hamilton2017inductive}. In addition, it requires
a storage in  to maintain nonparametric
estimates 's.

It remains to instantiate the estimators, variances
and rewards related to our two bandit settings.
We name our first algorithm \textbf{GNN-BS} under
adversary MAB setting, and the second \textbf{GNN-BS.M}
under adversary MAB with multiple plays setting.
We first assume the weights 's are fixed,
then extend to attentive GNNs that
's change.







\subsection{GNN-BS: Graph Neural Networks with Bandit Sampler}\label{sec:bs}


In this setting, we choose  arm and 
repeat  times. We have the
following Monte Carlo estimator 

This yields the variance

Following Eq.~\eqref{eqn:effective-variance} and Eq.~\eqref{eq:bandit_obj}, 
we have the reward of  picking neighbor  at step  as 



\begin{comment}
We describe our training algorithm in Algorithm~\ref{alg:train_gnn}.
At each iteration , we read a minibatch of labeled vertices.
We randomly sample the layers in a top-down manner by using our
maintained sampling distribution 's. 
We do feedward by using our Monte Carlo estimator~\eqref{eq:estimator1}.
After we update the GNN model, 
we collect rewards defined in Eq.~\eqref{eq:reward1}, 
and update the sampling distribution 's 
using \textbf{EXP3}~\cite{auer2002nonstochastic}, which 
is a simple adversary MAB algorithm.
The parameter  in EXP3 controls how fast the 
sampling distribution  deviates from the uniform,
and  controls how fast the probability mass 
on a single location can change.
\end{comment}

\begin{comment}
\begin{algorithm}
\caption{DepRound}
\label{alg:dep_round}
\begin{algorithmic}[1]
\State \textbf{Input:} Sample size , sample distribution  with 
\State \textbf{Output:} Subset of  with  elements
\While{there is an  with }
    \State Choose distinct  and  with  and 
    \State Set  and  
    \State Update  and  as
    
\EndWhile
\State \textbf{return} 
\end{algorithmic}
\end{algorithm}
\end{comment}

\subsection{GNN-BS.M: Graph Neural Networks with Multiple Plays Bandit Sampler}\label{sec:bsm}


Given a vertex , an important property of DepRound is that
it satisfies 
, 
where  is any subset of size .
We have the following unbiased estimator.

\begin{proposition}\label{proposition:estimator2}
 
is the unbiased estimator of  
given that  is sampled from  using the DepRound 
sampler , where  is 
the selected -subset neighbors of vertex .
\end{proposition}
The effective variance of this estimator is 
.
Since the derivative of this effective variance w.r.t
 does not factorize, we instead have 
the following approximated effective variance 
using Jensen's inequality.
\begin{proposition}\label{proposition:mp_var_bound}
The effective variance can be approximated by .
\end{proposition}
\begin{proposition}\label{proposition:mp_derivative}
The negative derivative of the approximated effective variance  w.r.t ,
i.e. the reward of  choosing  at  is
.
\end{proposition}
Follow EXP3.M we use the reward w.r.t each arm as 
.
Our proofs rely on the property of DepRound introduced above.


\begin{comment}
We describe our second training algorithm in Algorithm~\ref{alg:train_gnn}.
We sample the layers
in a top-down manner with DepRound. We calculate the unbiased estimator
using Proposition~\ref{proposition:estimator2}.
After we update the GNN model, we calculate the reward derived 
in Eq.~\eqref{eq:reward2}. Finally we update the sampling distribution based 
on Algorithm~\ref{alg:exp3m} (\textbf{EXP3.M}~\cite{uchiya2010algorithms}) 
in Appendix~\ref{appendix:alg}.
\end{comment}


\subsection{Extension to Attentive GNNs}
\begin{comment}
Another advantage of our formulation is that, our samplers
can be naturally applied to attentive graph neural networks
where the attention values 's vary constantly
during the training procedure. Existing approaches cannot
evaluate Eq.~\eqref{eq:optimal_q} since they need to calculate
those 's before calculating the sampling distribution.
However, in our case, as long as our reward or regret 
can factorize with respect to hidden embedding 
or these learned weights , our approaches
can still apply to such architectures.
\end{comment}

In this section, we extend our algorithms to
attentive GNNs.
The issue remained is that the attention value  
can not be evaluated with only sampled neighborhoods,
instead, we can only compute the unnormalized attentions .
We define the adjusted feedback attention values as follows:

where 's are the unnormalized attention values that can be obviously
evaluated when we have sampled .
We use  as a surrogate of 

so that we can approximate the truth attention values  by our
adjusted attention values .


\begin{comment}
\begin{algorithm*}[ht]
\caption{Node-wise bandit sampling}
\label{alg:node_wise_algo}
\begin{algorithmic}[1]
\Require , sample size , neighbor size , 
\State \textbf{Initialize:} , 
, 
,

\For { to }
	\State Take a minibatch of nodes in the last layer, 
then sample  neighbors in layer  for each node  in layer , with
		.
	\State Update model to receive rewards in the last layer:
.
	\State For  set
	
	
	\If {}
		\State Decide  so as to satisfy
		
		\State Set  and  for  
                \Else
		\State Set 
                \EndIf
	\State Set
		
	\State Set
		 for
		
\EndFor
\end{algorithmic}
\end{algorithm*}
\end{comment}


\section{Regret Analysis}\label{sec:regret_analysis}
As we described in section~\ref{sec:problem}, the regret is defined as . By choosing the reward as the negative derivative of the effective variance, we have the following theorem that our bandit sampling algorithms asymptotically approximate the optimal variance within a factor of 3.

\begin{theorem}
Using Algorithm~\ref{alg:train_gnn} with  
and  
to minimize the effective variance with respect 
to , we have 

where , .
\end{theorem}
Our proof follows \cite{salehi2017stochastic} by upper and lower bounding the potential function. The upper and lower bounds are the functions of the alternative sampling probability  and the reward  respectively. By multiplying the upper and lower bounds by the optimal sampling probability  and using the reward definition in \eqref{eq:reward1}, we have the upper bound of the effective variance.
The growth of this regret is sublinear in terms of .
The regret decreases in polynomial as sample size  grows.
Note that the number
of neighbors  is always well bounded in pratical graphs,
and can be considered as a moderate constant number.
Compared with existing layer sampling 
approaches~\cite{hamilton2017inductive,chen2018fastgcn,zou2019layer}
that have a fixed variance given the specific estimators, 
this is the first work optimizing the sampling
variance of GNNs towards optimum.
We will empirically show the sampling variances in experiments.

\setlength{\tabcolsep}{1pt}
{\footnotesize
\begin{table*}
  \centering
  \caption{Dataset summary. ``s'' dontes multi-class task, and ``m'' denotes multi-label task.}
  \label{tb:data}
  \begin{tabular}{ccccccccc}
    Dateset & V & E & Degree &  \# Classes & \# Features & \# train & \# validation & \# test \\
    \midrule
    Cora&   &  &  &  (s) &  &  &  &  \\
    Pubmed &  &  &  &  (s)  &  &  &  &  \\
    PPI &     &  &  &  (m) &  &  &  & \\
    Reddit &  &  &  &  (s) &  &  &  & \\
    Flickr &  &  &  &  (s) &  &  &  & \\
  \bottomrule
\end{tabular}
\end{table*}
}


\section{Experiments}
In this section, we conduct extensive experiments compared with 
state-of-the-art approaches to show the advantage of our training
approaches. We use the following rule
to name our approaches: GNN architecture plus bandit sampler.
For example, \textbf{GCN-BS}, \textbf{GAT-BS} and 
\textbf{GP-BS} denote the training
approaches for GCN, GAT~\cite{velivckovic2017graph} 
and GeniePath~\cite{liu2019geniepath} respectively.

The major purpose of this paper is to compare the effects
of our samplers with existing training algorithms, so 
we compare them by training the same GNN architecture.
We use the following architectures unless otherwise stated.
We fix the number of layers as  as 
in~\cite{kipf2016semi} for all comparison algorithms.
We set the dimension of hidden embeddings as  for Cora and Pubmed,
and  for PPI, Reddit and Flickr. For a fair comparison, we 
do not use the normalization layer~\cite{ba2016layer}
particularly used in some 
works~\cite{chen2017stochastic,zeng2019graphsaint}.
For attentive GNNs, we use the attention layer proposed 
in GAT. we set the number of multi-heads 
as  for simplicity. 

We report results on  benchmark data that include
\textit{Cora}~\cite{sen2008collective}, 
\textit{Pubmed}~\cite{sen2008collective},
\textit{PPI}~\cite{hamilton2017inductive},
\textit{Reddit}~\cite{hamilton2017inductive},
and \textit{Flickr}~\cite{zeng2019graphsaint}.
We follow the standard data splits,
and summarize the statistics in Table~\ref{tb:data}.

{\tiny
\begin{table*}[h]
\caption{Comparisons on the GCN architecture: testing Micro F1 scores.}
\label{tb:bench-gcn}
\begin{center}
\begin{tabular}{llllll}
\toprule
\textbf{Method}  &\textbf{Cora} &\textbf{Pubmed} &\textbf{PPI}& \textbf{Reddit}& \textbf{Flickr} \\
\midrule
GraphSAGE &  & &  &  &  \\
FastGCN &  & &  &  &  \\
LADIES &  & &  &  &  \\
AS-GCN &  & &  &  &  \\
S-GCN &  & &  &  &  \\
ClusterGCN &  & &  &  &  \\
GraphSAINT &  & &  &  &  \\
\midrule
GCN-BS &  & &  &  &  \\
\bottomrule
\end{tabular}
\end{center}
\end{table*}
}

{\tiny
\begin{table*}[h]
\caption{Comparisons on the attentive GNNs architecture: testing Micro F1 scores.}
\label{tb:bench-gat}
\begin{center}
\begin{tabular}{llllll}
\toprule
\textbf{Method}  &\textbf{Cora} &\textbf{Pubmed} &\textbf{PPI}& \textbf{Reddit}& \textbf{Flickr} \\
\midrule
AS-GAT &  & &  & NA &  \\
GraphSAINT-GAT &  & &  &  &  \\
\midrule
GAT-BS &  & &  &  &  \\
GAT-BS.M &  & &  &  &  \\
GP-BS &  & &  &  &  \\
GP-BS.M &  & &  &  &  \\
\bottomrule
\end{tabular}
\end{center}
\end{table*}
\vspace{-0.1cm}
}

We summarize the comparison algorithms as follows.
\textbf{(1)} GraphSAGE~\cite{hamilton2017inductive} is
a node-wise layer sampling approach with a random 
sampler.
\textbf{(2)} FastGCN~\cite{chen2018fastgcn}, 
LADIES~\cite{zou2019layer}, and 
AS-GCN~\cite{huang2018adaptive} are
layer sampling approaches based on importance sampling.
\textbf{(3)} S-GCN~\cite{chen2017stochastic} can be 
viewed as an optimization solver for training of 
GCN based on a simply random sampler.
\textbf{(4)} ClusterGCN~\cite{chiang2019cluster} and
GraphSAINT~\cite{zeng2019graphsaint} are 
``graph sampling'' techniques that first partition 
or sample the graph into small subgraphs, then train 
each subgraph using the batch algorithm~\cite{kipf2016semi}.
\textbf{(5)} The open source algorithms that support the training
of attentive GNNs are AS-GCN and GraphSAINT. We denote
them as AS-GAT and GraphSAINT-GAT.

We do grid search for the following hyperparameters 
in each algorithm, i.e., the learning rate ,
the penalty weight on the -norm regularizers 
,
the dropout rate . By following the exsiting
implementations\footnote{Checkout: 
\url{https://github.com/matenure/FastGCN} or 
\url{https://github.com/huangwb/AS-GCN}}, 
we save the model based on the best
results on validation, and restore the model to report 
results on testing data in Section~\ref{sec:benchmark}.
For the sample size  in GraphSAGE, S-GCN and our algorithms, 
we set  for Cora and Pubmed,
 for Flickr,  for PPI and reddit.
We set the sample size in the first and second layer
for FastGCN and AS-GCN/AS-GAT as 
256 and 256 for Cora and Pubmed,
 and  for PPI,  and 
for Flickr, and  and  for Reddit.
We set the batch size of all the layer sampling approaches
and S-GCN as  for all the datasets. For ClusterGCN, 
we set the partitions
according to the suggestions~\cite{chiang2019cluster} 
for PPI and Reddit. We set the number of partitions 
for Cora and Pubmed as 10, for flickr as 200
by doing grid search. We set the architecture of GraphSAINT as 
``0-1-1''\footnote{Checkout \url{https://github.com/GraphSAINT/} 
for more details.} which means MLP layer followed by two 
graph convolution layers. We use the ``rw'' sampling 
strategy that reported as the best in their original paper
to perform the graph sampling procedure. We set the number 
of root and walk length as the paper suggested.

\begin{figure*}[h]
\includegraphics[width=0.33\textwidth]{cora-gcn}
\includegraphics[width=0.33\textwidth]{pubmed-gcn}
\includegraphics[width=0.33\textwidth]{ppi-gcn}
\includegraphics[width=0.33\textwidth]{reddit-gcn}
\includegraphics[width=0.33\textwidth]{flickr-gcn}
\includegraphics[width=0.33\textwidth]{cora-gat}
\includegraphics[width=0.33\textwidth]{pubmed-gat}
\includegraphics[width=0.33\textwidth]{ppi-gat}
\includegraphics[width=0.33\textwidth]{reddit-gat}
\vspace{-0.2cm}
\caption{The convergence on validation in terms of epochs.}
\label{fig:convergence_gcn}
\end{figure*}
\vspace{-0.2cm}


\begin{comment}
\begin{figure*}[h]
\includegraphics[width=0.26\textwidth]{cora-gat}
\includegraphics[width=0.26\textwidth]{pubmed-gat}
\includegraphics[width=0.26\textwidth]{ppi-gat}
\includegraphics[width=0.26\textwidth]{reddit-gat}
\includegraphics[width=0.26\textwidth]{flickr-gat}
\vspace{-0.2cm}
\caption{The convergence on validation in terms of epochs.}
\label{fig:convergence_gat}
\end{figure*}
\vspace{-0.2cm}
\end{comment}

\subsection{Results on Benchmark Data}\label{sec:benchmark}
We report the testing results on GCN and attentive
GNN architectures in Table~\ref{tb:bench-gcn} and 
Table~\ref{tb:bench-gat} respectively.
We run the results of each algorithm  times and
report the mean and standard deviation.
The results on the two layer GCN architecture show that our
GCN-BS performs the best on most of datasets. 
The results on the two layer attentive GNN architecture show
the superiority of our algorithms on training more complex
GNN architectures. GraphSAINT or AS-GAT cannot compute 
the softmax of learned weights, but
simply use the unnormalized weights to perform the aggregation.
As a result, most of results from AS-GAT and GraphSAINT-GAT
in Table~\ref{tb:bench-gat} are worse than their results
in Table~\ref{tb:bench-gcn}. Thanks to the power of 
attentive structures in GNNs, our algorithms perform
better results on PPI and Reddit compared with
GCN-BS, and significantly outperform the results from 
AS-GAT and GraphSAINT-GAT.






\subsection{Convergence}
In this section, we analyze the convergences of 
comparison algorithms on the two layer GCN and 
attentive GNN architectures in Figure~\ref{fig:convergence_gcn} 
in terms of epoch.
We run all the algorithms  times and show the mean and standard 
deviation.
Our approaches consistently converge to better results 
with faster rates and lower variances in most of
datasets like Pubmed, PPI, Reddit and Flickr compared with
the state-of-the-art approaches.
The GNN-BS algorithms perform very similar to GNN-BS.M, even though
strictly speaking GNN-BS does not follow the rigorous MAB setting.
Furthermore, we show a huge improvement on the training
of attentive GNN architectures compared with GraphSAINT-GAT
and AS-GAT.
The convergences on validation in terms of 
timing (seconds), compared with layer sampling
approaches, in Appendix~\ref{appendix:convergences} 
show the similar results. 
We further give a discussion about timing
among layer sampling approaches and graph sampling approaches
in Appendix~\ref{appendix:layer_vs_graph}.


\begin{figure}[h]
\includegraphics[width=0.49\textwidth]{sample_size_f1}
\includegraphics[width=0.49\textwidth]{sample_size_variance}
\caption{Comparisons on PPI by varying the sample sizes: 
(\textbf{left}) F1 score, (\textbf{right}) sample variances.}
\label{fig:sample_size}
\end{figure}



\subsection{Sample Size Analysis}
We analyze the sampling variances and accuracy
as sample size varies using PPI data.
Note that existing layer sampling approaches do not
optimize the variances once the 
samplers are specified. As a result, their
variances are simply fixed~\cite{zou2019layer}, while
our approaches asymptotically appoach the optimum.
For comparison, we train our models until convergence,
then compute the average sampling variances. 
We show the results in Figure~\ref{fig:sample_size}.
The results are grouped into two categories, i.e.
results for GCNs and attentive GNNs respectively.
The sampling variances of our approaches are smaller 
in each group, and even be smaller than the 
variances of S-GCN that leverages a variance reduction solver. 
This explains the
performances of our approaches on testing Micro F1 scores.
We also find that the overall sampling variances of node-wise
approaches are way better than those of layer-wise
approaches.



\section{Conclusions}
In this paper, we show that the optimal layer samplers
based on importance sampling for training general 
graph neural networks are computationally intractable,
since it needs all the neighbors' hidden embeddings 
or learned weights.
Instead, we re-formulate the sampling problem as a bandit problem
that requires only partial knowledges from neighbors being sampled.
We propose two algorithms based on multi-armed bandit and MAB with
multiple plays, and show the variance of our bandit sampler 
asymptotically approaches the optimum within a factor of .
Furthermore, our algorithms are not only applicable to 
GCNs but more general architectures like attentive GNNs.
We empirically show that our algorithms can converge to 
better results with faster rates and lower variances
compared with state-of-the-art approaches.











\begin{thebibliography}{25}
\providecommand{\natexlab}[1]{#1}
\providecommand{\url}[1]{\texttt{#1}}
\expandafter\ifx\csname urlstyle\endcsname\relax
  \providecommand{\doi}[1]{doi: #1}\else
  \providecommand{\doi}{doi: \begingroup \urlstyle{rm}\Url}\fi

\bibitem[Auer et~al.(2002)Auer, Cesa-Bianchi, Freund, and
  Schapire]{auer2002nonstochastic}
P.~Auer, N.~Cesa-Bianchi, Y.~Freund, and R.~E. Schapire.
\newblock The nonstochastic multiarmed bandit problem.
\newblock \emph{SIAM journal on computing}, 32\penalty0 (1):\penalty0 48--77,
  2002.

\bibitem[Ba et~al.(2016)Ba, Kiros, and Hinton]{ba2016layer}
J.~L. Ba, J.~R. Kiros, and G.~E. Hinton.
\newblock Layer normalization.
\newblock \emph{arXiv preprint arXiv:1607.06450}, 2016.

\bibitem[Battaglia et~al.(2018)Battaglia, Hamrick, Bapst, Sanchez-Gonzalez,
  Zambaldi, Malinowski, Tacchetti, Raposo, Santoro, Faulkner,
  et~al.]{battaglia2018relational}
P.~W. Battaglia, J.~B. Hamrick, V.~Bapst, A.~Sanchez-Gonzalez, V.~Zambaldi,
  M.~Malinowski, A.~Tacchetti, D.~Raposo, A.~Santoro, R.~Faulkner, et~al.
\newblock Relational inductive biases, deep learning, and graph networks.
\newblock \emph{arXiv preprint arXiv:1806.01261}, 2018.

\bibitem[Burtini et~al.(2015)Burtini, Loeppky, and Lawrence]{burtini2015survey}
G.~Burtini, J.~Loeppky, and R.~Lawrence.
\newblock A survey of online experiment design with the stochastic multi-armed
  bandit.
\newblock \emph{arXiv preprint arXiv:1510.00757}, 2015.

\bibitem[Chen et~al.(2017)Chen, Zhu, and Song]{chen2017stochastic}
J.~Chen, J.~Zhu, and L.~Song.
\newblock Stochastic training of graph convolutional networks with variance
  reduction.
\newblock \emph{arXiv preprint arXiv:1710.10568}, 2017.

\bibitem[Chen et~al.(2018)Chen, Ma, and Xiao]{chen2018fastgcn}
J.~Chen, T.~Ma, and C.~Xiao.
\newblock Fastgcn: fast learning with graph convolutional networks via
  importance sampling.
\newblock \emph{arXiv preprint arXiv:1801.10247}, 2018.

\bibitem[Chiang et~al.(2019)Chiang, Liu, Si, Li, Bengio, and
  Hsieh]{chiang2019cluster}
W.-L. Chiang, X.~Liu, S.~Si, Y.~Li, S.~Bengio, and C.-J. Hsieh.
\newblock Cluster-gcn: An efficient algorithm for training deep and large graph
  convolutional networks.
\newblock In \emph{Proceedings of the 25th ACM SIGKDD International Conference
  on Knowledge Discovery \& Data Mining}, pages 257--266, 2019.

\bibitem[Dai et~al.(2016)Dai, Dai, and Song]{dai2016discriminative}
H.~Dai, B.~Dai, and L.~Song.
\newblock Discriminative embeddings of latent variable models for structured
  data.
\newblock In \emph{International conference on machine learning}, pages
  2702--2711, 2016.

\bibitem[Fout et~al.(2017)Fout, Byrd, Shariat, and Ben-Hur]{fout2017protein}
A.~Fout, J.~Byrd, B.~Shariat, and A.~Ben-Hur.
\newblock Protein interface prediction using graph convolutional networks.
\newblock In \emph{Advances in Neural Information Processing Systems}, pages
  6530--6539, 2017.

\bibitem[Gandhi et~al.(2006)Gandhi, Khuller, Parthasarathy, and
  Srinivasan]{gandhi2006dependent}
R.~Gandhi, S.~Khuller, S.~Parthasarathy, and A.~Srinivasan.
\newblock Dependent rounding and its applications to approximation algorithms.
\newblock \emph{Journal of the ACM (JACM)}, 53\penalty0 (3):\penalty0 324--360,
  2006.

\bibitem[Hamilton et~al.(2017)Hamilton, Ying, and
  Leskovec]{hamilton2017inductive}
W.~Hamilton, Z.~Ying, and J.~Leskovec.
\newblock Inductive representation learning on large graphs.
\newblock In \emph{Advances in Neural Information Processing Systems}, pages
  1024--1034, 2017.

\bibitem[Hu et~al.(2019)Hu, Zhang, Shi, Zhou, Li, and Qi]{hu2019cash}
B.~Hu, Z.~Zhang, C.~Shi, J.~Zhou, X.~Li, and Y.~Qi.
\newblock Cash-out user detection based on attributed heterogeneous information
  network with a hierarchical attention mechanism.
\newblock In \emph{Proceedings of the AAAI Conference on Artificial
  Intelligence}, volume~33, pages 946--953, 2019.

\bibitem[Hu et~al.(2020)Hu, Fey, Zitnik, Dong, Ren, Liu, Catasta, and
  Leskovec]{hu2020open}
W.~Hu, M.~Fey, M.~Zitnik, Y.~Dong, H.~Ren, B.~Liu, M.~Catasta, and J.~Leskovec.
\newblock Open graph benchmark: Datasets for machine learning on graphs.
\newblock \emph{arXiv preprint arXiv:2005.00687}, 2020.

\bibitem[Huang et~al.(2018)Huang, Zhang, Rong, and Huang]{huang2018adaptive}
W.~Huang, T.~Zhang, Y.~Rong, and J.~Huang.
\newblock Adaptive sampling towards fast graph representation learning.
\newblock In \emph{Advances in Neural Information Processing Systems}, pages
  4558--4567, 2018.

\bibitem[Kipf and Welling(2016)]{kipf2016semi}
T.~N. Kipf and M.~Welling.
\newblock Semi-supervised classification with graph convolutional networks.
\newblock \emph{arXiv preprint arXiv:1609.02907}, 2016.

\bibitem[Liu et~al.(2018)Liu, Chen, Yang, Zhou, Li, and
  Song]{liu2018heterogeneous}
Z.~Liu, C.~Chen, X.~Yang, J.~Zhou, X.~Li, and L.~Song.
\newblock Heterogeneous graph neural networks for malicious account detection.
\newblock In \emph{Proceedings of the 27th ACM International Conference on
  Information and Knowledge Management}, pages 2077--2085. ACM, 2018.

\bibitem[Liu et~al.(2019)Liu, Chen, Li, Zhou, Li, Song, and
  Qi]{liu2019geniepath}
Z.~Liu, C.~Chen, L.~Li, J.~Zhou, X.~Li, L.~Song, and Y.~Qi.
\newblock Geniepath: Graph neural networks with adaptive receptive paths.
\newblock In \emph{Proceedings of the AAAI Conference on Artificial
  Intelligence}, volume~33, pages 4424--4431, 2019.

\bibitem[Salehi et~al.(2017)Salehi, Celis, and Thiran]{salehi2017stochastic}
F.~Salehi, L.~E. Celis, and P.~Thiran.
\newblock Stochastic optimization with bandit sampling.
\newblock \emph{arXiv preprint arXiv:1708.02544}, 2017.

\bibitem[Schlichtkrull et~al.(2018)Schlichtkrull, Kipf, Bloem, Van Den~Berg,
  Titov, and Welling]{schlichtkrull2018modeling}
M.~Schlichtkrull, T.~N. Kipf, P.~Bloem, R.~Van Den~Berg, I.~Titov, and
  M.~Welling.
\newblock Modeling relational data with graph convolutional networks.
\newblock In \emph{European Semantic Web Conference}, pages 593--607. Springer,
  2018.

\bibitem[Sen et~al.(2008)Sen, Namata, Bilgic, Getoor, Galligher, and
  Eliassi-Rad]{sen2008collective}
P.~Sen, G.~Namata, M.~Bilgic, L.~Getoor, B.~Galligher, and T.~Eliassi-Rad.
\newblock Collective classification in network data.
\newblock \emph{AI magazine}, 29\penalty0 (3):\penalty0 93--93, 2008.

\bibitem[Uchiya et~al.(2010)Uchiya, Nakamura, and Kudo]{uchiya2010algorithms}
T.~Uchiya, A.~Nakamura, and M.~Kudo.
\newblock Algorithms for adversarial bandit problems with multiple plays.
\newblock In \emph{International Conference on Algorithmic Learning Theory},
  pages 375--389. Springer, 2010.

\bibitem[Veli{\v{c}}kovi{\'c} et~al.(2017)Veli{\v{c}}kovi{\'c}, Cucurull,
  Casanova, Romero, Lio, and Bengio]{velivckovic2017graph}
P.~Veli{\v{c}}kovi{\'c}, G.~Cucurull, A.~Casanova, A.~Romero, P.~Lio, and
  Y.~Bengio.
\newblock Graph attention networks.
\newblock \emph{arXiv preprint arXiv:1710.10903}, 2017.

\bibitem[Wu et~al.(2019)Wu, Pan, Chen, Long, Zhang, and
  Yu]{wu2019comprehensive}
Z.~Wu, S.~Pan, F.~Chen, G.~Long, C.~Zhang, and P.~S. Yu.
\newblock A comprehensive survey on graph neural networks.
\newblock \emph{arXiv preprint arXiv:1901.00596}, 2019.

\bibitem[Zeng et~al.(2019)Zeng, Zhou, Srivastava, Kannan, and
  Prasanna]{zeng2019graphsaint}
H.~Zeng, H.~Zhou, A.~Srivastava, R.~Kannan, and V.~Prasanna.
\newblock Graphsaint: Graph sampling based inductive learning method.
\newblock \emph{arXiv preprint arXiv:1907.04931}, 2019.

\bibitem[Zou et~al.(2019)Zou, Hu, Wang, Jiang, Sun, and Gu]{zou2019layer}
D.~Zou, Z.~Hu, Y.~Wang, S.~Jiang, Y.~Sun, and Q.~Gu.
\newblock Layer-dependent importance sampling for training deep and large graph
  convolutional networks.
\newblock In \emph{Advances in Neural Information Processing Systems}, pages
  11247--11256, 2019.

\end{thebibliography}

\clearpage
\appendix


\section{Algorithms}\label{appendix:alg}
\begin{algorithm}
\caption{.}
\label{alg:exp3}
\begin{algorithmic}[1]
\Require , sample size , neighbor size , .
\State Set 
	
	
\State Set

\end{algorithmic}
\end{algorithm}

\begin{algorithm}[ht]
\caption{EXP3.M}
\label{alg:exp3m}
\begin{algorithmic}[1]
\Require , sample size , neighbor size , , .
	\State For  set
	
	
	\If {}
		\State Decide  so as to satisfy
		
		\State Set 
                \Else
		\State Set 
                \EndIf
	\State Set
		
	\State Set
		 for
		
\end{algorithmic}
\end{algorithm}

\begin{algorithm}
\caption{DepRound}
\label{alg:dep_round}
\begin{algorithmic}[1]
\State \textbf{Input:} Sample size , sample distribution  with 
\State \textbf{Output:} Subset of  with  elements
\While{there is an  with }
    \State Choose distinct  and  with  and 
    \State Set  and  
    \State Update  and  as
    
\EndWhile
\State \textbf{return} 
\end{algorithmic}
\end{algorithm}

\section{Proofs}\label{appendix:proof}
\setcounter{theorem}{0}
\setcounter{property}{0}
\setcounter{proposition}{0}

\begin{proposition}\label{proposition:estimator2}
 
is the unbiased estimator of  
given that  is sampled from  using the DepRound 
sampler , where  is 
the selected -subset neighbors of vertex .
\end{proposition}
\begin{proof}
Let us denote  as the probability of vertex 
 choosing any -element subset  
from the -element set  using DepRound sampler .
This sampler follows the alternative sampling distribution
 where  denotes
the alternative probability of sampling neighbor .
This sampler is guaranteed to satisfy , i.e. the sum over the probabilities of all subsets  that
contains element  equals the probability .

\end{proof}

\begin{proposition}\label{proposition:mp_var_bound}
The effective variance can be approximated by .
\end{proposition}
\begin{proof}
The variance is


Therefore the effective variance has following upper bound:

\end{proof}


\begin{proposition}\label{proposition:mp_derivative}
The negative derivative of the approximated effective variance  w.r.t ,
i.e. the reward of  choosing  at , is
.
\end{proposition}
\begin{proof}
Define the upper bound as ,
then its derivative is 

\end{proof}





Before we give the proof of Theorem~\ref{theorem:bs},
we first prove the following Lemma~\ref{lemma:1} that
will be used later.
\begin{lemma}\label{lemma:1}
	For any real value constant  and any valid distributions  and  we have
		
\end{lemma}
\begin{proof}
The function  is convex with respect to , hence for any  and  we have


Multiplying both sides of this inequality by , we have


In the following, we prove this Lemma in our two bandit settings: \textit{adversary MAB setting} and \textit{adversary MAB with multiple plays setting}.

In \textit{adversary MAB setting}, we have


In \textit{adversary MAB with multiple plays setting}, we use the approximated effective variance  derived in Proposition~\ref{proposition:mp_var_bound}. For notational simplicity, we denote the approximated effective variance as  in the following. We have

The equation~\eqref{eq:effective_variance_derivative} holds because of Proposition~\ref{proposition:mp_derivative}.

At last, we conclude the proof

\end{proof}


\begin{theorem}\label{theorem:bs}
	Using Algorithm~\ref{alg:train_gnn} with  and  to minimize effective variance with respect to , we have 
	
	where  and .
\end{theorem}

\begin{proof}
First we explain why condition  ensures that ,
{\footnotesize

}
Assuming , inequality~\eqref{eq:apx1} holds because  and . Then replace  by the condition, we get .

Let ,  denote ,  respectively. Then for any ,
{\footnotesize

}

Inequality~\eqref{eq:apx2} uses  for . Equality~\eqref{eq:apx14} holds because of update equation of  defined in EXP3.M. Inequality~\eqref{eq:apx3} holds because . Since  for , we have
{\footnotesize

}

If we sum, for , we get the following telescopic sum
{\footnotesize

}

On the other hand, for any subset  containing k elements,
{\footnotesize

}
The inequality~\eqref{eq:apx6} uses the fact that
{\footnotesize

}

The equation~\eqref{eq:apx7} uses the fact that
{\footnotesize

}

From~\eqref{eq:apx5} and ~\eqref{eq:apx7}, we get
{\footnotesize

}

And we have the following inequality
{\footnotesize

}
The equality~\eqref{eq:apx11} holds beacuse  when  and  bacause  for all .

Then add inequality~\eqref{eq:apx12} in ~\eqref{eq:apx13} we have
{\footnotesize

}
Given  we have , hence, taking expectation of~\eqref{eq:apx8} yields that
{\footnotesize

}
By multiplying~\eqref{eq:apx9} by  and summing over , we get
{\footnotesize

}
As
{\footnotesize

}

By plugging \eqref{eq:apx15} in~\eqref{eq:apx10} and rearranging it, we find
{\footnotesize

}
Using Lemma~\ref{lemma:1}, we have
{\footnotesize

}
Finally, we know that 
{\footnotesize

}
By setting  and , we get the upper bound.

\end{proof}





\section{Experiments}


\subsection{Convergences}\label{appendix:convergences}
We show the convergences on validation in terms of timing (seconds)
in Figure~\ref{fig:timing_gcn} and Figure~\ref{fig:timing_gat}. Basically, our algorithms converge
to much better results in nearly same duration compared with other
``layer sampling'' approaches.

Note that we cannot complete the training of AS-GAT on Reddit
because of memory issues.

\begin{figure*}[h]
\includegraphics[width=0.33\textwidth,height=0.33\textwidth]{t-cora-gcn}
\includegraphics[width=0.33\textwidth,height=0.33\textwidth]{t-pubmed-gcn}
\includegraphics[width=0.33\textwidth,height=0.33\textwidth]{t-ppi-gcn}
\includegraphics[width=0.33\textwidth,height=0.33\textwidth]{t-reddit-gcn}
\includegraphics[width=0.33\textwidth,height=0.33\textwidth]{t-flickr-gcn}
\caption{The convergence in timing (seconds) on GCNs.}
\label{fig:timing_gcn}
\end{figure*}

\begin{figure*}[h]
\includegraphics[width=0.24\textwidth,height=0.24\textwidth]{t-cora-gat}
\includegraphics[width=0.24\textwidth,height=0.24\textwidth]{t-pubmed-gat}
\includegraphics[width=0.24\textwidth,height=0.24\textwidth]{t-ppi-gat}
\includegraphics[width=0.24\textwidth,height=0.24\textwidth]{t-flickr-gat}
\caption{The convergence in timing (seconds) on attentive GNNs.}
\label{fig:timing_gat}
\end{figure*}
\subsection{Discussions on Timings between Layer Sampling and Graph Sampling Paradigms}\label{appendix:layer_vs_graph}

Note that the comparisons of 
timing between ``graph sampling'' and ``layer sampling'' 
paradigms have been studied recently in~\cite{chiang2019cluster,zeng2019graphsaint}.
As a result, we do not compare the timing with ``graph sampling'' approaches.
Under certain conditions, the graph sampling approaches 
should be faster than layer sampling approaches. That is, graph sampling approaches
are designed for graph data that all vertices have labels.
Under such condition, the floating point operations analyzed in~\cite{chiang2019cluster}
are maximally utilized compared with the ``layer sampling'' paradigm.
However, in practice, there are large amount of graph data with labels
only on some types of vertices, such as the graphs in~\cite{liu2018heterogeneous}.
``Graph sampling'' approaches
are not applicable to cases where only partial vertices have labels.
To summarize, the
``layer sampling'' approaches are more flexible
and general compared with ``graph sampling''
approaches in many cases.

\subsection{Results on OGB}
We report our results on 
OGB protein dataset~\cite{hu2020open}. We set the learning rate as 1e-3, batch size as 256, the dimension of hidden embeddings as 64, sample size as 10 and epochs as 200. We save the model based on the best results on validation and report results on testing data. We run the experiment 10 times with different random seeds to compute the average and standard deviation of the results. Our result of GP-BS on protein dataset performs the best\footnote{Please refer to \url{https://ogb.stanford.edu/docs/leader_nodeprop/}.}
until we submitted this paper. Please find our implementations at \url{https://github.com/xavierzw/ogb-geniepath-bs}.

\setlength{\tabcolsep}{1pt}
\begin{table}[H]
  \centering
  \begin{tabular}{cccc}
  	\toprule
    Dateset & Mean & Std & \#experiments \\
    \midrule
    ogbn-proteins&   &  &  \\
  \bottomrule
\end{tabular}
\end{table}


\end{document}
