\documentclass[a4paper]{article}


\usepackage{amssymb}
\usepackage{amsthm}
\usepackage{graphicx}
\usepackage{subfigure}
\usepackage{xspace}

\newtheorem{theorem}{Theorem}
\newtheorem{lemma}{Lemma}
\newtheorem{corollary}{Corollary}
\newtheorem{observation}{Observation}
\newtheorem{property}{Property}

\newcommand{\SA}{\includegraphics{figures/SA.pdf}~}
\newcommand{\SB}{\includegraphics{figures/SB.pdf}~}
\newcommand{\SC}{\includegraphics{figures/SC.pdf}~}
\newcommand{\SD}{\includegraphics{figures/SD.pdf}~}
\newcommand{\shapes}{\xspace}

\begin{document}



\title{L-Visibility Drawings of IC-planar Graphs
\thanks{Research supported in part by the MIUR project AMANDA ``Algorithmics for MAssive and Networked DAta''.}}
    
\author{Giuseppe Liotta and Fabrizio Montecchiani\\Universit{\`a} degli Studi di Perugia, Italy\\ \texttt{\small\{giuseppe.liotta,fabrizio.montecchiani\}@unipg.it}} 

\date{}

\maketitle

\begin{abstract}
An IC-plane graph is a topological graph where every edge is crossed at most once and no two crossed edges share a vertex. We show that every IC-plane graph has a visibility drawing where every vertex is of the form \shapes, and every edge is either a horizontal or vertical segment. As a byproduct of our drawing technique, we prove that every IC-plane graph has a RAC drawing in quadratic area with at most two bends per edge.
\end{abstract}






\section{Introduction}

A \emph{visibility drawing}  of a planar graph  maps the vertices of  into non-overlapping horizontal segments (\emph{bars}), and the edges of  into  vertical segments (\emph{visibilities}), each connecting the two bars corresponding to its two end-vertices. Visibilities intersect bars only at their extreme points.  is a \emph{strong} visibility drawing if there exists a visibility between two bars if and only if there exists an edge in  between the corresponding vertices. Every biconnected planar graph admits a strong visibility drawing (see, e.g.,~\cite{TamassiaTollis86}). Conversely, if a visibility may not correspond to an edge of the graph, then  is a {\em weak} visibility drawing. Since every planar graph can be augmented to a biconnected planar graph by adding edges, every planar graph admits a weak visibility drawing. 

The problem of extending visibility drawings to non-planar graphs has been first studied by Dean {\em et al.}~\cite{DBLP:journals/jgaa/DeanEGLST07}. They introduce \emph{bar -visibility drawings}, which are visibility drawings where each bar can see through at most  distinct bars. In other words, each visibility segment can intersect at most  bars, while each bar can be intersected by arbitrary many visibility segments. The graphs that admit a bar -visibility drawing are called \emph{-visibile}. Brandenburg {\em et al.} and independently Evans {\em et al.} prove that \emph{-planar graphs}, i.e., those graphs that can be drawn with at most one crossing per edge, are -visible~\cite{DBLP:journals/jgaa/Brandenburg14,DBLP:journals/jgaa/Evans0LMW14}. They focus on a \emph{weak} model, where there is a visibility through at most  bars if there is an edge, while the converse may not be true. In fact, having a strong model would be too restrictive in the case of bar -visibility drawings. For example, it is easy to see that a cycle of length at least four does not admit a strong bar -visibility drawing~\cite{DBLP:journals/jgaa/Brandenburg14}. In terms of readability, a clear benefit of bar -visibility drawings is that the crossings form right angles. \emph{Right-angle crossing (RAC) drawings} and their advantages in terms of readability have been extensively studied in the graph drawing literature (see, e.g.,~\cite{dl-cargd-12,DBLP:conf/apvis/HuangHE08}). However, in a bar -visibility drawing crossings involve bars and visibilities, i.e., vertices and edges. These crossings are arguably less intuitive than crossings between edges.

Evans {\em et al.} introduce a new model of visibility drawings, called {\em L-visibility drawings}~\cite{elm-svrp+-15}. Their aim is to simultaneously represent two plane -graphs  and  (whose union might be non-planar). They assume a {\em strong} model, where each vertex is represented by a horizontal bar and a vertical bar that share an extreme point, i.e. it is an {\em L-shape} in the set \shapes.  Two L-shapes are connected by a vertical (horizontal) visibility segment if and only if there exists an edge in  () between the corresponding vertices,  no two L-shapes cross one another, and visibilities intersect bars only at their extreme points. A clear advantage of this kind of drawing is that the only possible crossings are between vertical and horizontal visibilites, i.e., between edges of the graph. Furthermore, similar to bar -visibilities, these crossings form right angles. 

\begin{figure}[t]
\centering
\subfigure[]{\includegraphics[scale=0.45,page=1]{figures/example}\label{fi:example-1}}\hfil
\subfigure[]{\includegraphics[scale=0.45,page=2]{figures/example}\label{fi:example-2}}\hfil
\subfigure[]{\includegraphics[scale=0.45,page=3]{figures/example}\label{fi:example-3}}
\caption{\small (a) An IC-plane graph . (b) A L-visibility drawing of . (c) A RAC drawing of  with at most two bends per edge.}
\end{figure}

In this paper we initiate the study of {\em weak} L-visibility drawings of non-planar graphs.  We focus on the class of graphs that admit a drawing where each edge is crossed at most once, and no two crossed edges share an end-vertex. These graphs are called \emph{IC-planar graphs} (see  Fig.~\ref{fi:example-1} for an example). Their chromatic number is at most five~\cite{ks-cpgic-JGT10}, and they have at most~ edges, which is a tight bound~\cite{zl-spgic-CEJM13}. Recognizing IC-planar graphs is NP-hard~\cite{bdek+-rdicg-15}. Our main contribution is summarized by the following theorem, proved in Section~\ref{se:proof}. See Fig.~\ref{fi:example-2} for an example of a drawing computed by using Theorem~\ref{th:main}.

\begin{theorem}\label{th:main}
Every -vertex IC-plane graph  admits a L-visibility drawing in  area, which can be computed in  time.
\end{theorem}

 
We remark that Theorem~\ref{th:main} contributes to the rapidly growing literature devoted to the problem of drawing graphs that are ``nearly planar'' in some sense, i.e. graphs where only some types of edge crossings are allowed (for example, an edge can be crossed at most a constant number of times); see e.g.,~\cite{DBLP:conf/ictcs/Liotta14} for references.  
In particular,  Brandenburg {\em et al.} have recently described a cubic-time algorithm that computes IC-planar drawings with right-angle crossings and straight-line edges~\cite{bdek+-rdicg-15}. However these drawings may require exponential area, which is proved to be worst-case optimal~\cite{bdek+-rdicg-15}.  Brandenburg {\em et al.} leave as an open problem to study techniques that compute IC-planar drawings in polynomial area and with good crossing resolution~\cite{bdek+-rdicg-15}. We also recall that every graph admits a RAC drawing with at most three bends per edge~\cite{DBLP:journals/tcs/DidimoEL11}, while testing whether a graph has a straight-line RAC drawing is NP-hard~\cite{DBLP:journals/jgaa/ArgyriouBS12}. The following corollary follows as a byproduct of Theorem~\ref{th:main} (see also Fig.~\ref{fi:example-3}).

\begin{corollary}\label{co:main}
Every -vertex IC-plane graph  admits a RAC drawing with at most two bends per edge in  area, which can be computed in  time.
\end{corollary}




\section{Preliminaries}\label{se:preliminaries}
We assume familiarity with basic graph drawing concepts, see also~\cite{dett-gd-99}. 

{\bf Planarity and connectivity.} A graph  is \emph{simple}, if it contains neither loops nor multiple edges. We consider simple graphs, if not otherwise specified. A \emph{drawing}  of  maps each vertex of  to a point of the plane and each edge of  to a Jordan arc between its two end-points. We only consider \emph{simple drawings}, i.e., drawings such that the arcs representing two edges have at most one point in common, which is either a common end-vertex or a common interior point where the two arcs properly cross. A drawing is \emph{planar} if no two arcs representing two edges cross. A planar drawing divides the plane into topologically connected regions, called \emph{faces}. The unbounded region is called the \emph{outer face}. A \emph{planar embedding} of a graph is an equivalence class of planar drawings that define the same set of faces. A graph with a given planar embedding is a \emph{plane} graph. For a non-planar drawing, we can still talk about embedding considering that the boundary of a face may consist of portions of arcs between vertices and/or crossing points. 

A graph is \emph{biconnected} if it remains connected after removing any one vertex. A directed graph (a digraph for short) is biconnected if its underlying undirected graph is biconnected. A \emph{topological numbering} of a digraph is an assignment, , of numbers to its vertices such that  for every edge . A graph admits a topological numbering if and only if it is acyclic. An acyclic digraph with a single source  and a single sink  is called an \emph{-graph}. A \emph{plane -graph} is an -graph that is planar and embedded such that  and  are on the boundary of the outer face. In any -graph, the presence of the edge  guarantees that the graph is biconnected. In the following we consider -graphs that contain the edge , as otherwise it can be added without violating planarity. Let  be a plane -graph, then for each vertex  of  the incoming edges appear consecutively around , and so do the outgoing edges. Vertex  only has outgoing edges, while vertex  only has incoming edges. This particular transversal structure is known as a \emph{bipolar orientation}~\cite{DBLP:journals/dcg/RosenstiehlT86,TamassiaTollis86}. Each face  of  is bounded by two directed paths with a common \emph{origin} and \emph{destination}, called the \emph{left path} and \emph{right path} of . 

\begin{figure}[t]
    \centering
    \subfigure[]{\includegraphics[scale=0.8,page=1]{figures/thomassen}\label{fi:thomassen-1}}
    \hfil
    \subfigure[]{\includegraphics[scale=0.8,page=2]{figures/thomassen}\label{fi:thomassen-2}}
  \caption{\small (a) An X-configuration and (b) a B-configuration.}
\end{figure}

{\bf IC-planar graphs.} We recall some definitions also given in~\cite{bdek+-rdicg-15}. A drawing is IC-planar if each edge is crossed at most once, and any two crossed edges do not share an end-vertex. See Fig.~\ref{fi:example-1} for an illustration. An \emph{IC-planar embedding} is an embedding derived from an IC-planar drawing. A graph with a given IC-planar embedding is an \emph{IC-plane} graph. Thomassen~\cite{t-rdg-JGT88} characterized the possible crossing configurations that occur in a 1-planar drawing, i.e., a drawing where each edge is crossed at most once. This characterization applied to IC-planar drawings gives rise to the following property, where an X-crossing is of the  type described in Fig.~\ref{fi:thomassen-1}, and a B-crossing is of the type described in Fig.~\ref{fi:thomassen-2} (the solid edges only). 

\begin{property}[\cite{bdek+-rdicg-15}]\label{pr:char-crossins}
  Every crossing of an IC-plane graph is either an X- or a B-crossing.
\end{property}
A \emph{kite}~ is a graph isomorphic to~ together with an embedding such that all the vertices are on the boundary of the outer face. This implies that two edges of~ cross each other, while the other four edges are not crossed and  belong to the boundary of the outer face; see Fig.~\ref{fi:thomassen-1}. Consider a pair of crossing edges of an IC-plane graph , such that their four end-vertices induce a kite . The kite  is \emph{empty}, if in  there is no other vertex inside the internal faces of . The following property is a consequence of the more general Lemma 1 in~\cite{bdek+-rdicg-15} (in particular of cases c1 and c2 of that lemma). 

\begin{property}[\cite{bdek+-rdicg-15}]\label{pr:augmentation}
Let  be an -vertex IC-plane graph. It is possible to augment  to a biconnected IC-plane graph  (with a possibly different embedding), where , such that the end-vertices of each pair of crossing edges of  induce an empty kite. This can be done in  time.
\end{property}


{\bf Visibility model.}  In a L-visibility drawing  of a graph , every vertex is represented by a horizontal and a vertical segment sharing an end-point, i.e., by an L-shape in the set \shapes. Each edge of  is drawn in  as either a horizontal or a vertical visibility segment joining the two L-shapes corresponding to its two end-vertices. Clearly, horizontal visibilities only cross vertical visibilities at right angles. Also, no two L-shapes intersect. If  is an IC-plane graph, then each visibility is crossed at most once and no two crossed visibilities are incident to the same L-shape. In Fig.~\ref{fi:example-2}, an L-visibility representation  of an IC-plane graph  is shown.  Finally, we adopt a weak model, where a visibility may not imply the existence of the corresponding edge in the graph. For example, in Fig.~\ref{fi:example-2} the L-shapes of vertices  and  can be joined by a horizontal visibility, although the edge  does not exist in .



\section{Proof of Theorem~\ref{th:main}}\label{se:proof}

The proof of Theorem~\ref{th:main} is constructive and is based on a drawing algorithm that takes as input an IC-plane graph  and returns a L-visibility drawing  of . By Property~\ref{pr:augmentation}, we assume that  is such that each crossing induces an empty kite (see Section~\ref{se:preliminaries}). In fact, the output of our drawing algorithm maintains the IC-planar embedding obtained by applying Property~\ref{pr:augmentation}. We begin by removing from  all pairs of crossing edges and orient the resulting graph  to an -graph. The computed orientation is such that, when reinserting a pair of crossing edges in the corresponding planar face of , one of them is always incident to the origin and the destination of the face. In other words, each face of  that corresponds to an empty kite of , is oriented so that its left and right paths contain exactly one vertex each. To prove that this is always the case, we first need to introduce additional notation. Let  be a face of a plane graph .  Let  be the  vertices that belong to the boundary of , and let  be the set of neighbors of a vertex  of . The \emph{contraction} of  is the  operation defined as follows. Add to  a vertex  and connect  to the vertices in . Then remove  from . The resulting (multi)graph is still planar. Moreover, the contraction operation can be performed so to preserve the planar embedding of . Namely, the circular order of the edges incident to  is the same circular order encountered walking along the boundary of . See also Fig.~\ref{fi:contraction} for an illustration. The original graph  can be obtained by applying the reverse operation, called the \emph{expansion} of . Namely, vertices  are reinserted along with their original edges and  is removed from the graph. 


\begin{figure}[t]
\centering
\subfigure[ Contraction of ]{\includegraphics[scale=0.5,page=1]{figures/contraction}\label{fi:contraction}}\hfil
\subfigure[{\bf Case 1c}]{\includegraphics[scale=0.5,page=2]{figures/contraction}\label{fi:expansion-1}}\hfil
\subfigure[{\bf Case 2b}]{\includegraphics[scale=0.5,page=3]{figures/contraction}\label{fi:expansion-2}}\hfil
\subfigure[{\bf Case 3a}]{\includegraphics[scale=0.5,page=4]{figures/contraction}\label{fi:expansion-3}}
\caption{\small Illustration for the proof of Lemma~\ref{le:st-orientation}.}
\end{figure}

\begin{lemma}\label{le:st-orientation}
Let  be an -vertex IC-plane graph such that the end-vertices of each pair of crossing edges induce an empty kite. Let  be the set of crossing edges in , and consider the plane graph . Graph  can be oriented to an -graph such that each pair of crossing edges of  has been removed from a face of  whose left and right paths contain exactly one vertex each. This operation can be done in  time.
\end{lemma}
\begin{proof}
Each pair of crossing edges of  induces an empty kite , and thus corresponds to a single face  in  having exactly four vertices on its boundary. As a first step, we contract each face  of  (corresponding to a kite  in ). See also Fig.~\ref{fi:contraction} for an illustration. Notice that, since  is an IC-plane graph, no two faces of  share a vertex, and thus each vertex  of  corresponds to exactly one face of . Hence, we can contract the faces following an arbitrary order. The resulting graph  is a plane (multi)graph. Indeed, observe that if a face  of  shares an edge with a triangular face , then  will contain two parallel edges between  and the vertex  of  not in  (see also Fig.~\ref{fi:contraction}). 

As a second step, we orient  to an -graph (observe that parallel edges must receive the same orientation). In the third step, we expand one by one all the vertices  corresponding to a contracted face . After expanding a vertex , we orient the four reinserted edges of the face   maintaining the following invariants: {\bf I1.} The resulting graph has a single source and a single sink; {\bf I2.} The left and right paths of  contain exactly one vertex each.


Invariants {\bf I1} and {\bf I2} imply that the graph after expanding  is still an -graph, as it has a single source and a single sink by {\bf I1} and is acyclic by {\bf I2}. 

Let  be the four vertices belonging to the boundary of face , encountered in this order clockwise around the boundary of the face. To maintain {\bf I2} we need to orient the edges of  such that the origin and the destination of  are two non-adjacent vertices, i.e., either  and  or  and . In order to maintain {\bf I1}, recall that the incoming edges of  appear consecutive around it and so do the outgoing edges, unless  is the source or the sink of the graph. Thus, if  is neither the source nor the sink of the graph, then at most two of the reinserted vertices will have both incoming and outgoing edges incident to , whereas at most three will have only incoming or only outgoing  edges incident to . We distinguish the following three cases.

\medskip

{\noindent \bf Case 1.} No vertex of  has both incoming and outgoing edges. See Fig.~\ref{fi:expansion-1} for an illustration. Then we consider the following subcases. {\bf Case 1a.} There are three vertices having only outgoing edges, say ,  and , which implies that all the edges of  are incoming. In this case, we orient the edges of  so that  is the destination and  is the origin of the face, which ensures {\bf I2}. All vertices has now both incoming and outgoing edges, and thus {\bf I1} is also maintained. 
{\bf Case 1b.} There are three vertices having only incoming edges, say ,  and , which implies that all the edges of  are outgoing. In this case, the orientation that ensures {\bf I1} and {\bf I2} is the one where  is the destination and  is the origin of the face. 
{\bf Case 1c.} Two (consecutive) vertices only have incoming edges, say  and , and two (consecutive) vertices only have outgoing edges,  and .  Then to maintain {\bf I1} and {\bf I2} we let  to be the destination and  the origin of the face. This particular case is shown in Fig.~\ref{fi:expansion-1}. 

\medskip

{\noindent \bf Case 2.} Only one vertex of  , say , has both incoming and outgoing edges. Moreover, assume that the incoming edges of  are between the edge  and the outgoing edges of , as in Fig.~\ref{fi:expansion-2}, since the case in which the incoming edges of  are between the outgoing edges of  and the edge  is symmetric. We have two subcases. 
{\bf Case 2a.} The other three vertices only have incoming (resp., outgoing) edges. In this case,  we orient the edges of  so that  (resp., ) is the origin and  (resp., ) is the destination. This choice ensures both {\bf I1} and {\bf I2}. 
{\bf Case 2b.} Two vertices only have incoming edges and the other vertex only have outgoing edges. Due to the bipolar orientation of , the two vertices having only incoming edges are  and . Then we choose  as origin of the face and  as destination, which ensures both {\bf I1} and {\bf I2}. This particular case is shown in Fig.~\ref{fi:expansion-2}. The same orientation works also if two vertices only have outgoing edges ( and ) and the other vertex only have incoming edges ().

\medskip

{\noindent \bf Case 3.} Two vertices of   have both incoming and outgoing edges. We have two subcases. {\bf Case 3a.} Suppose first that these two vertices, say  and , are adjacent in .   Moreover, assume that the incoming edges of  are between the edge  and the outgoing edges of , as in Fig.~\ref{fi:expansion-3}, since the other case is symmetric. This implies that the incoming edges of  are between the outgoing edges of  and the edge . Moreover,  and  only have incoming edges. Then, if we let  to be the destination of the face and  the origin, {\bf I1} and {\bf I2} are maintained. 
{\bf Case 3b.} Suppose now that the two vertices, say  and , are not adjacent in . Moreover, assume that the incoming edges of  are between the edge  and the outgoing edges of , as in Fig.~\ref{fi:expansion-3}, since the case in which the incoming edges of  are between the outgoing edges of  and the edge  is symmetric. This implies that the incoming edges of  are between the outgoing edges of  and the edge . Moreover,  and  only have outgoing and incoming edges, respectively. Then, if we orient the edges of  so that  is the destination of the face and  the origin, then {\bf I1} and {\bf I2} are ensured.

\medskip

Finally, suppose that  is the source (resp., sink) of the graph. Then all the vertices of  have either no incident oriented edges, or all outgoing (resp., incoming) edges. Also, at least one of them has at least one outgoing (resp., incoming) edge, say . We orient the edges of  so that  is the destination (resp., origin) and  the origin (resp., destination) of . 

\medskip

The described algorithm works in  time. Namely, the graph  can be constructed and oriented to an -graph in  time (see, e.g.,~\cite{Even1976339}). Furthermore, orienting the edges of an expanded face requires first to analyze the orientation of the edges incident to each vertex of the face, and then to orient the edges of the face. Since the faces are vertex-disjoint, this costs at most , where  is the number of edges of , which is .
\end{proof}

Let  be a face of the plane -graph  which corresponds to an (empty) kite  in . By Lemma~\ref{le:st-orientation},  is such that its left and right path have both length two. This implies that one of the two crossing edges of  is incident to the origin and to the destination of , whereas the other one is incident to the two vertices belonging one to the left and one to the right path of . Let  be the biconnected plane graph obtained from  by reinserting, for each pair of crossing edges of , the edge incident to the origin and to the destination of the corresponding face in . Furthermore, let  be a strong visibility drawing of  in  area, which can be computed in  time (see, e.g.,~\cite{TamassiaTollis86}).

\begin{figure}[t]
\centering
\subfigure[]{\includegraphics[scale=0.8,page=3]{figures/visrep}\label{fi:face}}\hfil
\subfigure[]{\includegraphics[scale=0.8,page=1]{figures/visrep}\label{fi:barvisrep}}\hfil
\subfigure[]{\includegraphics[scale=0.8,page=2]{figures/visrep}\label{fi:lvisrep}}\hfil
\caption{\small Illustration for the proof of Lemma~\ref{le:drawing}.}
\end{figure}

\begin{lemma}\label{le:drawing}
 can be transformed into a L-visibility drawing  of  that requires  area. This operation can be done in  time.
\end{lemma}
\begin{proof}
Consider a face  of  corresponding to an empty kite  in . Let  be the four vertices of , encountered in this order clockwise around the boundary of the face. See also Fig.~\ref{fi:face}. Without loss of generality, let  be the destination of the face. Then Lemma~\ref{le:st-orientation} implies that  is the origin of , and thus the edge reinserted in  for this face is . In other words,  is split in two faces in , and these two faces share the edge . Consider the subdrawing of  induced by the four vertices  . An illustration is also shown in Fig.~\ref{fi:barvisrep}. Let  be the bar representing a vertex  in . Either  and  are drawn at the same -coordinate, or, one is above the other. Also, the two bars do not overlap as there is no edge between  and  in  (and  is a strong visibility drawing). Between the two bars there is actually (at least) one unit gap needed to draw the visibility from  to . Moreover,  is below both  and , while  is above both of them. In any case, we first extend  (resp., ) by 0.25 units to the right (resp., left). Next, if  and  have the same -coordinate, then it suffice to draw two vertical bars  and , such that the bottomost end-point of  (resp., ) coincides with the rightmost end-point of  (resp., leftmost end-point of ), and such that the other end-point is 0.5 units above it.  If one is above the other, say  is above , and the difference in terms of -coordinates between the two bars is  units, then we draw two vertical bars  and , such that the end-point of  (resp., ) coincides with the rightmost end-point of  (resp., leftmost end-point of ), and such that the other end-point is  units above it (resp., below it). In both cases, the two resulting L-shapes see each other through a horizontal visibility segment. See also Fig.~\ref{fi:lvisrep}. Since every vertex is adjacent to at most one crossed edge  we have that the final drawing  is a L-visibility drawing of . Since  contains  segments which have to be transformed to L-shapes,  is computed in  time. Finally, in order to restore integer coordinates, we scale by a factor 4 the grid of , which thus takes  area.
\end{proof}

Lemmas~\ref{le:st-orientation} and~\ref{le:drawing} imply Theorem~\ref{th:main}. To prove Corollary~\ref{co:main}, consider a visibility drawing  of an -vertex IC-plane graph . By Theorem~\ref{th:main},  can be computed in  time and fits on a grid of  size. Let  be an L-shape of . The \emph{representative point}  of  is defined as follows. If both the horizontal and the vertical segments of  have non-zero length, then  is the point where they touch. Otherwise,  is the midpoint of the segment of   having non-zero length. Replace each vertical visibility segment with a polyline as follows. Let  be a visibility segment connecting the two L-shapes  and . Let  and  be the representative points of  and , respectively.  Also, let  and  be the points that  shares with  and , respectively. Suppose that  is below  (and thus  is below ). Replace  with the polyline starting at , bending 0.25 grid units above , bending again 0.25 units below , and ending at . With a symmetric operation we can also replace each horizontal visibility segment. Finally, replace each L-shape with its representative point. The resulting drawing is an IC-plane drawing of  where edges are polylines with (at most) two bends that cross at right-angles. Finally, scaling by a factor 4 the grid of the drawing we restore integer coordinates. Fig.~\ref{fi:example-3} shows a RAC drawing computed from the L-visibility drawing in Fig.~\ref{fi:example-2}.

\section{Conclusions and Open Problems}\label{se:conclusions}
We have proved that every IC-plane graph  has a L-visibility drawing which can be computed in linear time. As a corollary, our result implies that  has a RAC drawing in quadratic area and at most two bends per edge which can also be computed in linear time. We conclude the paper with two open problems:  Does every -planar graph admit a visibility drawing where the shape associated with each vertex is either an L-shape, a T-shape, or a +-shape?  Does every IC-plane graph admit a RAC drawing with at most one bend per edge in polynomial area?










{\small \bibliography{lshapes}}
\bibliographystyle{abbrv}

\end{document}
