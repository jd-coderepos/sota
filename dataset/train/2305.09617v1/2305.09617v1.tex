
\clearpage


\renewcommand{\thesection}{A.\arabic{section}}
\renewcommand{\thefigure}{A.\arabic{figure}}
\renewcommand{\thetable}{A.\arabic{table}}
\renewcommand{\theequation}{A.\arabic{equation}}

\setcounter{section}{0}
\setcounter{figure}{0}
\setcounter{table}{0}
\setcounter{equation}{0}



\noindent \textbf{\LARGE{Appendix}}\\
\normalfont

\section{Additional Results}

\begin{itemize}[leftmargin=1.5em,rightmargin=0em]
\setlength\itemsep{5pt}

\item \cref{tab-sup:1}: Overlap sensitivity analysis.
\item \cref{tab-sup:2}: Statistical analysis for independent evaluation of long-form answers with physician raters on MultiMedQA 140.
\item \cref{tab-sup:3}: Statistical analysis for independent evaluation of long-form answers with physician raters on adversarial questions.
\item \cref{tab-sup:4} Statistical analysis for independent evaluation of long-form answers with lay-person raters on MultiMedQA 140.
\item \cref{tab-sup:pairwise-mp2phys}: Statistical analysis of pairwise ranking evaluation using physician raters on MultiMedQA 1066, comparing Med-PaLM 2 to physician answers.
\item \cref{tab-sup:pairwise-mp2mp}: Statistical analysis of pairwise ranking evaluation using physician raters on MultiMedQA 1066, comparing Med-PaLM 2 to Med-PaLM answers.
\item \cref{tab-sup:ranking-examples}: Examples of Med-PaLM and Med-PaLM 2 responses on long-form answers.
\item \cref{tab-sup:ranking-example-ratings} Pairwise rankings between Med-PaLM and Med-PaLM 2 answers on the example questions.
\item \cref{tab-sup:8}: Summary statistics of answer lengths, in characters, for Med-PaLM 2, Med-PaLM, physicians who produced answers to questions in the MultiMedQA 140 and Adversarial sets.
\end{itemize}

\section{Inter-rater Reliability}
\label{sec-sup:inter-rater-reliability}
We performed inter-rater reliability (IRR) analysis for physician ratings of long-form answers on a subset of question-answer pairs (N=140) that were multi-rated by a set of three independent physicians. Inter-rater agreement was measured as Randolph's  ~\cite{randolph2005free}; this measurement was more appropriate than other measures such as Krippendorff's alpha given the low baseline positive rate for several axes, such as incorrect comprehension. Raters were in very good ( > 0.8) agreement for 10 out of 12 alignment questions, and good ( > 0.6) agreement for the remaining two questions, including whether the answer either misses important content, or contains unnecessary additional information. \cref{fig:inter-rater} illustrates agreement metrics for each of the 12 evaluation axes along with 95\% confidence intervals. 

\begin{figure*}[h]
\small
    \centering
    \includegraphics[width=0.7\textwidth]{figures/inter-rater.pdf}
    \vspace{10pt}
    \caption{\textbf{Inter-rater reliability} Illustration of inter-rater reliability for the 12 alignment questions on MultiMedQA 140. The green dotted line (=0.6) indicates good agreement and the green solid line (=0.8) indicates very good agreement. }
    \vspace{-0pt}
    \label{fig:inter-rater}
\end{figure*}


\begin{table}[]
\small
\centering
\caption{\textbf{Overlap sensitivity analysis} We define a question as overlapping if either the entire question or up to 120 characters overlap with any document in the training corpus of the LLM underlying Med-PaLM 2.}
\label{tab-sup:1}
\vspace{3pt}
\begin{tabular}{l|cccc}
\toprule
\textbf{Dataset} &
  \textbf{Overlap  Fraction} &
  \textbf{\begin{tabular}[c]{@{}c@{}}Performance \\ (without Overlap)\end{tabular}} &
  \textbf{\begin{tabular}[c]{@{}c@{}}Performance \\ (with Overlap)\end{tabular}} &
  \textbf{Delta} \\ \midrule
MedQA (USMLE) &
  {\color[HTML]{212121} \begin{tabular}[c]{@{}c@{}}142/1273 \\ (11.2\%)\end{tabular}} &
  {\color[HTML]{212121} \begin{tabular}[c]{@{}c@{}}85.3 \\ {[}83.3, 87.4{]}\end{tabular}} &
  {\color[HTML]{212121} \begin{tabular}[c]{@{}c@{}}85.9 \\ {[}80.2, 91.6{]}\end{tabular}} &
  {\color[HTML]{212121} \begin{tabular}[c]{@{}c@{}}-0.6 \\ {[}-5.8, 6.4{]}\end{tabular}} \\ \hline
PubMedQA &
  {\color[HTML]{212121} \begin{tabular}[c]{@{}c@{}}67/500 \\ (13.4\%)\end{tabular}} &
  {\color[HTML]{212121} \begin{tabular}[c]{@{}c@{}}74.1\\ {[}70.0, 78.3{]}\end{tabular}} &
  {\color[HTML]{212121} \begin{tabular}[c]{@{}c@{}}73.1 \\ {[}62.5, 83.7{]}\end{tabular}} &
  {\color[HTML]{212121} \begin{tabular}[c]{@{}c@{}}1.0 \\ {[}-9.1, 13.3{]}\end{tabular}} \\ \hline
MedMCQA &
  {\color[HTML]{212121} \begin{tabular}[c]{@{}c@{}}1021/4183 \\ (24.4\%)\end{tabular}} &
  {\color[HTML]{212121} \begin{tabular}[c]{@{}c@{}}70.5 \\ {[}68.9, 72.1{]}\end{tabular}} &
  {\color[HTML]{212121} \begin{tabular}[c]{@{}c@{}}74.4 \\ {[}71.8, 77.1{]}\end{tabular}} &
  {\color[HTML]{212121} \begin{tabular}[c]{@{}c@{}}-4.0 \\ {[}-7.0, -0.8{]}\end{tabular}} \\ \hline
MMLU Clinical knowledge &
  {\color[HTML]{212121} \begin{tabular}[c]{@{}c@{}}56/265 \\ (21.1\%)\end{tabular}} &
  {\color[HTML]{212121} \begin{tabular}[c]{@{}c@{}}88.5 \\ {[}84.2, 92.8{]}\end{tabular}} &
  {\color[HTML]{212121} \begin{tabular}[c]{@{}c@{}}87.5 \\ {[}78.8, 96.2{]}\end{tabular}} &
  {\color[HTML]{212121} \begin{tabular}[c]{@{}c@{}}1.0 \\ {[}-7.1, 12.7{]}\end{tabular}} \\ \hline
MMLU Medical genetics &
  {\color[HTML]{212121} \begin{tabular}[c]{@{}c@{}}56/100 \\ (56.0\%)\end{tabular}} &
  {\color[HTML]{212121} \begin{tabular}[c]{@{}c@{}}93.2 \\ {[}85.7, 100.0{]}\end{tabular}} &
  {\color[HTML]{212121} \begin{tabular}[c]{@{}c@{}}91.1 \\ {[}83.6, 98.5{]}\end{tabular}} &
  {\color[HTML]{212121} \begin{tabular}[c]{@{}c@{}}2.1 \\ {[}-10.4, 13.4{]}\end{tabular}} \\ \hline
MMLU Anatomy &
  {\color[HTML]{212121} \begin{tabular}[c]{@{}c@{}}39/135 \\ (28.9\%)\end{tabular}} &
  {\color[HTML]{212121} \begin{tabular}[c]{@{}c@{}}82.3 \\ {[}74.7, 89.9{]}\end{tabular}} &
  {\color[HTML]{212121} \begin{tabular}[c]{@{}c@{}}89.7 \\ {[}80.2, 99.3{]}\end{tabular}} &
  {\color[HTML]{212121} \begin{tabular}[c]{@{}c@{}}-7.5 \\ {[}-18.2, 7.3{]}\end{tabular}} \\ \hline
MMLU-Professional medicine &
  {\color[HTML]{212121} \begin{tabular}[c]{@{}c@{}}149/272 \\ (54.8\%)\end{tabular}} &
  {\color[HTML]{212121} \begin{tabular}[c]{@{}c@{}}84.6 \\ {[}78.2, 90.9{]}\end{tabular}} &
  {\color[HTML]{212121} \begin{tabular}[c]{@{}c@{}}94.6 \\ {[}91.0, 98.3{]}\end{tabular}} &
  {\color[HTML]{212121} \begin{tabular}[c]{@{}c@{}}-10.1 \\ {[}-18.0, -2.9{]}\end{tabular}} \\ \hline
MMLU-College biology &
  {\color[HTML]{212121} \begin{tabular}[c]{@{}c@{}}69/144 \\ (47.9\%)\end{tabular}} &
  {\color[HTML]{212121} \begin{tabular}[c]{@{}c@{}}94.7 \\ {[}89.6, 99.8{]}\end{tabular}} &
  {\color[HTML]{212121} \begin{tabular}[c]{@{}c@{}}97.1 \\ {[}93.1, 100.0{]}\end{tabular}} &
  {\color[HTML]{212121} \begin{tabular}[c]{@{}c@{}}-2.4 \\ {[}-10.3, 5.3{]}\end{tabular}} \\ \hline
MMLU-College medicine &
  {\color[HTML]{212121} \begin{tabular}[c]{@{}c@{}}70/173 \\ (40.5\%)\end{tabular}} &
  {\color[HTML]{212121} \begin{tabular}[c]{@{}c@{}}79.6 \\ {[}71.8, 87.4{]}\end{tabular}} &
  {\color[HTML]{212121} \begin{tabular}[c]{@{}c@{}}85.7 \\ {[}77.5, 93.9{]}\end{tabular}} &
  {\color[HTML]{212121} \begin{tabular}[c]{@{}c@{}}-6.1 \\ {[}-16.9, 5.9{]}\end{tabular}} \\ \bottomrule
\end{tabular}
\end{table}





\begin{table}[]
\small
\centering
\caption{\textbf{Statistical analysis for independent evaluation of long-form answers with physician raters on MultiMedQA 140.} 95\% confidence intervals were computed via bootstrapping. -values represent pairwise permutation tests between Med-PaLM 2 and Med-PaLM answer ratings (left column) and Med-PaLM 2 and Physician answers ratings (right column).}
\label{tab-sup:2}
\vspace{3pt}
\begin{tabular}{c|ccccc}
\toprule
\textbf{Rating type} &
  \textbf{\begin{tabular}[c]{@{}c@{}}Metric, \\ Med-PaLM 2\\ {[}CI{]}\end{tabular}} &
  \textbf{\begin{tabular}[c]{@{}c@{}}Metric, \\ Med-PaLM \\ {[}CI{]}\end{tabular}} &
  \textbf{\begin{tabular}[c]{@{}c@{}}Metric, \\ Physician\\ {[}CI{]}\end{tabular}} &
  \textbf{\begin{tabular}[c]{@{}c@{}}\\ Med-PaLM 2 \\ vs. Med-PaLM\end{tabular}} &
  \textbf{\begin{tabular}[c]{@{}c@{}}\\ Med-PaLM 2 \\ vs. Physician\end{tabular}} \\ \midrule
\begin{tabular}[c]{@{}c@{}}Answer supported by \\consensus\end{tabular} &
  \begin{tabular}[c]{@{}c@{}}0.917\\ {[}0.890, 0.943{]}\end{tabular} &
  \begin{tabular}[c]{@{}c@{}}0.929\\ {[}0.879, 0.971{]}\end{tabular} &
  \begin{tabular}[c]{@{}c@{}}0.921\\ {[}0.879, 0.964{]}\end{tabular} &
  0.725 &
  0.890 \\ \hline
\begin{tabular}[c]{@{}c@{}}Possible harm extent\\ = No  harm\end{tabular} &
  \begin{tabular}[c]{@{}c@{}}0.933\\ {[}0.910, 0.955{]}\end{tabular} &
  \begin{tabular}[c]{@{}c@{}}0.943\\ {[}0.900, 0.979{]}\end{tabular} &
  \begin{tabular}[c]{@{}c@{}}0.929\\ {[}0.886, 0.971{]}\end{tabular} &
  0.687 &
  0.950 \\ \hline
Low likelihood of harm &
  \begin{tabular}[c]{@{}c@{}}0.955\\ {[}0.936, 0.974{]}\end{tabular} &
  \begin{tabular}[c]{@{}c@{}}0.979\\ {[}0.950, 1.000{]}\end{tabular} &
  \begin{tabular}[c]{@{}c@{}}0.971\\ {[}0.943, 0.993{]}\end{tabular} &
  0.287 &
  0.439 \\ \hline
\begin{tabular}[c]{@{}c@{}}Shows evidence of \\question  comprehension\end{tabular} &
  \begin{tabular}[c]{@{}c@{}}0.983\\ {[}0.969, 0.995{]}\end{tabular} &
  \begin{tabular}[c]{@{}c@{}}0.936\\ {[}0.886, 0.971{]}\end{tabular} &
  \begin{tabular}[c]{@{}c@{}}0.971\\ {[}0.943, 0.993{]}\end{tabular} &
  0.056 &
  0.655 \\ \hline
\begin{tabular}[c]{@{}c@{}}Shows evidence of \\ knowledge recall\end{tabular} &
  \begin{tabular}[c]{@{}c@{}}0.971\\ {[}0.957, 0.988{]}\end{tabular} &
  \begin{tabular}[c]{@{}c@{}}0.936\\ {[}0.893, 0.971{]}\end{tabular} &
  \begin{tabular}[c]{@{}c@{}}0.971\\ {[}0.943, 0.993{]}\end{tabular} &
  0.313 &
  1.000 \\ \hline
\begin{tabular}[c]{@{}c@{}}Shows evidence of \\reasoning\end{tabular}  &
  \begin{tabular}[c]{@{}c@{}}0.974\\ {[}0.957, 0.988{]}\end{tabular} &
  \begin{tabular}[c]{@{}c@{}}0.914\\ {[}0.864, 0.964{]}\end{tabular} &
  \begin{tabular}[c]{@{}c@{}}0.971\\ {[}0.943, 0.993{]}\end{tabular} &
  0.030 &
  0.858 \\ \hline
\begin{tabular}[c]{@{}c@{}}No sign of incorrect \\ comprehension\end{tabular} &
  \begin{tabular}[c]{@{}c@{}}0.986\\ {[}0.974, 0.995{]}\end{tabular} &
  \begin{tabular}[c]{@{}c@{}}0.943\\ {[}0.900, 0.979{]}\end{tabular} &
  \begin{tabular}[c]{@{}c@{}}0.971\\ {[}0.943, 0.993{]}\end{tabular} &
  0.108 &
  0.713 \\ \hline
\begin{tabular}[c]{@{}c@{}}No sign of incorrect \\ knowledge recall\end{tabular} &
  \begin{tabular}[c]{@{}c@{}}0.933\\ {[}0.912, 0.955{]}\end{tabular} &
  \begin{tabular}[c]{@{}c@{}}0.829\\ {[}0.764, 0.886{]}\end{tabular} &
  \begin{tabular}[c]{@{}c@{}}0.950\\ {[}0.914, 0.986{]}\end{tabular} &
  0.022 &
  0.523 \\ \hline
\begin{tabular}[c]{@{}c@{}}No sign of incorrect \\reasoning\end{tabular}  &
  \begin{tabular}[c]{@{}c@{}}0.962\\ {[}0.943, 0.979{]}\end{tabular} &
  \begin{tabular}[c]{@{}c@{}}0.886\\ {[}0.829, 0.936{]}\end{tabular} &
  \begin{tabular}[c]{@{}c@{}}0.964\\ {[}0.929, 0.993{]}\end{tabular} &
  0.032 &
  0.820 \\ \hline
\begin{tabular}[c]{@{}c@{}}No inaccurate or \\ irrelevant information \end{tabular}  &
  \begin{tabular}[c]{@{}c@{}}0.900\\ {[}0.871, 0.926{]}\end{tabular} &
  \begin{tabular}[c]{@{}c@{}}0.814\\ {[}0.750, 0.879{]}\end{tabular} &
  \begin{tabular}[c]{@{}c@{}}0.971\\ {[}0.943, 0.993{]}\end{tabular} &
  0.066 &
  0.076 \\ \hline
\begin{tabular}[c]{@{}c@{}}No missing important \\content\end{tabular}  &
  \begin{tabular}[c]{@{}c@{}}0.881\\ {[}0.848, 0.914{]}\end{tabular} &
  \begin{tabular}[c]{@{}c@{}}0.850\\ {[}0.786, 0.907{]}\end{tabular} &
  \begin{tabular}[c]{@{}c@{}}0.871\\ {[}0.814, 0.921{]}\end{tabular} &
  0.427 &
  0.784 \\ \hline
\begin{tabular}[c]{@{}c@{}}No sign of bias  towards \\specific subgroups\end{tabular} &
  \begin{tabular}[c]{@{}c@{}}0.971\\ {[}0.955, 0.986{]}\end{tabular} &
  \begin{tabular}[c]{@{}c@{}}0.993\\ {[}0.979, 1.000{]}\end{tabular} &
  \begin{tabular}[c]{@{}c@{}}0.971\\ {[}0.943, 0.993{]}\end{tabular} &
  0.429 &
  1.000 \\ \bottomrule
\end{tabular}
\end{table}



\begin{table}[]
\small
\centering
\caption{\textbf{Statistical analysis for independent evaluation of long-form answers with physician raters on adversarial questions.} For each rating axis, the top row summarizes ratings across all adversarial questions, while the below rows show individual evaluation performance on two subsets: Health equity focused questions ( raters) and General questions ( raters).}
\label{tab-sup:3}
\vspace{3pt}
\begin{tabular}{l|lccc}
\toprule
\multicolumn{1}{l|}{\textbf{Rating}} &
  \multicolumn{1}{c}{\textbf{Question set}} &
  \textbf{\begin{tabular}[c]{@{}c@{}}Metric, \\ Med-PaLM 2\\ \end{tabular}} &
  \textbf{\begin{tabular}[c]{@{}c@{}}Metric, \\ Med-PaLM \\ \end{tabular}} &
  \textbf{ value} \\ \midrule
\multirow{3}{*}{\begin{tabular}[c]{@{}l@{}}Answer supported by \\ consensus\end{tabular}} &
  All Adversarial questions &
  \begin{tabular}[c]{@{}c@{}}0.769 {[}0.733, 0.803{]}\end{tabular} &
  \begin{tabular}[c]{@{}c@{}}0.585 {[}0.544, 0.626{]}\end{tabular} &
  0.000 \\
 &
  Health equity questions &
  \begin{tabular}[c]{@{}c@{}}0.784 {[}0.742, 0.826{]}\end{tabular} &
  \begin{tabular}[c]{@{}c@{}}0.590 {[}0.539, 0.640{]}\end{tabular} &
  0.000 \\
 &
  General questions &
  \begin{tabular}[c]{@{}c@{}}0.746 {[}0.690, 0.802{]}\end{tabular} &
  \begin{tabular}[c]{@{}c@{}}0.578 {[}0.513, 0.642{]}\end{tabular} &
  0.001 \\ \hline
\multirow{3}{*}{\begin{tabular}[c]{@{}l@{}}Possible harm extent\\  = No harm\end{tabular}} &
  All Adversarial questions &
  \begin{tabular}[c]{@{}c@{}}0.786 {[}0.752, 0.820{]}\end{tabular} &
  \begin{tabular}[c]{@{}c@{}}0.619 {[}0.580, 0.658{]}\end{tabular} &
  0.000 \\
 &
  Health equity questions &
  \begin{tabular}[c]{@{}c@{}}0.764 {[}0.719, 0.809{]}\end{tabular} &
  \begin{tabular}[c]{@{}c@{}}0.576 {[}0.525, 0.626{]}\end{tabular} &
  0.000 \\
 &
  General questions &
  \begin{tabular}[c]{@{}c@{}}0.819 {[}0.767, 0.866{]}\end{tabular} &
  \begin{tabular}[c]{@{}c@{}}0.685 {[}0.625, 0.746{]}\end{tabular} &
  0.005 \\ \hline
\multirow{3}{*}{Low likelihood of harm} &
  All Adversarial questions &
  \begin{tabular}[c]{@{}c@{}}0.906 {[}0.883, 0.929{]}\end{tabular} &
  \begin{tabular}[c]{@{}c@{}}0.794 {[}0.762, 0.827{]}\end{tabular} &
  0.000 \\
 &
  Health equity questions &
  \begin{tabular}[c]{@{}c@{}}0.913 {[}0.882, 0.941{]}\end{tabular} &
  \begin{tabular}[c]{@{}c@{}}0.784 {[}0.739, 0.826{]}\end{tabular} &
  0.000 \\
 &
  General questions &
  \begin{tabular}[c]{@{}c@{}}0.897 {[}0.853, 0.935{]}\end{tabular} &
  \begin{tabular}[c]{@{}c@{}}0.810 {[}0.759, 0.858{]}\end{tabular} &
  0.019 \\ \hline
\multirow{3}{*}{\begin{tabular}[c]{@{}l@{}}Shows evidence of \\ question comprehension\end{tabular}} &
  All Adversarial questions &
  \begin{tabular}[c]{@{}c@{}}0.949 {[}0.930, 0.966{]}\end{tabular} &
  \begin{tabular}[c]{@{}c@{}}0.871 {[}0.844, 0.896{]}\end{tabular} &
  0.000 \\
 &
  Health equity questions &
  \begin{tabular}[c]{@{}c@{}}0.949 {[}0.924, 0.972{]}\end{tabular} &
  \begin{tabular}[c]{@{}c@{}}0.868 {[}0.831, 0.902{]}\end{tabular} &
  0.000 \\
 &
  General questions &
  \begin{tabular}[c]{@{}c@{}}0.948 {[}0.918, 0.974{]}\end{tabular} &
  \begin{tabular}[c]{@{}c@{}}0.875 {[}0.832, 0.918{]}\end{tabular} &
  0.002 \\ \hline
\multirow{3}{*}{\begin{tabular}[c]{@{}l@{}}Shows evidence of \\ knowledge recall\end{tabular}} &
  All Adversarial questions &
  \begin{tabular}[c]{@{}c@{}}0.969 {[}0.956, 0.983{]}\end{tabular} &
  \begin{tabular}[c]{@{}c@{}}0.827 {[}0.796, 0.857{]}\end{tabular} &
  \textless 0.001 \\
 &
  Health equity questions &
  \begin{tabular}[c]{@{}c@{}}0.969 {[}0.949, 0.986{]}\end{tabular} &
  \begin{tabular}[c]{@{}c@{}}0.823 {[}0.781, 0.862{]}\end{tabular} &
  \textless 0.001 \\
 &
  General questions &
  \begin{tabular}[c]{@{}c@{}}0.970 {[}0.944, 0.991{]}\end{tabular} &
  \begin{tabular}[c]{@{}c@{}}0.832 {[}0.780, 0.879{]}\end{tabular} &
  \textless 0.001 \\ \hline
\multirow{3}{*}{\begin{tabular}[c]{@{}l@{}}Shows evidence of \\ reasoning\end{tabular}} &
  All Adversarial questions &
  \begin{tabular}[c]{@{}c@{}}0.959 {[}0.942, 0.974{]}\end{tabular} &
  \begin{tabular}[c]{@{}c@{}}0.811 {[}0.779, 0.842{]}\end{tabular} &
  \textless 0.001 \\
 &
  Health equity questions &
  \begin{tabular}[c]{@{}c@{}}0.955 {[}0.933, 0.975{]}\end{tabular} &
  \begin{tabular}[c]{@{}c@{}}0.806 {[}0.764, 0.846{]}\end{tabular} &
  \textless 0.001 \\
 &
  General questions &
  \begin{tabular}[c]{@{}c@{}}0.966 {[}0.940, 0.987{]}\end{tabular} &
  \begin{tabular}[c]{@{}c@{}}0.819 {[}0.767, 0.866{]}\end{tabular} &
  \textless 0.001 \\ \hline
\multirow{3}{*}{\begin{tabular}[c]{@{}l@{}}No sign of incorrect \\ comprehension\end{tabular}} &
  All Adversarial questions &
  \begin{tabular}[c]{@{}c@{}}0.947 {[}0.929, 0.964{]}\end{tabular} &
  \begin{tabular}[c]{@{}c@{}}0.855 {[}0.827, 0.883{]}\end{tabular} &
  \textless 0.001 \\
 &
  Health equity questions &
  \begin{tabular}[c]{@{}c@{}}0.947 {[}0.921, 0.969{]}\end{tabular} &
  \begin{tabular}[c]{@{}c@{}}0.854 {[}0.817, 0.890{]}\end{tabular} &
  \textless 0.001 \\
 &
  General questions &
  \begin{tabular}[c]{@{}c@{}}0.948 {[}0.918, 0.974{]}\end{tabular} &
  \begin{tabular}[c]{@{}c@{}}0.858 {[}0.810, 0.901{]}\end{tabular} &
  0.001 \\ \hline
\multirow{3}{*}{\begin{tabular}[c]{@{}l@{}}No sign of incorrect \\ knowledge recall\end{tabular}} &
  All Adversarial questions &
  \begin{tabular}[c]{@{}c@{}}0.857 {[}0.828, 0.884{]}\end{tabular} &
  \begin{tabular}[c]{@{}c@{}}0.709 {[}0.672, 0.745{]}\end{tabular} &
  \textless 0.001 \\
 &
  Health equity questions &
  \begin{tabular}[c]{@{}c@{}}0.868 {[}0.831, 0.902{]}\end{tabular} &
  \begin{tabular}[c]{@{}c@{}}0.722 {[}0.674, 0.770{]}\end{tabular} &
  \textless 0.001 \\
 &
  General questions &
  \begin{tabular}[c]{@{}c@{}}0.841 {[}0.793, 0.884{]}\end{tabular} &
  \begin{tabular}[c]{@{}c@{}}0.690 {[}0.629, 0.750{]}\end{tabular} &
  0.001 \\ \hline
\multirow{3}{*}{\begin{tabular}[c]{@{}l@{}}No sign of incorrect \\ reasoning\end{tabular}} &
  All Adversarial questions &
  \begin{tabular}[c]{@{}c@{}}0.961 {[}0.944, 0.976{]}\end{tabular} &
  \begin{tabular}[c]{@{}c@{}}0.798 {[}0.765, 0.830{]}\end{tabular} &
  \textless 0.001 \\
 &
  Health equity questions &
  \begin{tabular}[c]{@{}c@{}}0.955 {[}0.933, 0.975{]}\end{tabular} &
  \begin{tabular}[c]{@{}c@{}}0.795 {[}0.753, 0.837{]}\end{tabular} &
  \textless 0.001 \\
 &
  General questions &
  \begin{tabular}[c]{@{}c@{}}0.970 {[}0.944, 0.991{]}\end{tabular} &
  \begin{tabular}[c]{@{}c@{}}0.802 {[}0.750, 0.853{]}\end{tabular} &
  \textless 0.001 \\ \hline
\multirow{3}{*}{\begin{tabular}[c]{@{}l@{}}No inaccurate or  \\ irrelevant information\end{tabular}} &
  All Adversarial questions &
  \begin{tabular}[c]{@{}c@{}}0.847 {[}0.816, 0.874{]}\end{tabular} &
  \begin{tabular}[c]{@{}c@{}}0.651 {[}0.612, 0.690{]}\end{tabular} &
  \textless 0.001 \\
 &
  Health equity questions &
  \begin{tabular}[c]{@{}c@{}}0.848 {[}0.812, 0.882{]}\end{tabular} &
  \begin{tabular}[c]{@{}c@{}}0.638 {[}0.587, 0.685{]}\end{tabular} &
  \textless 0.001 \\
 &
  General questions &
  \begin{tabular}[c]{@{}c@{}}0.845 {[}0.797, 0.888{]}\end{tabular} &
  \begin{tabular}[c]{@{}c@{}}0.672 {[}0.612, 0.733{]}\end{tabular} &
  0.002 \\ \hline
  \multirow{3}{*}{\begin{tabular}[c]{@{}l@{}}No missing important \\ content\end{tabular}} &
  All Adversarial questions &
  \begin{tabular}[c]{@{}c@{}}0.808 {[}0.776, 0.838{]}\end{tabular} &
  \begin{tabular}[c]{@{}c@{}}0.614 {[}0.575, 0.653{]}\end{tabular} &
  \textless 0.001 \\
 &
  Health equity questions &
  \begin{tabular}[c]{@{}c@{}}0.806 {[}0.764, 0.846{]}\end{tabular} &
  \begin{tabular}[c]{@{}c@{}}0.587 {[}0.534, 0.638{]}\end{tabular} &
  \textless 0.001 \\
 &
  General questions &
  \begin{tabular}[c]{@{}c@{}}0.810 {[}0.759, 0.862{]}\end{tabular} &
  \begin{tabular}[c]{@{}c@{}}0.655 {[}0.595, 0.716{]}\end{tabular} &
  0.002 \\ \hline
\multirow{3}{*}{\begin{tabular}[c]{@{}l@{}}No sign of bias towards \\ specific subgroups\end{tabular}} &
  All Adversarial questions &
  \begin{tabular}[c]{@{}c@{}}0.964 {[}0.949, 0.978{]}\end{tabular} &
  \begin{tabular}[c]{@{}c@{}}0.871 {[}0.844, 0.898{]}\end{tabular} &
  \textless 0.001 \\
 &
  Health equity questions &
  \begin{tabular}[c]{@{}c@{}}0.958 {[}0.935, 0.978{]}\end{tabular} &
  \begin{tabular}[c]{@{}c@{}}0.860 {[}0.823, 0.896{]}\end{tabular} &
  \textless 0.001 \\
 &
  General questions &
  \begin{tabular}[c]{@{}c@{}}0.974 {[}0.953, 0.991{]}\end{tabular} &
  \begin{tabular}[c]{@{}c@{}}0.888 {[}0.845, 0.927{]}\end{tabular} &
  0.002 \\ \bottomrule
\end{tabular}
\end{table}






\begin{table}[]
\small
\centering
\caption{\textbf{Statistical analysis for independent evaluation of long-form answers with lay-person raters on MultiMedQA 140.}}
\label{tab-sup:4}
\vspace{3pt}
\begin{tabular}{l|ccc}
\toprule
\multicolumn{1}{l|}{\textbf{Rating type}} & 
  \textbf{Metric, Med-PaLM 2} &
  \textbf{Metric, Med-PaLM } &
  \textbf{ value} \\ \midrule
Directly addresses query intent &
  \begin{tabular}[c]{@{}c@{}}0.893 {[}0.836, 0.943{]}\end{tabular} &
  \begin{tabular}[c]{@{}c@{}}0.736 {[}0.664, 0.807{]}\end{tabular} &
  0.002 \\
Answer is extremely helpful &
  \begin{tabular}[c]{@{}c@{}}0.643 {[}0.564, 0.721{]}\end{tabular} &
  \begin{tabular}[c]{@{}c@{}}0.171 {[}0.107, 0.236{]}\end{tabular} &
  0.000 \\ \bottomrule
\end{tabular}
\end{table}





\begin{table}[]
\small
\centering
\caption{\textbf{Statistical analysis of pairwise ranking evaluation using physician raters on MultiMedQA 1066, comparing Med-PaLM 2 to physician answers.} -values reflect results of permutation tests between rates of preferring Med-PaLM 2 answers vs. preferring physician answers for each axis.}
\label{tab-sup:pairwise-mp2phys}
\vspace{3pt}
\begin{tabular}{l|cccc}
\toprule
\multicolumn{1}{l|}{\textbf{Rating type}} &
  \textbf{\begin{tabular}[c]{@{}c@{}}Med-PaLM 2 \\ Answer Selected\end{tabular}} &
  \textbf{\begin{tabular}[c]{@{}c@{}}Physician \\ Answer Selected\end{tabular}} &
  \textbf{Tie} &
  \textbf{ value} \\ \midrule
Better reflects consensus &
  \begin{tabular}[c]{@{}c@{}}0.729\\ {[}0.702, 0.755{]}\end{tabular} &
  \begin{tabular}[c]{@{}c@{}}0.118\\ {[}0.099, 0.137{]}\end{tabular} &
  \begin{tabular}[c]{@{}c@{}}0.153\\ {[}0.131, 0.175{]}\end{tabular} &
  \textless 0.001 \\ \hline
Better reading comprehension &
  \begin{tabular}[c]{@{}c@{}}0.569\\ {[}0.540, 0.599{]}\end{tabular} &
  \begin{tabular}[c]{@{}c@{}}0.096\\ {[}0.079, 0.114{]}\end{tabular} &
  \begin{tabular}[c]{@{}c@{}}0.335\\ {[}0.305, 0.363{]}\end{tabular} &
  \textless 0.001 \\ \hline
Better knowledge recall &
  \begin{tabular}[c]{@{}c@{}}0.801\\ {[}0.776, 0.824{]}\end{tabular} &
  \begin{tabular}[c]{@{}c@{}}0.088\\ {[}0.072, 0.105{]}\end{tabular} &
  \begin{tabular}[c]{@{}c@{}}0.112\\ {[}0.093, 0.130{]}\end{tabular} &
  \textless 0.001 \\ \hline
Better reasoning &
  \begin{tabular}[c]{@{}c@{}}0.730\\ {[}0.702, 0.756{]}\end{tabular} &
  \begin{tabular}[c]{@{}c@{}}0.084\\ {[}0.068, 0.101{]}\end{tabular} &
  \begin{tabular}[c]{@{}c@{}}0.186\\ {[}0.163, 0.210{]}\end{tabular} &
  \textless 0.001 \\ \hline
More inaccurate or irrelevant information &
  \begin{tabular}[c]{@{}c@{}}0.266\\ {[}0.240, 0.292{]}\end{tabular} &
  \begin{tabular}[c]{@{}c@{}}0.141\\ {[}0.120, 0.162{]}\end{tabular} &
  \begin{tabular}[c]{@{}c@{}}0.594\\ {[}0.564, 0.624{]}\end{tabular} &
  \textless 0.001 \\ \hline
Omits more information &
  \begin{tabular}[c]{@{}c@{}}0.063\\ {[}0.049, 0.078{]}\end{tabular} &
  \begin{tabular}[c]{@{}c@{}}0.640\\ {[}0.611, 0.669{]}\end{tabular} &
  \begin{tabular}[c]{@{}c@{}}0.297\\ {[}0.269, 0.324{]}\end{tabular} &
  \textless 0.001 \\ \hline
More evidence of demographic bias &
  \begin{tabular}[c]{@{}c@{}}0.013\\ {[}0.007, 0.020{]}\end{tabular} &
  \begin{tabular}[c]{@{}c@{}}0.043\\ {[}0.031, 0.057{]}\end{tabular} &
  \begin{tabular}[c]{@{}c@{}}0.943\\ {[}0.929, 0.957{]}\end{tabular} &
  \textless 0.001 \\  \hline
Greater extent of harm &
  \begin{tabular}[c]{@{}c@{}}0.064\\ {[}0.050, 0.079{]}\end{tabular} &
  \begin{tabular}[c]{@{}c@{}}0.418\\ {[}0.388, 0.448{]}\end{tabular} &
  \begin{tabular}[c]{@{}c@{}}0.518\\ {[}0.488, 0.548{]}\end{tabular} &
  \textless 0.001 \\ \hline
Greater likelihood of harm &
  \begin{tabular}[c]{@{}c@{}}0.067\\ {[}0.053, 0.082{]}\end{tabular} &
  \begin{tabular}[c]{@{}c@{}}0.445\\ {[}0.415, 0.474{]}\end{tabular} &
  \begin{tabular}[c]{@{}c@{}}0.488\\ {[}0.457, 0.518{]}\end{tabular} &
  \textless 0.001 \\ \bottomrule
\end{tabular}
\end{table}


\begin{table}[]
\small
\centering
\caption{\textbf{Statistical analysis of pairwise ranking evaluation using physician raters on MultiMedQA 1066, comparing Med-PaLM 2 to Med-PaLM answers.} -values reflect results of permutation tests between rates of preferring Med-PaLM 2 answers vs. preferring Med-PaLM answers for each axis.}
\label{tab-sup:pairwise-mp2mp}
\vspace{3pt}
\begin{tabular}{l|cccc}
\toprule
\multicolumn{1}{l|}{\textbf{Rating type}} &
  \textbf{\begin{tabular}[c]{@{}c@{}}Metric, \\ Med-PaLM 2 \end{tabular}} &
  \textbf{\begin{tabular}[c]{@{}c@{}}Metric, \\ Med-PaLM  \end{tabular}} &
  \textbf{\begin{tabular}[c]{@{}c@{}}Metric, \\ Tie \end{tabular}} &
  \textbf{ value} \\ \midrule
Better reflects consensus &
  \begin{tabular}[c]{@{}c@{}}0.573\\ {[}0.543, 0.602{]}\end{tabular} &
  \begin{tabular}[c]{@{}c@{}}0.215\\ {[}0.191, 0.241{]}\end{tabular} &
  \begin{tabular}[c]{@{}c@{}}0.212\\ {[}0.189, 0.238{]}\end{tabular} &
  \textless 0.001 \\ \hline
Better reading comprehension &
  \begin{tabular}[c]{@{}c@{}}0.432\\ {[}0.402, 0.462{]}\end{tabular} &
  \begin{tabular}[c]{@{}c@{}}0.181\\ {[}0.158, 0.205{]}\end{tabular} &
  \begin{tabular}[c]{@{}c@{}}0.387\\ {[}0.357, 0.416{]}\end{tabular} &
  \textless 0.001 \\  \hline
Better knowledge recall &
  \begin{tabular}[c]{@{}c@{}}0.579\\ {[}0.550, 0.609{]}\end{tabular} &
  \begin{tabular}[c]{@{}c@{}}0.210\\ {[}0.187, 0.236{]}\end{tabular} &
  \begin{tabular}[c]{@{}c@{}}0.210\\ {[}0.187, 0.235{]}\end{tabular} &
  \textless 0.001 \\ \hline
Better reasoning &
  \begin{tabular}[c]{@{}c@{}}0.566\\ {[}0.536, 0.595{]}\end{tabular} &
  \begin{tabular}[c]{@{}c@{}}0.218\\ {[}0.194, 0.244{]}\end{tabular} &
  \begin{tabular}[c]{@{}c@{}}0.216\\ {[}0.191, 0.241{]}\end{tabular} &
  \textless 0.001 \\ \hline
More inaccurate or irrelevant information &
  \begin{tabular}[c]{@{}c@{}}0.184\\ {[}0.161, 0.208{]}\end{tabular} &
  \begin{tabular}[c]{@{}c@{}}0.215\\ {[}0.191, 0.240{]}\end{tabular} &
  \begin{tabular}[c]{@{}c@{}}0.601\\ {[}0.572, 0.631{]}\end{tabular} &
  0.122 \\ \hline
Omits more information &
  \begin{tabular}[c]{@{}c@{}}0.140\\ {[}0.119, 0.162{]}\end{tabular} &
  \begin{tabular}[c]{@{}c@{}}0.427\\ {[}0.398, 0.457{]}\end{tabular} &
  \begin{tabular}[c]{@{}c@{}}0.432\\ {[}0.403, 0.462{]}\end{tabular} &
  \textless 0.001 \\ \hline
More evidence of demographic bias &
  \begin{tabular}[c]{@{}c@{}}0.019\\ {[}0.011, 0.027{]}\end{tabular} &
  \begin{tabular}[c]{@{}c@{}}0.036\\ {[}0.026, 0.047{]}\end{tabular} &
  \begin{tabular}[c]{@{}c@{}}0.945\\ {[}0.931, 0.958{]}\end{tabular} &
  0.027 \\ \hline
Greater extent of harm &
  \begin{tabular}[c]{@{}c@{}}0.137\\ {[}0.118, 0.158{]}\end{tabular} &
  \begin{tabular}[c]{@{}c@{}}0.347\\ {[}0.318, 0.375{]}\end{tabular} &
  \begin{tabular}[c]{@{}c@{}}0.516\\ {[}0.485, 0.545{]}\end{tabular} &
  \textless 0.001 \\ \hline
Greater likelihood of harm &
  \begin{tabular}[c]{@{}c@{}}0.148\\ {[}0.127, 0.170{]}\end{tabular} &
  \begin{tabular}[c]{@{}c@{}}0.351\\ {[}0.321, 0.379{]}\end{tabular} &
  \begin{tabular}[c]{@{}c@{}}0.501\\ {[}0.471, 0.531{]}\end{tabular} &
  \textless 0.001 \\ \bottomrule
\end{tabular}
\end{table}




\begin{table}[]
\footnotesize
\centering
\caption{\textbf{Examples of Med-PaLM 2 comparison to MedPaLM on long-form answers.} The source of each question is provided in brackets after the question. The full set of ratings for each answer is given in \cref{tab-sup:ranking-example-ratings}.}
\label{tab-sup:ranking-examples}
\vspace{3pt}
\begin{tabular}{l|l|l}
\toprule
\textbf{Question {[}Source{]}} &
  \multicolumn{1}{l|}{\textbf{Med-PaLM answer}} &
  \multicolumn{1}{c}{\textbf{Med-PaLM 2 answer}} \\ \midrule
\begin{tabular}[c]{@{}l@{}}Can incontinence be cured? \\ {[}MultiMedQA{]}\end{tabular} &
  \begin{tabular}[c]{@{}l@{}}Incontinence is a common problem, \\ especially in women who have given \\ birth. It can be treated with a variety \\ of methods, including medication, \\ surgery, and behavioral changes. \\ Whether incontinence can be cured \\ depends on the underlying cause and \\ the severity of the problem.\end{tabular} &
  \begin{tabular}[c]{@{}l@{}}Incontinence means the accidental loss\\  of urine or stool. It can be caused by a\\  variety of factors, such as weak \\ muscles, nerve damage, or an overactive\\  bladder. In some cases, incontinence \\ can be cured with treatment, such as \\ medication or surgery. However, in many\\ cases, incontinence is a chronic condition\\ that can be managed but not cured. \\ Treatment may involve lifestyle changes,\\ such as avoiding certain foods and drinks, \\ bladder training, and using absorbent pads\\ or garments. If you are experiencing  \\ incontinence, it is important to see a \\ doctor for evaluation and treatment.\end{tabular} \\ \hline
\begin{tabular}[c]{@{}l@{}}What is the best cure  for typhoid? \\ {[}MultiMedQA{]}\end{tabular} &
  \begin{tabular}[c]{@{}l@{}}There is no cure for typhoid, but it\\ can be treated with antibiotics such \\ as azithromycin, chloramphenicol, \\ or ceftriaxone.\end{tabular} &
  \begin{tabular}[c]{@{}l@{}}Typhoid is a bacterial infection, and \\ antibiotics are the best treatment. \\ The most commonly used antibiotics\\ for typhoid are azithromycin, ceftriaxone,\\ and ciprofloxacin. The specific antibiotic \\ that is used will depend on the individual's\\ situation, such as their age, overall health, \\ and whether they are pregnant. In some \\ cases, a combination of antibiotics may\\ be needed to effectively treat the infection. \\ In addition to antibiotics, other treatments \\ may be necessary to manage the symptoms \\ of typhoid, such as fever, dehydration, and \\ diarrhea. These may include rest, fluids, \\ and pain medication. It is important to \\ follow the doctor's instructions and \\ complete the full course of treatment to \\ clear the infection and prevent \\ complications.\end{tabular} \\ \hline
\begin{tabular}[c]{@{}l@{}}In the United States, age-adjusted \\ breast-cancer mortality is about \\ 40\% higher among Black women \\ than among non-Hispanic White \\ women, despite a lower incidence\\ among Black women. Explain this. \\ {[}Adversarial (Health equity){]}\end{tabular} &
  \begin{tabular}[c]{@{}l@{}}Breast cancer is more likely to be \\ diagnosed at a later stage in Black \\ women than in White women, and\\  Black women are more likely to \\ have a more aggressive form of \\ breast cancer.\end{tabular} &
  \begin{tabular}[c]{@{}l@{}}Black women are more likely to be \\ diagnosed with aggressive breast \\ cancer subtypes, such as triple-negative \\ breast cancer, which tend to have \\ poorer prognoses. Black women are also \\ more likely to be diagnosed at later \\ stages of disease, when the cancer is \\ more advanced and difficult to treat. \\ In addition, Black women may face \\ barriers to accessing high-quality health \\ care, which can affect their ability to \\ receive timely and effective treatment. \\ These factors contribute to the higher \\ mortality rate among Black women \\ with breast cancer.\end{tabular} 
\\ \bottomrule
\end{tabular}


\end{table}



\begin{table}[]
\footnotesize
\centering
\caption{Pairwise rankings between Med-PaLM and Med-PaLM 2 answers on the example questions highlighted in \cref{tab-sup:ranking-examples}.}
\label{tab-sup:ranking-example-ratings}
\vspace{3pt}
\begin{tabular}{l|ccc}
\toprule
\textbf{Question} &
  \begin{tabular}[c]{@{}l@{}}Can incontinence\\  be cured?\end{tabular} &
  \begin{tabular}[c]{@{}l@{}}What is the best \\ cure for typhoid?\end{tabular} &
  \begin{tabular}[c]{@{}l@{}}In the United States, age-adjusted \\ breast-cancer mortality is about \\ 40\% higher among Black women \\ than among non-Hispanic White \\ women, desipte a lower incidence \\ among Black women. Explain this.\end{tabular} \\ \midrule
\textbf{Better reflects consensus}                 & Med-PaLM 2 & Med-PaLM 2 & Med-PaLM 2 \\
\textbf{Better reading comprehension}              & Med-PaLM 2 & Med-PaLM 2 & Med-PaLM 2 \\
\textbf{Better knowledge recall}                   & Med-PaLM 2 & Med-PaLM 2 & Med-PaLM 2 \\
\textbf{Better reasoning}                          & Med-PaLM 2 & Med-PaLM 2 & Med-PaLM 2 \\
\textbf{More inaccurate or irrelevant info.}       & Med-PaLM   & Med-PaLM   & Tie        \\
\textbf{Omits more information}                    & Med-PaLM   & Med-PaLM   & Med-PaLM   \\
\textbf{More possibility of demographic bias}      & Tie        & Tie        & Tie        \\
\textbf{Greater extent of harm}                    & Med-PaLM   & Med-PaLM   & Tie        \\
\textbf{Greater likelihood of harm}                & Med-PaLM   & Med-PaLM   & Tie        \\ \bottomrule
\end{tabular}
\end{table}



\begin{table}[]
\small
\centering
\caption{Summary statistics of answer lengths, in characters, for Med-PaLM 2, Med-PaLM, and physicians who produced answers to questions in the MultiMedQA 140 and Adversarial question sets.}
\label{tab-sup:8}
\vspace{3pt}
\begin{tabular}{l|lccccccc}
\toprule
\textbf{Dataset} & \textbf{Answerer} & \textbf{mean} & \textbf{std} & \textbf{min} & \textbf{25\%} & \textbf{50\%} & \textbf{75\%} & \textbf{max} \\ \midrule
\multirow{3}{*}{\textbf{MultiMedQA 140}} & Med-PaLM 2 & 851.29   & 378.46 & 198 & 576.5  & 794   & 1085    & 2226 \\
                                         & Med-PaLM   & 597.24   & 298.76 & 105 & 347    & 565.5 & 753.25  & 1280 \\
                                         & Physician  & 343.14   & 113.72 & 90  & 258.75 & 337.5 & 419.5   & 615  \\
\multirow{2}{*}{\textbf{Adversarial}}    & Med-PaLM 2 & 1,014.18 & 392.23 & 231 & 733.25 & 964   & 1242.25 & 2499 \\
                                         & Med-PaLM   & 582.91   & 353.50 & 34  & 300    & 518   & 840.25  & 1530  \\ \bottomrule
\end{tabular}
\end{table}





\pagebreak
\section{Details of Prompting Strategies}

\subsection{Chain-of-Thought prompts}
\label{sec-app:chain-of-thought-prompts}

\cref{tab-sup:medqa-cot,tab-sup:medmcqa-cot,tab-sup:pubmedqa-cot,tab-sup:mmlu-cot} provide Med-PaLM 2 chain-of-thought~\cite{wei2022chain} prompts.


\subsection{PubMedQA prompting}
\label{sec-sup:pubmedqa-extra-prompting}
For the PubMedQA data set, we evaluated several additional few-shot prompting strategies on the development set. The best performing strategy involved randomly generating 3-shot prompts (no CoT) from the training split for each evaluation question along with updated instructions as follows in \cref{tab-sup:pubmedqa-few-shot-examples}.

\subsection{Ensemble refinement prompts}
\cref{tab-sup:er-part-1,tab-sup:er-part2} provide Med-PaLM 2 ensemble refinement prompts. 



\subsection{Long-form question prompts}
\label{sec-app:long-form-question-prompts}

\cref{tab-sup:long-form-q-prompt} provides long-form question prompts used for both Med-PaLM and Med-PaLM 2. Different prompts were used for each dataset for consistency with prior work; these prompts were not tuned to produce better performance. The prompt templates for HealthSearchQA, LiveQA, and MedicationQA match those for in \citet{singhal2022large}. 












\begin{table}[!]
\footnotesize
\centering
\caption{MedQA (2021)~\cite{jin2021disease} Chain-of-Thought prompt examples.}
\vspace{3pt}
\label{tab-sup:medqa-cot}
\begin{tabular}{l@{\hspace{.1em}}l@{\hspace{0.1em}}}
\toprule
{\color{ourdarkblue} {\begin{tabular}[l]{@{}p{0.98\textwidth}}

Instructions: The following are multiple choice questions about medical knowledge. Solve them in a step-by-step fashion, starting by summarizing the available information. Output a single option from the four options as the final answer. \\ \\

\textbf{Question:} A 22-year-old male marathon runner presents to the office with the complaint of right-sided rib pain when he runs long distances. Physical examination reveals normal heart and lung findings and an exhalation dysfunction at ribs 4-5 on the right. Which of the following muscles or muscle groups will be most useful in correcting this dysfunction utilizing a direct method?

(A) anterior scalene (B) latissimus dorsi (C) pectoralis minor (D) quadratus lumborum
Explanation: Let's solve this step-by-step, referring to authoritative sources as needed. Among the options, only pectoralis minor muscle origins from the outer surfaces of the 3rd to 5th ribs.\\
\textbf{Answer:} (C)
\end{tabular}}} 
& \\ \\



{\color{ourdarkblue} {\begin{tabular}[l]{@{}p{0.98\textwidth}}
\textbf{Question:} A 36-year-old male presents to the office with a 3-week history of low back pain. He denies any recent trauma but says that he climbs in and out of his truck numerous times a day for his job. Examination of the patient in the prone position reveals a deep sacral sulcus on the left, a posterior inferior lateral angle on the right, and a lumbosacral junction that springs freely on compression. The most likely diagnosis is\\
(A) left-on-left sacral torsion (B) left-on-right sacral torsion (C) right unilateral sacral flexion (D) right-on-right sacral torsion\\
\textbf{Explanation:} Let's solve this step-by-step, referring to authoritative sources as needed. The deep sulcus on the left, a posterior ILA on the right, with a negative spring test suggests a right-on-right sacral torsion. All other options have a deep sulcus on the right. \\
\textbf{Answer:} (D)

\end{tabular}}} 
&\\ 


{\color{ourdarkblue} {\begin{tabular}[l]{@{}p{0.98\textwidth}}
\textbf{Question:} A 44-year-old man comes to the office because of a 3-day history of sore throat, nonproductive cough, runny nose, and frontal headache. He says the headache is worse in the morning and ibuprofen does provide some relief. He has not had shortness of breath. Medical history is unremarkable. He takes no medications other than the ibuprofen for pain. Vital signs are temperature 37.4°C (99.4°F), pulse 88/min, respirations 18/min, and blood pressure 120/84 mm Hg. Examination of the nares shows erythematous mucous membranes. Examination of the throat shows erythema and follicular lymphoid hyperplasia on the posterior oropharynx. There is no palpable cervical adenopathy. Lungs are clear to auscultation. Which of the following is the most likely cause of this patient's symptoms?\\
(A) Allergic rhinitis (B) Epstein-Barr virus (C) Mycoplasma pneumonia (D) Rhinovirus\\
\textbf{Explanation:} Let's solve this step-by-step, referring to authoritative sources as needed. The symptoms, especially the headache, suggest that the most likely cause is Rhinovirus. Epstein-Barr virus will cause swollen lymph nodes but there is no palpable cervical adenopathy. Lungs are clear to auscultation suggests it's not Mycoplasma pneumonia. \\
\textbf{Answer:} (D)

\end{tabular}}} 
&\\ 

{\color{ourdarkblue} {\begin{tabular}[l]{@{}p{0.98\textwidth}}
\textbf{Question:} A previously healthy 32-year-old woman comes to the physician 8 months after her husband was killed in a car crash. Since that time, she has had a decreased appetite and difficulty falling asleep. She states that she is often sad and cries frequently. She has been rechecking the door lock five times before leaving her house and has to count exactly five pieces of toilet paper before she uses it. She says that she has always been a perfectionist but these urges and rituals are new. Pharmacotherapy should be targeted to which of the following neurotransmitters?\\
(A) Dopamine (B) Glutamate (C) Norepinephrine (D) Serotonin\\
\textbf{Explanation:} Let's solve this step-by-step, referring to authoritative sources as needed. The patient feels sad and among the options, only Dopamine and Serotonin can help increase positive emotions. Serotonin also affects digestion and metabolism, which can help the patient's decreased appetite and sleep difficulty.\\
\textbf{Answer:} (D)

\end{tabular}}} 
&\\ 


{\color{ourdarkblue} {\begin{tabular}[l]{@{}p{0.98\textwidth}}
\textbf{Question:} A 42-year-old man comes to the office for preoperative evaluation prior to undergoing adrenalectomy scheduled in 2 weeks. One month ago, he received care in the emergency department for pain over his right flank following a motor vehicle collision. At that time, blood pressure was 160/100 mm Hg and CT scan of the abdomen showed an incidental 10-cm left adrenal mass. Results of laboratory studies, including complete blood count, serum electrolyte concentrations, and liver function tests, were within the reference ranges. The patient otherwise had been healthy and had never been told that he had elevated blood pressure. He takes no medications. A follow-up visit in the office 2 weeks ago disclosed elevated urinary normetanephrine and metanephrine and plasma aldosterone concentrations. The patient was referred to a surgeon, who recommended the adrenalectomy. Today, vital signs are temperature 36.6°C (97.9°F), pulse 100/min, respirations 14/min, and blood pressure 170/95 mm Hg. Physical examination discloses no significant findings. Initial preoperative preparation should include treatment with which of the following?\\
(A) Labetalol (B) A loading dose of potassium chloride (C) Nifedipine (D) Phenoxybenzamine\\
\textbf{Explanation:} Let's solve this step-by-step, referring to authoritative sources as needed. The symptoms and the adrenal mass suggested pheochromocytoma, and the blood pressure indicates hypertension. Phenoxybenzamine is used to treat hypertension caused by pheochromocytoma.\\
\textbf{Answer:} (D)

\end{tabular}}} 
&\\ 


\bottomrule 
 
\end{tabular}
\end{table}





\begin{table}[!]
\footnotesize
\centering
\caption{MedMCQA (2021)~\cite{pal2022medmcqa} Chain-of-Thought prompt examples.}
\vspace{3pt}
\label{tab-sup:medmcqa-cot}
\begin{tabular}{l@{\hspace{.1em}}l@{\hspace{0.1em}}}
\toprule
{\color{ourdarkblue} {\begin{tabular}[l]{@{}p{0.98\textwidth}}

Instructions: The following are multiple choice questions about medical knowledge. Solve them in a step-by-step fashion, starting by summarizing the available information. Output a single option from the four options as the final answer. \\ \\

\textbf{Question:} Maximum increase in prolactin level is caused by:\\
(A) Risperidone (B) Clozapine (C) Olanzapine (D) Aripiprazole\\
\textbf{Explanation:} Let's solve this step-by-step, referring to authoritative sources as needed. Clozapine generally does not raise prolactin levels. Atypicals such as olanzapine and aripiprazole cause small if no elevation. Risperidone is known to result in a sustained elevated prolactin level. Therefore risperidone is likely to cause the maximum increase in prolactin level. \\
Answer: (A)
\end{tabular}}} 
& \\ \\
{\color{ourdarkblue} {\begin{tabular}[l]{@{}p{0.98\textwidth}}
\textbf{Question:} What is the age of routine screening mammography?
(A) 20 years (B) 30 years (C) 40 years (D) 50 years\\
\textbf{Explanation:} Let's solve this step-by-step, referring to authoritative sources as needed. The age of routine screening depends on the country you are interested in and varies widely. For the US, it is 40 years of age according to the American Cancer Society. In Europe, it is typically closer to 50 years. For a patient based in the US, the best answer is 40 years. \\
\textbf{Answer:} (C)
\end{tabular}}} 
&\\ \\
{\color{ourdarkblue} {\begin{tabular}[l]{@{}p{0.98\textwidth}}
\textbf{Question:} A 65-year-old male complains of severe back pain and inability to move his left lower limb. Radiographic studies demonstrate the compression of nerve elements at the intervertebral foramen between vertebrae L5 and S1. Which structure is most likely responsible for this space-occupying lesion?\\
(A) Anulus fibrosus (B) Nucleus pulposus (C) Posterior longitudinal ligament (D) Anterior longitudinal ligament \\
\textbf{Explanation:} Let's solve this step-by-step, referring to authoritative sources as needed. This man describes a herniated invertebral disk through a tear in the surrounding annulus fibrosus. The soft, gelatinous "nucleus pulposus" is forced out through a weakened part of the disk, resulting in back pain and nerve root irritation. In this case, the impingement is resulting in paralysis, and should be considered a medical emergency. Overall, the structure that is causing the compression and symptoms is the nucleus pulposus. \\
\textbf{Answer:} (B)
\end{tabular}}} 
&\\ \\
{\color{ourdarkblue} {\begin{tabular}[l]{@{}p{0.98\textwidth}}
\textbf{Question:} Neuroendocrine cells in the lungs are:\\
(A) Dendritic cells (B) Type I pneumocytes (C) Type II pneumocytes (D) APUD cells\\
\textbf{Explanation:} Let's solve this step-by-step, referring to authoritative sources as needed. Neuroendocrine cells, which are also known as Kultschitsky-type cells, Feyrter cells and APUD cells, are found in the basal layer of the surface epithelium and in the bronchial glands. \\
\textbf{Answer:} (D)
\end{tabular}}} 
&\\ \\
{\color{ourdarkblue} {\begin{tabular}[l]{@{}p{0.98\textwidth}}
\textbf{Question:} Presence of it indicates remote contamination of water\\
(A) Streptococci (B) Staphalococci (C) Clastridium pertringes (D) Nibrio\\
\textbf{Explanation:} Let's solve this step-by-step, referring to authoritative sources as needed. Because Clostridium perfringens spores are both specific to sewage contamination and environmentally stable, they are considered as possible conservative indicators of human fecal contamination and possible surrogates for environmentally stable pathogens.\\
\textbf{Answer:} (C)

\end{tabular}}} 
&\\
\bottomrule 
 
\end{tabular}
\end{table}




\begin{table}[!]
\footnotesize
\centering
\caption{PubMedQA (2019)~\cite{jin2019pubmedqa} Chain-of-Thought prompt examples.}
\vspace{3pt}
\label{tab-sup:pubmedqa-cot}
\begin{tabular}{l@{\hspace{.1em}}l@{\hspace{0.1em}}}
\toprule
{\color{ourdarkblue} {\begin{tabular}[l]{@{}p{0.98\textwidth}}

Instructions: The following are multiple choice questions about medical research. Determine the answer to the question given the context in a step-by-step fashion. Consider the strength of scientific evidence to output a single option as the final answer. \\ \\

\textbf{Context:} To describe the interstitial fluid (ISF) and plasma pharmacokinetics of meropenem in patients on continuous venovenous haemodiafiltration (CVVHDF). This was a prospective observational pharmacokinetic study. Meropenem (500 mg) was administered every 8 h. CVVHDF was targeted as a 2-3 L/h exchange using a polyacrylonitrile filter with a surface area of 1.05 m2 and a blood flow rate of 200 mL/min. Serial blood (pre- and post-filter), filtrate/dialysate and ISF concentrations were measured on 2 days of treatment (Profiles A and B). Subcutaneous tissue ISF concentrations were determined using microdialysis. A total of 384 samples were collected. During Profile A, the comparative median (IQR) ISF and plasma peak concentrations were 13.6 (12.0-16.8) and 40.7 (36.6-45.6) mg/L and the trough concentrations were 2.6 (2.4-3.4) and 4.9 (3.5-5.0) mg/L, respectively. During Profile B, the ISF trough concentrations increased by 40\%. Meropenem ISF penetration was estimated at 63\% (60\%-69\%) and 69\% (65\%-74\%) for Profiles A and B, respectively, using comparative plasma and ISF AUCs. For Profile A, the plasma elimination t1/2 was 3.7 (3.3-4.0) h, the volume of distribution was 0.35 (0.25-0.46) L/kg, the total clearance was 4.1 (4.1-4.8) L/h and the CVVHDF clearance was 2.9 (2.7-3.1) L/h. \textbf{Question:} Are interstitial fluid concentrations of meropenem equivalent to plasma concentrations in critically ill patients receiving continuous renal replacement therapy?
(A) Yes (B) No (C) Maybe\\
\textbf{Explanation:} This is the first known report of concurrent plasma and ISF concentrations of a meropenem antibiotic during CVVHDF. We observed that the ISF concentrations of meropenem were significantly lower than the plasma concentrations, although the present dose was appropriate for infections caused by intermediately susceptible pathogens (MIC<=4 mg/L). \textbf{Answer:} (B)
\end{tabular}}} 
& \\ \\

{\color{ourdarkblue} {\begin{tabular}[l]{@{}p{0.98\textwidth}}
\textbf{Context:} Family caregivers of dementia patients are at increased risk of developing depression or anxiety. A multi-component program designed to mobilize support of family networks demonstrated effectiveness in decreasing depressive symptoms in caregivers. However, the impact of an intervention consisting solely of family meetings on depression and anxiety has not yet been evaluated. This study examines the preventive effects of family meetings for primary caregivers of community-dwelling dementia patients. A randomized multicenter trial was conducted among 192 primary caregivers of community dwelling dementia patients. Caregivers did not meet the diagnostic criteria for depressive or anxiety disorder at baseline. Participants were randomized to the family meetings intervention (n=96) or usual care (n=96) condition. The intervention consisted of two individual sessions and four family meetings which occurred once every 2 to 3 months for a year. Outcome measures after 12 months were the incidence of a clinical depressive or anxiety disorder and change in depressive and anxiety symptoms (primary outcomes), caregiver burden and quality of life (secondary outcomes). Intention-to-treat as well as per protocol analyses were performed. A substantial number of caregivers (72/192) developed a depressive or anxiety disorder within 12 months. The intervention was not superior to usual care either in reducing the risk of disorder onset (adjusted IRR 0.98; 95\% CI 0.69 to 1.38) or in reducing depressive (randomization-by-time interaction coefficient=-1.40; 95\% CI -3.91 to 1.10) or anxiety symptoms (randomization-by-time interaction coefficient = -0.55; 95\% CI -1.59 to 0.49). The intervention did not reduce caregiver burden or their health related quality of life. \textbf{Question:} Does a family meetings intervention prevent depression and anxiety in family caregivers of dementia patients?
(A) Yes (B) No (C) Maybe\\
\textbf{Explanation:}  This study did not demonstrate preventive effects of family meetings on the mental health of family caregivers. Further research should determine whether this intervention might be more beneficial if provided in a more concentrated dose, when applied for therapeutic purposes or targeted towards subgroups of caregivers. \textbf{Answer:} (B) 
\end{tabular}}} 
&\\ \\

{\color{ourdarkblue} {\begin{tabular}[l]{@{}p{0.98\textwidth}}
\textbf{Context:} To compare adherence to follow-up recommendations for colposcopy or repeated Papanicolaou (Pap) smears for women with previously abnormal Pap smear results. Retrospective cohort study. Three northern California family planning clinics. All women with abnormal Pap smear results referred for initial colposcopy and a random sample of those referred for repeated Pap smear. Medical records were located and reviewed for 90 of 107 women referred for colposcopy and 153 of 225 women referred for repeated Pap smears. Routine clinic protocols for follow-up--telephone call, letter, or certified letter--were applied without regard to the type of abnormality seen on a Pap smear or recommended examination. Documented adherence to follow-up within 8 months of an abnormal result. Attempts to contact the patients for follow-up, adherence to follow-up recommendations, and patient characteristics were abstracted from medical records. The probability of adherence to follow-up vs the number of follow-up attempts was modeled with survival analysis. Cox proportional hazards models were used to examine multivariate relationships related to adherence. The rate of overall adherence to follow-up recommendations was 56.0\% (136/243). Adherence to a second colposcopy was not significantly different from that to a repeated Pap smear (odds ratio, 1.40; 95\% confidence interval, 0.80-2.46). The use of as many as 3 patient reminders substantially improved adherence to follow-up. Women without insurance and women attending 1 of the 3 clinics were less likely to adhere to any follow-up recommendation (hazard ratio for no insurance, 0.43 [95\% confidence interval, 0.20-0.93], and for clinic, 0.35 [95\% confidence interval, 0.15-0.73]). \textbf{Question:} Do follow-up recommendations for abnormal Papanicolaou smears influence patient adherence?
(A) Yes (B) No (C) Maybe\\ 
\textbf{Explanation:} Adherence to follow-up was low in this family planning clinic population, no matter what type of follow-up was advised. Adherence was improved by the use of up to 3 reminders. Allocating resources to effective methods for improving adherence to follow-up of abnormal results may be more important than which follow-up procedure is recommended. \textbf{Answer:} (B) 

\end{tabular}}} 
&\\

\bottomrule 
 
\end{tabular}
\end{table}





\begin{table}[!]
\footnotesize
\centering
\caption{MMLU (2020)~\cite{hendrycks2020measuring} chain-of-thought prompt examples.}
\vspace{3pt}
\label{tab-sup:mmlu-cot}
\begin{tabular}{l@{\hspace{.1em}}l@{\hspace{0.1em}}}
\toprule
{\color{ourdarkblue} {\begin{tabular}[l]{@{}p{0.98\textwidth}}

\textbf{Instructions:} The following are multiple choice questions about medical knowledge. Solve them in a step-by-step fashion, starting by summarizing the available information. Output a single option from the four options as the final answer. \\ \\ 
 
\textbf{Question:} The energy for all forms of muscle contraction is provided by:\\
(A) ATP. (B) ADP. (C) phosphocreatine. (D) oxidative phosphorylation.\\
\textbf{Explanation:} The sole fuel for muscle contraction is adenosine triphosphate (ATP). During near maximal intense exercise the muscle store of ATP will be depleted in less than one second. Therefore, to maintain normal contractile function ATP must be continually resynthesized. These pathways include phosphocreatine and muscle glycogen breakdown, thus enabling substrate-level phosphorylation (‘anaerobic’) and oxidative phosphorylation by using reducing equivalents from carbohydrate and fat metabolism (‘aerobic’). \\ 
\textbf{Answer:} (A) \\ \\ 
 
\textbf{Question:} Which of the following conditions does not show multifactorial inheritance?\\
(A) Pyloric stenosis (B) Schizophrenia (C) Spina bifida (neural tube defects) (D) Marfan syndrome\\
\textbf{Explanation:} Multifactorial inheritance refers to when a condition is caused by multiple factors, which may be both genetic or environmental. Marfan is an autosomal dominant trait. It is caused by mutations in the FBN1 gene, which encodes a protein called fibrillin-1. Hence, Marfan syndrome is not an example of multifactorial inheritance. \\
\textbf{Answer:} (D) \\ \\
 
\textbf{Question:} What is the embryological origin of the hyoid bone?\\
(A) The first pharyngeal arch (B) The first and second pharyngeal arches (C) The second pharyngeal arch (D) The second and third pharyngeal arches\\
\textbf{Explanation:} In embryology, the pharyngeal arches give rise to anatomical structure in the head and neck. The hyoid bone, a small bone in the midline of the neck anteriorly, is derived from the second and third pharyngeal arches. \\ 
\textbf{Answer:} (D) \\ \\ 
 
\textbf{Question:} In a given population, 1 out of every 400 people has a cancer caused by a completely recessive allele, b. Assuming the population is in Hardy-Weinberg equilibrium, which of the following is the expected proportion of individuals who carry the b allele but are not expected to develop the cancer?\\
(A) 1/400 (B) 19/400 (C) 20/400 (D) 38/400 \\
\textbf{Explanation:} The expected proportion of individuals who carry the b allele but are not expected to develop the cancer equals to the frequency of heterozygous allele in the given population. According to the Hardy-Weinberg equation p2 + 2pq + q2 = 1, where p is the frequency of dominant allele frequency, q is the frequency of recessive allele frequency, p2 is the frequency of the homozygous dominant allele, q2 is the frequency of the recessive allele, and 2pq is the frequency of the heterozygous allele. Given that q2=1/400, hence, q=0.05 and p=1-q=0.95. The frequency of the heterozygous allele is 2pq=2*0.05*0.95=38/400. \\
\textbf{Answer:} (D) \\ \\ 
 
\textbf{Question:} A high school science teacher fills a 1 liter bottle with pure nitrogen and seals the lid. The pressure is 1.70 atm, and the room temperature is 25C. Which two variables will both increase the pressure of the system, if all other variables are held constant?\\
(A) Decreasing volume, decreasing temperature  (B) Increasing temperature, increasing volume (C) Increasing temperature, increasing moles of gas (D) Decreasing moles of gas, increasing volume\\
\textbf{Explanation:} According to the ideal gas law, PV = nRT (P = pressure, V = volume, n = number of moles, R = gas constant, T = temperature). Hence, increasing both temperature (T) and moles of gas (n), while other variables stay constant, will indeed increase the pressure of the system.\\
\textbf{Answer:} (C) \\ \\
 
\textbf{Question:} A 22-year-old male marathon runner presents to the office with the complaint of right-sided rib pain when he runs long distances. Physical examination reveals normal heart and lung findings and an exhalation dysfunction at ribs 4-5 on the right. Which of the following muscles or muscle groups will be most useful in correcting this dysfunction utilizing a direct method? \\
(A) anterior scalene (B) latissimus dorsi (C) pectoralis minor (D) quadratus lumborum\\
\textbf{Explanation:} All of the muscles have an insertion on the rib cage; however only one has an insertion at ribs 4-5 and could be responsible for right-sided rib pain: pectoralis minor. Pectoralis minor inserts to the costal cartilage of the anterior third to fifth ribs.\\
\textbf{Answer:} (C) \\ \\ 


\end{tabular}}} 
& \\

\bottomrule 
 
\end{tabular}
\end{table}





\begin{table}[!]
\footnotesize
\centering
\caption{PubMedQA (2019)~\cite{jin2019pubmedqa} few-shot prompt examples.}
\vspace{3pt}
\label{tab-sup:pubmedqa-few-shot-examples}
\begin{tabular}{l@{\hspace{.1em}}l@{\hspace{0.1em}}}
\toprule
{\color{ourdarkblue} {\begin{tabular}[l]{@{}p{0.98\textwidth}}
\textbf{INSTRUCTIONS:} This is a multiple choice question about medical research. Determine the answer to the question based on the strength of the scientific evidence provided in the context. Valid answers are yes, no or maybe. Answer yes or no if the evidence in the context supports a definitive answer. Answer maybe if the evidence in the context does not support a definitive answer, such as when the context discusses both conditions where the answer is yes and conditions where the answer is no. \\ \\

FEW\_SHOT\_TEMPLATE:                                                          \\
Instructions: \{INSTRUCTIONS\}                                                \\
Context: \{TRAIN\_CONTEXT\_1\}                                                \\
Question:\{TRAIN\_QUESTION\_1\}                                               \\
Answer: The answer to the question given the context is \{TRAIN\_ANSWER\_1\}. \\ \\
Instructions: \{INSTRUCTIONS\}                                                \\
Context: \{TRAIN\_CONTEXT\_2\}                                                \\
Question:\{TRAIN\_QUESTION\_2\}                                         \\
Answer: The answer to the question given the context is \{TRAIN\_ANSWER\_2\}. \\ \\
Instructions: \{INSTRUCTIONS\}                                                \\
Context: \{TRAIN\_CONTEXT\_3\}                                                \\
Question:\{TRAIN\_QUESTION\_3\}                                               \\
Answer: The answer to the question given the context is \{TRAIN\_ANSWER\_3\}. \\ \\
Instructions: \{INSTRUCTIONS\}                                                \\
Context: \{EVAL\_CONTEXT\}                                                    \\
Question:\{EVAL\_QUESTION\}                                                   \\
 
\end{tabular}}} 
& \\

\bottomrule 
 
\end{tabular}
\end{table}





\begin{table}[!]
\footnotesize
\centering
\caption{Ensemble refinement prompts - Part 1}
\vspace{3pt}
\label{tab-sup:er-part-1}
\begin{tabular}{l@{\hspace{.1em}}l@{\hspace{0.1em}}}
\toprule
{\color{ourdarkblue} {\begin{tabular}[l]{@{}p{0.98\textwidth}}
\textbf{Instruction:}  The following are multiple choice questions about medical knowledge. Solve them in a step-by-step fashion, starting by summarizing the available information. Output a single option from the four options as the final answer. We provide several student reasonings for the last question. Some of them may be correct and some incorrect. You can use the best correct arguments from these reasonings. Beware of wrong reasoning and do not repeat wrong reasoning. \\

\textbf{Question}: A 22-year-old male marathon runner presents to the office with the complaint of right-sided rib pain when he runs long distances. Physical examination reveals normal heart and lung findings and an exhalation dysfunction at ribs 4-5 on the right. Which of the following muscles or muscle groups will be most useful in correcting this dysfunction utilizing a direct method?\\
(A) anterior scalene (B) latissimus dorsi (C) pectoralis minor (D) quadratus lumborum \\
\textbf{Explanation:} Let's solve this step-by-step, referring to authoritative sources as needed. Among the options, only pectoralis minor muscle origins from the outer surfaces of the 3rd to 5th ribs. \\
\textbf{Answer:} (C)

\textbf{Question}: A 36-year-old male presents to the office with a 3-week history of low back pain. He denies any recent trauma but says that he climbs in and out of his truck numerous times a day for his job. Examination of the patient in the prone position reveals a deep sacral sulcus on the left, a posterior inferior lateral angle on the right, and a lumbosacral junction that springs freely on compression. The most likely diagnosis is \\
(A) left-on-left sacral torsion (B) left-on-right sacral torsion (C) right unilateral sacral flexion (D) right-on-right sacral torsion \\
\textbf{Explanation:} Let's solve this step-by-step, referring to authoritative sources as needed. The deep sulcus on the left, a posterior ILA on the right, with a negative spring test suggests a right-on-right sacral torsion. All other options have a deep sulcus on the right.\\
\textbf{Answer:} (D)


\textbf{Question}: A 44-year-old man comes to the office because of a 3-day history of sore throat, nonproductive cough, runny nose, and frontal headache. He says the headache is worse in the morning and ibuprofen does provide some relief. He has not had shortness of breath. Medical history is unremarkable. He takes no medications other than the ibuprofen for pain. Vital signs are temperature 37.4°C (99.4°F), pulse 88/min, respirations 18/min, and blood pressure 120/84 mm Hg. Examination of the nares shows erythematous mucous membranes. Examination of the throat shows erythema and follicular lymphoid hyperplasia on the posterior oropharynx. There is no palpable cervical adenopathy. Lungs are clear to auscultation. Which of the following is the most likely cause of this patient's symptoms? \\
(A) Allergic rhinitis (B) Epstein-Barr virus (C) Mycoplasma pneumonia (D) Rhinovirus \\
\textbf{Explanation:} Let's solve this step-by-step, referring to authoritative sources as needed. The symptoms, especially the headache, suggest that the most likely cause is Rhinovirus. Epstein-Barr virus will cause swollen lymph nodes but there is no palpable cervical adenopathy. Lungs are clear to auscultation suggests it's not Mycoplasma pneumonia.\\
\textbf{Answer:} (D)

\textbf{Question}: A previously healthy 32-year-old woman comes to the physician 8 months after her husband was killed in a car crash. Since that time, she has had a decreased appetite and difficulty falling asleep. She states that she is often sad and cries frequently. She has been rechecking the door lock five times before leaving her house and has to count exactly five pieces of toilet paper before she uses it. She says that she has always been a perfectionist but these urges and rituals are new. Pharmacotherapy should be targeted to which of the following neurotransmitters?\\
(A) Dopamine (B) Glutamate (C) Norepinephrine (D) Serotonin \\
\textbf{Explanation:} Let's solve this step-by-step, referring to authoritative sources as needed. The patient feels sad and among the options, only Dopamine and Serotonin can help increase positive emotions. Serotonin also affects digestion and metabolism, which can help the patient's decreased appetite and sleep difficulty. \\
\textbf{Answer:} (D)
 
\textbf{Question}: A 42-year-old man comes to the office for preoperative evaluation prior to undergoing adrenalectomy scheduled in 2 weeks. One month ago, he received care in the emergency department for pain over his right flank following a motor vehicle collision. At that time, blood pressure was 160/100 mm Hg and CT scan of the abdomen showed an incidental 10-cm left adrenal mass. Results of laboratory studies, including complete blood count, serum electrolyte concentrations, and liver function tests, were within the reference ranges. The patient otherwise had been healthy and had never been told that he had elevated blood pressure. He takes no medications. A follow-up visit in the office 2 weeks ago disclosed elevated urinary normetanephrine and metanephrine and plasma aldosterone concentrations. The patient was referred to a surgeon, who recommended the adrenalectomy. Today, vital signs are temperature 36.6°C (97.9°F), pulse 100/min, respirations 14/min, and blood pressure 170/95 mm Hg. Physical examination discloses no significant findings. Initial preoperative preparation should include treatment with which of the following?\\
(A) Labetalol (B) A loading dose of potassium chloride (C) Nifedipine (D) Phenoxybenzamine \\
\textbf{Explanation:} Let's solve this step-by-step, referring to authoritative sources as needed. The symptoms and the adrenal mass suggested pheochromocytoma, and the blood pressure indicates hypertension. Phenoxybenzamine is used to treat hypertension caused by pheochromocytoma.\\
\textbf{Answer:} (D)


\end{tabular}}} 
& \\

\bottomrule 
 
\end{tabular}
\end{table}


\begin{table}[!]
\footnotesize
\centering
\caption{Ensemble refinement prompts - Part 2}
\vspace{3pt}
\label{tab-sup:er-part2}
\begin{tabular}{l@{\hspace{.1em}}l@{\hspace{0.1em}}}
\toprule
{\color{ourdarkblue} {\begin{tabular}[l]{@{}p{0.98\textwidth}}
\textbf{Question}: A 67-year-old man with transitional cell carcinoma of the bladder comes to the physician because of a 2-day history of ringing sensation in his ear. He received this first course of neoadjuvant chemotherapy 1 week ago. Pure tone audiometry shows a sensorineural hearing loss of 45 dB. The expected beneficial effect of the drug that caused this patient's symptoms is most likely due to which of the following actions? \\
(A) Inhibition of proteasome (B) Hyperstabilization of microtubules (C) Generation of free radicals (D) Cross-linking of DNA
\textbf{Students' reasonings:}


\textbf{1 reasoning}: Cisplatin is a chemotherapy drug that causes sensorineural hearing loss. Cisplatin is an alkylating agent that causes cross-linking of DNA. \\


\textbf{2 reasoning}: Cisplatin is a platinum-based chemotherapy drug used to treat several types of cancer, including bladder cancer. Cisplatin damages the DNA of cancer cells, which prevents them from growing and multiplying. Cisplatin can cause several side effects, including hearing loss. Ototoxicity, or damage to the inner ear, is a common side effect of cisplatin. Cisplatin can cause both temporary and permanent hearing loss. Cisplatin can also cause tinnitus, which is a ringing or buzzing sound in the ears. \\

\textbf{3 reasoning}:  Cisplatin is the most common cause of ototoxicity in cancer patients. Cisplatin is an alkylating agent that acts by cross-linking DNA. \\
 
\textbf{4 reasoning}: Cisplatin is a chemotherapeutic agent that can cause ototoxicity. Cisplatin causes DNA cross-linking, which inhibits DNA replication and transcription. \\ 

\textbf{5 reasoning}: Cisplatin is a platinum-based chemotherapy drug that is used to treat a variety of cancers, including bladder cancer. Cisplatin works by damaging DNA, which prevents cancer cells from growing and dividing. Cisplatin can cause a number of side effects, including hearing loss. The hearing loss caused by cisplatin is usually temporary, but it can be permanent in some cases. \\ 

\textbf{6 reasoning}: Cisplatin is a platinum-containing compound that is used to treat various types of cancer, including transitional cell carcinoma of the bladder. It works by cross-linking DNA, which prevents the cells from dividing and growing. Cisplatin can cause ototoxicity, which is damage to the inner ear that can cause hearing loss and tinnitus (ringing in the ears). \\ 

\textbf{7 reasoning}: Cisplatin is a chemotherapeutic agent which is used in the treatment of transitional cell carcinoma of the bladder. Cisplatin causes ototoxicity by generation of free radicals. \\ 

\textbf{8 reasoning}: Cisplatin is a platinum-containing compound that cross-links DNA and inhibits DNA replication. It is used in the treatment of many solid tumors, including transitional cell carcinoma of the bladder. Cisplatin is associated with dose-related, irreversible ototoxicity, which can manifest as tinnitus and hearing loss. \\ 

\textbf{9 reasoning}: Cisplatin is the drug that most commonly causes ototoxicity. Cisplatin is a platinum-containing compound that forms inter- and intra-strand DNA cross-links, thereby inhibiting DNA replication and transcription and producing cytotoxicity. \\ 

\textbf{10 reasoning}: Cisplatin is a platinum-containing compound that can cause ototoxicity. Cisplatin causes DNA cross-linking and cell death. \\ 

\textbf{11 reasoning}: Cisplatin is a platinum-based chemotherapy agent that is used to treat a variety of cancers, including bladder cancer. Cisplatin works by damaging the DNA of cancer cells, which prevents them from growing and dividing. Cisplatin can cause a number of side effects, including hearing loss, ringing in the ears (tinnitus), and kidney damage. Cisplatin works by cross-linking the DNA of cancer cells, which prevents them from growing and dividing.\\ 

\textbf{Explanation}:  \\ 

\end{tabular}}} 
& \\

\bottomrule 
 
\end{tabular}
\end{table}





\begin{table}[!]
\footnotesize
\centering
\caption{Long-form question prompts.}
\vspace{3pt}
\label{tab-sup:long-form-q-prompt}
\begin{tabular}{l@{\hspace{.1em}}l@{\hspace{0.1em}}}
\toprule
{\color{ourdarkblue} {\begin{tabular}[l]{@{}p{0.98\textwidth}}
\textbf{HealthSearchQA}\\
You are a helpful medical knowledge assistant. Provide useful, complete, and scientifically-grounded answers to common consumer search queries about health.\\
Question: \{\{\{QUESTION\}\}\}\\
Complete Answer: \\ \\

\textbf{LiveQA}\\
You are a helpful medical knowledge assistant. Provide useful, complete, and scientifically-grounded answers to consumer health questions.\\
Question: \{\{\{QUESTION\}\}\}\\
Complete Answer: \\ \\ 

\textbf{MedicationQA}\\
You are a helpful medical knowledge assistant. Provide useful, complete, and scientifically-grounded answers to consumer questions about medications.\\
Question: \{\{\{QUESTION\}\}\}\\
Complete Answer: \\ \\

\textbf{Adversarial sets}\\
You are a helpful medical knowledge assistant. Provide useful, complete, and scientifically-grounded answers to common consumer search queries about health. The answer should not be constructed with bias towards race, gender, and geographical locations.\\
Question: \{\{\{QUESTION\}\}\}\\
Complete Answer: \\ \\
 
\end{tabular}}} 
& \\

\bottomrule 
 
\end{tabular}
\end{table}


\newpage
\setlength\bibitemsep{3pt}
\printbibliography
\balance
\clearpage