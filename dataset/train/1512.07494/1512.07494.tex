This section introduces some basics about graphs, graph edit distance
and edit paths. We futher introduce different familly of edit paths
and show that one of this familly is in direct correspondence with a
familly of mapping functions between the set of nodes of two graphs.
\subsection{Graph Basics}\label{sec:mainDef}
\begin{definition}[Unlabeled graph]
  An unlabeled  graph  is defined by the couple
   where  is the set of nodes and  is the set of edges. Each edge is an orded couple
  of nodes  with . The direction of an edge is
  implicitly given by the order of its nodes, 
  i.e. the direction of  is from  to .
\end{definition}
\begin{definition}[Nodes Adjacency]
Given a graph  and two nodes .  and  are
  said to be adjacent iff .
\end{definition}
\begin{definition}[Unlabeled simple graph]
  An unlabeled graph is said to be simple iff:
  \begin{enumerate}
  \item It exists at most one edge between any pair of nodes,
  \item The graph does not contain self loops
    ()
  \end{enumerate}
\end{definition}
\begin{definition}[Labeled simple graph]\label{def:labeledsimplegraph}
  Let  be a finite alphabet of node and edge labels. A
  labeled simple graph  is a tuple  where 
  \begin{itemize}
  \item the couple  defines an unlabeled simple graph,
  \item  is a node labeling function,
  \item  is an edge labeling function.
  \end{itemize}
  The unlabeled graph associated to a given labeled graph
   is defined by the couple .
\end{definition}


In the following we will only consider simple graphs that we will
simply denote by unlabeled (resp. labeled) graphs.  The term graph
will denote indifferently a labeled or an unlabeled graph.
\begin{definition}[Undirected graph]\label{def:undirectedgraph}
~
\begin{itemize}
  \item A simple graph  is said to be undirected iff\\
  .
  \item A labeled simple graph  is said to be undirected iff\\
  .
\end{itemize}
\end{definition}
\begin{definition}[Bipartite graph]\label{def:bipartitegraphs}
  A graph , labeled or not, is said to be bipartite
  iff \\ .
\end{definition}
\begin{definition}[Subgraph]\label{def:subgraph}
~
  \begin{itemize}
  \item An unlabeled graph  is said to be an unlabeled
    subgraph of  if  and
    . The unlabeled subgraph  is 
    called an unlabeled proper subgraph if  or .
  \item If  and 
    are both labeled graphs then  is a (proper) subgraph of 
    if
     is an unlabeled (proper) subgraph of 
    and if the
    following additional constraint is fulfilled:  and , where  denotes the restriction of function  to a
    particular domain.
  \item A structural subgraph of a labeled graph  is an unlabeled
    subgraph of the unlabeled graph associated to .
  \end{itemize}
\end{definition}











\subsection{Edit operations, paths, and distance}\label{sec:mainDefEditDist}
\begin{definition}[Elementary edit operations]\label{def:elEditOp}  
  An elementary edit operation is one of the following operation
  applied on a graph:
  \begin{enumerate}
  \item[] Node/Edge removal. Such removals are defined as the
    removal of the considered element from sets  or .
  \item[] Node/Edge insertion. On labeled graphs, a vertex/edge
    insertion also associates a label to the inserted element.
  \item[] Node/Edge substitution if the graph is a labeled one.
    Such an operation modifies the label of a node or an edge and
    thus transforms the node or edge labeling functions.
  \end{enumerate}
\end{definition}
\begin{definition}[Cost of an elementary edit operation]\label{def:elEdOpCost}
  Each elementary operation  is associated to a cost encoded by a
  specific function for each type of operation:
  \begin{itemize}
  \item Node () and edge removal ()
  \item Node ( and edge () insertion,
  \item Node () and edge () substitution on labeled graphs.
  \end{itemize}
  By extension, we will consider that functions  and
   (resp.  and ) apply on the set of nodes
  (resp. the set of edges) of a graph. Hence, the cost 
  denotes the cost of removing node .

  We assume that a substitution transforming one label into the same
  label has zero cost: 
  
  where  denotes the substitution of label  into
   on some edge or node.
\end{definition}
\begin{definition}[Edit path]\label{def:editPath}
  An edit path of a graph  is a sequence of elementary
    operations applied on , where node removal and edge
    insertion have to satisfy the following constraints:
    \begin{enumerate}
    \item\label{item:vertexremov} A node removal implies a first removal of all its incident
      edges,
    \item\label{item:edgeinsertion} An edge insertion can be applied
      only between two existing or already inserted nodes. 
    \item\label{item:simplegraph} Edge insertions should not create
      more than one edge between two vertices nor create self-loops.
    \end{enumerate}
  An edit path that transforms a graph  into a graph  is an edit
    path of  whose last graph is . If  and  are unlabeled we assume that no node nor edge
  substitutions are performed.
\end{definition}
\begin{definition}[Cost of an edit path]
  The cost of an edit path , denoted  is the sum of
  the costs of its elementary edit operations.
\end{definition}
\begin{definition}[Edit distance]
  The edit distance from a graph  to a graph  is defined as
  the minimal cost of all edit paths from  to .
  
  where  is the set of all edit paths
  transforming  into . An edit path from  to  with a
  minimal cost is called an optimal path.
\end{definition}
\begin{proposition}
  \label{prop:resultEditPath}
  Given any graph , and any edit path  of , the
  transformation of  by  is still a simple graph. 
\end{proposition}
\begin{proof}
  Let  and  denote the
  initial graph and its transformation by . Since the insertion
  of nodes and edges induces the definition of their labels,
  function  (resp. ) defines a valid labeling function
  over  (resp. ). 

  Let us consider . Vertices  and  should either be
  present in  or have been inserted before the insertion of edge
   (Definition~\ref{def:editPath},
  condition~\ref{item:edgeinsertion}). Moreover, none of these nodes
  can be removed after the last insertion of edge  since such a
  removal would imply the removal of 
  (Definition~\ref{def:editPath}, condition~\ref{item:vertexremov}).
  Both  and  thus belong to .
Hence  . Moreover, according to
  Definition~\ref{def:editPath}, condition~\ref{item:simplegraph} the
  edge  can not be inserted if it already exists in  and
  .  It follows that  is a labeled
  simple graph according to Definition~\ref{def:labeledsimplegraph}.
\end{proof}
\begin{definition}[Independent edit path]
  \label{def:mineditpath}
  An independent edit path between two labeled graphs  and  is
  an edit path such that:
  \begin{enumerate}
  \item\label{item:subremov} No node nor edge is both substituted and
    removed,
  \item\label{item:subins} No node nor edge is simultaneously substituted and inserted,
  \item\label{item:insremov} Any inserted element is never removed,
  \item\label{intem:subsonce} Any node or edge is substituted at most
    once,
  \end{enumerate}    
\end{definition}
Note that an independent edit path is not minimal in the number of
operations. Indeed, Definition~\ref{def:mineditpath} still allows to
replace one substitution by one removal followed by one insertion
(but such an operation can be performed only once for each node or
edge thanks to condition~\ref{item:insremov}).  We however forbid useless
operations such as the substitution of one node followed by its
removal (condition~\ref{item:subremov}) or the insertion of a node
with a wrong label followed by its substitution
(condition~\ref{item:subins}).

In the following we will only
consider independent edit paths that we simply call edit paths.
\begin{proposition}
  The elementary operations of an independent edit path between two graphs
   and  may be ordered into a sequence of removals,
  followed by a sequence of substitutions and terminated by a
  sequence of insertions.
\end{proposition}
\begin{proof}
  Let  and  denote the sub-sequences of
  Removals, Substitutions and Insertions of an edit path ,
  respectively. Since no removal may be performed on a substituted
  element (condition~\ref{item:subremov} of
  Definition~\ref{def:mineditpath}) and no removal may be performed on
  an inserted element (condition~\ref{item:insremov}), removals only
  apply on elements which are neither substituted nor inserted. Such
  removals operations may thus be grouped at the beginning of the edit
  path. Now, since an element cannot be substituted after its
  insertion, substitutions can only apply on the remaining elements after
  the removal step and can be grouped after the removal operations.  The
  remaining operations only contain insertions.

  Let us consider the sequence of elementary operations 
  the order within sequences ,  and  being deduced from the
  one of . Such a sequence may be defined since operations in 
  apply on elements not in  and  while operations in  do
  not apply on the same elements than operations in . Such sets
  are independent, leading to the definition of independent edit
  paths. However, we still have to show that the sequence 
  defines a valid edit path.
  \begin{enumerate}
  \item Since  contains all the removal operations contained in
    , if  satisfies condition~\ref{item:vertexremov} of
    Definition~\ref{def:editPath}, so does the sequence .
  \item Let us suppose that an edge insertion is valid in sequence
     while it violates Definition~\ref{def:editPath},
    condition~\ref{item:edgeinsertion} in sequence . Let us
    denote by  such an edge. Edge  violates
    condition~\ref{item:edgeinsertion} in sequence  only if
    either the removal of  or  belongs to .  In such a case
    the insertion of  in  should be made before the
    removal of  or . But such a removal would imply the
    removal of all the incident edges of  or 
    (Definition~\ref{def:editPath},
    condition~\ref{item:vertexremov}) including the newly inserted
    edge .  Such an operation would violate the independence
    of  (Definition~\ref{def:mineditpath},
    condition~\ref{item:insremov}).
  \end{enumerate}
  The sequence  is thus a valid edit path which transforms
  a graph  into  if  do so.
  Furthermore, it is readily seen that all the conditions of Definition~\ref{def:mineditpath}
  are satisfied by the sequence  as soon as they are satisfied by .
  The sequence  is thus an independent edit path.
\end{proof}
\begin{proposition}\label{prop:editPathSubGraphs}
  Let  be an edit path between two graphs  and . Let us
  further denote by ,  and  the sequence of node and edge
  Removals, Substitutions and Insertions performed by , the order
  in each sequence being deduced from the one of . Then:
  \begin{itemize}
  \item the graph  obtained from  by applying
    removal operations  is a subgraph of ,
  \item the graph  obtained from  by applying
    the sequence of operations  is a subgraph of ,
  \item Both  and  correspond to a same common
    structural subgraph of  and .
  \end{itemize}
\end{proposition}
\noindent{\bf Proof:}
  \begin{enumerate}
  \item Since the sequence  is an edit path,  is a
    graph by Proposition~\ref{prop:resultEditPath}. Moreover, since
     is only composed of removal operations, we trivially have
     and . The fact that
     is induced
    by the fact that  is a graph. Moreover, if  is a
    labeled graph, since removal operations do not modify labels,
    labels on  are only the restriction of the ones on
     to  and .
  \item The graph  is deduced from  by the edit path
    , it is thus a graph. Moreover,  is deduced from
     by the sequence of insertions . We thus trivially
    have:  and .  Moreover, since insertion operations
    do not modify the label of existing elements, the restriction of
    the label functions of  to  and 
    corresponds to the label functions of .
  \item Sub graph  is deduced from  by the
    sequence of substitution operations . Since substitution
    operations only modify label functions, the structure of both
    graphs is the same and there exists a structural isomorphism
    between both graphs.~~{}
  \end{enumerate}
One should note that it may exist several structural isomorphisms
between  and . The set of substitutions 
fixes one of them, say  such that the image of any element of
 by  have the same label than the one defined by the
substitution. More precisely, let us suppose that we enlarge the set
of substitution  by  cost substitutions so that all the
nodes and edges of 
are substituted. In this case, we have:

where  and  correspond to the labels defined by the
substitutions of  and  and  and  define
respectively the node and edge labeling functions of .
\begin{corollary}\label{coro:costEditPath}
  Using the same notations than in
  Proposition~\ref{prop:editPathSubGraphs}, the cost  of an
  edit path is defined by:
  \renewcommand{\arraystretch}{2}
   
\end{corollary}
\noindent{\bf Proof:}
  The edit path  and its rewriting in  have the same set of
  operations and thus a same cost.
  \begin{description}
  \item[From  to :] Operations in  remove
    nodes in  and edges in .
  \item[From  to :] Substitutions of  apply
    between the two graphs  and . Let us
    consider the set of substitutions  which corresponds to the
    completion of  by  cost substitutions so that all nodes
    and edges of  are substituted. Both  and  have
    a same cost. The cost of  is defined as the sum of costs of
    the substituted nodes and edges of .
  \item[From  to :] Operations in  insert
    nodes of  and edges of  in
    order to obtain  from .~~{}
  \end{description}
\begin{remarque}\label{rem:doubleCounting}
  Using the same notations than
  Proposition~\ref{prop:editPathSubGraphs} if both  and  are
  undirected we have:
  
  with
   
\end{remarque}


  Indeed, if both graphs  and  are undirected both 
  and  belong to  while encoding a single edge .  The
  removal or the substitution of the edge  is thus counted twice in
  . In the same way  and  represent the
  same edge  of  which is thus inserted twice in
  . The factor  of  removes
   this double couting of edge operations.
\begin{corollary}\label{coro:costEditPathConst}
  If all costs do not depend on the node/edge involved in the operations, \textit{i.e.} edit cost functions , , , , , and  are constant, the cost of an edit path  is equal to:
  \renewcommand{\arraystretch}{1.5}
  
  where  (resp. ) denotes the number of
  nodes (resp. edges) substituted with a non-zero cost.
  
  Moreover, minimizing the cost of such an edit path is
  equivalent to maximizing the following formula:
  
\end{corollary}
\begin{proof}
  We deduce immediately from Corollary~\ref{coro:costEditPath} the
  following formula:
  \renewcommand{\arraystretch}{1.5}
  
  We obtain by grouping constant terms:
  \renewcommand{\arraystretch}{1.5}
  
  Since there is a structural isomorphism between  and
  , we have  and
  . So:
  
  The first part of the above equation being constant, the
  minimization of  is equivalent to the maximization of
  the last part of the equation.
\end{proof}
\begin{definition}[Restricted edit path]
  A restricted edit path is an independent edit path in which any node or
  any edge cannot be removed and then inserted.
\end{definition}
The term restricted should be understood as minimal number
of operations. The cost of a restricted edit path may not be minimal
(among all edit paths) if the cost of a removal operation followed by
an insertion is lower than the cost of the associated
substitution. However, such a drawback may be easily solved by setting
a new substitution cost equal to the minimum between the old substitution
cost and the sum of the costs of a removal and an insertion. In this
case all non-zero cost substitutions, for nodes and edges, may be equivalently
replaced by a removal followed by an insertion.
\begin{proposition}\label{prop:edit_path_mapping}
  If  and  are simple graphs, there is a one-to-one mapping
  between the set of restricted edit paths between  and  and
  the set of injective functions from a subset of  to . We
  denote by , the special function from the empty set onto
  the empty set.
\end{proposition}
\begin{proof}
  Let  denote an edit path. If no node substitution occurs, all
  node of  are first removed and then all nodes of 
  are inserted. We associate this edit path to . 

  If substitutions occur. We associate to  the function previously
  mentioned which maps each substituted node of  onto the
  corresponding node of . 

  Let  denotes this mapping. Let us consider an injective
  function  from a set  onto .  We
  associate to  the sets of node
  \begin{itemize}
  \item removals: ,
  \item insertions:  , and
  \item substitutions: .   
  \end{itemize}
  Moreover, since  and  are simple graphs it exists at most
  one edge between any pair of nodes. We thus define the following
  set of edge operations:
  \begin{itemize}
  \item removals:  such that  or  does not belong to  or  do not belongs to ,
  \item insertions  such that  or  does not belong to  or  do not belongs to ,

  \item substitutions  and .
  \end{itemize}
  Let us denote respectively by  and  the set of
  removals/substitutions/insertions defined both on node and
  edges. The sequence  defines a valid restricted edit path whose image
  by  is by construction equal to . The function  is
  thus surjective.

  Let us consider two different edit paths  and . Then:
  \begin{itemize}
  \item If . If node removals are not equal,  and
     are not defined on the same set and are consequently
    not equal. Otherwize, let us suppose that an edge  is
    removed in  and not in . Let us further suppose that
     and . Since
     is not removed in  we have
    . The
    edge  should thus be first removed and then inserted in 
    which contradicts the definition of a restricted edit
    path. Therefore, one of the two following conditions should holds:
    \begin{enumerate}
    \item      or ,
    \item the set of edge removal operations is the same in  and .
    \end{enumerate}
    If the first condition holds . If the second
    condition holds, since , the set of node removals is
    different in  and . In this case we get also
    .
  \item If . Node substitutions are different if they are
    either defined on different sets or do not correspond to the same
    node mapping. In both cases, . If node
    substitutions are identical but if  and  differ by the
    edge substitutions, an edge of  is substituted by  and  to an edge of
     with two different labels. Since an edge may
    be substituted at most once in an independent edit path, either
     or  is not a valid edit path between  and .
    Hence  and  differ only if their node substitutions differ, 
    in which case we have .
  \item If  (with  and ). If a node 
    is inserted by  but not by , this means that  is
    substituted by . Hence we contradict . In the same way,
    if an edge  is inserted by  and not by ,
     is substituted by  but not by . We again contradict
    .   
  \end{itemize}
  Note that the last item of the previous decomposition shows (as a
  side demonstration) that given the initial and final graphs, a
  restricted edit path is fully defined by its set of removals and
  substitutions.  Moreover, if the set of removals or substitutions of
  two restricted edit paths differ, then the associated mapping is
  different. The function  is thus injective and hence
  bijective.
\end{proof}
\begin{proposition}\label{prop:defSRIsets}
  Let  be a restricted edit path not associated with 
  (hence with some substitutions). Let us denote by
   the injective function
  associated to  and let us denote  by
  . We further introduce the following two sets:
  
  \begin{itemize}
  \item The set of substituted/removed/inserted nodes by  are
    respectively equal to: ,  and
    .
  \item The set of edges substituted/removed/inserted by  are
    respectively equal to:
    \begin{itemize}
    \item Substituted: \\
      with 
    \item Removed:       
    \item Inserted: 
    \end{itemize}
  \end{itemize}
\end{proposition}
\noindent{\textbf{Proof:}~}
  Node substitions/removal/insertions are direct consequences of the
  proof of Proposition~\ref{prop:edit_path_mapping} and the bijective
  mapping between injective functions from a subset
   onto  and restricted edit paths. 
  \begin{itemize}
  \item If
    
    and . Since  cannot be
    removed and then inserted it must be substituted. Conversely, if
     is substituted,  an  must be substituted
    (). Moreover, the edge
     should exists in . Hence
     and . 

    If , it exits
     such that  and . Hence
    . Since 
    and . Hence
    .
    Conversely, if
    , it exits
     such that  and 
    (). Since  we have
    . Finally,  implies
    that  hence  and
    .
  \item Any non substited edge of  must be removed. Hence, the
    set of removed edges is equal to . The remaining
    equation, is deduced by a negation of the conditions defining
    .
  \item In the same way, any edge of  which is not produced by a
    substitution of an edge of  must be inserted. Hence, the set
    of inserted edges is equal to . The negation of the
    condition defining  provides the remaining equation.~~
  \end{itemize}
\subsection{Linear and quadratic assignment problems}\label{sec:mainDefAssignment}
Within this report, an assignment corresponds to a bijective mapping  between two finite sets
 and  having the same size .
When these sets are reduced to the same set of integers, \textit{i.e.} , the bijection  reduces to the permutation . Any permutation  can be represented by a  permutation matrix  with 

More generally, recall that a permutation matrix is defined as follows.
\begin{definition}[Permutation Matrix]\label{def:permutationmatrix}
A  matrix  is a permutation matrix iff it satisfies the following contraints:

\end{definition}
These constraints ensure  to be binary and doubly stochastic (sum of rows and sum of columns equal to ).

The selection of a relevant assignment, among all possible ones from  to , depends on the problem. Nevertheless, each assignment is commonly penalized by a cost, and a relevant assignment becomes one having a minimal or a maximal cost. In this report, we consider minimal costs only. The cost of an assignment is usually defined as a sum of elementary costs. An elementary cost may penalize the assignment of an element of  to an element of , or the simultaneaous assignment of two elements  and  of  to two elements  and  of , respectively.

\begin{definition}[Linear Sum Assignment Problem (LSAP)]\label{def:lsap}
Let  be a matrix such that  corresponds to the cost of assigning the element  to the element . The Linear Sum Assignment Problem (LSAP) consists in finding
an optimal permutation 

where  is the set of all permutations of . Equivalently, the LSAP consists in finding an optimal  permutation matrix

\end{definition}


Let  be the vectorization of the cost matrix , obtained by concatenating its rows. Similarly, let  be the vectorization of . Then, the LSAP consists in finding an optimal vector

where the linear system  is the matrix version of the constraints defined by Eq.~\ref{eq-cts-mtx-perm}. The matrix  represents the node-edge incidence matrix of the complete bipartite graph  with node sets  and :

The system , together with the binary constraint on , selects exactly one edge of  for each element of . In other terms, these constraints represent a subgraph of , with node sets  and , such that each node has degree one. Indeed, the LSAP is a weighted bipartite graph matching problem.

More details on the LSAP can be found in \cite{sier01,bur09}. In particular, Eq.~\ref{eq:lsap_formulation} is a binary linear programming problem, efficiently solvable in polynomial time complexity, for instance with the Hungarian  algorithm~\cite{Kuhn1955,Munkres1957,law76} combined with pre-processing steps~\cite{Jonker1987}. In our experiments, we have used the  (time complexity) version of the Hungarian algorithm proposed in \cite{law76,bur09}.

Problems that incorporate pairwise constraints, \textit{i.e.} simultaneously assigning two elements of  to two elements of , can be cast as a quadratic assignment problem \cite{koop57,Lowler1963,law76,Loiola2007,bur09}. This is the case for the graph matching problem, and for the GED as demonstrated in Section~\ref{sec:qap_formulation}. In this paper, we consider the general expression of quadratic assignment problems \cite{Lowler1963}.
\begin{definition}[Quadratic Assignment Problem (QAP)]\label{def:qsap}
Let  be a matrix such that  corresponds to the cost of simultaneously assigning the elements  and  of  to the elements  and  of , respectively. The quadratic assignment problem (QAP) consists in finding an optimal permutation

Equivalently, the QAP consists in finding an optimal  permutation matrix

\end{definition}
Note that  is assigned to , and  is assigned to , simultaneously iff .

The QAP can be rewritten as a quadratic program:

where  is the vectorization of , and the right-hand side is the matrix version of the constraints defined by Eq.~\ref{eq-cts-mtx-perm}.

The quadratic term is able to incorporate a linear one. Indeed, any simultenous assignment of the same element  to the same element  is penalized by the cost , \textit{i.e.} a diagonal element of . Since , we have . Then the total contribution of diagonal elements to the quadratic cost is given by

where  denotes the diagonal vector. So, if , the quadratic functional incorporates a linear term which decribes the linear sum assignment between the elements of  and . When these constraints are also part of the underlying problem, it is sometimes more convenient to rewritte the QAP as

where , and  is the cost of assigning each element of  to each element of .

As the GED, the QAP is in general NP-hard, and exact algorithms can only be used with sets of small cardinality \cite{bur09}. Indeed, the cost functional is generally not convex, and methods based on relaxation and linearization are usually considered to find an approximate solution. See Section~\ref{sec:qap_solvers} for more details.
