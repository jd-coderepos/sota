This section introduces some basics about graphs, graph edit distance
and edit paths. We futher introduce different familly of edit paths
and show that one of this familly is in direct correspondence with a
familly of mapping functions between the set of nodes of two graphs.
\subsection{Graph Basics}\label{sec:mainDef}
\begin{definition}[Unlabeled graph]
  An unlabeled  graph $G$ is defined by the couple
  $G=(V,E)$ where $V$ is the set of nodes and $E\subseteq
  V\times V$ is the set of edges. Each edge is an orded couple
  of nodes $(i,j)$ with $i,j \in V$. The direction of an edge is
  implicitly given by the order of its nodes, 
  i.e. the direction of $(i,j)$ is from $i$ to $j$.
\end{definition}
\begin{definition}[Nodes Adjacency]
Given a graph $G=(V,E)$ and two nodes $i,j \in V$. $i$ and $j$ are
  said to be adjacent iff $\exists (i,j) \in E$.
\end{definition}
\begin{definition}[Unlabeled simple graph]
  An unlabeled graph is said to be simple iff:
  \begin{enumerate}
  \item It exists at most one edge between any pair of nodes,
  \item The graph does not contain self loops
    ($(i,i)\not\in E, \forall i \in V$)
  \end{enumerate}
\end{definition}
\begin{definition}[Labeled simple graph]\label{def:labeledsimplegraph}
  Let $\mathcal{L}$ be a finite alphabet of node and edge labels. A
  labeled simple graph  is a tuple $G=(V,E,\mu,\nu)$ where 
  \begin{itemize}
  \item the couple $(V,E)$ defines an unlabeled simple graph,
  \item $\mu : V \rightarrow \mathcal{L}$ is a node labeling function,
  \item $\nu : E\rightarrow \mathcal{L}$ is an edge labeling function.
  \end{itemize}
  The unlabeled graph associated to a given labeled graph
  $G=(V,E,\mu,\nu)$ is defined by the couple $(V,E)$.
\end{definition}


In the following we will only consider simple graphs that we will
simply denote by unlabeled (resp. labeled) graphs.  The term graph
will denote indifferently a labeled or an unlabeled graph.
\begin{definition}[Undirected graph]\label{def:undirectedgraph}
~
\begin{itemize}
  \item A simple graph $G=(V,E)$ is said to be undirected iff\\
  $\forall (i,j) \in E ~ \exists (j,i) \in E$.
  \item A labeled simple graph $G=(V,E,\mu,\nu)$ is said to be undirected iff\\
  $\forall (i,j) \in E ~ \exists (j,i) \in E \land \nu(i,j) = \nu(j,i)$.
\end{itemize}
\end{definition}
\begin{definition}[Bipartite graph]\label{def:bipartitegraphs}
  A graph $G$, labeled or not, is said to be bipartite
  iff \\ $\exists V_1, V_2 \subseteq V : \forall (i,j) \in E, (i \in V_1 \land j \in V_2) 
  \lor (i \in V_2 \land j \in V_1)$.
\end{definition}
\begin{definition}[Subgraph]\label{def:subgraph}
~
  \begin{itemize}
  \item An unlabeled graph $G_1=(V_1,E_1)$ is said to be an unlabeled
    subgraph of $G_2=(V_2,E_2)$ if $V_1\subseteq V_2$ and
    $E_1\subseteq E_2\cap (V_1\times V_1)$. The unlabeled subgraph $G_1$ is 
    called an unlabeled proper subgraph if $V_1\neq V_2$ or $E_1\neq
    E_2$.
  \item If $G_1=(V_1,E_1,\mu_1,\nu_1)$ and $G_2=(V_2,E_2,\mu_2,\nu_2)$
    are both labeled graphs then $G_1$ is a (proper) subgraph of $G_2$
    if
    $(V_1,E_1)$ is an unlabeled (proper) subgraph of $(V_2,E_2)$
    and if the
    following additional constraint is fulfilled: ${\mu_2}_{|V_1}=\mu_1$ and ${\nu_2}_{|E_1}=\nu_1$, where $f_{|}$ denotes the restriction of function $f$ to a
    particular domain.
  \item A structural subgraph of a labeled graph $G$ is an unlabeled
    subgraph of the unlabeled graph associated to $G$.
  \end{itemize}
\end{definition}











\subsection{Edit operations, paths, and distance}\label{sec:mainDefEditDist}
\begin{definition}[Elementary edit operations]\label{def:elEditOp}  
  An elementary edit operation is one of the following operation
  applied on a graph:
  \begin{enumerate}
  \item[$\bullet$] Node/Edge removal. Such removals are defined as the
    removal of the considered element from sets $V$ or $E$.
  \item[$\bullet$] Node/Edge insertion. On labeled graphs, a vertex/edge
    insertion also associates a label to the inserted element.
  \item[$\bullet$] Node/Edge substitution if the graph is a labeled one.
    Such an operation modifies the label of a node or an edge and
    thus transforms the node or edge labeling functions.
  \end{enumerate}
\end{definition}
\begin{definition}[Cost of an elementary edit operation]\label{def:elEdOpCost}
  Each elementary operation $x$ is associated to a cost encoded by a
  specific function for each type of operation:
  \begin{itemize}
  \item Node ($c_{vd}(x)$) and edge removal ($c_{ed}(x)$)
  \item Node ($c_{vi}(x))$ and edge ($c_{ei}(x)$) insertion,
  \item Node ($c_{vs}(x)$) and edge ($c_{es}(x)$) substitution on labeled graphs.
  \end{itemize}
  By extension, we will consider that functions $c_{vd}$ and
  $c_{vi}$ (resp. $c_{ed}$ and $c_{ei}$) apply on the set of nodes
  (resp. the set of edges) of a graph. Hence, the cost $c_{vd}(v)$
  denotes the cost of removing node $v$.

  We assume that a substitution transforming one label into the same
  label has zero cost: 
  \[
  \forall l\in \mathcal{L}, \;c_{vs}(l\rightarrow l)=c_{es}(l\rightarrow l)=0
  \]
  where $l\rightarrow l'$ denotes the substitution of label $l$ into
  $l'$ on some edge or node.
\end{definition}
\begin{definition}[Edit path]\label{def:editPath}
  An edit path of a graph $G$ is a sequence of elementary
    operations applied on $G$, where node removal and edge
    insertion have to satisfy the following constraints:
    \begin{enumerate}
    \item\label{item:vertexremov} A node removal implies a first removal of all its incident
      edges,
    \item\label{item:edgeinsertion} An edge insertion can be applied
      only between two existing or already inserted nodes. 
    \item\label{item:simplegraph} Edge insertions should not create
      more than one edge between two vertices nor create self-loops.
    \end{enumerate}
  An edit path that transforms a graph $G_1$ into a graph $G_2$ is an edit
    path of $G_1$ whose last graph is $G_2$. If $G_1$ and $G_2$ are unlabeled we assume that no node nor edge
  substitutions are performed.
\end{definition}
\begin{definition}[Cost of an edit path]
  The cost of an edit path $P$, denoted $\gamma(P)$ is the sum of
  the costs of its elementary edit operations.
\end{definition}
\begin{definition}[Edit distance]
  The edit distance from a graph $G_1$ to a graph $G_2$ is defined as
  the minimal cost of all edit paths from $G_1$ to $G_2$.
  \[
  d(G_1,G_2)=\min_{P\in \mathcal{P}(G_1,G_2)} \gamma(P)
  \]
  where $\mathcal{P}(G_1,G_2)$ is the set of all edit paths
  transforming $G_1$ into $G_2$. An edit path from $G_1$ to $G_2$ with a
  minimal cost is called an optimal path.
\end{definition}
\begin{proposition}
  \label{prop:resultEditPath}
  Given any graph $G$, and any edit path $P$ of $G$, the
  transformation of $G$ by $P$ is still a simple graph. 
\end{proposition}
\begin{proof}
  Let $G=(V,E,\mu,\nu)$ and $G'=(V',E',\mu',\nu')$ denote the
  initial graph and its transformation by $P$. Since the insertion
  of nodes and edges induces the definition of their labels,
  function $\mu'$ (resp. $\nu'$) defines a valid labeling function
  over $V'$ (resp. $E'$). 

  Let us consider $(u,v)\in E'$. Vertices $u$ and $v$ should either be
  present in $V$ or have been inserted before the insertion of edge
  $(u,v)$ (Definition~\ref{def:editPath},
  condition~\ref{item:edgeinsertion}). Moreover, none of these nodes
  can be removed after the last insertion of edge $(u,v)$ since such a
  removal would imply the removal of $(u,v)$
  (Definition~\ref{def:editPath}, condition~\ref{item:vertexremov}).
  Both $u$ and $v$ thus belong to $V'$.
Hence  $E'\subseteq V'\times V'$. Moreover, according to
  Definition~\ref{def:editPath}, condition~\ref{item:simplegraph} the
  edge $(u,v)$ can not be inserted if it already exists in $G$ and
  $u\neq v$.  It follows that $G'=(V',E',\mu',\nu')$ is a labeled
  simple graph according to Definition~\ref{def:labeledsimplegraph}.
\end{proof}
\begin{definition}[Independent edit path]
  \label{def:mineditpath}
  An independent edit path between two labeled graphs $G_1$ and $G_2$ is
  an edit path such that:
  \begin{enumerate}
  \item\label{item:subremov} No node nor edge is both substituted and
    removed,
  \item\label{item:subins} No node nor edge is simultaneously substituted and inserted,
  \item\label{item:insremov} Any inserted element is never removed,
  \item\label{intem:subsonce} Any node or edge is substituted at most
    once,
  \end{enumerate}    
\end{definition}
Note that an independent edit path is not minimal in the number of
operations. Indeed, Definition~\ref{def:mineditpath} still allows to
replace one substitution by one removal followed by one insertion
(but such an operation can be performed only once for each node or
edge thanks to condition~\ref{item:insremov}).  We however forbid useless
operations such as the substitution of one node followed by its
removal (condition~\ref{item:subremov}) or the insertion of a node
with a wrong label followed by its substitution
(condition~\ref{item:subins}).

In the following we will only
consider independent edit paths that we simply call edit paths.
\begin{proposition}
  The elementary operations of an independent edit path between two graphs
  $G_1$ and $G_2$ may be ordered into a sequence of removals,
  followed by a sequence of substitutions and terminated by a
  sequence of insertions.
\end{proposition}
\begin{proof}
  Let $R, S$ and $I$ denote the sub-sequences of
  Removals, Substitutions and Insertions of an edit path $P$,
  respectively. Since no removal may be performed on a substituted
  element (condition~\ref{item:subremov} of
  Definition~\ref{def:mineditpath}) and no removal may be performed on
  an inserted element (condition~\ref{item:insremov}), removals only
  apply on elements which are neither substituted nor inserted. Such
  removals operations may thus be grouped at the beginning of the edit
  path. Now, since an element cannot be substituted after its
  insertion, substitutions can only apply on the remaining elements after
  the removal step and can be grouped after the removal operations.  The
  remaining operations only contain insertions.

  Let us consider the sequence of elementary operations $(R,S,I)$
  the order within sequences $R$, $S$ and $I$ being deduced from the
  one of $P$. Such a sequence may be defined since operations in $R$
  apply on elements not in $S$ and $I$ while operations in $S$ do
  not apply on the same elements than operations in $I$. Such sets
  are independent, leading to the definition of independent edit
  paths. However, we still have to show that the sequence $(R,S,I)$
  defines a valid edit path.
  \begin{enumerate}
  \item Since $R$ contains all the removal operations contained in
    $P$, if $P$ satisfies condition~\ref{item:vertexremov} of
    Definition~\ref{def:editPath}, so does the sequence $R$.
  \item Let us suppose that an edge insertion is valid in sequence
    $P$ while it violates Definition~\ref{def:editPath},
    condition~\ref{item:edgeinsertion} in sequence $(R,S,I)$. Let us
    denote by $(u,v)$ such an edge. Edge $(u,v)$ violates
    condition~\ref{item:edgeinsertion} in sequence $(R,S,I)$ only if
    either the removal of $u$ or $v$ belongs to $R$.  In such a case
    the insertion of $(u,v)$ in $P$ should be made before the
    removal of $u$ or $v$. But such a removal would imply the
    removal of all the incident edges of $u$ or $v$
    (Definition~\ref{def:editPath},
    condition~\ref{item:vertexremov}) including the newly inserted
    edge $(u,v)$.  Such an operation would violate the independence
    of $P$ (Definition~\ref{def:mineditpath},
    condition~\ref{item:insremov}).
  \end{enumerate}
  The sequence $(R,S,I)$ is thus a valid edit path which transforms
  a graph $G_1$ into $G_2$ if $P$ do so.
  Furthermore, it is readily seen that all the conditions of Definition~\ref{def:mineditpath}
  are satisfied by the sequence $(R,S,I)$ as soon as they are satisfied by $P$.
  The sequence $(R,S,I)$ is thus an independent edit path.
\end{proof}
\begin{proposition}\label{prop:editPathSubGraphs}
  Let $P$ be an edit path between two graphs $G_1$ and $G_2$. Let us
  further denote by $R$, $S$ and $I$ the sequence of node and edge
  Removals, Substitutions and Insertions performed by $P$, the order
  in each sequence being deduced from the one of $P$. Then:
  \begin{itemize}
  \item the graph $\hat{G}_1$ obtained from $G_1$ by applying
    removal operations $R$ is a subgraph of $G_1$,
  \item the graph $\hat{G}_2$ obtained from $G_1$ by applying
    the sequence of operations $(R,S)$ is a subgraph of $G_2$,
  \item Both $\hat{G}_1$ and $\hat{G}_2$ correspond to a same common
    structural subgraph of $G_1$ and $G_2$.
  \end{itemize}
\end{proposition}
\noindent{\bf Proof:}
  \begin{enumerate}
  \item Since the sequence $R$ is an edit path, $\hat{G}_1$ is a
    graph by Proposition~\ref{prop:resultEditPath}. Moreover, since
    $R$ is only composed of removal operations, we trivially have
    $\hat{V}_1\subset V_1$ and $\hat{E}_1 \subset E_1$. The fact that
    $\hat{E}_1\subset E_1\cap \hat{V}_1\times \hat{V}_1 $ is induced
    by the fact that $\hat{G}_1$ is a graph. Moreover, if $G_1$ is a
    labeled graph, since removal operations do not modify labels,
    labels on $\hat{G}_1$ are only the restriction of the ones on
    $G_1$ to $\hat{V}_1$ and $\hat{E}_1$.
  \item The graph $\hat{G}_2$ is deduced from $G_1$ by the edit path
    $(R,S)$, it is thus a graph. Moreover, $G_2$ is deduced from
    $\hat{G}_2$ by the sequence of insertions $I$. We thus trivially
    have: $\hat{V}_2\subset V_2$ and $\hat{E}_2 \subset E_2\cap
    \hat{V}_2\times\hat{V}_2 $.  Moreover, since insertion operations
    do not modify the label of existing elements, the restriction of
    the label functions of $G_2$ to $\hat{V}_2$ and $\hat{E}_2$
    corresponds to the label functions of $\hat{G}_2$.
  \item Sub graph $\hat{G}_2$ is deduced from $\hat{G}_1$ by the
    sequence of substitution operations $S$. Since substitution
    operations only modify label functions, the structure of both
    graphs is the same and there exists a structural isomorphism
    between both graphs.~~{$\Box$}
  \end{enumerate}
One should note that it may exist several structural isomorphisms
between $\hat{G}_1$ and $\hat{G}_2$. The set of substitutions $S$
fixes one of them, say $f$ such that the image of any element of
$\hat{G}_1$ by $f$ have the same label than the one defined by the
substitution. More precisely, let us suppose that we enlarge the set
of substitution $S$ by $0$ cost substitutions so that all the
nodes and edges of $\hat{G}_1=(\hat{V}_1,\hat{E}_1,\mu_1,\nu_1)$
are substituted. In this case, we have:
\[
\left\{  \begin{array}{llll}
    \forall v\in \hat{V}_1, &\mu_2(f(v))&=&l_v\\
    \forall e\in \hat{E}_1, &\nu_2(f(e))&=&l_e\\
  \end{array}\right.
\]
where $l_v$ and $l_e$ correspond to the labels defined by the
substitutions of $v$ and $e$ and $\mu_2$ and $\nu_2$ define
respectively the node and edge labeling functions of $G_2$.
\begin{corollary}\label{coro:costEditPath}
  Using the same notations than in
  Proposition~\ref{prop:editPathSubGraphs}, the cost $\gamma(P)$ of an
  edit path is defined by:
  \renewcommand{\arraystretch}{2}
  \begin{equation*}
    \begin{array}{lcl}
      \gamma(P) & = & \displaystyle   \sum_{v\in V_1\setminus\hat{V}_1} c_{vd}(v)+
      \sum_{e\in E_1\setminus\hat{E}_1} c_{ed}(e)+
      \sum_{v\in \hat{V}_1} c_{vs}(v)+
      \sum_{e\in \hat{E}_1} c_{es}(e) \\
      & & \displaystyle + \sum_{v\in V_2\setminus\hat{V}_2} c_{vi}(v) + \sum_{e\in E_2\setminus\hat{E}_2} c_{ei}(e)
    \end{array}
  \end{equation*} 
\end{corollary}
\noindent{\bf Proof:}
  The edit path $P$ and its rewriting in $(R,S,I)$ have the same set of
  operations and thus a same cost.
  \begin{description}
  \item[From $G_1$ to $\hat{G}_1$:] Operations in $R$ remove
    nodes in $V_1\setminus\hat{V}_1$ and edges in $E_1\setminus\hat{E}_1$.
  \item[From $\hat{G}_1$ to $\hat{G}_2$:] Substitutions of $S$ apply
    between the two graphs $\hat{G}_1$ and $\hat{G}_2$. Let us
    consider the set of substitutions $S'$ which corresponds to the
    completion of $S$ by $0$ cost substitutions so that all nodes
    and edges of $\hat{G}_1$ are substituted. Both $S$ and $S'$ have
    a same cost. The cost of $S'$ is defined as the sum of costs of
    the substituted nodes and edges of $\hat{G}_1$.
  \item[From $\hat{G}_2$ to $G_2$:] Operations in $I$ insert
    nodes of $V_2\setminus\hat{V}_2$ and edges of $E_2\setminus\hat{E}_2$ in
    order to obtain $G_2$ from $\hat{G}_2$.~~{$\Box$}
  \end{description}
\begin{remarque}\label{rem:doubleCounting}
  Using the same notations than
  Proposition~\ref{prop:editPathSubGraphs} if both $G_1$ and $G_2$ are
  undirected we have:
  \[
  \gamma(P)=\gamma_v(P)+\gamma_e(P)
  \]
  with
  \begin{equation*}
    \begin{array}{lcl}
      \gamma_v(P) & = & \displaystyle   \sum_{i\in V_1\setminus\hat{V}_1} c_{vd}(i)+
      \sum_{i\in \hat{V}_1} c_{vs}(i)+\sum_{k\in V_2\setminus\hat{V}_2} c_{vi}(k)\\
      \gamma_e(P)&=&\displaystyle\frac{1}{2}\left(\sum_{(i,j)\in \hat{E}_1} c_{es}((i,j)) +
                     \sum_{(i,j)\in E_1\setminus\hat{E}_1} c_{ed}((i,j))+\sum_{(k,l)\in E_2\setminus\hat{E}_2} c_{ei}((k,l))\right)\\
    \end{array}
  \end{equation*} 
\end{remarque}


  Indeed, if both graphs $G_1$ and $G_2$ are undirected both $(i,j)$
  and $(j,i)$ belong to $E_1$ while encoding a single edge $e$.  The
  removal or the substitution of the edge $e$ is thus counted twice in
  $\gamma_e(P)$. In the same way $(k,l)$ and $(l,k)$ represent the
  same edge $e$ of $E_2\setminus\hat{E_2}$ which is thus inserted twice in
  $\gamma_e(P)$. The factor $\frac{1}{2}$ of $\gamma_e(P)$ removes
   this double couting of edge operations.
\begin{corollary}\label{coro:costEditPathConst}
  If all costs do not depend on the node/edge involved in the operations, \textit{i.e.} edit cost functions $c_{vd}$, $c_{ed}$, $c_{vs}$, $c_{es}$, $c_{vi}$, and $c_{ei}$ are constant, the cost of an edit path $P$ is equal to:
  \renewcommand{\arraystretch}{1.5}
  \begin{equation*}
    \begin{array}{lcl}
      \gamma(P) & = & (|V_1|-|\hat{V}_1|)c_{vd}+(|E_1|-|\hat{E}_1|)c_{ed}+V_fc_{vs}+E_fc_{es}\\
      &   & +\,(|V_2|-|\hat{V}_2|)c_{vi}+(|E_2|-|\hat{E}_2|)c_{ei}    
    \end{array}
  \end{equation*}
  where $V_f$ (resp. $E_f$) denotes the number of
  nodes (resp. edges) substituted with a non-zero cost.
  
  Moreover, minimizing the cost of such an edit path is
  equivalent to maximizing the following formula:
  \[
  M(P)\overset{\text{def.}}{=}|\hat{V}_1|(c_{vd}+c_{vi})+
  |\hat{E}_1|(c_{ed}+c_{ei})-V_fc_{vs}-E_fc_{es}
  \]
\end{corollary}
\begin{proof}
  We deduce immediately from Corollary~\ref{coro:costEditPath} the
  following formula:
  \renewcommand{\arraystretch}{1.5}
  \begin{equation*}
    \begin{array}{lcl}
      \gamma(P) & = & (|V_1|-|\hat{V}_1|)c_{vd}+(|E_1|-|\hat{E}_1|)c_{ed}+V_fc_{vs}+E_fc_{es}\\
      &   & +\,(|V_2|-|\hat{V}_2|)c_{vi}+(|E_2|-|\hat{E}_2|)c_{ei}
    \end{array}
  \end{equation*}
  We obtain by grouping constant terms:
  \renewcommand{\arraystretch}{1.5}
  \begin{equation*}
    \begin{array}{lcl}
      \gamma(P) & = & |V_1|c_{vd}+|E_1|c_{ed}+|V_2|c_{vi}+|E_2|c_{ei}\\
      & & - \left[
        |\hat{V}_1|c_{vd}+|\hat{E}_1|c_{ed}+|\hat{V}_2|c_{vi}+|\hat{E}_2|c_{ei}-V_fc_{vs}-E_fc_{es}\right]
    \end{array}
  \end{equation*}
  Since there is a structural isomorphism between $\hat{G}_1$ and
  $\hat{G}_2$, we have $\hat{V}_1=\hat{V}_2$ and
  $\hat{E}_1=\hat{E}_2$. So:
  \begin{equation*}
    \begin{array}{lcl}
      \gamma(P) & = & |V_1|c_{vd}+|E_1|c_{ed}+|V_2|c_{vi}+|E_2|c_{ei} \\
      & & - \left[
        |\hat{V}_1|(c_{vd}+c_{vi})+
        |\hat{E}_1|(c_{ed}+c_{ei})-V_fc_{vs}-E_fc_{es}\right]
    \end{array}
  \end{equation*}
  The first part of the above equation being constant, the
  minimization of $\gamma(P)$ is equivalent to the maximization of
  the last part of the equation.
\end{proof}
\begin{definition}[Restricted edit path]
  A restricted edit path is an independent edit path in which any node or
  any edge cannot be removed and then inserted.
\end{definition}
The term restricted should be understood as minimal number
of operations. The cost of a restricted edit path may not be minimal
(among all edit paths) if the cost of a removal operation followed by
an insertion is lower than the cost of the associated
substitution. However, such a drawback may be easily solved by setting
a new substitution cost equal to the minimum between the old substitution
cost and the sum of the costs of a removal and an insertion. In this
case all non-zero cost substitutions, for nodes and edges, may be equivalently
replaced by a removal followed by an insertion.
\begin{proposition}\label{prop:edit_path_mapping}
  If $G_1$ and $G_2$ are simple graphs, there is a one-to-one mapping
  between the set of restricted edit paths between $G_1$ and $G_2$ and
  the set of injective functions from a subset of $V_1$ to $V_2$. We
  denote by $\varphi_0$, the special function from the empty set onto
  the empty set.
\end{proposition}
\begin{proof}
  Let $P$ denote an edit path. If no node substitution occurs, all
  node of $G_1$ are first removed and then all nodes of $G_2$
  are inserted. We associate this edit path to $\varphi_0$. 

  If substitutions occur. We associate to $P$ the function previously
  mentioned which maps each substituted node of $V_1$ onto the
  corresponding node of $G_2$. 

  Let $\psi$ denotes this mapping. Let us consider an injective
  function $f\neq \varphi_0$ from a set $\hat{V_1}\subset V_1$ onto $V_2$.  We
  associate to $f$ the sets of node
  \begin{itemize}
  \item removals: $c_{vd}(v\rightarrow \epsilon), v\in V_1-\hat{V_1}$,
  \item insertions:  $c_{vi}(\epsilon\rightarrow v), v\in V_2-f(\hat{V_1})$, and
  \item substitutions: $c_{vs}(v\rightarrow f(v)), v\in \hat{V_1}$.   
  \end{itemize}
  Moreover, since $G_1$ and $G_2$ are simple graphs it exists at most
  one edge between any pair of nodes. We thus define the following
  set of edge operations:
  \begin{itemize}
  \item removals: $c_{ed}((i,j)\rightarrow \epsilon)$ such that $i$ or $j$ does not belong to $\hat{V_1}$ or $(f(i),f(j))$ do not belongs to $E_2$,
  \item insertions $c_{ei}(\epsilon \rightarrow (k,l) )$ such that $k$ or $l$ does not belong to $f(\hat{V_1})$ or $(f^{-1}(k),f^{-1}(l))$ do not belongs to $E_1$,

  \item substitutions $c_{es}((i,j)\rightarrow (f(i),f(j))), (i,j)\in (\hat{V_1}\times\hat{V_1})\cap E_1$ and $(f(i),f(j)))\in E_2$.
  \end{itemize}
  Let us denote respectively by $R,S$ and $I$ the set of
  removals/substitutions/insertions defined both on node and
  edges. The sequence $(R,S,I)$ defines a valid restricted edit path whose image
  by $\psi$ is by construction equal to $f$. The function $\psi$ is
  thus surjective.

  Let us consider two different edit paths $P=(R,S,I)$ and $P'=(R',S',I')$. Then:
  \begin{itemize}
  \item If $R\neq R'$. If node removals are not equal, $\psi(P)$ and
    $\psi(P')$ are not defined on the same set and are consequently
    not equal. Otherwize, let us suppose that an edge $(i,j)$ is
    removed in $P$ and not in $P'$. Let us further suppose that
    $\psi(P)(i)=\psi(P')(i)$ and $\psi(P)(j)=\psi(P')(j)$. Since
    $(i,j)$ is not removed in $P'$ we have
    $(\psi(P)(i),\psi(P)(j))=(\psi(P')(i),\psi(P')(j))\in E_2$. The
    edge $(i,j)$ should thus be first removed and then inserted in $P$
    which contradicts the definition of a restricted edit
    path. Therefore, one of the two following conditions should holds:
    \begin{enumerate}
    \item     $\psi(P)(i)\neq\psi(P')(i)$ or $\psi(P)(j)\neq\psi(P')(j)$,
    \item the set of edge removal operations is the same in $P$ and $P'$.
    \end{enumerate}
    If the first condition holds $\psi(P)\,{\neq}\,\psi(P')$. If the second
    condition holds, since $R\,{\neq}\,R'$, the set of node removals is
    different in $P$ and $P'$. In this case we get also
    $\psi(P)\,{\neq}\,\psi(P')$.
  \item If $S\neq S'$. Node substitutions are different if they are
    either defined on different sets or do not correspond to the same
    node mapping. In both cases, $\psi(P)\neq \psi(P')$. If node
    substitutions are identical but if $S$ and $S'$ differ by the
    edge substitutions, an edge of $G_1$ is substituted by $P$ and $P'$ to an edge of
    $G_2$ with two different labels. Since an edge may
    be substituted at most once in an independent edit path, either
    $P$ or $P'$ is not a valid edit path between $G_1$ and $G_2$.
    Hence $S$ and $S'$ differ only if their node substitutions differ, 
    in which case we have $\psi(P)\neq \psi(P')$.
  \item If $I\neq I'$ (with $R=R'$ and $S=S'$). If a node $k\in V_2$
    is inserted by $I$ but not by $I'$, this means that $k$ is
    substituted by $S'$. Hence we contradict $S=S'$. In the same way,
    if an edge $(k,l)\in E_2$ is inserted by $I$ and not by $I'$,
    $(k,l)$ is substituted by $S'$ but not by $S$. We again contradict
    $S=S'$.   
  \end{itemize}
  Note that the last item of the previous decomposition shows (as a
  side demonstration) that given the initial and final graphs, a
  restricted edit path is fully defined by its set of removals and
  substitutions.  Moreover, if the set of removals or substitutions of
  two restricted edit paths differ, then the associated mapping is
  different. The function $\psi$ is thus injective and hence
  bijective.
\end{proof}
\begin{proposition}\label{prop:defSRIsets}
  Let $P$ be a restricted edit path not associated with $\varphi_0$
  (hence with some substitutions). Let us denote by
  $\varphi: \hat{V_1}\rightarrow V_2$ the injective function
  associated to $P$ and let us denote $\varphi(\hat{V_1})$ by
  $\hat{V_2}$. We further introduce the following two sets:
  \[
  \left\{
    \begin{array}{lll}
      R_{12}&=&\{(i,j)\in E_1\cap(\hat{V_1}\times\hat{V_1})~|~(\varphi(i),\varphi(j))\not\in E_2\}\\
      I_{21}&=&\{(k,l)\in E_2\cap(\hat{V_2}\times\hat{V_2})~|~(\varphi^{-1}(k),\varphi^{-1}(k))\not\in E_1\}\\
    \end{array}
  \right.
  \]
  \begin{itemize}
  \item The set of substituted/removed/inserted nodes by $P$ are
    respectively equal to: $\hat{V_1}$, $V_1\setminus\hat{V_1}$ and
    $V_2\setminus\hat{V_2}$.
  \item The set of edges substituted/removed/inserted by $P$ are
    respectively equal to:
    \begin{itemize}
    \item Substituted: $\hat{E_1}=\left(E_1\cap (\hat{V_1}\times\hat{V_1})\right)\setminus R_{12}$\\
      with $\hat{E_2}=\varphi(\hat{E_1})=\left(E_2\cap (\hat{V_2}\times\hat{V_2})\right)\setminus I_{21}$
    \item Removed: $E_1\setminus\hat{E_1}=\left(E_1\cap\left(((V_1\setminus\hat{V_1})\times V_1)\cup ( V_1\times(V_1\setminus\hat{V_1}))\right)\right)\cup R_{12}$      
    \item Inserted: $E_2\setminus\hat{E_2}=\left(E_2\cap\left(((V_2\setminus\hat{V_2})\times V_2)\cup ( V_2\times(V_2\setminus\hat{V_2}))\right)\right)\cup I_{21}$
    \end{itemize}
  \end{itemize}
\end{proposition}
\noindent{\textbf{Proof:}~}
  Node substitions/removal/insertions are direct consequences of the
  proof of Proposition~\ref{prop:edit_path_mapping} and the bijective
  mapping between injective functions from a subset
  $\hat{V_1}\subset V_1$ onto $V_2$ and restricted edit paths. 
  \begin{itemize}
  \item If
    $(i,j)\in \hat{E_1}, \{\varphi(i),\varphi(j)\}\subset \hat{V_2}$
    and $(\varphi(i),\varphi(j))\in E_2$. Since $(i,j)$ cannot be
    removed and then inserted it must be substituted. Conversely, if
    $(i,j)\in E_1$ is substituted, $i$ an $j$ must be substituted
    ($\{i,j\}\subset \hat{V_1}$). Moreover, the edge
    $(\varphi(i),\varphi(j))$ should exists in $E_2$. Hence
    $(i,j)\not\in R_{12}$ and $(i,j)\in \hat{E_1}$. 

    If $(k,l)\in \hat{E_2}=\varphi(\hat{E_1})$, it exits
    $(i,j)\in E_1$ such that $k=\varphi(i)$ and $l=\varphi(j)$. Hence
    $(k,l)\not\in I_{21}$. Since $(i,j)\not\in R_{12}, (k,l)\in E_2$
    and $\{k,l\}\subset \hat{V_2}$. Hence
    $(k,l)\in \varphi(\hat{E_1})=E_2\cap
    (\hat{V_2}\times\hat{V_2})\setminus I_{21}$.
    Conversely, if
    $(k,l)\in E_2\cap (\hat{V_2}\times\hat{V_2})\setminus I_{21}$, it exits
    $(i,j)$ such that $k=\varphi(i)$ and $l=\varphi(j)$
    ($\{k,l\}\in \hat{V_2}$). Since $(k,l)\in E_2\setminus I_{21}$ we have
    $(i,j)\in E_1\setminus R_{12}$. Finally, $\{k,l\}\subset \hat{V_2}$ implies
    that $\{i,j\}\subset \hat{V_1}$ hence $(i,j)\in \hat{E_1}$ and
    $(k,l)\in \hat{E_2}=\varphi(\hat{E_1})$.
  \item Any non substited edge of $G_1$ must be removed. Hence, the
    set of removed edges is equal to $E_1\setminus\hat{E_1}$. The remaining
    equation, is deduced by a negation of the conditions defining
    $\hat{E_1}$.
  \item In the same way, any edge of $G_2$ which is not produced by a
    substitution of an edge of $G_1$ must be inserted. Hence, the set
    of inserted edges is equal to $E_2\setminus\hat{E_2}$. The negation of the
    condition defining $\hat{E_2}$ provides the remaining equation.~~$\Box$
  \end{itemize}
\subsection{Linear and quadratic assignment problems}\label{sec:mainDefAssignment}
Within this report, an assignment corresponds to a bijective mapping $\varphi:\mathcal{X}{\rightarrow}\mathcal{Y}$ between two finite sets
$\mathcal{X}$ and $\mathcal{Y}$ having the same size $|\mathcal{X}| = |\mathcal{Y}| = n$.
When these sets are reduced to the same set of integers, \textit{i.e.} $\mathcal{X}\,{=}\,\mathcal{Y}\,{=}\,\{1,\ldots,n\}$, the bijection $\varphi$ reduces to the permutation $(\varphi(1),\ldots,\varphi(n))$. Any permutation $\varphi$ can be represented by a $n\,{\times}\,n$ permutation matrix $\mathbf{X}\,{=}\,(x_{i,j})_{i,j=1,\ldots,n}$ with 
\begin{equation}
x_{i,j} = 
\begin{cases}
1 & \text{if } j =  \varphi(i) \\
0 & \text{else}
\end{cases}
\end{equation}
More generally, recall that a permutation matrix is defined as follows.
\begin{definition}[Permutation Matrix]\label{def:permutationmatrix}
A $n\,{\times}\,n$ matrix $\mathbf{X}$ is a permutation matrix iff it satisfies the following contraints:
\begin{equation}\label{eq-cts-mtx-perm}
  \left\lbrace\begin{aligned}
  &\forall j\,{=}\,1,\ldots,n,&&\sum_{i=1}^n x_{i,j} = 1 \\
  &\forall i\,{=}\,1,\ldots,n,&&\sum_{j=1}^n x_{i,j} = 1 \\
  &\forall i,j\,{=}\,1,\ldots,n,&&x_{i,j} \in \{0,1\}
  \end{aligned}\right.
  \end{equation}
\end{definition}
These constraints ensure $\mathbf{X}$ to be binary and doubly stochastic (sum of rows and sum of columns equal to $1$).

The selection of a relevant assignment, among all possible ones from $\mathcal{X}$ to $\mathcal{Y}$, depends on the problem. Nevertheless, each assignment is commonly penalized by a cost, and a relevant assignment becomes one having a minimal or a maximal cost. In this report, we consider minimal costs only. The cost of an assignment is usually defined as a sum of elementary costs. An elementary cost may penalize the assignment of an element of $\mathcal{X}$ to an element of $\mathcal{Y}$, or the simultaneaous assignment of two elements $i$ and $j$ of $\mathcal{X}$ to two elements $k$ and $l$ of $\mathcal{Y}$, respectively.

\begin{definition}[Linear Sum Assignment Problem (LSAP)]\label{def:lsap}
Let $\mathbf{C}\,{\in}\,[0,{+\infty})^{n\times n}$ be a matrix such that $c_{i,j}$ corresponds to the cost of assigning the element $i\,{\in}\,\mathcal{X}$ to the element $j\,{\in}\,\mathcal{Y}$. The Linear Sum Assignment Problem (LSAP) consists in finding
an optimal permutation 
\begin{equation}
\hat{\varphi}\in{\argmin}~\left\lbrace\sum_{i=1}^nc_{i,\varphi(i)}~|~\varphi\,{\in}\,\mathcal{S}_n\right\rbrace
\end{equation}
where $\mathcal{S}_n$ is the set of all permutations of $\{1,\ldots,n\}$. Equivalently, the LSAP consists in finding an optimal $n\,{\times}\,n$ permutation matrix
\begin{equation}
\hat{\mathbf{X}}\in\argmin~\left\{\sum_{i=1}^n\sum_{j=1}^n c_{i,j} x_{i,j}~|~\mathbf{X}~\text{satisfies Eq.~\ref{eq-cts-mtx-perm}}\right\}.
\end{equation}
\end{definition}


Let $\mathbf{c}\,{=}\,\text{vec}(\mathbf{C})\,{\in}\,[0,{+\infty})^{n^2}$ be the vectorization of the cost matrix $\mathbf{C}$, obtained by concatenating its rows. Similarly, let $\mathbf{x}\,{=}\,\text{vec}(\mathbf{X})\,{\in}\,\{0,1\}^{n^2}$ be the vectorization of $\mathbf{X}$. Then, the LSAP consists in finding an optimal vector
\begin{equation}\label{eq:lsap_formulation}
  \hat{\mathbf{x}}\in\argmin\left\{\mathbf{c}^T\mathbf{x}~|~\mathbf{L}\mathbf{x}\,{=}\,\mathbf{1}_{2n},~\mathbf{x}\,{\in}\,\{0,1\}^{n^2}\right\},
\end{equation}
where the linear system $\mathbf{L}\mathbf{x}\,{=}\,\mathbf{1}$ is the matrix version of the constraints defined by Eq.~\ref{eq-cts-mtx-perm}. The matrix $\mathbf{L}\,{\in}\,\{0,1\}^{2n\times n^2}$ represents the node-edge incidence matrix of the complete bipartite graph $K_{n,n}$ with node sets $\mathcal{X}$ and $\mathcal{Y}$:
\begin{equation}
	l_{i,(j,k)}=\begin{cases}
	1&\text{if}~(j\,{=}\,i)\vee(k\,{=}\,i)\\
	0&\text{else}
	\end{cases}
\end{equation}
The system $\mathbf{L}\mathbf{x}\,{=}\,\mathbf{1}$, together with the binary constraint on $\mathbf{x}$, selects exactly one edge of $K_{n,n}$ for each element of $\mathcal{X}\,{\cup}\,\mathcal{Y}$. In other terms, these constraints represent a subgraph of $K_{n,n}$, with node sets $\mathcal{X}$ and $\mathcal{Y}$, such that each node has degree one. Indeed, the LSAP is a weighted bipartite graph matching problem.

More details on the LSAP can be found in \cite{sier01,bur09}. In particular, Eq.~\ref{eq:lsap_formulation} is a binary linear programming problem, efficiently solvable in polynomial time complexity, for instance with the Hungarian  algorithm~\cite{Kuhn1955,Munkres1957,law76} combined with pre-processing steps~\cite{Jonker1987}. In our experiments, we have used the $O(n^3)$ (time complexity) version of the Hungarian algorithm proposed in \cite{law76,bur09}.

Problems that incorporate pairwise constraints, \textit{i.e.} simultaneously assigning two elements of $\mathcal{X}$ to two elements of $\mathcal{Y}$, can be cast as a quadratic assignment problem \cite{koop57,Lowler1963,law76,Loiola2007,bur09}. This is the case for the graph matching problem, and for the GED as demonstrated in Section~\ref{sec:qap_formulation}. In this paper, we consider the general expression of quadratic assignment problems \cite{Lowler1963}.
\begin{definition}[Quadratic Assignment Problem (QAP)]\label{def:qsap}
Let $\mathbf{D}\in[0,{+\infty})^{n^2\times n^2}$ be a matrix such that $d_{ik,jl}$ corresponds to the cost of simultaneously assigning the elements $i$ and $j$ of $\mathcal{X}$ to the elements $k$ and $l$ of $\mathcal{Y}$, respectively. The quadratic assignment problem (QAP) consists in finding an optimal permutation
\begin{equation}\label{eq:lowler_form}
\hat{\varphi}\in\argmin\left\{\sum_{i=1}^n\sum_{j=1}^n d_{i\varphi(i),j\varphi(j)}~|~\varphi\in\mathcal{S}_n\right\}.
\end{equation}
Equivalently, the QAP consists in finding an optimal $n\,{\times}\,n$ permutation matrix
\begin{equation*}
\hat{\mathbf{X}}\in\argmin\left\{\sum_{i=1}^n\sum_{k=1}^n\sum_{j=1}^n\sum_{l=1}^n d_{ik,jl}x_{i,k}x_{j,l}~|~\mathbf{X}~\text{satisfies Eq.~\ref{eq-cts-mtx-perm}}\right\}.
\end{equation*}
\end{definition}
Note that $i\,{\in}\,\mathcal{X}$ is assigned to $k\,{\in}\,\mathcal{Y}$, and $j\,{\in}\,\mathcal{X}$ is assigned to $l\,{\in}\,\mathcal{Y}$, simultaneously iff $x_{i,k}\,{=}\,x_{j,l}\,{=}\,1$.

The QAP can be rewritten as a quadratic program:
\begin{equation*}
\argmin\left\{\mathbf{x}^T\mathbf{D}\mathbf{x}~|~\mathbf{L}\mathbf{x}\,{=}\,\mathbf{1}_{2n},~\mathbf{x}\,{\in}\,\{0,1\}^{n^2}\right\}
\end{equation*}
where $\mathbf{x}$ is the vectorization of $\mathbf{X}$, and the right-hand side is the matrix version of the constraints defined by Eq.~\ref{eq-cts-mtx-perm}.

The quadratic term is able to incorporate a linear one. Indeed, any simultenous assignment of the same element $i\,{\in}\,\mathcal{X}$ to the same element $k\,{\in}\,\mathcal{Y}$ is penalized by the cost $d_{ik,ik}$, \textit{i.e.} a diagonal element of $\mathbf{D}$. Since $x_{i,k}\,{\in}\,\{0,1\}$, we have $d_{ik,ik}x_{i,k}x_{i,k}\,{=}\,d_{ik,ik}x_{i,k}$. Then the total contribution of diagonal elements to the quadratic cost is given by
\begin{equation*}
	\sum_{i=1}^n\sum_{k=1}^n d_{ik,ik}x_{i,k}=\text{diag}(\mathbf{D})^T\mathbf{x}
\end{equation*}
where $\text{diag}$ denotes the diagonal vector. So, if $\text{diag}(\mathbf{D})\,{\not=}\,\mathbf{0}$, the quadratic functional incorporates a linear term which decribes the linear sum assignment between the elements of $\mathcal{X}$ and $\mathcal{Y}$. When these constraints are also part of the underlying problem, it is sometimes more convenient to rewritte the QAP as
\begin{equation}\label{eq:matrix_form}
\argmin\left\{\mathbf{x}^T\mathbf{D}\mathbf{x} + \mathbf{c}^T\mathbf{x}~|~\mathbf{L}\mathbf{x}\,{=}\,\mathbf{1}_{2n},~\mathbf{x}\,{\in}\,\{0,1\}^{n^2}\right\}
\end{equation}
where $\text{diag}(\mathbf{D})\,{=}\,\mathbf{0}$, and $\mathbf{c}$ is the cost of assigning each element of $\mathcal{X}$ to each element of $\mathcal{Y}$.

As the GED, the QAP is in general NP-hard, and exact algorithms can only be used with sets of small cardinality \cite{bur09}. Indeed, the cost functional is generally not convex, and methods based on relaxation and linearization are usually considered to find an approximate solution. See Section~\ref{sec:qap_solvers} for more details.
