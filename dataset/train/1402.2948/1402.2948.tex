




\begin{section}{Experimental Results and Conclusions}
	\label{sec:exprescon}

\textbf{Evaluation.}
To evaluate the proposed approach, we present the experimental results obtained
on the dining cryptographers protocol~\cite{Cha88,LQR09} and a variant of the
cake-cutting problem~\cite{ES84}.
The experiments were run on an Intel Core i7-2600 CPU 3.40GHz machine with 8GB
RAM running Linux kernel version 3.8.0-34-generic.
Table~\ref{tab:dining-cryptographers} reports the results obtained when
verifying the dining cryptographers protocol against the specifications
 and , with
, where:




\vspace{-2em}
\begin{table}
\centering
\caption{Verification results for the dining cryptographers protocol.}
\begin{tabular}{|c|c|c|c|c|c|}
\hline
 \textbf{crypts} & \textbf{possible states} & \textbf{reachable states}
& \textbf{reachability} (s) & \textbf{CTLK} (s) & \textbf{SLK} (s) \\ \hline
 &  &  &  &  &  \\ \hline
 &  &  &  &  &  \\ \hline
 &  &  &  &  &  \\ \hline
 &  &  &  &  &  \\ \hline
 &  &  &  &  &  \\ \hline
 &  &  &  &  & 
\\ \hline
 &  &  &  &  & 
\\ \hline
 &  &  &  &  & 
\\ \hline
\end{tabular}
\label{tab:dining-cryptographers}
\end{table}
\vspace{-1em}

 is the usual epistemic specification for the
protocol~\cite{MS04,LQR09} and  is its natural extension where
strategies are quantified.
The results show that the checker can verify reasonably large state spaces.
The performance depends on the number of Boolean variables required to represent
the extended states.
In the case of \SLK specifications, the number of Boolean variables is
proportional to the number of strategies (here equal to the number of agents).
The last two columns of Table~\ref{tab:dining-cryptographers} show that the
tool's performance drops considerably faster when verifying \SLK formulas
compared to \CTLK ones.
This is because \CTLK requires no strategy assignments and extended states
collapse to plain states.
In contrast, the performance for \CTLK is dominated by the computation of the
reachable state space.

We now evaluate \MCMASSLK  with respect to strategy synthesis and specifications
expressing Nash equilibria.
Specifically, we consider a variation of the model for the classic cake-cutting
problem~\cite{ES84} in which a set of \emph{ agents} take turns to slice a
cake of size  and the \emph{environment} responds by trying to ensure the
cake is divided fairly.
We assume that at each even round the agents concurrently choose how to divide
the cake; at each odd round the environment decides how to cut the cake and how
to assign each of the pieces to a subset of the agents.
Therefore, the problem of cutting a cake of size  between  agents is
suitably divided into several simpler problems in which pieces of size 
have to be split between  agents.
The multi-player game terminates once each agent receives a slice.

The model uses as atomic propositions pairs

indicating that agent  gets a piece of cake of size .
The existence of a protocol for the cake-cutting problem is given by the
following \SL specification :

\begin{itemize}
\item
 ensures that the protocol  is fair, \ie, all Nash
equilibria  of the agents guarantee equity of
the splitting;
\item

ensures
that the protocol has a solution, \ie, there is at least one Nash equilibrium;
\item

ensures that if agent  has a strategy  allowing him to get from the
environment a slice of size  once the strategies of the other agents are
fixed, he is already able to obtain a slice of size  by means of his
original strategy  (this can be ensured by taking , , and );
\item
 ensures
that each agent  is able to obtain a piece of size  (
and  are the same as in the item above).
\end{itemize}
We were able to verify the formula  defined above on a system with  agents and a cake of size .
Moreover, we automatically synthesised a strategy  for the environment
(see~\cite{MCMASSLK} for more details).
We were not able to verify larger examples; for example with , there
are  reachable states; the encoding required  Boolean variables (most
of them represent the assignments in the sets of extended states), and the
intermediate BDDs were found to be in the order of  nodes.
This should not be surprising given the theoretical difficulty of the
cake-cutting problem.
Moreover, we are synthesising the entire protocol and not just the agents'
optimal behaviour.



\textbf{Conclusions.}
In this paper we presented \MCMASSLK, a novel symbolic model checker for the
verification of systems against specifications given in \SLK.
A notable feature of the approach is that it allows for the automatic
verification of sophisticated game concepts such as various forms of equilibria,
including Nash equilibria.
Since \MCMASSLK also supports epistemic modalities, this further enables us to
express specifications concerning individual and group knowledge of cooperation
properties; these are commonly employed when reasoning about multi-agent
systems.
Other tools supporting epistemic or plain \ATL specifications
exist~\cite{AHMQRT98,GM04,KNNPPSWZ07,LQR09}.
In our experiments we found that the performance of \MCMASSLK on the \ATL and
\CTLK fragments was comparable to that of \MCMAS, one of the leading checkers
for multi-agent systems.
This is because we adopted an approach in which the colouring with strategies is
specification-dependent and is only performed after the set of reachable states
is computed.

As described, a further notable feature of \MCMASSLK is the ability to
synthesise behaviours for multi-player games, thereby going beyond the
classical setting of two-player games.

We found that the main impediment to better performance of the tool is
the size of the BDDs required to encode sets of extended
states. Future efforts will be devoted to mitigate this problem as
well as to support other fragments of SL.



\end{section}
















