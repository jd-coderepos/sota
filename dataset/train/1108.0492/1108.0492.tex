\documentclass[11pt,letter]{article}



\usepackage{cite}
\usepackage{graphicx}
\usepackage{amsmath}
\usepackage{amsthm}
\usepackage{amsfonts}
\usepackage{amssymb}
\usepackage{fullpage}
\usepackage[margin=1in]{geometry}
\usepackage{color}
\usepackage{latexsym}
\usepackage{algorithm}
\usepackage[noend]{algorithmic}
\usepackage{subfigure}
\usepackage{soul}



\newtheorem{theorem}{Theorem}[section]
\newtheorem{corollary}[theorem]{Corollary}
\newtheorem{lemma}[theorem]{Lemma}
\newtheorem{observation}[theorem]{Observation}
\newtheorem{definition}[theorem]{Definition}
\newtheorem{conjecture}[theorem]{Conjecture}
\newtheorem{proposition}[theorem]{Proposition}
\newtheorem{problem}[theorem]{Problem}

\def\marrow{\marginpar[\hfill]{}}
\def\todo#1{\textsc{(TODO: \marrow\textsf{#1})}}
\def\rom#1{\textsc{(Rom says: \marrow\textsf{#1})}}
\def\matya#1{\textsc{(Matya says: \marrow\textsf{#1})}}
\def\gila#1{\textsc{(Gila says: \marrow\textsf{#1})}}


\renewcommand{\qedsymbol}{\hfill\rule{2mm}{2mm} \vspace{2mm}} \renewcommand{\algorithmicrequire}{\textbf{Input:}}
\renewcommand{\algorithmicensure}{\textbf{Output:}}


\newcommand{\old}[1]{{{}}}

\def\bd{{\partial}}
\def\segment#1#2{{\overline{#1#2}}}
\def\wedge#1{{W_{{#1}}}}
\def\halfplane#1{{h_{{#1}}}}
\def\leftray#1{{\rho{\stackrel{_\nwarrow}{_{#1}}}}}
\def\rightray#1{{\rho{\stackrel{_\nearrow}{_{#1}}}}}
\def\leftline#1{{l(\leftray{#1})}}
\def\rightline#1{{l(\rightray{#1})}}
\def\topregion#1{{{R}^{top}_{{#1}}}}
\def\bottomregion#1{{{R}^{bot}_{{#1}}}}
\def\clst#1{{\C_{{{F}}({{#1}})}}}

\newcommand{\newtext}[1]{\hl{#1}}
\newcommand{\oldtext}[1]{\st{#1}}


\def\C{{\cal{C}}}


\def\grid{{\cal{G}}}
\def\graph{{{G}}}
\def\UDG{{U\!DG}}
\def\nf{{n\!f}}


\begin{document}


\title{Symmetric Connectivity with Directional Antennas\thanks{Work by R. Aschner and G. Morgenstern was partially supported by the Lynn and William Frankel Center for Computer Sciences. Work by R. Aschner and M. Katz was partially supported by the Israel Ministry of Industry, Trade and Labor (consortium CORNET). Work by M. Katz was partially supported by grant 1045/10 from the Israel Science Foundation.
}}

\author{Rom Aschner \ \ \ Matthew J. Katz \ \ \ Gila Morgenstern
\\
\\
{\small Department of Computer Science, Ben-Gurion University, Israel} \\
{\small {\tt romas,matya,gilamor@cs.bgu.ac.il}}}


\old{
\author{Rom Aschner\thanks{Partially supported by the Lynn and William Frankel Center for Computer Sciences, and by the Israel Ministry of Industry, Trade and Labor (consortium CORNET).} \ \ \ Matthew J. Katz\thanks{Partially supported by grant 1045/10 from the Israel Science Foundation, and by the Israel Ministry of Industry, Trade and Labor (consortium CORNET).} \ \ \ Gila Morgenstern\thanks{Partially supported by the Lynn and William Frankel Center for Computer Sciences.}
\\
{\small Department of Computer Science, Ben-Gurion University, Israel} \\
{\small {\tt romas,matya,gilamor@cs.bgu.ac.il}}}
}
\old{
\author{Rom Aschner\thanks{Dept. of Computer Science, Ben-Gurion University, Beer-Sheva 84105, Israel, {\tt romas@cs.bgu.ac.il}. Partially supported by the Lynn and William Frankel Center for Computer Sciences, and by the Israel Ministry of Industry, Trade and Labor (consortium CORNET).} \and Matthew J. Katz\thanks{Dept. of Computer Science, Ben-Gurion University, Beer-Sheva 84105, Israel, {\tt matya@cs.bgu.ac.il}. Partially supported by grant 1045/10 from the Israel Science Foundation, and by the Israel Ministry of Industry, Trade and Labor (consortium CORNET).} \and Gila Morgenstern\thanks{Dept. of Computer Science, Ben-Gurion University, Beer-Sheva 84105, Israel, {\tt gilamor@cs.bgu.ac.il}. Partially supported by the Lynn and William Frankel Center for Computer Sciences.}}
}


\maketitle




\begin{abstract}
Let  be a set of points in the plane, representing transceivers equipped with a directional antenna of angle  and range . The coverage
area of the antenna at point  is a circular sector of angle  and radius , whose orientation can be adjusted.
For a given assignment of orientations,
the induced {\em symmetric communication graph} (SCG) of  is the undirected graph, in which two vertices (i.e., points)  and  are connected by an edge if and only if  lies in 's sector and vice versa. In this paper we ask what is the smallest angle  for which there exists an integer , such that for any set  of  antennas of angle  and unbounded range, one can orient the antennas so that (i) the induced SCG is connected, and (ii) the union of the corresponding wedges is the entire plane. We show (by construction) that the answer to this problem is , for which . Moreover, we prove that if  and  are two quadruplets of
antennas of angle  and unbounded range, separated by a line, to which one applies the above construction, independently, then
the induced SCG of  is connected. This latter result enables us to apply the construction locally, and to solve the following two further problems.

In the first problem ({\em replacing omni-directional antennas with directional antennas}), we are given a connected unit disk graph, corresponding to a set  of omni-directional antennas of range 1, and the goal is to replace the omni-directional antennas by directional antennas of angle  and range  and to orient them, such that the induced SCG is connected, and, moreover, is an -spanner of the unit disk graph, w.r.t. hop distance. In our solution  and the spanning ratio is 8. In the second problem ({\em orientation and power assignment}), we are given a set  of directional antennas of angle  and adjustable range. The goal is to assign to each antenna , an orientation and a range , such that the resulting SCG is (i) connected, and (ii)  is minimized, where  is a constant. For this problem, we present an -approximation algorithm.

\end{abstract}
\newpage

\section {Introduction}

Let  be a set of points in the plane, and assume that each point represents a transceiver equipped with a directional antenna. The {\em coverage area} of a directional antenna located at point  of angle  and range , is a sector of angle  of the disk of radius  centered at , where the orientation of the sector can be adjusted. We denote the coverage area of the antenna at  by  (since when assuming unbounded range the sector becomes a wedge).
The induced {\em symmetric communication graph} (SCG) of  is the undirected graph over , in which two vertices (i.e., points)  and  are connected by an edge if and only if  and .


The vast majority of the papers dealing with algorithmic problems motivated by wireless networks, consider omni-directional antennas, whose coverage area is often modeled by a disk. Many of these papers study problems, in which one has to assign radii (under some restrictions) to the underlying antennas so as to satisfy various coverage or communication requirements, while optimizing some measure, such as total power consumption.
Only very recently, researches have begun to study such problems for directional antennas. Directional antennas
have some noticeable advantages over omni-directional antennas. In particular, they require less energy to reach a point at a given distance, and they often reduce the level of interferences in the network.

In this paper we ask the following question:
\begin{problem}\label{prob:smallest_angle}
What is the smallest angle  for which there exists an integer , such that for any set  of  points in the plane,
representing transceivers equipped with directional antennas of angle  and unbounded range, one can orient the antennas so that (i) the induced SCG is connected, and (ii) the union of the corresponding wedges is the entire plane, i.e., for any point , there exists a point , such that .
\end{problem}

We would like to use the solution to this problem as a building block in the study of the following two important applications.
These applications have been studied under the asymmetric model of communication (where there is a directed edge from  to  if and only if ), but not under the (more natural) symmetric model of communication, where they are considerably more difficult.

\vspace{-2mm}
\paragraph{Replacing omni-directional antennas with directional antennas.}
Given a set  of points in the plane, let  be the {\em unit disk graph} of  (i.e., two points of  are connected by an edge if and only if the distance between them is at most 1), and assume that  is connected. Notice that  is the communication graph obtained by placing at each point of  an omni-directional antenna of range 1.
The goal is to replace the omni-directional antennas with directional antennas of some small angle  and range , such that (i) , (ii) the induced SCG is connected, and, moreover, (iii) the SCG is an -spanner of , w.r.t. hop distance (i.e., there exists a constant , such that, for each edge  of , there is a path between  and  in the SCG, consisting of at most  hops).

\vspace{-2mm}
\paragraph{Orientation and power assignment.}
Given a set  of directional antennas of angle  and adjustable range, the goal is to assign to each antenna , an orientation and a range , such that the resulting SCG is (i) connected, and (ii)  is minimized, where  is the distance-power gradient (typically between  and ).

\paragraph{Related work.}
A major challenge in the context of directional antennas is how to replace omni-directional antennas with directional antennas, such that (strong) connectivity is preserved, as well as other desirable properties, e.g., short range, similar hop distance, etc.
Several papers have considered this problem under the asymmetric model.
Caragiannis et al.~\cite{CKK+08} consider the problem of orienting the directional antennas and fixing their range, such that the induced graph is strongly connected and the assigned range is minimized.
They present a 3-approximation algorithm for any angle ; the maximum hop distance in their construction can be linear.
In their survey chapter, Kranakis et al.~\cite{KKM} consider this problem in a more general setting, where each transceiver is equipped with  directional antennas.
Damian and Flatland~\cite{DF10} show how to minimize both the range and the hop-ratio (w.r.t. the unit disk graph), for . Subsequently, Bose et al.~\cite{BCDFKM11} show how to do it for any .
Carmi et al.~\cite{CKLR09} were the first to study this problem under the symmetric model. They show that it is always possible to obtain a connected graph for , assuming the range is unbounded (i.e., equal to the diameter of the underlying point set). Later, a somewhat simpler construction was proposed by Ackerman et al.~\cite{AGP10}.
Carmi et al.~\cite{CKLR09} also observe that for  it is not always possible to orient the antennas such that the induced SCG is connected.
Ben-Moshe et al.~\cite{bcckms-dawn-10} investigate the problem of orienting quadrant antennas with only four possible orientations (, , , and ), and vertical half-strip antennas with only two possible orientations (up and down). Both problems are studied under the symmetric model.

The power assignment problem for omni-directional antennas is known to be NP-hard and was studied extensively; see, e.g.,~\cite{KKKP00,CPS99,CMZ02,C10}.
The orientation and power assignment problem, under the asymmetric model, was considered by Nijnatten~\cite{N08}, who observed that there exists a simple -approximation algorithm for any . His solution is based on -approximation algorithms for the energy-efficient traveling salesman tour problem. The quality of his approximation does not depend on . Notice that according to the observation of Carmi et al.~\cite{CKLR09} above, there does not always exist a solution to the problem under the symmetric model when .



\paragraph{Our results.}
In Section~\ref{sec:connected_coverage} we show that the solution to problem~\ref{prob:smallest_angle} is , for which . That is, we show how to orient any four antennas of angle , such that there is a path between any two of them and they collectively cover the entire plane (assuming unbounded range).
In order to use this construction as a building block in the solution of appropriate optimization problems, we need to be able to apply it locally, within a small geographic region, and to have a connection between nearby regions. Unfortunately, in Section~\ref{sec:A_cub_B} we give an example showing that we may not have such a connection. We overcome this difficulty by proving the following theorem. If  and  are two quadruplets of
antennas of angle  and unbounded range, separated by a line, to which one applies the above construction, independently, then
the induced SCG of  is connected.
In Section~\ref{sec:replacing_omni} we address the first application above. We show how to replace omni-directional antennas of range 1 with directional antennas of angle  and range  and orient them, such that the induced SCG is an 8-spanner of the unit disk graph, w.r.t. hop distance.
In Section~\ref{sec:power_assignment} we study the second application. We show how to assign an orientation and range to each antenna in a given set of directional antennas of angle , such that the induced SCG is connected and the total power consumption is at most some constant times the total power consumption in an optimal solution.
























\section{Connected coverage of the plane}\label{sec:connected_coverage}


In this section we consider Problem~1.1.
What is the smallest angle  with the property that there exists a positive integer ,
such that for any set  of at least  points,
one can place directional antennas of angle  and unbounded range at the points of , so that
(i)~the induced SCG is connected,
and (ii)~the plane is entirely covered by the antennas.

We show that the answer to the above question is .
We first show that for ,  is such an integer. Then, we show that for any , such an integer  does not exist.

\paragraph{Notation.}
We denote the antenna at point  by ,
the left ray bounding  (when looking from  into ) by ,
and the right ray by .
The lines containing these two rays are denoted by  and , respectively.

\subsection {}


\begin{theorem} \label{thm:fourpoints}
Let  be a set of four points in the plane representing the locations of four transceivers equipped with directional antennas of angle .
Then, one can assign orientations to the antennas, such that
the induced SCG is connected,
and the plane is entirely covered by the four corresponding (unbounded) wedges.
\end{theorem}

\begin{proof}
Denote the convex hull of  by .
We distinguish between the case where  is a convex quadrilateral and the case where it is a triangle.
If  is a convex quadrilateral, then one of its angles is of size at most .
Each of the two diagonals of  divides each of its corresponding two angles into two smaller angles,
such that at least one of these smaller angles is of size at most .
Thus, at least  of the  angles defined by  and its two diagonals are of size at most .
Denote the intersection point of the two diagonals by . Then, there exist two adjacent vertices  of , such that
 and . Therefore, one can orient the antennas, such that
the resulting SCG includes the two diagonals and the edge , and is thus connected.

\begin{figure}[htb]
 \centering
 \subfigure[]{
   \centering
       \includegraphics[width=0.40\textwidth]{fig/lemma1_convex}
		   \label{fig:four_pts_convex}
  }
 \subfigure[]{
    \centering
        \includegraphics[width=0.40\textwidth]{fig/lemma1_concave2}
        \label{fig:four_pts_concave}
 }
	\caption{Proof of Theorem~\ref{thm:fourpoints}.}
\end{figure}

\old{
\begin{figure}[htp]
   \centering
       \includegraphics[width=0.45\textwidth]{fig/lemma1_convex}
       \includegraphics[width=0.45\textwidth]{fig/lemma1_concave2}
   \caption{Proof of Theorem~\ref{thm:fourpoints}.}
   \label{fig:four_pts}
\end{figure}
}
Let  and  be the other two vertices of , such that  is adjacent to .
Then, we saw that one can orient the antennas, such that the resulting SCG includes
the edges  and  and the edge .
Denote by  the line that passes through  and , and
by  the closed half plane defined by  and containing .
Reorient the antenna  at  (resp.,  at ), such that it faces  and one of its bounding rays
passes through  (resp., ); see Figure~\ref{fig:four_pts_convex}.
Notice that by doing so, we do not lose any of the graph edges, and, moreover,
the half plane  is entirely covered by the two antennas.
To complete the proof we need to adjust the antennas at  and , so that the half plane 
(on the other side of ) is also covered.
Since  (resp. ) is covered by  (resp., ), we reorient  (resp., ) so that
it is opposite  (resp., ); see Figure~\ref{fig:four_pts_convex}. By doing so,
we do not lose any of the graph edges, and, moreover, the half plane  is covered by .

Assume now that  is a triangle  and that . Then,  has at least
two angles of size at most . W.l.o.g., assume  and .
Orient  and , as above, so that  is covered by .
Notice that both  and  contain , and therefore both cover  and .
Orient  and , so that  is covered by . The preceding observation implies that
either  covers  and  covers , or vice versa; see Figure~\ref{fig:four_pts_concave}.
Thus, in any case, the obtained graph is connected.

\end{proof}



Let us observe a few properties of the resulting structure.
First, notice that the orientation of each antenna differs from the orientations of the other three by , , and , respectively.
Second, each antenna is {\em coupled} with two of the others; namely, with those whose orientation differs from its own by  and , respectively. For example, in Figure~\ref{fig:four_pts_convex},  is coupled with  and with .
Notice that each such couple covers a half plane. E.g.,  and  cover the appropriate half plane defined by , and  and  cover the appropriate half plane defined by .


\paragraph{Remark.}
Clearly, at least four antennas of angle  are needed in order to cover the entire plane, so we cannot replace the number four in Theorem~\ref{thm:fourpoints} by a smaller number. However, if we are using antennas of angle  (resp., ), then it is relatively easy to show that three points (resp., two points) are sufficient (i.e.,  and ).


\subsection{}
As observed by Carmi et al.~\cite{CKLR09}, if ,
then it is not always possible to orient the antennas such that the resulting graph is connected.
(Consider, for example, three antennas located at the vertices of an equilateral triangle.
Each of these antennas can cover at most one of the other two vertices, thus it is impossible to obtain a connected graph in this setting.)

For ,
Carmi et al.~\cite{CKLR09}, and subsequently Ackerman et al.\cite{AGP10},
showed how to obtain a connected symmetric graph, for any set of antennas of angle .
However, their construction does not ensure that the union of the wedges covers the entire plane.
Actually, it is not always possible to orient a set of antennas with angle , so that the induced graph is connected and the entire plane is covered. (This is true even in the asymmetric model where one only requires strong connectivity, as observed by Bose et al.~\cite{BCDFKM11}.)
\old{
To see this, consider, e.g., a set  of points on a line.
At least one of the corresponding antennas must be oriented such that its wedge is empty of other points of .
See Figure~\ref{fig:counter_example};
in order to cover the plane, one needs, in particular, to cover the point . However, if  is far enough from the line,
then any antenna that covers  cannot cover any other point of  (except for the point at its apex).
Thus, the antenna that covers  is isolated in the resulting SCG.


\begin{figure}[htp]
   \centering
       \includegraphics[width=0.6\textwidth]{fig/example_figure1}
   \caption{The antenna that covers  is isolated in the resulting SCG.}
   \label{fig:counter_example}
\end{figure}
}
To see this, consider, e.g., a set  of points on a vertical line segment .
In order to cover a point that lies, e.g., far enough to the right of ,
at least one of the corresponding antennas must be oriented such that its wedge is empty of points of  (except for the point at its apex).
This antenna is isolated in the resulting SCG.

\section{Separated quadruplets are connected}\label{sec:A_cub_B}

Let  and  be two quadruplets of points (representing transceivers) in the plane, and assume that each of the transceivers
is equipped with a directional antennas of angle . Orient the antennas corresponding to the points in  (resp., in ),
such that they satisfy the conditions of Theorem~\ref{thm:fourpoints}.
Clearly, each point in  is covered by at least one antenna corresponding to a point in , and vice versa.
Unfortunately, this does not imply that the SCG induced by  is connected;
see Figure~\ref{fig:8pts_no_speartion} for an example where there is no edge between
 and  in the SCG of .
Theorem~\ref{thm:twosets} below is crucial for our subsequent applications. It states that if the quadruplets  and  can be separated
by a line, then the induced SCG is surely connected.





\begin{figure}[htp]
   \centering
       \includegraphics[width=0.5\textwidth]{fig/8pts_no_speartion}
   \caption{The induced SCG is not connected.}
   \label{fig:8pts_no_speartion}
\end{figure}


\begin{theorem} \label{thm:twosets}
Let ,  be two sets of four points each,
and
let the (antennas corresponding to the) points of  and, independently, the points of  be oriented as in the proof of Theorem~\ref{thm:fourpoints}.
If there exists a line  that separates between  and , then
the SCG induced by  is connected.
\end{theorem}


\begin{proof}
It is enough to show that there exist a point  and a point 
that cover each other (i.e.,  and ).
Assume w.l.o.g. that  is vertical and that the points of  (resp., )
lie to the left (resp., right) of .
Denote by  (resp., ) the half plane that is defined by  and contains  (resp., ).
Let  be the smallest number, such that one can pick  points of 
that (together) cover . Clearly,  is either 2 or 3.
We distinguish between two cases. In the first case, at least one of the two numbers  and  is 2, where  is defined analogously.
In the second case, both  and  are 3.

\begin{figure}[htb]
 \centering
 \subfigure[Case 1]{
   \centering
       \includegraphics[width=0.4\textwidth]{fig/thm_figure4}
   		 \label{fig:thm_case1}
  }
 \hspace{0.5cm}
 \subfigure[Case 2]{
    \centering
      \includegraphics[width=0.4\textwidth]{fig/thm_figure6}
      \label{fig:thm_case2}
 }
\caption{Proof of Theorem~\ref{thm:twosets} --- Cases 1 and 2.}
\end{figure}




\old{
\begin{figure}[htp]
   \centering
       \includegraphics[width=0.6\textwidth]{fig/thm_figure4}
   \caption{Proof of Theorem~\ref{thm:twosets} --- Case 1.}
   \label{fig:thm_case1}
\end{figure}
}
{\bf Case~1.}
There exist two points from one set that (together) cover the half plane containing the other set.
W.l.o.g., assume points  cover , and  is not below .
In this case,  and  form a couple (see above), and the half plane covered by them
contains . That is, the rays  and  are parallel to , where
 is pointing downwards and  is pointing upwards, and  and 
are perpendicular to  and pointing rightwards; see Figure~\ref{fig:thm_case1}.
Let  be a point that covers the point .
If  lies in , then we are done, since  and  cover each other.
Assume, therefore, that  lies in , and that  does not
cover  (since, otherwise,  and  cover each other and we are done).
It follows that  crosses the line segment .
Let  be the point in , such that  and  form a couple, and 's orientation is equal to 's orientation plus  (adding counterclockwise). Then,  lies above (or on)  (and, of course, ), and  lies in . But,
the former assertion implies that  lies in , hence  and  cover each other, and we are done.



\old{
\begin{figure}[htp]
   \centering
       \includegraphics[width=0.6\textwidth]{fig/thm_figure6}
   \caption{Proof of Theorem~\ref{thm:twosets} --- Case 2.}
   \label{fig:thm_case2}
\end{figure}
}

{\bf Case~2.}
 is covered by three points of  (but not by two), and  is covered by three points of  (but not by two).
Notice that on each side of  there exists a point, whose wedge
divides the half plane on the other side of  into three disjoint regions.
More precisely, there exists  that divides 
into three disjoint regions:
, , and ; see Figure~\ref{fig:thm_case2}.
We denote the ``complement'' regions of  and  by  and , respectively, where  (resp., ) is obtained by rotating  (resp., ) around its apex by .
Notice that .
Let  be the point in , such that  and  form a couple, and 's orientation is equal to 's orientation minus .
Then,  lies above  (i.e., ) and  covers .
Similarly, let  be the point in , such that  and  form a couple, and 's orientation is equal to 's orientation plus . Then,  and  covers ; see Figure~\ref{fig:thm_case2} for an illustration.

As above, let  be a point in  that divides  into three disjoint regions,
, , and ; denote the
``complement'' regions of  and  by  and , respectively;
let ,
 and
, such that
 covers  and  covers .

We now show that there exist two points, one in  and one in , that cover each other.
We distinguish between a few subcases.



\noindent
{\bf Case 2a.}  and . That is,  and  cover each other.


\begin{figure}[htb]
 \centering
 \subfigure[Case 2b]{
   \centering
       \includegraphics[width=0.4\textwidth]{fig/thm_figure7}
   \label{fig:thm_case2b}
  }
 \hspace{0.5cm}
 \subfigure[Case 2c]{
     \centering
       \includegraphics[width=0.4\textwidth]{fig/thm_figure8}
   \label{fig:thm_case2c}
 }
\caption{Proof of Theorem~\ref{thm:twosets} --- Cases 2b and 2c.}
\end{figure}



\old{
\begin{figure}[htp]
   \centering
       \includegraphics[width=0.6\textwidth]{fig/thm_figure7}
   \caption{Proof of Theorem~\ref{thm:twosets} --- Case 2b.}
   \label{fig:thm_case2b}
\end{figure}
}

\noindent
{\bf Case 2b.}  (that is,  is covered by )
and ; see Figure~\ref{fig:thm_case2b}.
Assume that  (otherwise,  and  cover each other),
then,  and thus covers .
Recall that  and covers .
We conclude that  and  cover each other.


\old{
\begin{figure}[htp]
   \centering
       \includegraphics[width=0.6\textwidth]{fig/thm_figure8}
   \caption{Proof of Theorem~\ref{thm:twosets} --- Case 2c.}
   \label{fig:thm_case2c}
\end{figure}
}

\noindent
{\bf Case 2c.}  and ; see Figure~\ref{fig:thm_case2c}.
Then, , and, therefore,  covers both  and .
Now, if  lies in , then  and  cover each other, and we are done.
Otherwise, , but then,  and  cover each other.

Finally, it is easy to verify that all other subcases are symmetric to either Case~2b or Case~2c.

\end{proof}



\section{Replacing omni-directional antennas with directional antennas}\label{sec:replacing_omni}

Let  be a set of  points in the plane. The {\em unit disk graph} of , denoted , is the graph over , in which there is an edge between two points if and only if the distance between them is at most 1. Notice that  is the communication graph obtained, when each point of  represents a transceiver with an omni-directional antenna of range 1. Assume that  is connected.
Our goal in this section is to replace the omni-directional antennas with directional antennas of angle  and range  and to orient them, such that the induced {\em symmetric} communication graph is (i) connected, and (ii) a -spanner of , with respect to hop distance, where  is an appropriate constant. We first show that this can be done for  and .
Then, in Section~\ref{subsec:reducing_hops}, we reduce the hop ratio (i.e., ) from  to .

The main idea underlying our construction is to apply Theorem~\ref{thm:fourpoints} multiple times, each time to a cluster of points within a small region, and to use Theorem~\ref{thm:twosets} to establish that the SCG induced by any two such clusters is connected (assuming unbounded range).

We now describe our construction.
Lay a regular grid  over , such that the length of a cell side is .
(If  is the bottom-left corner of a cell , then  is the semi-open square .)
For a cell  of , the {\em block} of  is the  portion
of  centered at .
Each of the  cells surrounding  is a {\em neighbor} of .
A cell of  is considered {\em full} if it contains at least four points of .
It is considered {\em non-full} if it contains at least one and at most three points of .
Proposition~\ref{prop:fullcell} below is analogous to a proposition of Bose et al.~\cite{BCDFKM11}, referring to a similar grid.

\begin{proposition}{(\!\!\cite{BCDFKM11})} \label{prop:fullcell}
Let  be a cell of . Then, any path in  that begins at a point in 
and exits the block of , must pass through a full cell in 's block (not including  itself, which may or may not be full).
In particular, if there are points of  outside 's block, then at least one of the 8 neighbors of  is full.
\end{proposition}

\begin{figure}[htp]
   \centering
   \subfigure[]{
       \includegraphics[width=0.45\textwidth]{fig/cell_size1}
       }
   \subfigure[]{
       \includegraphics[width=0.45\textwidth]{fig/cell_size2}
       }

   \caption{Proposition~\ref{prop:fullcell}.}
   \label{fig:fullcell}
\end{figure}


\old{
\begin{figure}[htp]
   \centering
       \includegraphics[width=0.45\textwidth]{fig/cell_size}
   \caption{Proposition~\ref{prop:fullcell}.}
   \label{fig:fullcell}
\end{figure}
}

\begin{proof}


\old{
Observe that since each cell is a semi-open square of side length , then any path connecting two non-neighboring cells
contains at least  middle-points.
Now, let  be a path that begins at a point 
and exits the block of , where  is the first point along  not in 's block.
Assume, e.g., that  lies above 's block,
and let  be the first point along  for which the points  all lie above the cell .
By our observation at the beginning of the proof,  (clearly , and  and  lies in non-neighboring cells).
Let , , and 
be the top cells in the block of , left to right,
then .
If these (at least ) points lies in one or in two of the three cells, then at least one cell contains at least  points, namely is full.
Otherwise, there exists a subpath of  that connects a point in  with a point in .
By our observation at the beginning of the proof,
such a path contains at least  middle-points lying in , implying that  in this case is full.
}

Let  be a path that begins at a point 
and exits 's block, where  is the first point in  that is not in 's block.
Assume, e.g., that  lies on or above the line containing the top side of 's block,
and let  be the last point in  that lies below the line containing the top side of ; see Figure~\ref{fig:fullcell}.
Then, all points in the subpath
 of , except for the two extreme ones, lie in the union of the top three cells of 's block, and their number (excluding the two extreme points) is at least 7. Let , , and  be the top three cells, from left to right, of 's block. If  is contained in only one of these cells, or, alternatively, in only two of these cells, then at least one of these cells contains at least 4 points, and we are done (Figure~\ref{fig:fullcell}(a)). Otherwise, at least one of the points lies in  and at least one of the points lies in , implying that  contains at least 7 points (Figure~\ref{fig:fullcell}(b)).



\end{proof}


If none of the grid cells is full, then, it is easy to see that  is contained in a  square. In this case, we can apply Theorem~\ref{thm:fourpoints} to an arbitrary subset  of four points of , and orient each of the other points of  towards a point of  that covers it. By setting , we obtain a SCG, in which the hop distance between any two points is at most 5 (see below).
We thus may assume that at least one of the grid cells is full.

For each cell  of ,
we orient the points in  as follows.
If  is full, then arbitrarily pick four points in  as 's {\em hub points},
and orient these hub points according to Theorem~\ref{thm:fourpoints}.
Next, orient each non-hub point in  towards one of the hub points of  that covers it.
If  is non-full, then, for each point ,
pick a full cell closest to ,
that is, a full cell containing a point , such that the hop distance (in ) from  to  is not greater than the hop distance from  to any other point of  lying in a full cell.
Denote this cell by .
Notice that by Proposition~\ref{prop:fullcell}  is a neighbor of .
Orient each  towards a hub point of  that covers it.
Finally, set , so that each antenna can reach any point in its own cell or in a neighboring cell, provided that this point is within the antenna's wedge.

Let  be the resulting SCG.
Lemma~\ref{lemma:hop_distance}, below, states that  is a -spanner of , w.r.t. hop distance.
In particular, since  is connected, then so is .
We first prove the following auxiliary lemma.

\begin{lemma}
\label{lemma:neighboring}
Let  be an edge of ,
and let  and  be the cells of , such that  and .
\begin{itemize}
	\item[]
	(i)  	If  and  are both full, then either  or  and  are neighbors.
	\item[]
	(ii) 	If  is full and  is non-full, then either  or  and  are neighbors.
	\item[]
	(iii) If  and  are both non-full, then either  or  and  are neighbors.
\end{itemize}			
\end{lemma}

\begin{proof}
(i) This is obvious, since the Euclidean distance between  and  is at most 1 and the side length of a cell is .
\\
(ii) Let  be a point that determines the hop distance between  and  (see above).
By construction, the hop distance in  from  to  is not greater than the hop distance from  to , which is .
Thus, the Euclidean distance between  and  is at most , whereas the side length of a cell is .
We conclude therefore that either  or  and  are neighbors.
\\
(iii) Assume to the contrary that  and that  and  are not neighbors.
Let  (resp., ) be a point that determines the hop distance between  and  (resp.,  and ).
Consider the path  in  that is obtained by concatenating the shortest path from  to , the edge , and the shortest path from  to .
By our assumption,
 is not in 's block (and vice versa),
thus  starts at  and exits its block. By
Proposition~\ref{prop:fullcell},
 passes through a full cell  in 's block,
other than  (and other than , which is not in 's block).
Since  is full, . Now, if  visits  before it visits the point ,
then we get that  is a full cell closer to  than is , and, if  visits  after it visits ,
then we get that  is a full cell closer to  than is . Thus,
in both cases we arrive at a contradiction.
\end{proof}



\begin{lemma}
\label{lemma:hop_distance}
 is a -spanner of , w.r.t. hop distance.
\end{lemma}


\begin{proof}
Let  be an edge of .
We show that  contains a path from  to  consisting of at most  edges.
Let  and  be the cells of , such that  and .
We distinguish between the three cases that are listed in Lemma~\ref{lemma:neighboring}.

\begin{figure}[htb]
 \centering
 \subfigure[]{
    \centering
      \includegraphics[width=0.25\textwidth]{fig/hopspanner3}
        \label{fig:hopspanner3}
   }
   \subfigure[]{
     \centering
     \includegraphics[width=0.25\textwidth]{fig/hopspanner2}
     \label{fig:hopspanner2}
    }
    \subfigure[]{
     \centering
     \includegraphics[width=0.25\textwidth]{fig/hopspanner1}
     \label{fig:hopspanner1}
    }
    \caption{ is a -spanner of , w.r.t. hop distance.}
\end{figure}

(i) Consider first the case where  and  are both full.
Then, by Lemma~\ref{lemma:neighboring}, either  or  and  are neighbors.
Notice that  is either a hub point of , or it is connected to one by a single edge; and the same holds for  and .
Assume first that ,
then, there exists a path in  from  to  consisting of at most 5 edges.
Indeed, such a path starts at , passes through at most four hub points of ,
and ends at .
Assume now that  and  are neighbors.
By Theorem~\ref{thm:twosets} and since the range of each antenna is sufficient to reach any point in any neighboring cell,
there exists an edge in  connecting a hub point of  with a hub point of .
Consequently,  contains a path from  to  consisting of at most 9 edges.
Indeed, such a path starts at ,
passes through at most four hub points of ,
continues to a hub point of ,
passes through at most four hub points of ,
and finally ends at ;
see Figure~\ref{fig:hopspanner3} for an illustration.

(ii) Consider now the case where one of the cells, say , is full and the other is non-full.
By construction,  is oriented to a hub point of a neighboring full cell  that covers it.
By Lemma~\ref{lemma:neighboring}, either  or  and  are neighbors.
Therefore, as in case (i) above,  contains a path from  to  consisting of at most 9 edges;
see Figure~\ref{fig:hopspanner2} for an illustration.

(iii) Finally, consider the case where  and  are both non-full.
By construction,  (resp., ) is connected by an edge to a hub point of a full cell  (resp., ) that covers it.
By Lemma~\ref{lemma:neighboring}, either  or  and  are neighbors.
Therefore, as in case (i) above,  contains a path from  to  consisting of at most 9 edges;
see Figure~\ref{fig:hopspanner1} for an illustration.
\end{proof}




\subsection{Reducing the hop ratio to 8}\label{subsec:reducing_hops}

We show how to reduce the hop ratio from 9 to 8, by picking the four hub points (in each full cell) more carefully, and by reorienting the points in non-full cells, so that whenever two such points are ``close'' to each other, they are oriented towards the same full cell.


\begin{figure}[htp]
   \centering
       \includegraphics[width=0.4\textwidth]{fig/four_pts_improved}
   \caption{The orientation assignment to  satisfies the conditions of Theorem~\ref{thm:fourpoints},
   and each of the points in  is covered by  or .}
   \label{fig:four_pts_improved}
\end{figure}




We first show how to carefully pick the four hub points of a full cell; see Figure~\ref{fig:four_pts_improved} for an illustration.
Let  be a full cell and put . Let  be the longest edge of , and assume, w.l.o.g., that  is horizontal and that  lies above it.
The half-strip with base  and facing upwards
covers at least one point of 
(otherwise,  has an edge longer than ). Let  be such a point, and let  be an arbitrary point in
. We pick  as the hub points of , where
 and  are the {\em supporting} hub points of .
Assume w.l.o.g. that  is to the left of  and that  is to the left of .
Assign  orientation up-right,  orientation up-left,  down-right, and  down-left.
This orientation assignment satisfies the conditions of Theorem~\ref{thm:fourpoints}.
Moreover, since , we can orient each non-hub point of 
towards one of the supporting hub points of .


Now, we would like to modify the rule by which we orient the points in non-full cells. For this we need Lemma~\ref{lemma:full_super_neighbor} below.
Let  be the subset of points of  lying in non-full cells, and
consider , the subgraph of  induced by .
Let  and  be two points in the same connected component of ,
and let  and  be the (non-full) cells of , such that  and .
As observed above, Proposition~\ref{prop:fullcell} implies that  is a neighbor of  and that  is a neighbor of .
Lemma~\ref{lemma:full_super_neighbor} below, shows that  is also a neighbor of  and that  is also a neighbor of .


\begin{lemma}
\label{lemma:full_super_neighbor}
Consider a connected component of , and let  be its corresponding set of points.
Let  be two points in , and let  be the cell of  such that .
Then  is a neighbor of .
\end{lemma}
\begin{proof}
Assume to the contrary that  is not a neighbor of .
Let  be a point that determines the hop distance between  and .
Consider the path  in  that is obtained by concatenating the subpaths  and ,
where  is a path in  from  to ,
and  is the shortest path in  from  to .
By our assumption,  exits 's block.
Therefore, by Proposition~\ref{prop:fullcell},  must pass through a full cell in 's block before exiting the block.
By definition,  does not pass through full cells, hence  must pass through a full cell  in 's block.
But,  visits  before it visits , thus  is a full cell closer to  than is 
--- a contradiction.
\end{proof}

\old{
\begin{corollary}
\label{col:super_neighbor_full_cell}
Let  corresponds to a connected component of ,
and let  be an arbitrary point.
Then, for any , the cell of  containing  is a neighbor of .
\end{corollary}
}

Given Lemma~\ref{lemma:full_super_neighbor}, we may reorient the points in  as follows.
For each connected component of  and its corresponding set of points , pick any point  and orient it,
as before, towards a hub point of  that covers it. Now, for each point , , orient  towards a hub point
of  that covers it.

Let  be the resulting SCG.
We are now able to reduce the number of hops in each of the paths described in the proof of Lemma~\ref{lemma:hop_distance} above.
Let  be an edge of , and let  and  be the cells of , such that  and .
(i)~Consider first the case where  and  are both full.
Then, there exists a path in  from  to  consisting of at most 7 edges.
Indeed, both  and  are connected to a supporting hub point by a single edge,
therefore, in the worst case, a path from  to  starts at , passes through
at most three hub points of ,
continues to a hub point of ,
passes through at most three hub points of ,
and finally ends at .


(ii)~Consider now the case where  is full and  is non-full.
Let  be the point in the connected component of  to which  belongs, such that each point in  was oriented towards a hub point of  that covers it. We claim that  is a neighbor of , even though it is possible that .
Indeed, assuming that  is not in 's block leads to a contradiction, as above. (Define the path in  that is obtained by concatenating the edge , the path in  from  to , and the shortest path in  from  to a point in . Then, by Proposition~\ref{prop:fullcell}, there is a full cell closer to  than is .)
However, the hub point of  towards which  is oriented, is not necessarily a supporting hub point,
thus the path from  to  consists of at most 8 edges.
Such a path starts at ,
passes through at most three hub points of ,
continues to a hub point of , passes through at most four hub points of ,
and finally ends at .

(iii)~Finally, consider the case where  and  are both non-full.
By construction,  and  are oriented to the same full cell ,
thus the path from  to  starts at ,
passes through at most four hub points of , and finally ends at ;
such a path consists of at most 5 edges.




Theorem~\ref{thm:replacing_summary} summarizes the main result of this section.
\begin{theorem}
\label{thm:replacing_summary}
Let  be a set of points, where each point represents a transceiver equipped with an omni-directional antenna of range 1,
and assume that  is connected.
Then, one can replace the omni-directional antennas with directional antennas of angle  and range ,
such that the induced SCG is (i) connected, and (ii) a 8-spanner of , w.r.t. hop distance.
\end{theorem}



\old{
\begin{figure}[htb]
 \centering
 \subfigure[]{
    \centering
      \includegraphics[width=0.25\textwidth]{fig/hopspanner6}
        \label{fig:hopspanner6}
   }
   \subfigure[]{
     \centering
     \includegraphics[width=0.25\textwidth]{fig/hopspanner5}
     \label{fig:hopspanner5}
    }
    \subfigure[]{
     \centering
     \includegraphics[width=0.25\textwidth]{fig/hopspanner4}
     \label{fig:hopspanner4}
    }
    \caption{-hop distance}
\end{figure}
}







\section{Orientation and power assignment}\label{sec:power_assignment}

In the construction described in Section~\ref{sec:replacing_omni}, each antenna is assigned range .
In this section we consider the problem of assigning to each antenna  an individual range, denoted .
Let  be a set of  points in the plane, representing the locations of directional antennas of angle .
In the {\em orientation and power assignment} problem one needs to assign to each of the antennas an orientation and a range, such that
(i) the resulting SCG is connected, and (ii)  is minimized, where  is the distance-power gradient (typically, ).

Assume, for convenience, that , for some integer .
Let  be a simple closed polygonal path whose vertices are the points in .
We partition  into subsets (called {\em sections}) of size 8, by traversing  from an arbitrary point . That is, each of the sections consists of eight consecutive points along .
For each section , we partition its points, according to their -coordinate, into a left subsets, , consisting of the four leftmost points of , and a right subset, , consisting of the four rightmost points of .
(Notice that the points in each quadruplet are not necessarily consecutive along .)
Next, we orient the antennas corresponding to the points in  (resp., ) according to Theorem~\ref{thm:fourpoints}.
By definition (of ), there exists a vertical line, , that separates between  and .
Therefore, by Theorem~\ref{thm:twosets}, the SCG induced by  is connected.

Moreover, consider two `adjacent' sections  and , and consider their corresponding lines  and .
Assume, e.g., that  lies to the left of ; see Figure~\ref{fig:power_assignment}. Then,  also separates between  and ,
and therefore, assuming unbounded range, the SCG induced by  is connected.
We conclude that any two adjacent sections are connected, and one can limit the range of the points in a section as long as they reach the points in the preceding and succeeding sections (and each other).


\begin{figure}[htp]
   \centering
       \includegraphics[width=0.45\textwidth]{fig/power_assignment}
   \caption{The points of  are shown as filled circles and the points of  as empty circles.}
   \label{fig:power_assignment}
\end{figure}

We take  to be an -approximation of an optimal solution to the traveling salesperson problem in the complete graph
 over , in which the weight of an edge  is ;  can be computed in polynomial time, see, e.g., \cite{BC00,BNSWW10,FLNL08,A01}.
For each section , , and for each point , we set  to be the maximum distance between  and a point in ,
where  are the preceding and succeeding sections of , respectively ( and ).
By construction, the induced SCG is connected.

Below we analyze the quality of the obtained power assignment, denoted , w.r.t. to the optimal power assignment using omni-directional antennas, denoted . Recall that the cost of  (denoted ) is .
Clearly,  is not greater than the cost of an optimal power assignment using directional antennas of angle .

\begin{theorem}
.
\end{theorem}

\begin{proof}
Let  be the 'th section. Then,

Let  denote a minimum spanning tree of the graph  (defined above), and
let  denote a minimum tour of . It is well known and easy to prove that
, where  is the sum of the weights of the edges in the appropriate structure.
Kirousis et al.~\cite{KKKP00} argued that . It follows that .
Thus,


\end{proof}

\paragraph{Remark.}
Our goal in this section was to establish that there exists an -approximation algorithm for the orientation and power assignment problem. We did not try to optimize the constant.













\begin{thebibliography}{1}

\bibitem{AGP10}
Eyal Ackerman, Tsachik Gelander, and Rom Pinchasi.
\newblock Ice-creams and wedge graphs.
\newblock CoRR abs/1106.0855, 2011.

\bibitem{A01}
Thomas Andreae.
\newblock On the traveling salesman problem restricted to inputs satisfying a relaxed triangle inequality.
\newblock {\em Networks}, 38:59--67, 2001.

\bibitem{BC00}
Michael A. Bender and Chandra Chekuri.
\newblock Performance guarantees for the TSP with a parameterized triangle inequality.
\newblock {\em Information Processing Letters}, 73(1-2):17--21, 2000.

\bibitem{bcckms-dawn-10}
Boaz Ben-Moshe, Paz Carmi, Lilach Chaitman, Matthew J. Katz, Gila Morgenstern, and Yael Stein.
\newblock Direction assignment in wireless networks.
\newblock In {\em Proc. 22nd Canadian Conf. on Computational Geometry}, pages 39--42, 2010.

\bibitem{BNSWW10}
Mark de Berg, Fred van Nijnatten, Ren{\'e} Sitters, Gerhard J. Woeginger, and Alexander Wolff.
\newblock The traveling salesman problem under squared Euclidean distances.
\newblock In {\em Proc. 27th Internat. Sympos. on Theoretical Aspects of Computer Science}, pages 239--250, 2010.



\bibitem{BCDFKM11}
Prosenjit Bose, Paz Carmi, Mirela Damian, Robin Flatland, Matthew J. Katz, and Anil Maheshwari.
\newblock Switching to directional antennas with constant increase in radius and hop distance.
\newblock {\em  Proc. 12th Algorithms and Data Structures Sympos.}, 2011.





\bibitem{C10}
Gruia Calinescu.
\newblock Min-power strong connectivity.
\newblock In {\em Proc. 13th Internat. Workshop on Approximation Algorithms for Combinatorial Optimization Problems},
pages 67--80, 2010.

\bibitem{CMZ02}
Gruia Calinescu, Ion I. Mandoiu, and Alexander Zelikovsky.
\newblock Symmetric connectivity with minimum power consumption in radio networks.
\newblock In {\em Proc. 2nd IFIP Internat. Conf. on Theoretical Computer Science}, pages 119--130, 2002.

\bibitem{CKK+08}
Ioannis Caragiannis, Christos Kaklamanis, Evangelos Kranakis, Danny Krizanc, and Andreas Wiese.
\newblock Communication in wireless networks with directional antennas.
\newblock In {\em 20th ACM Sympos. on Parallelism in Algorithms and Architectures}, pages 344--351, 2008.

\bibitem{CKLR09}
Paz Carmi, Matthew J. Katz, Zvi Lotker, and Adi Ros\'{e}n.
\newblock Connectivity guarantees for wireless networks with directional antennas.
\newblock {\em Computational Geometry: Theory and Applications}, 44(9):477--485, 2011.

\bibitem{CPS99}
Andrea E. F. Clementi, Paolo Penna, and Riccardo Silvestri.
\newblock Hardness results for the power range assignment problem in packet radio networks.
\newblock In {\em 2nd Internat. Workshop on Approximation Algorithms for Combinatorial Optimization Problems}, pages 197--208, 1999.

\bibitem{DF10}
Mirela Damian and Robin Flatland.
\newblock Spanning properties of graphs induced by directional antennas.
\newblock In {\em Electronic Proc. 20th Fall Workshop on Computational Geometry},
Stony Brook, NY, 2010.

\bibitem{FLNL08}
Stefan Funke, S\"{o}ren Laue, Zvi Lotker, and Rouven Naujoks.
\newblock Power assignment problems in wireless communication: Covering points by disks, reaching
few receivers quickly, and energy-efficient traveling salesman tours.
\newblock In {\em Proc. 4th IEEE internat. conf. on Distributed Computing in Sensor Systems}, pages 282--295, 2008.

\bibitem{KKKP00}
Lefteris M. Kirousis, Evangelos Kranakis, Danny Krizanc, and Andrzej Pelc.
\newblock Power consumption in packet radio networks.
\newblock {\em Theoretical Computer Science}, 243:289--305, 2000.









\bibitem{KKM}
Evangelos Kranakis, Danny Krizanc, and Oscar Morales.
\newblock Maintaining connectivity in sensor networks using directional antennae.
In Theoretical Aspects of Distributed Computing in Sensor Networks, Chapter 3, pages 83--110, S. Nikoletseas and J. D. P. Rolim (Eds.), Springer.

\bibitem{N08}
Fred van Nijnatten.
\newblock Range Assignment with Directional Antennas.
\newblock Master's Thesis, Technische Universiteit Eindhoven, 2008.

\end{thebibliography}
\end{document}
