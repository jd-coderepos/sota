\documentclass{article}




\usepackage{url}
\usepackage[english]{babel}
\usepackage[latin1]{inputenc}
\usepackage{color}
\usepackage{pgf}
\usepackage{graphicx}
\usepackage{times}
\usepackage{enumerate}
\usepackage{amsmath}
\usepackage{amsfonts}
\usepackage{amssymb}
\usepackage{amsthm}


\newtheorem{theorem}{Theorem}[section]
\newtheorem{corollary}[theorem]{Corollary}
\newtheorem{lemma}[theorem]{Lemma}
\newtheorem{proposition}[theorem]{Proposition}
\newtheorem{definition}[theorem]{Definition}
\newtheorem{remark}[theorem]{Remark}
\newtheorem{example}[theorem]{Example}



\newcommand{\Z}{{\mathbb{Z}}} \newcommand{\Q}{{\mathbb{Q}}} 
\newcommand{\C}{{\mathbb{C}}} \newcommand{\N}{{\mathbb{N}}}
\newcommand{\R}{{\mathbb{R}}} \newcommand{\K}{{\mathbb{K}}}


\newcommand{\ord}{\mathrm{ord}}
\newcommand{\gcrd}{\mathrm{GCRD}}
\newcommand{\Hom}{\mathrm{Hom}}
\newcommand{\cS}{\circledS}
\newcommand{\genexp}{\mathrm{GenExp}}
\newcommand{\valg}{\mathrm{ValG}}
\newcommand{\trunc}{\mathrm{Trunc}}
\newcommand{\ind}{\mathrm{Ind}}
\newcommand{\Seq}{\mathbf{S}}
\newcommand{\Gq}{\mathrm{Gquo}}
\newcommand{\SNF}{\mathrm{SNF}}
\newcommand{\todo}[1]{\marginpar{ToDo: #1 }}

\begin{document}


\title{Closed form solutions of linear difference equations in terms of symmetric products}



\author{Yongjae Cha\thanks{Supported by the Austrian Science Fund (FWF) under grant P22748-N18}\\
Johannes Kepler University\\
      4040 Linz (Austria)\\
\texttt{ycha@risc.jku.at}}


\maketitle

\begin{abstract}
In this paper we show how to find a closed form solution for third
order difference operators in terms of solutions of second order
operators. This work is an extension of previous results on finding
closed form solutions of recurrence equations and a counterpart to
existing results on differential equations. As motivation and
application for this work, we discuss the problem of proving
positivity of sequences given merely in terms of their defining
recurrence relation. The main advantage of the present approach to
earlier methods attacking the same problem is that our algorithm
provides human-readable and verifiable, i.e., certified proofs.
\end{abstract}






\section{Introduction}
This paper presents an extension of the algorithm {\em solver}~\cite{YC11,CH09,CHG10} that
returns closed form solutions for second order linear difference equations to third order linear 
difference equations.  The solutions that we are looking for are in terms of (finite) sums
of squares. This is motivated by applying the algorithm for proving inequalities on
special functions, i.e., on expressions that may be defined in terms of linear difference
equations with polynomial coefficients. Conjectures about positivity of special functions
inequalities arise in many applications in mathematics and science. Proving them usually
requires profound knowledge on relations between these special functions. It is well known
that there exist many algorithms for proving and finding special function
identities~\cite{Zeil90a,ChyzakDM,AeqB,KoutschHF}. For automated proving of special
functions inequalities only few approaches exist. Gerhold and
Kauers~\cite{GKIneq,MKSumCracker} introduced a method that is based on Cylindrical
Algebraic Decomposition (CAD). This method has been proven to work well on many non-trivial
examples~\cite{MKTuran,VPSI}, but even though correctness is easy to be seen, termination
cannot be guaranteed, hence it is not an algorithm in the strict sense. A first attempt to
clarify the latter issue has been made in~\cite{MKVP10}. One of the features of proofs of
special functions identities is that they usually come with a certificate, i.e., some easy
to check identity that verifies the proof. The CAD-based approach can not hope to have a
similar certificate in the near future. The method presented here is a first step toward
human readable proofs of special functions inequalities, although admittedly a
representation in terms of sums of squares with positive coefficients is not expected to
exist for any given input. Besides this application, the results presented
  are of independent interest as they provide difference case counterparts to results obtained
  for the differential case~\cite{MS85, vH07}. 

First we review the available results in the differential case.  Let  be a differential
field and  be a linear homogeneous third order
differential operator.  Singer~\cite{MS85} characterizes when solutions of  can be
written in terms of solutions of a second order operator in .  Van
Hoeij~\cite{vH07} handles the similar problem when the coefficients of the second order
operator are restricted to  and shows that it will be either of the following cases.
\begin{description}
\item[Case 1]  is the symmetric square of a second order operator 
\item[Case 2]  is gauge equivalent to a symmetric square of a second order operator
  
\end{description} 
The definitions of symmetric products and gauge equivalence are recalled in sections~\ref{sec:sp}
and~\ref{sec:ge} below.  The algorithm given in~\cite{vH07} returns a second order differential
operator, , and a gauge transformation in  that sends
solutions of the symmetric squares of  to solutions of  for Case 2.

In the differential case, the symmetric square of  has order~5 if and only if we are
in Case~1. In this case, there is a simple formula that gives .  Case~2 is equivalent to the
symmetric square of  having order~6 and a first order right-hand side factor
in~ as well as a certain conic of (\cite[Equation 4.2.1]{MS85}) having a
non-zero solution in~.  Since for  this conic is solvable over , the
last condition becomes trivial in this case.  The algorithm given in~\cite{vH07} in the
first step checks the order of the symmetric square of  to distinguish the cases.


The difference case behaves differently; here we denote by  the ring of
linear difference operators, where  denotes the shift operator.
Example~\ref{example:symp} shows that the cases can not be distinguished according to the
order of the symmetric squares when the coefficients are in~. 
To set up a counterpart theorem for difference equations, 
this example shows that we need one more transformation than that in the differential case. 
Furthermore in Case~1, the algorithm for
finding the second order operator is more complicated than in the differential case.




Summarizing, the ideas used in the differential case can not be carried over immediately to
the difference case. Furthermore our aim is to have a closed form solution of the given
input. Hence, if a factorization is found that is not solvable, this fails to satisfy our
goal. Thus we build on the ideas of the algorithm {\em solver}~\cite{YC11,CH09,CHG10}.
Here we say that a function is in closed form if it is a linear combination of elementary
functions, special functions or hypergeometric functions over .  For instance the
modified Bessel function of the first kind is a closed form solution of the second order
operator~.

The algorithm {\em solver} returns closed form solutions for second order linear
difference operators. The main idea of {\em solver} is to map the given operator~ to
an operator~ of which a solution is known. This transformation is a bijective map,
called GT-transformation, that sends solutions of  to solutions of . If a closed
form solution to one of the operators is known, then by means of this transformation the
solution of the second operator can be constructed. For this purpose a table with
second order operators including parameters together with characteristic data (local data)
has been constructed. This local data can be computed for the given operator, the
corresponding equivalent operator is found by table look-up. Then by comparing parameters
of the local data the GT-transformation can be constructed. The characteristic data is
described in Section~\ref{sec:ld}. To cover the extension described here the table has been extended 
so that we can give closed form solutions of certain third order linear difference operators.











\section{Preliminary}
In this section we introduce notations used in this paper and recall some known
facts~\cite{CH09, CHG10, AeqB, PS97} about difference operators. Additionally Cases~1
and~2 above are carried over to the difference case for algebraic extensions in
Theorem~\ref{thm:ord3dec} below.

\subsection{Ring of difference oprators}
Let  be the ring of linear difference operators with coefficients in , where  is the shift
operator acting on  by . Then  is a noncommutative ring where  For  with , we
say that  has order~ and write . If furthermore  then  is
said to be a {\em normal} operator.  In this paper we will assume all operators to be
normal.

The adjoint operator of  is defined by .
Suppose  for some .  Then , where  and thus right-hand side factors of 
correspond to left-hand side factors of~. We say that a third order operator  is
irreducible in  if both  and  have no first order right-hand side factor in .

A second order operator  is called a {\em
    full} operator if .  Thus, if  is a normal but not full
  operator, then .  





\subsection{Ring of sequences}

Let . Then an element 
  corresponds to a sequence .  is embedded in 
  as a subring via constant sequences.  Suppose , then we define

With the above termwise addition and multiplication,  forms a -algebra.  We
define the action of  on  by .

Let  where  if there exists  such
that, for all , .  Then it is easy to verify that  is a unit in
, i.e.  is invertible in , if and only if  has only finitely
many zeros.  If , then the image of  in  and the action of  on
 are well defined. This way we can embed  to  and call 
rational if there exist  and  such that  for all .   forms a ring of difference operators and  is embedded in
.


We say  for  if there is  such that


\begin{definition}
 is called hypergeometric if  is rational and 
 is called the certificate of 
\end{definition}
If  is hypergeometric then  where  is the certificate of~.
We define .

\begin{theorem}\cite[Theorem 8.2.1]{AeqB}
 for a normal difference
operator . 
\end{theorem}

Thus for a normal operator ,  forms a -vector space 
with a basis .






\subsection{Term equivalence}\label{sec:sp}

\begin{definition}\label{def:sp}
  The symmetric product, , of operators  and  is an
  order-minimal and monic operator such that  for all  and . 
\end{definition}

There is a simple formula if one of the operators has order~.  Let  and . Then

Thus,  for any .


Suppose  and . Then the above formula gives an operator  such that  
where  is a solution of . If  then it is easy to see that  is rational.


\begin{definition}
   are said to be term equivalent if there exists  such that
  , denoted by . Such a T is called the term
  transformation from  to .
\end{definition}

If  and  are term equivalent and  is the term transformation then
.  Suppose  and a term
transformation  are given, then  can be obtained by~\eqref{sym}.

\subsection{Gauge equivalence}\label{sec:ge}
Let  be two given operators, where a closed form solution~ of  is
known. If furthermore an operator  can be determined sucht that  is solution
of~, then a closed form solution of  can be written as a linear combination of
shifts of  over . Such a transformation~ is called a gauge transformation and
 and  are said to be gauge equivalent if such a transformation exists.




\begin{definition}\label{def:gt}
  Let  have the same order.  is called a {\em gauge
    transformation} from  to  iff  is a bijection.
\end{definition}

Note that  is not required to be a normal operator.  

Suppose we are given a gauge transformation  where . Then there exist 
with  such that~. The remainder  is
also a gauge transformation that acts in the same way as  on . Hence, 
w.l.o.g., we may assume that~.

Let  denote the greatest common right divisor of .  
Since  is a bijection, any non zero solution  of  does not map to zero by . Thus,  and  have no nontrivial common right hand factor, i.e. . 
Using the extended Euclidean algorithm  can be determined such
that . Then  is the identity on  and
 is an inverse of  that sends  bijectively.

\begin{definition}\label{def:ge}
  Two operators  and  with the same order are called gauge equivalent if there
  exists a gauge transformation  and we use the notation~.
\end{definition}


Suppose  where the gauge transformation from  to  is a single term operator,  for . 
Then  is term equivalent to  where the term transformation from  to  is .



\subsubsection{How to compute the gauge transformation}\label{hom}
Suppose  and  are gauge equivalent and  is a gauge transformation from 
to .  Then there is an operator  such that .

The algorithm that was used to find the gauge transformation in~\cite{CH09, CHG10, GH10}
works as follows:
\begin{enumerate}
\item For given operators  and , we set up the ansatz
  , where the  are undetermined
  coefficients.
\item\label{step3} Right divide  by  and set the remainder to zero. This way we obtain a
  system  of difference equations for the unknown coefficients .
\item  Compute the  rational  solutions of the  system    to determine  the  values for
  the~.
\end{enumerate}

This algorithm was efficient for second order operators, but for operators
  of order three and higher, computing a solution of the system , we get at
  Step~\ref{step3}, is very costly. Hence in the current implementation we use the new
algorithm HOM to compute the gauge transformations that give the set of homomorphisms
 in~ sending  to  for any~. This
means in particular that we can drop the condition on the orders,~.

In short, the algorithm HOM works as follows: For ,
, we define the -adjoint operator .  Then there is a one to one correspondence between 
and rational (invariant under the difference Galois group) elements of .  We define a space  that is isomorphic to
.  Then rational elements of 
correspond bijectively to elements of . Thus, we compute rational elements
of .  This is done by working directly with  and
, and we avoid computing large operators such as the symmetric product of 
and  (whose solution space is a homomorphic image of .)

Note that if  and  are of the same order, then HOM returns exactly the gauge
transformations. The algorithm HOM is available at
\url{http://www.risc.jku.at/people/ycha/Hom.txt} and more details can be found in \cite{TensorRatSol}. This is joint work of Yongjae Cha and Mark van Hoeij.

\subsection{GT-equivalence}

\begin{definition}
  Suppose there is a gauge transformation  and a term transformation  such that the
  composition of  and , , maps  to , i.e. 
   .  
  Then  and  are called GT-equivalent, denoted by~, and the composition of  and  is refered to as the GT-transformation from
   to .
\end{definition}



Suppose there is a map  which is a multiple composition of gauge transformations and term transformations. Then 
\cite[Theorem 3.3.]{Le10} shows that we can find a gauge transformation  and a term transformation  such that
.



\subsubsection{How to compute the GT-Transformation}

\begin{definition}\label{SNFdef}
  Let   be a subfield of  and , for some ,
  monic irreducible in , and let  equal
  the sum of the roots of .

   is said to be in {\em shift normal form} if , for . We denote by  the shift normalized form
  of , which is obtained by replacing each  by  for some  such that  is in shift normal form.
\end{definition}

 is unique up to the choice of . In the algorithm given in
Section~\ref{sec:algo} we assume .  For ,
we denote by  the determinant of the companion matrix of , which is
.

\begin{theorem}\cite[Theorem 2.3.9]{YC11}\label{tpsuff}
  Suppose  for  where  is a subfield of
  .  Then there exists a gauge transformation  from  to  for some  where 
\end{theorem}

The original statement of the above theorem uses , but the same proof works for any
subfield  of .

Suppose we know that  for  and we want to
find the GT-transformation.  By the above theorem there exists  such that
 where .  When  is even  or  can be gauge equivalent to .  Thus the
algorithm Hom will return a non-empty set for either of the two. Furthermore, when  is
odd,  is gauge equivalent to .





\subsection{Symmetric powers of operators}
Given an operator  that annihilates a function , then in order to obtain an
operator  that annihilates~ we need the symmetric square of~. In this
section we state some facts about these operators.

By  we denote the  symmetric power of , i.e.,we
define~ and .   is
called a symmetric square root of  if .

Suppose  is a differential operator of order 2 then it is known that the order of
 is ~\cite[Lemma 3.2, (b)]{MS85}.  However the following lemma
shows that this is not true for difference operators.

\begin{lemma}\cite[Lemma 3]{GH10}\label{lemma:sym-power2}
 Let . Then
\begin{enumerate}
 \item 
	if  then 
	 
        where
        
\item if  then .
\end{enumerate}
The formulas above give order-minimal operators for both cases.
\end{lemma}




If a full operator  is
a symmetric square root of a third order operator , then also  is a symmetric square root of~. If  is a solution
of , then  is a solution of~. We say  and  are conjugates if  where the term transformation is .



Solutions of an equation of type 2 are called Liouvillian solutions~\cite{CH09,HS99,GH10}.
Suppose  is a solution of  then , where , forms a basis of  and~.  Also, it is easy to verify that
for arbitrary orders~ it holds that 
with a similar proof to the one of Lemma~\ref{lemma:sym-power2}.






\begin{definition}
  A second order operator  is called a unity free operator if the solution space of
   does not admit a basis  such that  for some .
\end{definition}


Let .  Then a basis of the solution space of  is 
where  and  are solutions of  in .  Since  for ,  is not a unity free operator.



\begin{lemma}
\label{lm:unfr}
If  is an irreducible second order operator then it is a unity free operator.
\end{lemma}

\begin{proof}

  We prove this by contraposition.  Suppose  is not a unity free operator.
  Then we may assume  is monic and  admits a basis  such that
   for some .  Let  be the smallest integer that
  satisfies , then we may assume  for some  where  denotes th root of unity
  and  are pairwise relatively prime.  Thus
   where
  . Since  is an element in  and by equation~\eqref{sym},  is in
   and this implies  is reducible in D.
\end{proof}

\begin{lemma}
\label{lm:zd}
If  satisfies a full second order operator  then 
 is not a zero divisor in .
\end{lemma}

\begin{proof}
We will prove that  has only finitely many zeros.
Since  there is  such that 

and  has no poles or roots for all . Suppose 
for some . Then by~\eqref{eq:zd},  for all  and this
contradicts that . Suppose  for some .
Then again by~\eqref{eq:zd},  for all . Thus  for
 large enough and hence  is a unit.
\end{proof}

\begin{theorem}\label{thm:nonvan}

  If  for some irreducible full second order operator  then 

\end{theorem}
\begin{proof} 

   Let  be a basis of . We will show that then  are linearly independent.  Suppose there exist  in , not all zero
  such that .  
  
  By Lemma~\ref{lm:zd},  is not a unit and since  is irreducible operator, by
  Lemma~\ref{lm:unfr},  for any .  Let  and  then , i.e.  for
  all .  Thus,  for all  and
  . Suppose  is not a constant sequence. Since  is an irreducible full operator
  in , it contradicts that  is a solution of .  Suppose  is a constant
  sequence. Then it contradicts that  are  linearly independent.
\end{proof}




In the differential case it holds that if the symmetric square of a third order
differential operator  has order 5, then  for
some second order operator .  However, the following example
shows that this does not hold in the difference case.

\begin{example}
\label{example:symp}
Let . Then . A solution of  is  where  denotes the
modified Bessel function of the first kind.  Then the symmetric square roots of  are
 and
, which are not in~.  A solution of
 and  are  and , respectively.




Let . Then a solution of  is .  Then , but  and  are not gauge equivalent in~, i.e, there is no
operator in  that sends  to .
Since  is not a solution of 
any shift operator in , \cite[Theorem 5.2]{FlajoletGerholdSalvy05}and \cite[Lemma A.2]{Chen2012111},  is not a symmetric product of  and a difference
operator in~.


\end{example}





In the differential case, suppose , i.e, the
  solution of  can be obtained by multiplying a hyperexponential term  to the
  solutions of . Let  be a basis of the solution space of
  , then  admits a basis of the solution space .
  However, if  is hyperexponential then  is also hyperexponential. Thus,
   for  such that  where .  However if  is a
  hypergeometric term,  is not guaranteed to be a solution of an operator in .




\begin{definition}\label{def:gauge-equ}
  An irreducible operator  is said to be {\emph solvable in terms of second order in } if it is
  GT-equivalent to  where the 's
  are irreducible and full second order operators in .
\end{definition}

We need the following Lemma to prove Theorem~\ref{thm:ord3dec}.  


\begin{lemma}\label{symbs}
  Let  be full second order operators. 
If  then 
we can choose a basis  of , and a basis  of , such that .

\end{lemma}


\begin{proof}
  Let  be a basis of  and  be a basis of .
  Since , the -vector space generated by  has dimension 3. Then there exists , which are
  not all zero, such that  Suppose 
  and  then it contradicts that  and  are linearly independent.
  Likewise, if  and  then it contradicts that  and  are
  linearly independent.  If  and  then we have the desired form.
  So, the remaining cases are either only one of the coefficients  is zero,
  or all  are non-zero.  Here, we will prove the case when  is the
  only zero coefficient.  Let  be another basis of 
  such that

Then for  we have  as claimed.
\end{proof}






\begin{theorem}\label{thm:ord3dec}
  Let  be an operator of order 3, irreducible and solvable in terms of second order in~.
  Then  for some irreducible full second order operator 
  and furthermore
 

\begin{enumerate}[(a)]



\item\label{c1}   then .

 
\item\label{c2} if the gauge transformation of  is not a
  single term operator then .
\end{enumerate}

\end{theorem}

\begin{proof} Let  be a third order, irreducible operator that is
    solvable in terms of second order in~.  Then by definition (and the restriction of
    the order), there exist two irreducible full second order operators 
    such that . By Lemma~\ref{symbs}, a suitable basis 
    for , and a suitable basis  for  can be chosen, such
    that~.  Let , then  and , and
    .  Since  is a basis for an operator in ,  is
    hypergeometric and this implies that  with term transformation
    , where  is the certificate of the hypergeometric term~. Summarizing, by
    Lemma~\ref{lemma:sym-power2},  for some full operator .

  
  (a) Let  be a basis of  and  be the term transformation from  to  and  be a solution of . Then
   forms a basis of  and thus  by Lemma~\ref{thm:nonvan}.
  
 


  (b) Let  be a basis of . Then  forms a basis of , where  is a non
  single term operator and  is a hypergeometric term.  Then  and this implies .
 \end{proof}



 Suppose  is a differential operator of order 3 and  for some
 second order differential operator . Then it is well known that there exists a
 formula to construct this~, see~\cite[Lemma 3.4]{MS85}. The case where only
 gauge-equivalence holds, i.e., , is more interesting.
 In~\cite{vH07} third order operators are treated with a focus on determining both 
 and a gauge transformation.


 It is possible to implement a similar algorithm for difference equations which returns
 the second order operator  to which the given  can be reduced to and a gauge
 transformation.  However, in the difference case, in order to give a closed form solution
 of  other algorithms need to be applied or a table look-up. Also, even if we are in
 case~\eqref{c1}, finding  is not as simple as in the differential case, in particular
 if there is a parameter included in the input.  Morever to distinguish the cases, the
 symmetric square of a third order operator needs to be computed which can become costly
 if many parameters are involved.














\section{Local data}\label{sec:ld}

The local data that we are using are the valuation growths at finite singularities in
 and generalized exponents at the point of infinity.  This data is invariant under
GT transformations.  In this section, we review the definition and an invariance
property (Theorem~\ref{gp}, Theorem~\ref{genexp}) of local data from~\cite{YC11, CH09,
  CHG10, HO99}. We omit proofs in this paper.


\subsection{Finite singularities}
Valuation growth was first introduced in~\cite{HO99} and an algorithm to compute it was
given in the same paper. Let . After multiplying
 from the left by a suitable element of , we may assume that the
 are in  and gcd. Then  is called a {\em
   problem point} of  if  is a root of the polynomial  and  is called a {\em finite singularity} of  if  has a problem point in 
 (i.e.  for some problem point ).  Let .  For  we say  iff  is a positive integer.

 Let  be a new indeterminant, i.e., transcendental over .  We define
  which is obtained by substituting
  with  in . The map  defines an embedding
 (as non-commutative rings) of  in . Hence, if
 , then~.





\begin{definition} Let  and  be the ring of 
formal power series over  in . The -valuation
   of  at  is the element of 
  defined as follows: if  then  is the largest integer  such that , and
  . \end{definition}



We define an  dimensional -vector space 

Let  be the smallest root of  in , so
 is the smallest problem point
in . Likewise we define  to be the largest root of  in . If  is not a singularity,
that is, if  and  have no roots in ,
then choose two arbitrary elements in  and define  to be those two elements.


\begin{definition} \label{box} For non-zero  and for  if , where , we define the {\em box-valuation}
 \end{definition}

\begin{lemma}
    \label{vl}
With  chosen as above, we have

. \end{lemma}


We define  as  which, by Lemma~\ref{vl}, equals the box valuation of any box on the left of
. Likewise we define  as .

\begin{definition}
\label{def:valg}
 Define the {\em valuation growth} of non-zero  as

Define the {\em set of valuation growths} of  at  as
 \end{definition}
If  is a first operator operator then  has only one element. 



\begin{definition}
    \label{apartsing}
    Let  be a difference operator and  be a finite singularity of . If
     has more than one element then  is called an {\em essential
      singularity}. 
\end{definition}

The algorithm given in~\cite{HO99} determines the set 
 



\begin{theorem}
    \label{gp}\cite[Theorem 1]{CH09}
If  and  are gauge equivalent then 

for all .  
\end{theorem}



The following lemma is an immediate consequence of Definition~\ref{def:valg}.
\begin{lemma}
For each , 

\end{lemma}

The above theorem only gives invariance under gauge equivalence. 
To have invariance under GT-equivalence, we need to define one more set. 
Suppose , then  for some . 
Then 
where , . So  is invariant under GT-equivalence.
Thus, for a difference operator , we define a set of ordered pairs



\subsection{Singularity at infinity}

Let  be the field of formal Laurent series and  for . We define the valuation
for  as the smallest power of  whose coefficient is non-zero and denote it
by~.  This definition can be extended to , where
 denotes the forward difference, by setting

for any operator~.

\begin{lemma}
\label{ind}
Let .  There exists a polynomial  such that
for every  we have

where the dots refer to terms of valuation .
\end{lemma}


\begin{definition}
	, the {\em indicial polynomial} of , is the polynomial  in Lemma~\ref{ind}~\eqref{tn}.
\end{definition}

Lemma 9.2 in \cite{CHG10} states that if  is a root of  then there is  such that  and .  However, there is no one-to-one correspondence
between solutions of  in  and integer roots of
. For this matter, we introduce the ring , where  is a solution of , see~\cite{LF99} for existence of .  We extend valuation on  to  by: for
, , , and , we
let~.  With this notion we obtain the following theorem which is equivalent
to~\cite[Theorem 3.2.10]{YC11} and \cite[Lemma 6.1]{PS97}.

\begin{theorem}
  is a solution of  if and only if  has a solution  with . 
\end{theorem}

An immediate consequence of the above theorem is the following corollary.

\begin{corollary}
\label{indsym}
If  and  are the solutions of the indicial equations of  and ,
respectively, then  is a solution of the indicial equation of~.
\end{corollary}






Define the action of  on  as:


Since we have defined the action of  on , we can now apply the formula for the
term symmetric product in~\eqref{sym} to .  Let  and  be the
following subset and subgroup, respectively, of :



Now  is a set of representatives for . The composition of the
natural maps  defines a natural map

Let



\begin{definition}\label{def:req}
Let  then for , we say  is -equivalent to , , when .
\end{definition}


Note that  if and only if  with  as
in the definition of ,  for , and  matching as well.


\begin{definition}
  Let  for some . We say that  is a {\em generalized exponent} of
   with multiplicity  if and only if zero is a root of  with
  multiplicity m where . We denote by 
  the set of generalized exponents of~.
\end{definition}

Suppose  then .



\begin{theorem}
\label{genexp}
If two operators  and  are gauge equivalent then for each  there is a  such that  is equivalent to .
\end{theorem}
This theorem has been proven first in~\cite{CHG10}. An alternative proof can be found in~\cite{YC11}.




\begin{theorem}\label{symgen}
Suppose  then 

\end{theorem}

\begin{proof}
   and since 0 is a solution of  and , 0 is also a solution of
  the indicial equation of  by
  Lemma~\ref{indsym}
\end{proof}



Likewise for the valuation growth, we need to define one more set to have invariance for GT-equivalence.
Suppose  for some . Then 



Thus we define the following set,

and then  if .



\section{Table of base equations}



In \cite{YC11, CHG10}, we have formed a table of base equations of order 2, call it TB, as follows;
\begin{itemize}
\item collect equations with known solution from \cite{abst, AG76}.
\item for any closed form expression that shows up often in the literature, generate a base equation with existing algorithms \cite{ChyzakDM, KoutschHF}.
\end{itemize}
For the algorithm given in Section~\ref{sec:algo}, we have computed symmetric squares of each base equation in TB yielding an entry in TB2 of a base equations of third orders.  Moreover we have generated further base equations as follows:

Suppose  is a solution of an operator .
Then  is a solution of the operator

As input for our algorithm we accept only operators of order three and the above equation
may be of higher order. One way of obtaining the base equation for  in this case
is using  when it is a multiple of an operator  for which~.
Since  as constructed above is not guaranteed to be the minimial order operator
we compute~. If the algorithm HOM returns the identity map this means
that  is in fact order-minimal. These cases are neglected and
we use  as a base equation only if HOM returns a non-trivial map.

For instance for the squared hypergeometric function in the table below,
, an annihilating
operator  can be obtained starting from an operator  annihilating
 using~\eqref{eq:Tm}.  Then
the order-minimality of  is checked with the algorithm HOM. In this case HOM returns
a non-identity map and hence we save  in the table.



If , then , where  is a root of
 and .  Thus, we can detect whether an
input operator may have a solution  if a base equation for  is in our table.
However, it is more efficient to compute the base equation for small values of~.




\subsection{Example of base equations}
\label{sec:base}

Here we list a small part of the table which is needed in Section~\ref{sec:algo} and \ref{apps}.  In the
following table they are listed under (a) a solution (b) the corresponding , and (c) the . 
The full table can be found at
\url{http://www.risc.jku.at/people/ycha/TB2.txt}.



\begin{enumerate}


\item\label{ex2} 	
	\begin{enumerate}
	  \item 
	  \item 
	  \item 
	\end{enumerate}

\item 
	\begin{enumerate}\label{Legendre}
	  \item  (Legendre polynomials squared)
	  \item 
	  \item 
	\end{enumerate}

\item
	\begin{enumerate}
	  \item  (Hermite polynomials squared)
	  \item 
	  \item 
	\end{enumerate}
\end{enumerate}

\section{Algorithm}\label{sec:algo}

The basic structure of the algorithm is the same that was given in \cite{YC11}. Here we use an extended table of base equations and a more efficient algorithm for computing
the gauge transformation, as mentioned in Section~\ref{hom}.

Suppose  is the input operator with local data

for .
By comparing the corresponding data in TB2, we can find that local data of  matches with the data of \eqref{Legendre} in Section~\ref{sec:base}.
Let  be the operator of which  is a solution. To compute the parameter , we compare  with each entry of  
and compute the set of candidates of possible values for  which is, 

Substituting  by each of the values of the above set, a set of equations  is obtained. 
It remains to cheek for each of the equations in  whether there is a GT-transformation to  and 
if so then we return the closed form solution by applying the GT-transformation to 
. 
\\

\noindent {\bf Algorithm {\em solver2}}\\
\noindent {\bf Input}: A third order normal operator .\\
\noindent {\bf Output}: Either at least one closed form solution of  in the form of  where 
 are hypergeometric terms and  is a solution in TB2 or otherwise the empty set.

\begin{enumerate}
\item  .
\item Find the base equations in TB2 by comparing  and  with the corresponding data in the table.
\begin{enumerate}
\item if there is no match then return `Not solvable within the Table'.
\item if there is a matching equation , .
\end{enumerate}
\item\label{four} For each , compute candidate values for the parameters using  and .
\item Construct a set  by substituting parameters by the values determined in Step.\ref{four}
\item For each  check if there exists a GT-Transformation from  to .
\begin{enumerate}
\item if there is a GT-transformation then apply  to the known solution of  and return the solution.
\item if there is no GT-transformation found return `Not solvable within the Table'. 
\end{enumerate}

\end{enumerate}






\section{Improvement}
A similar approach can be applied to higher order operators that are solvable in terms
of order two. Suppose  is a fourth order operator that is solvable in terms of order 
two in~. Then  is equal or gauge equivalent to either  or 
for some second order operators  with nonvanishing coefficients. The candidates
for  can be detected analogously using Theorem~\ref{symgen}.

Concerning the applications to proving positivity of special functions inequalities it has
to be noted that representations in terms of finite linear combination of squares with non-negative
coefficients need not exist on the full range of validity of a given inequality, as can be
seen below. Further investigations of the applicability of this approach as well as an
implementation of the above mentioned extension to higher order recurrences are ongoing
work.

\section{Applications}
\label{apps}
Our main motivation to extend finding closed form solutions of difference equations in
terms of symmetric products is to develop an algorithmic approach for proving special
functions inequalities. Existing symbolic methods~\cite{GKIneq,MKSumCracker,MKVP10} are
based on using Cylindrical Algebraic Decomposition (CAD) which in several examples has
proven to be an effective way for proving positivity of sequences that are given only in
terms of their defining sequences. However, it is sometimes unsatisfiable to have a proof
that only comes with ``True'' without any certificate. Some classical proofs of
inequalities are using rewriting of the given expression as linear combination of easy to
verify positive objects such as sums of squares. The present work tries to make this
approach algorithmic. Certainly it will not provide answers for any special functions
inequality, but it is a first step in a new direction of automatic inequality proving.
Below we give two examples, one for each of the cases distinguished above, of classical
problems that can be solved fully or at least partially using the presented algorithm.
Note that all of these identities stated can be proven easily using existing algorithms
for symbolic summation. The novelty is the automatic discovery of certain closed form
expressions for sequences that are given only in terms of their defining recurrence
relation. In this sense it is comparable to the above mentioned algorithms based on CAD.

\subsection{Clausen's formula} \label{Clausen} 

Proofs of special function inequalities often depend on a variety of classical techniques
such as argument transformations, integral representations of hypergeometric series and
many more. For instance in the proof of the Askey-Gasper inequality~\cite{AG76},which
played a key role in the proof of the Bieberbach conjecture by
de~Branges~\cite{deBranges}, Clausen's formula

entered at a central point. Zeilberger~\cite{SBE93} has shown how this identity can be
proven using symbolic summation. By means of the algorithm presented here, Clausen's formula
can be discovered entirely automatic.

The hypergeometric function in~\eqref{cl} satisfies a third order recurrence that is given 
by the operator~ and is too large to be displayed here. It can however be found easily
common symbolic summation algorithms~\cite{Zeil90a, ChyzakDM, KoutschHF}. This difference operator
is the input for our procedure and we start by determining the local data given by

A table look-up shows that this local data is compatible with~\ref{ex2} in Section~\ref{sec:base}. Comparing local data and solving the system mod  
the following candidates for  and  can be found:

There is no term transformation for these operators and an application of HOM shows that we obtain a constant map if ,  and~.

\noindent 



\subsection{Tur\'{a}n inequality for Hermite polynomials}
\label{turan}
The positivity of Tur\'{a}n determinants has been proven for many different families of
orthogonal polynomials. The first Tur\'{a}n inequality was formulated for Legendre
polynomials~\cite{TP50} and Szeg\"o~\cite{SG48} has given four different proofs of this
inequality.  Szwarc~\cite{Szwarc} has provided a more general approach for proving
Tur\'{a}n type inequalities based on the mere knowledge of the recurrence coefficients
satisfied by the given sequence. Gerhold and Kauers~\cite{MKTuran} have proven and
improved this type of inequalities using their CAD-based method. The approach presented
here does not give a full proof for Tur\'{a}n type inequalities, however it gives a
representation of the given determinant in sums of squares derived from the third order
annihilating operator of the determinant. In the case of Hermite polynomials this yields a
representation that gives positivity in the limit for  tending to infinity.

Tur\'{a}n's inequality for Hermite polynomials  reads as follows:

Then an annihilating operator of 
is  and the local
data of this operator is



 are elements in  and
these are equivalent to  under , see Definition~\ref{def:req} for . 
Thus the local data of  correspond to those of the third entry of the table given in Section~\ref{sec:base}.


Using the algorithm described above a gauge transformation can be found that applied to 
 yields the following equivalent formulation

This representation gives the positivity of Tur\'{a}n's inequality for



\bibliographystyle{plain}
\bibliography{myrefs}

\end{document}
