\documentclass[twocolumn, 10pt]{svjour3}         \smartqed  \usepackage{graphicx}
\usepackage{bm}\usepackage{amsmath}
\usepackage{amssymb}
\usepackage{latexsym}
\usepackage{epsfig}
\usepackage{amsbsy}
\usepackage{array}
\usepackage{amssymb}
\usepackage{setspace}
\usepackage[caption=false,font=footnotesize]{subfig}
\usepackage{algorithm}
\usepackage{algorithmic}
\usepackage{cite}
\usepackage{setspace}
\doublespacing
\DeclareMathOperator*{\E}{\mathbb{E}}
\DeclareMathOperator{\arcsec}{arcsec}








\journalname{Wireless Networks}

\begin{document}

\title{Capacity and Delay-Throughput Tradeoff in ICMNs with Poisson Meeting Process}



\author{Yin~Chen \and Yulong~Shen \and Jinxiao~Zhu \and Xiaohong~Jiang}

\institute{Y.~Chen, J.~Zhu and X.~Jiang \at School of Systems Information Science, Future University Hakodate, Japan.\\
\email{ychen1986@gmail.com, jxzhu1986@gmail.com and jiang@fun.ac.jp}.   
\and
           Y.~Shen \at School of Computer Science and Technology, Xidian University,  China.\\
           \email{ylshen@mail.xidian.edu.cn}. \\
}







\date{Received: date / Accepted: date}



\maketitle

\begin{abstract}




Intermittently connected mobile networks \\label{eqn:beta_estimate}
	\beta_{\text{RW}} \approx \frac{ 2 c_1 \, d \E[V^*]}{L^2},~~\mbox{and}~~
	\beta_\text{RD} \approx \frac{2 d \E[V^*]}{L^2},
\label{eqn:capacity}
	\mu = \frac{n}{4} \beta.

	\mu = \frac{n}{4} \beta.
\label{eqn:stability}
	\lambda n - \epsilon \leq \frac{1}{T} \sum_{h=1}^{\infty} X_h(T),
\label{eqn:tran_upp_bound}
	\sum_{h=1}^{\infty} h X_h(T) \leq Y(T).

\frac{1}{T} Y(T) &\geq  \frac{1}{T} X_1 (T) +  \frac{2}{T} \sum_{h=2}^{\infty} X_h(T)\nonumber\\&\geq  \frac{1}{T} X_1 (T) + 2\left[(\lambda n - \epsilon) - \frac{1}{T} X_1 (T) \right],
\label{eqn:bound}
\lambda \leq \frac{1}{2n}\left [\frac{1}{T} Y(T) + \frac{1}{T} X_1 (T) + 2\epsilon \right].

\frac{1}{T}Y(T)  \xrightarrow{\text{a.s.}} \frac{(n-1) n}{2} \beta   \label{eqn:ex_no_tr}.

\frac{1}{T}Y_{sd}(T)  \xrightarrow{\text{a.s.}} \frac{n}{2}  \beta \label{eqn:ex_no_sdtr}.

\lambda \leq \frac{n}{4} \beta + \frac{\epsilon}{n}, \text{  as } T \to \infty.
\label{eqn:delay}
	\E\{D\} = \frac{n-1}{\mu-\lambda},

\mu &= \beta /2 + \beta (n-2) /4 \label{eqn:service_rate}\\
	&= \frac{n}{4}\beta,

	\E\{D_s\} = \frac{1}{\mu-\lambda}.

	\E\{D_r\} = \frac{1}{\mu'-\lambda/n}.

	\E\{D\} = \E\{D_s\} + \frac{n-2}{n} \E\{D_r\} = \frac{n-1}{\mu-\lambda},

	\frac{\E\{D\}}{\lambda} \geq \frac{1- \log(2) }{2 (n-1) \beta^2}.
\label{eqn:total_delay_ICN}
	\E\{D\} = \frac{1}{n} \sum_{i=1}^{n} \E\{D_i\}.

 \lambda n \cdot \frac{1}{n} \sum_{i=1}^{n} \E\{R_i\}	= \lambda \sum_{i}^{n} \E\{R_i\}.
\label{eqn:rate_redun_ICN}
	\lambda \sum_{i=1}^{n} \E\{R_i\} \leq \binom{n}{2} \beta = \frac{(n-1)n}{2} \beta.

	\E\{D_i\} &= \E  \left\{ D_i | R_i \leq 2 \E  \left\{ R_i \right \} \right\} \Pr \left \{R_i \leq 2 \E\{R_i\}   \right \} \nonumber\\
	&~~~+\E \{D_i | R_i > 2 \E\{R_i\} \} \Pr \{R_i > 2 \E\{R_i\}  \} \nonumber\\
	&\geq \E \{D_i | R_i \leq 2 \E\{R_i\} \} \Pr \{R_i \leq 2 \E\{R_i\}  \} \nonumber\\
	&\geq  \frac{1}{2} \E \{D_i | R_i \leq 2 \E\{R_i\} \},\label{eqn:one_half}
\label{eqn:mini_ineq}
	\E \{D_i | R_i \leq 2 \E\{R_i\} \} \geq \underset{\Theta}{\inf}  \E\{D_i^* | \Theta \},

	\underset{\Theta}{\inf}  \E\{D_i^* | \Theta \} &= \E\{D_i^* | D_i^* \leq \omega \}\nonumber\\
	&= \frac{\E\{D_i^*\} -\E\{D_i^* | D_i^* > \omega \}\Pr\{ D_i^* > \omega \} }{\Pr\{ D_i^* \leq \omega \}}\nonumber\\
	&=\frac{\frac{1}{2 \E\{R_i\} \beta} - \frac{1}{2}(\omega + \frac{1}{2 \E\{R_i\} \beta})}{1/2}\nonumber\\
	&= \frac{1-\log (2)}{{2 \E\{R_i\} \beta}}.\label{eqn:mini_R_star}

	\E\{D\} &\geq \frac{1- \log(2)}{4 \beta} \cdot \frac{1}{n} \sum_{i=1}^{n}  \frac{1}{\E\{ R_i \}}\label{eqn:jen1_ICN}\\
	&\geq  \frac{1- \log(2)}{4 \beta} \cdot \frac{1}{\frac{1}{n} \sum_{i=1}^{n}   {\E\{ R_i \}}},\label{eqn:jen2_ICN}

	\E\{D\} &\geq    \frac{1- \log(2)}{4 \beta} \cdot \frac{2 \lambda}{(n-1)\beta} =\frac{1- \log(2) }{2 (n-1) \beta^2} \cdot \lambda .\label{eqn:delay_bound_ICN}
\label{eqn:appr_capa}
	\mu_{\text{RW}} \approx \frac{ c_1 n d   \E[V^*]}{ 2 L^2}~~\mbox{and},~~
	\mu_\text{RD} \approx \frac{ n d   \E[V^*]}{2 L^2},

	\frac{\E\{D\}}{\lambda} \geq \frac{  (1-\log (2) )L^4}{8(n-1) (c_1 d  \E [V^*])^2   },

	\frac{\E\{D\}}{\lambda} \geq \frac{   (1-\log (2) ) L^4}{8(n-1) (  d  \E [V^*])^2   },  

	\frac{\E\{D\}}{\lambda} \geq \frac{  (1-\log (2) ) \pi^2 L^4}{128 (n-1) (c_1 d  v)^2   },\label{eqn:tradeoff_rw}

	\frac{\E\{D\}}{\lambda} \geq \frac{ (1-\log (2) ) L^4}{128 (n-1) ( d  v)^2   }.\label{eqn:tradeoff_rd}
 
\end{enumerate}



\begin{remark}\label{remark:corollary_1}
Notice that for both the random waypoint and random direction mobility models, if we consider that the    and  increase while  the node density  remains constant, then we have the following observations:
\begin{itemize}
	\item The results of~(\ref{eqn:appr_capa}) reduce to  and , indicating that a constant throughput capacity is still achievable in  a large scale ICMN.
Meanwhile, the result in~(\ref{eqn:delay}) indicates that the average end-to-end delay under Algorithm~\ref{alg:routing} will increase linearly with the number of nodes .
	\item The results in~(\ref{eqn:tradeoff_rw}) and~(\ref{eqn:tradeoff_rd})  indicate that the delay-throughput scales as .
\end{itemize}


\end{remark}



\section{Simulation and Numerical Results}\label{sec:numerical}
In this section, we first provide simulation results to validate the efficiency of the theoretical results developed in Section~\ref{sec:capacity}, and then apply these results to illustrate the performance of the concerned ICMNs under different settings of system parameters.
\subsection{Model Validation}
To validate the efficiency of our analytical results, we provide simulation results under the random waypoint and  the random direction mobility models in this section.
The simulation results were obtained from a self-developed  discrete event simulator that implements the packet delivery process under Algorithm~\ref{alg:routing} and accepts  mobility traces generated by the NS- code of the random waypoint and random direction mobility models as input.
\subsubsection{Mobility Models}\label{sec:numerical:mobility}
The mobility models considered in the simulation are summarized as follows.
\begin{itemize}


\item  Random waypoint mobility model~\cite{Groenevelt2005}:
Under this model, initially network nodes are uniformly distributed in the network area and each node travels at a  travel speed randomly and uniformly selected in  with  towards a destination randomly and uniformly selected in the network area. 
After arriving at the destination, the node may pause for a random amount of time and then chooses a new destination and a new travel speed, independently of previous ones.
It is notable that the  locations of the nodes in steady-state under the random waypoint model are not uniformly distributed.
Particularly, it was reported in~\cite{Bettstetter2002Proc.WMAN} that the stationary distribution of the location of a node  is more concentrated near the center of the network region.

\item  Random direction mobility model~\cite{Groenevelt2005}: Under this mobility model, initially network nodes are uniformly distributed in the network area and each node randomly selects a direction, a speed and a finite traveling time.
The node travels towards the direction at the given speed for the given duration of time. 
When the travel time duration has expired,  the node could pause for a random time, after which it selects a new set of direction, speed and time duration, independently of all previous ones.
When the node reaches a boundary, it is either reflected (i.e., it is bounced back to the network
area with the angle of  or ) or the area wraps around so that it appears on the other side.
It was shown in~\cite{Nain2005Proc.IEEEINFOCOM} that the stationary distribution of locations is uniformly distributed for arbitrary distributions of direction, speed and travel time duration, irrespective of the boundaries being reflecting or wrapped around.

\end{itemize}
\subsubsection{Simulation Setting}
In our simulation, we consider a square network  of side-length  m and  number of nodes .
The travel speed is constant and equals to  ms.
There is no pause time.
We consider transmission distances of , where according to~(\ref{eqn:beta_estimate}) the corresponding pairwise meeting rates are determined as  for the random waypoint mobility model and  for the random direction mobility model. 
For the simulation measurements of the throughput and average end-to-end delay under Algorithm~\ref{alg:routing}, we focus on a specific traffic flow and measure its throughput and average packet delay over a long time period
of  seconds for each system load .

\subsubsection{Simulation Results}
\begin{figure}
	\centering
	\subfloat[Random waypoint model.]{\includegraphics[width=3.0in]{throughput_vs_load.eps}\label{fig:throughput_vs_load_RW}}\\
	\subfloat[Random direction model.]{\includegraphics[width=3.0in]{throughput_vs_load_RD.eps}\label{fig:throughput_vs_load_RD}}
		\caption{Throughput vs. system load .}\label{fig:throughput_vs_load}
\end{figure}
\begin{figure}
	\centering
		\subfloat[Random waypoint model.]{\includegraphics[width=3.0in]{delay_vs_load.eps}\label{fig:delay_vs_load_RW}}\\
		\subfloat[Random direction model.]{\includegraphics[width=3.0in]{delay_vs_load_RD.eps}\label{fig:delay_vs_load_RD}}
		\caption{Average end-to-end delay vs. system load .}\label{fig:delay_vs_load}
\end{figure}
To validate the efficiency of the  developed throughput capacity model, we summarize in Fig.~\ref{fig:throughput_vs_load} the simulation results of throughput for different values of system load.
In Fig.~\ref{fig:throughput_vs_load}, the dots represent  the  simulation results and the dashed lines are the corresponding theoretical throughput capacities calculated by~(\ref{eqn:appr_capa}).
We can observe from Fig.~\ref{fig:throughput_vs_load} that for both the random waypoint and random direction mobility models, the throughput  increases linearly as  increases from  to  and approaches   when  grows further beyond .
This is expected since the queuing system in the network is underloaded when ,
and it saturates as  approaches  and beyond.
The results in Fig.~\ref{fig:throughput_vs_load} indicate clearly that our theoretical throughput capacity result developed based on the Poisson meeting process can accurately predict the throughput capacity for the concerned ICMNs with the random waypoint or random direction mobility model.
Moreover, it also indicates that this throughput capacity can be achieved by adopting Algorithm~\ref{alg:routing} as routing algorithm in the network.







We then proceed to  validate the efficiency of our end-to-end delay model.
Particularly, we compare in Fig.~\ref{fig:delay_vs_load} the simulation results of the average end-to-end packet delay to those of theoretical ones calculated by substituting the results in~(\ref{eqn:appr_capa}) into~(\ref{eqn:delay}).
We can see from Fig.~\ref{fig:delay_vs_load} that for both the considered mobility models, the theoretical results nicely agree with the simulation ones. 
This observation indicates that our delay model of~(\ref{eqn:delay}) is accurate and can efficiently capture the delay behavior under Algorithm~\ref{alg:routing} in the considered network.






\subsection{Numerical Results and Discussions}


\begin{figure}
	\centering
		\includegraphics[width=3.0in]{mu_vs_speed.eps}
		\caption{Capacity  vs. average speed .}
	\label{fig:mu_vs_speed}
\end{figure}
\begin{figure}
	\centering
		\includegraphics[width=3.0in]{delay_vs_speed.eps}
		\caption{Average end-to-end delay    vs. average speed .}
	\label{fig:delay_vs_speed}
\end{figure}
\begin{figure}
	\centering
		\includegraphics[width=3.0in]{mu_vs_d.eps}
		\caption{Capacity  vs. transmission distance .}
	\label{fig:mu_vs_d}
\end{figure}
\begin{figure}
	\centering
		\includegraphics[width=3.0in]{delay_vs_d.eps}
		\caption{Average end-to-end delay  vs. transmission distance .}
	\label{fig:delay_vs_d}
\end{figure}


Based on our theoretical models, we first explore the impact of nodel traveling speed on the throughput capacity and end-to-end delay. 
We summarize in Fig.~\ref{fig:mu_vs_speed}  how the  varies with average pairwise relative speed   in a network of  m and  m.
Fig.~\ref{fig:mu_vs_speed}  shows that as the  increases, the throughput capacities under  both the random waypoint and random direction models increase linearly.
This is mainly due to that a higher average travel speed will lead to an increase on the pairwise meeting rate as shown in~(\ref{eqn:beta_estimate}), and hence to a higher throughput capacity.
For the same network setting, we then present in Fig.~\ref{fig:delay_vs_speed} how the average delay  under Algorithm~\ref{alg:routing} varies with  under system load .
It can be observed in Fig.~\ref{fig:delay_vs_speed} that increasing  will cause a lower average delay, which is because the  is inverse proportional to the throughput capacity  as indicated in~(\ref{eqn:delay}).


We then present in Fig.~\ref{fig:mu_vs_d} and~\ref{fig:delay_vs_d} how the throughput capacity  and average end-to-end packet delay vary with transmission distance  for a network of  m/s,  m and  (for delay).
It can be seen from in Figs.~\ref{fig:mu_vs_d} and~\ref{fig:delay_vs_d} that the impacts  of the transmission distance  on the behavior of capacity and delay are similar to those of the , for the reason that as shown in~(\ref{eqn:beta_estimate}),   is also  a factor in the evaluation of .


It is also interesting to see that from Figs.~\ref{fig:mu_vs_speed}-\ref{fig:delay_vs_d} that the random waypoint mobility model provides a performance better than that of the random direction mobility model for the network settings here.
Recall that compared with the random direction model that has a uniform stationary distribution of  nodes location, the stationary distribution of the location of a node under the random waypoint mobility model is more concentrated near the center of the network region (see Section~\ref{sec:numerical:mobility}).
Therefore,  the random waypoint mobility model leads to a higher nodel pairwise meeting rate (see~(\ref{eqn:beta_estimate})) and hence a higher throughput capacity, for the same network setting of ,   and .



















\section{Conclusions}\label{sec:conclusion}
This paper studied the  throughput capacity and delay-throughput tradeoff in an ICMN with Poisson meeting process. Based on the pairwise meeting rate in the concerned ICMN, an exact expression of the throughput capacity is derived, which indicates the maximum throughput that the network can stably support.
To reveal the inherent relationship between the end-to-end packet delay and achievable throughput, a necessary condition on the delay-throughput tradeoff is also established.
To illustrate the applicability of these theoretical results developed based on the Poisson meeting process, we conducted parameter-matching to fit the random waypoint and random direction models to the Poisson meeting process and obtained approximations to the throughput capacity and delay-throughput tradeoff with these mobility models.
Simulation result demonstrates that the throughput capacity developed based on the Poisson meeting process can serve as a good approximation to that under the random waypoint or random direction mobility models.
It is expected that the theoretical analysis developed in this paper will be also helpful for exploring the throughput capacity and delay-throughput tradeoff in ICMNs under other types of mobility models as well. 
Remark~\ref{remark:corollary_1} indicates that under the random waypoint or random direction mobility, a constant  throughput capacity is achievable even in a large scale ICMN as far as the node density can be kept constant, but at the cost of a linearly increasing expected end-to-end delay. Our results also reveal that by increasing the average node traveling speed or transmission range in an ICMN, an improvement on both its throughput and end-to-end delay performance might be expected.



\bibliographystyle{ieeetran}
\bibliography{IEEEabrv,ref}



\end{document}
