\documentclass[a4paper,11pt]{article}
 
\usepackage{a4fullpage}
\usepackage{algorithm}
\usepackage{algorithmicx}
\usepackage[noend]{algpseudocode}

\usepackage{amssymb, amsmath, mathabx, subfigure,url}
\usepackage[pdftex]{graphicx,color}
\usepackage{picinpar}
\usepackage{picinpar}
\usepackage{xspace}
\usepackage{bbbl}
\usepackage[mathscr]{euscript}

\usepackage[newitem,neverdecrease]{paralist}
\setlength{\pltopsep}{1.0ex}
\setlength{\plitemsep}{0.3ex}
\setlength{\plparsep}{0.3ex}


\newtheorem{definition}{Definition}
\newtheorem{lemma}{Lemma}
\newtheorem{theorem}{Theorem}
\newtheorem{observation}{Observation}
\newtheorem{corollary}{Corollary}


\newcommand{\Oh}[1]{\ensuremath{\mathcal{O}\left(#1\right)}\xspace}
\newcommand{\statement}[2]{\vskip2ex\noindent\textbf{#1~\ref{#2}.}}




\newenvironment{proof}{\textbf{Proof:}}{\hspace*{0mm}\hfill\ensuremath{\Box}}

\usepackage[newitem,neverdecrease]{paralist}

\begin{document}

\title{Approximating the Integral Fr\'{e}chet Distance}

\author{\begin{tabular}{ c c c }
  Anil Maheshwari & J\"{o}rg-R\"{u}diger Sack & Christian Scheffer \\
  School of Computer Science & School of Computer Science & Department of Computer Science \\
  Carleton University & Carleton University & TU Braunschweig\\
  Ottawa, Canada K1S5B6 & Ottawa, Canada K1S5B6 & M\"{u}hlenpfordtstr. 23,\\
  && 38106 Braunschweig, Germany \\
  \texttt{anil@scs.calreton.cs} & \texttt{sack@scs.carleton.ca} & \texttt{scheffer@ibr.cs.tu-bs.de}
\end{tabular}
  }

\date{}
\maketitle



\begin{abstract}
  	A pseudo-polynomial time -approximation algorithm is presented for computing the integral and average Fr\'{e}chet distance between two given polygonal curves  and . In particular, the running time is upper-bounded by  where  is the complexity of  and~ and  is the maximal ratio of the lengths of any pair of segments from  and~. The Fr\'{e}chet distance captures the minimal cost of a continuous deformation of  into  and vice versa and defines the cost of a deformation as the maximal distance between two points that are related. The integral Fréchet distance defines the cost of a deformation as the integral of the distances between points that are related. The average Fréchet distance is defined as the integral Fréchet distance divided by the lengths of  and .

Furthermore, we give relations between weighted shortest paths inside a single parameter cell~ and the monotone free space axis of . As a result we present a simple construction of weighted shortest paths inside a parameter cell. Additionally, such a shortest path provides an optimal solution for the partial Fr\'{e}chet similarity of segments for all leash lengths. These two aspects are related to each other and are of independent interest.
\end{abstract}






\section{Introduction}\label{sec:intro}

	Measuring  similarity between geometric objects is a fundamental problem in many areas of science and engineering. Applications arise e.g., when studying animal behaviour, human movement, traffic management, surveillance and security, military and battlefield, sports scene analysis, and movement in abstract spaces~\cite{gudmundsson:movement,gudmundsson:gpu,gudmundsson:football}.  Due to its practical relevance, the resulting algorithmic problem of curve matching has become one of the well-studied problems in computational geometry.  One of the prominent measures of similarities between curves is given by the  \emph{Fr\'{e}chet distance} and its variants. \emph{Fr\'{e}chet}  measures have been applied e.g., in  hand-writing recognition~\cite{DBLP:conf/icdar/SriraghavendraKB07}, protein structure alignment~\cite{DBLP:journals/jbcb/JiangXZ08}, and vehicle tracking~\cite{wenk:vehicle}. 

	In the well-known dog-leash metaphor, the (standard) \emph{Fr\'{e}chet  distance} is described as follows:
suppose a person walks a dog, while both have to move from the starting point to the ending point on their respective curves~ and~. The \emph{Fr\'{e}chet}  distance is the minimum leash length required over all possible pairs of walks, if neither person nor dog is allowed to move backwards. Here, we see the  \emph{Fr\'{e}chet  distance} as capturing the cost of a continuous deformation of~ into  and vice versa. (A deformation is required to maintain the order along~ and~.) A specific deformation induces a relation  such that `` is deformed into ''. For  we say \emph{ is related to~} and vice versa. The \emph{Fr\'{e}chet distance} defines the cost of a deformation as the maximal distance between two related points. 

	
\begin{figure}[ht]
  \begin{center}
    \begin{tabular}{ccccc}
      \includegraphics[height=2.5cm]{deformation.pdf} & &\\
\end{tabular}
  \end{center}
  \vspace*{-12pt}
  \caption{A deformation between  and  and the relation between  and . The deformation maintains the order of points along the curves. The distances between related points on the peak are larger than the distances between related points that do not lie  on the peak.
}
  \label{fig:deformationDistanceVSintegral}
\end{figure}
	
	In this paper, we study the integral and average Fr\'{e}chet distance originally introduced by Buchin~\cite{buchin:phd}. The \emph{integral Fr\'{e}chet distance} defines the cost of a deformation as the integral of the distances between points that are related. The \emph{average Fr\'{e}chet distance} is defined as the integral Fr\'{e}chet distance divided by the lengths of  and~. Next, we define these notions formally.
\subsection{Problem Definition} 
	Let  by two polygonal curves. We denote the first derivative of a function~ by . By, , we denote the -norm and by  its induced  metric. The \emph{lengths~ and~} of  and~ are defined as  and , respectively.  To simplify the exposition, we assume that  and that  and  each have  segments. A \emph{reparametrization} is a continuous function  with  and . A reparameterization  is \emph{monotone} if  holds for all . A \emph{(monotone) matching} is a pair of (monotone) reparametrizations . The \emph{Fr\'echet distance} of  and~ w.r.t.  is defined as .
	
	For a given leash length , Buchin et al.~\cite{buchin:exact} define the \emph{partial Fr\'{e}chet similarity  w.r.t. a matching } as
	 	
	and the \emph{partial Fr\'{e}chet similarity} as .
		
	Given a monotone matching , the \emph{integral Fr\'echet distance  of  and  w.r.t. } is defined as:
		
	and the \emph{integral Fr\'{e}chet distance} as ~\cite{buchin:phd}. Note that the derivatives of  and  are measured w.r.t. the -norm because the lengths of  and  are measured in  Euclidean space.  The \emph{average Fr\'{e}chet distance} is defined as ~\cite{buchin:phd}.
	
	 While the integral Fr\'{e}chet distance has been studied ~\cite[p. 860]{wenk:vehicle}, no  efficient algorithm exists to compute this distance measure (see Subsection~\ref{subsec:rel} for details).  In this paper, we design  the first pseudo-polynomial time algorithm for computing an -approximation of the integral Fr\'{e}chet distance and consequently of the average Fr\'{e}chet distance.\\
\subsection{Related Work} \label{subsec:rel} 


In their seminal paper, Alt and Godau~\cite{alt:computing} provided an algorithm that computes the Fr\'{e}chet distance between two polygonal curves  and  in  time, where  is the complexity of  and . In the presence of outliers though, the Fr\'{e}chet distance may not provide an appropriate result. This is due to the fact that the Fr\'{e}chet distance measures the maximum of the distances between points that are related. This means that already one large  "peak"  may substantially increase the Fr\'{e}chet distance between  and  when the remainder of  and  are similar to each other, see Figure~\ref{fig:deformationDistanceVSintegral} for an example.
	
\begin{figure}[ht]
  \begin{center}
    \begin{tabular}{ccccc}
      \includegraphics[height=2.5cm]{deformation2.pdf} & &\\
\end{tabular}
  \end{center}
  \vspace*{-12pt}
  \caption{An optimal deformation between  and  for both, the Fr\'{e}chet distance and the partial Fr\'{e}chet similarity. Relative to the other portions, the Fr\'{e}chet distance is significantly increased by the peak on . The partial Fr\'{e}chet similarity is unstable for distance thresholds between  and  where~. The integral Fr\'{e}chet distance between  and  is robust w.r.t. to small changes of the distances between related points and the influence of the peak. }
  \label{fig:outlier}
\end{figure}
	
	To overcome the issue of outliers, Buchin et al.~\cite{buchin:exact} introduced the notion of \emph{partial Fr\'{e}chet similarity} and gave an algorithm running in  time, where distances are measured w.r.t. the  or  metric. The partial Fr\'{e}chet similarity measures the cost of a deformation as the lengths of the parts of  and  which are made up of points that fulfill the following: The distances that are induced by straightly deforming points into their related points are upper-bounded by a given threshold , see Figure~\ref{fig:outlier}. De Carufel et al.~\cite{carufel:similarity} showed that the partial Fr\'{e}chet similarity w.r.t. to the  metric cannot be computed exactly over the rational numbers. Motivated by that, they gave an -approximation algorithm guaranteeing a pseudo-polynomial running time. An alternative perspective on the partial Fr\'{e}chet similarity is the partial Fr\'{e}chet dissimilarity, i.e., the minimization of the portions on  and  which are involved in distances that are larger than . Observe that an exact solution for the similarity problem directly leads to an exact solution for the dissimilarity problem. In particular, the sum of both values is equal to the sum of the lengths of  and .
	
	Unfortunately, both the partial Fr\'{e}chet similarity and dissimilarity are highly dependent on  the choice of  as provided by the user. As a function of , the partial Fr\'{e}chet distance is unstable, i.e., arbitrary small changes of  can result in arbitrarily large changes of the partial Fr\'{e}chet (dis)similarly, see Figure~\ref{fig:outlier}. In particular, noisy data may yield  incorrect similarity results. For noisy data,   the computation of the Fr\'{e}chet distance in the presence of imprecise points has been explored in~\cite{ahn:imprecise}. The idea behind this approach is to model signal errors by replacing each vertex  of the considered chain  by a small ball centered at . Unfortunately, the above described outlier-problem cannot be resolved by such an approach because the distance of an outlier to the other chain  could be arbitrarily large. This would mean that the radii of the corresponding balls would have been chosen extremely large.
	


	An approach related to the integral Fr\'{e}chet distance is dynamic time warping (DTW), which arose in the context of speech recognition~\cite{rabiner:fundamentals}. Here, a discrete version of the integral Fr\'{e}chet distance is computed via dynamic programming. This is not suitable for general curve matching (see~\cite[p. 204]{efrat:mathching}). Efrat et al.~\cite{efrat:mathching} worked out an extension of the idea of DTW to a continuous version. In particular, they  compute shortest path distances on a combinatorial piecewise linear -manifold that is constructed by taking the Minkowski sum of  and . Furthermore, they gave two approaches dealing with that manifold. The first one does not yield an approximation of the integral Fr\'{e}chet distance. The second one does not lead to theoretically provable guarantees regarding both: polynomial running time and approximation quality of the integral Fr\'{e}chet distance.  
	
	More specifically, ~\cite{efrat:mathching} designed two approaches for continuous curve matching by computing shortest paths on a combinatorial piecewise linear -manifold . In particular, they consider shortest path lengths between the points  and  on the polyhedral structure which is induced by . The first approach is to compute in polynomial time the unweighted monotone shortest path length on  w.r.t. . This approach does not take into account the weights in form of the considered leash length. Therefore, it does not yield an approximation of the integral Fr\'{e}chet distance. In contrast to this, the second approach considers an arbitrarily chosen weight function  such that the minimum path integral over all connecting curves on  is approximated. In terms of Fr\'{e}chet distances, this approach is an approximation of the integral Fr\'{e}chet distances as described next. By flattening and rectifying , we have a representation of the parameter space in the space of  and , such that by setting  and considering shortest path length w.r.t.  instead of , we obtain the problem setting of computing the integral Fr\'{e}chet distance (the function  is defined in Section \ref{sec:prelim}). However, to compute the weighted shortest path length on , Efrat et al. apply the so-called \emph{Fast Marching Method}, ``to solve the Eikonal equation numerically''~\cite[p. 211]{efrat:mathching}. While ``the solution it (ed.: the algorithm) provides converges monotonically''~\cite[p. 211]{efrat:mathching}, the solution does not give a  approximation with pseudo-polynomial running-time.
\subsection{Contributions} 
\label{sec:our-result}
\begin{itemize}
\item
We present a (pseudo-)polynomial time algorithm that approximates the integral  Fr\'{e}chet Distance,   , up to an multiplicative error of . This measure is desirable because it integrates the inter-curve distances along the curve traversals, and is thus more stable (w.r.t. to the choice of ) than other Fr\'{e}chet Distance measures defined by the  maximal such distance.

\item
The running time of our approach is ,  where~ is the maximal ratio of the lengths of any pair of segments from  and . Note that achieving a running time that is independent of  seems to be quite challenging as  could be arbitrary small compared to . 

\item This  guarantees   an  approximation within pseudo-polynomial running time which was  not been achieved by the approach of \cite{efrat:mathching}.

\item  Our results thus answer the implicit question raised in \cite{wenk:vehicle}: ``Unfortunately there is no algorithm known that computes the integral Fr\'{e}chet distance.'' 





	
\item		As a by-product, we show that a shortest weighted path  between two points  and~ inside a parameter cell  can be computed in constant time. We also make the observation that  provides an optimal matching for the partial Fr\'{e}chet similarity for all leash length thresholds. This provides a natural extension of locally correct Fr\'{e}chet matchings that were first introduced by Buchin et al.~\cite{buchin:locally}. They suggest to: ``restrict to the locally correct matching that decreases the matched distance as quickly as possible.''\cite[p. 237]{buchin:locally}. The matching induced by  fulfils this requirement. 

\end{itemize}
	



\section{Preliminaries} \label{sec:prelim}
The \emph{parameter space } of  and~ is an axis aligned rectangle. The bottom-left corner  and  upper-right corner    correspond to  and , respectively. We denote the - and the -coordinate of a point  by  and , respectively. A point  \emph{dominates} a point , denoted by , if  and  hold. A path~ is \emph{(-) monotone} if  holds for all . Thus a monotone matching corresponds to a monotone path  with  and . By inserting  vertical and  horizontal \emph{parameter lines}, we refine  into  rows and  columns such that the -th row (column) has a height (resp., width) that corresponds to the length of the -th segment on  (resp., ).  This induces a partitioning of  into cells,  called \emph{parameter cells}.
	
	For  with , we have . This is equal to the sum of the lengths of the subcurves between  and  and between  and . Thus, we define the \emph{length  of a path } as . Note that for the paths inside the parameter space  the -norm is applied, while the lengths of the curves in the Euclidean space are measured w.r.t. the -norm. As  measures the length of  and  at which each  is weighted by , we consider the \emph{weighted length} of  defined as follows:
	
Let  be defined as  for all . The weighted length  of a path  is defined as 	


\begin{observation}[\cite{buchin:phd}]\label{obs:dualpaths}
	Let  be a shortest weighted monotone path between  and~ inside~. Then, we have .
\end{observation}	
	


	Motivated by Observation~\ref{obs:dualpaths}, we approximate  by approximating the length of a shortest weighted monotone path  connecting  and . 
	Let  be chosen arbitrarily but fixed. Inside each parameter cell~, the union of all points  with  is equal to the intersection of an ellipse  with .  Observe that  can be computed in constant time~\cite{alt:computing}.  is characterized by two focal points  and  and a radius  such that . The two axes  (monotone) and  (not monotone) of , called the \emph{free space axes}, are defined as the line induced by  and  and the bisector between  and . If  is a disc,  and  are the lines with gradients  and  and which cross each other in the middle of . Note that the axes are independent of the value of . 
	
\begin{figure}[ht]
  \begin{center}
    \begin{tabular}{ccccc}
      \includegraphics[height=1.8cm]{illustrationFreeSpaceAxis} & &\\
\end{tabular}
  \end{center}
  \vspace*{-12pt}
  \caption{A weighted shortest -monotone path  between two points , where . The subpaths of   that do not lie on  are minimal.}
  \label{fig:shortestVSaxis}
\end{figure}	
	
	To approximate  efficiently we make the following observation that is of independent interest: Let  be two parameter points that lie in the same parameter cell  such that~. The shortest weighted monotone path  between  and  (that induces an optimal solution for the integral Fr\'{e}chet distance) is the monotone path between  and  that maximizes its subpaths that lie on   (see Figure~\ref{fig:shortestVSaxis} and Lemma~\ref{lem:key}). Another interesting aspect of  is that it also provides an optimal matching for the partial Fr\'{e}chet similarity (between the corresponding (sub-)segments) for all leash lengths,  as  has the maximal length for all , where  for a specific . Next, we discuss our algorithms.
	

\section{An Algorithm for Approximating Integral Fr\'{e}chet Distance}\label{sec:pre}
We approximate the length of a shortest weighted monotone path between  and  as follows: We construct two weighted, directed, geometric graphs  and  that lie embedded in  such that  and . Then, in parallel, we compute for  and  the lengths of the shortest weighted paths between  and . Finally, we output the minimum of both values as an approximation for .
	


We introduce some additional terminology. 	A \emph{geometric graph } is a graph where each  is assigned to a point , its \emph{embedding}. The \emph{embedding} of an edge  (into ) is . The \emph{embedding of  (into )} is . For  and , we denote simultaneously the vertex , the edge , and the graph  and their embeddings by , , and , respectively.  is \emph{monotone (directed)} if  holds for all .  Let  be an arbitrarily chosen axis aligned rectangle with height  and width . The \emph{grid (graph) of  with mesh size } is the geometric graph that is induced by the segments that are given as the intersections of  with the following lines: Let  be the  equidistant horizontal lines and let  be the  equidistant vertical lines such that . \\ \\
\noindent {\bf Construction of :}  Let  be the length of a smallest segment from  and . We construct  as the monotone directed grid graph of  with a mesh size of . Furthermore, we set  for all .\\ \\
\noindent{\bf Construction of : }  For  and , we consider the ball  with its center at  and a radius of  w.r.t. the  metric. 
For the construction of  we need the free space axes of the parameter cells and so called grid balls:


\begin{definition}\label{def:gridball}
	Let  and  be chosen arbitrarily. The \emph{grid ball } is defined as the grid of  that has a mesh size of . We say~ \emph{approximates}~.
\end{definition}

We define  as the monotone directed graph that is induced by the arrangement that is made up of the following components restricted to : 

\noindent\begin{minipage}{0.52\linewidth}\vspace*{2ex}

	\begin{itemize}
		\item (1) All monotone free space axes restricted to their corresponding parameter cell.  
		\item (2) All grid balls  for  and any parameter edge . 
		\item (3) The segments  and  if the parameter cells  and  that contain  and  are intersected by their corresponding monotone free space axes  and , where  and  are defined as the  bottom-leftmost and  top-rightmost point of  and .
	\end{itemize}
	
\end{minipage}
\begin{minipage}{0.4\linewidth}\vspace*{2ex}
\begin{center}
    \begin{tabular}{p{6cm}}
      \includegraphics[height=2.3cm]{exampleG2merged.pdf}\\ 
{\small Exemplified construction of  for two given polygonal curves  and .  For simplicity we only illustrate four grid balls (with reduced radii) and the corresponding point pairs from .}
    \end{tabular}
  \end{center}
\end{minipage}\vspace*{2ex}

Finally, we set  for all . 
For each edge  we choose the point  as the center of the corresponding grid ball because the free space axes of the parameters cells adjacent to  lie close to . \\  \\
\noindent{\bf Analysis of our approach: }
Since  is monotone and each edge  is assigned to , we obtain that for each path  between~ and  holds .  The same argument applies to . Hence, we still have to ensure that there is a path  or  such that . We say that a path  is \emph{low} if  holds for all .  For our analysis, we show the following:\\
	{\bf Case A: }  There is a  with  if there is a shortest path  that is not low (see Subsection~\ref{subsubsec:anaG1}).\\
	{\bf Case B:}  Otherwise, there is a  with  (see Subsection~\ref{subsubsec:anaG2}).

\subsection{Analysis of Case A}\label{subsubsec:anaG1} 

	In this subsection, we assume that there is a shortest path  between  and  that is not low, i.e., there is a  with . Furthermore, for any , we denote the subpath of~ which is between  and~ by . 
First we prove a lower bound for  (Lemma~\ref{lem:lowerBoundForSummedFDcase1}). This lower bound ensures that the approximation error that we make for a path in  is upper-bounded by  (Lemma~\ref{lem:apprQualityG1}).

A \emph{cell  of } is the convex hull of four vertices  such that . As the mesh size of  is , we have  for any two points  and  that lie in the same cell of . The following property of  is the key in the analysis of the weighted shortest path length of :

\begin{definition}[\cite{funke:smooth}]\label{def:lip}
		 is -Lipschitz if  for all ~\footnote{The requirement  is also occasionally used to define -Lipschitz continuity. Note that this alternative definition is equivalent to Definition~\ref{def:lip}.}.
	\end{definition}

\begin{lemma}\label{lem:lip}
	 is -Lipschitz w.r.t. .
\end{lemma}
\begin{proof}
	Let  be chosen arbitrarily. The subcurves  between  and  and  between  and  have lengths no larger than  and . Thus  and . Furthermore,  is equal to . By triangle inequality,  , because , , , and . \end{proof}

	Lemma~\ref{lem:lip} allows us to prove the following lower bound for the weighted length of .

\begin{lemma}\label{lem:lowerBoundForSummedFDcase1}
	
\end{lemma}
\begin{proof}
	Let  such that . Let . We have  because  is -Lipschitz. Furthermore,  implies  which yields .
\end{proof}



\begin{lemma}\label{lem:apprQualityG1}
	There is a path  that connects  and  such that .
\end{lemma}
\begin{proof} Starting from , we construct  inductively as follows: If  crosses  a vertical
	
\noindent\begin{minipage}{0.8\linewidth}\vspace*{0.5ex}
 (horizontal) parameter line next,  goes one step to the right (top). For  let  be the line with gradient  such that  (see the figure on the right). As  and  are monotone,  is unique and well defined. For all ,   and  lie in the same cell of  and thus, . This implies  because . To be more precise, we consider  to be parametrized such that . We obtain,  for all .
\end{minipage}
\begin{minipage}{0.2\linewidth}
  \begin{center}
    \includegraphics[height=3cm]{constructionEidetildePiCase1}
  \end{center}
\end{minipage}\vspace*{0.5ex}
	  Furthermore, the above implies  . Thus:
	
\end{proof}




\subsection{Analysis of Case B}\label{subsubsec:anaG2}
 In this subsection, we assume that there is a monotone low path  between  and . 
First, we make a key observation that is also of independent interest. It states that a shortest path (that is not necessarily low) inside a parameter cell is uniquely determined by its monotone free space axis.
\begin{lemma}\label{lem:key}
	Let  be an arbitrarily chosen parameter cell and  such that . Furthermore, let  be the monotone free space axis of  and  the rectangle that is induced by  and . The shortest path  between  and  is given as:
		\begin{itemize}
			 \item  , if  intersects  in  and  such that  and as
			\item , otherwise, where  is defined as the closest point from  to .
		\end{itemize}
		
\begin{figure}[ht]
  \begin{center}
    \begin{tabular}{ccccccc}
      \includegraphics[height=2.2cm]{smallerWeightProjection2.pdf} & &
       \includegraphics[height=2.2cm]{smallerWeightProjection.pdf}&&\\ 
{\small (a) Construction of a curve  between} & &
      {\small (b) Projecting a point orthogonally onto}&&\\
      {\small  and  which is not longer than .}&&
      {\small a free space axis reduces its weight.}&&
    \end{tabular}
  \end{center}
  \vspace*{-12pt}
  \caption{A shortest weighted -monotone path between two points  and  with .}
  \label{fig:smallerWeightProjection}
\end{figure}
		
\end{lemma}
\begin{proof} Let  by an arbitrary monotone path that connects  and . In the following, we show that . For this, we prove the following: Let  be chosen arbitrarily and   be its orthogonal projection onto  (see Figure~\ref{fig:smallerWeightProjection}(b)). We show  for . This implies that there is an injective, continuous function  with  for all~. In particular,  is defined as the intersection point of  and the line  that lies perpendicular to  such that . The function  is well defined and injective as both  and  are monotone paths that connect  and . Similarly, as in the proof of Lemma~\ref{lem:apprQualityG1}, this implies  because .

	To be more precise, consider  to be parametrized such that . This implies  for all . Thus:
	
	
	Finally, we show:  , for . Note that  and  are the leash lengths for   and  that lie on the boundary of the white space inside , i.e., on the boundary of the ellipses  and~, resp. (see Figure~\ref{fig:smallerWeightProjection}). Since  we get , which implies .
\end{proof}

	We call a point  \emph{canonical} if . Let  and  be two parameter cells that share a parameter edge . Furthermore, let   and  be two canonical parameter points such that  where  and  are the monotone free space axis of  and , respectively. Let  be the top-right end point of  and  the bottom-left end point of . The following lemma is based on Lemma~\ref{lem:key} and characterizes how a shortest path passes through the parameter edges.\\ 

\noindent\begin{minipage}{0.5\linewidth}\vspace*{0.5ex}
\begin{lemma}\label{lem:canonicalOneVertex} If  and ,  is equal to the concatenation of the segments , , and   (see figure~(a) on right). Otherwise, there is a  such that  is equal to the concatenation of the segments , , and , where  and  such that  is the orthogonal projection of  and  onto  (see figure~(b)).
\end{lemma}
\end{minipage}
\begin{minipage}{0.4\linewidth}
\begin{center}
    \begin{tabular}{ccccccc}
      \includegraphics[height=2.8cm]{crossingParameterEdgeA.pdf} & &
       \includegraphics[height=2.8cm]{crossingParameterEdgeB.pdf}&&\\ 
{\small (a)  for } & &
      {\small (b)  for }&&
    \end{tabular}
  \end{center}
\end{minipage} 
\\ 

\noindent {\bf Outline of the analysis of Case B: }
In the following, we apply Lemmas~\ref{lem:key} and~\ref{lem:canonicalOneVertex} to subpaths  of  in order to ensure that~ is a subset of the union of a constant number of balls (that are approximated by grid balls in our approach) and monotone free space axes. In particular, we construct a discrete sequence of points from  which lie on the free space axes, see Subsection~\ref{subsec:Sep}. 
For each induced subpath~, we ensure that  crosses one or two perpendicular parameter edges. For the analysis we distinguish between the two cases which we consider separately:\\
{\bf Case 1:}  crosses one parameter edge and 
{\bf Case 2:}  crosses two parameter edges.

	
\begin{figure}[ht]
  \begin{center}
    \begin{tabular}{ccccccc}
       \includegraphics[height=3.9cm]{captureSubpathA.pdf} & &
       \includegraphics[height=3.9cm]{captureSubpathB.pdf} & &
       \includegraphics[height=3.8cm]{captureSubpathC2.pdf}&&\\ 
      {\small (a) Case (1.)} & &
      {\small (b) Case (2.1.)}&&
      {\small (c) Case (2.2.)}\\
\end{tabular}
  \end{center}
  \vspace*{-12pt}
  \caption{Three different subcases in which we  ensure, differently, that we capture a subpath  by balls and free space axes. Path  is approximated by a path  in the graph that is induced by these free space axis and the corresponding grid balls.}
  \label{fig:captureTheSubpath}
\end{figure}
	
    For Case 1, we show that, if  crosses one edge () then  is a subset of the union of the two monotone free space axes of the parameter cells that share  and the ball  for   (see Figure~\ref{fig:captureTheSubpath}(a) and Subsections~\ref{subsec:anaOneCrossing}).
	
	For Case 2,  (see Subsection~\ref{subsec:anaTwoCrossing}), we consider the case that  crosses two parameter edges~ and~. In particular,  runs through three parameter cells , , and , where  and~ share  and~ and  share . 
	
	We further distinguish further between two subcases. For this, let  and . \\
{\bf Case 2.1:}  We show that, if , then  is a subset of the union of the balls  and  and the monotone free space axes of , , and  (see Figure~\ref{fig:captureTheSubpath}(b) and Lemma~\ref{lem:shortestPathOneCrossing}).\\ 
{\bf Case 2.2:} We show that, if , then  is a subset of the union of the ball  and the monotone free space axes of  and  (see Figure~\ref{fig:captureTheSubpath}(c) and Lemma~\ref{lem:twoCrossingComplex}).
	
	For the analysis of the length of a shortest path  that lies between  and , we construct for  a path  between  and  such that . In particular,  is a subset of the grid balls that approximate the above considered balls and the free space axes that are involved in the individual (sub-)case for  (see, Figure~\ref{fig:captureTheSubpath}). Finally, we define  as the concatenation of the approximations  for all .
		
\subsubsection{Separation of a shortest path}\label{subsec:Sep}
	
	In the following, we determine a discrete sequence of canonical points  such that  crosses at most two parameter lines for each . First we need the following supporting lemma:
	
\begin{lemma}\label{lem:tech}
	For all  that lie in the same parameter cell with  we have .
\end{lemma}
\begin{proof}
	The triangle inequality implies:\\
	 . This implies , 
because\\ . Furthermore,  and  because  and  lie in the same cell. This implies . 	A corresponding argument yields .
\end{proof}

\begin{lemma}\label{lem:separatingPoints}
	There are canonical points  such that for all  the following holds: (P1)  crosses at most one vertical and at most one horizontal parameter line which are both not part of  and (P2) the distance of  to a parameter line is lower-bounded by  for all .
\end{lemma}
\begin{proof} First, we give the construction of . After that, we establish Properties (P1) and  (P2), for each .
\begin{itemize}
	\item Construction of : We construct  iteratively with . Point  is defined as the first point on  such that  or . For , let  be defined,  be the top-right corner of the parameter cell that contains , and  be the next intersection point of  (behind ) with the parameter grid, see Figure~\ref{fig:constructionSequenceCanonicalPoints}. W.l.o.g., we assume that  lies on a vertical parameter line.
	
	If , we define  as the first point on  with  or , see Figure~\ref{fig:constructionSequenceCanonicalPoints}(a).
	
	If , we consider the next intersection point  of  with a horizontal parameter line such that . We define  as the first point behind  such that .
	
\begin{figure}[ht]
  \begin{center}
    \begin{tabular}{ccccccc}
      \includegraphics[height=3.5cm]{constructionOfpi+1a.pdf} & &
       \includegraphics[height=3.5cm]{constructionOfpi+1b.pdf}&&\\ 
{\small (a) Construction of  for .} & &
      {\small (b) Construction of  for .}&&
    \end{tabular}
  \end{center}
  \vspace*{-12pt}
  \caption{The iterative construction of  distinguishes between the two cases depending on whether the next intersection  of  and the parameter grid lies close to a vertex  of the parameter grid or not.}
  \label{fig:constructionSequenceCanonicalPoints}
\end{figure}

	\item (P1) and (P2): W.l.o.g., we assume . For the configurations of  a symmetric argument applies. Assume  and  fulfil (P1) and (P2) for . We show that  fulfils (P1) and  (P2) for the two cases  and  separately. By induction it follows the statement of the lemma.
		\begin{itemize}
			\item : 
				\begin{itemize}
					\item (P1): In both subcases  or  it follows that  lies in the parameter cell  that lies to the right of the parameter cell that contains . In particular, in the first (second) subcase  lies by construction in the same parameter row (column). As  is monotone and  is defined as the first point that fulfils one of the two constraints, . This implies (P1).
					\item (P2): The above argument implies that the distances of  to the right and the top parameter line are lower-bounded by . If , we have . Thus, Lemma~\ref{lem:tech} implies  which yields that the distance of  to the left parameter line is lower-bounded by . If , we have . Thus, Lemma~\ref{lem:tech} implies  which yields that the distance of  to the bottom parameter line is lower-bounded by . Hence, (P2) is fulfilled.
				\end{itemize}
			\item : 
				\begin{itemize}
					\item (P1): Assume  crosses another vertical parameter line in a point  that lies before~. This implies,  and . Hence,  which is a contradiction to Lemma~\ref{lem:tech}. Thus, (P1) is fulfilled.
					\item (P2): The construction of  implies that the distances to the bottom and to the top parameter line is lower-bounded by . Finally, we lower-bound the distances of  to the left and to the right parameter line as follows:  By combining  and Lemma~\ref{lem:tech} we obtain . Another application of Lemma~\ref{lem:tech}, combined with  leads to . Thus, the distances of  to the  left and to the right parameter line are lower-bounded by . Hence, (P2) is fulfilled.
				\end{itemize}
		\end{itemize}
\end{itemize}
\end{proof}


\subsubsection{Analysis of subpaths that cross one parameter edge}\label{subsec:anaOneCrossing}
	We need to show that those parts of  that do not lie on the free space axes are covered by the balls~. For this, we use the following geometrical interpretation of the free space axes  and  of a parameter cell . Let  and  be the segments that correspond to . We denote the angular bisectors of  and  by  and  such that the start points  and  of  and  lie on different sides w.r.t. , see Figure~\ref{fig:dual}(b). If  and~ are parallel,  denotes the line between  and  and we declare  as undefined\footnote{There is a corresponding definition of  in the case of . However, considering  for  would unnecessarily complicate the presentation because  is not required.}. We observe (see Figure~\ref{fig:dual}):
	
\begin{observation}\label{obs:dual}
	 and  .
\end{observation}

\begin{figure}[ht]
  \begin{center}
    \begin{tabular}{ccccccc}
      \includegraphics[height=1.9cm]{dualB.pdf} & &
       \includegraphics[height=2.4cm]{dualA.pdf}&&\\ 
{\small (a) Point  and .} & &
      {\small (b) To  and  corresponding leashes.}&&
    \end{tabular}
  \end{center}
  \vspace*{-12pt}
  \caption{Duality of parameter points from  () and leashes that lie perpendicular to  ().}
  \label{fig:dual}
\end{figure}

	From now on, let  be two consecutive, canonical points that are given via Lemma~\ref{lem:separatingPoints} such that . Furthermore, let  and  be the free space axes of the parameter cells~ and  such that  and .
	
\begin{lemma}\label{lem:oneCrossing}
	If  crosses one parameter edge ,  exist and we have  where .
\end{lemma}
\begin{proof} W.l.o.g., we assume that  is horizontal. Let  and  be the segments that induce  parameter cells  and . First, we show  and, then, that . Let  and  such that  and , see Figure~\ref{fig:oneCrossing}(a).  implies . Furthermore,  implies:  corresponds to a leash  such that  and , see Figure~\ref{fig:oneCrossing}(b). Thus,  is upper-bounded by  which is upper-bounded by .




\begin{figure}[ht]
  \begin{center}
    \begin{tabular}{ccccccc}
      \includegraphics[height=3.2cm]{oneCrossing.pdf} & &
       \includegraphics[height=3.2cm]{oneCrossing2.pdf}&&\\ 
{\small (a) The subpath } & &
      {\small (b) The length of the subcurve of  that corresponds to}&&\\
      {\small lies in the convex hull of}& &
      {\small the -coordinates of the points from }&&\\
      {\small , , , and .}&&
      {\small depends on .}&&
    \end{tabular}
  \end{center}
  \vspace*{-12pt}
  \caption{Configuration of the Lemmas~\ref{lem:oneCrossing} and~\ref{lem:shortestPathOneCrossing}: The length of the subpath of  that does not necessarily lie on  is related to .}
  \label{fig:oneCrossing}
\end{figure}

	Finally, we show : We have  because  is low. Lemma~\ref{lem:separatingPoints} implies . Thus, . A similar argument implies that 
\end{proof}
	
\begin{lemma}\label{lem:shortestPathOneCrossing}
	 (see Figure~\ref{fig:captureTheSubpath}(a)).
\end{lemma} 
\begin{proof}
	We combine Lemmas~\ref{lem:canonicalOneVertex} and~\ref{lem:oneCrossing}. Lemma~\ref{lem:canonicalOneVertex} implies that  orthogonally crosses    at a point  that lies between  and  such that . Lemma~\ref{lem:oneCrossing} implies . Thus, . Furthermore,  and . This implies  because .
\end{proof}


\begin{lemma}\label{lem:shortestPathOneCrossingAppr}
	There is a path  between  and  such that .
\end{lemma}
\begin{proof}: By Lemma~\ref{lem:shortestPathOneCrossing}, the two following intersection points are well defined: Let  be the intersection point of  and  which lies on the left or bottom edge of . Analogously, let  be the intersection point of  and  which lies on the right or top edge of . By Lemma~\ref{lem:shortestPathOneCrossing}, we can subdivide  into the three pieces , , and . As , we just have to construct a path  between~ and  such that .
	
	We construct  by applying the same approach as used in the proof of Lemma~\ref{lem:apprQualityG1}, see Figure~\ref{fig:captureTheSubpath}(a).
	
	To upper-bound  by  we first lower-bound  by . Then, we apply an approach that is similar to the approach used in the proof of Lemma~\ref{lem:apprQualityG1} 
	\begin{itemize}
		\item : Let . As  and  is -Lipschitz, we obtain . This implies  because~.
		\item : We observe that   because  is monotone and . Furthermore, we parametrize  such that . This implies   and   for all . Thus:
			
	\end{itemize}
\end{proof}
\subsubsection{Analysis of subpaths that cross two parameter edges}\label{subsec:anaTwoCrossing}
	Let  and  be two consecutive parameter points from  such that  crosses two parameter edges  and . By Lemma~\ref{lem:separatingPoints},  and  are perpendicular to each other and are adjacent at a point . Let  be the parameter cell such that  and  are part of the boundary of . Furthermore, let  and  be the parameter cells such that  and . We denote the monotone free space axis of , , and  by , , and , respectively. Let  and .

\begin{lemma}\label{lem:twoCrossing}
	If , there is another canonical parameter point  such that .
	
\begin{figure}[ht]
  \begin{center}
    \begin{tabular}{ccccccc}
      \includegraphics[height=4cm]{twoCrossingA.pdf} & &
       \includegraphics[height=4cm]{twoCrossing.pdf}&&\\ 
{\small (a) Construction of } & &
      {\small (b) The matching that corresponds  has three}&&\\
      {\small s.t. ,}& &
      {\small parts in that the matched points from  lie}&&\\
      {\small , and}&&
      {\small perpendicular to the corresponding angular bisector}&&\\
      {\small }&&
      {\small }&&
    \end{tabular}
  \end{center}
  \vspace*{-12pt}
  \caption{Configuration of Lemma~\ref{lem:twoCrossing} in the Euclidean space of  and  and in the parameter space of : There are three supaths  that lie on monotone free space axis , , and  and, thus, three parts of the corresponding matching which are made up of leashes that are perpendicular to the diagonals , , and  that correspond to , , and .}
  \label{fig:twoCrossing}
\end{figure}

\end{lemma}
\begin{proof}
	W.l.o.g., assume that  crosses first a vertical parameter edge. Let  and  be the segments that induce  parameter cells , , and , see Figure~\ref{fig:twoCrossing}. Let  and  be the top-right end points of  and , respectively. Let   and  be the bottom-left end points of  and , respectively (see Figure~\ref{fig:twoCrossing}(a)). Let  such that  and . Furthermore, let  and  such that  and . In the following, we show that , , and . This implies  and concludes the proof.
	
	
	Below, we show . Then, a similar argument as used in Lemma~\ref{lem:oneCrossing} implies  and .
	Finally, we show :  implies that  and  are upper-bounded by . Thus, we obtain .
\end{proof}

\begin{lemma}\label{lem:shortestPathTwoCrossingApprSimplie}
	If , there is a path  between  and  such that .
\end{lemma}
\begin{proof}
	Lemma~\ref{lem:twoCrossing} implies that the following constructions are unique and well defined: Let  () be the intersection point of  and  () that lies on the left or bottom (respectively, right or top) edge of . Analogously, let  () be the intersection point of  and  () that lies on the left or bottom (respectively, right or top) edge of .  
	 By applying the approach of Lemma~\ref{lem:shortestPathOneCrossingAppr}, for  and , we obtain a path  between  and  and a path  between  and  such that  and .
	This concludes the proof because .
\end{proof}

\begin{lemma}\label{lem:twoCrossingComplex}
	If ,  we have  where .
\end{lemma}
\begin{proof} Let  be the last point that lies on , i.e., there is no point  such that , see Figure~\ref{fig:twoCrossingComplex}(b). In the following, we show . Analogously, we construct the first point  that lies on , i.e., there is no point , see Figure~\ref{fig:twoCrossingComplex}(b). A similar argument as above implies . The triangle inequality implies  for all  and . This concludes the proof.

	For the sake of contradiction we assume . Lemma~\ref{lem:key} implies that  crosses the boundary  of  in the orthogonal projection  of  onto the top edge of  or in the orthogonal projection  of  onto the right edge of , see Figure~\ref{fig:twoCrossingComplex}(b). Thus, even  or  because  is monotone. In the following, we show  where  such that . This implies . 
		
	Furthermore, we construct another path  connecting  and  such that . Additionally, we show . This is a contradiction to the fact that  is a shortest path between  and :
		
	
\begin{figure}[ht]
  \begin{center}
    \begin{tabular}{ccccccc}
      \includegraphics[height=2.6cm]{twoCrossingComplexC.pdf} & &
       \includegraphics[height=3cm]{twoCrossingComplexD.pdf}&&\\ 
{\small (a) The paths (1.)  (red), (2.)  (blue), (3.)  (green), and } & &
      {\small (b) Even  crosses }&&\\
      {\small (4.)  (orange) correspond to matchings such the matched}& &
      {\small orthogonally in  or}&&\\
      {\small points (1.) on  are just  (red), (2.) on  are}&&
      {\small in  and  orthogonally }&&\\
      {\small just  (blue), (3.) on  are just  (green), and }&&
      {\small even in  or in .}\\
      {\small (4.) on  are just  (orange)}&&
      {\small }\\
      \includegraphics[height=3cm]{twoCrossingComplexA2.pdf} & &
       \includegraphics[height=3cm]{twoCrossingComplexB.pdf}&&\\ 
{\small (c) Matching that corresponds to } & &
      {\small (d) Construction of }&&\\
    \end{tabular}
  \end{center}
  \vspace*{-12pt}
  \caption{Two paths  and  and the corresponding matchings: The shortest path  that ``leaves''  at the point  and enters  at the point  and the path  that is constructed such that .}
  \label{fig:twoCrossingComplex}
\end{figure}

	\begin{itemize} 
		\item Construction of : Let  be the top-right end point of  and  the bottom-left end point of , see Figure~\ref{fig:twoCrossingComplex}(d). We define . 
		\item Upper bound for : First of all we show . After that we show  for all . This implies . A similar argument implies . Furthermore, we show  and  for all . This implies . A similar argument implies  . Finally we upper bound
			 
			\begin{itemize}
				\item :  implies . As the gradient of  is  and , we have .
				\item  for all : First we show : By Lemma~\ref{lem:separatingPoints} we know . Let . This implies . Furthermore, we have  because  is low. This implies . Thus we have . This implies . As , we get .
				\item : By  it follows that  holds. Furthermore, we have  because . The triangle inequality implies .
				\item  for all : Above we already showed . By combining the -Lipschitz continuity of  and  we obtain .
			\end{itemize}
		\item : Above we showed . The triangle inequality implies .
		\item : First we lower-bound : Above we already showed . Combining this with  and the triangle inequality yields . Thus  because  has a gradient of  and . 
		
	Above we already showed  because . This implies  because .
		
	By combining  and  we get  as follows: Consider the subsegment . By  it follows  for all  because  is -Lipschitz. This implies . 
	\end{itemize}
\end{proof}


	Lemma~\ref{lem:twoCrossingComplex} implies that the  approach taken in the proof of Lemma~\ref{lem:shortestPathOneCrossingAppr} yields that there is a path  between  and  such that  If . Combining this with Lemmas~\ref{lem:shortestPathOneCrossingAppr} and~\ref{lem:shortestPathTwoCrossingApprSimplie} yields the following corollary:

\begin{corollary}\label{cor:apprC2}
	Let  be a shortest path. We have .
\end{corollary}

\subsection{``Bringing it all together''}

	In Subsections~\ref{subsubsec:anaG1} and~\ref{subsubsec:anaG2}, we showed that in both  cases, Case A and B,  the minimum of the shortest path lengths in  and  is upper-bounded by , where  is a shortest path in . 

	Next, we discuss that our algorithm has a running time of . Graph  is given by the arrangement that is induced by  horizontal and  vertical lines because the corresponding grid has a mesh of size  . Thus, . Graph	 is given by the arrangement that is induced by  free space axis and  grid balls. Each grid ball has a complexity of . Thus, . Applying Dijkstra's  shortest path algorithm on  and  takes time proportional to   and . 
As  and  we have to ensure that each edge of  can be computed in constant time to guarantee an overall running time of .

\begin{lemma}\label{lem:edgesComputable}
	All edges of  and  can be computed in  time.
\end{lemma}
\begin{proof}
	There are two types of edges used in  and : (1.) axis aligned edges and (2.) edges that lie on a monotone free space axis. We consider both cases separately:
		\begin{itemize}
			\item Axis aligned edge , see Figure~\ref{fig:edgeWeight}(a): W.l.o.g., we assume that  is horizontal. We have , see Figure~\ref{fig:edgeWeight}(a). Let  be the segment such that  for . W.l.o.g., we assume that  lies on the -axis.  can be calculated as follows:
					
				That value can be calculated in constant time.
			\item Edge  on a free space axis, see Figure~\ref{fig:edgeWeight}(b): Let  be the free space axis such that  and  the corresponding angular bisector that corresponds to . By observation~\ref{obs:dual}, we have  where  lies perpendicular to , see Figure~\ref{fig:edgeWeight}(b). Thus,  is equal to the area that is bounded by , , , and  which can be computed in  time.
		\end{itemize}
		
		\begin{figure}[ht]
  \begin{center}
    \begin{tabular}{ccccccc}
      \includegraphics[height=2cm]{calcHorizontalEdgeWeight.pdf} & &
       \includegraphics[height=1.8cm]{calcAxisEdgeWeight.pdf}&&\\ 
{\small (a) Matching of a horizontal edge.} & &
      {\small (b) Matching of an edge an a free space axis.}&&
    \end{tabular}
  \end{center}
  \vspace*{-12pt}
  \caption{The two types of matchings that correspond to the two types of edges from  and~.}
  \label{fig:edgeWeight}
\end{figure}
\end{proof}
This leads to our main result.
\begin{theorem}
	We can compute an -approximation of the integral Fr\'echet distance  in  time.
\end{theorem}

\section{Locally optimal Fr\'{e}chet matchings}

In this section, we discuss an application of Lemma~\ref{lem:key} to so-called \emph{locally correct (Fr\'{e}chet) matchings} as introduced by Buchin et al.~\cite{buchin:locally}. For  and , we denote the subcurve between  and  by .
\begin{definition}[\cite{buchin:locally}]\label{def:correct}
	A matching  is \emph{locally correct} if 
	, for all .
\end{definition}

	Buchin et al.~\cite{buchin:locally} suggested to extend the definition of locally correct (Fr\'{e}chet) matchings to ``locally optimal'' (Fr\'{e}chet) matchings as future work. ``The idea is to restrict to the locally correct matching that decreases the matched distance as quickly as possible.''\cite[p. 237]{buchin:locally}. To the best of our knowledge, such an extension of the definition of locally correct matchings has not been given until now. In the following, we give a definition of locally optimal matchings and show that each locally correct matching 
can be transformed, in  time, into a locally optimal matching.
	
	Buchin et al.~\cite{buchin:locally} require the leash length to decrease as fast possible. In general though, there is no matching that ensures a monotonically decreasing leash length. We therefore also consider increasing the leash length and extend the objective as follows: ``Computing a locally correct matching that locally decreases and increases the leash length as fast as possible between two maxima''. We measure how fast the leash length decreases (increases) as sum of the lengths of the subcurves that are needed to achieve a leash length of  (the next (local) maximum), then we continue from the point pair that realizes .
	
	 Thus, it seems to be natural to consider a matched point pair from  in that a local maxima is achieved as fixed. Note that requiring a fast reduction and a fast enlargement of the leash length between two pairs  and  of fixed points is equivalent to requiring a matching that is optimal w.r.t. the partial Fr\'{e}chet similarity between the curves between the points  and  and  and  for all thresholds . 
	 
	 In the following, we give a definition of locally correct matchings that considers the above described requirements.
	Let . , is the parameter of a \emph{local maximum} of  if the following is fulfilled: there is a  such that for all  and  or . 
	Given a matching , let  be the ordered sequence of parameters for all local maxima of the function . For any , we denote the restrictions of  and  to  as  and .
	We say  is \emph{locally optimal} if it is locally correct and for all ,  for all .
	
	By applying a similar approach as in the proof of Lemma~\ref{lem:key} we obtain the following:
	
\begin{lemma}\label{cor:partFS}
	Let  be an arbitrarily chosen parameter cell and  such that  and  the path induced by Lemma~\ref{lem:key}. Then,  for all , where  is the free space ellipse of  for the distance threshold .
\end{lemma}

Lemma~\ref{cor:partFS} implies that each locally correct matching  can be transformed into a locally optimal Fr\'{e}chet matching in  time as follows: Let  be the intersection points with the parameter grid. For each  we substitute the subpath  by the path between  and  which is induced by Lemma~\ref{lem:key}.
	The algorithm from~\cite{buchin:locally} computes a locally correct matching in  time. Thus, a locally optimal matching can be computed in  time.

\section{Conclusion}

	We presented pseudo-polynomial -approximation algorithms for the integral and average Fr\'{e}chet distance which have a running time of . In particular, in our approach we compute two geometric graphs and their weighted shortest path lengths in parallel. It remains open if one can reduce the complexity of  to polynomial with respect to the input parameters such that   still ensures an -approximation. 
	













\begin{thebibliography}{50}

\bibitem{ahn:imprecise}
H.-K. Ahn, C.~Knauer, M.~Scherfenberg, L.~Schlipf, and A.~Vigneron.
\newblock Computing the discrete {F}r{\'e}chet distance with imprecise input.
\newblock {\em Int. J. Comput. Geometry Appl.}, 22(1):27--44, 2012.

\bibitem{alt:computing}
H.~Alt and M.~Godau.
\newblock Computing the {F}r{\'e}chet distance between two polygonal curves.
\newblock {\em Int. J. Comput. Geometry Appl.}, 5:75--91, 1995.

\bibitem{wenk:vehicle}
S.~Brakatsoulas, D.~Pfoser, R.~Salas, and C.~Wenk.
\newblock On map-matching vehicle tracking data.
\newblock In K.~B{\"o}hm, C.~S. Jensen, L.~M. Haas, M.~L. Kersten, P.-{\AA}.
  Larson, and B.~C. Ooi, editors, {\em Proceedings of the 31st International
  Conference on Very Large Data Bases, Trondheim, Norway, August 30 - September
  2, 2005}, pages 853--864. ACM, 2005.

\bibitem{buchin:locally}
K.~Buchin, M.~Buchin, W.~Meulemans, and B.~Speckmann.
\newblock Locally correct {F}r{\'e}chet matchings.
\newblock In {\em ESA}, pages 229--240, 2012.

\bibitem{buchin:exact}
K.~Buchin, M.~Buchin, and Y.~Wang.
\newblock Exact algorithms for partial curve matching via the {F}r{\'e}chet
  distance.
\newblock In C.~Mathieu, editor, {\em SODA}, pages 645--654. SIAM, 2009.

\bibitem{buchin:phd}
M.~Buchin.
\newblock On the computability of the {F}r\'{e}chet distance between
  triangulated surfaces.
\newblock Ph.D. thesis, Dept. of Comput. Sci., Freie Universit\"{a}t, Berlin,
  2007.

\bibitem{carufel:similarity}
J.-L.~D. Carufel, A.~Gheibi, A.~Maheshwari, J.-R. Sack, and C.~Scheffer.
\newblock Similarity of polygonal curves in the presence of outliers.
\newblock {\em Comput. Geom.}, 47(5):625--641, 2014.

\bibitem{efrat:mathching}
A.~Efrat, Q.~Fan, and S.~Venkatasubramanian.
\newblock Curve matching, time warping, and light fields: New algorithms for
  computing similarity between curves.
\newblock {\em Journal of Mathematical Imaging and Vision}, 27(3):203--216,
  2007.

\bibitem{gudmundsson:movement}
J.~Gudmundsson, P.~Laube, and T.~Wolle.
\newblock Movement patterns in spatio-temporal data.
\newblock In S.~Shekhar and H.~Xiong, editors, {\em Encyclopedia of GIS}, pages
  726--732. Springer, 2008.

\bibitem{gudmundsson:gpu}
J.~Gudmundsson and N.~Valladares.
\newblock A {GPU} approach to subtrajectory clustering using the {F}r\'{e}chet
  distance.
\newblock In {\em SIGSPATIAL/GIS}, pages 259--268, 2012.

\bibitem{gudmundsson:football}
J.~Gudmundsson and T.~Wolle.
\newblock Football analysis using spatio-temporal tools.
\newblock In I.~F. Cruz, C.~A. Knoblock, P.~Kr{\"o}ger, E.~Tanin, and
  P.~Widmayer, editors, {\em SIGSPATIAL/GIS}, pages 566--569. ACM, 2012.

\bibitem{DBLP:journals/jbcb/JiangXZ08}
M.~Jiang, Y.~Xu, and B.~Zhu.
\newblock Protein structure-structure alignment with discrete fr{\'{e}}chet
  distance.
\newblock {\em J. Bioinformatics and Computational Biology}, 6(1):51--64, 2008.

\bibitem{rabiner:fundamentals}
L.~R. Rabiner and B.-H. Juang.
\newblock {\em Fundamentals of speech recognition}.
\newblock Prentice Hall {S}ignal {P}rocessing series. Prentice Hall, 1993.

\bibitem{DBLP:conf/icdar/SriraghavendraKB07}
E.~Sriraghavendra, K.~Karthik, and C.~Bhattacharyya.
\newblock Fr{\'{e}}chet distance based approach for searching online
  handwritten documents.
\newblock In {\em 9th International Conference on Document Analysis and
  Recognition {(ICDAR} 2007), 23-26 September, Curitiba, Paran{\'{a}}, Brazil},
  pages 461--465, 2007.
  
\bibitem{funke:smooth}
S.~Funke and E~A. Ramos.
\newblock Smooth-Surface Reconstruction in Near-Linear Time.
\newblock In {\em Proceedings of the Thirteenth Annual {ACM}-{SIAM} Symposium on Discrete Algorithms},
 pages 781--790, 2007.
               

\end{thebibliography}











\end{document}
