



\documentclass[11pt]{article}
\usepackage{coling2018}
\usepackage{times}
\usepackage{url}
\usepackage{latexsym}

\def\figref#1{Fig.~\ref{#1}}
\def\secref#1{Sec.~\ref{#1}}
\def\tabref#1{Table~\ref{#1}}
\def\eqnref#1{Eqn.~\ref{#1}}

\usepackage{graphicx}
\usepackage{color}
\usepackage{amsmath}
\usepackage{tabularx}
\usepackage{wrapfig}
\usepackage{arydshln}
\graphicspath{ {figs/} }





\newcommand\BibTeX{B{\sc ib}\TeX}


\title{Dynamic Multi-Level Multi-Task Learning for Sentence Simplification}

\author{Han Guo \;\;\;\;\;\;\; Ramakanth Pasunuru \;\;\;\;\;\;\; Mohit Bansal \\
  UNC Chapel Hill \\
  {\tt \{hanguo, ram, mbansal\}@cs.unc.edu} \\
 }

\date{}

\begin{document}
\maketitle


\begin{abstract}
Sentence simplification aims to improve readability and understandability, based on several operations such as splitting, deletion, and paraphrasing. However, a valid simplified sentence should also be logically entailed by its input sentence. In this work, we first present a strong pointer-copy mechanism based sequence-to-sequence sentence simplification model, and then improve its entailment and paraphrasing capabilities via multi-task learning with related auxiliary tasks of entailment and paraphrase generation. Moreover, we propose a novel `multi-level' layered soft sharing approach where each auxiliary task shares different (higher versus lower) level layers of the sentence simplification model, depending on the task's semantic versus lexico-syntactic nature. We also introduce a novel multi-armed bandit based training approach that dynamically learns how to effectively switch across tasks during multi-task learning. Experiments on multiple popular datasets demonstrate that our model outperforms competitive simplification systems in SARI and FKGL automatic metrics, and human evaluation. Further, we present several ablation analyses on alternative layer sharing methods, soft versus hard sharing, dynamic multi-armed bandit sampling approaches, and our model's learned entailment and paraphrasing skills. 
\end{abstract}

\section{Introduction}
\label{sec-intro}

\blfootnote{
\hspace{-0.65cm}  This work is licensed under a Creative Commons Attribution 4.0 International License. License details: \url{http://creativecommons.org/licenses/by/4.0/}.
}

Sentence simplification is the task of improving the readability and understandability of an input text. This challenging task has been the subject of research interest because it can address automatic ways of improving reading aids for people with limited language skills, or language impairments such as dyslexia~\cite{rello2013impact}, autism~\cite{evans2014evaluation}, and aphasia~\cite{carroll1999simplifying}. It also has wide applications in NLP tasks as a preprocessing step, for example, to improve the performance of parsers~\cite{chandrasekar1996motivations}, summarizers~\cite{klebanov2004text}, and semantic role labelers~\cite{vickrey2008sentence,woodsend2014text}.

Several sentence simplification systems focus on operations such as splitting a long sentence into shorter sentences~\cite{SIDDHARTHAN2006SyntacticSA,petersen2007text}, deletion of less important words/phrases~\cite{knight2002summarization,clarke2006models,filippova2008dependency}, and paraphrasing~\cite{devlin1999simplifying,inui2003text,kaji2002verb}. Inspired from machine translation based neural models, recent work has built end-to-end sentence simplification models along with attention mechanism, and further improved it with reinforcement-based policy gradient approaches~\cite{zhang2017dress}. Our baseline is a novel application of the pointer-copy mechanism~\cite{see2017get} for the sentence simplification task, which allows the model to directly copy words and phrases from the input to the output. We  further improve this strong baseline by bringing in auxiliary entailment and paraphrasing knowledge via soft and dynamic multi-level, multi-task learning.\footnote{All code and pretrained models available at:~\url{https://github.com/HanGuo97/MultitaskSimplification}.} 

Apart from the three simplification operations discussed above, we also ensure that the simplified output is a directed logical entailment w.r.t. the input text, i.e., does not generate any contradictory or unrelated information. We incorporate this entailment skill via multi-task learning~\cite{luong2015multi} with an auxiliary entailment generation task. 
Further, we also induce word/phrase-level paraphrasing knowledge via a paraphrase generation task, enabling parallel learning of these three tasks in a three-way multi-task learning setup. 
We employ a novel `multi-level' layered, soft sharing approach, where the parameters between the tasks are loosely coupled at different levels of layers; we share higher-level semantic layers between the sentence simplification and entailment generation tasks (which teaches the model to generate outputs that are entailed by the full input), while sharing the lower-level lexico-syntactic layers between the sentence simplification and paraphrase generation tasks (which teaches the model to paraphrase only the smaller sub-sentence pieces). 

Finally, we also propose a multi-armed bandit approach that dynamically learns an effective schedule (curriculum) of switching between tasks for optimization during multi-task learning, instead of the traditional approach with a manually-tuned, static (fixed) mixing ratio~\cite{luong2015multi}. This dynamic approach allows us to achieve not only  equal, but in fact better results than the manual approach, while importantly avoiding the hassle of tuning on the large space of mixing ratios over several different tasks.

Empirically, we evaluate our system on three standard datasets: Newsela, WikiSmall, and WikiLarge. First, we show that our pointer-copy baseline is significantly better than sequence-to-sequence models, and competitive w.r.t. the state-of-the-art. Next, we show that our multi-level, multi-task framework performs significantly better than our strong pointer baseline and other competitive sentence simplification models on both automatic evaluation as well as on human study simplicity criterion. Further, we show that the dynamic multi-armed bandit based switching of tasks during training improves over the traditional manually-tuned static mixing ratio. Lastly, we show several ablation studies based on different layer-sharing approaches (higher versus lower) with auxiliary tasks, hard versus soft sharing, dynamic mixing ratio sampling, as well as our model's learned entailment and paraphrasing skills.
 
\section{Related Work}
\label{section:related-works}

Previous approaches to sentence simplification systems range from hand-designed rules~\cite{SIDDHARTHAN2006SyntacticSA}, to syntactic and lexical simplification via synonyms and paraphrases~\cite{Siddharthan2014ASO,kaji2002verb,horn2014learning,glavavs2015simplifying}, as well as treating simplification as a monolingual MT task, where operations are learned from examples of complex-simple sentence pairs~\cite{specia2010translating,koehn2007moses,coster2011learning,Zhu2010AMT,Wubben2012SentenceSB,Narayan2014HybridSU}. 
Recently,~\newcite{Xu2016OptimizingSM} trained a syntax-based MT model using the newly proposed SARI as a simplification-specific objective. Further,~\newcite{zhang2017dress} used reinforcement learning in a sequence-to-sequence approach to directly optimize simplification metrics. In this work, we first introduce the pointer-copy mechanism~\cite{see2017get} as a novel application to sentence simplification, and then use multi-task learning to bring in auxiliary entailment and paraphrasing skills.


Multi-task learning, known for improving the generalization performance of a task with related tasks, has successful application to many domains of machine learning~\cite{caruana1998multitask,collobert2008unified,girshick2015fast,luong2015multi,pasunuru2017multitask,Pasunuru2017TowardsIA}. Although there are many variants of multi-task learning~\cite{ruder2017sluice,hashimoto2017ajm,luong2015multi}, 
our approach is similar to~\newcite{luong2015multi}, where different tasks share some common model parameters with alternating mini-batches optimization. 
In this work, we explore a multi-level (i.e., task-specific higher-level semantic versus lower-level lexico-syntactic layer sharing) and soft-sharing mechanism for improving sentence simplification via related tasks of entailment and paraphrase generation.

Recognizing Textual Entailment (RTE) is the task of predicting entailment, contradiction, or neutral relationships, and is useful for many downstream tasks like Q\&A, summarization, and information retrieval~\cite{harabagiu2006methods,dagan2006pascal,lai2014illinois,jimenez2014unal}. Neural network models~\cite{bowman2016snli,parikh2016decomposable} and large datasets~\cite{bowman2016snli,williams2017broad} enabled recent strong progress. Recently,~\newcite{Pasunuru2017TowardsIA} and~\newcite{han2017multitask} presented results using entailment generation as an auxiliary task for abstractive summarization; however, we use entailment as well as paraphrasing knowledge in a soft and multi-level layer sharing setup to improve sentence simplification.

Previous work~\cite{barzilay2001extracting,ganitkevitch2013ppdb,Wieting2017PushingTL} has developed methods and datasets for generating paraphrase pairs which can be useful for downstream applications such as question answering, semantic parsing, and information extraction~\cite{fader2013paraphrase,berant2014semantic,zhang2015exploiting}. \newcite{Wieting2017PushingTL} recently introduced a large sentential paraphrase dataset via back-translation, and showed promising results when applied to learning sentence embeddings. In this work, we use this paraphrase dataset as an auxiliary generation task to improve our sentence simplification model by teaching it about paraphrasing in a multi-task setting.

Many control problems can be cast as a multi-armed bandits algorithm, where the goal of the agent is to select the arm/action from one of the  choices that gives the maximum expected future reward~\cite{bubeck2012regret}. Optimal control and reinforcement learning have been used to find the trade-off between exploitation and exploration, and yield theoretically-sound regret bounds, e.g., Boltzmann exploration~\cite{kaelbling1996reinforcement}, UCB~\cite{auer2002finite}, Thompson sampling~\cite{chapelle2011empirical}, adversarial bandits~\cite{auer2002nonstochastic}, and information gain using variational approaches~\cite{houthooft2016vime}. Recently, \newcite{graves2017automated} use a non-stationary multi-armed bandit to automatically select the curriculum or syllabus that a neural network follows so as to maximize learning efficiency. \newcite{sharma2017online} use multi-armed bandit sampling to choose which domain data (harder vs. easier) to feed as input to a single model (using different Atari games), whereas we use multi-armed bandit sampling to decide the optimization curriculum (mixing ratio) among our three models for sentence simplification, entailment generation, and paraphrase generation (with different softly-shared layers).




 

\section{Models}
\label{sec:models}

In this section, we first describe our sentence simplification baseline model with attention mechanism, which is further improved by pointer-copy mechanism. Later, we introduce our two auxiliary tasks (entailment and paraphrase generation) and discuss how they can share specific lower/higher-level layers/parameters to improve the sentence simplification task in a multi-task learning setting. Finally, we discuss our new multi-armed bandit based dynamic multi-task learning approach.

\subsection{Pointer-Copy Baseline Sentence Simplification Model}
Our baseline is a 2-layer sequence-to-sequence model with both attention~\cite{bahdanau2014attention} and pointer-copy mechanism~\cite{see2017get}. Given the sequence of input/source tokens , the model learns an auto-regressive distribution over output/target tokens , which is defined as , 
where  represents model parameters and  is probability of generating token  at decoder time step  given the previous generated tokens  and input . Given encoder hidden states , and decoder's  time step hidden state (of last layer) , the context vector , where the attention weights  define an attention distribution over encoder hidden states: , where .
Finally, the conditional distribution at each time step  of the decoder is defined as , where the final hidden state  is a combination of context vector  and last layer hidden state  and is defined as , where  and  are trained parameters.



\noindent\textbf{Pointer-Copy Mechanism:} This helps in directly copying the words from the source inputs to the target outputs via merging the generative distribution and attention distribution (as a proxy of copy distribution). The goal of sentence simplification is to rewrite sentences more simply, while preserving important information; hence, it also involves significant amount of copying from the source. Our pointer mechanism approach is similar to~\newcite{see2017get}. At each time step of the decoder, the model makes a (soft) choice between words from the vocabulary distribution  and attention distribution  (based on words in the input) using the word generation probability , where  is sigmoid,  and  are trainable parameters, and  is decoder input. The final vocabulary distribution is defined as the weighted combination of vocabulary and attention distributions:
\vspace{-5pt}




\subsection{Auxiliary Tasks}

\paragraph{Entailment Generation} 
The task of entailment generation is to generate a hypothesis which is entailed by the given input premise. A good simplified sentence should be entailed by (follow from) the source sentence, and hence we incorporate such knowledge through an entailment generation task into our sentence simplification task. We share the higher-level semantic layers between the two tasks (see reasoning in Sec.~\ref{subsec:multi-level-sharing-mechanism} below). We use entailment pairs from SNLI~\cite{bowman2016snli} and Multi-NLI~\cite{williams2017broad} datasets for training our entailment generation model, where we use the same architecture as our sentence simplification model.



\paragraph{Paraphrase Generation} 
Paraphrase generation is the task of generating similar meaning phrases or sentences by reordering and modifying the syntax and/or lexicon. Paraphrasing is one of the common operations used in sentence simplification, i.e, by substituting complex words and phrases with their simpler paraphrase forms. Hence, we add this knowledge to the sentence simplification task via multi-task learning, by sharing the lower-level lexico-syntactic layers between the two tasks (see reasoning in Sec.~\ref{subsec:multi-level-sharing-mechanism} below). For this, we use the paraphrase pairs from ParaNMT~\cite{Wieting2017PushingTL}. Here, again, we use the same architecture as our sentence simplification model.


\begin{figure}
\centering
\begin{minipage}[t]{.48\linewidth}
\centering
\includegraphics[width=0.98\linewidth]{Figs/multitask_model.pdf}
\vspace{-10pt}
\caption{Overview of our 3-way multi-task model. Same color and dashed connections represent soft-shared parameters in different layers.
}
\vspace{-10pt}
\label{fig:multitask}
\end{minipage}
\hfill
\begin{minipage}[t]{.48\linewidth}
\centering
\includegraphics[width=0.98\linewidth]{Figs/MAB.pdf}
\vspace{-10pt}
\caption{Overview of our multi-armed bandits algorithm for dynamic mixing ratio learning. It consists of a controller with 3 arms/tasks.
}
\vspace{-10pt}
\label{fig:mab}
\end{minipage}
\end{figure}



\subsection{Multi-Task Learning}
\label{subsec:multi-level-sharing-mechanism}
In this subsection, we discuss our multi-task, multi-level soft sharing strategy with parallel training of sentence simplification and related auxiliary tasks (entailment and paraphrase generation). 

The predominant approach for multi-task learning in sequence-to-sequence models is to directly hard-share all encoder/decoder layers/parameters~\cite{luong2015multi,johnson2016google,pasunuru2017multitask,kaiser2017one}. However, this approach places very strong constraints/priors on the primary model to compress knowledge from diverse tasks. We believe that while the auxiliary tasks considered in this work share many similarities with the primary sentence simplification task, they are still different in either lower-level or higher-level representations (e.g., entailment will deal with higher-level, full-sentence logical inference, while paraphrasing will handle the lower-level intermediate word/phrase simplifications). 
In this section, we propose to relax the priors in two ways: (1) we share the model parameters in a finer-grained scale, i.e. layer-specific sharing, by keeping some of their parameters private, while sharing related representations; and (2) we encourage shared parameters to be close in certain distance metrics with a penalty term instead of hard-parameter-tying~\cite{luong2015multi}.


\paragraph{Multi-Level Sharing Mechanism} Fig.~\ref{fig:multitask} shows our multi-task model with parallel training of three tasks: sentence simplification (primary task), entailment generation (auxiliary task), and paraphrase generation (auxiliary task). 
Recently,~\newcite{belinkov2017neural} observed that different layers in a sequence-to-sequence model (trained on translation) exhibit different functionalities: lower-layers (closer to inputs) of the encoder learn to represent word structure while higher layers (farther from inputs) are more focused on semantics and meanings (\newcite{zeiler2014visualizing} observed similar findings for convolutional image features). Based on these findings, we share the higher-level layers\footnote{We found that sharing higher-level semantic layers (farther from input/output), i.e., encoder layer 2, attention, and decoder layer 1 (in~\figref{fig:multitask}), to work well. See Sec.~\ref{sec:analysis} for ablations on alternative layer sharing methods.} between the entailment generation and sentence simplification tasks, since they share higher semantic-level language inference skills (for full sentence-to-sentence logical directedness). On the other hand, we share the lower-level lexico-syntactic layers\footnote{We found that sharing lower-level lexico-syntactic layers (closer to input/output), i.e., encoder layer 1 and decoder layer 2 (in~\figref{fig:multitask}), to work well. See Sec.~\ref{sec:analysis} for ablations on alternative layer sharing methods.} between the paraphrase generation and sentence simplification tasks, since they share more word/phrase and syntactic level paraphrasing knowledge to simplify the smaller, intermediate sentence pieces.
Sec.~\ref{sec:analysis} present empirical ablations to support our intuitive layer sharing.\footnote{Note that even though entailment just tries to generate shorter, logical-subset sub-sentences, the overall saliency and quality of the simplified output is still balanced because the entailment task is flexibly (softly) shared with the paraphrasing and sentence simplification tasks, and the final model mixture is chosen based on simplification task metrics (see output examples in Fig.~\ref{fig:examples} where our multi-task model generates entailed sentences with important information).}


\paragraph{Soft Sharing}
In multi-task learning, we can do either hard sharing or soft sharing of parameters. Hard sharing directly ties the parameters to be shared, and receives gradient information from multiple tasks. On the other hand, soft sharing only loosely couples the parameters, and encourages them to be close in representation space. Hence the soft sharing approach gives more flexibility for parameter sharing, hence allowing different tasks to choose what parts of their parameters space to share. We minimize the  distance between shared parameters as a regularization along with the cross entropy loss. Hence, the final loss function of the primary task with a related auxiliary task is defined as follows:
\vspace{-6pt}

where  represents the full parameters of the primary task (sentence simplification),  and  are the subsets of shared parameters between the primary and auxiliary task resp., and  is a hyperparameter.

\paragraph{Multi-Task Training}
We employ multi-task learning with parallel training of related tasks in alternate mini-batches based on a mixing ratio , where we alternatively optimize , ,  mini-batches of sentence simplification, entailment generation, and paraphrase generation, respectively, until all models converge. In the next section, we discuss a new approach to replace this static mixing ratio with dynamically-learned task switching.

\subsection{Dynamic Mixing Ratio Learning}
\label{subsec:active-sampling}
Current multi-task models are trained via alternate mini-batch optimization based on a task `mixing ratio'~\cite{luong2015multi,pasunuru2017multitask}, i.e., how many iterations on each task relative to other tasks (see end of Sec.~\ref{subsec:multi-level-sharing-mechanism}). This is usually treated as a very important hyperparameter to be tuned, and the search space scales exponentially with the number of tasks. Hence, we importantly replace this manually-tuned and static mixing ratio with a `dynamic' mixing ratio learning approach, where a controller automatically switches between the tasks during training, based on the current state of the multi-task model. Specifically, we use a multi-armed bandits based controller with Boltzmann exploration~\cite{kaelbling1996reinforcement} with an exponential moving average update rule. 


We view the problem of learning the right mixing of tasks as a sequential control problem, where the controller's goal is to decide the next task/action after every  training steps in each task-sampling round .\footnote{We set  to  to reduce variance of estimates, i.e., the bandit controller's task/action will be trained for  mini-batches.} Let  represent the set of 3 tasks in our multi-task setting, i.e., sentence simplification, entailment generation, and paraphrase generation. We model the controller as a -armed bandits, where it selects a sequence of actions/arms over the current training trajectory to maximize the expected future payoffs (see Fig.~\ref{fig:mab}). At each round , the controller selects an arm based on noisy value estimates and observes rewards  for the selected arm (we use the  negative validation loss of the primary task as the reward in our setup). One problem in bandits learning is the trade-off between exploration and exploitation, where the agent needs to make a decision between taking the action that yields the best payoff on current estimates, or explore new actions whose payoffs are not yet certain. For this, we use the Boltzmann exploration~\cite{kaelbling1996reinforcement} with exponentially moving action value estimates. Let  be the policy of the bandit controller at round , we define this to be:
\vspace{-9pt}

where  is the estimated action value of each arm  at round , and  is the temperature.\footnote{We tried decaying the temperature variable, but we didn't find this to very beneficial, so we instead fix this to .} If  is the initial value estimate of arm , then  is the exponentially weighted mean with the decay rate :
\vspace{-9pt}

To further help the exploration process, we follow the principle of optimism under uncertainty~\cite{sutton1998reinforcement} and set  to be above the maximum empirical rewards. Empirically, we show that this approach of `dynamic mixing ratio' is equal or better than the traditional static mixing ratio (see Table~\ref{table:dynamic-mixing-results}).
Also, we further show ablation study in Sec.~\ref{sec:analysis} to show that this switching approach is better than the alternative approach of first using multi-armed bandits for finding an optimal `final' mixing ratio and then re-training the model based on this bandits-selected mixing ratio.

 


\section{Evaluation Setup}
\label{section:setup}

\paragraph{Datasets}
\label{subsec:datasets}
We first describe the three standard sentence simplification datasets we evaluate on: Newsela, WikiSmall, and WikiLarge; next, we describe datasets for our auxiliary entailment and paraphrase generation tasks. \textbf{Newsela}~\cite{xu2015problems} is acknowledged as a higher-quality dataset for studying sentence simplifications, as opposed to Wikipedia-based datasets which automatically align complex-simple sentence pairs and have generalization issues~\cite{zhang2017dress,xu2015problems,amancio2014analysis,hwang2015aligning,vstajner2015deeper}. Newsela consists of  news articles, 
and we follow previous work~\cite{zhang2017dress} to use the first  documents for training, and  documents each for development and test.
\textbf{WikiSmall}~\cite{Zhu2010AMT} contains automatically-aligned complex-simple sentences from the ordinary-simple English Wikipedias. The data has  pairs for training and  for test. We use the -pairs validation set from~\newcite{zhang2017dress}.
\textbf{WikiLarge}~\cite{zhang2017dress} is a larger Wikipedia corpus aggregating pairs from~\newcite{Kauchak2013ImprovingTS},~\newcite{Woodsend2011LearningTS}, and WikiSmall. We use the exact training/evaluation sets provided by~\newcite{zhang2017dress}.
\textbf{SNLI and MultiNLI}: For the task of entailment generation, we use the Stanford Natural Language Inference (SNLI) corpus~\cite{bowman2016snli} and MultiNLI~\cite{williams2017broad}. We use their entailment labeled pairs for our entailment generation task, following previous work~\cite{pasunuru2017multitask}. The combined SNLI and MultiNLI dataset has  entailment pairs, out of which we use  pairs for training, and the rest are divided into validation and test sets. 
\textbf{ParaNMT}: For the task of paraphrase generation, we use the back-translated paraphrase dataset provided by~\newcite{Wieting2017PushingTL}. The filtered version of the dataset has  million pairs of paraphrases.\footnote{We chose ParaNMT over other paraphrase datasets (e.g. the phrase-to-phrase PPDB dataset~\cite{ganitkevitch2013ppdb}), because ParaNMT is a sentence-to-sentence dataset and hence is a more natural fit for sentence-level multi-task RNN-layer sharing with our sentence-to-sentence simplification task.} We use  for training, and the rest are evenly divided into validation and test sets.

\paragraph{Evaluation Metrics}
\label{subsec:evaluation}

Following previous work~\cite{zhang2017dress}, we report all the standard evaluation metrics: SARI~\cite{Xu2016OptimizingSM}, FKGL~\cite{kincaid1975fkgl}, and BLEU~\cite{Papineni2002BleuAM}. However, several studies have shown that BLEU is poorly correlated w.r.t. simplicity~\cite{Zhu2010AMT,vstajner2015deeper,Xu2016OptimizingSM}. Moreover,~\newcite{shardlow2014survey} argues that FKGL~\cite{kincaid1975fkgl}, which measures readability of simpler output (lower is better), favors very short sentences even though longer/less coarse counterparts can be simpler. Further,~\newcite{Xu2016OptimizingSM} argues that BLEU tends to favor conservative systems that do not make many changes, and proposes SARI metric which explicitly measures the quality of words that are added and deleted.
SARI is shown to correlate well with human judgment in simplicity~\cite{Xu2016OptimizingSM},
and hence we primarily focus on this metric in our models' performance analysis.\footnote{We use the JOSHUA package for calculating SARI and BLEU score following~\newcite{zhang2017dress} and~\newcite{Xu2016OptimizingSM}. Our FKGL implementation is based on~\url{https://github.com/mmautner/readability.}} 
Further, we also do human evaluation based on: Fluency (`is the output grammatical and well formed?'), Adequacy (`to what extent is the meaning expressed in the original sentence preserved in the output?') and Simplicity (`is the output simpler than the original sentence?'), following guidelines suggested by~\newcite{Xu2016OptimizingSM} and~\newcite{zhang2017dress}.

\paragraph{Training Details}
\label{subsec:training}
All our soft/hard and layer-specific sharing decisions (\secref{sec:analysis}) were made on the validation/dev set. Our model selection (tuning) criteria is based on the average of our 3 metrics (SARI, BLEU, 1/FKGL) on the validation set. Please refer to the appendix for full training details (vocabulary overlap, mixing ratios and bandit sampler decay rates and reward, WikiLarge pre-training, etc.).


 
\begin{table}[t]
\begin{minipage}[t]{.365\linewidth}
\begin{center}
\begin{small}
\begin{tabular}{|l|c|c|c|c|}
\hline
Models & BLEU & FKGL & SARI\\
\hline
\multicolumn{4}{|c|}{\textsc{Previous Work}}\\
\hline
PBMT-R & 18.19 & 7.59 & 15.77 \\
Hybrid & 14.46 & 4.01 & 30.00 \\
EncDecA & 21.70 & 5.11 & 24.12 \\
DRESS & 23.21 & 4.13 & 27.37 \\
DRESS-LS & 24.30 & 4.21 & 26.63 \\
\hline
\multicolumn{4}{|c|}{\textsc{Our Models}}\\
\hline
Baseline  & 23.72 & 3.25 & 28.31 \\
 + Ent. & 16.82 & 2.21 & 31.55 \\
 + Paraphr. & 16.29 & 2.03 & 31.71 \\
+Ent+Par & 11.86 & 1.38 & 32.98 \\
\hline
\end{tabular}
\end{small}
\end{center}
\vspace{-10pt}
\caption{Newsela (FKGL: lower is better). Note that SARI is the primary, human-correlated metric for sentence simplification~\cite{Xu2016OptimizingSM}.}
\vspace{-5pt}
\label{table:newsela_results}
\end{minipage}
\hfill
\begin{minipage}[t]{.595\linewidth}
\begin{center}
\begin{small}
\begin{tabular}{|l|c|c|c|c|c|c|c|}
\hline
& \multicolumn{3}{c|}{\textsc{WikiSmall}} & \multicolumn{3}{c|}{\textsc{WikiLarge}}\\
Models & BLEU & FKGL & SARI &  BLEU & FKGL & SARI\\
\hline
\multicolumn{7}{|c|}{\textsc{Previous Work}}\\
\hline
PBMT-R  & 46.31 & 11.42 & 15.97 & 81.11 & 8.33 & 38.56 \\
Hybrid  & 53.94 & 9.21 & 30.46 & 48.97 & 4.56 & 31.40 \\
SBMT-SARI  & - & - & -  & 73.08 & 7.29 & 39.96\\
EncDecA  & 47.93 & 11.35 & 13.61 & 88.85 & 8.41 & 35.66 \\
DRESS  & 34.53 & 7.48 & 27.48 & 77.18 & 6.58 & 37.08 \\
DRESS-LS  & 36.32 & 7.55 & 27.24 & 80.12 & 6.62 & 37.27 \\
\hline
\multicolumn{7}{|c|}{\textsc{Our Models}}\\
\hline
Baseline   & 36.18 & 7.69 & 25.67 & 82.37 & 7.84 & 36.68 \\
+Ent+Par  & 29.70 & 6.93 & 28.24 & 81.49 & 7.41 & 37.45 \\
\hline
\end{tabular}
\end{small}
\end{center}
\vspace{-8pt}
\caption{WikiSmall/Large results (FKGL: lower is better). Note that SARI is the primary, human-correlated metric for sentence simplification~\cite{Xu2016OptimizingSM}.
}
\label{table:wikilarge_results}
\end{minipage}
\vspace{-5pt}
\end{table}


\section{Results}
\label{label:results}
We evaluate our models on three datasets and via several automatic metrics plus human evaluation.\footnote{As described in Sec.~\ref{subsec:datasets}, Newsela is considered as a higher quality dataset for text simplification, and thus we report ablation-style results (e.g., 2-way multi-task models and different layer-sharing ablations) and human evaluation on Newsela (since Wikipedia datasets are automatically-aligned). Moreover, we report SARI, FKGL, and BLEU for completeness, but as described in Sec.~\ref{section:setup}, SARI is the primary human-correlated metric for sentence simplification.} 

\paragraph{Pointer Baseline}
First, we compare our pointer baseline with various previous works: PBMT-R~\cite{Wubben2012SentenceSB}, Hybrid~\cite{Narayan2014HybridSU}, SBMT-SARI~\cite{Xu2016OptimizingSM}\footnote{We borrow the SBMT-SARI results for WikiLarge from~\newcite{zhang2017dress}.}, and EncDecA, DRESS, and DRESS-LS~\cite{zhang2017dress}. As shown in Table~\ref{table:newsela_results}, our pointer baseline already achieves the best score in FKGL and the second-best score in SARI on Newsela, and also achieves overall comparable results on both WikiSmall and WikiLarge (see Table~\ref{table:wikilarge_results}).

\paragraph{Multi-Task Models}
We further improve our strong pointer-based sentence simplification baseline model by multi-task learning it with entailment and paraphrase generation. First, we show that our 2-way multi-task models with auxiliary tasks (entailment and paraphrase generation) are statistically significantly better than our pointer baseline and previous works in both SARI and FKGL on Newsela (see Table~\ref{table:newsela_results}).\footnote{Stat. significance is computed via bootstrap test~\cite{noreen1989computer,efron1994introduction}. Both our 2-way multi-task models are statistically significantly better in SARI and FKGL with  w.r.t. our pointer baseline and previous works. Note the discussion in~\secref{subsec:evaluation} about why BLEU is not a good sentence simplification metric.} 
Next, Table~\ref{table:newsela_results} and Table~\ref{table:wikilarge_results} summarize the performance of our final 3-way multi-level, multi-task models with entailment generation and paraphrase generation on all three datasets.
Here, our 3-way multi-task models are statistically significantly better than our pointer baselines in both SARI and FKGL (with ) on Newsela and WikiSmall, and in SARI () on WikiLarge.
Also, our 3-way multi-task model is statistically significantly better than the 2-way multi-task models in SARI and FKGL with  (see Table~\ref{table:newsela_results}). In Sec.~\ref{sec:analysis}, we further provide a set of detailed ablation experiments investigating the effects of different (higher-level versus lower-level) layer sharing methods and soft- vs. hard-sharing in our multi-level, multi-task models; and we show the superiority of our final choice of higher-level semantic sharing for entailment generation and lower-level lexico-syntactic sharing for paraphrase generation.


\begin{wraptable}[8]{r}{0.45\textwidth}
\begin{small}
\begin{center}
\vspace{-4pt}
\begin{tabular}{|l|c|c|c|}
\hline
Models & BLEU & FKGL & SARI \\
\hline
\multicolumn{4}{|c|}{\textsc{Newsela}}\\
\hline
Static Mixing Ratio  & 11.86 & 1.38 & 32.98 \\
Dynamic Mixing Ratio & 11.14 & 1.32  & 33.22 \\
\hline
\multicolumn{4}{|c|}{\textsc{WikiSmall}}\\
\hline
Static Mixing Ratio  & 29.70 &  6.93 & 28.24 \\
Dynamic Mixing Ratio & 27.23 &  5.86 & 29.58 \\
\hline
\end{tabular}
\end{center}
\vspace{-12pt}
\caption{Results on dynamic vs. static mixing ratio (FKGL: lower is better).}
\label{table:dynamic-mixing-results}
\end{small}
\end{wraptable}
\paragraph{Dynamic Mixing Ratio Models}
Finally, we present results on our 3-way multi-task model with the new approach of using `dynamic' mixing ratios based on multi-armed bandits sampling (see~\secref{subsec:active-sampling}). As shown in Table~\ref{table:dynamic-mixing-results}, this dynamic multi-task approach achieves a stat. significant improvement in SARI  as compared to the traditional fixed and manually-tuned mixing ratio based 3-way multi-task model: 33.22 vs. 32.98 () on Newsela, and 29.58 vs. 28.24 () on WikiSmall. Hence, this allows us to achieve not only equal, but in fact better results than the manual approach, while importantly avoiding the hassle of tuning on the large space of mixing ratios over several different tasks. In Sec.~\ref{sec:analysis}, we further provide ablation analysis to study whether the improvements come from the bandit learning this dynamic curriculum or from the bandit finding the final optimal mixing-ratio at the end of the sampling procedure (and also compare it to a random curriculum).

\begin{table}[t]
\begin{center}
\begin{small}
\begin{tabular}{|l|c|c|c|c|||c|c|c|}
\hline
 & \multicolumn{4}{c|||}{HUMAN EVALUATION} & \multicolumn{3}{c|}{MATCH-WITH-INPUT} \\
Models & Fluency & Adequacy & Simplicity & Average & BLEU (\%) & ROUGE (\%) & Exact Match (\%) \\
\hline
Ground-truth & 4.97 & 4.08 & 3.83 & 4.29 & 18.25 & 43.74 & 0.00 \\
\hdashline
Hybrid & 3.88 & 3.82 & 3.92 & 3.87 & 25.74 & 56.20 & 3.34 \\
DRESS-LS & 4.84 & 4.18 & 3.21 & 4.08 & 42.93 & 67.61 & 14.48 \\
Pointer Baseline & 4.61 & 3.94 & 3.99 & 4.18 & 30.80 & 60.56 & 10.68 \\
3-way Multi-task &  4.73 & 3.18 & 4.62 & 4.18 & 8.74 & 37.82 & 2.41 \\
\hline
\end{tabular}
\end{small}
\end{center}
\vspace{-10pt}
\caption{Human evaluation results (on left) and closeness-to-input source results (on right), for Newsela. In Sec.~\ref{label:results} `Human Evaluation', we discuss the issue of high adequacy scores for outputs that are very similar to the input (see right part of the table).
}
\vspace{-7pt}
\label{table:newsela-human-eval}
\end{table}


\paragraph{Human Evaluation}
We also perform an anonymized human study comparing our pointer baseline, our multi-task model, some previous works (Hybrid~\cite{Narayan2014HybridSU} and state-of-the-art DRESS-LS~\cite{zhang2017dress}), and ground-truth references (see left part of Table~\ref{table:newsela-human-eval}), based on fluency, adequacy, and simplicity (see Sec.~\ref{subsec:evaluation} for more details about these criteria) using 5-point Likert scale. We asked annotators to evaluate the models (randomly shuffled to anonymize model identity) based on  samples from the representative and cleaner Newsela test set, and their scores are reported in Table~\ref{table:newsela-human-eval}.
Our 3-way multi-task model achieves a significantly higher () simplicity score compared to DRESS-LS, Hybrid, and our pointer baseline models. 
However, we next observe that our 3-way multi-task model has lower adequacy score as compared to DRESS-LS and the pointer model, but this is because our 3-way multi-task model focuses more strongly on simplification, which is the goal of the given task. 
Moreover, based on the overall average score of the three human evaluation criteria, our 3-way multi-task model is also significantly better () than the state-of-the-art DRESS-LS model (and  w.r.t. Hybrid model).\footnote{Note that our multi-task model is stat. equal to our pointer baseline on the overall-average score, showing the available trade-off between systems that simplify conservatively vs. strongly, based on one's desired downstream task application. Also refer to the high `match-with-input' issue with the adequacy metric discussed next.} 
Also, on further investigation, we found that a problem with the adequacy metric is that it gets artificially high scores for output sentences which are exact match (or a very close match) with the input source sentence, i.e., they have very little simplification and hence almost fully retain the exact meaning. In the right part of Table~\ref{table:newsela-human-eval}, we analyzed the matching scores of the outputs from different models w.r.t. the source input text, based on BLEU, ROUGE~\cite{lin2004rouge} and exact match. First, this shows that the ground-truth sentence-simplification references are in fact (as expected) very different from the input source (0\% exact match, 18\% BLEU, 44\% ROUGE). Next, we find that our multi-task model also has low match-with-input scores (2\% exact match, 9\% BLEU, 38\% ROUGE), similar to the behavior of the ground-truth references. On the other hand, DRESS-LS (and pointer baseline) model is generating output sentences which are substantially closer to the input and hence is not making enough changes (14\% exact match, 43\% BLEU, 68\% ROUGE) as compared to the references (which explains their higher adequacy but lower simplicity scores).
 
\section{Ablations and Analysis}
\label{sec:analysis}

In this section, we conduct several ablation analyses to study the different layer-sharing mechanisms (higher semantic vs. lower lexico-syntactic), soft- vs. hard-sharing, two dynamic multi-armed bandit approaches, and our model's learned entailment and paraphrasing skills. We also present and analyze some output examples from several models.\footnote{Since Newsela is considered as the more representative dataset for sentence simplification with lesser noise and human quality~\cite{xu2015problems,zhang2017dress}, we conduct our ablation studies on this dataset, but we observed similar patterns on the other two datasets as well.} Note that all our soft and layer sharing decisions were strictly made on the dev/validation set (see~\secref{subsec:training}).

\begin{wraptable}[7]{R}{0.51\textwidth}
    \vspace{-5pt}
\begin{small}
\begin{tabular}{|l|c|c|c|c|}
\hline
Models & BLEU & FKGL & SARI\\
\hline
\textbf{Final (High Ent + Low PP)} & \textbf{11.86} & \textbf{1.38} & 
\textbf{32.98} \\
\hline
Both lower-layer & 11.94 & 1.47 & 31.92 \\
Both higher-layer & 12.26 & 1.38 & 32.02 \\
Swapped (Low Ent + High PP) & 21.64 & 2.97 & 29.07 \\
\hline
Hard-sharing & 13.01 & 1.38 & 32.36 \\
\hline
\end{tabular}
\end{small}
\vspace{-12pt}
\caption{Multi-task layer ablation results on Newsela.}
\label{table:ablation_results}
\end{wraptable}

\paragraph{Different Layer Sharing Approaches}
We empirically show that our final multi-level layer sharing method (i.e., higher-level semantic layer sharing with entailment generation, while lower-level lexico-syntactic layer sharing with paraphrase generation) performs better than the following alternative layer sharing methods: (1) both auxiliary tasks with high-level layer sharing, (2) both with low-level layer sharing, and (3) reverse/swapped sharing (i.e., lower-level layer sharing for entailment, and higher-level layer sharing for paraphrasing). Results in Table~\ref{table:ablation_results} show that our approach of high-level sharing for entailment generation and low-level sharing for paraphrase generation is statistically significantly better than all other alternative approaches in SARI () (and statistically better or equal in FKGL).


\paragraph{Soft- vs. Hard-Sharing}
In this work, we use soft-sharing instead of hard-sharing approach (benefits discussed in Sec.~\ref{subsec:multi-level-sharing-mechanism}) in all of our models. Table~\ref{table:ablation_results} also presents empirical results comparing soft- vs. hard-sharing on our final 3-way multi-task model, and we observe that soft-sharing is statistically significantly better than hard-sharing in SARI with .



\paragraph{Quantitative Improvements in Entailment}
We employ a state-of-the-art entailment classifier~\cite{chen2017enhanced} to calculate the entailment probabilities of output sentence being entailed by the ground-truth.\footnote{For this entailment analysis, we use ground-truth output as premise instead of input source, because: (1) entailment w.r.t. input source can give artificially high scores even when the output doesn't simplify enough and just copies the source (see the discussion in Sec.~\ref{label:results} and Table~\ref{table:newsela-human-eval}); (2) By transitivity, if output is entailed by ground-truth, which in turn is entailed by source, then output should also be entailed by source (plus, we want the output to be closer to ground-truth than to input source).}
Table~\ref{table:entailment_paraphrase_analysis_results} summaries the average entailment scores for the Hybrid, DRESS-LS, Pointer baseline, and 2-way multi-task model (with entailment generation auxiliary task), showing that the 2-way multi-task model improves in the aspect of logical entailment (), demonstrating the inference skill acquired by the simplification model via the auxiliary knowledge from the entailment generation task.



\paragraph{Quantitative Improvements in Paraphrasing}
We use the paraphrase classifier from~\newcite{wieting17revisiting} to compute the paraphrase probability score between the generated output and the input source. The results in Table~\ref{table:entailment_paraphrase_analysis_results} show that our 2-way multi-task model (with paraphrasing generation auxiliary task) is closer to the ground-truth in terms of the amount of paraphrasing (w.r.t. input) required by the sentence-simplification task, while the pointer baseline and previous models have higher scores due to higher amount of copying from input source (see `Match-with-Input' discussion in Sec.~\ref{label:results}, Table~\ref{table:newsela-human-eval}).


\paragraph{Addition/Deletion Operations}
We also measured the performance of the various models in terms of the addition and deletion operations using SARI's sub-operation scores computed w.r.t. both the ground-truth and source~\cite{Xu2016OptimizingSM}. Table~\ref{table:individual_scores_results} shows that our multi-task model is equal or better in terms of both operations. 

\begin{table}[t]
\begin{minipage}[t]{.55\linewidth}
\begin{small}
\begin{center}
\begin{tabular}{|l|c|c|}
\hline
Models & Entailment & Paraphrasing \\
\hline
Ground-truth & N/A & 62.1 \\
\hdashline
Hybrid &  34.8 & 74.1 \\
DRESS-LS &  30.7 & 77.9 \\
Pointer Baseline &  36.9 &  76.6 \\
2-way Multi-Task & 41.4 & 63.9 \\
\hline
\end{tabular}
\end{center}
\vspace{-10pt}
\caption{Analysis: Entailment and paraphrase classification results (avg. probability scores as \%) on Newsela.}
\label{table:entailment_paraphrase_analysis_results}
\end{small}
\end{minipage}
\hfill
\begin{minipage}[t]{.41\linewidth}
\begin{small}
\begin{center}
\begin{tabular}{|l|c|c|}
\hline
Models & Deletions & Additions \\
\hline
Hybrid &  95.18 & 0.000 \\
DRESS-LS &  85.37 & 0.047 \\
Pointer Baseline &  88.91 &  0.026 \\
3-way Multi-Task & 97.54 & 0.049 \\
\hline
\end{tabular}
\end{center}
\vspace{-10pt}
\caption{Analysis: SARI's sub-operation scores on Newsela dataset.}
\label{table:individual_scores_results}
\end{small}
\end{minipage}
\vspace{-7pt}
\end{table}


\paragraph{Two Multi-Armed-Bandit Approaches}
As described in Sec.~\ref{subsec:active-sampling}, our multi-armed bandit approach with dynamic mixing ratio during multi-task training learns a sufficiently good curriculum to improve the sentence simplification task (see Sec.~\ref{label:results}). Here, we further show an ablation study on another alternative approach of using multi-armed bandits, where we record the last  of the actions from the 
\begin{wrapfigure}[14]{r}{0.45\textwidth}
  \centering
  \vspace{0pt}
  \includegraphics[width=0.45\textwidth]{Figs/bandits.pdf}
  \vspace{-26pt}
  \caption{Task selection probability over training trajectory, predicted by bandit controller.}
	\label{fig:mab-visualization}
\end{wrapfigure}
bandit controller\footnote{We choose the last  to avoid the noisy action-value estimates at the start of the training.}, then calculate the corresponding mixing ratio based on this 10\%, and run another independent model from scratch with this fixed mixing ratio. We found that the curriculum-style dynamic switching of tasks is in fact very effective as compared to this other 2-stage approach ( versus  in SARI with ). This is intuitive because the dynamic switching of tasks during multi-task training allows the model to choose the best next task to run based on the current state (as well as the previous curriculum path) of the model, as opposed to a fixed/static single mixing ratio for the full training period. 
In Fig.~\ref{fig:mab-visualization}, we visualize the (moving averages of) probabilities of selecting each task, which shows that in the 0-1000 \#rounds range, the bandit initially gives higher weight to the main task, but gradually redistributes the probabilities to the auxiliary tasks; and beyond 1000 \#rounds, it then alternates switching among the three different tasks periodically. 
We also experimented with replacing the bandit controller with random task choices, and our bandit-controller achieves statistically significantly better results than this approach in both SARI and FKGL with , which shows that the path learned by the bandit controller is meaningful.


\begin{wrapfigure}[14]{R}{0.57\textwidth}
    \vspace{-16pt}
\begin{small}
\begin{tabularx}{\linewidth}{|X|l|}
\hline
\textbf{Input:} \emph{he put henson in charge of escorting his slaves to his brother 's kentucky plantation .} \\
\hdashline
\textbf{Reference:} \emph{he sent henson to take his slaves to kentucky .}\\
\textbf{DRESS-LS:} \emph{he put henson in charge of escorting his slaves to his brother 's kentucky plantation .} \\
\textbf{Baseline:} \emph{he put his slaves to his brother 's kentucky plantation .} \\
\textbf{Multi-Task:} \emph{he put henson in charge of escorting .}\\
\hline
\hline
\textbf{Input:} \emph{northern states did not allow slavery , but escaped slaves were returned to their owners as property , so henson would have to flee to canada to be free .} \\
\hdashline
\textbf{Reference:} \emph{states in the north did not allow slavery .}\\
\textbf{DRESS-LS:} \emph{southern states did not allow slavery , but the guatemalans were returned to their owners as property .} \\
\textbf{Baseline:} \emph{he slaves were returned to their owners as property .} \\
\textbf{Multi-Task:} \emph{northern states did not allow slavery .}\\
\hline
\end{tabularx}
 \vspace{-12pt}
\caption{Output examples comparing DRESS-LS, our pointer baseline, and multi-task model.}
\label{fig:examples}
\vspace{-5pt}
\end{small}
\end{wrapfigure}
\paragraph{Multi-Task Learning vs. Data Augmentation}
To verify that our improvements come indeed from the auxiliary tasks' specific character/capabilities and not just due to adding more data, we separately trained word embeddings on each auxiliary dataset (i.e., SNLI+MultiNLI and ParaNMT) and incorporated them into the primary simplification model. We found that both our 2-way multi-task models perform stat. significantly better than these models (which use the auxiliary word-embeddings), suggesting that merely adding more data is not enough. Moreover, Table~\ref{table:ablation_results} shows that only specific intuitive (syntactic vs. semantic) layer sharing between the primary and auxiliary tasks helps results and not just adding data.


\paragraph{Output Examples}
Fig.~\ref{fig:examples} shows two output examples comparing DRESS-LS, pointer baseline, and multi-task models (and reference). We see that our multi-task model simplifies the input appropriately (similar extent to reference) while also keeping reasonably important information from the source. The pointer baseline and the DRESS-LS models simplify to a lesser extent and keep much more of the original input (as also suggested by our match-with-input investigation in Table~\ref{table:newsela-human-eval}). 
 
\section{Conclusion}
We presented a multi-level, multi-task learning approach to incorporate natural language inference and paraphrasing knowledge into sentence simplification models, via soft sharing at higher-level semantic and lower-level lexico-syntactic levels. We also introduced a multi-armed bandits approach for learning a dynamic mixing ratio of tasks. We demonstrated strong simplification improvements on three standard datasets via automatic and human evaluation, and also discussed several ablation and analysis studies. 
 
\section*{Acknowledgments}
We thank the reviewers for their helpful comments (and Xingxing Zhang for providing preprocessed datasets). This work was supported by DARPA (YFA17-D17AP00022), Google Faculty Research Award, Bloomberg Data Science Research Grant, and Nvidia GPU awards. The views contained in this article are those of the authors and not of the funding agency.

\bibliographystyle{acl}
\bibliography{citations}


\appendix
\section{Appendix}
\subsection{Training Details}
All LSTMs use hidden state size of . We train word vectors with embedding size of  with random initialization. 
We use gradient clipped norm of . Our model selection (tuning) criteria is based on the average of our 3 metrics (SARI, BLEU, 1/FKGL) on the validation set. The mixing ratios are  for Newsela,  for WikiSmall, and  for WikiLarge. The soft-sharing coefficient  is set such that we balance the cross-entropy and regularization losses (at convergence),
which is  for Newsela,  WikiSmall, and  for WikiLarge. We train models from scratch for Newsela and WikiSmall (using Adam~\cite{Kingma2014AdamAM} optimizer with learning rate of  and , respectively). However, because of the large size and computation overhead for WikiLarge, we first pre-train both main and auxiliary models on their own domain until they reach  convergence, and use these models to initialize the multi-task models, and set the learning rate to  of its original default value (). We set the decay rate  in the bandit controller to be . We use the  negative validation loss as the reward at each sampling step to the bandit algorithm. The validation loss is divided by two as a smoothing technique.\footnote{This constant serves the same purpose as the temperature variable in the softmax function.} All our soft/hard and layer-specific sharing decisions (\secref{sec:analysis}) were made on the validation/dev set.
We follow previous work~\cite{zhang2017dress} in their pre-processing and post-processing of named entities.
We capped vocabulary size to be  and replaced less frequent words with  token.\footnote{We measured the vocabulary overlap between the main and auxiliary tasks, and found that ``word-form-overlap'' (percentage of unique word types in auxiliary task that also appear in the main task) to be  (entailment) and  (paraphrase), and ``word-count-overlap'' (percentage of words in auxiliary task that also appear in the main task, based on token frequency counts) to be  (entailment) and  (paraphrase). Hence, this suggests that only rare words (which make up for very few counts) aren't considered in training process, and our pointer mechanism handles these extra UNK words by copying the actual word-form from the source to the output.} Unlike previous work~\cite{zhang2017dress}, we do not use -replacement at test time, but instead rely on our pointer-copy mechanism. We use beam search with beam size of . All other details provided in our released code. 
\end{document}