\pdfoutput=1
\documentclass[preprint, 12pt]{elsarticle}

\usepackage{my_preamble}
\usepackage{pdfpages}
\usepackage{hyperref}
\journal{Theoretical Computer Science}

\usepackage{lipsum}
\makeatletter
\def\ps@pprintTitle{\let\@oddhead\@empty
 \let\@evenhead\@empty
 \def\@oddfoot{\footnotesize \copyright 2018. \textit{This manuscript version is made available under the \href{http://creativecommons.org/licenses/by-nc-nd/4.0/}{CC-BY-NC-ND 4.0 license.}}
}\let\@evenfoot\@oddfoot}
\makeatother

\begin{document}
\begin{frontmatter}
\title{On the Gap Between Separating Words and Separating Their Reversals}

	\author[farzam]{Farzam Ebrahimnejad}
	\ead{febrahimnejad@ce.sharif.edu}

	\address[farzam]{Department of Computer Engineering, Sharif University of Technology, Tehran, Iran}
	\begin{abstract}
	A deterministic finite automaton (DFA) separates two strings  and  
if it accepts  and rejects .
The minimum number of states required 
for a DFA to separate  and  is denoted by .
The present paper shows that the difference  
is unbounded for a binary alphabet; 
here  stands for the mirror image of . 
This solves an open problem stated in 
[Demaine, Eisenstat, Shallit, Wilson: Remarks on separating words. 
DCFS 2011. LNCS vol. 6808, pp. 147-157.]
	\end{abstract}
	\begin{keyword}
		Words separation \sep Finite automata
	\end{keyword}


\end{frontmatter}

\bibliographystyle{unsrtnat}
	
\section{Introduction}
	
	In \citeyear{goralvcik1986discerning2}, \citet*{goralvcik1986discerning2} introduced the \textit{separating words problem}. Given two distinct strings  and , we define  to be the number of states in the smallest deterministic finite automaton (DFA) that accepts  and rejects  \cite{demaine2011remarks2}. This problem asks for good upper and lower bounds on 
	\citet{goralvcik1986discerning2} proved . Besides, the best known upper bound so far is , which was obtained by \citet*{Robson89,Robson96}. A recent paper by \citet*{demaine2011remarks2} surveys the latest results about this problem, and while proving several new theorems, it also introduces three new open problems, all of which have remained unsolved until now. In this paper, we solve the first open problem stated in that paper, which asks whether 
	
	is bounded or not. We prove that this difference is actually unbounded. In order to do so, in Theorem \ref{th:main} in subsection \ref{subsec:five}, for all positive integers  and , we will construct two strings 	
	
	for some , such that  approaches infinity as  and  approach infinity. As we will later see in Lemma \ref{lem:24} in subsection \ref{subsec:four}, under certain conditions, we can set  so that it requires relatively few states to separate . But while preserving these conditions, by using the function  and the regular language , which we will introduce in subsections \ref{subsec:two} and \ref{subsec:three}, respectively, we can set  so that it will require exponentially more states, with respect to , to separate  and . We will see how exactly to do so in the rest of the paper.	
\section{Results}	
	\subsection{Preliminaries}
	We assume the reader is familiar with the basic concepts and terminology of automata theory as in, for example, \cite{Hopcroft}. In this subsection, we present some definitions and notation, and prove a few simple lemmas which will be used in the subsequent subsections.
	
	In this paper, we let  denote the set of natural numbers, excluding .
	
	\begin{definition}
	We denote a DFA  by a -tuple , where  is the set of states of ,  is the alphabet that  is defined over,  is the transition function,  is the start state, and  is the set of accept states of .
	\begin{itemize}			
		\item
			For a state  and a string , we define  to be the state in  at which we end if we start reading  from .
			Also, we define . We say that  accepts  if , and otherwise we say that it rejects .  Moreover, for a subset of states  and a language , we define
			
			Finally, we define .
		\item
			For a positive integer , we define  to be the set of all DFAs  defined over , where . Clearly,  is finite.
		\item 	
			In this paper, we assume , unless stated otherwise explicitly.
			
		\end{itemize}		
	\end{definition}
		
	\begin{definition}
		Given a DFA  and two distinct strings , we say   \textit{separates} two strings  and , if it accepts  but rejects . Now we can define  as the minimum number of states required for a DFA to \separate  and .
		Also, we say that  \textit{distinguishes}  and  if .
	\end{definition}
	Notice that if a DFA \separates two strings, then it must also \distinguish them. The following simple lemma shows that a stronger connection exists between these two definitions.
	\begin{lemma}
		\label{lem:3}
		For any two arbitrary strings , if a DFA  \distinguishes  and , then .
	\end{lemma}
	\import{proofs/}{lem_3.tex}

	
	
	The following lemma shows that adding the same prefix and suffix to two distinct strings will not make it easier to separate them. 
	\begin{lemma}
		\label{lem:4}
		For any four strings  such that , we have .
	\end{lemma}
	\import{proofs/}{lem_4.tex}
	
	The next observation will be used several times throughout this paper, both in Lemma \ref{lem:9} and Theorem \ref{th:main}.
	\begin{proposition}
		\label{prp:five}
		Let  be a regular language. If , then  for every positive integer .
	\end{proposition}	
	
	Now let us consider the transitions on symbol . The following definition and proposition help us in the proof of Lemma \ref{lem:9} in the next subsection.
	\begin{definition}
		Assume  is a DFA over . For a state , 
		we say  is in a \textit{zero-cycle}, if there exists some positive integer  such that . We call the minimum such  the length of this cycle.  
		 
	
		 Also, for a non-negative integer , we define 
		 
		  Finally, we denote  by .

	\end{definition}
	
	Notice that if a state  is in a \zcycle,
	then for every  with , the state  is also in a \zcycle.
	Using this fact, we get the following observation.
	
	\begin{proposition}
		\label{prp:one}
		 Let  be a DFA and  be a positive integer. For convenience, we will drop the subscript  from . Then
		 \begin{enumerate}[label=(\alph*)]
		 	\item 
		 	\label{st:a}
		 		 and .
		 	\item  
		 	\label{st:b}
		 		If  is not in a \zcycle, then .
		 	\item 
		 	\label{st:c}
		 		.
		 \end{enumerate}
	\end{proposition}
	\import{proofs/}{prop_one.tex}
		
	
	\subsection{The Strings  and , and the Function }	
	\label{subsec:two}
	As explained in the Introduction section, our goal is to find some strings  and , so that by setting  and ,  becomes arbitrarily greater than . The purpose of this subsection is to help us set  and  so that  becomes large enough. Actually, it is not hard to show that . By Lemma \ref{lem:4}, it follows that regardless of what  and  are, the values  and  are at least . In Lemma \ref{lem:9}, we show that we can set  and  so that . However, this lemma does not guarantee a low value for , and so Lemma \ref{lem:9} alone does not solve the problem. But still, it plays a crucial role in the proof of Theorem \ref{th:main} in subsection \ref{subsec:five}, and in the next subsections, we will see how to fix this issue.

\begin{definition}
	Since  and  are used frequently throughout this paper, from now on, for convenience, we denote them by  and , respectively.
\end{definition}

	
	\begin{lemma}
		\label{lem:9}
		For all  and , there exists  such that . We denote the  corresponding to  by .
	\end{lemma}
	\import{proofs/}{lem_9.tex}
	
	
	\subsection{The Regular Language }
	\label{subsec:three}
	In this subsection, we introduce the regular language , which has some interesting characteristics. For all ,
	there exists a DFA with  states that accepts , while no DFA with less than  states accepts . 
	Similar regular languages that also have these two characteristics
 	have been defined before \cite{reverseGaoKY12a, reverseJiraskova08, reverseSebej10} but are not quite appropriate for our purposes. Another characteristic of  is that, as proven later in Lemma \ref{lem:17}, there exists  such that if a DFA with less than  states accepts , then it should also accept some string in . This, together with Lemma \ref{lem:9}, helps us construct the desired strings in Theorem \ref{th:main}. Recall that  denotes the set of positive integers.

	\begin{definition}
		 For every positive integer , we define languages  and  over  as follows:
		
		Finally, we define 
	\end{definition}
	
	\begin{lemma}
		\label{lem:11}
		For all , we have  if and only if .	
	\end{lemma}
	\import{proofs/}{lem_11.tex}
	
	\begin{definition}
		For a regular language , we define , or the state complexity of , to be the minimum number of states required for a DFA to accept . This concept has been studied for a long time; see, for example, \cite{maslov1970estimates, YuZS92, YuZS94}.
	\end{definition}

	
	\begin{lemma}
		\label{lem:13}
		For all integers , we have .
	\end{lemma}
	\import{proofs/}{lem_13.tex}
		
	\begin{lemma}
		\label{lem:14}
		For all integers , we have .
	\end{lemma}
	\import{proofs/}{lem_14.tex}
	

	
	\begin{definition}
		For  and a language  over , we define  as the minimum number of states of a DFA that accepts  and rejects all .
	\end{definition}
	
	\begin{definition}
		Since the set 
		 is referred to several times in the rest of this paper, for simplicity, we will denote it by .
\end{definition}
	
	
	\begin{lemma}
		\label{lem:17}
		There exists  such that .
	\end{lemma}
	\import{proofs/}{lem_17.tex}
	
	\begin{definition}
		The set 
		 is referred to several times in the rest of this paper. So, for simplicity, we will denote it by .
	\end{definition}
	
	\begin{remark}
		\label{rem:spider}
		For any two DFAs  and , some DFA  exists such that .	
	\end{remark}

	
	\begin{lemma}
		\label{lem:19}
		For every two DFAs , and every string , there exists  such that  and .
	\end{lemma}
	\import{proofs/}{lem_19.tex}
	

	\begin{lemma}
		\label{lem:20}
		Let . For any two DFAs , there exists some  such that  and , or in other words, neither  nor  \distinguishes  and .
	\end{lemma}
	\import{proofs/}{lem_20.tex}
	
	\begin{proposition}
		\label{prp:four}
		Let  be a DFA in , , and . There exists some  such that  and . 
	\end{proposition}
	\import{proofs/}{prp_four.tex}
				

	
	\subsection{Mapping  to }
	\label{subsec:four}
	The previous lemmas may help us to construct two strings in  with our desired characteristics. But our goal is to prove our result for an alphabet of size . To be able to construct the intended strings over , in this subsection we introduce the function  that maps strings in  to strings in , while preserving some of our desired characteristics in them.
	
	\begin{definition}
		For a string , we define  to be the string obtained from  by replacing all occurrences of  by  and all occurrences of  by . Clearly we have 		
	\end{definition}
	
	The following lemma shows that when two strings are mapped under , separating them would be at least as hard as separating the original ones.
	\begin{lemma}
		\label{lem:23}
		For all pairs of distinct strings , we have 
		
	\end{lemma}
	\import{proofs/}{lem_23.tex}
	
	\begin{lemma}
		\label{lem:24}
		Let  and  be a regular language such that .
		Also, let .
		For all , we have
		
		Recall that  and  .
	\end{lemma}
	\import{proofs/}{lem_24.tex}
	
	\subsection{The Main Result}
	\label{subsec:five}
	Now we are ready to prove our main result. As shown in Theorem \ref{th:final}, by substituting the appropriate values for  and  in Theorem \ref{th:main}, we can prove that the difference  is unbounded.
	
	
	\begin{theorem}
		\label{th:main}
		For all , there exist two unequal strings  such that 
		
		but 
		 
	\end{theorem}
	\import{proofs/}{theorem1.tex}
	
	
	
	\begin{theorem}
		\label{th:final}
		The difference 
		 is unbounded for an alphabet of size at least . 
	\end{theorem}
	\import{proofs/}{th_final.tex}



	\section{Conclusion}
	In this paper, we proved that the difference  can be unbounded. However, it remains open to determine whether there is a good upper bound on .

	\section*{Acknowledgments}
I wish to thank Jeffrey Shallit, Mohammad Izadi, Arseny Shur, MohammadTaghi Hajiaghayi, Keivan Alizadeh, Hooman Hashemi, Hadi Khodabandeh, and Mobin Yahyazadeh, who helped me write this paper. I would also like to thank the anonymous referees for their careful reading of this paper, and for their valuable comments and suggestions. 
	\bibliography{references/references}
\end{document}