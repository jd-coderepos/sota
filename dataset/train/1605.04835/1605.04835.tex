\pdfoutput=1
\documentclass[preprint, 12pt]{elsarticle}

\usepackage{my_preamble}
\usepackage{pdfpages}
\usepackage{hyperref}
\journal{Theoretical Computer Science}

\usepackage{lipsum}
\makeatletter
\def\ps@pprintTitle{\let\@oddhead\@empty
 \let\@evenhead\@empty
 \def\@oddfoot{\footnotesize \copyright 2018. \textit{This manuscript version is made available under the \href{http://creativecommons.org/licenses/by-nc-nd/4.0/}{CC-BY-NC-ND 4.0 license.}}
}\let\@evenfoot\@oddfoot}
\makeatother

\begin{document}
\begin{frontmatter}
\title{On the Gap Between Separating Words and Separating Their Reversals}

	\author[farzam]{Farzam Ebrahimnejad}
	\ead{febrahimnejad@ce.sharif.edu}

	\address[farzam]{Department of Computer Engineering, Sharif University of Technology, Tehran, Iran}
	\begin{abstract}
	A deterministic finite automaton (DFA) separates two strings $w$ and $x$ 
if it accepts $w$ and rejects $x$.
The minimum number of states required 
for a DFA to separate $w$ and $x$ is denoted by $\ssep(w,x)$.
The present paper shows that the difference $\left|\ssep(w,x)-\ssep(w^R,x^R)\right|$ 
is unbounded for a binary alphabet; 
here $w^R$ stands for the mirror image of $w$. 
This solves an open problem stated in 
[Demaine, Eisenstat, Shallit, Wilson: Remarks on separating words. 
DCFS 2011. LNCS vol. 6808, pp. 147-157.]
	\end{abstract}
	\begin{keyword}
		Words separation \sep Finite automata
	\end{keyword}


\end{frontmatter}

\bibliographystyle{unsrtnat}
	
\section{Introduction}
	
	In \citeyear{goralvcik1986discerning2}, \citet*{goralvcik1986discerning2} introduced the \textit{separating words problem}. Given two distinct strings $w$ and $x$, we define $\ssep(w,x)$ to be the number of states in the smallest deterministic finite automaton (DFA) that accepts $w$ and rejects $x$ \cite{demaine2011remarks2}. This problem asks for good upper and lower bounds on $$S(n) \coloneqq \max_{w \neq x \land \size{w}, \size{x} \leq n} \ssep(w, x).$$
	\citet{goralvcik1986discerning2} proved $S(n) = o(n)$. Besides, the best known upper bound so far is $O(n^{2/5} \left(\log n\right)^{3/5})$, which was obtained by \citet*{Robson89,Robson96}. A recent paper by \citet*{demaine2011remarks2} surveys the latest results about this problem, and while proving several new theorems, it also introduces three new open problems, all of which have remained unsolved until now. In this paper, we solve the first open problem stated in that paper, which asks whether 
	$$\size{\ssep(w, x) - \ssep(w^R, x^R)}$$
	is bounded or not. We prove that this difference is actually unbounded. In order to do so, in Theorem \ref{th:main} in subsection \ref{subsec:five}, for all positive integers $k$ and $n$, we will construct two strings 	
	$$w = u 0^n v, x = u 0^{n + (2n+1)!} v,$$
	for some $u, v \in \Sets{01, 11}^+ \left( 0^+ \Sets{01, 11}^+\right)^*$, such that $\ssep(w, x) - \ssep(w^R, x^R)$ approaches infinity as $k$ and $n$ approach infinity. As we will later see in Lemma \ref{lem:24} in subsection \ref{subsec:four}, under certain conditions, we can set $u, v$ so that it requires relatively few states to separate $w^R, x^R$. But while preserving these conditions, by using the function $C_n$ and the regular language $G_k$, which we will introduce in subsections \ref{subsec:two} and \ref{subsec:three}, respectively, we can set $u,v$ so that it will require exponentially more states, with respect to $k$, to separate $w$ and $x$. We will see how exactly to do so in the rest of the paper.	
\section{Results}	
	\subsection{Preliminaries}
	We assume the reader is familiar with the basic concepts and terminology of automata theory as in, for example, \cite{Hopcroft}. In this subsection, we present some definitions and notation, and prove a few simple lemmas which will be used in the subsequent subsections.
	
	In this paper, we let $\N$ denote the set of natural numbers, excluding $0$.
	
	\begin{definition}
	We denote a DFA $D$ by a $5$-tuple $(Q_D,\Sigma,\delta_D,q_0,F_D)$, where $Q_D$ is the set of states of $D$, $\Sigma$ is the alphabet that $D$ is defined over, $\delta_D$ is the transition function, $q_0 \in Q_D$ is the start state, and $F_D \subseteq Q_D$ is the set of accept states of $D$.
	\begin{itemize}			
		\item
			For a state $q \in Q_D$ and a string $w \in \Sigma^*$, we define $\delta_D(q, w)$ to be the state in $Q_D$ at which we end if we start reading $w$ from $q$.
			Also, we define $\delta_D(w) \coloneqq \delta_D(q_0,w)$. We say that $D$ accepts $w$ if $\delta_D(w) \in F_D$, and otherwise we say that it rejects $w$.  Moreover, for a subset of states $S \subseteq Q_D$ and a language $L \subseteq \Sigma^*$, we define
			$$\delta_D(S, L) \coloneqq \Set{q' \in Q_D}{\exists q \in S, x \in L : q' = \delta_D(q, x)}.$$
			Finally, we define $\delta_D(q, L) \coloneqq \delta_D(\Sets{q}, L)$.
		\item
			For a positive integer $i$, we define $M_i$ to be the set of all DFAs $E$ defined over $\Sets{0,1,2}$, where $\size{Q_E} \leq i$. Clearly, $M_i$ is finite.
		\item 	
			In this paper, we assume $\Sigma = \Sets{0, 1, 2}$, unless stated otherwise explicitly.
			
		\end{itemize}		
	\end{definition}
		
	\begin{definition}
		Given a DFA $D$ and two distinct strings $w,x \in \Sigma^*$, we say  $D$ \textit{separates} two strings $w$ and $x$, if it accepts $w$ but rejects $x$. Now we can define $\ssep(w,x)$ as the minimum number of states required for a DFA to \separate $w$ and $x$.
		Also, we say that $D$ \textit{distinguishes} $w$ and $x$ if $\delta_D(w) \neq \delta_D(x)$.
	\end{definition}
	Notice that if a DFA \separates two strings, then it must also \distinguish them. The following simple lemma shows that a stronger connection exists between these two definitions.
	\begin{lemma}
		\label{lem:3}
		For any two arbitrary strings $w,x \in \Sigma^*$, if a DFA $D$ \distinguishes $w$ and $x$, then $\ssep(w,x) \le \size{Q_D}$.
	\end{lemma}
	\import{proofs/}{lem_3.tex}

	
	
	The following lemma shows that adding the same prefix and suffix to two distinct strings will not make it easier to separate them. 
	\begin{lemma}
		\label{lem:4}
		For any four strings $w,x,u,v \in \Sigma^*$ such that $w \neq x$, we have $\ssep(uwv, uxv) \geq \ssep(w, x)$.
	\end{lemma}
	\import{proofs/}{lem_4.tex}
	
	The next observation will be used several times throughout this paper, both in Lemma \ref{lem:9} and Theorem \ref{th:main}.
	\begin{proposition}
		\label{prp:five}
		Let $R$ be a regular language. If $x, y \in R\left(0^+R\right)^*$, then $x 0^j y \in R\left(0^+R\right)^*$ for every positive integer $j$.
	\end{proposition}	
	
	Now let us consider the transitions on symbol $0$. The following definition and proposition help us in the proof of Lemma \ref{lem:9} in the next subsection.
	\begin{definition}
		Assume $D$ is a DFA over $\Sets{0,1,2}$. For a state $q \in Q_D$, 
		we say $q$ is in a \textit{zero-cycle}, if there exists some positive integer $i$ such that $\delta_D(q, 0^i)=q$. We call the minimum such $i$ the length of this cycle.  
		 
	
		 Also, for a non-negative integer $i$, we define 
		 $$\zpath_D(q,i) \coloneqq \Set{p=\delta_D(q,0^j)}{0 \le j \le i \text{ and } p \text{ is not in a zero-cycle}}.$$
		  Finally, we denote $\zpath_D(q, \size{Q_D})$ by $\zpath_D(q)$.

	\end{definition}
	
	Notice that if a state $\delta_D(q,0^i)$ is in a \zcycle,
	then for every $j$ with $j>i$, the state $\delta_D(q,0^j)$ is also in a \zcycle.
	Using this fact, we get the following observation.
	
	\begin{proposition}
		\label{prp:one}
		 Let $D = (Q,\Sigma,\delta,q_0,F)$ be a DFA and $i$ be a positive integer. For convenience, we will drop the subscript $D$ from $\zpath_D$. Then
		 \begin{enumerate}[label=(\alph*)]
		 	\item 
		 	\label{st:a}
		 		$\size{\zpath(q,i)} \le i+1$ and $\size{\zpath(q,i)} \le \size{\zpath(q)}$.
		 	\item  
		 	\label{st:b}
		 		If $\delta(q,0^i)$ is not in a \zcycle, then $\size{\zpath(q,i)} = i+1 \le \size{Q}$.
		 	\item 
		 	\label{st:c}
		 		$\size{\zpath(q)} = \size{\zpath(q,i-1)} + |\size{\zpath(\delta(q,0^i))}$.
		 \end{enumerate}
	\end{proposition}
	\import{proofs/}{prop_one.tex}
		
	
	\subsection{The Strings $f_n$ and $g_n$, and the Function $C_n$}	
	\label{subsec:two}
	As explained in the Introduction section, our goal is to find some strings $u$ and $v$, so that by setting $w = u 0^n v$ and $x = u 0^{n+(2n+1)!} v$, $\ssep(w,x)$ becomes arbitrarily greater than $\ssep(w^R, x^R)$. The purpose of this subsection is to help us set $u$ and $v$ so that $\ssep(w,x)$ becomes large enough. Actually, it is not hard to show that $\ssep(0^n, 0^{n+(2n+1)!}) = n + 2$. By Lemma \ref{lem:4}, it follows that regardless of what $u$ and $v$ are, the values $\ssep(w,x)$ and $\ssep(w^R,x^R)$ are at least $n+2$. In Lemma \ref{lem:9}, we show that we can set $u$ and $v$ so that $\ssep(w,x) \geq 2n+2$. However, this lemma does not guarantee a low value for $\ssep(w^R, x^R)$, and so Lemma \ref{lem:9} alone does not solve the problem. But still, it plays a crucial role in the proof of Theorem \ref{th:main} in subsection \ref{subsec:five}, and in the next subsections, we will see how to fix this issue.

\begin{definition}
	Since $0^n$ and $0^{n+(2n+1)!}$ are used frequently throughout this paper, from now on, for convenience, we denote them by $f_n$ and $g_n$, respectively.
\end{definition}

	
	\begin{lemma}
		\label{lem:9}
		For all $n \in \N$ and $w_0 \in \Sigma^+$, there exists $w \in w_0{\left( 0^+ w_0 \right)}^*$ such that $\ssep(wf_nw, wg_nw) \geq 2n + 2$. We denote the $w$ corresponding to $w_0$ by $C_n(w_0)$.
	\end{lemma}
	\import{proofs/}{lem_9.tex}
	
	
	\subsection{The Regular Language $G_k$}
	\label{subsec:three}
	In this subsection, we introduce the regular language $G_k \subseteq \Sets{1,2}^*$, which has some interesting characteristics. For all $k \in \N$,
	there exists a DFA with $O(k)$ states that accepts $G_k^R$, while no DFA with less than $2^k$ states accepts $G_k$. 
	Similar regular languages that also have these two characteristics
 	have been defined before \cite{reverseGaoKY12a, reverseJiraskova08, reverseSebej10} but are not quite appropriate for our purposes. Another characteristic of $G_k$ is that, as proven later in Lemma \ref{lem:17}, there exists $z_k \in G_k$ such that if a DFA with less than $2^k$ states accepts $z_k$, then it should also accept some string in $\Sets{1,2}^* - G_k$. This, together with Lemma \ref{lem:9}, helps us construct the desired strings in Theorem \ref{th:main}. Recall that $\N$ denotes the set of positive integers.

	\begin{definition}
		 For every positive integer $k$, we define languages $L_k$ and $G_k$ over $\Sets{1,2}$ as follows:
		\begin{equation*}
		 	\begin{aligned}
		 		L_k \coloneqq& \Set{1^{2i}2}{i \in \N \wedge i \leq k}
				\\
				\cup& \left\{1^{i_1}21^{i_2}2   \cdots 21^{i_{s - 1}}21^{i_s}2\right. &&\mid{} s, i_1, i_2, \ldots, i_s \in  \N \\
				&  &&\wedge{} i_1 + i_2 + \cdots + i_s = 2k + 1 \\
				& &&\wedge{} i_1, i_2, \ldots, i_{s - 1} \equiv 0  \left. \pmod{2} \right\}.
		 	\end{aligned}
		\end{equation*}
		Finally, we define $G_k \coloneqq L_k^*$
	\end{definition}
	
	\begin{lemma}
		\label{lem:11}
		For all $u, v \in \Sigma^*$, we have $u1^{2k+1}2v \in G_k$ if and only if $u,v \in G_k$.	
	\end{lemma}
	\import{proofs/}{lem_11.tex}
	
	\begin{definition}
		For a regular language $L \subseteq \Sigma^*$, we define $\stc(L)$, or the state complexity of $L$, to be the minimum number of states required for a DFA to accept $L$. This concept has been studied for a long time; see, for example, \cite{maslov1970estimates, YuZS92, YuZS94}.
	\end{definition}

	
	\begin{lemma}
		\label{lem:13}
		For all integers $k \in \N$, we have $\stc(G_k) \geq 2^k$.
	\end{lemma}
	\import{proofs/}{lem_13.tex}
		
	\begin{lemma}
		\label{lem:14}
		For all integers $k \in \N$, we have $\stc(G_k^R) \leq 5k + 3$.
	\end{lemma}
	\import{proofs/}{lem_14.tex}
	

	
	\begin{definition}
		For $w \in \Sigma^*$ and a language $L$ over $\Sigma$, we define $\lsep(w, L)$ as the minimum number of states of a DFA that accepts $w$ and rejects all $x \in L$.
	\end{definition}
	
	\begin{definition}
		Since the set $\Sets{1,2}^* - G_k$
		 is referred to several times in the rest of this paper, for simplicity, we will denote it by $H_k$.
\end{definition}
	
	
	\begin{lemma}
		\label{lem:17}
		There exists $z_k \in (G_k - \Sets{\epsilon})$ such that $\lsep(z_k, H_k) \geq 2^k$.
	\end{lemma}
	\import{proofs/}{lem_17.tex}
	
	\begin{definition}
		The set $H_k \cup \Sets{z_k}$
		 is referred to several times in the rest of this paper. So, for simplicity, we will denote it by $H'_k$.
	\end{definition}
	
	\begin{remark}
		\label{rem:spider}
		For any two DFAs $D \in M_i$ and $D' \in M_j$, some DFA $E \in M_{i \times j}$ exists such that $L(E) = L(D) \cap L(D')$.	
	\end{remark}

	
	\begin{lemma}
		\label{lem:19}
		For every two DFAs $D, D' \in M_{2^{k/2}-1}$, and every string $w \in \Sigma^*$, there exists $x \in H_k$ such that $\delta_D(w z_k) = \delta_D(w x)$ and $\delta_{D'}(w z_k) = \delta_{D'}(w x)$.
	\end{lemma}
	\import{proofs/}{lem_19.tex}
	

	\begin{lemma}
		\label{lem:20}
		Let $w \in H'_k \left(0^+ H'_k \right)^*$. For any two DFAs $D, D' \in M_{2^{k/2}-1}$, there exists some $w' \in H_k \left(0^+ H_k \right)^*$ such that $\delta_D(w) = \delta_D(w')$ and $\delta_{D'}(w) = \delta_{D'}(w')$, or in other words, neither $D$ nor $D'$ \distinguishes $w$ and $w'$.
	\end{lemma}
	\import{proofs/}{lem_20.tex}
	
	\begin{proposition}
		\label{prp:four}
		Let $D$ be a DFA in $M_{2^{k/2}-1}$, $q, q' \in Q_D$, and $w \in H'_k \left(0^+ H'_k \right)^*$. There exists some $w' \in H_k \left(0^+ H_k \right)^*$ such that $\delta_D(q, w) = \delta_D(q, w')$ and $\delta_D(q', w) = \delta_D(q', w')$. 
	\end{proposition}
	\import{proofs/}{prp_four.tex}
				

	
	\subsection{Mapping $\Sets{0,1,2}^*$ to $\Sets{0,1}^*$}
	\label{subsec:four}
	The previous lemmas may help us to construct two strings in $\Sigma^*=\Sets{0,1,2}^*$ with our desired characteristics. But our goal is to prove our result for an alphabet of size $2$. To be able to construct the intended strings over $\Sets{0,1}$, in this subsection we introduce the function $\tr$ that maps strings in $\Sigma^*$ to strings in $\Sets{0,1}^*$, while preserving some of our desired characteristics in them.
	
	\begin{definition}
		For a string $w \in \Sigma^*$, we define $\tr(w)$ to be the string obtained from $w$ by replacing all occurrences of $1$ by $11$ and all occurrences of $2$ by $10$. Clearly we have $\tr(w) \in \Sets{0,1}^*$		
	\end{definition}
	
	The following lemma shows that when two strings are mapped under $\tr^R$, separating them would be at least as hard as separating the original ones.
	\begin{lemma}
		\label{lem:23}
		For all pairs of distinct strings $w,x \in \Sigma^*$, we have 
		$$\ssep(\tr^R(w), \tr^R(x)) \geq \ssep(w, x).$$
	\end{lemma}
	\import{proofs/}{lem_23.tex}
	
	\begin{lemma}
		\label{lem:24}
		Let $t \in \N$ and $R \subseteq \Sets{1,2}^*$ be a regular language such that $\stc(R) \leq t$.
		Also, let $w \in \left(\left(\Sets{1,2}^* - R \right) 0^+ \right)^* (R - \Sets{\epsilon})$.
		For all $w' \in 1\Sets{0,1}^*$, we have
		$$\ssep(\tr(w) f_n w', \tr(w) g_n w') \leq 2t + n + 4.$$
		Recall that $f_n = 0^n$ and  $g_n = 0^{n+(2n+1)!}$.
	\end{lemma}
	\import{proofs/}{lem_24.tex}
	
	\subsection{The Main Result}
	\label{subsec:five}
	Now we are ready to prove our main result. As shown in Theorem \ref{th:final}, by substituting the appropriate values for $n$ and $k$ in Theorem \ref{th:main}, we can prove that the difference $\size{\ssep(w,x)-\ssep(w^R,x^R)}$ is unbounded.
	
	
	\begin{theorem}
		\label{th:main}
		For all $k, n \in \N$, there exist two unequal strings $w', x' \in \Sets{0,1}^*$ such that 
		$$\ssep(w', x') \geq \min(2n + 2, 2^{k/2}),$$
		but 
		$$\ssep((w')^R, (x')^R) \leq n + 10k + 10.$$ 
	\end{theorem}
	\import{proofs/}{theorem1.tex}
	
	
	
	\begin{theorem}
		\label{th:final}
		The difference 
		$$\size{\ssep(w,x)-\ssep(w^R,x^R)}$$ is unbounded for an alphabet of size at least $2$. 
	\end{theorem}
	\import{proofs/}{th_final.tex}



	\section{Conclusion}
	In this paper, we proved that the difference $\size{\ssep(w,x) - \ssep(w^R,x^R)}$ can be unbounded. However, it remains open to determine whether there is a good upper bound on $\ssep(w,x) / \ssep(w^R,x^R)$.

	\section*{Acknowledgments}
I wish to thank Jeffrey Shallit, Mohammad Izadi, Arseny Shur, MohammadTaghi Hajiaghayi, Keivan Alizadeh, Hooman Hashemi, Hadi Khodabandeh, and Mobin Yahyazadeh, who helped me write this paper. I would also like to thank the anonymous referees for their careful reading of this paper, and for their valuable comments and suggestions. 
	\bibliography{references/references}
\end{document}