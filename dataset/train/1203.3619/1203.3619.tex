We have implemented both the HWM and SHALE  algorithms described in
Section~\ref{sec:solution} and benchmarked their performance against
the full solution approach (known hereafter as XPRESS) on
historical booked contract sets. 
We have extensively tuned the parameters for XPRESS, so it is much
faster than just using it ``off-the-shelf.''
First we describe these datasets
and our chosen performance metrics and then present our evaluation
results.

\subsection{Experimental setup}
\label{experimental setup}
In order to test the ``real-world'' performance of all three
algorithms we considered 6 sets of real GD contracts booked and
active in the recent past. In particular, we chose three periods of
time, each for one to two weeks, 
and two ad positions
LREC and SKY for each of these time periods. 


We considered US region contracts booked to the aforementioned
positions and time periods and also excluded all frequency capped
contracts and all contracts with time-of-day and other custom targets.
Also, all remaining contracts that were
active for longer than the specified date ranges were truncated and
their demands were proportionally reduced. Next, 
we generated a bipartite graph for each contract set as in
Figure~\ref{fig:bipartite1}; by sampling  eligible impressions for each
contract in the set. This sampling procedure is described in detail in
\cite{vvs10}. We then ran HWM, SHALE and XPRESS on each of the 6 graphs
and evaluated the following metrics.
\begin{enumerate}
\item  {\bf Under-delivery Rate} : This represents the total
  under-delivered impressions as a proportion of the booked demand,
  i.e.,

\item {\bf Penalty Cost} : This represents the penalty incurred
  by the publisher for failing to deliver the guaranteed number of
  impressions to booked GD contracts. Note that the true long-term penalty 
  due to under-delivery  is not known since we cannot easily forecast
  how an advertiser's future business with the publisher will change due to
  under-delivery on a booked contract. Here we define the total
  penalty cost to be

where  is the number of under-delivered impressions to contract
 and  is the cost for each under-delivered
impression. For our experiments, we set  to be  where  is the revenue per delivered
impression from contract . Indeed, it is intuitive and reasonable
to expect that contracts that are more valuable to the advertiser
incur larger penalties for under-delivery. The offset (here 5 CPM\frac{1}{2}\sum_{i\in\neij} s_i\frac{V_j}{\theta_{ij}}(x_{ij} - \theta_{ij})^2\ell_2^2\gp_jp_j =
0.002 \ + \ 4*q_jq_jjs_id_jS_jj$.
In Figure~\ref{fig:expt 2}, we show the
under-delivery rate, penalty cost and L2 distance for SHALE as a
percentage of the corresponding metric for HWM for various levels of
ASC.
\begin{figure}[h!]
\centering
\includegraphics*[viewport=37 155 760 575,scale=0.32]{experiment2.pdf}
\caption{SHALE Vs. HWM}
\label{fig:expt 2}
\end{figure}
First we note that each of our metrics for SHALE is better than the
corresponding metric for HWM for all values of ASC. Indeed, the SHALE
L2 distance is less than 50\% of that for HWM. Also note 
that the SHALE penalty cost consistently improves compared to HWM as
the ASC increases.  This indicates that even though HWM appears to have better
pacing for some data sets, SHALE is still a more robust algorithm and is likely
preferrable in most situations.  (Indeed, we see very consistently that its
under-delivery penalty and revenue are both clearly better.)


