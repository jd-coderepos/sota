\documentclass{LMCS}

\def\doi{9(4:7)2013}
\lmcsheading {\doi}
{1--10}
{}
{}
{Mar.~\phantom08, 2013}
{Oct.~16, 2013}
{}




\usepackage{enumerate}
\usepackage{hyperref}


\usepackage{amssymb}
\usepackage{bussproofs}
\usepackage{tikz-qtree}

\theoremstyle{plain}\newtheorem{satz}[thm]{Satz}
\def\eg{{\em e.g.}}
\def\cf{{\em cf.}}

\begin{document}

\title[Implicit Resolution]{Implicit Resolution}

\author[Z.~C.~Wang]{Zi Chao Wang}	\address{Department of Algebra\\Faculty of Mathematics and Physics\\Charles University in Prague\\Sokolovsk\' a 83\\Praha 8-Karl\' in\\18675\\Czech Republic}	\email{wang@karlin.mff.cuni.cz}  \thanks{The research leading to these results has received funding from the European Community's Seventh Framework Programme FP7/2007-2013 under grant agreement n\ensuremath{^\circ} 238381.}	






\keywords{Implicit Proofs, Resolution, Extended Frege}
\subjclass{MANDATORY list of acm classifications}
\ACMCCS{[{\bf Theory of computiation}]:  Computational complexity and cryptography---Proof complexity}




\begin{abstract}
  Let  be a set of unsatisfiable clauses, an implicit
  resolution refutation of  is a circuit  with a
  resolution proof  of the statement `` describes a
  correct tree-like resolution refutation of ''. We show that
  such a system is p-equivalent to Extended Frege.

  More generally, let  be a tautology, a -proof of
   is a pair  s.t.  is a -proof of
  the statement `` is a circuit describing a correct -proof
  of ''. We prove that  for an arbitrary
  Cook-Reckhow proof system .
\end{abstract}

\maketitle



\section{Introduction}

In proof complexity one of the basic questions that remained open is whether or not there is an optimal proof system. Although there is no consensus whether such proof system should exist it is generally believed that Extended Frege is the pivotal case in the sense that if such optimal proof system exists then Extended Frege is currently the most natural candidate. This is because Extended Frege corresponds to the complexity class  and many attempts in constructing proof systems that are conjecturally stronger than Extended Frege ended up in producing systems that are equivalent to Extended Frege.

Implicit proofs were introduced by Kraj{\'{\i}}{\v{c}}ek \cite{MR2058178} as a general framework for direct combinatorial constructions of strong proof systems beyond Extended Frege. The idea is to succinctly describe an exponential size proof by some polynomial size circuit then supplement such circuit with an additional correctness proof.
Loosely speaking, let  and  be some existing proof systems and  be a tautology, a -proof of  is a pair  s.t.  is a -proof of the formalized statement `` is a circuit describing a correct -proof of ''. For any proof system  the implicit version of , denoted , is the proof system defined as .

Whilst a hierarchy of implicit proof systems based on Extended Frege were introduced in \cite{MR2058178}, the system , is of specific interest. We may think of  as the ``succinct'' version of exponential size Extended Frege proofs and it bears a correspondence to exponential time computation and serves as the base case of the iterated construction of a strong implicit proof system whose soundness is not provable in the theory  where the exponentiation function is total. We would therefore expect that insights into the problem whether  is indeed stronger than Extended Frege, shall contribute to the empirical evidences towards the study of the existence of optimal proof systems.

In contrast to strong proof systems such as Extended Frege of which we do not even have any candidate hard tautologies, resolution is a refutational proof system with the resolution rule as the only derivation rule. It had been extensively studied since its introduction and substantial progress had been made in understanding its limits. Resolution is known to be inefficient for proving a number of combinatorial principles. For example, Haken \cite{MR821207} first proved that the propositional pigeon hole principle requires exponential size resolution refutations. More recently, systematic treatment on lower bounds of resolution in terms of clause width was presented by Ben-Sasson and Widgerson in \cite{MR1868713}.

In this paper we are motivated to understand Extended Frege in terms of resolution and implicit proofs. In Theorem \ref{t_main} we show that Extended Frege is p-equivalent to a resolution based proof system in the framework of implicit proofs\footnote{This was first conjectured by Kraj{\'{\i}}{\v{c}}ek.}. We generalize the construction in Theorem \ref{t_gen} to prove that  for any proof system , hence showing that  collapses to , although we are not able to address the precise strength of the latter. As a by product, in Lemma \ref{t_c} we show that existence of an  search algorithm that is provably correct in Extended Frege implies existence of such algorithm provably correct in resolution.

The paper is organized as follows. We briefly review the definition of resolution and fix notation in Section \ref{s_pre}. In Section \ref{s_circuit} we present a prototype of the key technical construction in terms of correctness of  search algorithms. In Section \ref{s_ires} we give precise definition of implicit resolution and prove the main result that it is p-equivalent to Extended Frege. In Section \ref{s_gen} we outline the construction applied to general implicit proof systems and briefly discuss generalizations to subsystems of .

\section{Preliminaries}
\label{s_pre}

Recall that a propositional proof system , as defined by Cook and Reckhow in \cite{MR523487}, is a polynomial time Turing machine  s.t. for any propositional formula  there exists a string  with  if and only if  is a tautology, in which case we say that  is a -proof of . Let  and  be proof systems. We say that  simulates , denoted , if there exists some constant  s.t. for any tautology  with -proof of length  there exists a -proof of  of length less than . We also say that  p-simulates , denoted , if we are able to compute the required -proof from the given -proof in time . We say that a proof system  is optimal if  for any proof system  \cite{MR1011192}.

A literal is a propositional variable or the negation of a propositional variable, i.e., we say that  is a literal if  or  for some propositional variable . We also write  to denote  and  to denote the negation of l. A clause is a multiset of literals, which we shall interpret as a disjunction. Let  be a clause we would also write  for some literal  if  for literals . We interpret a set of clauses as a conjunction of disjunctions of literals.

The resolution rule and the weakening rule are the following:
\begin{prooftree}
\AxiomC{}
\AxiomC{}
\LeftLabel{r}
\BinaryInfC{}
\DisplayProof\hspace{4cm}
\AxiomC{}
\LeftLabel{w}
\UnaryInfC{}
\end{prooftree}
where , ,  and  are clauses,  is a propositional variable\footnote{The standard definition also requires that  and .}. We say that the clause  is derived from  and  by resolving the variable  and the clause  is a weakening of the clause , respectively.

Let  be a set of unsatisfiable clauses, a resolution refutation or -refutation of  is a sequence of clauses  s.t.  is the empty clause and for each  either  or  is derived from  and  with the resolution rule for some  or  is a weakening of  for some .

We may think of an -refutation as a directed acyclic graph and we say that an -refutation is tree-like, denoted a -refutation, if the directed acyclic graph is in fact a tree. We also say that an -refutation is regular if on every path from a source to a sink in the directed acyclic graph no propositional variable is resolved more than once. Tseitin \cite{0205.00402} first introduced regular resolution and proved that regular resolution p-simulates tree-like resolution.

Resolution is complete without the weakening rule, i.e., let  be a set of clauses, if there exists an -refutation of  with weakening then there exists an -refutation of  without weakening.

\section{Resolution Refutation of Circuit Correctness}
\label{s_circuit}

In this paper we work with Extended Resolution, denoted , which is known to be p-equivalent to Extended Frege. In order to fix notation we give precise definition of Extended Resolution.

\begin{defi}
Let  be a set of clauses we write  to denote the set of all propositional variables in .
\end{defi}

Recall that an extension clause  is an abbreviation for a group of clauses of the form  where  is a propositional variable and  and  are literals. Suppose  and  are the variables on which the literals  and  are defined, i.e.,  and  for some  then we say that the extension clause  defines two directed edges  and .

\begin{defi}
Let  be a set of extension clauses. We say that  is a circuit if the directed graph defined by the extension clauses is acyclic and every vertex has at most two incoming edges. We say that  is a free variable if  is a source, otherwise we say that  is an extension variable. We write  and  to denote the set of free variables and the set of extension variables of the circuit  respectively.
\end{defi}

\begin{defi}
Let  be a set of unsatisfiable clauses, an -refutation of  is an -refutation of  for some circuit  with  and .
\end{defi}

For convenience we do not restrict the outputs of a circuit to the sinks of the graph and we shall instead specify explicitly the variables for which we would label as outputs. It is easy to see that such relaxed definition is equivalent to the conventional definition. Let  be a circuit and let  be a vertex which is not a sink be our desired output. We take  for some  then  would be a sink in the new circuit . Similarly suppose we have some sink  yet we do not label  as an output we may take some output  and define .

We shall write  to denote the circuit  with variables  and  displayed. Note in particular that we do not necessarily display all variables. When we write  for some circuits  and  we always mean their union as sets of clauses. Therefore the resulting set of clauses  is not necessarily a circuit since conditions for extension variables might be violated.

\begin{defi}
Let  and  be circuits, we say that  embeds in  and  is an expansion of  if there exists an injection  s.t.  and  whenever  where  if  for some variable . We say that  is an embedding from  to . We also say that  is an isomorphism if  is bijective.
\end{defi}

\begin{lem}
\label{l_eq}
Let  and  be circuits, if  is an expansion of  and  is an embedding from  to  s.t  for all  then for any  there exists -refutations of  and  of size .
\end{lem}
\begin{proof}
By induction on the size of the circuit . The base case is trivial so suppose we have  for some extension variable  and literals  and .

From  we can derive the following clauses:

\begin{tabular}{ c c c }
\begin{tikzpicture}[grow'=up]
            \Tree [.   ]
\end{tikzpicture}
&
\begin{tikzpicture}[grow'=up]
            \Tree [.   ]
\end{tikzpicture}
&
\begin{tikzpicture}[grow'=up]
            \Tree [.   ]
\end{tikzpicture}
\end{tabular}

\noindent By induction hypothesis there is an -derivation of the singleton clause  from  of size . Then again by induction hypothesis there is an -refutation of  of size .

The -refutation of  is completely analogous.
\end{proof}

\noindent Suppose we have some circuit  and such circuit appears as subcircuit of some larger circuit more than once, possibly with different inputs, as  say, to avoid variable name collision it is mandatory for us to rename all the extension variables in  and sometimes it is also necessary to rename the free variables. Therefore in order to speak about isomorphic subcircuits with potentially different input variables we define a duplicate of a circuit  with variable substitutions  to be an isomorphic copy of the circuit  defined by some isomorphism  mapping the variables  to . This would allow us to be able to substitute the input variables and to specify output variables for different copies of the same circuit with different inputs.

\begin{defi}
Let  be a circuit and let . Suppose there is a circuit  and an isomorphism  s.t.  for all ,   and . Then we say that  is a duplicate of  with  defined by  and we write .
\end{defi}

The complexity class  was introduced by Megiddo and Papadimitriou in \cite{MR1107721}. This class characterizes the type of  search problems where a solution is guaranteed to exist. Let  be a binary relation computable in polynomial time. We say that  iff there exists a constant  s.t. for any string  of length  there exists a string  of length  s.t.  holds. We consider correctness proofs of nonuniform algorithms solving the  search problem defined by the relation  in propositional logic.

Let  be the sequence of uniformly generated circuits computing the relation  where  and  are bits of the input strings  and  respectively and  is the output s.t. the relation  holds iff  is true. A non-uniform algorithm for  is a sequence of circuits  with  where  are bits of the input string  and  are bits of the output string . Define  then the nonuniform algorithm is correct if and only if the set of clauses  is not satisfiable for any .

We show that Extended Resolution refutations of such encoded statements could be efficiently translated into resolution refutations. We shall construct from the original circuit  a circuit  of size linear in the size of the given -refutation s.t. the two circuits  and  are equivalent in the sense that they compute the same function yet the correctness of the new circuit  has efficient proofs in . This is straightforwardly done by expanding the original circuit  with the correctness definition  along with all the extension variables as defined in the -refutation.

\begin{lem}
\label{t_c}
Let  be an -refutation of  for some circuit  then there exists an -refutation of  of size  for some circuit  of size .
\end{lem}
\begin{proof}
By definition  is an -refutation of  for some circuit .

Take .

From  we can construct an -refutation of  of size . We also know that the set of clauses  is not refutable since it is a circuit. Therefore the singleton clause  is derivable from  with an -derivation of size .

Now  defines an embedding from  to . Therefore by Lemma \ref{l_eq} there is an -refutation of  of size . This completes the proof since .
\end{proof}

We shall briefly remark that similar results also hold in the uniform setting in bounded arithmetic.

Suppose for some polynomial time Turing machine  the bounded arithmetic theory  proves that

where  is a formula expressing that the string  encodes the transcript of computation of the Turing machine  on input string  and  outputs the string  and  is a formula expressing that the string  encodes the transcript of computation of the Turing machine  defining the  search problem on input strings  and  and output a boolean value  s.t.  if and only if the relation  holds.

Then there exists a polynomial time Turing machine  solving the same  search problem whose correctness is provable in the theory 

where -IND is a theory axiomatized by basic axioms and induction for bounded universal number quantifiers.

\section{Strength of Implicit Resolution}
\label{s_ires}
Let  be the number of propositional variables and let  be a set of unsatisfiable clauses. We may first assume without loss of generality that every -refutation is regular and weakening is only applied to initial clauses. Then by appropriate padding we may represent any -refutation of  as a balanced decision tree of depth  and size .

Our encoding of circuit representation of -refutation is as follows. Let  be a circuit with  inputs\footnote{To simplify definition we insert an additional input as placeholder although only  inputs are actually required.} and  outputs. The root of the balanced decision tree is computed by , and for each node  in the tree, the left child is computed by  and the right child is computed by . The circuit computes the propositional variable on which the decision tree branches: the left child represents the negative literal and the right child represents the positive literal.

Notice that with such encoding, any circuit describes a correct -refutation of some set of clauses by definition, hence one only needs to check that all the initial clauses are as prescribed.

Let  be propositional variables,  be a circuit with  inputs and  outputs,  be a clause, we say that  is an initial clause of the -refutation described by  if there exists a string  of length  s.t.
\begin{enumerate}[(a)]
\item

\item
 iff there exists  s.t.  and 
\item
 iff there exists  s.t.  and 
\end{enumerate}
That is, from a string  we can compute\footnote{Recall that  and  for any propositional variable .} the following initial clause 

We write  to denote the set of all initial clauses of the -refutation described by .

Let  be a set of clauses in  propositional variables . We could encode a propositional variable as a string of length . Similarly a literal could be encoded as a string of length  with an extra bit to denote whether it is negated.

We define a circuit  of size  expressing that  is true iff the clause  is a weakening of some clause in  where  is a new variable and  are literals encoded in variables , , i.e, for each   the literal  is encoded in the string  so that the literal  is  iff  codes number  and it is negated iff .

To check that the clause  is a weakening of some initial clause  it suffices to check that for each literal  there exists  s.t. . Therefore by enumerating all the clauses in  we are able to check whether  is a weakening of some .

Formally  is a set of clauses with extension variables of constant depth defined as follows:
\begin{enumerate}[]
\item
 is the -th bit of the binary representation of .
\item
 expresses that the literal  is not .
\item
 expresses that the clause  contains the literal .
\item
 expresses that the clause  is not a weakening of .
\item
 expresses that the clause  is a weakening of some .
\end{enumerate}
Recall that in order to compute some clause  from some string  we need to evaluate the circuit   times to compute all the propositional variables. We therefore define an auxiliary circuit  of size  with inputs  and outputs  s.t. for each ,  defines a string as input to the circuit  in order to compute the propositional variable at depth , i.e.

\begin{tabular}{c c c c c c}
 &   & \multicolumn{2}{c}{} &  & \\
 & \multicolumn{2}{c}{} &  &  & \\
 & \multicolumn{4}{c}{} & \\
 &  & \multicolumn{2}{c}{} &  & 
\end{tabular}

\noindent Let  be a set of clauses and let  be a circuit with inputs  and outputs  we define a set of clauses  of size  expressing that there exists  s.t.  is not a weakening of  for any .

where each  is a duplicate of  computing the propositional variable at depth , i.e.


\begin{defi}
\label{d_ires}
An implicit resolution refutation is a 4-tuple  where  is the number of propositional variables,  is the set of initial clauses,  is a circuit with  inputs,  outputs and  is an -refutation of .
\end{defi}

\begin{lem}
Implicit resolution is a Cook-Reckhow proof system.
\begin{proof}
The soundness of implicit resolution follows directly from soundness of  and .

To see that implicit resolution is complete we show that any unsatisfiable clause has an implicit resolution refutation. Let  be a set of unsatisfiable clauses. By completeness of  there exists a -refutation of . It is trivial to obtain a circuit  generating such refutation by hardwiring the outputs. Now the set of clauses  is unsatisfiable since  is indeed correct therefore it is refutable in  by completeness of .

In order to show that implicit resolution refutations could be verified in polynomial time we give description of a Turing machine as follows:
\begin{enumerate}[(i)]
\item\label{s_1}
decode the input in order to obtain a 4-tuple  as in Definition \ref{d_ires}.
\item\label{s_2}
verify that ,  and the output variables of the circuit  are correctly specified.
\item\label{s_3}
compute .
\item\label{s_4}
verify that  is an -refutation of .
\end{enumerate}

It is clear that steps (\ref{s_1}) (\ref{s_2}) (\ref{s_3}) runs in polynomial time and step (\ref{s_4}) also runs in polynomial time since  is also a Cook-Reckhow proof system.
\end{proof}
\end{lem}

The following is essentially a special case of Lemma 4.1 from \cite{MR2058178}, we sketch a proof here for completeness of presentation and refer the reader to \cite{MR2039508} for detailed treatment on reflection principle for resolution.

\begin{lem}
\label{l_er}
Let  be a set of clauses in  variables and let  be an -refutation of  then there exists a circuit  of size  and -refutation of the set of clauses  of size .
\begin{proof}
We construct a circuit  with inputs  and outputs . The circuit  is canonically constructed so that it describes a decision tree that always branch on variable  at depth . Such circuit is trivially described by identifying the largest  s.t.  for all  then enumerating all the possibilities and hardwiring the outputs .

It is clear that the set of clauses  is equivalent to a form of Tarski's truth definition. In other words, suppose that we have  for some  then those  are precisely the negations of the truth assignments and the entire set of clauses  is a formalized statement expressing that  is satisfied by the truth assignment . To see this, notice that by definition the clause computed from the string  is . This clause is a weakening of some clause  iff the truth assignment  satisfies the clause . Suppose on the contrary that  satisfies the set of clauses  then the clause  is not a weakening of any . Hence for any  there exists  s.t. , that is, the truth assignment  satisfies the set of initial clauses .

It is well known that such formalized truth definition could be refuted efficiently in , provided that the original set of clauses  has efficient -refutation.
\end{proof}
\end{lem}

\begin{thm}
\label{t_main}
Implicit resolution is p-equivalent to .
\begin{proof}
We first show that  p-simulates implicit resolution.

Let  be an implicit resolution refutation we construct polynomial size -refutation of .

We begin by defining all the required extension variables from the set of clauses . Let  be the propositional variables in . From truth assignments to  we compute an initial clause defined by the string .

where each  is a duplicate of  computing the propositional variable at depth , i.e.

and each  is an auxiliary circuit assigning  the truth value  where  is the number encoded by the string  in binary representation.

Now the auxiliary circuits  ensure that  defines a path that falsifies .
Therefore the simulation follows from the reflection principle of  which also has polynomial size -refutation.

In this part of the proof we show that implicit resolution p-simulates .

Let  be the set of initial clauses. By Lemma \ref{l_er} we may assume that we are given some circuit  with inputs  and outputs  and an -refutation  of the set of clauses . It suffices to construct polynomial size circuit  and -refutation  of .

The -refutation  is in fact an -refutation of  for some circuit . To display the output  let us suppose that we have  for some . We construct a new circuit  of size . Take  for some isomorphism . Then a  size -derivation of the singleton clause  from  could be analogously translated from . Now we see that the restriction  defines an embedding from  to . By choosing appropriate variables in  it is possible to force . Therefore we obtain by Lemma \ref{l_eq} an -refutation of  of size  as required.
\end{proof}
\end{thm}

\section{General Construction}
\label{s_gen}

Let  be a Cook-Reckhow proof system defined by some Turing machine  and let  be a tautology. Let  be a boolean circuit with  inputs. We may think of the outputs of the circuit  as a 0-1 matrix of size  for some . The intended meaning is that each row of the matrix gives a snapshot of the Turing machine  and the entire matrix describes a valid terminating computation of  that outputs  on some exponential size input.

Suppose for the sake of argument that  does not describe a valid computation of  that outputs . By definition of Turing machine we would be able to identify specific local properties on which  fails. Hence the correctness condition of the circuit  could be expressed by formalizing the negation of the disjunction of the following statements:
\begin{enumerate}[(i)]
\item
there exists some row  and some column  s.t. the tape at row  position  is modified without valid tape head transition.
\item
there exists some row  and some column  s.t. the tape head movement at row  from position  does not conform to the transition table.
\item
there exists some row  and some column  s.t. row  defines a terminating state and the output at position  does not match the encoded tautology .
\end{enumerate}

That is, given some  and  we are able to check that row  column  is a cell on which the circuit  violates the definition of the Turing machine  with output , in polynomial time, provided that this is actually the case. In fact we only need to evaluate the circuit  constantly many times.

Let  denote the canonically generated set of clauses with limited extension expressing the conditions above. We define a general implicit proof system  based on Cook-Reckhow proof systems  and . The idea is to have a -proof  of the correctness of an exponential size -proof described by some circuit . For convenience we shall have the correctness condition written as sets of unsatisfiable clauses instead of tautologies.

\begin{defi}[Kraj{\'{\i}}{\v{c}}ek \cite{MR2058178}]
We say that  is a -proof of the tautology  if  is a -refutation of the set of clauses .
\end{defi}

\begin{thm}
\label{t_gen}
 for arbitrary Cook-Reckhow proof system .
\begin{proof}
It suffices to show that  since  implies .

Let  be a tautology and let  be a -proof of . We know that  is in fact an -refutation of  for some circuit  with . By taking  where  is the output variable asserting the correctness in  the p-simulation follows by arguments similar to that of the proof of Theorem \ref{t_main}.
\end{proof}
\end{thm}

\begin{cor}

\end{cor}

We define a variant of implicit resolution which is p-equivalent to -Frege and similar constructions also apply to subsystems of  such as Frege and -Frege by substituting the circuit complexity classes  and  in place of .

\begin{defi}
Let  be a set of unsatisfiable clauses in  variables, we say that  is an implicit -resolution refutation of  if  is an implicit resolution refutation of  and  is an  circuit.
\end{defi}

\begin{thm}
Implicit -resolution is p-equivalent to -Frege.
\begin{proof}
The proof follows directly from the arguments in Theorem \ref{t_main} with the circuit class  in place of .
\end{proof}
\end{thm}

\section*{Acknowledgement}
I would like to thank Jan Kraj{\'{\i}}{\v{c}}ek for helpful discussions and suggestions and am extremely grateful to an anonymous referee for pointing out the simplified proof of the first part of Theorem \ref{t_main} without bounded arithmetic.



\bibliographystyle{plain}
\bibliography{wang}

\end{document}
