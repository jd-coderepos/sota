\documentclass[11pt]{article}
\usepackage{fullpage}
\usepackage{amssymb,amsmath}
\usepackage{graphicx, epsfig}



\def\id#1{\ensuremath{\mathit{#1}}}
\let\idit=\id
\def\idbf#1{\ensuremath{\mathbf{#1}}}
\def\idrm#1{\ensuremath{\mathrm{#1}}}
\def\idtt#1{\ensuremath{\mathtt{#1}}}
\def\idsf#1{\ensuremath{\mathsf{#1}}}
\def\idcal#1{\ensuremath{\mathcal{#1}}}  

\def\floor#1{\lfloor #1 \rfloor}
\def\ceil#1{\lceil #1 \rceil}
\def\etal{\emph{et~al.}}

\newcommand{\no}[1]{}
\newcommand{\nono}[1]{}
\newcommand{\fullv}[1]{ }

\newtheorem{theorem}{Theorem}
\newtheorem{lemma}{Lemma}
\newtheorem{fact}{Fact}
\newtheorem{claim}{Claim}
\newtheorem{corollary}{Corollary}
\newtheorem{problem}{Problem}
\newenvironment{proof}{\trivlist\item[]\emph{Proof}:}{\unskip\nobreak\hskip 1em plus 1fil\nobreak
\parfillskip=0pt\endtrivlist}

\newenvironment{itemize*}{\begin{itemize}\setlength{\itemsep}{0pt}\setlength{\parskip}{0pt}\setlength{\parsep}{0pt}\setlength{\topsep}{0pt}\setlength{\partopsep}{0pt}}{\end{itemize}}


\newcommand{\halfleftsect}[2]{(#1,#2]}
\newcommand{\cL}{{\cal L}}
\newcommand{\cA}{{\cal A}}
\newcommand{\cT}{{\cal T}}
\newcommand{\cP}{{\cal P}}
\newcommand{\cD}{{\cal D}}
\newcommand{\cM}{{\cal M}}
\newcommand{\cB}{{\cal B}}
\newcommand{\cF}{{\cal F}}
\newcommand{\cQ}{{\cal Q}}
\newcommand{\cH}{{\cal H}}
\newcommand{\cV}{{\cal V}}
\newcommand{\oS}{\overline{S}}
\newcommand{\tS}{\widetilde{S}}
\newcommand{\tI}{\widetilde{I}}
\newcommand{\tD}{\widetilde{D}}
\newcommand{\oD}{\overline{D}}

\newcommand{\todo}[1]{ TO DO:  #1 \newline}




\newcommand{\locus}{\idsf{locus}}
\newcommand{\Suf}{\idsf{Suf}}
\newcommand{\RMQ}{\idsf{rmq}}
\newcommand{\occ}{\idrm{occ}}
\newcommand{\lleft}{\idit{left}}
\newcommand{\rright}{\idit{right}}




\newcommand{\sindex}{\idrm{index}}
\newcommand{\rng}{\idrm{rng}}
\newcommand{\eps}{\varepsilon}




\begin{document}
\title{Sorted Range Reporting\thanks{Partially funded by Millennium Nucleus Information and Coordination in
Networks ICM/FIC P10-024F, Chile.} }
\author
{
Yakov Nekrich\thanks{Department of Computer Science, University of Chile.
Email: {\tt yakov.nekrich@googlemail.com}.}
\and 
Gonzalo Navarro \thanks{Department of Computer Science, University of Chile.
Email: {\tt  gnavarro@dcc.uchile.cl}.}
}


\date{}
\maketitle
\begin{abstract}
We consider a variant of the orthogonal range reporting problem
when all points should be reported in the sorted order of their
 -coordinates. We show that reporting two-dimensional points with this
 additional condition can be organized  
(almost) as efficiently as the standard range reporting.
Moreover, our results generalize and improve the previously known results 
 for the orthogonal range successor problem and can be used to obtain better 
solutions for some stringology problems. 
\end{abstract}

\section{Introduction}
\label{sec:introduction}
An orthogonal range reporting query  on a set of -dimensional points 
 asks for all points  that belong to the query 
rectangle .  
The orthogonal range reporting problem, that is, the problem of constructing 
a data structure that supports such queries, was studied extensively; see
for example~\cite{agarwal1999geometric}.  
In this paper we consider  a variant of the two-dimensional range reporting  in which reported points must be sorted by one of their coordinates.  
Moreover, our data structures can also work in the online modus: the query answering procedure reports all points from  in 
 increasing -coordinate order
until the procedure is terminated or all points in  are output.\footnote{We can get increasing/decreasing 
/-coordinate ordering via coordinate changes.}

Some simple database queries can be represented as orthogonal range reporting 
queries. For instance, identifying all company employees who are between 
20 and 40 years old and whose salary is in the range  is equivalent 
to answering a range reporting query  on a set of 
points with coordinates . 
Then reporting employees with the salary-age range 
sorted by their salary is equivalent to 
a sorted range reporting query. 

\tolerance=1000
Furthermore, the sorted reporting problem is a generalization of the 
orthogonal range successor problem (also known as the range next-value problem) \cite{LenhofS94,CrochemoreIR07,KKL07wads,IliopoulosCKRW08,Yu:2011}. 
The answer to an orthogonal range successor query  
is the point with smallest -coordinate\footnote{Previous works (e.g., \cite{CrochemoreIR07,Yu:2011}) use slightly different definitions, but all of them are equivalent up to a simple change of coordinate system or reduction to rank space~\cite{GabowBT84}.} among all points that are 
in the rectangle . The best previously known  space  data structure 
for the range 
successor queries uses  space and supports queries in
  time~\cite{Yu:2011}. The fastest previously described 
structure  supports range successor queries in  
time but needs  space. 
\no{
A data structure that answers range successor queries in time  can also 
answer sorted reporting queries in  time; see the proof of Theorem~\ref{theor:spaceeff}.
Thus the linear space structure of~\cite{Yu:2011} answers queries in  time.}
 In this paper we show that these results can be significantly improved. 


In Section~\ref{sec:linspace} we describe two data structures for range 
successor queries.  The first structure needs  space and answers 
 queries in  time; henceforth  denotes an arbitrarily small positive constant. The second structure  needs 
 space and supports  queries in  
time. Both data structures can be used to answer sorted reporting queries 
in  and  time, respectively, where  is the number of reported points.
In Sections~\ref{sec:3sided-rep} and~\ref{sec:2dim-range} we further improve the query time and describe a data 
structure that uses  space and supports sorted reporting 
queries in  time.  As follows from the reduction of~\cite{MNSW98} and the lower bound of~\cite{PT06}, any data structure that uses  space needs 
 time to answer  (unsorted) orthogonal range reporting queries. Thus we achieve optimal query time for the sorted range reporting 
problem.  We observe that the currently best data structure 
for unsorted range reporting in optimal time~\cite{ChanLP11} also 
uses  space.
In Section~\ref{sec:appl} we discuss applications of sorted reporting queries to some problems related to text indexing and some geometric problems. 
\nono{Further applications are described in the full version of this paper~\cite{NN12full}.}

Our results are valid in the word RAM model. Unless specified otherwise, we measure the space usage in words of  
bits. We denote by  and  the coordinates of a point . We assume that points lie on an  grid, i.e., that point coordinates are integers
integers in . We can reduce the more general case to this one by
reduction to rank space~\cite{GabowBT84}. The space usage will not change
and the query time would increase by an additive factor 
, where  is the time needed to search in a one-dimensional set of integers~\cite{BoasKZ77,PT06}.
\no{
\todo{Say that we can iterate and report only k leftmost points, k unknown in advance, say it better than in the 1st paragraph}
\todo{Say about applications,  say that we report points sorted by -coordinates}
}

\section{Compact Range Trees}

The range tree is a handbook data structure frequently used for
various orthogonal range reporting problems. Its leaves contain 
the -coordinates of points; a set  associated with each node
  contains all points whose -coordinates are stored in the subtree rooted 
at . We will assume that points of  are sorted by their -coordinates.
 will denote the -th point in ;
 will denote the sorted list of  points
 .

A standard range tree uses  space, but this can be reduced by 
storing compact representations of sets .
We will need to support the following two operations on 
compact range trees.
Given a range  and a node ,  finds
 the range   
such that  if and only if   for any .
Given an index  and a  node ,  returns  the coordinates of 
point . 

\begin{lemma}\cite{Chaz88,ChanLP11}\label{lemma:range}
There exists a compact range tree that uses  space and supports 
operations  and   in  and  time, respectively, for
(i)  and ;
(ii)  and ;
(iii)  and .
\end{lemma}
\begin{proof} 
We can support  in  time using  space as in variants  and  using a result from Chazelle \cite{Chaz88}; 
we can support  in  time and   space using a result from Chan et al.~\cite{ChanLP11}. In the same paper \cite[Lemma 2.4]{ChanLP11}, the 
authors also showed how to support  in 
 time and  additional space 
using a data structure that supports  in  time. 
\end{proof}



\section{Sorted Reporting in Linear Space}
\label{sec:linspace}
In this section we show how a range successor query  can be answered efficiently.
We combine the recursive approach of the van Emde Boas
structure~\cite{BoasKZ77} with compact structures for range maxima
queries. A combination of succinct range minima structures 
and range trees was also used in~\cite{ChanLP11}. A novel idea that distinguishes 
our data structure from the range reporting structure in~\cite{ChanLP11}, as well as
 from the previous range successor structures, is binary search on tree levels
 originally designed for one-dimensional searching~\cite{BoasKZ77}.
We essentially perform a  one-dimensional search for the successor 
of  and answer range maxima queries at each step.
 Let  denote the compact range tree on the
-coordinates of points.  is implemented as in variant  of 
Lemma~\ref{lemma:range}; hence, we can find the interval 
for any node  in  time. We also store a
compact structure for range maximum queries  in every node :
given a range ,  returns the index  of the
point  with the greatest -coordinate in .
We also store a structure for range minimum  queries .
 and  use  bits and answer queries 
in  time \cite{Fis10}. Hence all  and  for  use 
 space. 
Finally, an  space level ancestor structure enables us to 
find the depth- ancestor of any node  in  time~\cite{BenderF04}.


Let  denote the search path for  in the tree : 
connects the root of  with the leaf that contains the smallest
value . Our procedure looks for the lowest node  on
 such that .  For simplicity we assume
that the length of  is a power of .  We initialize  to
the leaf that contains ; we initialize  to the root
node. The node  is found by a binary search on .  We say
that a node  is the middle node between  and  if  is on
the path from  to  and the length of the path from  to 
equals to the length of the path from  to . We set the node
 to be the middle node between  and . Then we find the
index  of the maximal element in  and the
point .  If , then  is either 
or its descendant; hence, we set .  If , then  is
an ancestor of ; hence, we set . The search procedure
continues until  is the parent of .  Finally, we test nodes
 and  and identify  (if such  exists).
\begin{fact}
  If the child  of  belongs to , then  is the left
  child of .
\end{fact}
\begin{proof}
  Suppose that  is the right child of  and let  be the
  sibling of . By definition of , . Since  belongs to  and  is the left
  child,  for all points .  Since
  ,  and we obtain a
  contradiction.
\end{proof}
Since  is the left child of ,  for all  for the sibling  of .  Moreover,  for all
points  by definition of .  Therefore
the range successor is the point with minimal -coordinate in
.

The search procedure visits  nodes and spends
 time in each node, thus the total query time is
. By replacing  in the above
construction, we obtain the following result.
\begin{lemma}\label{lemma:linsucc}
  There exists a data structure that uses  space and answers
  orthogonal range successor queries in  time.
\end{lemma}
If we use the compact tree that needs  space,
then . Using the same structure as in the proof of 
Lemma~\ref{lemma:linsucc}, we obtain the following.
\begin{lemma}\label{lemma:llsucc}
  There exists a data structure that uses  space and
  answers orthogonal range successor queries in 
  time.
\end{lemma}


\paragraph{Sorted Reporting Queries.} 
We can answer sorted reporting queries by answering a sequence of 
range successor queries. Consider a query . 
Let  be the answer to the range successor query .
For , let  be the answer to the query 
.  
The sequence of points  is the sequence of  leftmost 
points in  sorted by their -coordinates. We observe 
that our procedure also works in the \emph{online modus} when   is not known in advance. That is, we can output the points of  in the left-to-right order until the procedure is stopped by the user or 
all points in  are reported.
\no{
 leftmost point,  then the next one, etc;   the query can be stopped at any time by 
the user. The total query time is the same as the time needed to answer 
 range successor queries. } 
\begin{theorem}\label{theor:spaceeff}
There exist a data structures that uses  space and answer
  sorted range reporting queries in  time, and that
use  space 
and answer those queries in  
time.
\end{theorem}




\section{Three-Sided Reporting in Optimal Time}
\label{sec:3sided-rep}
In this section we present optimal time data structures for two special 
cases of sorted two-dimensional queries.
In the first part of this section we describe a data structure that
answers sorted one-sided queries: for a query  we report all points
, , sorted in increasing order of their -coordinates. 
Then we will show how to answer three-sided queries, i.e., to report all points ,  and
, sorted in increasing order by their -coordinates. 



\paragraph{One-Sided Sorted Reporting.} We start by describing a data
structure that answers queries in  time; our solution is  based on a standard range tree
 decomposition of the query interval  into  intervals. Then we show how
to reduce the query time to . This improvement uses an additional data structure 
for the case when  points must be reported. 

We construct a range tree on the
-coordinates. For every node , the list  contains all
points that belong to  sorted by their -coordinates. Suppose
that we want to return  points  with smallest -coordinates
such that .  We can represent the interval  as a
union of  node ranges for nodes .  The search
procedure visits each  and finds the leftmost point (that is, the
first point) in every list .  Those points are kept in a data
structure . Then we repeat the following step  times: We find the 
leftmost point  stored in , output  and remove it from
. If  belongs to a list , we find the point  that
follows  in  and insert  into .  As  contains
 points, we support updates and find the leftmost point
in  in  time \cite{FW94}.  Hence, we can initialize  in
 time and then report  points in  time.


We can reduce the query time to  by constructing additional data
structures.  If  the data structure described above
already answers a query in  time.  The case  can be handled as follows.  We store for each  a list
. Among all points  such that  the list
 contains  points with the smallest
-coordinates. Points in  are sorted in increasing order by
their -coordinates.  To find  leftmost points in  for , we identify the highest point  such that  and report the first  points in . The point  can be
found in  time using the van Emde Boas data structure~\cite{BoasKZ77}.  If  is known, then a query can be answered in  time for any 
value of .

One last improvement will be important for the data structure of Lemma~\ref{lemma:3sid}. Let  denote the set of  lowest points in . We store the -coordinates of 
in the -heap F. Using , we can find the highest , 
such that , in  time \cite{FW94}. 
\no{Suppose that we want to report  leftmost points , , in sorted order.}   
Let . If , 
then . As described above, we can answer a query in  time 
when  is known. Hence, a query can be answered in  time if .
\no{If , then we can answer a query in  time.} 
\begin{lemma}\label{lemma:1sid}
  There exists an  space data structure that supports
  one-sided sorted range 
  reporting queries in  time. If the highest point
   with  is known, then one-sided sorted queries can be
  supported in  time. 
  If , a sorted range 
  reporting query  can be answered in  time.
\no{If , then  one-sided  sorted  queries  can be answered in  time.}
\no{ where .}
\end{lemma}

\paragraph{Three-Sided Sorted Queries.} We construct a range tree on
-coordinates of points. For any node , the data structure  of Lemma~\ref{lemma:1sid}
supports one-sided queries on  as described above. 
\no{We also store  lowest points from  
in a data structure .  Using the result of~\cite{FW94}, 
we can find for any 
the highest  such that . 
Hence, we can use  to determine whether  contains at least 
 points that are below . 
If this is not the case, we can find the highest , , 
using . 
}
For each
root-to-leaf path  we store two data structures,  and
.  Let  and  be defined as follows. If 
belongs to a path  and  is the left child of its parent, then
its sibling  belongs to .  If  belongs to  and 
is the right child of its parent, then its sibling  belongs to
.  The data structure  contains the lowest point in
 for each ; if  is a leaf, 
also contains the point stored in .  The data structure 
contains the lowest point in  for each ; if  is a leaf,  also contains the point stored in .  Let
 denote the level of a node  (the level of a node  is the 
length of the path from the root to ). If a point , , comes
from a node , then .  For a given query  the
data structure  () reports points  such that
 and  sorted in decreasing (increasing) order
by .  Since a data structure , , contain
 points, the point with the -th largest (smallest) value of 
among all  with  can be found in  time. 
\no{first  points can be found in  time.}
The implementation of structures  is based on standard bit 
techniques and will be described in the full version.


Consider a query .  Let  and  be
the paths from the root to  and  respectively. Suppose that the
lowest node  is situated on level . Then
all points  such that  belong to some node  such
that  and  or  and .
We start by finding the leftmost point  in  such that
 and . 
Since the -coordinates of points in
 decrease as  increases, this is equivalent to
finding the point  such that  and  is maximal.
If , we visit the node  that contains ; \no{We find the highest point  , . If  is the last (highest) point in , then   
contains at least  points , such that . Otherwise  is the highest among all  
such that .
Then, we use  to report the  leftmost points  such that . 

If  is not the highest point in , then  for any  such that . 
}
 using , we report the  leftmost points  such that . 
Then, we find the point  with the next largest value of 
 among all  such that ; we visit the node 
 that contains  and proceed as above. 
The procedure continues until  points are output or there are no 
more points ,  and . 
If  points were reported, we visit selected nodes  and report remaining  points using a symmetric procedure. 

Let  denote the number of reported points from the set  
and let . 
We spend  time in a visited node  if 
or . If   and , then we spend 
 time in the respective node . 
Thus we spend  time in a node 
only if , i.e., only if not all points from 
are reported.
Since at most one such 
node  is visited, the total time needed to answer all one-sided queries is . 
\no{
in the decreasing order of . For every found point , we visit the node  that contains .
Using , we report the  leftmost points  such that
. 
}
\begin{lemma}\label{lemma:3sid}
  There exists an  space data structure that answers
  three-sided sorted reporting queries in  time.
\end{lemma}

\paragraph{Online queries.}
We assumed in Lemmas~\ref{lemma:1sid} and~\ref{lemma:3sid} that 
parameter  is fixed and given with the query. 
Our data structures can also support queries in the online modus using 
the method originally described in~\cite{BrodalFGL09}. 
The main idea is that we find roughly  leftmost points from the 
query range for  and ; while  points are reported, we simultaneously compute the following  points in the background. 
For a more extensive description, refer to \cite[Section 4.1]{NavN12}, where the same method for a slightly different problem is described.   


\section{Two-Dimensional Range Reporting in Optimal Time}
\label{sec:2dim-range}
We store points in a compact range tree  on
-coordinates. We use the variant  of Lemma~\ref{lemma:range} that uses 
 space and retrieves the coordinates of the -th point
 from
  in  time. Moreover, the sets , , are divided into
groups . Each , except of the last one, contains
 points. For , each point assigned to  has smaller -coordinate than any point in . The set  contains selected elements from . If  is the right child of its
parent, then  contains  points with smallest
-coordinates from each group ; structure 
supports three-sided sorted queries of the form  on
points of .  If  is the left child of its parent, then
 contains  points with largest -coordinates from
each group ; data structure  supports three-sided
sorted queries of the form  on points of
.  For each point  we store the index  of the
group  that contains .  We also store the point with
the largest -coordinate from each  in a structure
 that supports  time searches \cite{BoasKZ77}.

For all points in each group  we store an array  that
contains  points sorted by their -coordinates.  Each point is
specified by the rank of its -coordinate in ; so each
entry uses  bits of space.
\no{
We answer a two-dimensional query by answering two three-sided 
queries in the online modus. Data structures  are main 
tools for answering relevant three-sided queries. We also use 
the structures for single groups 
}

To answer a query , we find the lowest common
ancestor  of the leaves that contain  and .  Let  and
 be the left and the right children of .  All points in
 belong to either  or
.  We generate the sorted list of 
leftmost points in  by merging the lists of  leftmost
points in  and . Thus it suffices to answer sorted three-sided
queries  and  in nodes
 and  respectively.

We consider a query ; query  is answered symmetrically. Assume
 fits into one group , i.e., all points  such that
 belong to one group .  We can find the
-rank  of the highest point , such that  in  time by binary search in .  Let 
and  be the ranks of  and  in .  We can find the
positions of  leftmost points in  using a data structure .
 contains the -ranks and -ranks of points in  
and answers sorted three-sided queries on . 
By Lemma~\ref{lemma:3sid},  uses  bits 
and supports  queries in  time.
 Actual coordinates of points can be obtained from their ranks in  in   time per point: if the -rank of a point 
is known, we can compute its position in ; we obtain 
-coordinates of the -th point in  using variant 
 of Lemma~\ref{lemma:range}.

Now assume  spans several groups 
for . That is, the -coordinates of all points in groups
 belong to ; the
-coordinate of at least one point in  () is
smaller than  (larger than ) but the -coordinate of at least
one point in  and  belongs to . Indices  and 
 are found in  time using .  We report
at most  leftmost points in 
just as described above.

 
Let ; if , the query
is answered. Otherwise, we report  leftmost points in

using the following method.  Let  and  be the minimal and
the maximal -coordinates of points in  and
, respectively.  The main idea is to answer the query
 in the online modus using the data structure
. If some group , , contains less than  points  with , then all such  belong to
 and will be reported by .  Suppose that 
reported  points that belong to the same group
. Then we find the rank  of  among the
-coordinates of points in . Using , we report
the positions of all points , such that the rank of
 in  is at most , in the left-to right order; we can
also identify the coordinates of every such  in  time per point.
The query to  is terminated when all such points are
reported or when the total number of reported points is .

We need  time to answer a query on , where 
 denotes the number of reported points from .  
Let 
If  is the last examined group, then ; otherwise 
.
We send a query to  only if  contains at least 
 points from  . 
Hence,  a query to  takes  time, 
unless  is the last examined group. 
Thus all necessary queries to  for  take  time.

Finally, if the total number of points in  is smaller than , we also report the remaining points from . 

\tolerance=1500
The compact tree  uses  words of space.  
A data structure  uses  words 
of space. Since all sets , , contain  
points, all  use  words of space.  
A data structure for a group   uses 
 bits. Since all  for all  contain  elements, data structures for all groups  use  
words of  bits. 
\begin{theorem}\label{theor:2dim}
  There exists a  space data structure that answers
  two-dimensional sorted reporting queries in  time.
\end{theorem}

\section{Applications}
\label{sec:appl}
\tolerance=1000
In this section we will describe data structures for several indexing
 and computational geometry problems. A text (string)  of length  
 is pre-processed and stored in a data structure so that 
certain queries concerning some substrings of  can be answered 
efficiently.
\paragraph{Preliminaries.}
In a suffix tree  for a text , every leaf of  is associated with a suffix of 
. If the leaves of  are listed from left to right, then the 
corresponding suffixes of  are lexicographically sorted. 
For any pattern , we can find in  time the special 
node , called the \emph{locus} of . 
The starting position of every suffix in the subtree of  is the 
location of an occurrence of . 
We define the rank of a suffix  as the number of 's suffixes 
that are lexicographically smaller than or equal to .  
The ranks of all suffixes in  belong to an interval 
, where  and  denote the ranks 
of the leftmost and the rightmost suffixes in the subtree of . 
Thus for any pattern  there is a unique range ; 
pattern  occurs at position  in  if and only if 
the rank of suffix  belongs to .  
Refer to~\cite{Gusfield1997} for a more extensive description of
suffix trees and related concepts. 

We will frequently use a special set of points, further called 
\emph{the position set for }.
Every point  in the position set corresponds to a unique suffix 
of a string ; 
the -coordinate of  equals to the rank of  and the -coordinate 
of  equals to the starting position of  in . 

  
\paragraph{Successive List Indexing.}
In this problem a query consists of a pattern  
and an index , . We want to find the first (leftmost) 
occurrence of   at position . 
A successive list indexing query  is equivalent to finding the 
point  from the position set such that  belongs to the range   and the -coordinate of  is minimal. 
Thus a list indexing query is equivalent to a range successor query on 
the position set. Using Theorems~\ref{theor:spaceeff} and~\ref{theor:2dim} 
to answer range successor queries, we obtain the following result.
\begin{corollary}\label{cor:succind}
We can store a string  in an  space data structure, so that 
for any pattern  and any index , , the leftmost 
occurrence of  at position  can be found in  time 
for 
 (i)  and ;
(ii)  and ;
(iii)  and .
\end{corollary}

\paragraph{Range Non-Overlapping Indexing.}
In the string statistics problem we want to find the maximum number of non-overlapping occurrences of a pattern . In~\cite{KKL07wads} the 
\emph{range non-overlapping indexing problem} was introduced: 
instead of just computing the maximum number of occurrences 
we want to find the longest sequence of non-overlapping occurrences of .
It was shown \cite{KKL07wads} that the range non-overlapping 
indexing problem can be solved via  successive list indexing 
queries; here  denotes the maximal number of non-overlapping occurrences. 

\begin{corollary}\label{cor:rangenon}
The range non-overlapping indexing problem can be solved  in  time with an  space data structure, where  and  
are defined as in Corollary~\ref{cor:succind}.
\end{corollary}

Other, more far-fetched applications, are described next.

\subsection{Pattern Matching with Variable-Length Don't Cares}
\label{sec:regpat}
We must determine whether a query pattern  
occurs in . The special symbol  is the Kleene star
 symbol; it corresponds to an arbitrary sequence of (zero or more) 
characters from the original alphabet of .
The parameter  can be specified at query time. 
In~\cite{YuWK10} the authors showed how to answer such queries in 
 and  space in the case when the alphabet size 
is . In this paper we describe  a data structure 
for an arbitrarily large alphabet.
Using the approach of~\cite{YuWK10}, we can reduce such a query for   
to answering  successive list indexing queries.  
First, we identify the leftmost occurrence of  in  by answering 
the successive list indexing query . 
Let  denote the leftmost position of .  
 occurs in  if and only if  occurs at position . We find the leftmost occurrence
  of  by answering the query .  
 occurs in  at position  
if and only if  occurs at position . 
Proceeding in the same way we find the leftmost possible positions 
for . Thus we answer  successive list indexing 
queries , ; here ,  
for , and  denotes the answer to the -th query.
\begin{corollary}\label{cor:dontcares}
We can determine whether a text  contains a substring    in  time using an  space data structure, where  and  are defined as in Corollaries~\ref{cor:succind} and~\ref{cor:rangenon}.
\end{corollary}
\nono{Further applications of sorted range reporting and orthogonal range successor queries are discussed in the full version of this 
paper~\cite{NN12full}.}


\subsection{Ordered Substring Searching}
\label{sec:sortsubstr}
Suppose that a data structure contains a text  and we want to report 
occurrences of a query pattern  in the left-to-right order, i.e., in the 
same order as they appear in ; in some case we may want to find only 
the  leftmost occurrences.
 In this section we describe two solutions for this problem. Then we show how sorted range reporting 
can be used to solve the position-restricted variant of this problem. We denote by  the number of 's occurrences in  that are reported when a query is answered.

\paragraph{Data Structure with Optimal Query Time.}
Such queries can be answered in 
 time and  space using the suffix tree and the data structure  of Brodal \etal~\cite{BrodalFGL09}. 
Positions of suffixes are stored in lexicographic order in the suffix array ;
the -th entry  contains the starting position of the -th suffix in 
the lexicographic order.  In~\cite{BrodalFGL09} the authors described an  space data structure that answers online sorted range reporting queries: 
for any , we can report in  time 
all entries , , sorted in increasing order by their 
values. Occurrences of a pattern  can be reported in the left-to-right order as follows.  Using a suffix tree, we find  and  in  time. Then we report all suffixes in the interval 
 sorted by their starting positions 
using the data structure of~\cite{BrodalFGL09} on . 
\begin{corollary}\label{cor:sortlin}
We can answer a sorted  substring matching query in  time using a  space data structure
\end{corollary}

\paragraph{Succinct Data Structure.}
The space usage of a data structure for sorted pattern matching can be 
further reduced.  We store a compressed suffix array for  and a 
succinct data structure for range minimum queries. 
We use the implementation of the compressed suffix array described in~\cite{GrossiGV03} that needs  bits for , where  denotes  the alphabet size  and  is the -th order entropy.
Using the results of~\cite{GrossiGV03}, we can find the position of the -th lexicographically smallest  suffix in  time.  
We can also find  and  for any  in 
 time. 
We also store the range minimum data structure~\cite{Fis10} for the 
array  defined above.  For any , we can find  such  that 
 for any . Observe that  itself is not stored; we only store the structure from~\cite{Fis10} that uses  bits of space. 
Now occurrences of  are reported as follows. 
An initially empty queue  contains suffix positions;
with every suffix position  we also store an interval  and the rank  of the suffix that starts at 
position . Let  and . We find 
 and  the position  of the suffix with 
rank . The position  with its rank  and the associated interval  is inserted into . We repeat the following steps until  is empty.
The item with the minimal value of  is extracted from 
. Let  and  denote the rank and interval stored with . We answer queries  and 
 and identify the positions ,  of suffixes with ranks , . Finally, we insert items 
 and  into 
. Using the van Emde Boas data structure, we can 
implement each operation on  in  time. 
We can find the position of a suffix with rank  in 
 time. Thus the total time that 
we need to answer a query is . 
Our data structure uses   bits. 
We observe however that we need  additional 
bits at the time when a query is answered. 
\begin{corollary}\label{cor:sortcomp}
If the alphabet size , then 
we can answer an ordered substring searching   query in  time 
using a -bit data structure
\end{corollary}

\paragraph{Position-Restricted Ordered Substring Searching}
The position restricted substring searching problem was introduced by  M{\"a}kinen and Navarro in~\cite{MakinN07}. Given a range  
we want to report all occurrences of  that start at 
position , . 
If we want to report occurrences of  at positions from 
 in the sorted order, then this is equivalent to 
answering a sorted range reporting query . Hence, we can obtain the same 
time-space trade-offs as in Theorems~\ref{theor:spaceeff} and~\ref{theor:2dim}.


\subsection{Maximal  Points in a 2D Range and Rectangular Visibility}

A point  \emph{dominates} another point  if  
and . A point  is \emph{a maximal point} 
if  is not dominated by any other point . 
In a two-dimensional maximal points range  query, we must find  
all maximal points in   for a query rectangle .
We refer to~\cite{BT11} and references therein for description of 
previous results.

We can answer such  queries using orthogonal range successor queries. 
For simplicity, we assume that all points have different - and -coordinates.
Suppose that maximal points in the range 
 must be listed. For , we report a point  
 such that  for any , where 
 and  for .
 Our reporting procedure is completed when .
Clearly, finding a point  or determining that no such  
exists is equivalent to answering a range successor query for 
. Thus we can find  maximal points in 
 time using an  space data structure, 
where  and  are again defined as in Corollary~\ref{cor:succind}.

A point  is rectangularly visible from a point  
if , where  is the rectangle 
with points  and  at its opposite corners. 
In the rectangle visibility problem, we must determine all 
points  that are visible from a query point .  
Rectangular visibility problem is equivalent to 
finding maximal points in  for .
Hence, we can find points rectangularly visible from 
 in  time using an  space data structure. 

\bibliographystyle{abbrv}
\bibliography{sorted-rep}

\end{document}
