\documentclass{llncs}

\usepackage{multicol}

\usepackage{amsmath}
\usepackage{amsfonts}
\usepackage{amssymb}

\usepackage{graphicx}
\usepackage{epic}
\usepackage{eepic}
\usepackage{epsfig,float}
\usepackage{verbatim}
\usepackage{pdfsync}


\pagestyle{plain}
\DeclareGraphicsRule{.tif}{png}{.png}{`convert #1 `dirname #1`/`basename #1 .tif`.png}

\newcommand{\spd}{} 
\newcommand{\ol}{\overline}
\newcommand{\eps}{\varepsilon}
\newcommand{\emp}{\emptyset}
\newcommand{\rhoR}{R}
\newcommand{\Sig}{\Sigma}
\newcommand{\sig}{\sigma}
\newcommand{\noin}{\noindent}
\newcommand{\pf}{prefix-focused}
\newcommand{\ur}{uniquely reachable}
\newcommand{\bi}{\begin{itemize}}
\newcommand{\ei}{\end{itemize}}
\newcommand{\be}{\begin{enumerate}}
\newcommand{\ee}{\end{enumerate}}
\newcommand{\bd}{\begin{description}}
\newcommand{\ed}{\end{description}}
\newcommand{\bq}{\begin{quote}}
\newcommand{\eq}{\end{quote}}
\newcommand{\txt}[1]{\mbox{ #1 }}




\newcommand{\floor}[1]{\lfloor #1 \rfloor}

\newcommand{\etc}{\mbox{\it etc.}}
\newcommand{\ie}{\mbox{\it i.e.}}
\newcommand{\eg}{\mbox{\it e.g.}}
\newcommand{\FigureDirectory}{FIGS}
\renewcommand{\Im}{\text{Im}}
\newcommand{\tid}{\mbox{{\bf 1}}}
\newcommand{\N}{\mathbb N}
\newcommand{\imp}{\Rightarrow}

\newcommand{\inv}[1]{\mbox{}}

\newcommand{\stress}[1]{{\fontfamily{cmtt}\selectfont #1}}

\def\shu{\mathbin{\mathchoice
{\rule{.3pt}{1ex}\rule{.3em}{.3pt}\rule{.3pt}{1ex}
\rule{.3em}{.3pt}\rule{.3pt}{1ex}}
{\rule{.3pt}{1ex}\rule{.3em}{.3pt}\rule{.3pt}{1ex}
\rule{.3em}{.3pt}\rule{.3pt}{1ex}}
{\rule{.2pt}{.7ex}\rule{.2em}{.2pt}\rule{.2pt}{.7ex}
\rule{.2em}{.2pt}\rule{.2pt}{.7ex}}
{\rule{.3pt}{1ex}\rule{.3em}{.3pt}\rule{.3pt}{1ex}
\rule{.3em}{.3pt}\rule{.3pt}{1ex}}\mkern2mu}}

\newcommand{\cA}{{\mathcal A}}
\newcommand{\cB}{{\mathcal B}}
\newcommand{\cC}{{\mathcal C}}
\newcommand{\cD}{{\mathcal D}}
\newcommand{\cL}{{\mathcal L}}
\newcommand{\cN}{{\mathcal N}}
\newcommand{\cP}{{\mathcal P}}
\newcommand{\cR}{{\mathcal R}}
\newcommand{\cS}{{\mathcal S}}
\newcommand{\cT}{{\mathcal T}}

\newcommand{\Z}{{\mathbb Z}}

\newcommand{\one}{{\mathbf 1}}

\newcommand{\Lra}{{\hspace{.1cm}\Leftrightarrow\hspace{.1cm}}}
\newcommand{\lra}{{\hspace{.1cm}\leftrightarrow\hspace{.1cm}}}
\newcommand{\la}{{\hspace{.1cm}\leftarrow\hspace{.1cm}}}
\newcommand{\raL}{{\hspace{.1cm}{\rightarrow_L} \hspace{.1cm}}}
\newcommand{\lraL}{{\hspace{.1cm}{\leftrightarrow_L} \hspace{.1cm}}}

\newcommand{\sn}{{semiautomaton}}
\newcommand{\sa}{{semiautomata}}
\newcommand{\Sn}{{Semiautomaton}}
\newcommand{\Sa}{{Semiautomata}}
\newcommand{\se}{{settable}}
\newcommand{\Se}{{Settable}}
\newcommand{\pc}{{prefix-continuous}}
\newcommand{\tr}{{transformation}}
\newcommand{\noni}{{non-increasing}}

\newcommand{\timg}{\mbox{img}}
\newcommand{\tdom}{\mbox{dom}}

\newcommand{\qedb}{\hfill} 

\newtheorem{open}[theorem]{Open problem}
\newtheorem{prop}[theorem]{Proposition}

\newtheorem{lem}{Lemma}
\newtheorem{cor}{Corollary}
\newtheorem{conj}[theorem]{Conjecture}
\newtheorem{task}{Task}
\title{Symmetric Groups and Quotient Complexity of Boolean 
Operations\thanks{This work was supported by the Natural Sciences and Engineering Research Council of Canada under grants No.~611456 and~OGP0000871, 
by  the  European Regional Development Fund through the programme COMPETE,    and by the Portuguese Government through the FCT under projects    PEst-C/MAT/UI0144/2011 and CANTE-PTDC/EIA-CCO/101904/2008.
}
}

\author{Jason Bell\inst{1} \and
Janusz~Brzozowski\inst{2} \and 
Nelma Moreira\inst{3} \and Rog\'erio Reis\inst{3}}

\authorrunning{Bell, Brzozowski, Moreira, Reis}   

\institute{Department of Pure Mathematics, 
University of Waterloo, \\
Waterloo, ON, Canada N2L 3G1\\
\{{\tt jpbell@uwaterloo.ca}\}  \and 
David R. Cheriton School of Computer Science, University of Waterloo, \\
Waterloo, ON, Canada N2L 3G1\\
\{{\tt brzozo@uwaterloo.ca}\}
\and
CMUP \& DCC, Faculdade de Ci{\^e}ncias da Universidade do Porto,\\
Rua do Campo Alegre, 4169--007 Porto Portugal\\
\{{\tt \{nam,rvr\}@dcc.fc.up.pt}\}
}


\begin{document}

\maketitle
\begin{abstract}
The quotient complexity of a regular language  is the number of left quotients of , which  is the same as the state complexity of .
Suppose that  and  are binary regular languages with quotient complexities  and ,  and that the transition semigroups of the minimal deterministic automata accepting  and  are  the symmetric groups  and  of degrees  and , respectively.
Denote by  any binary boolean operation that is not a constant and not a function of one argument only.
For  with  we prove that the quotient complexity of  is  if and only either (a)  or (b)  and the bases (ordered pairs of generators) of  and  are not conjugate.  For  we give examples to show that this need not hold.
In proving these results we generalize the notion of uniform minimality to direct products of automata. We also establish a non-trivial connection between complexity of boolean operations and group theory.
\medskip

\noin
{\bf Keywords:}
boolean operation, quotient complexity, regular language, state complexity,  symmetric group, transition semigroup 
\end{abstract}

\section{Motivation}

The \emph{left quotient,} or simply \emph{quotient,} of a regular language  over an alphabet  by a word 
is the regular language .
It is well known that a language is regular if and only if it has a finite number of quotients. 
Consequently, the number of quotients of a regular language, its \emph{quotient complexity,} is a natural measure of complexity of the language.
Quotient complexity is also known as \emph{state complexity,} which is the number of states in the complete minimal \emph{deterministic finite automaton} (\emph{DFA}) recognizing the language.
We prefer quotient complexity because it is a language-theoretic concept, whereas state complexity involves a completely different object, the DFA.
State complexity was first studied by Maslov~\cite{Mas70} in 1970, but it only attracted much interest after 1994 as a result of the paper by Yu, Zhuang and K.\ Salomaa~\cite{YZS94}.
For more details about state complexity see the survey by Yu~\cite{Yu01}. 
The quotient point of view was introduced in 2010 by Brzozowski~\cite{Brz10}.
In this paper we usually refer to quotient/state complexity simply as \emph{complexity}.

The problem of determining the \emph{(quotient) complexity of an operation}~\cite{Brz10,Mas70,Yu01,YZS94} on regular languages has received much attention. It is defined as the maximal complexity of the language resulting  from the operation, taken as a function of the complexities of the operands.
Languages that meet the upper bound on the complexity of an operation are \emph{witnesses} for this operation.
Although witnesses for common operations on regular languages are well known, there are occasions when one has to look for new witnesses:
\be
\item
One may be interested in  a \emph{class} of languages that have the same complexity with respect to a given operation. 
For example, let  and let  be the number of times the letter  appears in the word .
Then the intersection of the languages  and  has complexity . 
The languages  and  also meet this bound~\cite{BJL13}; hence   and   are in the same complexity class with respect to intersection.
\item
Whenever one studies complexity within a  \emph{proper subclass} of regular languages, one usually needs to find new witnesses.
For example, in the class of regular right ideals---languages  satisfying ---languages   and  are appropriate, but  and  are not.
\item
When one studies \emph{combined operations} --- operations that involve more than one basic operation,  for example, the intersection of reversed languages --- once again need new witnesses~\cite{LMSY08}. 
\ee
Before stating our result, we provide some additional background information.
The \emph{Myhill congruence}  of  is defined as follows~\cite{Myh57}: For all , 

The set  of equivalence classes of the relation  is a semigroup with concatenation as the operation; it is called the \emph{syntactic semigroup} of , which we denote by . 
It is well known that the syntactic semigroup is isomorphic to the semigroup  of transformations performed by  non-empty words on the set of states in the minimal DFA  recognizing ; this semigroup is known as the \emph{transition semigroup} of .
If  has  states, the cardinality of the transition semigroup is bounded from above by , and this bound is reachable.

The \emph{atoms}~\cite{BrTa11,BrTa12} of a regular language are non-empty intersections of left quotients of the language, some or all of which may be complemented. A regular language has at most  atoms, and their quotient complexities are known~\cite{BrTa12}.

The \emph{reverse} of a word is defined inductively: the reverse of the empty word  is , and the reverse of  with  and  is .
The reverse of a language  is .
The maximal complexity of  for  with  complexity  is , and this bound is reachable~\cite{Mir66}.
\medskip

Whenever new witnesses are used, it is necessary to prove that these witnesses  meet the required bound. It would be very useful to have results stating that \emph{if the languages in question have some  property , then they meet the upper bound for a given operation.} 
Some results of this type are now now briefly discussed. 


Let \textbf{MSC} denote the class of languages with \emph{maximal syntactic complexity} (languages with largest syntactic semigroups), let \textbf{STT} denote the class of languages whose minimal DFAs have \emph{set-transitive transition semigroups} (for any two sets of states of the same cardinality there is a transformation that maps one set to the other), let \textbf{MAL} denote the class of \emph{maximally atomic languages} (languages that have all  atoms, all of which have maximal possible quotient complexity), let \textbf{MNA} denote the class of languages with the \emph{maximal number {\rm () of atoms}}, and let \textbf{MCR} denote the class of languages with a \emph{maximally complex reverse} (reverse of complexity ). The following relations hold~\cite{BrDa13a}:

The fact that \subset
is a result of  A. Salomaa, Wood, and Yu~\cite{SWY04},
and the observation that  was made by Brzozowski and Tamm~\cite{BrTa11}.


 Our main theorem is a similar result for binary boolean operations on regular languages. We say that such a boolean operation is \emph{proper} if  is not a constant  and not a function of one variable only. 

Let  denote the symmetric group of degree . 
A \emph{basis}~\cite{Pic39} of 
is an ordered pair  of distinct transformations of  that generate .
Two bases  and  of  are \emph{conjugate} if there exists a transformation  such that , and  .

We are interested in DFAs whose transition semigroups are symmetric groups.
Assume that a DFA  (respectively, ) has state set  (), and transition semigroup  ().
Let  () be the language accepted by  ().
Our main result is the following:
\begin{theorem}
\label{thm:main}
Let  and  be binary DFAs with  and  states respectively, where 
 and .
If  the transition semigroups of  and  are  and  respectively, and  is a proper binary boolean operation,
then the quotient complexity of  is , unless
 and the bases of the transition semigroups of  and  are conjugate, in which case the quotient complexity of  is .
\end{theorem}


This theorem is a generalization of some results of Brzozowski and Liu~\cite{Brz12,BrLiu12} which will be stated in the next section.
\medskip

The proof that the quotient complexity of a binary boolean operation on two languages is maximal involves two steps. First, one proves that the direct product of the minimal DFAs of the languages is connected, meaning that all of its states are reachable from the initial state.
Second, one verifies that every two states in the direct product are distinguishable by some word, that is, that they are not equivalent.

The remainder of the paper is structured as follows: Section~\ref{sec:term} defines our terminology and notation. Section~\ref{sec:connect} deals with the conditions under which the direct product of two automata is connected. Section~\ref{sec:uniform} studies uniformly minimal semiautomata, that is, semiautomata which become minimal DFAs if one adds an arbitrary set of final states, other than the empty set and the set of all states. 
Section~\ref{sec:main} contains our main result relating symmetric groups to the complexity of boolean operations, for all except a few cases which are dealt with in Section~\ref{sec:small}.
Section~\ref{sec:conc} concludes the paper.

\section{Preliminaries}
\label{sec:term}

\subsection{Groups}
Many results in this paper rely heavily on the theory of finite groups. Here we provide only some basic definitions, and refer the reader to texts on group theory, for 
example~\cite{Rot65,Suz82}, for additional information.

A \emph{semigroup}  is a set  with an associative binary operation , which we call \emph{multiplication} and often omit. A \emph{monoid}  is a semigroup with an \emph{identity}
, which is an element of  satisfying  for all .
A \emph{group} is a monoid , such that every element  has an \emph{inverse}  that satisfies .
The \emph{order} of a group  is the number of elements in .
If , then  is a \emph{conjugate} of  (by ).

Let  and  be groups.
A \emph{homomorphism}  is a mapping satisfying 
.
If  is a homomorphism, the set  is the \emph{image} of , and the set  is the \emph{kernel} of .


A non-empty subset  of a group  is a \emph{subgroup} of , if 
 is a group under the operation of .
If  is a subset of a group , then the smallest subgroup of  containing  is \emph{the subgroup of  generated by }.

For non-empty subsets ,  of a group , define
 to be .
If  we write  for .
Let  be a subgroup of a group , and let ; then  () is a 
\emph{right coset} (\emph{left coset}) of  in , and  is a \emph{representative} of  and .
If  is a subgroup of , the number of right cosets of  is the same as the number of left cosets of . The \emph{index} of  in  is the number of right (or left) cosets of  in .
A subgroup  of  is \emph{normal} if  is a subset of  for all .


\subsection{Transformations}

A {\em transformation} of a set  is a mapping of  into itself. 
We consider only transformations of finite non-empty sets and, 
without loss of generality, assume that . If  is a transformation of 
and  , then  is the image of  under .  
An arbitrary transformation is written in the form

where , , and . 
The {\em composition} of two transformations  and  of  is a transformation  such that  for all . We usually omit the composition operator and write . 
The set of all transformations of  is a monoid with the identity transformation as the unit and composition as the operation.
A \emph{permutation} of  is a mapping of  \emph{onto} itself. 
In this paper we are concerned only with permutations.
The \emph{identity} transformation is denoted by .

A permutation  is a \emph{cycle of of length } or a  \emph{-cycle} , where , if there exist pairwise different elements ,~\dots,~ such that
, , \dots, , and , and  does not affect any other elements.
A~cycle is denoted by .
A \emph{transposition} is a 2-cycle.  
Every permutation is a product (composition) of transpositions, and the parity of the number of transpositions in the factorization is an invariant. A permutation is \emph{odd} (\emph{even}) if its factorization has an odd (even) number of factors.

The \emph{symmetric group}  of \emph{degree}  is the set of all permutations of , with composition as the group operation, and the identity transformation as .
The \emph{alternating group}  is the set of all even permutations of .

Given a subgroup  of , we say that  \emph{acts transitively} on  if for each  there is some  such that .  We say that  \emph{acts doubly transitively} on  if whenever  with  and  there is some  such that , .  


\subsection{Semiautomata and Automata}
A \emph{deterministic finite semiautomaton (DFS)} is a quadruple , where 
 is the set of \emph{states},  is a finite non-empty \emph{alphabet},  is the \emph{transition function}, and  is the \emph{initial state}. We extend  to  in the usual way.
A state  is  \emph{reachable} from the initial state if  there is a word  such that 
. 
A DFS is \emph{connected} if every state  is reachable.

For a  DFS  and a word , the transition function   is a transformation of , the transformation \emph{induced by }. 
The set of all transformations induced by non-empty words is the \emph{transition semigroup}  of .
For , we denote by  the transformation  of  induced by .

Given semiautomata  and ,
we define their direct product to be the DFS
.


A~\emph{deterministic finite automaton (DFA)} is a quintuple , where  is a DFS and 
 is the set of \emph{final states}. 
The DFA  \emph{accepts} a word  if . 
The set of all words accepted by  is the \emph{language}  of . 
The \emph{language accepted from a state}  of a DFA is the language  accepted by the DFA
.
Two states of a DFA are \emph{distinguishable} if there exists a word 
which is accepted from one of the states and rejected from the other. Otherwise,
the two states are \emph{equivalent}.
A DFA is \emph{minimal} if all of its states are reachable from the initial state and no two states are equivalent. 
Note that if  and  is minimal, then .

\subsection{An Earlier Result}
Let   and
, where   and , and let  be the language of .
Similarly,  let
, where   and , and let  be the language of .
Also,  let
, where   and , and let  be the language of .

Let  denote union, intersection, difference, or symmetric difference.
The following results were proved in~\cite{Brz12,BrLiu12}:
\begin{proposition}
\label{prop:BrzLiu}
For ,  and  as above and ,  (a) the  complexity of  is , and (b) if , the  complexity of  is .
 \end{proposition}
 
Our main theorem is a generalization of this result.
\section{Connectedness}
\label{sec:connect}

From now on we are interested in semiautomata  and  whose transition semigroups are symmetric groups generated by two-element bases.
We assume that permutations  and  are induced by   in  and , and permutations  and , by  , that is,
,  in , and ,  in .

\begin{example}
Let , , and , where
,  in , 
and ,  
in .
Then  and  are conjugate, since  and
 for .
On the other hand, if   and ,  then  and  are not conjugate.


The transition semigroups of ,  and  all have 6 elements.
Those of  and , when viewed as semigroups generated by  and , are identical, but those of  and  are not: for example,
 in , but  in .
\qedb
\end{example}

\begin{theorem}
\label{thm: reach}
Let  , let  and  be 
semiautomata with transition semigroups  that are  symmetric groups  of degrees  and  respectively, and let the corresponding bases be  and .  For , the direct product  is connected if and only if either (1)  or (2)  and  and  are not conjugate. 
\end{theorem}

\begin{proof}
Without loss of generality, assume that .
Let  denote the transition semigroup of ; then  is a subgroup of . Define homomorphisms  and  by  and . Observe that  and  are surjective, since the transition semigroups of  and  are  and  respectively.  We let  denote the subgroup of  consisting of all elements that map the set  to itself.  Then  has index  in  and thus  has index at most  in .
Thus the order of  is at least .  

Since a subgroup of  that does not act transitively on  is necessarily isomorphic to a subgroup of  for some ~\cite[Section 2.5.1]{Wil09},  a subgroup of  whose order is strictly greater than  acts transitively on .  Moreover, a subgroup of order  that does not act transitively on  is isomorphic to ; that is, it is the stabilizer of a point.  Thus  fails to act transitively on  if and only if  and  is the stabilizer of a point. 

Suppose that  or  and  is not the stabilizer of a point,  which is equivalent to assuming that  acts transitively on .  We claim that the direct product  is connected.  To see this, notice that given  and  in , we can find  (respectively ) in  that  sends  to  (respectively  to ) for some  (respectively ) in , since  acts transitively on .  
Since we have assumed that  acts transitively on , we can find  such that   sends  to . Hence 
 sends  to , and so  is connected. 




Suppose next that  and  is the stabilizer of a point.   By relabelling if necessary, we may assume that  stabilizes .   Then  cannot send  to  for  and so  is not connected.  We claim that the bases  and  are conjugate.  

To prove this claim, note that  has the property that if  and , then .  
We claim there is a permutation  with  such that if  sends  to , then . First suppose that  have the property that there is some  such that  and  are in the orbit of  under the action of .  Then we can pick  in  such that .  Then  and  are both in the orbit of , which means that , giving .  It follows that there is a map  with  such that,
if  sends  to , then .  Since  acts transitively on , the map  must be surjective and hence is a permutation, as claimed.

Let  denote the elements in the transition semigroup corresponding to , and let   correspond to .   
Let  be the group generated by .
Then  is conjugate to  (we conjugate  by  to obtain ); furthermore,   has the property that if  sends  to , then .  Thus  acts transitively on the diagonal of ; if  then  for all , which gives that . Hence, if , then   and so the bases  and  are conjugate.  
Thus if   
 is not connected, then  and the bases  and  are conjugate.


 Now we show the converse: If  and the bases  and  are conjugate, then  is not connected.
If  , and  , 
let  be the mapping that assigns to   the element 
.
For any , if , then 
.
Hence  the transition semigroups of  and  are isomorphic.

The direct product  is defined by , where   and
 for any .  If  is connected, then for all  there must exist a word  such that
 or, equivalently, there exists a permutation
 such that  and . 
There are now two cases:
\be 
\item
If , we prove that state  is unreachable for all .
If  is reachable, then there exists a permutation  such that
 and . But then , and so
, which is a contradiction.
\item
If , we prove that state  is unreachable for some .
Since  cannot be the identity, there must exist an  such 
that . Suppose  is reachable for that .
Then there exists a permutation
 such that  and . 
Thus  and 
, which is a contradiction.
\ee

In either case    is not connected. 
\qed
\end{proof}

\begin{remark}
\label{rem:sc}
If  is connected, then it is strongly connected, since the transition semigroup of  is a group.
\end{remark}

\section{Uniformly Minimal Semiautomata}
\label{sec:uniform}
Semiautomata that result in minimal DFAs under any non-trivial assignment of final states were studied by Restivo and Vaglica~\cite{ReVa12}.
We modify their definitions slightly to suit our purposes.
A strongly connected DFS  with  is \emph{uniformly minimal}
if the DFA  is minimal for each set  of final states, where .

Given a DFS , we define the \emph{pair graph}
of  to be the directed graph , where the set  of vertices  is the set of all two-element subsets  of , and the set  of edges consists of unordered pairs  such that 
.
The following result was proved in~\cite{ReVa12}:

\begin{proposition}[Restivo and Vaglica]
\label{prop:Restivo}
Let  be a strongly connected DFS with at least two states. If the pair graph
 is strongly connected, then  is uniformly minimal.
\end{proposition}

We prove a similar result for semiautomata   with transition semigroups  that are the symmetric groups. 

\begin{proposition}
\label{prop:DFS}
Suppose that  is a DFS and the transition semigroup  of  is the symmetric group . Then  is strongly connected and uniformly minimal.
\end{proposition}
\begin{proof}
If  , then  contains all permutations of , in particular, the cycle ; hence  is strongly connected. For any  with , , and ,  any permutation that maps  to  and  to  connects  to  in the pair graph of . Hence the pair graph is strongly connected, and  is uniformly minimal by Proposition~\ref{prop:Restivo}.
\qed
\end{proof}


Let the truth values of propositions be 1 (true) and 0 (false). Let  be a binary boolean function.
Extend  to a function
:
If  and , 
then 
Also, extend  to a function
:
If , , , and , 
then 


Suppose that  and  with  and  are uniformly minimal DFSs, and 
 is any proper boolean function.
The pair    is \emph{uniformly minimal for }
if the direct product 
is minimal for all valid assignments of  and  of sets of final states to  and , that is, sets  and  such that  and .

If ,  then  is isomorphic to  and  no
boolean function  is proper.
Hence this case, and also the case with , is of no interest.
Henceforth we assume that  .

We now consider  pair graphs of DFSs with symmetric groups as their transition semigroups.  

\begin{example}
\label{ex:22}
Suppose now that , and  and  both have  as their transition semigroup.
There are two permutations in :  and , and 
there are  three bases:  ,
, and
.
Note that no two of these bases are conjugate.

For each basis, there are two possible final states, 0 or 1, and  hence  two DFAs; thus there are six different DFAs.
There are then twelve direct products  with non-conjugate bases, where
 () uses basis  () and has  () as final state, for  and .

For each pair of DFAs accepting languages  and  respectively, we tested the complexity of five boolean functions: 
,  ,  ,  and . Note that the complexity of each remaining proper boolean function is the same as that of one of these five functions.
For all twelve direct products of DFAs with non-conjugate bases, all proper boolean functions reach the maximal complexity 4, except for the functions  and , which fail in all twelve cases.
Thus any two DFAs  and , where
, ,  () is defined by basis  (),
 and , are uniformly minimal for all proper boolean functions, except  and its complement. So our main result applies only in some cases if .
\qedb
\end{example}



\begin{proposition}
\label{prop:pairgraph}
Let  and , with  and , be 
DFSs with transition semigroups  that are symmetric groups, and let  be their direct product.
Then the following hold:
\be
\item
The pair graph of  consists of strongly connected components---which we will call simply \emph{components}---of one of the following three types:
\bi
\item
, 
\item
, 
\item
. 
\ei
\item
Every state  of the direct product  appears in at least one pair in each component.\item
Each component has at least  pairs.
\ee
\end{proposition}
\begin{proof}
The first claim follows since the transition semigroup of  is a group.
The second claim holds because the direct product is strongly connected, by Remark~\ref{rem:sc}.
For the third claim, note that there are  states in , but they can appear in pairs; hence the bound . Since we are assuming that , the last claim follows.
\qed
\end{proof}

Now consider DFAs  and , where
 and .
A~state  of the pair graph of the direct product  of  and  is \emph{distinguishing} if and only if  is final and  is not, or \emph{vice versa}.

\begin{example}
\label{ex:23}
Suppose  and ,  and  are as above,  is defined by the the basis   of , and  by the basis  of .
The direct product of  and  is connected as guaranteed by Theorem~\ref{thm: reach}, and has six states. The
 components in the pair graph are:
\bi
\item
, 
\item
,
\item
, 
\item
.
\ei
One verifies that if ,  and the boolean function is symmetric difference,  the distinguishing pairs are  in boldface. 
We return to this case in Section~\ref{sec:small}.\qedb
\end{example}

\begin{example}
\label{ex:34}
If  and ,  is defined by the basis  of ,   by  the basis  of . One verifies that these bases are not conjugate.
The direct product   is connected and has twelve states.

If ,  and intersection is the boolean function,
the component of the pair graph containing  is:\\
\mbox{\hspace{1cm} } \\
\\
and there are no distinguishing pairs. Hence states  and  are equivalent in , as are also any two states appearing in the same pair of .
Indeed, the minimal version of  has exactly six states.
For symmetric difference, there are only four states, but there are twelve states for union.
Here our theorem applies only in some cases if  and .
\qedb
\end{example}

\begin{example}
\label{ex:44}
Suppose ,   is defined by the basis , and   by  the basis
. If  and
,
then the complexity of  is 4, but all the other complexities are 12.
The same holds if  and .
Again, our theorem applies only in some cases if .
\qedb
\end{example}





\begin{lemma}
\label{lem:dist}
Let  and , with , be 
DFAs with transition semigroups  that are groups, and let  be their direct product.
Then  is minimal if and only if every component of the pair graph  of  has a distinguishing pair.
\end{lemma}
\begin{proof}  Let  be the transition group of the direct product .  
Suppose  corresponds to the transformation of  induced by some word ; 
then for , define  to be .

Suppose first that every component of  has a distinguishing pair, but  is not minimal.   Then there must be two distinct states  such that, for ,  is a final state if and only if  is also final.
By assumption, there is a distinguishing pair  in the component of  that contains .  By interchanging  and  if necessary, we may assume that there is some  such that 
 and .  But this is a contradiction.

Conversely, suppose there is a component  without a distinguishing pair. Then, if 
 and  appear in the same pair, they must be equivalent since they can only reach states that are both final or both non-final. 
\qed
\end{proof}



\section{Symmetric Groups and Complexity of Boolean Operations}
\label{sec:main}

We begin with a well-known but apparently unpublished result.
\begin{lemma}
Let  be a positive integer, let  be either  or , and let  be a subgroup of  of index .  Then the following hold:
\begin{enumerate}
\item[(i)] if  and , then  is either  or ;
\item[(ii)] if  and , then there is some  such that  is the set of permutations in  that fix .
\item[(iii)] if , then there is an automorphism  of  such that  is the set of elements that fix .\end{enumerate}
\label{lem: bertrand}
\end{lemma}
\begin{proof} For , both (i) and (ii) are clear.  Thus assume that .  
Let  be the set of left cosets of  in .   Note that  acts on  via left multiplication; more explicitly,  for , there is a permutation  such that  for all .  The map  gives a non-trivial homomorphism  from  into .  Furthermore, the kernel of  is necessarily contained in , since the kernel of  is  and this is contained in .    

If , then  and hence  must have a non-trivial kernel, which is a normal subgroup of .  For , the only normal subgroups of  are either  or . Since the kernel of  is a normal subgroup contained in ,   must be either  or , if .   This establishes (i).

On the other hand, if  and ,  we have a non-trivial homomorphism .  If the kernel is non-trivial, then  again  must be  or , which contradicts the fact that  has index .  If the kernel is trivial, then  gives an embedding of  into .  If  then  is an automorphism.  If  then the image of  is an index-two subgroup of  and hence necessarily .  Thus  gives an automorphism of  in either case.  For , all automorphisms of  or  are given by conjugation by an element of  (see \cite[Chapter 3.2]{Suz82}).  Since  stabilizes the coset , the definition of the map  now gives , and so  stabilizes 0.  Since  is given by conjugation by an element of , we see that  consists of all elements of  that stabilize some .  Thus we have proved that (ii) holds except when .  This argument also gives (iii) immediately.  

If , as before we have a non-trivial homomorphism  and the kernel must be one of , ,  (the Klein 4-group), or the trivial group.  Since the kernel of  is contained in  and  has order  or ,  the kernel is in fact trivial and  is an embedding.   The argument used above now proves (ii) in this case.  
\qed  
\end{proof}

The following lemma, like Theorem~\ref{thm: reach}, deals with reachability. The conditions in the lemma, however, are useful for determining reachability in the pair graph of , rather than in   itself.
\begin{lemma}
Let  , let  and  be 
semiautomata with transition semigroups  that are  symmetric groups  of degrees  and  respectively with ,  and .  
Let  be the transition semigroup of , and let  and  be the natural projections from  onto  and  respectively. If
 
then 
\be
\item  is either  or , or is the stabilizer of a point in . 
\item
 is the stabilizer of a point if and only if , and in this case the direct product  is not connected.
\ee
\label{lem: H0}
\end{lemma}
\begin{proof}
For Part 1, 
since , for each  there is some  such that  takes  to .  For a given ,  takes  to  for some , and thus  and so .  However, since  takes  to , we have  and thus  for . Thus the cosets  are distinct, and   has index  in .  Since 
 
 has index at most  in .   If  and  then  is either  or  by Lemma \ref{lem: bertrand}. 
If  and , then  has index  in  and hence must be the stabilizer of a some  by Lemma \ref{lem: bertrand}.  

For Part 2, suppose that  and  is the stabilizer of a point in .
By relabelling if necessary, we may assume that  stabilizes .  Hence, if  sends  to  then .  In particular, there is no  that sends  to  or that sends  to  and so  is necessarily not connected.
\qed
\end{proof}


 
\begin{lemma} 
\label{lem:main}
Let  and  be 
semiautomata with transition semigroups  that are the symmetric groups  of degrees  and , respectively with , , , and .  If  is connected then the pair graph of  has exactly three connected components:
, , and 
. 

\end{lemma}
\begin{proof}
We let  denote the transition semigroup of .  In addition to this, we let 
, , and 
.  We show that each of  is strongly connected.  Note that each of , ,  is necessarily a union of connected components. 


We show that  is strongly connected.  Suppose we have pairs  and  with  distinct,  distinct,  distinct, and  distinct.  Since  acts doubly transitively on  when ,  there is some  that sends  to  and  to  for some .

Thus we may assume without loss of generality that  and .  Let  be the subgroup of  consisting of all  such that  fixes .  By Lemma \ref{lem: H0},  since we assume that  is connected,
 is not a stabilizer of a point in .
Hence  is either  or .  
Let  denote the subgroup of  consisting of all  such that  fixes  and .  
By the argument used in Lemma \ref{lem: H0} to show that  has index  in , we see that  has index at most  in .  
Thus  is a subgroup of  or  of index at most , and hence must again be  or  by Lemma \ref{lem: bertrand}.  
Since  and  both act doubly transitively on , there is some  that sends  to  and  to  whenever  and  are distinct.  This proves that  is indeed a strongly connected component.

Next, consider pairs  with  distinct.  For given  with  distinct, there is some element  such that  and thus  sends
 to  and  to  for some  with .  Now note that  is either  or  by Lemma \ref{lem: H0}, and thus acts doubly transitively on .  It follows that there is some  such that
 sends  to  and  to .  Then  sends  to  and thus  is strongly connected.

Finally, consider pairs  and  with  distinct and  distinct.  From the argument used in proving  is strongly connected, we see that we can find  that sends
 to  for some .  As in the proof that  is strongly connected,  we see that the image of the set of  for which  stabilizes both  and  under  acts transitively on ; hence we can find  that  sends  to .  Thus  is strongly connected.
\qed
\end{proof} 

We are now in a position to prove our main result in all except a few cases which are dealt with in Section~\ref{sec:small}.


\begin{corollary}
Let  and  be positive integers with , , and , and let  and  be 
semiautomata with transition semigroups  that are the symmetric groups  of degrees  and . Suppose that the direct product  is connected and assume further that sets of final states are added to  and  and that  is a proper binary boolean function that defines the set of final states of the direct product . Then  is minimal for any such .
\label{cor:main}

\end{corollary}
\begin{proof} 
By Lemma~\ref{lem:main}, the pair graph of  has three strongly connected components:
, , and 
.  

For , define  to be  if  is a final state, and  , otherwise.  We first claim that  has a distinguishing pair, that is, there are pairs  and  in  with  and  such that .  

Suppose no distinguishing pair exists in .
Assume without loss of generality that .
 then  whenever  and .  
Given , we pick ; this is always possible since .  
Since  is in  and we have assumed that  has no distinguishing pairs, we must have .
But  must be 0, for otherwise we would have the distinguishing pair .
Hence .   
Thus we have
 for every  and every .  
Similarly, we must have  for ,  and hence  is the zero function, a contradiction.  

The fact that  and  both have distinguishing pairs follows from the fact that  is a proper boolean function.
By Lemma~\ref{lem:dist}, we conclude that   is uniformly minimal.
\qed
\end{proof}


\section{Results for Small Values of  and }
\label{sec:small}
We have proved our main result in the case that  and  if .  By symmetry we may always assume that .  The case  was handled in Example~\ref{ex:22}, that of , in Example~\ref{ex:34}, and that
of , in Example~\ref{ex:44}. Therefore the only cases that we need to consider are those with .  

In this section we prove the following result:
\begin{theorem}
\label{thm: small}
Let  and  be 
semiautomata with transition semigroups that are   and  respectively. Suppose that the direct product  is connected, sets of final states are added to  and , and   is a proper binary boolean function that defines the set of final states of the direct product . If , then  is minimal for any such .
\end{theorem}
The theorem is proved in four parts, since each case requires  a different argument.
The following remark, however, is common to all parts.

\begin{remark}
\label{rem:equiv}
If  there is a proper boolean function  for which  is not minimal, then there must be two distinct states   such that  is final if and only if  is final for any  in the transition semigroup of~.  

Define an equivalence relation on  by declaring that  precisely when   is final if and only if  is final for all .  
This equivalence relation partitions  into disjoint parts.  Moreover, each equivalence class must have the same size, since  is connected; in particular, each equivalence class has size equal to a divisor of .  If each part in the partition has size~, then  is minimal and there is nothing to prove.   
If there is exactly one part in the partition, then either all states of  are final or all non-final; in either case  is not proper, a contradiction. 
\end{remark}
\subsection{}

\begin{lemma}
Let  be an outer automorphism.  
\begin{enumerate}
\item[(1)] Let  be a subgroup of  of order . 
If  has a fixed point, then   does not, and 
if  has a fixed point, then  does not.
\item[(2)] 
If  is as in (1), then either  or  has a fixed point. 
\item[(3)]
If  is a subgroup of index  in the stabilizer subgroup of some , then  acts doubly transitively on .
\end{enumerate}
\label{lem:s6}
\end{lemma}


\begin{proof}
We first show (1).  If  stabilizes some point, then  contains a transposition since it has order 120.  
Since  is outer, it sends any transposition to a product of three disjoint transpositions~\cite{MR0096724}. 
Since the product of three disjoint transpositions has no fixed points,   cannot have a fixed point.  
Similarly,  if  has a fixed point then  cannot have a fixed point.  
Since  is inner~~\cite[p. 133]{Rot65},  is conjugate to  and hence  cannot have a fixed point.  

For (2), we must show that at least one of  and  fixes some point.  
Suppose that  does not have a fixed point.  
Then  acts on the left cosets of  by left multiplication, and this gives a map  (we think of the copy of  on the right-hand side as acting on cosets of ).  
Note that the kernel of  is contained in , and since  and  are the only non-trivial normal subgroups of , 
 the kernel of  is trivial and so  is an automorphism.  
 Note that  stabilizes the coset  by definition of our map and hence  has a fixed point in .  
 Since  does not have a fixed point, 
  cannot be an inner automorphism.  
 Since the inner automorphism group of  has index  in the full automorphism group~\cite[p. 133]{Rot65}, 
   can be obtained by composing  with an inner automorphism and so  has a fixed point.  Thus we have shown that if  has no fixed point,  then   does. 
 It follows that exactly one of  and  has a fixed point.

 
We now prove (3).  We first show that  acts transitively on .  If it did not, then  would be contained in a conjugate of a subgroup of  of the form  for some .  Since  for , we see that  would necessarily fix some .  By replacing  by  composed with some appropriate inner automorphism, we may assume that our outer automorphism  has the property that both  and  fix .  This means that  and  are both contained in the stabilizer subgroup, , of , which is a group of order .  We claim that .  If not, then  is normal of index  in both  and , and so by the second isomorphism theorem~\cite[p. 26]{Rot65},  generates a group of order  in .  But this is impossible by Lemma \ref{lem: bertrand}, since this group would have index  in .  It follows that , which contradicts (1), since both  and  fix .  Thus  acts transitively on .  

To show that  acts doubly transitively, it suffices to prove that the set of elements of  that stabilize  acts transitively on .   The orbit of  under the action of  has size ; hence the stabilizer is an index- subgroup of  and thus has size .   
Since it has size , it must contain a 5-cycle  on the elements , and for given , we have that  for some .  
It follows that  acts doubly transitively on .  
\qed
\end{proof}
\begin{proposition}
Theorem~\ref{thm: small} holds for .
\end{proposition}
\begin{proof}  Let  denote the transition semigroup of .  Then  is a subgroup of .  We let  and  denote the two natural surjections from  onto .  For , let  denote the subgroup of  obtained by applying  to the collection of  such that  fixes .  Then  has index at most  in  and hence must be one of , , or a group of order .  If each  is either  or , then we may follow the argument of Lemma \ref{lem:main} to show that the pair graph of  has exactly three connected components; namely,
, , and 
.  Then the argument from Corollary \ref{cor:main} shows that  is minimal whenever  is a proper binary boolean function.

After relabelling if necessary, it is sufficient to consider the case that  is a group of order .   Let .  Then  is a normal subgroup of  and hence must be one of  , , or the trivial subgroup.  
Since ,  must be trivial.  
If we define  to be ,
then  is an automorphism, and so .   Since  is connected,   cannot be an inner automorphism by Theorem~\ref{thm: reach}.  We claim that in this case,  is necessarily minimal for any proper boolean function.


Suppose  there is a proper boolean function  for which  is not minimal.  
Define the equivalence  as in Remark~\ref{rem:equiv};
then each equivalence class has size equal to a divisor of , and we can ignore the cases where that size is 1 or 36.    

Let .  Then  has size  and  acts transitively on , since  is outer.  Let ; then  has size . 
Since  has size , it contains an element of the form  where both  and  are -cycles that permute .  It follows that  and  both act transitively on . 

Now let  denote the equivalence class of ; then  has size at least .  Since  divides , we see that , but, as noted before, 36 can be ignored.  

We now do a case-by-case analysis.  

\item[Case 1: .]

In this case  the orbit of  under  has size .  Let  denote the set of elements of  that stabilize .  Then  has index  in , and hence must be equal to .

We now claim that if   and  are in  for some  and distinct , , then  is in  for all .  To see this, observe that the set of  in  for which  has index  in , and so it is a subgroup of order .  
Thus it is of the form  where  is the copy of  inside the set of elements of  that stabilize , which is isomorphic to .   
Notice that  acts doubly transitively on  by Lemma \ref{lem:s6}, since  is an outer automorphism of , and so we get the result.  A similar result holds for , which means that, for a fixed , the set of  for which  is in  is either  or empty, and so our boolean function is a function of the first variable, a contradiction.

\item[Case 2. .]

In this case, the orbit of  under  has size either  or  and thus the stabilizer of  has index  or  in .  But  is isomorphic to  and hence has no subgroups of index  or  by Lemma \ref{lem: bertrand} (i).

\item[Case 3. .]
By the remarks above, we have 
 for some permutation  of 

The orbit of  under  has size , and so the stabilizer, , of  in  has size .  This means  that  where  is a subgroup of  of order .  Either  or  must have a fixed point by Lemma \ref{lem:s6}.  Without loss of generality,  has a fixed point and  does not.  We extend  to a permutation of  by declaring that .  If  fixes , we have , and so  fixes , a contradiction.  The result follows.
\qed
\end{proof}

\subsection{}
\begin{lemma}   
Suppose that  is a -generated subgroup of  with the property that the two natural projections into  are surjective.  Then either  or there is some permutation  such that .  
\label{lem:s3}
\end{lemma}
\begin{proof}
Let .  Then  is a normal subgroup of  and hence must be one of , , or the trivial subgroup.  If  is trivial then the first projection is an isomorphism and hence  is isomorphic to .  Thus  is an automorphism of .  Since all automorphisms of  are inner, there exists some  such that .  

If , then we also know that  is .  Thus if ,  and  are either both even permutations or both odd permutations.  Every generating set for  must contain a transposition and thus one generator of  must be of the form  with  and  transpositions.  By conjugating by a permutation in the second coordinate, we may assume that our first generator is  for some transposition .  Let  be the second generator for .  Then either  and  are both 3-cycles or they are both transpositions not equal to .  In both cases, either  or . By conjugating by either  or  , we see that it is no loss of generality to assume that  is generated by  and  for two elements of .  But this contradicts the fact that .

If , then  has size  and hence must be . 
\qed
\end{proof}

\begin{proposition}
Theorem~\ref{thm: small} holds for .
\end{proposition}
\begin{proof}
Let  denote the transition semigroup of .  Since  is connected,   by Lemma \ref{lem:s3}.

Suppose  there is a proper boolean function  for which  is not minimal.  
Each equivalence class of Remark~\ref{rem:equiv} has size equal to a divisor of , and we can ignore the cases where that size is 1 or 9; hence the size must be 3. 

Let  be the part in the partition of  that contains  and let  be another element of .  
Since , there exists  that is not in .  
If  and , then  acts doubly transitively on . 
Hence there exists  such that  and , which is a contradiction since either  or  is empty.   
We conclude that if , then either  or .  
Next suppose that  contains an element of the form  with   and an element of the form  with .  
Then
.  
If we let  be the pair in which  is the identity and  is a 3-cycle that sends  to , then   has size 1, a contradiction, since it is either all of  or empty.  We conclude that  is either  or .  But then  is a constant function and hence not proper.   The result follows.
\qed
\end{proof}
\subsection{}
\begin{proposition}
Theorem~\ref{thm: small} holds for .
\end{proposition}
\begin{proof}
Let  denote the transition semigroup of .  Then the natural projections from  to  and  are both surjective.  In particular,  has size either  or , and it can be verified that it contains all elements of the form  in which  and  are either both even or both odd.  

Suppose  there is a proper boolean function  for which  is not minimal.  
Each equivalence class of Remark~\ref{rem:equiv} has size equal to a divisor of , and we can ignore the cases where that size is 1 or 6; hence the size must be 2 or 3.    

Let  be a part in our partition.  
If there exist  and distinct  such that , then by relabelling we may assume that  and , .  
Since , and , we see that  and so .  
Since ,  the partition consists of the two parts  and , contradicting the fact that  is proper.  
It follows that  and each part of our partition consists of an element of the form  and an element of the form  for some .  We cannot have , since then   would be a constant function.  By relabelling if necessary, we may assume that  is one part of our partition.  Letting  act on , we see that
 and  are the remaining parts that make up our partition of .  It is no loss of generality to assume that exactly two elements of  are final states.  Let  be these two final states of . Since  and  are either both final or both non-final, either all states in  are final or none of them are.  
It is no loss of generality to assume that all states of  are final.  
But   and  are both final, where  and  are taken modulo .  Then also  and  are either both final or both non-final.  Since, modulo ,  is non-empty and ,  all states of the form  are final, and thus all states are final, contradicting that  is proper.  The result follows.
\qed
\end{proof}
\section{}
\begin{proposition}
Theorem~\ref{thm: small} holds for .
\end{proposition}
\begin{proof}
Let  denote the transition semigroup of .  Then the natural projections from  to  and  are both surjective and so  has size either  or , and  contains all elements of the form  in which  and  are either both even or both odd.

Suppose there is a proper boolean function  for which  is not minimal.  
Each equivalence class of Remark~\ref{rem:equiv} has size equal to a divisor of , and we can ignore the cases where that size is 1 or 8; hence the size must be 2 or 4. 

Let  be a part in our partition.  
If there exist  and distinct  such that , then by relabelling we may assume that  and , .  
Since , and , we see that  and so .  
Similarly, , and ; hence  and so .  
Thus .  
Since , our partition must consist of the two parts  and , which contradicts the fact that  is proper.  

Thus   and each part of our partition consists of an element of the form  and an element of the form  for some .  We  cannot have , since then   would be a constant function.  Thus, by relabelling if necessary, we may assume that  is one part of our partition.  But , and since , we see that .  However , a contradiction.  The result follows.
\qed
\end{proof}
\section{Conclusions}
\label{sec:conc}
We have shown that if the inputs of two DFAs induce transformations that constitute non-conjugate bases of symmetric groups, then the quotient complexity of all non-trivial boolean operations on the languages accepted by the DFAs is maximal, except for a few special cases when the sizes of the DFAs are small. We believe that other similar results are possible and deserve further study.
\smallskip 

\noin
{\bf Acknowledgment}
We thank Gareth Davies for his careful reading of our manu\-script and for his constructive comments.


\providecommand{\noopsort}[1]{}
\begin{thebibliography}{10}
\providecommand{\url}[1]{\texttt{#1}}
\providecommand{\urlprefix}{URL }

\bibitem{Brz10}
Brzozowski, J.: Quotient complexity of regular languages. J. Autom. Lang. Comb.
   15(1/2),  71--89 (2010)

\bibitem{Brz12}
Brzozowski, J.: In search of the most complex regular languages. In: Moreira,
  N., Reis, R. (eds.) CIAA 2012. LNCS, vol. 7381, pp. 5--24. Springer (2012),
  full journal version to appear in Int. J. Found. Comput. Sc.

\bibitem{BrDa13a}
Brzozowski, J., Davies, G.: Maximally atomic languages (Aug 2013), {\small\tt
  http://arxiv.org/abs/1308.4368}

\bibitem{BJL13}
Brzozowski, J., Jir{\'a}skov{\'a}, G., Li, B.: Quotient complexity of ideal
  languages. Theoret. Comput. Sci.  470,  36--52 (2013)

\bibitem{BrLiu12}
Brzozowski, J., Liu, D.: Universal witnesses for state complexity of basic
  operations combined with reversal (Jul 2012), {\small\tt
  http://arxiv.org/abs/1207.0535}

\bibitem{BrTa11}
Brzozowski, J., Tamm, H.: Theory of \'atomata. In: Mauri, G., Leporati, A.
  (eds.) DLT 2011. LNCS, vol. 6795, pp. 105--116. Springer (2011), full version
  at {\small\tt http://arxiv.org/abs/1102.3901}, (Jul 2013)

\bibitem{BrTa12}
Brzozowski, J., Tamm, H.: Quotient complexities of atoms of regular languages.
  In: Yen, H.C., Ibarra, O.H. (eds.) DLT 2012. LNCS, vol. 7410, pp. 50--61.
  Springer (2012), full journal version to appear in Int. J. Found. Comput. Sc.

\bibitem{LMSY08}
Liu, G., Martin-Vide, C., Salomaa, A., Yu, S.: State complexity of basic
  language operations combined with reversal. Inform. and Comput.  206,
  1178--1186 (2008)

\bibitem{Mas70}
Maslov, A.N.: Estimates of the number of states of finite automata. Dokl. Akad.
  Nauk SSSR  194,  1266--1268 (Russian). (1970), english translation: Soviet
  Math. Dokl. {\bf 11} (1970), 1373--1375

\bibitem{MR0096724}
Miller, D.W.: On a theorem of {H}\"older. Amer. Math. Monthly  65,  252--254
  (1958)

\bibitem{Mir66}
Mirkin, B.G.: On dual automata. Kibernetika (Kiev)  2,  7--10 (Russian) (1970),
  english translation: Cybernetics {\bf 2}, (1966) 6--9

\bibitem{Myh57}
Myhill, J.: Finite automata and representation of events. Wright Air
  Development Center Technical Report  57--624 (1957)

\bibitem{Pic39}
Piccard, S.: Sur les bases du groupe sym\'etrique. \v{C}asopis pro
  p\v{e}stov\'an\'i matematiky a fysiky  68(1),  15--30 (1939)

\bibitem{ReVa12}
Restivo, A., Vaglica, R.: A graph theoretic approach to automata minimality.
  Theoret. Comput. Sc.  429,  282--291 (2012)

\bibitem{Rot65}
Rotman, J.: The Theory of Groups: An Introduction. Allyn and Bacon, Inc.,
  Boston (1965)

\bibitem{SWY04}
Salomaa, A., Wood, D., Yu, S.: On the state complexity of reversals of regular
  languages. Theoret. Comput. Sci.  320,  315--329 (2004)

\bibitem{Suz82}
Suzuki, M.: Group Theory, vol.~1. Springer, Berlin New York (1982)

\bibitem{Wil09}
Wilson, R.: The Finite Simple Groups. Springer, Berlin Heidelberg New York
  (2009)

\bibitem{Yu01}
Yu, S.: State complexity of regular languages. J. Autom. Lang. Comb.  6,
  221--234 (2001)

\bibitem{YZS94}
Yu, S., Zhuang, Q., Salomaa, K.: The state complexities of some basic
  operations on regular languages. Theoret. Comput. Sci.  125(2),  315--328
  (1994)

\end{thebibliography}

\end{document}
