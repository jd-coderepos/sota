\documentclass[letter,runningheads]{llncs}

\usepackage{amsmath}
\usepackage{amssymb}
\usepackage{graphicx}
\usepackage{xspace}
\usepackage{enumerate}
\usepackage{comment}
\usepackage{cite}
\usepackage{algorithm}
\usepackage{times}
\usepackage[noend]{algpseudocode}
\usepackage[usenames]{xcolor}
\usepackage{stmaryrd} \usepackage{mathtools} 

\spnewtheorem{prop}{Property}{\bfseries}{\itshape} 
\newcommand{\false}{\texttt{false}}
\newcommand{\true}{\texttt{true}}
\newcommand{\remove}[1]{}
\newcommand{\red}[1]{\color{red} #1 \color{black}\xspace}
\newcommand{\rephrase}[3]{\noindent\textbf{#1 #2}.~\emph{#3}}
\newcommand{\rephraset}[4]{\noindent\textbf{#1 #2 #3}.~\emph{#4}}

\renewenvironment{proof}
{{\em Proof.\ }}{\hspace*{\fill}\par\vspace{2mm}}

\newenvironment{proofsketch}
{{\em Proof sketch.\ }}{\hspace*{\fill}\par\vspace{2mm}}

\newcommand{\say}[2]{\marginpar{\footnotesize\emph{\color{blue}#1:} \color{red}#2}}

\newcommand{\conf}{\otimes}

\begin{document}
\title{Advances on Testing C-Planarity of\\ Embedded Flat Clustered Graphs
\thanks{Research partially supported by the Australian Research Council (grant DE140100708).}
}
\author{Markus Chimani\inst{1} \and Giuseppe Di Battista\inst{2} \and Fabrizio Frati\inst{3} \and Karsten Klein\inst{3}}
\institute{Theoretical Computer Science, University Osnabr\"uck, Germany\\
\email{markus.chimani@uni-osnabrueck.de}
\and Dipartimento di Ingegneria, University Roma Tre, Italy\\
\email{gdb@dia.uniroma3.it}
\and School of Information Technologies, The University of Sydney, Australia\\
\email {\{fabrizio.frati,karsten.klein\}@sydney.edu.au}}
\maketitle

\begin{abstract}
We show a polynomial-time algorithm for testing -planarity of embedded flat clustered graphs with at most two vertices per cluster on each face.
\end{abstract}


\section{Introduction}
A \emph{clustered graph}  consists of a graph , called {\em underlying graph}, and of a rooted tree , called {\em inclusion tree}, representing a cluster hierarchy on . The vertices in  are the leaves of , and the inner nodes of , except for the root, are called \emph{clusters}. The vertices that are descendants of a cluster  in  {\em belong to}  or {\em are in} . A \emph{-planar drawing} of  is a planar drawing of  together with a representation of each cluster   as a simple connected region  enclosing all and only the vertices that are in ; further, the boundaries of no two such regions  and  intersect; finally, only the edges connecting vertices in  to vertices not in  cross the boundary of , and each does so only once. A clustered graph is \emph{-planar} if it admits a -planar drawing.

Clustered graphs find numerous applications in computer science~\cite{s-gc-07}, thus theoretical questions on clustered graphs have been deeply investigated. From the visualization perspective, the most intriguing question is to determine the complexity of testing -planarity of clustered graphs. Unlike for other planarity variants~\cite{s-ttphtpv-13}, like {\em upward planarity}~\cite{GT01} and {\em partial embedding planarity}~\cite{adfjkpr-tppeg-10}, the complexity of testing -planarity remains unknown since the problem was posed nearly two decades ago~\cite{fce-pcg-95}.


Polynomial-time algorithms to test the -planarity of a clustered graph  are known if  belongs to special classes of clustered graphs~\cite{cw-cccg-06,cdpp-cccc-05,df-ectefcgsf-09,fce-pcg-95,d-ltarcgp-98,cdfpp-cccg-08,gls-cecg-05,gjlmpw-actcg-02,jkkpsv-scceg-09,jstv-cfoe-08}, including {\em -connected clustered graphs}, that are clustered graphs  in which, for each cluster , the subgraph  of  induced by the vertices in  is connected~\cite{fce-pcg-95,d-ltarcgp-98,cdfpp-cccg-08}. Effective ILP formulations and FPT algorithms for testing -planarity have been presented~\cite{cgjkm-ssscp-09,ck-ssscp-12}. Generalizations of the -planarity testing problem have also been considered~\cite{addfpr-rccp-14,afp-scgc-09,dgl-ocp-08}.

An important variant of the -planarity testing problem is the one in which the clustered graph  is {\em flat} and {\em embedded}. That is, every cluster is a child of the root of  and a planar embedding for  (an order of the edges incident to each vertex) is fixed in advance; then, the -planarity testing problem asks whether a -planar drawing exists in which  has the prescribed planar embedding. This setting can be highly regarded for several reasons. First, several NP-hard graph drawing problems are polynomial-time solvable in the fixed embedding scenario, e.g., {\em upward planarity testing}~\cite{bdlm-udtd-94,GT01} and {\em bend minimization in orthogonal drawings}~\cite{t-eggmnb-87,GT01}. Second, testing -planarity of embedded flat clustered graphs generalizes testing -planarity of triconnected flat clustered graphs. Third, testing -planarity of embedded flat clustered graphs is strongly related to a seemingly different problem, that we call {\em planar set of spanning trees in topological multigraphs} ({\sc pssttm}): Given a non-planar topological multigraph  with  connected components , do spanning trees  of  exist such that no two edges in  cross? Starting from an embedded flat clustered graph , an instance  of the {\sc pssttm} problem can be constructed that admits a solution if and only if  is -planar:  is composed of the edges that can be inserted inside the faces of  between vertices of the same cluster, where each cluster defines a multigraph . The {\sc pssttm} problem is NP-hard, even if ~\cite{kln-nstl-91}.



Testing -planarity of an embedded flat clustered graph  is a polynomial-time solvable problem if  has no face with more than five vertices and, more in general, if  is a {\em single-conflict} clustered graph~\cite{df-ectefcgsf-09}, i.e., the instance  of the {\sc pssttm} problem associated with  is such that each edge has at most one crossing. A polynomial-time algorithm is also known for testing -planarity of embedded flat clustered graphs such that the graph induced by each cluster has at most two connected components~\cite{jjkl-ecgtcc-09}. Finally, the -planarity of clustered cycles with at most three clusters~\cite{cdpp-cccc-05} or with each cluster containing at most three vertices~\cite{jkkpsv-scceg-09} can be tested in polynomial time.

\subsubsection*{Contribution and outline.} In this paper we show how to test -planarity in cubic time for embedded flat clustered graphs  such that at most two vertices of each cluster are incident to any face of . While this setting might seem unnatural at a first glance, its study led to a deep (in our opinion) exploration of some combinatorial properties of highly non-planar topological graphs. Namely, every instance  of the {\sc pssttm} problem arising from our setting is such that there exists no sequence  of edges in  with  and  in the same connected component of  and with  crossing , for every ; these instances might contain a quadratic number of crossings, which is not the case for single-conflict clustered graphs~\cite{df-ectefcgsf-09}. Within our setting, performing all the ``trivial local'' tests and simplifications results in the rise of nice global structures, called {\em -donuts}, whose study was interesting to us.

The paper is organized as follows. In Section~\ref{se:preliminaries} we introduce some preliminaries; in Section~\ref{se:outline} we give an outline of our algorithm; in Section~\ref{se:algorithm} we describe our algorithm and prove its correctness; finally, in Section~\ref{se:conclusions} we conclude.

\section{Saturators, Con-Edges, and Spanning Trees} \label{se:preliminaries}

A natural approach to test -planarity of a clustered graph  is to search for a {\em saturator} for . A set  is a saturator for  if  is a -connected -planar clustered graph. Determining the existence of a saturator for  is equivalent to testing the -planarity of ~\cite{fce-pcg-95}. Thus, the core of the problem consists of determining  so that  is connected, for each , and so that  is planar.

In the context of embedded flat clustered graphs (see Fig.~\ref{fig:con-edges}(a)), the problem of finding saturators becomes seemingly simpler. Since the embedding of  is fixed, the edges in  can only be embedded inside the faces of , in order to guarantee the planarity of . This implies that, for any two edges  and  that can be inserted inside a face  of , it is known {\em a priori} whether  and  can be both in , namely only if their end-vertices do not alternate along the boundary of . Also,  can be assumed to contain only edges connecting vertices that belong to the same cluster, as edges connecting vertices belonging to different clusters ``do not help'' to connect any cluster. For the same reason,  can be assumed to contain only edges connecting vertices belonging to distinct connected components of , for each cluster .

\begin{figure}[tb]
\begin{center}
\begin{tabular}{c c c c}
\mbox{\includegraphics[scale=0.315]{Saturator1.eps}} \hspace{1mm} &
\mbox{\includegraphics[scale=0.315]{Saturator2.eps}} \hspace{1mm} &
\mbox{\includegraphics[scale=0.315]{Saturator3.eps}} \hspace{1mm} &
\mbox{\includegraphics[scale=0.315]{Saturator4.eps}}\\
(a) \hspace{1mm} & (b) \hspace{1mm} & (c) \hspace{1mm} & (d)
\end{tabular}
\caption{(a) A clustered graph . (b) Con-edges in . (c) Multigraph . (d) A planar set  of spanning trees for . Edges in  are thick and solid, while edges in  are thin and dashed.}
\label{fig:con-edges}
\end{center}
\end{figure}

Consider a face  of  and let  be the clockwise order of the occurrences of vertices along the boundary of , where  and  might be occurrences of the same vertex  (this might happen if  is a cut-vertex of ). A {\em con-edge} (short for {\em connectivity-edge}) is a pair of occurrences  of distinct vertices both belonging to a cluster , both incident to , and belonging to different connected components of  (see Fig.~\ref{fig:con-edges}(b)). If there are  distinct pairs of occurrences of vertices  and  along a single face , then there are  con-edges connecting  and  in , one for each pair of occurrences. A {\em con-edge for } is a con-edge connecting vertices in a cluster . Two con-edges  and  in  {\em have a conflict} or {\em cross} (we write ) if the occurrences in  alternate with the occurrences in  along the boundary of .

The {\em multigraph  of the con-edges} is an embedded multigraph that is defined as follows. Starting from , insert all the con-edges inside the faces of ; then, for each cluster  and for each connected component  of , contract  into a single vertex; finally, remove all the edges of .  See Fig.~\ref{fig:con-edges}(c). With a slight abuse of notation, we denote by  both the multigraph of the con-edges and the set of its edges. For each cluster , we denote by  the subgraph of  induced by the con-edges for . A {\em planar set of spanning trees for } is a set  such that: (i) for each cluster , the subset  of  induced by the con-edges for  is a tree that spans the vertices belonging to ; and (ii) there exist no two edges in  that have a conflict. See Fig.~\ref{fig:con-edges}(d). The {\sc pssttm} problem asks whether a planar set of spanning trees for  exists.

The following lemma relates the -planarity problem for embedded flat clustered graphs to the {\sc pssttm} problem.

\begin{lemma}[\cite{df-ectefcgsf-09}] \label{le:acyclic-saturator}
An embedded flat clustered graph  is -planar if and only if: (1)  is planar; (2) there exists a face  in  such that when  is chosen as outer face for  no cycle composed of vertices of the same cluster encloses a vertex of a different cluster; and (3) a planar set of spanning trees for  exists.
\end{lemma}

We now introduce the concept of {\em conflict graph} , which is defined as follows. Graph  has a vertex for each con-edge in  and has an edge  if . In the remainder of the paper we will show how to decide whether a set of planar spanning trees for  exists by assuming that the following property holds for .

\begin{prop} \label{pr:no-two-edges-same-structure}
No two con-edges for the same cluster belong to the same connected component of .
\end{prop}

We now show that  can be assumed w.l.o.g. to satisfy Property~\ref{pr:no-two-edges-same-structure}, given that  has at most two vertices per cluster incident to each face of . Consider any face  of  and any cluster  such that two vertices  and  of  are incident to .

First, no con-edge for  in  that connects a pair of vertices different from  belongs to the connected component of  containing , given that no vertex of  different from  and  is incident to . However, it might be the case that several con-edges  belong to the same connected component of , which happens if , or , or both have several occurrences on the boundary of . We show a simple reduction that gets rid of these multiple con-edges.

Denote by  the clockwise order of the occurrences of vertices along the boundary of  and assume w.l.o.g. that , , and  are occurrences of  , , and , respectively, with .

Suppose that there exist occurrences  and  in  of vertices  and  belonging to a cluster  with , with , and with , as in Fig.~\ref{fig:property1}(a). We claim that, if any planar set  of spanning trees for  exists, then  does not contain the con-edge  connecting the occurrence  of  and the occurrence  of . Namely, all the con-edges  have a conflict with ; moreover, the con-edges  form a separating set for , hence at least one of them belongs to . Thus, , and this edge can be removed from , as in Fig.~\ref{fig:property1}(b). Similar reductions can be performed if  or , and by exchanging the roles of  and .

\begin{figure}[tb]
\begin{center}
\begin{tabular}{c c c c}
\mbox{\includegraphics[scale=0.43]{Property1-a.eps}} \hspace{2mm} &
\mbox{\includegraphics[scale=0.43]{Property1-b.eps}}  \hspace{2mm} &
\mbox{\includegraphics[scale=0.43]{Property1-c.eps}}  \hspace{2mm} &
\mbox{\includegraphics[scale=0.43]{Property1-d.eps}}\\
(a) \hspace{2mm} & (b) \hspace{2mm} & (c) \hspace{2mm} & (d)
\end{tabular}
\caption{Illustration for the reduction to a multigraph of the con-edges satisfying Property~\ref{pr:no-two-edges-same-structure}.}
\label{fig:property1}
\end{center}
\end{figure}

Rename the vertex occurrences in  so that  and  are the first and the last occurrence of  in , and so that  and  are the first and the last occurrence of  in , with . If no two occurrences  and  in  as described above exist, the only con-edges  left are crossed by con-edges connecting occurrences  and  of vertices  and  in , respectively, such that  and . That is, any two con-edges  cross the same set of con-edges for clusters different from  (see Fig.~\ref{fig:property1}(c)). Hence, a single edge  can be kept in , and all the other con-edges  can be removed from  (see Fig.~\ref{fig:property1}(d)).

After repeating this reduction for all the con-edges in , an equivalent instance is eventually obtained in which Property~\ref{pr:no-two-edges-same-structure} is satisfied by . Observe that the described simplification can be easily performed in  time. Thus, we get the following:

\begin{lemma} \label{le:pssttm}
Assume that the {\sc pssttm} problem can be solved in  time for instances satisfying Property~\ref{pr:no-two-edges-same-structure}. Then the -planarity of any embedded flat clustered graph  with at most two vertices per cluster on each face can be tested in ~time.
\end{lemma}

\begin{proof}
Consider any embedded flat clustered graph  with at most two vertices per cluster on each face. Conditions (1) and (2) in Lemma~\ref{le:acyclic-saturator} can be tested in  time (see~\cite{df-ectefcgsf-09}); hence, testing the -planarity of  is equivalent to solve the {\sc pssttm} problem for . Finally, as described before the lemma, there exists an -time algorithm that modifies multigraph  so that it satisfies Property~\ref{pr:no-two-edges-same-structure}.
\end{proof}

Before proceeding with the description of the algorithm, we state a direct consequence of Property~\ref{pr:no-two-edges-same-structure} that will be useful in the upcoming proofs. Refer to Fig.~\ref{fig:inside-a-face}. Consider a set  of con-edges all belonging to the same connected component of  and such that all their end-vertices are incident to the outer face of the subgraph of  induced by . Let  be the set of clusters that have con-edges in . Then, it is possible to draw a closed curve  that passes through the end-vertices of all the edges in , that contains all the con-edges in  in its interior, and all the other con-edges for clusters in  in its exterior.

\begin{figure}[htb]
\begin{center}
\mbox{\includegraphics[scale=0.5]{InsideAFace.eps}}
\caption{Curve  (gray) and edges in  (multi-colored).}
\label{fig:inside-a-face}
\end{center}
\end{figure}

\section{Algorithm Outline} \label{se:outline}

In this section we give an outline of our algorithm for testing the existence of a planar set  of spanning trees for , where we assume that no two con-edges for the same cluster belong to the same connected component of .

Our algorithm repeatedly tries to detect certain substructures in~. When it does find one of such substructures, the algorithm either ``simplifies''  or concludes that  does not admit any planar set of spanning trees. For example, if a cluster  exists such that  is not connected, then the algorithm concludes that no planar set of spanning trees exists and terminates; as another example, if conflicting con-edges  and  for clusters  and  exist in  such that  is a bridge for , then the algorithm determines that  has to be in  and that  can be assumed not to be in .

If the algorithm determines that certain edges have to be in  or can be assumed not to be in , these edges are contracted or removed, respectively. Given a set , the operation of {\em removing}  from  consists of updating . Given a set , the operation of {\em contracting} the edges in  consists of identifying the end-vertices of each con-edge  in  (all the con-edges different from  and incident to the end-vertices of  remain in ), and of updating .

The edges in  are removed from  (contracted in ) only when this operation does not alter the possibility of finding a planar set of spanning trees for . Also, contractions are only applied to con-edges that cross no other con-edges; hence after any contraction graph  only changes because of the removal of the isolated vertices corresponding to the contracted edges.

As a consequence of a removal or of a contraction operation, the number of edges in  decreases, that is,  is ``simplified''. After any simplification due to the detection of a certain substructure in , the algorithm will run again all previous tests for the detection of the other substructures. In fact, it is possible that a certain substructure arises from performing a simplification on  (e.g., a bridge might be present in  after a set of edges has been removed from ). Since detecting each substructure that leads to a simplification in  can be performed in quadratic time, and since the initial size of  is linear in the size of , the algorithm has a cubic running time.

If none of the four tests (called {\sc Test 1--4}) and none of the eight simplifications (called {\sc Simplification 1--8}), that will be fully described in Section~\ref{se:algorithm}, applies to , then  is a {\em single-conflict} multigraph. That is, each con-edge in  crosses at most one con-edge in . A linear-time algorithm for deciding the existence of a planar set of spanning trees in a single-conflict multigraph  is known~\cite{df-ectefcgsf-09}. Hence, our algorithm uses that algorithm~\cite{df-ectefcgsf-09} to conclude the test of the existence of a planar set of spanning trees in . A pseudo-code description of our algorithm is presented in Algorithm~\ref{a:alg}.



\newcommand{\lIf}[2]{\State\algorithmicif\ {#1}\ \algorithmicthen\ {#2}}
\newcommand{\lElse}[1]{\State\algorithmicelse\ {#1}}
\renewcommand{\algorithmiccomment}[1]{\hfill{\color[rgb]{0,0.5,0} #1}}
\begin{algorithm}
\begin{algorithmic}[1]
\State ;
 \While{ con-edge that crosses more than one con-edge in }\label{step:restart}
     \If{ cluster  such that  is disconnected}
        \State \Return `` planar set of spanning trees for '' \Comment{{\sc Test 1} (L\ref{le:disconnected})}
     \EndIf
     \If{  that is a bridge of }
        \State Remove  from , insert  in , contract  in , goto \eqref{step:restart} \Comment{{\sc Simpl.~1} (L\ref{le:bridge})}
     \EndIf
     \If{ , s.t. , for }
        \State \Return `` planar set of spanning trees for '' \Comment{{\sc Test 2} (L\ref{le:conflicts-are-bipartite})}
     \EndIf
     \If{ con-edge  that is a self-loop}
        \State Remove  from , goto \eqref{step:restart} \Comment{{\sc Simpl. 2} (L\ref{le:self-loop})}
     \EndIf
     \If{ con-edge  that does not cross any con-edge in }
       \State Insert  in , contract  in , goto \eqref{step:restart} \Comment{{\sc Simpl. 3} (L\ref{le:non-conf})}
     \EndIf
     \If{ , ,  with ,  facial cycle  of , and \\\hspace{0.38cm}  ,  ,  with  s.t. }
        \State Remove  from , goto \eqref{step:restart} \Comment{{\sc Simpl. 4} (L\ref{le:deviation})}
     \EndIf
     \If{  sharing a face of  delimited by cycle  and both crossed first \\\hspace{0.38cm} by a con-edge for  and then by a con-edge for  when traversing  clockwise}
        \State \Return `` planar set of spanning trees for '' \Comment{{\sc Test 3} (L\ref{le:same-order})}
     \EndIf
     \If{  in facial cycle  of , s.t. , ,  are encountered in this  order \\\hspace{0.38cm} when traversing  clockwise,  ,  , s.t. , \\\hspace{0.38cm} , and , and such that  is crossed first by  and then by  when \\\hspace{0.38cm} traversing  clockwise}
     \State \Return `` planar set of spanning trees for '' \Comment{{\sc Test 4} (L\ref{le:ababa})}
     \EndIf
     \If{ -donut with spokes  and  s.t.  is isomorphic to }
          \State Remove  from , insert   in , contract  in , \\\hspace{0.9cm} goto \eqref{step:restart} \Comment{{\sc Simpl. 5} (L\ref{le:isomorphic})}
     \EndIf
\If{ -donut  with spokes  and ,  ,  \\\hspace{0.38cm} ,   s.t. , and  is in }
         \If{ con-edge  for  s.t. }
            \State Remove  from , insert  in , contract  in , \\\hspace{1.35cm} goto \eqref{step:restart} \Comment{{\sc Simpl. 6} (L\ref{le:non-isomorphic-differentT1})}
         \EndIf
         \If{  s.t.  and  spoke  of }
            \State Remove  from , insert  in , contract  \\\hspace{1.35cm} in , goto \eqref{step:restart} \Comment{{\sc Simpl. 7} (L\ref{le:non-isomorphic-sameT1})}
         \EndIf
     \EndIf
     \If{ -donut with exactly two spokes  and ,    s.t. (1)   \\\hspace{0.38cm} and  s.t. , and    s.t.  with \\\hspace{0.38cm}  , or (2)   and  s.t. \\\hspace{0.38cm} , and   s.t.  with }
         \State Let  be the minimal integer satisfying (1) or (2)\Comment{{\sc Simpl. 8} (L\ref{le:non-isomorphic-differentTj})}
         \lIf{} remove  from , insert  in , \\\hspace{3.55cm} contract  in , goto \eqref{step:restart}
         \lIf{} remove  from , insert  in , \\\hspace{3.62cm} contract  in , goto \eqref{step:restart}
     \EndIf
\EndWhile
\State \Return the output of the algorithm in~\cite{df-ectefcgsf-09} on  \Comment{(L\ref{le:no-simplification-single-conflict})}
\end{algorithmic}
\caption{Testing for the existence of a planar set  of spanning trees for \label{a:alg}. The comments specify each test and simplification, and the lemma proving its correctness.}
\end{algorithm}


\section{Algorithm} \label{se:algorithm}

To ease the reading and avoid text duplication, when introducing a new lemma we always assume, without making it explicit, that all the previously defined simplifications do not apply, and that all the previously defined tests fail. Also, we do not make explicit the removal and contraction operations that we perform, as they straight-forwardly follow from the statement of each lemma. Refer also to the description in Algorithm~\ref{a:alg}.

We start with the following test.

\begin{lemma}[{\sc Test 1}]\label{le:disconnected}
Let  be a cluster such that  is disconnected. Then, there exists no planar set  of spanning trees for .
\end{lemma}

\begin{proof}
No set  is such that  induces a graph that spans the vertices belonging to . This proves the lemma.
\end{proof}


If a con-edge  is a bridge for some graph , then not choosing  to be in  would disconnect , which implies that no planar set of spanning trees for  exists.

\begin{lemma}[{\sc Simplification 1}]\label{le:bridge}
Let  be a bridge of . Then, for every planar set  of spanning trees for , we have .
\end{lemma}

\begin{proof}
Suppose, for a contradiction, that a planar set  of spanning trees for  exists such that . Then  is disconnected. By Lemma~\ref{le:disconnected}, no planar set of spanning trees for  exists, a contradiction.
\end{proof}



The following lemma is used massively in the remainder of the paper.

\begin{lemma} \label{le:one-or-the-other}
Let  be con-edges such that . Let  be a planar set of spanning trees for  and suppose that . Then, .
\end{lemma}

\begin{proof}
Assume, for a contradiction, that  contains neither  nor . Then, there exists a path  () all of whose edges belong to  connecting the end-vertices of  (resp.\ of ). Consider the cycle  composed of  and . We have that  cannot cross  . In fact,  cannot cross , as both such paths are composed of con-edges in , and it cannot cross  by Property~\ref{pr:no-two-edges-same-structure}, given that  and . However, the end-vertices of  are on different sides of , hence by the Jordan curve theorem  does cross , a contradiction.
\end{proof}



The algorithm continues with the following test.


\begin{lemma}[{\sc Test 2}]\label{le:conflicts-are-bipartite}
If the conflict graph  is not bipartite, then there exists no planar set  of spanning trees for~.
\end{lemma}

\begin{proof}
Assume, for a contradiction, that  is not bipartite and that  exists. Let  be a cycle in  with an odd number of vertices (recall that vertices in  are con-edges in ). Suppose that . Then, by repeated applications of Lemma~\ref{le:one-or-the-other} and of the fact that  does not contain two conflicting edges, we get , , , , , , , a contradiction. The case in which  can be discussed analogously.
\end{proof}



The contraction of con-edges in  that have been chosen to be in  might lead to self-loops in , a situation that is dealt with in the following.
\begin{lemma}[{\sc Simplification 2}]\label{le:self-loop}
Let  be a self-loop. Then, for every planar set  of spanning trees for , we have .
\end{lemma}

\begin{proof}
Since a tree does not contain any self-loop, the lemma follows.
\end{proof}


Next, we show a simplification that can be performed if a con-edge exists in  that does not have a conflict with any other con-edge in .
\begin{lemma}[{\sc Simplification 3}]\label{le:non-conf}
Let  be any con-edge in~ that does not have a conflict with any other con-edge in . Then, there exists a planar set  of spanning trees for  if and only if there exists a planar set  of spanning trees for  such that .
\end{lemma}

\begin{proof}
Let  be any planar set of spanning trees for . If , then there is nothing to prove. Suppose that . Since  does not cross any con-edge in , we have that  does not contain any two conflicting edges. Denote by  the cluster  is a con-edge for. Since  is a spanning tree,  contains a cycle . Since we can assume that {\sc Simplification 2} does not apply to  (it would have been performed before applying this lemma), we have that  contains at least one edge  different from . Then,  is a planar set of spanning trees for . \end{proof}

In the next three lemmata we deal with the following setting. Assume that there exist con-edges  for distinct clusters , , and , respectively, such that  and . Since {\sc Test 2} fails on ,  does not cross . Let  be any of the two facial cycles of  incident to , where a facial cycle of   is a simple cycle all of whose edges appear on the boundary of a single face of .  Assume w.l.o.g.\ that  is crossed first by  and then by  when  is traversed clockwise.  See Fig.~\ref{fig:many-conflicts}(a).

\begin{figure}[tb]
\begin{center}
\begin{tabular}{c c c}
\mbox{\includegraphics[scale=0.39]{ManyConflictSetting-M.eps}} \hspace{1mm} &
\mbox{\includegraphics[scale=0.39]{SameOrder-M.eps}} \hspace{1mm} &
\mbox{\includegraphics[scale=0.39]{Ababa-M.eps}}\\
(a) \hspace{1mm} & (b) \hspace{1mm} & (c)
\end{tabular}
\caption{The setting for (a) Lemma~\ref{le:deviation}, (b) Lemma~\ref{le:same-order}, and (c) Lemma~\ref{le:ababa}.}
\label{fig:many-conflicts}
\end{center}
\end{figure}



The next lemma presents a condition in which we can delete an edge  from .

\begin{lemma}[{\sc Simplification 4}]\label{le:deviation}
Suppose that there exists no con-edge of  different from  that has a conflict with both a con-edge for  and a con-edge for . Then, for every planar set  of spanning trees for , we have .
\end{lemma}

\begin{proof}
Denote by  and  (by  and , by  and ) the end-vertices of  (resp.\ of , resp.\ of ). By Property~\ref{pr:no-two-edges-same-structure}, it is possible to draw a closed curve  passing through , , , , , and  in (w.l.o.g.) clockwise order, containing edges , , and  in its interior, and containing every other con-edge for , , and  in its exterior. See Fig.~\ref{fig:lemma-deviation}.

Suppose, for a contradiction, that there exists a planar set  of spanning trees for  such that .
\begin{figure}[tb]
\begin{center}
\mbox{\includegraphics[scale=0.4]{LemmaDeviation-M.eps}}
\caption{Illustration for the proof of Lemma~\ref{le:deviation}.}
\label{fig:lemma-deviation}
\end{center}
\end{figure}
Then, there exists a path  () all of whose edges belong to  connecting  and  (resp.\  and ). Since  and  are on different sides of , by the Jordan curve theorem  crosses a con-edge  of . Since no con-edge of  different from  has a conflict with both a con-edge for  and a con-edge for , it follows that  does not cross . Consider the cycle  composed of , of , of the path  in  between  and  not containing , and of the path  in  between  and  not containing . There exist vertices of  on both sides of cycle  (e.g., the end-vertices of ). However, no con-edge  for  in  can cross . In fact,  cannot cross  and , as such paths are composed of con-edges in , and it cannot cross  and  by construction of . It follows that  does not connect , a contradiction to the fact that  is a planar set of spanning trees for .
\end{proof}



The next two lemmata state conditions in which no planar set of spanning trees for  exists. Their statements are illustrated in Figs.~\ref{fig:many-conflicts}(b) and~\ref{fig:many-conflicts}(c), respectively.

\begin{lemma}[{\sc Test 3}]\label{le:same-order}
Suppose that there exist con-edges  for clusters , , and , respectively, such that  ,  belongs to , and  as well as . Assume that  is crossed first by  and then by  when  is traversed clockwise. Then, no planar set of spanning trees for  exists.
\end{lemma}

\begin{proof}
Denote by  and  (by  and , by  and ) the end-vertices of  (resp.\ of , resp.\ of ). By Property~\ref{pr:no-two-edges-same-structure}, it is possible to draw a closed curve  passing through , , , , , and  in this clockwise order, containing edges , , and  in its interior, and containing every other con-edge for , , and  in its exterior.

Denote by  and  (by  and , by  and ) the end-vertices of  (resp.\ of , resp.\ of ). By Property~\ref{pr:no-two-edges-same-structure}, it is possible to draw a closed curve  passing through , , , , , and  in this clockwise order, containing edges , , and  in its interior, and containing every other con-edge for , , and  in its exterior. Assume, w.l.o.g., that the face of  delimited by  is to the right of  when traversing such a cycle in clockwise direction. Finally assume, w.l.o.g., that , , , and  appear in this clockwise order along  (possibly  and/or ).

Suppose, for a contradiction, that there exists a planar set  of spanning trees for . The proof distinguishes two cases.

\paragraph{Case 1.} Let . Refer to Fig.~\ref{fig:same-order}(a).

Consider the path  () all of whose edges belong to  connecting  and  (resp.\  and ). Since the end-vertices of  and  alternate along  and since  delimits a face of , it follows that  and  share vertices (and possibly edges). Thus, the union of  and  is a tree  (recall that  contains no cycle) whose only possible leaves are , , , and . Next, consider the path  all of whose edges belong to  connecting  and . We claim that this path contains . Indeed, if  does not contain , then it crosses the path connecting  and  in , thus contradicting the fact that  is a planar set of spanning trees. An analogous proof shows that  contains~.

Now, consider the cycle  composed of , of , of the path  in  between  and  not containing , and of the path  in  between  and  not containing . Cycle  contains vertices of  on both sides (e.g.,  and ). However, no con-edge  for  in  can cross . In fact,  cannot cross  or , as such paths are composed of con-edges in , and it cannot cross  and  by construction of  and . It follows that  does not connect , a contradiction to the fact that  is a planar set of spanning trees.

\begin{figure}[tb]
\begin{center}
\begin{tabular}{c c}
\mbox{\includegraphics[scale=0.38]{SameOrder-1-M.eps}} \hspace{1mm} &
\mbox{\includegraphics[scale=0.38]{SameOrder-2-M.eps}}\\
(a) \hspace{1mm} & (b)
\end{tabular}
\caption{Proof of Lemma~\ref{le:same-order}. (a) The case in which neither  nor  belongs to . (b) The case in which .}
\label{fig:same-order}
\end{center}
\end{figure}

\paragraph{Case 2.} Let . Suppose, w.l.o.g., that . Refer to Fig.~\ref{fig:same-order}(b).

Consider the path  () all of whose edges belong to  connecting  and  (resp.\  and ). Consider the cycle  composed of  and . We have that no con-edge  for  in  can cross . In fact,  cannot cross , as such a path is composed of con-edges in , and it cannot cross  by Property~\ref{pr:no-two-edges-same-structure}, given that  and  belong to the same connected component of  and do not cross, as otherwise {\sc Test 2} would succeed on . It follows that  has  and  on the same side, as otherwise  would not connect , a contradiction to the fact that  is a planar set of spanning trees. Since  and  are on the same side of , it follows that  is on one side of  (call it {\em the small side} of ), while , , and  are on the other side (call it {\em the large side} of ). Analogously, the cycle  composed of  and  has  on one side (call it {\em the small side} of ), and , , and  on the other side (call it {\em the large side} of ). Observe that the small side of  and the small side of  are disjoint, as otherwise  intersects  or  intersects .

Now consider the cycle  composed of , of , of the path  in  between  and  not containing , and of the path  in  between  and  not containing . Cycle  contains vertices of  on both sides. Namely, it contains  and  on one side (the side of  containing the small side of  and the small side of ), and  and  on the other side. However, no con-edge  for  in  crosses . In fact,  cannot cross  and , as such paths are composed of con-edges in , and it cannot cross  and  by construction of . It follows that  does not connect , a contradiction to the fact that  is a planar set of spanning trees.
\end{proof}



\begin{lemma}[{\sc Test 4}]\label{le:ababa}
Suppose that con-edges  for  exist in , and such that , , and  occur in this order along , when clockwise traversing . Suppose also that there exist con-edges  for  and  for  such that , , and . Then, no planar set of spanning trees for  exists.
\end{lemma}

\begin{proof}
Denote by  and  (by  and , by  and ) the end-vertices of  (resp.\ of , resp.\ of ). By Property~\ref{pr:no-two-edges-same-structure}, it is possible to draw a closed curve  passing through , , , , , and  in this clockwise order, containing edges , , and  in its interior, and containing every other con-edge for , , and  in its exterior.

Denote by  and  (by  and , by  and ) the end-vertices of  (resp.\ of , resp.\ of ), and by  and  (by  and ) the end-vertices of  (resp.\ of ). By Property~\ref{pr:no-two-edges-same-structure}, it is possible to draw a closed curve  passing through , , , , , and  in this clockwise order, containing edges , , and  in its interior, and containing every other con-edge for , , and  in its exterior. Also, it is possible to draw a closed curve  passing through , , , and  in this clockwise order, containing edges  and  in its interior, and containing every other con-edge for  and  in its exterior. We assume that  is arbitrarily close to the drawing of  and  so that  intersects a con-edge for a cluster different from  and  only if that con-edge intersects  or . Assume, w.l.o.g., that the face of  delimited by  is to the right of  when traversing such a cycle in clockwise direction. Finally assume, w.l.o.g., that , , , , , and  appear in this clockwise order along  (possibly , , and/or ).

First, suppose that  has a conflict with a con-edge  for . Then, if  is crossed first by  and then by  when  is traversed clockwise, we can conclude that no planar set of spanning trees for  exists by Lemma~\ref{le:same-order} (with  and  playing the role of the edges  and  in the statement of Lemma~\ref{le:same-order}). Analogously, if  is crossed first by  and then by  when  is traversed clockwise, we can conclude that no planar set of spanning trees for  exists by Lemma~\ref{le:same-order} (with  and  playing the role of the edges  and  in the statement of Lemma~\ref{le:same-order}). Thus, in what follows we assume that  does not have a conflict with any con-edge for .

Suppose, for a contradiction, that there exists a planar set  of spanning trees for . The proof distinguishes three cases.

\paragraph{Case 1.} Let . Refer to Fig.~\ref{fig:ababa}(a).

Consider the path  (, ) all of whose edges belong to  connecting  and  (resp.\  and , resp.\  and ). Since the end-vertices of  and  alternate along , for every  with , and since  delimits a face of , it follows that  and  share vertices (and possibly edges). Thus, the union of , , and  is a tree  whose leaves can only be from . Consider the path  all of whose edges belong to  connecting  and . We claim that this path contains . Indeed, if  does not contain , then it crosses the path connecting  and  in , thus contradicting the fact that  is a planar set of spanning trees for . Analogously,  contains . Further,  does not contain , as otherwise it would cross the path connecting  and  in . Next, consider the path  that is the path in  between  and  and not containing . Also, consider the path  that is the path in  between  and  and not containing . Not both  and  intersect a con-edge for , as otherwise by construction of  con-edge  would have a conflict with a con-edge for , which contradicts the assumptions. Assume that  does not cross any con-edge for , the other case being analogous.

Consider the cycle  composed of , of , of , and of the path  in  between  and  not containing . Cycle  contains vertices of  on both sides (e.g.,  and ). However, no con-edge  for  in  crosses . In fact,  cannot cross  or , as such paths are composed of con-edges in , it cannot cross  by construction of , and it cannot cross  by assumption. It follows that  does not connect , a contradiction to the fact that  is a planar set of spanning trees for .

\begin{figure}[tb]
\begin{center}
\begin{tabular}{c c}
\mbox{\includegraphics[scale=0.38]{Ababa-1-M.eps}} \hspace{1mm} &
\mbox{\includegraphics[scale=0.38]{Ababa-1-b-M.eps}}\\
(a) \hspace{1mm} & (b)
\end{tabular}
\caption{Proof of Lemma~\ref{le:ababa}. (a) The case in which neither , nor , nor  belongs to . (b) The case in which  and  do not belong to , while  belongs to .}
\label{fig:ababa}
\end{center}
\end{figure}

\paragraph{Case 2.} Let , and . Refer to Fig.~\ref{fig:ababa}(b).

Consider the path  () all of whose edges belong to  connecting  and  (resp.\  and ). Since the end-vertices of  and  alternate along  and since  delimits a face of , it follows that  and  share vertices (and possibly edges). Thus, the union of  and  is a tree  whose leaves can only be from . Also, let  be a path whose edges belong to  connecting  and any vertex in , say . Assume that  is minimal, i.e., no vertex of  different from  belongs to .
Next, consider the path  in  between  and  and not containing . Also, consider the path  in  between  and  and not containing . Not both  and  intersect a con-edge for , as otherwise by construction of  con-edge  would have a conflict with a con-edge for , which contradicts the assumptions. Assume that  does not cross any con-edge for , the other case being analogous.
Now consider the path  all of whose edges belong to  connecting  and . We claim that this path contains . Indeed, if  does not contain , then it crosses the path connecting  and  in , thus contradicting the fact that  is a planar set of spanning trees for .

Consider the cycle  composed of , of the path  in  between  and  not containing , of the path connecting  and  in , of , and of . Cycle  contains vertices of  on both sides (e.g.,  and ). However, no con-edge  for  in  can cross . In fact,  cannot cross , the path connecting  and  in , or , as such paths are composed of con-edges in , it cannot cross  by construction of , and it cannot cross  by assumption. It follows that  does not connect , a contradiction to the fact that  is a planar set of spanning trees for .

\begin{figure}[tb]
\begin{center}
\mbox{\includegraphics[scale=0.38]{Ababa-2-M.eps}}
\caption{Proof of Lemma~\ref{le:ababa}, in the case in which  belongs to .}
\label{fig:ababa-2}
\end{center}
\end{figure}

\paragraph{Case 3.} Let . Suppose, w.l.o.g., that . Refer to~Fig.~\ref{fig:ababa-2}. This case is similar to the second case in the proof of Lemma~\ref{le:same-order}.

Consider the path  () all of whose edges belong to  connecting  and  (resp.\  and ).
Consider the cycle  composed of  and . We have that no con-edge  for  in  crosses . In fact,  cannot cross , as such a path is composed of con-edges in , and it cannot cross  by Property~\ref{pr:no-two-edges-same-structure}, given that  and  belong to the same connected component of  and do not cross, as otherwise {\sc Test 2} would succeed on . It follows that  has  and  on the same side, as otherwise  would not connect , a contradiction to the fact that  is a planar set of spanning trees for .  Since  and  are on the same side of , it follows that  is on one side of  (call it {\em the small side} of ), while  and  are on the other side (call it {\em the large side} of ). Observe that, differently from the proof of Lemma~\ref{le:same-order}, it might be the case that  is in the small side of , if  contains con-edge .
An analogous argument proves that the cycle  composed of  and  has  on one side (call it {\em the small side} of ), and , , and  on the other side (call it {\em the large side} of ).
Observe that the small side of  and the small side of  are disjoint, as otherwise  intersects  or  intersects .

Now consider the cycle  composed of , of , of the path  in  between  and  not containing , and of the path  in  between  and  not containing . Cycle  contains vertices of  on both sides. Namely, it contains  and  on one side (the side of  containing the small side of  and the small side of ), and  on the other side. However, no con-edge  for  in  can cross . In fact,  cannot cross  and , as such paths are composed of con-edges in , and it cannot cross  and  by construction of . It follows that  does not connect , a contradiction to the fact that  is a planar set of spanning trees for .
\end{proof}


If {\sc Simplifications 1--4} do not apply to  and {\sc Tests 1--4} fail on , then the con-edges for a cluster  that are crossed by con-edges for (at least) two other clusters have a nice structure, that we call {\em -donut} (see Fig.~\ref{fig:donut}). 

Consider a con-edge  for  crossing con-edges  for clusters , with . An -donut for  consists of a sequence  of con-edges for  with , called {\em spokes} of the -donut, of a sequence  of facial cycles in , and of sequences  of con-edges for , for each , such that the following hold for every : 

\begin{itemize}
\item[(a)]  is one of edges ;
\item[(b)] , for every ;
\item[(c)]  and  share edge ;
\item[(d)] edge  is crossed by  in this order when  is traversed clockwise;
\item[(e)] all the con-edges of  encountered when clockwise traversing  from  to  do not cross any con-edge for ; and
\item[(f)] all the con-edges of  encountered when clockwise traversing  from  to  do not cross any con-edge for . 
\end{itemize}

We have the following.

\begin{lemma} \label{le:donut}
For every con-edge  for , there exists an -donut for .
\end{lemma}

\begin{figure}[tb]
\begin{center}
\mbox{\includegraphics[scale=0.4]{Donut.eps}}
\caption{The -donut for . Only the con-edges of , the con-edges for  crossing the spokes of the -donut for , the con-edges for  and  inside the faces delimited by , part of the con-edges for  incident to vertices in , and part of the con-edges of  and  crossing the con-edges of  are shown.}
\label{fig:donut}
\end{center}
\end{figure}

\begin{proof}
Let, w.l.o.g.,  and consider the two faces  and  of  incident to . Since {\sc Simplification 1} does not apply to , it follows that . Let  be the cycle delimiting , for . Let  be the con-edges for clusters , respectively, ordered as they cross  when clockwise traversing . Thus,  is crossed by  in this order when  is traversed clockwise.

Consider facial cycle .

Since {\sc Simplification 4} does not apply to , there exists at least one con-edge  in  that is different from  and that is crossed by con-edges  for  and  for . Since {\sc Test 3} fails on , it follows that  is crossed first by  and then by  when clockwise traversing . Since {\sc Test 4} fails on , it follows that all the con-edges of  different from  and  encountered when clockwise traversing  from  to  (from  to ) do not have a conflict with any con-edge for  (resp.\ for ).

\begin{figure}[tb]
\begin{center}
\begin{tabular}{c c}
\mbox{\includegraphics[scale=0.38]{Donut-Proof-1.eps}} \hspace{1cm} &
\mbox{\includegraphics[scale=0.38]{Donut-Proof-2.eps}}\\
(a) \hspace{1cm} & (b)
\end{tabular}
\caption{Illustration for the proof of Lemma~\ref{le:donut}.}
\label{fig:donut-proof}
\end{center}
\end{figure}


Now, since {\sc Simplification 4} does not apply to , there exists at least one con-edge  in  that is different from  and that is crossed by con-edges  for  and  for . We prove that . Suppose, for a contradiction, that . If  is encountered when clockwise traversing  from  to , as in Fig.~\ref{fig:donut-proof}(a), then {\sc Test 4} would succeed on , with , , and  playing the roles of , , and , respectively, in the statement of Lemma~\ref{le:ababa}, a contradiction. Hence, assume that  is encountered when clockwise traversing  from  to , as in Fig.~\ref{fig:donut-proof}(b). Since {\sc Test 3} fails on , it follows that  is crossed first by  and then by  when clockwise traversing . However, this implies that {\sc Test 4} succeeds on , with , , and  playing the roles of , , and , respectively, in the statement of Lemma~\ref{le:ababa}, a contradiction. Thus, we get that , hence  is crossed by a con-edge  for . Since {\sc Test 3} fails on , it follows that  is crossed first by , then by , and then by  when clockwise traversing . Since {\sc Test 4} fails on , it follows that all the con-edges of  different from  and  encountered when clockwise traversing  from  to  (from  to ) do not have a conflict with any con-edges for  (resp.\ for ).

The argument in the previous paragraph can be repeated for each , with , with , , , and  playing the roles of  , , , and . This leads to conclude that  is crossed by con-edges  for , respectively, in this order when clockwise traversing , and that all the con-edges of  encountered when clockwise traversing  from  to  (from  to ) do not have a conflict with any con-edges for  (resp.\ for ).

Now the same argument as the one we just presented for  is repeated for , that is the facial cycle that contains  and that is different from . Again, this leads to conclude that there exists a con-edge  for  that belongs to , that there exist con-edges  for clusters , respectively, that cross  in this order when clockwise traversing , and that all the con-edges of  encountered when clockwise traversing  from  to  (from  to ) do not have a conflict with any con-edges for  (resp.\ for ).

Since the number of edges of  is finite and since each facial cycle of  does not contain more than two con-edges crossed by con-edges for all of  (as otherwise {\sc Test 3} would succeed on ), eventually a facial cycle  of  is considered in which the two con-edges that are crossed by con-edges for all of  are  and . This concludes the proof of the lemma.
\end{proof}

Observe that the -donut for any con-edge  for  can be computed efficiently. The following is a consequence of Lemma~\ref{le:donut}.

\begin{lemma} \label{le:exactly-one}
Consider a con-edge  for  that has a conflict with  con-edges for other clusters. Let  be the spokes of the -donut for . Then, if a planar set  of spanning trees for  exists, it contains exactly one of .
\end{lemma}

\begin{proof}
First, by Lemma~\ref{le:donut},  belongs to both facial cycles  and , for every , where . It follows that removing all of  from  disconnects . Hence,  contains at least one of .

Suppose, for a contradiction, that a planar set  of spanning trees for  exists that contains at least two edges  and . Refer to Fig.~\ref{fig:exactly-one}. Denote by  and  (by  and ) the end-vertices of  (resp.\ of ), where we assume w.l.o.g. that edges  are crossed in this order when traversing  from  to , and that edges  are crossed in this order when traversing  from  to . Further, denote by  and , by  and , by  and , by  and ,  the end-vertices of , of , of , and of , respectively.


\begin{figure}[tb]
\begin{center}
\mbox{\includegraphics[scale=0.45]{Exactly-One-M.eps}}
\caption{Illustration for the proof of Lemma~\ref{le:exactly-one}.}
\label{fig:exactly-one}
\end{center}
\end{figure}

Consider the path  () all of whose edges belong to  connecting  and  (resp.\  and ). Consider the cycle  composed of  and . We have that no con-edge  for  in  crosses . In fact,  cannot cross , as such a path is composed of con-edges in , and it cannot cross  by Property~\ref{pr:no-two-edges-same-structure}, given that  and  belong to the same connected component of  and do not cross, as otherwise {\sc Test 2} would succeed on . It follows that  has  and  on the same side, as otherwise  would not connect , a contradiction to the fact that  is a planar set of spanning trees for .  Since  and  are on the same side of , since , , and  are on the same side of , and since  and  are on the same side of , it follows that  is on one side of  ({\em the small side} of ), while , , and  are on the other side ({\em the large side} of ). Analogously, the cycle  composed of  and  has  on one side ({\em the small side} of ), and , , and  on the other side ({\em the large side} of ). The small side of  and the small side of  are disjoint, as otherwise  intersects , or  intersects .

By Property~\ref{pr:no-two-edges-same-structure}, it is possible to draw a closed curve  passing through , , , , , and  in this circular order, containing edges , , and  in its interior, and containing every other con-edge for , , and  in its exterior. Now consider the cycle  composed of , of , of the path  in  between  and  not containing , and of the path  in  between  and  not containing . Cycle  contains vertices of  on both sides. Namely, it contains  and  on one side (the side of  containing the small side of  and the small side of ), and  and  on the other side. However, no con-edge  for  in  crosses . In fact,  cannot cross  and , as such paths are composed of con-edges in , and it cannot cross  and  by construction of . It follows that  does not connect , a contradiction to the fact that  is a planar set of spanning trees for .
\end{proof}


Consider a con-edge  for a cluster . The {\em conflicting structure}  of  is a sequence of sets  of con-edges which correspond to the layers of a BFS traversal starting at  of the connected component of  containing . That is: ; then, for ,  is the set of con-edges that cross con-edges in  and that are not in , and  is the set of con-edges that cross con-edges in  and that are not in .

We now study the conflicting structures of the spokes  of the -donut for a con-edge  for . No two edges in a set  or in a set  have a conflict, as otherwise {\sc Test 2} would succeed. Also, by Lemma~\ref{le:one-or-the-other}, any planar set  of spanning trees for  contains either all the edges in  or all the edges in .

Assume that  has a conflict with at least two con-edges for other clusters. For any , we say that  and  have {\em isomorphic conflicting structures} if  and  belong to isomorphic connected components of  and if the vertices of these components that are in correspondence under the isomorphism represent con-edges for the same cluster. Formally,  and  have isomorphic conflicting structures if there exists a bijective mapping  between the edges in  and the edges in  such that:
\begin{enumerate}
\item  is a con-edge for a cluster  if and only if  is a con-edge for , for every ;
\item  if and only if , for every ;
\item  if and only if , for every ; and
\item  if and only if , for every .
\end{enumerate}

Observe that the isomorphism of two conflicting structures can be tested efficiently.


We will prove in the following four lemmata that by examining the conflicting structures for the spokes of the -donut for , a decision on whether some spoke is or is not in  can be taken without loss of generality. We start with the following:

\begin{lemma}[{\sc Simplification 5}]\label{le:isomorphic}
Suppose that spokes  and  have isomorphic conflicting structures. Then, there exists a planar set  of spanning trees for  if and only if there exists a planar set  of spanning trees for  such that .
\end{lemma}

\begin{proof}
If there exists no planar set of spanning trees for , there is nothing to prove. Otherwise, consider any planar set  of spanning trees for . If , there is nothing to prove. Otherwise, suppose that . Since  does not contain any two con-edges that have a conflict and by Lemma~\ref{le:one-or-the-other}, we have  and .

By Lemma~\ref{le:exactly-one}, exactly one of  belongs to any planar set  of spanning trees for . Hence, . Since  does not contain any two con-edges that have a conflict and by Lemma~\ref{le:one-or-the-other}, we have  and .

Consider the set  of con-edges obtained from  by removing  and  and by adding  and . We claim that  is a planar set of spanning trees for . The claim directly implies the lemma.

First, we prove that no two con-edges in  have a conflict. Since  is a planar set of spanning trees for , no two con-edges in  have a conflict. Consider any con-edge , for some , and consider any con-edge . If  does not belong to , then  and  do not cross, since  and  belong to the same connected component of . Further, if  belongs to  and does not belong to  or to , then  and  do not cross, by definition of conflicting structure. Finally,  does not belong to  or to , given that . It can be analogously proved that no edge , for some , crosses any con-edge .

Second, we prove that, for each cluster , the graph induced by the con-edges in  is a tree that spans the vertices in . This is trivially proved for every cluster  that has no con-edge in , given that in this case . Moreover, by Property~\ref{pr:no-two-edges-same-structure} and since  and  are isomorphic, each cluster  having a con-edge in  has {\em exactly} one con-edge  in  and one con-edge  in . By the construction of  and by the fact that if  is in  (in ), then  is in  (resp.\ in ), it follows that  is obtained from  by removing  and by adding , for some distinct . Since  induces a spanning tree of the vertices in , in order to prove that  induces a spanning tree of the vertices in  it suffices to prove that the end-vertices of  belong to distinct connected components of . In the following we prove this statement.

For , denote by  and  the end-vertices of ; assume w.l.o.g. that edges  are crossed in this order when traversing  from  to . Denote by  and  the end-vertices of , for every  and .

\begin{itemize}
\item We start with cluster . Denote by  and  the two connected components of  obtained by removing  from . Since  are a separating set for , since  is the only edge among  that belongs to  (by assumption and by Lemma~\ref{le:exactly-one}), it follows that  and  are one in  and the other one in .


\item We now deal with cluster , for any . Denote by  and  the two connected components of  obtained by removing  from . In the following we prove that  and  are one in  and the other one in .

Suppose, for a contradiction, that both  and  are in . Then, there exists a path  between  and  all of whose edges belong to . See Fig.~\ref{fig:distinct-components-2}.

\begin{figure}[tb]
\begin{center}
\mbox{\includegraphics[scale=0.38]{distinct-components-2.eps}}
\caption{Illustration for the proof that the end-vertices of  belong to distinct connected components of .}
\label{fig:distinct-components-2}
\end{center}
\end{figure}

Since  are a separating set for , since , and since , it follows that the path  connecting  and  in  contains edge . Since edges  and  have a conflict, it follows that vertices  and  are on different sides of cycle , hence path  crosses cycle  by the Jordan curve theorem. However,  cannot cross , as all the edges of such paths belong to , by assumption; moreover,  cannot cross , as by Property~\ref{pr:no-two-edges-same-structure} this would imply that  contains , contradicting the fact that such an edge does not belong to .



\item We now deal with any cluster  such that there exists a con-edge  for  in , for some . Observe that . Denote by  any con-edge in  such that  and by  any con-edge in  such that ; these edges exist by definition of conflicting structure and since . Denote by  and  the two connected components of  obtained by removing  from . In the following we prove that the end-vertices of  are one in  and the other one in .

    Denote by  and  the clusters  and  are con-edges for. Since  and  have isomorphic conflicting structures, we have that , , and  are con-edges for , , and , respectively. Also, by assumption, , , and  belong to , while , , and  do not.

    Suppose, for a contradiction, that both the end-vertices of  are in . The end-vertices of  are one in  and the other one in , given that  is a tree and that  and  are obtained from  by removing edge . Refer to Fig.~\ref{fig:distinct-components-3}.

    First, consider the path  connecting the end-vertices of  and all of whose edges belong to . By assumption,  (and hence ) does not contain  nor . Then, consider the cycle  composed of  and of . We have that con-edges  and  for  do not cross  by Property~\ref{pr:no-two-edges-same-structure}; in fact,  () belongs to the same connected component of  as  (resp.\  ) and it does not cross  (resp.\ ), as otherwise {\sc Test 2} would succeed on , hence it does not cross any con-edge for . Also, we have that no con-edge  for  in  crosses . In fact,  cannot cross , as such a path is composed of con-edges in , and it cannot cross  by Property~\ref{pr:no-two-edges-same-structure}, since  and con-edge  for  belong to the same connected component of  and do not cross. These observations immediately lead to a contradiction in the case in which  and  are on different sides of , as in such a case no path whose edges belong to  can connect an end-vertex of  with an end-vertex of  without crossing . Hence, assume that  and  are on the same side of  (call it {\em the large side} of ). Call {\em the small side} of  the side of  that does not contain  and .


\begin{figure}[tb] 
\begin{center}
\begin{tabular}{c c}
\mbox{\includegraphics[scale=0.38]{distinct-components-3.eps}} \hspace{3mm} &
\mbox{\includegraphics[scale=0.38]{distinct-components-3-bis.eps}}\\
(a) \hspace{3mm} & (b)
\end{tabular}
\caption{Illustration for the proof that the end-vertices of  are one in  and the other one in .}
\label{fig:distinct-components-3}
\end{center}
\end{figure}

Next, consider the path  connecting the end-vertices of  and all of whose edges belong to . Observe that  does not coincide with , given that . (Observe that  can possibly contain edge , asymmetrically to   that does not contain .)  Then, consider the cycle  composed of  and of . Analogously as for , it can be concluded that  and  are on the same side of  (call it {\em the large side} of ). Call {\em the small side} of  the side of  that does not contain  and . Hence,  is in the large side of  and  is in the large side of , thus the small sides of   and  have disjoint interiors.

Now, consider edge . Since it crosses , then one of its end-vertices is in the small side of ; also, since it crosses , the other end-vertex is in the small side of . Hence, in order to obtain a contradiction, it suffices to prove that there exists a vertex  of  that is neither in the small side of   nor in the small side of  (that is,  is simultaneously in the large side of   and in the large side of  ). In fact, if that is the case, then no path whose edges belong to  can connect  with the end-vertices of , given that no con-edge for  in  can cross an edge of , hence  does not connect , a contradiction.

We claim that at least one of the end-vertices of  is simultaneously in the large side of   and in the large side of  . First, observe that both the end-vertices of  are in the large side of . In fact,  is in the large side of , by assumption, and hence all of , , and  are in the large side of , given that  does not contain . Analogously, if  does not contain  (as in Fig.~\ref{fig:distinct-components-3}(a)), then all of , , and  are in the large side of ; on the other hand, if  contains  (as in Fig.~\ref{fig:distinct-components-3}(b)), then  crosses , hence one of its end-vertices is in the small side of  and the other end-vertex is in the large side of . This proves the claim and hence the statement.


\item It remains to deal with any cluster  such that there exists a con-edge  for  in , for some . Observe that , while . Denote by  any con-edge in  such that  and by  any con-edge in  such that . All these edges exist by definition of conflicting structure and since . Denote by  and  the two connected components of  obtained by removing  from . The following statement can be proved: The end-vertices of  are one in  and the other one in . The proof is the same as the one for the case in which there exists a con-edge  for  in , for some , with , , , , , and  playing the roles of , , , , , and , respectively.
\end{itemize}

This concludes the proof of the lemma. Together with Lemma~\ref{le:one-or-the-other}, it establishes the correctness of {\sc Simplification 5}.
\end{proof}


Next, we study non-isomorphic spokes. Let  be a spoke of the -donut for . Assume that  contains a con-edge  for a cluster , and that  contains a con-edge  for a cluster , where  and . By Property~\ref{pr:no-two-edges-same-structure}, since  and  belong to the same connected component of  and do not cross (as otherwise {\sc Test 2} would succeed), it follows that  does not cross any con-edge for , hence it lies in one of the two faces  and  of  that  shares with spokes  and , respectively. Assume w.l.o.g. that  lies in . By Lemma~\ref{le:donut},  contains a con-edge  for , where .

The next two lemmata discuss the case in which  contains a con-edge for  that has a conflict with  and the case in which it does not. We start with the latter.

\begin{lemma}[{\sc Simplification 6}] \label{le:non-isomorphic-differentT1}
Suppose that no con-edge  for  exists such that , and that a planar set  of spanning trees for  exists. Then, .
\end{lemma}

\begin{proof}
Suppose, for a contradiction, that a planar set  of spanning trees for  exists with  . Since no two conflicting edges both belong to  and by Lemma~\ref{le:one-or-the-other}, we have  and . Refer to Fig.~\ref{fig:non-isomorphic-one}.

\begin{figure}[tb]
\begin{center}
\mbox{\includegraphics[scale=0.38]{Non-Isomorphic-One.eps}}
\caption{Illustration for the proof of Lemma~\ref{le:non-isomorphic-differentT1}.}
\label{fig:non-isomorphic-one}
\end{center}
\end{figure}

Consider the path  connecting the end-vertices of  and all of whose edges belong to . By assumption,  does not coincide with . Then, consider the cycle  composed of  and of . We have that  does not cross any con-edge for  (including those not in ). Indeed, suppose that  crosses a con-edge  for . Then, it contains vertices of  on both sides (e.g., the end-vertices of ). However, no con-edge  for  in  crosses . In fact,  cannot cross , as such a path is composed of con-edges in , and it cannot cross , as  and  belong to the same connected component of  and do not cross; it follows that  does not connect , a contradiction. We can hence assume that all the edges of  lie inside a single face of . Since  lies inside , we have that all the edges of  lie inside . We emphasize that  does not cross , given that the latter crosses no con-edge for , by assumption.

Next, consider the path  connecting the end-vertices of  and all of whose edges belong to . By assumption,  does not coincide with . By Lemma~\ref{le:exactly-one} and since  form a separating set for , we have that  contains exactly one of , say  with . Denote by  the con-edge for  that has a conflict with . This edge exists by Lemma~\ref{le:donut}. Also, denote by  the cycle .

Cycles  and  do not cross, since  lies inside . Then, call the {\em large side} of  the side that contains all the edges of  (call the other side of  its {\em small side}); also call the {\em large side} of  the side that contains all the edges of  (call the other side of  its {\em small side}). Thus, thus the small sides of  and  have disjoint interiors.

Now, consider edge . Since it crosses , one of its end-vertices is in the small side of ; also, since it crosses , the other end-vertex is in the small side of . Hence, in order to obtain a contradiction, it suffices to prove that there exists a vertex  of  that is simultaneously in the large side of  and in the large side of . In fact, if that is the case, then no path whose edges belong to  can connect  with the end-vertices of , given that no con-edge for  in  can cross an edge of ; thus,  does not connect , a contradiction.

We claim that one of the end-vertices of  is simultaneously in the large side of  and in the large side of . First, we prove that both the end-vertices of  are in the large side of . Namely, since  does not cross any con-edge for , the end-vertices of  are both in the large side of . Hence, if one of the end-vertices of  is not in the large side of , it follows that  crosses . Since all the edges of  lie in , we have that  crosses  only if . However, this contradicts the assumption that  does not cross any con-edge for . Second, since  crosses , one of its end-vertices is in the small side of , while the other one, say , is in the large side of . Hence,  is in the large side of both  and . This proves the claim and hence the lemma. Together with Lemma~\ref{le:one-or-the-other}, this lemma establishes the correctness of {\sc Simplification~6}.
\end{proof}



\begin{lemma}[{\sc Simplification 7}]\label{le:non-isomorphic-sameT1}
Suppose that a con-edge  for  exists with . If a planar set  of spanning trees for  exists, then either  or .
\end{lemma}

\begin{proof}
Suppose that a planar set  of spanning trees for  exists. By Lemma~\ref{le:exactly-one}, {\em exactly} one out of  belongs to . Hence, {\em at most} one out of  and  belongs to . It remains to prove that {\em at least} one out of  and  belongs to . By Lemma~\ref{le:one-or-the-other}, this is equivalent to prove that {\em at most} one out of  and  belongs to .

We prove that  is a spoke of the -donut for . By Lemma~\ref{le:exactly-one}, the statement implies that at most one out of  and  belongs to , and hence implies the lemma.

\begin{figure}[tb]
\begin{center}
\mbox{\includegraphics[scale=0.38]{Non-Isomorphic-Two.eps}}
\caption{Illustration for the proof of Lemma~\ref{le:non-isomorphic-sameT1}.}
\label{fig:non-isomorphic-two}
\end{center}
\end{figure}

Suppose, for a contradiction, that  is not a spoke of the -donut for . Refer to Fig.~\ref{fig:non-isomorphic-two}. First, the -donut for  exists by Lemma~\ref{le:donut}, given that  crosses at least two con-edges  and  for clusters  and , respectively, and given that {\sc Simplifications 1--4} do not apply to  and {\sc Tests 1--4} fail on . Second, denote by  the clusters whose con-edges cross , ordered as they cross  when clockwise traversing one of the two faces incident to . Observe that  and  are among . Third, we define two subgraphs  and  of , as the subgraphs of  whose edges delimit the -donut for . That is, consider the faces of  incident to spokes of the -donut for ; the union of the boundaries of such faces defines a connected subgraph of , from which we remove the spokes of the -donut for , thus obtaining a subgraph  of  composed of two connected components, that we denote by  and . By Lemma~\ref{le:donut}, the edges of  are not crossed by any con-edge for , and the edges of  are not crossed by any con-edge for  (up to renaming  with ). Denote by  the connected region defined by  that used to contain the spokes of the -donut for .

If  is not a spoke of the -donut for , then either  or  separates  from , given that the only edges of  in  are the spokes of the -donut for . Suppose w.l.o.g. that  separates  from . Observe that at least one of  and  is in , say that  is in . Then, either  is disconnected, or there exists a path that is composed of con-edges for , that connects an end-vertex of  with an end-vertex of , and that contains an edge crossing an edge of . In both cases we have a contradiction, which proves the statement and hence the lemma. Together with Lemma~\ref{le:one-or-the-other}, this lemma establishes the correctness of {\sc Simplification~7}.
\end{proof}

Observe that Simplification 7 can be applied in the case in which the -donut for  has at least three spokes. Namely, in that case, by Lemmata~\ref{le:exactly-one} and~\ref{le:non-isomorphic-sameT1} all the spokes different from  and  can be removed from .



Next, assume that there exists an -donut with exactly two spokes  and . Consider the smallest  such that one of the following holds:

\begin{enumerate}
\item there exist con-edges  and  for clusters  and , resp., such that , and there exists no con-edge  for  such that  with  con-edge for  in , for some  with ; or
\item there exist con-edges  and  for clusters  and , resp., such that , and there exists no con-edge  for  such that  with  con-edge for  in , for some  with .
\end{enumerate}

We have the following.

\begin{lemma}[{\sc Simplification 8}] \label{le:non-isomorphic-differentTj}
Assume that a planar set  of spanning trees for  exists. Then, .
\end{lemma}

\begin{proof}
We prove the lemma in the case in which  is determined by (1), i.e. there exist con-edges  and  for clusters  and , respectively, such that , and there exists no con-edge  for  such that  with  con-edge for , for some  with . The proof for the case in which the value of  is determined by (2) is analogous. Refer to Fig.~\ref{fig:two-spokes}.

Let , , and . Also, for , denote by  and  con-edges for clusters  and , respectively, where  for  and  for . Observe that . Further, for , denote by  and  con-edges for clusters  and , respectively, such that  for  and  for . All these edges exist by definition of conflicting structure and by the minimality of . Observe that .  By assumption, no con-edge  for  exists such that .

\begin{figure}[tb]
\begin{center}
\mbox{\includegraphics[scale=0.38]{Two-spokes-M.eps}}
\caption{Illustration for the proof of Lemma~\ref{le:non-isomorphic-differentTj}, with .}
\label{fig:two-spokes}
\end{center}
\end{figure}

First, we argue that no con-edge  for  exists such that . That is, not only  is not in , but no con-edge  for  such that  exists in  at all. By definition of conflicting structure and since {\sc Test 2} does not apply to , if  and , then . However, this contradicts the minimality of . Namely,  there exist con-edges  and  for clusters  and , respectively, such that  (in fact,  is any edge in  that crosses ; this edge exists by definition of conflicting structure), and there exists no con-edge  for  such that  with  con-edge for , since by Property~\ref{pr:no-two-edges-same-structure} no con-edge for  different from  belongs to .

Now suppose, for a contradiction, that a planar set  of spanning trees for  exists with . Since no two conflicting edges both belong to  and by Lemma~\ref{le:one-or-the-other}, we have that  and . Further, by Lemma~\ref{le:exactly-one} we have . Since no two conflicting edges both belong to  and by Lemma~\ref{le:one-or-the-other}, we have that  and .

For each , denote by  the path connecting the end-vertices of  and all of whose edges belong to . By assumption, . Hence, denote by  the cycle composed of  and . Analogously, for each , denote by  the path connecting the end-vertices of  and all of whose edges belong to . By assumption, . Hence, denote by  the cycle composed of  and .

We will iteratively prove the following statements (see  Fig.~\ref{fig:two-spokes-2}): (I) For every , edge  belongs to ; (II) for every , edge  belongs to .

\begin{figure}[tb]
\begin{center}
\mbox{\includegraphics[scale=0.38]{Two-spokes-2-M.eps}}
\caption{Illustration for statements (I) and (II).}
\label{fig:two-spokes-2}
\end{center}
\end{figure}

First,  belongs to , as a consequence of the fact that  and  form a separating pair of edges for  and that the end-vertices of  are in different connected components resulting from the removal of  and  from .

Now suppose that, for some , it holds that ; we prove that . Observe that the end-vertices of  are on different sides of , given that  and that . By the Jordan curve theorem, path  crosses an edge of . However,  does not cross any edge of  as all the edges of such paths belong to . Hence,  crosses , hence it contains edge , which proves the statement.

Analogously, suppose that, for some , it holds that ; we prove that . Observe that the end-vertices of  are on different sides of , given that  and that . By the Jordan curve theorem, path  crosses an edge of . However,  does not cross any edge of  as all the edges of such paths belong to . Hence,  crosses , and therefore it contains edge , which proves the statement.

This proves statements (I) and (II). Now observe that the end-vertices of  are on different sides of , given that  has a conflict with  and that  belongs to . Consider a path  that connects the end-vertices of  and all of whose edges belong to . Since the end-vertices of  are on different sides of , by the Jordan curve theorem  crosses an edge of . However,  does not cross any edge of  as all the edges of such paths belong to . Hence,  crosses , a contradiction to the assumption that no con-edge for  crosses . This proves the lemma. Together with Lemma~\ref{le:one-or-the-other}, this lemma establishes the correctness of {\sc Simplification~8}.
\end{proof}

We now prove that our simplifications form a ``complete set''.
\begin{lemma} \label{le:no-simplification-single-conflict}
Suppose that {\sc Simplifications 1--8} do not apply to  and that {\sc Tests 1--4} fail on . Then, every con-edge in  crosses exactly one con-edge in .
\end{lemma}

\begin{proof}
Since {\sc Simplification 2} and {\sc Simplification 3} do not apply to , every con-edge in  has a conflict with at least one con-edge in . Hence, we need to prove that there exists no con-edge in  that has a conflict with two or more con-edges in . Suppose, for a contradiction, that there exists a con-edge  for a cluster  that has a conflict with con-edges for clusters , for some .

Since {\sc Simplifications 1--4} do not apply to  and {\sc Tests 1--4} fail on , by Lemma~\ref{le:donut} there exists an -donut  having  as one of its spokes.

Suppose first that  has more than two spokes. Consider any two spokes  and  in . If , then  and  have isomorphic conflicting structures, hence {\sc Simplification 5} applies to , a contradiction. Hence,  or , say w.l.o.g. . Let  be any con-edge for a cluster , and let  be a con-edge for a cluster , for some , such that . Also, , given that . Since  and  belong to the same connected component of  and do not cross, by Property~\ref{pr:no-two-edges-same-structure}, it follows that  does not cross any con-edge for , hence it lies either in the face  shared by  and  or in the face  shared by  and , say w.l.o.g. that  lies in . If the con-edge  for  has no conflict with any con-edge for , then {\sc Simplification 6} applies to , while if  has a conflict with a con-edge for , then {\sc Simplification 7} applies to . In both cases, we get a contradiction to the fact that {\sc Simplifications 1--8} do not apply to .

Suppose next that  has exactly two spokes  and . If  and  have isomorphic conflicting structures, then {\sc Simplification 5} applies to , a contradiction. Otherwise, consider a minimal index  such that either (1) there exists a con-edge  for a cluster  that crosses a con-edge  for a cluster  , and there exists no con-edge  for  that crosses a con-edge  for , for some  with , or (2) there exists a con-edge  for a cluster  that crosses a con-edge  for a cluster  , and there exists no con-edge  for  that crosses a con-edge  for , for some  with . Observe that (1) or (2) has to apply (as otherwise  and  would have isomorphic conflicting structures). But then {\sc Simplification 8} applies to , a contradiction that proves the lemma.
\end{proof}

A linear-time algorithm to determine whether a planar set  of spanning trees exists for a single-conflict graph is known~\cite{df-ectefcgsf-09}. We thus finally get:

\begin{theorem} \label{th:main}
There exists an -time algorithm to test the -planarity of an embedded flat clustered graph  with at most two vertices per cluster on the boundary of each face.
\end{theorem}

\begin{proof}
The multigraph  of the con-edges can be easily constructed in  time, so that  has  vertices and edges and satisfies Property~\ref{pr:no-two-edges-same-structure}. By Lemma~\ref{le:pssttm}, it suffices to show how to solve the {\sc pssttm} problem for  in  time.

Algorithm~1 correctly determines whether a planar set  of spanning trees for  exists, by Lemmata~\ref{le:disconnected}--\ref{le:no-simplification-single-conflict}. By suitably equipping each con-edge  in  with pointers to the edges in  that have a conflict with , it can be easily tested in  time whether the pre-conditions of each of {\sc Simplifications 1--8} and {\sc Tests 1--4} are satisfied; also, the actual simplifications, that is, removing and contracting edges in , can be performed in  time. Furthermore, the algorithm in~\cite{df-ectefcgsf-09} runs in  time. Since the number of performed tests and simplifications is in , the total running time is , and hence .
\end{proof}

\section{Conclusions} \label{se:conclusions}

We presented a polynomial-time algorithm for testing -planarity of embedded flat clustered graphs with at most two vertices per cluster on each face. An interesting extension of our results would be to devise an FPT algorithm to test the -planarity of embedded flat clustered graphs, where the parameter is the maximum number  of vertices of the same cluster on the same face. Even an algorithm with running time  seems to be an elusive goal. Several key lemmata (e.g. Lemmata~\ref{le:one-or-the-other} and~\ref{le:conflicts-are-bipartite}) do not apply if , hence a deeper study of the combinatorial properties of embedded flat clustered graphs may be necessary.





\bibliographystyle{splncs_srt}
\bibliography{journal}


\end{document}
