\documentclass{patmorin}
\usepackage{amsfonts,amsopn,amsthm,graphicx}
\listfiles

\usepackage[mathlines]{lineno}


\newcommand{\Z}{\mathbb{Z}}
\newcommand{\etal}{\emph{et~al.}}
\pagestyle{empty}
\thispagestyle{empty}

\DeclareMathOperator{\ch}{ch}
\DeclareMathOperator{\ind}{indeg}
\DeclareMathOperator{\outd}{outdeg}

\newtheorem{lem}{Lemma}
\newtheorem{claim}{Claim}
\newtheorem{definition}{Definition}
\newtheorem{problem}{Problem}
\newtheorem{observation}{Observation}
\newtheorem{corollary}{Corollary}
\newtheorem{fact}{Fact}




\newcommand{\seclabel}[1]{\label{sec:#1}}
\newcommand{\Secref}[1]{Section~\ref{sec:#1}}
\newcommand{\secref}[1]{\mbox{Section~\ref{sec:#1}}}

\newcommand{\alglabel}[1]{\label{alg:#1}}
\newcommand{\Algref}[1]{Algorithm~\ref{alg:#1}}
\newcommand{\algref}[1]{\mbox{Algorithm~\ref{alg:#1}}}

\newcommand{\applabel}[1]{\label{app:#1}}
\newcommand{\Appref}[1]{Appendix~\ref{app:#1}}
\newcommand{\appref}[1]{\mbox{Appendix~\ref{app:#1}}}

\newcommand{\tablabel}[1]{\label{tab:#1}}
\newcommand{\Tabref}[1]{Table~\ref{tab:#1}}
\newcommand{\tabref}[1]{Table~\ref{tab:#1}}

\newcommand{\figlabel}[1]{\label{fig:#1}}
\newcommand{\Figref}[1]{Figure~\ref{fig:#1}}
\newcommand{\figref}[1]{\mbox{Figure~\ref{fig:#1}}}

\newcommand{\eqlabel}[1]{\label{eq:#1}}
\newcommand{\eqrefx}[1]{(\ref{eq:#1})}
\newcommand{\eqref}[1]{(\ref{eq:#1})}
\newcommand{\Eqref}[1]{Equation~(\ref{eq:#1})}

\newtheorem{thm}{Theorem}
\newcommand{\thmref}[1]{Theorem~\ref{thm:#1}}
\newcommand{\thmlabel}[1]{\label{thm:#1}}
\newcommand{\lemlabel}[1]{\label{lem:#1}}
\newcommand{\lemref}[1]{Lemma~\ref{lem:#1}}

\begin{document}



\title{Notes on Large Angle Crossing Graphs}


\author{Vida Dujmovi\'c\thanks{{Carleton University, Canada}, \texttt{vida@mcgill.ca}},\,
Joachim Gudmundsson\thanks{{NICTA Sydney, Australia}, \texttt{joachim.gudmundsson@nicta.com.au}},\,
Pat Morin\thanks{{Carleton University, Canada}, \texttt{morin@scs.carleton.ca}},\,
Thomas Wolle\thanks{{NICTA Sydney, Australia}, \texttt{thomas.wolle@nicta.com.au}}\,
}

\maketitle



\begin{abstract}
A geometric graph  is an  angle crossing (AC) graph if
every pair of crossing edges in  cross at an angle of at least
.  The concept of right angle crossing (RAC) graphs ()
was recently introduced by Didimo \etal\ \cite{del-dgrac-09}. It was
shown that any RAC graph with  vertices has at most  edges
and that there are infinitely many values of  for which there exists a RAC
graph with  vertices and  edges.  In this paper, we give upper
and lower bounds for the number of edges in AC graphs for all
.
\end{abstract}

\section{Introduction}

The problem of producing visually appealing graph drawings of relational data sets is a fundamental problem and has been studied extensively, see the books~\cite{bett-gd-99,jm-gds-03,kw-dgmm-01,nr-pgd-04}.
One measure of a graph drawing algorithm's quality is the number of edge crossings it draws~\cite{ew-ecdbg-94,jm-2lscm-97,kw-dgmm-01,n-ibosm-05}.
While some graphs cannot be drawn without edge crossings, some graphs can. These are called planar graphs.
According to this metric, ``good'' algorithms draw graphs with as few edge crossings as possible.
This intuition has some scientific validity:
experiments by Purchase \etal~\cite{p-eiv-00,pca-eeabg-02,wpcm-cmga-02}
have shown that the performance of humans in path tracing tasks is negatively correlated to the number of edge crossings and to the number of bends in the drawing.

However, recently Huang \etal\ \cite{h-uetig-07,h-etseg-08,hhe-eca-08} showed, through eye-tracking experiments, that crossings that occur at angles of greater than  have little effect on a human's ability to interpret graphs.  Therefore, graph drawings with crossing are not bad, as long as the crossings occur with large angles between them. This motivated Didimo \etal\ \cite{del-dgrac-09} to introduce so-called right angle crossing (RAC) graphs.  A \emph{geometric graph}~\cite{ae-degg-89} is a graph  such that the vertices are distinct points in  and edges are straight-line segments. A geometric graph  is a RAC graph if any two crossing segments are orthogonal with each other.

In this paper we generalize the concept to  angle crossing (AC) graphs.  A geometric graph  is an AC graph if every pair of crossing edges in  cross at an angle of at least .  Clearly, AC graphs are more general than planar graphs and RAC graphs, but how much more so?  One measure of generality is the maximum number of edges such a graph can represent.  Euler's Formula implies that a planar graph with  vertices has at most  edges.  How many edges can an AC graph have?


\subsection{Previous Work}
Didimo \etal\ \cite{del-dgrac-09} studied -angle crossing graphs, called right angle
crossing (RAC) graphs, and showed that any RAC graph with  vertices has at most  edges and
that there exist infinitely many values of  for
which there exists a RAC graph with  vertices and  edges.
Recently, Angelini \etal~\cite{acddfks-porac-09} considered some
special cases for drawing RAC graphs, for example, acyclic planar RAC
digraphs and upward RAC digraphs. They showed that there exist acyclic planar
digraphs not admitting any straight-line upward RAC drawing and that the
corresponding decision problem is NP-hard. They also gave a construction of
digraphs whose straight-line upward RAC drawings require exponential area.

For , an AC graph has no three edges that
mutually cross since, otherwise, one of the pairs of edges must
cross at an angle that is at most .  Geometric graphs with no
three pairwise crossing edges are known as \emph{quasiplanar graphs}
\cite{aapps-qpgln-97}. Ackerman and Tardos \cite[Theorem~5]{at-mneqp-07} have shown that
any quasiplanar graph on  vertices has at most  edges.

For , an AC graph has no four pairwise crossing edges.
Ackerman~\cite{a-mnetg-09} has shown that any such graph has at most  
edges.  It remains an open problem whether, for any , a graph with no
 pairwise crossing edges has a linear number of edges.  The best known
upper bound of  on the number of edges in such a graph is due
to Valtr \cite[Theorem~3]{v-epct-99}.

A recent paper of Arikushi \etal\ \cite{afkmt-dgoc-10} considers a similar question. They 
consider a drawing of a graph  on  vertices in the plane where each edge is represented 
by a polygonal arc joining its two respective vertices.  They consider the class of graphs in 
which each edge has at most one bend or two bends, respectively, and any two edges can cross 
only at right angle. It is shown that the number of edges of such graphs is at most  and 
, respectively. 

\subsection{New Results}

The current paper gives upper and lower bounds on the number of edges in
AC graphs.  In \secref{uniform} we show that, for any , the maximum number of edges in an AC graph is at most
.  In \secref{lower-bounds}, we give constructions that
essentially match this upper bound when , for
any integer  and any .  Finally, in~\secref{charging} we use
a charging argument similar to the one used by
Ackerman and Tardos~\cite{at-mneqp-07} to prove that, for
, the number of edges in an AC graph is
bounded by .
An overview of previous and new results is illustrated in \figref{fig:overview}.
\begin{figure*}[tbh]
  \begin{center}
    \includegraphics[width=14cm]{overview}
  \end{center}
  \caption{A plot of previous and new upper and lower bounds. Lower order terms are disregarded.}
  \figlabel{fig:overview}
\end{figure*}

\section{A Uniform Upper Bound}
\seclabel{uniform}

In this section, we give an upper bound of  on the
number of edges in an AC graph.
This upper bound captures the intuition that an AC graph can be viewed as the union of 
planar graphs. The only trouble with this intuition is that 
is not necessarily an integer.

\begin{thm}\thmlabel{uniform-upper-bound}
Let  be an AC graph with  vertices for some .
Then  has at most  edges.
\end{thm}

\begin{proof}
Define the \emph{direction} of an edge  whose lower endpoint is  (in
the case of a horizontal edge, take  as the left endpoint) as the angle 
 where . The direction of an edge  is
therefore a real number in the interval . Let . 
Now, take a random rotation  of  and partition the edge set of  
into spanning subgraphs , and , ,
contains all edges of  whose direction is in the interval .

Note that no two edges of  cross each other, so each  is
a planar graph that, by Euler's Formula, has at most  edges.
Furthermore, since  is a random rotation, the expected number of
edges in  is .
In particular, there must exist some rotation  of  such that .
Therefore,

Rearranging \eqref{uniform} yields

as required.
\end{proof}



\section{Lower Bounds}
\seclabel{lower-bounds}

In this section we give constructions that essentially match
\thmref{uniform-upper-bound} for any  of the form
, where  is an integer.  All our
lower-bounds are based on a general technique of defining a 3-dimensional
geometric graph and taking a 2-dimensional projection of this graph
to obtain an AC graph.  Before presenting the technique in
all its generality, we first present an \emph{ad hoc} construction
for  that still illustrates the main ingredients of the
technique.

\begin{thm} \label{thm:LB_thm2}
For any  and all sufficiently large integers , there exist
 graphs that have  vertices and  edges.
\end{thm}

\begin{proof}
Consider the infinite planar graph whose vertices are the integer lattice
 and in which each vertex  is adjacent to ,
 and  (see \figref{infinite}.a).  This graph is
6-regular, and if we consider the subgraph of this graph induced by an
 grid, then this graph has  edges.  We call
this graph .

We will draw  copies of  in  as follows.
For each , we draw one copy on the ``vertical'' grid
 and one copy on the ``horizontal''
grid  (see
\figref{infinite}.b). Note that there are 4 possible ways of drawing  on these
 grids.  We choose one of the two ways that have the property
that no edge of  is drawn parallel to the -axis.
The resulting 3-d geometric graph has  vertices and  edges.  All that remains is to show how to
project this 3-d geometric graph to the plane without introducing small
crossing angles.

Consider the orthogonal projection of the above graph onto a plane whose
normal is only slightly skewed from the -axis (\figref{infinite}.b
shows such a projection).  For example, we could choose a plane orthogonal
to the vector  for some arbitrarily small .
Since our projection is not parallel to the -axis, each individual copy of
 projects to a planar drawing of .  
Since our projection
is only slightly skewed, no edge of any horizontal (respectively,
vertical) graph intersects any edge of any other horizontal (respectively,
vertical) graph.  Finally, the projection of each edge in each horizontal
graph is a segment whose direction lies in the interval 
and the direction of each edge in each vertical graph is a segment whose
direction lies in the interval , where
 can be made arbitrarily close to 0 by choosing a sufficiently
small value of .  Thus,
any pair of crossing edges cross at an angle of at least  for .
\end{proof}

\begin{figure}
  \begin{center}
  \begin{tabular}{cc}
    \includegraphics[height=2in]{infinite-a} &
    \includegraphics[height=2in]{infinite-both} \\
    (a) & (b)
  \end{tabular}
  \end{center}
  \caption{The lower bound in \thmref{LB_thm2}.}
  \figlabel{infinite}
\end{figure}

\newcommand{\R}{\mathbb{R}}
The construction in the proof of \thmref{LB_thm2} can be viewed as first
starting with an arrangement of lines (in this case,  horizontal
and  vertical lines), extending each line into a plane in  and
drawing a copy of  on this plane so that the vertices lie
on the intersections of several planes.  The following lemma shows that
this can be done with any set of lines:

\begin{lem}\lemlabel{convert}
If there exists a set  of  lines in  such that
\begin{enumerate}
\item any two lines in  are parallel or cross at an angle of at least  and
\item there are  points in  each contained in at least  lines
of , 
\end{enumerate}
then, for any integer  and any , there exist
AC graphs that have  vertices and 
edges.
\end{lem}

For example, applying \lemref{convert} to a set of  horizontal lines
and  vertical lines with  gives a AC graph
with  vertices and  edges, yielding the result
of \thmref{LB_thm2}.

\begin{proof}[Proof of \lemref{convert}]
Refer to \figref{convert} for an example.  As before, we create a 3-d
geometric graph.  Let  be the set of points in  where  or
more lines of  intersect.  For each point , our graph
contains the vertices .  Notice that
each line of  that contains  points in  corresponds to a plane
in  that contains  points.  On each such plane, we draw a
copy of , creating  edges. Again,
we orient this drawing so that no edge of  is parallel
to the -axis.  All  of these graphs are drawn on the same set of
 vertices and the total number of edges in all  of these graphs
is .  Projecting the resulting 3-d geometric graph
in the same manner as the proof of \thmref{LB_thm2} yields the desired
2-d geometric graph.
\end{proof}

\begin{figure}
  \begin{center}
    \includegraphics{convert}
  \end{center}
  \caption{An arrangement of  lines having  vertices of
           degree  yields a graph with  vertices and 
            edges.}
  \figlabel{convert}
\end{figure}

The next result applies \lemref{convert} to some standard grid
constructions:

\begin{thm}\label{thm:finite}
For any , any , and
all sufficiently large integers , there exist AC
graphs that have  vertices and  edges.
\end{thm}

\begin{proof}
Refer to \figref{lattices}.
As we have already seen, for  we use a set of  horizontal
and vertical lines and apply \lemref{convert}.  For  we use
lines that support the triangular lattice.  For  we use
horizontal and vertical lines as well as lines of slope 1 and . For
, we use a further refinement of the triangular lattice.
\begin{figure}
  \begin{center}
    \includegraphics{lattices}
  \end{center}
  \caption{The lattices used to prove lower bounds in \thmref{finite}.}
  \figlabel{lattices}
\end{figure}
\end{proof}

Finally, we prove a lower bound for any  of the form
:
\begin{thm}
For any , any integer , and all sufficiently large
integers , there exist  graphs that have 
vertices and  edges.
\end{thm}

\begin{proof}
Set .  We prove this result by showing the
existence of a set of  lines satisfying the conditions of
\lemref{convert} with  and .  We can then
apply \lemref{convert} with , giving a AC
graph, with  and  edges, as required.

We prove the existence of this set of lines as follows: We select a set
of  lines whose slopes are rational, but very close to the directions
.  We then use 
translates of these  lines to cover each of the vertices of an  grid by lines with each of the  slopes.

The set of lines that we use will be such that all  vertices of
the  grid  will have
 lines passing through them.  We do this by first constructing a
\emph{-frame}  consisting of  lines through the origin so that
any two lines of the -frame meet at angles of at least .
This -frame also has the property that each line passes through a
grid point  such that  for some integer . (See \figref{bigproof}.a.)

\begin{figure}
  \begin{center}
    \begin{tabular}{ccc}
    \includegraphics[width=1.7in]{frame2}
    &
    \includegraphics[width=1.7in]{boundary}
    &
    \includegraphics[width=1.7in]{cover} \\
    (a) & (b) & (c)
    \end{tabular}
  \end{center}
  \caption{(a)~A 3-frame, , in which each line passes through a grid point
 with ; 
           (b)~The set , for ; and 
           (c)~Placing  at each point in  covers each point of
 with 3 lines}
  \figlabel{bigproof}
\end{figure}

\newcommand{\floor}[1]{\lfloor #1 \rfloor}

Next, we make  translated copies of the frame  by
translating  to each of the points in the set

(These are the points that are within distance  of the boundary of
the grid ; see \figref{bigproof}.b.)  Denote the resulting set of
 lines by .  Notice that, for every point  in
, there are  lines of  that contain .  This follows from
the fact that each line in the frame  intersects all lattice points
 for some  with 
(see \figref{bigproof}.c).  Thus, the set  of lines, combined with
\lemref{convert} proves the theorem.  To complete the proof, all that
remains is to describe the frame  used to construct .

The frame  consists of  lines , all passing
through the origin, and such that the direction of  is in .  Consider the \emph{ideal line} 
that contains the origin and that has direction exactly .
Take a point  on  whose distance from the origin is
 (see \figref{angle}).  There is a point
 whose distance from  is at most .
The angle  is at most .  The distance from 
to the origin is at most

Therefore .  For our frame , we take the line
 that contains the origin and .  This yields a set  of lines
such that the angle between any two lines is at most , and
completes the proof.
\end{proof}

\begin{figure}
  \begin{center}
    \includegraphics{angle}
  \end{center}
  \caption{For every line  through the origin with direction
, there exists a line  through the origin and  with
direction  and with }
  \figlabel{angle}
\end{figure}





\section{Charging Arguments}
\seclabel{charging}

In this section we derive upper bounds using charging arguments similar
to those used by Ackerman and Tardos \cite{at-mneqp-07} and Ackerman
\cite{a-mnetg-09}.  Let  be an AC graph.  We denote by 
the planar graph obtained by introducing a vertex at each point where
two or more edges in  cross (thereby subdividing) edges of .

For a face  of , we denote by  the length of the facial walk around 
so that, if we walk along an edge twice during the traversal, then it contributes twice to
.  Let  denote the number of steps of this traversal during
which a vertex of  (as opposed to a vertex introduced in ) is
encountered.  For each face  of  define the \emph{initial charge}
of~ as

Ackerman and Tardos show, using two applications of Euler's formula, that

We call a face  of  a -\emph{shape} if  and  is a
\emph{shape}.  For example, a 2-pentagon is a face of  with  and
.

As a warm-up, and introduction to charging arguments, we offer an alternate
proof to the upper bound presented by Didimo \etal\ in~\cite{del-dgrac-09}.

\begin{thm}
A RAC graph with  vertices has at most  edges.
\end{thm}
\begin{proof}
Let  be a maximal RAC graph on  vertices, and define  and 
as above.  We claim that, for every face  of , .
To see this, observe that the claim is certainly true if .  On
the other hand, if  then, by the RAC property, , so it
is also true in this case.
Therefore,

which proves that .

To improve the above bound, observe that, since  is maximal all vertices
on the outer face,~, of  are vertices of .  If  then
, so in this case, proceeding as above, we have

and we are done.  Otherwise, the outer face of  is a 3-triangle and
.
Consider the internal faces of  incident to the three edges of .
Because  is maximal, and , there must
be three such faces and each of these three faces, , has .
Furthermore, at most one of these faces is a 2-triangle.\footnote{This is proven by a simple geometric argument that shows for
any triangle , two right-angle triangles that are interior to 
and each share an edge with  must overlap. See, e.g., the proof of
Theorem~1 in Reference~\cite{del-dgrac-09}.}
Consider each of the other two faces, .  If  then

since .
Otherwise, , and we have 

since .
Therefore, we have

which, implies that  since  is an integer.
\end{proof}

Next, we prove an upper bound for  that improves on the
 upper bound that follows from Ackerman and Tardos' bound on
quasiplanar graphs.

\begin{thm}\label{thm:six-n}
Let  be an AC graph with  vertices, for .
Then  has at most  edges.
\end{thm}

\begin{proof}
We will redistribute the charge in the geometric graph  to obtain a new charge 
such that  for every face  of .  In this way, we
get

which we rewrite to get .

The charge  is obtained as follows.  Let  be any 1-triangle of
.  (Note that .)   That is,  is a triangle formed by two
edges  and  that meet at a vertex  of  and an edge  that
crosses  and .
Imagine walking along the bisector of  and  (starting in the
interior of ) until reaching a face  such that  is not a
0-quadrilateral.  To see why such an  exists, observe that if we
encounter nothing but 0-quadrilaterals we will eventually reach a face that
contains an endpoint of  or  and is therefore not a
0-quadrilateral.

Adjust the charges at  and  by subtracting  from  and
adding  to .  It is helpful to think of the charge as leaving
 through the last edge  traversed in the walk.  Note that neither
endpoint of  is a vertex of .  This implies that for a face ,
the amount of charge that leaves~ is at most

Let  be the charge obtained after performing this redistribution of
charge for every 1-triangle .  We claim that  for every face  of .  To
see this, we need only run through a few cases that can be verified using
\eqref{leaving} and the following observations:

\begin{enumerate}
\item If , then , so .

\item If , then  since, otherwise,  has two edges on
its boundary that cross at an angle of less than or equal to .

\item If , and  is a 0-quadrilateral then , by construction.

\item If , and  is a 1-triangle then , by construction.

\item If  then  since, otherwise,  has two edges on
its boundary that cross at an angle less of at most .
\end{enumerate}
This completes the proof.
\end{proof}


\section{Notes}

Theorem~\ref{thm:six-n} appears to be true even for , but we have not
been able to prove it. The problem occurs because 0-pentagons can finish with
a charge of  or . (See \figref{pentagons}.)

\begin{figure}
  \begin{center}
    \includegraphics{pentagons}
  \end{center}
  \caption{Pentagrams lead to 0-pentagons with negative charge.}
  \figlabel{pentagons}
\end{figure}

A possible proof could look for extra charge near the vertices of the
penta\emph{gram} that created this pentagon, but it is easy to make gadgets
so that the faces surrounding those vertices have no extra charge.  Another
option is to look for extra charge near the vertices of the penta\emph{gon}
itself. Again, it is not too hard to to make them have no extra charge.
(See \figref{p2}.)

\begin{figure} [bth]
  \begin{center}
    \includegraphics{p2}
  \end{center}
  \caption{Pentagrams can have 0 extra charge at their vertices and
           0 extra charge at the vertices of a pentagon.}
  \figlabel{p2}
\end{figure}


We also tried to follow the Ackerman-Tardos proof more closely. Namely, we
distribute the charge so that  and then prove that there
is leftover charge at the faces around each vertex.  For this to give a
bound of  we would need the extra charge at each vertex to be .
Unfortunately, the limiting case in Ackerman-Tardos is  and this is
realizable even with crossing angles arbitrarily close to .
(See \figref{at-bound}.)

\begin{figure}
  \begin{center}
    \includegraphics{at-bound}
  \end{center}
  \caption{The Ackerman-Tardos proof cannot even prove a bound of  for
          crossing angles of .}
  \figlabel{at-bound}
\end{figure}

Finally, we can take a more global approach.  Discharging rules define a
directed graph among the faces (and possibly vertices) of .  An edge
 indicates that a charge of  travels from  to , for some
number  ( in our argument).  The graph has to respect some flow
rules.  For example, in Theorem~\ref{thm:six-n} we have

where  and  denote the in and out degree.  The goal would be
to define discharging paths recursively and then show that the recursion
terminates (i.e.~that the resulting graph is acyclic) and that the flow rule is satisfied.


\subsection*{Acknowledgements}
NICTA is funded by the Australian Government as represented by the
Department of Broadband, Communications and the Digital Economy and the
Australian Research Council through the ICT Centre of Excellence program.
Pat Morin's research was supported by NSERC, CFI, the Ontario Innovation Trust, NICTA, and the University of Sydney.

\bibliographystyle{plain}



 \begin{thebibliography}{99}

 \bibitem{a-mnetg-09}
   E. Ackerman.
   \newblock On the maximum number of edges in topological graphs with no four pairwise crossing edges.
   \newblock Discrete and Computational Geometry, 41(3):365--375, 2009.

 \bibitem{at-mneqp-07}
   E. Ackerman and G. Tardos.
   \newblock On the maximum number of edges in quasi-planar graphs.
   \newblock Journal of Combinatorial Theory Ser. A 114(3):563--571, 2007.

 \bibitem{aapps-qpgln-97}
   P. K. Agarwal, B. Aronov, J. Pach, R. Pollack and M. Sharir.
   \newblock Quasi-planar graphs have a linear number of edges.
   \newblock Combinatorica, 17(1):1--9, 1997.

 \bibitem{ae-degg-89}
    N. Alon and P. Erd\"{o}s.
    \newblock Disjoint edges in geometric graphs.
    \newblock Discrete and Computational Geometry, 4:287--290, 1989.

 \bibitem{acddfks-porac-09}
   P. Angelini, L. Cittadini, G. Di Battista, W. Didimo, F. Frati, M. Kaufmann and A. Symvonis.
   \newblock On the perspectives opened by right angle crossing drawings.
   \newblock In Proceedings of the 17th International Symposium on Graph Drawing, pages 21--32, 2009. 
             Lecture Notes in Computer Science, 5664. Springer.

  \bibitem{afkmt-dgoc-10}
   K. Arikushi, R. Fulek, B. Keszegh, F. Mori\'{c} and C. D. T\'{o}th.
   \newblock Drawing graphs with orthogonal crossings.
   \newblock In Proceedings of the 36th Workshop on Graph Theoretic Concepts in Computer Science, 2010.

 \bibitem{bett-gd-99}
   G. Di Battista, P. Eades, R. Tamassia and I. G. Tollis.
   \newblock Graph drawing.
   \newblock Prentice Hall, Upper Saddle River, NJ, 1999.

 \bibitem{del-dgrac-09}
   W. Didimo, P. Eades and G. Liotta,
   \newblock Drawing graphs with right angle crossings.
   \newblock In Proceedings of the 11th International Symposium on Algorithms and Data Structures (WADS), pages 206--217, 2009.



 \bibitem{ew-ecdbg-94}
   P. Eades and N.C. Wormald.
   \newblock Edge crossing in drawing bipartite graphs.
   \newblock Algorithmica, 11:379--403, 1994.

 \bibitem{h-uetig-07}
   W. Huang.
   \newblock Using eye tracking to investigate graph layout effects.
   \newblock In Proceedings of the 6th International Asia-Pacific Symposium on
                Visualization, pages 97--100, 2007.

 \bibitem{h-etseg-08}
   W. Huang.
   \newblock An eye tracking study into the effects of graph layout.
   \newblock CoRR, abs/0810.4431,  2008.

 \bibitem{hhe-eca-08}
   W. Huang, S.-H. Hong and P. Eades.
   \newblock Effects of crossing angles.
   \newblock In Proceedings of the IEEE VGTC Pacific Visualization Symposium, pages 41--46, 2008.

 \bibitem{jm-gds-03}
   M. J\"unger and P. Mutzel (Eds.).
   \newblock Graph drawing software.
   \newblock Springer Verlag, 2003.

 \bibitem{jm-2lscm-97}
   M. J\"unger and P. Mutzel.
   \newblock 2-Layer straightline crossing minimization: performance of exact and heuristic algorithms.
   \newblock Journal of Graph Algorithms and Applications, 1(1):1--25, 1997.

 \bibitem{kw-dgmm-01}
   M. Kaufmann and D. Wagner.
   \newblock Drawing graphs, methods and models.
   \newblock Lecture Notes in Computer Science, Volume 2025, Springer, 2001.

 \bibitem{n-ibosm-05}
   H. Nagamochi.
   \newblock An improved bound on the one-sided minimum crossing number in two-layered drawings.
   \newblock Discrete and Computational Geometry, 33(4):569--591, 2005.

 \bibitem{nr-pgd-04}	
   T. Nishizeki and M. S. Rahman.
   \newblock Planar graph drawing.
   \newblock World Scientific, 2004

 \bibitem{p-eiv-00}
   H. C. Purchase.
   \newblock Effective information visualisation: a study of graph drawing aesthetics and algorithms.
   \newblock Interacting with Computers, 13(2):147--162, 2000.

 \bibitem{pca-eeabg-02}
   H. C. Purchase, D. A. Carrington and J.-A. Allder.
   \newblock Empirical evaluation of aesthetics-based graph layout.
   \newblock Empirical Software Engineering, 7(3):233--255, 2002.

 \bibitem{v-epct-99}
   P.~Valtr.
   \newblock On geometric graphs with no k pairwise parallel edges.
   \newblock Discrete and Computational Geometry, 19:461--469, 1998.

 \bibitem{wpcm-cmga-02}
   C. Ware, H. C. Purchase, L. Colpoys and M. McGill.
   \newblock Cognitive measurements of graph aesthetics.
   \newblock Information Visualization, 1(2):103--110, 2002.

 \end{thebibliography}

\end{document}
