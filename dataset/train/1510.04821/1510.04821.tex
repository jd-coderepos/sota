Our recent work~\cite{FOOL} presented a modification of many-sorted first-order logic that contains a boolean sort with a fixed interpretation and treats terms of the boolean sort as formulas. We called this logic FOOL, standing for first-order logic (FOL) + boolean sort. FOOL extends FOL by (1) treating boolean terms as formulas; (2) \ITE\ expressions; and (3) \LETIN\ expressions. There is a model-preserving transformation of FOOL formulas to FOL formulas, hence an implementation of this transformation makes it possible to prove FOOL formulas using a first-order theorem prover.

The language of FOOL is, essentially, a superset of the core language of SMT-LIB~2~\cite{SMT-LIB}, the library of problems for SMT solvers. The difference between FOOL and the core language is that the former has richer \LETIN\ expressions, which support local definitions of functions symbols of arbitrary arity, while the latter only supports local binding of variables.

FOOL can be regarded as the smallest superset of the SMT-LIB~2 Core language and TFF0. An implementation of a translation of FOOL to FOL thus also makes it possible to translate SMT-LIB problems to TPTP. Consider, for example, the following tautology, written in the SMT-LIB syntax: \verb'(exists ((x Bool)) x)'. It quantifies over boolean variables and uses a boolean variable as a formula. Neither is allowed in the standard TPTP language, but can be directly expressed in an extended TPTP that represents FOOL.

The rest of this section presents features of FOOL not included in FOL, explains how they are implemented in Vampire and how they can be represented in an extended TPTP syntax understood by Vampire.

\subsection{Proving with the Boolean Sort}

Vampire supports many-sorted predicate logic and the TFF0 syntax for this logic. In many-sorted predicate logic all sorts are uninterpreted, while the boolean sort should be interpreted as a two-element set. There are several ways to support the boolean sort in a first-order theorem prover, for example, one can axiomatise it by adding two constants  and  of this sort and two axioms:  and . However, as we discuss in \cite{FOOL}, using this axiomatisation in a superposition theorem prover may result in performance problems caused by self-paramodulation of .

To overcome this problem, in \cite{FOOL} we proposed the following modification of the superposition calculus.
\begin{enumerate}
  \item Use a special simplification ordering that makes the constants  and  smallest terms of the sort  and also makes  greater than .

\item Add the axiom .

\item Add a special inference rule, called \emph{FOOL paramodulation}, of the form
  
where
\begin{enumerate}
\item  is a term of the sort  other than  and ;
\item  is not a variable;
\end{enumerate}
\end{enumerate}

Both ways of dealing with the boolean sort are supported in Vampire. They are controlled by the option \verb|--fool_paramodulation|, which can be set to \verb|on| or \verb|off|. The default value is \verb|on|, which enables the modification.

Vampire uses the TFF0 subset of the TPTP syntax, which does not fully support FOOL. To write FOOL formulas in the input, one uses the standard TPTP notation: \tptpo\ for the boolean sort, \dtrue\ for  and \dfalse\ for . There are, however, two ways to output the boolean sort and the constants. One way will use the same notation as in the input and is the default, which is sufficient for most applications. The other way can be activated by the option \verb'--show_fool on', it will
\begin{enumerate}
  \item denote as \dbool\ every occurrence of  as a sort of a variable or an argument (to a function or a predicate symbol);
  \item denote as \ddtrue\ every occurrence of  as an argument; and
  \item denote as \ddfalse\ every occurrence of  as an argument.
\end{enumerate}
Note that an occurrence of any of the symbols \dbool, \ddtrue\ or \ddfalse\ anywhere in an input problem is not recognised as syntactically correct by Vampire.

Setting \verb'--show_fool' to \verb'on' might be necessary if Vampire is used as a front-end to other reasoning tools. For example, one can use Vampire not only for proving, but also for preprocessing the input problem or converting it to clausal normal form. To do so, one uses the options \verb|--mode preprocess| and \verb|--mode clausify|, respectively. The output of Vampire can then be passed to other theorem provers, that either only deal with clauses or do not have sophisticated preprocessing. Setting \verb'--show_fool' to \verb'on' appends a definition of a sort denoted by \dbool\ and constants denoted by \ddtrue\ and \ddfalse\ of this sort to the output. That way the output will always contain syntactically correct TFF0 formulas, which might not be true if the option is set to \verb'off' (the default value).

Every formula of the standard FOL is syntactically a FOOL formula and has the same models. Vampire does not reason in FOOL natively, but rather translates the input FOOL formulas into FOL formulas in a way that preserves models. This is done at the first stage of preprocessing of the input problem.

Vampire implements the translation of FOOL formulas to FOL given in~\cite{FOOL}. It involves replacing parts of the problem that are not syntactically correct in the standard FOL by applications of fresh function and predicate symbols. The set of assumptions is then extended by formulas that define these symbols. Individual steps of the translation are displayed when the \verb'--show_preprocessing' options is set to \verb'on'.

In the next subsections we present the features of FOOL that are not present in FOL together with their syntax in the extended TFF0 and their implementation in Vampire.

\subsection{Quantifiers over the Boolean Sort}

FOOL allows quantification over  and usage of boolean variables as formulas. For example, the formula  is a syntactically correct tautology in FOOL. It is not however syntactically correct in the standard FOL where variables can only occur as arguments.

Vampire translates boolean variables to FOL in the following way. First, every formula of the form , where  and  are boolean variables, is replaced by . Then, every occurrence of a boolean variable  anywhere other than in an argument is replaced by . For example, the tautology  will be converted to the FOL formula  during preprocessing.

Note that it is possible to directly express quantified boolean formulas (QBF) in FOOL, and use Vampire to reason about them.

TFF0 does not support quantification over booleans. Vampire supports an extended version of TFF0 where the sort symbol \tptpo\ is allowed to occur as the sort of a quantifier and boolean variables are allowed to occur as formulas. The formula  can be expressed in this syntax as \lstinline'![X:\mathit{impl}\bool \times \bool \to \bool\mathit{impl}o * o).
tff(impl_definition, axiom,
    ![X:o]: (impl(X,Y) <=> (~X | Y))).
\end{lstlisting}

\subsection{Formulas as Arguments}

Unlike the standard FOL, FOOL does not make a distinction between formulas and boolean terms. It means that a function or a predicate can take a formula as a boolean argument, and formulas can be used as arguments to equality between booleans. For example, with the definition of , given earlier, we can express in FOOL that
 is a graph of a (partial) function of the type  as follows:


Note that the definition of  could as well use equality instead of equivalence.

In order to support formulas occurring as arguments, Vampire does the following. First, every expression of the form  is replaced by . Then, for each formula  occurring as an argument the following translation is applied. If  is a nullary predicate  or , it is replaced by  or , respectively. If  is a boolean variable, it is left as is. Otherwise, the translation is done in several steps. Let  be all free variables of  and  be their sorts. Then Vampire
\begin{enumerate}
  \item introduces a fresh function symbol  of the type 
  \item adds the definition  to its set of assumptions;
  \item replaces  by .
\end{enumerate}

For example, after this translation has been applied for both arguments of , \eqref{eq:bool-arg-example} becomes  where  and  are fresh function symbol of the types  and , respectively, defined by the following formulas:
\begin{enumerate}
  \item ;
  \item .
\end{enumerate}

TFF0 does not allow formulas to occur as arguments. The extended version of TFF0, supported by Vampire, removes this restriction for arguments of the boolean sort. Formula~\eqref{eq:bool-arg-example} can be expressed in this syntax as follows:
\begin{lstlisting}
![X:s, Y:t, Z:t]: impl(p(X,Y) & p(X,Z), Y = Z)
\end{lstlisting}

For a more interesting example, consider the following logical puzzle taken from the TPTP problem \mbox{PUZ081}:
\begin{quote}
  A very special island is inhabited only by knights and knaves. Knights always tell the truth, and knaves always lie. You meet two inhabitants: Zoey and Mel. Zoey tells you that Mel is a knave. Mel says, `Neither Zoey nor I are knaves'. Who is a knight and who is a knave?
\end{quote}

\newcommand{\knight}{\mathit{Knight}}
\newcommand{\knave}{\mathit{Knave}}
\newcommand{\says}{\mathit{Says}}
\newcommand{\statement}{\mathit{statement}}
\newcommand{\person}{\mathit{person}}
\newcommand{\zoye}{\mathit{zoye}}
\newcommand{\mel}{\mathit{mel}}
To solve the puzzle, one can formalise it as a problem in FOOL and give a corresponding extended TFF0 representation to Vampire. Let  and  be terms of a fixed sort  that represent Zoye and Mel, respectively. Let  be a predicate that takes a term of the sort  and a boolean term. We will write  to denote that a person  made a logical statement . Let  and  be predicates that take a term of the sort . We will write  or  to denote that a person  is a knight or a knave, respectively. We will express the fact that knights only tell the truth and knaves only lie by axioms  and , respectively. We will express the fact that every person is either a knight or a knave by the axiom , where  is the ``exclusive or'' connective. Finally, we will express the statements that Zoye and Mel make in the puzzle by axioms  and , respectively.

The axioms and definitions, given above, can be written in the extended TFF0 syntax in the following way.
\begin{lstlisting}
tff(person, type, person: o) > o).
tff(knights_always_tell_truth, axiom,
    ![P:person, S:o).
tff(knaves_always_lie, axiom,
    ![P:person, S:\ite{\psi}{s}{t}\psist\mathit{max}\Z \times \Z \to \Zx_1,\ldots,x_n\psist\sigma_1,\ldots,\sigma_n\taust\tau\bool\tau\boolg\ite{\psi}{s}{t}g(x_1,\ldots,x_n)\tau\boolg\sigma_1 \times \ldots \times \sigma_ng\Z \times \Z \to \Z(\forall x:\Z)(\forall y:\Z)(x \geq y \implies g(x, y) \eql x)(\forall x:\Z)(\forall y:\Z)(x \not\geq y \implies g(x, y) \eql y).\mathit{max}int * int).
tff(max_definition, axiom,
    ![X:int]:
      (max(X,Y) = greatereq(X,Y),X,Y))).
\end{lstlisting}It uses the TPTP notation \dint\ for the sort of integers and \dgreatereq\ for the greater-than-or-equal-to comparison of two numbers.



Consider now the following valid property of :


Its encoding in the extended TFF0 can use the same \dite\ expression:
\begin{lstlisting}
![X:int]: greatereq(X,Y),
                           m \geq 1f_1,\ldots,f_mn_i \geq 01 \leq i \leq mx^i_1\ldots,x^i_{n_i}1 \leq i \leq ms_1,\ldots,s_mts_1,\ldots,s_mf_1,\ldots,f_mm\arrayt\Z \to \Zabf(b, a)f_i0 \leq i < ms_{i+1},\ldots,s_mg_if_itg_iy_1,\ldots,y_mst\tau_1,\ldots,\tau_mx_1,\ldots,x_ny_1,\ldots,y_m\sigma_0s\sigma_0\bool\sigma_0\boolgz_1,\ldots,z_ns'sx_1,\ldots,x_nz_1,\ldots,z_n\letin{f(x_1:\sigma_1,\ldots,x_n:\sigma_n)}{s}{t}t't'ty_1,\ldots,y_mf(t_1, \ldots, t_n)fg(t_1, \ldots, t_n,\allowbreak y_1, \ldots, y_m)\sigma_0\boolg(2) + g(3)g\Z \to \Zaba'f(a'', b')a''b'a'' \eql bb' \eql ai|).

Formula \eqref{eq:let-function-example} can be written in the extended TFF0 with the TPTP interpreted function \dsum, representing integer addition, as follows:
\begin{lstlisting}
int) := sum(array(2), array(3))).
\end{lstlisting}



The same \dlet\ expression can be used for multiple bindings. For that, the bindings should be separated by a semicolon and passed as the first argument. The formula~\eqref{eq:parallel-let-example} can be written using \dlet\ as follows.
\begin{lstlisting}
i]:
    i, Y2:let_tt(![Z1:let_ft(![Y3:i]: (q(Y3,Y4) <=>
          ite_t(q(b, b), f(a), f(X)))))).
\end{lstlisting}

It uses both of the TFF0 variations of \ITE\ and three different variations of \LETIN. The same snippet can be expressed more concisely using \dite\ and \dlet\ expressions.
\begin{lstlisting}
tff(let_binders, axiom, ![X:let(q(Y1,Y2) := p(Y1),
      q(let(q(Y3,Y4) :=
               ite(q(b,b), f(a), f(X)))))).
\end{lstlisting}
