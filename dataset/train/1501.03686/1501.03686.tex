\documentclass[11pt,a4paper]{article}

\usepackage{fullpage}
\usepackage[utf8x]{inputenc}
\usepackage{amsmath, amsthm}
\usepackage{wrapfig,graphicx,amssymb,textcomp,array,amsmath}
\usepackage{algpseudocode} 
\usepackage{enumerate}
\algtext*{EndWhile}\algtext*{EndIf}\usepackage{algorithm}
\usepackage{color}

\setlength{\arraycolsep}{0in}

\newcommand{\CH}[1]{\text{}}
\newcommand{\HC}[1]{\text{}}
\newcommand{\M}[2]{\text{}}
\newcommand{\Min}[2]{\text{\sf Min}}
\newcommand{\Cut}[2]{\text{\sf Cut}}
\newcommand{\Tangent}[2]{\text{\sf Tangent}}
\newcommand{\UT}[2]{\text{\em UpperTangent}}
\newcommand{\LT}[2]{\text{\em LowerTangent}}
\newcommand{\fn}[1]{f(#1)}
\newcommand{\Ln}[1]{L(#1)}
\newcommand{\bt}{\beta}
\newcommand{\gm}{\gamma}
\newcommand{\LC}[1]{{\em left}}
\newcommand{\RC}[1]{{\em right}}
\newcommand{\Kn}[1]{K#1}
\newcommand{\e}[2]{e_{#1}{(#2)}}
\newcommand{\pmp}[1]{pmp(#1)}

\title{Packing Plane Perfect Matchings into a Point Set\thanks{Research supported by NSERC.}}
\author{
Ahmad Biniaz\thanks{School of Computer Science, Carleton University, 
                    Ottawa, Canada.}
\and 
Prosenjit Bose\footnotemark[2]
\and
Anil Maheshwari\footnotemark[2]
\and 
Michiel Smid\footnotemark[2]
}\date{\today}
\newtheorem{lemma}{Lemma}
\newtheorem{corollary}{Corollary}
\newtheorem{conjecture}{Conjecture}
\newtheorem{theorem}{Theorem}
\newtheorem{observation}{Observation}
\begin{document}

\maketitle

\begin{abstract}
Given a set  of  points in the plane, where  is even, we consider the following question: How many plane perfect matchings can be packed into ? We prove that at least  plane perfect matchings can be packed into any point set . For some special configurations of point sets, we give the exact answer. We also consider some extensions of this problem.
\end{abstract}
\section{Introduction}
\label{introduction-section}
Let  be a set of  points in general position (no three points on a line) in the plane. A {\em geometric graph}  is a graph whose vertex set is  and whose edge set  is a set of straight-line segments with endpoints in . We say that two edges of  {\em cross} each other if they have a point in common that is interior to both edges. Two edges are {\em disjoint} if they have no point in common. A subgraph  of  is said to be {\em plane} ({\em non-crossing} or {\em crossing-free}) if its edges do not cross. A {\em plane matching} is a plane graph consisting of pairwise disjoint edges. Two subgraphs  and  are {\em edge-disjoint} if they do not share any edge. A {\em complete geometric graph}  is a geometric graph on  which contains a straight-line edge between every pair of points in . 

We say that the sequence  of subgraphs is {\em packed into} , if the subgraphs in this sequence are pairwise edge-disjoint. In a packing problem, we ask for the largest number of subgraphs of a given type that can be packed into . Among all subgraphs of , {\em plane perfect matchings}, {\em plane spanning trees}, and {\em plane spanning paths} are of interest. That is, one may look for the maximum number of plane spanning trees, plane Hamiltonian paths, or plane perfect matchings that can be packed into . Since  has  edges, at most  spanning trees, at most  spanning paths, and at most  perfect matchings can be packed into it. 

A long-standing open question is to determine if the edges of  (where  is even) can be partitioned into  plane spanning trees. In other words, is it possible to pack  plane spanning trees into ? If  is in convex position, the answer in the affirmative follows from the result of Bernhart and Kanien~\cite{Bernhart1979}. For  in general position, Aichholzer et al.~\cite{Aichholzer2014} prove that  plane spanning trees can be packed into . They also show the existence of at least 2 edge-disjoint plane spanning paths. 

In this paper we consider a closely related question: How many plane perfect matchings can be packed into , where  is a set of  points in general position in the plane, with  even? 
\subsection{Previous Work}
\label{previous-work-section}
\subsubsection{Existence of Plane Subgraphs}
The existence of certain plane subgraphs in a geometric graph on a set  of  points is one of the classical problems in combinatorial and computational geometry. 

One of the extremal problems in geometric graphs which was first studied by Avital and Hanani~\cite{Avital1966}, Kuptiz~\cite{Kupitz1979}, Erd\H{o}s~\cite{Erdos1946}, and Perles (see reference \cite{Toth1999}) is the following. What is the smallest number  such that any geometric graph with  vertices and more than  edges contains  pairwise disjoint edges, i.e., a plane matching of size at least . Note that . By a result of Hopf and Pannwitz~\cite{Hopf1934} and Erd\H{o}s~\cite{Erdos1946}, , i.e., any geometric graph with  edges contains a pair of disjoint edges. Specifically, they showed that the diameter of a point set, i.e., the maximum distance among  points in the plane, can be repeated  times. Let  be the geometric graph on  obtained by connecting each pair of points with the maximum distance by an edge. If  contains two disjoint edges  and , then the convex hull of  forms either a triangle or a quadrilateral. In both cases there is a distance longer than  and ; which is a contradiction. Thus  cannot have two disjoint edges. From this, it follows that .

Alon and Erd\H{o}s~\cite{Alon1989} proved that , i.e., any geometric graph with  vertices and at least  edges contains a plane matching of size three. This bound was improved to  by Goddard et al.~\cite{Goddard1996}. Recently {\v{C}}ern{\'{y}}~\cite{Cerny2005} proved that ; while the lower bound of  is due to Perles (see \cite{Cerny2005}). For , Goddard et al.~\cite{Goddard1996} showed that , which was improved by T{\'{o}}th and Valtr~\cite{Toth1999} to .

For general values of , Akiyama and Alon~\cite{Akiyama1989} gave the upper bound of . Goddard et al.~\cite{Goddard1996} improved the bound to . Pach and T{\"{o}}r{\H{o}}csik~\cite{Pach1994} obtained the upper bound of ; which is the first upper bound that is linear in . The upper bound was improved to  by T{\'{o}}th and Valtr~\cite{Toth1999}; they also gave the lower bound of . T{\'{o}}th~\cite{Toth2000} improved the upper bound to , where the constant has been improved to  by Felsner~\cite{Felsner2004}. It is conjectured that  for some constant .

For the maximum value of , i.e., , with  even, Aichholzer et al.~\cite{Aichholzer2010} showed that . That is, by removing  edges from any complete geometric graph, the resulting graph has  disjoint edges, i.e., a plane perfect matching. This bound is tight; there exist complete geometric graphs, such that by removing  edges, the resulting graph does not have any plane perfect matching. Similar bounds were obtained by Kupitz and Perles for complete convex graphs, i.e., complete graphs of point sets in convex position. 
Specifically, for convex geometric graphs, Kupitz and Perles showed that ; see \cite{Goddard1996} (see also \cite{Akiyama1989} and \cite{Alon1989}). In particular, in the convex case,  edges guarantee a plane matching of size three. In addition, Keller and Perles~\cite{Keller2012} gave a characterization of all sets of  edges whose removal prevents the resulting graph from having a plane perfect matching.

\v{C}ern{\'{y}} et al.~\cite{Cerny2007} considered the existence of Hamiltonian paths in geometric graphs. They showed that after removing at most  edges from any complete geometric graph of  vertices, the resulting graph still contains a plane Hamiltonian path. Aichholzer et al.~\cite{Aichholzer2010} obtained tight bounds on the maximum number of edges that can be removed from a complete geometric graph, such that the resulting graph contains a certain plane subgraph; they considered plane perfect matchings, plane subtrees of a given size, and triangulations. 
\subsubsection{Counting Plane Graphs}
The number of plane graphs of a given type in a set of  points is also of interest. In 1980, Newborn and Moser~\cite{Newborn1980} asked for the maximal number of plane Hamiltonian cycles; they give an upper bound of , but conjecture that it should be of the form , for some constant . In 1982, Ajtai et al.~\cite{Ajtai1982} proved that the number of plane graphs is at most . Every plane graph is a subgraph of some triangulation (with at most  edges). Since a triangulation has at most  plane subgraphs, as noted in~\cite{Garcia2000}, any bound of  on the number of triangulations implies a bound of  on the number of plane graphs. The best known upper bound of , for the number of triangulations is due to Santos and Seidel~\cite{Santos2003}. This implies the bound  for plane graphs. As for plane perfect matchings, since a perfect matching has  edges, Dumitrescu~\cite{Dumitrescu1999} obtained an upper bound of , where . Recently, Sharir and Welzl~\cite{Sharir2006} showed that the number of plane perfect matchings is at most . They also showed that the number of all (not necessarily perfect) plane matchings is at most . 

Garc{\'{\i}}a et al.~\cite{Garcia2000} showed that the number of plane perfect matchings of a fixed size set of points in the plane is minimum when the points are in convex position. Motzkin~\cite{Motzkin1948} showed that points in convex position have  many perfect matchings (classically referred to as non-crossing configurations of chords on a circle), where  is the  {\em Catalan number}; . Thus, the number of plane perfect matchings of  points in the plane is at least . Garc{\'{\i}}a et al.~\cite{Garcia2000} presented a configuration of  points in the plane which has  many plane perfect matchings. See Table~\ref{table1}.

\subsubsection{Counting Edge-Disjoint Plane Graphs}

The number of edge-disjoint plane graphs of a given type in a point set  of  points is also of interest. Nash-Williams~\cite{Nash-Williams1961} and Tutte~\cite{Tutte1961} independently considered the number of (not necessarily plane) spanning trees. They obtained necessary and sufficient conditions for a graph to have  edge-disjoint spanning trees. Kundu~\cite{Kundu1974} showed that any -edge-connected graph contains at least  edge-disjoint spanning trees. 

As for the plane spanning trees a long-standing open question is to determine if the edges of  (where  is even) can be partitioned into  plane spanning trees. In other words, is it possible to pack  plane spanning trees into ? If  is in convex position, the answer in the affirmative follows from the result of Bernhart and Kanien~\cite{Bernhart1979}. In \cite{Bose2006}, the authors characterize the partitions of the complete convex graph into plane spanning trees. They also describe a sufficient condition, which generalizes the convex case, for points in general position. Aichholzer et al.~\cite{Aichholzer2014} showed that if the convex hull of  contains  vertices, then  contains at least  edge-disjoint plane spanning trees, and if  is in a ``regular wheel configuration'',  can be partitioned into  spanning trees. For  in general position they showed that  contains  edge-disjoint plane spanning trees. They obtained the following trade-off between the number of edge-disjoint plane spanning trees and the maximum vertex degree in each tree: For any ,  has  edge-disjoint plane spanning trees with maximum vertex degree  and diameter . They also showed the existence of at least 2 edge-disjoint plane Hamiltonian paths. 

\subsection{Our Results}
\label{our-results-section}

Given a set  of  points in the plane, with  even, we consider the problem of packing plane perfect matchings into . 
From now on, a {\em matching} will be a {\em perfect matching}. 

In Section~\ref{edge-disjoint-plane-section} we prove bounds on the number of plane matchings that can be packed into . 
In Section~\ref{convex-position-section} we show that if  is in convex position, then  plane matchings can be packed into ; this bound is tight. 

The points in wheel configurations are considered in Section~\ref{wheel-section}. We show that if  is in regular wheel configuration, then  edge-disjoint plane matchings can be packed into ; this bound is tight as well. In addition, for a fixed size set of points, we give a wheel configuration of the points which contains at most  edge-disjoint plane matchings. 

Point sets in general position are considered in Section~\ref{general-position-section}. We show how to find three edge-disjoint plane matchings in any set of at least 8 points. If  is a power of two, we prove that  contains at least  many edge-disjoint plane matchings. For the general case, where  is an even number, we prove that  contains at least  edge-disjoint plane matchings. 

In Section~\ref{non-crossing-matching-section} we count the number of pairwise non-crossing plane matchings. Two plane matchings  and  are called {\em non-crossing} if the edges of  and  do not cross each other. We show that  contains at least two and at most five non-crossing plane matchings; these bounds are tight. Table~\ref{table1} summarizes the results. 

In Section~\ref{persistency-section} we define the concept of {\em matching persistency} in a graph. A graph  is called {\em matching-persistent}, if by removing any perfect matching  from , the resulting graph, , still contains a perfect matching. We define the {\em plane matching persistency} of a point set , denoted by , to be the smallest number of edge-disjoint plane matchings such that, if we remove them from  the resulting graph does not have any plane perfect matching. In other words, , where  is the smallest set of edge-disjoint plane matchings such that  does not have any plane perfect matching. Here, the challenge is to find point sets with high plane matching persistency. We show that  for all point sets . We give a configuration of  with . 
Concluding remarks and open problems are presented in Section~\ref{conclusion}.

\begin{table}
\caption{Number of plane perfect matchings in a point set  of  points ( is even).}
\label{table1}
\centering
    \begin{tabular}{|l||@{\hskip 0.15in}c@{\hskip 0.15in}|@{\hskip 0.15in}c@{\hskip 0.15in}|@{\hskip 0.15in}c@{\hskip 0.15in}|@{\hskip 0.15in}c@{\hskip 0.15in}|}
         \hline
             Matching 	&  &&&  \\ \hline\hline
             total& 	\cite{Garcia2000, Motzkin1948}&\cite{Motzkin1948}&\cite{Garcia2000}& \cite{Sharir2006}\\\hline\hline
		 edge-disjoint& && &\\
             non-crossing edge-disjoint& 2 & 2&5&5 \\
         \hline
    \end{tabular}
\end{table}

\section{Preliminaries}
\label{preliminaries}

\subsection{Graph-Theoretical Background}
\label{graph-background-section}
Consider a graph  with vertex set  and edge set . If  is a complete graph on a vertex set  of size , then  is denoted by .
A {\em -factor} is a regular graph of degree . If  is the union of pairwise edge-disjoint -factors, their union is called a {\em-factorization} and  itself is {\em-factorable}~\cite{Harary1991}. A {\em matching} in a graph  is a set of edges that do not share vertices. A {\em perfect matching} is a matching which matches all the vertices of . Since a perfect matching is a regular graph of degree one, it is a -factor. It is well-known that for  even, the complete graph  is 1-factorable (See~\cite{Harary1991}).
Note that  has  edges and every -factor has  edges. Thus,  can be partitioned into at most  edge-disjoint perfect matchings.

On the other hand it is well-known that the edges of a complete graph , where  is even, can be colored by  colors such that any two disjoint edges have a different color. Each color is assigned to  edges, so that each color defines a -factor. The following geometric construction of a coloring, which uses a ``regular wheel configuration'', is provided in \cite{Soifer2009}. In a regular wheel configuration,  regularly spaced points are placed on a circle and one point is placed at the center of the circle. For each color class, include an edge  from the center to one of the boundary vertices, and all of the edges perpendicular to the line through , connecting pairs of boundary vertices.

The number of perfect matchings in a complete graph  (with  even), denoted by , is given by the double factorial;  \cite{Callan2009}, where . We give the following recurrence for the number of perfect matchings in . Consider an edge . The number of perfect matchings in  is the number of perfect matchings containing , plus the number of perfect matchings which do not contain . One can easily derive the following recurrence for : 


\subsection{Plane Matchings in Colored Point Sets}
\label{colored-matching-section}
Let  be a set of  colored points in general position in the plane with  even. A {\em colored matching} of , is a perfect matching such that every edge connects two points of distinct colors. A {\em plane colored matching} is a colored matching which is non-crossing.  
A special case of a plane colored matching, where  is partitioned into a set  of  red points and a set  of  blue points, is called {\em plane bichromatic matching}, also known as {\em red-blue matching} ({\em-matching}). In other words, an -matching of  is a non-crossing perfect matching such that every edge connects a red point to a blue point. It is well-known that if no three points of  are collinear, then  has an -matching~\cite{Putnam1979}. As shown in Figure~\ref{RB-fig}(a), some point sets have a unique -matching. Hershberger and Suri~\cite{Hershberger1992} construct an -matching in  time, which is optimal.

\begin{figure}[htb]
  \centering
\setlength{\tabcolsep}{0in}
  
  \caption{(a) A point set with a unique -matching, (b) Recursive ham sandwich cuts: first cut is in solid, second-level cuts are in dashed, and third-level cuts are in dotted lines.}
\label{RB-fig}
\end{figure}

We review some proofs for the existence of a plane perfect matching between  and :
\begin{itemize}
\item {\Min{R}{B}:} Consider a matching  between  and  which minimizes the total Euclidean length of the edges. The matching  is plane. To prove this, suppose that two line segments  and  in  intersect. By the triangle inequality, . This implies that by replacing  and  in  by  and , the total length of the matching is decreased; which is a contradiction.

\item {\Cut{R}{B}:} The {\em ham sandwich theorem} implies that there is a line , known as a {\em ham sandwich cut}, that splits both  and  exactly in half; if the size of  and  is odd, the line passes through one of each. Match the two points on  (if there are any) and recursively solve the problem on both sides of ; the recursion stops when each subset has one red point and one blue point. By matching these two points in all subsets, a plane perfect matching for  is obtained. See Figure~\ref{RB-fig}(b). A ham sandwich cut can be computed in  time \cite{Lo1994}, and hence the running time can be expressed as the recurrence . Therefore, an -matching can be computed in  time. 

\item {\Tangent{R}{B}:} If  and  are separated by a line, we can compute an -matching in the following way. W.l.o.g. assume that  and  are separated by a vertical line . Let \CH{R} and \CH{B} denote the convex hulls of  and . Compute the upper tangent  of \CH{R} and \CH{B} where  and . Match  and , and recursively solve the problem for  and ; the recursion stops when the two subsets are empty. In each iteration, all the remaining points are below the line passing through  and , thus, the line segments representing a matched pair in the successor iterations do not cross . Therefore, the resulting matching is plane.
\end{itemize}

Consider a set  of  points where  is even, and a partition  of  into  color classes. Sufficient and necessary conditions for the existence of a colored matching in  follows from the following theorem by Sitton~\cite{Sitton1996}:

\begin{theorem}[Sitton~\cite{Sitton1996}]
\label{Sitton}
Let  be a complete multipartite graph with  vertices, where . If , then  has a matching of size . 
\end{theorem}

Aichholzer et al.~\cite{Aichholzer2010} showed that if  is a geometric graph corresponding to a colored point set , then the minimum-weight colored matching of  is non-crossing. Specifically, they extend the proof of 2-colored point sets to multi-colored point sets:

\begin{theorem}[Aichholzer et al.~\cite{Aichholzer2010}]
\label{Aichholzer}
Let  be a set of colored points in general position in the plane with  even. Then 
admits a non-crossing perfect matching such that every edge connects two points of distinct colors if and only if at most half the points in  have the same color.
\end{theorem}


\section{Packing Plane Matchings into Point Sets}
\label{edge-disjoint-plane-section}
Let  be a set of  points in the plane with  even. In this section we prove lower bounds on the number of plane matchings that can be packed into . It is obvious that every point set has at least one plane matching, because a minimum weight perfect matching in , denoted by , is plane. A trivial lower bound of 2 (for ) is obtained from a minimum weight Hamiltonian cycle in , because this cycle is plane and consists of two edge-disjoint matchings. We consider points in convex position (Section~\ref{convex-position-section}), wheel configuration (Section~\ref{wheel-section}), and general position (Section~\ref{general-position-section}). 


\subsection{Points in Convex Position}
\label{convex-position-section}
In this section we consider points in convex position. We show that if  is in convex position,  plane matchings can be packed into ; this bound is tight.
\begin{lemma}
\label{two-convex-edges}
 If  is in convex position, where  is even and , then every plane matching in  contains at least two edges of \CH{P}.
\end{lemma}
\begin{proof}
Let  be a plane matching in . We prove this lemma by induction on the size of . If , then . None of the diagonals of  can be in , thus, the two edges in  belong to \CH{P}. If  then . If all edges of  are edges of \CH{P}, then the claim in the lemma holds. Assume that  contains a diagonal edge . Let  and  be the sets of points of  on each side of  (both including  and ). Let  and  be the edges of  in  and , respectively. It is obvious that  (resp. ) is in convex position and  (resp. ) is a plane matching in  (resp. ). By the induction hypothesis  (resp. ) contains two edges of \CH{P_1} (resp. \CH{P_2}). Since  and ,  contains at least two edges of \CH{P}.
\end{proof}


\begin{theorem}
\label{convex}
For any set  of  points in convex position in the plane, with  even, the maximum number of plane matchings that can be packed into  is .
\end{theorem}
\begin{proof}
By Lemma~\ref{two-convex-edges}, every plane matching in  contains at least two edges of \CH{P}. On the other hand, \CH{P} has  edges. Therefore, the number of plane matchings that can be packed into  is at most .

Now we show how to pack  plane matchings into .
Let , and w.l.o.g. assume that  is the radial ordering of the points in  with respect to a fixed point in the interior of \CH{P}. For each  in the radial ordering, where , let  (all indices are modulo ). Informally speaking,  is a plane perfect matching obtained from edge  and all edges parallel to ; see Figure~\ref{convex-fig}. Let . The matchings in  are plane and pairwise edge-disjoint. Thus,  is a set of  plane matchings that can be packed into .  
\end{proof}

\begin{figure}[htb]
  \centering
\setlength{\tabcolsep}{0in}
\includegraphics[width=.32\columnwidth]{fig/convex.pdf}
  \caption{Points in convex position.}
\label{convex-fig}
\end{figure}

\subsection{Points in Wheel Configurations}
\label{wheel-section}
A point set  of  points is said to be in ``regular wheel configuration'' in the plane, if  points of  are regularly spaced on a circle  and one point of  is at the center of .
We introduce a variation of the regular wheel configuration as follows. 
Let the point set  be partitioned into  and  such that  and  is an odd number. The points in  are regularly spaced on a circle . For any two distinct points  let  be the line passing through  and . Since  is regularly spaced on  and  is an odd number,  does not contain the center of . Let  and  be the two half planes defined by  such that  contains the center of . Let . The points in  are in the interior of ; see Figure~\ref{n-over-3-fig}(a). Note that for any two points , the line segment  does not intersect the interior of . The special case when  is the regular wheel configuration.

\begin{figure}[htb]
  \centering
\setlength{\tabcolsep}{0in}
  
  \caption{(a) A variation of the regular wheel configuration. (b) Illustration of Theorem~\ref{n-over-3-thr}. The points of  and the edges of  are in blue, and the points of  and the edges of  are in red.}
\label{n-over-3-fig}
\end{figure}

\begin{lemma}
\label{two-wheel-edges}
 Let  be a set of points in the plane where  is an even number and . Let  be a partition of the points in   such that  is an odd number and . If  is in the wheel configuration described above, then any plane matching in  contains at least two edges of \CH{P}.
\end{lemma}
\begin{proof}
 Consider a plane matching  of . It is obvious that ; we show that  contains at least two edges of . Note that , and both  and  are odd numbers. Observe that ; which implies that . Thus, , and hence there is at least one edge in  with both endpoints in . Let  be the longest such edge. Recall that . Let  be the set of points of  in  (including  and ), and let  be the set of points of  in  (excluding  and ). By definition,  does not contain any point of . Thus,  and  is in convex position with  (note that  is an odd number). Let  and  be the edges of  induced by the points in  and , respectively. Clearly,  is a partition of the edges of , and hence  (resp. ) is a plane perfect matching for  (resp. ). We show that each of  and  contains at least one edge of \CH{X}. First we consider . If , then  is the only edge in  and it is an edge of \CH{X}. Assume that . By Lemma~\ref{two-convex-edges},  contains at least two edges of \CH{A}. On the other hand each edge of \CH{A}, except for , is also an edge of \CH{X}; see Figure~\ref{n-over-3-fig}(b). This implies that  contains at least one edge of \CH{X}. Now we consider . Let , that is,  is a partition of the points in . Since , we have . Recall that . Thus, , and hence there is an edge  with both  and  in . Let  be the set of points of  in  (including  and ). By definition,  does not contain any point of . Thus,  and  is in convex position. On the other hand, by the choice of  as the longest edge,  cannot be a subset of  and hence . Let   be the edges of  induced by the points in . We show that  contains at least one edge of \CH{X}. If , then  is the only edge in  and it is an edge of \CH{X}. Assume that . By Lemma~\ref{two-convex-edges},  contains at least two edges of \CH{B}. On the other hand, each edge of \CH{B}, except for , is also an edge of \CH{X}; see Figure~\ref{n-over-3-fig}(b). This implies that  contains at least one edge of \CH{X}. This completes the proof. 
\end{proof}

\begin{figure}[htb]
  \centering
\setlength{\tabcolsep}{0in}
  
  \caption{Points in the regular configuration with (a)  and (b) ; one of the edges in \CH{P} cannot be matched.}
\label{wheel-fig}
\end{figure}

\begin{theorem}
\label{wheel}
For a set  of  points in the regular wheel configuration in the plane with  even, the maximum number of plane matchings that can be packed into  is .
\end{theorem}
\begin{proof}
In the regular wheel configuration,  is partitioned into a point set  of size  and a point set  of size 1. The points of  are placed regularly on a circle  and the (only) point of  is the center of . By Lemma~\ref{two-wheel-edges}, every plane matching in  contains at least two edges of \CH{P}. On the other hand, \CH{P} has  edges. Therefore, the number of plane matchings that can be packed into  is at most . Since  is an even number and the number of plane matchings is an integer, we can pack at most  plane matchings into .

Now we show how to pack  plane matchings into .
Let , and w.l.o.g. assume that  is the center of . Let , and let  be the radial ordering of the points in  with respect to . For each  in the radial ordering, where , let  and  (all indices are modulo ). Let ; informally speaking,  is a plane perfect matching obtained from edge  and edges parallel to . See Figure~\ref{wheel-fig}(a) for the case where  and Figure~\ref{wheel-fig}(b) for the case where . Let . The matchings in  are plane and pairwise edge-disjoint. Thus,  is a set of  plane matchings that can be packed into . 
\end{proof}

In the following theorem we use the wheel configuration to show that for any even integer , there exists a set  of  points in the plane, such that no more than  plane matchings can be packed into . 

\begin{theorem}
\label{n-over-3-thr}
For any even number , there exists a set  of  points in the plane such that no more than  plane matchings can be packed into .
\end{theorem}
\begin{proof}
The set  of  points is partitioned into  and , where  and . The points in  are regularly placed on a circle  and the points in  are in the interior . By Lemma~\ref{two-wheel-edges}, any plane matching in  contains at least two edges of \CH{P}.
Since , any plane matching of  contains at least two edges of \CH{X}. Thus, if  denotes any set of plane matchings which can be packed into , we have (note that  is odd)

\end{proof}

\subsection{Points in General Position}
\label{general-position-section}
In this section we consider the problem of packing plane matchings for point sets in general position (no three points on a line) in the plane. 
Let  be a set of  points in general position in the plane, with  even. Let  denote the maximum number of plane matchings that can be packed into . As mentioned earlier, a trivial lower bound of  (when ) is obtained from a minimum weight Hamiltonian cycle, which is plane and consists of two edge-disjoint perfect matchings. 

In this section we show that at least  plane matchings can be packed into . As a warm-up, we first show that if  is a power of two, then  plane matchings can be packed into . Then we extend this result to get a lower bound of  for every point set with an even number of points. We also show that if , then at least three plane matchings can be packed into , which improves the result for , , and . Note that, as a result of Theorem~\ref{n-over-3-thr}, there exists a set of  points such that no more than  plane matchings can be packed into . First consider the following observation.

\begin{observation}
\label{partition-obs}
 Let  be a partition of the point set , such that  is even and  for all  where . Let  be an index such that, . Then, .
\end{observation}

\begin{theorem}
\label{n-power2}
For a set  of  points in general position in the plane, where  is a power of 2, at least  plane matchings can be packed into .
\end{theorem}
\begin{proof}
We prove this theorem by induction. The statement of the theorem holds for the base case, where . Assume that . Recall that  denotes the maximum number of plane matchings that can be packed into . W.l.o.g. assume that a vertical line  partitions  into sets  and , each of size . By the induction hypothesis, . By Observation~\ref{partition-obs}, . That is, by pairing a matching  in  with a matching  in  we get a plane matching  in , such that each edge in  has both endpoints in  or in . If we consider the points in  as red and the points in  as blue, \Cut{R}{B} (see Section~\ref{colored-matching-section}) gives us a plane perfect matching  in , such that each edge in  has one endpoint in  and one endpoint in . That is . Therefore, we obtain one more plane matching in , which implies that .
\end{proof}

Let  and  be two point sets which are separated by a line. A {\em crossing tangent} between  and  is a line  touching \CH{R} and \CH{B} such that  and  lie on different sides of . Note that  contains a point , a point , and consequently the line segment ; we say that  is subtended from . It is obvious that there are two (intersecting) crossing tangents between  and ; see Figure~\ref{three-matching-fig}. 

\begin{figure}[htb]
  \centering
\setlength{\tabcolsep}{0in}
  
  \caption{(a) The crossing tangents intersect at a point :  and  are sorted clockwise around , (b) The crossing tangents intersect at a point :  is sorted clockwise around .  and  are shown by green and gray line segments.}
\label{three-matching-fig}
\end{figure}

\begin{theorem}
\label{3-matching-theorem}
For a set  of  points in general position in the plane with  even, at least three plane matchings can be packed into .
\end{theorem}
\begin{proof}
We describe how to extract three edge-disjoint plane matchings, , from . Let  be a vertical line which splits  into sets  and , each of size . Consider the points in  as red and the points in  as blue. We differentiate between two cases: (a)  and (b) , for some integer .

In case (a), both  and  have an even number of points. Let \M{1}{R} and \M{2}{R} (resp. \M{1}{B} and \M{2}{B}) be two edge-disjoint plane matchings in  (resp. ) obtained by a minimum length Hamiltonian cycle in  (resp. ). Let  and . Clearly  and  are edge-disjoint plane matchings for . Let . It is obvious that  is edge-disjoint from  and , which completes the proof in the first case.

In case (b), both  and  have an odd number of points and we cannot get a perfect matching in each of them. Let  and  be the two crossing tangents between  and , subtended from  and , respectively. We differentiate between two cases: (i)  and  intersect in the interior of  and , (ii)  and  intersect at an endpoint of both  and ; see Figure~\ref{three-matching-fig}.
\begin{itemize}
\item In case (i), let  be the intersection point; see Figure~\ref{three-matching-fig}(a). Let  and  be the points of  and , respectively, sorted clockwise around , where , . Consider the Hamiltonian cycle . Let  and  be the two edge-disjoint matchings obtained from . Note that  and  cannot be in the same matching, thus,  and  are plane. Let . As described in Section~\ref{colored-matching-section},  is a plane matching for . In order to prove that , we show that  and \textemdash which are the only edges in  that connect a point in  to a point in \textemdash do not belong to . Note that \Tangent{R}{B} iteratively selects an edge which has the same number of red and blue points below its supporting line, whereas the supporting lines of  and  have different numbers of red and blue points below them. Thus  and  are not considered by \Tangent{R}{B}. Therefore  is edge-disjoint from  and . 

\item In case (ii), w.l.o.g. assume that  and  intersect at the red endpoint of  and , i.e., ; See Figure~\ref{three-matching-fig}(b). Let  and . Note that both  and  have an even number of points and . Let \M{1}{R'} and \M{2}{R'} be two edge-disjoint plane matchings in  obtained by a minimum length Hamiltonian cycle in . Let  be the points of  sorted clockwise around , where , . Consider the Hamiltonian cycle . Let  and  be the two edge-disjoint plane matchings in  obtained from . Let  and . Clearly  and  are edge-disjoint plane matchings in . Let . As described in case (i),  is a plane matching in  and . Therefore,  is edge-disjoint from  and .
\end{itemize}
\end{proof}
As a direct consequence of Theorem~\ref{n-power2} and Theorem~\ref{3-matching-theorem} we have the following corollary.
\begin{corollary}
For a set  of  points in general position in the plane with , at least  plane matchings can be packed into .
\end{corollary}
\begin{proof}
 Partition  by vertical lines, into  point sets, each of size . By Theorem~\ref{3-matching-theorem}, at least three plane matchings can be packed into each set. Considering these sets as the base cases in Theorem~\ref{n-power2}, we obtain  plane matchings between these sets. Thus, in total,  plane matchings can be packed into 
\end{proof}


\begin{theorem}
For a set  of  points in general position in the plane, with  even, at least  plane matchings can be packed into .
\end{theorem}
\begin{proof}
We describe how to pack a set  of  plane perfect matchings into . The construction consists of the following three main steps which we will describe in detail.

\begin{enumerate}
  \item Building a binary tree .
  \item Assigning the points of  to the leaves of .
  \item Extracting  from  using internal nodes of .
\end{enumerate}

\begin{paragraph}{\em \small 1. Building the tree T.}
In this step we build a binary tree  such that each node of  stores an even number, and each internal node of  has a left and a right child. For an internal node , let \LC{u} and \RC{u} denote the left child and the right child of , respectively. Given an even number , we build  in the following way:
\begin{itemize}
  \item The root of  stores .
  \item If a node of  stores , then that node is a leaf.
  \item For a node  storing , with  even and , we store the following even numbers into \LC{u} and \RC{u}:
    \begin{itemize}
	\item If  is divisible by , we store  in both \LC{u} and \RC{u}; see Figure~\ref{tree-construction-fig}(a).
	\item If  is not divisible by  and  is the root or the left child of its parent then we store  in \LC{u} and  in \RC{u}; see Figure~\ref{tree-construction-fig}(b).
	\item If  is not divisible by  and  is the right child of its parent then we store  in \LC{u} and  in \RC{u}; see Figure~\ref{tree-construction-fig}(c).
    \end{itemize}
Note that in the last two cases\textemdash where  is not divisible by four\textemdash the absolute difference between the values stored in \LC{u} and \RC{u} is exactly 2. See Figure~\ref{matching-example-fig}.
\end{itemize}
\end{paragraph}

\begin{figure}[htb]
  \centering
\setlength{\tabcolsep}{0in}
  
  \caption{(a)  is divisible by four, (b)  is not divisible by four and  is a left child, and (c)  is not divisible by four and  is a right child.}
\label{tree-construction-fig}
\end{figure}

\begin{paragraph}{\em \small 2. Assigning the points to the leaves of the tree.}
In this step we describe how to assign the points of , in pairs, to the leaves of . We may assume without loss of generality that no two points of  have the same -coordinate. Sort the points of  in a increasing order of their -coordinate.
Assign the first two points to the leftmost leaf, the next two points to the second leftmost leaf, and so on. Note that  has  leaves, and hence all the points of  are assigned to the leaves of . See Figure~\ref{matching-example-fig}.  
\end{paragraph}

\begin{paragraph}{\em \small 3. Extracting the matchings.}
Let  be the number of edges in a shortest path from the root to any leaf in ; in Figure~\ref{matching-example-fig}, . For an internal node , let  be the subtree rooted at . Let  and  be the set of points assigned to the left and right subtrees of , respectively, and let . Consider the points in  as red and the points in  as blue. Since the points in  have smaller -coordinates than the points in , we say that  and  are separated by a vertical line . For each internal node  where  is in level  in \textemdash assuming the root is in level \textemdash we construct a plane perfect matching  in  in the following way. Let  be the even number stored at .
\begin{itemize}
  \item If  is divisible by  (Figure~\ref{tree-construction-fig}(a)), then let ; see Section~\ref{colored-matching-section}. Since ,  is a plane perfect matching for . See vertices  in Figure~\ref{matching-example-fig}.

  \item If  is not divisible by  and  is the root or a left child (Figure~\ref{tree-construction-fig}(b)), then . Let  be the two points assigned to the rightmost leaf in , and let . Since ,  is a perfect matching in . In addition,  and  are the two rightmost points in , thus,  does not intersect any edge in , and hence  is plane. See vertices  in Figure~\ref{matching-example-fig}.

  \item If  is not divisible by  and  is a right child (Figure~\ref{tree-construction-fig}(c)), then . Let  be the two points assigned to the leftmost leaf in  and let . Since ,  is a perfect matching in . In addition,  and  are the two leftmost points in , thus,  does not intersect any edge in , and hence  is plane. See vertices  in Figure~\ref{matching-example-fig}.
\end{itemize}

For each , where , let  be the set of vertices of  in level ; see Figure~\ref{matching-example-fig}. For each level  let . Let . 
In the rest of the proof, we show that  contains  edge-disjoint plane matchings in .
\end{paragraph}

\begin{figure}[htb]
  \centering
  \includegraphics[width=.9\columnwidth]{fig/TP2.pdf}
 \caption{The points in  are assigned, in pairs, to the leaves of , from left to right. The point set  with the edge-disjoint plane matchings is shown as well.  contains the bold red edge and the red edges crossing .  contains the bold green edge and the green edges crossing .  contains the bold blue edges and the blue edges crossing .}
  \label{matching-example-fig}
\end{figure}

\vspace{10pt}
{\em {\bf Claim 1:} For each , where ,  is a plane perfect matching in .} 
Note that if  is the root of the tree, then . In addition, for each internal node  (including the root),  is a partition of the point set . This implies that in each level  of the tree, where , we have . Moreover, the points in  are assigned to the leaves of  in non-decreasing order of their -coordinate. Thus,  is a partition of the point set ; the sets  with  are separated by vertical lines; see Figure~\ref{matching-example-fig}. Therefore,  is a perfect plane matching in ; which proves the claim.
\vspace{10pt}

\vspace{10pt}
{\em {\bf Claim 2:} For all , where  and , .}
In order to prove that  and  are edge-disjoint, we show that for each pair of distinct internal nodes  and , . If  and  are in the same level, then  and  are separated by , thus,  and  do not share any edge. Thus, assume that  and  such that , , and w.l.o.g. assume that . If , then  and  are separated by line , where  is the lowest common ancestor of  and ; this implies that  and  do not share any edge. Therefore, assume that , and w.l.o.g. assume that  is in the left subtree of . Thus, \textemdash and consequently \textemdash is to the left of . The case where  is in the right subtree of  is symmetric. Let  be the number stored at . We differentiate between three cases:

\begin{itemize}
  \item If  is divisible by , then all the edges in  cross , while the edges in  are to the left of . This implies that  and  are disjoint. 

  \item If  is not divisible by  and  is the root or a left child, then all the edges of  cross , except the rightmost edge  which is to the right of . Since  is to the left of , it follows that  and  are disjoint. 

  \item If  is not divisible by  and  is a right child, then all the edges of  cross , except the leftmost edge . If , then , and hence  and  are disjoint. If  then  is the left child of its parent and all the edges in  cross  (possibly except one edge which is to the right of ), while  is to the left of . Therefore  and  do not share any edge. This completes the proof of the claim.
\end{itemize}
\vspace{10pt}

\vspace{10pt}
{\em {\bf Claim 3:} For every two nodes  and  in the same level of , storing  and , respectively, .}

We prove the claim inductively for each level  of . For the base case, where : (a) if  is divisible by four, then both  and  store  and the claim holds, (b) if  is not divisible by four then  stores  and  stores ; as , the claim holds for . 
Now we show that if the claim is true for the th level of , then the claim is true for the th level of .
Let  and , storing  and , respectively, be in the th level of . By the induction hypothesis, the claim holds for the th level, i.e., . We prove that the claim holds for the th level of , i.e., for the children of  and . Since  and  are even numbers, . If , then , and by a similar argument as in the base case, the claim holds for the children of  and . If , then w.l.o.g. assume that . Let  be the smallest number and  be the largest number stored at the children of  and  (which are at the th level). We show that . It is obvious that  and . Thus,

Now, we differentiate between two cases, where  or . If , then by Equation~\ref{alpha-beta},

If , then by Equation~\ref{alpha-beta},

which completes the proof of the claim.
\vspace{10pt}

\vspace{10pt}
{\em {\bf Claim 4:} .}

It follows from Claim 3 that all the leaves of  are in the last two levels. Since  has  leaves,  has  nodes. Recall that  is the number of edges in a shortest path from the root to any leaf in . Thus, , where  is the height of . To give a lower bound on , one may assume that the last level of  is also full, thus,


and hence, . Therefore, .  Since  is an integer, ; which proves the claim.

\vspace{20pt}

Claim 1 and Claim 2 imply that  contains  edge-disjoint plane perfect matchings. Claim 4 implies that , which proves the statement of the theorem.
\end{proof}


\subsection{Non-crossing Plane Matchings}
\label{non-crossing-matching-section}
In this section we consider the problem of packing plane matchings into  such that any two different matchings in the packing are non-crossing.
Two edge-disjoint plane matchings  and  are {\em non-crossing}, if no edge in  crosses any edge in , and vice versa. For a set  of  points in general position in the plane, with  even, we give tight lower and upper bounds on the number of pairwise non-crossing plane perfect matchings that can be packed into . 

\begin{lemma}
\label{5-non-crossing}
For a set  of  points in general position in the plane, with  even, at most five pairwise non-crossing plane matchings can be packed into .
\end{lemma}
\begin{proof}
 Let  be any maximal set of non-crossing edge-disjoint plane matchings in . Let , and let  be the induced subgraph of  by . It is obvious that  is an -regular graph. Since  are plane and pairwise non-crossing,  is an -regular plane graph. It is well known that every plane graph has a vertex of degree at most 5. Thus,  has a vertex of degree at most five and hence ; which implies at most five pairwise non-crossing plane matchings can be packed into .
\end{proof}

\begin{figure}[htb]
  \centering
  \includegraphics[width=.4\columnwidth]{fig/5-regular.pdf}
 \caption{A point set with five non-crossing edge-disjoint plane perfect matchings.}
  \label{5-regular-fig}
\end{figure}

Figure~\ref{5-regular-fig} shows a -regular geometric graph on a set of 12 points in the plane which contains five non-crossing edge-disjoint plane matchings. In \cite{Hasheminezhad2011}, the authors showed how to generate an infinite family of 5-regular planar graphs using the graph in Figure~\ref{5-regular-fig}. By an extension of the five matchings shown in Figure~\ref{5-regular-fig}, five non-crossing matchings for this family of graphs is obtained. Thus, the bound provided by Lemma~\ref{5-non-crossing} is tight.  

It is obvious that if  contains at least four points, the minimum length Hamiltonian cycle in  contains two non-crossing edge-disjoint plane matchings. In the following lemma we show that there exist point sets which contain at most two non-crossing edge-disjoint plane matchings.


\begin{observation}
\label{even-cycle-obs}
The union of two edge-disjoint perfect matchings in any graph is a set of even cycles.
\end{observation}

\begin{lemma}
 \label{2-non-crossing}
For a set  of  points in convex position in the plane, with  even, at most two pairwise non-crossing plane matchings can be packed into .
\end{lemma}
\begin{proof}
 The proof is by contradiction. Consider three pairwise non-crossing plane matchings . Let  be the union of  and . By Observation~\ref{even-cycle-obs},  is a union of cycles, say , each with an even number of points. Observe that , where , is plane and convex, and has at least 4 vertices. 
In addition, the regions enclosed by  are pairwise disjoint.


If , then . By Lemma~\ref{two-convex-edges},  contains two edges of , which contradicts that , , and  are pairwise disjoint. 

From now on, we assume that . Then, each  contains an edge that is not an edge of \CH{P}.


\vspace{10pt}
{\em {\bf Claim 1:} There exists a cycle  such that at most one edge of  is not an edge of \CH{P}.}

Let  be an edge of  such that  is not an edge of \CH{P} and  is minimum, where  and  are the two half planes defined by a line through  and . The cycle containing  satisfies the statement in the claim.
\vspace{10pt}

Let  be a cycle in  which satisfies the statement of Claim 1. Since ,  and  are pairwise non-crossing, none of the edges in  intersect . Thus, in  we have one of the following two cases:
\begin{itemize}
 \item Both  and  are matched to the points of . Let  be the points of . Note that  is an even number and . In addition . Let  be the edges in  which are induced by the points in . By Lemma~\ref{two-convex-edges},  contains two edges of \CH{P'}. Since , both of these edges belong to . This contradicts the fact that , ,  are pairwise edge-disjoint.
 \item None of  and  are matched to the points of . Let  be the points of  except  and . Note that  is an even number. Let  be the edges in  which are induced by the points in . If , then the only edge in  is an edge of . If , then by Lemma~\ref{two-convex-edges},  contains two edges of \CH{P'}. At least one of these edges belong to . Either case contradicts the fact that , ,  are pairwise edge-disjoint.
\end{itemize}
\end{proof}
We conclude this section with the following theorem.
\begin{theorem}
For a set  of  points in general position in the plane, with  even, at least two and at most five pairwise non-crossing plane matchings can be packed into . These bounds are tight.
\end{theorem}

\section{Matching Removal Persistency}
\label{persistency-section}
In this section we define the matching persistency of a graph. A graph  is {\em matching persistent} if by removing any perfect matching  from , the resulting graph, , has a perfect matching. We define the {\em matching persistency} of , denoted by , as the size of the smallest set  of edge-disjoint perfect matchings that can be removed from  such that  does not have any perfect matching. 
In other words, if , then

\begin{enumerate}
 \item by removing an arbitrary set of  edge-disjoint perfect matchings from , the resulting graph still contains a perfect matching, and 
\item there exists a set of  edge-disjoint perfect matchings such that by removing these matchings from , the resulting graph does not have any perfect matching.
\end{enumerate}

In particular,  is matching persistent iff .

\begin{figure}[htb]
  \centering
\setlength{\tabcolsep}{0in}
  
  \caption{(a) By removing any matching (red, green, or blue) from , at most two paths between  and  disappear. (b) The edges of  are partitioned to  perfect matchings, thus,  does not have any perfect matching.}
\label{path-matching}
\end{figure}

\begin{lemma}
\label{regularity-connectivity}
Let  be a complete graph with  vertices, where  is even, and let  be a set of  edge-disjoint perfect matchings in . Then,  is an -regular graph which is -connected.
\end{lemma}
\begin{proof}
The regularity is trivial, because  is -regular and every vertex has degree  in , thus,  is an -regular graph. Now we prove the connectivity. Consider two vertices  and  in . There are  many edge-disjoint paths between  and  in :  many paths of length two of the form , where  and a path  of length one; see Figure~\ref{path-matching}(a). By removing any matching in  from , at most two paths disappear, because  and  have degree one in each matching. Thus in , there are  many edge-disjoint paths between  and , which implies that  is -connected.
\end{proof}

\begin{theorem}
\label{mp-thr}
 .
\end{theorem}
\begin{proof}
In order to prove the theorem, we show that by removing any set  of  edge-disjoint perfect matchings from , where , the resulting graph still has a perfect matching. By Lemma~\ref{regularity-connectivity}, the graph  is an -regular graph which is -connected. Since ,  is a connected graph and the degree of each vertex is at least . Thus, by a result of 
Dirac~\cite{Dirac1952},  has a Hamiltonian cycle and consequently a perfect matching. Therefore, by removing  arbitrary perfect matchings from , where , the resulting graph still has a perfect matching, which proves the claim.
\end{proof}

\begin{lemma}
\label{bipartite-matchings-lemma}
 If , then .
\end{lemma}
\begin{proof}
Let  be a complete bipartite subgraph of . Note that  is an odd number and  is an -regular graph.
According to Hall's marriage theorem \cite{Hall1935}, for , every -regular bipartite graph contains a perfect matching \cite{Harary1991}. Since by the iterative removal of perfect matchings  from  the resulting graph is still regular, the edges of  can be partitioned into  perfect matchings; see Figure~\ref{path-matching}(a). It is obvious that  consists of two connected components of odd size. Thus, by removing the  matchings in , the resulting graph, , does not have any perfect matching. This proves the claim.
\end{proof}

By Theorem~\ref{mp-thr} and Lemma~\ref{bipartite-matchings-lemma} we have the following corollary.

\begin{corollary}
If , then . 
\end{corollary}

In the rest of this section we consider plane matching removal from geometric graphs.

Let  be a set of  points in general position in the plane, with  even. Given a geometric graph  on , we say that  is {\em plane matching persistent} if by removing any plane perfect matching  from , the resulting graph, , has a plane perfect matching. We define the {\em plane matching persistency} of , denoted by  as the size of the smallest set  of edge-disjoint plane perfect matchings that can be removed from  such that  does not have any plane perfect matching. In particular,  is plane matching persistent iff .

Aichholzer et al.~\cite{Aichholzer2010} and Perles (see~\cite{Keller2012}) showed that by removing any set of at most  edges from , the resulting graph has a plane perfect matching. This bound is tight~\cite{Aichholzer2010}; that is, there exists a point set  such that by removing a set  of  edges from  the resulting graph does not have any plane perfect matching. In the examples provided by~\cite{Aichholzer2010}, the  edges in  form a connected component which has  vertices. 

Thus, one may think if the removed edges are disjoint, it may be possible to remove more than  edges while the resulting graph has a plane perfect matching.
In the following lemma we show that by removing any plane perfect matching, i.e., a set of  disjoint edges, from , the resulting graph still has a perfect matching.

\begin{lemma}
\label{pmp2-lemma}
Let  be a set of  points in general position in the plane with  even, then .
\end{lemma}
\begin{proof}
 Let  be any plane perfect matching in . Assign  distinct colors to the points in  such that both endpoints of every edge in  have the same color. By Theorem~\ref{Aichholzer},  has a plane colored matching, say . Since both endpoints of every edge in  have the same color while the endpoints of every edge in  have distinct colors,  and  are edge-disjoint. Therefore, by removing any plane perfect matching from , the resulting graph still has a plane perfect matching, which implies that .
\end{proof}

\begin{theorem}
For a set  of  points in convex position in the plane with  even, .
\end{theorem}
\begin{proof}
By Lemma~\ref{pmp2-lemma}, . In order to prove the theorem, we need to show that . Let  and  be two edge-disjoint plane matchings obtained from \CH{P}. By Lemma~\ref{two-convex-edges}, any plane perfect matching in  contains at least two edges of \CH{P}, while  does not have convex hull edges, and hence does not have any plane perfect matching. Therefore, . 
\end{proof}


\begin{lemma}
There exists a point set  in general position such that .
\end{lemma}
\begin{proof}
We prove this lemma by providing an example. Figure~\ref{pmp3-fig}(a) shows a set  of  points in general position, where  is an even number. In order to prove that , we show that by removing any two edge-disjoint plane matchings from , the resulting graph still has a plane perfect matching. Let  and  be any two plane perfect matchings in . Let  be the subgraph of  induced by the edges in . Note that  is a 2-regular graph and by Observation~\ref{even-cycle-obs} does not contain any odd cycle. For each , let  be the triangle which is defined by the three points , , and . Let  be the set of these  (nested) triangles. Since  does not have any odd cycle, for each , at least one edge of  is not in . Let  be the matching containing an edge  from each  such that . See Figure~\ref{pmp3-fig}(b). Now we describe how to complete , i.e., complete it to a perfect matching. Partition the triangles in  into  pairs of consecutive triangles. For each pair  of consecutive triangles we complete  locally\textemdash on \textemdash in the following way. Let  and . See Figure~\ref{pmp3-fig}(c). W.l.o.g. assume that  contains  and , that is  and . If , then we complete  by adding . If , then  or  because  has degree two in . W.l.o.g. assume that . Then we modify  by removing  and adding . Now, if , then we complete  by adding . If , then by Observation~\ref{even-cycle-obs}, . We modify  by removing  and adding . At this point, since  and  are in ,  and we complete  by adding .
\end{proof}
\begin{figure}[htb]
  \centering
\setlength{\tabcolsep}{0in}
  
\caption{(a) Set  of  points in general position. (b)  contains one edge from each triangle. (c) Locally converting  to a perfect matching, for  and .}
\label{pmp3-fig}
\end{figure}

\section{Conclusion}
\label{conclusion}
In this paper, we considered the problem of packing edge-disjoint plane perfect matchings in a complete geometric graph  on a set  of  points in general position in the plane. We proved that
\begin{itemize}
  \item at least  plane matchings can be packed into ,
  \item at least two and at most five non-crossing plane matchings can be packed into ,
  \item the plane matching persistency of  is at least two.
\end{itemize}
In addition, for some special configurations of  we showed that
\begin{itemize}
  \item if  is in convex position, then  plane matchings can be packed into ,
  \item if  is in a regular wheel configuration, then  plane matchings can be packed into ,
  \item there exists a set  such that no more than  plane matchings can be packed into ,
  \item if  is in convex position, the plane matching persistency of  is two,
  \item there exists a set  such that the plane matching persistency of  is at least three.
\end{itemize}

We leave a number of open problems:
\begin{itemize}
  \item We believe that the number of plane matchings that can be packed into  is linear in . Thus, improving the lower bound of  is the main open problem.
  \item Is there an upper bound better than , on the number of plane matchings that can be packed into , where ?
  \item Providing point sets with large plane matching persistency.
\end{itemize}
 
\bibliographystyle{abbrv}
\bibliography{Matching-Packing.bib}
\end{document}
