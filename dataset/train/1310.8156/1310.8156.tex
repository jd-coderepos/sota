\documentclass{LMCS}

\def\dOi{9(4:24)2013}
\lmcsheading {\dOi}
{1--25}
{}
{}
{Jan.~29, 2013}
{Dec.~18, 2013}
{}

\subjclass{F.4.1.~Mathematical Logic, F.4.2.~Grammars and Other
  Rewriting Systems, F.1.1.~Models of Computation}

\ACMCCS{[{\bf Theory of computation}]: Logic---Proof theory; Formal
  languages and automata theory---Tree languages}

\usepackage{proof}
\usepackage{stmaryrd}
\usepackage{amsmath}
\usepackage{amssymb}
\usepackage{amsthm}
\usepackage{latexsym}
\usepackage{mathdots}
\usepackage{mathabx}
\usepackage{helvet}
\usepackage{calrsfs}
\usepackage{color}
\usepackage{fixltx2e}
\usepackage{url}




\usepackage[pdftex]{virginialake}
\vlnosmallleftlabels
\vlnostructuressyntax
\vlstemheight=12pt
\newdimen\myvltreesize
\myvltreesize=6ex
\renewcommand{\vltrauxx}[5]{\vldaux
   {#1}
   {#2}
   {\vlhy{#3}}
   {}
   {\hbox{}}
   {\kern\deropen}}

\newcommand{\vlstrf}[4]{\vltrf{\hbox{\small}}{#2}{\vlshy{}}{#3}{\vlshy{}}{#4}}
\newcommand{\vlbtrf}[4]{\vlstrf{#1}{#2}{\vlshy{\hskip#3}}{#4}}
\newcommand{\vlhtr}[2]{\vlbtrf{#1}{#2}{2em}{1}}
\newcommand{\vlshy}[1]{\vlhyaux{}}

\newcommand{\vlfa}[1]{\forall #1\,}
\newcommand{\vlex}[1]{\exists #1\,}



\let\theorem=\thm
\let\endtheorem=\endthm

\theoremstyle{plain}
\newtheorem{proposition}[thm]{Proposition}
\newtheorem{lemma}[thm]{Lemma}
\newtheorem{corollary}[thm]{Corollary}
\theoremstyle{definition}
\newtheorem{definition}[thm]{Definition}
\newtheorem{example}[thm]{Example}
\newtheorem{observation}[thm]{Observation}
\newtheorem{notation}[thm]{Notation}
\newtheorem{remark}[thm]{Remark}

\let\amsproof=\proof
\let\endamsproof=\endproof
 



\def\tA{\tilde{A}}
\def\tB{\tilde{B}}
\def\tC{\tilde{C}}
\def\tP{\tilde{P}}
\def\tQ{\tilde{Q}}
\def\tR{\tilde{R}}
\def\tS{\tilde{S}}
\def\tT{\tilde{T}}
\def\tU{\tilde{U}}
\def\tW{\tilde{W}}
\def\tDelta{\tilde{\Delta}}
\def\tGamma{\tilde{\Gamma}}
\def\tPi{\tilde{\Pi}}
\def\tpi{\tilde{\pi}}
\def\tsigma{\tilde{\sigma}}
\def\ts{\tilde{s}}

\def\ha{\hat{a}}

\def\hA{\hat{A}}
\def\hB{\hat{B}}
\def\hP{\hat{P}}
\def\hQ{\hat{Q}}
\def\hR{\hat{R}}
\def\hS{\hat{S}}
\def\hT{\hat{T}}
\def\hDelta{\hat{\Delta}}
\def\hGamma{\hat{\Gamma}}
\def\hrho{\hat{\rho}}

\def\cA{{\mathcal A}}
\def\cB{{\mathcal B}}
\def\Cc{{\mathcal C}}
\def\cD{{\mathcal D}}
\def\cE{{\mathcal E}}
\def\cF{{\mathcal F}}
\def\cG{{\mathcal G}}
\def\cH{{\mathcal H}}
\def\cI{{\mathcal I}}
\def\cJ{{\mathcal J}}
\def\cK{{\mathcal K}}
\def\cL{{\mathcal L}}
\def\cM{{\mathcal M}}
\def\cN{{\mathcal N}}
\def\cO{{\mathcal O}}
\def\cP{{\mathcal P}}
\def\cQ{{\mathcal Q}}
\def\cR{{\mathcal R}}
\def\cS{{\mathcal S}}
\def\cT{{\mathcal T}}
\def\cU{{\mathcal U}}
\def\cV{{\mathcal V}}
\def\cW{{\mathcal W}}
\def\cX{{\mathcal X}}
\def\cY{{\mathcal Y}}
\def\cZ{{\mathcal Z}}

\def\Bool{\mathbb{B}}
\def\Nat{\mathbb{N}}



\def\SKS{\mathsf{SKS}}
\def\SKSd{\mathsf{SKS}\mathord{\downarrow}}
\def\SKSu{\mathsf{SKS}\mathord{\uparrow}}
\def\KS{\mathsf{KS}}
\def\SKSg{\mathsf{SKSg}}
\def\SKSgd{\mathsf{SKSg}\mathord{\downarrow}}
\def\SKSgu{\mathsf{SKSg}\mathord{\uparrow}}
\def\KSg{\mathsf{KSg}}
\def\K{\mathsf{K}}
\def\MLL{\mathsf{MLL}}
\def\MALL{\mathsf{MALL}}




\def\set#1{\{#1\}}
\def\bigset#1{\big\{#1\big\}}
\def\dset#1{\left\{#1\right\}}
\def\mset#1{\{#1\}}
\def\cons#1{\{#1\}}
\def\conhole      {\cons{\enspace}}

\def\disunion{\mathbin{\uplus}}
\def\directsum{\mathbin{\oplus}}

\def\fcomp{\mathbin{\circ}}

\def\scope{\mathrel{.}}

\def\tuple#1{\langle#1\rangle}



\def\grammareq {\mathrel{\raise.4pt\hbox{::}{=}}}



\def\id{{\mathrm{1}}}
\def\idh{{\hat{\id}}}
\def\idch{{\check{\id}}}


\def\codiag{\nabla}
\def\diag{\Delta}
\def\proj{\Pi}
\def\coproj{\amalg}

\newcommand{\sto}[1][]{\stackrel{#1}{\to}}
\newcommand{\lsto}[1][]{\stackrel{#1}{\longrightarrow}}
\newcommand{\lsfrom}[1][]{\stackrel{#1}{\longleftarrow}}

\newcommand{\dotto}[1][]{\mathrel{\!\xy\ar@{.>}^-{#1}(5,0)\endxy\!}}
\newcommand{\solto}[1][]{\mathrel{\!\xy\ar@{->}^-{#1}(5,0)\endxy\!}}
\newcommand{\longsolto}[1][]{\mathrel{\!\xy\ar@{->}^-{#1}(11,0)\endxy\!}}
\newcommand{\longdotto}[1][]{\mathrel{\!\xy\ar@{.>}^-{#1}(11,0)\endxy\!}}
\newcommand{\xldotto}[2][]{\mathrel{\!\xy\ar@{.>}^-{#1}(#2,0)\endxy\!}}

\newcommand{\downmap}[1]{\begin{array}{c}#1\end{array}}
\newcommand{\vcdownmap}[1]{\vcenter{\begin{array}{c}#1\end{array}}}
\newcommand{\ldownto}[1][]{
  \\
  \llap{}\clapm{\Big\downarrow}
  \\}
\newcommand{\rdownto}[1][]{
  \\
  \clapm{\Big\downarrow}\rlap{}
  \\}

\def\biproj{p}
\def\bicoproj{i}
\def\bidiag{\blacktriangle}
\def\bicodiag{\blacktriangledown}




\def\Hom{}
\renewcommand{\Hom}[1][]{\mathrm{Hom}_{#1}}
\def\Obj{{\mathrm{Obj}}}
\def\isom{\cong}

\def\name#1{\hat{#1}}
\def\coname#1{\check{#1}}
\def\widename#1{\widehat{#1}}
\def\wideconame#1{\widecheck{#1}}

\def\Set{{\mathbf{Set}}}



\def\sqn  #1{{\;\turnstile #1\;}}\def\ssqn#1#2{{\;#1\turnstile #2\;}}

\def\cutr{\mathsf{cut}}
\def\idr{\mathsf{id}}
\def\axr{\mathsf{ax}}
\def\mixr{\mathsf{mix}}
\def\mixzr{\mathsf{mix_0}}
\def\weakr{\mathsf{weak}}
\def\conr{\mathsf{cont}}
\def\exr{\mathsf{exch}}
\def\cyclr{\mathsf{cycl}}
\def\rr{\mathsf{r}}
\def\swir{\mathsf{s}}
\def\medr{\mathsf{m}}

\def\MLL{\mathsf{MLL}}
\def\LK{\mathsf{LK}}
\def\Gthree{\mathsf{G3}}
\def\LC{\mathsf{LC}}
\def\LLP{\mathsf{LLP}}



\newcommand{\ird}{\mathsf{i}\mathord{\downarrow}}
\newcommand{\iru}{\mathsf{i}\mathord{\uparrow}}
\newcommand{\atird}{\mathsf{ai}\mathord{\downarrow}}
\newcommand{\atiru}{\mathsf{ai}\mathord{\uparrow}}
\newcommand{\seqrd}{\mathsf{q}\mathord{\downarrow}}
\newcommand{\seqru}{\mathsf{q}\mathord{\uparrow}}
\newcommand{\conrd}{\mathsf{c}\mathord{\downarrow}}
\newcommand{\conru}{\mathsf{c}\mathord{\uparrow}}
\newcommand{\weakrd}{\mathsf{w}\mathord{\downarrow}}
\newcommand{\weakru}{\mathsf{w}\mathord{\uparrow}}
\newcommand{\assrd}[1][]{\alpha\mathord{\downarrow}_{#1}}
\newcommand{\assru}[1][]{\alpha\mathord{\uparrow}_{#1}}
\newcommand{\sassrd}[1][]{\alpha^\triangleleft\mathord{\downarrow}_{#1}}
\newcommand{\sassru}[1][]{\alpha^\triangleleft\mathord{\uparrow}_{#1}}
\newcommand{\comrd}[1][]{\sigma\mathord{\downarrow}_{#1}}
\newcommand{\comru}[1][]{\sigma\mathord{\uparrow}_{#1}}
\newcommand{\assoc}[1][]{\alpha_{#1}}
\newcommand{\twist}[1][]{\sigma_{#1}}

\def\ltensl{\mathord{\ltens}\mathsf{L}}
\def\ltensr{\mathord{\ltens}\mathsf{R}}
\def\lparl{\mathord{\lpar}\mathsf{L}}
\def\lparr{\mathord{\lpar}\mathsf{R}}
\def\limpl{\mathord{\limp}\mathsf{L}}
\def\limpr{\mathord{\limp}\mathsf{R}}
\def\lnegl{\mathord{\cdot^\lbot}\mathsf{L}}
\def\lnegr{\mathord{\cdot^\lbot}\mathsf{R}}
\def\exrl{\mathsf{exchL}}
\def\exrr{\mathsf{exchR}}
\def\markl{\mathord{\mathsf{:}}\mathsf{L}}
\def\markr{\mathord{\mathsf{:}}\mathsf{R}}

\def\poll{\bullet}
\def\polr{\circ}
\def\markpl{^\poll}
\def\markpr{^\polr}
\def\lnegpl{^{\lbot\poll}}
\def\lnegpr{^{\lbot\polr}}


\def\wlpar{\;\lpar\;}
\def\wltens{\;\ltens\;}

\def\un{\circ}




\def\wvskipvi{\mathchoice{\hskip-2ex}{\hskip-1.5ex}{\hskip-1ex}{\hskip-.8ex}}
\def\wvskipx{\mathchoice{\hskip-2.5ex}{\hskip-2ex}{\hskip-1.5ex}{\hskip-1ex}}
\def\wvskipxii{\mathchoice{\hskip-2.9ex}{\hskip-2.25ex}{\hskip-2ex}{\hskip-1.5ex}}
\let\wvskip\wvskipx
\def\dbigand#1{\displaystyle\mathop{\bigwedge\wvskip\bigwedge}_{#1}}
\def\dbigor#1{\displaystyle\mathop{\bigvee\wvskip\bigvee}_{#1}}
\def\bigand#1{\mathop{\bigwedge\wvskip\bigwedge}_{\smash{#1}}}
\def\bigor#1{\mathop{\bigvee\wvskip\bigvee}_{\smash{#1}}}
\def\ctrue{{\mathbf t}}
\def\cfalse{{\mathbf f}}
\def\cand{\wedge}
\def\cor{\vee}
\def\candor{\mathchoice {\mathbin{\rlap{}\wedge}}
  {\mathbin{\rlap{}\wedge}}
  {\mathbin{\rlap{}\wedge}}
  {\mathbin{\rlap{}\wedge}}
}
\def\cplus{\oplus}
\def\ccon{\blacktriangledown}
\def\cneg#1{\bar{#1}}
\def\widecneg#1{\overline{#1}}
\def\wcneg#1{\overline{#1}}
\def\cimp{\mathop{\Rightarrow}}
\def\ciff{\mathop{\Leftrightarrow}}
\def\fneg{\widecneg{(-)}}

\newbox\cutbox
\newdimen\cutwd
\newdimen\cutht
\newdimen\cutdp
\def\ccut{\setbox\cutbox\hbox{}
  \cutwd=\wd\cutbox
  \cutht=\ht\cutbox
  \cutdp=\dp\cutbox
  \setbox\cutbox\hbox to\cutwd{\hss\vrule width.3pt height\cutht depth\cutdp\hss}
\mathbin{\lozenge\hskip-\cutwd\copy\cutbox}}
\def\scriptcut{\setbox\cutbox\hbox{}
  \cutwd=\wd\cutbox
  \cutht=\ht\cutbox
  \cutdp=\dp\cutbox
  \setbox\cutbox\hbox to\cutwd{\hss\vrule width.3pt height\cutht depth\cutdp\hss}
  \mathord{\lozenge\hskip-\cutwd\copy\cutbox}}

\def\vccut{\setbox\cutbox\hbox{}
  \cutwd=\wd\cutbox
  \cutht=\ht\cutbox
  \cutdp=\dp\cutbox
  \setbox\cutbox\hbox to\cutwd{\hss\hskip.3pt\vrule width.3pt height\cutht depth\cutdp\hss}
\mathbin{\lozenge\hskip-\cutwd\copy\cutbox}}




\def\conldel {\{}\def\conrdel {\}}\def\lrgldel {\mathchoice{(}{(}{\langle}{\langle}}\def\lrgrdel {\mathchoice{)}{)}{\rangle}{\rangle}}\def\aprldel {\mathchoice
    {\mathopen {\setbox0=\hbox{}\hbox to\wd0
                         {\hfil\hfil}}}{\mathopen {\setbox0=\hbox{}\hbox to\wd0
                         {\hfil\hfil}}}{\mathopen {\setbox0=\hbox{}\hbox to\wd0
                         {\hfil\hfil}}}{\mathopen {\setbox0=\hbox{}\hbox to\wd0
                         {\hfil\hfil}}}}\def\aprrdel {\mathchoice
    {\mathclose{\setbox0=\hbox{}\hbox to\wd0
                         {\hfil\hfil}}}{\mathclose{\setbox0=\hbox{}\hbox to\wd0
                         {\hfil\hfil}}}{\mathclose{\setbox0=\hbox{}\hbox to\wd0
                         {\hfil\hfil}}}{\mathclose{\setbox0=\hbox{}\hbox to\wd0
                         {\hfil\hfil}}}}\def\seqldel {\mathchoice
    {\mathopen {\setbox0=\hbox{}\hbox to\wd0
                         {\hfil\hfil}}}{\mathopen {\setbox0=\hbox{}\hbox to\wd0
                         {\hfil\hfil}}}{\mathopen {\setbox0=\hbox{}\hbox to\wd0
                         {\hfil\hfil}}}{\mathopen {\setbox0=\hbox{}\hbox to\wd0
                         {\hfil\hfil}}}}\def\seqrdel {\mathchoice
    {\mathclose{\setbox0=\hbox{}\hbox to\wd0
                         {\hfil\hfil}}}{\mathclose{\setbox0=\hbox{}\hbox to\wd0
                         {\hfil\hfil}}}{\mathclose{\setbox0=\hbox{}\hbox to\wd0
                         {\hfil\hfil}}}{\mathclose{\setbox0=\hbox{}\hbox to\wd0
                         {\hfil\hfil}}}}\def\parldel {\mathchoice
    {\mathopen {\setbox0=\hbox{}\hbox to\wd0
                         {\hfil\hfil}}}{\mathopen {\setbox0=\hbox{}\hbox to\wd0
                         {\hfil\hfil}}}{\mathopen {\setbox0=\hbox{}\hbox to\wd0
                         {\hfil\hfil}}}{\mathopen {\setbox0=\hbox{}\hbox to\wd0
                         {\hfil\hfil}}}}\def\parrdel {\mathchoice
    {\mathclose{\setbox0=\hbox{}\hbox to\wd0
                         {\hfil\hfil}}}{\mathclose{\setbox0=\hbox{}\hbox to\wd0
                         {\hfil\hfil}}}{\mathclose{\setbox0=\hbox{}\hbox to\wd0
                         {\hfil\hfil}}}{\mathclose{\setbox0=\hbox{}\hbox to\wd0
                         {\hfil\hfil}}}}

\def\eightpoint{\small}                         
\def\pluldel {\mathchoice
   {\mathopen {\setbox0=\hbox{}\hbox to\wd0
                        {\hfil\hfil}\kern-\wd0\hbox to\wd0
                        {\hss\scriptscriptstyle\bullet\hss}}}{\mathopen {\setbox0=\hbox{}\hbox to\wd0
                        {\hfil\hfil}\kern-\wd0\hbox to\wd0
                        {\hss\scriptscriptstyle\bullet\hss}}}{\mathopen {\setbox0=\hbox{}\hbox to\wd0
                        {\hfil\hfil}\kern-\wd0\hbox to\wd0
                        {\hss\scriptscriptstyle\bullet\hss}}}{\mathopen {\setbox0=\hbox{}\hbox to\wd0
                        {\hfil\hfil}\kern-\wd0\hbox to\wd0
                        {\hss\scriptscriptstyle\bullet\hss}}}}\def\plurdel {\mathchoice
   {\mathclose{\setbox0=\hbox{}\hbox to\wd0
                        {\hfil\hfil}\kern-\wd0\hbox to\wd0
                        {\hss\scriptscriptstyle\bullet\hss}}}{\mathclose{\setbox0=\hbox{}\hbox to\wd0
                        {\hfil\hfil}\kern-\wd0\hbox to\wd0
                        {\hss\scriptscriptstyle\bullet\hss}}}{\mathclose{\setbox0=\hbox{}\hbox to\wd0
                        {\hfil\hfil}\kern-\wd0\hbox to\wd0
                        {\hss\scriptscriptstyle\bullet\hss}}}{\mathclose{\setbox0=\hbox{}\hbox to\wd0
                        {\hfil\hfil}\kern-\wd0\hbox to\wd0
                        {\hss\scriptscriptstyle\bullet\hss}}}}\def\witldel {\mathchoice
   {\mathopen {\setbox0=\hbox{}\hbox to\wd0
                        {\hfil\hfil}\kern-\wd0\hbox to\wd0
                        {\hss\scriptscriptstyle\bullet\mkern3.2mu\hss}}}{\mathopen {\setbox0=\hbox{}\hbox to\wd0
                        {\hfil\hfil}\kern-\wd0\hbox to\wd0
                        {\hss\scriptscriptstyle\bullet\mkern3.2mu\hss}}}{\mathopen {\setbox0=\hbox{}\hbox to\wd0
                        {\hfil\hfil}\kern-\wd0\hbox to\wd0
                        {\hss\scriptscriptstyle\bullet\mkern3.2mu\hss}}}{\mathopen {\setbox0=\hbox{}\hbox to\wd0
                        {\hfil\hfil}\kern-\wd0\hbox to\wd0
                        {\hss\scriptscriptstyle\bullet\mkern3.2mu\hss}}}}\def\witrdel {\mathchoice
   {\mathclose{\setbox0=\hbox{}\hbox to\wd0
                        {\hfil\hfil}\kern-\wd0\hbox to\wd0
                        {\hss\scriptscriptstyle\mkern3.2mu\bullet\hss}}}{\mathclose{\setbox0=\hbox{}\hbox to\wd0
                        {\hfil\hfil}\kern-\wd0\hbox to\wd0
                        {\hss\scriptscriptstyle\mkern3.2mu\bullet\hss}}}{\mathclose{\setbox0=\hbox{}\hbox to\wd0
                        {\hfil\hfil}\kern-\wd0\hbox to\wd0
                        {\hss\scriptscriptstyle\mkern3.2mu\bullet\hss}}}{\mathclose{\setbox0=\hbox{}\hbox to\wd0
                        {\hfil\hfil}\kern-\wd0\hbox to\wd0
                        {\hss\scriptscriptstyle\mkern3.2mu\bullet\hss}}}}

\newbox\ldelbox
\setbox\ldelbox=\hbox{}
\def\ldelskip{\hskip\wd\ldelbox}
\newbox\rdelbox
\setbox\rdelbox=\hbox{}
\def\rdelskip{\hskip\wd\rdelbox}

\def\wits #1{\witldel #1\witrdel}\def\plus #1{\pluldel #1\plurdel}

\def\aprs #1{\aprldel #1\aprrdel}\def\pars #1{\parldel #1\parrdel}\def\seqs #1{\seqldel #1\seqrdel}\def\cons #1{\conldel #1\conrdel}\def\copt #1{#1}





\let\turnstile=\vdash
\def\lone{1}
\def\lbot{\bot}
\def\lzero{0}
\def\ltop{\top}
\def\ltens{\mathop\varotimes}
\def\lcut{\mathop\varobar}
\def\lpar{\mathop\bindnasrepma}
\def\lplus{\mathop\varoplus}
\def\lwith{\mathop\binampersand}
\def\lwn{\mathord{?}}
\def\loc{\mathord{!}}
\def\lneg{^\bot}
\def\lnegneg{^{\bot\bot}}
\def\limp{\multimap}
\def\lseq{\mathop\vartriangleleft}
\def\lcoseq{\mathop\vartriangleright}

\def\lweakr{\mathord{\lwn}\mathsf{w}}
\def\lconr{\mathord{\lwn}\mathsf{c}}
\def\lderr{\mathord{\lwn}\mathsf{d}}





\def\quadfs {\rlap{\rm\quad.}}\def\quadcm {\rlap{\rm\quad,}}\def\quadsc {\rlap{\rm\quad;}}\def\quadcl {\rlap{\rm\quad:}}\def\quadqm {\rlap{\rm\quad?}}

\def\quby {\quad\mbox{by}\quad}\def\qquby {\qquad\mbox{by}\qquad}\def\qqquby {\quad\qquad\mbox{by}\qquad\quad}\def\qqqquby {\qquad\qquad\mbox{by}\qquad\qquad}

\def\quato {\quad\to\quad}\def\qquato {\qquad\to\qquad}\def\qqquato {\quad\qquad\to\qquad\quad}\def\qqqquato {\qquad\qquad\to\qquad\qquad}

\def\dqualto {\hskip.5em\leadsto\hskip.5em}\def\qualto {\quad\leadsto\quad}\def\qqualto {\qquad\leadsto\qquad}\def\qqqualto {\quad\qquad\leadsto\qquad\quad}\def\qqqqualto {\qquad\qquad\leadsto\qquad\qquad}

\def\quand {\quad\mbox{and}\quad}\def\qquand {\qquad\mbox{and}\qquad}\def\qqquand {\quad\qquad\mbox{and}\qquad\quad}\def\qqqquand {\qquad\qquad\mbox{and}\qquad\qquad}

\def\nquand {\quad\mbox{\normalsize and}\quad}\def\nqquand {\qquad\mbox{\normalsize and}\qquad}\def\nqqquand {\quad\qquad\mbox{\normalsize and}\qquad\quad}\def\nqqqquand {\qquad\qquad\mbox{\normalsize and}\qquad\qquad}

\def\quor {\quad\mbox{or}\quad}\def\qquor {\qquad\mbox{or}\qquad}\def\qqquor {\quad\qquad\mbox{or}\qquad\quad}\def\qqqquor {\qquad\qquad\mbox{or}\qquad\qquad}

\def\nquor {\quad\mbox{\normalsize or}\quad}\def\nqquor {\qquad\mbox{\normalsize or}\qquad}\def\nqqquor {\quad\qquad\mbox{\normalsize or}\qquad\quad}\def\nqqqquor {\qquad\qquad\mbox{\normalsize or}\qquad\qquad}

\def\quwith {\quad\mbox{with}\quad}\def\qquwith {\qquad\mbox{with}\qquad}\def\qqquwith {\quad\qquad\mbox{with}\qquad\quad}\def\qqqquwith {\qquad\qquad\mbox{with}\qquad\qquad}

\def\quiff {\quad\mbox{iff}\quad}\def\qquiff {\qquad\mbox{iff}\qquad}\def\qqquiff {\quad\qquad\mbox{iff}\qquad\quad}\def\qqqquiff {\qquad\qquad\mbox{iff}\qquad\qquad}

\def\quifof {\quad\mbox{if and only if}\quad}\def\qquifof {\qquad\mbox{if and only if}\qquad}\def\qqquifof {\quad\qquad\mbox{if and only if}\qquad\quad}\def\qqqquifof {\qquad\qquad\mbox{if and only if}\qquad\qquad}

\def\queq {\quad=\quad}\def\qqueq {\qquad=\qquad}

\def\qcdots{\quad\cdots\quad}

\def\qquadfs {\rlap{\rm\qquad.}}\def\qquadcm {\rlap{\rm\qquad,}}

\def\yields{\quad\mbox{yields}\quad}

\def\mydisplaybreak{\hspace*{\fill} \cr \hspace*{\fill}}

\def\vcvdots{\vbox{\baselineskip=4pt \lineskiplimit=0pt
    \hbox{.}\hbox{.}\hbox{.}}}




\def\nlap#1{#1}
\def\clap#1{\hbox to 0pt{\hss#1\hss}}
\def\sqlap#1{\hbox to .5em{\hss#1\hss}}
\def\qlap#1{\hbox to 1em{\hss#1\hss}}
\def\qqlap#1{\hbox to 2em{\hss#1\hss}}
\def\qqqlap#1{\hbox to 3em{\hss#1\hss}}
\def\qqqqlap#1{\hbox to 4em{\hss#1\hss}}
\def\qqqqqlap#1{\hbox to 5em{\hss#1\hss}}
\def\qqqqqqlap#1{\hbox to 6em{\hss#1\hss}}
\def\qqqqqqqlap#1{\hbox to 7em{\hss#1\hss}}
\def\qqqqqqqqlap#1{\hbox to 8em{\hss#1\hss}}
\def\qqqqqqqqqlap#1{\hbox to 9em{\hss#1\hss}}
\newcommand{\wlap}[2][10ex]{\hbox to#1{\hss#2\hss}}
\def\nlapm#1{#1}
\def\clapm#1{\clap{}}
\def\sqlapm#1{\sqlap{}}
\def\qlapm#1{\qlap{}}
\def\qqlapm#1{\qqlap{}}
\def\qqqlapm#1{\qqqlap{}}
\def\qqqqlapm#1{\qqqqlap{}}
\def\qqqqqlapm#1{\qqqqqlap{}}
\def\qqqqqqlapm#1{\qqqqqqlap{}}
\def\qqqqqqqlapm#1{\qqqqqqqlap{}}
\def\qqqqqqqqlapm#1{\qqqqqqqqlap{}}
\def\qqqqqqqqqlapm#1{\qqqqqqqqqlap{}}
\newcommand{\wlapm}[2][10ex]{\hbox to#1{\hss\hss}}
\def\rlapm#1{\hbox to 0pt{\hss}}
\def\llapm#1{\hbox to 0pt{\hss}}

\def\qqquad{\quad\qquad}
\def\qqqquad{\qquad\qquad}
\def\qqqqquad{\qqquad\qquad}
\def\qqqqqquad{\qqquad\qqquad}

\newcommand{\vclap}[2][0pt]{\hbox to #1{\hss#2\hss}}
\newcommand{\vclapm}[2][0pt]{\hbox to #1{\hss\hss}}



\def\proofadjust{\vadjust{\nobreak\vskip-2.7ex\nobreak}}
\def\semiproofadjust{\vadjust{\nobreak\vskip-1.3ex\nobreak}}
\def\doubleproofadjust{\vadjust{\nobreak\vskip-4ex\nobreak}}
\def\tripleproofadjust{\vadjust{\nobreak\vskip-5.3ex\nobreak}}


\def\interdisplayskip{.5ex}
\newskip\mydisplaywidth
\newcommand{\twolinedisplay}[3][10pt]{\mydisplaywidth=\displaywidth
  \advance\mydisplaywidth-#1
  \begin{array}{c}
    \clap{\hbox to\mydisplaywidth{\hss}}\
\vlinf{}{\axr}{\;A,\dual{A}\;}{}
\qqquad
\vlinf{}{\top}{\;\top\;}{}
\qqquad
\vlinf{}{\weakr}{\Gamma, A}{\Gamma}
\qqquad
\vlinf{}{\conr}{\Gamma, A}{\Gamma, A, A}
\qqquad
\vliinf{}{\cutrr{}}{\Gamma,\Delta}{\Gamma, A\;}{\;\dual{A}, \Delta}

\vlinf{}{\lor}{\Gamma, A\lor B}{\Gamma, A, B}
\qqquad
\vliinf{}{\land}{\Gamma,\Delta,A\land B}{\Gamma,A\;\;}{\;\;\Delta, B}
\qqquad
\vlinf{}{\forall}{\Gamma, \forall x\, A}{\Gamma, A\unsubst{x}{\alpha}}
\qqquad
\vlinf{}{\exists}{\Gamma, \exists x\, A}{\Gamma, A\unsubst{x}{t}}
\myskip]
      \qqquad\qqquad&
      \vlderivation{
        \vliin{}{\cutrr{}}{\Gamma,A}{
          \vlhtr{\psi}{\Gamma,A}}{
          \vlin{}{\axr}{\dual{A},A}{
            \vlhy{}}}}
      &\scutred&
      \vlderivation{
        \vlhtr{\psi}{\Gamma,A}}
      \\\\
      \rlap{Quantifier reduction:}\\myskip]
      &
      \vlderivation{
        \vliin{}{\cutrr{}}{\Gamma,\Delta,\Pi}{
          \vliin{}{\land}{\Gamma, \Delta, A\land B}{
            \vlhtr{\psi_1}{\Gamma,A}}{
            \vlhtr{\psi_2}{\Delta,B}}}{
          \vlin{}{\lor}{\dual{A}\lor \dual{B}, \Pi}{
            \vlhtr{\psi_3}{\dual{A},\dual{B},\Pi}}}}
      &\scutred&
      \vlderivation{
        \vliin{}{\cutrr{}}{\Gamma,\Delta,\Pi}{
          \vlhtr{\psi_2}{\Delta,B}}{
          \vliin{}{\cutrr{}}{\dual{B}, \Gamma, \Pi}{
            \vlhtr{\psi_1}{\Gamma,A}}{
            \vlhtr{\psi_3}{\dual{A},\dual{B},\Pi}}}}
      \\\\
      \rlap{Contraction reduction:}\\ &
      \vlderivation{
        \vliin{}{\cutrr{}}{\Gamma,\Delta}{
          \vlin{}{\conr}{\Gamma, A}{
            \vlhtr{\psi_1}{\Gamma,A,A}}}{
          \vlhtr{\psi_2}{\dual{A},\Delta}}}
      &\scutred&
      \vlderivation{
        \vliq{}{\conr^\ast}{\Gamma,\Delta}{
          \vliin{}{\cutrr{}}{\Gamma,\Delta,\Delta}{
            \vliin{}{\cutrr{}}{\Gamma, \Delta, A}{
              \vlhtr{\psi_1}{\Gamma,A,A}}{
              \vlhtr{~\psi_2\rho'}{\dual{A},\Delta}}}{
            \vlhtr{~\psi_2\rho''}{\dual{A},\Delta}}}}
      \\\\
      \rlap{Weakening reduction:}\\myskip]
      &
      \vlderivation{
        \vliin{}{\cutrr{}}{\Gamma,\Delta}{
          \vlin{}{\rr}{\Gamma,A}{
            \vlhtr{\psi_1}{\Gamma',A}}}{
          \vlhtr{\psi_2}{\dual{A},\Delta}}}
      &\scutred&
      \vlderivation{
        \vlin{}{\rr}{\Gamma,\Delta}{
          \vliin{}{\cutrr{}}{\Gamma',\Delta}{
            \vlhtr{\psi_1}{\Gamma',A}}{
            \vlhtr{\psi_2}{\dual{A},\Delta}}}}
      \\\\
      \rlap{Binary inference permutation:}\\rho'=\unsubst{\alpha}{\alpha'}_{\alpha\in \EV(\psi_2)}
  \qqquand
  \rho''=\unsubst{\alpha}{\alpha''}_{\alpha\in \EV(\psi_2)}
  G&=&\left\langle \set{\theta,\alpha,\beta}, \set{\alpha} , \set{ 0/0, s/1, f/2 }, \theta,P\right\rangle
  \quad\text{where}\\
  P&=&\{\; \theta \rightarrow f(\alpha,\alpha),\\
  &&\phantom{\{\;}\alpha\rightarrow 0 \mid s(\beta), \\
  &&\phantom{\{\;}\beta \rightarrow 0 \mid s(\beta)\; \}
\Hseq(\pi)=\bigcup_{P\in\Gamma}\Hseq(P)
    \vlinf{}{\exists}{\Gamma, \exists x\, A}{\Gamma,A\unsubst{x}{t}}
  
    \label{eq:simple}
    \vcenter{\lkcut{\Gamma,\Delta}{
        \Gamma, B
        &&
        \dual{B},\Delta
    }}
    \qqquor
    \vcenter{\lkcut{\Gamma,\Delta}{
        \Gamma,\exists x\, B
        &&
        \lku{\forall x\, \dual{B},\Delta}{
          \dual{B}\unsubst{x}{\alpha},\Delta
    }}}
  \Bsub(\pi)=\Union_{\alpha\in\EVc(\pi)} \Bsub(\alpha)\quadfs
  N_R&=\EVc(\pi)\cup\set{\theta}\\
  \Sigma&=\Sigma(\pi) \union \set{\land,\lor,\top,\bot}\\
  P&=\set{\theta\to A \mid A\in \Hseq(\pi)}\union 
\set{\alpha\to t \mid \unsubst{\alpha}{t}\in\Bsub(\pi)}

  \vlderivation{ 
    \vliiin{}{\cutrr{\beta}}{\qcdots}{
      \vliin{}{\cutrr{\alpha}}{\qcdots}{ 
        \vlhy{\qcdots}}{
        \vlin{}{\forallrr{\alpha}}{\qcdots}{
          \vlhy{\qcdots}}}}{
      \vlhy{\;\;}}{
      \vlin{}{\forallrr{\beta}}{\qcdots}{
        \vlhy{\qcdots}}}}
  \qqualto
  \vlderivation{
    \vliin{}{\cutrr{\alpha}}{\qcdots}{
      \vlhy{\qcdots}}{
\vliiin{}{\cutrr{\beta}}{\qcdots}{
        \vlin{}{\forallrr{\alpha}}{\qcdots}{
          \vlhy{\qcdots}}}{
        \vlhy{\;\;}}{
        \vlin{}{\forallrr{\beta}}{\qcdots}{
          \vlhy{\qcdots}}}}}

  \vlderivation{
    \vliin{}{\cutrr{\alpha}}{\qcdots}{
      \vlhy{\qcdots}}{
\vlin{}{\forallrr{\alpha}}{\qcdots}{
        \vliiin{}{\cutrr{\beta}}{\qcdots}{
          \vlhy{\qcdots}}{
          \vlhy{\;\;}}{
          \vlin{}{\forallrr{\beta}}{\qcdots}{
            \vlhy{\qcdots}}}}}}

    \cA\Hsub\cB \qqquiff \text{for all  there
      is a  with }\quad.    
  
\renewcommand{\vlvruler}{\vss\hbox{}\kern1pt}
  \vlderivation{
    \vliiin{}{\cutrr{\beta}}{\Gamma,\Delta}{
      \vlde{}{}{\Gamma,\vlex x A}{
        \vlin{}{\existsrr{t}}{\Gamma',\vlex x A}{
          \vlhy{\Gamma',A\unsubst{x}{t}}}}}{
      \vlhy{\qquad}}{
      \vlde{}{}{\vlfa x \dual A,\Delta}{
        \vlin{}{\forallrr\beta}{\vlfa x \dual A,\Delta'}{
          \vlhy{\dual A\unsubst{x}{\beta},\Delta'}}}}}

    \vlderivation{
      \vliiin{}{\rr_3}{\vdots}{
        \vlin{}{\rr_1}{\quad\ddots}{
          \vlhy{\vdots}}}{
        \vlhy{\quad}}{
        \vlin{}{\rr_2}{\iddots\quad}{
          \vlhy{\vdots}}}}
  
    \vlderivation{
        \vliiin{}{\cutrr{\alpha_l}}{\vdots}{
          \vlin{}{\existsrr{t_{n}}}{\quad\ddots}{
            \vlhy{\vdots}}}{
          \vlhy{\quad}}{
          \vlin{}{\forallrr{\alpha_1}}{\iddots\quad}{
            \vlhy{\vdots}}}}
  
    \vlderivation{
      \vliin{}{\cutrr{\alpha_{n+1}}}{\vdots}{
        \vlin{}{\existsrr{t_{n+1}}}{\quad\ddots}{
          \vlhy{\vdots}}}{
        \vliin{}{\cutrr{\alpha_l}}{\iddots\quad}{
          \vlin{}{\existsrr{t_{n}}}{\quad\ddots}{
            \vlhy{\vdots}}}{
          \vlin{}{\forallrr{\alpha_1}}{\iddots\quad}{
            \vlhy{\vdots}}}}}
    \nqquor
    \vlderivation{
      \vliin{}{\cutrr{\alpha_l}}{\vdots}{
        \vliin{}{\cutrr{\alpha_{n+1}}}{\quad\ddots}{
          \vlin{}{\existsrr{t_{n+1}}}{\quad\ddots}{
            \vlhy{\vdots}}}{
          \vlin{}{\existsrr{t_{n}}}{\iddots\quad}{
            \vlhy{\vdots}}}}{
        \vlin{}{\forallrr{\alpha_1}}{\iddots\quad}{
          \vlhy{\vdots}}}}
  
    &&\beta\to t\ni\gamma_0\to s_0\ni\gamma_1\to s_1\ni\ldots
    \to s_{n-1}\ni\gamma_n=\alpha\to s_n\\
    &&\beta\to t\ni\delta_0\to r_0\ni\delta_1\to r_1\ni\ldots
    \to r_{m-1}\ni\delta_m=\alpha\to r_m
  
      \vlderivation{
      \vliin{}{\cutrr{\delta_0}}{\vdots}{
        \vliin{}{\cutrr{\alpha}}{\quad\ddots}{
          \vlin{}{\existsrr{s_n}}{\quad\ddots}{
            \vlhy{\vdots}}}{
          \vlin{}{\forallrr{\alpha}}{\iddots\quad}{
            \vlhy{\vdots}}}}{
        \vliin{}{\cutrr{\gamma_0}}{\iddots\quad}{
          \vlin{}{\existsrr{s_0}}{\quad\ddots}{
            \vlhy{\vdots}}}{
          \vlin{}{\forallrr{\gamma_0}}{\iddots\quad}{
            \vlhy{\vdots}}}}}
    
      \vlderivation{
      \vliin{\qquad}{\cutrr{\alpha}}{\vdots}{
        \vlin{}{\existsrr{s_n}}{\quad\ddots}{
          \vlhy{\vdots}}}{
        \vliin{}{\cutrr{\delta_0}}{\iddots\quad}{
          \vlhy{\quad\ddots}}{
          \vliin{}{\cutrr{\gamma_0}}{\iddots\quad}{
            \vlhy{\quad\ddots}}{
            \vlin{}{\forallrr{\gamma_0}}{\iddots\quad}{
              \vlhy{\vdots}}}}}}
      \nqquand
      \vlderivation{
      \vliin{}{\cutrr{\alpha}}{\vdots}{
        \vlin{}{\existsrr{r_m}}{\quad\ddots}{
          \vlhy{\vdots}}}{
        \vliin{}{\cutrr{\delta_0}}{\iddots\quad}{
          \vlhy{\quad\ddots}}{
          \vlin{}{\forallrr{\delta_0}}{\iddots\quad}{
              \vlhy{\vdots}}}}}
     
      \vlderivation{
        \vliin{}{\rr}{\vdots}{
          \vliin{}{\cutrr{\alpha}}{\quad\ddots}{
            \vlin{}{\existsrr{r_m}}{\quad\ddots}{
              \vlhy{\vdots}}}{
            \vlhy{\iddots\quad}}}{          
          \vliin{}{\cutrr{\delta_0}}{\iddots\quad}{
            \vlhy{\quad\ddots\quad}}{
            \vlin{}{\forallrr{\delta_0}}{\iddots\quad}{
              \vlhy{\vdots}}}}}
    
      \vlderivation{
        \vliin{}{\rr}{\vdots}{
          \vliin{}{\cutrr{\alpha}}{\quad\ddots}{
            \vlin{}{\existsrr{s_n}}{\quad\ddots}{
              \vlhy{\vdots}}}{
            \vlhy{\iddots\quad}}}{
          \vliin{}{\cutrr{\delta_0}}{\iddots\quad}{
            \vlhy{\quad\ddots}}{
            \vliin{}{\cutrr{\gamma_0}}{\iddots\quad}{
              \vlhy{\quad\ddots\quad}}{
              \vlin{}{\forallrr{\gamma_0}}{\iddots\quad}{
                \vlhy{\vdots}}}}}}
    
      \vlderivation{
        \vliin{}{\cutrr{\delta_l}}{\vdots}{
          \vliin{}{\cutrr{\alpha}}{\quad\ddots}{
            \vlin{}{\existsrr{r_m}}{\quad\ddots}{
              \vlhy{\vdots}}}{
            \vlin{}{\forallrr{\alpha}}{\iddots\quad}{
              \vlhy{\vdots}}}}{          
          \vliin{}{\cutrr{\delta_0}}{\iddots\quad}{
            \vlhy{\quad\ddots\quad}}{
            \vlin{}{\forallrr{\delta_0}}{\iddots\quad}{
              \vlhy{\vdots}}}}}
  
  \psi
  \quad = \quad
  \vlderivation{
    \vliin{}{\cutr}{\Gamma,\Delta}{
      \vlin{}{\conr}{\Gamma,A}{
        \vlbtrf{\psi_1}{\Gamma,A,A}{2em}{.8}}}{
      \vlbtrf{\psi_2}{\dual A,\Delta}{2em}{.8}}}
  \qqualto
  \vlderivation{
    \vliq{}{\conr^\ast}{\Gamma,\Delta}{
      \vliin{}{\cutr}{\Gamma,\Delta,\Delta}{
        \vliin{}{\cutr}{\Gamma,\Delta,A}{
          \vlbtrf{\psi_1}{\Gamma,A,A}{2em}{.8}}{
          \vlbtrf{\psi_2\rho'}{\dual A,\Delta}{2em}{.8}}}{
        \vlbtrf{\psi_2\rho''}{\dual A,\Delta}{2em}{.8}}}}
  \quad = \quad
  \psi'

  P'=(P\setminus Q)\rho'\union\{ \alpha'\to t_1,\ldots,\alpha'\to t_k \}
  \union 
  (P\setminus Q)\rho'' \union \{ \alpha''\to t_{k+1},\ldots,
  \alpha''\to t_n \}\quad.

  \begin{array}{c}
    \beta\to t\ni\gamma_0\to s_0\ni\gamma_1\to s_1\ni\ldots
    \to s_{n-1}\ni\gamma_n=\alpha\to s_n
    \
where  and  occur at two different positions
in~. Thus, we can apply Lemma~\ref{lem.alphapos}, giving us the
following two cases:
\begin{itemize}
\item We have  for some  and . Say , and let  and
   be the positions of  and 
  (respectively) in . Since  we know that
   does not violate the rigidity condition (we chose  to
  be minimal), and therefore . Let
   and
   be the two
  subderivations of  starting in positions  and
  , respectively. Without loss of generality, we can
  assume that . Then let
   be the derivation obtained from  by replacing
   by . Then  is still a derivation for
  , but .
\item For all  and  we have
   and
  . So all inferences
  of the path  as well as all
  inferences of  are in
  . Therefore all variables of of these paths are in
  . As  violates the rigidity in  one of
   must be a -position and the other a
  -position in  because  does satisfy the
  rigidity condition. Without loss of generality we can assume
  that~ is the -position and  the
  -position. As the paths are contained completely in
   we have  and
   which is a contradiction as no term
  can contain both a variable from  and one from
  .  \qedhere
\end{itemize}
\end{proof}


\begin{proof}[Proof of Lemma~\ref{lem.inv_weak}]
By induction on the length of the reduction 
or  respectively using one of Lemmas~\ref{lem:H-invar}, 
\ref{lem:H-invar-quant}, \ref{lem:H-invar-weak} or \ref{lem.subset_contraction_reduction}
depending on the current reduction step.
\end{proof}



\section{Skolemization and Deskolemization}\label{sec.skol_deskol}

In this section we will describe some results that allow one to extend the above invariance lemma
to proofs of arbitrary end-sequents
(including -quantifiers). Carrying out the above argument directly 
for arbitrary end-sequents would require dealing with variable-names on
the level of the grammar in order to describe the changes of eigenvariables
of the -quantifiers in the end-sequent. This can be avoided completely
by skolemizing proofs to reduce the general case to that of weak sequents and then
translating back the results by deskolemization. Skolemization and deskolemization
are simple operations on the level of Herbrand-disjunctions or
expansion trees~\cite{Miller87Compact} and their use in this context suffices
for our purposes. In contrast, they have surprising complexity-effects
on the level of proofs, see e.g.~\cite{Baaz12Complexity}. The reason why this transfer
is possible is that the form of the end-sequent,
and in particular the question whether it contains universal quantifiers,
does not have an effect on the dynamics of cut-elimination. This 
observation has been well known for a long time and is apparent already in Gentzen's consistency
proof for Peano Arithmetic~\cite{Gentzen38Neue} which is carried out on a (hypothetical) proof of
the empty sequent as well as in the proof of the second -Theorem
from the first -Theorem by
deskolemization~\cite{Hilbert39Grundlagen2}.

Let us now first define the notion of Herbrand-disjunction precisely. We assume
w.l.o.g.\ that in a formula every variable is bound by at most one quantifier.

\begin{defi}
For a given formula , we write  for the formula obtained from
 by removing all quantifiers. Now let  be the
existentially bound variables in , and let  be the
universally bound variables in . Then any formula of the shape 

where  is an arbitrary formula with , where  are arbitrary terms, and where
 are fresh variables, is called an
\emph{instance of }.
If  is a sequent we say that a set  of formulas is a \emph{set
of instances of } if for every  there is a , s.t.\  is instance
of .
\end{defi}

Often we will work in the context of a proof  of a sequent  and
consider the instances of the formulas in  that are induced by .
Then the above fresh variables  will be eigenvariables
of the proof and their occurrences in terms will be restricted by an acylicity-condition, see below.

Let  be a sequent, let  be a set of
instances of , let  be the number of quantifiers in
, and let  be the number of instances of  in . If
we impose an arbitrary linear ordering on the instances of  in
, then a tuple  for  and
 and  uniquely identifies the
term which is substituted for the quantifier  in the -th
instance of the formula . We will write  for this term
(which could just be an eigenvariable if  happens to be an
-quantifier).  The -th instance of  can hence be
written as , where
 are the bound variables in , and 
is some formula with .  Such a tuple  is called \emph{existential position} if  is
bound existentially in , and \emph{universal position} if 
is bound universally in .

A position  is said to \emph{dominate}
another position , if , and
, and the quantifier  is in the scope of the
quantifier  in . A set  of instances induces a
relation  on its existential positions as:  if there is
a universal position , such that the term
 contains a variable  with  and  dominates . Furthermore we define the \emph{dependency
  relation } on the existential positions of  as
transitive closure of~.

\begin{rem}
  A proof  with the property that  is sometimes
  called a \emph{sequentialization of }.  If  has positions
   and  with
  ,
  then in each sequentialization of  the inference corresponding
  to  is below that of . In the literature on proof nets, relations like
   are known as \emph{jumps}.
\end{rem}
 


\begin{defi}
  A set  of instances of  is called
  \emph{Herbrand-disjunction of } if
  \begin{itemize}
  \item the dependency relation  of  is acyclic, and
  \item  is a tautology.
  \end{itemize}
\end{defi}

This notion of Herbrand-disjunction is essentially a flat (as opposed to tree-like)
formulation of expansion tree proofs~\cite{Miller87Compact}. A similar flat formulation
can, for instance, be found in~\cite{Baaz94Skolemization}.

\begin{theorem}
 is valid iff it has a Herbrand-disjunction.
\end{theorem}
\begin{proof}[Proof Sketch]
Via translating back and forth with cut-free sequent calculus or alternatively via
expansion tree proofs.
\end{proof}
\begin{exa}\label{ex.drinker}
Let , let
} and
fix the numbering of quantifiers and instances to be from the left to the
right. Then there are the two existential positions 
with  and  with 
and two universal positions  with 
and  with . As 
dominates , we have ,
but not the other way round because  is variable-free. Therefore
 is acyclic. Furthermore  is a tautology and hence a Herbrand-disjunction.
\end{exa}
Note that for a weak sequent , the induced dependency ordering  is empty and
hence trivially acyclic. The Herbrand-disjunctions of weak sequents are therefore exactly
the tautologies of instances.
\begin{defi}
Let  be a formula containing a universal quantifier
and let , \ldots,  be the existential quantifiers in whose scope
 is. Then define the \emph{Skolemization} of this universal quantifier~as

where  is a fresh -ary function symbol, called a {\em Skolem
  function symbol}.  The term  is called
\emph{Skolem-term}.  For a formula  define its \emph{Skolemization
  } to be the iteration of  until no universal
quantifier is left, such that no Skolem function symbol is used for
two different universal quantifiers in . For a sequent  define its \emph{Skolemization }, where no Skolem function symbol is used
for two different universal quantifiers in .
\end{defi}

\begin{rem}
Sometimes the above operation on formulas is also called Herbrandization. We
prefer to use the name Skolemization due to the simple duality between the
satisfiability-preserving replacement of existential quantifiers and the
validity-preserving replacement of universal quantifiers by new function
symbols. There is no danger of confusion as, in the proof-theoretic context
of this work, we are clearly dealing with validity only. This use of terminology
is due to~\cite{Hilbert39Grundlagen2}, see in particular Section~3.5.a.
\end{rem}

The above side condition on the choice of Skolem function symbols results in
a 1-1 mapping between universal quantifiers
in the sequent we skolemize and the Skolem function symbols. It could be
made formally more precise by equipping the -operation with such a bijection
as second argument. However, for the sake of notational simplicity we refrain
from doing so here.

The Skolemization of formulas and sequents can be extended to a Skolemization of proofs. When
skolemizing a proof, all universal quantifiers in the end-sequent are removed and their variables are replaced
by Skolem-terms. In contrast, the cut-formulas remain unchanged, more precisely:
\begin{defi}
  Let  be a proof of a sequent , and let 
  be the variables that are bound by a -quantifier in
  . Furthermore, for each  let
   be the eigenvariables
  introduced in  by an -rule whose main formula is of the
  shape .  Then the \emph{Skolemization of the proof },
  denoted by , is the proof with end-sequent 
  that is obtained from  by 
  \begin{enumerate}
  \item removing all -quantifiers binding one of
     everywhere, and
  \item replacing each occurrence of  (for
    ) and  (for
     and ) by the
    corresponding Skolem-term. This term is in each case uniquely
    determined if we proceed from the end-sequent of  upwards to
    the axioms and demand that each rule application remains valid,
    or, in the case of the -rule, becomes void (i.e., premise
    and conclusion coincide), and
  \item removing the void rule instances.
  \end{enumerate}
  Note that  still can contain -quantifiers, namely
  those coming from a cut.
\end{defi}

The Skolemization of a proof  also affects the quantifier-free
formulas in  through the replacement of eigenvariables by Skolem
terms. In the context of proof Skolemization we hence extend the notation
 to formulas  from which some (or all) -quantifiers
have been removed; then  denotes the formula obtained from
skolemizing the remaining -quantifiers \emph{and} carrying out
the replacement of eigenvariables by Skolem-terms.
Skolemization of proofs has the following useful commutation properties.

\begin{lem}\label{lem.sk_cred_commute}
If  then . If  then
.
\end{lem}
\begin{proof}
By induction on the number of reductions in  or ,
respectively, making a case distinction on the reduction step. The most
interesting case is that of the permutation of a -inference over a cut

where the main formula of the -inference is an ancestor of the end-sequent.
This reduction step is translated to an identity-step as Skolemization maps
both of the above proofs to

Each of the other reduction steps translates directly into exactly one reduction
step in the skolemized sequence.
\end{proof}

\begin{lem}\label{lem.sk_LangGram_commute}
.
\end{lem}
\begin{proof}\sloppy
First note that  hence  and
 have the same non-terminals. Furthermore, to each
 corresponds a unique Skolem-term
in , hence to each  and  corresponds
a unique  and  and therefore to
each production  in  corresponds a unique production
 in  that is obtained from replacing
eigenvariables by their respective Skolem-terms.
If  then by
Lemma~\ref{lem.totrig_acyclic_language} we have . Now for
 being the
productions in , letting  be the corresponding productions in
 we obtain
. Thus,
.
For the other direction, note that every Skolem-term has at least one
corresponding , and as before,
this relation translates to productions. So, if  then by
Lemma~\ref{lem.totrig_acyclic_language} we have  for  being the productions
in .  By choosing one corresponding set of
productions 
where Skolem-terms are replaced by the eigenvariables from which they
originate we obtain
.
\end{proof}

As we have seen in the above proof, Skolemization can identify instances that
differ only in their variable names. The reason for this ability lies in the use
of variable names which can be chosen in a redundant way. These superfluous instances
can also be removed by an appropriate variable renaming as shown in the following
example.

\begin{exa}
Let . Then the set of instances
obtained from a sequent calculus proof that ends with an -inference is

Skolemizing would produce the following set of instances

by implicitly identifying the two formulas that become equal. A similar effect
(but without using Skolemization) can be achieved by directly
identifying  and  as in

\end{exa}

We now generalize the observations made in the above example.  For
every Herbrand-disjunction  there is a substitution , such
that  is a Herbrand-disjunction having the following
property: If two universal positions  and
 have different variables then there is a
, such that the quantifier  dominates  in  and . This follows for
example from the formulation of expansion trees
in~\cite{Chaudhuri12Systematic,ChaudhuriXXIsomorphism} which use sets
of terms for the -quantifier and a single variable for the
-quantifier.  A Herbrand-disjunction with this property is
-equivalent to one with {\em canonical variable names} in the
following sense.
\begin{defi}
Let  be a set of instances. The {\em canonical name} of the eigenvariable
of the universal position 
is  where  are the terms of the existential
positions that dominate . The {\em canonical variable
renaming}  of  is the substitution which replaces all variable
names by their canonical names.
\end{defi}

\begin{rem}
  Note that this relationship is significantly more complex than
  -equivalence, as differently named variables are identified
  according to certain criteria external to variable names. In
  particular, for some fixed , there are  of unbounded
  size such that .  This can be seen, for
  example, by continuing Example~\ref{ex.drinker}: take .
\end{rem}

We now turn to \emph{deskolemization}, the inverse operation of Skolemization. In our
setting, we only consider deskolemization of sequents and their instances,
but not of proofs. Furthermore we always assume that the original sequent
with -quantifiers is known. Hence the deskolemization of a sequent
trivially replaces it by the original sequent. More
interesting is the deskolemization of instances which will consist of
replacing Skolem-terms by (canonically named) variables.

\begin{defi}
Let  be a sequent with Skolem function symbol 
for the universal quantifier  in . Let  be a set of
instances of  and define its \emph{deskolemization}  by
repeating the replacement

on maximal Skolem-terms (w.r.t.\ the subterm ordering).
\end{defi}
In the deskolemization of a Herbrand-disjunction, the acyclicity of
the dependency relation is obtained from the acyclicity of the subterm
ordering on the Skolem-terms. Conversely, during Skolemization, the
Skolem-terms are well-defined due to the acyclicity of the dependency
relation (see
e.g.~\cite{Miller87Compact,Weller11Elimination,Baaz12Complexity} for
more details).  We hence obtain the following properties:


\begin{lem}\label{lem.desk_canonic}
Let  be a sequent and  be a weak sequent with .
\begin{enumerate}
\item If  is a Herbrand-disjunction of , then  is
a Herbrand-disjunction of .
\item If  is a Herbrand-disjunction of , then 
is a Herbrand-disjunction of .
\item If  is a Herbrand-disjunction of , then
.
\end{enumerate}
\end{lem}



\section{Herbrand-Content}\label{sec.herbrand_content}


\begin{defi}
For a simple proof , we define its
\emph{Herbrand-content} as .
\end{defi}

Note that for a cut-free proof  we have , i.e.\ the Herbrand-content is nothing other than
the Herbrand-disjunction of the proof after variable
normalization. Also note that for a proof  of a weak sequent we
have , and hence, for a cut-free proof
of a weak sequent we have . We can now lift the
main invariance lemma, Lemma~\ref{lem.inv_weak}, to proofs of
arbitrary end-sequents and formulate this result in terms of the
Herbrand-content.

\begin{theorem}\label{thm.main}
If  is a reduction sequence of simple proofs, then
.
If  is a reduction sequence of simple proofs,
then .
\end{theorem}

\begin{proof}
If  then  by Lemma~\ref{lem.sk_cred_commute}.
So, by Lemma~\ref{lem.inv_weak}, we have .
By Lemma~\ref{lem.sk_LangGram_commute}, we get
. Using
Lemma~\ref{lem.desk_canonic} and the observation that  commutes with 
we see that

The proof for  is step-by-step the same, replacing 
by .
\end{proof}

\begin{cOr}\label{cor.upperbound}
If  is a reduction sequence of simple proofs and
 is cut-free, then 
\end{cOr}
\begin{proof}
This is a direct consequence of Theorem~\ref{thm.main}.
\end{proof}

This corollary shows that  is an upper bound on the Herbrand-disjunctions obtainable by
cut-elimination from . Let us now compare this result with
another upper bound that has previously been obtained
in~\cite{Hetzl10Form}. To that aim let  denote the
regular tree grammar underlying  which can be obtained by
setting all non-terminals to non-rigid. In this notation, a central
result of~\cite{Hetzl10Form}, adapted to this paper's setting is

\begin{theorem}\label{thm.formwitness}
Let  be a proof of a formula of the shape  with  quantifier-free, and let  with  cut-free. Then .
\end{theorem}

While the Theorem~\ref{thm.formwitness} applies also to non-simple proofs,
Corollary~\ref{cor.upperbound} is stronger in several respects:

First, the size of the Herbrand-content is by an exponential smaller
than the size of the bound given by Theorem~\ref{thm.formwitness}.
Indeed, it is a straightforward consequence of
Lemma~\ref{lem.totrig_acyclic_language} that the language of a totally rigid
acyclic tree grammar with  production rules is bound by  but on
the other hand:

\begin{prop}
There is an acyclic regular tree grammar  with  productions and .
\end{prop}

\begin{proof}
Let  be an -ary function symbol, then the productions
, \ldots, 
create a tree with  leaves. Let  be terminal symbols, then by adding
the productions

we obtain the desired grammar .
\end{proof}

Secondly, the class of totally rigid acyclic tree grammars can be
shown to be in exact correspondence with the class of simple proofs in
the following sense.  Not only can we use a totally rigid acyclic tree
grammar to simulate the process of cut-elimination, we can also---in
the other direction---use cut-elimination to simulate the process of
calculating the language of a grammar. It is shown
in~\cite{Hetzl12Applying} how to transform an arbitrary acyclic
totally rigid tree grammar  into a simple proof that has a 
normal form whose Herbrand-disjunction is essentially the language of
.

The third and---for the purposes of this paper---most important
difference is that the bound of Corollary~\ref{cor.upperbound} is
{\em tight} in the sense that it can actually be reached by a cut-elimination
strategy, namely . In fact, an even stronger statement is true: not only is
there a normal form of  that reaches the bound but all of them do.
This property leads naturally to the following confluence result for
classical logic.

\begin{defi}[Herbrand-confluence]
A relation  on a set of proofs is called \emph{Herbrand-confluent}
if  and  with  and  being
normal forms for  implies that
.
\end{defi}

\begin{cOr}\label{cor.Hconfluence}
The relation  is Herbrand-confluent on the set of simple proofs.
\end{cOr}
\begin{proof}
This is a direct consequence of Theorem~\ref{thm.main}.
\end{proof}

How does this result fit together with  being neither
confluent nor strongly normalizing? In fact, note that it is possible to construct a simple proof which permits
an infinite  reduction sequence from which one can obtain
normal forms of arbitrary size by bailing out from time to time. This can be done by building on the propositional double-contraction
example found e.g.\ in~\cite{Danos97New,Gallier93Constructive,Urban00Classical} and
in a similar form in~\cite{Zucker74Correspondence}. While these infinitely many normal forms do have pairwise different Herbrand-disjunctions
when regarded as {\em multisets}, Corollary~\ref{cor.Hconfluence} shows that as {\em sets} they are all the same.
This set-character of Herbrand-disjunctions is assured by using canonical variable
names (or equivalently: Skolemization) and thus identifying repeated instances.
This observation shows that the lack of strong normalization is taken care of
by using sets instead of multisets as data structure. But what about the lack of
confluence? Results like~\cite{Baaz11Nonconfluence} and~\cite{Hetzl12Computational} show
that the number of  normal forms with different Herbrand-disjunctions
can be enormous. On the other hand we have just seen that  induces only
{\em a single} Herbrand-disjunction: . The relation between 
and the many Herbrand-disjunctions induced by  is explained by Corollary~\ref{cor.upperbound}:
 contains them all.

\section{Conclusion}\label{sec.conclusion}

We have shown that non-erasing cut-elimination for the class of simple
proofs is Herbrand-confluent. While there are different and possibly
infinitely many normal forms, they all induce the same
Herbrand-disjunction. This result motivates the definition of this
unique Herbrand-disjunction as Herbrand-{\em content} of the proof
with cut.

As future work, the authors plan to extend this result to arbitrary
first-order proofs. The treatment of blocks of quantifiers is straightforward:
the rigidity condition must be changed to apply to vectors of non-terminals.
Treating quantifier alternations is more difficult: the current
results suggest to use a \emph{stack} of totally rigid tree grammars,
each layer of which corresponds to one layer of quantifiers (and is
hence acyclic). Concerning further generalizations, note that the 
method of describing a cut-free proof by a tree language is
applicable to any proof system with quantifiers that has a
Herbrand-like theorem, e.g., even full higher-order logic as
in~\cite{Miller87Compact}. The difficulty consists in finding an
appropriate type of grammars.

Given the wealth of different methods for the extraction of
constructive content from classical proofs, what we learn from our work
about the class of simple proofs is this: the
first-order structure possesses (in contrast to the propositional
structure) a unique and canonical unfolding. The various extraction
methods hence do not differ in the choice of how to unfold the
first-order structure but only in choosing {\em which part} of it to
unfold. We therefore see that the effect of the underspecification of
algorithmic detail in classical logic is redundancy.


\section*{Acknowledgments}
The authors would like to thank Paul-Andr\'{e} Melli\`{e}s for helpful comments
on this work. The first author was supported by a Marie Curie Intra European
Fellowship within the 7th European Community Framework Programme, by the
projects I603, P22028 and P25160 of the Austrian Science Fund (FWF) and
the WWTF Vienna Research Group 12-04.



\bibliographystyle{alpha}
\bibliography{references}


\end{document}
