\documentclass{llncs}
\usepackage{amsfonts}
\usepackage{amsmath}
\usepackage{amssymb}
\usepackage[all]{xy}

\newtheorem{teo}{Theorem}
\newtheorem{prop}{Proposition}
\newtheorem{cor}{Corollary}
\newtheorem{rem}{Remark}
\newtheorem{lem}{Lemma}
\newtheorem{defi}{Definition}
\newcommand{\M}{{\mathbb M}}
\newcommand{\am}{A^\M}
\newcommand{\az}{A^\Zset}
\newcommand{\Zset}{{\mathbb Z}}
\newcommand{\Cset}{{\mathbb C}}
\newcommand{\Qset}{{\mathbb Q}}
\newcommand{\Nset}{{\mathbb N}}
\newcommand{\Rset}{{\mathbb R}}
\newcommand{\lang}{\mathcal L}
\newcommand{\length}[1]{\left|#1\right|}
\newcommand{\card}[1]{\left|#1\right|}
\newcommand{\appl}[5]{#1:\begin{array}{rcl}#2&\to&#3\\
#4&\mapsto&\displaystyle#5\end{array}}
\newcommand{\et}{\textrm{ and }}

\newcommand{\xpr}[1]{"#1"}
\usepackage{stmaryrd} \newcommand{\abs}[1]{\left|#1\right|}
\newcommand{\soit}[1]{\left|\everymath{\displaystyle\everymath{}}\begin{array}{ll}#1\end{array}\right.}
\newcommand{\both}[1]{\left\{\everymath{\displaystyle\everymath{}}\begin{array}{l}#1\end{array}\right.}

\newcommand{\co}[2]{\left\llbracket #1,#2\right\llbracket}\newcommand{\cc}[2]{\left\llbracket #1,#2\right\rrbracket}\newcommand{\oo}[2]{\left\rrbracket #1,#2\right\llbracket}\newcommand{\oc}[2]{\left\rrbracket #1,#2\right\rrbracket}\newcommand{\ci}[1]{\co{#1}\infty}\newcommand{\io}[1]{\oo{-\infty}{#1}}\newcommand{\oi}[1]{\oo{#1}\infty}\newcommand{\ic}[1]{\oc{-\infty}{#1}}
\newcommand{\scc}[2]{_{\cc{#1}{#2}}}\newcommand{\sco}[2]{_{\co{#1}{#2}}}\newcommand{\soo}[2]{_{\oo{#1}{#2}}}\newcommand{\soc}[2]{_{\oc{#1}{#2}}}\newcommand{\sci}[1]{_{\ci{#1}}}\newcommand{\sio}[1]{_{\io{#1}}}\newcommand{\soi}[1]{_{\oi{#1}}}\newcommand{\sic}[1]{_{\ic{#1}}}
\newcommand{\sett}[2]{\left\{\left.#1\vphantom{#2}\right|#2\right\}}
\newcommand{\set}[3]{\sett{#1\in#2}{#3}}
\newcommand{\ie}{\textit{i.e.}\ }
\newcommand{\ipart}[1]{\left\lfloor #1\right\rfloor}


\title{Zigzags in Turing machines\thanks{This work has been supported by ECOS-Sud Project and CONICYT FONDECYT \#1090568.}}
\author{Anah\'i Gajardo\inst1 \and Pierre Guillon\inst2}
\institute{Departamento de Ingenier\'ia Matem\'atica, Universidad de Concepci\'on,
Casilla 160-C, Concepci\'on, Chile
\email{anahi@ing-mat.udec.cl}
\and
DIM - CMM, UMI CNRS 2807, Universidad de Chile, Av. Blanco Encalada 2120,
Santiago, Chile
\email{pguillon@dim.uchile.cl}
}

\bibliographystyle{splncs}

\begin{document}

\maketitle

\begin{abstract}
We study one-head machines through symbolic and topological dynamics.
In particular, a subshift is associated to the system, and we are interested in its complexity in terms of realtime recognition.
We emphasize the class of one-head machines whose subshift can be recognized by a deterministic pushdown automaton.
We prove that this class corresponds to particular restrictions on the head movement, and to equicontinuity in associated dynamical systems.
\end{abstract}

\noindent\textbf{Keywords:} Turing machines, discrete dynamical systems, subshifts, formal languages.

We study the dynamics of a system consisting in a finite automaton (the head) that can write and move over an infinite tape, like a Turing machine.
We use the approach of symbolic and topological dynamics.
Our interest is to understand its properties and limitations, and how dynamical properties are related to computational complexity.

This approach was initiated by K\r urka in \cite{Kurk} with two different topologies: one focused on the machine head, and the other on the tape.
The first approach was further developed in~\cite{Nich,Opro}.
More recently, in \cite{Gaja07,GajaJAC}, a third kind of dynamical system was associated to Turing machines, taking advantage of the following specificity: changes happen only in the head position whilst the rest of the configuration remains unaltered.
The whole evolution can therefore be described by the sequence of states taken by the head and the symbols that it reads.
This observation actually yields a factor map between K\r urka's first dynamical system and a one-sided subshift.

In \cite{Gaja07}, it has been proved that machines with a sofic subshift correspond to machines whose head makes only bounded cycles.
We prove here a similar characterization of machines with a shift that can be recognized by a deterministic pushdown automaton.
Moreover, we establish links between these two properties and equicontinuity in all three spaces.


In the first section, we recall the definitions and fundamental results.
The second section is devoted to defining the different dynamical systems associated to one-head machines, and to stating basic results about equicontinuity within these systems.
In the last section, we define the class of bounded-zigzag machines and state our main results.
\section{Preliminaries}
Consider a finite alphabet , and  to stand either for  or for .
For a finite word , we will note  its length, and index its letters from  to , unless specified otherwise.
We denote  the set of words on  of length at most .
If  and ,  will denote the closed interval of integers , , etc.
A point  will be called \emph{configuration}.
For a configuration or a word , we define .
 will denote the disjoint union of two sets  and .
\subsection{Topological dynamics}\label{sec:top}
A \emph{dynamical system} (DS) is a pair  where  is a metric space and  a continuous self-map of . Sometimes the space will be implicit.

The orbit of a point  is the set of the  for all \emph{iteration} .
A point  is called \emph{preperiodic} if there exist two naturals ,  such that .
If  and  are minimal, then  is called the \emph{transient} and  the \emph{period}.
When ,  is called \emph{periodic}.


A point  is \emph{isolated} if there is an  such that the ball of radius  and center  contains only .
A point  is \emph{equicontinuous} for  if, for any , there exists some  such that, for any  with , we have that, for all , .
The DS  is \emph{equicontinuous} if, for any , there exists some  such that, for any  with , we have that, for all , .
When  is compact, this is equivalent to having only equicontinuous points.
The DS  is \emph{almost equicontinuous} if it has a residual set of equicontinuous points.

A DS  is a \emph{factor} of a DS  if  for some continuous onto map , then called a \emph{factor map}.
\subsection{Subshifts}
We can endow the space  of {configurations} with the product of the discrete topology of .
It is based on the cylinders , where ,  and ; this notation shall be extended to semi-infinite words.
If ,  and , we note .

This topology corresponds to the metric .
In other words, ; two points are \xpr{close to each other} if they coincide \xpr{around position 0}.
It is easy to extend this metric to spaces  and .
In that setting,  and  are compact, but  is not.



The \emph{shift} map is the function  defined by .
A \emph{subshift}  is a closed subset of  which is also invariant by . It can be seen as a compact DS where the map is .

A subshift  is characterized by its \emph{language}, containing all finite patterns that appear in some of its configurations: . We denote .
If the language  is regular, then we say that  is \emph{sofic}.
Equivalently, a sofic subshift can be seen as the set of labels of infinite paths in some finite arc-labeled graph; this graph basically corresponds to the finite automaton that recognizes its language, without initial nor terminal state.

Any subshift can also be defined from a set of forbidden finite patterns  by .
If  can be chosen to be finite, then  is a subshift \emph{of finite type} (SFT).

A DS  on  is completely determined by the family of its factor subshifts, \ie the factors which are also subshifts in some alphabet.
Up to some letter renaming, all factor subshifts of  are of the form , where  is a finite partition of  into closed open sets, and  denotes the unique element of this partition which contains .
\subsection{Deterministic pushdown automata}
\begin{defi}
A \emph{deterministic pushdown automaton} (DPDA) is a tuple\break  where  is the \emph{input alphabet},  is the \emph{set of states },  is the \emph{stack alphabet},  is the \emph{stack bottom},  is the \emph{initial state},  is the \emph{subset of terminal states } and  is the \emph{transition function} such that: if , then  contains exactly one , which is on its end, and if  with , then  does not contain any .

An (infinite) arc-labeled graph  is associated to the automaton.
Its set of vertices is , and there exists an arc from  to  labeled  if and only if  and . The word  is called the \emph{stack content}.

The language  recognized by the automaton consists of all words  in  such that there exists a finite path in  with label , starting on vertex  and ending in some vertex  with . A subshift is recognized by the automaton if its language is recognized by the automaton.
\end{defi}
\newcommand{\pile}[2][]{(o_{#2},#1\mu^{#2})}
\newcommand{\pilo}[2][]{(o_{#2},#1\mu_0^{#2})}
\section{Turing Machines}
In this article, a Turing Machine (TM) is a triple , where  and  are the finite \emph{tape alphabet} and \emph{set of state}, and  the \emph{rule}.
We do not particularize any halting state.
We can see the TM as evolving on a bi-infinite tape.
The phase space is .
Any element of  is called a configuration and represents the state of the tape, the state of the head and its position.
We consider here the topology introduced in Section \ref{sec:top}.
Thus, the farther the head is from the center, the less important become the read symbols, but the head state and position remain important.
On this (non-compact) space,  by  if  gives the corresponding DS.
We can extend the shift function to TM configurations by , and it clearly commutes with .

We can represent the head state and position by adding a ``mark'' on the tape.
If we want a compact space, this corresponds to the following phase space:

where the head position is implicitly given by the only cell with a symbol in , and the function  is defined by , where  and , and  if  does not contain any symbol in .
With the topology of  as a subshift of , the head state and movement are less important when the head is far from .
This model corresponds to the TM \emph{with moving head} defined by K\r urka in \cite{Kurk}, which highlights the tape configuration. It is a particular case of cellular automaton, \ie based on some uniformly-applied local rule.
We can intuitively see a continuous injection  such that  and .

Focusing on the movements and states of the head, \cite{Kurk} also defines the system \emph{with moving tape}  on the (compact) space\break  by  if 
Here the head is assumed to be always at position , and the tape is shifted at each step according to the rule.
There is a continuous non-injective surjection  such that .

Finally, we can have a vision centered on the head and which emphasizes only the relevant part of the configuration, as in \cite{Gaja07,GajaJAC}.
The system  is the one-sided subshift on alphabet , which is the image of the factor map  defined by  if .
In other words, it represents the sequence of pairs corresponding to the successive states of the head and the letters that it reads.
{
}
Similarly, we will note  the one-sided subshift on alphabet  which is the image of the factor map  defined by . Unlike , this subshift does not always contain the relevant information, since the head can be completely absent.
\subsection{Equicontinuous configurations}\label{ss:eqpt}
Topological notions can actually formalize various types of head movements.
One first example is equicontinuity of the DS .
It is strongly related with periodicity, as the next remark establishes.
This is natural since the symbol that the head reads in  is always at position .
Hence, if the head visits an infinite number of cells, say to the right, any perturbation on the initial configuration will get to position , and thus will become largely significant for this topology.
We conclude the following. \begin{rem}\label{r:eqpt}
Let  be a configuration and  a machine over . The following statements are equivalent:
\begin{enumerate}
\item The head position on  is bounded.
\item  is \textbf{preperiodic} for .
\item  is \textbf{preperiodic} for .
\item  is \textbf{equicontinuous} for .
\item  is \textbf{preperiodic} and \textbf{isolated} --\ie \textbf{equicontinuous}-- in .
\end{enumerate}
Moreover, if one of the above occurs, then  is preperiodic for ,  is equicontinuous for  and  is equicontinuous for .
The set of equicontinuous configurations for  is a union of cylinders of .
\end{rem}
If  is preperiodic for , then  is also periodic (for ), but  need not be periodic for .
For example, a machine that simply moves to the left on every symbol will produce a periodic point for  if the initial configuration  is spatially periodic.
From the previous remark, such a point is not equicontinuous, and  is a non-isolated periodic point in , because any perturbation of  will produce a neighbor of  in .
Periodic points for  generate isolated periodic points in  because, once the system falls in the periodic behavior, its future is fixed.

Preperiodicity in  also implies equicontinuity in , but  may have other equicontinuous points.
The previously mentioned machine which always go to the left produces equicontinuous points for  which are not equicontinuous nor preperiodic for .

The following proposition states that the equicontinuity of preperiodic configurations is transmitted to factor subshifts of , which will be helpful in the sequel.
\begin{prop}\label{p:shper}
 If  is a preperiodic word involving the machine head infinitely often, then it is isolated.
\end{prop}
\begin{proof}
 We can assume that  is periodic, and then include the transient evolution in a larger ball.
 Let  be the period of ; let us prove that the ball  of  is equal to .
 Let  and . It can be seen that the head computing over  always remains between the positions  and , which correspond to at most  distinct finite patterns. Hence there are  such that ; as a consequence  is -periodic. Together with , they are both -periodic and coincide on their first  letters, since . As a conclusion, .
\qed\end{proof} \subsection{Preperiodic machines}
When all the configurations are uniformly preperiodic, we say that the system is preperiodic, \ie there exist ,  such that .
In the present case, global preperiodicity of each of the considered systems comes directly from local preperiodicity of ; and it is equivalent to global equicontinuity of each of the systems as the next theorem establishes.
\begin{teo}
Considering a machine, the following statements are equivalent:
\begin{enumerate}
 \item\label{bounded} The head position is (uniformly) bounded.
\item Any configuration of  (or , ) is {\bf preperiodic}.
 \item\label{prep} (or , , , ) is {\bf preperiodic}.
 \item\label{equi} (or , , , ) is {\bf equicontinuous}.
 \item\label{finite}  (or ) is {\bf finite}.
\end{enumerate}
\end{teo}
\begin{proof}We give only a sketch of the main implications.
\begin{itemize}
 \item It is quite obvious from the commutation diagrams that the preperiodicity of ,  and  are equivalent, and they imply those of  and .
They also imply, from Remark~\ref{r:eqpt}, that the head position is bounded.
 \item Clearly, the equicontinuity of  and  are equivalent.
 \item It is known from cellular automata theory that the equicontinuity of , its preperiodicity, that of all its configuration and the finiteness of  are equivalent.
 \item If the head position on all configurations is bounded, then from Remark~\ref{r:eqpt} they are all equicontinuous for .  being compact,  is equicontinuous.
 \item It is obvious that  is finite if and only if the head reads a bounded part of the initial configuration.
\qed
\end{itemize}
\end{proof}
\subsection{Sofic machines}
Now we allow computations where the head can go arbitrarily \xpr{far}, but without ever making \xpr{large} movements back.
\begin{defi}
 We say that a machine makes a \emph{right-cycle} (\emph{left-cycle}) of width  over a configuration  and a cell  if there exist time steps  such that the head position is  at time  and , and is  () at time .
\end{defi}
In this section, we consider machines whose cycles have bounded width, \ie there exists an integer  such that the machine cannot make any cycle wider than .
These machines have been studied in~\cite{GajaJAC,Gaja07}, where it was proved that they are exactly the machines for which  is sofic.
\begin{teo}~\label{t:sofic}
Considering a machine, the following statements are equivalent:
\begin{enumerate}
 \item\label{sofic}  is {\bf sofic}.
 \item\label{i:tmheq} All configurations of  that contain the head are {\bf equicontinuous}.
\end{enumerate}
\end{teo}
\begin{proof}~\begin{itemize}
 \item[\ref{sofic}\ref{i:tmheq}]
 From~\cite{Gaja07}, we know that there exists an integer  such that the machine cannot make any cycle wider than , and let  a configuration containing the head within , for some .
 Let us show that if , then for every  we have .
 Let us remark that while the head is inside , we necessarily have .
 Let us suppose that there exists  such that the head is outside  at time  and let us take this  minimal.
 Then the heads of  and  are outside .
 At some moment, the head has gone from  to  (or from ); if it comes back to , it would make a cycle.
 Therefore, the head cannot come back to , and this is true both for  and , and we have the result.
\item[\ref{i:tmheq}\ref{sofic}]
  Conversely, assume that the head can do arbitrarily wide right-cycles in cell , \ie for each  there exists a cylinder  of  with , with , such that over each configuration of , the head starts at , it visits the whole interval  and comes back to cell .
 Let us take some configuration  in each cylinder .
 By compactness, the sequence  admits an adhering value , on which the head necessarily goes infinitely far to the right without ever coming back to cell .
 By construction, for any , there is some  such that the configuration .
 But there exists a time  such that  has the head in cell , whilst  has not; hence  is not equicontinuous.
 \qed
\end{itemize}
\end{proof}
From \cite{GajaJAC}, any of the former properties implies that any configuration is either preperiodic or gives rise to a movement of the head arbitrarily far in some direction, but the converse is not true.
Any configuration of  is preperiodic, hence this subshift is numerable.
\section{Bounded-zigzag machines}
Whilst the sofic machines did not allow any large cycle, we can wonder what happens when allowing a single one, or a finite number of these.
The first step is to allow one cycle of arbitrary width but to forbid two overlapped unbounded cycles (zigzags).
We remark that two independent cycles, each on a different direction, are allowed in this case.
\begin{defi}
 We say that a machine makes a \emph{right-zigzag} (resp., \emph{left-zigzag}) of width  over a configuration  and a cell , if there exist time steps  such that the machine position is  at times  and , and  (resp., ) at time .
We say that a machine is \emph{bounded-zigzag} if the maximal width of the zigzags that it can make is finite.
\end{defi}
\subsection{Complexity of }
While bounded cycle machines have a sofic shift , the bounded-zigzag machines have a subshift recognized by a deterministic pushdown automata.
The words of  contain information about the tape symbols and the head state.
From this data, it is possible to deduce the tape symbol of the visited cells and the relative position of the head at each time step.
In order to recognize , we can register this information and check its coherence at each time step.
When the width of the cycles is bounded, we only need to register a finite part of the tape (bounded-cycle machines have a subshift that can be recognized by a finite state automaton).

When only one ``wide'' cycle can be done, we can register the information in a stack, from which it can be read exactly once (and is lost forever once read).
This corresponds to the fact that the cells registered in the stack cannot be visited any more and zigzags cannot be allowed. The complete proof can be found in the appendix.\begin{teo}\label{t:zigstack}
A machine  is bounded-zigzag if and only if  is recognized by some deterministic pushdown automaton.
\end{teo}
\subsection{Complexity of }
If we now adopt a point of view fixed on the tape ---- rather than the head, a cycle in the subshift corresponds to a waiting time during which cell  does not change.
We can adapt the previously built DPDA so that it recognizes exactly these waiting words between two visits of the head.
The key point here is that these languages are unary, and unary context-free languages are regular (see for example~\cite{Gins62}), and thus they can be recognized with a finite automaton.

When the machine is bounded-zigzag, the head can make at most one long cycle by side.
The rest of the time, the head is either moving closer to or farther from cell , or staying in some finite window around cell .
All of these behaviors can be recognized by a finite automaton, thus the language of  is regular.
Therefore, we obtain a surprising reduction in language complexity when changing the point of view: if  is recognized by some DPDA, then  is sofic. The complete proof can be found in the appendix.Note that, up to a rescaling of the tape alphabet, all factor subshifts can be reduced to the case of .
\begin{teo}\label{t:zigsof}
 For any bounded-zigzag machine, all the factor subshifts of  are sofic.
\end{teo}
The converse of this theorem is false: we can construct a machine with a tape with  levels, where the head vertically shifts down the content of each level while moving right.
It rebounds when it finds a wall in the lowest level (which is erased in the same way), and does the same in the opposite direction.
We can see that the machine can make arbitrarily wide -zigzags, each of independent length, in such a way that the factor subshifts of  are sofic.

Nevertheless, we can prove that this kind of construction is possible only with a bounded .
Let us introduce this formally.
\begin{defi}
 We say that a machine makes an \emph{-cycle} of width  over configuration  and cell , if there exist  time steps  such that the head is in position  at time  and outside  at time , for each .
We say that the machine is -bounded-cycle if there is some  such that the head cannot make -cycles of width larger than .
\end{defi}
When  is sofic, the machine is -bounded cycle.
Considering some machine , we denote  the maximum  such that the machine can make an -cycle of width  over configuration .
Clearly,  is -bounded cycle if and only if for some ,  is bounded by .


Let us call  the set of time steps for which the head has position  when computing over configuration .
This set is linked to cycles by the following intuitive observation.
\begin{prop}\label{p:zzbound}
If  is an -bounded-cycle machine, then there exists  such that for any cell  and any non-preperiodic configuration , .
\end{prop}
\begin{proof}
Let  be such that , and  such that  -- the case  can be obtained by shifting.
Consider  with .
If we consider an -cycle over  in cell , we can see that there exist  such that for any , the head goes beyond  or  between time steps  and , but not between (possibly equal) times  and .
This means that  is the last time that the head is in  before going beyond .
Let  and , in such a way that , where  for . 
There are  such intervals, so one of them, say , has at least  elements; this is all the more the case for .
Hence, between time steps  and  there are at least  consecutive time steps in  such that the head stays within the interval of cells .
As a result, there are  with  and , which implies that  is preperiodic.
\qed\end{proof}
\begin{teo}
If  is sofic, then  is -bounded-cycle for some .
\end{teo}
\begin{proof}
Assume that  is recognized by some finite automaton with  states, and that there exists some configuration  on which the machine makes some -cycle of width .
Let  be as in the definition of -cycles, and .
Let  be the corresponding path of the finite automaton.
We can see that there are  such that , hence there is some periodic infinite word  corresponding to the path  that repeats the cycle .
From Proposition~\ref{p:shper},  is isolated.
As a consequence,  is the only path to start from .
Therefore, its vertices are all different, and , but in this case the head does not have the time to go beyond  between these two iterations, which is a contradiction.
We have proved that  is -bounded-cycle.
\qed
\end{proof}
Here, too, the converse is false, since it is easy to build a machine doing a given number of arbitrarily wide rebounds on specific wall characters before stopping.
The language of such a machine cannot be regular because the time intervals between two rebounds are not independent.
\subsection{Almost equicontinuity}
We have already seen that in sofic machines, almost all configurations of  are equicontinuous.
It is still so when allowing -cycles, though in this case there are some configurations with head which are not equicontinuous -- recall that Theorem~\ref{t:sofic} is an equivalence.
\begin{teo}
 If  is an -bounded-cycle machine for some , then  is almost equicontinuous.
\end{teo}
\begin{proof}
 By compactness of the space, it is enough to prove that for any cylinder  and any , there exist some  and some  such that for any  and any , .
Let  be as in the definition of -bounded-cycle machine,  a cylinder of  and .
If  contains some preperiodic configuration with the head, then we can easily find  thanks to Remark~\ref{r:eqpt}.
 Otherwise, let us consider some configuration  (with the head) maximizing , which is finite thanks to Proposition~\ref{p:zzbound}.
 Let  be such that  and the interval  contains all the cells visited, when computing from , up to time step .
 Then we can see that any configuration  has the same evolution as  until this time step, and that after that, its head cannot visit cell  nor , otherwise it would contradict the maximality of .
 We can deduce that the head of  (then also ) is outside  after iteration , otherwise it would be trapped between  and  and would become periodic.
 We observe, then, that the cells of  evolve exactly in the same way for configurations  and .
\qed\end{proof}
The converse is untrue: imagine a machine whose head rebounds between two walls, each time shifting them to the left.
Every configuration where the head starts enclosed between two walls is equicontinuous.
Any finite pattern can be extended by adding walls to enclose the head, therefore equicontinuous points are dense, but the head can make an arbitrary number of arbitrarily wide cycles.
\section*{Conclusion}
The complexity of the Turing machine will always be very hard to understand.
In our attempt to treat this issue through the theories of topological and symbolic dynamics, we have found interesting relations between:
\begin{itemize}
 \item the head movements that can be observed during the computation;
 \item the density of equicontinuous points;
 \item the language complexity of the associated subshifts  and .
\end{itemize} 
These relations introduce a new point of view on how computation is performed.
In addition to generalizing them to more machines, the next step would be to study Turing machines as computing model by introducing a halting state, and to link all of these considerations to the result itself of the computation, and eventually the temporal or spatial complexity of the computation.

\bibliography{Xbib}

\newpage

\appendix\section*{Proofs}
The Ogden Lemma \cite{Odge} is a well-know generalization to the case of pushdown automata of the pumping lemma on finite automata. It can be expressed on paths of the graph as follows.
\begin{lem}\label{l:bomba}
 Consider a DPDA , and  some path of its graph and  a subset of distinguished positions of size .
Then there exist four positions  and such that:
\begin{enumerate}
 \item ;
 \item ;
 \item ;
 \item .
 \item  is also a valid path of the graph, where ;
 \item ;
 \item ;
 \item Either  or .
\end{enumerate}
\end{lem}


If  is a machine with rule  and , then we note ,  and  if .
If , then we can define the corresponding cylinder in space :

Let  denote the empty word.

Theorem~\ref{t:zigstack} comes from the following lemmas.

\begin{lem}\label{l:centerDFA}
Let  be a fixed natural number and  a Turing machine.
Given two partial configurations , there exists a DFA  that recognizes the language  of the words  for ,  such that  and for any  the head position of  is in .

Moreover, if  satisfies the conditions of , then every  also does, with the same time .
\end{lem}

The language  can be either empty, a singleton or, when  is periodic for , infinite.
The automaton  simply simulates  by loading  on its memory, and making the partial configuration over cells  evolve simply by applying the machine rule.
The next lemma corresponds to similar and more evolved proof.

\begin{lem}\label{l:LRcycleDPDA}
Let  be a fixed natural number and let  be a Turing machine that cannot do -zigzags of width .
If we have three partial configurations  such that  and , then there exists a DPDA  that recognizes the language  of the words  for ,  such that  and for any , the head position of  is strictly positive.

Moreover, if  satisfies the conditions of , then every  such that  also does, with the same time .

Symetrically, if  such that  and , then there exists a DPDA  that recognizes the language  of the words  for ,  such that  and for any , the head position of  is strictly negative.

Moreover, if  satisfies the conditions of , then every  such that  also does, with the same time .
\end{lem}
\begin{proof}
We will do the proof only for .
The automaton registers the states of the tape and updates them at each step.
The states of the cells at the right of the head will be registered in the internal state of the automaton, while the states of the cells at the left will be stocked in the stack.
The position of the head is given by the stack depth; in this way the head is always reading the symbol .

We define actually an automaton in a slightly different model than previously defined. The initial and terminal states actually involve the content of the stack: we initially push a given finite word into the stack, and to accept a word, we verify if both the terminal state and the stack content are in some given finite sets. It is easy to see, by considering some complex encoding in the stack alphabet , that this model can be simulated by the previous one.
The automaton  has input alphabet , states set , stack alphabet ; its initial state is  and initial stack content the mirror of ; it terminates when the pair composed of the internal state and the stack content is in ; its transition function  is defined by:




Let us denote by  and  the respectively stack content and internal state at iteration .
\begin{itemize}
\item
We will prove by induction on , that if  satifies the conditions of , then .
\\
For  it is clear, because .
Let us suppose that it is true for a given , and let us prove it for .
\\
If , the head is reading the symbol  and is in state , hence the only input accepted is .
In this case, the head will pass to state  and will move to .
If  the automaton must push  and ``erase'' .
If  the automaton must replace  by , pop a symbol and concatenate it to .
\\
If , the automaton will accept  only if ; in this case it will work, assuming that .
\item
Now we need to prove that every word recognized by  is in fact in .
We use recurrence to define the configuration  that certifies this.
The first condition is that , it follows from the first verification: .
Let us suppose that we have defined  such that  for every  and that the set of cells visited by the head is  for some  and .
Let us prove that the same is true for  for a suitable .
We can note that the condition  holds for any  satisfying .
\\
If , then the automaton will accept any pair  with  if cell  has already been visited; the value of  is important and cannot be defined to be . But if  was visited, its value was registered in  for somme , and it has been erased because the head has moved to  in some moment (then ).
The existence of  such that  insures that the head has moved from  to , which means that, the head has made a -zigzag to the right between cells  and , with is forbiden by hypothesis. Hence  has not been visited before () and we can define .
\\
When , we know that the value of cell  in  is . The automaton will only accept the pair .
This and the former construction insure that .
\qed\end{itemize}\end{proof}

\begin{proof}[of Theorem~\ref{t:zigstack}]

()
Since  does not regards the head position, we can suppose that the head starts at 0.
Let  be a configuration.
\begin{itemize}
\item If the head does not exit the interval  during the whole evolution, then only  is needed to recognize , we conclude that , for every .
\item If the head exits  for the first time at iteration , by the right side, and never comes back to cell  after that, then , for any .
\item If the head exits  for the first time at iteration , by the right side, comes back to  at iteration , and never exit  again, then 
, for any .
\item If the head exits  for the first time at iteration , by the right side, comes back to  at iteration , and exits  again at  and does not ever come back to , then  is in the concatenation of the languages , ,  and , for any .
\item If the head exits  for the first time at iteration , by the right side, comes back to  at iteration , exits  again at , and comes back to  at , then  is in the concatenation of , , ,  and , for any .
\end{itemize}

The analogous case when the head first exits  through cell  can be treated in a similar way. We conclude that for any  and any , the word  is in the language

where , , , and .
This language is recognizable by a DPDA since it is a concatenation and union of languages which are recognizable by DPDAs, thanks to Lemmas~\ref{l:centerDFA} and~\ref{l:LRcycleDPDA}.

We have to prove now that this union of languages contains only words of . The proof is similar for each of the listed languages; we will develop it only for
.

From Lemma~\ref{l:centerDFA}, we know that if , then any  will satisfy  if the head position at time  is .
From Lemma~\ref{l:LRcycleDPDA}, if , then there exists  and  such that  and .
We define , which will satisfy .
From the same lemmas, we know that the values of  on  are still ``free'' and  gives , where  is the instant in which the head reaches the cell  for the first time.

We can suppose that  is not empty -- otherwise the result is trivial. Then there exists  such that  and .
The values of  over  are not important and we can fix them to those of , or in other words, to define . We obtain

This completes the proof.

()
Let us assume that the language of  is recognized by some DPDA , that  is as in Lemma~\ref{l:bomba}, and that the machine can do a -zigzag of width ;  we can easily find some configuration  with time steps  such that the machine visits cell  at time , cell  at times  and , and cell  at time .
It can also be assumed that the zigzag is minimal, in the sense that no other configuration satisfies the condition with a lower . Moreover, we can assume that  is the last time when cell  is visited before , and  is the first time when cell  is visited. Note that . Let  the corresponding path in the graph of .

The key point of the proof is that, thanks to the determinism of the automaton, given , the  cell is visited by the head for the fist time if and only if the corresponding vertex in the graph of  has out-degree more than .
Since the out-degree of a vertex  of  depends only on .
Let  be the set of vertices with out-degree , and  be the subset of  corresponding to vertices whose unique out-neighbor is not in  and such that this unique transition corresponds to a left movement of the head. These vertices represent cells which are at the left extremity of some visited zone.
For instance, note that the vertices  are in  since the corresponding visited cells are between  and , and  is the first vertex of the path  to belong to .

If we apply Lemma~\ref{l:bomba} with , we obtain time steps  such that 

is a valid path in the graph of , \ie it can be obtained from some configuration , which we can suppose to have the head in cell  without loss of generality.

\begin{itemize}
\item First, suppose . 
Since , we must have .
Moreover, the nonemptiness of  gives .
The vertices of  are in , then  is the only subpath of this length starting at .
Thus, .
In particular, .
From the lemma, ; the same automaton rule is applied in both vertices, and since  is not in V, we conclude that  .
This results in , and from the Ogden Lemma we get , which is a contradiction.

\item Now suppose that . As no vertex of  is in  before , we can see that, in this path too, the vertex  corresponds, at time , to the first visit of cell  -- at the first time we go more to the left than the visited zone.
Since both paths coincide after that, the head does the same movements, and we obtain that its position at the last vertex  of path  is . But, from path , we know that , so on  too the machine had already visited cell  before arriving on this vertex. It could not be after time , since from then on we have followed the same positions as in , hence  represents a -zigzag; from the last point of the lemma, it is shorter than , so  satisfies the construction hypotheses of  but contradicts its minimality.
\qed
\end{itemize}
\end{proof}



\begin{proof}[of Theorem~\ref{t:zigsof}]
Let us define the following languages.




It is a context-free language since it is the transformation of a context-free language through a letter morphism.
It is also a regular language because it uses a single symbol .
If  and  satisfy the conditions of , then .

We also define the language  of the words  with  and  such that  and for any , the head of  is in .

It is recognized by an automaton that simulates  and accepts a pair  if and only if the current head position is , and  and  match the simulation.

If the head starts at cell , the analogous concatenation and union of the s, s and s would represent . But if the head does not start at , we need to consider, for , the language  of the words  for which there exists  with  and for any , the head of  is not in cell .
 represents the set of sequences of states observed at cell  until the head reaches it, when the partial configuration  is observed in .
 is always an nonempty ``interval'', \ie  for some  which may be  -- if  is a ``garden of Eden''.



Since  is either finite or equal to , it can be recognized with a DFA .
 will be the concatenation and union of s and the other languages.

Globally, we obtain that  is in the following union:

\end{proof}

\end{document}
