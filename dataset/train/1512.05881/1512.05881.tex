


Finite automata, of any kind, are widely used for their algorithmic
properties in many fields of computer science like model-checking,
pattern matching and machine learning. Developing new efficient algorithms
for finite automata is therefore a challenging problem still addressed by
many recent papers. New algorithms are frequently motivated by improvement
of worst cases bound. However, several examples, such as sorting algorithms,
primality testing or solving linear problems, show that worst case
complexity is not always the right way to evaluate the practical performance
of an algorithm. When benchmarks are not available, random testing, with a
controlled distribution, represents an efficient mean of performance
testing. In this context, the problem of uniformly generating finite
automata is a challenging problem.


This paper tackles with the problem of the uniform random generation of
real-time deterministic pushdown automata. Using classical combinatorial
techniques, we expose how to extend existing works on the generation of
finite deterministic automata to pushdown automata. More precisely, we show
in Section~\ref{sec:random} how to uniformly generate and enumerate (in the
complete case) accessible real-time deterministic pushdown automata. In
Section~\ref{sec:PDA}, it is shown that using a rejection algorithm it is
possible to efficiently generate pushdown automata that don't accept an
empty language. The influence of the accepting condition (final state or
empty-stack) on the reachability
of the generated pushdown automata is also experimentally studied  in
Section~\ref{sec:PDA}.



\paragraph{Related work.}  The 
enumeration of deterministic finite automata has been first investigated
in~\cite{Vyssotsky} and was applied to several subclasses of deterministic
finite
automata~\cite{Korshunov,DBLP:journals/eik/Korshunov86,robinson,DBLP:journals/dam/Liskovets06}. 
The uniform random generation of accessible deterministic complete automata
was initially proposed in~\cite{thesecril} for two-letter alphabets and the
approach was extended to larger alphabets
in~\cite{DBLP:journals/tcs/ChamparnaudP05}. Better algorithms can be found
in~\cite{DBLP:journals/tcs/BassinoN07,DBLP:conf/stacs/CarayolN12}. The
random generation of possibly incomplete automata is analyzed
in~\cite{incomplet}. The recent paper~\cite{DBLP:conf/wia/CarninoF11}
presents how to use Monte-Carlo approaches to generate deterministic acyclic
automata. As far as we know, the only work focusing on the random generation
of deterministic transducers is~\cite{DBLP:journals/tcs/HeamNS10}. This work
can be applied to the random generation of deterministic real-time
pushdown automata that can be possibly incomplete. However the requirement to fix the size of the stack operation on each transition, represents a major restriction.
The reader interested in the random generation of deterministic automata is
 referred to the survey~\cite{DBLP:conf/mfcs/Nicaud14}. 






\section{Formal Background}\label{sec:bg}


We assume that the reader is familiar with classical notions on formal
languages. For more information on automata theory or on pushdown automata
the reader is referred to~\cite{Hopcroft} or to~\cite{Saka}. For a general
reference on random generation and enumeration of combinatorial structures
see~\cite{DBLP:journals/tcs/FlajoletZC94}. For any word $w$ on an alphabet $\Sigma$, $|w|$ denotes
its length. The empty word is denoted~$\varepsilon$.
The cardinal of a finite set $X$ is denoted $|X|$.

\paragraph{Deterministic Finite Automata.}
A {\it deterministic finite automaton} on $\Sigma$ is a tuple
$(Q,\Sigma,\delta,q_{\rm init},F)$ where $Q$ is a finite set of states,
$\Sigma$ is a finite alphabet, $q_{\rm init}\in Q$ is the initial state,
$F\subseteq Q$ is the set of final states and $\delta$ is a partial function
from $Q\times\Sigma$ into $Q$. If $\delta$ is not partial, i.e., defined for
each $(q,a)\in Q\times\Sigma$, the automaton is said {\it complete}. A
triplet of the form $(q,a,\delta(p,a))$ is called a {\it transition}. A
finite automaton is graphically represented by a labeled finite graph whose
vertices are the states of the automaton and edges are the transitions. A
deterministic finite automaton is {\it accessible} if for each state $q$
there exists a path from the initial state to $q$. Two finite automata
$(Q_1,\Sigma,\delta_1,q_{\rm init1},F_1)$ and $(Q_2,\Sigma,\delta_2,q_{\rm
init2},F_2)$ are isomorphic if they are identical up to the state's names,
formally if there exists a one-to-one function $\varphi$ from $Q_1$ into
$Q_2$ such that (1) $\varphi(q_{\rm init1})=q_{\rm init2}$,  (2)
$\varphi(F_1)=F_2$, and (3) $\delta_1(q,a)=p$ iff
$\delta_2(\varphi(q),a)=\varphi(p)$.

\paragraph{Pushdown Automata.}
A {\it real-time deterministic pushdown automaton}, RDPDA for short, is a
tuple $(Q,\Sigma,\Gamma,Z_{\rm init},\delta,q_{\rm init},F)$ where $Q$ is a
finite set of states, $\Sigma$ and $\Gamma$ are finite disjoint alphabets,
$q_{\rm init}\in Q$ is the initial state, $F\subseteq Q$ is the set of final
states, $Z_{\rm init}$ is the initial stack symbol and $\delta$ is a partial
function from $Q\times(\Sigma\times\Gamma)$ into $Q\times\Gamma^*$. If
$\delta$ is not partial, i.e., defined for each $(q,(a,X))\in Q\times(\Sigma\times\Gamma)$, the
RDPDA is said to be {\it complete}. 
A triplet of the form $(q,(a,X),w,p)$ with
$\delta(q,(a,X))=(p,w)$ is called a {\it transition} and $w$ is the {\it
output} of the transition. The {\it output size} of a transition
$(q,(a,X),w,p)$ is the length of $w$.  The {\it output size} of
a RDPDA is the sum of the sizes of its transitions.
The {\it underlying automaton} of an
RDPDA, is the finite automaton $(Q,\Sigma\times\Gamma,\delta^\prime,q_{\rm
init},F)$, with $\delta^\prime(q,(a,X))=p$ iff $\delta((q,(a,X)))=(p,w)$
for some $w\in\Gamma^*$. An RDPDA is {\it accessible} if its underlying
automaton is accessible. A transition whose output is $\varepsilon$ is
called a {\it pop transition}. An example of a complete accessible RDPDA is
depicted in Fig.~\ref{figAutoPile1}. The related underlying  finite
automaton is depicted in Fig.~\ref{figUnder}.


\begin{figure}[t!]
\centering
\subfigure{
\begin{tikzpicture}
\node (0) [state,fill=black!20,initial,initial text=] at (0,0) {$0$};
\node (1) [state,fill=black!20, accepting]at (3,0) {$1$};

\path[->,>=triangle 90] (0) [bend left] edge[above] node {$(a,X),Z$} (1);
\path[->,>=triangle 90] (1) [bend left] edge[below, pos=0.3] node
     {\begin{tabular}{c}$(b,X),\varepsilon$\\$(a,Z),XZX$\end{tabular}} (0);

\path[->,>=triangle 90] (0) [loop above] edge[above] node
     {\begin{tabular}{c}$(a,Z),ZZX$\\$(b,Z),ZX$\end{tabular}} ();
\path[->,>=triangle 90] (0) [loop below] edge[below] node
     {$(b,X),X$} (0);

\path[->,>=triangle 90] (1) [loop above] edge[above] node
     {\begin{tabular}{c}$(a,X),XX$\\$(b,Z),\varepsilon$\end{tabular}} ();


\node (txt) at (7.5,0.3){
$\begin{array}{rcl}
Q &= & \{0,1\}\\
\Sigma&=&\{a,b\}\\
\Gamma&=&\{Z,X\}\\
\delta&=&\{(0,(a,X))\mapsto (1,Z),(0,(b,X))\mapsto (0,X)\\
&&(0,(a,Z)) \mapsto (0,ZZX),(0,(b,Z))\mapsto (0,ZX)\\
&&(1,(a,X))\mapsto (1,XX), (1,(b,X)) \mapsto (0,\varepsilon)\\
&&(1,(a,Z))\mapsto(0, XZX),(1,(b,Z))\mapsto (1,\varepsilon)\\
q_{\rm init}&=&0,\quad Z_{\rm init}=Z,\quad F=\{1\}
\end{array}$};
\end{tikzpicture}
}
\caption{$P_{\rm toy}$, a complete RDPDA.\label{figAutoPile1}}
\end{figure}

A {\it configuration} of a {RDPDA} is an element of $Q\times\Gamma^*$. The
{\it initial configuration} is $(q_{\rm init},Z_{\rm init})$. Two
configurations $(q_1,w_1)$ and $(q_2,w_2)$ are $a$-consecutive, denoted
$(q_1,w_1)\models_a(q_2,w_2)$ if the following conditions are satisfied:
\begin{itemize}
\item $w_1\neq \varepsilon$, and let $w_1=w_3X$ with $X\in \Gamma$,
\item  $\delta(q_1,(X,a))=(q_2,w_4)$ and $w_2=w_3w_4$.
\end{itemize}
Two configurations are {\it consecutive} if there is a letter $a$ such that there
are $a$-consecutive. A state $p$ of a RDPDA is {\it reachable} if there
exists a sequence of consecutive configurations $(p_1,w_1),\ldots,(p_n,w_n)$
such that $(p_1,w_1)$ is the initial configuration and $p_n=p$. Moreover, if
$w_n=\varepsilon$, $p$ is said to be {\it reachable with an empty stack}. A
RDPDA is {\it reachable} if all its states are reachable. Consider for
instance the RDPDA of Fig.~\ref{fig:acceptation}, where the initial stack
symbol is $X$. State $3$ is not reachable since the transition from $0$ to
$3$ cannot be fired. State $1$ is reachable with an empty stack. State $2$ is
reachable, but not reachable with an empty stack. Note that a reachable
state is accessible, but the converse is not true in general: accessibility
is a notion defined on the underlying finite automaton. 

 The configurations $(1,XZ)$ and $(2,XXZX)$ are $a$-consecutive on the RDPDA
depicted in Fig.~\ref{figAutoPile1}. 

\begin{figure}[h!]
\centering
\subfigure{
\begin{tikzpicture}
\node (0) [state,fill=black!20,initial,initial text=] at (0,0) {$0$};
\node (1) [state,fill=black!20, accepting]at (3,0) {$1$};

\path[->,>=triangle 90] (0) [bend left] edge[above] node {$(a,X)$} (1);
\path[->,>=triangle 90] (1) [bend left] edge[below, pos=0.3] node
     {\begin{tabular}{c}$(b,X)$\\$(a,Z)$\end{tabular}} (0);

\path[->,>=triangle 90] (0) [loop above] edge[above] node
     {\begin{tabular}{c}$(a,Z)$\\$(b,Z)$\end{tabular}} ();
\path[->,>=triangle 90] (0) [loop below] edge[below] node
     {$(b,X)$} (0);

\path[->,>=triangle 90] (1) [loop above] edge[above] node
     {\begin{tabular}{c}$(a,X)$\\$(b,Z)$\end{tabular}} ();


\node (txt) at (7.5,0.3){
$\begin{array}{rcl}
Q &= & \{0,1\}\\
\text{The alphabet}& \text{is} &\{a,b\}\times\{Z,X\}\\
\delta^\prime&=&\{(0,(a,X))\mapsto 1,(0,(b,X))\mapsto 0\\
&&(0,(a,Z)) \mapsto 0,(0,(b,Z))\mapsto 0\\
&&(1,(a,X))\mapsto 1, (1,(b,X)) \mapsto 0\\
&&(1,(a,Z))\mapsto 0,(1,(b,Z))\mapsto 1\\
&&q_{\rm init}=0\quad F=\{1\}
\end{array}$};
\end{tikzpicture}
}
\caption{Underlying automaton of $P_{\rm toy}$.\label{figUnder}}
\end{figure}
There are three main kinds of accepting
conditions for a word $u=a_1\ldots a_k\in\Sigma^*$ by an RDPDA:
\begin{itemize}
\item Under the {\it empty-stack
 condition},  $u$ is accepted if there exists 
configurations $c_1,\ldots,c_{k+1}$ such that $c_1$ is the initial
configuration, $c_i$ and $c_{i+1}$ are $a_i$-consecutive, and $c_{k+1}$ is
of the form $(q,\varepsilon)$. 
\item Under the {\it final-state
 condition}, $u$ is accepted  if there exists 
configurations $c_1,\ldots,c_{k+1}$ such that $c_1$ is the initial
configuration, $c_i$ and $c_{i+1}$ are $a_i$-consecutive, and $c_{k+1}$ is
of the form $(q,w)$, with $q\in F$.
 \item Under the {\it final-state and empty-stack
 condition}, $u$ is accepted  if there exists 
configurations $c_1,\ldots,c_{k+1}$ such that $c_1$ is the initial
configuration, $c_i$ and $c_{i+1}$ are $a_i$-consecutive, and $c_{k+1}$ is
of the form $(q,\varepsilon)$, with $q\in F$.
 \end{itemize}

\begin{figure}[h!]
\begin{center}
\begin{tikzpicture}
\node (0) [state,fill=black!20,initial,initial text=] at (0,0) {$0$};
\node (1) [state,fill=black!20, accepting]at (3,0) {$1$};
\node (2) [state,fill=black!20, accepting]at (6,0) {$2$};
\node (3) [state,fill=black!20, accepting]at (-3,0) {$3$};

\path[->,>=triangle 90] (0) [] edge[above] node {$(b,X),\varepsilon$} (1);
\path[->,>=triangle 90] (1) [] edge[above] node
     {$(b,X),XX$} (2);

\path[->,>=triangle 90] (0) [loop above] edge[above] node
     {$(a,X),X$} ();
\path[->,>=triangle 90] (0) [bend right] edge[above] node
     {$(a,Z),\varepsilon$} (3);

\end{tikzpicture}
\end{center}
\caption{Acceptance conditions.}\label{fig:acceptation}
\end{figure}
Consider for instance the RDPDA of Fig.~\ref{fig:acceptation}, where the
initial stack symbol is $X$. With the empty-stack condition only the word
$b$ is accepted, as well as for the empty-stack and final state condition.
With the final state condition, the accepted language is $a^*(b+bb)$.

Two RDPDA $(Q_1,\Sigma,\Gamma,\delta_1,q_{\rm init1},F_1)$
and $(Q_2,\Sigma,\Gamma,\delta_2,q_{\rm init2},F_2)$ are isomorphic if
there exists a one-to-one function $\varphi$ from $Q_1$ into $Q_2$ such that
(i) $\varphi(q_{\rm init1})=q_{\rm init2}$, and (ii) $\varphi(F_1)=F_2$, and
 $$(iii)\quad \delta_1(q,(a,X))=(p,w)\quad \text{iff}\quad \delta_2(\varphi(q),(a,X))=(\varphi(p),w).$$
Note that if two RDPDA are isomorphic, then their underlying automata are
isomorphic too.



\paragraph{Generating Functions.}
A {\it combinatorial class} is a class $\mathfrak{C}$ of objects associated
with a size function $|.|$ from $\mathfrak{C}$ into $\mathbb{N}$ such that
for any integer $n$ there are finitely many elements of $\mathfrak{C}$ of
size $n$. The ordinary generating function for $\mathfrak{C}$ is
$C(z)=\sum_{c\in \mathfrak{C}} z^{|c|}$. The $n$-th coefficient of $C(z)$ is
exactly the number of objects of size $n$ and is denoted $[z^n]C(z)$. The
reader is referred to~\cite{FSbook} for the general methodology of analytic
combinatorics, ans especially the use of generating functions to count
objects. The following result~\cite[Theorem~VIII.8]{FSbook} will be useful
in this paper.

\begin{theorem}\label{thm:VIII.8}
Let $C(z)$ be an ordinary generating function satisfying: (1) $C(z)$ is
analytic at $0$ and have only positive coefficients, $(2)$ $C(0)\neq 0$ and
$(3)$ $C(z)$ is aperiodic. Let $R$ be the radius of convergence of $C(z)$
and $T=\lim_{x\to R^{-}} x\frac{C'(x)}{C(x)}$.
Let $\lambda\in ]0,T[$ and $\zeta$ be the  unique
solution of $x\frac{C'(x)}{C(x)}=\lambda$. Then, for $N=\lambda n$ an
integer, one has
$$[z^N]C(z)^n=\frac{C(\zeta)^n}{\zeta^{N+1}\sqrt{2\pi n\xi}}(1+o(1)),$$
where $\xi=\frac{d^2}{d z^2}\left(\log C(z)-\lambda\log (z)\right)|_{z=\zeta}.$
\end{theorem}
In the above theorem, $T$ is called the \textit{spread} of the $C(z)$. 

\paragraph{Rejection Algorithms.}
A rejection algorithm is a probabilistic algorithm to randomly
generate an element in a set $X$, using an algorithm $A$ for
generating an element of $Y$ in the simple way: repeat $A$ until it
returns an element of $X$. Such an algorithm is tractable if the
expected number of iterations can be kept under control (for instance is
fixed): the probability that an element of $Y$ is in $X$ has to be large
enough.

\paragraph{Random Generation.}
 The theory of Generating Functions provides an efficient way to randomly
 and uniformly generate an element of size $n$ of a combinatorial class
 $\mathfrak{C}$ using a recursive approach~\cite{FSbook}. It requires a
 $O(n^2)$ precomputation time and each random sample is obtained in time
 $O(n\log n)$. Another efficient way to uniformly generate element of
 $\mathfrak{C}$ is to use Boltzmann
 samplers~\cite{DBLP:journals/cpc/DuchonFLS04}: the random generation of an
 object with a size about $n$ (approximate sampling) is performed in $O(n)$,
 while the random generation of an object with a size exactly $n$ is
 performed in expected time $O(n^2)$ using a rejection algorithm (without
 precomputation). Boltzmann samplers are quite easy to implement but are
 restricted to a limited number of combinatorial constructions. They also
 require the evaluation of some generating functions at some values of the
 variable.



