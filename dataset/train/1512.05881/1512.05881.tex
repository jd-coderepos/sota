


Finite automata, of any kind, are widely used for their algorithmic
properties in many fields of computer science like model-checking,
pattern matching and machine learning. Developing new efficient algorithms
for finite automata is therefore a challenging problem still addressed by
many recent papers. New algorithms are frequently motivated by improvement
of worst cases bound. However, several examples, such as sorting algorithms,
primality testing or solving linear problems, show that worst case
complexity is not always the right way to evaluate the practical performance
of an algorithm. When benchmarks are not available, random testing, with a
controlled distribution, represents an efficient mean of performance
testing. In this context, the problem of uniformly generating finite
automata is a challenging problem.


This paper tackles with the problem of the uniform random generation of
real-time deterministic pushdown automata. Using classical combinatorial
techniques, we expose how to extend existing works on the generation of
finite deterministic automata to pushdown automata. More precisely, we show
in Section~\ref{sec:random} how to uniformly generate and enumerate (in the
complete case) accessible real-time deterministic pushdown automata. In
Section~\ref{sec:PDA}, it is shown that using a rejection algorithm it is
possible to efficiently generate pushdown automata that don't accept an
empty language. The influence of the accepting condition (final state or
empty-stack) on the reachability
of the generated pushdown automata is also experimentally studied  in
Section~\ref{sec:PDA}.



\paragraph{Related work.}  The 
enumeration of deterministic finite automata has been first investigated
in~\cite{Vyssotsky} and was applied to several subclasses of deterministic
finite
automata~\cite{Korshunov,DBLP:journals/eik/Korshunov86,robinson,DBLP:journals/dam/Liskovets06}. 
The uniform random generation of accessible deterministic complete automata
was initially proposed in~\cite{thesecril} for two-letter alphabets and the
approach was extended to larger alphabets
in~\cite{DBLP:journals/tcs/ChamparnaudP05}. Better algorithms can be found
in~\cite{DBLP:journals/tcs/BassinoN07,DBLP:conf/stacs/CarayolN12}. The
random generation of possibly incomplete automata is analyzed
in~\cite{incomplet}. The recent paper~\cite{DBLP:conf/wia/CarninoF11}
presents how to use Monte-Carlo approaches to generate deterministic acyclic
automata. As far as we know, the only work focusing on the random generation
of deterministic transducers is~\cite{DBLP:journals/tcs/HeamNS10}. This work
can be applied to the random generation of deterministic real-time
pushdown automata that can be possibly incomplete. However the requirement to fix the size of the stack operation on each transition, represents a major restriction.
The reader interested in the random generation of deterministic automata is
 referred to the survey~\cite{DBLP:conf/mfcs/Nicaud14}. 






\section{Formal Background}\label{sec:bg}


We assume that the reader is familiar with classical notions on formal
languages. For more information on automata theory or on pushdown automata
the reader is referred to~\cite{Hopcroft} or to~\cite{Saka}. For a general
reference on random generation and enumeration of combinatorial structures
see~\cite{DBLP:journals/tcs/FlajoletZC94}. For any word  on an alphabet ,  denotes
its length. The empty word is denoted~.
The cardinal of a finite set  is denoted .

\paragraph{Deterministic Finite Automata.}
A {\it deterministic finite automaton} on  is a tuple
 where  is a finite set of states,
 is a finite alphabet,  is the initial state,
 is the set of final states and  is a partial function
from  into . If  is not partial, i.e., defined for
each , the automaton is said {\it complete}. A
triplet of the form  is called a {\it transition}. A
finite automaton is graphically represented by a labeled finite graph whose
vertices are the states of the automaton and edges are the transitions. A
deterministic finite automaton is {\it accessible} if for each state 
there exists a path from the initial state to . Two finite automata
 and  are isomorphic if they are identical up to the state's names,
formally if there exists a one-to-one function  from  into
 such that (1) ,  (2)
, and (3)  iff
.

\paragraph{Pushdown Automata.}
A {\it real-time deterministic pushdown automaton}, RDPDA for short, is a
tuple  where  is a
finite set of states,  and  are finite disjoint alphabets,
 is the initial state,  is the set of final
states,  is the initial stack symbol and  is a partial
function from  into . If
 is not partial, i.e., defined for each , the
RDPDA is said to be {\it complete}. 
A triplet of the form  with
 is called a {\it transition} and  is the {\it
output} of the transition. The {\it output size} of a transition
 is the length of .  The {\it output size} of
a RDPDA is the sum of the sizes of its transitions.
The {\it underlying automaton} of an
RDPDA, is the finite automaton , with  iff 
for some . An RDPDA is {\it accessible} if its underlying
automaton is accessible. A transition whose output is  is
called a {\it pop transition}. An example of a complete accessible RDPDA is
depicted in Fig.~\ref{figAutoPile1}. The related underlying  finite
automaton is depicted in Fig.~\ref{figUnder}.


\begin{figure}[t!]
\centering
\subfigure{
\begin{tikzpicture}
\node (0) [state,fill=black!20,initial,initial text=] at (0,0) {};
\node (1) [state,fill=black!20, accepting]at (3,0) {};

\path[->,>=triangle 90] (0) [bend left] edge[above] node {} (1);
\path[->,>=triangle 90] (1) [bend left] edge[below, pos=0.3] node
     {\begin{tabular}{c}\\\end{tabular}} (0);

\path[->,>=triangle 90] (0) [loop above] edge[above] node
     {\begin{tabular}{c}\\\end{tabular}} ();
\path[->,>=triangle 90] (0) [loop below] edge[below] node
     {} (0);

\path[->,>=triangle 90] (1) [loop above] edge[above] node
     {\begin{tabular}{c}\\\end{tabular}} ();


\node (txt) at (7.5,0.3){
};
\end{tikzpicture}
}
\caption{, a complete RDPDA.\label{figAutoPile1}}
\end{figure}

A {\it configuration} of a {RDPDA} is an element of . The
{\it initial configuration} is . Two
configurations  and  are -consecutive, denoted
 if the following conditions are satisfied:
\begin{itemize}
\item , and let  with ,
\item   and .
\end{itemize}
Two configurations are {\it consecutive} if there is a letter  such that there
are -consecutive. A state  of a RDPDA is {\it reachable} if there
exists a sequence of consecutive configurations 
such that  is the initial configuration and . Moreover, if
,  is said to be {\it reachable with an empty stack}. A
RDPDA is {\it reachable} if all its states are reachable. Consider for
instance the RDPDA of Fig.~\ref{fig:acceptation}, where the initial stack
symbol is . State  is not reachable since the transition from  to
 cannot be fired. State  is reachable with an empty stack. State  is
reachable, but not reachable with an empty stack. Note that a reachable
state is accessible, but the converse is not true in general: accessibility
is a notion defined on the underlying finite automaton. 

 The configurations  and  are -consecutive on the RDPDA
depicted in Fig.~\ref{figAutoPile1}. 

\begin{figure}[h!]
\centering
\subfigure{
\begin{tikzpicture}
\node (0) [state,fill=black!20,initial,initial text=] at (0,0) {};
\node (1) [state,fill=black!20, accepting]at (3,0) {};

\path[->,>=triangle 90] (0) [bend left] edge[above] node {} (1);
\path[->,>=triangle 90] (1) [bend left] edge[below, pos=0.3] node
     {\begin{tabular}{c}\\\end{tabular}} (0);

\path[->,>=triangle 90] (0) [loop above] edge[above] node
     {\begin{tabular}{c}\\\end{tabular}} ();
\path[->,>=triangle 90] (0) [loop below] edge[below] node
     {} (0);

\path[->,>=triangle 90] (1) [loop above] edge[above] node
     {\begin{tabular}{c}\\\end{tabular}} ();


\node (txt) at (7.5,0.3){
};
\end{tikzpicture}
}
\caption{Underlying automaton of .\label{figUnder}}
\end{figure}
There are three main kinds of accepting
conditions for a word  by an RDPDA:
\begin{itemize}
\item Under the {\it empty-stack
 condition},   is accepted if there exists 
configurations  such that  is the initial
configuration,  and  are -consecutive, and  is
of the form . 
\item Under the {\it final-state
 condition},  is accepted  if there exists 
configurations  such that  is the initial
configuration,  and  are -consecutive, and  is
of the form , with .
 \item Under the {\it final-state and empty-stack
 condition},  is accepted  if there exists 
configurations  such that  is the initial
configuration,  and  are -consecutive, and  is
of the form , with .
 \end{itemize}

\begin{figure}[h!]
\begin{center}
\begin{tikzpicture}
\node (0) [state,fill=black!20,initial,initial text=] at (0,0) {};
\node (1) [state,fill=black!20, accepting]at (3,0) {};
\node (2) [state,fill=black!20, accepting]at (6,0) {};
\node (3) [state,fill=black!20, accepting]at (-3,0) {};

\path[->,>=triangle 90] (0) [] edge[above] node {} (1);
\path[->,>=triangle 90] (1) [] edge[above] node
     {} (2);

\path[->,>=triangle 90] (0) [loop above] edge[above] node
     {} ();
\path[->,>=triangle 90] (0) [bend right] edge[above] node
     {} (3);

\end{tikzpicture}
\end{center}
\caption{Acceptance conditions.}\label{fig:acceptation}
\end{figure}
Consider for instance the RDPDA of Fig.~\ref{fig:acceptation}, where the
initial stack symbol is . With the empty-stack condition only the word
 is accepted, as well as for the empty-stack and final state condition.
With the final state condition, the accepted language is .

Two RDPDA 
and  are isomorphic if
there exists a one-to-one function  from  into  such that
(i) , and (ii) , and
 
Note that if two RDPDA are isomorphic, then their underlying automata are
isomorphic too.



\paragraph{Generating Functions.}
A {\it combinatorial class} is a class  of objects associated
with a size function  from  into  such that
for any integer  there are finitely many elements of  of
size . The ordinary generating function for  is
. The -th coefficient of  is
exactly the number of objects of size  and is denoted . The
reader is referred to~\cite{FSbook} for the general methodology of analytic
combinatorics, ans especially the use of generating functions to count
objects. The following result~\cite[Theorem~VIII.8]{FSbook} will be useful
in this paper.

\begin{theorem}\label{thm:VIII.8}
Let  be an ordinary generating function satisfying: (1)  is
analytic at  and have only positive coefficients,   and
  is aperiodic. Let  be the radius of convergence of 
and .
Let  and  be the  unique
solution of . Then, for  an
integer, one has

where 
\end{theorem}
In the above theorem,  is called the \textit{spread} of the . 

\paragraph{Rejection Algorithms.}
A rejection algorithm is a probabilistic algorithm to randomly
generate an element in a set , using an algorithm  for
generating an element of  in the simple way: repeat  until it
returns an element of . Such an algorithm is tractable if the
expected number of iterations can be kept under control (for instance is
fixed): the probability that an element of  is in  has to be large
enough.

\paragraph{Random Generation.}
 The theory of Generating Functions provides an efficient way to randomly
 and uniformly generate an element of size  of a combinatorial class
  using a recursive approach~\cite{FSbook}. It requires a
  precomputation time and each random sample is obtained in time
 . Another efficient way to uniformly generate element of
  is to use Boltzmann
 samplers~\cite{DBLP:journals/cpc/DuchonFLS04}: the random generation of an
 object with a size about  (approximate sampling) is performed in ,
 while the random generation of an object with a size exactly  is
 performed in expected time  using a rejection algorithm (without
 precomputation). Boltzmann samplers are quite easy to implement but are
 restricted to a limited number of combinatorial constructions. They also
 require the evaluation of some generating functions at some values of the
 variable.



