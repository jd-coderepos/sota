\documentclass[journal,onecolumn,11pt]{IEEEtran}
\usepackage{amsthm}
\theoremstyle{plain}
\newtheorem{theorem}{Theorem}
\newtheorem{lemma}{Lemma}
\newtheorem{proposition}{Proposition}
\newtheorem{fact}{Fact}
\newtheorem{corollary}{Corollary}
\theoremstyle{definition}
\newtheorem{definition}{Definition}
\newtheorem{example}{Example}
\newtheorem{remark}{Remark}
\usepackage{subfig}
\usepackage{setspace}
\doublespacing

\def\x{{\mathbf x}}
\def\s{{\mathbf s}}
\def\u{{\mathbf u}}
\def\v{{\mathbf v}}
\def\w{{\mathbf w}}
\def\y{{\mathbf y}}
\def\z{{\mathbf z}}
\def\bm{{\mathbf m}}
\def\bp{{\mathbf p}}
\def\bx{{\mathbf x}}
\def\bX{{\mathbf X}}
\def\bC{{\mathbf C}}
\def\bP{{\mathbf P}}
\def\bmu{{\boldsymbol{\mu}}}
\def\bz{{\mathbf z}}
\def\bZ{{\mathbf Z}}
\def\bE{{\mathbf E}}
\def\bbE{{\mathbb E}}
\def\bg{{\mathbf g}}
\def\bb{{\mathbf b}}
\def\bs{{\mathbf s}}
\def\btheta{{\mathbf \theta}}
\def\bgamma{\bm{\gamma}}
\def\L{{\cal L}}
\def\O{{\cal O}}
\def\N{{\cal N}}
\def\cU{{\mathcal U}}
\def\cS{{\mathcal S}}
\def\cN{{\mathcal N}}


\usepackage[cmex10]{amsmath}
\usepackage{amssymb}
\usepackage{array}

\usepackage{algorithm}
\usepackage{algorithmic}
\usepackage{setspace}
\usepackage{epstopdf}
\usepackage[export]{adjustbox}
\usepackage{subfig}
\usepackage{graphicx}
\usepackage{stfloats}
\usepackage{filecontents,lipsum}
\usepackage[noadjust]{cite}

\ifCLASSINFOpdf
\else
\fi

\hyphenation{op-tical net-works semi-conduc-tor}


\begin{document}
\title{Optimal Forwarding in Opportunistic Delay Tolerant Networks with Meeting Rate Estimations}

\author{Shohreh Shaghaghian and~Mark Coates\thanks{The authors are with the Computer Networks Research Laboratory, Department of Electrical and Computer Engineering, McGill University, Montreal, QC H3A 0E9, Canada (e-mail:
shohreh.shaghaghian@mail.mcgill.ca;
mark.coates@mcgill.ca). This research was supported by the Natural
Sciences and Engineering Research Council of Canada (NSERC) via grant
application 261528.}\thanks{Manuscript received April 3, 2015}}

\markboth{}{Shell \MakeLowercase{\textit{et al.}}: Bare Demo of IEEEtran.cls for Journals}

\maketitle

\begin{abstract}
  Data transfer in opportunistic Delay Tolerant Networks (DTNs) must
  rely on unscheduled sporadic meetings between nodes. The main
  challenge in these networks is to develop a mechanism based on which
  nodes can learn to make nearly optimal forwarding decision rules despite having
  no a-priori knowledge of the network topology. The forwarding
  mechanism should ideally result in a high delivery probability, low
  average latency and efficient usage of the network resources. 
  In this paper, we propose both centralized and
  decentralized single-copy message forwarding algorithms that, under relatively
  strong assumptions about the network’s behaviour, minimize the
  expected latencies from any node in the network to a particular
  destination. After proving the optimality of our proposed
  algorithms, we develop a decentralized algorithm that involves a
  recursive maximum likelihood procedure to estimate the meeting
  rates.  We confirm the improvement that our proposed algorithms make
  in the system performance through numerical simulations on datasets
  from synthetic and real-world opportunistic networks.
\end{abstract}

\begin{IEEEkeywords}
Delay Tolerant Networks, Opportunistic Forwarding, Meeting Rate Estimation
\end{IEEEkeywords}

\IEEEpeerreviewmaketitle



\section{Introduction}\label{sec:intro}
\IEEEPARstart{D}{\textit{elay}} or \textit{Disruption Tolerant Networks}
(DTNs) are a class of wireless mobile node networks in which the
communication path between any pair of nodes is frequently
unavailable. Nodes are thus only intermittently connected. DTNs were
first studied in the 1990s when the research community considered how
the Internet could be adapted for space
communications~\cite{voyiatzis2012survey}. Later, it was recognized
that DTNs were a suitable model for several terrestrial networks.

DTNs can be categorized according to whether the
node connections are scheduled (thus predictable) or random (hence
unpredictable). Space communication networks fall into the first
group. Networks belonging to the second category, which are the focus
of this paper, are also referred to as {\em opportunistic networks},
because nodes seize the opportunity to transfer data when a
communication channel becomes available. Opportunistic networks have been studied intensively in
recent years (e.g.,
\cite{khabbaz2012disruption,cao2013routing,wei2014survey}) because
they can fulfil a number of useful purposes, such as non-intrusive
wildlife tracking (e.g., ZebraNet~\cite{juang2002energy} and
SWIM~\cite{small2003shared}), emergency response in disaster scenarios
(e.g., ChaosFIRE~\cite{pataki2014sensor}), provision of data
communication to remote and rural areas (e.g.,
DakNet~\cite{pentland2004daknet}), and traffic offloading in cellular
networks (e.g., \cite{han2012mobile}).

In most opportunistic networks, the nodes are highly mobile and have a
short radio range, and the density of nodes is low. In many cases,
nodes have limited power and memory resources. These attributes
combine with the intermittent connections to make routing
traffic challenging. Routing is usually based on a store-carry-forward mechanism that
exploits node mobility. In this mechanism, the source transmits its
message to a node it meets. This intermediate node stores and then
carries the received message until it meets another node to which it
can forward the message. This process is repeated until the message
reaches its destination. The key ingredients in designing an
opportunistic network routing protocol are the forwarding decisions:
should a node forward a message to a neighbour it meets?  should it
retain a copy for itself?

Although much research effort has been devoted to the development of
opportunistic network routing
algorithms~\cite{vahdat2000epidemic,grossglauser2001mobility,spyropoulos2005spray,
  davis2001wearable,lindgren2003probabilistic,burgess2006maxprop,
  jones2007practical,daly2007social,hui2011bubble,sharma2013contact,
  xiao2013community,li2011impact,sermpezis2014understanding,zhang2013gossip,
  conan2008fixed,xiao2013tour,boldrini2010modelling,boldrini2012performance},
the algorithms are either centralized, have no performance guarantees,
or ignore the need to estimate network parameters. Our work focuses on the mobile ad-hoc network (MANET)
setting, where node speed is much reduced compared to the vehicular
ad-hoc (VANET) case, and we can assume that there are fewer
restrictions on the amount of data that nodes can transfer when they meet. In this paper, we
derive a decentralized routing algorithm that has performance
guarantees (under simplifying assumptions about the network
behaviour). When the meeting times between nodes are
independent and exponentially distributed, the routing algorithm
minimizes the expected latency in sending a packet from any source
node to a specific destination. We examine the behaviour of the
routing algorithm when the meeting rates are learned online using a
recursive maximum likelihood procedure. We show that, for a stationary
network, the decision rules and achieved expected latencies converge
to those obtained when there is exact knowledge of the meeting
rates. We present the results of simulations that compare the
performance of the proposed algorithm to previous approaches, and
examine how the algorithm is affected by practical network limitations
(finite buffers, restrictions on data exchange, message expiry times).

{\em Organization}: The paper is organized as follows. In the following
subsection, we discuss related work. In
Section~\ref{sec:system_model}, we describe our system model and
formulate the routing problem. In Section~\ref{sec:algo}, we
present the forwarding algorithms and discuss their
optimality under the network modeling assumptions. We present
numerical simulation results in Section~\ref{sec:res} and make
concluding remarks in Section~\ref{sec:conc}.

\subsection{Related Work}

The first proposed approaches for routing in opportunistic networks
were based on extending the concept of flooding to intermittently
connected mobile networks. In these {\em replication-based} methods, a
node forwards the messages stored in its buffer to all of (or to a
fraction of) the nodes it encounters. There is no attempt to evaluate
the capability of a given node to expedite the delivery. These routing
algorithms have few parameters: they determine only how much
replication can occur and which nodes can make copies of packets. One
of the earliest algorithms was {\em epidemic
  routing}~\cite{vahdat2000epidemic}, in which a node forwards a
message to any node it meets, provided that node has not previously
received a copy of the message. Thus messages are quickly distributed
through the connected portions of the network. Other
replication-based approaches (\cite{grossglauser2001mobility,
  spyropoulos2005spray,ramanathan2007prioritized,khouzani2012optimal})
manage to reduce the transmission overhead of epidemic routing and
improve its delivery performance through modification of the
replication process and prioritization of messages. Models have been
developed that allow an analytical characterization of the performance
of the epidemic routing techniques~\cite{klein2010reaction,wang2012analytical}.
Replication-based approaches result in a high probability of
message delivery since more nodes have a copy of each message, but
they can produce network congestion.

A step towards achieving more efficient routing approaches is to
consider the history of node contacts in the network instead of
blindly forwarding packets. {\em History-based} (also called {\em
  utility-based}) routing algorithms assume that nodes' movement
patterns are not completely random and that future contacts depend on
the frequency and duration of past encounters. Based on these past
observations, both the source and the intermediate nodes decide
whether to forward a message to nodes they encounter or to store it
and wait for a better opportunity. An early example is
\cite{davis2001wearable}, which extends epidemic routing to situations
with limited resources, incorporating a dropping strategy for the case when the buffer
of a node is full. The dropping decisions are based on the meeting
history of the node. PRoPHET~\cite{lindgren2003probabilistic} assigns a delivery
probability metric to each node which indicates how likely it is that
the message will be delivered to the destination by that particular
node. This metric is updated each time two nodes meet, and thus takes
into account the history of meetings in the network.
MaxProp~\cite{burgess2006maxprop} and MEED~\cite{jones2007practical}
are other examples of history-based algorithms proposed for vehicular
DTNs. In these networks, nodes move with higher speeds, reducing the
amount of time they are in each other's radio range. Hence, the two
main limiting resources are the duration of time that nodes are able
to transfer data and their storage capacities.

Other researchers have examined whether it is possible to exploit
other characteristics of opportunistic networks to improve the
performance of routing algorithms. Since social
interactions often determine when connections between nodes occur,
several algorithms strive to use social network concepts like
betweenness centralities (e.g. \cite{daly2007social}), or community
formations (e.g.
\cite{hui2011bubble,sharma2013contact,xiao2013community}). Other
algorithms have attempted to take advantage of the strategic behaviour
of nodes (e.g. \cite{li2011impact,sermpezis2014understanding}). Our
work focuses on routing a message to a single destination, but there
are connections to research that addresses the task of spreading
information to multiple nodes in a network. Of particular interest is
the gossip-based approach in~\cite{zhang2013gossip}, which greatly reduces the number of
message copies in the network while achieving near-optimal
dissemination.

The experimental-based studies demonstrate the efficiency of their
proposed methods by running simulations on traces recorded from real
world opportunistic networks. Experimental analyses are valuable and
take into account practical considerations, but they can leave us with
an incomplete understanding of how an algorithm operates and how it
will perform in other untested network conditions. For example, the
behaviour of PRoPHET has been shown to be very sensitive to parameter
choice~\cite{grasic2011evolution}. It is also useful to design an
optimal algorithm under slightly less realistic modeling assumptions, and
then consider how it can be adapted to address the practical
limitations, without completely losing its desirable features.
More recent studies have focused on deriving a forwarding process
whose optimality (in some sense) can be mathematically proved under
assumptions about network behaviour.
\cite{conan2008fixed} extends the two hop relay strategy of
\cite{grossglauser2001mobility} by considering the expected delivery
time to the destination as a metric to find the best set of candidate
relays. By increasing the number of relaying steps recursively, a
centralized single-copy multi-hop opportunistic routing scheme is
proposed for sparse DTNs. The main defect of a centralized
approach is that global knowledge of the network is required in
order to make forwarding decisions.

There have been some efforts towards migrating to decentralized
solutions that still provide performance guarantees.
\cite{xiao2013tour} proposes a decentralized time-sensitive algorithm
called TOUR in which message priority is taken into account in
addition to nodes' expected latencies when making forwarding decisions.
Although in TOUR each node only needs to be aware of the local
information about the rates of contacts with its own set of
neighbours, the algorithm assumes that the node knows the exact
contact rates. In most practical scenarios, this assumption is not valid.

Some researchers have explored how imprecision in the measurement
or estimation of network parameters can impact the performance of
opportunistic network routing algorithms. In \cite
{boldrini2010modelling, boldrini2012performance}, Boldrini
et al. discuss different sources of errors that may exist in parameter
estimation like missed encounters, incorrect combination of short
contacts, and memory limitations. They model these errors
as a random variable with a normal distribution and evaluate the
performance of four different forwarding schemes under this
model. Although this error analysis is useful, Boldrini et al. do not
specify how parameters should be estimated in order to obtain a
performance that approaches what can be achieved when perfect a-priori
knowledge of the network parameters is available.

Some of the results in this paper were presented in an earlier
conference paper~\cite{shaghaghian2014opportunistic}, but here we include more extensive experimental
analysis and additional theoretical results.

\section{System Model} \label{sec:system_model}
We consider a network of  mobile nodes which aim to send messages
to a particular destination node . The set of nodes is denoted by
. We assume that the random inter-meeting times of nodes
are independent and exponentially distributed with parameter
 for nodes  and . 

Although the aggregate intermeeting distributions of nodes in mobile
ad-hoc networks often follow a truncated power
law~\cite{cai2009,chaintreau2007impact}, there is evidence that the
intermeeting times of individual pairs of nodes can be adequately
modeled by exponential distributions with heterogeneous
coefficients~\cite{conan2007,gao2009,lee2010,zhu2010}. In particular,
Conan et al.~\cite{conan2007} and Gao et al.~\cite{gao2009} conduct
statistical analyses of mobile social network data traces, including
the Infocom data set~\cite{cambridge-haggle-2006-01-31} that we
analyze in Section~\ref{sec:res}. They demonstrate that most pairs of
nodes have intermeeting times that are approximately exponentially
distributed. In~\cite{lee2010,zhu2010}, approximately exponential
distributions of individual meeting times are detected through
statistical analyses of car/taxi mobility traces.

We associate with the network a {\em contact graph} which is formed by
adding a link between any two nodes that meet. We assume that the
contact graph is connected and denote the set of neighbors of node 
in this graph by . Since the contacts between nodes are
not pre-scheduled, we cannot identify end-to-end paths ahead of
time. Hence, solving the routing task is equivalent to identifying the
forwarding decisions that nodes should make when meeting each
other. We assume that nodes' buffer sizes are unlimited, message Time
To Live (TTL) is infinity and that nodes' speed and message lengths
are such that any number of messages can be forwarded during each
meeting.  We consider only algorithms that do not involve
replication. In the class of algorithms we consider, each time node
 meets one of its neighbors , it forwards a
message destined for  with probability . Considering the
matrix  comprised of all pairs  and ,
we set  if nodes  and  never meet and are thus not
neighbors in the contact graph. We denote the forwarding probabilities
of node  by the vector ; this is the -th row of
the matrix .

The expected latency from node  to destination  is a function of
the probability decision matrix  and we denote it by
. Our goal is to find the matrix 
such that the sum of the expected latencies of all the nodes in the
network to the specified destination  is minimized. Let us call
this utility function . We assume that the network topologies and
meeting rates are such that the solution  is unique. If
not, our algorithms guarantee that we reach one of the optimal
matrices, but the proofs are more complicated.
The first step towards achieving this goal and finding matrix  is to discover how the expected latency of an arbitrary node , , depends on the elements of the probability decision matrix  in general. Lemma \ref{Lid} provides an expression for  in
terms of  and . The proof is available in Appendix \ref{pl1}.


\begin{lemma} \label{Lid}
The expected latency of a node  to the destination  is

\end{lemma}

Based on the expression derived in Lemma \ref{Lid}, the expected latency of each node to the destination depends on the expected latencies of its neighbours. This result raises a substantial question: Does the probability decision  matrix that minimizes the sum of expected latencies of all nodes of the network (), also minimize the expected latency of each individual node? Before continuing to propose algorithms for finding ,
we answer this question and make two points about the structure of  through the
following theorem. The proof is provided in Appendix \ref{pt1}.
\begin{theorem}\label{theorem1}
Suppose . Then:
\begin{itemize}
\item[(1)]  is a binary matrix (its components are either  or ).
\item[(2)] For any , the matrix  also
  minimizes : 

\end{itemize}
\end{theorem}

Theorem \ref{theorem1} shows that the minimization problem is actually a binary problem. Each time node  meets one of its neighbours , it either forwards the message or keeps it. From now on, we change our notation and use the binary indicator matrix  instead of  to capture this binary decision. Therefore, the optimization takes the form:

Theorem \ref{theorem1} also states that the optimum solution matrix  can be equivalently achieved by minimizing the expected latency of each of the network nodes to the destination. This is the main idea of developing centralized and decentralized algorithms for finding . In the next section, we introduce the algorithms we have proposed for solving this optimization problem and prove that they find the optimal solution.


\section{Algorithms}\label{sec:algo}

In the first part of this section, we try to to find  in
a centralized fashion where the whole topology and meeting rates of
the network are available at a central unit. This unit calculates a
binary matrix  and informs the nodes about the neighbours
they should forward their buffered messages to. We prove that the
solution achieved upon completion of this algorithm is the same as the
optimum solution . In the second part of the section,
we introduce a decentralized algorithm and prove that it converges to
the same global solution with
probability . The advantage of the decentralized approach is that
no node needs to have a global knowledge of the network and each node
can learn its own optimal forwarding decisions. The only piece of
information a node needs to know is its meeting rates with its own
neighbours. Finally in the last part of the section, we make our
model more realistic by assuming that nodes have no a-priori knowledge
of any meeting rates. In this more practically realistic scenario,
each node estimates the meeting rates with its neighbours, updating
its estimates each time a contact occurs.

\begin{algorithm}[!htb]
\caption{Centralized Greedy Latency Minimization} \label{algo:cen}
\begin{spacing}{1.0}
\begin{algorithmic}[1]
		\vspace{3mm}
		\STATE{ \texttt{// Initialization}}
		\STATE{, , ,
                   for }
		\STATE{ \texttt{// Iteratively add nodes to the set}}
 		\WHILE{}
			\FOR{each node }
				\STATE{Identify }

				\STATE{Calculate  for  \\
				\vspace{0.15cm}
 }
				\vspace{0.15cm}
				\STATE{Denote }
			\ENDFOR
 			\STATE{Identify }
			\STATE{Set }
			\STATE{Set  for all }
			\STATE{Set }
		\ENDWHILE

\end{algorithmic}
\end{spacing}
\end{algorithm}

\subsection{Centralized Approach with Global Knowledge}
Suppose for each node , the set of neighbours
 and their meeting rates  are known at a central calculation unit. Algorithm
\ref{algo:cen} presents an iterative procedure to identify a binary
decision matrix . In this algorithm, we first decide on
the forwarding rules of the node that has the most frequent direct
contacts with the destination. We refer to this node as node . In
order to achieve the minimum expected latency to the destination, node
 should forward its generated or received messages only to the
destination, ignoring its meetings with any other nodes. All other
nodes that  encounters meet the destination less frequently and, if
they forward their messages to the destination through other nodes,
these other nodes also meet the destination less frequently than
. In subsequent steps of the algorithm, we consider all the nodes
that have direct contacts with the nodes whose forwarding decision
rules have already been made (the set ) and calculate the
minimum latency that each of them can obtain by forwarding through
nodes in  to the destination. At the end of each
iteration, we finalize the forwarding decision for the one node that
can achieve the minimum latency and add it to . We repeat
the procedure until the decision has been made for all the nodes of
the network and the elements of the binary matrix  have
all been specified.  The next theorem states that the binary matrix
 resulting from applying this procedure, as specified
concretely in Algorithm \ref{algo:cen}, achieves the minimum sum of
expected latencies to the destination. The proof can be found in
Appendix \ref{pt2}.

\begin{theorem}\label{theorem2}
  Suppose all meeting rates are different and there exists
  a unique solution  for the optimization problem \eqref{opt}.
\begin{enumerate}
\item After each iteration of Algorithm \ref{algo:cen},
\begin{enumerate}
\item 
\item 
\item 
\end{enumerate}
\item Upon completion, Algorithm \ref{algo:cen} identifies a labelling  and associated expected latencies  such that 
\end{enumerate}
\end{theorem}
Theorem \ref{theorem2} demonstrates that the iterative optimization
procedure expressed in Algorithm \ref{algo:cen} finds the solution of
the minimization problem in (\ref{opt}). If there is not a unique
solution, then at some point in Algorithm \ref{algo:cen}, there will
be multiple  that solve the optimization in line 7. It is
straightforward to show that choosing any one of these  leads to a
decision matrix  that achieves the minimum expected latencies. 

\subsection{Decentralized Approach with Partial A-priori Knowledge}\label{dec_partial}
Suppose no central unit exists and each node is just aware of its own
 and the meeting rates . Algorithm \ref{algo:DistLatMin} demonstrates how nodes
can make their binary forwarding decisions based on this local
information. Since the expected latency of each node depends on the
expected latency values of its neighbours, nodes need to have an
estimation of their neighbours' expected latencies to be able to make
forwarding decisions. We denote by  the estimate
at node  of the latency from node  to the destination. In
Algorithm \ref{algo:DistLatMin}, each time two nodes meet, they update
these estimates and then recalculate their optimum forwarding rules. Theorem
\ref{theorem3} proves that this decentralized approach results in the
same global optimum solution. The proof of Theorem \ref{theorem3} is provided in Appendix \ref{pt3}.
\begin{algorithm}[!htb]
\caption{Decentralized Greedy Latency Minimization} \label{algo:DistLatMin}
\begin{spacing}{1.0}
\begin{algorithmic}[1]
		\vspace{3mm}
		\STATE{ \texttt{// Initialization}}
		\STATE{}
		\STATE{,  }
                      \WHILE{ Nodes continue to meet}
	      		\STATE{ \texttt{// Nodes  and  meet at time }}
		           \STATE{ Set  }
		            \STATE{ Set  }
			\STATE{Update  and identify the minimizing  }
			\STATE{Update  and
                          identify the minimizing   }
			\STATE{Set  and }
			\vspace{0.2cm}
		\ENDWHILE
\end{algorithmic}
\end{spacing}
\end{algorithm}

\begin{theorem}\label{theorem3}
The decision matrix  identified by Algorithm \ref{algo:DistLatMin} converges to  with probability .
\end{theorem}


We refer to our proposed decentralized greedy latency minimization algorithm (Algorithm \ref{algo:DistLatMin}) as MinLat and evaluate its efficiency in different random and real-world networks based on certain performance metrics in Section \ref{sec:res}. Regarding the computational complexity of finding the minimum expected latency in MinLat, the following lemma shows that the optimizations in lines 8 and 9 of this algorithm are linear fractional programs and can be solved quickly using variants from linear programming. Further details are available in Appendix \ref{pl2}. 

\begin{lemma} \label{lem2}
The minimization problem in Algorithm \ref{algo:DistLatMin},

can be converted to a linear programming problem.
\end{lemma}
Assuming that \eqref{opt_prob} can be solved in polynomial order
, the worst case complexity order of Algorithm
\ref{algo:cen} is  because in the th round of this
algorithm, \eqref{opt_prob} should be solved for each of the 
nodes that are not in the set . In Algorithm
\ref{algo:DistLatMin}, each time node  meets one of its
neighbours,it solves a problem of complexity . The only
information that a node needs to share when it meets another node is
its estimate of its own expected latency to the destination. In the
general case where messages can be destined to any node in the
network, this exchangable message could be a length  vector of
expected latencies to all nodes.

The following proposition provides a
bound on the expected convergence time of Algorithm
\ref{algo:DistLatMin}. The brief proof is provided in
Appendix~\ref{pt3}. The bound depends on the slowest meeting rate
between each node and its candidate relay nodes. This is a
conservative bound, since in practice, a node only needs to meet the
relay nodes to which it actually forwards data under the optimum forwarding rule.
\begin{proposition}\label{prop1}
The expected convergence time, , of Algorithm
\ref{algo:DistLatMin} is bounded as 
.
\end{proposition}

\subsection{Decentralized Approach with No A-priori Knowledge}\label{est}
In part \ref{dec_partial}, we assumed that as soon as a node meets
another node, it has a perfect knowledge of its meeting rate with that
node. In practice, a node will need to estimate its meeting rates with
the neighbours and periodically revise the estimation as meetings
occur (or fail to occur).
Consider an arbitrary pair of nodes that meet each other with rate
. We denote the  intermeeting time, which is
the time between  and  meetings, by
. For this specific pair of nodes,  is an independent
sample of an exponentially distributed random variable with parameter
. Using the maximum likelihood (ML) approach we can estimate the
parameter  after  samples. The
likelihood function  is maximized by 

Hence, under the exponential model, a node only needs to remember the
last time it met its neighbour and the number of times it has met that
neighbour. With these two pieces of information, it can update its
estimation of the meeting rate () from the previously estimated value () using the following equation.

Based on this argument, we develop a more practical version of MinLat
in which an arbitrary pair of nodes  and  use their estimated
meeting rates  in their calculations and
modify this estimation each time they meet. We refer to this version
of MinLat as MinLat-E.

Let  denote the time since the network began operating, and denote
by  the decision matrix achieved by MinLat-E
at time . Further, denote by
 the estimate at
node  at time  of the expected latency to the destination, when
the forwarding decision matrix is . This
estimate differs from that obtained in Algorithm 2,
, because the distributed algorithm calculates
them using estimated meeting rates . The
following theorem states that the achieved expected latencies,
 and the estimated expected
latencies, , converge in probability to the
optimum expected latencies . The proof is
provided in Appendix \ref{pt4}.

\begin{theorem}\label{theorem4}
 For any node in the network, the sequences of estimated and achieved latencies converge to its optimum expected latency in probability, i.e.,  and . More precisely for any ,


\end{theorem}
 We check the claims of Theorem \ref{theorem4} and investigate the
 convergence speed of MinLat-E through simulations in Section
\ref{sec:res}. 

\section{Simulation Results}\label{sec:res} 
In this section, we investigate the efficiency of our
proposed approach in modeling and solving the forwarding/routing problem in
different opportunistic network scenarios. 
We first test our algorithms using three different
networks to model the contacts between  mobile nodes.
The characteristics of the networks are derived from the Infocom05 dataset~\cite{cambridge-haggle-2006-01-31}. This data set is based on an
experiment conducted during the IEEE Infocom 2005 conference in Miami
where 41 Bluetooth enabled devices (Intel iMote) were carried by
attendees for 3 days. The start and end times of each contact
between participants were recorded.  The average time between node
contacts in the Infocom05 dataset is  seconds (
hours). In our processing, we only consider the contacts in which both devices
recognized each other so that an acknowledged message could be
transfered between them. 

In the first network, ({\em Net I}), we construct a contact graph using
an evolving undirected network model based on the preferential
attachment mechanism. We start with a small fully connected graph of
 vertices and add vertices to it one by one until the
graph consists of  nodes. At each step, the new vertex is
connected to  previously existing vertices. The probability
that the new vertex is connected to vertex  is  where  is the degree of  up to this stage. After
building the contact graph, we assign a parameter  to
each pair of nodes  and  which are connected in the contact
graph and assume that they meet with exponentially distributed
intermeeting times with parameter . We choose the parameters  from a uniform distribution with the same expectation as the average of node meeting rates observed in the Infocom05 dataset. 

In {\em Net II}, we set  to be equal to the inverse of
the average intermeeting time between nodes  and  in the
Infocom05 dataset. We are interested in the behaviour of the
algorithms in relatively sparse networks, so we limit the number of
neighbours of each node: node  is only connected to node  in the
contact graph if the meeting rate  is among the largest
 meeting rates of either node  or node . In our simulations, the meeting
times between nodes  and  for {\em Net II} are then chosen from an exponential
distribution with parameter . In the third experimental
network, {\em Net III}, we use the actual meeting times recorded in
the Infocom05 dataset. The analysis in~\cite{gao2009} indicates that
the distribution of individual intermeeting times for most pairs of
nodes can be approximated reasonably well by an exponential
distribution; on the other hand, the aggregate distribution of contact
times shows heavy-tailed behaviour and is better approximated using a
truncated power distribution~\cite{chaintreau2007impact, cai2009}.
Table \ref{table:networks} summarizes the properties of the test networks.

\begin{table}[hbt] 
\small
\caption{Test Network Properties} \centering \begin{tabular}{|c|c|c|c|c|} \hline \begin{tabular}{@{}c@{}}Net- \\work\end{tabular} & \begin{tabular}{@{}c@{}}Contact \\Graph\end{tabular} &  Parameters & \begin{tabular}{@{}c@{}}Intermeeting \\Times\end{tabular}& \begin{tabular}{@{}c@{}}Number \\of nodes\end{tabular}\\ [0.5ex] 
\hline {\em I} & \begin{tabular}{@{}c@{}} Preferential  \\Attachment\end{tabular} & \begin{tabular}{@{}c@{}}   \\ \end{tabular} & \begin{tabular}{@{}c@{}} Exponential  \\ : uniform\end{tabular} &  \0.3ex]
\hline
{\em III} & \begin{tabular}{@{}c@{}} Sparsified  \\Infocom \end{tabular} && Data-set times &  \\label{Twait}
E(T_{w})=\frac{1}{\sum\limits_{k \in \mathcal{S}_i} \lambda_{ik}}
 \label{Lid_expansion} L_{id}(\bP)=E(T_{w})+\sum_{j
    \in \mathcal{S}_i}\frac{\lambda_{ij}}{\sum\limits_{k \in \mathcal{S}_i}
    \lambda_{ik}} [ p_{ij}L_{jd}(\bP)+(1-p_{ij})L_{id}(\bP)]

\frac{\partial L_{id}}{\partial p_{ij}}=\frac{\lambda_{ij}[\sum_{k\in \mathcal{S}_i}\lambda_{ik} p_{ik} (L_{jd}(\bP^\prime)-L_{kd}(\bP^\prime))-1]}{(\sum_{k\in \mathcal{S}_i}\lambda_{ik} p_{ik})^2}

\forall i \in \mathcal{N}': \quad L_{id}^* < L_{id}(\mathbf{P}^*)
\label{Tk}
E(T_k-T_{k-1})=E(\underset{i \in \{1,...,k-1\}}{\max} x_i)

E(T_k-T_{k-1})=\sum_{i=1}^{k-1} \frac{1}{\lambda_{ki}}-\sum_{i=1}^{k-1} \sum_{j=i+1}^{k-1} \frac{1}{\lambda_{ki}+\lambda_{kj}}
+ \sum_{i=1}^{k-1} \sum_{j=i+1}^{k-1}\sum_{l=j+1}^{k-1} \frac{1}{\lambda_{ki}+\lambda_{kj}+\lambda_{kl}}-\dots
\label{optLin}
\begin{aligned}
& \widehat{L}_{id}(i)=\underset{\mathbf{p}^i}{\text{min}}
& & \frac{\mathbf{c}^T \mathbf{p}^i +\alpha}{\mathbf{d}^T \mathbf{p}^i +\beta} \\
& \text{subject to}
& & \mathbf{A}\mathbf{p}^i\leq \mathbf{b}
\end{aligned}

 \mathbf{A}_{ij} =
  \begin{cases}
   1 & \text{if } i<|\cS_i| \quad \& \quad j=i \\
   -1 & \text{if } i>|\cS_i| \quad \& \quad j=i-|\cS_i| \\
   0       & \text{otherwise }
  \end{cases}

\bx=\frac{1}{\mathbf{d}^T\bp^i}\bp^i \qquad y=\frac{1}{\mathbf{d}^T\bp^i}

\begin{aligned}
& \underset{\mathbf{x},y}{\text{min}}
& & \mathbf{c}^T\mathbf{x}+\alpha y \\
& \text{subject to}
& & \mathbf{d}^T\mathbf{x}=1\\
& & & \mathbf{A}\mathbf{x} \leq \mathbf{b}y\\
& & & y\geq 0
\end{aligned}
\label{consistency}
\forall \epsilon_{ij}>0 : \underset{t -> \infty}{\lim} P(|\hat{\lambda}_{ij}-\lambda_{ij}|< \epsilon_{ij})=1

P((1-\epsilon_0)\lambda_{ij}<\widehat{\lambda}_{ij,t}<
(1+\epsilon_0)\lambda_{ij}) > (1-\delta) \label{prob_bound}
\label{tighbound}
\frac{(1-\epsilon_0)^{i-1}}{(1+\epsilon_0)^i} L_{id}(\mathbf{B}^*) < \widetilde{L}_{id,t}(\mathbf{B}^*,i) < \frac{(1+\epsilon_0)^{i-1}}{(1-\epsilon_0)^i} L_{id}(\mathbf{B}^*)

\frac{L_{1d}(\mathbf{B}^*)}{1+\epsilon_0}=\frac{1}{1+\epsilon_0}\frac{1}{\lambda_{1d}}<\widetilde{L}_{1d,t}(\mathbf{B}^*,1)=\frac{1}{\widehat{\lambda}_{1d,t}}
<\frac{1}{1-\epsilon_0}\frac{1}{\lambda_{1d}}=\frac{L_{1d}(\mathbf{B}^*)}{1-\epsilon_0}


\label{lhat}
\begin{aligned}
\frac{(1-\epsilon_0)^{\hat{k}-1}}{(1+\epsilon_0)^{\hat{k}}} L_{\hat{k}d}(\widetilde{\mathbf{B}}_{t})<\widetilde{L}_{\hat{k}d}(\widetilde{\mathbf{B}}_{t},\hat{k})<\frac{(1+\epsilon_0)^{\hat{k}-1}}{(1-\epsilon_0)^{\hat{k}}} L_{\hat{k}d}(\widetilde{\mathbf{B}}_{t})
\end{aligned}

\begin{aligned}
\frac{(1-\epsilon_0)^{N-1}}{(1+\epsilon_0)^{N}} L_{kd}(\mathbf{B})<\widetilde{L}_{kd}(\mathbf{B},k)<\frac{(1+\epsilon_0)^{N-1}}{(1-\epsilon_0)^{N}} L_{kd}(\mathbf{B})
\end{aligned}

L_{id}(\mathbf{B}^*)<(\frac{1-\epsilon_0}{1+\epsilon_0})^{2N-1} L_{id}(\widetilde{\mathbf{B}}_t)\,\,,

\epsilon_0<\frac{e^{\frac{\ln(K)}{2N-1}}-1}{e^{\frac{\ln(K)}{2N-1}}+1}\,,
\label{2conv}
\begin{aligned}
&\forall j \in \{1,...,k-1,d\}\cap \mathcal{S}_k: \\
&\quad \quad \{\widehat{\lambda}_{kj} \widetilde{L}_{jd}(\widetilde{\mathbf{B}}_t,k)\}\xrightarrow{d} \lambda_{kj} L_{jd}(\mathbf{B}^*) \in \mathbb{R}\,\,,\\
\end{aligned}
\label{2conv2}
\{\Sigma_{j \in \{1,...,k-1,d\}} \hat{b}^*_{kj} \widehat{\lambda}_{kj}\} \xrightarrow{p} \Sigma_{j \in \{1,...,k-1,d\}} b^*_{kj} \lambda_{kj}\,\,.

Property 1 in combination with \eqref{2conv} and \eqref{2conv2} results in the statement of the theorem.

For the achieved expected latencies, , as opposed to those estimated at
the nodes, the proof is more straightforward. For a given decision
matrix, , the expected latencies
s are functions of the true
meeting rates  and are thus not random variables. Thus, the sequence
 is only a function of the random sequences ,  via the optimization
that determines . Since we have
 established that  converges in probability to
, it follows that  converges in
probability to  due to property 1.

\end{proof}

\ifCLASSOPTIONcaptionsoff
  \newpage
\fi

\bibliographystyle{IEEEtran}
\bibliography{references}

\end{document}
