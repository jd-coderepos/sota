\documentclass{article}

\usepackage{spconf,amsmath,epsfig,amssymb,mathrsfs,psfrag,epsf,enumerate,bm}
\usepackage{amstext,amsfonts,amssymb}
\usepackage{float}
\usepackage{graphicx}
\usepackage{cite}
\usepackage{subfigure}
\usepackage{url}
\usepackage{shadow,color,pifont,times,rotate}





\def\pth#1{\left(#1\right)}                \def\stdpth#1{(#1)}
\def\acc#1{\left\{#1\right\}}              \def\stdacc#1{\{#1\}}
\def\cro#1{\left[#1\right]}                \def\stdcro#1{[#1]}
\def\bars#1{\left|#1\right|}               \def\stdbars#1{|#1|}
\def\norm#1{\left\|#1\right\|}             \def\stdnorm#1{\|#1\|}
\def\scal#1{\left\langle#1\right\rangle}   \def\stdscal#1{\langle#1\rangle}
 
\def\bigpth#1{\bigl(#1\bigr)}              \def\biggpth#1{\biggl(#1\biggr)}
\def\bigacc#1{\bigl\{#1\bigr\}}            \def\biggacc#1{\biggl\{#1\biggr\}}
\def\bigcro#1{\bigl[#1\bigr]}              \def\biggcro#1{\biggl[#1\biggr]}
\def\bigbars#1{\bigl|#1\bigr|}             \def\biggbars#1{\biggl|#1\biggr|}
\def\bignorm#1{\bigl\|#1\bigr\|}           \def\biggnorm#1{\biggl\|#1\biggr\|}
\def\bigscal#1{\bigl\langle#1\bigr\rangle} \def\biggscal#1{\biggl\langle#1\biggr\rangle}

\def\Bigpth#1{\Bigl(#1\Bigr)}              \def\Biggpth#1{\Biggl(#1\Biggr)}
\def\Bigacc#1{\Bigl\{#1\Bigr\}}            \def\Biggacc#1{\Biggl\{#1\Biggr\}}
\def\Bigcro#1{\Bigl[#1\Bigr]}              \def\Biggcro#1{\Biggl[#1\Biggr]}
\def\Bigbars#1{\Bigl|#1\Bigr|}             \def\Biggbars#1{\Biggl|#1\Biggr|}
\def\Bignorm#1{\Bigl\|#1\Bigr\|}           \def\Biggnorm#1{\Biggl\|#1\Biggr\|}
\def\Bigscal#1{\Bigl\langle#1\Bigr\rangle} \def\Biggscal#1{\Biggl\langle#1\Biggr\rangle}


\def\diag{{\mathrm{diag}}}              \def\Diag#1{{\mathrm{diag}}\bigcro{#1}} \def\Diagold#1{{\mathrm{diag}}\cro{#1}}
\def\tr{{\mathrm{tr}}\,}                \def\Tr#1{{\mathrm{tr}}\bigcro{#1}}                     \def\Trold#1{{\mathrm{tr}}\cro{#1}}
\def\rg{{\mathrm{rg}}\,}                \def\Rg#1{{\mathrm{rg}}\bigcro{#1}}                     \def\Rgold#1{{\mathrm{rg}}\cro{#1}}
\def\esp{{\mathrm{E}}\,}              \def\Esp#1{{\mathrm{E}}\bigcro{#1}}  \def\Esph#1#2{\underset{{#1}}{\mathrm{E}}\bigcro{#2}}               \def\Espold#1{{\mathrm{E}}\cro{#1}} 
\def\var{{\mathrm{var}}\,}              \def\Var#1{{\mathrm{var}}\bigcro{#1}}           \def\Varold#1{{\mathrm{var}}\cro{#1}}
\def\cov{{\mathrm{Cov}}\,}              \def\Cov#1{{\mathrm{Cov}}\bigcro{#1}}           \def\Covold#1{{\mathrm{Cov}}\cro{#1}}




\def\Cos#1{\cos\cro{#1}}
\def\Sin#1{\sin\cro{#1}}
\def\Exp#1{\exp\cro{#1}}
\def\Log#1{\log\cro{#1}}
\def\Ln#1{\ln\cro{#1}}                  
\def\Det#1{\det\bigcro{#1}}
\def\Erf#1{\mathrm{erf}\cro{#1}} 
\def\Erfc#1{\mathrm{erfc}\cro{#1}}
\def\Erfcx#1{\mathrm{erfcx}\cro{#1}}

\def\cond{\textrm{Cond}\,}              \def\Cond#1{\cond\bigpth{#1}}

\def\sinc{{\mathrm{sinc}}\,}    \def\Sinc#1{{\mathrm{sinc}}\bigcro{#1}} \def\Sincold#1{{\mathrm{sinc}}\cro{#1}}
\def\rang{{\mathrm{rang}}\,}    \def\Rang#1{\rang\bigcro{#1}}                                   \def\Rangold#1{\rang\cro{#1}}
\def\ker{\textrm{Ker}\,}                \def\Ker#1{\ker\bigcro{#1}}                                     \def\Kerold#1{\ker\cro{#1}}
\def\img{\textrm{Im}\,}                 \def\Img#1{\img\bigcro{#1}}                                     \def\Imgold#1{\img\cro{#1}}
\def\vect{{\mathrm{Vect}}\,}    \def\Vect#1{\vect\bigcro{#1}}                                   \def\Vectold#1{\vect\cro{#1}}
\def\sgn{{\mathrm{sgn}}}                \def\Sgn#1{\sgn\bigcro{#1}}                                     \def\Sgnold#1{\sgn\cro{#1}}

\def\Pr{\mathop{\textrm{Pr}}}

\def\reel{{\mathrm{Re}}}                \def\Reel#1{{\mathrm{Re}}\cro{#1}}
\def\card{{\mathrm{Card}}}              \def\Card#1{{\mathrm{Card}}\cro{#1}}

\def\sh{{\mathrm{sh}}}                \def\ch{{\mathrm{ch}}}                \def\th{{\mathrm{th}}}                \def\coth{{\mathrm{coth}}}            \def\coeffbin#1#2{\pth{\setlength{\arraycolsep}{.1em}\barr{c}#1\\#2\earr}}


\def\Div{{\mathrm{div}}}                                                                                        \def\Rotv{\overrightarrow{\mathop{{\mathrm{rot}}}}}             \def\Gradv{\overrightarrow{\mathop{{\mathrm{grad}}}}}           

\def\IF{\text{if\:}}             \def\SI{\text{si\:}}
\def\If{\text{If\:}}             \def\Si{\text{Si\:}}
\def\AND{\text{and\:}}           \def\ET{\text{et\:}}
\def\OR{\text{or\:}}             \def\OU{\text{ou\:}}
\def\THEN{\text{then\:}}         \def\ALORS{\text{alors\:}}
                                 \def\DOU{\text{d'o\`u\:}}
\def\WHERE{\text{where\:}}       \def\Ou{\text{o\`u\:}}
\def\WHEN{\text{when\:}}         \def\QUAND{\text{quand\:}}
\def\FOR{\text{for\:}}           \def\POUR{\text{pour\:}}
\def\FORALL{\text{for all\:}}    \def\POURTOUT{\text{pour tout\:}}
\def\ST{\text{s.t.\:}}           \def\SC{\text{s.c.\:}}
\def\SUBJTO{\text{subject to\:}} \def\SOUSC{\text{sous contraintes\:}}
\def\OTHERWISE{\text{otherwise}} \def\SINON{\text{sinon}}
\def\WITH{\text{with\:}}         \def\AVEC{\text{avec\:}}
\def\IN{\text{in\:}}             \def\DANS{\text{dans\:}}



\def\arrayp{\renewcommand{\arraystretch}{.7}\setlength{\arraycolsep}{2pt}}
\def\tabp{\renewcommand{\arraystretch}{.7}\setlength{\tabcolsep}{2pt}}



\newsavebox{\fminibox}
\newlength{\fminilength}
\newenvironment{fminipage}[1][\linewidth]
  {\setlength{\fminilength}{#1}\begin{lrbox}{\fminibox}\begin{minipage}{\fminilength}}
  {\end{minipage}\end{lrbox}\noindent\fbox{\usebox{\fminibox}}}



\def\M{^{-1}} \def\T{^\tD} \def\+{^\dagger}
\def\I{\,|\,}           \def\J{\mathop{\,;\,}}  \def\w{,\thinspace}
\def\W{,\thickspace}
\def\ldotsv{,\,\ldots,\,}
\def\V{,}               \def\e#1{.10^{#1}}      

\def\egdef{\stackrel{\Delta}{=}}
\def\nequiv{\not\kern-.05em\equiv}
\def\egal{\kern-.5em=\kern-.5em}        \def\propt{\kern-.2em\propto\kern-.2em} \def\dans{\in\!}                        \def\pourtt{\forall\,}                  \def\wh#1{\widehat{#1}}                 \def\wt#1{\widetilde{#1}} 

\def\argmax{\mathop{\mathrm{arg\,max}}} \def\argmin{\mathop{\mathrm{arg\,min}}} \def\Argmax#1#2{\displaystyle \argmax_{#1}\left\{{#2}\right\}} \def\Argmin#1#2{\displaystyle \argmin_{#1}\left\{{#2}\right\}}  

\def\EL{\mathrm{EL}}
\def\Vect{\mathop{\text{Vect}}}

\def\froc#1#2{{#1/#2}}                  \def\fric#1#2{\frac1{#2}#1}
\def\fracds#1#2{\frac{\displaystyle#1}{\displaystyle#2}}
\def\diff#1#2{{\frac{d#1}{d#2}}}
\def\dd{\,d}                            \def\derpar#1#2{{\frac{\partial #1}{\partial #2}}}
\def\derpor#1#2{{\froc{\partial #1}{\partial #2}}}
\def\parsec#1#2#3{{\frac{\partial^2 #1}{\partial #2\,\partial #3}}}
\def\parsecd#1#2{{\frac{\partial^2 #1}{{\partial #2}^2}}}
\def\porsec#1#2#3{{\froc{\partial^2 #1}{\partial #2\,\partial #3}}}
\def\porsecd#1#2{{\froc{\partial^2 #1}{{\partial #2}^2}}}
\def\pornd#1#2#3{{\froc{\partial^{#3} #1}{{\partial #2}^{#3}}}}
\def\intdouble{\int\kern-0.3em\int}
\def\inttriple{\int\kern-0.3em\int\kern-0.3em\int}
\def\prods{\mathop{\text{\footnotesize}}}

\def\rond#1{\overset{\kern-0.33em~_\circ}{#1}}
\def\rondit[#1]#2{\overset{\kern#1~_\circ}{#2}}

\def\incirc#1{\pscirclebox[framesep=1pt]{\scriptsize#1}}
\def\incircp#1{\pscirclebox[framesep=1pt]{\tiny#1}}
\def\outcirc#1{\raisebox{-3pt}{\huge}\hspace*{-.45cm}\text{\incirc{#1}}}



 \def\x{{\bm x}}
\def\L{{\cal L}}
\def\H{{\mathbf H}}
\def\HH{\widehat{{\mathbf H}}}
\def\tg{\mathcal{H}}
\def\y{{\bm y }}
\def\s{{\bm s }}
\def\z{{\bm z }}
\def\W{{ W}}
\def\WT{{\widetilde{ W}}}
\def\esp{{\mathbb E}}
\def\eye{{\mathbb I}}
\def\err{\boldsymbol{\mathcal E}}
\def\sig{\mathbf \Sigma}
\def\Zset{\mathbb{Z}}
\DeclareMathAlphabet{\mathpzc}{OT1}{pzc}{m}{it}
\newcommand{\deq}{\stackrel{d}{=}}
\newcommand{\mb}{\mathbf}
\newcommand{\ul}{\underline}
\newcommand{\ol}{\overline}
\newcommand{\bs}{\boldsymbol}
\newcommand{\mc}{\mathcal}
\newcommand{\tc}{\textcolor}
\newcommand{\tb}{\textbf}
\newtheorem{theorem}{Theorem}[section]
\newtheorem{lemma}[theorem]{Lemma}
\newtheorem{proposition}[theorem]{Proposition}
\newtheorem{corollary}[theorem]{Corollary}
\newtheorem{remark}[theorem]{Remark}
\newtheorem{definition}[theorem]{Definition}
\title{MIMO-OFDM Optimal Decoding And Achievable Information Rates \\Under Imperfect Channel Estimation}

\name{Sajad Sadough, Pablo Piantanida, and Pierre Duhamel}

\address{Ecole Nationale Sup\'erieure de Techniques Avanc\'ees, 75739 Paris Cedex 15, France\\
 Laboratoire des Signaux et Syst\`emes, CNRS/Sup\'{e}lec, F-91192 Gif-sur-Yvette, France\\ Email:\{sadough, piantanida, pierre.duhamel\}@lss.supelec.fr \
\label{eq:model1}
\y = \H \, \s + \z

\label{eq:model2}
\y_k = \H_k \, \s_k + \z_k \;\;\;\; k=1,...,M,

   \label{eq:model3}
         \mathbf{Y}_T = \H_k \,\mathbf{S}_T + \mathbf{Z}_T
  
\label{eq:model4}
\HH_k^{\rm ML}=\mathbf{Y}_T \,\mathbf{S}_T^\tg\,(\mathbf{S}_T\mathbf{S}_T^\tg)^{-1}=\H_k + \mathbf{\mathcal{E}}

\label{eq:model5}
f(\H_k|\HH_k^{\rm ML}) = \mathcal{CN}( \sig_{\Delta} \HH_{\rm ML} ,\, \eye_{M_T} \otimes \sig_{\Delta} \sig_{ \mathcal{E} } )

\label{eq:metric1}
\hat{\s}_k^{\rm ML}(\H_k)= \argmin_{\s_k \in \, \mathbb{C}^{M_T \times 1}}\, \big \{ \,\mathcal{D}_{\rm ML}(\s_k,\y_k,\H_k) \big \},

\label{eq:metric3}
\hat{\s}_k^{\rm ML}(\HH_k)= \argmin_{\s_k \in \, \mathbb{C}^{M_T \times 1}}\, \big \{ \,\|\y_k - \H_k \s_k \|^2 \big \}_{\big |_{\H_k=\HH_k} }

	\label{eq:metric4}
             \WT(\y_k|\HH_k,\s_k) &= \esp_{\H_k | \HH_k} \big[ \W(\y_k|\H_k,\s_k) \big | \, \HH_k \big] \notag \\
             &= \int_{\H_k \in \mathbb{C}^{M_R \times M_T}} \W(\y_k|\H_k,\s_k) \; f(\H_k|\HH_k) \;\;\; {\rm d}\H_k.

  \label{eq:metric5}
  \left\{\begin{array}{ll}
       \boldsymbol{\mu}_{_\mathcal{M}} &= \delta \, \HH_k \, \s_k, \\
       \sig_{_\mathcal{M}} &= \sig_z  + \delta \, \sig_{\mathcal{E}} \, \| \s_k \|^2.
    \end{array} \right.

\label{eq:metric6}
  \hat{\s}_k^{\mathcal{M}}(\HH_k)= \argmin_{\s_k \in \, \mathbb{C}^{M_T \times 1}}\, \big \{ \,\mathcal{D}_{\mathcal{M}}(\s_k,\y_k,\H_k) \big \},

\label{eq:metric7}
& \mathcal{D}_{_\mathcal{M}}(\s_k,\y_k,\HH_k) \triangleq - \ln \WT(\y_k|\s_k,\HH_k) \notag \\
&= M_R \ln \, \pi (\sigma^2_z + \delta \, \sigma^2_\mathcal{E}\, \|\s_k\|^2)+ \frac{\|\y_k-\delta\,\HH_k\,\s_k\|^2}{\sigma^2_z+\delta\,\sigma^2_\mathcal{E}\,\|\s_k\|^2}

\label{eq:mimoRx1}
L(d_k^{i,m})=\log \frac{P_{\rm dem}(d_k^{i,m}= 1)|\y_k,\H_k)}{P_{\rm dem}(d_k^{i,m}=0|\y_k,\H_k)}.

\label{eq:mimoRx2}
L(d_k^{i,m})= \log \frac{\sum \limits_{\s_k:d_k^{i,m}=1} \W(\y_k|\s_k,\H_k) \prod \limits_{\stackrel{n=1}{n\neq m}}^{B M_T} P^1_{\rm dec}(d_k^{i,n})}{\sum \limits_{\s_k:d_k^{i,m}=0} \W(\y_k|\s_k,\H_k) \prod \limits_{\stackrel{n=1}{n\neq m}}^{B M_T} P^0_{\rm dec}(d_k^{i,n})}\vspace{-2mm}

\label{mutinf}
I(S;Y)\triangleq \frac{1}{M} \sum_{k=1}^M \log_2 \det \left(\eye_{M_R}+ \Upsilon_k \sig_{\s_k} \Upsilon_k^{\tg}\sig_k^{-1} \right)

&\mathrm{(c_{1})}: \det \left(\Upsilon \sig_\s \Upsilon^\tg + \sig \right)=\det\left(\H \sig_\s \H^\tg + \sig_{\z}\right) \label{c1}\\
&\mathrm{(c_{2,k})}: \| \Upsilon_k + a_k \widehat{\H}_k \|^2\leq  \| \H_k +a_k \widehat{\H}_k \|^2 + \textrm{Cst}_k,\label{c2}

\Gamma (-n,t)=\frac{(-1)^n}{n !} \Big[\Gamma (0,t)-\exp(-t)\sum\limits_{i=0}^{n-1}(-1)^i \frac{i !}{t^{i+1}} \Big], \vspace{-1mm}

C_{\mc{M}}(\H,\widehat{\H})=\left \{ \begin{array}{ll} \min \limits_{\underline{\mu}} \,\,\,\,\,\,  \frac{1}{M}\sum\limits_{k=1}^{M} \sum\limits_{i=1}^{M_R}\log_2 \left(1+\displaystyle{\frac{\bar{P}|\mu_{k,i}|^2}{\sigma^2( \underline{\mu_k})}}\right), \\ \textrm{subject to} \,\,\,\,\,\,  \| \underline{\mu}_k +a  \mb{\tilde{h}_k}  \|^2\leq b_k. \end{array}\right.\label{final_opt}

C_{\mc{M}}(\H,\widehat{\H})=\frac{1}{M}\sum\limits_{k=1}^M\log_2 \textrm{det}\left(\mathbb{I}_{M_R}+ \frac{\bar{P} \Upsilon_{\textrm{opt},k}\Upsilon_{\textrm{opt},k}^\tg}{ \sigma_k^{2}( \underline{\mu}^{\textrm{opt}}_{\mc{M},k})}   \right),\label{acievable_rates}

\underline{\mu}^{\textrm{opt}}_{\mc{M},k}=\left\{ \begin{array}{ll} \displaystyle{\left(\frac{\sqrt{b_k}}{\|\mb{\tilde{h}}_k\|}-|a_k|\right)\mb{\widetilde{h}}_k} & \,\, \textrm{if ,} \\
  \underline{0} & \,\, \textrm{otherwise}. \end{array}  \right. \label{solution_mu}

\label{mismached_ML_capa}
\underline{\mu}^{\textrm{opt}}_{\textrm{ML},k}=\frac{ \mathbb{R}e\{{\rm Tr}(\Lambda_k^\tg \tilde{\mb{h}}_k) \}}{\|\tilde{\mb{h}}_k\|^2}\tilde{\mb{h}}_k,

P_{_{\mathcal{M}}}^{\mathrm{out}}(R,\widehat{\mathbf{H}})=\mathrm{Pr}_{\mathbf{H}| \widehat{\mathbf{H}}} \big(\mathbf{H}\in \Lambda_{_\mathcal{M}}(R,\widehat{\mathbf{H}})|\widehat{\mathbf{H}}\big), \notag

\label{capout}
C_{\mathcal{M}}^{\mathrm{out}}(\gamma,\widehat{\mathbf{H}})=\sup_R\big\{R \geq 0: P_{_{\mathcal{M}}}^{\mathrm{out}}(R,\widehat{\mathbf{H}})\leq \gamma\big\}

\overline{C}_{\mathcal{M}}^{\; \mathrm{out}}(\gamma)=\mathbb{E}_{\widehat{\mathbf{H}}}\big[C_{_\mathcal{M}}^{\mathrm{out}}(\gamma,\widehat{\mathbf{H}})\big]. \label{EQ-capacity}

\overline{C}_{_\mathcal{G}}^{\; \mathrm{out}}(\gamma)=\mathbb{E}_{\widehat{\mathbf{H}}}\big[C_{_\mathcal{G}}^{\mathrm{out}}(\gamma,\widehat{\mathbf{H}})\big], \label{perfect-capacity}

C_{_\mathcal{G}}(\mathbf{H},\widehat{\mathbf{H}})=\frac{1}{M}\sum\limits_{k=1}^{M} \log_2 \det \left(\eye_{M_R} +\frac{\bar{P}\;\H_k\H_k^\tg}{\sigma^2_z}\right),

and  are random channels drawn from the posterior distribution of \eqref{eq:model5}.
\vspace{-3mm}
\section{Numerical Results}
\label{sec:simul}


Next, the performance of the improved decoder is measured in terms of bit error rate (BER) and achievable outage rates. The binary information data are encoded by a rate  NRNSC code with constraint length 3 defined in octal form by (5,7). Throughout the simulations, each frame is assumed to consists of one OFDM symbol with 50 subcarriers belonging to a 16-QAM constellation with Gray labeling. The interleaver is a pseudo-random one operating over the entire frame with size  bits.
For each transmitted frame, a different realization of a Rayleigh distributed channel has been drawn and remains constant during the whole frame. Besides, it is assumed that the average pilot symbol energy is equal to the average data symbol energy. Moreover, the number of decoding iterations are set to 4.

Fig. \ref{bermimofdm} depicts the BER performance over a  MIMO-OFDM channel estimated by  pilot symbols per frame. As observed, the increase in the required  caused by CEE is an
important effect of imperfect CSIR in the case of mismatched ML decoding.
The figure shows that the SNR to obtain a BER of  with  pilots is reduced by about  dB if the improved decoder is used instead of the mismatched decoder. We also notice that the performance loss of the mismatched receiver with respect to the derived receiver becomes insignificant for . This can be explained from the expression of the metric \eqref{eq:metric7}, where we note that by increasing the number of pilot symbols, this expression tends to the classical Euclidean distance metric. However, the proposed decoder outperforms the mismatched decoder especially when few numbers of pilot symbols are dedicated for channel estimation.



Fig. \ref{capmimofdm22} shows the expected outage rates (in bits per channel use) corresponding to the transmission of one OFDM symbol with  subcarriers and  antennas, achieved by adopting mismatched ML decoding and the improved decoder (expression \eqref{EQ-capacity}). For comparison, we also display the upper bounds on these achievable outage rates (expression \eqref{perfect-capacity}) and the ergodic capacity.
 pilot symbols are sent per frame for CSIR acquisition and the outage probability has been fixed to . At a mean outage rate of  bits, we note that the achievable rate of the mismatched decoder is about  dB of SNR far from the capacity provided by the theoretical decoder. As observed, by using the improved decoder, higher rates are obtained for any SNR values and the aforementioned SNR gap is reduced by about  dB.

Similarly, Fig. \ref{capmimofdm44} compares the outage rates in the case of a  MIMO-OFDM channel estimated by  training symbols. Again, it can be observed that the proposed decoder achieves higher rates than the mismatched decoder. However, we note that the increase in the SNR (at a given mean outage rate) induced by using the mismatched decoder rather than the improved decoder, is less than that obtained for a  MIMO-OFDM channel (see Fig. \ref{capmimofdm22}). This can be explained by noting that when , we must send  pilot symbols per frame which yields a more accurate estimate of the channel and consequently bring closer the performance of the two decoders. This observation is in consistence with those presented in \cite{garg05} where it is reported that the performance degradation due to imperfect channel estimation can be reduced by increasing the number of antennas.
\vspace{-1.5mm}
\section{Conclusion}
\label{sec:concl}
A Bayesian approach for the design of a decoding method that is robust to channel estimation errors was presented.
The robustness of our method comes from the averaging of the decoding rule, that would be used if the channel were perfectly known, over all channel estimation errors.
We also derived the expression of the achievable rates associated to our proposed decoder and compared it to that provided by the classical mismatched decoding approach. As a practical application, the proposed decoder was exploited for iterative BICM decoding of MIMO-OFDM under imperfect channel knowledge.
Simulation results showed that, without introducing any additional complexity, the proposed decoder outperforms the classical mismatched approach in terms of BER and achievable outage rates for short training sequences.
Although our proposed method outperforms mismatched decoding, the derivation of a practical decoder achieving the maximum outage rate under imperfect channel estimation is still an open problem.
\vspace{-1.5mm}
\bibliographystyle{IEEEbib}
\bibliography{biblio}
\begin{figure}[!htb]
\vspace{-3mm}
\centering
\includegraphics[width=0.5\textwidth,height=0.29\textheight]{ber-MIMO-spawc07.eps}
\caption{BER performance of the proposed and mismatched decoders in the case of a  MIMO-OFDM Rayleigh fading channel with  subcarriers for training sequence length .} \label{bermimofdm} \vspace{-4mm}
\end{figure}
\begin{figure}[!htb]
\vspace{2mm}
\centering
\includegraphics[width=0.5\textwidth,height=0.28\textheight]{cap22_16_2pil.eps}
\caption{Expected outage rates of a  MIMO-OFDM system with  subcarriers versus SNR for  pilots.}\label{capmimofdm22}\vspace{-4mm}
\end{figure}
\begin{figure}[!htb]
\vspace{2mm}
\centering
\includegraphics[width=0.5\textwidth,height=0.28\textheight]{cap44_16_4pil.eps}
\caption{Expected outage rates of a  MIMO-OFDM system with  subcarriers versus SNR for  pilots.}\label{capmimofdm44}\vspace{-4mm}
\end{figure}
\end{document}
