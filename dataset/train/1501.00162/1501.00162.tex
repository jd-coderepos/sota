\documentclass{article}

\usepackage[utf8]{inputenc}

\usepackage{amsmath}
\usepackage{amsfonts}
\usepackage{amssymb}
\usepackage{amsthm}
\usepackage{mathabx}
\usepackage{geometry}
\usepackage{graphicx}
\geometry{
	b5paper,
	margin=1.5cm,
	top=1.75cm,
	bottom=1.75cm
}

\newcommand{\llinnm}[2]{\operatorname{L}_{\operatorname{lin}}({#1}, {#2})}
\newcommand{\llinn}[1]{\llinnm{#1}{#1}}
\newcommand{\hlinr}[1]{\operatorname{H}_{\operatorname{lin}}^{#1}}
\newcommand{\hlin}{\operatorname{H}_{\operatorname{lin}}}
\newcommand{\hfact}[2]{\operatorname{H}_{\operatorname{factor}}({#1}, {#2})}
\newcommand{\rot}[2]{\operatorname{H}_{\operatorname{rot}}^{{#1}, {#2}}}

\newcommand{\bin}[3]{\operatorname{\mathbf{bin}}({#1}, {#2}, {#3})}
\newcommand{\lbin}[2]{\operatorname{\mathbf{lbin}}({#1}, {#2})}
\newcommand{\vbin}[2]{\operatorname{\mathbf{bin}}({#1}, {#2})}
\newcommand{\vlbin}[1]{\operatorname{\mathbf{lbin}}({#1})}


\newcommand{\probs}[2]{\operatorname{\mathbf{Pr}}_{{#1}}\left[{#2}\right]}
\newcommand{\prob}[1]{\probs{}{#1}}
\newcommand{\expects}[2]{\operatorname{\mathbf{E}}_{{#1}}\left[{#2}\right]}
\newcommand{\expect}[1]{\expects{}{#1}}
\newcommand{\inu}{\in_U}

\newtheorem{lemma}{Lemma}
\newtheorem{theorem}{Theorem}
\newtheorem{claim}{Claim}
\newtheorem{corollary}{Corollary}


\title{Expected number of uniformly distributed balls in a most loaded bin using placement with simple linear functions}
\author{Martin Babka\thanks{Research supported by the Czech Science Foundation grant GA14-10799S.}}

\begin{document}

\maketitle

\begin{abstract}
We estimate the size of a most loaded bin in the setting when the balls are placed into the bins using a random linear function in a finite field.
The balls are chosen from a transformed interval. 
We show that in this setting the expected load of the most loaded bins is constant.

This is an interesting fact because using fully random hash functions with the same class of input sets leads to an expectation of  balls in most loaded bins where  is the number of balls and bins.

Although the family of the functions is quite common the size of largest bins was not known even in this simple case.
\end{abstract}

\section{Introduction}

Our basic task is to estimate estimate the size of a largest bin in a special case of the balls and bins model. This models simply means that the balls are randomly thrown into bins. The process of their placement is of a various study -- its randomness, independence and other properties lead to various bin sizes.
The most simple model is to use fully random functions or some kind of their approximation to place the balls. There is a plenty of results, i.e. estimates of bin sizes, for various placement processes.

When the balls are thrown independently at random to the bins the expected size of the largest bin is .

One of the first results were shown by Carter and Wegman \cite{cw} and this model was used to design universal and perfect hashing. They showed that the expected size of a bin is a constant when the placement is done by the functions which we will refer to as simple linear functions. These functions are two-wise independent and thus achieve  expected size of a largest bin.

It is also possible to use functions with higher degrees of independence and obtain better bounds. There are lower bounds for on the speed of such functions, size needed to represent and the size of the largest bin and independence they achieve given by Siegel \cite{siegel}.

The need to improve the size of the largest bins lead to two-choice paradigm. Out of two bins, hence we use two functions, the balls is placed into the smaller one. In this model the size of the largest bin is  where  is the number of balls and bins shown by Azar et al \cite{azar} and improved by V\"{o}cking \cite{vocking}.

Nowadays more complicated family of functions are studied in \cite{wieder}. The functions no longer rely on high degree of independence but are designed so that they achieve small largest bins even with high probability.

Our model exhibits the use of simple linear functions and the balls are chosen from an interval in . Such model has a constant size of largest bins.

\section{Notation and definitions}
\label{sec:notation}
We refer to the set  as to . 
In the whole text we assume that  is a fixed prime. 
The set of chosen balls is denoted by .
The number of bins is the same as the number of balls and is denoted by , i.e. .

For each pair  we define the function  as  and the function  as .

The multiset of simple linear functions mapping  to the range  is denoted by  and is defined as .
For a function  we define the size of -th bin as  and the maximal size of the bin as .

In the following text we fix the probability space to be formed by a uniform choice of .
The notation  and  then refers to the random variables formed by the mentioned random uniform choice.

For an element  we define the value ; that is how many ``leaps'' are created by applying the function  on the element  in the field .

\section{Collision probability for three elements}

We first study the probability of collision of three arbitrary elements.
By collision of the elements we understand the event when all of the elements are mapped to the same element in  by the randomly chosen linear function.

We fix three different elements  and we count the number of pairs  such that .

We start by simplifying to the case when  and the third element  for a suitable  such that  depending on the choice of .

\begin{lemma}[Transformation lemma]
\label{lemma:transformation}
Let  be arbitrary different elements. Moreover assume that . Then there exist an element  such that

\end{lemma}
\begin{proof}
The idea of the proof is simple. We show that there is a one-to-one map between simple linear functions mapping  to  and simple linear functions transforming  to the same elements.

In the first part of the proof we observe that combining simple linear functions with a linear function in  does not change the the probability space.
There is a single linear function transforming  to  in  which we refer to as .
Finally we choose  so that  and the proof is finished.

We show that the elements  can be transformed to the elements  so that the probability of the mappings from the statement of the lemma remains the same.

Choose  so that .
Observe that the mapping  is a one-to-one map on .
If there is another pair  such that , then  and . Thus the mapping is injective.
Also for arbitrary  the element  is mapped to .

The compound function  is exactly equal to the function ; this also follows from the fact that the set of all linear functions in  forms a group with the operation of compounding functions. 

Let .
From the previous we can conclude that the combination of a function  with a fixed function  is a one-to-one map in the space of functions .
Also observe that the composition of a function  with  can not change the probability (count of the functions) of mapping arbitrary three elements to a their prescribed images.

There is also a single function , i.e. a single function , transforming the elements  and  to  and  in the field  without taking modulo . It is the function  and .
To prove the lemma we choose  such that , i.e. .
\end{proof}

Lemma \ref{lemma:transformation} shows that the probability properties, e.g. collision, mapping to the prescribed elements, for the elements  are the same as for the elements  where  comes from the previous lemma.

Next we estimate the collision probability for the elements .

\begin{lemma}[Probability of collision of three elements]
\label{lemma:probability-3-elements}
Let  be arbitrary element.

\end{lemma}
\begin{proof}
We count the number of functions  such that  for some .
For each  it holds that  and .

Whenever the elements  and  are mapped to the same element  it must hold that  and . Hence

From which we obtain the following sequence of equations



Since  is a prime we conclude the fact that .
We estimate the collision probabilities from the two statements following from the previous formulas:


The statement (\ref{3-prob-2-statement}) roughly means that out of  possible values for  only the  fraction may generate the collision of the three elements. Notice that for a fixed  it holds that  equals is a subinterval of .
From (\ref{3-prob-1-statement}) we can observe that only the  fraction from the possible values of  lying in the appropriate intervals allowed by valid values of  are causing collisions.

For the rest of the proof fix the value of . 
First, we show that the values of  such that  form disjoint intervals in  each of size at most .
Then we count the number of values  in an interval causing collisions -- using (\ref{3-prob-1-statement}).
And finally we count the number of the valid intervals.

Let , then it holds that . Immediately we get that . The total number of values of , i.e. integers, in each valid interval is at most . The ceiling must be applied. For example assume an interval of length of 1.5 starting at point 0.8 -- it contains two integer points 1 and 2. This happens whenever  is an integer.

Now fix the value  such that .
In order to estimate the number of values of  causing the collisions we split into two cases according to the value of .

\subparagraph{The first case, .} 
From the two previous statements we conclude that


\subparagraph{The second case, .}
As in the first case it must hold that


In both cases, there are at most  values of  satisfying the second condition.
Also for each satisfying value of  there are at most  values of  causing the collision.

In both cases and for each  it holds that the probability of collision of the three elements is bounded by

\end{proof}

The worst possible case is for  and the probability is roughly . 
When , the formula is a great overestimate as shown in Figure~1.

\begin{figure}[h]
	\label{fig:probability-3}
	\centering
	\includegraphics[width=8cm]{coll-3-21787-512-scaled}
	\caption{The function of probability of collision of the elements  with respect to . Notice that the probability is decreasing in the part when  and is almost symmetric. In this figure  and .}
\end{figure}

\begin{corollary}
\label{co:d-elements}
Let . Then .
\end{corollary}
\begin{proof}
When all the elements from  collide, then the elements  must collide as well. The probability of the collision of  is hence a valid upper bound on the probability of the collision of the whole interval. The statement is then a direct application of Lemma~\ref{lemma:probability-3-elements}.
\end{proof}

For completeness we just show a simple fact that our probability estimate is tight when we have a stronger assumption, namely we assume .

\begin{lemma}
\label{lm:0-d-prob-lower-bound}
If  and , then 
\end{lemma}
\begin{proof}
For a fixed , if  and , then the elements  collide.
For each  there are at least  such values of .

We conclude that the number of pairs  making the elements collide is at least


Thus the resulting probability is at least .
\end{proof}

\section{The expected size of most loaded bins}

First we study the role of the parameter  in the hash function .

The following lemma states that the effect of  on  is not asymptotic since it more or less only shifts the largest bin.
\begin{lemma}
\label{lm:b-zero}
Assume that   and . Then 
\end{lemma}
\begin{proof}
Let  be elements of bin , i.e. .
For each  we have that 


Notice that the two possible new bins are either  or .
The lemma now follows from the following two observation.
First each original bin is either shifted and keeps its size or is split into two possibly uneven shifted bins -- hence .
And notice that each new bin can only contain elements from at most two different original bins and thus .
\end{proof}

For completeness let us mention that the change of the sign of  has almost no effect on .
\begin{lemma}
\label{lemma:sign-a}
Assume that   and  such that . Then 
\end{lemma}
\begin{proof}
Similarly as in the proof of the previous lemma. Let  be elements of bin , i.e. .
Let , then .
Observe that  holds only when .
The bin  is thus moved to the bin  and the lemma holds.
\end{proof}

Obviously allowing zero makes only a negligible change.
\begin{corollary}
Let , then

\end{corollary}

For the choice of  we show that the expected size of a most loaded bin is within . 
This can be compactly formulated as follows.
\begin{theorem}
\label{thm:interval-constant}
Assume that , then

\end{theorem}
\begin{proof}
By Lemma~\ref{lm:b-zero} we may assume that the chosen function has  without asymptotically increasing the expected size of the largest bin. In the proof of the claims we thus assume that the chosen linear function is exactly the function . Moreover we assume that . Notice that this assumption adds exactly  to the computed expected value which is .

Observe that each bin is formed by a single arithmetic progression. Notice that since  is a prime it holds that  is co-prime with .
The reason can be stated as follows.
Let  be two elements in a single bin, then for  it holds that  or .

All the solutions of the equation  where  form a finite arithmetic progression.
For the proof of the previous statement notice that since  is a prime it holds that  is co-prime with . 

In addition a difference  and a given length , , there is a canonical value  such that if there is a bin of size at least , then there is another bin formed by an arithmetic progression of length at least  with the same difference  having  as the minimal element. If , we choose . Otherwise we put .

After establishing the previous facts we simply compute the expected value of  using the following idea. Now we allow  to have arbitrary value.

Assume that , then there is an arithmetic progression chosen from  of size at least  collapsing into a single bin, here we use Lemma~\ref{lm:b-zero}. Since for a fixed difference and length we have its canonical position there are at most  possible arithmetic progressions from which we choose from. By Corollary~\ref{co:d-elements} we upper bound the probability of the collapse of the arithmetic progression as 



Hence for  we have


Then we simply conclude that


\end{proof}

We can conclude the main result, i.e. each set transformable to  in  has constant sized largest bins.
\begin{corollary}
Let , . 
If , then .
\end{corollary}
\begin{proof}
Direct corollary of Theorem~\ref{thm:interval-constant} since by Lemma~\ref{lemma:transformation} (extended to all the elements of ) the probabilistic properties of  do not change under the transformation .
\end{proof}


\begin{thebibliography}{32}
\bibitem{cw}
J.L. Carter, and M.N. Wegman. 
\newblock Universal Classes of Hash Functions.
\newblock Journal of Computer and System Sciences, 18. pages 143--154, 1979.

\bibitem{siegel}
A. Siegel. 
\newblock On universal classes of extremely random constant-time hash functions.
\newblock SIAM Journal on Computing, 33(3). pages 505--543 (electronic), 2004

\bibitem{wieder}
L.E. Celis, O. Reingold, G. Segen, and U. Wieder
\newblock Balls and Bins: Smaller Hash Families and Faster Evaluation
\newblock Foundations of Computer Science (FOCS), 2011 IEEE 52nd Annual Symposium, pages 599 -- 608, 2011

\bibitem{azar}
Y. Azar, A. Broder, A. Karlin, and E. Upfal
\newblock Balanced allocations.
\newblock SIAM Journal on Computing, 29(1). pages 180--200, 1999.

\bibitem{vocking}
B. V\"{o}cking
\newblock How asymmetry helps load balancing.
\newblock In Proceedings of the Fortieth Annual Symposium on Foundations of Computer Science. pages 131--140, 1999.
\end{thebibliography}

\end{document}
