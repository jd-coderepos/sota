\documentclass{llncs}

\usepackage{graphicx}
\usepackage{tabularx}
\pagestyle{plain}
\begin{document}

\newcolumntype{Z}{>{\centering\arraybackslash}X} 
\mainmatter              
\title{Dynamic Top- Dominating Queries}
\titlerunning{Dynamic Top- Dominating Queries}  \author{Andreas Kosmatopoulos \and Kostas Tsichlas}
\authorrunning{Kosmatopoulos and Tsichlas} \tocauthor{Andreas Kosmatopoulos and Kostas Tsichlas}
\institute{Informatics Department, Aristotle University of Thessaloniki, Greece\\
\email{\{akosmato, tsichlas\}@csd.auth.gr}}

\maketitle              
\begin{abstract}
Let  be a dataset of  -dimensional points. The top- dominating query aims to report the  points that dominate the most points in . A point  dominates a point  iff all coordinates of  are smaller than or equal to those of  and at least one of them is strictly smaller. The top- dominating query combines the dominance concept of maxima queries with the ranking function of top- queries and can be used as an important tool in multi-criteria decision making systems. In this work, we propose novel algorithms for answering semi-dynamic (insertions only) and fully dynamic (insertions and deletions) top- dominating queries. To the best of our knowledge, this is the first work towards handling (semi-)dynamic top- dominating queries that offers algorithms with asymptotic guarantees regarding their time and space~cost.
\keywords{computational geometry, dynamic data structures, dominance, top- preference query}
\end{abstract}

\section{Introduction} \label{section:Introduction}
In recent years, there has been an increasing research interest in preference queries, due to their ability to select the most interesting objects of a given dataset. The dataset objects are characterized by a number of often contradictory attributes; thus selecting a suitable subset becomes a challenging task.

As an example, consider a dataset containing hotels which will accommodate researchers during a conference. Assume that each hotel is represented as a -dimensional point of two attributes: its distance from the conference venue and its room price. Generally, a potential customer would be interested in hotels that have both of these attributes as minimized as possible. The solution would then consist of all the hotels that are in a sense more ``preferred'' than the others. There have been three different kinds of preference queries analyzed in literature which we discuss below. For the remainder of this introductory section we assume that  is a dataset of points in the -dimensional plane . Furthermore, Figure~\ref{fig:Hotels} illustrates a dataset of hotels that will serve as a running example.

\begin{figure}
\centering
\includegraphics[width=0.43\textwidth]{hotel_dataset}
\vspace{-0.4cm}
\caption{Hotel Dataset} \label{fig:Hotels}
\vspace{-0.6cm}
\end{figure}

A \emph{Top- query}~\cite{FaginLN01} provides a preference ranking for all the objects in  using a user-defined ranking function. Let  be a monotone ranking function used to assign a value to each point in . The top- query will return the  points in  with the smallest  value. For example, in Figure~\ref{fig:Hotels} the top- points for  are  and . We assume that  is small when compared to the dataset size (imagine a researcher asking for the top- among the  hotels in New York!). An advantage of top- queries is that the user-defined parameter  controls the output size. However, they are based on a preference function and as such different user-defined ranking functions may produce different rankings.

A \emph{Maxima (Skyline) query}~\cite{BorzsonyiKS01} returns all the points in the dataset that are not dominated by any other point. A point  dominates a point  if all the coordinates of  are smaller than the coordinates of  and at least one of the coordinates of  is strictly smaller than the respective coordinate of . The above definition assumes that smaller values are preferable to larger at all dimensions. For datasets where larger values are preferable, the definition may be altered accordingly. As an example, the maximal points of Figure~\ref{fig:Hotels} are ,  and . The major advantage of maxima queries is that they are based on the dominance concept and as a result, they do not require a user-defined preference ranking function. Furthermore, due to the definition of the maxima query, the results are not affected by potentially different scales at different dimensions. On the other hand, maxima queries do not control the size of their output and in extreme cases it can be as large as the dataset size.

A \emph{Top- Dominating query}~\cite{Papadias05} combines the dominance concept of maxima queries with the ranking function of top- queries. More specifically, each point  is assigned a dominance score which is equal to the number of points it dominates. A top- dominating query returns the  points in  with the highest dominance score. For instance, the top- dominating points of Figure \ref{fig:Hotels} are  () and  (). Top- dominating queries provide an intuitive way of ranking a dataset's objects by using their dominance relationship instead of a user-defined ranking function. In addition, their output size is bounded by the parameter  and they are unaffected by different data scales at different dimensions. In a sense, they combine the merits of maxima and top- queries. Consequently, top- dominating queries can be used as an important tool in multi-criteria optimization applications (e.g. in \cite{SkoutasSSKS09} they use a form of the dominance concept to rank web services under multiple criteria).

We define a query  in  to be decomposable~\cite{BenSax80} if its output can be computed accurately by performing  in a partition of . In contrast to top- and maxima queries, a top- dominating query is a \emph{non-decomposable query} since the score of each point is dependent on the coordinates of all the other points in . This fact greatly increases the difficulty of the problem.

In this work, we propose a novel solution for semi-dynamic and fully dynamic top- dominating queries on a dataset  of -dimensional points, where  is a user-defined parameter that is fixed between queries. In the semi-dynamic setting we support the insertion of new points, while in the fully dynamic setting we further support the deletion of existing points. To the extent of our knowledge, this is the first work that offers asymptotic time and space bounds for these particular queries. Note, that this paper concentrates on -dimensional data for two reasons. First, there is no previous work with asymtpotic guarantees and as a result, this paper provides a deeper understanding of the complexity of the problem. The second, more practical, reason is that many applications are inherently -dimensional. This is because, one often faces the situation of having to strike a balance between a pair of naturally contradicting
factors (e.g., price vs quality, space vs query time).

In the remainder of the introductory section we review related work and present our contributions. In Section~\ref{section:Preliminaries} we provide an overview of the basic concepts that will be used throughout the rest of this work. In Sections \ref{section:SemiDynTopKDom} and \ref{section:DynamicTopKDom} we formally describe our proposed solution for the semi-dynamic and fully dynamic top- dominating query, respectively. Finally, in Section \ref{section:ConclusionsFutureWork} we conclude and offer directions for future work.





\subsection{Related Work and Our Results} \label{subsection:RelatedWork}
The simplest approach to answering the top- dominating query would to be to compare each point  with every other point  in the dataset and increment 's score if it dominates . This results in  time cost and  space cost. An approach with lower time complexity would be to use a -dimensional range counting data structure. Assuming word size , for each point  in a dataset , the points lying in the query rectangle  can be counted in  time and  space using the method described in \cite{HeMunroWADS11}. The number of points found in  is equal to 's dominance score. In order to compute the dominance score of each point, we repeat the process for all the points in  in  total time. Insertions and deletions can be trivially supported in  time since one has to update the dominance scores of all points in the worst-case. In the following, we describe more elaborate methods to answer a top- dominating query.

Papadias et al.~\cite{Papadias05}, first proposed the -dimensional top- dominating query along with a solution based on the iterative computation of a dataset's maxima points. More specifically, they observed that the top- dominating point of a dataset is contained in the dataset maxima's points. This stems from the observation that for every point  not in the maxima, there exists a point  in the maxima that dominates it and, as a result,  has a larger score than . Thus, in their approach, they compute the set of maxima  (using the BBS algorithm \cite{Papadias05}) and compute the dominance score of all the points in . The point  with the highest score is the top- dominating point and is thereby reported. Finally,  is removed from the dataset and the procedure is repeated until  points have been reported. Since in the worst case the set of maxima may be the whole dataset, the algorithm requires  time and  space\footnote{The BBS algorithm is based on the use of R-trees which require linear space}.


Yiu and Mamoulis~\cite{YiuM09} suggested using aggregate R-trees (aR-trees) to efficiently compute -dimensional top- dominating queries. They provided various algorithms based on aR-trees that proved experimentally to be quite fast. They also make an analytic study making the assumption that the data points are uniformly and independently distributed in a domain space. The authors do not make any statement for the worst-case time complexity of the query but it is certainly .

 Both methods \cite{Papadias05,YiuM09} focus on the top- dominating query, where  is arbitrary. Update operations can be applied in both cases with a linear time cost. However, the top- dominating query has to be re-evaluated in both cases. Finally, both prove the efficiency of their approach experimentally (extensive experiments can be found in~\cite{YiuM09}).
\vspace{-0.5cm}
\begin{table}\centering\footnotesize
\caption{SD stands for Semi-Dynamic, where only insertions are supported while FD stands for Fully-Dynamic where both insertions and deletions are supported. \vspace{-0.3cm}}\label{table:OurResults}
\begin{tabular}{l|c|c|c}
\hline
Algorithm & Preprocessing & Query & Update (amortized)\\
\hline
SD/-list &  &  &  \\
SD/-list &  &  &  \\
FD/-list &  &  &  \\
FD/-list &  &  &  \\
\hline
\end{tabular}
\end{table}

\vspace{-0.5cm}
\noindent In this paper we tackle for the first time the problem of answering dynamic top- dominating queries providing a complexity analysis of all supported operations. Our algorithms work well under the assumption that  is a fixed user specified parameter which is small when compared to the size  of the dataset. We also attack the problem in two dimensions for reasons previously mentioned. Finally, our algorithms are based on a novel restricted dynamization of layers of maxima~\cite{BlunckVAlgorithmica10}. This is of independent interest in case we only need to access the first  layers of maxima.

We consider the problem in the semi-dynamic case (insertions only), where logarithmic  complexities are attained. However, in the fully-dynamic case, we are only able to attain polynomial complexities for update operations. For each case we provide two solutions (-list and -list) that provide a trade-off between update and query time. All our algorithms use linear space. Table \ref{table:OurResults} provides a detailed  overview of our results.




\section{Preliminaries} \label{section:Preliminaries}

Let  be a set of  -dimensional points  where  with  being the -coordinate of  and  the -coordinate of . A point  \emph{dominates} another point  iff . We use () to denote that  dominates . The \emph{dominance score} of a point  is equal to . We augment the definition of each point  to also include its score .
Thus, a \emph{top- dominating query} in  aims to report the  points of  with the highest dominance score.

The algorithms presented in the remaining sections are based on the concept of \emph{layers of maxima}. In order to compute the layers of maxima for a dataset  we perform a maxima query on , remove the answer set of points from  and repeat the process until no points remain in . The set that results from the -th maxima query forms the -th layer of maxima. By collecting all the layers, we form the layers of maxima of .







\section{Semi-Dynamic Top- Dominating Points} \label{section:SemiDynTopKDom}
In this section we propose a solution to the semi-dynamic top- dominating query problem and describe in detail the data structures and algorithms we use to achieve it.
Let  be a set of  -dimensional points. Recall that the semi-dynamic top- dominating query aims to report the  points in  with the highest dominance score where  is a fixed user-provided parameter. Furthermore,  is subject to insertions of new points. This poses an additional challenge since after inserting a new point, it is possible that the dominance score of many (or even all) the points in  must be updated. Individually updating the score of each such point would be computationally prohibitive so we follow a different approach and only update lazily the score of groups of points that are candidates for being in the final answer.

We first note that when a point  dominates another point , 's score is strictly greater than the score of :
\vspace{-0.2cm}


\vspace{-0.2cm}
Organizing  into layers of maxima offers an intuitive way of using the above property to eliminate points that are not possible to belong in the final answer. As an example, consider a top- dominating query in . The point with the highest dominance score is found in the first layer of maxima of  since all the points in the second and subsequent layers are dominated by at least one other point. Similarly, in a top- dominating query, the first point is found in the first layer and the second point is found in either the first or the second layer. In general, the following lemma holds for the top- dominating points (see Appendix~\ref{app:lemma1} for the proof):

\begin{lemma}\label{lemma:FirstKLayers}
The top- dominating points of  are found in the first  layers of maxima of .
\end{lemma}

A direct consequence of Lemma \ref{lemma:FirstKLayers} is that, when inserting a new point , we only need to update the scores of some points in the first  layers of maxima. However, some of the layers may have many points and thus individually updating the score of these points would result in a high update cost. To avoid this, after inserting a new point  in , we find only the first and last point that dominate  in each layer. This pair of points denotes an interval that marks all the points in each layer whose score must be updated. Consequently, by examining only  points in each layer the total update cost is reduced.

Lastly, an issue brought up by the use of layers of maxima is that the insertion of a new point  may create cascading changes to the structure of the layers. In particular, by inserting  into ,  must also be inserted in one of the layers of . Let  be that layer. The insertion of  in  may cause some of its points to be discarded as a result of them being dominated by . This group of points must be inserted into the next layer  possibly discarding some of the points in  in the process. Due to Lemma \ref{lemma:FirstKLayers} and the fact that only insertions are allowed, this chain of operations only has to be performed up until the -th layer.

To achieve efficient insertion, we model each layer as an -tree. In the following, we provide a detailed overview of the data structure and the operations it supports and then we describe the update and query algorithms for the semi-dynamic top- dominating query.




\subsection{The augmented -tree} \label{subsection:AugABTree}
We use an augmented leaf-oriented -tree to model each layer of maxima. Assume that  is a layer of maxima containing  points, i.e.  where . Since the points in  are totally ordered on each dimension\footnote{For two points  and  iff  then }, we can use a single -tree to search among the points in both dimensions. To achieve that, each inner node stores representative keys for both dimensions, instead of storing keys for only one of them.

For each node  of the tree with  we maintain a field . The field's contents denote a score that has to be added to the score of all the points in 's subtree. Finally, each node  with  is augmented with an -sized list  which stores the  points with the highest score in 's subtree\footnote{The height is measured from the leaves to the root of the tree}. The points in  are sorted according to their score. For the remainder of this work we assume that  since we present main memory algorithms. The following lemma provides the tree's total space cost (for a proof see Appendix~\ref{app:lemma2}).
\begin{lemma} \label{lemma:AugABTreeTotalSpace}
The total space required by an augmented -tree storing  points is .
\end{lemma}


The following lemma provides the time complexity for the construction of an augmented -tree over  points (for a proof see Appendix~\ref{app:construction}).
\begin{lemma} \label{lemma:construction}
The construction of an augmented -tree over  points can be carried out in  time, where  is a user-defined parameter.
\end{lemma}

\vspace{-0.5cm}
\subsubsection{Operations.} \label{subsubsection:AugABTreeOperations}
In this section all the operations supported by the augmented -tree are formally described. More specifically, the augmented -tree supports searching for a point, inserting a new point, or deleting an existing one. Furthermore, splits and concatenations between two different -trees are also supported.

The search operation \texttt{search} locates in the augmented -tree  a specific point  and can be performed with respect to either dimension of  by using the appropriate set of keys. Let  be a node of ,  be the -representative keys of 's children and  be the -representative keys of 's children. In order to search for a point  in , we begin at the root and search down until we reach a leaf. If the search is performed on the  dimension, we select the -th child of  such that . Otherwise, if the search is performed on the  dimension, we select the -th child of  such that . Since  is height-balanced, a search operation requires  time.

The rest of the operations are based on node splits and node merges. For reasons of clarity, we first describe how node splits and node merges are handled on the augmented -tree in relation to typical -trees.

The node split operation \texttt{node\_split} is performed similarly to the split operation of typical -trees with a few modifications. More specifically, before dividing a node  into two nodes  and  we check the contents of . If  stores a value different than , we add the contents of  to the  variable of 's children and set  to . Afterwards,  is divided into  and  and the keys for the  and  dimensions of  are ``shared'' between  and  in  time. After sharing the keys,  for  and  for  must be recomputed. If  we can compute  and  in  time by simultaneously traversing the   lists of 's and 's children respectively. As a result, the split operation requires  time in this case.

If  then the children of  and  are not augmented with  lists and thus computing  and  cannot be performed using the above procedure. In this case, we can compute  and  by simply traversing the  list and assigning each of its points  to either  or . In order to do this, we first find the child  of  that contains  in  time using 's representative keys. Afterwards, we discover whether  is a child of  or  using the representative keys of  and , and then assign  to  or  respectively. Since these steps are repeated for each point in  the split operation totally requires  time. Finally, if  then the split operation requires  time since  is not augmented with . In conclusion, a split operation is performed in  time in every case.



For the merge operation \texttt{node\_merge} we follow a similar procedure to the merge operation of standard -trees. More specifically, before merging two nodes  and  into  we check the contents of  and . If  stores a value different than  we add the contents of  to 's children and set  to . We follow the same procedure for . Then,  and  are merged into  and the keys for the  and  dimensions of  are derived from the keys of  and  in  time. After merging the keys,  must be recomputed. To achieve this, we simultaneously traverse  and  and store the  points with the highest score in  in  time.
As in the split operation, if  the merge operation requires  time due to the fact that  is not augmented with . As a result, the merge operation requires  time in all cases.

Operation \texttt{insert}, inserts a point  in . The point is inserted as a leaf in  and the tree is rebalanced using node splits. Since there are  node splits the time cost to insert a point is .

Operation \texttt{delete}, removes a point  from . The leaf corresponding to the point is removed and the resulting tree is rebalanced. There are  merges and a possible terminating split and as a result the time cost to delete a point is .

Using the node split and node merge operations as building blocks, we can define two additional operations on the augmented -trees: Tree Concatenation and Tree Split. For both the operations, we use the definition and algorithms provided in \cite{Mehlhorn84}.

Operation \texttt{ConCat}, concatenates two augmented -trees  and  into a third augmented -tree . This operation can only be performed if . In a tree concatenation there is one merge operation and up to  split operations performed. Since merge and split operations cost  time, a tree concatenation operation requires  time.

Operation \texttt{Split}, splits an augmented -tree  into two augmented -trees  and  at element  with respect to the one of the two dimensions, so that  and . In a tree split operation the starting -tree is first split into two forests of trees. Then, the roots of the trees in each forest are merged with each other recursively. Splitting the tree into two forests requires  time and since there are  merges for both forests, each requiring  time, a tree split operation requires  time. The following theorem summarizes the discussion on the -tree.

\begin{theorem} \label{theorem:AugABTreeCombined}
Given  -dimensional points  where  and a parameter , we can construct in  time an augmented -tree  that uses  space, and supports \texttt{search} in  time, \texttt{insert} and \texttt{delete} in  time and \texttt{Split} in  time. Furthermore, given an augmented -tree  where , \texttt{ConCat} is supported in  time.
\end{theorem}




\subsection{Insertion} \label{subsection:Insertion}
Let  be a point to be inserted into . Furthermore, let  be the first  layers of maxima of . Before inserting  we compute its dominance score using the dynamic range counting data structure proposed in \cite{HeMunroWADS11}\footnote{The data structure is built only once as a preprocessing step before the first insertion}. The data structure supports queries in  worst-case time and insertions and deletions in  amortized time (assuming word size  in the word RAM model).

Afterwards, we find if  must be inserted in one of  by searching each of the  respective -trees for . Starting from  and iterating towards , we search each tree for  both in the  and in the  dimension and retrieve the predecessor of  in the  dimension and the successor of  in the  dimension. If neither of those two points dominate , we insert  in the tree's respective layer and stop the iteration. Otherwise, the iteration may end without any layer satisfying the above condition. In that case,  does not become a member of any of the  first layers.

If we do not insert  in any of the  first layers then we only have to update the scores of some points in each of . Otherwise, assume that  is inserted into  where . Then we have to update scores of points in  and alter the structure of . We first describe how to handle \emph{score updating} on a layer and afterwards how to \emph{alter a layer's structure} using tree splits and tree concatenations.


To update the score of the points in a layer , we perform this procedure. We search the augmented -tree of  for  and . All points whose score must be updated lie to the left of  and to the right of . Since updating the score of each point would be time consuming, we only find the two boundary points that define the above interval and mark the subtrees between them.

We start from  of the two search paths and move up towards the root, adding  to  if  is a node hanging to the left of the search path for  or to the right of the search path for . Using this method we denote that the score of all the points in 's subtree must be incremented by one, without actually visiting the points themselves. Adding  to  does not change  since we increment the score of all the points in 's subtree and thus their relative order according to score remains unchanged. For each node  with , instead of incrementing , we exhaustively check the points in  and individually update their score based on if they are dominating . For each node with  no action is necessary since all the points in its subtree can be found in the  list of its ancestor with . Thus, at the end, we have indirectly marked all the points between  and  for score increment.

Finally, we update the  lists of the nodes in the search path as a result of modifying the  fields of their children. Starting from  and moving towards the root, we recursively compute the  list of each node  by simultaneously merging the  lists of its children. While merging the lists, we also add the contents of each node's  field to the score of the node's  list points so as to take into account the score changes caused by the insertion of . At the end, the  list found in the root of the -tree will have the correct top- points for that layer of maxima.

Updating the  fields requires  total time while merging the top- lists of a node's children requires  time and as a result the total cost for all the nodes in the search paths of the tree is  time. Thus, the total time required to update scores in a layer is .

On the second case, the point  may have to be inserted in a layer of maxima . Since inserting or deleting points from the layer of maxima one-by-one would be time consuming, we insert the point and remove the now-dominated points with a series of tree splits and tree concatenations. First, we find the interval as previously by querying the layer of maxima tree  for  and . Then we perform the following sequence of operations in order: 1) \texttt{Split}, 2) \texttt{Split}, 3) \texttt{insert} and 4) \texttt{ConCat}.
The layer of maxima tree  for  now correctly has  inserted and every point previously in  that is now dominated by  (i.e. ) has been discarded.

The tree  is now propagated to the next layer of maxima  where we repeat the above procedure with  as the input. Since  may have more than one points, instead of inserting them one-by-one we perform a tree concatenation at step (3) instead of an insertion. Finally, the insertion spot of  in  can be found by querying the tree of  for  where  is the  coordinate of the leftmost point in  and  is the  coordinate of the rightmost point in . In each layer of maxima we perform a series of  splits and concatenations. Thus, the total time required to alter a layer's structure given a point or an -tree as an input is .

As described in the beginning of the section, after an insertion a layer must either update the score of some of its points or alter its structure. Since either case requires  time, the time cost of manipulating the  first layers after an insertion is . Adding the cost of maintaining the dynamic range counting data structure, the total insertion cost is  amortized time (assuming word size ).


\subsection{Query} \label{subsection:Query}
To find the top- dominating points of , we apply Lemma \ref{lemma:TopKDomLemma} on all the top- lists found in the root of each -tree of each of the  first layers of maxima. Let  be the list returned by Lemma \ref{lemma:TopKDomLemma}. By selecting the -th order statistic of  we obtain the dominance score  of the -th top dominating point. Finally, we traverse all the top- lists we previously collected and report all points with score larger than . Since the lists are sorted according to their score, we can stop traversing a list when a point with score lower than  has been found. Applying Lemma \ref{lemma:TopKDomLemma} requires  time while finding the -th order statistic of  requires  time. Finally, traversing all requires  time in total. By combining all of the above, we achieve  query time.

\subsection{Reducing the Update Cost} \label{subsection:ReduceUpdateCost}
We can reduce the algorithm's update cost by shrinking the size of the  list in each node of each -tree. In particular, instead of storing  points in each  list we only store . This removes the cost of computing  lists during each node split or merge since each  list can be computed using  comparisons. As a result, node splits and merges cost  time and updating the score of points in a layer or altering its structure costs  time. This brings the total insertion cost down to  amortized time.

This change also implies that at the time of a query, each -tree's root only stores  element with the highest score in the layer and as a result we can no longer directly apply Lemma \ref{lemma:TopKDomLemma}. To overcome this we build a Strict Fibonacci Heap~\cite{BrodalLT12} by inserting each point with the highest score from each layer. By querying the heap we are able to find (and delete) the top- dominating point. After deleting a point  (belonging in a layer ) from the heap, we have to replace it by the point of  with the next highest score. This point can be found by querying 's tree for . Due to the definition of  lists, the point with the next highest score in  is guaranteed to be amongst the   lists of each node in the search path. We insert all  such points in the heap and repeat the process until  points have been deleted from the heap. In order to not have any duplicate points in the heap, we also employ a marking process. Deleting a point from the heap requires  time, while adding  points also requires  time. Since there aren't any duplicate points in the heap and the process is repeated  times, the query phase of the algorithm requires  time. The discussion of this section can be summarized in the following theorem:

\begin{theorem} \label{thm:sd}
Given a set  of  -dimensional points, we can support update operations in  amortized time and top- dominating queries in  time. Alternatively, we can support update operations in  amortized time and top- dominating queries in  time.
\end{theorem}



\section{Dynamic Top- Dominating Points} \label{section:DynamicTopKDom}

The algorithms presented so far only support insertions due to the fact that all operations could be restricted in the first  layers of maxima. However, assume the deletion of a point  in layer . It is possible that some points from  might have to be inserted in  as a result of them not being dominated by any other point in  apart from . This brings a cascading of restructuring operations since some of the points in  might have to be inserted in . Thus, a deletion operation may reach the last layer of  in the worst case.

Our algorithm for semi-dynamic settings can be extended to fully dynamic settings through the use of the global rebuilding technique~\cite{OvermarsL81}. More specifically in an update operation, instead of manipulating only the first  layers we perform score updates and layer restructuring operations in the first  layers. A deletion of an existing point can be defined in a similar way to the insertion of a point with each layer requiring either score updating or restructuring. Since we stop restructuring operations on a predefined point, after the -th deletion the -th layer will have become invalid. As a result, after  deletions, only the first  layers remain valid and at that point we rebuild the entire layers of maxima data structure. We also recompute the score of each point and reconstruct the -trees. Further details and the proof of the following theorem can be found in Appendix~\ref{app:global}.


\begin{theorem} \label{thm:fd}
Given a set  of  -dimensional points, we can support update operations in  amortized time and top- dominating queries in  time. Alternatively, we can support update operations in  amortized time and top- dominating queries in  time.
\end{theorem}


\section{Conclusions and Future Work} \label{section:ConclusionsFutureWork}
In this work we proposed algorithms for answering semi-dynamic and fully dynamic top- dominating queries where  is a parameter that is fixed between queries. The algorithms we described offer asymptotic guarantees for both their time and space cost and they are the first to do so. An interesting future work direction would be to lower the update cost for the full dynamic algorithms by avoiding the global rebuilding technique. Another research direction would be to provide solutions for this problem appropriate for secondary memory.


\bibliographystyle{splncs}
\bibliography{references}

\newpage
\appendix

\section{Reporting Lemma}

We use the following lemma from \cite{Frederickson82}:
\begin{lemma}\label{lemma:TopKDomLemma}
    Let  be arrays of values from a totally ordered set such that each array is sorted.
    Given an integer , there is a comparison-based algorithm that finds in  time a value  that is greater than at least  but at most  values in .
\end{lemma}

This lemma forms the basis in allowing us to efficiently find the -th point with the highest score out of a collection of ordered lists and is used in Sections \ref{section:SemiDynTopKDom} and \ref{section:DynamicTopKDom}.


\section{Proof of Lemma~\ref{lemma:FirstKLayers}} \label{app:lemma1}

\textbf{Lemma} \textit{The top- dominating points of  are found in the first  layers of maxima of .}

\begin{proof}
If  has only  or less layers of maxima, the lemma obviously holds. Otherwise, assume that a point  belongs in the -th layer of maxima, where . There are at least  points dominating  and due to Equation \ref{eq:DomScoreProperty} all of them have a larger score than . As a result,  is not included in the top- dominating points of .\qed
\end{proof}



\section{Proof of Lemma~\ref{lemma:AugABTreeTotalSpace}} \label{app:lemma2}

\textbf{Lemma} \textit{The total space required by an augmented -tree storing  points is .}

\begin{proof}
All the nodes with height lower than  only store  additional information so their total space cost is . There are  nodes with height higher than or equal to  each augmented with a -sized list. The total space cost of this part of the data structure is . As a result, the total space cost of the entire data structure is .\qed
\end{proof}


\section{Proof of Lemma~\ref{lemma:construction}} \label{app:construction}


\textbf{Lemma} \textit{The construction of an augmented -tree over  points can be carried out in  time, where  is a user-defined parameter.}

\begin{proof}
In order to construct the leaf-oriented augmented -tree we follow a bottom-up approach and assume that the input points are sorted according to their dimensions. The augmented -tree is constructed in a similar way to a typical -tree with an additional issue. At first, the nodes of the augmented -tree are constructed by scanning the input points, creating the leaves and then recursively creating the inner nodes from bottom to top. Each node is only visited once so the procedure up to this point requires  time.

The last step is to compute the  lists for all nodes with . For each node  with , the  list must be computed from the  lists of 's children. By simultaneously traversing the   lists of 's children we can compute  in  time. There are  nodes with  and since this process is repeated for every node, the time required is .

Finally, we compute the  lists for each node  with . Since 's children are not augmented with  lists we follow a different approach. We sort all the points found in 's subtree\footnote{There are  points in 's subtree since } in  time and store them in . There are  nodes with  and thus this step requires  total time.

The lemma follows by the above discussion.
\end{proof}


\section{Global Reconstruction for the Fully Dynamic Case} \label{app:global}

At the beginning, we compute the dominance score of each point in  using the -dimensional range counting data structure. This data structure does not need to be rebuilt since it is not related to the layers of maxima of .

The next step is to construct the layers of maxima for all the points in . This can be performed using the in-place algorithms proposed in \cite{BlunckVAlgorithmica10}. Afterwards, we build an augmented -tree on the points of each layer of maxima.

Computing the dominance score of all the points in  using \cite{HeMunroWADS11} requires  time and  space. The construction of all the layers of maxima can be done in  time and  space \cite{BlunckVAlgorithmica10}. Constructing the augmented -tree for all layers-of-maxima requires  time and  space.

We perform the global rebuilding step once in every  updates. A update up to the -th layer, requires  amortized time. We perform  such updates and then we globally rebuild the data structures in  time so the amortized time for an update is . Finally, the methods described in Section \ref{subsection:ReduceUpdateCost} can also be applied in the full dynamic setting.



\end{document}
