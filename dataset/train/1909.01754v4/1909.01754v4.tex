\section{Results and Discussion}
\label{sec:results}

In this section, we report the experiments carried out to verify the effectiveness of the proposed \gls*{alpr} system.
We first assess the detection stages separately since the regions used in the \gls*{lp} recognition stage are from the detection results, rather than cropped directly from the ground truth.
This is done to provide a realistic evaluation of the entire \gls*{alpr} system, in which well-performed vehicle and \gls*{lp} detections are essential for achieving outstanding recognition results.
Afterward, our system is evaluated in an end-to-end manner and the results achieved are compared with those obtained in previous works and by commercial~systems.

\subsection{Vehicle Detection}
\label{sec:results:vehicle_detection}

In this stage, we employed a confidence threshold of~$0.25$ (defined empirically) to detect as many vehicles as possible, while avoiding high \gls*{fp} rates and, consequently, a higher cost of the proposed \gls*{alpr} system.
The following parameters were used for training the network:~$60$K iterations~(max batches) and learning rate~=~[$10$\textsuperscript{-$3$},~$10$\textsuperscript{-$4$},~$10$\textsuperscript{-$5$}] with steps at $48$K and $54$K~iterations.

The vehicle detection results are presented in Table~\ref{tab:results:vehicle_detection}.
In the average of five runs, our approach achieved a recall rate of $99.92$\% and a precision rate of $98.37$\%. 
It is remarkable that the network was able to correctly detect all vehicles (i.e., recall = $100$\%) in $5$ of the $8$ datasets used in the experiments.
Some detection results are shown in Fig.~\ref{fig:results:veicle_detection_tps}. 
As can be seen, well-located predictions were attained on vehicles of different types and under different conditions.

\begin{table}[!htb]
\centering
\caption{Vehicle detection results achieved across all datasets.}
\label{tab:results:vehicle_detection}
\vspace{1mm}
\resizebox{0.725\columnwidth}{!}{
\begin{tabular}{@{}ccc@{}}
\toprule
\textbf{Dataset} & \textbf{Precision (\%)} & \textbf{Recall (\%)} \\ \midrule
\caltech & $100.00\pm0.00$ & $100.00\pm0.00$ \\
\englishlpd & $99.04\pm0.96$ & $100.00\pm0.00$ \\
\stills & $97.42\pm1.40$ & $100.00\pm0.00$ \\
\chinese & $99.26\pm1.00$ & $99.50\pm0.52$ \\
\aolp & $96.92\pm0.37$ & $99.91\pm0.08$ \\
\openalpreu & $99.27\pm0.76$ & $100.00\pm0.00$ \\
\ssig & $95.47\pm0.62$ & $99.98\pm0.06$ \\
\dataset & $99.57\pm0.07$ & $100.00\pm0.00$ \\ \midrule
\textbf{Average} & $\textbf{98.37}\boldsymbol{\pm}\textbf{0.65}$ & $\textbf{99.92}\boldsymbol{\pm}\textbf{0.08}$ \\ \bottomrule
\end{tabular}
}
\end{table}

\begin{figure}[!htb]
    \centering
    
    \includegraphics[width=0.99\columnwidth]{Figure08.pdf} 
    
    \vspace{-2mm}
    
    \caption{Some vehicle detection results achieved in distinct datasets. Observe that vehicles of different types were correctly detected regardless of lighting conditions~(daytime and nighttime), occlusion, camera distance, and other factors.}
    \label{fig:results:veicle_detection_tps}
\end{figure}

To the best of our knowledge, with the exception of the preliminary version of this work~\citep{laroca2018robust}, there is no other work in the \gls*{alpr} context where both cars and motorcycles are detected at this stage.
This is of paramount importance since motorcycles are one of the most popular transportation means in metropolitan areas, especially in Asia~\citep{hsu2016comparison}.
Although motorcycle \glspl*{lp} may be correctly located by \gls*{lp} detection approaches that work directly on the frames, they can be detected with fewer false positives if the motorcycles are detected~first~\citep{hsu2015comparison}. 

The precision rates obtained by the network were only not higher due to unlabeled vehicles present in the background of the images, especially in the \aolp and \ssig datasets. 
Three examples are shown in Fig.~\ref{fig:results:vehicle_detection_fps_fns}a.
In Fig.~\ref{fig:results:vehicle_detection_fps_fns}b, we show some of the few cases where our network failed to detect one or more vehicles in the image. 
As can be seen, such cases are challenging since only a small part of each undetected vehicle is visible.

\begin{figure}[!htb]
    \centering
    
    \includegraphics[width=0.99\columnwidth]{Figure09.pdf}
    
    \vspace{-2mm}
    
    \caption{\gls*{fp} and \gls*{fn} predictions obtained in the vehicle detection stage. As can be seen in (a), the predicted \glspl*{fp} are mostly unlabelled vehicles in the background. In (b), one can see that the vehicles not predicted by the network (i.e.,~the \glspl*{fn}) are predominantly those occluded or in the~background.}
    \label{fig:results:vehicle_detection_fps_fns}
\end{figure}

\subsection{License Plate Detection and Layout Classification}
\label{sec:results:lp_detection}

In Table~\ref{tab:results:lp_detection}, we report the results obtained by the modified Fast-YOLOv2 network in the \gls*{lp} detection and layout classification stage.
As we consider only one \gls*{lp} per vehicle image, the precision and recall rates are identical.
The average recall rate obtained in all datasets was $99.51$\% when disregarding the vehicles not detected in the previous stage and $99.45$\% when considering the entire test set.
This result is particularly impressive since we considered as incorrect the predictions in which the \gls*{lp}~layout was incorrectly classified with a high confidence value, even in cases where the \gls*{lp} position was predicted~correctly.

\begin{table}[!htb]
	\centering
	
	\captionsetup{position=top}
	
	\caption{Results attained in the \gls*{lp} detection and layout classification stage. The recall rates achieved in all datasets when disregarding the vehicles not detected in the previous stage are presented in~(a), while the recall rates obtained when considering the entire test set are listed in~(b).}
	\label{tab:results:lp_detection}
    \resizebox{0.9\columnwidth}{!}{
	\subfloat[][\label{tab:results:lp_detection_a}]{\begin{tabular}{@{}cc@{}}
		\toprule
		\textbf{Dataset} & \textbf{Recall (\%)} \\ \midrule
		\caltech & $99.13\pm1.19$ \\
		\englishlpd & $100.00\pm0.00$ \\
		\stills & $100.00\pm0.00$ \\
		\chinese & $100.00\pm0.00$ \\
		\aolp & $99.94\pm0.08$ \\
		\openalpreu & $98.52\pm0.51$ \\
		\ssig & $99.83\pm0.26$ \\
		\dataset & $98.67\pm0.25$ \\ \midrule
		\textbf{Average} & $\textbf{99.51}\boldsymbol{\pm}\textbf{0.29}$ \\ \bottomrule
	\end{tabular}} \quad \vline \quad
	\subfloat[][\label{tab:results:lp_detection_b}]{\begin{tabular}{@{}cc@{}}
		\toprule
		\textbf{Dataset} & \textbf{Recall (\%)} \\ \midrule
		\caltech & $99.13\pm1.19$ \\
		\englishlpd & $100.00\pm0.00$ \\
		\stills & $100.00\pm0.00$ \\
		\chinese & $99.63\pm0.34$ \\
		\aolp & $99.85\pm0.10$ \\
		\openalpreu & $98.52\pm0.51$ \\
		\ssig & $99.80\pm0.24$ \\
		\dataset & $98.67\pm0.25$ \\ \midrule
		\textbf{Average} & $\textbf{99.45}\boldsymbol{\pm}\textbf{0.33}$ \\ \bottomrule
	\end{tabular}} \,}
\end{table}

According to Fig.~\ref{fig:results:lp_detection_tps}, the proposed approach was able to successfully detect and classify \glspl*{lp} of various layouts, including those with few examples in the training set such as \glspl*{lp} issued in the U.S. states of Connecticut and Utah, or \glspl*{lp} of motorcycles \minor{registered in the Taiwan~region}.


\begin{figure}[!htb]
    \centering
    
    \includegraphics[width=0.99\columnwidth]{Figure10.pdf}
    
    \vspace{-2mm}
    
    \caption[\glspl*{lp} correctly detected and classified by the proposed approach]{\glspl*{lp} correctly detected and classified by the proposed approach. Observe the robustness for this task regardless of vehicle type, lighting conditions, camera distance, and other factors.}
    \label{fig:results:lp_detection_tps}
    
\end{figure}

\colored{It should be noted that (i)~the \glspl*{lp} may occupy a very small portion of the original image and that (ii)~textual blocks (e.g., phone numbers) on the vehicles or in the background can be confused with \glspl*{lp}.
Therefore, as can be seen in Fig.~\ref{fig:results:lp_detection_without_vehicle_detection}, the vehicle detection stage is \emph{\textbf{crucial}} for the effectiveness of our \gls*{alpr} system, as it helps to prevent both \glspl*{fp} and~\glspl*{fn}.}

\begin{figure}[!htb]
    \centering
    \captionsetup[subfloat]{captionskip=2pt,font={scriptsize}}
    
    \resizebox{\linewidth}{!}{
    \subfloat[][\colored{Examples of results obtained by detecting the \glspl*{lp} directly in the original image.} \label{fig:results:lp_detection_without_vehicle_detection:a}]{
    \includegraphics[height=11.5ex]{imgs/5-results/lp_detection/samples-lp-detection-without-vehicle-detection/no-vehicle-detection/002168_track0097_26_.jpg}
    \includegraphics[height=11.5ex]{imgs/5-results/lp_detection/samples-lp-detection-without-vehicle-detection/no-vehicle-detection/001835_Track36_23_.jpg}
    \includegraphics[height=11.5ex]{imgs/5-results/lp_detection/samples-lp-detection-without-vehicle-detection/no-vehicle-detection/000111_ac_39.jpg}
    } \,
    }
    
    \vspace{1.5mm}
    
    \resizebox{\linewidth}{!}{
    \subfloat[][\colored{Examples of results obtained by detecting the \glspl*{lp} in the vehicle patches.}\label{fig:results:lp_detection_without_vehicle_detection:b}]{
    \includegraphics[height=11.5ex]{imgs/5-results/lp_detection/samples-lp-detection-without-vehicle-detection/full-pipeline/002168_track0097_26_.jpg}
    \includegraphics[height=11.5ex]{imgs/5-results/lp_detection/samples-lp-detection-without-vehicle-detection/full-pipeline/001835_Track36_23_.jpg}
    \includegraphics[height=11.5ex]{imgs/5-results/lp_detection/samples-lp-detection-without-vehicle-detection/full-pipeline/000111_ac_39.jpg}
    } \,
    } 
    
    \vspace{-1mm}
    
    \caption{\colored{Comparison of the results achieved by detecting/classifying the \glspl*{lp} directly in the original image~(a) and in the vehicle regions predicted in the vehicle detection stage~(b).}}
    \label{fig:results:lp_detection_without_vehicle_detection}
\end{figure}

Some images where our network failed either to detect the \gls*{lp} or to classify the \gls*{lp} layout are shown in Fig.~\ref{fig:results:lp_detection_wrong}. 
As can be seen in Fig.~\ref{fig:results:lp_detection_wrong}a, our network failed to detect the \gls*{lp} in cases where there is a textual block very similar to an \gls*{lp} in the vehicle patch, or even when the \gls*{lp} of another vehicle appears within the patch (a single case in our experiments).
This is due to the fact that one vehicle can be almost totally occluded by another.
Regarding the errors in which the \gls*{lp} layout was misclassified, they occurred mainly in cases where the \gls*{lp} is considerably similar to \gls*{lp} of other layouts. 
For example, the left image in Fig.~\ref{fig:results:lp_detection_wrong}b shows a European \gls*{lp} (which has exactly the same colors and number of characters as standard Chinese \glspl*{lp}) incorrectly classified as Chinese.

\begin{figure}[!htb]
    \centering
    
    \includegraphics[width=0.99\columnwidth]{Figure11.pdf}
    
    \vspace{-2mm}
    
    \caption[Some images in which our network failed either to detect the \gls*{lp} or to classify the \gls*{lp} layout]{Some images in which our network failed either to detect the \gls*{lp} or to classify the \gls*{lp} layout.}
    \label{fig:results:lp_detection_wrong}
\end{figure}

It is important to note that it is still possible to correctly recognize the characters in some cases where our network has failed at this stage. For example, in the right image in Fig.~\ref{fig:results:lp_detection_wrong}a, the detected region contains exactly the same text as the ground truth (i.e., the~\gls*{lp}). Moreover, a Brazilian~\gls*{lp} classified as European (e.g., the middle image in Fig.~\ref{fig:results:lp_detection_wrong}b) can still be correctly recognized in the next stage since the only post-processing rule we apply to European~\glspl*{lp} is that they have between $5$ and $8$~characters.

As mentioned earlier, in this stage we disabled the color-related data augmentation of the Darknet framework. 
In this way, we eliminated more than half of the layout classification errors obtained when the model was trained using images with changed colors. 
We believe this is due to the fact that the network leverages color information (which may be distorted with some data augmentation approaches) for layout classification, as well as other characteristics such as the position of the characters and symbols on the~\gls*{lp}.

\subsection{License Plate Recognition (end-to-end)}
\label{sec:results:lp_recognition}

\begin{table*}[!htb]
\centering
\setlength{\tabcolsep}{9pt}
\caption{Recognition rates (\%) obtained by the proposed system, \colored{modified versions of our system}, previous works, and commercial systems in all datasets used in our experiments. To the best of our knowledge, in the literature, only algorithms for \gls*{lp} detection and character segmentation were evaluated in the \caltech, \stills and \chinese datasets. 
Therefore, our approaches are compared only with the commercial systems in these~datasets.
}
\vspace{1mm}

\label{tab:results:lp_recognition}
\resizebox{\textwidth}{!}{
\begin{tabular}{@{}ccccccccccc@{}}
\toprule
\diagbox[trim=l,trim=r,innerrightsep=-2.5pt,font=\footnotesize]{Dataset}{Approach} & \citep{panahi2017accurate} & \citep{zhuang2018towards} & \citep{silva2018license} & \colored{\citep{silva2020realtime}} & \citep{laroca2018robust} & Sighthound & OpenALPR & \colored{\begin{tabular}[c]{@{}c@{}}No Vehicle\\\phantom{i}Detection\footnotesize{$^\ast$}\end{tabular}} &  \colored{\begin{tabular}[c]{@{}c@{}}No Layout\\\phantom{i}Classification\footnotesize{$^\dagger$}\end{tabular}} & Proposed \\ \midrule
\caltech & $-$ & $-$ & $-$ & \colored{$-$} & $-$ & $95.7\pm2.7$ & $\textbf{99.1}\boldsymbol{\pm}\textbf{1.2}$ & \colored{$98.3\pm1.8$} & $96.1\pm1.8$ & $98.7\pm1.2$ \\
\englishlpd & $\textbf{97.0}$ & $-$ & $-$ & \colored{$-$} & $-$ & $92.5\pm3.7$ & $78.6\pm3.6$ & \colored{$95.3\pm1.6$} & $95.5\pm2.4$ & $95.7\pm2.3$ \\
\stills & $-$ & $-$ & $-$ & \colored{$-$} & $-$ & $\textbf{98.3}$ & $\textbf{98.3}$ & \colored{$98.0\pm0.7$} & $97.3\pm1.9$ & $98.0\pm1.4$ \\
\chinese & $-$ & $-$ & $-$ & \colored{$-$} & $-$ & $90.4\pm2.4$ & $92.6\pm1.9$ & \colored{$97.0\pm0.7$} & $95.4\pm1.1$ & $\textbf{97.5}\boldsymbol{\pm}\textbf{0.9}$ \\
\aolp & $-$ & $\textbf{99.8}$\footnotesize{$^{\ddagger}$} & $-$ & \colored{$-$} & $-$ & $87.1\pm0.8$ & $-$ & \colored{$98.8\pm0.3$} & $98.4\pm0.7$ & $99.2\pm0.4$ \\
\openalpreu & $-$ & $-$ & $93.5$ & \colored{$85.2$} & $-$ & $\colored{93.5}$ & $\colored{91.7}$ & \colored{$\textbf{97.8}\boldsymbol{\pm}\textbf{0.5}$} & $\colored{96.7\pm1.9}$ & $\colored{\textbf{97.8}\boldsymbol{\pm}\textbf{0.5}}$ \\
\ssig & $-$ & $-$ & $88.6$ & \colored{$89.2$} & $85.5$ & $82.8$ & $92.0$ & \colored{$96.5\pm0.9$} & $96.9\pm0.5$ & $\textbf{98.2}\boldsymbol{\pm}\textbf{0.5}$ \\
\dataset & $-$ & $-$ & $-$ & \colored{$-$} & $64.9$ & $62.3$ & $82.2$ & \colored{$59.6\pm0.9$} & $82.5\pm1.1$ & $\textbf{90.0}\boldsymbol{\pm}\textbf{0.7}$ \\ \midrule
Average & $-$ & $-$ & $-$ & \colored{$-$} & $-$ & $\colored{87.8\pm2.4}$ & $\colored{90.7\pm2.3}$ & \colored{$92.7\pm0.9$} & \colored{$94.8\pm1.4$} & $\colored{\textbf{96.9}\boldsymbol{\pm}\textbf{1.0}}$ \\ \bottomrule \\[-9pt]
\multicolumn{11}{l}{\footnotesize \colored{$^\ast$ A modified version of our approach in which the \glspl*{lp} are detected (and their layouts classified) directly in the original image (i.e., without vehicle detection).}} \\
\multicolumn{11}{l}{\footnotesize $^\dagger$ The proposed \gls*{alpr} system assuming that all \gls*{lp} layouts were classified as undefined (i.e., without layout classification and heuristic rules).} \\
\multicolumn{11}{l}{\footnotesize $^{\ddagger}$ The \gls*{lp} patches for the \gls*{lp} recognition stage were cropped directly from the ground truth in~\citep{zhuang2018towards}.}
\end{tabular}
}
\end{table*}

As in the vehicle detection stage, we first evaluated different confidence threshold values in the validation set in order to miss as few characters as possible, while avoiding high \gls*{fp}~rates.
We adopted a~$0.5$ confidence threshold for all \glspl*{lp} except European ones, where a higher threshold (i.e.,~$0.65$) was adopted since European \glspl*{lp} can have up to $8$ characters and several \glspl*{fp} were predicted on \glspl*{lp} with fewer characters when using a lower confidence threshold.

We considered the `$1$' and `I' characters as a single class in the assessments performed in the \ssig and \dataset datasets, as those characters are identical but occupy different positions on Brazilian~\glspl*{lp}.
The same procedure was done in~\citep{laroca2018robust,silva2018license}.

For each dataset, we compared the proposed \gls*{alpr} system with state-of-the-art methods that were evaluated using the same protocol as the one described in Section~\ref{sec:experiments:evaluation_protocol}. In addition, our results are compared with those obtained by Sighthound~\citep{masood2017sighthound} and OpenALPR~\citep{openalprapi}, which are two commercial systems often used as baselines in the \gls*{alpr} literature~\cite{spanhel2017holistic,laroca2018robust,goncalves2018realtime,silva2018license,silva2020realtime}.
According to the authors, both systems are robust for the detection and recognition of \glspl*{lp} of different layouts. 
It is important to emphasize that although the commercial systems were not tuned specifically for the datasets employed in our experiments, they are trained in much larger private datasets, which is a great advantage, especially in deep learning~approaches.

OpenALPR contains specialized solutions for \glspl*{lp} from different regions (e.g.,~\minor{mainland China}, Europe, among others) and the user must enter the correct region before using its~\acrshort*{api}, that is, it requires prior knowledge regarding the \gls*{lp}~layout. Sighthound, on the other hand, uses a single model/approach for \glspl*{lp} from different countries/regions, as well as the proposed system.

\colored{
The remainder of this section is divided into two parts.
First, in Section~\ref{sec:results:overall}, we conduct an overall evaluation of the proposed method across the eight datasets used in our experiments. 
The time required for our system to process an input image is also presented.
Afterward, in Section~\ref{sec:results:detailed}, we briefly present and discuss the results achieved by both the baselines and our \gls*{alpr} system on each dataset individually.
Such an analysis is very important to find out where the proposed system fails and the baselines do not, and vice versa.
}

\colored{
\subsection{Overall Evaluation}
\label{sec:results:overall}
}

The results obtained in all datasets by the proposed \gls*{alpr} system, previous works, and commercial systems are shown in Table~\ref{tab:results:lp_recognition}.
In the average of five runs, across all datasets, our end-to-end system correctly recognized \accavgmath of the \glspl*{lp}, outperforming Sighthound and OpenALPR by \outsighthound and \outopenalpr, respectively.
More specifically, the proposed system outperformed both previous works and commercial systems in the \chinese, \openalpreu, \ssig and \dataset datasets, and yielded competitive results to those attained by the~baselines in the other datasets. 

The proposed system attained results similar to those obtained by OpenALPR in the \caltech dataset ($98.7$\% against $99.1$\%, which represents a difference of less than one \gls*{lp} per run, on average, as there are only $46$ testing images), even though our system does not require prior knowledge. Regarding the \englishlpd dataset, our system performed better than the best baseline~\citep{panahi2017accurate} in~$2$ of the~$5$ runs \colored{(this evaluation highlights the importance of executing the proposed method five times and then averaging the results)}. Although we used the same number of images for testing, in~\citep{panahi2017accurate} the dataset was divided only once and the images used for testing were not specified. In the \stills dataset, both commercial systems reached a recognition rate of $98.3$\% while our system achieved $98$\% on average (with a standard deviation of $1.4$\%). 
Lastly, in the \aolp dataset, the proposed approach obtained similar results to those reported by~\cite{zhuang2018towards}, even though in their work the \gls*{lp} patches used as input in the \gls*{lp} recognition stage were cropped directly from the ground truth~(simplifying the problem, as explained in Section~\ref{sec:related_work}); in other words, they did not take into account vehicles or \glspl*{lp}  not detected in the earlier stages, nor background noise in the \gls*{lp} patches due to less accurate \gls*{lp}~detections.

\colored{To further highlight the importance of the vehicle detection stage, we included in Table~\ref{tab:results:lp_recognition} the results achieved by a modified version of our approach in which the \glspl*{lp} are detected (and their layouts classified) directly in the original image (i.e., without vehicle detection).
Although comparable results were achieved on datasets where the images were acquired on well-controlled scenarios, the modified version failed to detect/classify \glspl*{lp} in various images captured under less controlled conditions (as illustrated in Fig.~\ref{fig:results:lp_detection_without_vehicle_detection:b}), e.g. with vehicles far from the camera and shadows on the \glspl*{lp}, which explains the low recognition rate achieved by that approach in the challenging \dataset dataset -- where the images were taken from inside a vehicle driving through regular traffic in an urban environment, and most \glspl*{lp} occupy a very small region of the image~\cite{laroca2018robust}.}

Similarly, to evaluate the impact of classifying the \gls*{lp}~layout prior to \gls*{lp}~recognition (i.e., our main proposal), we also report in Table~\ref{tab:results:lp_recognition} the results obtained when assuming that all \gls*{lp} layouts were classified as undefined and that a generic approach (i.e., without heuristic rules) was employed in the \gls*{lp}~recognition stage.
The mean recognition rate was improved by~\improvelayoutclassification.
We consider this strategy (layout classification~+ heuristic rules) \emph{\textbf{essential}} for accomplishing outstanding results in datasets that contain \glspl*{lp} with fixed positions for letters and digits (e.g., Brazilian and Chinese \glspl*{lp}), as the recognition rates attained in the \chinese, \ssig and \dataset datasets were improved by~$3.6$\% on average.

The robustness of our \gls*{alpr} system is remarkable since it achieved recognition rates higher than $95$\% in all datasets except \dataset (\textit{where it outperformed the best baseline by $7.8$\%}).
The commercial systems, on the other hand, achieved similar results only in the \caltech and \stills datasets, which contain exclusively American~\glspl*{lp}, and performed poorly (i.e., recognition rates below $85$\%) in at least two datasets. 
This suggests that the commercial systems are not so well trained for \glspl*{lp} of other~layouts \minor{and highlights the importance of carrying out experiments on multiple datasets (with different characteristics) and not just on one or two, as is generally done in most works in the~literature.}

Although OpenALPR achieved better results than Sighthound (on average across all datasets), the latter system can be seen as more robust than the former since it does not require prior knowledge regarding the \gls*{lp} layout. 
In addition, OpenALPR does not support \minor{\glspl*{lp} from the Taiwan region}.
In this sense, we tried to employ OpenALPR solutions designed for \glspl*{lp} from other \minor{regions (including mainland China)} in the experiments performed in the \aolp~dataset; however, very low detection and recognition rates were obtained.



Fig.~\ref{fig:results:lp_recognition_correct} shows some examples of \glspl*{lp} that were correctly recognized by the proposed approach. As can be seen, our system can generalize well and correctly recognize \glspl*{lp} of different layouts, even when the images were captured under challenging conditions. It is noteworthy that, unlike~\cite{panahi2017accurate,laroca2018robust,zhuang2018towards}, the exact same networks were applied to all datasets; in other words, no specific training procedure was used to tune the networks for a given dataset or layout~class.
\colored{Instead, we use heuristic rules in cases where the LP layout is classified with a high confidence~value}.

\begin{figure}[!htb]
	\centering
    
    \includegraphics[width=0.99\columnwidth]{Figure12.pdf}
    
    \vspace{-2mm}

	\caption{Examples of \glspl*{lp} that were correctly recognized by the proposed \gls*{alpr} system. From top to bottom: American, Brazilian, Chinese, European and Taiwanese~\glspl*{lp}.}
    \label{fig:results:lp_recognition_correct}  
\end{figure}

Some \glspl*{lp} in which our system failed to correctly detect/recognize all characters are shown in Fig.~\ref{fig:results:lp_recognition_wrong}. As one may see, the errors occurred mainly in challenging \gls*{lp} images, where even humans can make mistakes since, in some cases, one character might become very similar to another due to the inclination of the \gls*{lp}, the \gls*{lp} frame, shadows, blur, among other factors. 
Note that, in this work, we did not apply preprocessing techniques to the \gls*{lp} image in order not to increase the overall cost of the proposed system.

\begin{figure}[!htb]
	\centering
    
    \includegraphics[width=0.99\columnwidth]{Figure13.pdf}
    
    \vspace{-2mm}

	\caption{Examples of \glspl*{lp} that were incorrectly recognized by the proposed \gls*{alpr} system.
	The ground truth is shown in parentheses.}
    \label{fig:results:lp_recognition_wrong}  
\end{figure}



In Table~\ref{tab:times}, we report the time required for each network in our system to process an input.
As in~\citep{silva2017realtime,laroca2018robust,silva2020realtime}, the reported time is the average time spent processing all inputs in each stage, assuming that the network weights are already loaded and that there is a single vehicle in the scene.
\minor{Although a relatively deep model is explored for vehicle detection, our system is still able to process $73$~\gls*{fps} using a high-end~\acrshort*{gpu}.
In this sense, we believe that it can be employed for several real-world applications, such as parking and toll monitoring systems, even in cheaper setups (e.g., with a mid-end~GPU).}

\begin{table}[!htb]
	\centering
	\caption{The time required for each network in our system to process an input on an NVIDIA Titan Xp \acrshort*{gpu}.}
	\label{tab:times}
    
    \vspace{1mm}

    \resizebox{0.8\columnwidth}{!}{
    \begin{tabular}{@{}cccc@{}}
		\toprule
		\gls*{alpr} Stage & Adapted Model & Time (ms) & \gls*{fps} \\ \midrule
		Vehicle Detection & YOLOv2 &  $8.5382$ & $117$ \\[0.5ex]
		\multicolumn{1}{c}{\begin{tabular}[c]{@{}c@{}}\gls*{lp} Detection and\\[-0.3ex]Layout Classification\end{tabular}} & Fast-YOLOv2 & $3.0854$ & $324$ \\[1.8ex]
		\gls*{lp} Recognition & CR-NET & $1.9935$ & $502$ \\ \midrule
		\textbf{End-to-end} & $\textbf{-}$ & $\textbf{13.6171}$ & $\textbf{73}$ \\ \bottomrule
	\end{tabular}
	}
\end{table}

It should be noted that practically all images from the datasets used in our experiments contain only one labeled vehicle.
However, to perform a more realistic analysis of the execution time, we listed in Table~\ref{tab:times_more_lps} the time required for the proposed system to process images assuming that there is a certain number of vehicles in every image (note that vehicle detection is performed only once, regardless of the number of vehicles in the image).
According to the results, our system can process more than $30$ \gls*{fps} even when there are $4$ vehicles in the~scene.
This information is relevant since some \gls*{alpr} approaches, including the one proposed in our previous work~\citep{laroca2018robust}, can only run in real time if there is at most one vehicle in the~scene.

\begin{table}[!htb]
    \centering
    \caption{Execution times considering that there is a certain number of vehicles in every image.}
    \label{tab:times_more_lps}
    
    \vspace{1mm}
    
    \resizebox{0.45\columnwidth}{!}{ \begin{tabular}{@{}ccc@{}}
    \toprule
    \# Vehicles & Time~(ms) & \gls*{fps} \\ \midrule
    $1$ & $13.6171$ & $73$ \\
    $2$ & $18.6960$ & $53$ \\
    $3$ & $23.7749$ & $42$ \\
    $4$ & $28.8538$ & $35$ \\[2pt] \cdashline{1-3} \\[-6pt]
    $5$ & $33.9327$ & $29$ \\ \bottomrule
    \end{tabular}
    }\end{table}

The proposed approach achieved an outstanding trade-off between accuracy and speed, unlike others recently proposed in the literature. For example, the methods proposed in~\citep{silva2017realtime,goncalves2018realtime} are capable of processing more images per second than our system but reached poor recognition rates (i.e., below $65$\%) in at least one dataset in which they were evaluated. On the other hand, impressive results were achieved on different scenarios in~\citep{li2018toward,li2018reading,silva2018license}.
However, the methods presented in these works are computationally expensive and cannot be applied in real time.
The Sighthound and OpenALPR commercial systems do not report the execution time.

\minor{We remark that real-time processing may be affected by many factors in practice.
For example, we measured our system's execution time when there was no other process consuming machine resources significantly. 
This is the standard procedure in the literature since it enables/facilitates the comparison of different approaches, despite the fact that it may not accurately represent some real-world applications, where other tasks must be performed simultaneously.
Some other factors that may affect real-time processing are the time it takes to transfer the image from the camera to the processing unit, hardware characteristics (e.g., CPU architecture, read/write speeds, and data transfer time between CPU and GPUs), and the versions of the frameworks and libraries used (e.g., OpenCV, Darknet and~CUDA).
}

\colored{
It is important to emphasize that, according to our experiments, the proposed \gls*{alpr} system is robust under different conditions while being efficient essentially due to the meticulous way in which we designed, optimized and combined its different parts, always seeking the best trade-off between accuracy and speed.
All strategies adopted are very important in some way for the robustness and/or efficiency of the proposed approach, and no specific part contributes more than the others in every scenario.
For example, as shown in Table~\ref{tab:results:lp_recognition} and Fig.~\ref{fig:results:lp_detection_without_vehicle_detection}, vehicle detection mainly helps to prevent false positives and false~negatives on complex scenarios, while layout classification (along with heuristic rules) mainly improves the recognition of \glspl*{lp} with a fixed number of characters and/or fixed positions for letters and~digits.
In the same way, both tasks and also \gls*{lp} recognition would not have been accomplished so successfully, or so efficiently, if not for careful modifications to the networks and exploration of data augmentation techniques (all details were given in Section~\ref{sec:proposed}).
}

\colored{
\subsection{Evaluation by Dataset}
\label{sec:results:detailed}
}

\colored{
In this section, we briefly discuss the results achieved by both the baselines and our \gls*{alpr} system on each dataset individually, striving to clearly identify what types of errors are generally made by each system.
For each dataset, we show some qualitative results obtained by the commercial systems and the proposed approach, since we know exactly which images/\glspl*{lp} these systems recognized correctly or not.
In the \openalpreu, \ssig and \dataset datasets, we also show some predictions obtained by the methods introduced in~\cite{silva2018license,laroca2018robust}, as their architectures and pre-trained weights were made publicly available by the respective authors.
Note that, as we are comparing different \gls*{alpr} systems, the \gls*{lp} images shown in this section were cropped directly from the ground~truth.
We focus on the recognition stage for visualization purposes and also because we consider this stage as the current bottleneck of \gls*{alpr} systems.
However, we pointed out cases where one or more systems did not return any predictions on multiple images from a given dataset, which may indicate that the \glspl*{lp} were not properly detected.\\
}

\colored{
\noindent
\textbf{\caltech~\cite{caltech}:} this is the dataset with fewer images for testing (only~$46$).
Hence, a single image recognized incorrectly reduces the accuracy of the system being evaluated by more than $2$\%.
By carefully analyzing the results, we found out that there is a challenging image in this dataset that neither the commercial systems nor the proposed system could correctly recognize.
Note that, in some executions, this image was not in the test subset, which explains the mean recognition rates above $98$\% attained by both our system and OpenALPR.
As illustrated in Fig.~\ref{fig:detailed-caltech}, while OpenALPR only made mistakes in that image, the proposed system failed in another image as well (where an~`F' looks like an~`E' due to the \gls*{lp}'s frame), and Sighthound failed in some other \glspl*{lp} due to very similar characters (e.g.,~`$1$'~and~`I') or false positives.\\
}

\begin{figure}[!htb]
    
    \centering
    \captionsetup[subfigure]{labelformat=empty,font={scriptsize}}


    \resizebox{\linewidth}{!}{
	\subfloat[][\centering \phantom{\,} \resizebox{\adj}{!}{\textbf{\phantom{1}\cite{masood2017sighthound}:}} \texttt{\textcolor{red}{L}3WAZ301}\hspace{\textwidth} \phantom{\,} \resizebox{\adj}{!}{\textbf{\phantom{1}\cite{openalprapi}:}} \texttt{\phantom{L}3WAZ301}\hspace{\textwidth} \phantom{\,} \resizebox{\adj}{!}{\textbf{Ours:}} \texttt{\phantom{L}3WAZ301}]{
		\includegraphics[width=0.23\linewidth]{imgs/5-results/detailed/caltech/000005_image_0016.jpg}} \, \subfloat[][\centering \phantom{\,} \resizebox{\adj}{!}{\textbf{\phantom{1}\cite{masood2017sighthound}:}} \texttt{\textcolor{red}{I}KCM356}\hspace{\textwidth} \phantom{\,} \resizebox{\adj}{!}{\textbf{\phantom{1}\cite{openalprapi}:}} \texttt{1KCM356}\hspace{\textwidth} \phantom{\,} \resizebox{\adj}{!}{\textbf{Ours:}} \texttt{1KCM356}]{
		\includegraphics[width=0.23\linewidth]{imgs/5-results/detailed/caltech/000023_image_0067.jpg}} \, \subfloat[][\centering \phantom{\,} \resizebox{\adj}{!}{\textbf{\phantom{1}\cite{masood2017sighthound}:}} \texttt{2MFF674}\hspace{\textwidth} \phantom{\,} \resizebox{\adj}{!}{\textbf{\phantom{1}\cite{openalprapi}:}} \texttt{2MFF674}\hspace{\textwidth} \phantom{\,} \resizebox{\adj}{!}{\textbf{Ours:}} \texttt{2MF\textcolor{red}{E}674}]{
		\includegraphics[width=0.23\linewidth]{imgs/5-results/detailed/caltech/000012_image_0033.jpg}} \, \subfloat[][\centering \phantom{\,} \resizebox{\adj}{!}{\textbf{\phantom{1}\cite{masood2017sighthound}:}} \texttt{VZW818}\hspace{\textwidth} \phantom{\,} \resizebox{\adj}{!}{\textbf{\phantom{1}\cite{openalprapi}:}} \texttt{VZW818}\hspace{\textwidth} \phantom{\,} \resizebox{\adj}{!}{\textbf{Ours:}} \texttt{VZW818}]{
    		\includegraphics[width=0.23\linewidth]{imgs/5-results/detailed/caltech/000036_image_0089.jpg}} \hspace{1.5mm}
	}
	
\vspace{2.25mm}
	
	\resizebox{\linewidth}{!}{
	\subfloat[][\centering \phantom{\,} \resizebox{\adj}{!}{\textbf{\phantom{1}\cite{masood2017sighthound}:}} \texttt{997JDG}\hspace{\textwidth} \phantom{\,} \resizebox{\adj}{!}{\textbf{\phantom{1}\cite{openalprapi}:}} \texttt{997JDG}\hspace{\textwidth} \phantom{\,} \resizebox{\adj}{!}{\textbf{Ours:}} \texttt{997JDG}]{
		\includegraphics[width=0.23\linewidth]{imgs/5-results/detailed/caltech/000034_image_0092.jpg}} \, \subfloat[][\centering \phantom{\,} \resizebox{\adj}{!}{\textbf{\phantom{1}\cite{masood2017sighthound}:}} \texttt{4CY\textcolor{red}{2}275}\hspace{\textwidth} \phantom{\,} \resizebox{\adj}{!}{\textbf{\phantom{1}\cite{openalprapi}:}} \texttt{4CYE275}\hspace{\textwidth} \phantom{\,} \resizebox{\adj}{!}{\textbf{Ours:}} \texttt{4CYE275}]{
		\includegraphics[width=0.23\linewidth]{imgs/5-results/detailed/caltech/000002_image_0006.jpg}} \, \subfloat[][\centering \phantom{\,} \resizebox{\adj}{!}{\textbf{\phantom{1}\cite{masood2017sighthound}:}} \texttt{VFY818}\hspace{\textwidth} \phantom{\,} \resizebox{\adj}{!}{\textbf{\phantom{1}\cite{openalprapi}:}} \texttt{VFY818}\hspace{\textwidth} \phantom{\,} \resizebox{\adj}{!}{\textbf{Ours:}} \texttt{VFY818}]{
		\includegraphics[width=0.23\linewidth]{imgs/5-results/detailed/caltech/000046_image_0110.jpg}} \, \subfloat[][\centering \phantom{aa} \resizebox{\adj}{!}{\textbf{\phantom{1}\cite{masood2017sighthound}:}} \texttt{\phantom{a}\textcolor{red}{F118}\phantom{a}}\phantom{i} \hspace{\textwidth} \phantom{aa} \resizebox{\adj}{!}{\textbf{\phantom{1}\cite{openalprapi}:}} \texttt{\phantom{aa}\textcolor{red}{n/a}\phantom{a}}\phantom{i} \hspace{\textwidth} \phantom{aa} \resizebox{\adj}{!}{\textbf{Ours:}} \texttt{\phantom{a}\textcolor{red}{IR69}\phantom{a}}\phantom{i}]{
		\includegraphics[width=0.23\linewidth]{imgs/5-results/detailed/caltech/000039_image_0109.jpg}} \hspace{1.5mm}
	}
	
	\vspace{-0.5mm}

    \caption{\colored{Some qualitative results obtained on \caltech~\cite{caltech} by Sighthound~\cite{masood2017sighthound}, OpenALPR~\cite{openalprapi} and the proposed system.}}
    \label{fig:detailed-caltech}
\end{figure}

\colored{
\noindent \textbf{\englishlpd~\cite{englishlpd}:} this dataset has several \gls*{lp} layouts and different types of vehicles such as cars, buses and trucks.
Panahi \& Gholampour~\cite{panahi2017accurate} reported a recognition rate of $97.0$\% in this dataset, however, their method was executed only once and the images used for testing were not specified.
As can be seen in Table~\ref{tab:detailed-english}, using the same number of test images, our method achieved recognition rates above $97$\% in two out of five executions (Sighthound also surpassed $97$\% in one run).
In this sense, we consider that our system is as robust as the one presented in~\cite{panahi2017accurate}.
According to Fig.~\ref{fig:detailed-english}, neither the commercial systems nor the proposed system had difficulty in recognizing \glspl*{lp} with two rows of characters in this dataset.
Instead, as there are many different \gls*{lp} layouts in Europe and thus the number of characters on each \gls*{lp} is not fixed, most errors refer to a character being lost (i.e., false negatives) or, conversely, a non-existent character being predicted (i.e., false positives).
The low recognition rates achieved by OpenALPR are due to the fact that it did not return any predictions in some cases (as if there were no vehicles/\glspl*{lp} in the image).
In this sense, we conjecture that OpenALPR only returns predictions obtained with a high confidence value and that it is not as well trained for European \glspl*{lp} as it is for American/Brazilian~ones.
}

\begin{table}[!htb]
\centering
\setlength{\tabcolsep}{8pt}

\caption{\colored{Recognition rates (\%) achieved by Panahi \& Gholampour~\cite{panahi2017accurate}, Sighthound~\cite{masood2017sighthound}, OpenALPR~\cite{openalprapi}, and our system on \englishlpd~\cite{englishlpd}.}}
\label{tab:detailed-english}
\vspace{1mm}
\colored{
\resizebox{0.65\linewidth}{!}{\begin{tabular}{@{}ccccc@{}}
\toprule
Run & \cite{panahi2017accurate} & \cite{masood2017sighthound} & \cite{openalprapi} & Proposed \\ \midrule
\# $1$   & $-$     & $\textbf{98.0}$       & $82.4$     & $96.1$       \\
\# $2$   & $-$     & $94.1$       & $79.4$     &  $\textbf{97.1}$      \\
\# $3$   & $-$     & $91.2$          & $76.5$         &   $\textbf{98.0}$     \\
\# $4$   & $-$     & $91.2$       & $73.5$     & $\textbf{95.1}$       \\
\# $5$   & $-$          &  $88.2$          &   $81.4$       &   $\textbf{92.2}$     \\ \midrule
Average &  $\textbf{97.0}$        &  $92.5$          &  $78.6$        &  $95.7$       \\ \bottomrule
\end{tabular}
}}
\end{table}

\begin{figure}[!htb]
    \centering
    \captionsetup[subfigure]{labelformat=empty,font={scriptsize}}


    \resizebox{\linewidth}{!}{
	\subfloat[][\centering \phantom{\,} \resizebox{\adj}{!}{\textbf{\phantom{1}\cite{masood2017sighthound}:}} \texttt{ZG200ID}\hspace{\textwidth} \phantom{\,} \resizebox{\adj}{!}{\textbf{\phantom{1}\cite{openalprapi}:}} \texttt{ZG200ID}\hspace{\textwidth} \phantom{\,} \resizebox{\adj}{!}{\textbf{Ours:}} \texttt{ZG200ID}]{
		\includegraphics[width=0.23\linewidth]{imgs/5-results/detailed/englishlpd/000036_070603_P6070093.jpg}} \, \subfloat[][\centering \phantom{\,} \resizebox{\adj}{!}{\textbf{\phantom{1}\cite{masood2017sighthound}:}} \texttt{ZG594TS\phantom{H}}\hspace{\textwidth} \phantom{\,} \resizebox{\adj}{!}{\textbf{\phantom{1}\cite{openalprapi}:}} \texttt{ZG594TS\phantom{H}}\hspace{\textwidth} \phantom{\,} \resizebox{\adj}{!}{\textbf{Ours:}} \texttt{ZG594TS\textcolor{red}{H}}]{
		\includegraphics[width=0.23\linewidth]{imgs/5-results/detailed/englishlpd/000102_280503_P5280124.jpg}} \, \subfloat[][\centering \phantom{\,} \resizebox{\adj}{!}{\textbf{\phantom{1}\cite{masood2017sighthound}:}} \texttt{ZG511S\textcolor{red}{-}}\hspace{\textwidth} \phantom{\,} \resizebox{\adj}{!}{\textbf{\phantom{1}\cite{openalprapi}:}} \texttt{ZG511\textcolor{red}{9-}}\hspace{\textwidth} \phantom{\,} \resizebox{\adj}{!}{\textbf{Ours:}} \texttt{ZG511SF}]{
		\includegraphics[width=0.23\linewidth]{imgs/5-results/detailed/englishlpd/000037_141002_Pa140041.jpg}} \, \subfloat[][\centering \phantom{a} \resizebox{\adj}{!}{\textbf{\phantom{1}\cite{masood2017sighthound}:}} \texttt{AHV\textcolor{red}{8}9002}\phantom{i} \hspace{\textwidth} \phantom{a} \resizebox{\adj}{!}{\textbf{\phantom{1}\cite{openalprapi}:}} \texttt{\phantom{aa}\textcolor{red}{n/a}\phantom{aaa}}\phantom{i} \hspace{\textwidth} \phantom{a} \resizebox{\adj}{!}{\textbf{Ours:}} \texttt{AHV\textcolor{red}{8}9002}\phantom{i}]{
		\includegraphics[width=0.23\linewidth]{imgs/5-results/detailed/englishlpd/000035_070603_P6070085.jpg}} \hspace{1.5mm}
	}
	
\vspace{2.25mm}
	
	\resizebox{\linewidth}{!}{
	\subfloat[][\centering \phantom{\,} \resizebox{\adj}{!}{\textbf{\phantom{1}\cite{masood2017sighthound}:}} \texttt{VU279A\textcolor{red}{-}}\hspace{\textwidth} \phantom{\,} \resizebox{\adj}{!}{\textbf{\phantom{1}\cite{openalprapi}:}} \texttt{VU279AE}\hspace{\textwidth} \phantom{\,} \resizebox{\adj}{!}{\textbf{Ours:}} \texttt{VU279AE}]{
		\includegraphics[width=0.23\linewidth]{imgs/5-results/detailed/englishlpd/000079_280503_P5280008.jpg}} \, \subfloat[][\centering \phantom{\,} \resizebox{\adj}{!}{\textbf{\phantom{1}\cite{masood2017sighthound}:}} \texttt{HGAS1802}\hspace{\textwidth} \phantom{\,} \resizebox{\adj}{!}{\textbf{\phantom{1}\cite{openalprapi}:}} \texttt{HGAS180\textcolor{red}{-}}\hspace{\textwidth} \phantom{\,} \resizebox{\adj}{!}{\textbf{Ours:}} \texttt{HGAS1802}]{
		\includegraphics[width=0.23\linewidth]{imgs/5-results/detailed/englishlpd/000007_040603_P6040062.jpg}} \, \subfloat[][\centering  \phantom{\,} \resizebox{\adj}{!}{\textbf{\phantom{1}\cite{masood2017sighthound}:}}~\texttt{RI393BD}\hspace{\textwidth} \phantom{\,} \resizebox{\adj}{!}{\textbf{\phantom{1}\cite{openalprapi}:}}~\texttt{RI393BD}\hspace{\textwidth} \phantom{\,} \resizebox{\adj}{!}{\textbf{Ours:}}~\texttt{RI393BD}]{
		\includegraphics[width=0.23\linewidth]{imgs/5-results/detailed/englishlpd/000090_280503_P5280077.jpg}} \, \subfloat[][\centering  \phantom{\,} \resizebox{\adj}{!}{\textbf{\phantom{1}\cite{masood2017sighthound}:}}~\texttt{VZ876C\textcolor{red}{T}}\hspace{\textwidth} \phantom{\,} \resizebox{\adj}{!}{\textbf{\phantom{1}\cite{openalprapi}:}}~\texttt{VZ876C\textcolor{red}{T}}\hspace{\textwidth} \phantom{\,} \resizebox{\adj}{!}{\textbf{Ours:}}~\texttt{V\textcolor{red}{2}876C\textcolor{red}{1}}]{
		\includegraphics[width=0.23\linewidth]{imgs/5-results/detailed/englishlpd/000071_180902_P9190060.jpg}} \hspace{1.5mm}
	}
	
	\vspace{-0.5mm}

    \caption{\colored{Some qualitative results obtained on \englishlpd~\cite{englishlpd} by Sighthound~\cite{masood2017sighthound}, OpenALPR~\cite{openalprapi} and the proposed system.}}
    \label{fig:detailed-english}
\end{figure}

\colored{
\noindent \textbf{\stills~\cite{ucsd}:} as \caltech, the \stills dataset also has few test images (only $60$).
Despite containing \glspl*{lp} from distinct U.S. states (i.e., different \gls*{lp} layouts) and under several lighting conditions, all \gls*{alpr} systems evaluated by us achieved excellent results in this dataset.
More specifically, both Sighthound and OpenALPR failed in just one image (interestingly, not in the same one).
This is another indication that these commercial systems are very well trained for American \glspl*{lp}.
Also very robustly, our system failed in just two images \underline{over $5$ runs}, remarkably recognizing all $60$ images correctly in one of them.
All images in which at least one system failed, as well as other representative ones, are shown in Fig.~\ref{fig:detailed-stills}.\\
}

\begin{figure}[!htb]
    \centering
    \captionsetup[subfigure]{labelformat=empty,font={scriptsize}}


    \resizebox{\linewidth}{!}{
	\subfloat[][\centering \phantom{\,} \resizebox{\adj}{!}{\textbf{\phantom{1}\cite{masood2017sighthound}:}}  \texttt{C\textcolor{red}{N}C3951}\hspace{\textwidth} \phantom{\,} \resizebox{\adj}{!}{\textbf{\phantom{1}\cite{openalprapi}:}} \texttt{CKC3951}\hspace{\textwidth}  \phantom{\,} \resizebox{\adj}{!}{\textbf{Ours:}} \texttt{CKC3951}]{
		\includegraphics[width=0.23\linewidth]{imgs/5-results/detailed/stills/000006_cars062.jpg}} \, \subfloat[][\centering \phantom{\,} \resizebox{\adj}{!}{\textbf{\phantom{1}\cite{masood2017sighthound}:}} \texttt{RNM25X}\hspace{\textwidth} \phantom{\,} \resizebox{\adj}{!}{\textbf{\phantom{1}\cite{openalprapi}:}} \texttt{RNM25X}\hspace{\textwidth} \phantom{\,} \resizebox{\adj}{!}{\textbf{Ours:}} \texttt{RNM25X}]{
		\includegraphics[width=0.23\linewidth]{imgs/5-results/detailed/stills/000058_cars4_102.jpg}} \, \subfloat[][\centering \phantom{\,} \resizebox{\adj}{!}{\textbf{\phantom{1}\cite{masood2017sighthound}:}} \texttt{1TM115\phantom{a}}\hspace{\textwidth} \phantom{\,} \resizebox{\adj}{!}{\textbf{\phantom{1}\cite{openalprapi}:}} \texttt{1TM115\phantom{a}}\hspace{\textwidth} \phantom{\,} \resizebox{\adj}{!}{\textbf{Ours:}} \texttt{1TM115\textcolor{red}{I}}]{
		\includegraphics[width=0.23\linewidth]{imgs/5-results/detailed/stills/000020_cars2_058.jpg}} \, \subfloat[][\centering \phantom{\,} \resizebox{\adj}{!}{\textbf{\phantom{1}\cite{masood2017sighthound}:}} \texttt{5CGP522}\hspace{\textwidth} \phantom{\,} \resizebox{\adj}{!}{\textbf{\phantom{1}\cite{openalprapi}:}} \texttt{5CGP522}\hspace{\textwidth} \phantom{\,} \resizebox{\adj}{!}{\textbf{Ours:}} \texttt{5CGP522}]{
		\includegraphics[width=0.23\linewidth]{imgs/5-results/detailed/stills/000033_cars4_006.jpg}} \hspace{1.5mm}
	}
	
\vspace{2.25mm}
	
	\resizebox{\linewidth}{!}{
	\subfloat[][\centering \phantom{\,} \resizebox{\adj}{!}{\textbf{\phantom{1}\cite{masood2017sighthound}:}}  \texttt{3J66282}\hspace{\textwidth} \phantom{\,} \resizebox{\adj}{!}{\textbf{\phantom{1}\cite{openalprapi}:}} \texttt{3J66282}\hspace{\textwidth}  \phantom{\,} \resizebox{\adj}{!}{\textbf{Ours:}} \texttt{3J66282}]{
		\includegraphics[width=0.23\linewidth]{imgs/5-results/detailed/stills/000005_cars053.jpg}} \, \subfloat[][\centering \phantom{\,} \resizebox{\adj}{!}{\textbf{\phantom{1}\cite{masood2017sighthound}:}} \texttt{4NFU116}\hspace{\textwidth} \phantom{\,} \resizebox{\adj}{!}{\textbf{\phantom{1}\cite{openalprapi}:}} \texttt{4NF\textcolor{red}{-}116}\hspace{\textwidth} \phantom{\,} \resizebox{\adj}{!}{\textbf{Ours:}} \texttt{4NFU116}]{
		\includegraphics[width=0.23\linewidth]{imgs/5-results/detailed/stills/000012_cars2_022.jpg}} \, \subfloat[][\centering \phantom{\,} \resizebox{\adj}{!}{\textbf{\phantom{1}\cite{masood2017sighthound}:}} \texttt{AHA6497}\hspace{\textwidth} \phantom{\,} \resizebox{\adj}{!}{\textbf{\phantom{1}\cite{openalprapi}:}} \texttt{AHA6497}\hspace{\textwidth} \phantom{\,} \resizebox{\adj}{!}{\textbf{Ours:}} \texttt{AHA6497}]{
		\includegraphics[width=0.23\linewidth]{imgs/5-results/detailed/stills/000052_cars4_079.jpg}} \, \subfloat[][\centering \phantom{\,} \resizebox{\adj}{!}{\textbf{\phantom{1}\cite{masood2017sighthound}:}} \texttt{4NIU770}\hspace{\textwidth} \phantom{\,} \resizebox{\adj}{!}{\textbf{\phantom{1}\cite{openalprapi}:}} \texttt{4NIU770}\hspace{\textwidth} \phantom{\,} \resizebox{\adj}{!}{\textbf{Ours:}} \texttt{4N\textcolor{red}{T}U770}]{
		\includegraphics[width=0.23\linewidth]{imgs/5-results/detailed/stills/000016_cars2_041.jpg}} \hspace{1.5mm}
	}
	
	\vspace{-0.5mm}

    \caption{\colored{Some qualitative results obtained on \stills~\cite{ucsd} by Sighthound~\cite{masood2017sighthound}, OpenALPR~\cite{openalprapi} and the proposed system.}}
    \label{fig:detailed-stills}
\end{figure}

\colored{
\noindent \textbf{\chinese~\cite{zhou2012principal}:} this dataset contains both images captured by the authors and downloaded from the Internet. We used $159$ images for testing in each run.
An important feature of \chinese is that it has several images in which the \glspl*{lp} are tilted or inclined, as shown in Fig.~\ref{fig:detailed-chinese}.
In fact, most of the prediction errors obtained by commercial systems were in such images.
Our system, on the other hand, handled tilted/inclined \glspl*{lp} well and mostly failed in cases where one character become
very similar to another due to the \gls*{lp}
frame, shadows, blur, etc. 
It should be noted that Sighthound~($90.4$\%) misclassified the Chinese character (see Section~\ref{sec:proposed:lp_recognition} for details) as an English letter on some occasions.
This kind of recognition error was rarely made by the proposed system~($97.5$\%) and OpenALPR~($92.6$\%).
}

\begin{figure}[!htb]
    \centering
    \captionsetup[subfigure]{labelformat=empty,font={scriptsize}}


    \resizebox{\linewidth}{!}{
	\subfloat[][\centering \phantom{\,} \resizebox{\adj}{!}{\textbf{\phantom{1}\cite{masood2017sighthound}:}} \texttt{ALA8\textcolor{red}{-{}-}}\hspace{\textwidth} \phantom{\,} \resizebox{\adj}{!}{\textbf{\phantom{1}\cite{openalprapi}:}} \texttt{A\textcolor{red}{I}A82I}\hspace{\textwidth} \phantom{\,} \resizebox{\adj}{!}{\textbf{Ours:}} \texttt{ALA82I}]{
		\includegraphics[width=0.23\linewidth]{imgs/5-results/detailed/chinese/000095_camera_SDC13400.JPG}} \, \subfloat[][\centering \phantom{\,} \resizebox{\adj}{!}{\textbf{\phantom{1}\cite{masood2017sighthound}:}} \texttt{ADT444}\hspace{\textwidth} \phantom{\,} \resizebox{\adj}{!}{\textbf{\phantom{1}\cite{openalprapi}:}} \texttt{AD\textcolor{red}{I}444}\hspace{\textwidth} \phantom{\,} \resizebox{\adj}{!}{\textbf{Ours:}} \texttt{AD\textcolor{red}{I}444}]{
		\includegraphics[width=0.23\linewidth]{imgs/5-results/detailed/chinese/000101_internet_1.jpg}} \, \subfloat[][\centering \phantom{\,} \resizebox{\adj}{!}{\textbf{\phantom{1}\cite{masood2017sighthound}:}} \texttt{\textcolor{red}{B}ABII57}\hspace{\textwidth} \phantom{\,} \resizebox{\adj}{!}{\textbf{\phantom{1}\cite{openalprapi}:}} \texttt{\phantom{a}ABII57}\hspace{\textwidth} \phantom{\,} \resizebox{\adj}{!}{\textbf{Ours:}} \texttt{\phantom{a}ABII57}]{
		\includegraphics[width=0.23\linewidth]{imgs/5-results/detailed/chinese/000052_camera_SDC13270.JPG}} \, \subfloat[][\centering \phantom{\,} \resizebox{\adj}{!}{\textbf{\phantom{1}\cite{masood2017sighthound}:}} \texttt{AK0473}\hspace{\textwidth} \phantom{\,} \resizebox{\adj}{!}{\textbf{\phantom{1}\cite{openalprapi}:}} \texttt{AK0473}\hspace{\textwidth} \phantom{\,} \resizebox{\adj}{!}{\textbf{Ours:}} \texttt{AK04\textcolor{red}{I}3}]{
		\includegraphics[width=0.23\linewidth]{imgs/5-results/detailed/chinese/000079_camera_SDC13323.JPG}} \hspace{1.5mm}
	}
	
\vspace{2.25mm}
	
	\resizebox{\linewidth}{!}{
	\subfloat[][\centering \phantom{\,} \resizebox{\adj}{!}{\textbf{\phantom{1}\cite{masood2017sighthound}:}} \texttt{C44444}\hspace{\textwidth} \phantom{\,} \resizebox{\adj}{!}{\textbf{\phantom{1}\cite{openalprapi}:}} \texttt{\textcolor{red}{G}44444}\hspace{\textwidth} \phantom{\,} \resizebox{\adj}{!}{\textbf{Ours:}} \texttt{C44444}]{
		\includegraphics[width=0.23\linewidth]{imgs/5-results/detailed/chinese/000105_internet_13.jpg}} \, \subfloat[][\centering \phantom{\,} \resizebox{\adj}{!}{\textbf{\phantom{1}\cite{masood2017sighthound}:}} \texttt{AEI4\textcolor{red}{L}I}\hspace{\textwidth} \phantom{\,} \resizebox{\adj}{!}{\textbf{\phantom{1}\cite{openalprapi}:}} \texttt{AEI4II}\hspace{\textwidth} \phantom{\,} \resizebox{\adj}{!}{\textbf{Ours:}} \texttt{AEI4II}]{
		\includegraphics[width=0.23\linewidth]{imgs/5-results/detailed/chinese/000031_camera_IMG_2687.JPG}} \, \subfloat[][\centering \phantom{\,} \resizebox{\adj}{!}{\textbf{\phantom{1}\cite{masood2017sighthound}:}} \texttt{\textcolor{red}{I}A6ITII}\hspace{\textwidth} \phantom{\,} \resizebox{\adj}{!}{\textbf{\phantom{1}\cite{openalprapi}:}} \texttt{\phantom{a}A6ITI\textcolor{red}{7}}\hspace{\textwidth} \phantom{\,} \resizebox{\adj}{!}{\textbf{Ours:}} \texttt{\phantom{a}A6ITII}]{
		\includegraphics[width=0.23\linewidth]{imgs/5-results/detailed/chinese/000156_internet_97.jpg}} \, \subfloat[][\centering  \phantom{\,} \resizebox{\adj}{!}{\textbf{\phantom{1}\cite{masood2017sighthound}:}}~\texttt{\textcolor{red}{R}L0020I}\hspace{\textwidth} \phantom{\,} \resizebox{\adj}{!}{\textbf{\phantom{1}\cite{openalprapi}:}}~\texttt{\phantom{a}L0020\textcolor{red}{N}}\hspace{\textwidth} \phantom{\,} \resizebox{\adj}{!}{\textbf{Ours:}}~\texttt{\textcolor{red}{R}L0020\textcolor{red}{-}}]{
		\includegraphics[width=0.23\linewidth]{imgs/5-results/detailed/chinese/000139_internet_65.jpg}} \hspace{1.5mm}
	}
	
	\vspace{-0.5mm}

    \caption{\colored{Some qualitative results obtained on \chinese~\cite{zhou2012principal} by Sighthound~\cite{masood2017sighthound}, OpenALPR~\cite{openalprapi} and the proposed system.}}
    \label{fig:detailed-chinese}
\end{figure}

\colored{
\noindent \textbf{\aolp~\cite{hsu2013application}:} this dataset has images collected \minor{in the Taiwan region} from front/rear views of vehicles and various locations, time, traffic, and weather conditions.
In our experiments, $683$ images were used for testing in each run.
As OpenALPR does not support \minor{\glspl*{lp} from the Taiwan region} (as pointed out in Section~\ref{sec:results:overall}), here we compare the results obtained by Sighthound~($87.1$\%) and the proposed system~($99.2$\%).
As shown in Fig.~\ref{fig:detailed-aolp}, different from what we expected, both systems dealt well with inclined \glspl*{lp} in this dataset.
While our system failed mostly in challenging cases, such as very similar characters (`E'~and~`F', `B'~and~`$8$', etc.), Sighthound also failed in simpler cases where our system had no difficulty in correctly recognizing all \gls*{lp}~characters.\\
}

\begin{figure}[!htb]
    \centering
    \captionsetup[subfigure]{labelformat=empty,font={scriptsize}}


    \resizebox{\linewidth}{!}{
	\subfloat[][\centering \phantom{\,} \resizebox{\adj}{!}{\textbf{\phantom{1}\cite{masood2017sighthound}:}} \texttt{C\textcolor{red}{8}8117}\hspace{\textwidth} \phantom{\,} \resizebox{\adj}{!}{\textbf{Ours:}} \texttt{C38117}]{
		\includegraphics[width=0.23\linewidth]{imgs/5-results/detailed/aolp/000462_le_721.jpg}} \, \subfloat[][\centering \phantom{\,} \resizebox{\adj}{!}{\textbf{\phantom{1}\cite{masood2017sighthound}:}} \texttt{RE9302}\hspace{\textwidth} \phantom{\,}  \resizebox{\adj}{!}{\textbf{Ours:}} \texttt{R\textcolor{red}{F}9302}]{
		\includegraphics[width=0.23\linewidth]{imgs/5-results/detailed/aolp/000212_ac_675.jpg}} \, \subfloat[][\centering \phantom{\,} \resizebox{\adj}{!}{\textbf{\phantom{1}\cite{masood2017sighthound}:}} \texttt{8695LS}\hspace{\textwidth} \phantom{\,}  \resizebox{\adj}{!}{\textbf{Ours:}} \texttt{8695LS}]{
		\includegraphics[width=0.23\linewidth]{imgs/5-results/detailed/aolp/000558_rp_320.jpg}} \, \subfloat[][\centering \phantom{\,} \resizebox{\adj}{!}{\textbf{\phantom{1}\cite{masood2017sighthound}:}} \texttt{Y\textcolor{red}{B}8096}\hspace{\textwidth} \phantom{\,}  \resizebox{\adj}{!}{\textbf{Ours:}} \texttt{Y\textcolor{red}{B}8096}]{
		\includegraphics[width=0.23\linewidth]{imgs/5-results/detailed/aolp/000136_ac_495.jpg}} \hspace{1.5mm}
	}
	
\vspace{2.25mm}
	
	\resizebox{\linewidth}{!}{
	\subfloat[][\centering \phantom{\,} \resizebox{\adj}{!}{\textbf{\phantom{1}\cite{masood2017sighthound}:}} \texttt{\textcolor{red}{I}51735}\hspace{\textwidth} \phantom{\,}  \resizebox{\adj}{!}{\textbf{Ours:}} \texttt{\textcolor{red}{I}51735}]{
		\includegraphics[width=0.23\linewidth]{imgs/5-results/detailed/aolp/000013_ac_15.jpg}} \, \subfloat[][\centering  \phantom{\,} \resizebox{\adj}{!}{\textbf{\phantom{1}\cite{masood2017sighthound}:}}~\texttt{9J3167}\hspace{\textwidth} \phantom{\,} \resizebox{\adj}{!}{\textbf{Ours:}}~\texttt{9J3167}]{
	\includegraphics[width=0.23\linewidth]{imgs/5-results/detailed/aolp/000493_rp_118.jpg}} \, \subfloat[][\centering \phantom{\,} \resizebox{\adj}{!}{\textbf{\phantom{1}\cite{masood2017sighthound}:}} \texttt{\textcolor{red}{1}2N4202}\hspace{\textwidth} \phantom{\,}  \resizebox{\adj}{!}{\textbf{Ours:}} \texttt{\phantom{a}2N4202}]{
		\includegraphics[width=0.23\linewidth]{imgs/5-results/detailed/aolp/000112_ac_424.jpg}} \, \subfloat[][\centering  \phantom{\,} \resizebox{\adj}{!}{\textbf{\phantom{1}\cite{masood2017sighthound}:}}~\texttt{\textcolor{red}{D}750J0}\hspace{\textwidth} \phantom{\,}  \resizebox{\adj}{!}{\textbf{Ours:}}~\texttt{0750J0}]{
		\includegraphics[width=0.23\linewidth]{imgs/5-results/detailed/aolp/000590_rp_390.jpg}} \hspace{1.5mm}
	}
	
	\vspace{-0.5mm}

    \caption{\colored{Some qualitative results obtained on the \aolp~\cite{hsu2013application} dataset by Sighthound~\cite{masood2017sighthound} and the proposed system.}}
    \label{fig:detailed-aolp}
\end{figure}

\colored{
\noindent \textbf{\openalpreu~\cite{openalpreu}:} this dataset consists of $108$ testing images, generally with the vehicle well centered and occupying a large portion of the image.
Therefore, both our \gls*{alpr} system and the baselines performed well on this dataset.
Over five executions, the proposed system~($97.8$\%) failed in just $3$ different images, while the baselines failed in a few more.
Surprisingly, as can be seen in Fig.~\ref{fig:detailed-openalpreu}, the systems made distinct recognition errors and we were unable to find an explicit pattern among the incorrect predictions made by each of them.
In this sense, we believe that the errors in this dataset are mainly due to the great variability in the fonts of the characters in different \gls*{lp} layouts.
As an example, note in Fig.~\ref{fig:detailed-openalpreu} that the `W' character varies considerably depending on the \gls*{lp}~layout.
}

\begin{figure}[!htb]
    \centering
    \captionsetup[subfigure]{labelformat=empty,font={scriptsize}}


    \resizebox{\linewidth}{!}{
	\subfloat[][\centering  \phantom{\,} \resizebox{\adj}{!}{\textbf{\phantom{1}\cite{masood2017sighthound}:}}~\texttt{BSE5579}\hspace{\textwidth} \phantom{\,} \resizebox{\adj}{!}{\textbf{\phantom{1}\cite{openalprapi}:}}~\texttt{BSE5579}\hspace{\textwidth} \phantom{\,}
    \resizebox{\adj}{!}{\textbf{\phantom{10}\cite{silva2018license}:}}~\texttt{BSE5579}\hspace{\textwidth} \phantom{\,} \resizebox{\adj}{!}{\textbf{Ours:}}~\texttt{BSE5579}]{
		\includegraphics[width=0.23\linewidth]{imgs/5-results/detailed/openalpr-eu/000077_test_066.jpg}} \, \subfloat[][\centering  \phantom{\,} \resizebox{\adj}{!}{\textbf{\phantom{1}\cite{masood2017sighthound}:}}~\texttt{BA2\textcolor{red}{2}8IM}\hspace{\textwidth} \phantom{\,} \resizebox{\adj}{!}{\textbf{\phantom{1}\cite{openalprapi}:}}~\texttt{BA268IM}\hspace{\textwidth} \phantom{\,}
    \resizebox{\adj}{!}{\textbf{\phantom{10}\cite{silva2018license}:}}~\texttt{\textcolor{red}{3}A268IM}\hspace{\textwidth} \phantom{\,} \resizebox{\adj}{!}{\textbf{Ours:}}~\texttt{BA268IM}]{
		\includegraphics[width=0.23\linewidth]{imgs/5-results/detailed/openalpr-eu/000053_test_042.jpg}} \, \subfloat[][\centering  \phantom{\,} \resizebox{\adj}{!}{\textbf{\phantom{1}\cite{masood2017sighthound}:}}~\texttt{VW4X4WP}\hspace{\textwidth} \phantom{\,} \resizebox{\adj}{!}{\textbf{\phantom{1}\cite{openalprapi}:}}~\texttt{VW4\textcolor{red}{7}4WP}\hspace{\textwidth} \phantom{\,}
    \resizebox{\adj}{!}{\textbf{\phantom{10}\cite{silva2018license}:}}~\texttt{VW4X4WP}\hspace{\textwidth} \phantom{\,} \resizebox{\adj}{!}{\textbf{Ours:}}~\texttt{VW4X4WP}]{
		\includegraphics[width=0.23\linewidth]{imgs/5-results/detailed/openalpr-eu/000005_eu07.jpg}} \, \subfloat[][\centering  \phantom{\,} \resizebox{\adj}{!}{\textbf{\phantom{1}\cite{masood2017sighthound}:}}~\texttt{GWAGEN}\hspace{\textwidth} \phantom{\,} \resizebox{\adj}{!}{\textbf{\phantom{1}\cite{openalprapi}:}}~\texttt{\textcolor{red}{-}WAGEN}\hspace{\textwidth} \phantom{\,}
    \resizebox{\adj}{!}{\textbf{\phantom{10}\cite{silva2018license}:}}~\texttt{G\textcolor{red}{VN}AGEN}\hspace{\textwidth} \phantom{\,} \resizebox{\adj}{!}{\textbf{Ours:}}~\texttt{G\textcolor{red}{VN}AGEN}]{
		\includegraphics[width=0.23\linewidth]{imgs/5-results/detailed/openalpr-eu/000002_eu02.jpg}} \hspace{1.5mm}
	}
	
\vspace{2.25mm}
	
	\resizebox{\linewidth}{!}{
\subfloat[][\centering  \phantom{\,} \resizebox{\adj}{!}{\textbf{\phantom{1}\cite{masood2017sighthound}:}}~\texttt{WSQ3021}\hspace{\textwidth} \phantom{\,} \resizebox{\adj}{!}{\textbf{\phantom{1}\cite{openalprapi}:}}~\texttt{WS\textcolor{red}{0}3021}\hspace{\textwidth} \phantom{\,}
    \resizebox{\adj}{!}{\textbf{\phantom{10}\cite{silva2018license}:}}~\texttt{WSQ302\textcolor{red}{-}}\hspace{\textwidth} \phantom{\,} \resizebox{\adj}{!}{\textbf{Ours:}}~\texttt{WSQ3021}]{
		\includegraphics[width=0.23\linewidth]{imgs/5-results/detailed/openalpr-eu/000006_eu08.jpg}} \, \subfloat[][\centering  \phantom{\,} \resizebox{\adj}{!}{\textbf{\phantom{1}\cite{masood2017sighthound}:}}~\texttt{RK60\textcolor{red}{0}AB}\hspace{\textwidth} \phantom{\,} \resizebox{\adj}{!}{\textbf{\phantom{1}\cite{openalprapi}:}}~\texttt{RK605AB}\hspace{\textwidth} \phantom{\,}
    \resizebox{\adj}{!}{\textbf{\phantom{10}\cite{silva2018license}:}}~\texttt{RK605AB}\hspace{\textwidth} \phantom{\,} \resizebox{\adj}{!}{\textbf{Ours:}}~\texttt{RK605AB}]{
		\includegraphics[width=0.23\linewidth]{imgs/5-results/detailed/openalpr-eu/000062_test_051.jpg}} \, \subfloat[][\centering  \phantom{\,} \resizebox{\adj}{!}{\textbf{\phantom{1}\cite{masood2017sighthound}:}}~\texttt{1Z7\phantom{a}5233}\hspace{\textwidth} \phantom{\,} \resizebox{\adj}{!}{\textbf{\phantom{1}\cite{openalprapi}:}}~\texttt{1Z7\phantom{a}5233}\hspace{\textwidth} \phantom{\,}
    \resizebox{\adj}{!}{\textbf{\phantom{10}\cite{silva2018license}:}}~\texttt{1Z7\phantom{a}5233}\hspace{\textwidth} \phantom{\,} \resizebox{\adj}{!}{\textbf{Ours:}}~\texttt{1Z7\textcolor{red}{8}5233}]{
		\includegraphics[width=0.23\linewidth]{imgs/5-results/detailed/openalpr-eu/000082_test_071.jpg}} \, \subfloat[][\centering  \phantom{\,} \resizebox{\adj}{!}{\textbf{\phantom{1}\cite{masood2017sighthound}:}}~\texttt{RK161AG}\hspace{\textwidth} \phantom{\,} \resizebox{\adj}{!}{\textbf{\phantom{1}\cite{openalprapi}:}}~\texttt{\textcolor{red}{B}K161AG}\hspace{\textwidth} \phantom{\,}
    \resizebox{\adj}{!}{\textbf{\phantom{10}\cite{silva2018license}:}}~\texttt{RK161AG}\hspace{\textwidth} \phantom{\,} \resizebox{\adj}{!}{\textbf{Ours:}}~\texttt{RK161AG}]{
		\includegraphics[width=0.23\linewidth]{imgs/5-results/detailed/openalpr-eu/000059_test_048.jpg}} \hspace{1.5mm}
	}
	
	\vspace{-0.5mm}

    \caption{\colored{Some qualitative results obtained on \openalpreu~\cite{openalpreu} by Sighthound~\cite{masood2017sighthound}, OpenALPR~\cite{openalprapi}, Silva \& Jung~\cite{silva2018license}, and the proposed system.}}
    \label{fig:detailed-openalpreu}
\end{figure}

\colored{
\noindent \textbf{\ssig~\cite{goncalves2016benchmark}:} this dataset contains $800$ images for testing.
All images were taken with a static camera on the campus of a Brazilian university.
Here, the proposed system achieved a high recognition rate of~$98.2$\%, outperforming the best baseline by~$6.2$\%.
As shown in Fig.~\ref{fig:detailed-ssig}, as well as in other datasets, our system failed mostly in challenging cases where one character becomes very similar to another due to motion blur, the position of the camera, and other factors.
This was also the reason for most of the errors made by OpenALPR and the system designed by Silva \& Jung~\cite{silva2018license}.
However, these systems also struggled to correctly recognize degraded~\glspl*{lp} in which some characters are distorted or erased.
In addition to such errors, Sighthound predicted $6$ characters instead of $7$ on several occasions, probably because it does not take advantage of information regarding the \gls*{lp}~layout.
Lastly, the preliminary version of our approach~\cite{laroca2018robust}, where the \gls*{lp} characters are first segmented and then individually recognized, had difficulty segmenting the characters `I'~and~`$1$' in some cases, which resulted in recognition~errors.
}

\begin{figure}[!htb]
    \centering
    \captionsetup[subfigure]{labelformat=empty,font={scriptsize}}


    \resizebox{\linewidth}{!}{
	\subfloat[][\centering  \phantom{\,} \resizebox{\adj}{!}{\textbf{\phantom{1}\cite{masood2017sighthound}:}}~\texttt{HIM2848}\hspace{\textwidth} \phantom{\,} \resizebox{\adj}{!}{\textbf{\phantom{1}\cite{openalprapi}:}}~\texttt{HIM2848}\hspace{\textwidth} \phantom{\,}
    \resizebox{\adj}{!}{\textbf{\phantom{10}\cite{silva2018license}:}}~\texttt{HIM2848}\hspace{\textwidth} \phantom{\,}
    \resizebox{\adj}{!}{\textbf{\phantom{1}\cite{laroca2018robust}:}}~\texttt{HIM2848}\hspace{\textwidth} \phantom{\,}
    \resizebox{\adj}{!}{\textbf{Ours:}}~\texttt{HIM2848}]{
		\includegraphics[width=0.23\linewidth]{imgs/5-results/detailed/ssig/000702_Track35_06_.png}} \, \subfloat[][\centering  \phantom{\,} \resizebox{\adj}{!}{\textbf{\phantom{1}\cite{masood2017sighthound}:}}~\texttt{OQ\textcolor{red}{D}541\textcolor{red}{D}}\hspace{\textwidth} \phantom{\,} \resizebox{\adj}{!}{\textbf{\phantom{1}\cite{openalprapi}:}}~\texttt{\textcolor{red}{D}QO5410}\hspace{\textwidth} \phantom{\,}
    \resizebox{\adj}{!}{\textbf{\phantom{10}\cite{silva2018license}:}}~\texttt{O\textcolor{red}{O}O5410}\hspace{\textwidth} \phantom{\,}
    \resizebox{\adj}{!}{\textbf{\phantom{1}\cite{laroca2018robust}:}}~\texttt{O\textcolor{red}{O}O5410}\hspace{\textwidth} \phantom{\,}
    \resizebox{\adj}{!}{\textbf{Ours:}}~\texttt{O\textcolor{red}{O}O5410}]{
		\includegraphics[width=0.23\linewidth]{imgs/5-results/detailed/ssig/000742_Track37_05_.png}} \, \subfloat[][\centering  \phantom{\,} \resizebox{\adj}{!}{\textbf{\phantom{1}\cite{masood2017sighthound}:}}~\texttt{H\textcolor{red}{O}R8361}\hspace{\textwidth} \phantom{\,} \resizebox{\adj}{!}{\textbf{\phantom{1}\cite{openalprapi}:}}~\texttt{HDR8361}\hspace{\textwidth} \phantom{\,}
    \resizebox{\adj}{!}{\textbf{\phantom{10}\cite{silva2018license}:}}~\texttt{HDR8361}\hspace{\textwidth} \phantom{\,}
    \resizebox{\adj}{!}{\textbf{\phantom{1}\cite{laroca2018robust}:}}~\texttt{HDR8361}\hspace{\textwidth} \phantom{\,}
    \resizebox{\adj}{!}{\textbf{Ours:}}~\texttt{H\textcolor{red}{O}R8361}]{
		\includegraphics[width=0.23\linewidth]{imgs/5-results/detailed/ssig/000508_Track25_02_.png}} \, \subfloat[][\centering  \phantom{\,} \resizebox{\adj}{!}{\textbf{\phantom{1}\cite{masood2017sighthound}:}}~\texttt{HJN208\textcolor{red}{-}}\hspace{\textwidth} \phantom{\,} \resizebox{\adj}{!}{\textbf{\phantom{1}\cite{openalprapi}:}}~\texttt{\textcolor{red}{R}JN2081}\hspace{\textwidth} \phantom{\,}
    \resizebox{\adj}{!}{\textbf{\phantom{10}\cite{silva2018license}:}}~\texttt{H\textcolor{red}{L}N2081}\hspace{\textwidth} \phantom{\,}
    \resizebox{\adj}{!}{\textbf{\phantom{1}\cite{laroca2018robust}:}}~\texttt{HJN208\textcolor{red}{-}}\hspace{\textwidth} \phantom{\,}
    \resizebox{\adj}{!}{\textbf{Ours:}}~\texttt{HJN2081}]{
		\includegraphics[width=0.23\linewidth]{imgs/5-results/detailed/ssig/000754_Track38_10_.png}} \hspace{1.5mm}
	}
	
    \vspace{2.25mm}
	
	\resizebox{\linewidth}{!}{
	\subfloat[][\centering  \phantom{\,} \resizebox{\adj}{!}{\textbf{\phantom{1}\cite{masood2017sighthound}:}}~\texttt{\textcolor{red}{D}GQ6370}\hspace{\textwidth} \phantom{\,} \resizebox{\adj}{!}{\textbf{\phantom{1}\cite{openalprapi}:}}~\texttt{OG\textcolor{red}{O}6370}\hspace{\textwidth} \phantom{\,}
    \resizebox{\adj}{!}{\textbf{\phantom{10}\cite{silva2018license}:}}~\texttt{OG\textcolor{red}{O}6370}\hspace{\textwidth} \phantom{\,}
    \resizebox{\adj}{!}{\textbf{\phantom{1}\cite{laroca2018robust}:}}~\texttt{OG\textcolor{red}{O}6370}\hspace{\textwidth} \phantom{\,}
    \resizebox{\adj}{!}{\textbf{Ours:}}~\texttt{OG\textcolor{red}{O}6370}]{
		\includegraphics[width=0.23\linewidth]{imgs/5-results/detailed/ssig/000586_Track29_08_.png}} \, \subfloat[][\centering  \phantom{\,} \resizebox{\adj}{!}{\textbf{\phantom{1}\cite{masood2017sighthound}:}}~\texttt{\textcolor{red}{D}XH86\textcolor{red}{J}7}\hspace{\textwidth} \phantom{\,} \resizebox{\adj}{!}{\textbf{\phantom{1}\cite{openalprapi}:}}~\texttt{OXH8617}\hspace{\textwidth} \phantom{\,}
    \resizebox{\adj}{!}{\textbf{\phantom{10}\cite{silva2018license}:}}~\texttt{OXH8617}\hspace{\textwidth} \phantom{\,}
    \resizebox{\adj}{!}{\textbf{\phantom{1}\cite{laroca2018robust}:}}~\texttt{OXH86\textcolor{red}{-}7}\hspace{\textwidth} \phantom{\,}
    \resizebox{\adj}{!}{\textbf{Ours:}}~\texttt{OXH8617}]{
		\includegraphics[width=0.23\linewidth]{imgs/5-results/detailed/ssig/000378_Track19_04_.png}} \, \subfloat[][\centering  \phantom{\,} \resizebox{\adj}{!}{\textbf{\phantom{1}\cite{masood2017sighthound}:}}~\texttt{\textcolor{red}{-}TV7556}\hspace{\textwidth} \phantom{\,} \resizebox{\adj}{!}{\textbf{\phantom{1}\cite{openalprapi}:}}~\texttt{\textcolor{red}{K}TV7556}\hspace{\textwidth} \phantom{\,}
    \resizebox{\adj}{!}{\textbf{\phantom{10}\cite{silva2018license}:}}~\texttt{\textcolor{red}{4}TV7556}\hspace{\textwidth} \phantom{\,}
    \resizebox{\adj}{!}{\textbf{\phantom{1}\cite{laroca2018robust}:}}~\texttt{A\textcolor{red}{I}V7556}\hspace{\textwidth} \phantom{\,}
    \resizebox{\adj}{!}{\textbf{Ours:}}~\texttt{ATV7556}]{
		\includegraphics[width=0.23\linewidth]{imgs/5-results/detailed/ssig/000095_Track6_01_.png}} \, \subfloat[][\centering  \phantom{\,} \resizebox{\adj}{!}{\textbf{\phantom{1}\cite{masood2017sighthound}:}}~\texttt{GMF\textcolor{red}{-}862}\hspace{\textwidth} \phantom{\,} \resizebox{\adj}{!}{\textbf{\phantom{1}\cite{openalprapi}:}}~\texttt{GMF2862}\hspace{\textwidth} \phantom{\,}
    \resizebox{\adj}{!}{\textbf{\phantom{10}\cite{silva2018license}:}}~\texttt{G\textcolor{red}{N}F2862}\hspace{\textwidth} \phantom{\,}
    \resizebox{\adj}{!}{\textbf{\phantom{1}\cite{laroca2018robust}:}}~\texttt{G\textcolor{red}{N}F2862}\hspace{\textwidth} \phantom{\,}
    \resizebox{\adj}{!}{\textbf{Ours:}}~\texttt{GMF2862}]{
		\includegraphics[width=0.23\linewidth]{imgs/5-results/detailed/ssig/000428_Track22_02_.png}} \hspace{1.5mm}
	}
	
	\vspace{-0.5mm}

    \caption{\colored{Some qualitative results obtained on \ssig~\cite{goncalves2016benchmark} by Sighthound~\cite{masood2017sighthound}, OpenALPR~\cite{openalprapi}, Silva \& Jung~\cite{silva2018license}, the preliminary version of our approach~\cite{laroca2018robust}, and the proposed system.}}
    \label{fig:detailed-ssig}
\end{figure}

 \vspace{1.5mm}
\colored{
\noindent \textbf{\dataset~\cite{laroca2018robust}:} this challenging dataset includes $1{,}800$ testing images acquired from inside a vehicle driving through regular traffic in an urban environment, that is, both the vehicles and the camera (inside another vehicle) were moving and most \glspl*{lp} occupy a very small region of the image.
In this sense, the commercial systems did not return any prediction in some images from this dataset where the vehicles are far from the camera.
Regarding the recognition errors, they are very similar to those observed in the \ssig dataset.
Sighthound often confused similar letters and digits, while segmentation failures impaired the results obtained by the approach proposed in our previous work~\cite{laroca2018robust}.
According to Fig.~\ref{fig:detailed-ufpralpr}, the images were collected under different lighting conditions and the four \gls*{alpr} systems found it difficult to correctly recognize certain \glspl*{lp} with shadows or high exposure.
It should be noted that motorcycle \glspl*{lp} (those with two rows of characters) are challenging in nature, as the characters are smaller and closely spaced.
In this context, some authors have evaluated their methods, which do not work for motorcycles or for \glspl*{lp} with two rows of characters, exclusively in images containing cars, overlooking those with motorcycles~\cite{goncalves2018realtime,silva2020realtime}.
}

\begin{figure}[!htb]
    \centering
    \captionsetup[subfigure]{labelformat=empty,font={scriptsize}}


    \resizebox{\linewidth}{!}{
	\subfloat[][\centering  \phantom{\,} \resizebox{\adj}{!}{\textbf{\phantom{1}\cite{masood2017sighthound}:}}~\texttt{ABN8528}\hspace{\textwidth} \phantom{\,} \resizebox{\adj}{!}{\textbf{\phantom{1}\cite{openalprapi}:}}~\texttt{ABN8528}\hspace{\textwidth} \phantom{\,}
    \resizebox{\adj}{!}{\textbf{\phantom{1}\cite{laroca2018robust}:}}~\texttt{ABN8528}\hspace{\textwidth} \phantom{\,} \resizebox{\adj}{!}{\textbf{Ours:}}~\texttt{ABN8528}]{
		\includegraphics[width=0.23\linewidth]{imgs/5-results/detailed/ufpr/000458_track0106_08_.png}} \, \subfloat[][\centering  \phantom{\,} \resizebox{\adj}{!}{\textbf{\phantom{1}\cite{masood2017sighthound}:}}~\texttt{AM\textcolor{red}{DD}663}\hspace{\textwidth} \phantom{\,} \resizebox{\adj}{!}{\textbf{\phantom{1}\cite{openalprapi}:}}~\texttt{AM\textcolor{red}{D}0663}\hspace{\textwidth} \phantom{\,}
    \resizebox{\adj}{!}{\textbf{\phantom{1}\cite{laroca2018robust}:}}~\texttt{AMO0663}\hspace{\textwidth} \phantom{\,} \resizebox{\adj}{!}{\textbf{Ours:}}~\texttt{AMO0663}]{
		\includegraphics[width=0.23\linewidth]{imgs/5-results/detailed/ufpr/001356_track0136_06_.png}} \, \subfloat[][\centering  \phantom{\,} \resizebox{\adj}{!}{\textbf{\phantom{1}\cite{masood2017sighthound}:}}~\texttt{ATT4025}\hspace{\textwidth} \phantom{\,} \resizebox{\adj}{!}{\textbf{\phantom{1}\cite{openalprapi}:}}~\texttt{ATT4025}\hspace{\textwidth} \phantom{\,}
    \resizebox{\adj}{!}{\textbf{\phantom{1}\cite{laroca2018robust}:}}~\texttt{AT\textcolor{red}{U}4025}\hspace{\textwidth} \phantom{\,} \resizebox{\adj}{!}{\textbf{Ours:}}~\texttt{ATT402\textcolor{red}{6}}]{
		\includegraphics[width=0.23\linewidth]{imgs/5-results/detailed/ufpr/001205_track0131_05_.png}} \, \subfloat[][\centering  \phantom{\,} \resizebox{\adj}{!}{\textbf{\phantom{1}\cite{masood2017sighthound}:}}~\texttt{AUG\textcolor{red}{-}936}\hspace{\textwidth} \phantom{\,} \resizebox{\adj}{!}{\textbf{\phantom{1}\cite{openalprapi}:}}~\texttt{A\textcolor{red}{D}G0936}\hspace{\textwidth} \phantom{\,}
    \resizebox{\adj}{!}{\textbf{\phantom{1}\cite{laroca2018robust}:}}~\texttt{AU\textcolor{red}{S}0936}\hspace{\textwidth} \phantom{\,} \resizebox{\adj}{!}{\textbf{Ours:}}~\texttt{AUG0936}]{
		\includegraphics[width=0.23\linewidth]{imgs/5-results/detailed/ufpr/001419_track0138_09_.png}} \hspace{1.5mm}
	}
	
\vspace{2.25mm}
	
	\resizebox{\linewidth}{!}{
    \subfloat[][\centering  \phantom{\,} \resizebox{\adj}{!}{\textbf{\phantom{1}\cite{masood2017sighthound}:}}~\texttt{BBO851\textcolor{red}{-}}\hspace{\textwidth} \phantom{\,} \resizebox{\adj}{!}{\textbf{\phantom{1}\cite{openalprapi}:}}~\texttt{BBO8514}\hspace{\textwidth} \phantom{\,}
    \resizebox{\adj}{!}{\textbf{\phantom{1}\cite{laroca2018robust}:}}~\texttt{BBO85\textcolor{red}{-}4}\hspace{\textwidth} \phantom{\,} \resizebox{\adj}{!}{\textbf{Ours:}}~\texttt{BBO8514}]{
		\includegraphics[width=0.23\linewidth]{imgs/5-results/detailed/ufpr/000591_track0110_21_.png}} \, \subfloat[][\centering  \phantom{\,} \resizebox{\adj}{!}{\textbf{\phantom{1}\cite{masood2017sighthound}:}}~\texttt{IO\textcolor{red}{2}3616}\hspace{\textwidth} \phantom{\,} \resizebox{\adj}{!}{\textbf{\phantom{1}\cite{openalprapi}:}}~\texttt{IOZ3616}\hspace{\textwidth} \phantom{\,}
    \resizebox{\adj}{!}{\textbf{\phantom{1}\cite{laroca2018robust}:}}~\texttt{IOZ3616}\hspace{\textwidth} \phantom{\,} \resizebox{\adj}{!}{\textbf{Ours:}}~\texttt{IOZ3616}]{
		\includegraphics[width=0.23\linewidth]{imgs/5-results/detailed/ufpr/000864_track0119_24_.png}} \, \subfloat[][\centering  \phantom{\,} \resizebox{\adj}{!}{\textbf{\phantom{1}\cite{masood2017sighthound}:}}~\texttt{A\textcolor{red}{O}W1379}\hspace{\textwidth} \phantom{\,} \resizebox{\adj}{!}{\textbf{\phantom{1}\cite{openalprapi}:}}~\texttt{\textcolor{red}{N}QW1379}\hspace{\textwidth} \phantom{\,}
    \resizebox{\adj}{!}{\textbf{\phantom{1}\cite{laroca2018robust}:}}~\texttt{\textcolor{red}{-}OW\textcolor{red}{7}379}\hspace{\textwidth} \phantom{\,} \resizebox{\adj}{!}{\textbf{Ours:}}~\texttt{A\textcolor{red}{O}W1379}]{
		\includegraphics[width=0.23\linewidth]{imgs/5-results/detailed/ufpr/001083_track0127_03_.png}} \, \subfloat[][\centering  \phantom{\,} \resizebox{\adj}{!}{\textbf{\phantom{1}\cite{masood2017sighthound}:}}~\texttt{AIQ\textcolor{red}{-Q}56}\hspace{\textwidth} \phantom{\,} \resizebox{\adj}{!}{\textbf{\phantom{1}\cite{openalprapi}:}}~\texttt{A\textcolor{red}{T}Q1056}\hspace{\textwidth} \phantom{\,}
    \resizebox{\adj}{!}{\textbf{\phantom{1}\cite{laroca2018robust}:}}~\texttt{A\textcolor{red}{UC}1056}\hspace{\textwidth} \phantom{\,} \resizebox{\adj}{!}{\textbf{Ours:}}~\texttt{A\textcolor{red}{T}Q1056}]{
		\includegraphics[width=0.23\linewidth]{imgs/5-results/detailed/ufpr/000602_track0111_02_.png}} \hspace{1.5mm}
	}
	
	\vspace{-0.5mm}

    \caption{\colored{Some qualitative results obtained on \dataset~\cite{laroca2018robust} by Sighthound~\cite{masood2017sighthound}, OpenALPR~\cite{openalprapi}, the preliminary version of our approach~\cite{laroca2018robust}, and the proposed system.}}
    \label{fig:detailed-ufpralpr}
\end{figure}

\vspace{1.5mm}
\colored{
\noindent \textbf{Final remarks:} while being able to process in real time, the proposed system is also capable of correctly recognizing \glspl*{lp} from several countries/regions in images taken under different conditions.
In general, our \gls*{alpr} system failed in challenging cases where one character becomes very similar to another due to factors such as shadows and occlusions (note that some of the baselines also failed in most of these cases).
We believe that vehicle information, such as make and model, can be explored in our system's pipeline in order to make it even more robust and prevent errors in such~cases.
}