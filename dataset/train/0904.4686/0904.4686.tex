\documentclass{llncs}

\usepackage{amsfonts}
\usepackage{amsmath}
\usepackage{amssymb}
\usepackage{enumerate}
\usepackage{epsfig}
\usepackage{pseudocode}

\def\Kset{\mathbb{K}}
\def\Bset{\mathbb{B}}
\def\Nset{\mathbb{N}}
\def\Rset{\mathbb{R}}
\def\argmin{\mathop{\rm argmin}}
\def\argmax{\mathop{\rm argmax}\limits}
\def\Tset{\Rset_+ \cup \{\infty\}}

\newcommand{\bi}{\begin{itemize}}
\newcommand{\ei}{\end{itemize}}
\newcommand{\be}{\begin{enumerate}}
\newcommand{\ee}{\end{enumerate}}
\newcommand{\bd}{\begin{description}}
\newcommand{\ed}{\end{description}}

\newcommand{\set}[1]{\{#1\}}
\newcommand{\0}{\overline{0}}
\newcommand{\1}{\overline{1}}
\newcommand{\e}{\epsilon}
\newcommand{\+}{\oplus}
\renewcommand{\.}{\otimes}
\newcommand{\K}{(\Kset, \+, \., \0, \1)}
\newcommand{\B}{(\Bset, \vee, \wedge, 0, 1)}
\newcommand{\T}{(\Rset_+ \cup \set{\infty}, \min, +, \infty, 0)}
\newcommand{\R}{(\Rset_+ \cup \set{\infty}, +, \times, 0, 1)}

\newcommand{\eq}[1]{Eq.~(\ref{#1})}
\newcommand{\eqs}[2]{Eqs.~(\ref{#1})~and~(\ref{#2})}
\newcommand{\fig}[1]{Fig.~(\ref{#1})}
\newcommand{\figs}[2]{Figs.~(\ref{#1}){#2}}
\newcommand{\sect}[1]{Section~(\ref{#1})}

\newcommand{\level}{\textrm{level}}
\newcommand{\floor}[1]{\lfloor #1 \rfloor}
\newcommand{\ceiling}[1]{\lceil #1 \rceil}

\newcommand{\Aut}{A = (Q, I, F, \Sigma, \delta, \sigma, \lambda, \rho)}
\newcommand{\Trans}{T = (\Sigma, \Delta, Q, I, F, E, \lambda, \rho)}
\newcommand{\dom}[1]{{\it Dom}(#1)}
\newcommand{\ipsfig}[2]{\scalebox{#1}{\psfig{#2}}}
\newcommand{\ipsfigc}[2]{\centerline{\scalebox{#1}{\psfig{#2}}}}
\newcommand{\ignore}[1]{}

\title{Linear-Space Computation of the Edit-Distance between a String 
  and a Finite Automaton}
\author{Cyril Allauzen \inst{1} \and Mehryar Mohri \inst{2,1}}
\institute{
        Google Research\\
        76 Ninth Avenue, New York, NY 10011, US.\\
        \and
	Courant Institute of Mathematical Sciences\\
	251 Mercer Street, New York, NY 10012, US.\\
}

\begin{document}
\maketitle

\begin{abstract}
  The problem of computing the edit-distance between a string and a
  finite automaton arises in a variety of applications in
  computational biology, text processing, and speech recognition. This
  paper presents linear-space algorithms for computing the
  edit-distance between a string and an arbitrary weighted automaton
  over the tropical semiring, or an unambiguous weighted automaton
  over an arbitrary semiring. It also gives an efficient linear-space
  algorithm for finding an optimal alignment of a string and such a
  weighted automaton.
\end{abstract}

\section{Introduction}

The problem of computing the edit-distance between a string and a
finite automaton arises in a variety of applications in computational
biology, text processing, and speech recognition
\cite{durbin,gusfield,navarro-raffinot,pevzner,edit}. This may be to
compute the edit-distance between a protein sequence and a family of
protein sequences compactly represented by a finite automaton
\cite{durbin,gusfield,pevzner}, or to compute the error rate of a word
lattice output by a speech recognition with respect to a reference
transcription \cite{edit}. A word lattice is a weighted automaton,
thus this further motivates the need for computing the edit-distance
between a string and a weighted automaton. In all these cases, an
optimal alignment is also typically sought. In computational biology,
this may be to infer the function and various properties of the
original protein sequence from the one it is best aligned with. In
speech recognition, this determines the best transcription hypothesis
contained in the lattice.

This paper presents linear-space algorithms for computing the
edit-distance between a string and an arbitrary weighted automaton over
the tropical semiring, or an unambiguous weighted automaton over an
arbitrary semiring. It also gives an efficient linear-space algorithm
for finding an optimal alignment of a string and such a weighted
automaton. Our linear-space algorithms are obtained by using the same
generic shortest-distance algorithm but by carefully defining
different queue disciplines. More precisely, our meta-queue
disciplines are derived in the same way from an underling queue
discipline defined over states with the same level.

The connection between the edit-distance and the shortest distance in
a directed graph was made very early on (see
\cite{gusfield,crochemore-hancart-lecroq,crochemore-rytter,crochemore-rytter02}
for a survey of string algorithms). This paper revisits some of these
algorithms and shows that they are all special instances of the same
generic shortest-distance algorithm using different queue
disciplines. We also show that the linear-space algorithms all
correspond to using the same meta-queue discipline using different
underlying queues. Our approach thus provides a better understanding
of these classical algorithms and makes it possible to easily
generalize them, in particular to weighted automata.

The first algorithm to compute the edit-distance between a string 
and a finite automaton  as well as their alignment was due to
Wagner~\cite{wagner} (see also \cite{wagner-seiferas}). Its time
complexity was in  and its space complexity in
, where  denotes the alphabet
and  the number of states of . Sankoff and
Kruskal~\cite{sankoff} pointed out that the time and space complexity
 can be achieved when the automaton  is acyclic. Myers
and Miller~\cite{myers-miller89} significantly improved on previous
results. They showed that when  is acyclic or when it is a
\emph{Thompson automaton}, that is an automaton obtained from a
regular expression using Thompson's construction \cite{thompson}, the
edit-distance between  and  can be computed in  time
and  space. They also showed, using a technique due to
Hirschberg \cite{hirschberg}, that the optimal alignment between 
and  can be obtained in  space, and in 
time if  is acyclic, and in  time when  is
a Thompson automaton.

The remainder of the paper is organized as
follows. Section~\ref{sec:preliminaries} introduces the definition of
semirings, and weighted automata and transducers. In
Section~\ref{sec:edit-distance}, we give a formal definition of the
edit-distance between a string and a finite automaton, or a weighted
automaton. Section~\ref{sec:algo} presents our linear-space
algorithms, including the proof of their space and time complexity and
a discussion of an improvement of the time complexity for automata
with some favorable graph structure property.

\section{Preliminaries}
\label{sec:preliminaries}

This section gives the standard definition and specifies the notation
used for weighted transducers and automata which we use in our
computation of the edit-distance.

\emph{Finite-state transducers} are finite automata \cite{perrin} in
which each transition is augmented with an output label in addition to
the familiar input label \cite{berstel,eilenberg}. Output labels are
concatenated along a path to form an output sequence and similarly
input labels define an input sequence. \emph{Weighted transducers} are
finite-state transducers in which each transition carries some weight
in addition to the input and output labels
\cite{soittola,kuich}. Similarly, \emph{weighted automata} are finite
automata in which each transition carries some weight in addition to
the input label. A path from an initial state to a final state is
called an \emph{accepting path}. A weighted transducer or weighted
automaton is said to be \emph{unambiguous} if it admits no two
accepting paths with the same input sequence.

The weights are elements of a semiring , that is a ring that may
lack negation \cite{kuich}. Some familiar semirings are the tropical
semiring  and the probability semiring , where 
denotes the set of non-negative real numbers. In the following, we
will only consider weighted automata and transducers over the tropical
semiring. However, all the results of section~\ref{sec:algo:edit} hold
for an unambiguous weighted automaton  over an arbitrary semiring.

The following gives a formal definition of weighted transducers.

\begin{definition}
  A {\em weighted finite-state transducer}  over the tropical
  semiring  is an 8-tuple  where  is the finite
  input alphabet of the transducer,  its finite output
  alphabet,  is a finite set of states,  the set of
  initial states,  the set of final states,  a finite set of
  transitions,  the initial weight
  function, and  the final weight function
  mapping  to .

\end{definition}
We define the {\em size} of  as  where  is the number of states and  the number of
transitions of .

The weight of a path  in  is obtained by summing the weights
of its constituent transitions and is denoted by . The weight
of a pair of input and output strings  is obtained by taking
the minimum of the weights of the paths labeled with  from an
initial state to a final state.

For a path , we denote by  its origin state and by
 its destination state. We also denote by  the
set of paths from the initial states  to the final states 
labeled with input string  and output string .  The weight  associated by  to a pair of strings  is defined by:

Figure~\ref{fig:examples}(a) shows an example of weighted transducer
over the tropical semiring.

\begin{figure*}[t]
\begin{center}
\begin{tabular}{c@{\hspace{2cm}}c}
\ipsfig{.4}{figure=t1.ps} & \ipsfig{.4}{figure=a1.ps}\\
(a) & (b)
\end{tabular}
\end{center}
\caption[]{(a) Example of a weighted transducer . (b) Example of a
weighted automaton . . A bold
circle indicates an initial state and a double-circle a final state.
The final weight  of a final state  is indicated after the
slash symbol representing . }
\label{fig:examples}
\end{figure*}
\emph{Weighted automata} can be defined as
weighted transducers  with identical input and output labels, for
any transition. Thus, only pairs of the form  can have a
non-zero weight by , which is why the weight associated by  to
 is abusively denoted by  and identified with the
\emph{weight associated by  to }.  Similarly, in the graph
representation of weighted automata, the output (or input) label is
omitted. Figure~\ref{fig:examples}(b) shows an example.

\section{Edit-distance}
\label{sec:edit-distance}

We first give the definition of the edit-distance between a string and
a finite automaton.

Let  be a finite alphabet, and let  be defined by
. An element of 
can be seen as a symbol edit operation:  is a deletion,
 an insertion, and  with  a
substitution.  We will denote by  the natural morphism between
 and  defined by . An {\em
  alignment}  between two strings  and  is an element of
 such that .

Let  be a function associating a
non-negative cost to each edit operation. The cost of an alignment
 is defined as .

\begin{definition}
  The {\em edit-distance}  of two strings  and  is the
  minimal cost of a sequence of symbols insertions, deletions or
  substitutions transforming one string into the other:

  When  is the function defined by  and  for all ,  in
   such that , the edit-distance is also known as
  the Levenshtein distance. The {\em edit-distance  between a
  string  and a finite automaton } can then be defined as

  where  denotes the regular language accepted by . The {\em
  edit-distance  between a string  and a weighted automaton
   over the tropical semiring} is defined as:

\end{definition}

\section{Algorithms}
\label{sec:algo}

In this section, we present linear-space algorithms both for computing
the edit-distance  between an arbitrary string  and an
automaton , and an optimal alignment between  and , that is
an alignment  such that .

We first briefly describe two general algorithms that we will use as
subroutines.

\subsection{General algorithms}

\subsubsection{Composition.}
\label{sec:composition}

The {\em composition} of two weighted transducers  and  over
the tropical semiring with matching input and output alphabets
, is a weighted transducer denoted by  defined by:

 can be computed from  and  using the
composition algorithm for weighted transducers
\cite{pereira-riley,ecai}. States in the composition 
are identified with pairs of a state of  and a state of . In
the absence of transitions with  inputs or outputs, the
transitions of  are obtained as a result of the
following matching operation applied to the transitions of  and
:

A state  of  is initial (resp. final) iff
 and  are initial (resp. final) and, when it is final, its
initial (resp.final) weight is the sum of the initial (resp. final)
weights of  and . In the worst case, all transitions of
 leaving a state  match all those of  leaving state
, thus the space and time complexity of composition is quadratic,
that is .

\subsubsection{Shortest distance.}
\label{sec:sd}

Let  be a weighted automaton over the tropical semiring. The {\em
shortest distance} from  to  is defined as

It can be computed using the generic single-source shortest-distance
algorithm of \cite{shortest-distance}, a generalization of the
classical shortest-distance algorithms. This generic shortest-distance
algorithm works with an arbitrary \emph{queue discipline}, that is the
order according to which elements are extracted from a queue.  We
shall make use of this key property in our algorithms. The pseudocode
of a simplified version of the generic algorithm for the tropical
semiring is given in Figure~\ref{fig:alg:shortest}.

\begin{figure}[ht]
{\small 

\bc
\li \FOR \EACH  \DO\\
\li \> \\
\li \\
\li \\
\li \WHILE  \DO\\
\li \>  \\
\li \>  \\
\li \> \FOR \EACH  \DO \\
\li \> \> \IF  \THEN\\
\li \> \> \> \\
\li \> \> \> \IF    \THEN\\
\li \> \> \> \> \\
\ec
}
\caption{Pseudocode of the generic shortest-distance algorithm.}
\label{fig:alg:shortest}
\end{figure}

The complexity of the algorithm depends on the queue discipline
selected for . Its general expression is

where  denotes the number of times state  is
extracted from queue ,  the cost of extracting a
state from ,  the cost of inserting a state in ,
and  the cost of an assignment.

With a shortest-first queue discipline implemented using a heap, the
algorithm coincides with Dijkstra's algorithm \cite{dijkstra} and its
complexity is . For an acyclic automaton and
with the topological order queue discipline, the algorithm coincides
with the standard linear-time () shortest-distance
algorithm \cite{rivest}.

\subsection{Edit-distance algorithms}
\label{sec:algo:edit}

The edit cost function  can be naturally represented by a one-state
weighted transducer over the tropical semiring , or  in the absence of
ambiguity, with each transition corresponding to an edit operation:
.

\begin{lemma}
\label{lem:edit}
  Let  be a weighted automaton over the tropical semiring and let
   be the finite automaton representing a string . Then, the
  edit-distance between  and  is the shortest-distance from the
  initial state to a final state in the weighted transducer .
\end{lemma}
\begin{proof}
  Each transition  in  corresponds to an edit operation , and each path  corresponds to an alignment
   between  and . The cost of that alignment
  is, by definition of , . Thus,  defines
  the function:

  for any strings ,  in . Since  is an automaton and
   is the only string accepted by , it follows from the definition
  of composition that .
  The shortest-distance from the initial state to a final state in  is
  then:

  that is the edit-distance between  and .\qed
\end{proof}

\begin{figure*}[t]
\begin{center}
\begin{tabular}{c@{\hspace{2cm}}c@{\hspace{2cm}}c}
\ipsfig{.45}{figure=aba.ps} & \ipsfig{.45}{figure=ab_star.ps}\\
(a) & (b)   \\
\ipsfig{.45}{figure=t.ps} & \hspace{-5cm}\ipsfig{.5}{figure=u.ps}\\
(c) & (d)
\end{tabular}
\end{center}
\caption[]{ (a) Finite automaton  representing the string . (b) Finite automaton . (c) Edit transducer  over the
alphabet  where the cost of any insertion, deletion and
substitution is 1.  (d) Weighted transducer .}
\label{fig:edit}
\end{figure*}

Figure~\ref{fig:edit} shows an example illustrating Lemma~\ref{lem:edit}.
Using the lateral strategy of the 3-way composition algorithm of
\cite{3way} or an \emph{ad hoc} algorithm exploiting the structure of
,  can be computed in  time. The
shortest-distance algorithm presented in Section~\ref{sec:sd} can then
be used to compute the shortest distance from an initial state of 
to a final state and thus the edit distance of  and . Let us
point out that different queue disciplines in the computation of that
shortest distance lead to different algorithms and complexities.  In
the next section, we shall give a queue discipline enabling us to
achieve a linear-space complexity.

\subsection{Edit-distance computation in linear space}

Using the shortest-distance algorithm described in
Section~\ref{sec:sd} leads to an algorithm with space complexity
linear in the size of , i.e. in . However, taking
advantage of the topology of , it is possible to design a queue
discipline that leads to a linear space complexity .

We assume that the finite automaton  representing the string  is
topologically sorted. A state  in the composition  can be identified with a triplet  where  is a
state of , 0 the unique state of , and  a state of . Since
 has a unique state, we further simplify the notation by
identifying each state  with a pair . For a state  of , we will refer to  by the \emph{level of }. A key
property of the levels is that there is a transition in  from 
to  iff  or .
Indeed, a transition from  to  in  corresponds to
taking a transition in  (in that case  since  is
topologically sorted) or staying at the same state in  and
taking an input- transition in  (in that case ).

From any queue discipline  on the states of , we can derive
a new queue discipline  over  defined for all  in
 as follows:


\begin{proposition}
\label{prop:algo}
Let  be a queue discipline that requires at most  space
to maintain a queue over any set of states . Then, the
edit-distance between  and  can be computed in linear space,
, using the queue discipline .
\end{proposition}
\begin{proof}
  The benefit of the queue discipline  is that when computing
  the shortest distance to  in , only the shortest
  distances to the states in  of level  and  need to be
  stored in memory. The shortest distances to the states of level
  strictly less than  can be safely discarded. Thus, the space
  required to store the shortest distances is in .
 
  Similarly, there is no need to store in memory the full transducer
  . Instead, we can keep in memory the last two levels active in
  the shortest-distance algorithm. This is possible because the
  computation of the outgoing transitions of a state with level 
  only requires knowledge about the states with level  and . Therefore, the space used to store the active part of  is in
  .  Thus, it follows that the space
  required to compute the edit-distance of  and  is linear, that
  is in .\qed
\end{proof}

The time complexity of the algorithm depends on the underlying queue
discipline .  A natural choice is for  is the
shortest-first queue discipline, that is the queue discipline used in
Dijkstra's algorithm. This yields the following corollary.
\begin{corollary}
\label{cor:cyclic}
The edit-distance between a string  and an automaton  can be
computed in time  and space 
using the queue discipline .
\end{corollary}
\begin{proof}
  A shortest-first queue is maintained for each level and contains at
  most  states. The cost for the global queue of an insertion,
  , or an assignment, , is in  since it corresponds to inserting in or updating one of the
  underlying level queues. Since , the general
  expression of the complexity (\ref{eq:gen_exp}) leads to an overall
  time complexity of  for the shortest-distance
  algorithm.\qed
\end{proof}

When the automaton  is acyclic, the time complexity can be further
improved by using for  the topological order queue discipline.
\begin{corollary}
\label{cor:acyclic}
If the automaton  is acyclic, the edit-distance between  and 
can be computed in time  and space  using
the queue discipline  with the topological order queue
discipline for .
\end{corollary}
\begin{proof}
  Computing the topological order for  would require 
  space.  Instead, we use the topological order on , which can be
  computed in , to define the underlying queue discipline.
  The order inferred by (\ref{eq:queue}) is then a topological order
  on .  \qed
\end{proof}

Myers and Miller \cite{myers-miller89} showed that when  is a
Thompson automaton, the time complexity can be reduced to 
even when  is not acyclic.  This is possible because of the
following observation: in a weighted automaton over the tropical
semiring, there exists always a shortest path that is \emph{simple},
that is with no cycle, since cycle weights cannot decrease 
path weight.

In general, it is not clear how to take advantage of this
observation. However, a Thompson automaton has additionally the
following structural property: a \emph{loop-connectedness} of one.
The \emph{loop-connectedness} of  is  if in any depth-first
search of , a simple path goes through at most  back
edges. \cite{myers-miller89} showed that this property, combined with
the observation made previously, can be used to improve the time
complexity of the algorithm.  The results of \cite{myers-miller89} can
be generalized as follows.

\begin{corollary}
  If the loop-connectedness of  is , then the edit-distance
  between  and  can be computed in  time and
   space.
\end{corollary}
\begin{proof}
  We first use a depth-first search of , identify back edges, and
  mark them as such. We then compute the topological order for ,
  ignoring these back edges.  Our underlying queue discipline 
  is defined such that a state  is ordered first based on
  the number of times it has been enqueued and secondly based on the
  order of  in the topological order ignoring back edges. This
  underlying queue can be implemented in  space with
  constant time costs for the insertion, extraction and updating
  operations. The order  derived from  is then not
  topological for a transition  iff  was obtained by matching a
  back edge in  and . When such a
  transition  is visited,  is reinserted in the queue.

  When state  is dequeued for the th time, the value of 
  is the weight of the shortest path from the initial state to 
  that goes through at most  back edges.  Thus, the inequality
   holds for all  and, since the costs for
  managing the queue, , , and
  , are constant, the time complexity of the algorithm
  is in .  \qed
\end{proof}


\subsection{Optimal alignment computation in linear space}

The algorithm presented in the previous section can also be used to
compute an optimal alignment by storing a back pointer at each state
in . However, this can increase the space complexity up to .  The use of back pointers to compute the best alignment can
be avoided by using a technique due to Hirschberg \cite{hirschberg},
also used by \cite{myers-miller88,myers-miller89}.

As pointed out in previous sections, an optimal alignment between 
and  corresponds to a shortest path in .
We will say that a state  in  is a {\em midpoint} of an optimal
alignment between  and  if  belongs to a shortest path in
 and .

\begin{lemma}
Given a pair , a midpoint of the optimal alignment between 
and  can be computed in  space with a time complexity in
 if  is acyclic and in 
otherwise.
\end{lemma}
\begin{proof}
  Let us consider . For a state  in  let
   denote the shortest distance from the initial state to ,
  and by  the shortest distance from  to a final state.
  For a given state  in ,  is the
  cost of the shortest path going through .  Thus, for any ,
  the edit-distance between  and  is .

  For a fixed , we can compute both  and
   for all  in  time (or
   time if  is acyclic) and in linear space 
  using the algorithm from the previous section forward and backward
  and stopping at level  in each case. Running the algorithm
  backward (exchanging initial and final states and permuting the
  origin and destination of every transition) can be seen as computing
  the edit-distance between  and , the {\em mirror images}
  of  and .
 
  Let us now set  and . It then follows that  is a midpoint of
  the optimal alignment. Hence, for a pair , the running-time
  complexity of determining the midpoint of the alignment is in  if  is acyclic and  otherwise. \qed
\end{proof}

The algorithm proceeds recursively by first determining the midpoint
of the optimal alignment.  At step 0 of the recursion, we first find
the midpoint  between  and . Let  and  be
such that  and , and let  and  be
the automaton obtained from  by respectively changing the final
state to  in  and the initial state to  in . We
can now recursively find the alignment between  and  and
between  and .

\begin{theorem}
  An optimal alignment between a string  and an automaton  can
  be computed in linear space  and in time 
  if  is acyclic,  otherwise.
\end{theorem}
\begin{proof}
  We can assume without loss of generality that the length of  is a
  power of 2.  At step  of the recursion, we need to compute the
  midpoints for  string-automaton pairs . Thus, the complexity of step  is in
   since
   for all .  When  is acyclic, the 
  factor can be avoided and the equality  holds, thus the time complexity of step  is in
  . In the general case, each  can be in the
  order of , thus the complexity of step  is in .

  Since there are at most  steps in the recursion, this
  leads to an overall time complexity in  if  is
  acyclic and  in general.\qed
\end{proof}
When the loop-connectedness of  is , the time complexity can
be improved to  in the general case.

\section{Conclusion}

We presented general algorithms for computing in linear space both the
edit-distance between a string and a finite automaton and their
optimal alignment. Our algorithms are conceptually simple and make use
of existing generic algorithms. Our results further provide a better
understanding of previous algorithms for more restricted automata by
relating them to shortest-distance algorithms and general queue
disciplines.

\bibliographystyle{abbrv}
\bibliography{lse}
\end{document}
