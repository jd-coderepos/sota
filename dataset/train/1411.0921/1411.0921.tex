\RequirePackage{ifpdf}
\newif\ifpdf
\ifx\pdfoutput\undefined
\pdffalse
\else
\pdfoutput=1
\pdfcompresslevel=9
\pdftrue
\fi

\ifpdf
\documentclass[pdftex]{llncs}
\else
\documentclass[dvips]{llncs}
\fi

\usepackage{fullpage}
\usepackage{fancyhdr}
\usepackage[T1]{fontenc}
\usepackage{amsmath}
\usepackage{amssymb}
\usepackage{graphicx}
\usepackage{placeins}
\usepackage{color}
\usepackage{colortbl}
\usepackage[dvips]{epsfig}
\usepackage{xspace}
\usepackage{algorithm}
\usepackage{algorithmicx}
\usepackage{algpseudocode}
\usepackage{float}
\newfloat{algorithm}{t}{lop}
\usepackage{floatflt}
\definecolor{grey}{gray}{0.7}
\usepackage{url}
\usepackage{lineno}
\usepackage{wrapfig}


\usepackage{ae, aecompl}
\usepackage{colortbl}
\usepackage{numprint}

\pagestyle{plain}



\renewcommand{\topfraction}{0.9}

\newcommand{\R}{\mathbb{R}}
\newcommand{\N}{\mathbb{N}}
\newcommand{\eg}{e.\,g.\xspace}
\newcommand{\Eg}{E.\,g.\xspace}
\newcommand{\ie}{i.\,e.\xspace}
\newcommand{\Ie}{I.\,e.\xspace}
\renewcommand{\floatpagefraction}{0.9}





\newcommand{\Pro}[1]{\mathbf{Pr} \left[\,#1\,\right]}
\newcommand{\pro}[1]{\mathbf{Pr} [\,#1\,]}
\newcommand{\Ex}[1]{\mathbb{E} \left[\,#1\,\right]}
\newcommand{\ex}[1]{\mathbb{E} [\,#1\,]}
\newcommand{\hit}{- [\pi]_s (H[v,s]-H[u,s])}

\newcommand{\centre}[1]{z_{#1}}
\newcommand{\centres}{Z}
\newcommand{\degree}{\operatorname{deg}}
\newcommand{\maxdeg}{\operatorname{maxdeg}}
\newcommand{\diam}{\operatorname{diam}}
\newcommand{\res}{\operatorname{res}}
\newcommand{\Res}{\operatorname{Res}}
\newcommand{\cond}{\operatorname{cond}}
\newcommand{\Cond}{\operatorname{Cond}}
\newcommand{\proxy}{\operatorname{proxy}}
\newcommand{\excond}{\operatorname{ex\_cond}}
\newcommand{\exres}{\operatorname{ex\_res}}
\newcommand{\exalg}{\operatorname{ex\_alg}}
\newcommand{\dist}{\operatorname{dist}}
\newcommand{\ord}{\operatorname{ord}}
\newcommand{\Vor}{\operatorname{Vor}}
\newcommand{\M}{\mathbf{M}}
\newcommand{\LL}{\mathbf{L}}
\newcommand{\RR}{\mathbb{R}}
\newcommand{\f}{\hat{f}}
\newcommand{\e}{\mathbf{e}}
\newcommand{\dhat}{d}
\newcommand{\llminimal}{-minimal\xspace}
\newcommand{\NP}{}
\newcommand{\disjbigcup}{\mathop{\dot{\bigcup}}} 
\newcommand{\disjcup}{\mathop{\dot{\cup}}} 
\newcommand{\bigO}{\mathcal{O}} 
\newcommand{\polylog}{\operatorname{polylog}} 

\DeclareMathOperator{\intraWeight}{\it intraWeight}
\DeclareMathOperator{\interWeight}{\it interWeight}
\DeclareMathOperator{\subtreeVol}{\it subtreeVol}
\DeclareMathOperator{\cutWeight}{\it cutWeight}
\DeclareMathOperator{\conduct}{\it conduct}
\DeclareMathOperator{\inCutSet}{\it inCutSet}

\newcommand{\iec}{\textit{i.\,e.},\xspace}
\newcommand{\egc}{\textit{e.\,g.},\xspace}
\newcommand{\etal}{\textit{et al.}\xspace}
\newcommand{\Wlog}{w.\,l.\,o.\,g.\xspace}
\newcommand{\wrt}{w.\,r.\,t.\xspace}
\newcommand{\st}{s.\,t.\xspace}
\newcommand{\cf}{cf.\xspace}

\newcommand{\dibap}{\textsc{DibaP}\xspace}
\newcommand{\pdibap}{\textsc{PDibaP}\xspace}
\newcommand{\bubble}{\textsc{Bubble}\xspace}
\newcommand{\smooth}{\textsc{Smooth}}
\newcommand{\bubfosc}{\textsc{Bubble-FOS/C}\xspace}
\newcommand{\bubfost}{\textsc{Bubble-FOS/T}\xspace}
\newcommand{\metis}{\textsc{METIS}\xspace}
\newcommand{\gpmetis}{\textsc{gpMetis}\xspace}
\newcommand{\ndmetis}{\textsc{ndMetis}\xspace}
\newcommand{\kmetis}{\textsc{kMeTiS}\xspace}
\newcommand{\parmetis}{\textsc{ParMETIS}\xspace}
\newcommand{\kappart}{\textsc{KaPPa}\xspace}
\newcommand{\kahip}{\textsc{KaHIP}\xspace}
\newcommand{\initial}{\textsc{Initial}\xspace}
\newcommand{\initialM}{\textsc{Initial}_{\textsc{g}} \xspace}
\newcommand{\initialMO}{\textsc{Initial}_{\textsc{n}} \xspace}
\newcommand{\random}{\textsc{Random}\xspace}
\newcommand{\randomM}{\textsc{Random}_{\textsc{g}} \xspace}
\newcommand{\randomMO}{\textsc{Random}_{\textsc{n}} \xspace}
\newcommand{\rcm}{\textsc{RCM}\xspace}
\newcommand{\rcmM}{\textsc{RCM}_{\textsc{g}} \xspace}
\newcommand{\rcmMO}{\textsc{RCM}_{\textsc{n}} \xspace}
\newcommand{\durebi}{\textsc{DRB}\xspace}
\newcommand{\durebiM}{\textsc{DRB}_{\textsc{g}} \xspace}
\newcommand{\durebiMO}{\textsc{DRB}_{\textsc{n}} \xspace}
\newcommand{\greedyall}{\textsc{GreedyAll}\xspace}
\newcommand{\greedyallM}{\textsc{GreedyAll}_{\textsc{g}} \xspace}
\newcommand{\greedyallMO}{\textsc{GreedyAll}_{\textsc{n}} \xspace}
\newcommand{\greedyallc}{\textsc{GreedyAllC}\xspace}
\newcommand{\greedyallcM}{\textsc{GreedyAllC}_{\textsc{g}} \xspace}
\newcommand{\greedyallcMO}{\textsc{greedyAllC}_{\textsc{n}} \xspace}
\newcommand{\greedymin}{\textsc{GreedyMin}\xspace}
\newcommand{\greedyminM}{\textsc{GreedyMin}_{\textsc{g}} \xspace}
\newcommand{\greedyminMO}{\textsc{GreedyMin}_{\textsc{n}} \xspace}
\newcommand{\greedyminc}{\textsc{GreedyMinC}\xspace}
\newcommand{\greedymincM}{\textsc{GreedyMinC}_{\textsc{g}} \xspace}
\newcommand{\greedymincMO}{\textsc{GreedyMinC}_{\textsc{n}} \xspace}
\newcommand{\walshawlarge}{\textsc{WalshawLarge}\xspace}
\newcommand{\complexnets}{\textsc{ComplexNets}\xspace}
\newcommand{\jostle}{\textsc{Jostle}\xspace}
\newcommand{\zoltan}{\textsc{Zoltan}\xspace}
\newcommand{\parkway}{\textsc{Parkway}\xspace}
\newcommand{\graclus}{\textsc{Graclus}\xspace}
\newcommand{\party}{\textsc{Party}\xspace}
\newcommand{\scotch}{\textsc{Scotch}\xspace}
\newcommand{\thrsh}{\xspace}
\newcommand{\trunccons}{\textsc{TruncCons}\xspace}
\newcommand{\consol}{\texttt{Consolidation}\xspace}
\newcommand{\consols}{\texttt{Consolidations}\xspace}
\newcommand{\asspart}{\texttt{AssignPartition}\xspace}
\newcommand{\assclus}{\texttt{AssignCluster}\xspace}
\newcommand{\asssd}{\texttt{AssignSubdomain}\xspace}
\newcommand{\compcen}{\texttt{ComputeCenters}\xspace}
\newcommand{\initcen}{\textsc{LoadBasedInitialCenters}\xspace}
\newcommand{\NN}{\mbox{\rm IN}}
\newcommand{\closu}[1]{\overline{#1}}
\newcommand{\djoko}{Djokovi\'{c} relation\xspace}
\newcommand{\subE}[1]{{#1}_{\mathcal{E}}}
\newcommand{\argmin}{\operatornamewithlimits{argmin}}
\newcommand{\argmax}{\operatornamewithlimits{argmax}}
\newcommand{\intmax}{\textsc{int\_max}\xspace}

\newcommand{\anoe}[1]{\textcolor{blue}{[AN: #1]}\xspace}
\newcommand{\hmey}[1]{\textcolor{red}{[HM: #1]}\xspace}
\newcommand{\rgla}[1]{\textcolor{grey}{[RG: #1]}\xspace}
\renewcommand{\anoe}[1]{}
\renewcommand{\hmey}[1]{}
\renewcommand{\rgla}[1]{}

\renewcommand{\baselinestretch}{1.03}

\renewcommand\dblfloatpagefraction{.90}













\begin{document}


\title{Algorithms for Mapping Parallel Processes onto\\Grid and Torus Architectures}

\author{Roland Glantz \and Henning Meyerhenke \and Alexander Noe}

\institute{Karlsruhe Institute of Technology (KIT), Karlsruhe, Germany}
\maketitle
\vspace{-4mm}

\begin{abstract}
Static mapping is the assignment of parallel processes to the
processing elements (PEs) of a parallel system, where the assignment
does not change during the application's lifetime. In our scenario we
model an application's computations and their dependencies by an
application graph. This graph is first partitioned into (nearly)
equally sized blocks. These blocks need to communicate at block
boundaries.  To assign the processes to PEs, our goal is to compute a
communication-efficient bijective mapping between the blocks and the
PEs.

This approach of partitioning followed by bijective mapping has many
degrees of freedom. Thus, users and developers of parallel
applications need to know more about which choices work for which
application graphs and which parallel architectures. To this end, we
not only develop new mapping algorithms (derived from known greedy
methods).  We also perform extensive experiments involving different
classes of application graphs (meshes and complex networks),
architectures of parallel computers (grids and tori), as well as
different partitioners and mapping algorithms. Surprisingly, the
quality of the partitions, unless very poor, has little influence on
the quality of the mapping.

More importantly, one of our new mapping algorithms always yields the
best results in terms of the quality measure maximum congestion when
the application graphs are complex networks.  In case of meshes as
application graphs, this mapping algorithm always leads in terms of
maximum congestion \emph{and} maximum dilation, another common quality
measure.
\end{abstract}




\section{Introduction}
\label{sec:intro}
Symmetric dependencies of computations within a parallel application can be modeled by an
undirected graph , called \emph{application graph}, \eg the
mesh of a numerical simulation. Iterative algorithms in such a
simulation act upon the vertices of , and for each such vertex
require the values of the neighboring vertices from the previous
iteration. Thus, a vertex of  represents some computation, and an
edge of  indicates a dependency between computations, \ie
an exchange of data. It is important to note that this modeling is not restricted 
to simulations at all. In fact, the nodes of  could represent arbitrary parallel processes
and the edges symmetric communication requirements between the processes.

Typically, running an application on computers with distributed parallelism requires the application graph to be spread over 
the computer's processing elements. One
way to carry out this task, called \emph{static mapping}, is to (i)
partition the application graph  into blocks of equal size (or of equal weight
in case the computational requirements at the nodes are not homogeneous) for
load balancing purposes and (ii) map the blocks of  onto the
processing elements (PEs) of a parallel computer, see
Figure~\ref{fig:overview}. 
Mapping may involve the communication
graph , whose vertices represent the blocks of 's
partition and whose edges indicate block neighborhood and therefore
communication between different PEs. 

The parallel computer is often represented as a graph , called \emph{processor
  graph} (or topology graph), the vertices of which represent the PEs,
and the edges of which represent physical communication links between
the PEs. We require that  has the same number of vertices as
 and make the assumption that  is sparse,
which is true for many real architectures today~\cite{top500-13-06}.
In this paper we address the problem of finding a bijective mapping
 of 's vertex set onto 's vertex set
(processors) that is communication-efficient. We refer to  as
\emph{bijective topology mapping} or simply \emph{mapping}. One can also
see the problem as embedding the guest graph  into the host graph .


\begin{figure}[tb]
\centering{}(a) \includegraphics[width=2.3cm]{figures/G_a} (b)
\includegraphics[width=2.3cm]{figures/G_c} (c)
\includegraphics[width=2.3cm]{figures/G_p}\caption{\label{fig:overview}(a)
  Application graph  with -way partition indicated by
  colors. (b) Com\-mu\-ni\-ca\-tion graph  induced by 
  and the partition.  expresses the neighborhood relations of
  's blocks. Edge weights (shown through width) indicate
  communication volumes between blocks. (c) Processor graph .
  Nodes and edges represent the PEs and the communication links,
  respectively. Com\-mu\-ni\-ca\-tion between the green and the red
  block in , \ie via , requires two hops in .  }
\vspace{-3ex}
\end{figure}



\noindent \emph{Motivation.} Communication costs are crucial for the
scalability of many parallel applications. Static mapping, in turn, is
crucial when it comes to keeping communication costs under control
through (i) providing a partitioning with few edges between blocks and
(ii) mapping nearby blocks onto nearby PEs: due to the sparse nature
of many large-scale parallel computers,
communication costs may vary by several orders of magnitude depending
on the distance between the PEs
involved~\cite{Teresco2000269}. Also, numerous recent applications involve
massive \emph{complex networks} such as social networks or web graphs~\cite{costa2011analyzing}. These
networks usually lead to denser communication graphs and make improved
mapping strategies even more desirable.

\noindent \emph{Contribution.}
We investigate numerous algorithms for static mapping,
the scenario being that an application graph is first partitioned into
blocks, followed by a bijective mapping of the blocks onto the nodes
of a processor graph. The graph partitioners we employ are the
state-of-the-art packages \metis~\cite{Karypis13a} and
\kahip~\cite{Sanders2013a}. While \metis is widely used for graph
partitioning and has been employed for mapping before, it is the first
time that the high-quality partitioner \kahip is used in the mapping
context.

To assess and improve the performance of mapping algorithms, we
implement several state-of-the-art methods. Moreover and more
importantly, we develop and implement two new algorithms as
straightforward, yet very effective adaptations of existing greedy
algorithms.

The three most striking results of our extensive mapping experiments
on meshes and complex networks as application graphs, as well as grids
and tori as processor graphs, are: First, the strengths
and weaknesses of the mapping algorithms are, to a large extent,
independent of the class of application graphs (mesh or complex
network) and the processor graphs. Second, the graph partitioner and
its partitioning quality is of minor importance for the quality
of the mapping. Third, for complex networks as application graphs,
one of our new mapping algorithms always yields the best quality in terms 
maximum congestion. In case of meshes, this mapping algorithm always 
leads in terms of maximum congestion \emph{and} maximum dilation.

\section{Preliminaries}
\label{sec:prelim}
\subsection{Problem Description}
\label{sub:problem}

We represent the communication of a parallel appli\-ca\-tion as a
graph , where a weight , , indicates the volume of communication
between  and , \ie between the corresponding blocks of the
application graph.

The parallel computer takes the form of a graph , the \emph{processor graph}. Here,  indicates the bandwidths of the physical communication links. We
require .

Our aim is to find a bijective topology mapping (short \emph{mapping})
 that minimizes the overhead due to
communication between the processes. A first graph-theoretic
definition of the overhead (costs) was given
in~\cite{Rosenberg1980a}. In the following we present three aspects of
overhead (for more in-depth definitions see~\cite{hoefler-topomap}).

An edge  of  gives rise to communication
between  and  on . Sending a unit of
information along a path  in  with edges 
takes time at least . Sending
all information via an edge , \ie from
processes in  to processes in , then takes time at least


Thus, \emph{maximum} and \emph{average dilation}, defined as
 
respectively, provide lower bounds for the communication time of a
parallel application,  being the tighter lower bound.

When multiple messages are exchanged at the same time, more than one
of them may be routed via the same edge. Hence, if  denotes
the total volume of communication routed via , divided by
the bandwidth , then the maximum (weighted) congestion
 
provides another lower bound for the time. Minimizing ,
 and  is NP-hard, cf.\ Garey and
Johnson~\cite{Garey:1979:CIG:578533} and more recent
work~\cite{hoefler-topomap,ManKim1991246}. Due to the problem's
complexity, exact mapping methods are only practical in special
cases. Leighton's book~\cite{Leighton92introduction} discusses
embeddings between arrays, trees, and hypercubes.

As in previous studies~\cite{hoefler-topomap}, we assume that the
routing algorithm sends the messages on uniformly distributed shortest
paths in . In particular, the routing algorithm is oblivious to
the utilization of the parallel system. 



\subsection{Graph partitioning}
\label{sub:part_quality}

Given a graph  and a number of blocks , the 
graph partitioning problem asks for a division of  into  pairwise
disjoint subsets  (\emph{blocks}) such that no block
is larger than

where  is the allowed imbalance. The most widely
used objective function 
is the \emph{edge cut} (whose minimization is
-hard~\cite{Garey:1979:CIG:578533}), \ie, the total weight of the edges between
different blocks.  Yet, a more important factor for modeling the communication cost of parallel 
iterative graph algorithms seems to be the \emph{maximum communication volume} (MCV)~\cite{Hendrickson_graphpartitioning}, 
which has received growing attention recently, \eg in the 10th DIMACS Implementation Challenge
on graph partitioning. 
MCV considers the worst
communication volume taken over all blocks  ()
and thus penalizes imbalanced communication:



\section{Related Work}
\label{sub:related}
In this section we give a brief overview of algorithms for static
mapping. 
More on topology mapping can be found
in~\cite{Aubanel09resource,6495451} and particularly in Pellegrini's
survey~\cite{Pellegrini11static}.

It should be mentioned that partitioning and mapping can be done
simultaneously, \ie communication between PEs is taken into account
already during
partitioning~\cite{DBLP:journals/fgcs/WalshawC01,HuangAB06pagrid,MoulitsasK08architecture}.
In this paper, however, we focus on the complementary approach where
partitioning and topology mapping form different stages of a software
pipeline.

One can apply a wide range of optimization techniques to the topology
mapping problem. Hoefler and Snir~\cite{hoefler-topomap} employ (among
others) the Reverse Cuthill-McKee (RCM) algorithm, originally devised
for minimizing the bandwidth of a sparse matrix~\cite{Cuthill69a}. If
both  and  are sparse, the simultaneous optimization of both
graph layouts can lead to good mapping
results~\cite{Pellegrini07scotch}.

A common approach to static mapping, \ie, partitioning and topology
mapping combined, is to recursively partition  \emph{and} 
in the same fashion, \ie such that the number of blocks and sub-blocks
per block is equal on each level~\cite{Pellegrini94static}. Such a
hierarchical approach to mapping may take into account the actual
hierarchy of a heterogeneous multi-core
cluster~\cite{chan2012impact}. Typically, the number of sub-blocks per
block is small. Thus, on the scope of an individual block, an optimal
mapping of a block's sub-blocks can be found by evaluating all
possibilities. If the number of sub-blocks is two, the method is
called \emph{dual recursive bisection}. It has been shown effective in
the software \scotch\cite{Pellegrini07scotch}. While an optimal
mapping of a block's sub-blocks on the scope of an individual block is
not an issue in dual recursive bisection, neighboring relations
between sub-blocks of different blocks still pose a challenge.
In this paper we apply dual recursive bisection to the pair  instead of . This (basic) form of dual recursive
bisection does not take into account neighboring relations between the
sub-blocks of different blocks (as in~\cite{hoefler-topomap}).

Greedy approaches such as the ones by Hoefler and Snir~\cite{hoefler-topomap}
and Brandfass~\etal~\cite{Brandfass2013372} build on the idea of
increasing a mapping by successively adding new maps  such that (i)  has maximal communication volume with one or
all of the already mapped vertices of  and (ii)  has minimal
distance to one or all of the already mapped vertices of . For
more details see Sections~\ref{subsec:greedy}.

Hoefler and Snir~\cite{hoefler-topomap} compare RCM, DRB and a greedy
approach experimentally on abstractions of three real architectures. While their
results do not show a clear winner, they confirm previous
studies~\cite{Pellegrini11static} in that performing mapping at all is
worthwhile.  It is important to note, however, that Hoefler and Snir
perform mapping from reordered matrices, not from partitioned graphs
as we do here.

Many metaheuristics have been used to solve the mapping problem.
U\c{c}ar \etal~\cite{Ucar200632} implement a large variety of methods
within a clustering approach, among them genetic algorithms, simulated
annealing, tabu search, and particle swarm optimization. The authors
require, however, that the processor graph is homogeneous, \ie  depends only on whether  or not. Our approach is more
general than theirs in that we allow  to take different
values for  (see Equation~\ref{eq:dt}). 

Bhatele \etal~\cite{Bhatele:2011:AHT:2063384.2063486} discuss topology-aware
mappings of different MPI communication patterns on
emerging architectures. Better mappings avoid communication hot spots
and reduce communication times significantly. Geometric information
can also be helpful for finding good mappings on regular architectures
such as tori~\cite{6063073}.


\section{Methods for Topology Mapping}
\label{sec:algo}
The simplest topology mapping is the identity, \ie when block  of
the application graph (or vertex  of the communication graph )
is mapped onto node  of the processor graph , . We refer to this mapping as . It depends on how the
graph partitioner, in our case \metis or \kahip, numbers the blocks
and on how the nodes of  are numbered. In our experiments 
is a 2D or 3D grid or torus since such topologies are used in real architectures,
\eg tori for BlueGene~\cite{BlueGene02overview}.
The nodes are ordered
lexicographically \wrt the nodes' canonical integer coordinates. We
also carry along a mapping called , where the bijection  is random. The latter is done
for comparison purposes, keeping in mind that  is usually a
very bad solution.

Four algorithms in our collection, \ie, , ,
 and  are from the literature (for \rcm and
 see Section~\ref{sec:intro}
and~\cite{Cuthill69a,hoefler-topomap}). Algorithms 
and  are described in Section~\ref{subsec:greedy} 
(also see the references therein). There we also specify the last two
algorithms,  and , which are variants of
 and  and which, to our knowledge, are new.

\subsection{Greedy Algorithms}
\label{subsec:greedy}
As a prerequisite for the algorithms described in this section we need
to compute  once for a given processor graph 
(see Equation~\ref{eq:dt}). Using Johnson's
algorithm~\cite{Johnson77a,Cormen2001a} we can do so in time
  . Since  is sparse, this amounts to . This running time is not included in the running times
for the greedy algorithms in this section, as  is
computed only once for a given processor graph.

The mapping algorithm  consists of the ``construction
method'' proposed in~\cite{Brandfass2013372}. Using our terminology,
the algorithm starts by picking a node  of  such that
 is maximal, \ie 
is a vertex whose communication with neighboring vertices is
heaviest. Then, it computes for each vertex  of  the term
. Here,  is the (minimum)
time needed to send a unit of information from  to  (see
Section~\ref{sub:problem}). A vertex  for which this sum is
minimal (a most central node in  \wrt communication time) then
becomes the vertex onto which  is mapped. The experiments of
this paper involve processor graphs which are grids and tori. On the
latter all nodes are equally central.

The remaining pairs , , are formed as
follows. First, a not yet mapped vertex  of  is found such
that  is maximal, \ie
 is a vertex that communicates most heavily with the already
mapped vertices. Then, a not yet mapped vertex  of  is
found such that  is minimal, \ie a
vertex that is most central \wrt the already mapped vertices of
. Note that the choices of  and  are independent of
each other. Our implementation of  has running time
. This running time is achieved by updating
vectors  () that, for each vertex  () which
has not been mapped yet, stores the sum of the edge weights
(distances) to the vertices in  () that have been mapped
already. We use the same two vectors in \greedyallc, see
Algorithm~\ref{algo:greedyallc}.

\label{sec:greedymin}
The mapping algorithm  stems
from~\cite{hoefler-topomap}. Its general idea is the same as that
behind . The only differences are that (i)  is
picked randomly, (ii)  () is chosen such that
 is maximal, and (iii)
 () is chosen such that  is
minimal. Again, as in , the choices of  and 
are independent of each other. Our implementation of 
(which is less generic than that in~\cite{hoefler-topomap}) has
running time .

\subsection{ and }
\label{sec:greedy+}
Neither  nor  link the choices of  and
. Both algorithms aim at (i) a high communication volume of
 with all or one of the already mapped vertices of  and
(ii) a high centrality of  \wrt all or one of the already
mapped vertices of . The actual increase of communication times
caused by the new pair  (increase \wrt the partial
mapping defined so far) is not considered.
 
We therefore propose new variants  and
. They take this increase of communication time into
account. Specifically, the choice of  depends on the choice of
 (same as in ). Let
 be a candidate for being mapped onto by . Then, (minimal)
times of communication between  and the vertices of  that
have been mapped before, \ie , amount to



Analogous to  and , we set  to some
 such that the expression in Equation~\ref{eq+} is
\emph{minimal}. Thus, our objective function for choosing , \ie
Equation~\ref{eq+}, is about actual communication times and not just
distances on . We have experimented with replacing the sum in
Equation~\ref{eq+} by the maximum and found out that this tends to
decrease the quality of the mappings. For the pseudocode of
 see Algorithm~\ref{algo:greedyallc}.

\hmey{Latex warning in pseudocode}

\begin{algorithm}[!h]
\caption{The algorithm . \newline \underline{Input}:
  Communication graph  and processor graph
   with .\newline \underline{Output}: Pairs , , such that  defined by
   is a bijective mapping with low values of
  ,  and .}

\label{algo:greedyallc}
\begin{algorithmic}[1]
\State Find  with maximal 
\State Find  with minimal 
\State Create vectors  and  of length 
\State Initialize entries of  to zero and entries of   to one
\For{}
\State  /* Mark  as \emph{assigned} */
\State   /* Mark  as \emph{assigned} */
\ForAll{}
\If{} /*  is not yet assigned */
\State 
\EndIf
\EndFor
\State Pick  such that  is maximal
\For{}
\If{}
\State /*  is not yet assigned */
\State 
\ForAll{}
\If{}
\State /*  has already been assigned, \ie  is defined */
\State 
\EndIf
\EndFor
\EndIf
\EndFor
\State Pick  such that  is maximal
\EndFor
\end{algorithmic}
\end{algorithm}

\begin{proposition}
 The running time of  is .
\label{prop:time}
\end{proposition}

\begin{proof}
The outermost loop from line 5 to line 27 and the inner loop from line
8 to line 12 take amortized time . So does the
outer loop from line 14 to line 25 and the inner loop from
line 18 to line 23. Since the latter two loops are contained in the
outermost loop from line 5 to line 27, the running time of
Algorithm~\ref{algo:greedyallc} is indeed . Even a trivial implementation of lines 13 and 26 (with
running time ) does not change the result.
\end{proof}

The running time for  is the same as for 
because the two algorithms differ only at lines 1 to 4, and the
running times of both algorithms are not determined by this part.


\section{Experiments}
\label{sec:exp}
In this section we specify our test instances, our experimental setup
and the way we evaluate the mapping algorithms.

\paragraph{Test Instances.}
\label{sub:exp-instances}
The application graphs fall into two classes: The class \walshawlarge consists of the eight largest graphs in
Walshaw's graph partitioning archive~\cite{SoperWC04combined}, and the
class \complexnets consists of 12 complex networks (see
Tables~\ref{tab:walshaw} and~\ref{tab:complex}). The latter form a
subset of the 15 complex networks used in~\cite{Safro2012a}
for partitioning experiments. It turned out, however, that 
[ with -way partitioning, respectively], while respecting
the allowed imbalance, occasionally generated empty blocks for the
complex network \emph{p2p-Gnutella} [\emph{as-22july06} and
  \emph{loc-gowalla\_edges}]. Using  with \emph{recursive
  bisection} instead of k-way partitioning was not an option because
 then quite often violated the balance constraint and
produced blocks heavier than  times the average block
size (only on complex networks).
For each of the classes \walshawlarge and
\complexnets the benchmarking comprises the following processor graphs.
\begin{itemize}
    \item {\small 2DGrid(), 2DGrid(), 3DGrid()}
    \item {\small 2DTorus(), 2DTorus(), 3DTorus()}
\end{itemize}


\begin{table}[]
\caption{Meshes used for benchmarking}
\begin{center}
\begin{tabular}{ l | r | r }
    Name & \#vertices & \#edges\\ \hline \hline
fe\_tooth  & \numprint{78136}   & \numprint{452591} \\\hline
fe\_rotor  & \numprint{99617}   & \numprint{662431} \\\hline
598a       & \numprint{110971}   & \numprint{741934} \\\hline
fe\_ocean  & \numprint{143437}  & \numprint{409593} \\\hline
144        & \numprint{144649}   & \numprint{1074391} \\\hline
wave       & \numprint{156317}   & \numprint{1059331} \\\hline
m14b       & \numprint{214765}   & \numprint{1679018} \\\hline
auto       & \numprint{448695}   & \numprint{3314611} \\\hline
  \end{tabular}
\end{center}
\label{tab:walshaw}
\end{table}

\begin{table*}[]
\caption{Complex networks used for benchmarking.}
\begin{center}
\scalebox{0.8}{
  \begin{tabular}{ l | r | r | c }
    Name & \#vertices & \#edges & Type\\ \hline \hline
PGPgiantcompo         & \numprint{10680}  & \numprint{24316}    & largest connected component in network of PGP users\\\hline
email-EuAll           & \numprint{16805}  & \numprint{60260}    & network of connections via email\\\hline
soc-Slashdot0902      & \numprint{28550}  & \numprint{379445}   & news network\\\hline
loc-brightkite\_edges & \numprint{56739}  & \numprint{212945}   & location-based friendship network\\\hline
coAuthorsCiteseer     & \numprint{227320} & \numprint{814134}   & citation network\\\hline
wiki-Talk             & \numprint{232314} & \numprint{1458806}  & network of user interactions through edits\\\hline
citationCiteseer      & \numprint{268495} & \numprint{1156647}  & citation network\\\hline
coAuthorsDBLP         & \numprint{299067} & \numprint{977676}   & citation network\\\hline
web-Google            & \numprint{356648} & \numprint{2093324}  & hyperlink network of web pages\\\hline
coPapersCiteseer      & \numprint{434102} & \numprint{16036720} & citation network\\\hline
coPapersDBLP          & \numprint{540486} & \numprint{15245729} & citation network\\\hline
as-skitter            & \numprint{554930} & \numprint{5797663}  & network of internet service providers\\\hline
  \end{tabular}}
\end{center}
\label{tab:complex}
\end{table*}

\paragraph{Experimental Setup.}
\label{sub:exp-setup}
All computations are sequential and done on a workstation with two
4-core Intel(R) Core(TM) i7-2600K processors at 3.40GHz. Our code is written
in C++ and compiled with GCC 4.7.1. 

\paragraph{Evaluation.}
\label{sub:exp-evaluation}
The benchmarking of the mapping algorithms described in
Section~\ref{sec:algo} is done separately on the classes \walshawlarge
and \complexnets. First, graphs from both classes are partitioned into
256, 512 and 1024 parts using the graph partitioner \kahip v. 0.62
({\tt http://algo2.iti.kit.edu/documents/kahip/})
\cite{dissSchulz}.
In particular,
the meshes and social networks are partitioned with the configuration
\emph{eco} and \emph{ecosocial}, respectively. The allowed imbalance
is always , \ie . To recursively bipartition  and  during
, we also use \kahip (configurations \emph{fast} and
\emph{ecofast}, perfect balance).

Since the partitioning process depends on random choices, we run \kahip with 20
different seeds.  For each seed we construct a communication graph
 from the partition, map  onto all processor graphs with the
same number of vertices and then compute the minimum, the arithmetic
mean and the maximum of the mapping's runtime ,  (see
Equation~\ref{eq:maxCon}),  and  (see
Equation~\ref{eq:dil}). Thus we arrive at the values ,
, , , etc. (twelve values for each
combination of , , and a mapping algorithm).

Next we form the geometric means of the twelve values over all graphs
in \walshawlarge and \complexnets, respectively. Thus we arrive at
twelve values ,  for any combination of a graph
class (\walshawlarge or \complexnets), a processor graph, and a
mapping algorithm. Finally, the last 9 values (all except runtimes) are
set into proportion to the corresponding values for . This
yields the values , ,
, , ,
, ,  and
. A -value smaller than one means that the
quality is higher than that of  because we are minimizing.

We also investigate the influence of graph partitioning on the quality
of the mapping algorithms. In addition to using \kahip as described
above, we apply two variants of \metis v. 5.1.0~\cite{Karypis13a}

\begin{enumerate}
\item We run  with the option of -way partitioning, an
  allowed imbalance of  and  seeds (imbalance and 
  seed number are as for \kahip).
\item We run \ndmetis with  seeds. This results in a fill-reducing
  ordering of 's adjacency matrix. The ordering is then turned
  into a partitioning of  by going through the vertices in the
  new order and assigning block numbers such that all blocks have
  almost equal size (maximal deviation is one vertex). We are aware
  that using  in this way is not a good choice in view of
  partitioning quality ( is made for other purposes). We
  proceed like this, however, \emph{because} we wish to test our
  collection of mapping algorithms on partitions with mediocre edge cut
  and MCV.
\end{enumerate}

We indicate the \metis-based graph partitioning that is underlying a
mapping algorithm by using the subscripts  and
 when employing  and ,
respectively. As an example,  means that we applied
 to partitions obtained via .

Finally, for each 
(this set has 12 values), we form quotients  of the form



As an example,  for 
in Table~\ref{tab:app:meshesComp:2DGrid1} means that 
is worse by a factor of  if  is used instead of
 for mapping meshes onto 2DGrid().

\section{Results}
\label{sec:results}
\subsection{Mapping of Meshes onto Grids and Tori}
\label{sub:exp-grids-tori-meshes}

\begin{table}[htb]
  \caption{Performance of  and  on meshes compared
    to . Values smaller than one indicate that
     is faster or that the quality the
    -partitions is higher.}
\begin{center}
\begin{tabular}{ l | c c c | c c c | c c c}
           &  &  &
     &  &  &
     &  &  &
     \\\hline \hline

     & 0.0462 & 0.0451 & 0.0440 & 1.0101 & 1.0101 & 1.0121 & 0.9970 & 1.0449 & 1.1601\\
     & 0.1026 & 0.0999 & 0.0985 & 2.2075 & 2.2371 & 2.2472 & 6.0976 & 5.9880 & 5.7471  
\end{tabular}
\end{center}
\label{tab:meansQuot}
\end{table}

Table~\ref{tab:meansQuot} shows a comparison of  partitions
with partitions from  and . We measure running
time, edge cut and MCV. As above, we record the best, mean and worst
result over  seeds and calculate the geometric means of these
numbers over all meshes in our collection --- giving rise to the
numbers  in
Table~\ref{tab:meansQuot}. In terms of partition quality, 
performs significantly poorer than  only in terms of
 and . Here  is worse by
 and , respectively.
The partitions that we derived from  (in a deliberately
sub-optimal way) fall back drastically both in terms of the edge cut
and MCV. In particular,  and
, which means that the edge cut from
 is more than double and that MCV increases almost six times
if  is used instead of .

Table~\ref{tab:app:meshes:2DTorus1} shows the quality of the mapping
algorithms for (i) partitions based on  and (ii) mapping onto
the  torus. Tables~I - V in the Appendix support the
following results.

\begin{table}[htb]
\caption{Mapping of meshes onto 2DTorus(). Times
  ,  and  are in {\bf milliseconds}.}
\begin{center}
\scalebox{0.9}{
  \begin{tabular}{l || c c c | c  c  c | c  c  c  | c  c  c }
    Algo &  &  &  &  &
     &  &  &
     &  &  &
     & \\ \hline
         & 0.028 & 0.033 & 0.044 & 2.087 & 2.059 & 2.030 & 1.389 & 1.397 & 1.432 & 1.667 & 1.471 & 1.249\\
            & 0.060 & 0.070 & 0.088 & 1.634 & 1.640 & 1.645 & 1.357 & 1.454 & 1.586 & 1.509 & 1.389 & 1.242\\
         & 50.75 & 52.18 & 54.20 & 0.862 & 0.904 & 0.966 & 0.821 & 0.886 & 0.987 & 1.039 & 1.010 & 0.962\\
      & 0.948 & 0.971 & 0.997 & 1.310 & 1.316 & 1.324 & 1.252 & 1.291 & 1.369 & 1.359 & 1.263 & 1.164\\
      & 0.163 & 0.168 & 0.183 & 1.139 & 1.162 & 1.199 & 1.025 & 1.080 & 1.149 & 0.870 & 0.776 & {\bf 0.654}\\
     & 0.918 & 0.953 & 0.987 & {\bf 0.683} & {\bf 0.707} & {\bf 0.736} & {\bf 0.665} & {\bf 0.706} & {\bf 0.766} & {\bf 0.730} & 0.780 & 0.871\\
     & 0.869 & 0.939 & 0.100 & 0.793 & 0.813 & 0.844 & 0.739 & 0.789 & 0.849 & 0.745 & {\bf 0.756} & 0.780\\ \hline
  \end{tabular}
}
\end{center}
\label{tab:app:meshes:2DTorus1}
\end{table}

\begin{enumerate}
\item The mapping algorithms ,  and  are worse than
   on all accounts. While this was expected for , our data show
  that very simple mapping strategies are not worthwhile if the underlying
  partition is good.

\item The algorithm  beats  only in terms of
  average dilation. The improvement is, however, a major one in some
  cases, \eg  and 
  for the  2D torus (see
  Table~\ref{tab:app:meshes:2DTorus1}). Another strong point of
   is its low running time.


\item On all six processor graphs our new mapping algorithm
   yields the best maximum congestion, , and the best
  maximum dilation, . This holds not only for the (geometric mean
  over all meshes of the) average over all seeds, but also if the best
  or the worst result is taken over all seeds. The quotients are
  between  and . In terms of running time, we are
  in-between that of  and .

\item  yields many major improvements over  and,
  discarding average dilation, is worse only once (in terms of
   on the 3D torus, see Table~V in the
  Appendix).  often comes close to  and
  sometimes beats it on average dilation.

\item  has its strengths on tori and often beats
   on average dilation (on grids \emph{and}
  tori). Interestingly, the overall quality of  is much
  worse than that of  (both from previous work),
  while this trend is reversed if we
  look at the modified versions  and .
\end{enumerate}

We now look at the influence of the partitioning quality on the
quality of the mapping algorithms (see
Table~\ref{tab:app:meshesComp:2DGrid1} (Table~VI in the Appendix
provides more evidence). As for  vs. , the small
lead of  over  \wrt MCV translates into an even
smaller lead of the corresponding mappings. Moreover, this small lead
is only on average, and there are cases where  partitions
lead to better mapping results.  As for  vs. , poor
edge cut andor MCV seem to have a deteriorating effect on
mapping quality.

\begin{table}[htb]
\caption{Mapping of meshes onto 2DGrid().}
\begin{center}
\scalebox{0.85}{
  \begin{tabular}{l || c c c | c  c  c | c  c  c  | c  c  c }
    Algo &  &  &  &  &
     &  &  &
     &  &  &
     & \\ \hline
         & 1.0205 & 1.0152 & 1.1762 & 0.9930 & 0.9963 & 0.9881 & 1.0010 & 1.0002 & 0.9985 & 1.0042 & 1.0071 & 0.9690\\
        & 1.0346 & 1.0506 & 1.1932 & 1.9068 & 1.9594 & 2.0013 & 2.7783 & 2.8391 & 2.8141 & 3.9285 & 4.2721 & 4.2889\\
          & 0.9944 & 0.9749 & 0.9809 & 0.9961 & 0.9983 & 1.0027 & 0.9968 & 0.9847 & 0.9431 & 1.0251 & 1.0307 & 1.0524\\
         & 1.0120 & 0.9887 & 0.9510 & 1.9659 & 2.0315 & 2.0960 & 2.5159 & 2.5489 & 2.5157 & 3.8256 & 3.9547 & 4.3704\\
             & 1.0287 & 1.0087 & 1.0103 & 0.9980 & 1.0020 & 0.9955 & 1.0033 & 1.0169 & 1.0310 & 1.0473 & 1.0225 & 1.0019\\
            & 1.1701 & 1.1362 & 1.1554 & 2.0780 & 2.1376 & 2.1856 & 2.8376 & 2.8998 & 3.0228 & 3.6763 & 3.8054 & 4.2111\\
          & 1.0083 & 1.0034 & 0.9956 & 0.9991 & 1.0096 & 1.0258 & 0.9951 & 1.0112 & 1.0131 & 1.0179 & 1.0441 & 0.9811\\
         & 0.9980 & 1.0119 & 1.0194 & 1.9035 & 2.0223 & 1.9517 & 2.6668 & 2.8051 & 2.7540 & 2.5027 & 2.9897 & 3.1595\\
       & 1.0039 & 1.0021 & 1.0182 & 1.0110 & 0.9953 & 0.9948 & 1.0295 & 0.9959 & 0.9923 & 1.0031 & 1.0088 & 1.0034\\
      & 1.0095 & 1.0177 & 1.1114 & 1.5176 & 1.4785 & 1.4779 & 2.5373 & 2.5238 & 2.5750 & 1.5832 & 1.5819 & 1.6219\\
      & 1.0091 & 1.0074 & 1.0236 & 0.9999 & 1.0135 & 1.0144 & 1.0491 & 1.0200 & 1.0085 & 1.0153 & 0.9919 & 0.9771\\
     & 1.1313 & 1.1499 & 1.1819 & 1.9840 & 1.9265 & 1.8896 & 2.9915 & 2.9117 & 2.9811 & 2.1990 & 2.6233 & 3.2098\\
      & 1.0002 & 1.0073 & 1.0205 & 1.0121 & 0.9887 & 0.9809 & 1.0166 & 0.9987 & 0.9979 & 1.0018 & 1.0013 & 0.9449\\
     & 1.1869 & 1.1879 & 1.1993 & 1.6175 & 1.5741 & 1.5145 & 2.5819 & 2.7487 & 2.8079 & 1.6267 & 2.2717 & 3.5661\\ \hline
  \end{tabular}
}
\end{center}
\label{tab:app:meshesComp:2DGrid1}
\end{table}

\subsection{Mapping of Complex Networks onto Grids and Tori.}
\label{sub:exp-complex-grids-tori}

Table~\ref{tab:meansQuotComplex} shows a comparison of
 partitions with partitions from  and
. For a description of the table see the explanation of
Table~\ref{tab:meansQuot} in Section~\ref{sub:exp-grids-tori-meshes}.

\begin{table}[htb]
  \caption{Performance of  and  on complex
    networks compared to . Values smaller than one indicate
    that  is faster or that the quality the
    -partitions is higher.}
\begin{center}
\begin{tabular}{ l | c c c | c c c | c c c}
           &  &  &
     &  &  &
     &  &  &
     \\\hline \hline
     & 0.0083 & 0.0081 & 0.0078 & 1.0634 & 1.0619 & 1.0560 & 1.2531 & 1.2066 & 1.1536\\
     & 0.0262 & 0.0268 & 0.0257 & 2.0284 & 2.0202 & 2.0121 & 1.8416 & 1.9157 & 2.0040
\end{tabular}
\end{center}
\label{tab:meansQuotComplex}
\end{table}

Compared to the picture we saw on meshes,  now also leads in
terms of the edge cut. Moreover, the lead of  in terms of MCV
compared to  and  is even more pronounced (about
).

Regarding topology mapping based on  partitions, we only
comment on results that deviate from those that we have described for
meshes (especially running times show the same trends). The main
differences are in the maximum and average dilation. Sometimes 
and even  yield even lower maximum dilation than
. Moreover, average dilation behaves quite erratically,
as is revealed by a comparison between the -values of
 in Table~\ref{tab:app:social:2DTorus1} and
Tables~VII through~XI in the Appendix.

\begin{table}[htb]
\caption{Mapping of complex networks onto 2DTorus(). Times
  ,  and  are in {\bf milliseconds}.}
\begin{center}
\scalebox{0.9}{
  \begin{tabular}{l || c c c | c  c  c | c  c  c  | c  c  c }
    Algo &  &  &  &  &
     &  &  &
     &  &  &
     & \\ \hline
         & 0.028 & 0.032 & 0.042 & 1.511 & 1.509 & 1.513 & {\bf 0.777} & {\bf 0.780} & 0.810 & 3.291 & 3.241 & 2.840\\
            & 0.104 & 0.122 & 0.161 & 1.366 & 1.416 & 1.455 & 0.822 & 0.868 & 0.931 & 2.672 & 2.919 & 2.699\\
         & 124.8 & 138.5 & 154.0 & 0.982 & 1.003 & 1.021 & 0.853 & 0.876 & 0.926 & 1.084 & 1.250 & 1.476\\
      & 5.182 & 5.344 & 5.684 & 1.100 & 1.119 & 1.131 & 1.068 & 1.011 & 0.985 & 1.189 & 1.315 & 1.301\\
      & 0.248 & 0.258 & 0.292 & 1.057 & 1.054 & 1.056 & 1.109 & 1.056 & 1.015 & 0.858 & 0.676 & 0.496\\
     & 5.647 & 5.871 & 6.268 & {\bf 0.841} & {\bf 0.839} & {\bf 0.839} & 0.863 & 0.820 & {\bf 0.801} & {\bf 0.531} & {\bf 0.442} & {\bf 0.351}\\
     & 5.251 & 5.550 & 6.153 & 0.858 & 0.856 & 0.855 & 0.857 & 0.829 & 0.808 & 0.644 & 0.575 & 0.481\\ \hline
  \end{tabular}
}
\end{center}
\label{tab:app:social:2DTorus1}
\end{table}

Regarding maximum congestion, , the picture is the same as we saw
for meshes: Our new algorithm  always yields the best results.

Regarding the influence of partitioning quality on the quality of the
mappings we see that the higher partitioning quality of 
compared to  (in terms of the edge cut and MCV) does not
translate into considerably better mappings, see
Table~\ref{tab:app:socialComp:2DGrid1} (for additional evidence see
Table~XII in the Appendix). As in the case of meshes, the
partitions that we derived from  (in a deliberately
sub-optimal way) lead to poor mappings.

\begin{table}[htb]
\caption{Mapping of complex networks onto 2DGrid().}
\begin{center}
\scalebox{0.85}{
  \begin{tabular}{l || c c c | c  c  c | c  c  c  | c  c  c }
    Algo &  &  &  &  &
     &  &  &
     &  &  &
     & \\ \hline
         & 0.9723 & 0.9838 & 0.9812 & 1.0243 & 1.0407 & 1.0623 & 0.9914 & 0.9955 & 1.0209 & 1.2791 & 1.4033 & 1.4395\\
        & 0.9684 & 0.9963 & 0.9832 & 10.004 & 10.146 & 10.195 & 2.9752 & 2.8494 & 2.8637 & 3.9055 & 3.7684 & 3.5497\\
          & 1.0292 & 1.0234 & 1.0285 & 0.9832 & 0.9910 & 0.9983 & 1.0866 & 1.1270 & 1.1680 & 1.0194 & 1.0167 & 1.1733\\
         & 1.0127 & 1.0272 & 1.0027 & 7.5795 & 7.7739 & 7.9801 & 2.7521 & 2.8475 & 2.9019 & 1.8046 & 1.5884 & 1.6585\\
             & 1.0970 & 1.0851 & 0.9894 & 1.0087 & 0.9939 & 0.9991 & 1.1129 & 1.1367 & 1.1333 & 0.9571 & 0.9581 & 1.0146\\
            & 0.9611 & 0.9963 & 0.9008 & 7.5698 & 7.8690 & 8.2057 & 3.2024 & 3.2317 & 2.9598 & 1.6997 & 1.6083 & 1.4822\\
          & 1.0486 & 1.0492 & 1.0620 & 1.0076 & 1.0147 & 1.0027 & 1.0624 & 1.0745 & 1.0913 & 0.9775 & 1.0832 & 1.0793\\
         & 0.4242 & 0.4094 & 0.3885 & 7.6865 & 8.1313 & 8.5734 & 3.4637 & 3.7612 & 3.9759 & 2.0400 & 2.1615 & 1.7987\\
       & 1.0375 & 1.0341 & 1.0102 & 1.0284 & 1.0267 & 1.0278 & 1.0060 & 1.0440 & 1.0791 & 1.0846 & 1.1068 & 1.0990\\
      & 0.6850 & 0.6860 & 0.7321 & 9.0648 & 9.1580 & 9.4715 & 4.4637 & 4.3656 & 4.4046 & 2.5795 & 2.5664 & 2.5504\\
      & 1.0637 & 1.0637 & 1.0608 & 1.0450 & 1.0392 & 1.0392 & 1.0527 & 1.0582 & 1.0594 & 1.3140 & 1.0434 & 0.9078\\
     & 0.3690 & 0.3700 & 0.3559 & 7.9916 & 8.0772 & 8.1791 & 3.5868 & 3.5635 & 3.6554 & 4.0989 & 2.5944 & 1.6957\\
      & 1.0847 & 1.0731 & 1.0698 & 1.0454 & 1.0411 & 1.0347 & 1.1240 & 1.1579 & 1.2136 & 1.0790 & 1.0542 & 0.9809\\
     & 0.3688 & 0.3721 & 0.3769 & 9.2443 & 9.3823 & 9.8506 & 5.2981 & 5.1511 & 5.1905 & 2.4001 & 1.7601 & 1.3172\\ \hline
  \end{tabular}
}
\end{center}
\label{tab:app:socialComp:2DGrid1}
\end{table}

\FloatBarrier


\section{Conclusions and Future Work}
\label{sec:conclusions}
We performed extensive static mapping experiments, our scenario being
a consecutive pipeline of graph partitioning and bijective topology mapping.
These experiments involved  two classes of application
graphs (8 meshes, 12 complex networks), three ways to partition the
application graphs (one by , two by \metis), six
processor graphs (3 grids, 3 tori) and 8 mapping algorithms.

Our results indicate that the strengths and weaknesses of the mapping algorithms are, to
a large extent, independent of the class of application graphs (mesh
or complex network) and the processor graphs. The main differences are
in the maximum and average dilation. Especially the latter behaves
erratically in the case of complex networks.

Second, the quality of the partitions, both in terms of edge cut and
MCV, has little influence on the quality of the mapping, except in
cases where MCV is very poor. Thus, even MCV is not a good indicator
of how well a partition can be mapped onto a processor graph --- at
least within the realm of our experiments.

Third, our variant of a greedy mapping algorithm by Brandfass \etal,
\ie , clearly dominates all state-of-the art algorithms we considered
in terms of maximum congestion. The running time of our algorithm is , where  and  is the vertex and the
edge set of the communication graph, respectively (and therefore usually fairly small).

If the weak influence of partition quality on mapping quality is
affirmed for more classes of application graphs and more parallel
architectures, improvements of static mapping
are likely to come only out of new combinations of
partitioning and mapping. In the future we will investigate how 
to minimize the communication volume specified in Equation~\ref{eq+}
by such a coupled approach.



\bibliographystyle{IEEEtran}
\bibliography{roland,refs-parco,paper2.bib}


\section*{Appendix}
\begin{table*}[!h]
\caption{Mapping of meshes onto 2DGrid(). Times
  ,  and  are in {\bf milliseconds}.}
\begin{center}
\scalebox{0.9}{
  \begin{tabular}{l || c c c | c  c  c | c  c  c  | c  c  c }
    Algo &  &  &  &  &
     &  &  &
     &  &  &
     & \\ \hline
         & 0.029 & 0.033 & 0.040 & 2.295 & 2.264 & 2.243 & 1.941 & 1.926 & 1.861 & 1.744 & 1.550 & 1.423\\
            & 0.057 & 0.068 & 0.079 & 1.564 & 1.563 & 1.572 & 1.288 & 1.297 & 1.304 & 1.670 & 1.530 & 1.393\\
         & 49.40 & 50.82 & 52.68 & 0.756 & 0.818 & 0.881 & 0.697 & 0.750 & 0.773 & 0.913 & 0.974 & 1.012\\
      & 0.951 & 0.973 & 0.992 & 1.679 & 1.680 & 1.678 & 1.359 & 1.361 & 1.292 & 1.782 & 1.720 & 1.579\\
      & 0.160 & 0.164 & 0.177 & 1.120 & 1.192 & 1.274 & 1.092 & 1.136 & 1.175 & 0.871 & 0.803 & {\bf 0.714}\\
     & 0.906 & 0.942 & 0.995 & {\bf 0.665} & {\bf 0.722} & {\bf 0.789} & {\bf 0.626} & {\bf 0.665} & {\bf 0.736} & 0.882 & 1.027 & 1.253\\
     & 0.828 & 0.875 & 0.919 & 0.817 & 0.890 & 0.957 & 0.750 & 0.785 & 0.817 & {\bf 0.787} & {\bf 0.792} & 0.901\\ \hline
  \end{tabular}
}
\end{center}
\label{tab:app:meshes:2DGrid1}
\end{table*}

\begin{table*}[!h]
\caption{Mapping of meshes onto 2DGrid(). Times
  ,  and  are in {\bf milliseconds}.}
\begin{center}
\scalebox{0.9}{
  \begin{tabular}{l || c c c | c  c  c | c  c  c  | c  c  c }
    Algo &  &  &  &  &
     &  &  &
     &  &  &
     & \\ \hline
         & 0.093 & 0.110 & 0.135 & 2.933 & 2.871 & 2.806 & 2.268 & 2.169 & 2.084 & 1.769 & 1.637 & 1.440\\
            & 0.205 & 0.239 & 0.273 & 1.776 & 1.782 & 1.781 & 1.441 & 1.433 & 1.418 & 1.784 & 1.683 & 1.481\\
         & 216.5 & 221.6 & 227.7 & 0.715 & 0.757 & 0.806 & 0.655 & 0.690 & 0.756 & 0.952 & 1.060 & 1.125\\
      & 17.55 & 18.53 & 18.97 & 2.116 & 2.084 & 2.068 & 1.609 & 1.565 & 1.522 & 1.956 & 1.838 & 1.632\\
      & 3.229 & 3.867 & 3.947 & 1.294 & 1.371 & 1.442 & 1.289 & 1.320 & 1.408 & 0.936 & {\bf 0.858} & {\bf 0.732}\\
     & 17.20 & 18.05 & 18.89 & {\bf 0.569} & {\bf 0.626} & {\bf 0.681} & {\bf 0.556} & {\bf 0.615} & {\bf 0.689} & 0.876 & 1.137 & 1.381\\
     & 15.98 & 16.71 & 17.70 & 0.887 & 0.948 & 1.040 & 0.770 & 0.837 & 0.916 & {\bf 0.869} & 0.892 & 0.966\\ \hline
  \end{tabular}
}
\end{center}
\label{tab:meshes:2DGrid2}
\end{table*}

\begin{table*}[!h]
\caption{Mapping of meshes onto 3DGrid(). Times
  ,  and  are in {\bf milliseconds}.}
\begin{center}
\scalebox{0.9}{
  \begin{tabular}{l || c c c | c  c  c | c  c  c  | c  c  c }
    Algo &  &  &  &  &
     &  &  &
     &  &  &
     & \\ \hline
         & 0.047 & 0.056 & 0.070 & 2.169 & 2.131 & 2.094 & 1.749 & 1.739 & 1.690 & 1.664 & 1.508 & 1.329\\
            & 0.113 & 0.129 & 0.148 & 1.631 & 1.622 & 1.616 & 1.289 & 1.314 & 1.305 & 1.476 & 1.372 & 1.216\\
         & 112.6 & 115.6 & 119.8 & 0.814 & 0.856 & 0.920 & 0.719 & 0.769 & 0.819 & 0.935 & 0.995 & 0.946\\
      & 4.006 & 4.050 & 4.101 & 1.723 & 1.704 & 1.688 & 1.348 & 1.326 & 1.267 & 1.670 & 1.618 & 1.470\\
      & 0.687 & 0.708 & 0.722 & 1.214 & 1.232 & 1.255 & 1.051 & 1.061 & 1.082 & 0.976 & {\bf 0.887} & {\bf 0.780}\\
     & 3.875 & 3.996 & 4.104 & {\bf 0.683} & {\bf 0.713} & {\bf 0.736} & {\bf 0.617} & {\bf 0.633} & {\bf 0.638} & {\bf 0.876} & 1.055 & 1.151\\
     & 3.683 & 3.871 & 4.067 & 0.827 & 0.859 & 0.891 & 0.695 & 0.718 & 0.757 & 1.268 & 1.339 & 1.388\\ \hline
  \end{tabular}
}
\end{center}
\label{tab:meshes:3DGrid}
\end{table*}

\begin{table*}[!h]
\caption{Mapping of meshes onto 2DTorus(). Times
  ,  and  are in {\bf milliseconds}.}
\begin{center}
\scalebox{0.9}{
  \begin{tabular}{l || c c c | c  c  c | c  c  c  | c  c  c }
    Algo &  &  &  &  &
     &  &  &
     &  &  &
     & \\ \hline
         & 0.090 & 0.111 & 0.139 & 2.656 & 2.609 & 2.566 & 1.487 & 1.4728 & 1.444 & 1.661 & 1.491 & 1.369\\
            & 0.213 & 0.249 & 0.295 & 1.942 & 1.942 & 1.932 & 1.480 & 1.5121 & 1.552 & 1.536 & 1.400 & 1.312\\
         & 212.9 & 217.0 & 221.6 & 0.794 & 0.843 & 0.896 & 0.782 & 0.8642 & 0.938 & 1.059 & 1.068 & 1.111\\
      & 17.79 & 18.07 & 18.42 & 1.529 & 1.526 & 1.533 & 1.638 & 1.7147 & 1.765 & 1.407 & 1.319 & 1.215\\
      & 4.031 & 4.068 & 4.110 & 1.303 & 1.336 & 1.360 & 1.166 & 1.2162 & 1.259 & 0.892 & {\bf 0.787} & {\bf 0.683}\\
     & 17.53 & 18.08 & 18.94 & {\bf 0.569} & {\bf 0.611} & {\bf 0.647} & {\bf 0.609} & {\bf 0.6843} & {\bf 0.752} & {\bf 0.726} & 0.815 & 0.899\\
     & 16.93 & 17.68 & 18.30 & 0.778 & 0.820 & 0.859 & 0.752 & 0.8148 & 0.886 & 0.892 & 0.973 & 1.028\\ \hline
  \end{tabular}
}
\end{center}
\label{tab:meshes:2DTorus2}
\end{table*}

\begin{table*}[!h]
\caption{Mapping of meshes onto 3DTorus(). Times
  ,  and  are in {\bf milliseconds}.}
\begin{center}
\scalebox{0.9}{
  \begin{tabular}{l || c c c | c  c  c | c  c  c  | c  c  c }
    Algo &  &  &  &  &
     &  &  &
     &  &  &
     & \\ \hline
         & 0.049 & 0.058 & 0.075 & 2.052 & 2.013 & 2.000 & 1.307 & 1.331 & 1.337 & 1.571 & 1.447 & 1.392\\
            & 0.115 & 0.131 & 0.154 & 1.638 & 1.637 & 1.641 & 1.317 & 1.395 & 1.502 & 1.485 & 1.378 & 1.360\\
         & 115.8 & 118.5 & 121.2 & 0.931 & 0.972 & 1.049 & 0.859 & 0.943 & 1.039 & 1.057 & 1.028 & 1.045\\
      & 3.848 & 3.910 & 4.028 & 1.303 & 1.230 & 1.312 & 1.161 & 1.214 & 1.221 & 1.262 & 1.196 & 1.173\\
      & 0.656 & 0.670 & 0.684 & 1.209 & 1.217 & 1.240 & 1.002 & 1.049 & 1.073 & 0.903 & 0.797 & {\bf 0.726}\\
     & 3.882 & 4.050 & 4.178 & {\bf 0.751} & {\bf 0.757} & {\bf 0.767} & {\bf 0.683} & {\bf 0.719} & {\bf 0.711} & 0.755 & 0.806 & 0.913\\
     & 3.897 & 4.125 & 4.463 & 0.840 & 0.842 & 0.858 & 0.727 & 0.761 & 0.790 & {\bf 0.739} & {\bf 0.743} & 0.783\\ \hline
  \end{tabular}
}
\end{center}
\label{tab:app:meshes:3DTorus}
\end{table*}

\begin{table*}[!h]
\caption{Mapping of meshes onto 2DTorus().}
\begin{center}
\scalebox{0.85}{
  \begin{tabular}{l || c c c | c  c  c | c  c  c  | c  c  c }
    Algo &  &  &  &  &
     &  &  &
     &  &  &
     & \\ \hline
         & 1.0361 & 0.9916 & 0.8434 & 0.9987 & 0.9964 & 0.9968 & 1.0042 & 1.0032 & 0.9917 & 0.9844 & 1.0209 & 1.0812\\
        & 0.9799 & 0.9805 & 0.8702 & 1.6731 & 1.6923 & 1.7122 & 2.5803 & 2.7067 & 2.8812 & 1.5955 & 1.5739 & 1.5854\\
          & 1.0016 & 1.0165 & 0.9776 & 1.0019 & 1.0014 & 1.0027 & 1.0179 & 1.0257 & 1.0847 & 1.0188 & 1.0256 & 1.0516\\
         & 1.0269 & 1.0453 & 1.0364 & 1.9834 & 2.0339 & 2.0892 & 2.6819 & 2.7297 & 2.9002 & 3.8130 & 4.0997 & 4.3485\\
             & 1.0212 & 1.0093 & 1.1226 & 1.0140 & 1.0005 & 1.0014 & 1.0219 & 1.0256 & 0.9761 & 1.0133 & 1.0188 & 1.0131\\
            & 1.1623 & 1.1217 & 1.1519 & 2.0952 & 2.1371 & 2.1744 & 2.7086 & 2.7540 & 2.7208 & 3.7498 & 4.0711 & 4.3358\\
          & 1.0058 & 0.9997 & 0.9919 & 1.0062 & 1.0070 & 1.0161 & 1.0041 & 1.0061 & 1.0642 & 1.0238 & 1.0044 & 1.0358\\
         & 0.9996 & 1.0072 & 1.0058 & 1.9550 & 2.0459 & 1.9538 & 2.8690 & 2.9701 & 3.1304 & 2.5652 & 3.1615 & 3.8322\\
       & 1.0065 & 1.0013 & 1.0038 & 1.0162 & 0.9976 & 1.0011 & 0.9978 & 0.9865 & 0.9606 & 1.0100 & 1.0115 & 1.0202\\
      & 1.0185 & 1.0110 & 1.0252 & 1.5790 & 1.5512 & 1.5467 & 2.4673 & 2.5399 & 2.5795 & 1.6415 & 1.6700 & 1.7920\\
      & 1.0063 & 1.0052 & 1.0014 & 0.9941 & 0.9996 & 0.9941 & 1.0085 & 0.9930 & 0.9828 & 1.0682 & 1.0244 & 0.9929\\
     & 1.1169 & 1.1379 & 1.1481 & 2.0367 & 1.9924 & 1.9810 & 2.8488 & 3.0034 & 3.0069 & 2.0127 & 2.3230 & 2.9728\\
      & 1.0059 & 1.0101 & 1.0793 & 0.9915 & 0.9928 & 0.9991 & 0.9775 & 0.9935 & 1.0491 & 1.0182 & 1.0197 & 1.0345\\
     & 1.1779 & 1.1475 & 1.1270 & 1.7459 & 1.7276 & 1.6917 & 2.4912 & 2.6535 & 2.7047 & 1.7459 & 2.4074 & 3.3671\\ \hline
  \end{tabular}
}
\end{center}
\label{tab:app:meshesComp:2DTorus1}
\end{table*}



\begin{table*}[!h]
\caption{Mapping of complex networks onto 2DGrid(). Times
  ,  and  are in {\bf milliseconds}.}
\begin{center}
\scalebox{0.9}{
  \begin{tabular}{l || c c c | c  c  c | c  c  c  | c  c  c }
    Algo &  &  &  &  &
     &  &  &
     &  &  &
     & \\ \hline
         & 0.028 & 0.032 & 0.042 & 1.593 & 1.597 & 1.597 & 0.987 & 0.925 & 0.943 & 3.737 & 3.831 & 3.361\\
            & 0.101 & 0.120 & 0.155 & 1.362 & 1.419 & 1.457 & 0.968 & 0.956 & 1.032 & 2.783 & 3.267 & 3.411\\
         & 124.8 & 138.4 & 154.8 & 0.937 & 0.957 & 0.989 & 0.762 & 0.750 & 0.778 & 1.012 & 1.154 & 1.562\\
      & 5.176 & 5.337 & 5.610 & 1.078 & 1.098 & 1.104 & 1.096 & 0.975 & 0.925 & 1.365 & 1.625 & 1.630\\
      & 0.245 & 0.256 & 0.230 & 1.043 & 1.045 & 1.038 & 1.003 & 0.929 & 0.897 & 0.774 & 0.627 & {\bf 0.453}\\
     & 5.622 & 5.837 & 6.173 & {\bf 0.799} & {\bf 0.813} & {\bf 0.827} & 0.798 & 0.738 & 0.730 & 0.847 & 1.197 & 1.461\\
     & 5.243 & 5.488 & 5.836 & 0.849 & 0.854 & 0.856 & {\bf 0.711} & {\bf 0.674} & {\bf 0.669} & {\bf 0.554} & {\bf 0.548} & 0.557\\ \hline
  \end{tabular}
}
\end{center}
\label{tab:app:social:2DGrid1}
\end{table*}

\begin{table*}[!h]
\caption{Mapping of complex networks onto 2DGrid(). Times
  ,  and  are in {\bf milliseconds}.}
\begin{center}
\scalebox{0.9}{
  \begin{tabular}{l || c c c | c  c  c | c  c  c  | c  c  c }
    Algo &  &  &  &  &
     &  &  &
     &  &  &
     & \\ \hline
         & 0.094 & 0.112 & 0.145 & 1.849 & 1.836 & 1.817 & 0.820 & 0.812 & 0.858 & 5.086 & 5.603 & 5.046\\
            & 0.412 & 0.504 & 0.641 & 1.510 & 1.575 & 1.623 & 0.824 & 0.871 & 0.951 & 3.840 & 4.773 & 4.528\\
         & 415.0 & 445.1 & 484.8 & 0.881 & 0.911 & 0.948 & {\bf 0.617} & 0.636 & 0.685 & 0.951 & 1.207 & 1.477\\
      & 72.81 & 75.82 & 79.82 & 1.152 & 1.150 & 1.143 & 0.914 & 0.875 & 0.872 & 1.992 & 2.448 & 2.514\\
      & 3.307 & 4.141 & 4.316 & 1.047 & 1.050 & 1.046 & 0.867 & 0.849 & 0.858 & {\bf 0.681} & {\bf 0.584} & {\bf 0.437}\\
     & 96.07 & 99.59 & 103.2 & {\bf 0.720} & {\bf 0.728} & {\bf 0.730} & 0.650 & {\bf 0.631} & {\bf 0.642} & 0.784 & 0.804 & 0.830\\
     & 85.73 & 88.87 & 92.26 & 0.802 & 0.843 & 0.873 & 0.642 & 0.668 & 0.715 & 2.900 & 3.304 & 2.860\\ \hline
  \end{tabular}
}
\end{center}
\label{tab:social:2DGrid2}
\end{table*}

\begin{table*}[!h]
\caption{Mapping of complex networks onto 3DGrid(). Times
  ,  and  are in {\bf milliseconds}.}
\begin{center}
\scalebox{0.9}{
  \begin{tabular}{l || c c c | c  c  c | c  c  c  | c  c  c }
    Algo &  &  &  &  &
     &  &  &
     &  &  &
     & \\ \hline
         & 0.049 & 0.057 & 0.074 & 1.539 & 1.527 & 1.519 & 0.878 & 0.850 & 0.884 & 3.459 & 3.872 & 3.479\\
            & 0.209 & 0.258 & 0.335 & 1.381 & 1.410 & 1.437 & 0.889 & 0.880 & 0.900 & 2.686 & 3.307 & 3.177\\
         & 254.0 & 275.0 & 230.0 & 0.935 & 0.952 & 0.967 & {\bf 0.751} & {\bf 0.757} & {\bf 0.751} & {\bf 0.911} & 1.197 & 1.504\\
      & 20,78 & 21.49 & 22.47 & 1.108 & 1.104 & 1.112 & 1.134 & 1.083 & 1.035 & 1.412 & 1.538 & 1.548\\
      & 0.882 & 0.905 & 0.927 & 1.097 & 1.100 & 1.100 & 0.930 & 0.883 & 0.879 & 0.937 & {\bf 0.856} & {\bf 0.658}\\
     & 25.42 & 26.63 & 28.10 & {\bf 0.791} & {\bf 0.793} & {\bf 0.802} & 0.849 & 0.804 & 0.794 & 0.988 & 1.008 & 0.978\\
     & 23.38 & 24.43 & 25.65 & 0.857 & 0.886 & 0.907 & 0.811 & 0.791 & 0.804 & 3.522 & 4.509 & 4.087\\ \hline
  \end{tabular}
}
\end{center}
\label{tab:app:social:3DGrid}
\end{table*}

\begin{table*}[!h]
\caption{Mapping of complex networks onto 2DTorus(). Times
  ,  and  are in {\bf milliseconds}.}
\begin{center}
\scalebox{0.9}{
  \begin{tabular}{l || c c c | c  c  c | c  c  c  | c  c  c }
    Algo &  &  &  &  &
     &  &  &
     &  &  &
     & \\ \hline
         & 0.093 & 0.110 & 0.138 & 1.748 & 1.733 & 1.720 & {\bf 0.627} & {\bf 0.644} & {\bf 0.691} & 4.906 & 5.019 & 4.675\\
            & 0.420 & 0.513 & 0.650 & 1.524 & 1.577 & 1.618 & 0.723 & 0.762 & 0.824 & 3.820 & 4.422 & 4.475\\
         & 412.5 & 442.1 & 480.1 & 0.925 & 0.946 & 0.958 & 0.697 & 0.764 & 0.826 & 0.942 & 1.318 & 1.971\\
      & 74.24 & 75.93 & 78.76 & 1.171 & 1.181 & 1.200 & 0.989 & 0.973 & 0.993 & 1.750 & 2.176 & 2.493\\
      & 4.428 & 4.475 & 4.532 & 1.115 & 1.116 & 1.107 & 0.941 & 0.958 & 0.993 & 0.785 & 0.710 & 0.557\\
     & 96.29 & 99.57 & 103.6 & {\bf 0.770} & {\bf 0.769} & {\bf 0.764} & 0.749 & 0.755 & 0.771 & {\bf 0.578} & {\bf 0.522} & {\bf 0.439}\\
     & 86.33 & 89.39 & 93.14 & 0.837 & 0.838 & 0.839 & 0.735 & 0.760 & 0.797 & 2.732 & 3.256 & 3.034\\ \hline
  \end{tabular}
}
\end{center}
\label{tab:social:2DTorus2}
\end{table*}

\begin{table*}[!h]
\caption{Mapping of complex networks onto 3DTorus(). Times
  ,  and  are in {\bf
    milliseconds}.}
\begin{center}
\scalebox{0.9}{
  \begin{tabular}{l || c c c | c  c  c | c  c  c  | c  c  c }
    Algo &  &  &  &  &
     &  &  &
     &  &  &
     & \\ \hline
         & 0.050 & 0.058 & 0.072 & 1.479 & 1.474 & 1.469 & {\bf 0.762} & {\bf 0.753} & {\bf 0.773} & 3.029 & 3.498 & 3.251\\
            & 0.211 & 259.9 & 342.2 & 1.383 & 1.404 & 1.419 & 0.813 & 0.831 & 0.910 & 2.558 & 3.255 & 3.130\\
         & 257.3 & 279.1 & 302.5 & 0.995 & 1.014 & 1.028 & 0.893 & 0.925 & 0.994 & 1.003 & 1.418 & 1.804\\
      & 20.72 & 21.36 & 22.59 & 1.114 & 1.118 & 1.127 & 1.187 & 1.139 & 1.089 & 1.184 & 1.380 & 1.477\\
      & 0.841 & 0.866 & 0.886 & 1.100 & 1.099 & 1.095 & 0.956 & 0.945 & 0.945 & 0.893 & 0.816 & 0.634\\
     & 25.48 & 26.71 & 28.15 & {\bf 0.852} & {\bf 0.847} & {\bf 0.839} & 0.974 & 0.958 & 0.943 & {\bf 0.660} & {\bf 0.620} & {\bf 0.519}\\
     & 23.57 & 24.56 & 25.74 & 0.875 & 0.873 & 0.870 & 0.971 & 0.968 & 0.988 & 1.215 & 1.618 & 1.582\\ \hline
  \end{tabular}
}
\end{center}
\label{tab:app:social:3DTorus}
\end{table*}

\begin{table*}[!h]
\caption{Mapping of complex networks onto 2DTorus().}
\begin{center}
\scalebox{0.85}{
  \begin{tabular}{l || c c c | c  c  c | c  c  c  | c  c  c }
    Algo &  &  &  &  &
     &  &  &
     &  &  &
     & \\ \hline
         & 0.9714 & 1.0045 & 1.0365 & 1.0369 & 1.0357 & 1.0463 & 0.9555 & 0.9859 & 1.0147 & 1.2189 & 1.2921 & 1.3684\\
        & 1.0000 & 0.9888 & 1.0098 & 8.7294 & 8.9164 & 9.0664 & 3.2390 & 3.0742 & 2.9958 & 3.2217 & 2.9771 & 2.5263\\
          & 1.0197 & 1.0232 & 1.0457 & 0.9888 & 0.9887 & 0.9873 & 1.0984 & 1.1191 & 1.1202 & 0.9969 & 1.0419 & 1.1427\\
         & 1.0153 & 1.0125 & 1.1086 & 7.6895 & 7.7816 & 7.8792 & 2.8851 & 2.9030 & 2.8530 & 1.8989 & 1.7575 & 1.5831\\
             & 1.1057 & 1.0787 & 0.9371 & 0.9843 & 0.9860 & 0.9910 & 1.1042 & 1.1155 & 1.1473 & 0.9775 & 1.0073 & 1.1084\\
            & 0.9755 & 0.9936 & 0.8855 & 7.2693 & 7.3256 & 7.5483 & 3.1131 & 3.3055 & 3.2221 & 1.5371 & 1.5224 & 1.6418\\
          & 1.0430 & 1.0454 & 1.0633 & 1.0021 & 1.0092 & 1.0118 & 1.0575 & 1.0732 & 1.1035 & 0.9560 & 1.0777 & 1.1059\\
         & 0.4275 & 0.4097 & 0.3889 & 8.0650 & 8.1325 & 8.4128 & 3.2878 & 3.5476 & 3.5615 & 2.1035 & 2.0679 & 1.6679\\
       & 1.0334 & 1.0333 & 1.0187 & 1.0192 & 1.0181 & 1.0180 & 1.0293 & 1.0652 & 1.1093 & 1.1002 & 1.1122 & 1.1105\\
      & 0.6853 & 0.6817 & 0.6872 & 8.3516 & 8.4982 & 8.7077 & 3.4357 & 3.4170 & 3.3840 & 2.5083 & 2.4816 & 2.3758\\
      & 1.0608 & 1.0673 & 1.0932 & 1.0340 & 1.0304 & 1.0287 & 0.9963 & 1.0285 & 1.0338 & 1.1039 & 1.1171 & 1.1198\\
     & 0.3670 & 0.3733 & 0.3912 & 8.1861 & 8.3093 & 8.4513 & 4.1153 & 4.1375 & 4.1129 & 2.4514 & 2.3375 & 2.3238\\
      & 1.0827 & 1.0731 & 0.9870 & 1.0304 & 1.0283 & 1.0335 & 1.0148 & 1.0280 & 1.0331 & 1.0229 & 1.0711 & 1.1837\\
     & 0.3723 & 0.3708 & 0.3484 & 8.1740 & 8.3004 & 8.4825 & 3.7402 & 3.7732 & 3.7904 & 2.0304 & 1.7733 & 1.6501\\ \hline
  \end{tabular}
}
\end{center}
\label{tab:app:socialComp:2DTorus1}
\end{table*}

\end{document}
