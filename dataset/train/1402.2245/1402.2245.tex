\subsection{\Ppterms}
\label{sec:ppterm}
\denotationDistributed{In this section we will introduce a particular subclass of the set of proof terms, namely the \emph{\ppterms}. 
Just like \imsteps\ are in correspondence with maximal developments of \orthoredexsets, so that any maximal development can be \emph{denoted} by an \imstep (\confer\ \refsec{mstep-orthoredexset}); \ppterms\ are related with the set of all \redseqs\ for a given \TRS, so that \emph{any \redseq} can be denoted by a \ppterm.
A formal characterisation of the idea of denoting a \redseq\ by a (stepwise) proof term will be given in the following Section~\ref{sec:pterm-denotation}.
}\denotationInOwnChapter{In the following, we introduce the set of \ppterms, give some additional related definitions and state some basic properties of this subset of the set of valid proof terms.
}

\begin{definition}[One-step]
\label{dfn:one-step}
A \emph{one-step} is an \imstep\ including exactly one occurrence of a rule symbol.
If  is a one-step, then we define the redex position of , notation , as the position of the unique rule symbol occurrence in , and the depth of , notation , as ; \confer\ \refdfn{step} for the analogy with the corresponding notions as defined for a reduction step.
\end{definition}

\begin{definition}[\Ppterm, \Pnpterm]
\label{dfn:ppterm}
A \emph{\ppterm} is any proof term  whose formation satisfies any of the following conditions, where we refer to cases in \refdfn{layer-pterm}:
\begin{itemize}
	\item  is a one-step, so it is built by case \ref{rule:ptmstep},
	\item  is built by case \ref{rule:ptinfC}, so that , and all of the  are \ppterms, or
	\item  is built by case \ref{rule:ptbinC}, so that , and both  and  are \ppterms.
\end{itemize}
A \emph{\pnpterm} is any proof term  such that either  is a \ppterm\ or .
\end{definition}

\begin{definition}[Steps of a \pnpterm]
\label{dfn:steps}
For any  \pnpterm, we define the number of \emph{steps} of , notation , as the countable ordinal defined as follows: \\
\begin{tabular}{l}
if , then . \\
if  is a one-step, then . \\
if  then ; \confer \refdfn{ordinal-infAdd}. \\
if  then .
\end{tabular}
\end{definition}

\begin{lemma}
\label{rsl:steps-ordinal-coherence}
Let  be a \ppterm, and let  the ordinal such that . Then  is a limit ordinal iff  is.
\end{lemma}

\begin{proof}
Easy induction on  where .
\end{proof}

\begin{definition}[-th component of a \ppterm]
\label{dfn:ppterm-component}
Let  be a \ppterm\ and  an ordinal such that . We define the \emph{-th component} of , notation , as the one-step defined as follows: \\
\begin{tabular}{p{.95\textwidth}}
if  is a one-step, then . \\
if , then there are unique  and  such that  and ; \confer\ \reflem{ordinal-lt-infAdd-then-unique-representation}. We define . \\
if  and  then . \\
if  and , then  such that .
\end{tabular}
\end{definition}

\begin{definition}
\label{dfn:maxd}
Let  be a \ppterm\ such that .
Then we define the \emph{maximal depth activity} of  as .
We also define the \emph{maximal step depth} of  as  where  is the set of all the rule symbols occurring in .
\end{definition}


\medskip
We show some expected properties of the components of a \ppterm.
These properties particularly entail that a \ppterm\ can be seen as the concatenation of its components, so that the particular way in which they are associated is irrelevant. 
\denotationDistributed{More on this in Section~\ref{sec:peqence}, specifically in Section~\ref{sec:peqence-basic-properties} and Section~\ref{sec:frso}.}

\begin{lemma}
\label{rsl:ppterm-mind-big-then-tdist-little}
Let  be a \ppterm,  an ordinal and , such that  and . Then 
\begin{enumerate}
	\item \label{it:ppterm-mind-big-then-step-mind-big}
	.
	\item \label{it:ppterm-mind-big-then-tdist-little-step}
	.
	\item \label{it:ppterm-mind-big-then-tdist-little-src}
	.
\end{enumerate}
\end{lemma}

\begin{proof}
We proceed by induction on , \confer\ Prop.~\ref{rsl:pterm-induction-principle}.
If  is a one-step then  and .
Then we conclude immediately; \confer\ Lemma~\ref{rsl:mind-big-then-tdist-little} for (\ref{it:ppterm-mind-big-then-tdist-little-step}) and (\ref{it:ppterm-mind-big-then-tdist-little-src}).

Assume .
If , so that , then we conclude by \ih\ on .
Otherwise , so that . Then by applying \ih\ on  we obtain (\ref{it:ppterm-mind-big-then-step-mind-big}) and (\ref{it:ppterm-mind-big-then-tdist-little-step}) immediately, and also .
On the other hand we can apply Lemma~\ref{rsl:mind-big-then-tdist-little} to , obtaining . Thus we conclude by Lemma~\ref{rsl:tdist-is-ultrametric} since .

Assume . Let ,  such that , so that .
Then \ih\ on  yields immediately (\ref{it:ppterm-mind-big-then-step-mind-big}) and (\ref{it:ppterm-mind-big-then-tdist-little-step}), and also .
On the other hand, for each  it is immediate that , then an easy induction on  using Lemma~\ref{rsl:mind-big-then-tdist-little} and Lemma~\ref{rsl:tdist-is-ultrametric} yields . Thus we conclude by Lemma~\ref{rsl:tdist-is-ultrametric}.
\end{proof}


\begin{lemma}
\label{rsl:ppterm-mind-big-then-tdist-little-tgt}
Let  be a convergent \ppterm\ such that , and .
Then .
\end{lemma}

\begin{proof}
We proceed by induction on .
If  is a one-step then  and it suffices to observe that .

Assume .
If , then \ih\ on  yields .
On the other hand, Lemma~\ref{rsl:mind-big-then-tdist-little} implies .
We conclude by Lemma~\ref{rsl:tdist-is-ultrametric} since .
Otherwise, , then .
In this case we can apply \ih\ on  obtaining , thus we conclude.

Assume  and let ,  such that .
Then \ih\ on  yields .
Moreover, Lemma~\ref{rsl:mind-big-then-tdist-little} on  implies . Ths we conclude by Lemma~\ref{rsl:tdist-is-ultrametric}.
\end{proof}

\begin{lemma}
\label{rsl:ppterm-src}
Let  be a \ppterm. Then .
\end{lemma}

\begin{proof}
Easy induction on .
\end{proof}

\begin{lemma}
\label{rsl:ppterm-tgt-successor}
Let  be a \ppterm\ such that . Then .
\end{lemma}

\begin{proof}
We proceed by induction on .
If  is a one-step then  and we conclude immediately.

Assume .
Then  would imply , which is not possible since .
Then let  be the ordinal verifying , so that . 
We observe that , then .
We conclude by \ih\ on .

Finally,  contradicts  to be a successor ordinal. Thus we conclude.
\end{proof}


\begin{lemma}
\label{rsl:ppterm-tgt-limit}
Let  be a convergent \ppterm\ such that  is a limit ordinal. 
Then .
\end{lemma}

\begin{proof}
Observe  being a limit ordinal implies 
\denotationInOwnChapter{(\confer\ Lem.~\ref{rsl:steps-ordinal-coherence} and Lem.~\ref{rsl:ptinfC-iff-limit})}, so that  is defined to be equal to . Observe that Lem~\ref{rsl:mind-big-then-tdist-little}:(\ref{it:convergent-then-has-tgt}) implies this limit to be defined.
Let , let  such that  implies ,  such that  if , and .

Let  and .
Then  where  and , so that . 
Then , so that Lemma~\ref{rsl:ppterm-mind-big-then-tdist-little-tgt} implies .
On the other hand,  implies .
Hence Lemma~\ref{rsl:tdist-is-ultrametric} yields .
Consequently, we conclude.
\end{proof}


\begin{lemma}
\label{rsl:ppterm-tgt-src-coherence}
Let  be a \ppterm\ and  such that . Then .
\end{lemma}

\begin{proof}
We proceed by induction on . Observe  is a one-step would imply , contradicting .

Assume . We consider three cases
\begin{itemize}
\item 
If  then we conclude just by \ih\ on .
\item
If , then  and  where . 
Then  and , by Lemma~\ref{rsl:ppterm-tgt-successor} and Lemma~\ref{rsl:ppterm-src} respectively. Thus we conclude.
\item
If , then  and , therefore  and . Observe that  implies . Hence we conclude by \ih\ on .
\end{itemize}

Assume . Let ,  such that  and , so that .
If , then  for some , and , so that .
Therefore  and , by Lemma~\ref{rsl:ppterm-src} and Lemma~\ref{rsl:ppterm-tgt-successor} respectively. Thus we conclude.
Otherwise ; notice that  being a limit ordinal would contradict  being a successor one.
In this case , thus we conclude by \ih\ on .
\end{proof}



\begin{lemma}
\label{rsl:ppterm-seq-mind}
Let  be a \ppterm. 
Then \\ 
 \denotationDistributed{, \\
where if  is a set of positions, then  is the element of  verifying  if .} \end{lemma}

\begin{proof}
We prove that .
The rest of the statement follows immediately since it is trivial to verify  for any ; \confer\ Dfn.~\ref{dfn:dmin-imstep}.

We proceed by induction on ; \confer\ Prop.~\ref{rsl:pterm-induction-principle}.
We define , so we must verify .
If  is a one-step then the result holds immediately.

Assume .
In this case, \ih\ on  yields  for each , and Dfn.~\ref{dfn:layer-pterm} implies .
Then it suffices to verify .
From the definition of , it is immediate that  for .
Assume . Notice  would imply the existence of some  verifying , contradicting either the definition of  (if ) or the assertion  (otherwise). Hence .
A similar argument for the case  is enough to conclude.

If , then an argument similar to that used for binary composition applies.
To verify that , observe that  for all , and consider  such that  for all .
Then we can contradict  proceeding as in the previous case, hence . Thus we conclude.
\end{proof}


\includeStandardisation{
\medskip
The view of a \ppterm\ as the concatenation of its components will be used extensively to obtain standardisation results in Section~\ref{sec:peqence}, where the ability of separating a \ppterm\ in head (\ie\ first component) and tail (\ie\ the concatenation of components from the second one on) will be needed as well. This consideration motivates the following formalisation of the concept of tail of a \ppterm.

\begin{definition}
\label{dfn:ppterm-tail}
Let  be a proof term. We define the \emph{tail} of , notation , as follows: \\
\begin{tabular}{@{}p{.9\textwidth}}
If  is a one-step, then . \\
If  and  is a one-step, then . \\
If  and  is not a one-step, then . \\
If  and  is a one-step, then . \\
If  and  is not a one-step, then . \\
\end{tabular}
\end{definition}
}



\denotationInOwnChapter{\subsection{Denotation -- formal definition and proof of existence}
\label{sec:pterm-denotation-defs-existence}
}\denotationDistributed{
\subsection{Denotation of \redseqs}
\label{sec:pterm-denotation}
}In this section, we formalise the notion of a \pnpterm\ \emph{denoting} a \redseq, resorting to the definitions of length and -th component of \pnpterms, given in the presentation of such terms.
Then we prove the existence, for any \redseq\ having a countable ordinal length, of a \pnpterm\ which denotes it. 


\denotationInOwnChapter{As we have discussed in the introduction to Section~\ref{sec:pterm-denotation}, denotation of a \redseq\ is not unique. 
In the next subsection, we will investigate how to characterise the proof terms denoting the same \redseq.}


\begin{definition}[Denotation for reduction steps]
\label{dfn:redstep-denotation}
Let  be a reduction step, and  a one-step. 
Then  \emph{denotes}  iff all the following apply: , , and , therefore .
\end{definition}

\begin{definition}[Mapping from one-steps to reduction steps]
\label{dfn:rstepden}
Let \trst\ be a \TRS. We define the mapping  from the set of one-steps for \trst\ to the set of reduction steps for \trst, as follows: 
.
\end{definition}

\begin{lemma}
\label{rsl:rstepden-denotes}
Let  be a one-step and  a reduction step. 
Then  denotes  iff .
\end{lemma}

\begin{proof}
We prove each direction of the biconditional. 

\noindent
:
Let us say . Hypotheses imply immediately , and also , so that  and . Thus we conclude.
\noindent
:
Let us say  and .
Then it is immediate from Dfn.~\ref{dfn:rstepden} to verify  and .
In turn, observe that  where , and
, so that it is straightforward to verify . Thus we conclude.
\end{proof}



\begin{definition}[Denotation for \redseqs]
\label{dfn:redseq-denotation}
Let \reda\ be a \redseq, and  a \pnpterm. 
We will say that  \emph{denotes} \reda\ iff ,  and  denotes  for all .
\end{definition}


\begin{lemma}
\label{rsl:redseq-denotation-implications}
Let \reda\ be a \redseq, and  a \pnpterm, such that  denotes \reda.
Then ,  is convergent iff \reda\ is, and in that case, .
\end{lemma}

\begin{proof}
If , then the result holds immediately.

Otherwise, the result about  stems immediately from \reflem{ppterm-seq-mind}.

We prove the result about convergence.
Assume that  is a limit ordinal, then ; \confer\ Lem.~\ref{rsl:steps-ordinal-coherence} and Lem.~\ref{rsl:ptinfC-iff-limit}.
Assume \reda\ convergent, consider some , and  such that  if .
\refLem{ordinal-lt-infAdd-then-unique-representation} implies that  and  for some ; so that .
Consider , and . Observe  where , so that . Therefore .
Hence \reflem{ppterm-seq-mind} implies that . Consequently,  is convergent.

Conversely, assume  convergent, let , consider  such that  if . 
Let , and take  such that . 
Then Lem.~\ref{rsl:ordinal-lt-infAdd-then-unique-representation} implies  and , moreover,  implies .
Hence  by \reflem{ppterm-seq-mind}. Consequently, the requirement about depths in the characterisation of convergent \redseqs, \ie\ condition~(\ref{it:dfn-sred-depth}) in Dfn.~\ref{dfn:sred}, holds for \reda.
To prove the existence of , \ie\ condition~(\ref{it:dfn-sred-limit-existence}) in Dfn.~\ref{dfn:sred}, it suffices to observe that Lem.~\ref{rsl:mind-big-then-tdist-little}:(\ref{it:convergent-then-has-tgt}) implies that  is defined, and in turn Lem.~\ref{rsl:ppterm-tgt-limit} implies the desired limit to equal . Hence  is convergent.

If  is a successor ordinal, then assuming  is convergent, a straightforward induction on  suffices to prove that  is convergent as well; observe that Lem.~\ref{rsl:steps-ordinal-coherence} and Lem~\ref{rsl:ptinfC-iff-limit} imply that only one-step and binary concatenation must be considered. For the other direction, it is enough to observe that  being a successor ordinal implies immediately convergence of .

Finally, the result about targets stems immediately from \reflem{ppterm-tgt-limit} and \reflem{ppterm-tgt-successor}.
\end{proof}


\begin{proposition}
\label{rsl:denotation-existence}
Let  be a \redseq\ having a countable length. Then there exists a \pnpterm\  such that  denotes .
\end{proposition}

\begin{proof}
We proceed by induction on .

If , \ie\ , then it suffices to take .

Assume that . Let us say  where , implying that .
Take .
It is immediate to verify that  is a \ppterm\ verifying . Moreover, a simple analysis yields .
Furthermore, , and ; therefore  denotes . Hence  denotes .

Assume  and . 
In this case, applying twice \ih\ yields the existence of ,  such that  denotes  and  denotes .
Then a straightforward analysis allows to obtain that  denotes .

Assume  is a limit ordinal; recall that  is countable. Then Prop.~\ref{rsl:cofinality-omega} implies  where  for all .
Therefore, for any , \ih\ can be applied to obtain some  denoting .
We take .

Let . It is easy to verify that  is convergent, then Lem.~\ref{rsl:redseq-denotation-implications} implies 
; \confer\ conditions about sources and targets in Dfn.~\ref{dfn:sred}.
Hence  is a well-formed proof term.
Recalling that , it is straightforward to obtain .
Moreover, , recall that  denotes .
Let . Then Lem.~\ref{rsl:ordinal-lt-infAdd-then-unique-representation} implies the existence of unique  and  such that  and .
Therefore  and , \confer\ Dfn.~\ref{dfn:ppterm-component} and Dfn.~\ref{dfn:redseq-section}.
Hence  denoting  implies that  denotes .
Consequently, we conclude.
\end{proof}











