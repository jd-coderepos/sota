\subsection{\Ppterms}
\label{sec:ppterm}
\denotationDistributed{In this section we will introduce a particular subclass of the set of proof terms, namely the \emph{\ppterms}. 
Just like \imsteps\ are in correspondence with maximal developments of \orthoredexsets, so that any maximal development can be \emph{denoted} by an \imstep (\confer\ \refsec{mstep-orthoredexset}); \ppterms\ are related with the set of all \redseqs\ for a given \TRS, so that \emph{any \redseq} can be denoted by a \ppterm.
A formal characterisation of the idea of denoting a \redseq\ by a (stepwise) proof term will be given in the following Section~\ref{sec:pterm-denotation}.
}\denotationInOwnChapter{In the following, we introduce the set of \ppterms, give some additional related definitions and state some basic properties of this subset of the set of valid proof terms.
}

\begin{definition}[One-step]
\label{dfn:one-step}
A \emph{one-step} is an \imstep\ including exactly one occurrence of a rule symbol.
If $\psi$ is a one-step, then we define the redex position of $\psi$, notation $\RPos{\psi}$, as the position of the unique rule symbol occurrence in $\psi$, and the depth of $\psi$, notation $\sdepth{\psi}$, as $\posln{\RPos{\psi}}$; \confer\ \refdfn{step} for the analogy with the corresponding notions as defined for a reduction step.
\end{definition}

\begin{definition}[\Ppterm, \Pnpterm]
\label{dfn:ppterm}
A \emph{\ppterm} is any proof term $\psi$ whose formation satisfies any of the following conditions, where we refer to cases in \refdfn{layer-pterm}:
\begin{itemize}
	\item $\psi$ is a one-step, so it is built by case \ref{rule:ptmstep},
	\item $\psi$ is built by case \ref{rule:ptinfC}, so that $\psi = \icomp \psi_i$, and all of the $\psi_i$ are \ppterms, or
	\item $\psi$ is built by case \ref{rule:ptbinC}, so that $\psi = \psi_1 \comp \psi_2$, and both $\psi_1$ and $\psi_2$ are \ppterms.
\end{itemize}
A \emph{\pnpterm} is any proof term $\psi$ such that either $\psi$ is a \ppterm\ or $\psi \in \iSigmaTerms$.
\end{definition}

\begin{definition}[Steps of a \pnpterm]
\label{dfn:steps}
For any $\psi$ \pnpterm, we define the number of \emph{steps} of $\psi$, notation $\ppsteps{\psi}$, as the countable ordinal defined as follows: \\
\begin{tabular}{l}
if $\psi \in \iSigmaTerms$, then $\ppsteps{\psi} \eqdef 0$. \\
if $\psi$ is a one-step, then $\ppsteps{\psi} \eqdef 1$. \\
if $\psi = \icomp \psi_i$ then $\ppsteps{\psi} \eqdef \sum_{i < \omega} \ppsteps{\psi_i}$; \confer \refdfn{ordinal-infAdd}. \\
if $\psi = \psi_1 \comp \psi_2$ then $\ppsteps{\psi} \eqdef \ppsteps{\psi_1} + \ppsteps{\psi_2}$.
\end{tabular}
\end{definition}

\begin{lemma}
\label{rsl:steps-ordinal-coherence}
Let $\psi$ be a \ppterm, and let $\alpha$ the ordinal such that $\psi \in \layerpterm{\alpha}$. Then $\ppsteps{\psi}$ is a limit ordinal iff $\alpha$ is.
\end{lemma}

\begin{proof}
Easy induction on $\alpha$ where $\psi \in \layerpterm{\alpha}$.
\end{proof}

\begin{definition}[$\alpha$-th component of a \ppterm]
\label{dfn:ppterm-component}
Let $\psi$ be a \ppterm\ and $\alpha$ an ordinal such that $\alpha < \ppsteps{\psi}$. We define the \emph{$\alpha$-th component} of $\psi$, notation $\psi[\alpha]$, as the one-step defined as follows: \\
\begin{tabular}{p{.95\textwidth}}
if $\psi$ is a one-step, then $\psi[0] \eqdef \psi$. \\
if $\psi = \icomp \psi_i$, then there are unique $k$ and $\gamma$ such that $\alpha = \ppsteps{\psi_0} + \ldots + \ppsteps{\psi_{k-1}} + \gamma$ and $\gamma < \ppsteps{\psi_k}$; \confer\ \reflem{ordinal-lt-infAdd-then-unique-representation}. We define $\psi[\alpha] \eqdef \psi_k[\gamma]$. \\
if $\psi = \psi_1 \comp \psi_2$ and $\alpha < \ppsteps{\psi_1}$ then $\psi[\alpha] \eqdef \psi_1[\alpha]$. \\
if $\psi = \psi_1 \comp \psi_2$ and $\ppsteps{\psi_1} \leq \alpha$, then $\psi[\alpha] \eqdef \psi_2[\beta]$ such that $\ppsteps{\psi_1} + \beta = \alpha$.
\end{tabular}
\end{definition}

\begin{definition}
\label{dfn:maxd}
Let $\psi$ be a \ppterm\ such that $\ppsteps{\psi} < \omega$.
Then we define the \emph{maximal depth activity} of $\psi$ as $\maxd{\psi} \eqdef max(\sdepth{\redel{\psi}{n}} \setsthat n < \ppsteps{\psi})$.
We also define the \emph{maximal step depth} of $\psi$ as $\maxsd{\psi} \eqdef max(\Pdepth{\mu} \setsthat \mu \in R)$ where $R$ is the set of all the rule symbols occurring in $\psi$.
\end{definition}


\medskip
We show some expected properties of the components of a \ppterm.
These properties particularly entail that a \ppterm\ can be seen as the concatenation of its components, so that the particular way in which they are associated is irrelevant. 
\denotationDistributed{More on this in Section~\ref{sec:peqence}, specifically in Section~\ref{sec:peqence-basic-properties} and Section~\ref{sec:frso}.}

\begin{lemma}
\label{rsl:ppterm-mind-big-then-tdist-little}
Let $\psi$ be a \ppterm, $\alpha$ an ordinal and $n < \omega$, such that $\mind{\psi} > n$ and $\alpha < \ppsteps{\psi}$. Then 
\begin{enumerate}
	\item \label{it:ppterm-mind-big-then-step-mind-big}
	$\sdepth{\redel{\psi}{\alpha}} > n$.
	\item \label{it:ppterm-mind-big-then-tdist-little-step}
	$\tdist{src(\redel{\psi}{\alpha})}{tgt(\redel{\psi}{\alpha})} < 2^{-n}$.
	\item \label{it:ppterm-mind-big-then-tdist-little-src}
	$\tdist{src(\psi)}{tgt(\redel{\psi}{\alpha})} < 2^{-n}$.
\end{enumerate}
\end{lemma}

\begin{proof}
We proceed by induction on $\psi$, \confer\ Prop.~\ref{rsl:pterm-induction-principle}.
If $\psi$ is a one-step then $\alpha = 0$ and $\redel{\psi}{\alpha} = \psi$.
Then we conclude immediately; \confer\ Lemma~\ref{rsl:mind-big-then-tdist-little} for (\ref{it:ppterm-mind-big-then-tdist-little-step}) and (\ref{it:ppterm-mind-big-then-tdist-little-src}).

Assume $\psi = \psi_1 \comp \psi_2$.
If $\alpha < \ppsteps{\psi_1}$, so that $\redel{\psi}{\alpha} = \redel{\psi_1}{\alpha}$, then we conclude by \ih\ on $\psi_1$.
Otherwise $\alpha = \ppsteps{\psi_1} + \beta$, so that $\redel{\psi}{\alpha} = \redel{\psi_2}{\beta}$. Then by applying \ih\ on $\psi_2$ we obtain (\ref{it:ppterm-mind-big-then-step-mind-big}) and (\ref{it:ppterm-mind-big-then-tdist-little-step}) immediately, and also $\tdist{src(\psi_2)}{tgt(\redel{\psi}{\alpha})} < 2^{-n}$.
On the other hand we can apply Lemma~\ref{rsl:mind-big-then-tdist-little} to $\psi_1$, obtaining $\tdist{src(\psi)}{tgt(\psi_1)} < 2^{-n}$. Thus we conclude by Lemma~\ref{rsl:tdist-is-ultrametric} since $tgt(\psi_1) = src(\psi_2)$.

Assume $\psi = \icomp \psi_i$. Let $k$, $\beta$ such that $\redel{\psi}{\alpha} = \redel{\psi_k}{\beta}$, so that $\beta < \ppsteps{\psi_k}$.
Then \ih\ on $\psi_k$ yields immediately (\ref{it:ppterm-mind-big-then-step-mind-big}) and (\ref{it:ppterm-mind-big-then-tdist-little-step}), and also $\tdist{src(\psi_k)}{tgt(\redel{\psi}{\alpha})} < 2^{-n}$.
On the other hand, for each $i < k$ it is immediate that $\mind{\psi_i} \geq \mind{\psi} > n$, then an easy induction on $k$ using Lemma~\ref{rsl:mind-big-then-tdist-little} and Lemma~\ref{rsl:tdist-is-ultrametric} yields $\tdist{src(\psi)}{src(\psi_{k})} < 2^{-n}$. Thus we conclude by Lemma~\ref{rsl:tdist-is-ultrametric}.
\end{proof}


\begin{lemma}
\label{rsl:ppterm-mind-big-then-tdist-little-tgt}
Let $\psi$ be a convergent \ppterm\ such that $\mind{\psi} > p$, and $\alpha < \ppsteps{\psi}$.
Then $\tdist{tgt(\redel{\psi}{\alpha})}{tgt(\psi)} < 2^{-p}$.
\end{lemma}

\begin{proof}
We proceed by induction on $\psi$.
If $\psi$ is a one-step then $\alpha = 0$ and it suffices to observe that $\redel{\psi}{0} = \psi$.

Assume $\psi = \psi_1 \comp \psi_2$.
If $\alpha < \ppsteps{\psi_1}$, then \ih\ on $\psi_1$ yields $\tdist{tgt(\redel{\psi}{\alpha})}{tgt(\psi_1)} < 2^{-p}$.
On the other hand, Lemma~\ref{rsl:mind-big-then-tdist-little} implies $\tdist{src(\psi_2)}{tgt(\psi)} < 2^{-p}$.
We conclude by Lemma~\ref{rsl:tdist-is-ultrametric} since $tgt(\psi_1) = src(\psi_2)$.
Otherwise, $\alpha = \ppsteps{\psi_1} + \beta$, then $\redel{\psi}{\alpha} = \redel{\psi_2}{\beta}$.
In this case we can apply \ih\ on $\psi_2$ obtaining $\tdist{tgt(\redel{\psi_2}{\beta})}{tgt(\psi_2)} < 2^{-p}$, thus we conclude.

Assume $\psi = \icomp \psi_i$ and let $k$, $\gamma$ such that $\redel{\psi}{\alpha} = \redel{\psi_k}{\gamma}$.
Then \ih\ on $\psi_k$ yields $\tdist{tgt(\redel{\psi}{\alpha})}{tgt(\psi_k)} < 2^{-p}$.
Moreover, Lemma~\ref{rsl:mind-big-then-tdist-little} on $\icomp \psi_{k+1+i}$ implies $\tdist{src(\psi_{k+1})}{tgt(\psi)} < 2^{-p}$. Ths we conclude by Lemma~\ref{rsl:tdist-is-ultrametric}.
\end{proof}

\begin{lemma}
\label{rsl:ppterm-src}
Let $\psi$ be a \ppterm. Then $src(\redel{\psi}{0}) = src(\psi)$.
\end{lemma}

\begin{proof}
Easy induction on $\psi$.
\end{proof}

\begin{lemma}
\label{rsl:ppterm-tgt-successor}
Let $\psi$ be a \ppterm\ such that $\ppsteps{\psi} = \alpha + 1$. Then $tgt(\psi) = tgt(\redel{\psi}{\alpha})$.
\end{lemma}

\begin{proof}
We proceed by induction on $\psi$.
If $\psi$ is a one-step then $\alpha = 0$ and we conclude immediately.

Assume $\psi = \psi_1 \comp \psi_2$.
Then $\alpha < \ppsteps{\psi_1}$ would imply $\alpha + 1 = \ppsteps{\psi} \leq \ppsteps{\psi_1}$, which is not possible since $\ppsteps{\psi_2} > 0$.
Then let $\beta$ be the ordinal verifying $\ppsteps{\psi_1} + \beta = \alpha$, so that $\redel{\psi}{\alpha} = \redel{\psi_2}{\beta}$. 
We observe that $\ppsteps{\psi_1} + \beta + 1 = \alpha + 1 = \ppsteps{\psi}$, then $\ppsteps{\psi_2} = \beta + 1$.
We conclude by \ih\ on $\psi_2$.

Finally, $\psi = \icomp \psi_i$ contradicts $\ppsteps{\psi}$ to be a successor ordinal. Thus we conclude.
\end{proof}


\begin{lemma}
\label{rsl:ppterm-tgt-limit}
Let $\psi$ be a convergent \ppterm\ such that $\ppsteps{\psi}$ is a limit ordinal. 
Then $tgt(\psi) = \lim_{\alpha \to \ppsteps{\psi}} tgt(\redel{\psi}{\alpha})$.
\end{lemma}

\begin{proof}
Observe $\ppsteps{\psi}$ being a limit ordinal implies $\psi = \icomp \psi_i$
\denotationInOwnChapter{(\confer\ Lem.~\ref{rsl:steps-ordinal-coherence} and Lem.~\ref{rsl:ptinfC-iff-limit})}, so that $tgt(\psi)$ is defined to be equal to $\lim_{i \to \omega} tgt(\psi_i)$. Observe that Lem~\ref{rsl:mind-big-then-tdist-little}:(\ref{it:convergent-then-has-tgt}) implies this limit to be defined.
Let $p \in \Nat$, let $k'$ such that $k' < j < \omega$ implies $\tdist{tgt(\psi_j)}{tgt(\psi)} < 2^{-p}$, $k''$ such that $\mind{\psi_j} > p$ if $j > k''$, and $k \eqdef max(k', k'')$.

Let $\beta = \ppsteps{\psi_0} + \ldots + \ppsteps{\psi_k}$ and $\gamma > \beta$.
Then $\gamma = \ppsteps{\psi_0} + \ldots + \ppsteps{\psi_j} + \gamma'$ where $\gamma' < \ppsteps{\psi_{j+1}}$ and $j \geq k$, so that $\redel{\psi}{\gamma} = \redel{\psi_{j+1}}{\gamma'}$. 
Then $j + 1 > k \geq k''$, so that Lemma~\ref{rsl:ppterm-mind-big-then-tdist-little-tgt} implies $\tdist{tgt(\redel{\psi}{\gamma})}{tgt(\psi_{j+1})} < 2^{-p}$.
On the other hand, $j + 1 > k \geq k'$ implies $\tdist{tgt(\psi_{j+1})}{tgt(\psi)} < 2^{-p}$.
Hence Lemma~\ref{rsl:tdist-is-ultrametric} yields $\tdist{tgt(\redel{\psi}{\gamma})}{tgt(\psi)} < 2^{-p}$.
Consequently, we conclude.
\end{proof}


\begin{lemma}
\label{rsl:ppterm-tgt-src-coherence}
Let $\psi$ be a \ppterm\ and $\alpha < \ppsteps{\psi}$ such that $\alpha = \alpha' + 1$. Then $src(\redel{\psi}{\alpha}) = tgt(\redel{\psi}{\alpha'})$.
\end{lemma}

\begin{proof}
We proceed by induction on $\psi$. Observe $\psi$ is a one-step would imply $\alpha = 0$, contradicting $\alpha = \alpha' + 1$.

Assume $\psi = \psi_1 \comp \psi_2$. We consider three cases
\begin{itemize}
\item 
If $\alpha < \ppsteps{\psi_1}$ then we conclude just by \ih\ on $\psi_1$.
\item
If $\alpha = \ppsteps{\psi_1}$, then $\redel{\psi}{\alpha} = \redel{\psi_2}{0}$ and $\redel{\psi}{\alpha'} = \redel{\psi_1}{\alpha'}$ where $\alpha' + 1 = \alpha = \ppsteps{\psi_1}$. 
Then $tgt(\redel{\psi}{\alpha'}) = tgt(\psi_1)$ and $src(\redel{\psi}{\alpha}) = src(\psi_2)$, by Lemma~\ref{rsl:ppterm-tgt-successor} and Lemma~\ref{rsl:ppterm-src} respectively. Thus we conclude.
\item
If $\alpha > \ppsteps{\psi_1}$, then $\alpha' = \ppsteps{\psi_1} + \beta'$ and $\alpha = \ppsteps{\psi_1} + (\beta' + 1)$, therefore $\redel{\psi}{\alpha} = \redel{\psi_2}{\beta' + 1}$ and $\redel{\psi}{\alpha'} = \redel{\psi_2}{\beta'}$. Observe that $\alpha < \ppsteps{\psi}$ implies $\beta' + 1 < \ppsteps{\psi_2}$. Hence we conclude by \ih\ on $\psi_2$.
\end{itemize}

Assume $\psi = \icomp \psi_i$. Let $k$, $\gamma$ such that $\alpha = \ppsteps{\psi_0} + \ldots + \ppsteps{\psi_{k-1}} + \gamma$ and $\gamma < \ppsteps{\psi_k}$, so that $\redel{\psi}{\alpha} = \redel{\psi_k}{\gamma}$.
If $\gamma = 0$, then $\ppsteps{\psi_{k-1}} = \beta + 1$ for some $\beta$, and $\alpha' = \ppsteps{\psi_0} + \ldots + \ppsteps{\psi_{k-2}} + \beta$, so that $\redel{\psi}{\alpha'} = \redel{\psi_{k-1}}{\beta}$.
Therefore $src(\redel{\psi}{\alpha}) = src(\psi_k)$ and $tgt(\redel{\psi}{\alpha'}) = tgt(\psi_{k-1})$, by Lemma~\ref{rsl:ppterm-src} and Lemma~\ref{rsl:ppterm-tgt-successor} respectively. Thus we conclude.
Otherwise $\gamma = \gamma' + 1$; notice that $\gamma$ being a limit ordinal would contradict $\alpha$ being a successor one.
In this case $\redel{\psi}{\alpha'} = \redel{\psi_k}{\gamma'}$, thus we conclude by \ih\ on $\psi_k$.
\end{proof}



\begin{lemma}
\label{rsl:ppterm-seq-mind}
Let $\psi$ be a \ppterm. 
Then \\ 
$\begin{array}{rcl}
\mind{\psi} & = &
min(\sdepth{\redel{\psi}{\alpha}} \setsthat \alpha < \ppsteps{\psi}) 
\denotationDistributed{
\\ 
& = &
\posln{\,min_{\leq_{DL}} (\RPos{\redel{\psi}{\alpha}} \setsthat \alpha < \ppsteps{\psi})}
}
\\
& = &
min(\mind{\redel{\psi}{\alpha}} \setsthat \alpha < \ppsteps{\psi}) 
\end{array}$ \denotationDistributed{, \\
where if $P$ is a set of positions, then $p \eqdef min_{\leq_{DL}} (P)$ is the element of $P$ verifying $p \leq_{DL} q$ if $q \in P$.} \end{lemma}

\begin{proof}
We prove that $\mind{\psi} = min(\mind{\redel{\psi}{\alpha}} \setsthat \alpha < \ppsteps{\psi})$.
The rest of the statement follows immediately since it is trivial to verify $\sdepth{\redel{\psi}{\alpha}} = \mind{\redel{\psi}{\alpha}}$ for any $\alpha$; \confer\ Dfn.~\ref{dfn:dmin-imstep}.

We proceed by induction on $\psi$; \confer\ Prop.~\ref{rsl:pterm-induction-principle}.
We define $\mindp{\psi} \eqdef min(\mind{\redel{\psi}{\alpha}} \setsthat \alpha < \ppsteps{\psi})$, so we must verify $\mind{\psi} = \mindp{\psi}$.
If $\psi$ is a one-step then the result holds immediately.

Assume $\psi = \psi_1 \comp \psi_2$.
In this case, \ih\ on $\psi_i$ yields $\mind{\psi_i} = \mindp{\psi_i}$ for each $i = 1,2$, and Dfn.~\ref{dfn:layer-pterm} implies $\mind{\psi} = min(\mind{\psi_1}, \mind{\psi_2})$.
Then it suffices to verify $\mindp{\psi} = min(\mindp{\psi_1}, \mindp{\psi_2})$.
From the definition of $\mindpfn$, it is immediate that $\mindp{\psi} \leq \mindp{\psi_i}$ for $i = 1,2$.
Assume $\mindp{\psi_1} \leq \mindp{\psi_2}$. Notice $\mindp{\psi} < \mindp{\psi_1}$ would imply the existence of some $\gamma$ verifying $\mindp{\redel{\psi}{\gamma}} < \mindp{\psi_1}$, contradicting either the definition of $\mindp{\psi_1}$ (if $\gamma < \ppsteps{\psi_1}$) or the assertion $\mindp{\psi_1} \leq \mindp{\psi_2}$ (otherwise). Hence $\mindp{\psi} = \mindp{\psi_1}$.
A similar argument for the case $\mindp{\psi_2} < \mindp{\psi_1}$ is enough to conclude.

If $\psi = \icomp \psi_i$, then an argument similar to that used for binary composition applies.
To verify that $\mindp{\psi} = min_{i < \omega}(\mindp{\psi_i})$, observe that $\mindp{\psi} \leq \mindp{\psi_i}$ for all $i$, and consider $n$ such that $\mindp{\psi_n} \leq \mindp{\psi_i}$ for all $i$.
Then we can contradict $\mindp{\psi} < \mindp{\psi_n}$ proceeding as in the previous case, hence $\mindp{\psi} = \mindp{\psi_n}$. Thus we conclude.
\end{proof}


\includeStandardisation{
\medskip
The view of a \ppterm\ as the concatenation of its components will be used extensively to obtain standardisation results in Section~\ref{sec:peqence}, where the ability of separating a \ppterm\ in head (\ie\ first component) and tail (\ie\ the concatenation of components from the second one on) will be needed as well. This consideration motivates the following formalisation of the concept of tail of a \ppterm.

\begin{definition}
\label{dfn:ppterm-tail}
Let $\psi$ be a proof term. We define the \emph{tail} of $\psi$, notation $\redfrom{\psi}{1}$, as follows: \\
\begin{tabular}{@{$\ \ \bullet\ \ $}p{.9\textwidth}}
If $\psi$ is a one-step, then $\redfrom{\psi}{1} \eqdef tgt(\psi)$. \\
If $\psi = \psi_1 \comp \psi_2$ and $\psi_1$ is a one-step, then $\redfrom{\psi}{1} \eqdef \psi_2$. \\
If $\psi = \psi_1 \comp \psi_2$ and $\psi_1$ is not a one-step, then $\redfrom{\psi}{1} \eqdef \redfrom{\psi_1}{1} \comp \psi_2$. \\
If $\psi = \icomp \psi_i$ and $\psi_0$ is a one-step, then $\redfrom{\psi}{1} \eqdef \icomp \psi_{1+i}$. \\
If $\psi = \icomp \psi_i$ and $\psi_0$ is not a one-step, then $\redfrom{\psi}{1} \eqdef \redfrom{\psi_0}{1} \comp (\icomp \psi_{1+i})$. \\
\end{tabular}
\end{definition}
}



\denotationInOwnChapter{\subsection{Denotation -- formal definition and proof of existence}
\label{sec:pterm-denotation-defs-existence}
}\denotationDistributed{
\subsection{Denotation of \redseqs}
\label{sec:pterm-denotation}
}In this section, we formalise the notion of a \pnpterm\ \emph{denoting} a \redseq, resorting to the definitions of length and $\alpha$-th component of \pnpterms, given in the presentation of such terms.
Then we prove the existence, for any \redseq\ having a countable ordinal length, of a \pnpterm\ which denotes it. 


\denotationInOwnChapter{As we have discussed in the introduction to Section~\ref{sec:pterm-denotation}, denotation of a \redseq\ is not unique. 
In the next subsection, we will investigate how to characterise the proof terms denoting the same \redseq.}


\begin{definition}[Denotation for reduction steps]
\label{dfn:redstep-denotation}
Let $a = \langle t, p, \mu \rangle$ be a reduction step, and $\psi$ a one-step. 
Then $\psi$ \emph{denotes} $a$ iff all the following apply: $src(\psi) = t$, $tgt(\psi) = tgt(a)$, and $\psi(p) = \mu$, therefore $\sdepth{a} = \mind{\psi}$.
\end{definition}

\begin{definition}[Mapping from one-steps to reduction steps]
\label{dfn:rstepden}
Let \trst\ be a \TRS. We define the mapping $\rstepdenfn$ from the set of one-steps for \trst\ to the set of reduction steps for \trst, as follows: 
$\rstepden{\psi} \eqdef \langle src(\psi), \RPos{\psi}, \psi(\RPos{\psi}) \rangle$.
\end{definition}

\begin{lemma}
\label{rsl:rstepden-denotes}
Let $\psi$ be a one-step and $\stepa$ a reduction step. 
Then $\psi$ denotes $a$ iff $a = \rstepden{\psi}$.
\end{lemma}

\begin{proof}
We prove each direction of the biconditional. 

\noindent
$\Rightarrow )$:
Let us say $a = \langle t, p, \mu \rangle$. Hypotheses imply immediately $t = src(\psi)$, and also $\psi(p) = \mu$, so that $p = \RPos{\psi}$ and $\mu = \psi(\RPos{\psi})$. Thus we conclude.
\noindent
$\Leftarrow )$:
Let us say $\rstepden{\psi} = \langle t, p, \mu \rangle$ and $\mu: l \to h$.
Then it is immediate from Dfn.~\ref{dfn:rstepden} to verify $src(\psi) = t$ and $\psi(p) = \mu$.
In turn, observe that $tgt(\psi) = \repl{\psi}{h[t_1, \ldots, t_m]}{p}$ where $\subtat{\psi}{p} = \mu(t_1, \ldots, t_m)$, and
$t = src(\psi) = \repl{\psi}{l[t_1, \ldots, t_m]}{p}$, so that it is straightforward to verify $tgt(\rstepden{\psi}) = tgt(\psi)$. Thus we conclude.
\end{proof}



\begin{definition}[Denotation for \redseqs]
\label{dfn:redseq-denotation}
Let \reda\ be a \redseq, and $\psi$ a \pnpterm. 
We will say that $\psi$ \emph{denotes} \reda\ iff $\ppsteps{\psi} = \redln{\reda}$, $src(\psi) = src(\reda)$ and $\redel{\psi}{\alpha}$ denotes $\redel{\reda}{\alpha}$ for all $\alpha < \redln{\reda}$.
\end{definition}


\begin{lemma}
\label{rsl:redseq-denotation-implications}
Let \reda\ be a \redseq, and $\psi$ a \pnpterm, such that $\psi$ denotes \reda.
Then $\mind{\psi} = \mind{\reda}$, $\psi$ is convergent iff \reda\ is, and in that case, $tgt(\psi) = tgt(\reda)$.
\end{lemma}

\begin{proof}
If $\psi \in \iSigmaTerms$, then the result holds immediately.

Otherwise, the result about $\mindfn$ stems immediately from \reflem{ppterm-seq-mind}.

We prove the result about convergence.
Assume that $\ppsteps{\psi}$ is a limit ordinal, then $\psi = \icomp \psi_i$; \confer\ Lem.~\ref{rsl:steps-ordinal-coherence} and Lem.~\ref{rsl:ptinfC-iff-limit}.
Assume \reda\ convergent, consider some $k < \omega$, and $\alpha$ such that $\sdepth{\redel{\reda}{\beta}} > k$ if $\beta > \alpha$.
\refLem{ordinal-lt-infAdd-then-unique-representation} implies that $\alpha = \sum_{i < n} \ppsteps{\psi_i} + \gamma$ and $\gamma < \ppsteps{\psi_n}$ for some $n$; so that $\alpha < \sum_{i \leq n} \ppsteps{\psi_i}$.
Consider $j > n$, and $\gamma < \ppsteps{\psi_j}$. Observe $\redel{\psi_j}{\gamma} = \redel{\psi}{\beta}$ where $\beta = \sum_{i < j} \ppsteps{\psi_i} + \gamma$, so that $\beta \geq \sum_{i \leq n} \ppsteps{\psi_i} > \alpha$. Therefore $\mind{\redel{\psi_j}{\gamma}} = \mind{\redel{\psi}{\beta}} = \sdepth{\redel{\reda}{\beta}} > k$.
Hence \reflem{ppterm-seq-mind} implies that $\mind{\psi_j} > k$. Consequently, $\psi$ is convergent.

Conversely, assume $\psi$ convergent, let $k < \omega$, consider $n < \omega$ such that $\mind{\psi_j} > k$ if $j > n$. 
Let $\alpha \eqdef \sum_{i \leq n} \ppsteps{\psi_i}$, and take $\beta$ such that $ \alpha < \beta < \redln{\reda}$. 
Then Lem.~\ref{rsl:ordinal-lt-infAdd-then-unique-representation} implies $\beta = \sum_{i < j} \ppsteps{\psi_i} + \gamma$ and $\gamma < \ppsteps{\psi_j}$, moreover, $\beta > \alpha$ implies $j > n$.
Hence $\sdepth{\redel{\reda}{\beta}} = \mind{\redel{\psi_j}{\gamma}} > k$ by \reflem{ppterm-seq-mind}. Consequently, the requirement about depths in the characterisation of convergent \redseqs, \ie\ condition~(\ref{it:dfn-sred-depth}) in Dfn.~\ref{dfn:sred}, holds for \reda.
To prove the existence of $\lim_{\alpha \to \redln{\reda}} tgt(\redel{\reda}{\alpha})$, \ie\ condition~(\ref{it:dfn-sred-limit-existence}) in Dfn.~\ref{dfn:sred}, it suffices to observe that Lem.~\ref{rsl:mind-big-then-tdist-little}:(\ref{it:convergent-then-has-tgt}) implies that $tgt(\psi)$ is defined, and in turn Lem.~\ref{rsl:ppterm-tgt-limit} implies the desired limit to equal $tgt(\psi)$. Hence $\reda$ is convergent.

If $\ppsteps{\psi}$ is a successor ordinal, then assuming $\reda$ is convergent, a straightforward induction on $\psi$ suffices to prove that $\psi$ is convergent as well; observe that Lem.~\ref{rsl:steps-ordinal-coherence} and Lem~\ref{rsl:ptinfC-iff-limit} imply that only one-step and binary concatenation must be considered. For the other direction, it is enough to observe that $\redln{\reda}$ being a successor ordinal implies immediately convergence of $\reda$.

Finally, the result about targets stems immediately from \reflem{ppterm-tgt-limit} and \reflem{ppterm-tgt-successor}.
\end{proof}


\begin{proposition}
\label{rsl:denotation-existence}
Let $\reda$ be a \redseq\ having a countable length. Then there exists a \pnpterm\ $\psi$ such that $\psi$ denotes $\reda$.
\end{proposition}

\begin{proof}
We proceed by induction on $\redln{\reda}$.

If $\redln{\reda} = 0$, \ie\ $\reda = \redid{t}$, then it suffices to take $\psi \eqdef t$.

Assume that $\redln{\reda} = 1$. Let us say $\redel{\reda}{0} = \langle t, p, \mu \rangle$ where $\mu: l \to h$, implying that $\subtat{t}{p} = l[t_1, \ldots, t_m]$.
Take $\psi \eqdef \repl{t}{\mu(t_1, \ldots, t_m)}{p}$.
It is immediate to verify that $\psi$ is a \ppterm\ verifying $\ppsteps{\psi} = 1$. Moreover, a simple analysis yields $src(\psi) = src(\redel{\reda}{0}) = src(\reda) = t$.
Furthermore, $\psi(p) = \mu$, and $tgt(\psi) = tgt(\redel{\reda}{0}) = \repl{t}{h[t_1, \ldots, t_m]}{p}$; therefore $\redel{\psi}{0} = \psi$ denotes $\redel{\reda}{0}$. Hence $\psi$ denotes $\reda$.

Assume $\redln{\reda} = \alpha+1$ and $\alpha > 0$. 
In this case, applying twice \ih\ yields the existence of $\psi_1$, $\psi_2$ such that $\psi_1$ denotes $\redupto{\reda}{\alpha}$ and $\psi_2$ denotes $\redsublt{\reda}{\alpha}{\alpha+1}$.
Then a straightforward analysis allows to obtain that $\psi \eqdef \psi_1 \comp \psi_2$ denotes $\reda$.

Assume $\alpha \eqdef \redln{\reda}$ is a limit ordinal; recall that $\alpha$ is countable. Then Prop.~\ref{rsl:cofinality-omega} implies $\alpha = \sum_{i < \omega} \alpha_i$ where $\alpha_i < \alpha$ for all $i < \omega$.
Therefore, for any $n < \omega$, \ih\ can be applied to obtain some $\psi_n$ denoting $\redsublt{\reda}{\sum_{i < n} \alpha_i}{\sum_{i \leq n} \alpha_i}$.
We take $\psi \eqdef \icomp \psi_i$.

Let $n < \omega$. It is easy to verify that $\redsublt{\reda}{\sum_{i < n} \alpha_i}{\sum_{i \leq n} \alpha_i}$ is convergent, then Lem.~\ref{rsl:redseq-denotation-implications} implies 
$tgt(\psi_n) 
    = tgt(\redsublt{\reda}{\sum_{i < n} \alpha_i}{\sum_{i \leq n} \alpha_i})
		= src(\redsublt{\reda}{\sum_{i \leq n} \alpha_i}{\sum_{i \leq n+1} \alpha_i}
		= src(\psi_{n+1})$; \confer\ conditions about sources and targets in Dfn.~\ref{dfn:sred}.
Hence $\psi$ is a well-formed proof term.
Recalling that $\redln{\redsublt{\reda}{\sum_{i < n} \alpha_i}{\sum_{i \leq n} \alpha_i}} = \alpha_n$, it is straightforward to obtain $\ppsteps{\psi} = \redln{\reda} = \alpha$.
Moreover, $src(\psi) = src(\psi_0) = src(\redupto{\reda}{\alpha_0}) = src(\reda)$, recall that $\psi_0$ denotes $\redupto{\reda}{\alpha_0}$.
Let $\beta < \alpha$. Then Lem.~\ref{rsl:ordinal-lt-infAdd-then-unique-representation} implies the existence of unique $k$ and $\gamma$ such that $\beta = \sum_{i < k} \alpha_i + \gamma$ and $\gamma < \alpha_k$.
Therefore $\redel{\psi}{\beta} = \redel{\psi_k}{\gamma}$ and $\redel{\reda}{\beta} = \redel{\redsublt{\reda}{\sum_{i<k} \alpha_i}{\sum_{i \leq k} \alpha_i}}{\gamma}$, \confer\ Dfn.~\ref{dfn:ppterm-component} and Dfn.~\ref{dfn:redseq-section}.
Hence $\psi_k$ denoting $\redsublt{\reda}{\sum_{i<k} \alpha_i}{\sum_{i \leq k} \alpha_i}$ implies that $\redel{\psi}{\beta}$ denotes $\redel{\reda}{\beta}$.
Consequently, we conclude.
\end{proof}











