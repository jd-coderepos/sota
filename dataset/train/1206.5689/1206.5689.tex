\documentclass[11pt]{article}



\usepackage{latexsym}
\usepackage{graphicx}
\usepackage{amssymb,eepic,epic,latexsym}
\usepackage{amsthm,amsmath}



\newtheorem{definition}{Definition}
\newtheorem{lemma}{Lemma}
\newtheorem{remark}{Remark}
\newtheorem{theorem}{Theorem}
\newtheorem{corollary}{Corollary}
\newtheorem{claim}{Claim}

\newcommand{\ie}{i.\,e.}
\newcommand{\eg}{e.\,g.}


\begin{document}



\title{A New Upper Bound for the VC-Dimension of Visibility Regions\footnote{An extended abstract of this paper appeared at SoCG '11 \cite{gk-nubvcvr-11}.}}


\author{Alexander Gilbers\\
Institute of Computer Science\\
          University of Bonn\\
           53113 Bonn, Germany \\
           gilbers@cs.uni-bonn.de
       \and Rolf Klein\\
       Institute of Computer Science\\
          University of Bonn\\
           53113 Bonn, Germany \\
           rolf.klein@uni-bonn.de
       }

\maketitle

\begin{abstract}
In this paper we are proving the following fact.
Let  be an arbitrary simple polygon, and let  be an arbitrary set of 15 points inside~.
Then there exists a subset  of  that is not ``visually discernible'', that is, 
 holds for the visibility regions  of all points  in .
In other words, the VC-dimension   of visibility regions in a simple polygon cannot exceed .
Since Valtr~\cite{v-ggwps-98} proved in 1998 that  holds, 
no progress has been made on this bound.
By -net theorems our reduction immediately implies a smaller upper bound to the number of guards
needed to cover .
\end{abstract}






\section{Introduction}  

Visibility is among the central issues in computational geometry, see, e.~g., 
Asano et al.~\cite{ags-vip-00}, Ghosh~\cite{g-vap-07} and Urrutia~\cite{u-agip-00}.
Many problems involve visibility inside simple polygons,
among them the famous art gallery problem: Given a simple polygon~, find a minimum set of 
guards whose visibility regions together cover ; see O'Rourke~\cite{r-agta-87}. 

In this paper we study a visibility problem that is related to the art gallery problem, and interesting
in its own right. Given a simple polygon  and a finite set  of points in , we call a subset 
of  {\em  discernible} if there exists a point  such that  holds.
In general, one cannot expect all subsets of a given point set in a given polygon to be discernible.
If all subsets of a given point set  are discernible we say that  is \textit{shattered}.

Let us call a number  {\em realizable} if there exists a simple polygon , and a set  of 
points in , such that all subsets of  are discernible. If  is realizable, so is .
The example in Figure~\ref{VierPunkte-fig} shows that~4 is realizable. 

\begin{figure}[hbtp]\begin{center}\includegraphics[scale=1.2,keepaspectratio]{VierPunkte}\caption{All subsets consisting of elements that form a contiguous substring of  can be discerned from the 
    lower cave of , all others from the left. Hence, 4~is realizable.}\label{VierPunkte-fig}
  \end{center}\end{figure}

The biggest realizable number  is called
the {\em VC-dimension} of visibility regions in simple polygons. 
Valtr~\cite{v-ggwps-98} showed that , and these were the best bounds on 
known until today. 
In this paper we show that 15~is not realizable, which implies .

\begin{theorem}   \label{statement-theo}
For the VC-dimension  of visibility  regions in simple polygons,  holds.
\end{theorem}

The classic -net theorem implies that  many stationary guards
with  degree view are sufficient to cover , provided that each point in  sees
at least an th part of 's area. For sufficiently large  the constant hidden in  is very close to~1; 
see Kalai and Matou\v sek~\cite{km-ggwep-97} and Koml{\'o}s et al.~\cite{kpw-atben-92}. Decreasing the upper bound on the VC-dimension  directly leads to more interesting 
upper bounds on the number of guards. 
For a textbook treatment of VC-dimension we refer the reader to Matou\v sek~\cite{m-ldg-02}.


\section{Related Work}         \label{related-sec}

The VC-dimension of range spaces of visibility regions was first considered by Kalai and Matou\v sek \cite{km-ggwep-97}.
They showed that the VC-dimension of visibility regions of a simply connected gallery (i.e. a compact set in the plane) is finite.
In their proof they start with assuming that a large set (of size about ) of points  inside a gallery is shattered by the visibility regions of the points of a set . They then derive a configuration as in Figure \ref{KalaiMatousekProof-fig}. Here, points  and  should not see each other but the segment  is encircled by visibility segments, a contradiction. This kind of argument plays an important role in our proof of the new bound.
\begin{figure}[hbtp]\begin{center}\includegraphics[scale=0.7,keepaspectratio]{KalaiMatousekProof}\caption{Segment  is encircled by visibility segments.}\label{KalaiMatousekProof-fig}
  \end{center}\end{figure}
They also gave an example of a gallery with VC-dimension~. Furthermore, they showed that there is no constant that bounds the VC-dimension
if the gallery has got holes.  
For simple polygons, Valtr~\cite{v-ggwps-98} gave an example of a gallery with VC-dimension~ and proved an upper bound of  by subdividing the gallery into
cells and bounding the number of subsets that can be seen from within one cell. In the same paper he showed an upper bound for the VC-dimension of a 
gallery with holes of  where  is the number of holes.

Since then there has been no progress on these general bounds. However, some variations of the original problem have been considered.
Isler et al. \cite{ikdv-vcdev-04} examined the case of exterior visibility. In this setting the points of~ lie on the boundary of a polygon~
and the ranges are sets of the form  where  is a point outside the convex hull of . They showed that the VC-dimension is .
The result can also be seen as a statement about wedges, as we will see later.
They also considered a more restricted version of exterior visibility where the view points  all must lie on a circle around , 
with VC-dimension . For a -dimensional version of exterior visibility with  on the boundary of a polyhedron  they found that the
VC-dimension is in  as  is the number of vertices of .
King \cite{k-vcdvt-08} examined the VC-dimension of visibility regions on polygonal terrains. 
For 1.5-dimensional terrains he proved that the VC-dimension equals  and on 2.5-dimensional terrains there is no constant bound.
In \cite{gk-nrvsp-09} we considered the original setting and showed upper bounds of 13 for the number of points on the boundary and 15 for the number of points in convex position that can be shattered by interior visibility regions.



Without using the -net theorem, Kirkpatrick~\cite{k-ggnn-00} obtained a  upper
bound to the number of {\em boundary} guards needed to cover the {\em boundary} of . This raises the question 
if the factor  in the  bound for -nets in other geometric range spaces 
can be lowered to  as well, as was shown to be true by Aronov et al.~\cite{bes-ssena-10} for 
special cases; see also King and Kirkpatrick~\cite{kk-iagsgp-10}. 




\bigskip



\section{Proof Technique}         \label{technique-sec}

Theorem~\ref{statement-theo} will be proven by contradiction. 
Throughout Sections~\ref{technique-sec} and~\ref{inner-sec}, we shall assume that there exists a simple
polygon  containing a set  of 15~points each of whose subsets is discernible.
That is, for each  there is a view point  in  such that 

holds, where, as usual,
   
denotes the visibility domain of a point  in the (closed) set . 

We may assume that the points in  and the view points  are in general position,
by the following argument. If ,
then segment  is properly crossed by the boundary of , that is, the segment and the complement
of  have an interior point in common. On the other hand,
a visibility segment , where , can be touched by the boundary of , because
this does just not block visibility. By finitely many, arbitrarily small local enlargements of  we can remove
these touching boundary parts from the visibility segments without losing any proper crossing of a non-visibility segment.
Afterwards, all points and view points can be perturbed in small disks.

Property~\ref{prop1} can be rewritten as

This means, if we form the arrangement  of all visibility regions , where ,
then for each  there is a cell (containing the view point ) which is contained
in exactly the visibility regions of the points in .
To obtain a contradiction, one would like to argue that the number of cells in 
arrangement  is smaller than , the number of subsets of . But as we do not have
an upper bound on the number of vertices of ,  the complexity of  cannot be bounded from above.

For this reason we shall replace complex visibility regions with simple {\em wedges};
for wedge arrangements, a good upper complexity bound exists; see Theorem~\ref{isler-theo} below.
To illustrate this technique, let  be a point of . We assume that there are 
\begin{enumerate}
\item  points  of ,
\item  a view point  that sees  and , but not , such that
\item   is contained in the  triangle defined by ;
\end{enumerate}
see Figure~\ref{wedge-fig}~(i). We denote by  the wedge containing  formed 
by the lines through  and  and , respectively.
\begin{figure}[hbtp]\begin{center}\includegraphics[width=\textwidth,keepaspectratio]{wedgefig}\caption{Solid lines connect points that are mutually visible; such ``visibility segments''  
    are contained in polygon . Dashed style indicates that the line of vision
    is blocked; these segments are crossed by the boundary of .}\label{wedge-fig}
  \end{center}\end{figure}
Any view point  that sees  and  must be contained in wedge . Otherwise, the chain of
visibility segments  would encircle the line segment  connecting  and , 
preventing the boundary of  from blocking the view from  to ; see Figure~\ref{wedge-fig}~(ii). 

Let  denote the outermost view points in  that see  and include a maximum angle
(by assumption, such view points exist; by the previous reasoning, they lie in ).
Then  define a sub-wedge  of , as shown in Figure~\ref{wedge-fig}~(iii). 
We claim that in this situation

holds, where  denotes the set of all view points that see at least  and .
Indeed, each view point that sees  lies in . If it sees , too, it must lie in ,
by definition of . Conversely, let  be a view point in  that sees . Then line segment
 is encircled by the visibility segments ,
as depicted in Figure~\ref{wedge-fig}~(iv). Thus, .

Fact~\ref{wedge-fact} can be interpreted in the following way. We ``sacrifice'' two of the 15 points of ,
namely  and , and restrict ourselves to studying only those  view points  
that see both .  As a benefit, the visibility region  behaves like a wedge
when restricted to .

This technique will be applied as follows.
In Section~\ref{inner-sec} we prove, as a direct consequence, that at most~5 points can be situated {\em inside} the 
convex hull of . Then, in Section~\ref{outer-sec}, we show that at most 9~points can be located {\em on} the
convex hull. Together, these claims imply Theorem~\ref{statement-theo}.


\bigskip


\section{Interior points}         \label{inner-sec}

The goal of this section is in proving the following fact.

\begin{lemma}            \label{inner-lem}
At most five points of  can lie inside the convex hull of .
\end{lemma}
\begin{proof}
Suppose there are at least six interior points  in the convex hull, .
Each of the remaining points of  is a vertex of the convex hull of .
Let  an enumeration of these points in cyclic order. Let  (where )
be the view point that sees only these vertices but no interior point; see Figure~\ref{inner-fig}.
\begin{figure}[hbtp]\begin{center}\includegraphics[scale=0.9,keepaspectratio]{innerfig}\caption{Each interior point  is contained in some triangle defined by \protect{}.}\label{inner-fig}
  \end{center}\end{figure}
Each interior point  is contained in a triangle defined by , for some~ 
(where the indices are taken modulo ). 
Since properties~{1.--3.} mentioned in Section~\ref{technique-sec} are fulfilled, Fact~\ref{wedge-fact}
implies that there exists a wedge  such that 
 holds. 
If  denotes the set of view points that see at least the points of , 
 we obtain

 which implies the following. For each subset  of  the view point 
  lies in exactly those wedges  where . But the arrangement
 of six or more wedges does not contain that many combinatorially different cells, as 
 an argument by Isler et al.~\cite{ikdv-vcdev-04} shows; see Theorem~\ref{isler-theo}. 
Thus, the convex hull of  cannot contain six interior points.
\end{proof}

Therefore, at least~10 points of  must be vertices of the convex hull of~.

\begin{theorem}         \label{isler-theo}
(Isler et al.) For any arrangement of six or more wedges, there is a subset  of wedges for which
no cell exists that is contained in exactly the wedges of .
\end{theorem}
For convenience, we include a short proof based on the ideas in~\cite{ikdv-vcdev-04}.
\begin{proof}
By Euler's formula, an arrangement of  wedges has  many cells, where 
denotes the number of half-line intersections. Since two wedges intersect in at most 4~points---
in which case they are said to {\em cross} each other---we have
. Thus, an arrangement of  wedges has at most  cells.
We are going to provide an accounting argument which shows that for each wedge
one cell is missing from a maximum size arrangement (due to a shortage of intersections), or
one of the existing cells is redundant (because it stabs the same subset of wedges as some
other cell does). This will imply that at most  many of all  different
subsets can be stabbed by a cell, thus proving the theorem.


Let  be a wedge that is crossed by all other wedges, as shown in Figure~\ref{isler-fig}~(i).
Since the two shaded cells at the apex of  and at infinity are both stabbing the subset ,
we can write off one cell of the arrangement as redundant, and exclude  from further consideration.

The remaining  wedges are used as the vertices of a graph . Two vertices are connected
by an edge if their wedges do not cross. For each edge of  there is one cell less in the arrangement,
as (at least) one of four possible intersection points is missing.
By construction, each vertex of  has degree at least~1. Suppose that vertex 
is of degree~1, and let  denote the adjacent vertex in .
If  and  have at most two of four possible intersections,
even two cells are missing from the arrangement. If  and  intersect in three points,
there is a redundant cell in , in addition to the missing one; see Figure~\ref{isler-fig} (ii).
In either case, we may double the edge connecting  and , as we obtain two savings
from this pair. In the resulting graph  each vertex is of degree at least two. Thus,
 contains at least  edges, each of which represents a cell that is missing or redundant.
\begin{figure}[hbtp]\begin{center}\includegraphics[scale=0.9,keepaspectratio]{islerfig}\caption{In (i) and (ii), respectively, the shaded cells are contained in wedge  only.}\label{isler-fig}
 \end{center}\end{figure}
\end{proof}

\bigskip
\section{Points on the boundary of the convex hull}         \label{outer-sec}Ignoring interior points, we prove, in this section, the following fact.

\begin{lemma}            \label{outer-lem}
Let  be a set of~10 points in convex position inside a simple polygon, . 
Then not all of the subsets of  are discernible.
\end{lemma}
\begin{proof}
Again, the proof is by contradiction. So let  be a set of~10 points
in convex position inside a simple polygon . Assume that every subset of
 is discernible.

First, we enumerate the points around the convex hull.\footnote{The edges of the convex hull of  may intersect the boundary of .
} 
Let  denote the set of even indexed points. 
Let  be the view point that sees exactly the even indexed points. If  lies outside the convex hull, 
, of , we draw the 
two tangents from  at . The points between the two tangent points 
facing  are called {\em front points}, all other
points are named  {\em back points} of ; see Figure~\ref{front-fig}.
(If  then all points of  are called back points.)
 
 
\begin{figure}[htbp]\begin{center}\includegraphics[scale=0.7,keepaspectratio]{frontfig}\caption{Front points appear in white, back points in black. View point 
    sees exactly the points of even index.}\label{front-fig}
  \end{center}\end{figure}


We are going to discuss the case depicted in Figure~\ref{front-fig} first, namely:

\noindent{\bf Case~1: There exists an odd front point.}

It follows from the definition of front points that in this case  lies outside the convex hull of .
Let  and  be the outermost left and right front points with odd index, as seen from ;
and let  and  denote their outer neighbors, as shown in Figure~\ref{front-fig}.
While  is possible, we always have . Observe that  and
 may be front or back points; this will require some case analysis later on.

{\bf Notation.}
For two points , let   denote the open half-plane to the left of the
ray  from  through , and  the open half-plane to its right.
\begin{figure*}[hbtp]\begin{center}\includegraphics[width=\textwidth,keepaspectratio]{front2fig}\caption{(i) As segment  must be intersected by the boundary of , it cannot
    be encircled by visibility segments. (ii) Defining subsets  and  of .}\label{front2-fig}
  \end{center}\end{figure*}


\begin{claim}         \label{newfront-claim}
Each view point  that sees  and  lies in .
\end{claim}
\begin{proof}
If  were contained in  then the chain of visibility segments 
 would encircle the segment ---a contradiction,
because  does not see the odd indexed point ; see  Figure~\ref{front2-fig}~(i).
\end{proof}


We now define two subsets  and  of  that will be crucial in our proof.
\begin{definition}    \label{lr-defi}
(i) Let  and 
denote the view points that see all of  except  or , respectively.\\
(ii) Let  and .
\end{definition}
By Claim~\ref{newfront-claim}, the points of  contained in the triangle  are front points with respect
to , too; see Figure~\ref{front2-fig}.
\begin{claim}         \label{lr-claim}
None of the sets  are empty. The sets  and  are disjoint. 
\end{claim}
\begin{proof}
By construction, we have , , and .
If  then , obviously. Otherwise, there is at least one
even indexed point, , between  and  on . Assume 
that there exists a point  of  in the intersection of  and . Then segment
 would be encircled by the visibility chain , 
contradicting the fact that  sees every point {\em but} ; see Figure~\ref{lrdis-fig}. 
\end{proof}
\begin{figure}[hbtp]\begin{center}\includegraphics[scale=0.6,keepaspectratio]{lrdisfig}\caption{ and  are disjoint.}\label{lrdis-fig}
  \end{center}\end{figure}


The purpose of the sets  and  will now become clear: They contain points 
like  in Section~\ref{technique-sec}, that help us reduce visibility regions
to wedges. The precise property will be stated for  in Lemma~\ref{case-lem};
a symmetric property holds for . The proof of Lemma~\ref{case-lem} will be 
postponed. First, we shall derive a conclusion in Lemma~\ref{main-lem},
and use it in completing the proof of Lemma~\ref{outer-lem} in Case~1.


\begin{lemma}      \label{case-lem}
There exist points  in  such that the following holds either for  
or for .
For each  different from , 
each view point that (i) sees , (ii) lies in , and (iii) sees at least one point of , 
is contained in the half-plane .
\end{lemma}

Here,  denotes the complement of a set .
A symmetric lemma holds for points , a set 

and the half-plane . Adding up these facts yields the following.


\begin{lemma}     \label{main-lem}
Let . Then each view point in  that 
sees  lies in the wedge .
\end{lemma}

Now we can proceed as in Section~\ref{technique-sec}; see Figure~\ref{wedge-fig}~(iii) and (iv).
Within wedge  we find a sub-wedge  satisfying
 
with the same arguments that led to Fact~\ref{wedge-fact}, replacing  with
. Since membership in  only prescribes the visibility of ,
Fact~\ref{conclu} implies the following. For each subset 
 there exists a cell in the arrangement of the remaining 
six wedges , where , that is contained in precisely the wedges related to . 
As in Section~\ref{inner-sec}, this contradicts Theorem~\ref{isler-theo} and proves Lemma~\ref{outer-lem} in Case~1.
\end{proof}
It remains to show how to find  and  in Lemma~\ref{case-lem}. 

\begin{proof} (of Lemma~\ref{case-lem}) Before starting a case analysis depending on properties of
 and  we list some helpful facts.

\begin{claim}         \label{prep1-claim}
If a view point  sees a point  and a point  then
. A symmetric claim holds for .
\end{claim}
\begin{proof}
Otherwise,  would be encircled by , 
since  lies in the triangle defined by ; see Figure~\ref{prep-fig}~(i).
\end{proof}
\begin{figure*}[hbtp]\begin{center}\includegraphics[width=\textwidth,keepaspectratio]{prepfig}\caption{Illustration to Claims~\ref{prep1-claim} and~\ref{prep2-claim}.}\label{prep-fig}
  \end{center}\end{figure*}
The next fact narrows the locus from which two points, one from  and  each,
are visible. 
\begin{claim}         \label{prep2-claim}
If a view point  sees points  and  then
 lies in the wedge , and on the same side of 
as  and  do.
\end{claim}
\begin{proof}
If , or if  were situated on the opposite side of , 
then  would be encircled by ; see points  and  
in Figure~\ref{prep-fig}~(ii).
\end{proof}

Now we start on the case analysis. In each case, we need to define  and 
a set  or .
Then we must prove that the following assertion of Lemma~\ref{case-lem} holds.

{\em {\bf Assertion}\\
 If  is different from , and if  is a view point that sees  and
some  point , then .}

{\bf Case 1a:} Point set  contains at most two points.\\
We define  and let .\\
Let  and  be as in the Assertion.
If  then Claim~\ref{prep1-claim} implies . 
If  we obtain   by the first statement in Claim~\ref{prep2-claim}.
\begin{figure}[hbtp]\begin{center}\includegraphics[width=\textwidth,keepaspectratio]{case1bfig}\caption{Illustrations of Case 1b.}\label{case1b-fig}
  \end{center}\end{figure}


{\bf Case 1b:} Point set  contains more than two points, and  is tangent point of 
as seen from ; compare Figure~\ref{front-fig}.\\
We set  and let  be the odd indexed back point  counterclockwise next to .
Moreover, .\\
For each  the proof of Case~1a applies. Let  be different from .
Assume, by way of contradiction, that  holds. Since the second statement of 
Claim~\ref{prep2-claim} implies
, we obtain ;
see Figure~\ref{case1b-fig}.
Now we discuss the location of view point . If it lies in the wedge  
then segment  is encircled by ; see Figure~\ref{case1b-fig}~(i).
If  does not lie in this wedge, let  be the counterclockwise neighbor of  in .
If  lies on the same side of  as , then 
 protects segment ; see~(ii).
If it lies on the opposite side, then  intersects  
at some point , and  encircles segment~; see (iii).
In either situation, we obtain a contradiction.

Before continuing the case analysis we prove a simple fact.
\begin{lemma}          \label{tria-lem}
Let  denote the vertices of a triangle, in counterclockwise order. Suppose there exists 
a view point  in  that sees  and . Then,
each view point  that sees  and  but not  lies in .
\end{lemma}
\begin{proof}
Otherwise, segment  would be encircled by ; see Figure~\ref{tria-fig}.
\end{proof}
\begin{figure}[hbtp]\begin{center}\includegraphics[scale=0.7,keepaspectratio]{triafig}\caption{Proof of Lemma~\ref{tria-lem}}\label{tria-fig}
  \end{center}\end{figure}
{\bf Case 1c:} Point set  contains more than two points, and the 
counterclockwise neighbor, , of , is tangent point as seen from .
Let  denote the counterclockwise neighbor of , and let 
denote the view point that sees all of  except .
We consider three subcases, depending on the location of .

(1ci) If , we set 

To prove the Assertion, let , and let  be a view point that 
sees  but not , for some . 

As both  and  see  and , 
Claim~\ref{prep2-claim} implies .
The latter inclusion allows us to apply Lemma~\ref{tria-lem} to ,
which yields . Now  follows; 
see Figure~\ref{case1c-fig}~(i).
\begin{figure*}[hbtp]\begin{center}\includegraphics[width=\textwidth,keepaspectratio]{case1cfigneu}\caption{Illustrations of Case 1c.}\label{case1c-fig}
  \end{center}\end{figure*}


(1cii)  If , and if  and  are situated on opposite sides
of , we set  .\\
All points of  lie on the same side of  as .
A view point  that sees some point  and  must be in .
Otherwise,  would be encircled by ; 
see Figure~\ref{case1c-fig}~(ii).

(1ciii)  If , and if  and  are situated on the same side
of , we set  .\\
Clearly, . Each view point  that sees  and some 
---in particular point  of the Assertion--- must 
lie in , or  would be enclosed by ;
see Figure~\ref{case1c-fig}~(iii) (observe that  must be contained in 
, 
by Claim~\ref{prep2-claim}, as depicted in the figure).

Let  denote the view point that sees exactly . By Claim~\ref{prep2-claim},
.
We file for later use that  holds, for the same reason.
Since  is tangent point from , we have . Thus, we
can apply Lemma~\ref{tria-lem} to  and obtain
. 

We have just shown that  holds. Moreover,
Claim~\ref{prep2-claim} implies 
since  sees  and . Since  does not see  
we can apply Lemma~\ref{tria-lem} to  and obtain .
Together with the first finding in (1ciii), this implies
 for all .

This completes the proof of Lemma~\ref{outer-lem} in Case~1.

\end{proof}
Now we discuss the second case of Lemma \ref{outer-lem}, thereby completing its proof.
This also completes the proof of our main result, Theorem~\ref{statement-theo}.

\bigskip

{\bf Case~2: There is no odd front point.}

In this situation, view point  either lies inside , so that no front point exists, or  lies outside ,
and at most one front point is visible from  between the two tangent points on ; if so, its index is even. 

Independently of the position of ,  we introduce some notation.
Let  denote the point that sees all points in . 
The line  through  and  divides
 into two subsets,  and  (not to be confused with  and  in case 1), one of which may possibly be empty.
We cut  at , and rotate the half-line passing through  over ; see
Figure~\ref{Nomenklatur3-fig}. The first and the last odd indexed points of  encountered
during this rotational sweep are named  and , respectively. 
Similarly,  and  are defined in .

We observe that, e.~g.,  and  need not exist, or that  may hold; these cases will be taken
care of in the subsequent analysis. Also, the half-line rotating about  may cut through  in its
start position, depending on the position of . This is of no concern for our proof, which is literally the
same for either situation.
\begin{figure}[hbtp]\begin{minipage}[t]{0.4\textwidth}
\includegraphics[width=\textwidth,keepaspectratio]{Nomenklatur3.pdf}\end{minipage}
\hspace{2cm}
\begin{minipage}[t]{0.4\textwidth}
\includegraphics[width=\textwidth,keepaspectratio]{Nomenklatur4.pdf}\end{minipage}
\caption{The half-line is rotated about  over . The first odd point encountered is named , the last one . }
\label{Nomenklatur3-fig}
\end{figure}
\begin{lemma} If there are odd-indexed points in both  and , then exactly one point lies 
between  and  on the boundary of the convex hull of .
This point has even index and will be called . Similarly, there is exactly one even-indexed point between  and , 
called .
\end{lemma}

\begin{proof}
If there was more than one point on  between  and , 
one of them would have an odd index.
Let w.l.o.g  lie on the same side of  as . Being odd,  and  must be back points, since no
odd front points exist in Case~2.
As the order in which back points of  are encountered by the rotating rays 
coincides with their order on the boundary of the convex hull,
 would have to be hit by the rotating ray before , contradicting the definition of . 
The same argument applies to  and .
\end{proof}


We will deal with a somewhat special subcase first.
\vspace{0.5\baselineskip}

{\bf Case~2A:  } Both of the following properties hold.

\noindent
1. One of ,  contains exactly one point of ; its index is odd.\\
2. Point  is a front point with respect to .


\begin{figure}[hbtp]\begin{minipage}[t]{0.28\textwidth}
\begin{center}\includegraphics[width=\textwidth,keepaspectratio]{NoOddFrontpointsEvenFrontpoint1.pdf}

(i)
\end{center}\end{minipage}
\hfill
\begin{minipage}[t]{0.34\textwidth}
\begin{center}\includegraphics[width=\textwidth,keepaspectratio]{NoOddFrontpointsEvenFrontpoint2.pdf}

(ii)
\end{center}\end{minipage}
\hfill
\begin{minipage}[t]{0.34\textwidth}
\begin{center}\includegraphics[width=\textwidth,keepaspectratio]{NoOddFrontpointsEvenFrontpoint3.pdf}

(iii)
\end{center}\end{minipage}
\caption{The proof of Case 2A.}
\label{NoOddFrontpointsEvenFrontpoint-fig}
\end{figure}


In this case,  must lie outside , and  is the only front point with respect to , so that  and  
are tangent points, as seen from .  Moreover,  lies in the triangle given by  and  
because otherwise  would be a back point.


W.l.o.g., let  be the only point in , and let  be situated to the right of the directed line from  through ;
see Figure~\ref{NoOddFrontpointsEvenFrontpoint-fig}.
Then every view point  that sees  and some point  of  lies on the same side of  as  does,
or  would be encircled by . 
For the same reason every view point  that sees  and some point  of  lies on the same side of  as  does,
see Figure \ref{NoOddFrontpointsEvenFrontpoint-fig} (i).

Also,  does not see any other point , otherwise  would be encircled by . 


Now let  be the even-indexed neighbour of  that is a back point, and let us set 
. 

Next, we want to show that every view point  lies in ;
see Figure~\ref{NoOddFrontpointsEvenFrontpoint-fig}~(ii). If this were wrong,
 would imply
. Since  obviously lies in ,
we could apply Lemma~\ref{tria-lem} to , and obtain
---a contradiction.

Now let us assume that, in addition to being in , view point  sees a 
point .
As  lies in , it follows that , see~(iii). 

On the other hand,  also lies in  as already shown.

Summarizing, we have obtained a result analogous to Lemma~\ref{main-lem}. 

\begin{lemma} \label{case2A-lem}
Let . Then each view point in  that 
sees  lies in the wedge .
\end{lemma}

Now the proof of Case~2A is completed by exactly the same arguments used
subsequently to Lemma~\ref{main-lem} in Section~\ref{outer-sec}.  

\bigskip

If one of the properties of Case~2A is violated, we obtain
the following, by logical negation.
\vspace{0.5\baselineskip}

{\bf Case~2B:  } At least one of the following properties holds.

\noindent
1. None of ,  is a singleton set containing an odd indexed point.

\noindent 2. Point  is a back point, as seen from .

\medskip
Other than in the previous cases, we will now reduce visibility regions to {\em half-planes}, rather than to wedges.
We will show the existence of three points,  in , and of a halfplane  
for each, such that the following holds. 
Let  denote the set of view points that see at least .
Then,

Property~\ref{3halfplanes} leads to a contradiction, due to the following analogon of Theorem~\ref{isler-theo}.

\begin{lemma}     \label{isler-lem}
For any arrangement of three (or more) half-planes, there is a subset  of half-planes for which no cell
is contained in exactly the half-planes of .
\end{lemma}
\begin{proof}
With three half-planes, we have eight subsets, but at most seven cells.
\end{proof}
While this fact is easier to prove, and somewhat more efficient, as we need only three points to derive a contradiction, 
it is harder to find points fulfilling Property~\ref{3halfplanes}. This will be our next task.
Again, we consider points in  and points in  separately. Let us discuss the situation for .

We start by defining two points,  and . 
Suppose there is a point with odd index in . We set . 
As we are in Case~2B, point ---situated between  and ---is a back point,
or there is some point  with even index in . 
In the first case we set , in the latter case we set ; see Figure~\ref{NomenklaturCaseB-fig}~(i) and~(ii).

If there is no point with odd index in  then there are five points with odd index in . We then set  to be the second point with 
odd index that was hit during the rotation of the half-line from  through . 
Then  and   are distinct back points with respect to . 
Between  and  on the boundary of the convex hull there lies exactly one point  that has even index. We set .
In this case, there are three points with even index on the convex hull between  and .
Notice that  is a back point with respect to ; see Figure~\ref{NomenklaturCaseB-fig}~(iii).

In either case the points  and  have odd indices, and the point  has even index and is either 
a back point with respect to , or it lies in .


\begin{figure}[hbtp]\begin{minipage}[t]{0.26\textwidth}
\begin{center}\includegraphics[width=\textwidth,keepaspectratio]{NomenklaturCaseB3.pdf}

(i)
\end{center}\end{minipage}
\hfill
\begin{minipage}[t]{0.37\textwidth}
\begin{center}\includegraphics[width=\textwidth,keepaspectratio]{NomenklaturCaseB2.pdf}

(ii)
\end{center}\end{minipage}
\hfill
\begin{minipage}[t]{0.28\textwidth}
\begin{center}\includegraphics[width=\textwidth,keepaspectratio]{NomenklaturCaseB1.pdf}

(iii)
\end{center}\end{minipage}
\caption{(i) If  is a back point with respect to  we set . 
(ii) Otherwise there is an even indexed point in  we will
call .
(iii) If there is no (odd) point in R then  is the second odd indexed point and  is the (back) point between  and .}
\label{NomenklaturCaseB-fig}
\end{figure}


We will now prove the following.

\begin{lemma} \label{existenceHalfplanes-lem}
For all back points  with even index that lie in the wedge given by the rays from  through  and  the following holds.
There is a half-plane  such that every view point  that sees  and  sees  if and only if .
The analogue holds if we replace  and  by  and .
\end{lemma}
 
 Before we prove Lemma~\ref{existenceHalfplanes-lem}, we first use it to derive the following consequence. As explained before, it 
 provides us with a contradiction, thus proving Case~2B of Lemma \ref{outer-lem} and completing all proofs.
 
 \begin{lemma}
 There are three points  and half-planes  that satisfy Property~\ref{3halfplanes}.
 \end{lemma}
\begin{proof}
 If there are odd points in both  and , then there is exactly one even point between  and  and one even point 
 between  and  and all other even points lie between the rays from  through  and  and through  and , respectively. By Lemma \ref{existenceHalfplanes-lem}, we get that the remaining three even-indexed points have the desired property.
 
 If there is no odd point in  or in , then there are four even-indexed points between  and  or between  and  and therefore there are three points with the desired property between  and  or between  and .
 \end{proof}
 
 
\begin{proof}
To prove Lemma \ref{existenceHalfplanes-lem} let  be a point with even index that lies between  and . Points  and  lie on opposite sides of , by the definition of .

\begin{figure}[hbtp]\begin{minipage}[t]{0.30\textwidth}
\begin{center}\includegraphics[width=\textwidth,keepaspectratio]{CaseBIntersectClaim1.pdf}

(i)
\end{center}\end{minipage}
\hfill
\begin{minipage}[t]{0.23\textwidth}
\begin{center}\includegraphics[width=\textwidth,keepaspectratio]{CaseBIntersectClaim2.pdf}

(ii)
\end{center}\end{minipage}
\hfill
\begin{minipage}[t]{0.28\textwidth}
\begin{center}\includegraphics[width=\textwidth,keepaspectratio]{CaseBIntersectClaim3.pdf}

(iii)
\end{center}\end{minipage}
\caption{(i)  cannot lie between the rays  and . 
(ii)  and  can not lie on the same side of .
(iii) So there must be an intersection between  and .}
\label{CaseBIntersectClaim-fig}
\end{figure}

\begin{claim}In this situation the segments  and  intersect in a point . 
\end{claim}

\begin{proof}The segment  intersects the line  by definition of . It remains to show that  does neither lie on the side of  opposite to  nor on the side of  opposite to .
As  and  both belong to , the first assertion follows.
For the second one, notice that  cannot lie on the same side of  as  does because otherwise   would be encircled by , see Figure \ref{CaseBIntersectClaim-fig} (i). But  and  cannot both lie on the side of  opposite to : Because  are backpoints,  and  are the corners of a convex quadrilateral. If  and  lie on the same side of a line through , this line must be a tangent to this quadrilateral and therefore  and  would have to lie on the same side of this line, see (ii).
So the segment  crosses  in a point .
\end{proof}


\begin{figure}[hbtp]\begin{minipage}[t]{0.30\textwidth}
\begin{center}\includegraphics[width=\textwidth,keepaspectratio]{CaseBWedgeLemma1.pdf}

(i)
\end{center}\end{minipage}
\hfill
\begin{minipage}[t]{0.20\textwidth}
\begin{center}\includegraphics[width=\textwidth,keepaspectratio]{CaseBWedgeLemma2.pdf}

(ii)
\end{center}\end{minipage}
\hfill
\begin{minipage}[t]{0.28\textwidth}
\begin{center}\includegraphics[width=\textwidth,keepaspectratio]{CaseBWedgeLemma3.pdf}

(iii)
\end{center}\end{minipage}
\caption{ (i) and  must lie on the same side of .  
(ii)  and  must lie on the same side of .
(iii)  and  must lie on the same side of .}
\label{CaseBWedgeLemma-fig}
\end{figure}


Now it follows that every view point  that sees  and  lies on the same side of  as  does, because otherwise the segment  would be encircled by , see Figure \ref{CaseBWedgeLemma-fig} (i).

It also follows that every view point that sees  and  has to lie on the same side of  as  does, because otherwise  would be encircled by , see Figure \ref{CaseBWedgeLemma-fig} (ii).

\begin{claim} \label{l_1l_2-claim} Every view point  that sees  and  lies on the same side of  as  does. 
\end{claim}

\begin{proof}
Assume  and  lay on opposite sides of  . We already showed that  must lie on the same side of  as  does. So  would be encircled by , see Figure \ref{CaseBWedgeLemma-fig} (iii).
\end{proof}

\begin{lemma}\label{Case1aWedgeSummary} All view points  that see 
lie in the wedge  given by the two rays originating in  and going through  and , respectively
\end{lemma}

\begin{proof}
We just showed that all such view points  lie in the wedge  
given by the two rays originating in  and going through  and , respectively. 
As  is a subset of , the lemma follows.
\end{proof}


\begin{figure}[hbtp]\begin{minipage}[t]{0.30\textwidth}
\begin{center}\includegraphics[width=\textwidth,keepaspectratio]{CaseBHalfplaneLemma1.pdf}

(i)
\end{center}\end{minipage}
\hfill
\begin{minipage}[t]{0.20\textwidth}
\begin{center}\includegraphics[width=\textwidth,keepaspectratio]{CaseBHalfplaneLemma2.pdf}

(ii)
\end{center}\end{minipage}
\hfill
\begin{minipage}[t]{0.28\textwidth}
\begin{center}\includegraphics[width=\textwidth,keepaspectratio]{CaseBHalfplaneLemma3.pdf}

(iii)
\end{center}\end{minipage}
\caption{(i) We rotate the rays through  and  until they encounter  and .  
(ii) The area between  and .
(iii) No point  that lies in this area sees  and  but not .}
\label{CaseBHalfplaneLemma-fig}
\end{figure}

Let us now rotate the ray with origin  through   over the wedge , towards . Let us denote the first view point we encounter that sees  and   by . Let us then rotate the ray with origin  through   over the wedge , towards . Let us denote the first view point we encounter this time that sees  and   by , 
see Figure \ref{CaseBHalfplaneLemma-fig} (i).
 
We now obtain the two following facts.

\begin{claim} \label{v_1v_2-claim}
All view points that see  and  lie in the wedge originating in  and going through  and .
\end{claim}

\begin{proof}
By Lemma \ref{Case1aWedgeSummary} we know that all such points lie between  and . By construction of  and  there is no such point between  and  or between  and .
\end{proof}

\begin{lemma}\label{Case1aHalfplaneSummary}There is no view point on the side of  opposite to  that sees  and  but not .
\end{lemma}

\begin{proof}
By Claim \ref{l_1l_2-claim}, all view points that see  and  lie on the same side of  as  does.  
As  also lies on this side of the line and moreover inside the wedge between the rays from  through  and , respectively, it follows, that a point that sees  and  and that lies on the side of  opposite to  must lie in the wedge given by the rays from  through  and , which in turn is contained in the wedge between the rays from  through  and , see Figure \ref{CaseBHalfplaneLemma-fig} (ii). 

Now assume there was a view point  in this wedge, that saw  and  but not . If we take  to be the intersection of  and , then the segment  would be encircled by , 
see Figure \ref{CaseBHalfplaneLemma-fig} (iii).

\end{proof}

Now we are able to complete the proof of Lemma \ref{existenceHalfplanes-lem}.

We define  to be the closed halfplane to the side of the line through  and  in which  lies. By Claim \ref{v_1v_2-claim} all view points that see  and  lie in . Assume now there was a view point  in  
that sees  and  but not . 
By Lemma \ref{Case1aHalfplaneSummary} and the assumption that  lies in , it follows that then  must lie in the wedge with origin  and rays through  and . 

This again leads to a contradiction because the segement  then would be encircled by .
So a view point  that sees  and  sees  if and only if .
\end{proof}
Now all proofs are complete.



\section{Conclusions}      \label{conclu-sec}                                           

In his classical proof in~\cite{m-ldg-02}, Matou\v sek used
a particular type of enclosing cycle of length~4 to show that the VC-dimension of visibility
regions in simple polygons is finite (obtaining a bound in the thousands). Valtr~\cite{v-ggwps-98}
was able to prove an upper bound of~23 by combining enclosing chain arguments with a
cell decomposition technique. Our proof yields an upper bound of~14, using enclosing
cycles of length~6. The natural question is if better bounds can be obtained by considering
even more complex enclosing configurations, and if there is a systematic way to approach
this problem. One would expect that the true value of the VC-dimension is
closer to 6 than to 14.




\section{Acknowledgement}
The second author would like to thank Boris Aronov and David Kirkpatrick for interesting discussions. We would also like to thank the anonymous referees who carefully read the SoCG '11 version of this paper.

\bigskip

\bibliography{VcProc}{}
\bibliographystyle{plain}



\end{document}