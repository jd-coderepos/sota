A program is productive if it evaluates to a finite or infinite constructor normal form.
This rather vague description leaves open several choices 
that can be made to obtain a more formal definition. 
We explore several definitions and determine the degree of undecidability for each of them.
See~\cite{endr:grab:hend:isih:klop:2007}
for more pointers to the literature on productivity.

The following is a productive specification of the (infinite) stream of zeros:
\begin{align*}
  \zeros &\to \strcns{\zer}{\zeros}
\end{align*}
Indeed, there exists only one maximal 
rewrite sequence from $\zeros$ and this ends in the infinite constructor normal form 
$\strcns{0}{\strcns{0}{\strcns{0}{\ldots}}}$.
Here and later we say that a rewrite sequence $\rho : t_0 \to t_1 \to t_2 \to \ldots$ 
\emph{ends in} a term $s$ if either $\rho$ is finite with its last term being $s$,
or $\rho$ is infinite and then $s$ is the limit of the sequence of terms~$t_i$, i.e.\
$s = \lim_{i\to\infty}t_i$.
We consider only rewrite sequences starting from finite terms, thus all
terms occurring in $\rho$ are finite. Nevertheless, the limit $s$ of the terms $t_i$
may be an infinite term.
Note that, if $\rho$ ends in a constructor normal form, 
then every finite prefix will be evaluated after finitely many steps.

The following is a slightly modified specification of the stream of zeros:
\begin{align*}
  \zeros &\to \strcns{\zer}{\funap{\strff{id}}{\zeros}} & \funap{\strff{id}}{\astr} &\to \astr
\end{align*}
This specification is considered productive as well,
although there are infinite rewrite sequences that do not even end in a normal form,
let alone in a constructor normal form: e.g.\ by 
unfolding $\zeros$ only we get the limit term 
$\strcns{0}{\funap{\strff{id}}{\strcns{0}{\funap{\strff{id}}{\strcns{0}{\funap{\strff{id}}{\ldots}}}}}}$.
In general, normal forms can only be reached by outermost-fair rewriting sequences.
A rewrite sequence $\rho : t_0 \to t_1 \to t_2 \to \ldots$ is \emph{outermost-fair}~\cite{terese:2003}
if there is no $t_n$ containing an outermost redex which remains an outermost redex infinitely long,
and which is never contracted.
For this reason it is natural to consider productivity of terms with respect to outermost-fair strategies.

What about stream specifications that admit rewrite sequences to constructor normal forms, 
but that also have divergent rewrite sequences:
\begin{align*}
  \maybe &\to \strcns{0}{\maybe} &
  \maybe &\to \sink &
  \sink &\to \sink
\end{align*}
This example illustrates that, for non-orthogonal stream specifications,
reachability of a constructor normal form depends on the evaluation strategy.
The term $\maybe$ is only productive
with respect to strategies that always apply the first rule.

For this reason we propose to think of productivity as
a property of individual terms 
\emph{with respect to} a given rewrite strategy.
This reflects the situation in functional programming,
where expressions are evaluated according to an inbuilt strategy.
These strategies are usually based on a form of outermost-needed rewriting 
with a priority order on the rules.


\subsection{Productivity with respect to Strategies}
  \label{sec:productivity:subsec:strategies}

For term rewriting systems (TRSs) \cite{terese:2003}
we now fix definitions of the notions of 
(history-free) strategy and history-aware strategy.
Examples for the latter notion are outermost-fair strategies,
which typically have to take history into account.


\begin{definition}\normalfont
Let $\atrs$ be a TRS with rewrite relation $\redR{\atrs}$.

  A \emph{strategy for $\redR{\atrs}$} is a relation 
  $\sastrat\subseteq\sredR{\atrs}$ with the same normal forms
  as $\redR{\atrs}$.

  The \emph{history-aware rewrite relation $\sredh{\atrs}$ for $\atrs$}
  is the binary relation on $\ter{\asig} \times (\atrs \times \nat^*)^*$ 
  that is defined by:
\begin{align*}
\pair{s}{\ahist_s} \redh{\atrs} \pair{t}{\lstcns{\ahist_s}{\pair{\rho}{\apos}}} &\Longleftrightarrow
    s \red t \text{ via rule $\rho \in \atrs$ at position $\apos$}
    \punc.
\end{align*}
We identify $t \in \ter{\asig}$ with $\pair{t}{\posemp}$,
  and for $s,t \in \ter{\asig}$ we write $s \redh{\atrs} t$ 
  whenever $\pair{s}{\posemp} \redh{\atrs} \pair{t}{\ahist}$ for some history $h \in (\atrs \times \nat^*)^*$.
A \emph{history-aware strategy for $\atrs$} is a strategy for
  $\sredh{\atrs}$.

  A strategy $\astrat$ is \emph{deterministic} if 
  $s \astrat t$ and $s \astrat t'$ implies $t = t'$. 
  A strategy $\astrat$ is \emph{computable} if the function mapping
  a term (a term/history pair) to its set of $\astrat$-successors
  is a total recursive function, after coding into natural numbers. 
\end{definition}

\begin{remark}
Our definition of strategy for a rewrite relation
  follows \cite{toya:1992}.
  For abstract rewriting systems, 
  in which rewrite steps are first-class citizens,
  a definition of strategy is given in \cite[Ch.\,9]{terese:2003}.
  There, history-aware strategies for a TRS~$\atrs$ are defined in terms of 
  `labellings' for the `abstract rewriting system' underlying $\atrs$.
  While that approach is conceptually advantageous, our definition
  of history-aware strategy is equally expressive.  
\end{remark}

\begin{definition}\normalfont
A \emph{(TRS-indexed) family of strategies $\stratfam$} 
  is a function that assigns to every TRS $\atrs$ 
  a set $\funap{\stratfam}{\atrs}$  of strategies for $\atrs$.
We call such a family $\stratfam$ of strategies \emph{admissible} 
  if $\funap{\stratfam}{\atrs}$ 
  is non-empty for every orthogonal TRS $\atrs$.
\end{definition}

\newcommand{\strategy}{\leadsto}

Now we give the definition of productivity with respect to a strategy.
\begin{definition}\normalfont\label{def:productivity}
  A term $t$ is called \emph{productive with respect to a strategy $\strategy$}
  if all maximal $\strategy$ rewrite sequences starting from $t$ end in a constructor normal form.
\end{definition}
In the case of non-deterministic strategies we require here that all maximal rewrite sequences end in a constructor normal form.
Another possible choice could be to require only the existence of 
one such rewrite sequence (see Section~\ref{sec:productivity:subsec:strong}).
However, we think that productivity should be a practical notion.
Productivity of a term should entail that arbitrary finite parts of the constructor normal form can indeed be evaluated.
The mere requirement that a constructor normal form exists
leaves open the possibility that such a normal form 
cannot be approximated to every finite precision in a computable way.

For orthogonal TRSs outermost-fair (or fair) rewrite strategies
are the natural choice for investigating productivity
because
they guarantee to find (the unique) infinitary constructor normal form whenever it exists (see~\cite{terese:2003}).

Pairs and finite lists of natural numbers can be encoded using the well-known G\"{o}del encoding.
Likewise terms and finite TRSs over a countable set of variables
can be encoded.
A TRS is called \emph{finite} if its signature and set of rules
are finite.
In the sequel we restrict to (families of) computable strategies,
and assume that strategies are represented by
appropriate encodings.

Now we define the productivity problem in TRSs with respect to families
of computable strategies, and prove a $\cpi{0}{2}$-completeness result.


\paragraph{\textsc{Productivity Problem} \normalfont 
  with respect to a family $\stratfam$ of computable strategies}\ 

\emph{Instance}:
  Encodings of a finite TRS $\atrs$, 
  a strategy ${\strategy} \in \funap{\stratfam}{\atrs}$ 
  and a term $t$.

\emph{Answer}:
  `Yes' if $t$ is productive with respect to $\strategy$. 

\begin{theorem}\label{thm:strategy}
  For every family of admissible, computable strategies $\stratfam$,
  the productivity problem with respect to $\stratfam$ 
  is $\cpi{0}{2}$-complete.
\end{theorem}

\begin{proof}{
  \newcommand{\from}{\funap{\msf{from}}}
  \newcommand{\run}{\funap{\msf{M}}}
  A (deterministic) Turing machine is called \emph{total} (encodes a total function $\nat \to \nat$) 
  if it halts on all inputs encoding natural numbers.
  The problem of deciding whether a deterministic Turing machine is \emph{total} is well-known to be $\cpi{0}{2}$-complete, see~\cite{hinman:1978}.
  Let $\tm$ be an arbitrary Turing machine.
  Employing the encoding of deterministic Turing machines into orthogonal TRSs from~\cite{jw:handbook},
  we can define a TRS $\atrs_\tm$ that simulates $\tm$ such that
  for every $n \in \nat$ it holds:
  every reduct of the term $\run{\funap{\ssuc^n}{\zer}}$ contains at most one redex occurrence,
  and the term $\run{\funap{\ssuc^n}{\zer}}$ rewrites to $\zer$ 
  if and only if the Turing machine $\tm$ halts on the input $n$.
  Note that the rewrite sequence starting from $\run{\funap{\ssuc^n}{\zer}}$ is deterministic.
  We extend the TRS $\atrs_\tm$ to a TRS $\atrs'_\tm$ with the following rule:
\begin{align*}
\funap{\strff{go}}{\zer,x} &\to \strcns{\zer}{\funap{\strff{go}}{\run{x},\suc{x}}}
\end{align*}
and choose the term $t = \funap{\strff{go}}{\zer,\zer}$.
  Then $\atrs'_\tm$ is orthogonal and 
  by construction every reduct of $t$ contains at most one redex occurrence
  (consequently all strategies for $\atrs$ coincide on every reduct of $t$).
  The term $t$ is productive
  if and only if
  $\run{\funap{\ssuc^n}{\zer}}$ rewrites to $\zer$ for every $n \in \nat$
  which in turn holds if and only if
  the Turing machine $\tm$ is total.
  This concludes $\cpi{0}{2}$-hardness.

  For $\cpi{0}{2}$-completeness let $\stratfam$ be a family of computable strategies,
  $\atrs$ a TRS,~${\strategy} \in \funap{\stratfam}{\atrs}$ and $t$ a term. 
  Then productivity of $t$ can be characterised as:
\begin{equation}
    \tag{$\star$}\label{PI02form:thm:strategy}
\begin{aligned}
      \myall{d \in \nat}{\myex{n \in \nat}{&\text{every $n$-step $\strategy$-reducts of $t$}}}\\[-0.5ex]
      &\text{is a constructor normal form up to depth $d$}
\end{aligned}
\end{equation}
Since the strategy ${\strategy}$ is computable and finitely branching, 
  all $n$-step reducts of $t$ can be computed.
  Obviously, if the formula \eqref{PI02form:thm:strategy}
  holds, then $t$ is productive w.r.t.\ $\strategy$.
  Conversely, assume that $t$ is productive w.r.t.\ $\strategy$.
  For showing \eqref{PI02form:thm:strategy}, let $d\in\nat$ be arbitrary.
  By productivity of $t$ w.r.t.\ $\strategy$, on every path in the reduction 
  graph of $t$ w.r.t.\ $\strategy$ eventually a term with a constructor normal
  form up to depth $d$ is encountered. Since reduction graphs in TRSs
  always are finitely branching, Koenig's lemma implies that there 
  exists an $n\in\nat$ such that all terms on depth greater or equal to $n$ 
  in the reduction graph of $t$ are constructor prefixes of depth at least $d$. 
  Since $d$ was arbitrary, \eqref{PI02form:thm:strategy} has been established. 
Because \eqref{PI02form:thm:strategy} is a $\cpi{0}{2}$-formula, 
  the productivity problem with respect to $\stratfam$
  also belongs to $\cpi{0}{2}$.
\qed
}\end{proof}


\begin{corollary}
  In orthogonal TRSs, productivity is $\cpi{0}{2}$-complete 
  with respect to outermost-fair rewriting. 
\end{corollary}
\begin{proof}
  Apply Theorem~\ref{thm:strategy} to the family of strategies 
  that assigns to every orthogonal TRS $\atrs$
  the set of computable, outermost-fair rewriting strategies for $\atrs$,
  and $\setemp$ to non-orthogonal~TRSs.
\end{proof}

The definition of productivity with respect to computable strategies
reflects the situation in functional programming.
Nevertheless, we now investigate variants of this notion, and determine
their respective computational complexity.



\subsection{Strong Productivity}\label{sec:productivity:subsec:strong}

As already discussed, only outermost-fair rewrite sequences can reach a constructor normal form.
Dropping the fine tuning device `strategies', 
we obtain the following stricter notion of productivity.
\begin{definition}\normalfont\label{def:strongproductivity}
A term $t$ is called \emph{strongly productive}
  if all maximal outermost-fair rewrite sequences starting from $t$ end in a constructor normal form.
\end{definition}
The definition requires all outermost-fair rewrite sequences to end in a constructor normal form,
including non-computable rewrite sequences.
This catapults 
productivity into a much higher class of undecidability:
$\cpi{1}{1}$, a class of the analytical hierarchy.
The analytical hierarchy continues the classification
of the arithmetical hierarchy using second order formulas.
The computational complexity of strong productivity therefore
exceeds the expressive power of first-order logic 
to define sets from recursive sets.

A well-known result of recursion theory states
that for a given computable relation ${>} \subseteq \nat \times \nat$
it is $\cpi{1}{1}$-hard to decide whether $>$ is well-founded, see~\cite{hinman:1978}.
Our proof is based on a construction from~\cite{endr:geuv:zant:2009}.
There a translation from deterministic Turing machines $\tm$ to TRSs $\roottrs$ (which we explain below)
together with a term $t_\tm$ 
is given such that: $t_\tm$ is root-terminating (i.e., $t_\tm$
admits no rewrite sequences containing an infinite number of root steps)
if and only if the binary relation $>_\tm$ encoded by $\tm$ is well-founded.
The TRS $\roottrs$ consists of rules
for simulating the Turing machine $\tm$ such that $\bfunap{\tm}{x}{y} \red^* \tmT$ iff $x >_\tm y$ holds
(which basically uses a standard encoding of deterministic Turing machines, see~\cite{jw:handbook}), a rule:
\begin{gather*}
  \tfunap{\msf{run}}{\tmT}{\picknok{x}}{\picknok{y}} \to \tfunap{\msf{run}}{\bfunap{\tm}{x}{y}}{\picknok{y}}{\pickn}
\end{gather*}
and rules for randomly generating a natural number:
\begin{align*}
\pickn &\to \fc{\pickn} &\quad
\pickn &\to \picknok{\tmzer{\tmiblank}} &\quad
\fc{\picknok{x}} &\to \picknok{\fs{x}}\punc.
\end{align*}
The term $t_\tm = \tfunap{\msf{run}}{\tmT}{\pickn}{\pickn}$ admits a rewrite sequence containing infinitely
many root steps if and only if $>_\tm$ is not well-founded.
More precisely, whenever there is an infinite decreasing sequence $x_1 >_\tm x_2 >_\tm x_3 >_\tm\ldots$,
then $t_\tm$ admits a rewrite sequence
$\tfunap{\msf{run}}{\tmT}{\pickn}{\pickn} 
 \red^* \tfunap{\msf{run}}{\tmT}{\picknok{x_1}}{\picknok{x_2}}
 \red \tfunap{\msf{run}}{\bfunap{\tm}{x_1}{x_2}}{\picknok{x_2}}{\pickn}
 \red^* \tfunap{\msf{run}}{\tmT}{\picknok{x_2}}{\picknok{x_3}}
 \red^* \ldots$.
We further note that $t_\tm$ and all of its reducts contain
exactly one occurrence of the symbol $\msf{run}$, namely at the root position.

\begin{theorem}\label{thm:omfair}
  The recognition problem for strong productivity is $\cpi{1}{1}$-complete.
\end{theorem}
\begin{proof}
  For the proof of $\cpi{1}{1}$-hardness, let $\tm$ be a deterministic
  Turing machine.
  We extend the TRS $\roottrs$ from~\cite{endr:geuv:zant:2009}
  with the rule $\funap{\msf{run}}{x,y,z} \to \strcns{0}{\funap{\msf{run}}{x,y,z}}$.
  As a consequence the term $\tfunap{\msf{run}}{\tmT}{\pickn}{\pickn}$ is strongly productive
  if and only if ${>_\tm}$ is well-founded (which is $\cpi{1}{1}$-hard to decide).
  If $>_\tm$ is not well-founded, then by the result in~\cite{endr:geuv:zant:2009}
  $t_\tm$ admits a rewrite sequence containing infinitely many root steps which obviously does not end in a constructor normal form.
  On the other hand if $>_\tm$ is well-founded, then $t_\tm$ admits only finitely many root steps with respect to $\roottrs$,
  and thus by outermost-fairness the freshly added rule has to be applied infinitely often.
  This concludes $\cpi{1}{1}$-hardness.

  Rewrite sequences of length $\omega$ can be represented by functions $r \funin \nat \to \nat$
  where $\funap{r}{n}$ represents the $n$-th term of the sequence together with the position and rule
  applied in step $n$. 
  Then for all $r$ (one universal $\forall r$ function quantifier) we have to check
  that $r$ converges towards a constructor normal form whenever $r$ is outermost-fair;
  this can be checked by a first order formula.
  We refer to~\cite{endr:geuv:zant:2009} for the details of the encoding.
  Hence strong productivity is in $\cpi{1}{1}$.
  \qed
\end{proof}

\subsection{Weak Productivity}\label{sec:productivity:subsec:weak}

A natural counterpart to strong productivity is the notion of `weak productivity':
the existence of a rewrite sequence to a constructor normal form.
Here outermost-fairness does not need to be required, 
because rewrite sequences that reach normal forms
are always outermost-fair. 


\begin{definition}\normalfont\label{def:weakproductivity}
A term $t$ is called \emph{weakly productive}
  if there exists a rewrite sequence starting from $t$ that ends in a constructor normal form.
\end{definition}
For non-orthogonal TRSs the practical relevance of this definition is questionable
since, in the absence of a computable strategy to reach normal forms,
mere knowledge that a term $t$ is productive does typically not help
to find a constructor normal form of $t$. 
For orthogonal TRSs computable, normalising strategies exist,
but then also all of the variants of productivity coincide
(see Section~\ref{sec:productivity:subsec:discussion}).

\begin{theorem}\label{thm:weak}
  The recognition problem for weak productivity is $\csig{1}{1}$-complete.
\end{theorem}
\begin{proof}
  For the proof of $\csig{1}{1}$-hardness, let $\tm$ be a Turing machine.
  We exchange the rule
  $\tfunap{\msf{run}}{\tmT}{\picknok{x}}{\picknok{y}} \to \tfunap{\msf{run}}{\bfunap{\tm}{x}{y}}{\picknok{y}}{\pickn}$
  in the TRS $\roottrs$ from~\cite{endr:geuv:zant:2009}
  by 
  the rule $\tfunap{\msf{run}}{\tmT}{\picknok{x}}{\picknok{y}} \to \strcns{0}{\tfunap{\msf{run}}{\bfunap{\tm}{x}{y}}{\picknok{y}}{\pickn}}$.
  Then we obtain that the term $\tfunap{\msf{run}}{\tmT}{\pickn}{\pickn}$ is weakly productive
  if and only if ${>_\tm}$ is not well-founded (which is $\csig{1}{1}$-hard to decide).
  This concludes $\csig{1}{1}$-hardness.

  The remainder of the proof proceeds analogously
  to the proof of Theorem~\ref{thm:omfair},
  except that we now have an existential function quantifier $\exists r$ 
  to quantify over all rewrite sequences of length $\omega$.
  Hence weak productivity is in $\csig{1}{1}$.
  \qed
 \end{proof}


\subsection{Discussion}\label{sec:productivity:subsec:discussion}

For orthogonal TRSs all of the variants of productivity coincide.
That is, if we restrict the first variant to computable outermost-fair strategies;
as already discussed, other strategies are not very reasonable.
For orthogonal TRSs there always exist computable outermost-fair strategies,
and whenever for a term there exists a constructor normal form, 
then it is unique and all outermost-fair rewrite sequences will end in this unique constructor normal form.

This raises the question whether uniqueness of the constructor normal forms
should be part of the definition of productivity.
We consider a specification of the stream of random bits:
\begin{align*}
  \random &\to \strcns{0}{\random} &
  \random &\to \strcns{1}{\random}
\end{align*}
Every rewrite sequence starting from $\random$ ends in a normal form. 
However, these normal forms are not unique. In fact, there are uncountably many of them.
We did not include uniqueness of normal forms
into the definition of productivity since
non-uniqueness only arises in non-orthogonal TRSs 
when using non-deterministic strategies.
However, one might want to require uniqueness of normal forms even in the case of non-orthogonal TRSs.

\begin{theorem}\label{thm:unique}
The problem of determining, for TRSs~$\atrs$ and terms $t$ in $\atrs$, 
  whether $t$ has a unique (finite or infinite) normal form 
  is $\cpi{1}{1}$-complete.
\end{theorem}

\begin{proof}
  For $\cpi{1}{1}$-hardness, we extend the TRS constructed in the proof of Theorem~\ref{thm:weak} by the rules:
  $\msf{start} \to \tfunap{\msf{run}}{\tmT}{\pickn}{\pickn}$,
  $\tfunap{\msf{run}}{x}{y}{z} \to \tfunap{\msf{run}}{x}{y}{z}$,
  $\msf{start} \to \strff{ones}$, and
  $\strff{ones} \to \strcns{1}{\strff{ones}}$.
  Then $\msf{start}$ has a unique normal form 
  if and only if $>_\tm$ is well-founded.
  For $\cpi{1}{1}$-completeness, we observe that
  the property can be characterised by a $\cpi{1}{1}$-formula:
  we quantify over two infinite rewrite sequences, 
  and, in case both of them end in a normal form, we compare them.
  Note that consecutive universal quantifiers can be compressed into one. \qed
\end{proof}

Let us consider the impact on computational complexity 
of taking up the condition of uniqueness of normal forms
into the definition of productivity.
Including uniqueness of normal forms without considering the strategy
would increase the complexity of productivity with respect to a family of strategies to $\cpi{1}{1}$.
However, we think that doing so would be contrary to the spirit 
of the notion of productivity.
Uniqueness of normal forms should only be required 
for the normal forms
reachable by the given (non-deterministic) strategy.
But then the complexity of productivity remains unchanged,
$\cpi{0}{2}$-complete.
The complexity of strong productivity
remains unaltered, $\cpi{1}{1}$-complete,
when including uniqueness of normal forms.
However, the degree of undecidability of weak productivity increases.
From the proofs of Theorems~\ref{thm:weak} and~\ref{thm:unique}
it follows that the property would then both be
$\csig{1}{1}$-hard and $\cpi{1}{1}$-hard,
then being in $\cdel{1}{1}$.

