\documentclass[fleqn,12pt,twoside]{article}
\usepackage{tipa}



\usepackage[headings]{espcrc1}
\readRCS

\ProvidesFile{espcrc1.tex}[\filedate \space v\fileversion
     \space Elsevier 1-column CRC Author Instructions]



\usepackage{graphicx}
\usepackage[figuresright]{rotating}

\newtheorem{theorem}{Theorem}
\newtheorem{algorithm}{Algorithm}
\newtheorem{axiom}{Axiom}
\newtheorem{case}{Case}
\newtheorem{claim}{Claim}
\newtheorem{conclusion}[theorem]{Conclusion}
\newtheorem{condition}[theorem]{Condition}
\newtheorem{conjecture}{Conjecture}
\newtheorem{corollary}{Corollary}
\newtheorem{criterion}[theorem]{Criterion}
\newtheorem{definition}{Definition}
\newtheorem{example}{Example}
\newtheorem{exercise}[theorem]{Exercise}
\newtheorem{lemma}{Lemma}
\newtheorem{notation}[theorem]{Notation}
\newtheorem{problem}{Problem}
\newtheorem{proposition}{Proposition}
\newtheorem{remark}{Remark}
\newtheorem{solution}{Solution}
\newtheorem{summary}[theorem]{Summary}

\newenvironment{dedication}[1][]{\begin{trivlist}
\item[\hskip \labelsep {\bfseries #1}]}{\end{trivlist}}

\newenvironment{proof}[1][Proof.]{\begin{trivlist}
\item[\hskip \labelsep {\bfseries #1}]}{\end{trivlist}}

\newenvironment{acknowledgement}[1][Acknowledgement]{\begin{trivlist}
\item[\hskip \labelsep {\bfseries #1}]}{\end{trivlist}}

\newcommand{\ttbs}{\char'134}
\newcommand{\AmS}{{\protect\the\textfont2
  A\kern-.1667em\lower.5ex\hbox{M}\kern-.125emS}}

\hyphenation{author another created financial paper re-commend-ed Post-Script}

\usepackage{amsmath}
\usepackage{amsfonts}
\title{On disjoint matchings in cubic graphs}

\author{Vahan V. Mkrtchyan\address[MCSD]{Department of Informatics and Applied Mathematics,\\
Yerevan State University, Yerevan, 0025, Armenia}\address{Institute for Informatics and Automation Problems,\\
National Academy of Sciences of Republic of Armenia, 0014, Armenia}
\thanks{The author is supported by a grant of Armenian National Science and
Education Fund}
\thanks{email: vahanmkrtchyan2002@\{ysu.am, ipia.sci.am,
yahoo.com\}},
        Samvel S. Petrosyan\addressmark[MCSD]\thanks{email: samvelpetrosyan2008@yahoo.com
},
                and
        Gagik N. Vardanyan\addressmark[MCSD]\thanks{email: vgagik@gmail.com.}}



\runtitle{On disjoint matchings in cubic graphs} \runauthor{Vahan
Mkrtchyan, Samvel Petrosyan, Gagik Vardanyan}

\begin{document}

\maketitle

\begin{abstract}
For  and a cubic graph  let  denote the
maximum number of edges that can be covered by  matchings. We
show that 
and .
Moreover, it turns out that .
\end{abstract}



\section{Introduction}

In this paper graphs are assumed to be finite, undirected and without
loops, though they may contain multiple edges. We will also consider
pseudo-graphs, which, in contrast with graphs, may contain loops.
Thus graphs are pseudo-graphs. We accept the convention that a loop
contributes to the degree of a vertex by two.

The set of vertices and edges of a pseudo-graph  will be denoted
by  and , respectively. We also define:  and
. We will also use the following scheme for notations:
if  is a pseudo-graph and  is a graph-theoretic
parameter, we will write just  instead of . So, for
example, if we would like to deal with the edge-set of a
pseudo-graph , we will write  instead of
; moreover we will write  for the
number of edges in this graph.

A connected -regular graph with at least two vertices will be
called a \textit{cycle}. Thus, a loop is not considered to be a
cycle in a pseudo-graph. Note that our notion of cycle differs from
the cycles that people working on nowhere-zero flows and cycle
double covers are used to deal with.

The length of a path or a cycle is the number of edges lying on it.
The path or cycle is even (odd) if its length is even (odd). Thus,
an isolated vertex is a path of length zero, and it is an even path.

For a graph  let  and 
denote the maximum and minimum degrees of vertices in ,
respectively. Let  denote the
chromatic class of the graph .

The classical theorem of Shannon states:

\begin{theorem}
(Shannon \cite{Shannon}). For every graph 
\end{theorem}

In 1965 Vizing proved:

\begin{theorem}
(Vizing, \cite{Vizing}): , where  denotes the maximum multiplicity of an edge in .
\end{theorem}

Note that Shannon's theorem implies that if we consider a cubic
graph , then , thus  can take
only two values. In 1981 Holyer proved that the problem of deciding whether  or not for cubic graphs  is NP-complete \cite {Holyer}, thus the calculation of  is already hard
for cubic graphs.

For a graph  and a positive integer  define

\begin{center}
 are pairwise
edge-disjoint matchings of ,
\end{center}

and let

\begin{center}
.
\end{center}

Define:

\begin{center}
  and
.
\end{center}

If  denotes the cardinality of the largest matching of ,
then it
is clear that  for all  and . Moreover,  for all . Let us also note that  and  coincide with
.

In contrast with the theory of -matchings, where every graph 
admits a maximum -matching that includes a maximum matching
\cite{Lov}, there are graphs that do not have a \textquotedblleft
maximum\textquotedblright\ pair of disjoint matchings (a pair
 with ) that includes a maximum matching.

The following is the best result that can be stated about the ratio
 for any graph  (see \cite{FiveFourth}):

Very deep characterization of graphs  satisfying  is given in \cite{FivefourthCharacter}.

Let us also note that by Mkrtchyan's result \cite{MPP0-1},
reformulated as in \cite{HararyPlummer}, if  is a matching
covered tree, then . Note that a graph is said to be matching covered (see \cite {Perfect}), if its every edge belongs to a maximum matching (not
necessarily a perfect matching as it is usually defined, see e.g.
\cite{Lov}).

The basic problem that we are interested is the following: what is
the proportion of edges of an -regular graph (particularly, cubic
graph), that we can cover by its  matchings? The formulation of
our problem stems from the recent paper \cite{KaiserKralNorine},
where the authors investigate the proportion of edges of a
bridgeless cubic graph that can be covered by  of its perfect
matchings.

The aim of the present paper is the investigation of the ratios  (or equivalently, ) in the class of cubic graphs for . Note that for cubic graphs  Shannon's theorem implies
that  .

The case  has attracted much attention in the literature. See \cite {Hobbs} for the investigation of the ratio in the class of simple
cubic graphs, and
\cite{Bella,HenningYeo,TakaoBaybars,Takao,Weinstein} for the
general case. Let us also note that the relation between  and  has also been investigated in the regular
graphs of high girth \cite{GirthBound}.

The same is true for the case . Albertson and Haas
investigate these ratios in the class of simple cubic graphs (i.e.
graphs without multiple edges)in \cite
{AlbertsonHaasFirst,AlbertsonHaasSecond}, and Steffen investigates
the general case in \cite{Steffen}.

\section{Some auxiliary results}

If  is a pseudo-graph, and  is an edge of , then -subdivision of the edge  results a new pseudo-graph  which
is obtained from  by replacing the edge  with a path 
of length , for which . Usually, we
will say that  is obtained from  by -subdividing the edge
.

If  is a path or cycle of a pseudo-graph , and the pseudo-graph  is obtained from  by -subdividing the edge , then
sometimes we will speak about the path or cycle  corresponding
to , which roughly, can be defined as , if  does not lie on
, and
the path or cycle obtained from  by replacing its edge  with the path , if  lies on .

Our interest towards subdivisions is motivated by the following

\begin{proposition}
\label{CubicPseudoGraph}Let  be a connected graph with . Then, there exists a connected cubic pseudo-graph  and a mapping , such that  is
obtained from  by -subdividing each edge ,
where  is the set of non-negative integers.
\end{proposition}
\begin{proof}The existence of such a cubic pseudo-graph  can be verified,
for example, as follows; as the vertex-set of , we take the set
of vertices of  having degree three, and connect two vertices
 of  by an edge , if these vertices are connected
by a path  of length  in , whose end-vertices are
 and , and whose internal vertices are of degree two. We also
define . Finally, if a vertex  of  lies on a cycle
 of length  in , whose all vertices, except , are
of degree two, then in  we add a loop  incident to , and
define . Now, it is not hard to verify, that  is a
cubic pseudo-graph, and if we -subdivide each edge  of
, then the resulting graph is isomorphic to .
\end{proof}

Let  be a cubic pseudo-graph, and let  be a loop of . Let  be the edge of  adjacent to  (note that  is not a
loop). Let  be\ the vertex of  that is incident to 
and , and
let . Assume that  is not incident to a loop of , and let  and  be the other () edges of  incident to , and assume  and  be the endpoints
of  and , that are not incident to , respectively.
Consider the cubic pseudo-graph  obtained from  as
follows ((a)
of figure \ref{loopcut}):

\begin{figure}[h]
\begin{center}
\includegraphics{LoopCut.eps}\\
\caption{Cutting a loop }\label{loopcut}
\end{center}
\end{figure}


Note that  and  may coincide. In this case  is a loop of . We will say that  is obtained from  by
cutting the loop .

People dealing especially with bridgeless cubic graphs would have
already recognized Fleischner's splitting off operation. Completely
realizing this, we would like to keep the name "cutting the loops",
in order to keep the basic idea, that has helped us to come to its
definition!

\begin{remark}
\label{SuccessiveCut}If  is a connected cubic pseudo-graph,
then the successive cut of loops of  in any order of loops
leads either to a connected graph (that is, connected pseudo-graph
without loops), or to the cubic pseudo-graph shown on the figure
\ref{TrivialCase}. Sometimes, we will prefer to restate this
property in terms of applicability of the operation of cutting the
loop. More specifically, if  is a connected cubic
pseudo-graph, for which the operation of cutting the loop is not
applicable, then either  does not contain a loop or it is the
mentioned trivial graph.
\end{remark}

\begin{figure}[h]
\begin{center}
\includegraphics[height=5pc,width=15pc]{TrivialCase.eps}\\
\caption{The trivial case}\label{TrivialCase}
\end{center}
\end{figure}

Before we move on, we would like to state some properties of the
operation of cutting the loops.

\begin{proposition}
\label{ConnectedInvariance}If  is connected, then  is
connected, too.
\end{proposition}

\begin{proposition}
\label{CycleInvariance}If a connected cubic pseudo-graph 
contains a cycle, and a cubic pseudo-graph  obtained from
 by cutting a loop  of  does not, then  is
adjacent to an edge , which, in its turn, is adjacent to two
edges  and , that form the only cycle of  with length
two ((b) of figure \ref{loopcut}).
\end{proposition}

The following will be used frequently:

\begin{proposition}
\label{FractionInequality}Let be  be positive numbers with , . Then:
\end{proposition}

\begin{proposition}
\label{LinearInequality}Suppose that    and . Then:
\end{proposition}

\begin{proof}
Note that
thus
or
\end{proof}

\begin{theorem}\label{Gallai}(Gallai \cite{Lov})Let  be a connected graph
with  for any . Then  is factor-critical,
and particularly:

\end{theorem}
Terms and concepts that we do not define can be found in
\cite{Harary,Lov,West}.

\section{Maximum matchings and unsaturated vertices}

In this section we prove a lemma, which states that, under some
conditions, one can always pick up a maximum matching of a graph,
such that the unsaturated vertices with respect to this matching
"are not placed very close".

Before we present our result, we would like to deduce a lower bound
for  in the class of regular graphs using the theorem
\ref{Gallai} of Gallai.

Observe that Shannon's theorem implies that  for every cubic graph , thus . Now, it turns out, that there are no cubic
graphs , for which , thus .
Next we
prove a generalization of this statement, that originally appeared in \cite {Monthly} as a problem:

\begin{lemma}
\label{OddRegulars}

\begin{description}
\item[(a)] No -regular graph  contains  pairwise
edge-disjoint maximum matchings;

\item[(b)] If  is a connected simple -regular graph with 
pairwise edge-disjoint maximum matchings, then  is even and 
is the complete graph.
\end{description}
\end{lemma}

\begin{proof}
(a) Assume  to contain  pairwise edge-disjoint maximum matchings . Note that we may assume  to be connected.
Clearly, for every  there is  such that  does not saturate the
vertex . By a theorem \ref{Gallai} of Gallai, it follows that , that is,  is odd, which is impossible.

(b) Assume  to contain  pairwise edge-disjoint maximum matchings . (a) implies that  and  is even. Since, by Vizing's theorem , we have
thus

or

hence  is the complete graph.
\end{proof}

\begin{remark}
As the example of the "fat triangle" shows, the complete graph with
odd number of vertices is not the only graph, that prevents us to
generalize (a) to even regular graphs.
\end{remark}

Next we prove the main result of the section, which is interesting
not only on its own, but also will help us to derive better bounds
in the theorem \ref{MainTheoremCubics}.

\begin{lemma}
\label{Max Matching 2-3} Every graph , with , contains a maximum matching, such that the
unsaturated vertices (with respect to this maximum matching) do not
share a neighbour.
\end{lemma}

\begin{proof}
Let  be a maximum matching of , for which there are minimum
number of pairs of unsaturated vertices, which have a common
neighbour. The lemma will be proved, if we show that this number is
zero.

Suppose that there are vertices  and  of  which are not
saturated (by ) and have a common neighbour . Clearly,  is
saturated by an edge . Consider the edge . Note
that it lies in a maximum matching of  (an example of such a
maximum matching is ). Moreover,
for every maximum matching 
of  with , the alternating component  of  which contains the edge , is a path of
even length. Now, choose a maximum matching  of 
containing the edge  for which the length of  is maximum.

Let  be the other () end-vertex of the path . Note
that since  is even, there is a vertex  of  such that .

\begin{claim}
\label{Neighbours V}The neighbours of  lie on  and are
different from  and .
\end{claim}

\begin{proof}
First of all let us show that the neighbours of  lie on .
On the opposite assumption, consider a vertex  which is
adjacent to  and which does not lie on . Clearly
. As  is a
maximum matching, there is an edge  incident to
. Define:
Note that  is a maximum matching of  with
 for which the length of the alternating component of , which contains the edge ,
exceeds the length of  contradicting the choice of . Thus the
neighbours of  lie on . Let us show that they are different from  and . If there is an edge  connecting the vertices  and , then define:
Clearly,  is a matching of  for which ,
which is impossible. Thus, there are no edges connecting  and . As  is adjacent to  and ,  can be adjacent to  if and
only if , that is, if the length of  is two. But this is
impossible, too, since , hence there should be an
edge connecting  and . The proof of claim \ref{Neighbours V}
is completed.
\end{proof}

\begin{corollary}
The length of  is at least four.
\end{corollary}

To complete the proof of the lemma we need to consider two cases:

Case 1: .

Consider a maximum matching  of  which is obtained from
 by
shifting the edges of  on , that is,

Note that  saturates all vertices of  except .
Consider a vertex  which is a neighbour of . Due to claim
\ref{Neighbours V},  is a vertex of , which is
different from  and . Note that the neighbours of  are
the vertex  and one or two other vertices of  which are
saturated by . Thus there is no unsaturated vertex of ,
which has a common neighbour with . This implies that the number
of pairs of vertices of  which are not saturated by  and
have a common neighbour is less than the corresponding number for
, which contradicts the choice of .

Case 2: .

Consider a maximum matching  of , defined as:

Note that  saturates  and does not saturate . Consider
a vertex  which is a neighbour of . Due to claim
\ref{Neighbours V},  is a vertex of , which is
different from  and . Note that the neighbours of  are
the vertex  and two other vertices of 
which are saturated by . Thus there is no unsaturated vertex of , which has a common neighbour with . This implies that the
number of pairs of vertices of  which are not saturated by
 and have a common neighbour is less than the corresponding
number for , which contradicts the choice of . The proof of
lemma \ref{Max Matching 2-3} is completed.
\end{proof}

It would be interesting to generalize the statement of lemma
\ref{Max Matching 2-3} to almost regular graphs. In other words, we
would like to suggest the following

\begin{conjecture}
Let  be graph with . Then  contains a
maximum matching such that the unsaturated vertices (with respect to
this maximum matching) do not share a neighbour.
\end{conjecture}

We would like to note that we do not even know, whether the
conjecture holds for -regular graphs with .

\section{The system of cycles and paths}

In this section we prove two lemmas. For graphs that belong to a
very peculiar family, the first of them allows us to find a system
of cycles and paths that satisfy some explicitly stated properties.
The second lemma helps in finding a system with the same properties
in graphs that are subdivisions of the graphs from the mentioned
peculiar class. Moreover, due to the second lemma, it turns out that
if there is a system of the original graph that includes a maximum
matching, then there is a system of the subdivided graph preserving
this property!

\begin{lemma}
\label{Bipartite 2->=3}Let  be a graph with .
Suppose that every edge of  connects a vertex of degree two to
one with degree at least three. Then
\end{lemma}

\begin{enumerate}
\item[(1)] \textit{there exists a vertex-disjoint system of even paths }\textit{\ and cycles }\textit{\ of }\textit{\ such that }

\begin{enumerate}
\item[(1.1)] 

\item[(1.2)] \textit{all vertices of }\textit{\ lie on these paths or
cycles;}

\item[(1.3)] \textit{the end-vertices of the paths }\textit{\ are of degree two and these end-vertices are adjacent to vertices
of degree at least three;}

\end{enumerate}

\item[(2)] \textit{for every maximum matching }\textit{\ of }\textit{,
every pair of edge-disjoint matchings }\textit{\ with }\textit{\ every vertex }\textit{\ with }\textit{\ }\textit{\ is incident to one edge from }\textit{, one from }\textit{\ and one from }\textit{.}

\item[(3)]  \textit{contains two edge-disjoint maximum matchings};

\item[(4)] \label{NuIN(2,k)graphs}\textit{If} , ,  \textit{for every vertex}  \textit{then}


\end{enumerate}

\begin{proof}
(1) Clearly,  is a bipartite graph, since the sets

form a bipartition of . We intend to construct a system of
pairwise
vertex-disjoint cycles and even paths of  such that the all vertices of  lie on them. Of course, the cycles will be of even length since  is bipartite.

Choose a system of cycles  of  such that
,  and the graph
 does not contain a
cycle. Clearly,  is a forest, that is, a graph every
component of which is a tree. Moreover, for every 

\begin{description}
\item[(a)] if  then ;

\item[(b)] if  then .
\end{description}

If  contains no edge then add the remaining isolated vertices
(paths of length zero) to the system to obtain the mentioned system
of cycles and even paths of . Otherwise, consider a non-trivial
component of . Let  be a path of this component
connecting two vertices which have degree one in . Since 
is bipartite, (b) implies that  is of even length. Consider a
graph  obtained from  by removing
the path , that is,

Note that  is a forest. Moreover, it satisfies the properties
(a) and (b) like  does, that is, for every 

\begin{description}
\item[(a)] if  then ;

\item[(b)] if  then .
\end{description}

Clearly, by the repeated application of this procedure we will get a
system of even paths  of  such that the
graph 
contains no edge. Now, add the remaining isolated vertices (paths of
length zero) to  to obtained a system of even
paths .

Note that by the construction  and
 are vertex-disjoint. Moreover, the paths
 are of even length. As  is bipartite, the
cycles  are of even length, too.

Again, by the construction of  and
 we have (1.2) and that the end-vertices of
 are of degree two. As every edge of  connects a
vertex of degree two to one with degree at least three, the system
 satisfies (1.3).

Let us show that (1.1) holds, too. Since the number of vertices of
degree two and at least three is the same on the cycles
, and the difference of these two numbers is one on
each path from ,
then taking into account (1.2) and (1.3) we get:

(2) Define a pair of edge-disjoint matchings  in
the following way: alternatively add the edges of  and  to  and . Note that every vertex  is incident to one edge from , one from , and

On the other hand, for every pair of edge-disjoint matchings
, every vertex  is incident to at
most one edge
from  and at most two edges from , therefore

thus (see (\ref{eq1}))

and for every maximum matching  of , every pair of
edge-disjoint matchings  with , every
vertex  is incident to one edge from , one from
 and one from .

(3) directly follows from (2). (4) follows from (2) and the
bipartiteness of .

The proof of the lemma \ref{Bipartite 2->=3} is completed.
\end{proof}


\begin{lemma}
\label{SystemInSubdivision}Let  be a connected graph satisfying
the conditions:
\end{lemma}

\begin{description}
\item[(a)] ;

\item[(b)] \textit{no edge of }\textit{\ connects two vertices having degree at
least three.}

\item[ ] \textit{Let }\textit{\ be a graph obtained from }\textit{\ by a }\textit{-subdivision of an edge. If }\textit{\
contains a system of paths }\textit{\ and even cycles }\textit{\ such that }

\begin{enumerate}
\item[(1)] \textit{the degrees of vertices of a cycle from }\textit{\ are two and at least three alternatively,}

\item[(2)] \textit{all vertices of }\textit{\ lie on these paths or
cycles;}

\item[(3)] \textit{the end-vertices of the paths }\textit{\
are of degree two, and the vertices that are adjacent to these
end-vertices and do not lie on } \textit{\ are of
degree at least three;}

\item[(4)] \textit{every edge that does not lie on }\textit{\ and }\textit{\ is incident to one vertex of
degree two and one of degree at least three;}

\item[(5)] \textit{there is a maximum matching }\textit{\ of }\textit{\ such that every edge }\textit{\ lies on }\textit{\ and },
\end{enumerate}

\item \textit{then there is} \textit{a system of paths }\textit{\ and even cycles
}\textit{\ of the
graph } \textit{with} 
\textit{satisfying (1)-(5)}.
\end{description}

\begin{proof}
Let \textit{\ }and  be a system of
paths
and even cycles satisfying (1)-(5) and let  be the edge of  whose -subdivision led to the graph . First of all we will
construct a system of paths and even cycles of 
satisfying the conditions (1)-(4).

We need to consider three cases:

Case 1:  lies on a path .

Let  be the path of  corresponding to 
(that is, the path obtained from  by the -subdivision of the
edge ). Consider a system of paths and even cycles of  defined as:

Clearly, . It can be easily verified that the system \textit{\ }and\textit{\ } satisfies (1)-(4).

Case 2:  does not lie on either of \textit{\ }and  .

Let  be the new vertex of  and let  be the new edges of , that is:

(4) implies that  is incident to a vertex  of degree two and a vertex  of degree at least three, and suppose that .\newline

Since  and  does not lie on \textit{\ }and  , (2) implies that there is a path  such that  is an end-vertex of .
Consider the path  defined as:
and a system of paths and even cycles of  defined as:
Clearly, . Note that the new system satisfies (1) and
(2). Let us show that it satisfies (3) and (4), too. Since
,  is adjacent to the vertex  of
degree at least three
and  is an end-vertex of , we imply that the system \textit{\ }and\textit{\ } satisfies (3).

Note that we need to verify (4) only for the edge . As , , we imply that
the
system \textit{\ }and\textit{\ } satisfies (4), too.

Case 3:  lies on a cycle .

Let  be the new vertex of  and let  be the new edges of , that is:


(1) implies that the edge  is incident to a vertex  of degree
two and to a vertex  of degree at least three, and suppose that
, . Since
, (b)
implies that there is a vertex  such that  and . Note that since  and the edge 
does not lie on either of \textit{\ }and
 (2) implies that there is a path  such that  is an end-vertex of .

Let  be the path  of  starting from the vertex .
Consider a path  of  defined as:
and a system of paths and even cycles of  defined as:
Clearly, . Note that the new system satisfies (1) and
(2). Let us show that it satisfies (3) and (4), too. Since
,  is adjacent to the vertex  of
degree at least three, we imply that the system \textit{\ }and\textit{\
} satisfies (3).

Note that we need to verify (4) only for the edge . As ,  we imply that
the
system \textit{\ }and\textit{\ } satisfies (4), too.

The consideration of these three cases implies that there is a system \textit{\ }and\textit{\ } of paths and even cycles of  with  satisfying the conditions (1)-(4).
Let us show that among such systems there is at least one satisfying
(5), too.

Consider all pairs  in the
graph  where  is a system \textit{\ }and\textit{\ } of paths and even cycles of  with  satisfying the conditions (1)-(4) and  is a maximum matching of . Among these
choose a pair  for which the
number of edges of  which lie on cycles and paths of
 is maximum.We claim that all edges of
 lie on cycles and paths of .

\begin{claim}
\label{CycleCase}If  is a cycle from 
with length  then there are exactly  edges of 
lying on .
\end{claim}

\begin{proof}
Let  be the number of vertices of  which are saturated by an
edge from . (1) implies that if we
remove these  vertices from  we will get  paths with an odd
number of vertices. Thus
each of these  paths contains a vertex that is not saturated by . Thus the total number of edges from  is at most
Consider a maximum matching  of  defined
as:
where  is the set of edges of  that
are
incident to a vertex of , and  is a 1-factor of . Note that if  thenand therefore for the pair  we would have that  contains more edges
lying on cycles and paths of  then
 does, contradicting the choice of the pair
, thus , and on the cycle
 from  with length  there are exactly
 edges of . The proof of claim \ref{CycleCase} is
completed.
\end{proof}

Now, we are ready to prove that all edges of  lie on
cycles and paths of . Suppose, on the
contrary, that there is an edge  that
does not lie on cycles and paths of . (4)
implies that  is incident to a vertex  of degree at
least three and to a vertex  of degree two. (2)
implies that there is a path  of  such that  is an end-vertex of . (2) and claim \ref{CycleCase} imply
that
there is a path  of  such that  lies on . Let  and  be the end-vertices of , and let  and  be the subpaths of the path  connecting  and 
to , respectively. Consider a system  of paths and even cycles of  defined as follows:
where the path  is defined as:\ Note that  contains exactly
 paths. It can be easily verified that the new system
 of paths and even cycles of
 satisfies (1)-(4).

Now if we consider the pair  we would have that the paths and even cycles of

include more edges of  then the paths and even cycles of  do, contradicting the choice of the pair . Thus, all edges of  lie on
cycles
and paths of . The proof of the lemma \ref {SystemInSubdivision} is completed.
\end{proof}

\section{The subdivision and the main parameters}

The aim of this section is to prove a lemma, which claims that,
under some conditions, the subdivision of an edge increases the size
of the maximum 2-edge-colorable subgraph of a graph by one. This is
important for us, since it enables us to control our parameters,
while considering many graphs that are subdivisions of the others.

\begin{lemma}
\label{Edge Subdivision}Let  be a connected graph satisfying the
conditions:
\end{lemma}

\begin{description}
\item[(a)] ;

\item[(b)] \textit{\ is not an even cycle;}

\item[(c)] \textit{no edge of }\textit{\ connects vertices with degree at
least three.}

\item[ ] \textit{Let }\textit{\ be a graph obtained from }\textit{\ by a }\textit{-subdivision of an edge. Then}
\end{description}

\begin{enumerate}
\item[(1)] ;

\item[(2)] 
\end{enumerate}

\begin{proof}
(1) Let  be a pair of edge-disjoint matchings of  with  and let  be the edge of  whose -subdivision led to
the graph . We will consider three cases:

Case 1:  lies on a  alternating
cycle .

As  is connected and is not an even cycle, there is a
vertex  with . Clearly, there is a vertex  with  and . Let  be the other () edge incident to  and  be
an edge of  incident to . Note that since  is incident to
two edges lying
on  we, without loss of generality, may assume  to be different from . Let  be a path in  whose edge-set coincides with  and which starts from the vertex . Now,
assume  to be a path obtained from  by adding the edge
 to it, and let  be the path of 
corresponding to  (that is, the path obtained from  by the
-subdivision of the edge ).

Now, consider a pair of edge-disjoint matchings
 of  obtained in the following
way:

\begin{itemize}
\item if  then alternatively add the edges of 
to  and  beginning from ;

\item if  then alternatively add the edges of  to  and  beginning from
.
\end{itemize}

Define a pair of edge-disjoint matchings  of  as follows:

Clearly,


Case 2:  lies on a  alternating path
.

Let  be the path of  corresponding to 
(that is, the path obtained from  by the -subdivision of the
edge ).
Consider a pair of edge-disjoint matchings  of  obtained in the following way: alternatively add the edges of  to  and . Define:

Clearly,


Case 3: .

Due to (c) there is  with , such that  is
incident to . Let  be the other () edge of  that is
incident to , and assume  to be the edge of
 that is incident to  in  and is
different from . Now, add the edge  to  if
, and to  if . Clearly,
we constructed a pair of edge-disjoint matchings of ,
which contains  edges, therefore

(2) Note that if  is an odd cycle then  is an even one and , therefore, taking into account (1) and (b),
it suffices to show that if  is not a cycle then .

Let  be a pair of edge-disjoint matchings of
 with  and let  be the new vertex of , that is, assume . We need to
consider three cases:

Case 1:  contains at most one edge incident to
the vertex .

Note that

or

Case 2: The vertex  belongs to an alternating component of  which is a path .

Let  be a path of  containing the edge  and corresponding to , that is, let  be obtained from
 by the -subdivision of the edge . Consider a pair of
edge-disjoint matchings  of  defined as
follows: alternatively
add the edges of  to  and . Define:

Note that  is a pair of edge-disjoint matchings of . Moreover,

or

Case 3: The vertex  belongs to an alternating component of  which is a cycle .

Let  be a cycle of  containing the edge  and corresponding to , that is, let  be obtained from
 by the -subdivision of the edge . As  is not a
cycle, we imply that there is a vertex 
with . Clearly, there is a vertex
 such that  and . Let  be the other () edge of  incident to . Since  is
incident
to two edges lying on , we imply that there is an edge   such that  is incident to . Let  be a path of
, whose set of edges coincides with 
and starts from . Now consider the path  obtained from
 by adding the edge  to it.

Consider a pair of edge-disjoint matchings 
of  defined as follows:

\begin{itemize}
\item if  \ then alternatively add the edges of  to 
and  beginning from ;

\item if  then alternatively add the edges of 
to  and  beginning from .
\end{itemize}

Define

Note that  is a pair of edge-disjoint matchings of . Moreover,

or

The proof of the lemma \ref{Edge Subdivision} is completed.
\end{proof}

\section{The lemma}

In this section we prove a lemma that presents some lower bounds for
our parameters while we consider various subdivisions of graphs. The
aim of this lemma is the preparation of adequate theoretical tools
for understanding the growth of our parameters depending on the
numbers that the edges of graphs are subdivided. In contrast with
the proofs of the statements (a), (b), (c), (h), (i), that do not
include any induction, the proofs of the others significantly rely
on induction. Moreover, the basic tools for proving these statements
by induction are the proposition \ref{CubicPseudoGraph} and the
"loop-cut", the operation that helps us to reduce the number of
loops in a pseudo-graph. To understand the dynamics of the growth of
our parameters, we heavily use the lemma \ref{Edge Subdivision}.

Before we move on, we would like to define a class of graphs which
will play a crucial role in the proof of the main result of the
paper.

If  is a cubic pseudo-graph such that the removal (not cut)
of its loops leaves a tree (if we adopt the convention presented in
\cite{Harary}, then we may say that the "underlying graph" of
 is a tree; the simplest example of such a cubic pseudo-graph
is one from figure \ref{TrivialCase}), then consider the graph 
obtained from  by -subdividing each edge  of
, where
Define  to be the class of all those graphs  that
can be
obtained in the mentioned way. Note that the members of the class  are connected graphs.

\begin{lemma}
\label{PseudoGraphSubdivision} Let  be a connected cubic
pseudo-graph, and consider the graph  obtained from  by -subdividing each edge  of , . Suppose that,
for every edge  of , which is not a loop, we have:
. Then:

\begin{description}
\item[(a)] If  does not contain a loop then

\begin{description}
\item[(a1)] 

\item[(a2)] 
\end{description}

\item[(b)] If  contains an edge  which is adjacent to two loops 
and , then  is the cubic pseudo-graph from figure
\ref{TrivialCase} and

\item[(c)] If  contains a loop , then consider the cubic
pseudo-graph  obtained from  by cutting the
loop  and the graph  obtained from 
by -subdividing each edge  of
, where
Then:

\begin{description}
\item[(c1)] 

\item[(c2)] 

\item[(c3)] 

\item[(c4)] 
\end{description}

\item[(d)]

\begin{description}
\item[(d1)] 

\item[(d2)] 
\end{description}

\item[(e)]

\begin{description}
\item[(e1)] If  contains a loop  such that  then  and 

\item[(e2)] If  contains an edge  such that  is not a loop and  then  and 
\end{description}

\item[(f)] 

\item[(g)] If  then 

\item[(h)] If a cubic pseudo-graph  is obtained from 
by cutting its loop  and if a graph  is obtained from  by -subdividing each edge  of , where 
is defined according to (\ref{KPrimeDefinition}), then if  then ; in other words, the
property  is an invariant
for the operation of cutting a loop and defining  according to (\ref{KPrimeDefinition});

\item[(i)] If  then .
\end{description}
\end{lemma}

\begin{proof}
(a) For the proof of (a1) consider a graph  obtained from  by -subdividing each edge of . Note that
 satisfies the conditions of (\ref{NuIN(2,k)graphs}) of
the lemma \ref{Bipartite 2->=3}, thus (see
the equality (\ref{eq3}))therefore due to lemma \ref{Edge Subdivision} we have:

Note that for each  , hence
Taking into account (\ref{asterik}) we get:
thus


For the proof of (a2) let us note that as  does not contain a
loop, for each edge  of  we have , thus


(b) Note that

Since  is not a loop, we have  thus

and


(c) The proof of (c1) follows directly from the definition of the
operation of cutting loops. For the proof of (c2) note that

since  (see (\ref{KPrimeDefinition})).

For the proof of (c3) and (c4) let us introduce some additional
notations. Let  be the cycle and paths of

corresponding to the edges  of the cubic pseudo-graph . Let  be the cycle or a path of  corresponding to
the edge  of the cubic pseudo-graph .

Let  be a maximum matching of the graph . Define
 as the number of vertices from
 which
are saturated by an edge from . Note that if  then  and if  then .

Consider a subset of edges of the graph  defined as:
where  is a maximum matching of a path  obtained
from the paths  and  as follows:
 is a maximum matching of , and  is a maximum matching of .

Note that if  and  then we define the path
 in two ways. We would like to stress that our
results do not depend on the way the path  is
defined.

By the construction of ,  is a matching of . Moreover,as

Now, let us turn to the proof of (c4). Let  be a pair of edge-disjoint matchings of  such that . Define  as the number of vertices from  which
are saturated by an edge from . Note that if  then 
and if  then . We need to consider two
cases:

Case 1: ;

Define a pair of edge-disjoint matchings  of  as follows:
where , are obtained from a path  alternatively adding its edges to  and ; , are obtained from a path

alternatively adding its edges to  and , and the paths  and  are defined as
Again, let us note that if  and  then we define the
path  in two ways. We would like to stress that
our results do not depend on the way the path  is
defined.

Note that


Case 2: ;

Define a pair of edge-disjoint matchings  of  as follows:
where , are obtained from a path 
alternatively
adding its edges to  and ; , are obtained from the path  alternatively adding its edges to  and , and the path  is defined as
Note that

(d) We will give a simultaneous proof of the statements (d1) and
(d2). Note that if  does not contain a loop then (a1) and
(a2) imply that
thus without loss of generality, we may assume that  contains
a loop. Our proof is by induction on . Clearly, if  then
 is the pseudo-graph from figure \ref{TrivialCase}, thus (b)
implies that


as  and . Note that  or  if  and .

Now, by induction, assume that for every graph 
obtained from a cubic pseudo-graph  ()
by -subdividing each edge 
of , we have
and consider the cubic pseudo-graph  () and its
corresponding graph .

Let  be a loop of , and consider a cubic pseudo-graph  obtained from  by cutting the loop 
((a) of figure \ref{loopcut}). Note that  is
well-defined, since . As  due to induction
hypothesis, we have
where  is obtained from  by -subdividing each edge  of
,
and the mapping  is defined according to (\ref{KPrimeDefinition}). On the other hand, due to (c1), (c2) and (c4), we haveSince ,  we have
and therefore due to (\ref{InductionBound}) and proposition \ref {FractionInequality}, we get:


(e) We will prove (e1) by induction on . Note that if 
then  is the pseudo-graph from figure \ref{TrivialCase}, thusand due to (b)Now if  satisfies (e1), then taking into account that , ,  and , we get , and therefore


Now, by induction, assume that for every graph 
obtained from a cubic pseudo-graph  (),
by -subdividing each edge 
of , we have
provided that  satisfies (e1), and consider the
cubic pseudo-graph  () and its corresponding
graph . We need to consider two cases:

Case 1:  contains at least two loops.

Let  be a loop of  that differs from . Consider the
cubic pseudo-graph  obtained from  by
cutting the loop  ((a) of figure \ref{loopcut}), and the graph  obtained from a cubic pseudo-graph  by -subdividing each edge  of , where the mapping  is defined according to (\ref{KPrimeDefinition}).

Since  and , due to induction
hypothesis, we have
(c1), (c2) and (c4) imply thatSince ,  we have
and therefore due to proposition \ref{FractionInequality}, we get:


Case 2:  contains exactly one loop.

Let the only loop of  be adjacent to the edge . Let
 be the vertex of  that is incident to  and , and let . Let  and  () be
two edges that differ from  and are incident to . Finally,
let  and  be the endpoints of  and  that are
not incident to , respectively.

Subcase 2.1: .

Consider a cubic pseudo-graph  obtained from
 by cutting the loop  and the graph  obtained
from a cubic pseudo-graph  by -subdividing each edge  of
, where the mapping  is defined
according to (\ref{KPrimeDefinition}). As  does
not contain a loop, due to (a1) and (a2), we have(c1), (c2) and (c4) imply thatSince ,  we have
thus
Due to (\ref{7-8Bound}) and proposition \ref{FractionInequality}, we
get:


Subcase 2.2: .

Let  be the edge which is incident to  and is
different from  and , and let  (figure \ref{Reduction}).

\begin{figure}[h]
\begin{center}
\includegraphics[width=24pc]{ReductionPseudoCubics.eps}\\
\caption{Reducing  to }\label{Reduction}
\end{center}
\end{figure}

Define a cubic pseudo-graph  as follows:and consider the graph  obtained from  by -subdividing each edge  of , where
Note that ,  and  thus, due to induction hypothesis, we have:

It is not hard to see that

As , we have
therefore due to (\ref{Induction6-7}) and proposition \ref {FractionInequality}, we get:The proof of (e1) is completed. Now, let us turn to the proof of
(e2). Note that if  does not contain a loop then (a1) and
(a2) imply that
thus, without loss of generality, we may assume that 
contains a loop.
Our proof is by induction on . Clearly, if  then  is the pseudo-graph from figure
\ref{TrivialCase},
and due to (b)Now, if  satisfies (e2) then  and taking into
account that , , we get , therefore


Now, by induction, assume that for every graph 
obtained from a cubic pseudo-graph  ()
by -subdividing each edge 
of , we have
and consider the cubic pseudo-graph  () and its
corresponding graph .

Case 1: There is an edge  such that  and  form a cycle of the length two (figure \ref{Case of
multiple edge})

\begin{figure}[h]
\begin{center}
\includegraphics[height=15pc]{CaseOfMultipleEdge.eps}\\
\caption{The case of multiple edge}\label{Case of multiple edge}
\end{center}
\end{figure}

Let  be the edges and vertices as
on
figure \ref{Case of multiple edge}. Consider a cubic pseudo-graph , defined as follows:and consider the graph  obtained from  by -subdividing each edge  of , where
Note that
Let us show that
First of all note that  and 
therefore if  is not a loop of  () then
the inequalities follow directly from the induction hypothesis. On
the other hand, if  is a loop of  () then
the same inequalities hold due to (e1).

Since
proposition \ref{FractionInequality} implies that


Case 2:  contains at least two loops and does not satisfy the
condition of the case 1.

As  is connected and ,
there is a loop  of  such that  is not adjacent to . Let  be the edge adjacent to the edge . Let  be
the vertex
of  that is incident to  and , and let . Let  and  be two edges that differ from  and are incident to . Finally, let  and  be the endpoints of  and
 that are not incident to , respectively.

Consider the cubic pseudo-graph  obtained from
 by cutting the loop  and the graph  obtained
from a cubic pseudo-graph  by -subdividing each edge  of
, where the mapping  is defined
according to (\ref{KPrimeDefinition}). Note that .

Let us show that  satisfies the condition of (e2).
Clearly, if  then we are done, thus we may assume that . Since , we imply that . As  does not satisfy the condition of the
case 1, the edge  is not a loop of  and

Thus  satisfies the condition of (e2), therefore,
due to induction hypothesis, we get:
(c1), (c2) and (c4) imply thatSince ,  we have
therefore, due to proposition \ref{FractionInequality}, we get:


Case 3:  contains exactly one loop  and does not satisfy
the condition of the case 1.

Let  be the edge adjacent to the edge . Let  be the vertex of  that is incident to  and , and let .
Let 
and  be two edges that differ from  and are incident to . Finally, let  and  be the endpoints of  and
 that are not incident to , respectively.

Subcase 3.1:  and .

Define a cubic pseudo-graph  as follows (figure \ref {Reduction}):and consider the graph  obtained from  by -subdividing each edge  of , where
Note that  and 
thus, due to induction hypothesis, we have:

On the other hand, it is not hard to see that
As , we have
therefore, due to proposition \ref{FractionInequality}, we get:

Subcase 3.2:  or .

Consider the cubic pseudo-graph  obtained from
 by cutting the loop  and the graph  obtained
from a cubic pseudo-graph  by -subdividing each edge  of
, where the mapping  is defined
according to (\ref{KPrimeDefinition}). Note that .

Let us show that  and its corresponding graph
 satisfy
Note that if  then, since  and , (\ref{BasicInequalities}) follows
directly from the induction hypothesis. So, let us assume, that
. If  then  does not
contain a loop as . Thus (\ref{BasicInequalities}) follows
from (a1) and (a2). Thus, we may also assume that . As
, we deduce that .
As  does not satisfy the condition of the case 1, we have
 and  does not contain a loop. Thus
(\ref{BasicInequalities}) again follows from (a1) and (a2).

Now, (c1), (c2) and (c4) imply thatSince , , we have
therefore, due to (\ref{BasicInequalities}) and proposition \ref {FractionInequality}, we get:


(f) Note that if  satisfies at least one of the conditions of
(a), (e1), (e2), then, taking into account the inequality , we get:
thus, without loss of generality, we may assume that 
satisfies none of the conditions of (a), (e1), (e2), hence 
contains at least one loop, and for each loop  and for each edge
 of , that is not a loop, we have:  and .
For these cubic pseudo-graphs, we will prove the inequality (f) by
induction on . If  then  is the cubic
pseudo-graph from the figure \ref{TrivialCase} and, as  and ,  contains a perfect matching, thusNow, by induction, assume that for every graph 
obtained from a cubic pseudo-graph  ()
by -subdividing each edge 
of , we have
and consider the cubic pseudo-graph  () and its
corresponding graph .

Let  be a loop of , and consider a cubic pseudo-graph  obtained from  by cutting the loop  and a graph  obtained from  by -subdividing each edge  of , where the
mapping  is defined according to
(\ref{KPrimeDefinition}). As , due to induction
hypothesis, we have
(c2) and (c3) imply thatDue to proposition \ref{FractionInequality}, we get:


(g) Let  be the connected cubic pseudo-graph corresponding to
 and let  be the tree obtained from  by
removing its loops (see the definition of the class ).
Assume  and  to be the numbers of internal (non-pendant) and pendant vertices of . Clearly, . On the other
hand,
Since , we get
or
We prove the inequality by induction on . Note that if  then
 is the cubic pseudo-graph from the figure \ref{TrivialCase},
therefore
On the other hand, if  then  is the cubic pseudo-graph
shown on the figure \ref{Case k1}, thus


\begin{figure}[h]
\begin{center}
\includegraphics[height=15pc]{CaseK1.eps}\\
\caption{The case }\label{Case k1}
\end{center}
\end{figure}

Now, by induction, assume that for every graph  we have , if the tree
 contains less than  internal vertices,
and let us consider the graph  the corresponding
tree  of which contains () internal
vertices. We need to consider two cases:

Case 1: There is  such that ,  and the subtree of

induced by  is the tree shown on the figure \ref{Case of two branches}.

\begin{figure}[h]
\begin{center}
\includegraphics[height=15pc]{CaseOfTwoBranches.eps}\\
\caption{The case of two branches}\label{Case of two branches}
\end{center}
\end{figure}

Let  be the tree  and let  be the cubic
pseudo-graph obtained from  by removing the vertices
 and adding a new loop incident to the vertex
. Note that  contains less than 
internal vertices, thus for the graph 
corresponding to , we have
On the other hand, since
due to (\ref{InductionBound_6_13}) and proposition
\ref{FractionInequality}, we get:
since


Case 2: There is  such that 
and
the subtree of  induced by  is the tree shown on the figure \ref{Case of branch and a leave}.

\begin{figure}[h]
\begin{center}
\includegraphics[height=15pc]{CaseOfABranchLeave.eps}\\
\caption{The case of a branch and a leave}\label{Case of branch and
a leave}
\end{center}
\end{figure}

Let  be the tree  and let  be the
cubic pseudo-graph obtained from  by removing the vertices  and  and adding the edge .
Note that  contains less than  internal
vertices, thus
for the graph  corresponding to , we have
On the other hand, since
due to (\ref{InductionBound_6_13}) and proposition
\ref{FractionInequality}, we get:
since
To complete the proof of the inequality, let us note that, since the tree  contains , () internal vertices,
 satisfies at least one of the conditions of case 1 and
case 2.

(h) (c1) and (c2) imply that

Since ,  we have
thus, due to proposition \ref{FractionInequality}, we get:

(i) Note that as  due to (e1) and (e2), for every
edge 
of  we haveLet us show that . Consider a maximal (with
respect to
the operation of cutting loops) sequence of cubic pseudo-graphs , where , and  is obtained from  by cutting a loop  of ,. Note that proposition
\ref{ConnectedInvariance} implies that for  the graph
 is connected.

Consider the sequence of graphs , where , and for  the graph  is obtained from  by -subdividing each edge  of , where the mapping  is defined from  according to (\ref {KPrimeDefinition}) and . As the sequence  is maximal, the operation
of cutting the loops is not applicable to , thus due to
remark \ref{SuccessiveCut},  is either the trivial
cubic pseudo-graph from the figure \ref{TrivialCase} or a connected
graph (i.e. a connected pseudo-graph without loops). On the other
hand, (h) implies that for , we have
thus, taking into account (a2), we deduce that  is the
trivial cubic pseudo-graph from the figure \ref{TrivialCase}.

Note that for the proof of , it suffices to show
that if we remove all loops of  then we will get a tree,
which is equivalent to proving that  does not contain a
cycle. Suppose that  contains a cycle. As ,
which is the pseudo-graph from the
figure \ref{TrivialCase}, does not contain a cycle, we imply that there is  such that  contains a cycle and  does not. Proposition \ref{CycleInvariance} implies
that the
loop  of , whose cut led to the cubic pseudo-graph  is adjacent to an edge  which, in its turn, is
adjacent to two edges  and  that form the
only cycle of .

As the edges  and  form a cycle of
,
the cut of the loop  leads to a loop  of  (see the definition of the operation of the cut of
loops). Due to (\ref{KPrimeDefinition}), we have
thus, due to (e1), we havecontradicting (\ref{InequalityLIST}).The proof of the lemma is
completed.
\end{proof}

\section{The main results}

We are ready to prove the first result of the paper. The basic idea
of the proof of this theorem can be roughly described as follows:
proving a lower bound for the main parameters of a cubic graph 
is just proving a bound for the graph  obtained by
removing a maximum matching  of . Next, according to lemma
\ref{Max Matching 2-3}, there is a maximum matching of a cubic graph
such that its removal leaves a graph, in which each degree is either
two or three. Moreover, the vertices of degree three are not placed
very closed. This allows us to consider this graph as a subdivision
of a cubic pseudo-graph, in which each edge is subdivided
sufficiently many times. The word "sufficiently" here should be
understood as big enough to allow us to apply the main results of
the lemma \ref{PseudoGraphSubdivision}. Next, by considering the
connected components of , we divide them into two or
three groups. For each of this groups, thanks to lemma
\ref{PseudoGraphSubdivision}, we find a bound for our parameters.
Then, due to proposition \ref{LinearInequality}, we not only
estimate the total contribution of the connected components to the
main parameters, but also keep this estimations big enough, which
allows us to get the main results of the theorem.

\begin{theorem}
\label{MainTheoremCubics}Let  be a cubic graph. Then:
\end{theorem}

\begin{proof}
In \cite{Takao} it is shown that every odd regular graph 
contains a matching of size at least , where  is the degree of
vertices of . Particularly, for a cubic graph  we have:

Now, let us show that the other two inequalities are also true. Let
 be a
maximum matching of  such that the unsaturated vertices (with respect to ) do not have a common neighbour (see lemma \ref{Max Matching 2-3}). Let  be a rational number such that  and
Note that to complete the proof, it suffices to show that
Consider the graph . Clearly,


Let  and  be the numbers of vertices of  with
degree
two and three, respectively. Clearly,

which implies that
Let  be the connected components of . For a vertex ,  define:


Note that
and similarly
where  are vertices of  with
, .

By the choice of , we have that for   is

\begin{description}
\item[(a)] either a cycle,

\item[(b)] or a connected graph, with 
which does not contain two vertices of degree three that are
adjacent or share a neighbour.
\end{description}

Note that if  is of type (b), then there is a cubic pseudo-graph  such that  can be obtained from  by -subdividing each edge  of  (proposition \ref{CubicPseudoGraph}). Of course, if  is not a loop then .

Let  be the numbers of vertices of  that lie
on its connected components , which are cycles,
graphs of type (b) that are from the class , graphs of
type (b) which are not from the class , respectively.

It is clear that if  is a vertex of  lying on
a cycle of length  then
If  is a vertex of  lying on a connected component  of  which is from the class ,
then (g) of lemma \ref{PseudoGraphSubdivision} implies that
If  is a vertex of  lying on a connected component  of  which is of type (b) and does not belong
to the class , then (f) of lemma
\ref{PseudoGraphSubdivision} implies
that

Let  and  be the number of vertices of 
with degree three that lie on connected components
, which are graphs from the class  or
are graphs of type (b), which are not from the class ,
respectively. Clearly,
(d2) of lemma \ref{PseudoGraphSubdivision} implies that(i) of lemma \ref{PseudoGraphSubdivision} implies that
Thus, due to (\ref{Nhu1Weight})As  we get: . Since , due to proposition \ref{LinearInequality}, we have:and therefore(\ref{ThreeVertices}) implies that
which is equivalent toNote that if , then
, which means that  and among the components  there are no
representatives of the class .

Now, let us turn to the proof of the inequality .

Let  be the numbers of vertices of  that lie on
its connected components , which are cycles and
graphs of type (b), respectively. It is clear that if  is a
vertex of  lying on a cycle of the length  then
If  is a vertex of  lying on a connected component  of  which is of type (b), then (d1) of lemma \ref {PseudoGraphSubdivision} implies that
As the number of vertices of  which are of degree three is , (d2) of lemma \ref {PseudoGraphSubdivision} implies that
Thus, due to (\ref{Nhu2Weight})As , (\ref{Bbound}) implies that
Since , due to proposition
\ref{LinearInequality}, we get
and thereforewhich is equivalent to

The proof of the theorem is completed.
\end{proof}

\begin{remark}
There are graphs attaining bounds of the theorem
\ref{MainTheoremCubics}. The graph from figure \ref{Examples
attaining the bounds}a attains the first two bounds and the graph
from figure \ref{Examples attaining the bounds}b the last bound.
\end{remark}

\begin{center}
\begin{figure}[h]
\begin{center}
\includegraphics[height=10pc]{Examples.eps}\\
\caption{Examples attaining the bounds of the theorem
\protect\ref{MainTheoremCubics}}\label{Examples attaining the
bounds}
\end{center}
\end{figure}
\end{center}

Recently, we managed to prove:

\begin{theorem}
For every cubic graph 
\end{theorem}

Note that this implies that there is no graph attaining
all the bounds of theorem \ref{MainTheoremCubics} at the same time.

The methodology developed above allows us to prove the second result
of the paper, which is an inequality among our main parameters. To
prove it, again we reduce the inequality to another one considered
in the class of graphs, that are obtained from a cubic graph by
removing a matching of . Note that this time matching need not to
be maximum, nevertheless, its removal keeps the vertices of degree
three "far enough". Next, by considering any of connected components
of this graph, we look at it as a subdivision of a cubic
pseudo-graph. This allows us to apply the results from the section
on system of cycles and paths, and find a suitable system, which not
only captures the essence of the inequality that we were trying to
prove, but also is very simple in its structure, and this allows us
to complete the proof.

\begin{theorem}
\label{ArithmeticalMean}For every cubic graph  the following inequality holds:
\end{theorem}

\begin{proof}
Let  be a pair of edge-disjoint matchings of
 with . Without loss of generality we may assume that  is
maximal (not necessarily maximum). Let  be the
connected components of ,  be the
number of vertices of  having
degree three, , and let  be the number of vertices of  having degree three. Note that

Let us show that for each , , the following
inequality is
true:

Note that, if  is a cycle, then  and , thus (\ref{InequalityInComponents}) is true for the cycles.
Now, let us assume  to contain a vertex of degree three. As
 is a maximal matching, no two vertices of degree three are
adjacent in . Proposition
\ref{CubicPseudoGraph} implies that there is a cubic pseudo-graph  such that  can be obtained from  by -subdividing each edge  of  where . Let  be the graph obtained from
 by -subdividing each edge  of . Note
that 
contains  vertices of degree three,  vertices of degree two and no two vertices of
the same degree are adjacent in . Due to lemma
\ref{Bipartite 2->=3}, there is a system  of even cycles and paths of  satisfying the
conditions
(1.2),(1.3) of the lemma \ref{Bipartite 2->=3} and containing  paths (see (1.1) of the lemma \ref {Bipartite 2->=3}). (2) of lemma \ref{Bipartite 2->=3} implies that  includes a maximum matching of
.

Now, note that  can be obtained from  by a
sequence of -subdivisions. Lemma \ref{SystemInSubdivision}
implies that there is a system  of paths and even
cycles of  satisfying the conditions (1)-(5) of the lemma
\ref{Bipartite 2->=3} and containing exactly 
paths!

Let  be the number of paths from  containing an
odd number
of edges. Note that since , we have:
Summing up the inequalities (\ref{InequalityInComponents}) from 
to 
we get:Thus
Taking into account that
we get:
or
The proof of the theorem \ref{ArithmeticalMean} is completed.
\end{proof}


\begin{acknowledgement}
We would like to thank our reviewers for their useful comments that helped us to improve the paper.
\end{acknowledgement}

\begin{thebibliography}{99}

\bibitem{AlbertsonHaasFirst} M.O. Albertson, R. Haas, Parsimonious edge
coloring, Discrete Math. 148 (1996) 1--7.

\bibitem{AlbertsonHaasSecond} M.O. Albertson, R. Haas, The edge chromatic
difference sequence of a cubic graph, Discrete Math. 177, (1997)
1--8.

\bibitem{Bella} B. Bollob\'as, Extremal graph theory, Academic Press,
London-New York-San Francisco, 1978.

\bibitem{GirthBound} A. D. Flaxman, S. Hoory, Maximum matchings in regular
graphs of high girth, The Electronic Journal of Combinatorics, 14, N
1, 2007, pp. 1-4.

\bibitem{Harary} F. Harary, Graph Theory, Addison-Wesley, Reading, MA, 1969.

\bibitem{HararyPlummer} F. Harary, M.D. Plummer, On the core of a graph,
Proc. London Math. Soc. 17 (1967), pp. 305--314.

\bibitem{HenningYeo} M.A. Henning, A. Yeo, Tight Lower Bounds on the Size of
a Maximum Matching in a Regular Graph, Graphs and Combinatorics,
vol. 23, N 6, 2007, pp. 647-657

\bibitem{Hobbs} A. M. Hobbs, E. Schmeichel, On the maximum number of
independent edges in cubic graphs, Discrete Mathematics 42, 1982,
pp. 317-320.

\bibitem{Holyer} I. Holyer, The NP-completeness of edge coloring, SIAM J.
Comput. 10, N4, 718-720, 1981 (available at:
http://cs.bris.ac.uk/ian/graphs).

\bibitem{KaiserKralNorine} T. Kaiser, D. Kr\'al, S. Norine, Unions of perfect
matchings in cubic graphs, Electronic Notes in Discrete Mathematics,
22: 341-- 345, 2005.

\bibitem{Lov} L. Lov\'asz, M.D. Plummer, Matching theory, Ann. Discrete Math.
29 (1986).

\bibitem{MPP0-1} V. V. Mkrtchyan, On trees with a maximum proper partial 0-1
coloring containing a maximum matching, Discrete Mathematics 306,
(2006), pp. 456-459.

\bibitem{Perfect} V. V. Mkrtchyan, A note on minimal matching covered
graphs, Discrete Mathematics 306, (2006), pp. 452-455.

\bibitem{FiveFourth} V. V. Mkrtchyan, V. L. Musoyan, A. V. Tserunyan, On
edge-disjoint pairs of matchings, Discrete Mathematics 308, (2008),
pp. 5823-5828 (available at: http://arxiv.org/abs/0708.1903).

\bibitem{Monthly} V. V. Mkrtchyan, Problem 11305, American Mathematical
Monthly, v. 114, 2007, p. 640.

\bibitem{TakaoBaybars} T. Nishizeki and I. Baybars, Lower bounds on the
cardinality of the maximum matchings of planar graphs, Discrete
Math., 28, 255-267, 1979.

\bibitem{Takao} T. Nishizeki, On the maximum matchings of regular
multigraphs, Discrete Mathematics 37, 1981, pp. 105-114.


\bibitem{Shannon} C. E. Shannon, A theorem on coloring the lines of a
network, J. Math. And Phys., 28 (1949), pp. 148-151.

\bibitem{Steffen} E. Steffen, Measurements of edge-uncolorability, Discrete
Mathematics 280 (2004), pp. 191 -- 214.

\bibitem{FivefourthCharacter} A. V. Tserunyan, Characterization of a class
of graphs related to pairs of disjoint matchings, Discrete
Mathematics 309, (2009), pp. 693-713, (available at:
http://arxiv.org/abs/0712.1014)

\bibitem{Vizing} V. G. Vizing, The chromatic class of a multigraph,
Kibernetika No. 3, (1965) Kiev, pp. 29-39 (in Russian)

\bibitem{Weinstein} J. Weinstein, Large matchings in graphs, Canadian J.
Math., 26, 6,(1974), pp. 1498-1508.

\bibitem{West} D. B. West, Introduction to Graph Theory, Prentice-Hall,
Englewood Cliffs, 1996.
\end{thebibliography}
\end{document}
