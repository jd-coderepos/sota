\documentclass{LMCS}

\def\doi{8 (1:14) 2012}
\lmcsheading {\doi}
{1--33}
{}
{}
{Jan.~27, 2011}
{Feb.~29, 2012}
{}

\usepackage{amsmath,stmaryrd,amsfonts,amssymb,latexsym,url}
\usepackage{graphicx}
\usepackage[usenames]{color}
\usepackage[all]{xy}
\usepackage{ifpdf}
\usepackage{enumerate,hyperref}
\DeclareGraphicsRule{.pdftex}{pdf}{*}{}  


\DeclareFontFamily{U}{matha}{\hyphenchar\font45} \DeclareFontShape{U}{matha}{m}{n}{
      <5> <6> <7> <8> <9> <10> gen * matha
      <10.95> matha10 <12> <14.4> <17.28> <20.74> <24.88> matha12
      }{}
\DeclareSymbolFont{matha}{U}{matha}{m}{n}
\DeclareFontSubstitution{U}{matha}{m}{n}
\DeclareMathSymbol{\cll}{3}{matha}{"CE} 

\newcommand\Min{\mathop{\text{Min}}}

\newcommand\Eval[1]{\left\llbracket{#1}\right\rrbracket}
\newcommand\open[1]{\widehat{#1}}
\newcommand\code[1]{ulcorner{#1}\urcorner}
\newcommand\limp{\Rightarrow}
\newcommand\nat{\mathbb{N}}
\newcommand\bool{\mathbb{B}}
\newcommand\pow{\mathbb{P}}
\newcommand\Pfin{\pow_{fin}}
\newcommand\Smyth{\mathcal Q}
\newcommand\V{{\mathcal V}}
\newcommand\SV{\Smyth_\V}
\newcommand\Hoare{\mathcal H}
\newcommand\HV{\Hoare_\V}
\newcommand\Sober{\mathcal S}
\newcommand\pt{\mathsf{pt}}
\newcommand\Img{\mathop{\mathrm{Im}}}
\newcommand\img{\mathop{\mathrm{im}}}
\newcommand\Open{\mathcal O}
\newcommand\Cred{\mathbf{Cd}}

\newcommand\mopen{\{\mkern-\thinmuskip|}
\newcommand\mclose{|\mkern-\thinmuskip\}}
\newcommand\munion{\uplus}

\newcommand\upc{\mathop{\uparrow}\nolimits}
\newcommand\dc{\mathop{\downarrow}\nolimits}
\newcommand\uuarrow{\rlap{}\raise.5ex\hbox{}}\newcommand\ddarrow{\rlap{}\raise.5ex\hbox{}}\newcommand\Fin{\mathop{\text{Fin}}}
\newcommand\fin{\text{fin}}

\newcommand{\interior}[1]{int ({#1})} \newcommand{\biginterior}[1]{\interior{#1}} 

\newcommand\QRB{\mathbf{QRB}}
\newcommand\B{\mathbf{B}}
\newcommand\RB{\mathbf{RB}}
\newcommand\FS{\mathbf{FS}}
\newcommand{\identity}[1]{\mathrm{id}_{#1}}

\newcommand\qs{\varsigma}

\newcommand\Val{\mathbf V}
\newcommand{\real}{\mathbb{R}}
\newcommand{\creal}{\overline{\real^+_\sigma}}
\newcommand\Prev{\mathbf P} \newcommand\Prevlin{\Prev^{\triangle}}
\newcommand\Above[1]{\bigtriangleup{#1}}
\newcommand\Below[1]{\bigtriangledown{#1}}
\newcommand\Prevhaute{\Above\Prev}
\newcommand\Prevbasse{\Below\Prev}
\newcommand\smallchoquet{\rlap{}\int}
\newcommand\choquet{\rlap{}\int}
\newcommand\ugame[1]{\mathfrak{u}_{#1}}

\newcommand\dG{{\mathsf{d}}}
\newcommand\patch{{\mathsf{patch}}}

\newcommand\pll{\mathrel{\prec\mskip-5mu\prec}}

\begin{document}

\title[QRB-Domains]{-Domains and the Probabilistic Powerdomain\rsuper*}

\author[J.~Goubault-Larrecq]{Jean Goubault-Larrecq}	\address{LSV, ENS Cachan, CNRS, INRIA, France}	\email{goubault@lsv.ens-cachan.fr}  







\keywords{Domain theory, quasi-continuous domains, probabilistic powerdomain.}
\subjclass{D.3.1, F.1.2, F.3.2}

\titlecomment{{\lsuper*}An extended abstract already appeared in Proc. 25th {A}nnual {IEEE} {S}ymposium on {L}ogic in {C}omputer {S}cience ({LICS}'10).}





\begin{abstract}
  Is there any Cartesian-closed category of continuous domains that
  would be closed under Jones and Plotkin's probabilistic powerdomain
  construction?  This is a major open problem in the area of
  denotational semantics of probabilistic higher-order languages.  We
  relax the question, and look for quasi-continuous dcpos instead.
We introduce a natural class of such quasi-continuous dcpos, the
  omega-QRB-domains.  We show that they form a category omega-QRB with
  pleasing properties: omega-QRB is closed under the probabilistic
  powerdomain functor, under finite products, under taking bilimits of
  expanding sequences, under retracts, and even under so-called
  quasi-retracts.  But\ldots{} omega-QRB is not Cartesian closed.  We
  conclude by showing that the QRB domains are just one half of an
  FS-domain, merely lacking control.
\end{abstract}

\maketitle




\section{Introduction}
\label{sec:intro}

\subsection{The Jung-Tix Problem}
A famous open problem in denotational semantics is whether the
probabilistic powerdomain  of an -domain  is again
an -domain \cite{JT:troublesome}, and similarly with
-domains in lieu of -domains.   (resp.\
) is the dcpo of all continuous probability (resp.,
subprobability) valuations over : this construction was introduced
by Jones and Plotkin to give a denotational semantics to higher-order
probabilistic languages \cite{JP:proba}.


More generally, is there a category of nice enough dcpos that would be
Cartesian-closed and closed under ?  We call this the {\em
  Jung-Tix problem\/}.  By ``nice enough'', we mean nice enough to do
any serious mathematics with, e.g., to establish definability or full
abstraction results in extensional models of higher-order,
probabilistic languages.  It is traditional to equate ``nice enough''
with ``continuous'', and this is justified by the rich theory of
continuous domains \cite{GHKLMS:contlatt}.

However, {\em quasi-continuous\/} dcpos (see \cite{GLS:quasicont}, or
\cite[III-3]{GHKLMS:contlatt}) generalize continuous dcpos and are
almost as well-behaved.  We propose to widen the scope of the problem,
and ask for a category of quasi-continuous dcpos that would be closed
under .  We show that, by mimicking the construction of
-domains \cite{AJ:domains},
with some flavor of ``quasi'', we obtain a category  of
so-called -domains that not only has many desired, nice
mathematical properties (e.g., it is closed under taking bilimits of
expanding sequences, and every -domain is stably compact),
but is also closed under .

We failed to solve the Jung-Tix problem:  is indeed not
Cartesian-closed.  In spite of this, we believe our contribution to
bring some progress towards settling the question, and at least to
understand the structure of  better.  To appreciate this,
recall what is currently known about .  There are two landmark
results:  is a continuous dcpo as soon as  is
(\cite{Edalat:int}, building on Jones
\cite{JP:proba}), and  is stably compact (with its weak
topology) whenever  is \cite{JT:troublesome,AMJK:scs:prob}.  Since
then, no significant progress has been made.  When it comes to solving
the Jung-Tix problem, we must realize that there is {\em little
  choice\/}: the only known Cartesian-closed categories of (pointed)
continuous dcpos that may suit our needs are  and 
\cite{JT:troublesome}.  I.e., all other known Cartesian-closed
categories of continuous dcpos, e.g., bc-domains or L-domains, are
{\em not\/} closed under .
Next, we must recognize that {\em little is known\/} about the
(sub)probabilistic powerdomain of an  or -domain.  In trying
to show that either  or  was closed under , Jung and
Tix \cite{JT:troublesome} only managed to show that the
subprobabilistic powerdomain  of a {\em finite
  tree\/}  was an -domain, and that the subprobabilistic
powerdomain of a {\em reversed finite tree\/} was an -domain.
This is still far from the goal.

\begin{figure}
  \centering
  \ifpdf
  \begin{picture}(0,0)\includegraphics{v3.pdf}\end{picture}\setlength{\unitlength}{1450sp}\begingroup\makeatletter\ifx\SetFigFont\undefined \gdef\SetFigFont#1#2#3#4#5{\reset@font\fontsize{#1}{#2pt}\fontfamily{#3}\fontseries{#4}\fontshape{#5}\selectfont}\fi\endgroup \begin{picture}(13106,8580)(259,-11503)
\put(9181,-3976){\makebox(0,0)[lb]{\smash{{\SetFigFont{11}{13.2}{\rmdefault}{\mddefault}{\updefault}{\color[rgb]{0,0,0}Here,}}}}}
\put(9181,-7531){\makebox(0,0)[lb]{\smash{{\SetFigFont{11}{13.2}{\rmdefault}{\mddefault}{\updefault}{\color[rgb]{0,0,0}, indicative}}}}}
\put(9181,-8071){\makebox(0,0)[lb]{\smash{{\SetFigFont{11}{13.2}{\rmdefault}{\mddefault}{\updefault}{\color[rgb]{0,0,0}of the weight of }}}}}
\put(9766,-8611){\makebox(0,0)[lb]{\smash{{\SetFigFont{11}{13.2}{\rmdefault}{\mddefault}{\updefault}{\color[rgb]{0,0,0},}}}}}
\put(11071,-10456){\makebox(0,0)[lb]{\smash{{\SetFigFont{11}{13.2}{\rmdefault}{\mddefault}{\updefault}{\color[rgb]{0,0,0}}}}}}
\put(9181,-10456){\makebox(0,0)[lb]{\smash{{\SetFigFont{11}{13.2}{\rmdefault}{\mddefault}{\updefault}{\color[rgb]{0,0,0}E.g., }}}}}
\put(9181,-8611){\makebox(0,0)[lb]{\smash{{\SetFigFont{11}{13.2}{\rmdefault}{\mddefault}{\updefault}{\color[rgb]{0,0,0}(}}}}}
\put(9766,-9286){\makebox(0,0)[lb]{\smash{{\SetFigFont{11}{13.2}{\rmdefault}{\mddefault}{\updefault}{\color[rgb]{0,0,0},}}}}}
\put(10621,-3976){\makebox(0,0)[lb]{\smash{{\SetFigFont{11}{13.2}{\rmdefault}{\mddefault}{\updefault}{\color[rgb]{0,0,0}}}}}}
\put(11791,-3931){\makebox(0,0)[lb]{\smash{{\SetFigFont{11}{13.2}{\rmdefault}{\mddefault}{\updefault}{\color[rgb]{0,0,0}}}}}}
\put(12601,-3931){\makebox(0,0)[lb]{\smash{{\SetFigFont{11}{13.2}{\rmdefault}{\mddefault}{\updefault}{\color[rgb]{0,0,0}}}}}}
\put(12151,-3346){\makebox(0,0)[lb]{\smash{{\SetFigFont{11}{13.2}{\rmdefault}{\mddefault}{\updefault}{\color[rgb]{0,0,0}}}}}}
\put(12151,-4381){\makebox(0,0)[lb]{\smash{{\SetFigFont{11}{13.2}{\rmdefault}{\mddefault}{\updefault}{\color[rgb]{0,0,0}}}}}}
\put(9181,-6451){\makebox(0,0)[lb]{\smash{{\SetFigFont{11}{13.2}{\rmdefault}{\mddefault}{\updefault}{\color[rgb]{0,0,0}is drawn just as  itself,}}}}}
\put(11386,-9286){\makebox(0,0)[lb]{\smash{{\SetFigFont{11}{13.2}{\rmdefault}{\mddefault}{\updefault}{\color[rgb]{0,0,0}).}}}}}
\put(11431,-8611){\makebox(0,0)[lb]{\smash{{\SetFigFont{11}{13.2}{\rmdefault}{\mddefault}{\updefault}{\color[rgb]{0,0,0},}}}}}
\put(9181,-5191){\makebox(0,0)[lb]{\smash{{\SetFigFont{11}{13.2}{\rmdefault}{\mddefault}{\updefault}{\color[rgb]{0,0,0}Legend:}}}}}
\put(9181,-5911){\makebox(0,0)[lb]{\smash{{\SetFigFont{11}{13.2}{\rmdefault}{\mddefault}{\updefault}{\color[rgb]{0,0,0}Each valuation}}}}}
\put(9181,-6991){\makebox(0,0)[lb]{\smash{{\SetFigFont{11}{13.2}{\rmdefault}{\mddefault}{\updefault}{\color[rgb]{0,0,0}with blobs on each}}}}}
\end{picture}   \else
  \begin{picture}(0,0)\includegraphics{v3.pstex}\end{picture}\setlength{\unitlength}{1450sp}\begingroup\makeatletter\ifx\SetFigFont\undefined \gdef\SetFigFont#1#2#3#4#5{\reset@font\fontsize{#1}{#2pt}\fontfamily{#3}\fontseries{#4}\fontshape{#5}\selectfont}\fi\endgroup \begin{picture}(13106,8580)(259,-11503)
\put(9181,-6451){\makebox(0,0)[lb]{\smash{{\SetFigFont{11}{13.2}{\rmdefault}{\mddefault}{\updefault}{\color[rgb]{0,0,0}is drawn just as  itself,}}}}}
\put(9181,-7531){\makebox(0,0)[lb]{\smash{{\SetFigFont{11}{13.2}{\rmdefault}{\mddefault}{\updefault}{\color[rgb]{0,0,0}, indicative}}}}}
\put(9181,-8071){\makebox(0,0)[lb]{\smash{{\SetFigFont{11}{13.2}{\rmdefault}{\mddefault}{\updefault}{\color[rgb]{0,0,0}of the weight of }}}}}
\put(9766,-8611){\makebox(0,0)[lb]{\smash{{\SetFigFont{11}{13.2}{\rmdefault}{\mddefault}{\updefault}{\color[rgb]{0,0,0},}}}}}
\put(11071,-10456){\makebox(0,0)[lb]{\smash{{\SetFigFont{11}{13.2}{\rmdefault}{\mddefault}{\updefault}{\color[rgb]{0,0,0}}}}}}
\put(9181,-10456){\makebox(0,0)[lb]{\smash{{\SetFigFont{11}{13.2}{\rmdefault}{\mddefault}{\updefault}{\color[rgb]{0,0,0}E.g., }}}}}
\put(9181,-8611){\makebox(0,0)[lb]{\smash{{\SetFigFont{11}{13.2}{\rmdefault}{\mddefault}{\updefault}{\color[rgb]{0,0,0}(}}}}}
\put(11881,-3346){\makebox(0,0)[lb]{\smash{{\SetFigFont{11}{13.2}{\rmdefault}{\mddefault}{\updefault}{\color[rgb]{0,0,0}}}}}}
\put(11386,-9286){\makebox(0,0)[lb]{\smash{{\SetFigFont{11}{13.2}{\rmdefault}{\mddefault}{\updefault}{\color[rgb]{0,0,0}).}}}}}
\put(12331,-3931){\makebox(0,0)[lb]{\smash{{\SetFigFont{11}{13.2}{\rmdefault}{\mddefault}{\updefault}{\color[rgb]{0,0,0}}}}}}
\put(11431,-8611){\makebox(0,0)[lb]{\smash{{\SetFigFont{11}{13.2}{\rmdefault}{\mddefault}{\updefault}{\color[rgb]{0,0,0},}}}}}
\put(9181,-5191){\makebox(0,0)[lb]{\smash{{\SetFigFont{11}{13.2}{\rmdefault}{\mddefault}{\updefault}{\color[rgb]{0,0,0}Legend:}}}}}
\put(11881,-4381){\makebox(0,0)[lb]{\smash{{\SetFigFont{11}{13.2}{\rmdefault}{\mddefault}{\updefault}{\color[rgb]{0,0,0}}}}}}
\put(11521,-3931){\makebox(0,0)[lb]{\smash{{\SetFigFont{11}{13.2}{\rmdefault}{\mddefault}{\updefault}{\color[rgb]{0,0,0}}}}}}
\put(9766,-9286){\makebox(0,0)[lb]{\smash{{\SetFigFont{11}{13.2}{\rmdefault}{\mddefault}{\updefault}{\color[rgb]{0,0,0},}}}}}
\put(10351,-3976){\makebox(0,0)[lb]{\smash{{\SetFigFont{11}{13.2}{\rmdefault}{\mddefault}{\updefault}{\color[rgb]{0,0,0}}}}}}
\put(9181,-3976){\makebox(0,0)[lb]{\smash{{\SetFigFont{11}{13.2}{\rmdefault}{\mddefault}{\updefault}{\color[rgb]{0,0,0}Here,}}}}}
\put(9181,-5911){\makebox(0,0)[lb]{\smash{{\SetFigFont{11}{13.2}{\rmdefault}{\mddefault}{\updefault}{\color[rgb]{0,0,0}Each valuation}}}}}
\put(9181,-6991){\makebox(0,0)[lb]{\smash{{\SetFigFont{11}{13.2}{\rmdefault}{\mddefault}{\updefault}{\color[rgb]{0,0,0}with blobs on each}}}}}
\end{picture}   \fi
  \caption{Part of the Hasse Diagram of }
  \label{fig:v3}
\end{figure}

In fact, we do not know whether  is an -domain when
 is even the simple poset  ( and 
incomparable, , see Figure~\ref{fig:v3},
right)---but it is an -domain.  For a more complex (arbitrarily
chosen) example, take  to be the finite pointed poset of
Figure~\ref{fig:ex1}~: then  and 
are continuous and stably compact, but not known to be -domains
or -domains (and they are much harder to visualize, too).

\begin{figure}
  \centering
  \ifpdf
  \begin{picture}(0,0)\includegraphics{ex1.pdftex}\end{picture}\setlength{\unitlength}{2368sp}\begingroup\makeatletter\ifx\SetFigFont\undefined \gdef\SetFigFont#1#2#3#4#5{\reset@font\fontsize{#1}{#2pt}\fontfamily{#3}\fontseries{#4}\fontshape{#5}\selectfont}\fi\endgroup \begin{picture}(11076,2847)(480,-4642)
\put(2176,-3361){\makebox(0,0)[lb]{\smash{{\SetFigFont{11}{13.2}{\rmdefault}{\mddefault}{\updefault}{\color[rgb]{0,0,0}}}}}}
\put(2701,-3361){\makebox(0,0)[lb]{\smash{{\SetFigFont{11}{13.2}{\rmdefault}{\mddefault}{\updefault}{\color[rgb]{0,0,0}}}}}}
\put(2176,-2761){\makebox(0,0)[lb]{\smash{{\SetFigFont{11}{13.2}{\rmdefault}{\mddefault}{\updefault}{\color[rgb]{0,0,0}}}}}}
\put(2701,-2761){\makebox(0,0)[lb]{\smash{{\SetFigFont{11}{13.2}{\rmdefault}{\mddefault}{\updefault}{\color[rgb]{0,0,0}}}}}}
\put(2401,-2491){\makebox(0,0)[lb]{\smash{{\SetFigFont{11}{13.2}{\rmdefault}{\mddefault}{\updefault}{\color[rgb]{0,0,0}}}}}}
\put(2026,-2116){\makebox(0,0)[lb]{\smash{{\SetFigFont{11}{13.2}{\rmdefault}{\mddefault}{\updefault}{\color[rgb]{0,0,0}}}}}}
\put(2101,-3886){\makebox(0,0)[lb]{\smash{{\SetFigFont{11}{13.2}{\rmdefault}{\mddefault}{\updefault}{\color[rgb]{0,0,0}}}}}}
\put(6451,-3307){\makebox(0,0)[lb]{\smash{{\SetFigFont{11}{13.2}{\familydefault}{\mddefault}{\updefault}{\color[rgb]{0,0,0}}}}}}
\put(6451,-3672){\makebox(0,0)[lb]{\smash{{\SetFigFont{11}{13.2}{\familydefault}{\mddefault}{\updefault}{\color[rgb]{0,0,0}}}}}}
\put(6451,-4036){\makebox(0,0)[lb]{\smash{{\SetFigFont{11}{13.2}{\familydefault}{\mddefault}{\updefault}{\color[rgb]{0,0,0}}}}}}
\put(4726,-3307){\makebox(0,0)[lb]{\smash{{\SetFigFont{11}{13.2}{\familydefault}{\mddefault}{\updefault}{\color[rgb]{0,0,0}}}}}}
\put(4726,-3672){\makebox(0,0)[lb]{\smash{{\SetFigFont{11}{13.2}{\familydefault}{\mddefault}{\updefault}{\color[rgb]{0,0,0}}}}}}
\put(4726,-4036){\makebox(0,0)[lb]{\smash{{\SetFigFont{11}{13.2}{\familydefault}{\mddefault}{\updefault}{\color[rgb]{0,0,0}}}}}}
\put(10726,-3307){\makebox(0,0)[lb]{\smash{{\SetFigFont{11}{13.2}{\familydefault}{\mddefault}{\updefault}{\color[rgb]{0,0,0}}}}}}
\put(10726,-3672){\makebox(0,0)[lb]{\smash{{\SetFigFont{11}{13.2}{\familydefault}{\mddefault}{\updefault}{\color[rgb]{0,0,0}}}}}}
\put(10726,-4036){\makebox(0,0)[lb]{\smash{{\SetFigFont{11}{13.2}{\familydefault}{\mddefault}{\updefault}{\color[rgb]{0,0,0}}}}}}
\put(9001,-3307){\makebox(0,0)[lb]{\smash{{\SetFigFont{11}{13.2}{\familydefault}{\mddefault}{\updefault}{\color[rgb]{0,0,0}}}}}}
\put(9001,-3672){\makebox(0,0)[lb]{\smash{{\SetFigFont{11}{13.2}{\familydefault}{\mddefault}{\updefault}{\color[rgb]{0,0,0}}}}}}
\put(9001,-4036){\makebox(0,0)[lb]{\smash{{\SetFigFont{11}{13.2}{\familydefault}{\mddefault}{\updefault}{\color[rgb]{0,0,0}}}}}}
\put(901,-2686){\makebox(0,0)[lb]{\smash{{\SetFigFont{11}{13.2}{\rmdefault}{\mddefault}{\updefault}{\color[rgb]{0,0,0}}}}}}
\put(676,-2086){\makebox(0,0)[lb]{\smash{{\SetFigFont{11}{13.2}{\rmdefault}{\mddefault}{\updefault}{\color[rgb]{0,0,0}}}}}}
\put(1276,-2086){\makebox(0,0)[lb]{\smash{{\SetFigFont{11}{13.2}{\rmdefault}{\mddefault}{\updefault}{\color[rgb]{0,0,0}}}}}}
\put(826,-3361){\makebox(0,0)[lb]{\smash{{\SetFigFont{11}{13.2}{\rmdefault}{\mddefault}{\updefault}{\color[rgb]{0,0,0}}}}}}
\put(5701,-2011){\makebox(0,0)[lb]{\smash{{\SetFigFont{11}{13.2}{\familydefault}{\mddefault}{\updefault}{\color[rgb]{0,0,0}}}}}}
\put(9001,-2011){\makebox(0,0)[lb]{\smash{{\SetFigFont{11}{13.2}{\familydefault}{\mddefault}{\updefault}{\color[rgb]{0,0,0}}}}}}
\put(10726,-2011){\makebox(0,0)[lb]{\smash{{\SetFigFont{11}{13.2}{\familydefault}{\mddefault}{\updefault}{\color[rgb]{0,0,0}}}}}}
\put(751,-4561){\makebox(0,0)[lb]{\smash{{\SetFigFont{11}{13.2}{\familydefault}{\mddefault}{\updefault}{\color[rgb]{0,0,0} A finite poset}}}}}
\put(4051,-4561){\makebox(0,0)[lb]{\smash{{\SetFigFont{11}{13.2}{\familydefault}{\mddefault}{\updefault}{\color[rgb]{0,0,0} The non-continuous dcpo }}}}}
\put(9376,-4561){\makebox(0,0)[lb]{\smash{{\SetFigFont{11}{13.2}{\familydefault}{\mddefault}{\updefault}{\color[rgb]{0,0,0} }}}}}
\end{picture}   \else
  \begin{picture}(0,0)\includegraphics{ex1.pstex}\end{picture}\setlength{\unitlength}{2368sp}\begingroup\makeatletter\ifx\SetFigFont\undefined \gdef\SetFigFont#1#2#3#4#5{\reset@font\fontsize{#1}{#2pt}\fontfamily{#3}\fontseries{#4}\fontshape{#5}\selectfont}\fi\endgroup \begin{picture}(11076,2847)(480,-4642)
\put(2176,-3361){\makebox(0,0)[lb]{\smash{{\SetFigFont{11}{13.2}{\rmdefault}{\mddefault}{\updefault}{\color[rgb]{0,0,0}}}}}}
\put(2701,-3361){\makebox(0,0)[lb]{\smash{{\SetFigFont{11}{13.2}{\rmdefault}{\mddefault}{\updefault}{\color[rgb]{0,0,0}}}}}}
\put(2176,-2761){\makebox(0,0)[lb]{\smash{{\SetFigFont{11}{13.2}{\rmdefault}{\mddefault}{\updefault}{\color[rgb]{0,0,0}}}}}}
\put(2701,-2761){\makebox(0,0)[lb]{\smash{{\SetFigFont{11}{13.2}{\rmdefault}{\mddefault}{\updefault}{\color[rgb]{0,0,0}}}}}}
\put(2401,-2491){\makebox(0,0)[lb]{\smash{{\SetFigFont{11}{13.2}{\rmdefault}{\mddefault}{\updefault}{\color[rgb]{0,0,0}}}}}}
\put(2026,-2116){\makebox(0,0)[lb]{\smash{{\SetFigFont{11}{13.2}{\rmdefault}{\mddefault}{\updefault}{\color[rgb]{0,0,0}}}}}}
\put(2101,-3886){\makebox(0,0)[lb]{\smash{{\SetFigFont{11}{13.2}{\rmdefault}{\mddefault}{\updefault}{\color[rgb]{0,0,0}}}}}}
\put(6451,-3307){\makebox(0,0)[lb]{\smash{{\SetFigFont{11}{13.2}{\familydefault}{\mddefault}{\updefault}{\color[rgb]{0,0,0}}}}}}
\put(6451,-3672){\makebox(0,0)[lb]{\smash{{\SetFigFont{11}{13.2}{\familydefault}{\mddefault}{\updefault}{\color[rgb]{0,0,0}}}}}}
\put(6451,-4036){\makebox(0,0)[lb]{\smash{{\SetFigFont{11}{13.2}{\familydefault}{\mddefault}{\updefault}{\color[rgb]{0,0,0}}}}}}
\put(4726,-3307){\makebox(0,0)[lb]{\smash{{\SetFigFont{11}{13.2}{\familydefault}{\mddefault}{\updefault}{\color[rgb]{0,0,0}}}}}}
\put(4726,-3672){\makebox(0,0)[lb]{\smash{{\SetFigFont{11}{13.2}{\familydefault}{\mddefault}{\updefault}{\color[rgb]{0,0,0}}}}}}
\put(4726,-4036){\makebox(0,0)[lb]{\smash{{\SetFigFont{11}{13.2}{\familydefault}{\mddefault}{\updefault}{\color[rgb]{0,0,0}}}}}}
\put(10726,-3307){\makebox(0,0)[lb]{\smash{{\SetFigFont{11}{13.2}{\familydefault}{\mddefault}{\updefault}{\color[rgb]{0,0,0}}}}}}
\put(10726,-3672){\makebox(0,0)[lb]{\smash{{\SetFigFont{11}{13.2}{\familydefault}{\mddefault}{\updefault}{\color[rgb]{0,0,0}}}}}}
\put(10726,-4036){\makebox(0,0)[lb]{\smash{{\SetFigFont{11}{13.2}{\familydefault}{\mddefault}{\updefault}{\color[rgb]{0,0,0}}}}}}
\put(9001,-3307){\makebox(0,0)[lb]{\smash{{\SetFigFont{11}{13.2}{\familydefault}{\mddefault}{\updefault}{\color[rgb]{0,0,0}}}}}}
\put(9001,-3672){\makebox(0,0)[lb]{\smash{{\SetFigFont{11}{13.2}{\familydefault}{\mddefault}{\updefault}{\color[rgb]{0,0,0}}}}}}
\put(9001,-4036){\makebox(0,0)[lb]{\smash{{\SetFigFont{11}{13.2}{\familydefault}{\mddefault}{\updefault}{\color[rgb]{0,0,0}}}}}}
\put(901,-2686){\makebox(0,0)[lb]{\smash{{\SetFigFont{11}{13.2}{\rmdefault}{\mddefault}{\updefault}{\color[rgb]{0,0,0}}}}}}
\put(676,-2086){\makebox(0,0)[lb]{\smash{{\SetFigFont{11}{13.2}{\rmdefault}{\mddefault}{\updefault}{\color[rgb]{0,0,0}}}}}}
\put(1276,-2086){\makebox(0,0)[lb]{\smash{{\SetFigFont{11}{13.2}{\rmdefault}{\mddefault}{\updefault}{\color[rgb]{0,0,0}}}}}}
\put(826,-3361){\makebox(0,0)[lb]{\smash{{\SetFigFont{11}{13.2}{\rmdefault}{\mddefault}{\updefault}{\color[rgb]{0,0,0}}}}}}
\put(5701,-2011){\makebox(0,0)[lb]{\smash{{\SetFigFont{11}{13.2}{\familydefault}{\mddefault}{\updefault}{\color[rgb]{0,0,0}}}}}}
\put(9001,-2011){\makebox(0,0)[lb]{\smash{{\SetFigFont{11}{13.2}{\familydefault}{\mddefault}{\updefault}{\color[rgb]{0,0,0}}}}}}
\put(10726,-2011){\makebox(0,0)[lb]{\smash{{\SetFigFont{11}{13.2}{\familydefault}{\mddefault}{\updefault}{\color[rgb]{0,0,0}}}}}}
\put(751,-4561){\makebox(0,0)[lb]{\smash{{\SetFigFont{11}{13.2}{\familydefault}{\mddefault}{\updefault}{\color[rgb]{0,0,0} A finite poset}}}}}
\put(4051,-4561){\makebox(0,0)[lb]{\smash{{\SetFigFont{11}{13.2}{\familydefault}{\mddefault}{\updefault}{\color[rgb]{0,0,0} The non-continuous dcpo }}}}}
\put(9376,-4561){\makebox(0,0)[lb]{\smash{{\SetFigFont{11}{13.2}{\familydefault}{\mddefault}{\updefault}{\color[rgb]{0,0,0} }}}}}
\end{picture}   \fi
  \caption{Poset Examples}
  \label{fig:ex1}
\end{figure}

No progress seems to have been made on the question since Jung and
Tix' 1998 attempt.  As part of our results, we show that for every
finite pointed poset , e.g.\ Figure~\ref{fig:ex1}~,  is a continuous -domain.
This is also one of the basic results that we then leverage to show
that  is an -domain for {\em any\/}
-domain, in particular every -domain, not just every
finite pointed poset, .



One may obtain some intuition as to why this should be so, and at the
same time give an idea of what ()-domains are.  Let 
be a finite pointed poset.  In attempting to show that  is
an -domain, we are led to study the so-called {\em deflations\/}
, i.e., the continuous maps  with
{\em finite\/} range such that  for every continuous
probability valuation  on , and we must try to find deflations
 such that  is as close as one desires to .  All
natural definitions of  fail to be continuous, and in fact to be
monotonic.  (E.g., Graham's construction \cite{Graham:rb:V} is not
monotonic, see Jung and Tix.)  Looking for maps  such that  is instead a finite, non-empty {\em set\/} of valuations below
 shows more promise---the monotonicity requirements are slightly
more relaxed.  Such a set-valued function is what we call a {\em
  quasi-deflation\/} below.  For example, one may think of fixing  ( in Figure~\ref{fig:v3}), and mapping  to the
collection of all valuations  below  such that the measure
of any subset is a multiple of , keeping only those  that
are maximal.  (Pick them from the left of Figure~\ref{fig:v3}, in our
example.)  This still does not provide anything monotonic, but we
managed to show that one can indeed approximate every element  of
, continuously in , using quasi-deflations.  The
proof is non-trivial, and rests on deep properties relating
-domains and \emph{quasi-retractions}, all notions that we
define and study.



\subsection{Outline}

We introduce most of the required notions in Section~2.  Since we
shall only start studying the probabilistic powerdomain in
Section~\ref{sec:qretr:V}, we shall refrain from defining valuations,
probabilities, and related concepts until then.

We introduce -domains in Section~\ref{sec:qrb}.  They are
defined just as -domains are, only with a flavor of ``quasi'',
i.e., replacing approximating elements by approximating {\em sets\/}
of elements.  We establish their main properties there, in particular
that they are quasi-continuous, stably compact, and Lawson-compact.
Much as -domains are also characterized as the retracts of
bifinite domains, we show that, up to a few details, the
-domains are the quasi-retracts of bifinite domains in
Section~\ref{sec:qretr}.  This allows us to parenthesize  as
quasi-(retract of bifinite domain) or as (quasi-retract) of bifinite
domain.  Quasi-retractions are an essential concept in the study of
-domains, and we introduce them here, as well as the related
notion of quasi-projections---images by proper maps.

We also show that the category of countably-based -domains is
closed under finite products (easy) and taking bilimits of expanding
sequences (hard, but similar to the case of -domains) in
Section~\ref{sec:bilim}.

The core of the paper is Section~\ref{sec:qretr:V}, where we show that
the category  of countably-based -domains is closed
under the probabilistic powerdomain construction.  This capitalizes on
all previous sections, and will follow from a variant of Jung and Tix'
result that  is an -domain whenever  is a finite
tree, and applying suitable quasi-projections and bilimits.
The key result will then be Theorem~\ref{thm:qretr:V}, which shows
that for any quasi-projection  of a stably compact space ,
 is again a quasi-projection of , again up to
a few details.


We conclude in Section~\ref{sec:conc}.

\subsection{Other Related Work}

Instead of solving the Jung-Tix problem, one may try to circumvent it.
One of the most successful such attempts led to the discovery of {\em
  qcb-spaces\/} \cite{BSS:qcb} and to compactly generated
countably-based monotone convergence spaces \cite{BSS:cgdom}, as
Cartesian-closed categories of topological spaces where a reasonable
amount of semantics can be done.  This provides exciting new
perspectives.  The category of qcb-spaces accommodates two
probabilistic powerdomains \cite{BS:twoval}.  The observationally
induced one is essentially  (with the weak topology), but
differs from the one obtained as a free algebra.


\section{Preliminaries}
\label{sec:prelim}

We refer to \cite{AJ:domains,GHKLMS:contlatt,Mislove:topo:CS} for
background material.
A {\em poset\/}  is a set with a partial ordering .  Let  be the downward closure ; we write  for , when .  The
upward closures ,  are defined similarly.  When ,  is {\em below\/}  and  is {\em above\/} .  
is {\em pointed\/} iff it has a least element .  A {\em dcpo\/}
is a poset  where every directed family  has a
least upper bound ; directedness means that  and for every , there is an  such
that .

Every poset, and more generally each preordered set  comes with a
topology, whose opens  are the upward closed subsets such that, for
every directed family  that has a least upper bound
in ,  for some .  This is the {\em Scott
  topology\/}.  When we see a poset or dcpo  as a topological
space, we will implicitly assume the latter, unless marked otherwise.

There is a deep connection between order and topology.  Given any topological space , its {\em specialization preorder\/}
 is defined by  iff every open containing  also
contains .   is  iff  is an ordering, i.e.,  and  imply .  The specialization preorder of a
dcpo  (with ordering , and equipped with its Scott topology),
is the original ordering .

A subset  of a topological space  is {\em saturated\/} iff it is
the intersection of all opens  containing .  Equivalently, 
is upward closed in the specialization preorder \cite[Remark
after Definition~4.34]{Mislove:topo:CS}.  So we can, and shall often
prove inclusions  where  is upward closed by showing
that every open  containing  also contains .

A map  between topological spaces is {\em continuous\/}
iff  is open for every open subset  of .  Every
continuous map is monotonic with respect to the underlying
specialization preorders.  When  and  are preordered sets, it is
equivalent to require  to be {\em Scott-continuous\/}, i.e., to be
monotonic and to preserve existing directed least upper bounds.  A
{\em homeomorphism\/} is a bijective continuous map whose inverse is
also continuous.

Given a set , and a family  of subsets of , there is
a smallest topology containing : then  is a
{\em subbase\/} of the topology, and its elements are the {\em
  subbasic opens\/}.  To show that  is continuous, it is
enough to show that the inverse image of every subbasic open of  is
open in .  A subbase  is a \emph{base} if and only if
every open is a union of elements of .  This is the case,
for example, if  is closed under finite intersections.

The {\em interior\/}  of a subset  of a topological
space  is the largest open contained in .   is a
\emph{neighborhood} of  if and only if , and a
neighborhood of a subset  if and only if .
A subset  of a topological space  is {\em compact\/} iff one can
extract a finite subcover from every open cover of .  The important
ones are the {\em saturated\/} compacts.
 is {\em locally compact\/} iff for each open  and each , there is a compact saturated subset  such that  and .  In any locally compact space, we have the
following interpolation property: whenever  is a compact subset of
some open , then there is a compact saturated subset  such
that .

 is {\em sober\/} iff every irreducible closed subset is the
closure of a unique point; in the presence of local compactness (and
when  is ),
it is equivalent to require that  be {\em well-filtered\/}
\cite[Theorem~II-1.21]{GHKLMS:contlatt}, i.e., to require that, for
every open , for every filtered family  of
saturated compacts such that ,  for some  already.  We say
that the family is {\em filtered\/} iff it is directed in the
 ordering, and make it explicit by using
 as superscript.  (Symmetrically, we write  for
directed unions.)

Given a topological space , let  be the collection of
all non-empty compact saturated subsets  of .  There are two
prominent topologies one can put on .  The {\em upper
  Vietoris\/} topology has a subbase of opens of the form ,
 open in , where we write  for the collection of compact
saturated subsets  included in .  We shall write  for
the space  with the upper Vietoris topology, and call it
the \emph{Smyth powerspace}.  The specialization ordering of 
is reverse inclusion .  On the other hand, we shall reserve
the notation  for the {\em Smyth powerdomain\/} of
, which is equipped with the Scott topology of  instead.
When  is well-filtered,  is a dcpo, with least upper
bounds of directed families computed as filtered intersections, and
 is Scott-open for every open subset  of , i.e., the
Scott topology is finer than the upper Vietoris topology.  When  is
locally compact and sober (in particular, well-filtered), the two
topologies coincide, and  is then a continuous dcpo
(see below), where  iff 
\cite[Proposition~I-1.24.2]{GHKLMS:contlatt}.  Schalk
\cite[Chapter~7]{Schalk:PhD} provides a deep study of these spaces.

For every finite subset  of a topological space ,  is compact
and  is saturated compact in .  We call {\em finitary
  compact\/} those subsets of the form  with  finite, and
let  be the subset of  consisting of the
non-empty finitary compacts.   can be topologized with the
subspace topology from , in which case we obtain a space we
write , or with the Scott topology of reverse inclusion
, yielding a space that we write .

Given any poset , any finite subset  of , and any element 
of , we write  iff , i.e., iff there is a
 such that .  Given any upward closed subset  of
, we shall write  iff for every directed family
 that has a least upper bound above , then 
is in  for some .  Then a finite set  {\em
  approximates\/}  iff .  This is usually written  in the literature.
We shall also write , when , as shorthand for .  This is the more familiar way-below relation, and a poset
is {\em continuous\/} if and only if the set  of all
elements  such that  is directed and has  as least
upper bound.  One should be aware that  means that the
elements of  approximate  {\em collectively\/}, while none in
particular may approximate  individually.  E.g., in the poset
 (Figure~\ref{fig:ex1}~), the sets  approximate , for all ; but , .

It may be helpful to realize that  can also be presented in
the following equivalent way.  Given two finitary compacts 
and ,  if and only if for every , there is an  such that , and then we
write : this is the so-called {\em Smyth
  preorder\/}.  Then we can equate the finitary compacts  with the equivalence classes of finite subsets , up to the
equivalence  defined by  iff 
iff  and , declare that 
is the set of equivalence classes of non-empty finite sets, ordered by
.  But the approach based on finitary compacts is
mathematically smoother.

Among the Cartesian-closed categories of continuous dcpos, one finds
in particular the -domains (a.k.a., the bifinite domains), the
-domains, i.e., the retracts of bifinite domains
\cite[Section~4.2.1]{AJ:domains}, and the -domains
\cite[Section~4.2.2]{AJ:domains}\cite[Section~II.2]{GHKLMS:contlatt}.
There are several equivalent definitions of the first two.

For our purposes, an {\em -domain\/} is a pointed dcpo  with a
directed family  of deflations such that  \cite[Exercise~4.3.11(9)]{AJ:domains}.  A
{\em deflation\/}  on  is a continuous map from  to  such
that  for every , and that has {\em finite
  image\/}.  We order deflations, as well as all maps with codomain a
poset, pointwise: i.e.,  iff  for every ; knowing this, directed families and least upper bounds of
deflations make sense.  Every -domain is a continuous dcpo, and
 for every  and every .

A {\em -domain\/} is defined similarly, except the deflations
 are now required to be {\em idempotent\/}, i.e.,  \cite[Theorem~4.2.6]{AJ:domains}.  This implies that , i.e., that all the elements  are finite; hence
all bifinite domains are also algebraic.  Every bifinite domain is an
-domain.  Conversely, the -domains are exactly the retracts
of bifinite domains: we shall define what this means and extend this
in Section~\ref{sec:qretr}.

An {\em -domain\/} is defined similarly again, except the
functions  are no longer deflations, but continuous functions
that are {\em finitely separated from \/}.  That is, we
now require that there is a finite set  such that for every , there is an  such that .  We
say that  is {\em finitely separating\/} for  on .

Every deflation is finitely separated from : take 
to be the image of .  The converse fails.  E.g., for every
, the function  is
finitely separated from the identity on , but is not a
deflation \cite[Section~3.2]{JT:troublesome}.  Every -domain is
an -domain.  The converse is not known.

A {\em quasi-continuous dcpo\/}  (see \cite{GLS:quasicont} or
\cite[Definition~III-3.2]{GHKLMS:contlatt}) is a dcpo such that, for
every , the collection of all  that
approximate  () is directed (w.r.t.\ )
and their least upper bound in  is , i.e.,
.  The theory of quasi-continuous dcpos is less well explored than that
of {\em continuous dcpos\/}, but quasi-continuous dcpos retain many of
the properties of the latter.  (Every continuous dcpo is
quasi-continuous, but not conversely.  A counterexample is given by
, see Figure~\ref{fig:ex1}~.)
Every quasi-continuous dcpo  is locally compact and sober in its
Scott topology \cite[III-3.7]{GHKLMS:contlatt}.  In a quasi-continuous
dcpo , for every , the set 
defined as , is open, and equals the
interior  \cite[III-3.6(ii)]{GHKLMS:contlatt};
every open  is the union of all the subsets  with  contained in  \cite[III-5.6]{GHKLMS:contlatt}; and
for every compact saturated subset  and every open subset 
containing , there is a finitary compact subset  of 
such that  and 
\cite[III-5.7]{GHKLMS:contlatt}.
In particular, .  Another consequence is {\em
  interpolation\/}: writing  for 
for every  in  (equivalently, ),
if  in a quasi-continuous dcpo , for some , and , then  for
some .

If  is a quasi-continuous dcpo, the formula , valid
for every , shows that  is the filtered
intersection of its finitary compact neighborhoods, equivalently the
directed least upper bound of those non-empty finitary compacts  () that are way-below .  In other words, the
finitary compacts form a {\em basis\/} of .



\section{-Domains}
\label{sec:qrb}

We model -domains after -domains, replacing single
approximating elements , where  is a deflation, by
finite subsets, as in quasi-continuous dcpos.

\begin{defi}[-Domain]
  \label{defn:qrb}
  A {\em quasi-deflation\/} on a poset  is a continuous map
   such that  for
  every , and 
  is finite.

  A {\em -domain\/} is a pointed dcpo  with a {\em generating
    family of quasi-deflations\/}, i.e., a directed family of
  quasi-deflations  with  for each .
\end{defi}
We order quasi-deflations pointwise, i.e.,  iff
 for every .  Above, we write
 instead of  to stress the fact that the
family  of which we are taking the
intersection is {\em filtered\/}, i.e., for any two ,
there is an  such that  is contained
in both  and .  It is
equivalent to say that  is directed in
the  ordering of .

One can see the finitary compacts  as being smaller and
smaller upward closed sets containing .  The intersection
 is then just the least
upper bound of  in the Smyth
powerdomain .  On the other hand,  embeds into  by equating  with .  Modulo this
identification, the condition  requires that  is the least upper bound of  in .

That  is continuous means that  is monotonic ( implies ), and that for
every directed family  of elements of ,  is equal to ---this implies that the latter is finitary compact, in
particular.

\begin{prop}
  \label{prop:FS:QRB}
  Every -domain is a -domain.
\end{prop}
\begin{proof}
  Given a directed family of deflations ,
  define  as .  If , then
   for every , so
   is directed.  Also,  is the set of upper bounds of
  , of which the least is .  So this set is
  exactly .
\end{proof}
We shall improve on this in Theorem~\ref{thm:ctrlQRB}, which implies
that not only the -domains, but all -domains, are
-domains.







For any deflation , and more generally whenever  is finitely
separated from the identity,  is way-below 
\cite[Lemma~II-2.16]{GHKLMS:contlatt}.  Similarly:
\begin{lem}
  \label{lemma:qdefl:ll}
  Let  be a poset, and  be a quasi-deflation on .  For
  every , .
\end{lem}
\begin{proof}
  Let  be a directed family having a least upper
  bound above .  Since  is continuous, .  But since
   is finite, there are only finitely many sets , .  So  for
  some .  Since , .
\end{proof}

\begin{cor}
  \label{corl:qrb:qcont}
  Every -domain is quasi-continuous.
\end{cor}
In general, -domains are not continuous.  E.g., 
(Figure~\ref{fig:ex1}~) is not continuous.  However,  is a -domain: for all , take , , .  Then
 is the desired directed family of
quasi-deflations.

-domains, and more generally -domains, are not just
continuous domains, they are {\em stably compact\/}, i.e., locally
compact, sober, compact and coherent (see, e.g.,
\cite[Theorem~4.2.18]{AJ:domains}).  We say that a topological space
is {\em coherent\/} iff the intersection of any two compact saturated
subsets is compact (and saturated).  In a stably compact space, the
intersection of any family of compact saturated subsets is compact.
We show that -domains are stably compact as well.

Since every quasi-continuous dcpo is locally compact and sober
\cite[Proposition~III-3.7]{GHKLMS:contlatt}, and also compact since
pointed, only coherence remains to be shown.  For this, we need the
following consequence of Rudin's Lemma, a finitary form of
well-filteredness:
\begin{prop}[\protect{\cite[Corollary~III-3.4]{GHKLMS:contlatt}}]
  \label{prop:heckmann}
  Let  be a dcpo,  be a directed family in
  .  For every open subset  of , if , then 
  for some .
\end{prop}

It follows that, if  is a dcpo, then the Scott topology on  is finer than the upper Vietoris topology.  Indeed, this reduces
to showing that  is Scott-open in ,
for every open subset  of .  And this is
Proposition~\ref{prop:heckmann}, plus the easily checked fact that
 is upward closed in .
\begin{cor} \label{corl:BoxU}
  Let  be a dcpo.  The Scott topology is finer than the upper
  Vietoris topology on , and coincides with it whenever 
  is quasi-continuous.
\end{cor}
\begin{proof}
  It remains to show that, if  is a quasi-continuous dcpo, then
  every Scott-open  of  is open in the upper
  Vietoris topology.  Let  be in .
  It suffices to show that there is an open subset  of  such
  that .  Write .  For each , ,  is the
  filtered intersection of all finitary compacts .
  The unions , with , \ldots, , also form a directed family in , and their
  intersection is .  So there are finitary compacts , \ldots,  whose union is in .  Since  for each , each  is in the
  Scott-open , so  with .  Moreover, : for each ,  is included in
  ; since  is in  and  is
  upward-closed in ,  is in .  \end{proof}

Schalk \cite[Chapter~7]{Schalk:PhD} proved that  defines a monad
on the category of topology spaces (see \cite{Mog91} for an
introduction to monads and their importance in programming language
semantics).  This means first that there is a {\em unit map\/}
---here,  maps  to , and
this is continuous because .  That  is
a monad also means that every continuous map  has
an {\em extension\/} , i.e.,
 is continuous and .  This is
defined by  in our case.
Again,  is continuous, because .  And the {\em monad laws\/} are satisfied:
, ,
and .  One
should be careful here:  is a monad, but  is not a
monad, except on specific subcategories, e.g., sober locally compact
spaces , where  anyway.

The continuity claims in the following lemma are then obvious.
\begin{lem}
  \label{lemma:dagger:new}
  Let ,  be topological spaces.  Given any continuous map , its extension  restricts to a
  continuous map .  If
   is finite, then  maps 
  continuously into .
\end{lem}
\begin{proof}
  In each case, one only needs to show that  maps
  relevant compacts to finitary compacts.  In the first case, for
  every finitary compact , 
  (because  is monotonic), and this is finitary compact.  In the
  second case,  is a
  finite union of finitary compacts since  is finite.
\end{proof}
One would also like  to be continuous from
 to , in the face of the
importance of the Scott topology.  This is a consequence of the above
when  is sober and locally compact, and  is a quasi-continuous
dcpo, since  and  in this case.  However, one can also prove this in a more
general setting, using the following observation.  For each
topological space , write  for  with the Scott
topology of its specialization preorder.  For short, we shall
call {\em quasi monotone convergence space\/} any space  such that
the (Scott) topology on  is finer than that of , i.e.,
such that every open subset of  is open is Scott-open.  This is a
slight relaxation of the notion of {\em monotone convergence space\/},
i.e., of a quasi monotone convergence space that is a dcpo in its
specialization preorder
\cite[Definition~II-3.12]{GHKLMS:contlatt}.  E.g., every sober space
is a monotone convergence space, and in particular a quasi monotone
convergence space.
\begin{lem}
  \label{lemma:fin:cont}
  Let  be a quasi monotone convergence space and  be a
  topological space.  Every continuous map  is
  Scott-continuous, i.e., continuous from  to .
\end{lem}
\begin{proof}
  Since  is continuous, it is monotonic with respect to the
  underlying specialization preorders.  Let  be any
  directed family of elements of , with least upper bound .
  Certainly  is an upper bound of .  Let
  us show that, for any other upper bound , .  It is
  enough to show that every open neighborhood  of  contains
  .  Since ,  is in the open subset ,
  which is Scott-open by assumption, so  for some
  .  It follows that  is in , hence also 
  since  is upward closed.
\end{proof}

When  is sober and locally compact, the topology of  coincides with that of .  In particular, 
is a quasi-monotone convergence space.  Taking  in
Lemma~\ref{lemma:fin:cont}, one obtains the following corollary.
\begin{cor}
  \label{corl:fin:cont:Smyth}
  Let  be a sober, locally compact space, and  be a topological
  space.  Every continuous map from  to  is also
  Scott-continuous from  to .
\end{cor}
Similarly, with :
\begin{cor}
  \label{corl:fin:cont:Fin}
  Let  be a topological space,  be a quasi monotone convergence
  space.  Every continuous map from  to  is
  Scott-continuous, i.e., continuous from  to .
\end{cor}

\begin{lem}
  \label{lemma:psi*}
  Let  be a -domain, and  a
  generating family of quasi-deflations.  For every open subset  of
  , .
\end{lem}
\begin{proof}
  The union is directed, since  whenever  is pointwise below
  , i.e., when  for all .  For every , : every element  of  is
  indeed such that .  Conversely, we
  claim that every element  of  is in 
  for some .  Indeed, , so .  By
  Proposition~\ref{prop:heckmann},  for
  some , i.e., .
\end{proof}

\begin{lem}
  \label{lemma:psi*:comp}
  Let  be a -domain, and  a
  generating family of quasi-deflations.  For every compact saturated
  subset  of , .
\end{lem}
\begin{proof}
  Since  for every
  ,  contains  for every .  So .
  Conversely, since  is saturated, it is enough to show that every
  open  containing  also contains .  Since , by
  Lemma~\ref{lemma:psi*}, .  By compactness,  for some , i.e., for every , .  So .
\end{proof}

\begin{prop}
  \label{prop:qrb:Smyth}
  For every -domain ,  is an -domain.
\end{prop}
\begin{proof}
  Assume  is a -domain, with generating family of
  quasi-deflations .  The family
   is directed, since if 
  is below , i.e., if  for every , then .  Since  is quasi-continuous
  (Corollary~\ref{corl:qrb:qcont}), it is sober and locally compact.
  So Corollary~\ref{corl:fin:cont:Smyth} applies, showing that
   is Scott-continuous from  to .  Lemma~\ref{lemma:psi*:comp} states that the least upper bound
  of  is the identity on .  Clearly,  has finite image.  So  is an -domain.
\end{proof}

\begin{thm}
  \label{thm:qrb:scomp}
  Every -domain is stably compact.
\end{thm}
\begin{proof}
  Let  be a -domain, with generating family of
  quasi-deflations .  We claim that, given
  any two compact saturated subsets  and  of , 
  is again compact saturated.  This is obvious if  is
  empty.  Otherwise, writing  for the upward closure of an
  element  of a poset ,  is an intersection of two finitary compacts in .
  Since  is a quasi-continuous dcpo by
  Corollary~\ref{corl:qrb:qcont},  is sober and locally compact, so
  .  Moreover,  is an
  -domain (Proposition~\ref{prop:qrb:Smyth}), so  is
  coherent.  Therefore 
  is compact saturated in .  It is also non-empty: pick , then  is in .  So  is in .  Now there is a (continuous)
  map  defined as ---this is the so-called {\em multiplication\/} of
  the monad---and  is then an element of , i.e., a compact subset of
  .  We now observe that  is equal to : the left to
  right inclusion is obvious, and conversely every 
  defines an element  of  that is included
  in  and .  So  is compact saturated.  We conclude
  that  is coherent.

   is compact since pointed, and also locally compact and sober, as
  a quasi-continuous dcpo, hence stably compact.  \end{proof}


The {\em Lawson topology\/} is the smallest topology containing both
the Scott-opens and the complements of all finitary compacts .  When  is a quasi-continuous dcpo, since  is
compact saturated and every non-empty compact saturated subset is a
filtered intersection of such sets , the Lawson topology
coincides with the {\em patch topology\/}, i.e., the smallest topology
containing the original Scott topology and all complements of compact
saturated subsets.  Every stably compact space is patch-compact, i.e.,
compact in its patch topology
\cite[Section~VI-6]{GHKLMS:contlatt}.  So:
\begin{cor}
  \label{corl:qrb:lcomp}
  Every -domain is Lawson-compact.
\end{cor}

In the sequel, we shall need some form of countability:
\begin{defi} \label{defn:omega:qrb}
  An {\em -domain\/} is a -domain with a {\em
    countable\/} generating family of quasi-deflations.
\end{defi}

\begin{prop}
  \label{prop:omega:qrb}
  A pointed dcpo  is an -domain iff there is a
  generating {\em sequence\/} of quasi-deflations , i.e., for every , ,
   for every , and
   for every
  .
\end{prop}
\begin{proof}
  Let  be an -domain, and  be
  a countable generating family of quasi-deflations.  Build a sequence
   by letting , and  be any  such that  is above  and , by
  directedness.  Then let  for every .  By construction, whenever ,  is below
  .  And for every ,  is below
  , so  for every .  So
   is the desired sequence.
\end{proof}

Recall that a topological space is \emph{countably-based} if and only
if it has a countable subbase, or equivalently, a countable base.
\begin{prop}
  \label{prop:omega:qrb:2}
  A -domain  is an -domain iff it is
  countably-based.
\end{prop}
\begin{proof}
  Only if: let  be a generating sequence
  of quasi-deflations on .  For each , enumerate  as , and let  be the finite set .
  We claim that the countably many subsets , , , form a base of the topology.

  It is enough to show that, for every open  and every element ,  for some , , such that : since , use
  Proposition~\ref{prop:heckmann} to find  such that
  .  Since  and
   for some , there is a  such that , and .  Repeating the
  argument on , we find  such that .  By Lemma~\ref{lemma:qdefl:ll}, , i.e.,  is in  since  is
  quasi-continuous.  Since ,  is in .

  If: let  be a generating family of
  quasi-deflations on , and assume that the topology of  has a
  countable base .  Assume without loss of
  generality that  for every .  For
  every pair  such that  for some finite set , pick one such finite set and
  call it .  One can enumerate all such pairs as , .  By Lemma~\ref{lemma:psi*:comp}, .  By Proposition~\ref{prop:heckmann},
   for some
  : pick such an  and call it .  By directedness, we
  may also assume that  is also above ,
  .  Define  as .  This yields a
  non-decreasing sequence of quasi-deflations .

  We claim that it is generating.  On one hand,  since each  is a
  quasi-deflation.  Conversely, every open neighborhood  of 
  contains some , , with .  Then , so
   for some .  Write  as , where  is finite.  By
  Lemma~\ref{lemma:qdefl:ll}, , so .  As  is
  open,  for some .
  In particular, .  So  is a pair of the form .  By definition
  .  Since , .  So every open neighborhood  of  contains
   for some , hence .  So , whence the equality.
\end{proof}

\section{Quasi-Retracts of Bifinite Domains}
\label{sec:qretr}

The -domains can be characterized as the retracts of bifinite
domains.  Recall that a {\em retraction\/} of  onto  is a
continuous map  such that there is continuous map  (the {\em section\/}) with  for every .

We shall show that ()-domains are not just closed under
retractions, but under a more relaxed notion that we shall call {\em
  quasi-retractions\/}.  More precisely, our aim in this section is to
show that the -domains are exactly the quasi-retracts of
bifinite domains, up to some details.

\begin{figure}
  \centering
  \ifpdf
  \begin{picture}(0,0)\includegraphics{qretr.pdftex}\end{picture}\setlength{\unitlength}{2368sp}\begingroup\makeatletter\ifx\SetFigFont\undefined \gdef\SetFigFont#1#2#3#4#5{\reset@font\fontsize{#1}{#2pt}\fontfamily{#3}\fontseries{#4}\fontshape{#5}\selectfont}\fi\endgroup \begin{picture}(3675,3906)(2476,-5425)
\put(2626,-1711){\makebox(0,0)[lb]{\smash{{\SetFigFont{11}{13.2}{\rmdefault}{\mddefault}{\updefault}{\color[rgb]{0,0,0}}}}}}
\put(6151,-2161){\makebox(0,0)[lb]{\smash{{\SetFigFont{11}{13.2}{\rmdefault}{\mddefault}{\updefault}{\color[rgb]{0,0,0}}}}}}
\put(5776,-4111){\makebox(0,0)[lb]{\smash{{\SetFigFont{11}{13.2}{\rmdefault}{\mddefault}{\updefault}{\color[rgb]{0,0,0}}}}}}
\put(4426,-3061){\makebox(0,0)[lb]{\smash{{\SetFigFont{11}{13.2}{\rmdefault}{\mddefault}{\updefault}{\color[rgb]{0,0,0}}}}}}
\put(4426,-4486){\makebox(0,0)[lb]{\smash{{\SetFigFont{11}{13.2}{\rmdefault}{\mddefault}{\updefault}{\color[rgb]{0,0,0}}}}}}
\put(4276,-5161){\makebox(0,0)[lb]{\smash{{\SetFigFont{11}{13.2}{\rmdefault}{\mddefault}{\updefault}{\color[rgb]{0,0,0}}}}}}
\put(2476,-5161){\makebox(0,0)[lb]{\smash{{\SetFigFont{11}{13.2}{\rmdefault}{\mddefault}{\updefault}{\color[rgb]{0,0,0}}}}}}
\put(2851,-4336){\makebox(0,0)[lb]{\smash{{\SetFigFont{11}{13.2}{\rmdefault}{\mddefault}{\updefault}{\color[rgb]{0,0,0}}}}}}
\end{picture}   \else
  \begin{picture}(0,0)\includegraphics{qretr.pstex}\end{picture}\setlength{\unitlength}{2368sp}\begingroup\makeatletter\ifx\SetFigFont\undefined \gdef\SetFigFont#1#2#3#4#5{\reset@font\fontsize{#1}{#2pt}\fontfamily{#3}\fontseries{#4}\fontshape{#5}\selectfont}\fi\endgroup \begin{picture}(3675,3906)(2476,-5425)
\put(2626,-1711){\makebox(0,0)[lb]{\smash{{\SetFigFont{11}{13.2}{\rmdefault}{\mddefault}{\updefault}{\color[rgb]{0,0,0}}}}}}
\put(6151,-2161){\makebox(0,0)[lb]{\smash{{\SetFigFont{11}{13.2}{\rmdefault}{\mddefault}{\updefault}{\color[rgb]{0,0,0}}}}}}
\put(5776,-4111){\makebox(0,0)[lb]{\smash{{\SetFigFont{11}{13.2}{\rmdefault}{\mddefault}{\updefault}{\color[rgb]{0,0,0}}}}}}
\put(4426,-3061){\makebox(0,0)[lb]{\smash{{\SetFigFont{11}{13.2}{\rmdefault}{\mddefault}{\updefault}{\color[rgb]{0,0,0}}}}}}
\put(4426,-4486){\makebox(0,0)[lb]{\smash{{\SetFigFont{11}{13.2}{\rmdefault}{\mddefault}{\updefault}{\color[rgb]{0,0,0}}}}}}
\put(4276,-5161){\makebox(0,0)[lb]{\smash{{\SetFigFont{11}{13.2}{\rmdefault}{\mddefault}{\updefault}{\color[rgb]{0,0,0}}}}}}
\put(2476,-5161){\makebox(0,0)[lb]{\smash{{\SetFigFont{11}{13.2}{\rmdefault}{\mddefault}{\updefault}{\color[rgb]{0,0,0}}}}}}
\put(2851,-4336){\makebox(0,0)[lb]{\smash{{\SetFigFont{11}{13.2}{\rmdefault}{\mddefault}{\updefault}{\color[rgb]{0,0,0}}}}}}
\end{picture}   \fi
  \caption{A quasi-retraction}
  \label{fig:qretr}
\end{figure}
For each continuous , define  by .   is
continuous, since  for every
open .  This is the action of the  functor of the Smyth
powerspace monad \cite[Chapter~7]{Schalk:PhD}, equivalently .

\begin{defi}[Quasi-retract]
  \label{defn:qretr}
  A {\em quasi-retraction\/}  of  onto  is a
  continuous map such that there is a continuous map  (the {\em quasi-section\/}) such that  for every .

  A topological space  is a {\em quasi-retract\/} of  iff there
  is a quasi-retraction of  onto .
\end{defi}
In diagram notation, we require the bottom right triangle to commute,
but \emph{not} the top left triangle---what the puncture {\LARGE+}
indicates; the outer square always commutes:

While a section  picks an element  in the inverse
image , continuously, a quasi-section is only required to
pick a non-empty compact saturated collection of elements from  meeting  (see Figure~\ref{fig:qretr}),
continuously again.

Every retraction  (with section ) defines a canonical
quasi-retraction: let , then .

The converse fails.  For example,  is a quasi-retract of
 (see Figure~\ref{fig:ex1}~): 
maps both  and  to ,
and  for every .  But  is not a
retract of :  is a continuous dcpo, and every retract of a
continuous dcpo is again one; recall that  is not
continuous.

Every quasi-retraction  induces a continuous map , which is then a retraction in the Kleisli
category .  A retraction in a category is a morphism
 such that there is a section morphism ,
i.e., one with .  It is easy to see that the
quasi-retractions are exactly those continuous maps  such
that  is a retraction in .

\begin{lem}
  \label{lemma:qretr:surj}
  Every quasi-retraction  onto a  space  is
  surjective.  More precisely, if  is a matching quasi-section,
  then every element  is of the form  for some .
\end{lem}
\begin{proof}
  For every , .  Since ,  for some .  But
   is then in , so .
  Therefore .
\end{proof}

The following is reminiscent of the fact that every retract of a
stably compact space is again stably compact \cite[Proposition, bottom
of p.153, and subsequent discussion]{Lawson:versatile}: we shall show
that any  quasi-retract of a stably compact space is stably
compact.  We start with compactness.
\begin{lem}
  \label{lemma:qretr:comp}
  Every  quasi-retract  of a compact space  is compact.
\end{lem}
\begin{proof}
  The image of a compact set by a continuous map is compact.  Now
  apply Lemma~\ref{lemma:qretr:surj}.
\end{proof}

\begin{lem}
  \label{lemma:qretr:wf}
  Any quasi-retract  of a well-filtered space  is well-filtered.
\end{lem}
\begin{proof}
  Let  be the quasi-retraction, with quasi-section .

  Let  be a filtered family of compact saturated
  subsets of , and assume that , where  is open in .  Let .  This is compact saturated, and forms a directed family,
  since  is monotonic.  We claim that .  Indeed, every  is such that, for every , there is a 
  such that ; then , so , for every .  Since
  ,  is in ,
  whence the claim.

  Since  is well-filtered,  for some .  Then, for every , , so .  So .
\end{proof}

\begin{lem}
  \label{lemma:qretr:coh}
  Any  quasi-retract  of a coherent space  is coherent.
\end{lem}
\begin{proof}
  Let  be the quasi-retraction, with quasi-section .

  We use the fact that  is the identity on
  .  This is a well-known identity on monads: by the monad
  law , and
  since , , and this is
   by the first monad law.

  Let ,  be two compact saturated subsets of .  Then
   is compact saturated in
  , using the fact that  is coherent.  So  is compact saturated in .  We
  claim that , which will finish the proof.  In one direction, every
  element  of  is in : by Lemma~\ref{lemma:qretr:surj}, pick  such that , and observe that  (indeed , where ) and
  .  In the other direction, , since  is the identity on .
\end{proof}

\begin{lem}
  \label{lemma:qretr:lcomp}
  Any quasi-retract  of a locally compact space  is locally
  compact.
\end{lem}
\begin{proof}
  Let  be the quasi-retraction, with quasi-section .  Let  be any point of , and  be an open
  neighborhood of .  Since , , so .  Observe that  is compact saturated and  is open in .  Use
  interpolation in the locally compact space : there is a compact
  saturated subset  such that .

  In particular, , so  is in the
  open subset .  The latter is
  included in the compact subset , since every element
   of it is such that , hence .  In particular,  is in the interior of .  Finally, since , .
\end{proof}

\begin{prop}
  \label{prop:qretr:scomp}
  Every  quasi-retract  of a stably compact space  is
  stably compact.
\end{prop}
\begin{proof}
   is  by assumption, and locally compact, well-filtered,
  compact, and coherent by Lemma~\ref{lemma:qretr:comp},
  Lemma~\ref{lemma:qretr:wf}, Lemma~\ref{lemma:qretr:coh}, and
  Lemma~\ref{lemma:qretr:lcomp}.  In the presence of local
  compactness, it is equivalent to require sobriety or to require the
  space to be  and well-filtered
  \cite[Theorem~II-1.21]{GHKLMS:contlatt}.
\end{proof}

Call a space  \emph{locally finitary} if and only if for every  and every open neighborhood  of , there is a finitary
compact  such that  and .  This is the same definition as for local compactness,
replacing compact saturated subsets by finitary compacts.  The
interpolation property of locally compact spaces refines to the
following: In a locally finitary space , if  is compact
saturated and included in some open subset , then there is a
finitary compact  such that 
and .  The proof is as for interpolation in
locally compact spaces: for each , pick a finitary compact
 such that  and .   is an open cover
of .  Since  is compact, it has a finite subcover ,
\ldots, .  Then take .

We observe right away the following analog of
Lemma~\ref{lemma:qretr:lcomp}.
\begin{lem}
  \label{lemma:qretr:lfin}
  Any quasi-retract  of a locally finitary space  is locally
  finitary.
\end{lem}
\begin{proof}
  As in the proof of Lemma~\ref{lemma:qretr:lcomp}, let  and
   be an open neighborhood of .  By interpolation between  and  in the locally finitary space , we
  find a finitary compact subset  of  such that
  .  The rest of the proof is as for
  Lemma~\ref{lemma:qretr:lcomp}, only noticing that  is finitary compact.
\end{proof}
The importance of locally finitary spaces lies in the following
result: see Banaschewski \cite{Banaschewski:essn:ext}, or the
equivalence between Items (6) and (11) in Lawson
\cite[Theorem~2]{Lawson:T0:pw:conv}.  See also Isbell
\cite{Isbell:meetcont} for the notion of locally finitary space, up to
change of names.
\begin{prop}
  \label{prop:locfin=qcont}
  The locally finitary sober spaces are exactly the quasi-continuous dcpos in their Scott
  topology.
\end{prop}












We use this, in particular, in the following proposition.
\begin{prop}
  \label{prop:qretr:qrb}
  Every  quasi-retract of an ()-domain is an
  ()-domain.
\end{prop}
\begin{proof}
  Let  be a -domain,  be a  space,  be a
  quasi-retraction, and  be a matching
  quasi-section.  We first note that  is stably compact, by
  Proposition~\ref{prop:qretr:scomp}, using the fact that  is
  itself stably compact (Theorem~\ref{thm:qrb:scomp}).  So  is
  sober. By Proposition~\ref{prop:locfin=qcont},  is locally finitary, so
   is, too, by Lemma~\ref{lemma:qretr:lfin}.  By
  Proposition~\ref{prop:locfin=qcont} again,  is a quasi-continuous
  dcpo, and its topology is the Scott topology.

  Note that  is pointed.  Letting  be the least element of
  ,  is the least element of : for every ,
  pick some  such that  by
  Lemma~\ref{lemma:qretr:surj}, then .

  For each quasi-deflation  on ,  is continuous
  from  to : indeed it is continuous from  to
   and  by
  Corollary~\ref{corl:BoxU}, since  is quasi-continuous
  (Corollary~\ref{corl:qrb:qcont}).  So  makes sense.
  Let  map  to ;  is in 
  because 
  (Lemma~\ref{lemma:dagger:new}, second part), and  is finitary compact for every finite
  set .

  Explicitly, .


  For every open subset  of ,  is
  the set of all  such that for every , for
  every , .  I.e., for every , , that is, .  So .  Since
  the latter is open, and the sets  form a subbase of the
  topology of ,  is continuous from  to
  .  Since  is a quasi-continuous dcpo and its
  topology is Scott, by Corollary~\ref{corl:BoxU} , so  is also Scott-continuous from 
  to .  (Alternatively, apply Corollary~\ref{corl:fin:cont:Fin}.)

  We claim that  for every .
  Since , , so
  there is an  such that .  Now , so taking  in the definition of
  ,  is in .

  Let now  be a generating family of
  quasi-deflations on .  Clearly, if  is below
  , then  is below ,
  so  is directed.

  It remains to show that  for every .  Since , it remains to show : we show
  that every open  containing  contains .  Since  and , , so , i.e., . By Lemma~\ref{lemma:psi*}, .  Since  is
  compact,  for
  some .  So  is in , which is equal to 
  (see above).  It follows that  contains ,
  hence .  So 
  is a -domain.

  The case of -domains is similar, where now
   is a generating {\em sequence\/} of
  quasi-deflations.
\end{proof}

Later, we shall need a refinement of the notion of quasi-retraction,
which is to the latter as projections are to retractions.  Recall that
a {\em projection\/} is a retraction , with section ,
such that additionally .  Similarly, it is
tempting to define a \emph{quasi-projection} as a quasi-retraction
(with quasi-section ) such that  for every .  If  is a retraction, with section , and we see  as a
quasi-retraction in the canonical way, defining  as , then the quasi-projection condition  is
equivalent to the projection condition .



The point  shown in Figure~\ref{fig:qretr} satisfies the condition
:  is in the gray area , where .  However, Lemma~\ref{lemma:qproj:unique} below shows that  is
not a quasi-projection: for this to be the case, the gray area  should fill the whole of .

There is no need to invent a new term, though:
Lemma~\ref{lemma:qproj:unique} shows that quasi-projections are
nothing else than proper surjective maps.
A map  is {\em proper\/} if and only if it is continuous,
 is closed in  for every closed subset  of , and
 is compact in  for every element  of 
\cite[Lemma~VI-6.21~]{GHKLMS:contlatt}.
\begin{lem}
  \label{lemma:qproj:unique}
  Let  be a topological space, and  be a  topological
  space.  For a map , the following two conditions are
  equivalent:
  \begin{enumerate}[\em(1)]
  \item\label{q:qproj}  is a quasi-retraction, with matching
    quasi-section , such that additionally  for every ;
  \item  is proper and surjective.
  \end{enumerate}
  Then the quasi-section  in (\ref{q:qproj}) is unique, and it is
  defined by .
\end{lem}
\begin{proof}
  We first prove the following fact, which will serve in both
  directions of proof:  assume  for
  every , then for every open subset  of , the
  complement of  in  is , where 
  is the complement of  in .  Indeed, the complement of
   is the set of elements  such that  is not included in , i.e., such that there is an  that is not in , i.e., in .  Since , this is the set of elements  such that there is an  such that , namely, .

  Assume  is a quasi-retraction, and  is a matching
  quasi-section such that  for every .  We
  have seen that  is surjective (Lemma~\ref{lemma:qretr:surj}).

  Since , every element  of 
  is such that  is in , so .  Conversely, for every , i.e., if
  , then  since  is
  monotonic.  We have assumed that  was in , so .  It follows that , which proves
  the last claim in the Lemma.

  It also follows that  is compact in .  And,
  using , for every closed subset  of , with complement
  ,  is the complement of , which
  is open since  is continuous, so  is closed.
  Therefore  is proper.

  Conversely, assume that  is proper and surjective.  Define  as .  Since  is surjective,  is
  non-empty.  It is saturated, i.e., upward closed, because  is
  monotonic.  Since  is compact,  is an
  element of .  For every open subset  of , with
  complement ,  is the complement of 
  by , hence is open since  is proper.  So  is
  continuous.

  The equation  follows from  and the fact
  that  is surjective.  It is clear that  is in  for every .
\end{proof}



Let us turn to bifinite domains, or rather to their countably-based
variant.  Countability will be needed in a few crucial places.

A pointed dcpo  is an {\em -domain\/} (a.k.a.\ an
SFP-domain) iff there is a non-decreasing sequence of idempotent
deflations  such that, for every , .  I.e., an -domain is just like a
-domain, except that we take a non-decreasing sequence, not a
general directed family of idempotent deflations.





The key lemma to prove Theorem~\ref{thm:qrb:qretr} below is the
following refinement of Rudin's Lemma \cite[III-3.3]{GHKLMS:contlatt}.
Note that Rudin's Lemma would only secure the existence of a directed
family  whose least upper bound is , and which intersects each
; but  may intersect each  in more than one element
.  We pick exactly one element  in each , and for
this countability seems to be needed.
\begin{lem}
  \label{lemma:qs:nonempty}
  Let  be a dcpo, , and  a
  non-decreasing sequence in  (w.r.t.\ ) such that .
There is a non-decreasing sequence  in 
  such that  for every , and .
\end{lem}
\begin{proof}
  Let  for every .   is a non-decreasing sequence in  such that , and .

  Build a tree as follows.  Informally, there is a root node, all
  (non-root) nodes at distance  from the root node are
  labeled by some element of , and each such node ,
  labeled , say, has as many successors as there are elements
   in  such that .  Formally, one can
  define the nodes as being the sequences ,
  , where , , \ldots, , and .  Such a
  node is labeled  (if ), and its successors are
  all the sequences  with  chosen
  in , and above  if .

  This tree has arbitrarily long branches (paths from the root).
  Indeed, for every , pick an element ---this
  is possible since , hence  is non-empty---,
  then an element  below ---since ---, then an element  below , \ldots, and finally an element  below .  This is a node at distance  from the root.

  It follows that the tree is infinite.  It is finitely-branching,
  meaning that every node has only finitely many successors---because
   is finite.  K\H{o}nig's Lemma then states that this tree must
  have an infinite branch.  Reading the labels on non-root nodes in
  this branch, we obtain an infinite sequence  of elements , .  Clearly,  for each .  In
  particular, , so .  Since  for every ,
  the converse inequality holds.  So .
\end{proof}

\begin{thm}
  \label{thm:qrb:qretr}
  The following are equivalent for a dcpo :
  \begin{description}
  \item[]  is an -domain;
  \item[]  is a quasi-retract of an -domain;
  \item[]  is the image of an -domain under a proper map.
  \end{description}
\end{thm}
\begin{proof}
  .  Because any proper surjective map is a
  quasi-retraction (Lemma~\ref{lemma:qproj:unique}).

  .  Write  as a quasi-retract of an
  -domain .   is trivially an -domain.
  Since , as a dcpo, is , Proposition~\ref{prop:qretr:qrb}
  applies, so  is an -domain.

  .  Let  be an -domain, with
  generating sequence of quasi-deflations .  Let , and define  as the finite set , for each .  Let  be the set of all
  non-decreasing sequences  in  such
  that , and .  Order  componentwise.  As in
  \cite[Theorem~4.9, Theorem~4.1]{Jung:CCC},  is an
  -domain: for each , consider the idempotent
  deflation  defined by .  To show that this is well-defined, we
  must show that , i.e., that .  If , then 
  since  and  is monotonic, else  since  is a
  quasi-deflation.  It is easy to see that  is
  Scott-continuous.

  Let now  map  to .  This
  is evidently Scott-continuous.  For any fixed , apply
  Lemma~\ref{lemma:qs:nonempty} with  to
  obtain a non-decreasing sequence 
  such that  for every  and
  : in particular,  is in , and
  .  So  is surjective.  Let us show that it is
  proper.

  To this end, we first remark that .  Indeed, if
   is in , then , and since ,
  ,
  using the fact that  is monotonic.  Conversely, if  for every , then .

  This remark makes it easier for us to show that  is
  compact for every .  For each , let .  Let  be the set of all elements  of
   such that  for every .  Note
  that  is finite, (recall that each  with 
  is taken from the finite set ), and that
  .  Indeed, for every ,
  its image  by the idempotent deflation 
  is in , and is below .  So  is (finitary)
  compact.  Every -domain is stably compact
  \cite[Theorem~4.2.18]{AJ:domains}, and any intersection of saturated
  compacts in a stably compact space is compact, so  is compact.

  Let us now show that  is closed for every closed subset
   of .  Consider a directed family  of
  elements of , and let .  Since
  ,  intersects .  The family
   is a filtered family of compact
  saturated subsets of , each of which intersects the closed set
  .  Since  is an -domain, it is stably compact, hence
  well-filtered: so 
  intersects .  (Explicitly: if it did not, it would be included in
  the open complement  of , hence some  would
  be included in , contradicting the fact that it intersects .)
  Let  be any element of .  Then  for every ,
  so , hence .
\end{proof}

\section{Products, Bilimits}
\label{sec:bilim}

We first show that finite products of -domains are again
-domains.
\begin{lem}
  \label{lem:qrb:prod}
  If  (resp.\ ) is a
  generating family of quasi-deflations on  (resp.\ ), then
   is one on , where
  .
\end{lem}
\begin{proof}
Clearly, ,  is finitary
  compact, and  is finite.  For all ,
   is easily seen to be Scott-continuous, and .
\end{proof}
So:
\begin{lem}
  \label{lemma:qrb:prod}
  For any two ()-domains , , , with
  the product ordering, is an ()-domain.
\end{lem}

Recall that a retraction , with section , is
a projection iff, additionally,  for every ; then  is usually called an {\em embedding\/}, and is determined
uniquely from .  An \emph{expanding system} of dcpos is a family
, where  is a directed poset (with ordering
), with projection maps  where
, , and  whenever 
\cite[Section~3.3.2]{AJ:domains}. This is nothing else than a
projective system of dcpos, where the connecting maps  must be
projections.  If  is the associated embedding,
then one checks that  and  whenever , so that 
together with  forms an inductive
system of dcpos as well.  In the category of dcpos, the projective
limit of the former coincides with the inductive limit of the latter
(up to natural isomorphism), and is called the \emph{bilimit} of the
expanding system of dcpos.  We write this bilimit as , leaving the dependence on , , , implicit.
This can be built as the dcpo of all those elements  such that  for all  with , with the componentwise
ordering.

General bilimits of countably-based dcpos will fail to be
countably-based in general, so we shall restrict to bilimits of
\emph{expanding sequences} of dcpos
\cite[Definition~3.3.6]{AJ:domains}: these are expanding systems of
dcpos where the index poset  is , with its usual ordering.
To make it clear what we are referring to, we shall call
\emph{-bilimit} of spaces any bilimit of an expanding sequence
(not system) of spaces.

One can appreciate bilimits by realizing that the -domains are (up
to isomorphism) the bilimits of expanding systems of finite, pointed
posets \cite[Theorem~4.2.7]{AJ:domains}.  Similarly, the
-domains are the -bilimits of expanding sequences of
finite, pointed posets.

Bilimits are harder to deal with than products.  But the difficulty
was solved by Jung \cite[Section~4.1]{Jung:CCC} in the case of
-domains and deflations, and we proceed in a very similar way.
We first recapitulate the notion of bilimit.

Consider any set  of functions  from  to  such
that , i.e., , for every .  We say that  is {\em qfs\/} (for {\em quasi-finitely
  separating\/}) iff given any finitely many pairs  with , ,
there is a  that {\em separates\/} the pairs, i.e., such
that  (equivalently,
) for every , .
\begin{prop}
  \label{prop:qfs}
  Let  be a poset.  Then  is a -domain iff  is a
  quasi-continuous dcpo and the set  of quasi-deflations on  is
  qfs.
\end{prop}
\begin{proof}
  If  is a -domain, then let  be such that  for every , , and  be a generating family of
  quasi-deflations.  For each , , , so by Proposition~\ref{prop:heckmann} there is an 
  such that .  And we may pick the same  for every , by directedness.
  So  is the desired .

  Also,  is a quasi-continuous dcpo by
  Corollary~\ref{corl:qrb:qcont}.

  Conversely, assume that  is a quasi-continuous dcpo and  is
  qfs.  We show that  is a generating family of quasi-deflations.  Using
  Corollary~\ref{corl:BoxU}, .  Write
  it , for short.  For each ,  is
  continuous from  to , and  is
  continuous from  to  by
  Lemma~\ref{lemma:dagger:new}, so  is
  continuous from  to .  Since , 
  is also in .  Also,  is finite, since all its elements are unions of
  elements of the finite set .  So  is a quasi-deflation.

  Let us show that  is directed.  Pick  and 
  from .  Let , and
  .  Similarly, let  and .  For
  each ,  by Lemma~\ref{lemma:qdefl:ll}.
  Since  is quasi-continuous, use interpolation, and pick a
  finitary compact  such that .  Similarly, let  be a finitary compact such
  that  and 
  for each .

  Consider the finite collection of all pairs ,
  , , and ,
  where , , , .  Since
   is qfs, there is a  such that  for all the above pairs .  In
  particular, looking at the pair , we get:   for every .  And looking at the
  pair ,  for all , .  So .  We have proved:   for every .  Then, for every ,
   (by )
   (by
  )  (since )  (by one of the monad laws).  So
   is below .
  Similarly,  is below
  , so  is directed.

  Finally, we claim that .  In the  direction, this is
  because  is a quasi-retraction.
  Conversely, let  be such that .
  By interpolation, find  such that .  Since  is qfs, applied to the pairs  and  for each , there is an element
   such that  and  for every .  So .  So , as  is quasi-continuous.
\end{proof}

\begin{thm}
  \label{thm:qrb:bilimit}
  Any (-)bilimit of ()-domains is an
  ()-domain.
\end{thm}
\begin{proof}
  Let  be an expanding system of -domains,
  with projections  and embeddings , .  Let .  There is a
  projection , given by  (where
  ), and an embedding 
  for every .

  We observe that:  if  in , then
   for every .
  Indeed, consider any directed family  such that
  .  Then , so for some , there is a  with .  Then .  We conclude since .

  We now claim that the family  of all finitary
  compacts of the form , where  and , , is directed.
  Given  and  in , find some  such that , by
  directedness.  Then , and by  ,
  and similarly , with .
  Replacing  by ,  by the finitary compact ,  by , and  by  if necessary, we can therefore simply assume that .  Since
   is quasi-continuous, there is an  such
  that , and then
   is an element of  above both
   and .

  Moreover, we claim that  equals .  That it
  contains  is obvious: whenever ,
  pick  with , so that , hence .
  Conversely, every  must be such that , since  is
  quasi-continuous.  As this holds for every , .  So .

  In particular,  is a quasi-continuous dcpo.


  We check that the set of quasi-deflations on  is qfs.  Consider a
  finite collection of pairs  with , .
  Recall that  can be rephrased
  equivalently as:  is in the open subset .
Since , by
  Proposition~\ref{prop:heckmann}, for each , pick  included in , in particular above .  I.e., pick  and
   such that ,
  and such that .
(We can pick the same  for every , by directedness, as above.)
  Since  is a -domain, and ,
  using Proposition~\ref{prop:heckmann}, there is a quasi-deflation
   on  such that .  So ,
  for every , .  Consider 
  defined as .  ,
  restricted to , takes its values in , using
  Lemma~\ref{lemma:dagger:new} and the fact that .  Moreover,  is continuous from
   to , hence to  since  is
  quasi-continuous, by Corollary~\ref{corl:BoxU}.  For every , , since 
  is a quasi-deflation.  Then  is below ,
  and is in , so .  So 
  is a quasi-deflation.

  Moreover, by construction, for each , , , so , so , since .  So the set
  of quasi-deflations on  is qfs.

  By Proposition~\ref{prop:qfs},  is then a -domain.

  To deal with -bilimits of -domains, observe that
  any bilimit of a countable expanding system (in particular, an
  expanding sequence) of countably-based quasi-continuous dcpos is
  countably-based.  Indeed, a countably based quasi-continuous dcpo
   has a countable base of sets of the form ,
  , .  The  construction above, suitably modified, shows that the sets
  , where , , form a, necessarily countable, base of
  the topology on .  By Proposition~\ref{prop:omega:qrb:2},  is
  an -domain.
\end{proof}

\section{The Probabilistic Powerdomain}
\label{sec:qretr:V}

Let  be a fixed topological space, and let  be the
lattice of open subsets of .
A {\em continuous valuation\/}  on  \cite{JP:proba} is a map
from  to  such that , which is
{\em monotonic\/} ( whenever ),
{\em modular\/} (
for all opens ), and {\em continuous\/} ( for every directed
family  of opens).  A {\em (sub)probability\/}
valuation  is additionally such that  is {\em
  (sub)normalized\/}, i.e., that  ().  Let
 () be the dcpo of all (sub)probability
valuations on , ordered pointwise, i.e.,  iff  for every open .   ()
defines a endofunctor on the category of dcpos, and its action is
defined on morphisms  by .

We write  for the \emph{Dirac valuation} at , a.k.a., the
point mass at .  This is the continuous valuation such that
 if ,  otherwise.

The probabilistic powerdomain construction  is an elusive one,
and natural intuitions are often wrong.  For example, one might
imagine that if  has all binary least upper bounds, then so has
.  This was dispelled by Jones and Plotkin
\cite{JP:proba}.  Consider , with  and
 incomparable,  below every element and  above every
element (see Figure~\ref{fig:v3}, right).  Then the upper bounds of
 and  in  are the probability
valuations of the form  where , , and .  The minimal upper bounds are those of the form , .  So there is no
unique least upper bound; in fact, there are uncountably many of them,
even on this small example.

\begin{figure}
  \centering
  \ifpdf
  \begin{picture}(0,0)\includegraphics{V-RB.pdftex}\end{picture}\setlength{\unitlength}{395sp}\begingroup\makeatletter\ifx\SetFigFont\undefined \gdef\SetFigFont#1#2#3#4#5{\reset@font\fontsize{#1}{#2pt}\fontfamily{#3}\fontseries{#4}\fontshape{#5}\selectfont}\fi\endgroup \begin{picture}(66082,12572)(-13518,-24507)
\put(-9824,-13111){\makebox(0,0)[lb]{\smash{{\SetFigFont{11}{13.2}{\rmdefault}{\mddefault}{\updefault}{\color[rgb]{0,0,0}}}}}}
\put(601,-13111){\makebox(0,0)[lb]{\smash{{\SetFigFont{11}{13.2}{\rmdefault}{\mddefault}{\updefault}{\color[rgb]{0,0,0}}}}}}
\put(12676,-13111){\makebox(0,0)[lb]{\smash{{\SetFigFont{11}{13.2}{\rmdefault}{\mddefault}{\updefault}{\color[rgb]{0,0,0}}}}}}
\put(27151,-13111){\makebox(0,0)[lb]{\smash{{\SetFigFont{11}{13.2}{\rmdefault}{\mddefault}{\updefault}{\color[rgb]{0,0,0}}}}}}
\put(42751,-13111){\makebox(0,0)[lb]{\smash{{\SetFigFont{11}{13.2}{\rmdefault}{\mddefault}{\updefault}{\color[rgb]{0,0,0}}}}}}
\end{picture}   \else
  \begin{picture}(0,0)\includegraphics{V-RB.pstex}\end{picture}\setlength{\unitlength}{395sp}\begingroup\makeatletter\ifx\SetFigFont\undefined \gdef\SetFigFont#1#2#3#4#5{\reset@font\fontsize{#1}{#2pt}\fontfamily{#3}\fontseries{#4}\fontshape{#5}\selectfont}\fi\endgroup \begin{picture}(66082,12572)(-13518,-24507)
\put(-9824,-13111){\makebox(0,0)[lb]{\smash{{\SetFigFont{11}{13.2}{\rmdefault}{\mddefault}{\updefault}{\color[rgb]{0,0,0}}}}}}
\put(601,-13111){\makebox(0,0)[lb]{\smash{{\SetFigFont{11}{13.2}{\rmdefault}{\mddefault}{\updefault}{\color[rgb]{0,0,0}}}}}}
\put(12676,-13111){\makebox(0,0)[lb]{\smash{{\SetFigFont{11}{13.2}{\rmdefault}{\mddefault}{\updefault}{\color[rgb]{0,0,0}}}}}}
\put(27151,-13111){\makebox(0,0)[lb]{\smash{{\SetFigFont{11}{13.2}{\rmdefault}{\mddefault}{\updefault}{\color[rgb]{0,0,0}}}}}}
\put(42751,-13111){\makebox(0,0)[lb]{\smash{{\SetFigFont{11}{13.2}{\rmdefault}{\mddefault}{\updefault}{\color[rgb]{0,0,0}}}}}}
\end{picture}   \fi
  \caption{Discretizations of , }
  \label{fig:V-RB}
\end{figure}

It is unknown whether , with  is
an -domain, although it is an -domain, as a consequence of
\cite[Theorem~17]{JT:troublesome}.  Again, some of the most natural
ideas one can have about  are flawed.  It seems obvious
indeed that  should be the bilimit of the sequence of
finite posets , defined as those probability
valuations  where
, ,  are integer multiples of .  See Figure~\ref{fig:V-RB} for Hasse diagrams of a few of these
posets, for  small.

That  is such a bilimit is necessarily wrong, because any
bilimit of finite posets is an -domain, hence is algebraic,
but  is not algebraic, since no element except
 is finite.

However, one may imagine to define (non-idempotent) deflations 
on  directly, which would send  to
some discretized probability valuation in .
However, all known attempts fail.  A careful study of
\cite{JT:troublesome} will make this precise.  Let us only note that
if we decide to define  through its values on open sets,
typically letting  be the largest integer multiple of
 that is zero-or-strictly-below , we obtain a set
function that is not modular.  If we decide to define  as 
where for each   is the largest integer multiple
of  that is zero-or-strictly-below , then 
is not monotonic.  If we decide to define  as the largest
probability valuation way-below  in , we
run into the problem that there is no {\em unique\/} such largest
probability valuation.  For example,  admits four largest probability
valuations in  way-below it: , , , and , see
Figure~\ref{fig:V-approx}.

\begin{figure}
  \centering
  \ifpdf
  \begin{picture}(0,0)\includegraphics{V-approx.pdftex}\end{picture}\setlength{\unitlength}{592sp}\begingroup\makeatletter\ifx\SetFigFont\undefined \gdef\SetFigFont#1#2#3#4#5{\reset@font\fontsize{#1}{#2pt}\fontfamily{#3}\fontseries{#4}\fontshape{#5}\selectfont}\fi\endgroup \begin{picture}(25550,8201)(-17249,-23068)
\put(-12524,-15811){\makebox(0,0)[lb]{\smash{{\SetFigFont{11}{13.2}{\rmdefault}{\mddefault}{\updefault}{\color[rgb]{0,0,0}}}}}}
\put(-17249,-18586){\makebox(0,0)[lb]{\smash{{\SetFigFont{11}{13.2}{\rmdefault}{\mddefault}{\updefault}{\color[rgb]{0,0,0}Best discretizations:}}}}}
\end{picture}   \else
  \begin{picture}(0,0)\includegraphics{V-approx.pstex}\end{picture}\setlength{\unitlength}{592sp}\begingroup\makeatletter\ifx\SetFigFont\undefined \gdef\SetFigFont#1#2#3#4#5{\reset@font\fontsize{#1}{#2pt}\fontfamily{#3}\fontseries{#4}\fontshape{#5}\selectfont}\fi\endgroup \begin{picture}(25550,8201)(-17249,-23068)
\put(-12524,-15811){\makebox(0,0)[lb]{\smash{{\SetFigFont{11}{13.2}{\rmdefault}{\mddefault}{\updefault}{\color[rgb]{0,0,0}}}}}}
\put(-17249,-18586){\makebox(0,0)[lb]{\smash{{\SetFigFont{11}{13.2}{\rmdefault}{\mddefault}{\updefault}{\color[rgb]{0,0,0}Best discretizations:}}}}}
\end{picture}   \fi
  \caption{Largest discretizations below  fail to be unique}
  \label{fig:V-approx}
\end{figure}



Observe that the number of largest discretizations of  in
 is always finite, provided  is finite.
This was our original intuition that replacing deflations by
quasi-deflations, hence moving from -domains to -domains,
might provide a nice enough category of domains that would be stable
under the probabilistic powerdomain functor .  However,
defining quasi-deflations directly, as hinted above, does not work
either: monotonicity fails again.  This is where the characterization
of -domains as quasi-retracts of bifinite domains (up to details
we have already mentioned) will be decisive.


If  is a retract of , then  is easily seen to be a
retract of , using the  endofunctor.  We wish to
show a similar result for quasi-retracts.  We have not managed to do
so.  Instead we shall rely on the stronger assumptions that  is
stably compact, that  is a quasi-projection of , not just a
quasi-retract (i.e., the image of  under a proper map).

Moreover, we shall need to replace the Scott topology on 
by the {\em weak topology\/}, which is the smallest one containing the
subbasic opens , defined as , for each open subset  of  and .  When
 is a continuous pointed dcpo, the \emph{Kirch-Tix Theorem} states
that it coincides with the Scott topology (see \cite{AMJK:scs:prob},
who attribute it to Tix \cite[Satz~4.10]{Tix:bewertung}, who in turn
attributes it to Kirch \cite[Satz~8.6]{Kirch:bewertung}).

However, the weak topology is better behaved in the general case.  For
example, writing  for  with the
Scott topology, and  for the space of all
continuous maps from  to  with the Isbell topology, there
is a natural homeomorphism between the space of linear continuous maps
from  to  and the space of of
(extended, i.e., possibly taking the value ) continuous
valuations on , with the weak topology
\cite[Theorem~8.1]{Heckmann:space:val}.  This is an analog of the
Riesz Representation Theorem in measure theory, of which one can find
variants in \cite{Tix:bewertung,Gou-csl07} among others, and which we
shall use silently in the proof of Theorem~\ref{thm:qretr:V}.  Let
 be  with its weak topology.

 defines an endofunctor on the category of topological
spaces, by , where , , and .  That
 is continuous for every continuous , in
particular, is obvious, since for every open subset  of ,
.

As we have said above, we shall also require  to be stably compact.
If this is so, then the \emph{cocompact topology} on  consists of
all complements of compact saturated subsets.  Write , the
\emph{de Groot dual} of , for  with its cocompact topology.
Then  is again stably compact, and  (see
\cite[Corollary~12]{AMJK:scs:prob} or
\cite[Corollary~VI-6.19]{GHKLMS:contlatt}).  The patch topology on
, mentioned earlier, is nothing else than the join of the two
topologies of  and .

Write  for  equipped with its patch topology.  If  is
stably compact, then  is not only compact Hausdorff, but the
graph of the specialization preorder  of  is closed in
: one says that  is a \emph{compact
  pospace}.  The study of compact pospaces originates in Nachbin's
classic work \cite{Nachbin:toporder}.  Conversely, given a compact
pospace , i.e., a compact space with a closed ordering
 on it, the \emph{upwards topology} on  consists of those
open subsets of  that are upward closed in .  The space
, obtained as  with the upwards topology, is then
stably compact.  Moreover, the two constructions are inverse of each
other.  (See \cite[Section~VI-6]{GHKLMS:contlatt}.)

If  and  are stably compact, then  is proper if and
only if  is continuous, and monotonic with
respect to the specialization orderings of  and 
\cite[Proposition~VI.6.23]{GHKLMS:contlatt}, i.e., if and only if 
is a morphism of compact pospaces.

Now, the structure of the cocompact topology on ,
when  is stably compact, is as follows.  For every continuous
valuation  on , following Tix \cite{Tix:bewertung}, define
 as ,
for every compact saturated subset  of .
Define  as the set of probability valuations
 such that .  The sets  are compact saturated in , and
Proposition~6.8 of \cite{JGL-mscs09} even states that they form a
subbase of compact saturated subsets.  This means that the complements
of the sets of the form ,  compact
saturated in , , form a base of the topology of
.  A similar claim was already stated in
\cite[last lines]{Jung:scs:prob}.

\begin{lem}
  \label{lemma:Vf:dagger}
  Let ,  be stably compact spaces, and  be a proper
  surjective map from  to .  Then , for every compact
  saturated subset  of .
\end{lem}
\begin{proof}
  We must show that , where  ranges over opens in 
  and  over opens in .

  For every open  containing ,  is an open
  subset of  containing the compact saturated subset ,
  so .

  Conversely, for every open  containing , we shall
  build an open subset  containing  such that .  This will establish , hence the
  equality.

  Recall from Lemma~\ref{lemma:qproj:unique} that  forms a
  quasi-retraction, with a unique matching quasi-section  such that  for every , and such
  that  for every .  We let .  Since ,  is
  in .  For every ,  is then also in , so  is in
  .  So .  On the other hand,
  for every element  of ,  is in , so  is in .  Then .  So , and we are done.
\end{proof}

Similarly to the formula , which allowed us to conclude that  was
continuous for every continuous , we obtain:
\begin{lem}
  \label{lemma:Vf:dagger:inv}
  Let ,  be stably compact spaces, and  be a proper
  surjective map from  to .  Then  for every
  compact saturated subset  of , and .
\end{lem}
\begin{proof}
  Using Lemma~\ref{lemma:Vf:dagger}, .
\end{proof}

\begin{prop}
  \label{prop:Vf:proper}
  Let  be a stably compact space,  be a  space, and  be
  a proper surjective map from  to .  Then  is
  a proper map from  to .
\end{prop}
\begin{proof}
  First, since  is proper and surjective,  is a quasi-retraction
  (Lemma~\ref{lemma:qproj:unique}), so  is stably compact by
  Proposition~\ref{prop:qretr:scomp}.   is
  continuous from  to .
  Lemma~\ref{lemma:Vf:dagger:inv} implies that  is
  also continuous from  to : it suffices to check that the inverse images of
  subbasic patch-open subsets, of the form  or whose
  complements are of the form , are
  patch-open.  Also,  is monotonic with respect to
  the specialization orderings of  and , being continuous.  So  is proper.
\end{proof}

Let us establish surjectivity.  One possible proof goes as follows.
Let  denote the space of all Radon probability
measures on the space .  If  is stably compact, then  is compact in the vague topology, and forms a compact
pospace with the stochastic ordering, where  is below  if
and only if  for every open subset  of 
\cite[Theorem~31]{AMJK:scs:prob}.  By
\cite[Theorem~36]{AMJK:scs:prob}, there is an isomorphism between
 and .

Now assume a second stably compact space .  For two measurable
spaces  and , and  measurable, let 
map the Radon measure  to its image measure, whose value on the
Borel subset  of  is .  A standard result
\cite[2.4, Lemma~1]{Bourbaki:int:IX} states that for any two compact
Hausdorff spaces  and , if  is continuous surjective from 
to , then  is surjective.  The desired result
follows, up to a few technical details, by taking ,
, remembering that since  is proper from  to , it
is continuous from  to .

Instead of working out the---technically subtle but boring---technical
details, let us give a direct proof, similar to the above cited
Lemma~1, 2.4 \cite{Bourbaki:int:IX}.  Instead of using the Hahn-Banach
Theorem, we rest on the following Keimel Sandwich Theorem
\cite[Theorem~8.2]{Keimel:topcones}: let  be a topological cone,  be a continuous superlinear map,  be
a sublinear map, and assume ; then there is a continuous
linear map  such that .
Here, a \emph{cone} is an additive commutative monoid, with a scalar
multiplication by elements of  satisfying ,
, , ,  for all , .  A cone is \emph{topological} if and only if
addition and multiplication are continuous.  The continuous maps  are sometimes called lower semi-continuous in the
literature.  Such a map is \emph{superlinear} (resp.,
\emph{sublinear}, \emph{linear}) if and only if  for
all ,  and  for all
 (resp., , ).  It is easy to see that the space
 of all continuous maps from  to , equipped
with the obvious addition and scalar multiplication and with the Scott
topology of the pointwise ordering, is a topological cone.

\begin{prop}
  \label{prop:qretr:surj}
  Let ,  be stably compact spaces, and  be a proper
  surjective map from  to .  Then  is
  surjective.
\end{prop}
\begin{proof}
  Fix some continuous probability valuation  on .  Let  be
  .  Since  is proper, it has an associated
  quasi-section , with  for every ,
  by Lemma~\ref{lemma:qproj:unique}.  Define  by , where , and integration of continuous maps from  to
   is defined by a Choquet formula
  \cite{Tix:bewertung,JGL-icalp07}, or equivalently by Heckmann's
  general construction \cite{Heckmann:space:val}.

  Note that  is well-defined as , since
   is compact saturated hence of the form 
  for some ---then .  Moreover,  is
  continuous from  to , because .  So  is
  continuous, whence the integral defining  makes sense.  We now
  claim that the map  is (Scott-)continuous.  First,  is clearly monotonic.  Now let  be a
  directed family in  with a least upper bound .
  By monotonicity, for every , , so  exists and is below .
  Conversely, we must show that for every  such that , .  The elements  such that  are those such that for every , there is an  such that ,
  i.e., they are the elements of .  Since  commutes with directed unions, if  then  for some ,
  i.e., , and we are done.  Since  is
  continuous, and since the Choquet integral is Scott-continuous in
  the integrated function (see \cite[Satz~4.4]{Tix:bewertung}, or
  \cite[Theorem~7.1~(3)]{Heckmann:space:val}), we obtain that  is
  (Scott-)continuous.

  For every , .  Moreover, since
  , and integration is linear, 
  is superlinear.

  Define  as .  Clearly,  is
  sublinear.  Notably,
  
  Whenever , we claim that .  Indeed, since  for some 
  (Lemma~\ref{lemma:qretr:surj}), and since , .

  It follows that .  By taking infs over , .

  So Keimel's Sandwich Theorem applies.  There is a continuous linear
  map  such that .
  Define  by , where  is the characteristic function of .
  Then  is a continuous valuation on ; in particular,  because
  .

  Now, given an open subset  of , take .
  Then  iff , so , and therefore , using the fact that
   is surjective.  It follows that .  On the other hand, take  in the
  definition of , and check that .  It follows
  that .  Since
  , .  This holds for every open subset  of .  In
  particular, taking , we obtain that  is a probability
  valuation: .  And
  finally, that  holds for every open
   of  means that .
\end{proof}

Putting together Proposition~\ref{prop:Vf:proper} and
Proposition~\ref{prop:qretr:surj}, we obtain:
\begin{thm}[Key Claim]
  \label{thm:qretr:V}
  Let  be a stably compact space, and  be a  space.  If 
  is a proper surjective map from  to , then 
  is a proper surjective map from  to .
\end{thm}
In particular, if  is a quasi-projection of , then  is a quasi-projection of .

\begin{figure}
  \centering
  \ifpdf
  \begin{picture}(0,0)\includegraphics{ex1-path.pdftex}\end{picture}\setlength{\unitlength}{2368sp}\begingroup\makeatletter\ifx\SetFigFont\undefined \gdef\SetFigFont#1#2#3#4#5{\reset@font\fontsize{#1}{#2pt}\fontfamily{#3}\fontseries{#4}\fontshape{#5}\selectfont}\fi\endgroup \begin{picture}(4569,2733)(5551,-4408)
\put(9751,-1891){\makebox(0,0)[lb]{\smash{{\SetFigFont{11}{13.2}{\rmdefault}{\mddefault}{\updefault}{\color[rgb]{0,0,0}}}}}}
\put(9751,-2416){\makebox(0,0)[lb]{\smash{{\SetFigFont{11}{13.2}{\rmdefault}{\mddefault}{\updefault}{\color[rgb]{0,0,0}}}}}}
\put(9751,-3136){\makebox(0,0)[lb]{\smash{{\SetFigFont{11}{13.2}{\rmdefault}{\mddefault}{\updefault}{\color[rgb]{0,0,0}}}}}}
\put(6676,-2461){\makebox(0,0)[lb]{\smash{{\SetFigFont{11}{13.2}{\rmdefault}{\mddefault}{\updefault}{\color[rgb]{0,0,0}}}}}}
\put(6526,-3061){\makebox(0,0)[lb]{\smash{{\SetFigFont{11}{13.2}{\rmdefault}{\mddefault}{\updefault}{\color[rgb]{0,0,0}}}}}}
\put(6826,-3661){\makebox(0,0)[lb]{\smash{{\SetFigFont{11}{13.2}{\rmdefault}{\mddefault}{\updefault}{\color[rgb]{0,0,0}}}}}}
\put(7501,-3136){\makebox(0,0)[lb]{\smash{{\SetFigFont{11}{13.2}{\rmdefault}{\mddefault}{\updefault}{\color[rgb]{0,0,0}}}}}}
\put(8326,-3736){\makebox(0,0)[lb]{\smash{{\SetFigFont{11}{13.2}{\rmdefault}{\mddefault}{\updefault}{\color[rgb]{0,0,0}}}}}}
\put(9526,-3736){\makebox(0,0)[lb]{\smash{{\SetFigFont{11}{13.2}{\rmdefault}{\mddefault}{\updefault}{\color[rgb]{0,0,0}}}}}}
\put(5551,-2461){\makebox(0,0)[lb]{\smash{{\SetFigFont{11}{13.2}{\rmdefault}{\mddefault}{\updefault}{\color[rgb]{0,0,0}}}}}}
\put(6451,-4336){\makebox(0,0)[lb]{\smash{{\SetFigFont{11}{13.2}{\rmdefault}{\mddefault}{\updefault}{\color[rgb]{0,0,0}}}}}}
\end{picture}   \else
  \begin{picture}(0,0)\includegraphics{ex1-path.pstex}\end{picture}\setlength{\unitlength}{2368sp}\begingroup\makeatletter\ifx\SetFigFont\undefined \gdef\SetFigFont#1#2#3#4#5{\reset@font\fontsize{#1}{#2pt}\fontfamily{#3}\fontseries{#4}\fontshape{#5}\selectfont}\fi\endgroup \begin{picture}(4569,2733)(5551,-4408)
\put(9751,-1891){\makebox(0,0)[lb]{\smash{{\SetFigFont{11}{13.2}{\rmdefault}{\mddefault}{\updefault}{\color[rgb]{0,0,0}}}}}}
\put(9751,-2416){\makebox(0,0)[lb]{\smash{{\SetFigFont{11}{13.2}{\rmdefault}{\mddefault}{\updefault}{\color[rgb]{0,0,0}}}}}}
\put(9751,-3136){\makebox(0,0)[lb]{\smash{{\SetFigFont{11}{13.2}{\rmdefault}{\mddefault}{\updefault}{\color[rgb]{0,0,0}}}}}}
\put(6676,-2461){\makebox(0,0)[lb]{\smash{{\SetFigFont{11}{13.2}{\rmdefault}{\mddefault}{\updefault}{\color[rgb]{0,0,0}}}}}}
\put(6526,-3061){\makebox(0,0)[lb]{\smash{{\SetFigFont{11}{13.2}{\rmdefault}{\mddefault}{\updefault}{\color[rgb]{0,0,0}}}}}}
\put(6826,-3661){\makebox(0,0)[lb]{\smash{{\SetFigFont{11}{13.2}{\rmdefault}{\mddefault}{\updefault}{\color[rgb]{0,0,0}}}}}}
\put(7501,-3136){\makebox(0,0)[lb]{\smash{{\SetFigFont{11}{13.2}{\rmdefault}{\mddefault}{\updefault}{\color[rgb]{0,0,0}}}}}}
\put(8326,-3736){\makebox(0,0)[lb]{\smash{{\SetFigFont{11}{13.2}{\rmdefault}{\mddefault}{\updefault}{\color[rgb]{0,0,0}}}}}}
\put(9526,-3736){\makebox(0,0)[lb]{\smash{{\SetFigFont{11}{13.2}{\rmdefault}{\mddefault}{\updefault}{\color[rgb]{0,0,0}}}}}}
\put(5551,-2461){\makebox(0,0)[lb]{\smash{{\SetFigFont{11}{13.2}{\rmdefault}{\mddefault}{\updefault}{\color[rgb]{0,0,0}}}}}}
\put(6451,-4336){\makebox(0,0)[lb]{\smash{{\SetFigFont{11}{13.2}{\rmdefault}{\mddefault}{\updefault}{\color[rgb]{0,0,0}}}}}}
\end{picture}   \fi
  \caption{The path space of Figure~\ref{fig:ex1}~}
  \label{fig:ex1-path}
\end{figure}

We shall apply this theorem twice, and first, to finite pointed
posets.  Let  be the strict part of .
\begin{defi}[Path Space]
  \label{defn:path}
  Let  be any finite pointed poset.  Write  iff  is {\em immediately below\/} , i.e.,  and there is
  no  such that .  A {\em path\/}  in  is
  any set  with .  The {\em path space\/}  is the
  set of paths in , ordered by .
\end{defi}
Alternatively, the ordering on paths 
is the prefix ordering on sequences .

Note that  is always a finite tree, i.e., a finite pointed
poset such that the downward closure of a point is always totally
ordered.  Up to questions of finiteness, this is exactly how we built
a tree from an ordering in the proof of Lemma~\ref{lemma:qs:nonempty},
by the way.

We observe that every finite pointed poset  is a quasi-projection
of its path space .
\begin{lem}
  \label{lemma:path:qretr}
  For every finite pointed poset , the map 
  defined by  is proper and surjective.
\end{lem}
\begin{proof}
  See Figure~\ref{fig:ex1-path}, which displays the path space of the
  space  of Figure~\ref{fig:ex1}~.  Each gray region is
  labeled with an element from , which is the image by  of every
  point in the region; e.g., the top right, -element region is
  mapped to  in .

  Formally, let , and define  by , i.e., .  The map
   is surjective, and monotonic.  Since  and  are finite,  is
  then trivially proper.
\end{proof}
 is certainly not a retract of  in general: it is, if and
only if  is a tree.  Indeed, if  is a tree, then  is
isomorphic to , and conversely, every retract of a tree is a
tree.

Finite trees are very special.  Jung and Tix proved that  is an -domain \cite[Theorem~13]{JT:troublesome} for
every finite tree .  They noted (comment after op.cit.) that
 is even a bc-domain in this case, i.e., a pointed
continuous dcpo in which every pair of elements with an upper bound
has a least upper bound.  It is well-known that every bc-domain is an
-domain: given any finite subset  of a basis  of a
bc-domain , the map  is a
deflation, the family of these deflations is directed, and their least
upper bound is the identity map.
\begin{lem}
  \label{lemma:V1:bc}
  For every finite tree ,  is a countably-based
  bc-domain.
\end{lem}
\begin{proof}
  Since  is a finite tree, it is trivially a continuous pointed
  dcpo, so  is again continuous
  \cite[Section~3]{Edalat:int}.  A basis is given by the valuations of
  the form  with ,
   and each  rational.  Since 
  has a countable basis , its topology has a countable base
  consisting of the subsets , .  So 
  is countably-based.

  To show that  is a bc-domain, we observe that every
  probability valuation  on  is entirely characterized by the
  values , .  Indeed, for every open subset 
  of , let  be the (finite) set of minimal elements of ;
  the sets , , are pairwise disjoint, so .  The map 
  defined by  satisfies  and  for every .  Let us
  call such maps {\em admissible\/}.  Given any admissible map ,
  there is a unique probability valuation  such that  for every , namely 
  with .  So  is order-isomorphic to the poset of admissible maps, with the
  pointwise ordering.  Therefore we only have to show that any two
  admissible maps ,  below a third one  have a least
  upper bound .  As a least upper bound,  must be above , , and , so define
   by descending induction on  by .  (By descending
  induction, we mean induction on the largest length  of a sequence
   in  such that .)  This is
  admissible if and only if , and in this case will be the
  least upper bound of , .  By definition .
  It is easy to see that  for every , by
  descending induction on : so , hence
   is admissible.
\end{proof}
We retrieve the Jung-Tix result that  is a
bc-domain for every tree : let  be  with an extra bottom
element added below all elements of , and apply
Lemma~\ref{lemma:V1:bc} to .

\begin{prop}
  \label{prop:path:V}
  For every finite pointed poset ,  is a continuous
  -domain.
\end{prop}
\begin{proof}
   is trivially a continuous pointed dcpo.  Then we know that
   is again continuous \cite[Section~3]{Edalat:int}, and
  that  by the Kirch-Tix Theorem.
  Similarly for .   is clearly stably
  compact, since finite.  By Theorem~\ref{thm:qretr:V}, using
  Lemma~\ref{lemma:path:qretr},  is the image of  under some proper surjective map.  But  is a tree,
  so  is a countably-based bc-domain by
  Lemma~\ref{lemma:V1:bc}, hence a countably-based -domain, hence
  an -domain, by Proposition~\ref{prop:FS:QRB} and
  Proposition~\ref{prop:omega:qrb:2}.  By
  Proposition~\ref{prop:qretr:qrb},  must also be an
  -domain.
\end{proof}
We can finally prove the main theorem of this paper.
\begin{thm}
  \label{thm:qrb:V}
  The probabilistic powerdomain of any -domain is an
  -domain.
\end{thm}
\begin{proof}
  Let  be an -domain.  By Theorem~\ref{thm:qrb:qretr},
   is the image of some -domain  under some proper surjective map.  Since  is a locally
  continuous functor on the category of dcpos, (as mentioned in proof
  of \cite[Lemma~11]{JT:troublesome}),  is also a bilimit
  of the spaces , .  Each  is a
  continuous -domain by Proposition~\ref{prop:path:V},
  hence so is , by Theorem~\ref{thm:qrb:bilimit} and since
  bilimits of continuous dcpos are continuous
  \cite[Theorem~3.3.11]{AJ:domains}.

  Since  is bifinite, it is stably compact, (use, e.g.,
  Theorem~\ref{thm:qrb:scomp}), and 
  because  is continuous and pointed, using the Kirch-Tix Theorem.
  So  is the image of  under a proper
  surjective map, by Theorem~\ref{thm:qretr:V}.  It is clear that
   is , so by Proposition~\ref{prop:qretr:qrb}
   is an -domain in its specialization
  preorder , and its topology must be the Scott topology of
  .

  But it is easy to see that  is the usual ordering on
  , i.e.,  iff 
  for every open  of : note that if , then
   for every .  So , and we conclude.
\end{proof}

Using the fact that  is continuous whenever  is
continuous and pointed \cite[Section~3]{Edalat:int}, it also follows:
\begin{cor}
  \label{corl:qrb:V}
  The probabilistic powerdomain of any continuous -domain
  (in particular, every -domain) is again a continuous
  -domain.  \end{cor}


\section{Conclusion, Failures and Perspectives}
\label{sec:conc}

We have shown that the category  of -domains
and continuous maps is a category of quasi-continuous, stably compact
dcpos that is closed, not only under finite products, bilimits of
expanding sequences, retracts (and even quasi-retracts), but also
under the probabilistic powerdomain functor .  It is thus
reasonably well-behaved.

\begin{figure}
  \centering
  \ifpdf
  \begin{picture}(0,0)\includegraphics{T.pdftex}\end{picture}\setlength{\unitlength}{2763sp}\begingroup\makeatletter\ifx\SetFigFont\undefined \gdef\SetFigFont#1#2#3#4#5{\reset@font\fontsize{#1}{#2pt}\fontfamily{#3}\fontseries{#4}\fontshape{#5}\selectfont}\fi\endgroup \begin{picture}(2180,4338)(2101,-4483)
\put(2101,-3811){\makebox(0,0)[lb]{\smash{{\SetFigFont{12}{14.4}{\rmdefault}{\mddefault}{\updefault}{\color[rgb]{0,0,0}}}}}}
\put(3451,-3811){\makebox(0,0)[lb]{\smash{{\SetFigFont{12}{14.4}{\rmdefault}{\mddefault}{\updefault}{\color[rgb]{0,0,0}}}}}}
\put(2101,-3061){\makebox(0,0)[lb]{\smash{{\SetFigFont{12}{14.4}{\rmdefault}{\mddefault}{\updefault}{\color[rgb]{0,0,0}}}}}}
\put(3451,-3061){\makebox(0,0)[lb]{\smash{{\SetFigFont{12}{14.4}{\rmdefault}{\mddefault}{\updefault}{\color[rgb]{0,0,0}}}}}}
\put(2101,-2311){\makebox(0,0)[lb]{\smash{{\SetFigFont{12}{14.4}{\rmdefault}{\mddefault}{\updefault}{\color[rgb]{0,0,0}}}}}}
\put(3451,-2311){\makebox(0,0)[lb]{\smash{{\SetFigFont{12}{14.4}{\rmdefault}{\mddefault}{\updefault}{\color[rgb]{0,0,0}}}}}}
\put(2101,-1561){\makebox(0,0)[lb]{\smash{{\SetFigFont{12}{14.4}{\rmdefault}{\mddefault}{\updefault}{\color[rgb]{0,0,0}}}}}}
\put(3451,-1561){\makebox(0,0)[lb]{\smash{{\SetFigFont{12}{14.4}{\rmdefault}{\mddefault}{\updefault}{\color[rgb]{0,0,0}}}}}}
\put(3151,-361){\makebox(0,0)[lb]{\smash{{\SetFigFont{12}{14.4}{\rmdefault}{\mddefault}{\updefault}{\color[rgb]{0,0,0}}}}}}
\put(3226,-4411){\makebox(0,0)[lb]{\smash{{\SetFigFont{12}{14.4}{\rmdefault}{\mddefault}{\updefault}{\color[rgb]{0,0,0}}}}}}
\end{picture}   \else
  \begin{picture}(0,0)\includegraphics{T.pstex}\end{picture}\setlength{\unitlength}{2763sp}\begingroup\makeatletter\ifx\SetFigFont\undefined \gdef\SetFigFont#1#2#3#4#5{\reset@font\fontsize{#1}{#2pt}\fontfamily{#3}\fontseries{#4}\fontshape{#5}\selectfont}\fi\endgroup \begin{picture}(2180,4338)(2101,-4483)
\put(2101,-3811){\makebox(0,0)[lb]{\smash{{\SetFigFont{12}{14.4}{\rmdefault}{\mddefault}{\updefault}{\color[rgb]{0,0,0}}}}}}
\put(3451,-3811){\makebox(0,0)[lb]{\smash{{\SetFigFont{12}{14.4}{\rmdefault}{\mddefault}{\updefault}{\color[rgb]{0,0,0}}}}}}
\put(2101,-3061){\makebox(0,0)[lb]{\smash{{\SetFigFont{12}{14.4}{\rmdefault}{\mddefault}{\updefault}{\color[rgb]{0,0,0}}}}}}
\put(3451,-3061){\makebox(0,0)[lb]{\smash{{\SetFigFont{12}{14.4}{\rmdefault}{\mddefault}{\updefault}{\color[rgb]{0,0,0}}}}}}
\put(2101,-2311){\makebox(0,0)[lb]{\smash{{\SetFigFont{12}{14.4}{\rmdefault}{\mddefault}{\updefault}{\color[rgb]{0,0,0}}}}}}
\put(3451,-2311){\makebox(0,0)[lb]{\smash{{\SetFigFont{12}{14.4}{\rmdefault}{\mddefault}{\updefault}{\color[rgb]{0,0,0}}}}}}
\put(2101,-1561){\makebox(0,0)[lb]{\smash{{\SetFigFont{12}{14.4}{\rmdefault}{\mddefault}{\updefault}{\color[rgb]{0,0,0}}}}}}
\put(3451,-1561){\makebox(0,0)[lb]{\smash{{\SetFigFont{12}{14.4}{\rmdefault}{\mddefault}{\updefault}{\color[rgb]{0,0,0}}}}}}
\put(3151,-361){\makebox(0,0)[lb]{\smash{{\SetFigFont{12}{14.4}{\rmdefault}{\mddefault}{\updefault}{\color[rgb]{0,0,0}}}}}}
\put(3226,-4411){\makebox(0,0)[lb]{\smash{{\SetFigFont{12}{14.4}{\rmdefault}{\mddefault}{\updefault}{\color[rgb]{0,0,0}}}}}}
\end{picture}   \fi
  \caption{Plotkin's Domain }
  \label{fig:T}
\end{figure}

But  is {\em not\/} Cartesian-closed.  Consider the
space~ of \cite[Figure~12]{AJ:domains}, see Figure~\ref{fig:T}.
This is an -domain: define the quasi-deflations
, , as mapping  to , any
element  to  if , and any other
element to .

However,  is not an -domain.

Assume  were a generating sequence of
quasi-deflations on .  For each function , there is a continuous map  that sends
 to ,  to ,  to  and
 to  ( exchanges  and  if
, leaves them unswapped if ).  Write  as , where  is finite.

We claim that:  for each , there is an
index  such that .  If there were an
element  of  such that , for
infinitely many values of , then this would hold for every
; but the map sending  and  to , and all other
elements to  would be in , which is impossible.  So, for  large enough,
no element  of  maps  to .
Similarly, for  large enough, no element  of  maps  to .  Since ,
for  large enough we find  with ,
, and .

We check that .  First, 
so .  Next,  is an element below
 and different from , and the only
element satisfying this is .  Similarly, .  By induction on , we show that .  At rank ,  is an element below
 and above both  and .
The only such element is .
Similarly, .  Finally,  is an
element above all , hence must equal .

Since , and , Claim  is proved.

However, there are uncountably many functions of the form ,
and only countably many elements of , since each set  is
finite, and for each , there are only finitely many
distinct sets  with .  We have
reached a contradiction.

Since exponentials in any full subcategory of the category of dcpos
must be isomorphic to the ordinary continuous function space
\cite{Smyth:CCC}, it follows:
\begin{prop}
   is not Cartesian-closed.
\end{prop}
The above argument also shows that, although  is both continuous
(even algebraic) and an -domain,  is not an
-domain: so Corollary~\ref{corl:qrb:V} is not enough to settle
the Jung-Tix problem in the positive either.

One might hope that countability would be the problem.  However, we
required countability in at least two places.  The first one is
Lemma~\ref{lemma:qs:nonempty}, which would fail in case we allowed for
directed families  instead of non-decreasing
sequences.  The second one is in the  direction of
Theorem~\ref{thm:qrb:qretr}, where we need countability to obtain 
as a quasi-projection, and not just a quasi-retract.  (This is similar
to an open problem in the theory of -domains, see \cite[Remark
after Theorem~4.9]{Jung:CCC}.)  In turn, we need quasi-projections,
not just quasi-retracts, in the Key Claim, Theorem~\ref{thm:qretr:V}.

To get around the Jung-Tix problem using our results, one might shift
the focus towards the Kleisli category , for
example.  This is a full subcategory of Jung, Kegelmann and Moshier's
pleasing category  of stably compact spaces and closed
relations \cite{DBLP:journals/entcs/JungKM01}.

On the other hand, we would like to point out the following deep
connection between -domains and -domains.
\begin{defi}
  \label{defi:ctrlQRB}
  A {\em controlled quasi-deflation\/} on a poset  is a pair of a
  Scott-continuous map  and of a quasi-deflation  such that  for
  every .  The map  is the {\em control\/}.

  A {\em controlled -domain\/} is a pointed dcpo  with a {\em
    generating family of controlled quasi-deflations\/}, i.e., a
  directed family of controlled quasi-deflations (in the pointwise
  ordering)  such that  for every .
\end{defi}
So a controlled quasi-deflation is a pair  with the
property that  for
every .  Every controlled -domain is a -domain:
given a generating family of controlled quasi-deflations , , and the converse inclusion  is obvious; so
 is a generating family of quasi-deflations.

\begin{thm}
  \label{thm:ctrlQRB}
  The controlled -domains are exactly the -domains, and
  hence form a Cartesian-closed category.
\end{thm}
\begin{proof}
  If  is a generating family of
  controlled quasi-deflations on a pointed dcpo , then  is
  finitely separated from : indeed, let  and , then for every , since ,
  there is a point  where 
  such that , and since , we have ; so  is a finitely separating
  set for  on .

  Conversely, assume  is an -domain, and let  be a directed family of continuous maps, finitely separated
  from , and whose pointwise least upper bound is
  .  Let .  By
  \cite[Lemma~2]{Jung:CCC:LICS},  is {\em strongly finitely
    separated\/} from , i.e., there is a finite set 
  of pairs of elements  such that for every , one
  can find a pair  such that .  Moreover, the pointwise least upper bound of  is again .

  For each pair  with , .
  Indeed, for every , since  and  is Scott-open,  for
  some .  Since  is compact, and the family
   is directed,  for some .  By
  directedness again, we can take the same  for all pairs  in .  But now 
  implies that whenever , then .  Using the
  separation property of , for every , one can find a
  pair  such that .  In
  particular, letting  be the set of elements  such that  for some ,  is {\em finitely separated
    from \/}, with separating set , with the obvious
  meaning: for every , there is an  such that .  In this case, we write .

  We now define  as .
  Since  is finite,  is a map from  to .
  It is monotonic, and we claim it is Scott-continuous.  Let
   be a directed family in .  Then  (since  is continuous) .  The latter intersection is an intersection of finite sets,
  hence there is a  such that , from which Scott-continuity is
  immediate.

  We must now check that  is directed.  Given
  , one can find  so that  and .  By directedness, there is
  an  such that .  We claim that
  for every , .  For every , .  So .  Since , there
  is an element  such that .  So  is in , and below ,
  whence .  Similarly,  is in .

  Finally,  for every : by definition,
  there is a pair  such that ; so , , and  is below .
  And , while the
  converse inclusion is obvious.  So  is a
  generating family of quasi-deflations.
\end{proof}

Defining the {\em controlled -domains\/} as the
-domains, except with sequences of controlled
quasi-deflations instead of directed families, and similarly for the
-domains (a.k.a., the countably-based -domains, again
a Cartesian-closed category \cite[Theorem~11]{Jung:CCC:LICS}), we
prove similarly:
\begin{thm}
  \label{thm:omega:ctrlQRB}
  The controlled -domains are exactly the
  -domains, and hence form a Cartesian-closed category.
\end{thm}

Using this last observation, Corollary~\ref{corl:qrb:V} settles half
of the conjecture that the probabilistic powerdomain of an
-domain would be an -domain again.  We are only
lacking {\em control\/}.

\section*{Open Problems}

\begin{enumerate}[(1)]
\item Is countability necessary in Theorem~\ref{thm:qrb:qretr}?
  Precisely, can one show that the -domains are exactly the
  quasi-retracts of -domains?  The main difficulty seems to lie in
  the fact that a non-countable analog of
  Lemma~\ref{lemma:qs:nonempty} is missing---and Rudin's Lemma does
  not quite give us what we need, as discussed before the statement
  of the lemma.
\item If  is a quasi-retract of ,  is stably compact, and 
  is , then is  a quasi-retract of
  ?  This would be the analog of
  Theorem~\ref{thm:qretr:V}, only with quasi-retractions instead of
  quasi-projections.
\item Is stable compactness necessary to derive Theorem~\ref{thm:qretr:V}?
\item One way of trying to prove that the probabilistic powerdomain of
  an -domain is again an -domain would be by
  inventing a new notion, say of \emph{good maps}, and show that the
  -domains, or alternatively the controlled
  -domains, are exactly the images under good maps of
  -domains.  Good maps should intuitively be intermediate
  between projections and proper surjective maps, in the sense that
  every projection should be good, and that every good map should be
  proper and surjective.  Indeed surjective proper maps preserve the
   part, but not the control, while projections preserve too
  much, in the sense that not all -domains, only the
  -domains, are retracts of -domains.  Such a
  characterization of -domains would be of independent
  interest, too.
\end{enumerate}

\section*{Acknowledgment}

I would like to thank G. Plotkin for discussions, the anonymous
referees at LICS'2010, and the anonymous referees of this paper, who
offered counterexamples to conjectures, and numerous simplifications
and improvements in almost every part of the paper.  I am greatly
indebted to them.





\bibliographystyle{alpha}
\bibliography{qrb}

\end{document}
