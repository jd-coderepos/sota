\subsection{Linear Lower Bound}
\label{sec:linbound}
Consider systems with  anonymous processes and an arbitrary correct consensus algorithm 
  satisfying the nondeterministic solo termination property. 
We will assume that no execution of the algorithm uses more than  registers (otherwise, we are trivially done),
  and prove that such an algorithm has to use  registers, which completes the proof.
For notational convenience, let us define  to be .

The argument in~\lemmaref{lem:sqrt} relies on a new set of clones in each iteration 
  to overwrite the changes to the contents of the registers made during the inductive step. 
This is the primary reason why we only get an  lower bound.
As the authors of~\cite{FHS98} also mention,
  to get a stronger lower bound we would instead have to reuse existing processes.
In order to do so, these existing processes need to cover the registers in our inductive configurations
  (we must also ensure proper valency conditions on what they are about to write, but let us focus on the covering).
Now, even if we reach such a configuration, 
  during a solo execution interval of some process in the subsequent induction step, all the registers may get written to,
  and we would have to use all the covering existing processes to overwrite the changes.
Therefore, in the next configuration, there is no way to guarantee that the existing processes 
  would still cover various registers.

This is the primary reason why we have to replace solo executions in the proof with a different class of executions
  that we call \emph{reserving}.
Intuitively, reserving executions ensure that for the registers that are written to, 
  some processes are reserved to cover them.
This way, we can have reserved processes cover the registers in subsequent inductive configurations.
Notice that the definition of valency used in the proof of~\lemmaref{lem:sqrt} was based on solo executions.
Thus, we also redefine valency based on reserving executions. 
\subsubsection{Reserving executions}
The following is a formal definition of a reserving execution interval.
\begin{definition}
Let  be some configuration reachable by the algorithm, 
  and let  be a set of at least  processes.
We call an execution interval  that starts from configuration  \emph{reserving} from  by  if:
\begin{itemize}[noitemsep, nolistsep]
\item Every step in  is by a process in .  
\item At any time during the execution of :  
  if we let  be the set of registers written to so far during , 
then, for each register in , there is a \emph{reserved} process  covering that register, one per register. 
\item If a process  returns during  then it does so in the last step of . 
\end{itemize}
\end{definition}
Notice that by definition any prefix of a reserving execution interval is also a reserving execution interval. 
Let  be the set of all reserving execution intervals from  by processes in 
  that end with a process  returning.
We first show that given sufficiently many processes, such an execution interval exists.
This is essential for defining the valency later.
Recall that we assumed a strict upper bound of  on the number of registers that can ever be written.
\begin{claim}
For any reachable configuration  and a set of at least  processes , none of which have returned yet, 
  we have that .
\label{clm:reserved}
\end{claim}
\begin{proof}
For a given  and , we will prove the claim by constructing a particular reserving execution interval  
  that ends when some process  returns.
We start with an empty  and continuously extend it.
In the first stage, one by one, for each process : 
\begin{itemize}[noitemsep, nolistsep]
\item Due to the nondeterministic solo termination, there exists a solo execution of  where  returns.
  \begin{itemize}[noitemsep, nolistsep]
  \item If  ever writes to any register during this solo execution, 
    extend  by the prefix of the execution before this write, and move to the next process in .
  \item Otherwise, complete  by extending it with the whole solo execution of .
  \end{itemize}
\end{itemize}
We have finitely many processes and the first stage described above consists of extending the execution interval 
  at most  times.
Each time, because of the nondeterministic solo termination for some process , 
  we extend  by a prefix of a finite solo execution of .
Moreover, all operations are reads by processes in , and therefore the prefix of  constructed so far is reserving.

If some process returns in the first stage, the construction of  is complete.
Otherwise, since the first stage is finite, we move on to the second stage described below.
In the configuration after the first stage each of the at least  processes in  is covering a register 
  (by their next write operation after the first stage).
From that configuration, the execution interval  is extended by repeatedly doing the following:
\begin{itemize}[noitemsep, nolistsep]
\item[1.] Let  be the set of covered registers by processes of .
  Since , we can find two processes  covering the same register in .
\item[2.] Due to the nondeterministic solo termination, there exists a solo execution of  where  returns.
  \begin{itemize}[noitemsep, nolistsep]
  \item If  ever writes to a register outside of  during this solo execution, 
    extend  by the prefix of the execution before this write, and continue from the first step.
    Notice that at the beginning of the next iteration, process  still covers a register as required.
  \item Otherwise, complete  by extending it with the whole solo execution of .
  \end{itemize}
\end{itemize}
In the second stage, each iteration terminates, since for any process , 
  we can extend by at most the terminating solo execution of , which exists and is finite.
After each iteration, if the construction is not complete, the size of  increases by one.
But there are at most  registers in the system and . 
Thus, after at most  finite extensions, we will complete the construction of  when some process returns.

The execution is reserving because at all times, the registers that were written-to are in .
Moreover, for each register in , there is always a process covering it
  starting from the time it was first covered by some process  in the second step of some iteration 
  all the way until the end of .
\end{proof}
\noindent The next claim follows immediately from the definition of reserving executions.
\begin{claim}
\label{clm:prefix}
Consider a reachable configuration , a set of at least  processes  none of which have returned yet, 
  and another configuration  reached after some process  performs a write operation  in .  
Moreover, assume that another process  with  is covering the same register that  writes to.
Then if , then  is in  where . 
\end{claim}
\subsubsection{New definition of valency}
We say that a configuration  is -valent with respect to the set of processes ,
  if there exists a subset of at least  processes  and a reserving execution in  
  that finishes when some process in  returns .
We call  -valent w.r.t. , 
  if there exists a subset of \emph{exactly}  processes  ,
  and a reserving execution interval in  returning . 
We define -valent and -valent analogously.
If  contains at least  processes that have not returned,~\claimref{clm:reserved} implies that 
  every configuration is -valent or -valent (and thus -valent or -valent).



As in our earlier definition in~\sectionref{sec:sqrt}, but unlike the standard definition,
  a configuration that is -valent can still also be -valent
  in which case we call it bivalent.
Basically, a configuration is bivalent if it is both 
  -valent due to some  and 
  -valent due to some .
A configuration that is not bivalent is called univalent.
Finally, similar to our earlier convention,
  we define a configuration to be -univalent if it is -valent but not -valent.
On the other hand, a configuration that is -valent but not -valent
  is called -univalent.
Terms bivalent, univalent, -univalent and -univalent are defined analogously.

Next we prove a claim that lets us find reserving executions consisting of disjoint processes. 
\begin{claim}
\label{clm:disjoint}
Consider a configuration  which is bivalent w.r.t. .
Assume that there are (possibly intersecting) sets of at least  processes each  and  
  such that ,
  and some reserving execution in  ends when  returns ,
  while some reserving execution in  ends when  returns .
Then there are also disjoint sets of processes  and  (), 
  such that an execution in  returns  and an execution in  returns . 
Moreover,  and . 
\end{claim}
\begin{proof}
None of the processes in  may have already returned in configuration ,
  as that would contradict the existence of a reserving execution returning the other output. 
If  and  do not intersect then we set  and .
Otherwise, we can find a set  of  processes.
By~\claimref{clm:reserved},  is non-empty, 
  and without loss of generality, some execution in  returns .
Then, we set  and  (if all executions in  return , we would set  and ).
\end{proof}
\subsubsection{The process-clone pairs and the proof}
As mentioned earlier, it is obviously not sufficient to simply cover registers with existing processes
  without any knowledge of what they are about to write.
In the proof of~\lemmaref{lem:sqrt} we used new clones that covered registers to block-overwrite these registers 
  back to the contents whose valency we knew.
In order to do something similar with existing processes, we associate a dedicated clone to each process.
The process and its clone remain in the same states and perform the same operations during the whole execution.

Usually, when we schedule a process to perform an operation,
  its clone performs the same operation immediately after the process.  
  Thus the pair of the process and the clone remain in the same state.
Under these circumstances, we can treat the pair of the process and its clone as a single process,
  because no process can distinguish the execution from when the clone would not take steps.
However, sometimes we will \emph{split} the pair by having only the process perform a write operation 
  and let the clone cover the register.
We will explicitly say when this is the case.
After we split the pair of process and clone in such a way, 
  we will not schedule the process to take any more steps and thus the clone will remain poised to write to the covered register.
After some delay, we will schedule the clone of the process to write, 
  effectively resetting the register to the value it had when the process wrote. 
Moreover, because meanwhile the process did not take any steps,
  after the write the clone will again be in the same state as its associated process.
Hence the pair of the process and clone will no longer be split, 
  and will continue taking steps in sync like a single process.

This is different from the way clones were used in the proof of~\lemmaref{lem:sqrt},
  because after the pair of the process and its clone is united, it can be split again.
Therefore, the same clone can reset the contents of registers written by its associated process multiple times, 
instead of requiring a new clone every time.

We call a split pair of a process and a clone \emph{fresh} as long as the register that 
  the process wrote to, and its clone is covering, has not been overwritten.
After the register is overwritten, we call the split pair \emph{stale}.

In addition, we also use cloning in a way similar to the proof of~\lemmaref{lem:sqrt}, 
  except that we do this at most constantly many times, as opposed to  times, to reach the next configuration .
Moreover, each time when we do this, we create duplicates of both the process and its corresponding clone.
This new process-clone pair is in the same state as the original pair, 
  and from there on behaves like a single new process similar to all other pairs.
We will always consider valency with respect to sets of processes whose pairs are not split. 
Therefore, the definition of valency does not need to change 
  when the clones keep taking steps immediately after their processes.

Sometimes, when considering process-clone pairs, none of which are split, 
  we may refer to them as processes, i.e. we may talk about a process taking steps or returning a value. 
As mentioned earlier, it is assumed that as long as the pair is not split, 
  the clone always follows and takes the same steps right after the process. 
Hence, in this context, a process taking a step means a pair taking a step.   

Now we are ready to prove the main result.
\begin{theorem}
\label{thm:linclone}
In the system of anonymous processes, 
  consider any correct consensus algorithm satisfying nondeterministic solo termination,
  with the property that every execution uses at most  registers.
For each  with , there exists a set  containing  process-clone pairs  
  such that a configuration  is reachable through an execution  
  by processes and clones in  with the following properties:
\begin{itemize}[noitemsep, nolistsep]
\item[1.] There exists a set  of  registers, that can be partitioned in two disjoint subsets , where:
\begin{itemize}[noitemsep, nolistsep]
\item  consists of all registers in the system that each have one fresh split pair on them, 
  last written by some process whose clone has not yet performed the write and is covering the register.
\item .
Each register in  is covered by an unique pair of both a process and its clone.
\end{itemize}
Thus, each fresh pair is split on a different register in ,
  and an additional  pairs are covering the registers in .
Let  be the set of these  pairs.
\item[2.] There are at most  stale split pairs in the system,
  that are all split on pairwise different registers from .
Let  be the set of these at most  stale split pairs.
\item[3.] There exist disjoint sets of process-clone pairs that are not split 
   with ,
  such that an execution in  returns  and an execution in  returns .\footnote{The pairs of processes in  and  are not split, because all split pairs belong to  (fresh to  and stale to ). Also, the third condition implies that the configuration  is bivalent.} 
\end{itemize}
\end{theorem}
\begin{proof}
The proof is by induction on , with the base case .
Out of the  processes-clone pairs, 
  half of them start with an input  and half start with an input .
We let  be the initial state,
   be a set of some  pairs with input , and
   be a set of some  pairs with input .
The first two properties are trivially satisfied; also  and .
By~\claimref{clm:reserved} and correctness of consensus,
  there is a reserving execution in  that decides ,
  and a reserving execution in  that decides  ( is bivalent).
Observe that the pairs are not split and for the purposes of valency we can just consider the steps of processes.

Now, let us assume induction hypothesis for some , i.e. the existence of  and  with the required three properties,
  and prove the step for  by extending  to , resulting in the configuration .
Let , , , ,  and  all be defined as in the theorem statement for . 
  Our goal is to construct sets , , , ,  and  for . 
In  we have two more process-clone pairs available that have not taken steps and 
  can be used to clone an existing process-clone pair.
Let  denote .
Since ,  and , we have .

For all but  split pairs both processes and clones are in the same states, 
  about to perform the same operations.
By definition, each stale pair in  is split on a different register from .
In the following argument, we extend the execution from  to  by steps of processes and clones not in .
This can introduce new stale split pairs and the resulting configuration  
  may not immediately satisfy the second property.
We will then show how to modify the extension and unite some stale split pairs, 
  such that the resulting configuration satisfies all properties, including the second property with the new .
	
Let  be the reserving execution interval that returns ,
  and let  be the reserving execution interval that returns .
Notice that each time a process in  or  takes a step in  or , 
  its clone performs an identical step immediately after.
The execution  ends with a process-clone pair  returning  and 
  the execution  ends with a process-clone pair  returning . 

Each register in  was covered by some pair of both a process and its clone in .
Let  be a block write to all registers in  by only the processes but not the clones 
  of these respective covering pairs: i.e. after each write we get a new fresh split pair.
Consider a configuration  reached from  by executing this block write, i.e. a configuration reached after .
Assume that  is -valent, without loss of generality, because it has a valency.
For any execution interval , let us denote by  the set of registers written to during .
Hence,  is the set of registers in  that are written-to during .
Each register in  is covered by a clone of a split pair whose process has already performed the write and is stopped.
Define  as a block write to all registers in  by these trailing clones of the split pairs in : 
  i.e. after each write another clone catches up with its process and a previously split pair is united.
Basically, if we run an execution interval  from  that changes contents of some registers in ,
  we can then clean these changes up by executing , 
  which leads to all registers in  having the same contents as in .

Using a crude covering argument we can first show that
\begin{repclaim}{clm:alphaout}
The execution interval  must contain a write operation outside .
\end{repclaim}
\noindent Based on this we can write , 
  where  is the write operation to a register , performed by some process-clone pair .
    
Looking ahead, when we reach , the new set of registers  will be .
Next, we prove the following claim using an FLP-like case analysis:
\begin{repclaim}{clm:casean}
We can extend execution  (i.e. from ) with an execution interval  and reach a configuration 
  satisfying the first and the third inductive requirements to be  
  with a properly defined , , ,  and ,
  and with all process-clone pairs that are not split being in sync.
But the second property is not immediately satisfied.
All stale split pairs from  remain stale and split, but some pairs that were fresh and split on registers 
  in  may have become stale in  
  (because neither the process nor the clone in the split pair has taken steps while the register was overwritten in ). 
However, these are the only possible new stale split pairs in , and they do not belong to the new sets .
\end{repclaim}
The proofs of these claims are provided later. 
In order to finish the proof of the theorem, we need to show how to construct .
According to the above~\claimref{clm:casean} we can extend the execution to reach the next configuration  
  satisfying first and third but not the second property about the stale split pairs .
In  we had at most  stale pairs in the system, each split on a different register, and  was the set of these pairs.
But on the way to reaching , we may have introduced new stale pairs in the system.
According to~\claimref{clm:casean} these must be the pairs that were fresh and split on registers in  in ,
  and whose associated register in  has been overwritten during , making them stale in .

The set of all stale pairs in  may not satisfy the requirements imposed for ,
  since there could already have been a stale pair split on a register in  in  (in ).
Then two stale pairs would be split on this register in , violating the second property.
However, for each such register in , 
  we know a stale pair  was split on it in ,
  and that this register was written-to during extension .
We now modify the extension ; we add a single write by the clone of the stale split pair 
  immediately before a write operation to the same register that was already in .
This way, no pair other than the clone of  observes a difference between the two executions, 
  and we will use the configuration reached by the modified execution as .
Because of this indistinguishability, the new  still satisfies other required properties.
Moreover, the pair  is not split anymore; it is united since the clone has caught up with its process.

We can do the above modification to the execution for each register in  
  that previously ended up with two stale split processes in .
Let the modified execution extension be .
In , some stale split pairs from  are united, indistinguishably to all other processes and clones, 
  leading to a configuration , that still satisfies the first and third properties,
  and has at most one stale pair split on any register.
We take  to be the set of stale split pairs.
By construction, all stale pairs are split on registers in  and no two on the same register, 
  so we do have  as desired.
Hence, we have reached configuration  satisfying all properties and completing the proof.
\end{proof}
\begin{corollary}
In a system of  anonymous processes, 
  any consensus algorithm satisfying non-deterministic solo termination must use  registers.
\end{corollary}
\begin{proof}
\theoremref{thm:linclone} directly implies the  lower bound on the number of registers used in some execution.
If  is the number of anonymous processes and no execution uses more than  registers,
  by~\theoremref{thm:linclone} we can reach  for large enough , and we have enough processes .
In  there are  registers in , each of which has either already been written-to ()
  or are covered by unique processes ().
We could perform a block write to  by covering processes from  in , 
  after which in the resulting execution  different registers would have been written to.
\end{proof}
We now provide the delayed proofs for the claims.
\begin{claim}
\label{clm:alphaout}
The execution interval  must contain a write operation outside .		 
\end{claim}
\begin{proof}
Assume the contrary.
We know that the execution  decides .
No process or clone that takes a step in  or  appears in 
  (they belong to , disjoint from  and ),  
  and by definition, no process or clone from  takes a step in ,  or . 
Thus, to all processes (and clones) in , 
  the configurations after  and after ,
  which is configuration , are indistinguishable.
This is because no process (or clone) in  has taken steps, the registers in  contain the same values, 
  and other registers were not touched during ,  or .
Configuration  is -valent, so some extension from  by 
  an execution interval from  decides .
This contradicts the correctness of the algorithm.
\end{proof}
\begin{claim}
\label{clm:casean}
We can extend execution  (i.e. from ) with execution interval  and reach a configuration 
  satisfying the first and the third inductive requirements to be  
  with properly defined , , ,  and ,
  and with all process-clone pairs that are not split being in sync.
But the second property is not immediately satisfied.
All stale split pairs from  remain stale and split, but some pairs that were fresh and split on registers 
  in  may have become stale in  
  (because neither the process nor the clone in the split pair has taken steps while the register has been overwritten in ). 
However, these can be the only new stale split pairs in  and they do not belong to the new sets .
\end{claim}
\begin{proof}
The proof works by case analysis. We use the notation from~\theoremref{thm:linclone}.
 does not contain any split pairs, so we can consider valency with respect to processes in .  

\paragraph{Case 1: the configuration reached by the execution  is -valent:}
Let  be the length of  and  be a prefix of  of size  for .
Here we consider steps of process-clone pairs from , 
  i.e. the difference between  and  is the same operation performed twice by a process and its clone,
  and  counts these couples of identical operations,
  as illustrated in~\figureref{fig:case1}.
Pairs in  are not split, as by the inductive hypothesis only pairs in  are split, 
  and .

\begin{figure}
\label{fig:case1}
\caption{Case 1}
\resizebox{\textwidth}{!}{\includegraphics{case1.eps}}
\end{figure}

We consider two further subcases.

\paragraph{Case 1.1: for some , 
  the configuration reached by the execution  is bivalent:}
In this case, we let  be precisely the configuration reached by , 
  and let .
Clearly, .
Moreover, since  is bivalent there are two subsets of  process-clone pairs in  
  and respective reserving execution intervals from  that return  and .
Since , by~\claimref{clm:disjoint}, we can actually find  and , 
  with  and , such that
  an execution in  returns  and an execution in  returns .
These sets  and  are the new sets of pairs for , and as required .

The new set  will be , i.e. the registers from  
  that have not been touched during our execution extension  from  to .
For each of these registers we still have the same one pair from  split on it, 
  and since this pair was fresh in  and the register has not been written to during the extension,
  it is still fresh in  as required.  
There are no other fresh split pairs in :
  no new split pairs were introduced during the extension,
  and the rest of fresh pairs in  were split on .
These pairs are not fresh in , as their register was written to during the extension.

The set  is simply . 	
We must show that there is a unique process-clone pair covering each of these registers.
For each register in , we take the same pair from  that was associated and covering it in .
For each register in , 
  we find a unique pair from  covering it in .
Since  is a reserving execution interval from , 
  all its prefixes including  are also reserving.
Thus, in  for each register that has been ever written to during , 
  in particular for registers in , 
  we find and associate a unique covering pair in .
Technically, if , register  is not yet written, 
  but the next operation in  is  by a pair covering .

The set  contains all  pairs that we associated with registers in , so .
The number of stale split pairs may have increased, however.
We still have pairs in  plus  of the previously fresh pairs that are now stale.
We deal with this in~\theoremref{thm:linclone} by modifying the execution extension to unite some stale pairs from , 
  leaving us with a desired subset  of at most  stale pairs.

Finally, as , and  was disjoint from , ,  and ,
   as required.

\paragraph{Case 1.2: for every , 
  the configuration reached by the execution  is univalent:}
By \emph{\textbf{Case 1}} assumption  is -valent, 
  so it must be -univalent.
On the other hand,  ends with a process (and its clone) returning , 
  so the configuration reached by  must be -univalent.
No intermediate configuration is bivalent, so we can find a  such that 
   leads to a -univalent configuration and 
   leads to a -univalent configuration. 
We take  to be the configuration reached after ,
  and define sets , ,  and  the same way as if  was bivalent in \emph{\textbf{Case 1.1}}.
This still works, but we need a new way to find  and  with desired properties.

Let  be the operation by a process-clone pair in  separating  and .
 may not be a read, as no process (or clone) in  can distinguish between configurations 
  after  or ,
  making it impossible for these configurations to have different univalencies w.r.t. to .
Let  be a set of any  pairs in . 
By~\claimref{clm:reserved},  is non-empty and since  is -univalent,
  all executions in  return .
Recall that  is  plus two process-clone pairs that have not taken any steps,
  so that . 
Let us use these process-clone pairs to create two duplicates of the process-clone pair performing the write operation .
Both of these new pairs will be in the same state as the original pair performing .
These duplicate processes (and their clones) are thus poised on the same register about to perform 
  write operations  and  identical to the operation  at configuration . 

Recall that  and let  be a set of  pairs from .
Let  be the union of  and the two new duplicate pairs,  in total.
Let  be the -univalent configuration reached after .
Due to -univalency, there is a reserving execution  that returns .
Having one duplicate pair perform  from  while another covers the same register with the same operation , 
  we reach the state indistinguishable from  for all  pairs in .
Thus, execution  from  returns , and by~\claimref{clm:prefix}, .
By construction  and , as desired and 
  the intersection of  and  with  or  is empty like in \emph{\textbf{Case 1.1}}.

The remaining case is when the configuration reached by the execution  is -univalent.
It is a bit more involved but utilizes the same general ideas and techniques.
One difference is that we also split pairs.
The construction is given in~\appendixref{app:case2}.
\end{proof}
 