\subsection{Linear Lower Bound}
\label{sec:linbound}
Consider systems with $n$ anonymous processes and an arbitrary correct consensus algorithm 
  satisfying the nondeterministic solo termination property. 
We will assume that no execution of the algorithm uses more than $n/20$ registers (otherwise, we are trivially done),
  and prove that such an algorithm has to use $\Omega(n)$ registers, which completes the proof.
For notational convenience, let us define $m$ to be $n/20$.

The argument in~\lemmaref{lem:sqrt} relies on a new set of clones in each iteration 
  to overwrite the changes to the contents of the registers made during the inductive step. 
This is the primary reason why we only get an $\Omega(\sqrt{n})$ lower bound.
As the authors of~\cite{FHS98} also mention,
  to get a stronger lower bound we would instead have to reuse existing processes.
In order to do so, these existing processes need to cover the registers in our inductive configurations
  (we must also ensure proper valency conditions on what they are about to write, but let us focus on the covering).
Now, even if we reach such a configuration, 
  during a solo execution interval of some process in the subsequent induction step, all the registers may get written to,
  and we would have to use all the covering existing processes to overwrite the changes.
Therefore, in the next configuration, there is no way to guarantee that the existing processes 
  would still cover various registers.

This is the primary reason why we have to replace solo executions in the proof with a different class of executions
  that we call \emph{reserving}.
Intuitively, reserving executions ensure that for the registers that are written to, 
  some processes are reserved to cover them.
This way, we can have reserved processes cover the registers in subsequent inductive configurations.
Notice that the definition of valency used in the proof of~\lemmaref{lem:sqrt} was based on solo executions.
Thus, we also redefine valency based on reserving executions. 
\subsubsection{Reserving executions}
The following is a formal definition of a reserving execution interval.
\begin{definition}
Let $C$ be some configuration reachable by the algorithm, 
  and let $P$ be a set of at least $m+1$ processes.
We call an execution interval $\gamma$ that starts from configuration $C$ \emph{reserving} from $C$ by $P$ if:
\begin{itemize}[noitemsep, nolistsep]
\item Every step in $\gamma$ is by a process in $P$.  
\item At any time during the execution of $\gamma$:  
  if we let $R_w$ be the set of registers written to so far during $\gamma$, 
then, for each register in $R_w$, there is a \emph{reserved} process $p \in P$ covering that register, one per register. 
\item If a process $p \in P$ returns during $\gamma$ then it does so in the last step of $\gamma$. 
\end{itemize}
\end{definition}
Notice that by definition any prefix of a reserving execution interval is also a reserving execution interval. 
Let $\lit{Res}(C, P)$ be the set of all reserving execution intervals from $C$ by processes in $P$
  that end with a process $p \in P$ returning.
We first show that given sufficiently many processes, such an execution interval exists.
This is essential for defining the valency later.
Recall that we assumed a strict upper bound of $m$ on the number of registers that can ever be written.
\begin{claim}
For any reachable configuration $C$ and a set of at least $m+1$ processes $P$, none of which have returned yet, 
  we have that $\lit{Res}(C, P) \neq \emptyset$.
\label{clm:reserved}
\end{claim}
\begin{proof}
For a given $C$ and $P$, we will prove the claim by constructing a particular reserving execution interval $\gamma$ 
  that ends when some process $p \in P$ returns.
We start with an empty $\gamma$ and continuously extend it.
In the first stage, one by one, for each process $p \in P$: 
\begin{itemize}[noitemsep, nolistsep]
\item Due to the nondeterministic solo termination, there exists a solo execution of $p$ where $p$ returns.
  \begin{itemize}[noitemsep, nolistsep]
  \item If $p$ ever writes to any register during this solo execution, 
    extend $\gamma$ by the prefix of the execution before this write, and move to the next process in $P$.
  \item Otherwise, complete $\gamma$ by extending it with the whole solo execution of $p$.
  \end{itemize}
\end{itemize}
We have finitely many processes and the first stage described above consists of extending the execution interval 
  at most $|P|$ times.
Each time, because of the nondeterministic solo termination for some process $p \in P$, 
  we extend $\gamma$ by a prefix of a finite solo execution of $p$.
Moreover, all operations are reads by processes in $P$, and therefore the prefix of $\gamma$ constructed so far is reserving.

If some process returns in the first stage, the construction of $\gamma$ is complete.
Otherwise, since the first stage is finite, we move on to the second stage described below.
In the configuration after the first stage each of the at least $m+1$ processes in $P$ is covering a register 
  (by their next write operation after the first stage).
From that configuration, the execution interval $\gamma$ is extended by repeatedly doing the following:
\begin{itemize}[noitemsep, nolistsep]
\item[1.] Let $R$ be the set of covered registers by processes of $P$.
  Since $|R| \leq m < |P|$, we can find two processes $p, q \in P$ covering the same register in $R$.
\item[2.] Due to the nondeterministic solo termination, there exists a solo execution of $p$ where $p$ returns.
  \begin{itemize}[noitemsep, nolistsep]
  \item If $p$ ever writes to a register outside of $R$ during this solo execution, 
    extend $\gamma$ by the prefix of the execution before this write, and continue from the first step.
    Notice that at the beginning of the next iteration, process $p$ still covers a register as required.
  \item Otherwise, complete $\gamma$ by extending it with the whole solo execution of $p$.
  \end{itemize}
\end{itemize}
In the second stage, each iteration terminates, since for any process $p \in P$, 
  we can extend by at most the terminating solo execution of $p$, which exists and is finite.
After each iteration, if the construction is not complete, the size of $R$ increases by one.
But there are at most $m$ registers in the system and $|R| \leq m$. 
Thus, after at most $m$ finite extensions, we will complete the construction of $\gamma$ when some process returns.

The execution is reserving because at all times, the registers that were written-to are in $R$.
Moreover, for each register in $R$, there is always a process covering it
  starting from the time it was first covered by some process $p$ in the second step of some iteration 
  all the way until the end of $\gamma$.
\end{proof}
\noindent The next claim follows immediately from the definition of reserving executions.
\begin{claim}
\label{clm:prefix}
Consider a reachable configuration $C$, a set of at least $m+1$ processes $P'$ none of which have returned yet, 
  and another configuration $C'$ reached after some process $p \not \in P'$ performs a write operation $w_p$ in $C$.  
Moreover, assume that another process $q \neq p$ with $q \not \in P'$ is covering the same register that $w_p$ writes to.
Then if $\gamma \in \lit{Res}(C', P')$, then $w_p \gamma$ is in $\lit{Res}(C, P)$ where $P = P' \cup \{p\} \cup \{q\}$. 
\end{claim}
\subsubsection{New definition of valency}
We say that a configuration $C$ is $0$-valent$_U$ with respect to the set of processes $U$,
  if there exists a subset of at least $m+1$ processes $P \subseteq U$ and a reserving execution in $\lit{Res}(C, P)$ 
  that finishes when some process in $P$ returns $0$.
We call $C$ $0$-valent$_U^{m+1}$ w.r.t. $U$, 
  if there exists a subset of \emph{exactly} $m+1$ processes $P \subseteq U$ $(|P| = m+1)$,
  and a reserving execution interval in $\lit{Res}(C, P)$ returning $0$. 
We define $1$-valent$_U$ and $1$-valent$_U^{m+1}$ analogously.
If $U$ contains at least $m+1$ processes that have not returned,~\claimref{clm:reserved} implies that 
  every configuration is $0$-valent$_U^{m+1}$ or $1$-valent$_U^{m+1}$ (and thus $0$-valent$_U$ or $1$-valent$_U$).



As in our earlier definition in~\sectionref{sec:sqrt}, but unlike the standard definition,
  a configuration that is $0$-valent$_U^{m+1}$ can still also be $1$-valent$_U^{m+1}$
  in which case we call it bivalent$_U^{m+1}$.
Basically, a configuration is bivalent$_U^{m+1}$ if it is both 
  $0$-valent$_U^{m+1}$ due to some $P \subseteq U$ and 
  $1$-valent$_U^{m+1}$ due to some $Q \subseteq U$.
A configuration that is not bivalent$_U^{m+1}$ is called univalent$_U^{m+1}$.
Finally, similar to our earlier convention,
  we define a configuration to be $0$-univalent$_U^{m+1}$ if it is $0$-valent$_U^{m+1}$ but not $1$-valent$_U^{m+1}$.
On the other hand, a configuration that is $1$-valent$_U^{m+1}$ but not $0$-valent$_U^{m+1}$
  is called $1$-univalent$_U^{m+1}$.
Terms bivalent$_U$, univalent$_U$, $0$-univalent$_U$ and $1$-univalent$_U$ are defined analogously.

Next we prove a claim that lets us find reserving executions consisting of disjoint processes. 
\begin{claim}
\label{clm:disjoint}
Consider a configuration $C$ which is bivalent$_U$ w.r.t. $U$.
Assume that there are (possibly intersecting) sets of at least $m+1$ processes each $P \subseteq U$ and $Q \subseteq U$ 
  such that $|U| \geq |P| + |Q| + m$,
  and some reserving execution in $\lit{Res}(C, P)$ ends when $p \in P$ returns $0$,
  while some reserving execution in $\lit{Res}(C, Q)$ ends when $q \in Q$ returns $1$.
Then there are also disjoint sets of processes $P' \subseteq U$ and $Q' \subseteq U$ ($P' \cap Q' = \emptyset$), 
  such that an execution in $\lit{Res}(C, P')$ returns $0$ and an execution in $\lit{Res}(C, Q')$ returns $1$. 
Moreover, $m+1 \leq \min(|P'|, |Q'|) \leq \min(|P|, |Q|)$ and $\max(|P'|, |Q'|) \leq \max(|P|, |Q|)$. 
\end{claim}
\begin{proof}
None of the processes in $U$ may have already returned in configuration $C$,
  as that would contradict the existence of a reserving execution returning the other output. 
If $P$ and $Q$ do not intersect then we set $P' = P$ and $Q' = Q$.
Otherwise, we can find a set $H \subseteq U \setminus P \setminus Q$ of $m+1$ processes.
By~\claimref{clm:reserved}, $\lit{Res}(C, H)$ is non-empty, 
  and without loss of generality, some execution in $\lit{Res}(C, H)$ returns $0$.
Then, we set $P' = H$ and $Q' = Q$ (if all executions in $\lit{Res}(C, H)$ return $1$, we would set $P' = P$ and $Q' = H$).
\end{proof}
\subsubsection{The process-clone pairs and the proof}
As mentioned earlier, it is obviously not sufficient to simply cover registers with existing processes
  without any knowledge of what they are about to write.
In the proof of~\lemmaref{lem:sqrt} we used new clones that covered registers to block-overwrite these registers 
  back to the contents whose valency we knew.
In order to do something similar with existing processes, we associate a dedicated clone to each process.
The process and its clone remain in the same states and perform the same operations during the whole execution.

Usually, when we schedule a process to perform an operation,
  its clone performs the same operation immediately after the process.  
  Thus the pair of the process and the clone remain in the same state.
Under these circumstances, we can treat the pair of the process and its clone as a single process,
  because no process can distinguish the execution from when the clone would not take steps.
However, sometimes we will \emph{split} the pair by having only the process perform a write operation 
  and let the clone cover the register.
We will explicitly say when this is the case.
After we split the pair of process and clone in such a way, 
  we will not schedule the process to take any more steps and thus the clone will remain poised to write to the covered register.
After some delay, we will schedule the clone of the process to write, 
  effectively resetting the register to the value it had when the process wrote. 
Moreover, because meanwhile the process did not take any steps,
  after the write the clone will again be in the same state as its associated process.
Hence the pair of the process and clone will no longer be split, 
  and will continue taking steps in sync like a single process.

This is different from the way clones were used in the proof of~\lemmaref{lem:sqrt},
  because after the pair of the process and its clone is united, it can be split again.
Therefore, the same clone can reset the contents of registers written by its associated process multiple times, 
instead of requiring a new clone every time.

We call a split pair of a process and a clone \emph{fresh} as long as the register that 
  the process wrote to, and its clone is covering, has not been overwritten.
After the register is overwritten, we call the split pair \emph{stale}.

In addition, we also use cloning in a way similar to the proof of~\lemmaref{lem:sqrt}, 
  except that we do this at most constantly many times, as opposed to $r$ times, to reach the next configuration $C_{r+1}$.
Moreover, each time when we do this, we create duplicates of both the process and its corresponding clone.
This new process-clone pair is in the same state as the original pair, 
  and from there on behaves like a single new process similar to all other pairs.
We will always consider valency with respect to sets of processes whose pairs are not split. 
Therefore, the definition of valency does not need to change 
  when the clones keep taking steps immediately after their processes.

Sometimes, when considering process-clone pairs, none of which are split, 
  we may refer to them as processes, i.e. we may talk about a process taking steps or returning a value. 
As mentioned earlier, it is assumed that as long as the pair is not split, 
  the clone always follows and takes the same steps right after the process. 
Hence, in this context, a process taking a step means a pair taking a step.   

Now we are ready to prove the main result.
\begin{theorem}
\label{thm:linclone}
In the system of anonymous processes, 
  consider any correct consensus algorithm satisfying nondeterministic solo termination,
  with the property that every execution uses at most $m$ registers.
For each $r$ with $0 \leq r \leq m$, there exists a set $U$ containing $5m+6+2r$ process-clone pairs  
  such that a configuration $C_r$ is reachable through an execution $E_r$ 
  by processes and clones in $U$ with the following properties:
\begin{itemize}[noitemsep, nolistsep]
\item[1.] There exists a set $R$ of $r$ registers, that can be partitioned in two disjoint subsets $R = R_s \cup R_c$, where:
\begin{itemize}[noitemsep, nolistsep]
\item $R_s$ consists of all registers in the system that each have one fresh split pair on them, 
  last written by some process whose clone has not yet performed the write and is covering the register.
\item $R_c = R \setminus R_s$.
Each register in $R_c$ is covered by an unique pair of both a process and its clone.
\end{itemize}
Thus, each fresh pair is split on a different register in $R_s$,
  and an additional $|R_c|$ pairs are covering the registers in $R_c$.
Let $V$ be the set of these $|R_s| + |R_c| = r$ pairs.
\item[2.] There are at most $r$ stale split pairs in the system,
  that are all split on pairwise different registers from $R$.
Let $L$ be the set of these at most $r$ stale split pairs.
\item[3.] There exist disjoint sets of process-clone pairs that are not split 
  $P, Q \subseteq U \setminus V \setminus L$ with $|P| + |Q| \leq 2m+4$,
  such that an execution in $\lit{Res}(C_r, P)$ returns $0$ and an execution in $\lit{Res}(C_r, Q)$ returns $1$.\footnote{The pairs of processes in $P$ and $Q$ are not split, because all split pairs belong to $V \cup L$ (fresh to $V$ and stale to $L$). Also, the third condition implies that the configuration $C_r$ is bivalent$_{U \setminus V \setminus L}$.} 
\end{itemize}
\end{theorem}
\begin{proof}
The proof is by induction on $r$, with the base case $r = 0$.
Out of the $5m + 6$ processes-clone pairs, 
  half of them start with an input $0$ and half start with an input $1$.
We let $C_0$ be the initial state,
  $P$ be a set of some $m+1$ pairs with input $0$, and
  $Q$ be a set of some $m+1$ pairs with input $1$.
The first two properties are trivially satisfied; also $P \cap Q = \emptyset$ and $|P| + |Q| = 2m+2$.
By~\claimref{clm:reserved} and correctness of consensus,
  there is a reserving execution in $\lit{Res}(C_0, P)$ that decides $0$,
  and a reserving execution in $\lit{Res}(C_0, Q)$ that decides $1$ ($C_0$ is bivalent$_U$).
Observe that the pairs are not split and for the purposes of valency we can just consider the steps of processes.

Now, let us assume induction hypothesis for some $r$, i.e. the existence of $E_r$ and $C_r$ with the required three properties,
  and prove the step for $r+1$ by extending $E_r$ to $E_{r+1}$, resulting in the configuration $C_{r+1}$.
Let $U$, $P$, $Q$, $V$, $L$ and $R = R_s \cup R_c$ all be defined as in the theorem statement for $r$. 
  Our goal is to construct sets $U'$, $P'$, $Q'$, $V'$, $L'$ and $R' = R_s' \cup R_c'$ for $r+1$. 
In $U' \setminus U$ we have two more process-clone pairs available that have not taken steps and 
  can be used to clone an existing process-clone pair.
Let $T$ denote $U \setminus V \setminus L \setminus P \setminus Q$.
Since $|V| = r$, $L \leq r$ and $|P| + |Q| \leq 2m+4$, we have $|T| \geq 3m+2$.

For all but $|R_s| + |L|$ split pairs both processes and clones are in the same states, 
  about to perform the same operations.
By definition, each stale pair in $L$ is split on a different register from $R$.
In the following argument, we extend the execution from $E_r$ to $E_{r+1}$ by steps of processes and clones not in $L$.
This can introduce new stale split pairs and the resulting configuration $C_{r+1}$ 
  may not immediately satisfy the second property.
We will then show how to modify the extension and unite some stale split pairs, 
  such that the resulting configuration satisfies all properties, including the second property with the new $L'$.
	
Let $\alpha \in \lit{Res}(C_r, P)$ be the reserving execution interval that returns $0$,
  and let $\beta \in \lit{Res}(C_r, Q)$ be the reserving execution interval that returns $1$.
Notice that each time a process in $P$ or $Q$ takes a step in $\alpha$ or $\beta$, 
  its clone performs an identical step immediately after.
The execution $E_r \alpha$ ends with a process-clone pair $p \in P$ returning $0$ and 
  the execution $E_r \beta$ ends with a process-clone pair $q \in Q$ returning $1$. 

Each register in $R_c$ was covered by some pair of both a process and its clone in $V$.
Let $\gamma_c$ be a block write to all registers in $R_c$ by only the processes but not the clones 
  of these respective covering pairs: i.e. after each write we get a new fresh split pair.
Consider a configuration $D$ reached from $C_r$ by executing this block write, i.e. a configuration reached after $E_r \gamma_c$.
Assume that $D$ is $1$-valent$_T^{m+1}$, without loss of generality, because it has a valency.
For any execution interval $e$, let us denote by $W(e)$ the set of registers written to during $e$.
Hence, $R_s \cap W(e)$ is the set of registers in $R_s$ that are written-to during $e$.
Each register in $R_s$ is covered by a clone of a split pair whose process has already performed the write and is stopped.
Define $\gamma_s(e)$ as a block write to all registers in $R_s \cap W(e)$ by these trailing clones of the split pairs in $V$: 
  i.e. after each write another clone catches up with its process and a previously split pair is united.
Basically, if we run an execution interval $e$ from $C_r$ that changes contents of some registers in $R_s$,
  we can then clean these changes up by executing $\gamma_s(e)$, 
  which leads to all registers in $R_s$ having the same contents as in $C_r$.

Using a crude covering argument we can first show that
\begin{repclaim}{clm:alphaout}
The execution interval $\alpha$ must contain a write operation outside $R$.
\end{repclaim}
\noindent Based on this we can write $\alpha = \alpha' w_p \alpha''$, 
  where $w_p$ is the write operation to a register $\lit{reg} \not \in R$, performed by some process-clone pair $p \in P$.
    
Looking ahead, when we reach $C_{r+1}$, the new set of registers $R'$ will be $R \cup \{\lit{reg}\}$.
Next, we prove the following claim using an FLP-like case analysis:
\begin{repclaim}{clm:casean}
We can extend execution $E_r$ (i.e. from $C_r$) with an execution interval $e$ and reach a configuration 
  satisfying the first and the third inductive requirements to be $C_{r+1}$ 
  with a properly defined $U'$, $P'$, $Q'$, $V'$ and $R' = R_s' \cup R_c'$,
  and with all process-clone pairs that are not split being in sync.
But the second property is not immediately satisfied.
All stale split pairs from $L$ remain stale and split, but some pairs that were fresh and split on registers 
  in $R_s \cap W(e)$ may have become stale in $C_{r+1}$ 
  (because neither the process nor the clone in the split pair has taken steps while the register was overwritten in $e$). 
However, these are the only possible new stale split pairs in $C_{r+1}$, and they do not belong to the new sets $V' \cup P' \cup Q'$.
\end{repclaim}
The proofs of these claims are provided later. 
In order to finish the proof of the theorem, we need to show how to construct $L'$.
According to the above~\claimref{clm:casean} we can extend the execution to reach the next configuration $C_{r+1}$ 
  satisfying first and third but not the second property about the stale split pairs $L'$.
In $C_r$ we had at most $r$ stale pairs in the system, each split on a different register, and $L$ was the set of these pairs.
But on the way to reaching $C_{r+1}$, we may have introduced new stale pairs in the system.
According to~\claimref{clm:casean} these must be the pairs that were fresh and split on registers in $R_s \cap W(e)$ in $C_r$,
  and whose associated register in $R_s$ has been overwritten during $e$, making them stale in $C_{r+1}$.

The set of all stale pairs in $C_{r+1}$ may not satisfy the requirements imposed for $L'$,
  since there could already have been a stale pair split on a register in $R_s \cap W(e)$ in $L$ (in $C_r$).
Then two stale pairs would be split on this register in $C_{r+1}$, violating the second property.
However, for each such register in $R_s \cap W(e)$, 
  we know a stale pair $\rho \in L$ was split on it in $C_r$,
  and that this register was written-to during extension $W(e)$.
We now modify the extension $e$; we add a single write by the clone of the stale split pair $\rho$
  immediately before a write operation to the same register that was already in $e$.
This way, no pair other than the clone of $\rho$ observes a difference between the two executions, 
  and we will use the configuration reached by the modified execution as $C_{r+1}$.
Because of this indistinguishability, the new $C_{r+1}$ still satisfies other required properties.
Moreover, the pair $\rho$ is not split anymore; it is united since the clone has caught up with its process.

We can do the above modification to the execution for each register in $R_s \cap W(e)$ 
  that previously ended up with two stale split processes in $C_{r+1}$.
Let the modified execution extension be $e'$.
In $e'$, some stale split pairs from $L$ are united, indistinguishably to all other processes and clones, 
  leading to a configuration $C_{r+1}$, that still satisfies the first and third properties,
  and has at most one stale pair split on any register.
We take $L'$ to be the set of stale split pairs.
By construction, all stale pairs are split on registers in $R'$ and no two on the same register, 
  so we do have $|L'| \leq r+1$ as desired.
Hence, we have reached configuration $C_{r+1}$ satisfying all properties and completing the proof.
\end{proof}
\begin{corollary}
In a system of $n$ anonymous processes, 
  any consensus algorithm satisfying non-deterministic solo termination must use $\Omega(n)$ registers.
\end{corollary}
\begin{proof}
\theoremref{thm:linclone} directly implies the $\Omega(n)$ lower bound on the number of registers used in some execution.
If $n$ is the number of anonymous processes and no execution uses more than $m = n/20$ registers,
  by~\theoremref{thm:linclone} we can reach $C_{m}$ for large enough $n$, and we have enough processes $n \geq 10m + 12 + 4m$.
In $C_{m}$ there are $m$ registers in $R$, each of which has either already been written-to ($R_s$)
  or are covered by unique processes ($R_c$).
We could perform a block write to $R_c$ by covering processes from $V$ in $C_{m}$, 
  after which in the resulting execution $m = n/20 = \Omega(n)$ different registers would have been written to.
\end{proof}
We now provide the delayed proofs for the claims.
\begin{claim}
\label{clm:alphaout}
The execution interval $\alpha$ must contain a write operation outside $R$.		 
\end{claim}
\begin{proof}
Assume the contrary.
We know that the execution $E_r \alpha$ decides $0$.
No process or clone that takes a step in $\gamma_c$ or $\gamma_s(\alpha)$ appears in $\alpha$
  (they belong to $V$, disjoint from $P$ and $Q$),  
  and by definition, no process or clone from $T$ takes a step in $\alpha$, $\gamma_c$ or $\gamma_s(\alpha)$. 
Thus, to all processes (and clones) in $T$, 
  the configurations after $E_r \alpha \gamma_s(\alpha) \gamma_c$ and after $E_r \gamma_c$,
  which is configuration $D$, are indistinguishable.
This is because no process (or clone) in $T$ has taken steps, the registers in $R$ contain the same values, 
  and other registers were not touched during $\alpha$, $\gamma_s(\alpha)$ or $\gamma_c$.
Configuration $D$ is $1$-valent$_T^{m+1}$, so some extension from $E_r \alpha \gamma_s(\alpha) \gamma_c$ by 
  an execution interval from $\lit{Res}(D, T)$ decides $1$.
This contradicts the correctness of the algorithm.
\end{proof}
\begin{claim}
\label{clm:casean}
We can extend execution $E_r$ (i.e. from $C_r$) with execution interval $e$ and reach a configuration 
  satisfying the first and the third inductive requirements to be $C_{r+1}$ 
  with properly defined $U'$, $P'$, $Q'$, $V'$ and $R' = R_s' \cup R_c'$,
  and with all process-clone pairs that are not split being in sync.
But the second property is not immediately satisfied.
All stale split pairs from $L$ remain stale and split, but some pairs that were fresh and split on registers 
  in $R_s \cap W(e)$ may have become stale in $C_{r+1}$ 
  (because neither the process nor the clone in the split pair has taken steps while the register has been overwritten in $e$). 
However, these can be the only new stale split pairs in $C_{r+1}$ and they do not belong to the new sets $V' \cup P' \cup Q'$.
\end{claim}
\begin{proof}
The proof works by case analysis. We use the notation from~\theoremref{thm:linclone}.
$T$ does not contain any split pairs, so we can consider valency with respect to processes in $T$.  

\paragraph{Case 1: the configuration reached by the execution $E_r \alpha'$ is $1$-valent$_T^{m+1}$:}
Let $\ell$ be the length of $w_p \alpha''$ and $A_j$ be a prefix of $w_p \alpha''$ of size $j$ for $0 \leq j \leq \ell$.
Here we consider steps of process-clone pairs from $P$, 
  i.e. the difference between $A_j$ and $A_{j+1}$ is the same operation performed twice by a process and its clone,
  and $l$ counts these couples of identical operations,
  as illustrated in~\figureref{fig:case1}.
Pairs in $P$ are not split, as by the inductive hypothesis only pairs in $V \cup L$ are split, 
  and $(V \cup L) \cap P = \emptyset$.

\begin{figure}
\label{fig:case1}
\caption{Case 1}
\resizebox{\textwidth}{!}{\includegraphics{case1.eps}}
\end{figure}

We consider two further subcases.

\paragraph{Case 1.1: for some $0 \leq j \leq \ell$, 
  the configuration reached by the execution $E_r \alpha' A_j$ is bivalent$_T^{m+1}$:}
In this case, we let $C_{r+1}$ be precisely the configuration reached by $E_{r+1} = E_r \alpha' A_j$, 
  and let $R' = R \cup \{\lit{reg}\}$.
Clearly, $|R'| = r+1$.
Moreover, since $C_{r+1}$ is bivalent$_T^{m+1}$ there are two subsets of $m+1$ process-clone pairs in $T$ 
  and respective reserving execution intervals from $C_{r+1}$ that return $0$ and $1$.
Since $|T| \geq 3m+2$, by~\claimref{clm:disjoint}, we can actually find $P' \subseteq T$ and $Q' \subseteq T$, 
  with $P' \cap Q' = \emptyset$ and $|P'| = |Q'| = m+1$, such that
  an execution in $\lit{Res}(C_{r+1}, P')$ returns $0$ and an execution in $\lit{Res}(C_{r+1}, Q')$ returns $1$.
These sets $P'$ and $Q'$ are the new sets of pairs for $C_{r+1}$, and as required $|P'| + |Q'| = 2m+2 \leq 2m+4$.

The new set $R_s'$ will be $R_s \setminus (R_s \cap W(\alpha' A_j))$, i.e. the registers from $R_s$ 
  that have not been touched during our execution extension $\alpha' A_j$ from $C_r$ to $C_{r+1}$.
For each of these registers we still have the same one pair from $V$ split on it, 
  and since this pair was fresh in $C_r$ and the register has not been written to during the extension,
  it is still fresh in $C_{r+1}$ as required.  
There are no other fresh split pairs in $C_{r+1}$:
  no new split pairs were introduced during the extension,
  and the rest of fresh pairs in $C_r$ were split on $R_s \cap \alpha' A_j$.
These pairs are not fresh in $C_{r+1}$, as their register was written to during the extension.

The set $R_c'$ is simply $(R \cup \{\lit{reg}\}) \setminus R_s' = R_c \cup (R_s \cap W(\alpha' A_j)) \cup \{\lit{reg}\}$. 	
We must show that there is a unique process-clone pair covering each of these registers.
For each register in $R_c$, we take the same pair from $V$ that was associated and covering it in $C_r$.
For each register in $(R_s \cap W(\alpha' A_j)) \cup \{\lit{reg}\}$, 
  we find a unique pair from $P$ covering it in $C_{r+1}$.
Since $\alpha$ is a reserving execution interval from $C_r$, 
  all its prefixes including $\alpha' A_j$ are also reserving.
Thus, in $C_{r+1}$ for each register that has been ever written to during $\alpha' A_j$, 
  in particular for registers in $(R_s \cap W(\alpha' A_j)) \cup \{\lit{reg}\}$, 
  we find and associate a unique covering pair in $P$.
Technically, if $j=0$, register $\lit{reg}$ is not yet written, 
  but the next operation in $\alpha$ is $w_p$ by a pair covering $\lit{reg}$.

The set $V'$ contains all $r+1$ pairs that we associated with registers in $R'$, so $V' \subseteq V \cup P$.
The number of stale split pairs may have increased, however.
We still have pairs in $L$ plus $|R_s \cap W(\alpha' A_j)|$ of the previously fresh pairs that are now stale.
We deal with this in~\theoremref{thm:linclone} by modifying the execution extension to unite some stale pairs from $L$, 
  leaving us with a desired subset $L'$ of at most $r+1$ stale pairs.

Finally, as $P', Q' \subseteq T$, and $T$ was disjoint from $V$, $P$, $Q$ and $L$,
  $(P' \cup Q') \cap (V' \cup L') = \emptyset$ as required.

\paragraph{Case 1.2: for every $0 \leq j \leq \ell$, 
  the configuration reached by the execution $E_r \alpha' A_j$ is univalent$_T^{m+1}$:}
By \emph{\textbf{Case 1}} assumption $E_r \alpha' A_0$ is $1$-valent$_T^{m+1}$, 
  so it must be $1$-univalent$_T^{m+1}$.
On the other hand, $\alpha$ ends with a process (and its clone) returning $0$, 
  so the configuration reached by $E_r \alpha' A_{\ell}$ must be $0$-univalent$_T^{m+1}$.
No intermediate configuration is bivalent, so we can find a $0 \leq j < \ell$ such that 
  $E_r \alpha' A_j$ leads to a $1$-univalent$_T^{m+1}$ configuration and 
  $E_r \alpha' A_{j+1}$ leads to a $0$-univalent$_T^{m+1}$ configuration. 
We take $C_{r+1}$ to be the configuration reached after $E_{r+1} = E_r \alpha' A_j$,
  and define sets $R_c'$, $R_s'$, $V'$ and $L'$ the same way as if $C_{r+1}$ was bivalent in \emph{\textbf{Case 1.1}}.
This still works, but we need a new way to find $P'$ and $Q'$ with desired properties.

Let $o$ be the operation by a process-clone pair in $P$ separating $A_j$ and $A_{j+1}$.
$o$ may not be a read, as no process (or clone) in $T$ can distinguish between configurations 
  after $E_r \alpha' A_j$ or $E_r \alpha' A_j o$,
  making it impossible for these configurations to have different univalencies w.r.t. to $T$.
Let $Q'$ be a set of any $m+1$ pairs in $T$. 
By~\claimref{clm:reserved}, $\lit{Res}(C_{r+1}, Q')$ is non-empty and since $C_{r+1}$ is $1$-univalent$_T^{m+1}$,
  all executions in $\lit{Res}(C_{r+1}, Q')$ return $1$.
Recall that $U'$ is $U$ plus two process-clone pairs that have not taken any steps,
  so that $|U'| = |U| + 2 \leq 5m + 6 + 2(r+1)$. 
Let us use these process-clone pairs to create two duplicates of the process-clone pair performing the write operation $o$.
Both of these new pairs will be in the same state as the original pair performing $o$.
These duplicate processes (and their clones) are thus poised on the same register about to perform 
  write operations $o'$ and $o''$ identical to the operation $o$ at configuration $C_{r+1}$. 

Recall that $|T| \geq 3m+2$ and let $F$ be a set of $m+1$ pairs from $T \setminus Q'$.
Let $P'$ be the union of $F$ and the two new duplicate pairs, $|P'| = m+3$ in total.
Let $O$ be the $0$-univalent$_T^{m+1}$ configuration reached after $E_r \alpha' A_{j+1} = E_r \alpha' A_j o = E_{r+1} o$.
Due to $0$-univalency, there is a reserving execution $\xi \in \lit{Res}(O, F)$ that returns $0$.
Having one duplicate pair perform $o'$ from $C_{r+1}$ while another covers the same register with the same operation $o''$, 
  we reach the state indistinguishable from $O$ for all $m+1$ pairs in $F$.
Thus, execution $o'\xi$ from $C_{r+1}$ returns $0$, and by~\claimref{clm:prefix}, $o'\xi \in \lit{Res}(C_{r+1}, P')$.
By construction $|P'| + |Q'| = 2m+4$ and $P' \cap Q' = \emptyset$, as desired and 
  the intersection of $P'$ and $Q'$ with $V'$ or $L'$ is empty like in \emph{\textbf{Case 1.1}}.

The remaining case is when the configuration reached by the execution $E_r \alpha'$ is $0$-univalent$_T^{m+1}$.
It is a bit more involved but utilizes the same general ideas and techniques.
One difference is that we also split pairs.
The construction is given in~\appendixref{app:case2}.
\end{proof}
 