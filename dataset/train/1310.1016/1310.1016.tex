

\documentclass{LMCS}

\def\dOi{11(3:9)2015}
\lmcsheading {\dOi}
{1--28}
{}
{}
{Apr.~25, 2014}
{Sep.~11, 2015}
{}

\ACMCCS{[{\bf Theory of computation}]: Logic---Finite Model Theory}

\usepackage{amsmath}
\usepackage{amssymb}
\usepackage{mathrsfs}  
\usepackage{hyperref}
\usepackage{graphicx}
\usepackage{color}
\usepackage[all]{xy}
\usepackage{subfig}


\newtheorem{definition}{Definition}
\newtheorem{ex}{Example}
\newtheorem*{remarks}{Remarks}
\newtheorem*{remark}{Remark}

\newcommand{\csplogic}{\ensuremath{\{\exists, \wedge \}
    \mbox{-}\mathrm{FO}}}
\newcommand{\qcsplogic}{\ensuremath{\{\exists, \forall, \wedge \}
    \mbox{-}\mathrm{FO}}}
\newcommand{\qcsplogiceq}{\ensuremath{\{\exists, \forall, \wedge,= \} \mbox{-}\mathrm{FO}}}
\newcommand{\mylogic}{\ensuremath{\{\exists, \forall, \wedge,\vee \} \mbox{-}\mathrm{FO}}}
\newcommand{\posFO}{\ensuremath{\{\exists, \forall, \wedge,\vee,= \}
    \mbox{-}\mathrm{FO}}}
\newcommand{\tuple}[1]{\ensuremath{\mathbf{#1}}}
\newcommand{\notsubseteq}{\ensuremath{ \subseteq \hspace{-3mm} / \hspace{2mm} }}
\newcommand{\notmodels}{\ensuremath{ \models \hspace{-3mm} / \hspace{2mm} }}
\newcommand{\rank}[0]{\ensuremath{\mathsf{rank}}}
\newcommand{\depth}[0]{\ensuremath{\mathsf{depth}}}
\newcommand{\seed}[0]{\ensuremath{\mathsf{Seed}}}
\newcommand{\image}[0]{\ensuremath{\mathsf{im}}}
\newcommand{\skolem}[0]{\ensuremath{\mathsf{Skolem}}}
\newcommand{\surhom}{
  \ensuremath{
      \negthinspace 
      \longrightarrow
      \hspace{-5mm} \rightarrow \hspace{1mm}
  }
}
\newcommand{\homm}{
  \ensuremath{
      \negthinspace 
      \longrightarrow
      \negthinspace
  }
}
\newcommand{\nosurhom}{
  \ensuremath{
      \negthinspace 
      \longrightarrow
      \hspace{-5mm} \rightarrow \hspace{-4mm} / \hspace{3mm}
  }
}
\renewcommand{\phi}{\varphi}


\title[Quantified Constraints and Containment Problems]
        {Quantified Constraints and Containment Problems\rsuper*}



\author[H.~Chen]{Hubie Chen\rsuper a}
\address{{\lsuper a}Departamento LSI,
Facultad de Inform\'{a}tica,
Universidad del Pa\'{i}s Vasco,
E-20018 San Sebasti\'{a}n,
Spain
and
IKERBASQUE, Basque Foundation for Science,
E-48011 Bilbao,
Spain}
\email{hubie.chen@ehu.es}

\author[F.~Madelaine]{Florent Madelaine\rsuper b}
\address{{\lsuper b}Clermont Universit{\'e}, Universit{\'e} d'Auvergne, 
LIMOS, BP 10448, F-63000 Clermont-Ferrand, France}
\email{florent.madelaine@udamail.fr}

\author[B.~Martin]{Barnaby Martin\rsuper c}
\address{{\lsuper c}Science and Technology, Middlesex University,
The Burroughs, Hendon, London NW4 4BT, U.K.}
\email{barnabymartin@gmail.com}



\keywords{Quantified Constraints, Finite Model Theory}
\thanks{{\lsuper c}The third author was supported by EPSRC grant EP/L005654/1.}
\titlecomment{{\lsuper*}The main result of this paper has appeared in the extended abstracts \cite{LICS2008} and \cite{QCores}.}







\begin{document}

\maketitle

\begin{abstract}
The quantified constraint satisfaction problem  is the problem to decide whether a positive Horn sentence, involving nothing more than the two quantifiers and conjunction, is true on some fixed structure . We study two containment problems related to the QCSP.

Firstly, we give a combinatorial condition on finite structures  and  that is necessary and sufficient to render .  We prove that , that is all sentences of positive Horn logic true on  are true on , iff there is a surjective homomorphism from  to . This can be seen as improving an old result of Keisler that shows the former equivalent to there being a surjective homomorphism from  to .  We note that this condition is already necessary to guarantee containment of the  restriction of the QCSP, that is --. 
The exponent's bound of  places the decision procedure for the model containment problem in non-deterministic double-exponential time complexity. We further show the exponent's bound  to be close to tight by giving a sequence of structures  together with a fixed , , such that there is a surjective homomorphism from  to  only when .

Secondly, we prove that the entailment problem for positive Horn fragment of first-order logic is decidable. That is, given two sentences  and  of positive Horn,  we give an algorithm that determines whether  is true in all structures (models). Our result is in some sense tight, since we show that the entailment problem for positive first-order logic (\mbox{i.e.} positive Horn plus disjunction) is undecidable.

In the final part of the paper we ponder a notion of Q-core that is some canonical representative among the class of templates that engender the same QCSP. Although the Q-core is not as well-behaved as its better known cousin the core, we demonstrate that it is still a useful notion in the realm of QCSP complexity classifications.
\end{abstract}

\section{Introduction}

The \emph{constraint satisfaction problem} (CSP), much studied in artificial intelligence, is known to admit several equivalent formulations, two of the most popular of which are the model-checking problem for primitive positive first-order sentences and the homomorphism problem (see, e.g., \cite{KolaitisVardiBook05}). The CSP is NP-complete in general, and a great deal of effort has been expended in classifying its complexity for certain restricted cases, in particular where it is parameterised by the \emph{constraint language} (which corresponds to the model in the model-checking problem and the right-hand structure of the homomorphism problem). The problems  thereby obtained, sometimes termed non-uniform \cite{FederVardi}, are conjectured \cite{FederVardi,Bulatov00:algebras} to be always polynomial-time tractable or NP-complete. While this has not been settled in general, a number of partial results are known (e.g. over structures of size  \cite{Schaefer,BulatovJACM} and over smooth digraphs graphs \cite{HellNesetril,barto:1782}). Most of the great advances in these complexity classifications in the past decade have been driven by the algebraic method (\mbox{e.g.} \cite{BulatovJACM,barto:1782}). This involves studying indirectly the relations of a structure through certain operations called polymorphisms that preserve them.

The model containment problem for CSP is the question, for finite structures  and , whether ? It is easy to see that this is equivalent to the question of existence of a homomorphism from  to . Thus the model containment problem for CSP is, essentially, a CSP itself. The condition for  is, therefore, that  and  are homomorphically equivalent. This in turn is equivalent to the condition that  and  share the same, or rather isomorphic, \emph{cores} (where the core of a structure  is a minimal substructure that is homomorphically equivalent to ). The complexity classification problem for  is greatly facilitated by the fact that we may, therefore, assume that  is a core -- i.e. that  is a minimal representative of its equivalence class under the equivalence relation induced by homomorphic equivalence.

A useful generalisation of the CSP involves considering the model-checking problem for positive Horn (pH)  sentences (where we add to primitive positive logic universal quantification). This allows for a broader class of problems, used in artificial intelligence to capture non-monotonic reasoning, whose complexities rise through the polynomial hierarchy up to Pspace. 
When the quantifier prefix is restricted to , with all universal quantifiers preceding existential quantifiers, we obtain the -CSP; when the prefix is unrestricted, we obtain the \emph{quantified constraint satisfaction problem} (QCSP). In general, the -CSP and QCSP  are -complete and Pspace-complete, respectively (for more on these complexity classes, we direct the reader to \cite{ComputationalComplexity}).
As with the CSP, it has become popular to consider the QCSP parameterised by the constraint language, \mbox{i.e.} the model in the model-checking problem, and there is a growing body of results delineating the tractable instances from those that are (probably) intractable \cite{OxfordQuantifiedConstraints,chen-2006}. 

The model containment problem for QCSP takes as input two finite structures  and  and asks whether . Unlike the situation with the CSP, it is not apparent that this containment problem is in any way similar to the QCSP itself. As far as we know, neither a characterisation nor an algorithm for this problem had been known. In this paper we provide both, \mbox{i.e.} we settle the question as to when exactly  by giving a characterising morphism from  to .  It turns out that  exactly when there exists a positive integer  such that  there is a surjective homomorphism from the power structure  to . 

We note that this condition is already necessary to guarantee containment of --. Thus we can say on finite structures that positive Horn collapses to its  fragment. If the sizes of the structures  and  are  and , respectively, then we may take . Thus to decide whether , it suffices to verify whether or not there is a surjective homomorphism from  to . This provides a decision procedure for the model containment problem with non-deterministic double-exponential time complexity.

Keisler had already established in \cite{Keisler65} that a necessary and sufficient condition for countable  and  to satisfy   is a surjective homomorphism from  to . Thus our result can be seen as complementing his with a bound on  in the case that  and  are finite. Keisler's result holds also for infinite structures, and appears as part of a much more general result (with remarkably elegant proof) whose principal object of study is in fact the Horn fragment of first-order logic. His methods are typical of those used in (Classical) Model Theory: a back-and-forth argument making use of the benevolent properties of infinity. In the transfinite case his results rely on the Continuum Hypothesis.  

It is possible to prove our model containment result using the traditional back-and-forth proof method. However, we show that the positive Horn collapse to  is not observable via the back-and-forth because it does not hold on suitably chosen infinite structures. Indeed, if one allows for  to be not quite finite, but still -categorical (while  remains finite) then already the  collapse fails.


We demonstrate that our combinatorial result extends to give the  collapse in the case where  remains finite but  is -categorical, but, as mentioned, show that it can not be extended to the case where  is finite and  is -categorical. 

We demonstrate a near-matching lower bound to the exponent of , by giving a sequence of structures  together with an fixed , , such that there is a surjective homomorphism from  to  only when . This is only a square away from the upper bound  . The simplest structures we use have a growing signature, but we detail a fixed finite signature variant with the same properties.

The Classical Decision Problem, known also as Hilbert's \emph{Entscheidungsproblem}, is the question, given a first-order sentence , whether  is true in all models (is logically valid) or, dually, is true in some model (is satisfiable). It is well-known that this problem is undecidable in general. The entailment problem for first-order logic asks, given sentences  and , whether we have the logical validity of  (denoted ). The equivalence problem is defined similarly, with  substituted by . Both problems are easily seen to be equivalent to the Classical Decision Problem, and are therefore undecidable. A great literature exists on decidable and undecidable cases of the Classical Decision Problem, particularly under restrictions of quantifier prefixes and (arity and number of) relation and function symbols -- see the monograph \cite{CDP}. However, for certain natural fragments of first-order logic, it seems the entailment and equivalence problems are not well-studied. The query containment problem is closely related to the entailment problem, but with truth in all finite models substituted for truth in all models. Query containment problems are fundamental to many aspects of database systems, including query optimisation, determining independence of queries and rewriting queries using views. The query containment problem for first-order logic is also undecidable.

The sentence containment problem for the CSP -- \mbox{a.k.a.} the query containment problem for primitive positive logic -- is the question, given primitive positive sentences  and , whether, for all finite structures ,  implies  (\mbox{i.e.} ). It is easily seen that this problem is decidable and NP-complete, in fact it is an instance of the homomorphism problem (equivalently, the CSP itself). It is also easy to demonstrate, in this case, that the condition of finiteness is irrelevant. That is,  if, and only if, . Thus we have here the decidability and NP-completeness of the entailment problem for primitive positive logic.

The second part of this paper is motivated by the sentence containment problem for the QCSP -- \mbox{a.k.a.} the query containment problem for positive Horn -- that is, given positive Horn sentences  and , to determine whether . In this case it is not clear as to whether this coincides with the condition of entailment, .
Our principle contribution here is to give a decision procedure, with triple-exponential time complexity, for the entailment problem, \mbox{i.e.} the problem to determine, for two pH-sentences  and , whether . Since primitive positive sentences are positive Horn, it follows from the comments of the previous paragraph that this entailment problem is NP-hard.

We will make particular use of a certain canonical model for the sentence , built on the Herbrand universe of terms derived from Skolem functions over a countably infinite set of (new) constants. Herbrand models are commonplace in algorithmic results on logical validity and equivalence in both first-order logic (e.g. \cite{Kozen81}) and logic programming (e.g. \cite{lifschitz01strongly,EiterFTW07,EiterFW07}). However, our method differs significantly from those in the citations.

We also prove that the related entailment problem for positive logic -- even without equality -- is undecidable. Since the difference between positive Horn and positive logic is simply the addition of disjunction, we suggest that our decidability result is somehow tight.

In the last part of the paper, we go on to consider canonical representatives of classes of the equivalence relation  induced by  iff  . The similar relation for pp-logic always has a unique minimal element, the so-called \emph{core}, which is minimal with regard to both cardinality and induced submodel. The consideration of only cores simplifies considerably many CSP classifications, and is tantamount to considering the related polymorphism algebra to be idempotent. The situation for QCSP we show to be somewhat murkier, and we contrast positive Horn in this regard to primitive positive logic, positive equality-free logic and positive logic. We introduce the \emph{Q-cores} and show that, although their behaviour is difficult to pin down, this notion is able to greatly simplify known QCSP classification. We comment finally on the role of idempotency in the algebraic method applied to QCSPs.

This paper is organised as follows. After the preliminaries, we address the QCSP model containment problem in Section~\ref{sec:RHS}. Then we address the positive Horn entailment problem in Section~\ref{sec:LHS}. Finally, we expose the nature of Q-cores in Section~\ref{sec:Q-cores}. We conclude with some final remarks and open questions.

\textbf{Related work}. This paper is an expanded journal version of \cite{LICS2008} together with the most significant parts of \cite{QCores}. In particular, the discussion of the -categorical case is new to this paper.

For a structure , let  be the set of relations positive Horn definable on . Let  be that subset of these relations that are already definable in the  fragment. It follows from \cite{OxfordAndHubie} (see \cite{HubieSIGACT} for details though not a proof) that, for all ,   and   actually coincide. This phenomenon is related to our  collapse.

\section{Global Preliminaries}

Let  be a fixed, finite relational signature. If  is a -structure, then its domain is denoted  and the cardinality of that domain . The stipulation that  contains no constants is purely for technical convenience, as we will wish to consider structures over the expanded signature , where  is a set of (an arbitrary number)  constant symbols. These constants will be used specifically to name elements of the structure that correspond to the evaluation of universal variables. Structures over the expanded  will be written in Fraktur, , whereupon their -reducts become , in the natural way. For  and a -structure , we sometimes write  to indicate .

A \emph{homomorphism} from  to  is a function  that preserves positive relations. That is, for  a -ary relation symbol of , if  then . A homomorphism  must also preserve the constants, i.e. if  then .
Existence of a homomorphism (resp., surjective homomorphism) from  to  is denoted  (resp., ). If both  and , then we describe  and  as \emph{homomorphically equivalent}. If  is a function, and  then we denote by  the image of  under  (i.e. ). When  is omitted, it is considered to be the whole set .

A first-order (fo) sentence  is \emph{positive} if it contains no instances of negation and is \emph{positive Horn} (pH) if, further, it contains no instances of disjunction. Thus, pH involves precisely ,  and  (and , a topic we will return to later in the paper). If we further forbid universal quantifiers then we arrive at a sentence that is \emph{primitive positive} (pp). A priori, pp and pH sentences may contain equalities, though it is easy to see these may be propagated out in all but trivial cases (a topic we will return to later). It is clear that a positive (resp., pH, pp) sentence may be put in the prenex normal form

where  is positive (resp., a conjunction of atoms). If  contains only variables  and  (i.e. one quantifier alternation) then it is said to be ; if  contains only (the existential) variables  then it is said to be .
The \emph{quantified constraint satisfaction problem}  has
\begin{itemize}
\item Input: a positive Horn sentence .
\item Question: does ?
\end{itemize}
If  is restricted to being  (resp., ) then the resulting problem is - (resp., ). We identify a problem with the set of its yes-instances in the obvious way. The \emph{model containment problem} for QCSP takes as input two finite structures  and , and has as its yes-instances those pairs for which  . The model containment problem for CSP and -CSP is defined analogously.



Let  be a prenex sentence of the form , and let  be a finite structure. We will identify a variable tuple  with its underlying set of variables. The \emph{-game on } is a two-player game that pitches
\emph{Universal} (male) against \emph{Existential} (female). 
The game goes as follows.
For  ascending: 
\begin{itemize}
\item for every variable in , Universal chooses an element in
  : i.e. he gives a function ;
  and, 
\item for every variable in , Existential chooses an element in
  : i.e. she gives a function .
\end{itemize}
Existential wins if, and only if, 

where  is the natural pointwise action of  on the coordinates of . 


A \emph{strategy}  for Existential (resp.,  for Universal) tells
her (resp., him) how to play a variable tuple given what has been played
before. That is,  is a function from 

to  and  is a function from 

to  (note that elements of  and  are themselves functions specifying how the game was played on the previous variable tuples). 
A strategy for Existential is \emph{winning} if it beats all possible strategies of Universal. The -game on  is nothing other than a model-checking (Hintikka) game, and it is straightforward to verify that Existential has a winning strategy if, and only if, . In the case where  is a conjunction of atoms, then the winning condition may be recast as their being a homomorphism to  from the structure specified by the atomic conjunction  (this construction will be resurrected in the sequel). 

Given two -structures  and , we define their \emph{direct} (or categorical) \emph{product}  to have domain  and relations
 iff  and . The constant  is the element  s.t.  and . Note that the operator  is associative and commutative, up to isomorphism. Bearing this in mind,  indicates the power structure , from  copies of , where  may be any cardinal.

The \emph{orbit} of an -tuple  of elements in a structure  is the set 

A countably infinite structure is said to be \emph{-categorical} if it is the unique countable model of its first-order theory. It is known by the theorem of Engeler, Ryll-Nardzewsky and Svenonius (see \cite{Hodges}) that a structure that is -categorical has a finite number of orbits of -tuples, for each . This is one of several ways in which an -categorical structure may be said to be ``finite'' in its behaviour.

The \emph{Continuum Hypothesis} (CH) is the assertion that there is no cardinal strictly between  and , i.e. .

\section{The QCSP Model Containment Problem}
\label{sec:RHS}


The following lemma is a restriction of the well-known fact that surjective homomorphisms preserve positive formulae (see, e.g., \cite{Hodges}) -- we sketch the proof for the sake of completeness.
\begin{lem}\label{RHS:lem:pos-pres}
For all  and , if  then .
\end{lem}
\proof[(Sketch)]
If  is a surjective homomorphism, then let  be s.t.  is the identity on . Let  be a pH sentence of the form  . Given a winning strategy  for Existential in the -game on , we build a winning strategy  for her in the -game on . For , let  be a mapping from  to  and let  be a variable of . We set . The result follows from the positivity of 
\qed

\begin{ex}
\label{ex:graphs}
Consider the graphs drawn in Figure~\ref{fig:H1H2K3}. Both  and  have a surjective homomorphism to ; therefore we can derive both  and .
\end{ex}
\begin{figure*}
  \centering
  \resizebox{!}{2cm}{\begin{picture}(0,0)\includegraphics{H1H2K3.pdf}\end{picture}\setlength{\unitlength}{3947sp}\begingroup\makeatletter\ifx\SetFigFont\undefined \gdef\SetFigFont#1#2#3#4#5{\reset@font\fontsize{#1}{#2pt}\fontfamily{#3}\fontseries{#4}\fontshape{#5}\selectfont}\fi\endgroup \begin{picture}(5887,1282)(104,-627)
\put(1891, 12){\makebox(0,0)[lb]{\smash{{\SetFigFont{12}{14.4}{\rmdefault}{\mddefault}{\updefault}{\color[rgb]{0,0,0}}}}}}
\put(119, 12){\makebox(0,0)[lb]{\smash{{\SetFigFont{12}{14.4}{\rmdefault}{\mddefault}{\updefault}{\color[rgb]{0,0,0}}}}}}
\put(4476,-11){\makebox(0,0)[lb]{\smash{{\SetFigFont{12}{14.4}{\rmdefault}{\mddefault}{\updefault}{\color[rgb]{0,0,0}}}}}}
\put(3739,-11){\makebox(0,0)[lb]{\smash{{\SetFigFont{12}{14.4}{\rmdefault}{\mddefault}{\updefault}{\color[rgb]{0,0,0}}}}}}
\end{picture} }  
  \caption{Two graphs and a homomorphic image.}
  \label{fig:H1H2K3}
\end{figure*}
\begin{lem}
\label{lem:product:strategy}
For all  and , .
\end{lem}
\proof
Let  be a pH sentence of the form   . Let  be a winning strategy for Existential in the -game on . The product strategy  for Existential in the -game on  is defined as follows.
For , let  be a mapping from  to  and let  be a variable of . We set 
, 
where  denote the natural projections from  to . That  is a winning strategy for Existential in the -game on  follows from the fact that  is a conjunction of atoms (because every atom must have been true in every one of the  components).
\qed
\begin{remarks}
Lemma~\ref{lem:product:strategy} holds with the same proof for any ordinal exponent .
While Lemma~\ref{RHS:lem:pos-pres} holds for all positive sentences (not just pH), Lemma~\ref{lem:product:strategy} does not hold for positive sentences in general. Consider the directed -path , \mbox{i.e.} the digraph with vertex set  and edge set . Take .  but . In Figure~\ref{fig:H1H2K3},  (homomorphism exhibited in Figure~\ref{fig:k23toh2}), so we can deduce that  and  agree on all pH sentences.
\end{remarks}


\begin{figure*}
  \centering
 \hspace{1cm}  \hspace{1cm}

  \caption{Surjective homomorphism from  to .}
  \label{fig:k23toh2}
\end{figure*}
\noindent We note the following which essentially appears in \cite{Keisler65}.
\begin{thm}[\cite{Keisler65}]
Let  be finite and  of any cardinality. Then  iff .
\end{thm}

\subsection{Combinatorial characterisation}
\label{sec:comb}
\begin{thm}\label{theo:main:result}
  Let  and  be finite -structures. The following are equivalent.
  \begin{enumerate}[label=\Roman*.]
  \item[I.] .
  \item[II.] There exists  s.t. .
  \item[III.] .
  \item[IV.] .
  \end{enumerate}
\end{thm}
\noindent We now set out to prove Theorem~\ref{theo:main:result}, essentially through combinatorial means. Recall the signature , where . We will associate  with , in the natural way. Given a mapping  from  to a structure , we
write  for the -structure induced naturally by  and the
interpretation of the constant symbols given by . Let 
denote the set of  all possible interpretations. We call \emph{Superprodukt} the -structure . Note that this is well-defined since  is associative and commutative, up to isomorphism; and does not produce a clash of notation, as no structure  has been defined. From its definition it is clear to see that  is some kind of enriched power structure of  (indeed, it shares a  domain with ).

There is a natural correspondence between  pH sentences  with  universally quantified variables and
-structures. Recall  is of the form , where  and  is a conjunction of atoms.
From , we build the -structure  in the following way. The elements of  are the variables of , and the relation tuples of  are exactly the facts of the conjunction  (indeed if all the quantifiers of  were switched to being existential then one would obtain the so-called canonical query -- see \cite{KolaitisVardiBook05} -- of the structure ). Finally, the elements  interpret the constants . 
Conversely, given a -structure
, we build the  pH sentence  as follows. The variables of  are the elements of , and the quantifier-free part of  is the conjunction of the facts of . Finally, the variables (whose elements interpreted the constants)  are universally quantified, while all other variables are existentially quantified (to the inside of the universal quantification).  This correspondence is essentially bijective, and is illustrated in the following example. 
\begin{ex}
\label{ex:one}
. 

The sentence , depicted on the left, gives rise to the -structure , depicted on the right. The existential variables and their corresponding elements are not labelled. 

\vspace{0.5cm}
\hspace{2cm}
\begin{picture}(0,0)\includegraphics{Pi2Dir2.pdf}\end{picture}\setlength{\unitlength}{3947sp}\begingroup\makeatletter\ifx\SetFigFont\undefined \gdef\SetFigFont#1#2#3#4#5{\reset@font\fontsize{#1}{#2pt}\fontfamily{#3}\fontseries{#4}\fontshape{#5}\selectfont}\fi\endgroup \begin{picture}(3534,853)(12,-112)
\put(2776,529){\makebox(0,0)[lb]{\smash{{\SetFigFont{12}{14.4}{\rmdefault}{\mddefault}{\updefault}{\color[rgb]{0,0,0}}}}}}
\put(2299,524){\makebox(0,0)[lb]{\smash{{\SetFigFont{12}{14.4}{\rmdefault}{\mddefault}{\updefault}{\color[rgb]{0,0,0}}}}}}
\put(3272,539){\makebox(0,0)[lb]{\smash{{\SetFigFont{12}{14.4}{\rmdefault}{\mddefault}{\updefault}{\color[rgb]{0,0,0}}}}}}
\put(539,523){\makebox(0,0)[lb]{\smash{{\SetFigFont{12}{14.4}{\rmdefault}{\mddefault}{\updefault}{\color[rgb]{0,0,0}}}}}}
\put(1384,515){\makebox(0,0)[lb]{\smash{{\SetFigFont{12}{14.4}{\rmdefault}{\mddefault}{\updefault}{\color[rgb]{0,0,0}}}}}}
\put(1034,532){\makebox(0,0)[lb]{\smash{{\SetFigFont{12}{14.4}{\rmdefault}{\mddefault}{\updefault}{\color[rgb]{0,0,0}}}}}}
\put( 27,358){\makebox(0,0)[lb]{\smash{{\SetFigFont{12}{14.4}{\rmdefault}{\mddefault}{\updefault}{\color[rgb]{0,0,0}}}}}}
\put( 39, 13){\makebox(0,0)[lb]{\smash{{\SetFigFont{12}{14.4}{\rmdefault}{\mddefault}{\updefault}{\color[rgb]{0,0,0}}}}}}
\end{picture} \end{ex}
\begin{lem}
\label{RHS:thm:methodology}
  Let  be of the form , where  is a conjunction of positive atoms and . Let  be 's corresponding
  -structure. The following are equivalent:
  \begin{enumerate}[label=(\roman*)]
  \item 
  \item 
  \end{enumerate}
\end{lem}
\proof
   iff for every mapping  from  to
  , there exists a mapping  from  to  such that
.
From the definition, this is equivalent to there existing a homomorphism from  to , for every  (indeed, when  coincides with , under the natural substitution of the domain  by , then   provides the homomorphism).
 By construction of  as a product of such
  , we have equivalently that there exists a homomorphism
  from  to .
\qed
\proof[(of Theorem~\ref{theo:main:result})]
I  II  is trivial. II  III follows from Lemmas~\ref{RHS:lem:pos-pres} and \ref{lem:product:strategy}.  III  IV is trivial. 

It remains to prove  IV  I. Assume . Consider . Clearly, , by the upward direction of Lemma~\ref{RHS:thm:methodology}. It follows from our assumption that . Let  be any surjective function and  be given according to a winning strategy for Existential in the -game on . But now  gives a surjective homomorphism from  to  which proves our result.
\qed
\begin{ex}\label{ex:bip}
  Consider an undirected bipartite graph with at least one edge
   and  the graph that consists of a single double-edge.
  There is a surjective homomorphism from  to .
  Note also that  (where  stands for disjoint union)
  which we write as . Thus,  (as 
  distributes over ). Hence, if  has no isolated element and
   edges there is a surjective homomorphism from  to . It follows from Theorem~\ref{theo:main:result}) that  .
\end{ex}

\subsection{Complexity}
\label{RHS:Complexity}

Having established a combinatorial characterisation for the QCSP model containment problem, we make the following observation as to its complexity -- as can be seen the twin bounds are far from tight.
\begin{thm}
The model containment problem for QCSP, that is the problem which, given finite structures  and , decides whether  is 1.) in nondeterministic double-exponential time, and 2.) NP-hard (under polynomial-time reductions).
\end{thm}
\proof
Membership of nondeterministic double-exponential time follows from Theorem~\ref{theo:main:result} by building  and guessing a surjective homomorphism to  (which can easily be verified as such in double-exponential time). NP-hardness follows by a reduction from the problem \emph{graph -colourability}, as we will demonstrate.

Let  and  be the (irreflexive) - and -clique, respectively. That is,  is a single loopless vertex and  is the triangle. 
Recall  is the graph . It is well-known that  is -colourable iff . It is easy to see that this is equivalent to . We claim that this is equivalent to the existence of an  s.t. . 
To see this, use first the fact that  is an induced substructure of  (note that for any ,  is a substructure of ) to derive the existence of a homomorphism from  to . This homomorphism can be used in turn to construct a surjective homomorphism from  to .
The result now follows from Theorem~\ref{theo:main:result}. 
\qed

\subsection{Extending Theorem~\ref{theo:main:result}}
\label{sec:omega-cat}

The exponent  of Theorem~\ref{theo:main:result} corresponds to the number of functions . Suppose  and  are distinct functions s.t. there is an automorphism of  mapping  to , then it can be seen that we do not in fact need both of these in the Superprodukt , as the th and
th components are isomorphic. A first upper bound on the
exponent is therefore the number of distinct orbits of -tuples in
, and we will now see how this will enable us to derive a
version of Theorem~\ref{theo:main:result} when  is finite
and  is potentially infinite. The application of
K\"onig's Lemma in the following proof is based on that in
\cite{BodirskyNesetrilJLC}.
\newpage
\begin{thm}\label{theo:omega-cat}
  Let  be -categorical and  a finite -structure. The following are equivalent.
  \begin{enumerate}[label=\Roman*.]
  \item[I.] .
  \item[II.] There exists  s.t. .
  \item[III.] .
  \item[IV.] .
  \end{enumerate}
\end{thm}
\proof
Again: I  II  is trivial. II  III follows from Lemmas~\ref{RHS:lem:pos-pres} and \ref{lem:product:strategy}. III  IV is trivial. 

In Theorem~\ref{theo:main:result} we proved IV  I, but here we prefer IV  II (knowing that II  I is trivial).  Assume IV.  Let the number of distinct orbits of -tuples in  be .





Enumerate the countable domain  by . For , consider the set  of finite partial homomorphisms from  restricted to  to . That such always exist is attested by the fact that the canonical query of restricted to , itself a pp-sentence and true on  , is also by assumption true on . We introduce an equivalence relation on  whereby  if there is an automorphism  of  s.t. . These equivalence classes will form nodes of a forest in which there are edges joining (the equivalence class of) a finite partial homomorphism  on domain  with (the equivalence class of) its extension on domain . By assumption, this forest has nodes representing all finite domains that ultimately cover . It has a  finite number of trees, since there is a bounded number of -types in  (which is -categorical since  is), and each tree is infinite. Further, each tree is finitely branching, since the number of distinct orbits of -tuples in  is finite. It follows from K\"onig's Lemma that there is an infinite branch in each tree that gives a homomorphism from  to .
\qed


\subsection{Limit of the method}
\label{sec:limit}

We will now show that we do not observe the  collapse, that manifests in Theorem~\ref{theo:main:result}, in the general case. A fuller statement of the result of \cite{Keisler65} would be as follows.
\begin{thm}[\cite{Keisler65}]
\label{thm:keisler2}
Assume the CH. Let  be of cardinality  and saturated (or finite), and let  be of cardinality at most . Then  iff .
\end{thm}
\noindent We will establish the following.
\begin{prop}
\label{prop:counterexample}
There is a finite  and -categorical  s.t.  but not .
\end{prop}
\noindent Assuming the infinite part of Theorem~\ref{thm:keisler2} is not vacuous -- i.e. assuming the CH -- the  can be substituted by a saturated elementary extension of cardinality  (see \cite{Marker}). So, assuming the CH, Theorem~\ref{thm:keisler2} is actually untrue with pH substituted by -pH. We will begin by establishing the following.
\begin{lem}
\label{lem:NandQ}
.
\end{lem}
\proof
For  positive Horn, and given a winning strategy  for Existential in the -game on , we will build a winning strategy  for her in the -game on . Let  be given and let  be the least common multiple of the denominators in .
Set  (where  indicates the function of multiplication by  concatenated on ).
\qed
\noindent Let  be the digraph with vertex set  and edge set .
\begin{lem}
\label{lem:lately}
There is a surjective homomorphism  from  to .
\end{lem}
\proof
Indeed, we give a surjective homomorphism from  to . Set   and . Now, for  of the form  and , set .
\qed
\noindent Note that one can even argue there is a surjective homomorphism from  to .
\begin{lem}
\label{lem:DP1andN}
.
\end{lem}
\proof
For  positive Horn, and given a winning strategy  for Existential in the -game on , we will build a winning strategy  for her in the -game on . Let  be given and let  be the maximum of . Let  a surjective homomorphism  from  to  as given by Lemma~\ref{lem:lately}.
Set , where inverse images under  are chosen arbitrarily.
\qed
\proof[Proof of Proposition~\ref{prop:counterexample}]
That  follows from Lemmas~\ref{RHS:lem:pos-pres}, \ref{lem:product:strategy}, \ref{lem:NandQ} and \ref{lem:DP1andN}. However, the positive Horn sentence  holds on the former, but not on the latter (when the edge relation of  is identified with an order).
\qed
\noindent Proposition~\ref{prop:counterexample} may also be seen as limiting the methods used in the previous section.

\subsection{Lower bounds on the exponent}
\label{sec:lowerbound}

The Example~\ref{ex:bip} of bipartite graphs gives us a lower bound on the exponent that we now seek to improve.
Let  be a signature involving  unary relations. Let  be the -structure with domain  where , for each . Let  be the -structure  with domain , where  and , for all . It is clear that  (rainbow elements of the for  where  can map to ) while . For , this gives us a lower bound on the exponent of  where the upper bound would give .

It is not too demanding to construct a finite signature variant of this. Consider the signature  involving a binary relation  and a unary relation . Let  be the directed cycle on  vertices, such that all except one of these vertices is in the relation . Let  have domain  with  and . This has the property that  while . For , this once again gives us a lower bound on the exponent of  where the upper bound would give .

\section{The Entailment Problem}
\label{sec:LHS}

For a simpler exposition, we will assume throughout this section that all pH sentences have strict quantifier alternation, i.e. are of the form

where  is a conjunction of positive atoms. Of course, any pH sentence may be readily converted to an equivalent sentence in this form by the introduction of dummy variables. If  contains any atomic instance  () or  () then we describe  as \emph{degenerate}. It is clear that all models of a degenerate  are of cardinality , and that there is a finite set of normalised -structures over the domain . It follows that, if  is degenerate, we may establish directly whether  by evaluating  over all normalised models of . 

Note that instances of equality in a non-degenerate  may be propagated out by substitution. In order to answer the question  in general, we will wish to build a canonical model of . Henceforth, we will assume that  (but not necessarily ) contains no instances of equality.

\subsubsection*{The Canonical Model}

Let  be a pH sentence of the form 

We consider  to be the \emph{depth} of , denoted . We wish to build a \emph{canonical model} of , and we shall do this via its Skolem normal form. Let  be a set of function symbols, in which the arity of  is . Let 

be the derivative sentence over the signature . Each atom of  induces what we designate a \emph{quantified atom} in . It is well-known that the models of  and  are intimately related, indeed they are identical up to the additional interpretation of the new function symbols of .

If  is a positive integer, let ; if , let . 
Define  to be the set of (closed) \emph{terms} obtained from all compositions of the functions of  on themselves and on the constants of . The \emph{rank} of a term , denoted , is the maximum nesting depth of its function symbols;  is precisely that subset of  of terms of rank . Define  to be the subset of  induced by terms whose rank is .
Note that  is exactly the domain of the \emph{term algebra} of  (see, e.g., \cite{Hodges}).

Considering all instantiations of  by the terms of , we see that  becomes an infinite set of positive atoms , exactly the instantiations of the quantified atoms of . These immediately give rise to a canonical (sometimes known as Herbrand) model of  over the domain   in the standard way (see, e.g., \cite{Hodges}); we denote this model  (\mbox{i.e.} with calligraphic ). Note that  is the positive (Robinson) diagram of . Rather sloppily, we will consider  to be at once a -structure (a bona fide model of ) and a -structure -- this should cause no confusion. By further abuse of nomenclature, we will also continue referring to the elements of  as `terms' and elements of  as `constants'. Let  be the \emph{truncation} (submodel) of  induced by the domain . Note that  is generally not a model of ; however, the following is immediate from the construction.
\begin{fact}
For all , .
\end{fact}
\begin{ex}
\label{ex:main}
Let  contain a single binary relation (i.e. -structures are digraphs).
Let . In this case,\footnote{The reader may notice that  is not in the correct form as it fails to have strict alternation of quantifiers. While the introduction of a dummy existential quantifier (and consequent dummy unary Skolem function in ) would rectify this, it would also make the example rather hard to follow.} 

The quantified atoms of  are 


\noindent The following are depictions of the truncations  and , respectively.





%
 \end{ex}

\subsubsection{A Surjective Diagram Lemma}

Let  be a pH sentence,  its associated set of Skolem functions and  its Skolem normal form. The canonical model , with a countably infinite set of constants, plays a key role in our discourse.
The following is a variant of the Diagram Lemma (see, e.g., \cite{Hodges}).
\begin{lem}
\label{lemma:surjective-diagram}
Let  be a pH sentence. Then, for all countable (not necessarily infinite) structures , if  then there is a surjective homomorphism  s.t. .
\end{lem}
\proof
Let  be an enumeration of the elements of . Let  be the expansion of , over the signature  s.t. the elements  interpret the constants  (if  is finite interpret all remaining constants as, e.g., ). Since  contains no constants, . It follows that there is a further expansion  over the signature , s.t. 

Considering  as a -structure, we now uncover the canonical function . Each  is a syntactic term over . Set  to be the element (which interprets)  in .

The function  is manifestly a homomorphism, since  (actually, it is also unique).  

By once again considering  to be a -structure, we see that  is a surjective homomorphism from  to , s.t. .
\qed

\subsection{Characterisation}
\label{LHS:sec:characterisation}

We are now in a position to derive a model-theoretic characterisation for .
\begin{thm}\label{thm:methodology}
Let  and  be pH sentences. The following are equivalent:
\begin{itemize}
\item[] , i.e.  is logically valid, and
\item[] .
\end{itemize}
\end{thm}
\proof
(Downwards.) Since , we derive , whence, since , we derive .

(Upwards.) Suppose  and, for some , we have . If  is infinite and uncountable, then we apply the  Downward L\"{o}wenheim-Skolem Theorem to find another, countable, model  that agrees with  on all first-order sentences. It follows from Lemma~\ref{lemma:surjective-diagram} that there is a surjective homomorphism . It now follows from  Lemma~\ref{RHS:lem:pos-pres} that  and hence so does .
\qed

\subsubsection{Restricting Universal's Play}

Now let  be a pH sentence of which  is a canonical model. Let  be a pH sentence of the form   . The \emph{-rel-game on } is defined similarly to the -game on , except Universal is now restricted to playing elements of . In this case, Existential has a winning strategy in the -rel-game on  iff  

that is, if  models  with the universal variables relativised to .
\begin{prop}
\label{prop:restrict-universal}
Let  and  be pH sentences, with  a canonical model of . Then,
Existential has a winning strategy in the -game on , i.e. , iff Existential has a winning strategy in the -rel-game on .
\end{prop}
\proof
The forward direction is trivial. The backward direction may be proved in a similar manner to Lemma~\ref{RHS:lem:pos-pres}, given that Lemma~\ref{lemma:surjective-diagram} provides us with a surjective endomorphism  s.t. .
\qed

\subsubsection{Substitution Lemmas}

Given a term  one may consider the various \emph{subterms} of which it is composed. For example, the term   of rank  contains both  and  as subterms. We will talk of a term  as \emph{containing} the constants that are its subterms.
We adopt the notation  to denote the term obtained by replacing, in , all instances of  by  (nota bene  by ).

Consider terms .
Suppose that  holds in the canonical model ; might it always be the case that    holds in ? The answer is no; for example, in the case of digraphs, if , then we have no reason to conclude that , even though the latter corresponds to . However, we can make substitutions subject to certain rules, as the following lemmata attest. 

\begin{lem}[Substitution of terms of distinct rank]
\label{lem:sub-terms}
Let  be a -ary relation symbol of , and consider  s.t.  is distinct from each of , \ldots, . For all terms , if  then .
\end{lem}
\proof
Consider the quantified atom of  that caused  to be in  (via its instantiation in the positive diagram ). It must have been of the form

where  are not required to be disjoint, and each  is either
\begin{itemize}
\item the identity  (in which case  is a singleton) or 
\item some  (in which case  is a -tuple).
\end{itemize}
Since  is distinct in rank from each of , it can be easily seen that all occurrences of  in the  of   must have come from occurrences of  in the instantiations of the variables . It follows that the related instantiation  yields , and the result follows.
\qed

\begin{lem}[Substitution of constants]
\label{lem:sub-consts}
Let  be a -ary relation symbol of , consider  and . If  then .
\end{lem}
\proof
Similar to the previous lemma.
\qed
Let  be some (partial) bijection. For a term , let  be the term obtained by simultaneously switching each constant  for , in the obvious manner.
\begin{lem}[Permutation of constants]
\label{lem:aut}
Let  be a -ary relation symbol of , and consider .
Then,  iff   .
\end{lem}
\proof
It is evident from the construction that, for each permutation ,  has an automorphism that maps each term  to .
\qed

The structure  has the useful property that any finite substructure  has a homomorphism to the truncation . In fact, we are able to derive a stronger property. Call a partial function  \emph{constant-conservative} if, for all ,  contains no constants that are not contained in .
\begin{lem}
\label{lem:AhomT}
For , there is a constant-conservative homomorphism .
\end{lem}
\noindent The general idea of the proof is, in the (worst) case that the terms of  have distinct ranks, that they can still all be mapped to the first  ranks in a way that preserves the rank-order. The proof uses Lemma~\ref{lem:sub-terms} in order to explain what we do when a rank has been `missed out' in . Indeed, when a rank has been missed out, then we may reduce the rank of all higher terms in the rank-order, in an almost arbitrary way, while preserving homomorphism. However, to ensure that the homomorphism is constant-conservative, we reduce rank in a more particular manner. 
\proof
Let  be the elements of  ordered by non-decreasing rank. If the maximal rank is  then there exists some  of rank  s.t. no  is of rank , and  is of the form  for some terms  of which (at least) one is of rank . Suppose one that is of rank  is . Pick any subterm  of  of rank . Let  be that substructure of  derived by substituting  for  in all the terms of . Clearly this substitution is constant-conservative. We claim that the function from  to  induced by this substitution is a homomorphism, whereupon we may iterate the above reasoning until the obtained structure has maximal rank .

(Proof that .) 
Consider the elements  of  and the natural map that takes them to  in . We will demonstrate that this is a homomorphism. Let  be a -ary relation symbol of . 
Suppose , by Lemma~\ref{lem:sub-terms} we have  , whereupon the result follows (since  is an induced substructure of ).

It remains to argue that the iterative procedure we have given terminates, \mbox{i.e.} that eventually we end up in a situation in which the elements of  have ranks . We show why the iteration of our procedure ultimately produces a model in which all terms are of rank .
Let , and suppose  contains a term  of rank .
We claim that after  iterations of our procedure (where ) we must obtain a  s.t. , whereupon convergence of our procedure is implied.
Suppose, as before, that  contains no term of rank , but contains some , of rank  and of the form , s.t.  contains  subterms of rank  (i.e.  are of rank , and ). Either  is s.t.  or  also contains no term of rank , but contains , of rank  s.t.  contains  subterms of rank . The result follows.
\qed

\subsubsection{Restricting Existential's Play}

Proposition~\ref{prop:restrict-universal} tells us that we may consider Universal's play restricted to the set  in the -game on . Now we detail how we may make a certain assumption about Existential's play, without affecting her ability to win.

Let  be pH sentences, with  of the form , and let  be a canonical model of . Define the \emph{-rel-cc-game on } as the -rel-game on  but now restrict Existential to only playing terms  containing constants that Universal has already played (the cc abbreviates constant-conservative). In other words, if Universal has played  for variables , then Existential must play some  for . Legitimate strategies for Existential in this game will be termed \emph{constant-conservative}. Winning strategies for Existential in the -rel-cc-game on  are central to our discourse.

Consider the -rel-game (resp., -rel-cc-game) on the truncation  defined in the obvious way.
\begin{prop}
\label{prop:cc}
Let  be pH sentences, with  of the form 

 The following are equivalent.
  \begin{enumerate}[label=(\roman*)]
\item  Existential has a winning strategy in the -rel-game on .
\item  Existential has a winning strategy in the -rel-cc-game on .
\item Existential has a winning strategy in the -rel-cc-game on .
\item Existential has a winning strategy in the -rel-cc-game on .
\end{enumerate}
\end{prop}
\proof We break the proof into a cyclic system of implications.
\paragraph{}
Consider a \emph{game tree}  for the -rel-game on  under Existential strategy .  is an out-tree, branching on all possible Universal moves over  when Existential plays according to . The branching factor of  from the root to the leaves is alternately  and , and the distance from the root to the leaves is . The nodes at distance  (resp., ) from the root are labelled with Universal's (resp., Existential's) th move. The root is unlabelled. If  is a winning strategy, then when we read off valuations for  on a path, we will always have .

We will modify  inductively from the root to the leaves, in such a way as to ultimately enforce that Existential's moves are constant-conservative while keeping her strategy winning. The property  that we will maintain is that, at distance  from the root, there is no node  labelled by an Existential play  containing a constant  that Universal has not played on the path from the root to . When  this is clearly true; and when  we have that Existential's play was always constant-conservative. 

Suppose the inductive hypothesis  holds at distance  from the root. While there is a node , at distance  from the root, labelled by an Existential play  containing a constant  that Universal has not played on the path from the root to , we undertake the following procedure.
\begin{itemize}
\item Remove all subtrees beyond  whose roots are labelled with Universal plays . 
\item Pick a constant  that has been already played by Universal on the path from the root to , and substitute all terms  labelling nodes in the subtree rooted at  with .
\end{itemize}
It follows from Lemma~\ref{lem:sub-consts} that this modified game tree still represents a winning strategy for Existential, \emph{so long as Universal never plays  beyond node }. 

Now consider all missing subtrees corresponding to Universal plays of  after . These follow Existential plays at nodes , , \ldots,  at distances  beyond . For each , consider what Universal plays for :
\begin{itemize}
\item Pick some next Universal play that is a constant  s.t.  has not appeared on the path from the root to  (such a constant must exist since only a finite number of constants can be mentioned on any path). 
\item Take the bijection  that swaps  and . Duplicate the subtree corresponding to the choice  (i.e. rooted at the node labelled  immediately after ) but reset all the node labels  to . Now reintroduce this subtree as the choice  (immediately after ). 
\end{itemize}
Since neither  nor  is mentioned before , it follows from Lemma~\ref{lem:aut} that this modified game tree still represents a winning strategy for Existential.

An example for case  follows the remainder of the proof.


\paragraph{}
Existential may use the same winning strategy in the -rel-cc-game on  as she used in the -rel-cc-game on . This is because her play is constant-conservative.

\paragraph{}
Consider a winning strategy  in the -rel-cc-game on . We will construct a winning strategy  for her in the -rel-cc-game on .
Recall  are the ordered universal variables of ; there are at most  ways in which they may be, in order, played on to the set . This means that Existential needs at most  elements of  to beat any strategy of Universal. This means that there is a substructure  that contains at most  elements other than those of  s.t. Existential has the winning strategy  in the -rel-cc-game on . Note that .

Let  be a (constant-conservative) homomorphism, as guaranteed by Lemma~\ref{lem:AhomT}. It follows that  suffices.


\paragraph{}
Suppose Existential has a winning strategy  in the -rel-cc-game on , we will construct a (constant-conservative) winning strategy  for her in the -rel-game on . At the th round, Existential has in mind a partial bijection .\enlargethispage{2\baselineskip}

Universal plays first, with some constant  for . Existential sets  (i.e. the partial bijection that maps  to ), and responds with  for . At the th round, Universal plays some  for . If Universal has already played this, then Existential sets ; otherwise Existential sets  (which also equals ). In both cases she responds with 

for .
Since the strategy  is constant-conservative, no new constants are introduced through , and it follows from Lemma~\ref{lem:aut} that the strategy  is winning.
\qed
\begin{remark}
Although the constant-conservative nature of Existential's play is used in the proof of  above, it is only a truly vital component in the proof of . Imagine the play were not constant-conservative in that proof. Universal begins in the -rel-game on  by playing  for , and Existential sets . In the auxiliary -rel-cc-game on , Existential now looks up what she would have played in her winning strategy if Universal had played  for . But, she might have played a response for  that contains more than one constant. Clearly there is now the possibility to overload on constants in the auxiliary game.
\end{remark}

\paragraph{Illustration of the proof of Proposition~\ref{prop:cc}  by example}
Let  be as in Example~\ref{ex:main} and let  

Note that  is essentially a dummy variable in , but that  (unlike ) is in the correct normal form. Note also that  (in fact, ).

The following is part of a game tree  for the -rel-game on  corresponding to a certain winning Existential strategy . Only the branches corresponding to Universal plays of the first three constants  are depicted, and, even then, dashed arrows designate parts of the tree not expanded beyond their destination.



 \medskip

It is easily seen that the strategy  is not constant-conservative, as attested by the boxed play of  for . Below, we illustrate the technique for amending , so as to make it constant-conservative. At this node, the problem arises from Existential playing a term involving , when Universal has not yet played . Two branches beyond this node, corresponding to Universal plays of  for  and , must be removed. And, in this node and any beyond,  must be substituted by  (the only constant thus far played by Universal). The tree so obtained is illustrated below, with the amended nodes highlighted.

 \medskip

\noindent It is now necessary to return the two branches corresponding to Universal plays of  for  and .


 \medskip

\noindent Note that  has not been played on the path that now reads . We may therefore take the permutation that swaps  and  to replace the missing branch at . Neither is  played on the path , and we may take the same permutation to replace the missing branch at .


\subsection{An Algorithm for Entailment}
\label{LHS:sec:alg}

Our decision procedure for the entailment problem makes use of the following fact, which may be proved by induction on .
\begin{fact}
\label{fact2}
If  is a pH sentence of depth , then .
\end{fact}
\proof
Let  and . Clearly,  (as ) and the result follows.
\qed

\begin{thm}
The entailment problem for pH sentences is decidable in triple exponential time.
\end{thm}
\proof
Consider the input sentences  and  of depth  and , respectively. By Theorem~\ref{thm:methodology} and Proposition~\ref{prop:cc}, it suffices to verify whether Existential has a winning strategy in the -rel-cc-game on . The structure  is of size bounded by

where the  denotes exponentiation (with precedence to the right).
We may search through all -tuples that could be played in the -rel-cc-game on , in time  to determine whether Existential has a winning strategy. Noting that

the result follows.
\qed

\subsection{Undecidability of Entailment for Positive (equality-free) fo}
\label{LHS:sec:undecidable}

The \emph{entailment problem for positive fo} (\textsc{EPPFO}) is defined as follows.

We consider also its dual problem, \textsc{Dual-EPPFO}.

These problems are clearly Turing equivalent ( is satisfiable iff it is not the case that  is valid), and undecidability of the latter implies undecidability of the former.

We introduce one further problem, which may be seen as the satisfiability version of the (pure predicate) Classical Decision Problem, \textsc{Sat-CDP}.

It is well-known that this problem is undecidable (see, e.g., \cite{CDP}). We are now in a position to prove the main result of this section.
\begin{thm}
The entailment problem for positive (equality-free) fo-logic, \\ \textsc{EPPFO}, is undecidable.
\end{thm}
\proof
By reduction from the \textsc{Sat-CDP} to the problem \textsc{Dual-EPPFO} defined above. Let  be some input to the \textsc{Sat-CDP}, containing relation symbols , of respective arities . We introduce  new relation symbols , also of respective arities . We will now use these -relations to axiomatise negation. Consider



where each  is an -tuple. Note that  is logically equivalent to . Now note that  is logically equivalent to 

which we designate  (where  is positive). Finally, derive  from  by first propagating all negations to atomic level and then substituting any instances of negated relations  with . It is easy to see that  is satisfiable iff  is satisfiable. Furthermore,  and  are (equality-free) positive, and the result follows.
\qed

\section{Introducing Q-cores}
\label{sec:Q-cores}

\subsection{Canonical representatives and Core-ness}
\label{sec:core-ness}

A \emph{core} can be defined in various way, for example on finite structures one may say it is any structure all of whose endomorphisms are automorphisms. Consider the equivalence relation  for finite structures induced by  iff   (\mbox{i.e.}  and  agree on all pp sentences). It is well-known that every member of each equivalence class of   contains, as an induced substructure, an isomorphic copy of the same core, which is (of course) also a member of that class. The core is thus uniquely minimal in its class with respect to both size and inclusion. Thus, for CSP and primitive positive logic the problem to find a canonical representative of the class induced by  is straightforward (although still NP-hard!). Furthermore, each core  of size  enjoys the property that there is a pp-formula , so that  is an  isomorphic copy of that core, whose evaluation on  induces an isomorphism from  to  (we will paraphrase this by saying \emph{the constants are pp-definable in }). In particular, each element of  is individually pp-definable up to isomorphism.

What of a similar canonical representative for  ? We \textbf{might} try to call a structure  a ``Q-core'' if there is no pH-equivalent  of strictly smaller cardinality. We will discover that this ``Q-core'' would be a more cumbersome beast than its cousin the core; it need not be unique nor sit as an induced substructure of the templates in its class. However, in several cases we shall see in Section~\ref{sec:QcoresUseful} that its behaviour is reasonable and that -- like the core -- it can be very useful in delineating complexity classifications. 

We return to consider the following increasingly stronger fragments of fo logic:
\begin{enumerate}
\item primitive positive ()
\item positive Horn, equality-free ()\footnote{We specifically choose the equality-free version so that these four logics form a chain.}
\item positive equality-free fo (\mylogic); and,
\item positive fo (\posFO)
\end{enumerate}


The erratic behaviour of Q-cores sits in contrast not just to that of
cores, but also that of the \emph{--cores} of~\cite{LICS2011},
which are the canonical representatives of the equivalence classes
associated with positive equality-free logic,  and were instrumental in deriving a full complexity
classification -- a tetrachotomy -- for its associated model-checking problem in \cite{LICS2011}. 
Like cores, they are unique 
and sit as induced substructures in all templates in their
class. Thus, primitive positive logic and positive equality-free logic
behave genially in comparison to their wilder cousin positive Horn. In
fact this manifests on the algebraic side also -- polymorphisms and
surjective hyper-endomorphisms 
are preserved under composition, while surjective polymorphisms are not. 

Continuing to add to our logics, in restoring equality, we might
arrive at positive logic. Two finite structures agree on all sentences
of positive logic iff they are isomorphic -- so here every finite
structure satisfies the ideal of ``core''. 
When computing a/the smallest substructure with the same behaviour with
respect to the four decreasingly weaker logics -- positive logic,
positive equality-free, positive Horn, and primitive positive --
we will obtain potentially structure decreasing in size. In the case
of positive equality-free and primitive positive logic, as pointed
out, these are unique up to isomorphism; and for the --core
and the core, these will be induced substructures. A ``Q-core'' will
necessarily contain the core and be included in the --core.  
This phenomenon is illustrated on Table~\ref{tab:different-cores} and
will serve as our running example.

\begin{table}[h]
  \centering
  \begin{tabular}[m]{|c|c|c|c|}
    \hline
    \posFO& \mylogic& \qcsplogic& \csplogic\\
    \hline
    &&&\\\begin{minipage}[c]{.25\textwidth}
      \centering
      \begin{picture}(0,0)\includegraphics{DifCores4.pdf}\end{picture}\setlength{\unitlength}{3108sp}\begingroup\makeatletter\ifx\SetFigFont\undefined \gdef\SetFigFont#1#2#3#4#5{\reset@font\fontsize{#1}{#2pt}\fontfamily{#3}\fontseries{#4}\fontshape{#5}\selectfont}\fi\endgroup \begin{picture}(1577,1588)(2092,-6554)
\put(2566,-5911){\makebox(0,0)[lb]{\smash{{\SetFigFont{9}{10.8}{\rmdefault}{\mddefault}{\updefault}{\color[rgb]{0,0,0}3}}}}}
\put(2116,-5911){\makebox(0,0)[lb]{\smash{{\SetFigFont{9}{10.8}{\rmdefault}{\mddefault}{\updefault}{\color[rgb]{0,0,0}1}}}}}
\put(2161,-5191){\makebox(0,0)[b]{\smash{{\SetFigFont{9}{10.8}{\rmdefault}{\mddefault}{\updefault}{\color[rgb]{0,0,0}2}}}}}
\put(3061,-5236){\makebox(0,0)[b]{\smash{{\SetFigFont{9}{10.8}{\rmdefault}{\mddefault}{\updefault}{\color[rgb]{0,0,0}5}}}}}
\put(3151,-5911){\makebox(0,0)[b]{\smash{{\SetFigFont{9}{10.8}{\rmdefault}{\mddefault}{\updefault}{\color[rgb]{0,0,0}4}}}}}
\put(3601,-5911){\makebox(0,0)[b]{\smash{{\SetFigFont{9}{10.8}{\rmdefault}{\mddefault}{\updefault}{\color[rgb]{0,0,0}6}}}}}
\put(2836,-6451){\makebox(0,0)[b]{\smash{{\SetFigFont{9}{10.8}{\rmdefault}{\mddefault}{\updefault}{\color[rgb]{0,0,0}0}}}}}
\end{picture}     \end{minipage}
    & 
    \begin{minipage}[c]{.2\textwidth}
      \centering
      \begin{picture}(0,0)\includegraphics{DifCores3.pdf}\end{picture}\setlength{\unitlength}{3108sp}\begingroup\makeatletter\ifx\SetFigFont\undefined \gdef\SetFigFont#1#2#3#4#5{\reset@font\fontsize{#1}{#2pt}\fontfamily{#3}\fontseries{#4}\fontshape{#5}\selectfont}\fi\endgroup \begin{picture}(1127,1588)(2092,-6554)
\put(2566,-5911){\makebox(0,0)[lb]{\smash{{\SetFigFont{9}{10.8}{\rmdefault}{\mddefault}{\updefault}{\color[rgb]{0,0,0}3}}}}}
\put(2116,-5911){\makebox(0,0)[lb]{\smash{{\SetFigFont{9}{10.8}{\rmdefault}{\mddefault}{\updefault}{\color[rgb]{0,0,0}1}}}}}
\put(2161,-5191){\makebox(0,0)[b]{\smash{{\SetFigFont{9}{10.8}{\rmdefault}{\mddefault}{\updefault}{\color[rgb]{0,0,0}2}}}}}
\put(3061,-5236){\makebox(0,0)[b]{\smash{{\SetFigFont{9}{10.8}{\rmdefault}{\mddefault}{\updefault}{\color[rgb]{0,0,0}5}}}}}
\put(3151,-5911){\makebox(0,0)[b]{\smash{{\SetFigFont{9}{10.8}{\rmdefault}{\mddefault}{\updefault}{\color[rgb]{0,0,0}4}}}}}
\put(2836,-6406){\makebox(0,0)[b]{\smash{{\SetFigFont{9}{10.8}{\rmdefault}{\mddefault}{\updefault}{\color[rgb]{0,0,0}0}}}}}
\end{picture}     \end{minipage}
    &
    \begin{minipage}[c]{.2\textwidth}
      \centering
      \begin{picture}(0,0)\includegraphics{DifCores2.pdf}\end{picture}\setlength{\unitlength}{3108sp}\begingroup\makeatletter\ifx\SetFigFont\undefined \gdef\SetFigFont#1#2#3#4#5{\reset@font\fontsize{#1}{#2pt}\fontfamily{#3}\fontseries{#4}\fontshape{#5}\selectfont}\fi\endgroup \begin{picture}(218,1588)(2542,-6554)
\put(2611,-5236){\makebox(0,0)[b]{\smash{{\SetFigFont{9}{10.8}{\rmdefault}{\mddefault}{\updefault}{\color[rgb]{0,0,0}2}}}}}
\put(2611,-6296){\makebox(0,0)[b]{\smash{{\SetFigFont{9}{10.8}{\rmdefault}{\mddefault}{\updefault}{\color[rgb]{0,0,0}0}}}}}
\put(2566,-5911){\makebox(0,0)[lb]{\smash{{\SetFigFont{9}{10.8}{\rmdefault}{\mddefault}{\updefault}{\color[rgb]{0,0,0}1}}}}}
\end{picture}     \end{minipage}
    & 
    \begin{minipage}[c]{.15\textwidth}
            \centering
      \begin{picture}(0,0)\includegraphics{DifCores1.pdf}\end{picture}\setlength{\unitlength}{3108sp}\begingroup\makeatletter\ifx\SetFigFont\undefined \gdef\SetFigFont#1#2#3#4#5{\reset@font\fontsize{#1}{#2pt}\fontfamily{#3}\fontseries{#4}\fontshape{#5}\selectfont}\fi\endgroup \begin{picture}(388,373)(4351,-5926)
\put(4366,-5911){\makebox(0,0)[lb]{\smash{{\SetFigFont{9}{10.8}{\rmdefault}{\mddefault}{\updefault}{\color[rgb]{0,0,0}0}}}}}
\end{picture}     \end{minipage}
    \\
    &&&\ 
  \begin{array}{ccc}
    E^{\mathcal{A}}:=\{(1,1),(2,3),(3,2)\} & R^\mathcal{A}:=\{1,2\} & G^\mathcal{A}:=\{1,3\} \\
    E^{\mathcal{B}}:=\{(1,1),(2,3),(3,2)\} & R^\mathcal{B}:=\{1\} & G^\mathcal{B}:=\{1\}
  \end{array}
  
\begin{array}{ll}
\exists x_1,x_2,\ldots,x_{e-1},x_{e} & E(x_1,x_2) \wedge \ldots \wedge \\
& E(x_{e-1},x_{e}) \wedge E(x_{e},x_{1}) 
\end{array}

\begin{array}{ll}
\forall x \exists y \exists z & E(x,y) \wedge E(y,x) \wedge E(y,z) \wedge \\
& E(z,y) \wedge E(z,x) \wedge E(x,z), 
\end{array}
 \forall x \forall y \exists w\exists z \ E(x,y) \wedge E(y,w) \wedge E(w,z) \wedge E(z,y); 
 is not logically valid, as  models the former but not the latter, but . 

This problem has been registered at \cite{FMT-open-problem-garden}.

\textbf{Q-cores}.
There are two outstanding questions here. Firstly, is the Q-core of a finite structure unique up to isomorphism (when one considers non-induced substructure). Secondly, for every finite  does there exists a finite  so that QCSP and QCSP are polynomial-time equivalent and the constants are (all-at-once) pH-definable in  (up to isomorphism). We know this is false with ``polynomial-time equivalent'' replace by ``equal'', but this indirect method may yet salvage the legitimacy to assume we can deal with idempotent polymorphisms alone.

\section*{Acknowledgements}

We are very grateful to Arnaud Durand for supplying the undecidability proof of the entailment problem for positive (equality-free) fo-logic. We are also grateful to an anonymous referee from the conference version for directing us to the paper of Keisler. Finally, we are grateful to a number of referees of the journal version for their corrections and patience.

\bibliographystyle{acm}
\begin{thebibliography}{10}

\bibitem{barto:1782}
{\sc Barto, L., Kozik, M., and Niven, T.}
\newblock The {CSP} dichotomy holds for digraphs with no sources and no sinks
  (a positive answer to a conjecture of {Bang-Jensen} and {Hell}).
\newblock {\em SIAM Journal on Computing 38}, 5 (2009), 1782--1802.

\bibitem{BodirskyNesetrilJLC}
{\sc Bodirsky, M., and Ne\v{s}et\v{r}il, J.}
\newblock Constraint satisfaction with countable homogeneous templates.
\newblock {\em Journal of Logic and Computation 16}, 3 (2006), 359--373.

\bibitem{CDP}
{\sc B{\"o}rger, E., Gr{\"a}del, E., and Gurevich, Y.}
\newblock {\em The Classical Decision Problem}.
\newblock Springer-Verlag, Berlin, 1997.

\bibitem{OxfordAndHubie}
{\sc B\"{o}rner, F., Bulatov, A., Chen, H., Jeavons, P., and Krokhin, A.}
\newblock The complexity of constraint satisfaction games and {QCSP}.
\newblock {\em Inf. Comput. 207}, 9 (2009), 923--944.

\bibitem{OxfordQuantifiedConstraints}
{\sc B\"orner, F., Krokhin, A., Bulatov, A., and Jeavons, P.}
\newblock Quantified constraints and surjective polymorphisms.
\newblock Tech. Rep. PRG-RR-02-11, Oxford University, 2002.

\bibitem{Bulatov00:algebras}
{\sc Bulatov, A., Krokhin, A., and Jeavons, P.}
\newblock Constraint satisfaction problems and finite algebras.
\newblock In {\em ICALP\/} (2000), vol.~1853 of {\em LNCS}, Springer-Verlag,
  pp.~272--282.

\bibitem{BulatovJACM}
{\sc Bulatov, A.~A.}
\newblock A dichotomy theorem for constraint satisfaction problems on a
  3-element set.
\newblock {\em J. ACM 53}, 1 (2006), 66--120.

\bibitem{HubieSIGACT}
{\sc Chen, H.}
\newblock A rendez-vous of logic, complexity, and algebra.
\newblock {\em ACM SIGACT News\/} (2006).

\bibitem{chen-2006}
{\sc Chen, H.}
\newblock The complexity of quantified constraint satisfaction: Collapsibility,
  sink algebras, and the three-element case.
\newblock {\em SIAM J. Comput. 37}, 5 (2008), 1674--1701.

\bibitem{LICS2008}
{\sc Chen, H., Madelaine, F., and Martin, B.}
\newblock Quantified constraints and containment problems.
\newblock In {\em 23rd Annual IEEE Symposium on Logic in Computer Science\/}
  (2008), pp.~317--328.

\bibitem{EiterFTW07}
{\sc Eiter, T., Fink, M., Tompits, H., and Woltran, S.}
\newblock Complexity results for checking equivalence of stratified logic
  programs.
\newblock In {\em IJCAI\/} (2007), M.~M. Veloso, Ed., pp.~330--335.

\bibitem{EiterFW07}
{\sc Eiter, T., Fink, M., and Woltran, S.}
\newblock Semantical characterizations and complexity of equivalences in answer
  set programming.
\newblock {\em ACM Trans. Comput. Logic 8}, 3 (2007), 17.

\bibitem{FederVardi}
{\sc Feder, T., and Vardi, M.~Y.}
\newblock The computational structure of monotone monadic {SNP} and constraint
  satisfaction: a study through datalog and group theory.
\newblock {\em SIAM J. Comput. 28\/} (1999).

\bibitem{HellNesetril}
{\sc Hell, P., and Ne\v{s}et\v{r}il, J.}
\newblock On the complexity of {H}-coloring.
\newblock {\em J. Combin. Theory Ser. B 48\/} (1990).

\bibitem{Hodges}
{\sc Hodges, W.}
\newblock {\em Model theory}.
\newblock Cambridge University Press, 1993.

\bibitem{Keisler65}
{\sc Keisler, H.~J.}
\newblock Reduced products and horn classes.
\newblock {\em Trans. AMS 117\/} (1965), 307--328.

\bibitem{KolaitisVardiBook05}
{\sc Kolaitis, P.~G., and Vardi, M.~Y.}
\newblock {\em Finite Model Theory and Its Applications}.
\newblock Springer-Verlag, 2005, ch.~A logical Approach to Constraint
  Satisfaction.

\bibitem{Kozen81}
{\sc Kozen, D.}
\newblock Communication: Positive first-order logic is {NP}-complete.
\newblock {\em IBM J. Res. Dev. 25}, 4 (1981), 327--332.

\bibitem{lifschitz01strongly}
{\sc Lifschitz, V., Pearce, D., and Valverde, A.}
\newblock Strongly equivalent logic programs.
\newblock {\em Computational Logic 2}, 4 (2001), 526--541.

\bibitem{LICS2011}
{\sc Madelaine, F., and Martin, B.}
\newblock A tetrachotomy for positive equality-free logic.
\newblock In {\em Proceedings of the 26th Annual IEEE Symposium on Logic in
  Computer Science, LICS 2011\/} (2011), pp.~311--320.

\bibitem{QCores}
{\sc Madelaine, F., and Martin, B.}
\newblock Containment, equivalence and coreness from {CSP} to {QCSP} and
  beyond.
\newblock In {\em Principles and Practice of Constraint Programming - 18th
  International Conference, CP 2012\/} (2012).

\bibitem{Marker}
{\sc Marker, D.}
\newblock {\em Model Theory: An Introduction}.
\newblock Springer, 2002.

\bibitem{QCSPforests}
{\sc Martin, B.}
\newblock {QCSP} on partially reflexive forests.
\newblock In {\em Principles and Practice of Constraint Programming - 17th
  International Conference, CP 2011\/} (2011).

\bibitem{CiE2006}
{\sc Martin, B., and Madelaine, F.}
\newblock Towards a trichotomy for quantified {H}-coloring.
\newblock In {\em 2nd Conf. on Computatibility in Europe, LNCS 3988\/} (2006),
  pp.~342--352.

\bibitem{ComputationalComplexity}
{\sc Papadimitriou, C.}
\newblock {\em Computational Complexity}.
\newblock Addison-Wesley, 1994.

\bibitem{Schaefer}
{\sc Schaefer, T.}
\newblock The complexity of satisfiability problems.
\newblock In {\em STOC\/} (1978).

\bibitem{FMT-open-problem-garden}
{\sc Segoufin, L.}
\newblock Finite entailment of positive horn logic.
\newblock
  http://www.openproblemgarden.org/op/finite\_ satisfiability\_of\_positive\_horn\_logic\_entailment,
  2012.
\newblock In Open Problem Garden, Finite Model Theory.

\end{thebibliography}


\end{document}
