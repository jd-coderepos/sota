\subsection{Hard cases}

  The following lemma gives our general upper bounds for various combinations of temporal operators.
It establishes that the known upper complexity bounds for the case
  where only the propositional operators \AND, \OR, and negation are
  allowed to appear in the formulae still hold for the more general
  cases that we consider. This does not follow trivially, since there
  is no obvious strategy that converts every $B$-formula into a
  formula using only the standard connectives without leading to an
  exponential increase in formula length. The issues here are similar
  to the ``succinctness gap'' between the logics LTL+Past and LTL
  discussed in~\cite{mar04}. The proof of Parts (1) and (2) of the
  following lemma is a variation of the proof for Theorem 3.4 in
  \cite{bhss06}, where, using a similar reduction, an analogous result
  for circuits was proved.



  \begin{lemma}\label{lemma:PSPACE_ub_all}
  Let $B$ be a finite set of Boolean functions. Then the following holds:
  \begin{enumerate}[\em(1)]
   \item If $M \subseteq \set{\F,\G,\U,\S,\X}$, then $\tsat{M,B}$ is in \PSPACE,
   \item if $M \subseteq \set{\F,\G}$, then $\tsat{M,B}$ is in \NP, and
   \item if $M \subseteq \set{\X}$, then $\tsat{M,B}$ is also in \NP.
  \end{enumerate}
  \end{lemma}

  \proof For (1), we will show that $\tsat{M,B} \redlogm \tsat{\set{\U,\S,\X},\set{\wedge,\vee,\neg}},$ and
    for (2), we will show that $\tsat{M,B} \redlogm \tsat{\set{\F},\set{\wedge,\vee,\neg}}.$ The complexity result for these cases then follows from Theorem~\ref{theorem:sicl85}.
    \ifreport
    \else
      The proof for case (3) is omitted and given in \cite{bsssv06}.
    \fi

    The construction for (1) and (2) is nearly identical: Let $\varphi$ be a formula with arbitrary temporal operators and Boolean functions from $B$.
    We recursively transform the formula to a new formula using only the Boolean operators $\wedge$, $\vee$, and $\neg$, and the temporal
    operators \U, \S, and \X\ for the first case
    and the temporal operator \F\ for the second case\ifreport s\fi.
    For this we construct several formulae, which will be connected via conjunction. Let $k$ be the number of subformulae of $\varphi$.
Accordingly let $\varphi_1, \dots, \varphi_k$ be those subformulae with $\varphi=\varphi_1$. Let $x_1,\dots,x_k$ be new
    variables, \ie, distinct from the input variables of $\varphi$. For all $i$ from 1 to $k$ we make the following case
    distinction:

    \begin{enumerate}[$\bullet$]
    \item If $\varphi_i=y$ for a variable $y$, then let $f_i(\varphi) = x_i \leftrightarrow y$.
    \item If $\varphi_i=\X \varphi_j$, then let $f_i(\varphi) = x_i \leftrightarrow \X x_j$.
    \item If $\varphi_i=\F \varphi_j$, then let $f_i(\varphi) = x_i \leftrightarrow \F x_j$.
    \item If $\varphi_i=\G \varphi_j$, then let $f_i(\varphi) = x_i \leftrightarrow \G x_j$.
    \item If $\varphi_i=\varphi_j \U \varphi_\ell$, then let $f_i(\varphi) = x_i \leftrightarrow x_j \U x_\ell$.
    \item If $\varphi_i=\varphi_j \S \varphi_\ell$, then let $f_i(\varphi) = x_i \leftrightarrow x_j \S x_\ell$.
    \item If $\varphi_i=g(\varphi_{i_1},\dots, \varphi_{i_n})$ for some $g \in B$,
          then let $f_i(\varphi) = x_i \leftrightarrow h(x_{i_1},\dots,x_{i_n})$,
          where $h$ is a formula using only $\wedge$, $\vee$, and $\neg$, representing the function $g$.
    \end{enumerate}

    Such a formula $h$ always exists with constant length, because the set $B$ is fixed and does not depend on the input.
    Now let $f(\varphi)=x_1 \wedge \bigwedge_{i=1}^k (\G f_i(\varphi) \wedge \neg (\mathop{true} \S \neg f_i(\varphi)))$ for case (1) and
    $f(\varphi)=x_1 \wedge \bigwedge_{i=1}^k \G f_i(\varphi)$ for case (2). The part $\G f_i(\varphi)$ makes sure that $f_i(\varphi)$ holds in every
    future state of the structure and $\neg (\mathop{true} \S \neg f_i(\varphi)))$ does the same for the past states of the structure.
    Additionally we consider $x \leftrightarrow y$ as a shorthand for $(x \wedge y) \vee (\neg x \wedge \neg y)$.
For case (1) we consider $\F x$ as a shorthand for $\mathop{true} \U x$ and $\G x$
    as a shorthand for $\neg(\mathop{true} \U \neg x)$, and for case (2) we consider $\G x$ as a shorthand for $\neg\F\neg x$. Thus we have that $f(\varphi)$ is from $\LL{\{\U,\S,\X\},\{\wedge,\vee,\neg\}}$ in case (1) and from $\LL{\{\F\},\{\wedge,\vee,\neg\}}$ in case (2).
Furthermore $f$ is computable in logarithmic space, because the length of $f_i$ is polynomial
    and neither $\leftrightarrow$ nor the formulae $h$ occur nested. In order to show that
    $f$ is the reduction we are looking for, we still need to prove that $\varphi$ is satisfiable if and only if
    $f(\varphi)$ is satisfiable. Assume an arbitrary structure $S$, such that
    $S,s_i \vDash f(\varphi)$ for some $s_i$. We first prove by induction on the structure of the formula that $x_i$ holds if and only if $\varphi_i$ holds
    in every state $s$ of $S$ (for (1)) respectively in every state which lies in the future of $s_i$ (for (2)).
    Therefore for (1) let $s$ be an arbitrary state and for (2) let $s$ be an arbitrary state in the future of $s_i$. Thus by construction
    of $f(\varphi)$ the formulae $f_p(\varphi)$ hold at $s$ for all $1 \leq p \leq k$. Then the following holds:

    \begin{enumerate}[$\bullet$]
    \item
      If $\varphi_p=y$ for a variable $y$, then $f_p(\varphi) = x_p \leftrightarrow y$ and trivially $S, s \vDash x_p$
      iff $S, s \vDash y$.
    \item
      If $\varphi_p=\X\varphi_j$, then $f_p(\varphi) = x_p \leftrightarrow \X x_j$. Thus $S,s \vDash x_p$ iff for the
      successor state $s'$ of $s$, we have $S,s' \vDash x_j$. By induction this is equivalent to $S,s' \vDash \varphi_j$ and
      therefore $S,s \vDash \varphi_p$ iff $S,s \vDash x_p$.
    \item
      The cases for the temporal operator $\F$ or $\G$ work analogously.
    \item
      If $\varphi_p = \varphi_j \U \varphi_\ell$, then $f_p(\varphi) = x_p \leftrightarrow x_j \U x_\ell$. Thus $S,s \vDash x_p$
      iff there exists a state $s'$ in the future of $s$, such that $S,s' \vDash x_\ell$ and in all states $s_m$ in between
      (including $s$) $S,s_m \vDash x_j$. By induction this is equivalent to $S,s' \vDash \varphi_\ell$ and for all states
      in between $S,s_m \vDash \varphi_j$ and therefore $S,s \vDash \varphi_p$ iff $S,s \vDash x_p$.
    \item  \ifreport
      If $\varphi_p = \varphi_j \S \varphi_\ell$, then $f_p(\varphi) = x_p \leftrightarrow x_j \S x_\ell$. Thus $S,s \vDash x_p$
      iff there exists a state $s'$ in the past of $s$, such that $S,s' \vDash x_\ell$ and in all states $s_m$ in between
      (including $s$) $S,s_m \vDash x_j$. By induction this is equivalent to $S,s' \vDash \varphi_\ell$ and for all states
      in between $S,s_m \vDash \varphi_j$ and therefore $S,s \vDash \varphi_p$ iff $S,s \vDash x_p$.
     \else The case for the temporal operator \S\ works analogously to \U. \fi
    \item
      If $\varphi_p=g(\varphi_{i_1},\dots, \varphi_{i_n})$, then $f_p(\varphi) = x_p \leftrightarrow h(x_{i_1},\dots,x_{i_n})$,
      where $h$ is a formula using only $\wedge$, $\vee$, and $\neg$, representing the function $g$. Thus
      $S,s \vDash x_p$ iff $S,s \vDash h(x_{i_1},\dots,x_{i_n})$. Let $I$ be the subset of $I^n=\{i_1,\dots,i_n\}$, such that
      $S,s \vDash x_m$ for all $m \in I$ and $S,s \vDash \neg x_m$ for all $m \in I^n \setminus I$.
      By induction $S,s \vDash \varphi_m$ for all $m \in I$ and $S,s \vDash \neg \varphi_m$ for all
      $m \in I^n \setminus I$ and therefore $S,s \vDash h(\varphi_{i_1},\dots,\varphi_{i_n})$. Since
      $h$ represents the function $g$, we have that $S,s \vDash \varphi_p$ iff $S,s \vDash x_p$.
    \end{enumerate}

    Now, assume that $f(\varphi)$ is satisfiable. Then there exists a structure
    $S, s_i \vDash f(\varphi)$ and thus $S, s_i \vDash x_1$. Since in every state $x_j$ holds if and only if $\varphi_j$ holds,
    we have that $S, s_i \vDash \varphi = \varphi_1$. For the other direction, assume that $\varphi$ is satisfiable.
    Then there exists a structure $S,s_i \vDash \varphi=\varphi_1$. Now we can extend $S$ by adding new variables
    $x_1,\dots,x_k$ in such a way, that $x_j$ holds in a state $s$ from $S$ if and only if $\varphi_j$ holds in that state.
    Call this new structure $S'$. Then by construction of $f(\varphi)$, we have $S',s_i \vDash f(\varphi)$, since in every state
    $x_j$ holds if and only if $\varphi_j$ holds.
    \ifreport
      This concludes the proof of the first two cases.

      \bigskip\noindent
    \newcommand{\xdepth}[1]{\mathtext{depth}_{\X}\!\left(#1\right)}We now show (3). For a formula $\varphi$ in which $\X$ is the only temporal operator, let $\xdepth\varphi$ denote the maximal nesting degree of the $\X$-operator in $\varphi,$ which we call the $\X$-depth of $\varphi.$ It is obvious that this number is linear in the length of $\varphi.$ Therefore, to show that the problem can be solved in \NP, it suffices to prove the following:

    \begin{enumerate}[(a)]
	\item Such a formula $\varphi$ is satisfiable if and only if there is a structure $S$ with the sequence $(s_i)_{i\in\mathbb N}$ such that for every $i>\xdepth\varphi,$ every variable in $s_i$ is false, and $S,s_0\models\varphi.$
        \item Given the assignments to the variables in the first $\xdepth\varphi$ states in the structure above, it can be verified in polynomial time if $S,s_0\models\varphi.$
    \end{enumerate}

    These claims immediately imply the complexity result. For the first point, it obviously suffices to show one direction. Therefore, let $S$ be an arbitrary structure with sequence $(s_i)_{i\in\mathbb{N}}$ such that $S,s_0\models\varphi,$ and let $S'$ be the structure with sequence $(s'_i)_{i\in\mathbb N}$ obtained from $S$ as follows: For $i\leq\xdepth\varphi,$ the assignment of the variables in the state $s'_i$ is the same as in $s_i.$ For $i>\xdepth\varphi,$ every variable is false in $s'_i.$ To prove claim (a) above, it suffices to prove that $S',s'_0\models\varphi.$

    To show this, we prove that for every subformula $\psi$ of $\varphi$ and every $i\leq\xdepth\varphi,$ if $\xdepth\psi\leq\xdepth\varphi-i,$ then $S,s_i\models\psi$ if and only if $S',s'_i\models\psi.$ For $i=0$ and $\psi=\varphi,$ this implies the desired result $S',s'_0\models\varphi.$

    We show the claim by induction on the formula $\psi.$ If $\psi$ is a variable, then, by construction, $S',s'_i\models\psi$ if and only if $S,s_i\models\psi,$ since the truth assignments of $s'_i$ and $s_i$ are identical. Now let $\psi$ be of the form $f(\psi_1,\dots,\psi_n)$ for an $n$-ary function $f\in B.$ In this case, it immediately follows that $\xdepth\psi=\max\set{\xdepth{\psi_1},\dots,\xdepth{\psi_n}}.$  Because of the prerequisites, $\xdepth\psi\leq\xdepth\varphi-i,$ and hence we know that for each $j\in\set{1,\dots,n},$ it holds that $\xdepth{\psi_j}\leq\xdepth\varphi-i.$ Therefore, we can apply the induction hypothesis to all of the $\psi_j,$ and we know that $S,s_i\models\psi_j$ if and only if $S',s'_i\models\psi_j.$ This immediately implies that $S,s_i\models\psi$ if and only if $S',s'_i\models\psi,$ since $f$ is a Boolean function.

    Finally, let $\psi$ be of the form $\X\xi$ for some formula $\xi.$ Hence, $\xdepth\psi=\xdepth\xi+1.$ Since $\xdepth\psi\leq\xdepth\varphi-i,$ this implies that $\xdepth\xi\leq\xdepth\varphi-(i+1).$ Hence, we can apply the induction hypothesis, and conclude that $S,s_{i+1}\models\xi$ if and only if $S',s'_{i+1}\models\xi.$ This immediately implies that $S,s_i\models\psi$ if and only if $S',s'_i\models\psi,$ and hence concludes the induction and the proof of claim (a).

    For claim (b), assume that $\varphi$ and the truth assignments for the first $\xdepth\varphi$ states in the structure $S$ are given, where all variables are assumed to be false in all further states. We can now, for each subformula $\psi$ of $\varphi,$ mark those states $s_i$ (for $i\leq\xdepth\varphi$) in which $\psi$ holds. Starting with $j=0,$ consider the subformulae of $\X$-depth $j.$ The question if a formula of $\X$-depth $j$ holds at a given state can easily be decided when this is known for all formulae of lower $\X$-depth. For $j=0,$ this can be decided easily, since the subformulae of $\X$-depth $0$ are exactly the propositional subformulae, and for these, each state can be considered separately. Additionally, observe that in the structure $S,$ all states beyond the first $\xdepth\varphi$ states satisfy exactly the same set of subformulae of $\varphi,$ hence only $\xdepth\varphi+1$ many states need to be considered.\qed
    \fi


  The following two theorems show that the case in which our Boolean operators are able to express the function $x\wedge\overline{y},$
  leads to \PSPACE-complete problems in the same cases as for the full set of Boolean operators. This function
  already played an important role in the classification result from \cite{lew79}, where it also marked the
  ``jump'' in complexity from polynomial time to \NP-complete.

  \begin{theorem}\label{theorem:SAT(G,X|F,X|U,X;S1) PSPACE-h}
    Let $B$ be a finite set of Boolean functions such that $\cS_1\subseteq\clone B$. Then $\tsat{\{\G,\X\},B}$ and $\tsat{\{\F,\X\},B}$ are \PSPACE-complete.
  \end{theorem}

  \proof {Since it is possible to express \F\ using \G\ and negation, Theorem~\ref{theorem:sicl85}
implies that $\tsat{\{\G,\X\},\set{\wedge,\vee,\neg}}$ and $\tsat{\{\F,\X\},\set{\wedge,\vee,\neg}}$ are \PSPACE-hard.
Now, let $\varphi$ be
      a formula in which only temporal operators $\G$ and $\X$, or $\F$ and $\X$, and
the Boolean connectives $\wedge,\vee,$ and $\neg$  appear.
Let $B'=B\cup\set{1}$. The complete structure of Post's lattice~\cite{bcrv03} shows that $\clone{B'}=\cBF$.
      Now we can rewrite $\varphi$ as a $B'$-formula with the
      same temporal operators appearing. Due to Lemma~\ref{lemma:short formulae}, we can express the crucial operators
      $\wedge,\vee,\neg$ with short $B'$-formulae, \ie, formulae in which every relevant variable occurs only once.
      Therefore, this transformation can be performed in polynomial time. Now, in the $B'$-representation of $\varphi$, we
      exchange every occurrence of $1$ with a new variable $t$, and call the result $\varphi'$, which is a $B$-formula. It
      is obvious that $\varphi$ is satisfiable if and only if the $B$-formula $\varphi'\wedge t\wedge\G t$ is. Since
      $B\supseteq\cS_1$, we can express the occurring conjunctions using operators from $B$ (since these are a constant
      number of conjunctions, we do not need to worry about needing long $B$-formulae to express conjunction). This finishes
      the proof for $\tsat{\{\G,\X\},B}$. For the problem $\tsat{\{\F,\X\},B}$, observe that the function
      $g(x,y)=x\wedge\overline{y}$ generates the clone $\cS_1$, and therefore there is some $B$-formula equivalent to $g$.
      Now observe that the formula $t\wedge\overline{\F(t\wedge\overline{\X t})}=g(t,\F(g(t, \X t)))$ is equivalent to $\G t.$ Since this formula is independent of the input formula $\varphi$, this can be computed in polynomial time, and therefore this formula can be used to express $\varphi'\wedge t\wedge\G t$ in the same way as in the first case. Additionally, observe that if the operator $\F$ appears in the original formula $\varphi,$ then a subformula $\F\psi$ can be expressed as $(1\U\psi).$ Hence we conclude from Theorem~\ref{theorem:sicl85}\ref{theorem:sicl85:SAT(F,X|U|U,S,X;BF) PSPACE-c} that $\tsat{\{\U,\X\},\cBF}$ is \PSPACE-complete.\qed
      }




The construction in the proof of Theorem~\ref{theorem:SAT(G,X|F,X|U,X;S1) PSPACE-h} does not seem to be applicable to the languages
with \U\ and/or \S, as it requires a way to express $\G t$ using these operators. Hence, proving the desired completeness result
requires significantly more work.
Note that the case where $B$ contains the usual operators \AND, \OR, and negation, has already been proved in~\cite{mar04}. Our construction shows that hardness already holds for a class of propositional operators with less expressive power.



\begin{theorem}\label{theorem:SAT(S;BF) SAT(S|U;S1) PSPACE-h}
Let $B$ be a finite set of Boolean functions with $\cS_1 \subseteq \clone B$.
        Then $\tsat{\{\S\},B}$ and $\tsat{\{\U\},B}$ are \PSPACE-complete.
\end{theorem}


  \proof Since membership for \PSPACE\ is shown in Lemma~\ref{lemma:PSPACE_ub_all} we only need to show hardness. To do this, we give a reduction from $\mathtext{QBF}$. The main idea is to construct a temporal $B$-formula that requires satisfying models to simulate, in a linear structure, the quantifier evaluation tree of a quantified Boolean formula. Once we have ensured that models for the formula in fact are of this structure, we can prove that the quantified formula evaluation problem reduces to $\tsat{\{\S\},B}$.

    First we prove an auxiliary proposition for formulae of a special form which we use as building blocks in the construction. Intuitively the claim states that, given some propositional formulae $\varphi_1,\dots,\varphi_n$ that are pairwise contradictory, we can express that a model has a subsequence of states such that $\varphi_i$ holds in the $i$-th of these states.

    We cannot enforce that the $i$-th state always satisfies the $i$-th formula, since the truth of an LTL-formula using only $\S$ as a temporal operator is invariant under transformations of models that simply repeat a state finitely many times in the sequence.

        \begin{claim}
          Let $\varphi_1,\dots,\varphi_n$ be satisfiable propositional formulae such that $\varphi_i\rightarrow\neg\varphi_j$ is valid for all $i,j\in\{1,\dots,n\}$ with $i\neq j$. Then the formula
          \begin{equation*}
            \begin{split}
              \varphi =\varphi_1
                 \wedge(\varphi_1\S(\varphi_2\S(\dots\S(\varphi_{n-1}\S\varphi_n)\dots)))
                 \wedge((\dots((\varphi_1\S\varphi_2)\S\varphi_3)\S\dots)\S\varphi_n)
\end{split}
          \end{equation*}
          is satisfiable and every structure $S$ that satisfies $\varphi$ in a state $s_m$ fulfills the following property: there exist natural numbers $0=a_0<a_1<\dots<a_n\leq m+1$ such that $m-a_i< j\leq m-a_{i-1}$ implies $S,s_j\vDash\varphi_i$ for every $i\in\{1\dots,n\}$.
        \end{claim}
        \begin{proof}
          Clearly $\varphi$ is satisfiable: since all formulae $\varphi_i$ are satisfiable we can find a structure $S$ such that $S,s_i\vDash\varphi_{n-i}$ for all $i\in\{0,\dots,n-1\}$. One can verify that $S$ satisfies $\varphi$ in $s_{n-1}$.

          Let $S$ be a structure that satisfies $\varphi$ in a state $s_m$. Since $\varphi_i\rightarrow\neg\varphi_j$ is valid for all $i,j\in\{1,\dots,n\}$ with $i\neq j$, in every state only one of the formulae $\varphi_i$ can be satisfied by $S$. Therefore and since $S,s_m\vDash \varphi_1\S(\varphi_2\S(\dots\S(\varphi_{n-1}\S\varphi_n)\dots))$ holds, there are natural numbers $0=a_0\leq a_1\leq\dots\leq a_{n-1}<a_n\leq m+1$ such that $m-a_i< l\leq m-a_{i-1}$ implies $S,s_l\vDash\varphi_i$ for every $i\in\{1\dots,n\}$. Since $S,s_m\vDash\varphi_1$, it holds that $a_1>0$. Because $S,s_m\vDash(\dots((\varphi_1\S\varphi_2)\S\varphi_3)\S\dots)\S\varphi_n$ we conclude that $a_1<\dots <a_{n-1}$, which proves the claim.
        \end{proof}
        Now we give the reduction from $\mathtext{QBF}$, which is \PSPACE-complete due to~\cite{sto77}, to $\tsat{\{\S\},B}$. Let
$\psi=Q_1x_1\dots Q_nx_n\varphi$
        for some propositional $\{\wedge,\vee,\neg\}$-formula $\varphi$ with variables $x_1,\dots,x_n$ and for quantifiers $Q_1,\dots,Q_n\in\{\forall,\exists\}$.

        Let $I_{\forall}=\{p_1,\dots,p_k\}=\{i\mid Q_i=\forall\}$ and
        $I_{\exists}=\{q_1,\dots,q_l\}=\{i\mid Q_i=\exists\}$ such that $p_1<\dots<p_k$ and $q_1<\dots<q_l$.

        We construct a temporal formula $\psi'\in \LL{\{\S\},B}$ such that $\psi$ is valid if and only if $\psi'$ is satisfiable. Let $t_0,\dots,t_n,u_0,\dots,u_n$ be new variables. We start with defining some subformulae using propositional operators from $\set{\neg,\vee,\wedge}$, then we combine them to obtain $\psi'$, and afterwards turn $\psi'$ into a temporal $B$-formula.

        \begin{small}
          ~\par\vspace{-1.3\baselineskip}
          \begin{equation*}
              \alpha= u_0\wedge\overline{t_0}
                  \wedge(u_0\wedge\overline{t_0})\S((
                  \overline{u_0}\wedge\overline{t_0})\S(\overline{u_0}\wedge t_0)))
                  \wedge(((u_0\wedge\overline{t_0})\S(
                  \overline{u_0}\wedge\overline{t_0}))\S(\overline{u_0}\wedge t_0))
          \end{equation*}
          ~\par\vspace{-1.0\baselineskip}
          \begin{minipage}{.52\textwidth}
            \begin{equation*} \begin{split}
              \beta&^1[i] =\\ \ & (u_{i-1}\wedge\overline{t_{i-1}}\wedge u_i\wedge\overline{t_i}\wedge \overline{x_i})
            \S \\
            & \ \ \ \
            ((\overline{u_{i-1}}\wedge\overline{t_{i-1}}\wedge\overline{u_i}\wedge\overline{t_i}\wedge \overline{x_i})
            \S \\
            & \ \ \ \ \ \ \ \
            ((\overline{u_{i-1}}\wedge\overline{t_{i-1}}\wedge\overline{u_i}\wedge t_i\wedge \overline{x_i})
            \S \\
            & \ \ \ \ \ \ \ \ \ \ \ \
            ((\overline{u_{i-1}}\wedge\overline{t_{i-1}}\wedge u_i\wedge\overline{t_i}\wedge x_i)
            \S \\
            & \ \ \ \ \ \ \ \ \ \ \ \ \ \ \ \
            ((\overline{u_{i-1}}\wedge\overline{t_{i-1}}\wedge\overline{u_i}\wedge\overline{t_i}\wedge x_i)
            \S \\
            & \ \ \ \ \ \ \ \ \ \ \ \ \ \ \ \ \ \ \ \
            (\overline{u_{i-1}}\wedge t_{i-1}\wedge\overline{u_i}\wedge t_i\wedge x_i )))))
            \end{split}\end{equation*}
          \end{minipage}
          \hspace{-.12\textwidth}
          \begin{minipage}{.53\textwidth}
            \begin{equation*} \begin{split}
              \beta^2[i] &=\\ (((( &(u_{i-1}\wedge\overline{t_{i-1}}\wedge u_i \wedge\overline{t_i}\wedge \overline{x_i})
            \\
            & \ \ \ \ \S
            (\overline{u_{i-1}}\wedge\overline{t_{i-1}}\wedge\overline{u_i}\wedge\overline{t_i}\wedge \overline{x_i}))
            \\
            & \ \ \ \ \ \ \ \ \S
            (\overline{u_{i-1}}\wedge\overline{t_{i-1}}\wedge\overline{u_i}\wedge t_i\wedge \overline{x_i}))
            \\
            & \ \ \ \ \ \ \ \ \ \ \ \ \S
            (\overline{u_{i-1}}\wedge\overline{t_{i-1}}\wedge u_i\wedge\overline{t_i}\wedge x_i ))
            \\
            & \ \ \ \ \ \ \ \ \ \ \ \ \ \ \ \ \S
            (\overline{u_{i-1}}\wedge\overline{t_{i-1}}\wedge\overline{u_i}\wedge\overline{t_i}\wedge x_i))
            \\
            & \ \ \ \ \ \ \ \ \ \ \ \ \ \ \ \ \ \ \ \ \S
            (\overline{u_{i-1}}\wedge t_{i-1}\wedge\overline{u_i}\wedge t_i\wedge x_i)
            \end{split}\end{equation*}
          \end{minipage}
          ~\par\vspace{.3\baselineskip}
          \begin{minipage}{.48\textwidth}
            \begin{equation*} \begin{split}
              \gamma^1[i] = \ & (u_{i-1}\wedge\overline{t_{i-1}}\wedge u_i\wedge\overline{t_i}\wedge\overline{x_i})
            \S \\
            & \ \ \ \
            ((\overline{u_{i-1}}\wedge\overline{t_{i-1}}\wedge\overline{u_i}\wedge\overline{t_i}\wedge\overline{x_i})
            \S \\
            & \ \ \ \ \ \ \ \
            ((\overline{u_{i-1}}\wedge t_{i-1}\wedge\overline{u_i}\wedge t_i\wedge\overline{x_i})))
            \end{split}\end{equation*}
          \end{minipage}
          \hspace{-.02\textwidth}
          \begin{minipage}{.48\textwidth}
            \begin{equation*} \begin{split}
              \gamma^2[i] = \ & (u_{i-1}\wedge\overline{t_{i-1}}\wedge u_i\wedge\overline{t_i}\wedge x_i)
            \S \\
            & \ \ \ \
            ((\overline{u_{i-1}}\wedge\overline{t_{i-1}}\wedge\overline{u_i}\wedge\overline{t_i}\wedge x_i)
            \S \\
            & \ \ \ \ \ \ \ \
            ((\overline{u_{i-1}}\wedge t_{i-1}\wedge\overline{u_i}\wedge t_i\wedge x_i)))
            \end{split}\end{equation*}
          \end{minipage}
        \par\bigskip
        \end{small}

        The formula $\alpha$ initializes a model as follows: it sets $u_0\overline{t_0}$ in the current state and requires that in the past there is a state with $\overline{u_0}t_0$ and all states in between satisfy $\overline{u_0}\overline{t_0}$. We will use $\beta^1[i]$ and $\beta^2[i]$ for $\forall$-quantified variables $x_i$ to partition the states such that $x_i$ is true in one partition and false in the other. Finally, we need $\gamma^1[i]$ and $\gamma^2[i]$ to set the values for the $\exists$-quantified variables.

        We now define the formula $\psi'$, which constitutes the reduction.

        $$\psi'= \alpha \;
            \wedge \displaystyle \bigwedge_{i\in I_\forall}\!((\beta^1[i] \wedge \beta^2[i])\S\, t_0)\;
            \wedge \displaystyle \bigwedge_{i\in I_\exists}\!((\gamma^1[i] \vee \gamma^2[i])\S\, t_0)\;
            \wedge \,(\varphi \S\, t_0)$$

        The formula $\psi'$ as defined above is specified as a formula using the connectives \AND, \OR, and \NOT. Before proving the correctness of the reduction, we show how $\psi'$ can be rewritten using only the available connectives from $B$. Due to the prerequisites, we know that $\cS_1\subseteq\clone B$. From the complete structure of Post's lattice~\cite{bcrv03}, it follows that $\clone{B\cup\{1\}}=\cBF$. Let $B'$ denote the set $B\cup\{1\}$. Since, due to Lemma~\ref{lemma:short formulae}, conjunction, disjunction, and negation can be written as $B'$-formulae such that every relevant variable appears only once, we can rewrite $\psi'$ into a temporal $B'$-formula with the result growing only polynomially in size (and the transformation can be carried out in polynomial time). Hence we can regard $\psi'$ as a temporal $B'$-formula. Now, since $\clone B\supseteq\cS_1$, and the \AND-function is an element of $\cS_1$, there is a $B$-formula $\mathtext{and}_B(x,y)$ which is equivalent to $x\wedge y$ (but both $x$ and $y$ might occur more than once in $\mathtext{and}_B(x,y)$). Now consider the propositional conjunctions of up to $5$ literals occurring in the subformulae $\beta^j[i]$, $\gamma^j[i]$, and $\alpha$ of $\psi'$, and recall that in the above step, we have rewritten these into formulae that only use connectives from $B$ and the constant $1$. For each such conjunction $\psi_{\mathtext{lit}}$, let $\psi^t_{\mathtext{lit}}$ be the formula obtained from $\psi_{\mathtext{lit}}$ by exchanging each occurrence of the constant $1$ with the new variable $t$. Now the formula $\mathtext{and}_B(t,\psi^t_{\mathtext{lit}})$ is equivalent to $\psi_{\mathtext{lit}}\wedge t$. We can therefore replace all formulae $\psi_{\mathtext{lit}}$ with $\mathtext{and}_B(t,\psi^t_{\mathtext{lit}})$, and obtain a formula which is equivalent to $\psi'$, but additionally forces the new variable $t$ to true in all the affected states. The remaining conjunctions occurring in the subformula $\alpha$ can simply be rewritten using the $\mathtext{and}_B(x,y)$-formula---there is only a constant number of these, hence this rewriting can be done in polynomial time.

        It remains to deal with conjunctions on the outmost level of $\psi'$, i.e., the three conjunctions connecting the different parts of the formula and the conjunctions over all ${i\in I_\forall}$ and ${i\in I_\exists}$. We first re-arrange these conjunctions as a formula which is a binary tree of logarithmic depth. Then each conjunction can be replaced by using the formula $\mathtext{and}_B(x,y)$ defined above. Since the nesting degree of the conjunction (and hence of applications of $\mathtext{and}_B(x,y)$) is only logarithmic, this transformation leads to a formula which is polynomial in the length of the original representation of $\psi'$, and can be carried out in polynomial time.

        The result of these transformations is a temporal $B$-formula which is equivalent to $\psi'$, apart from forcing the newly-introduced variable $t$ to true in all worlds in all models of $\psi'$ that lie in the scope of the relevant temporal operators. In particular, this formula is satisfiability-equivalent to $\psi'$. Hence it suffices to prove that the reduction is correct with respect to $\psi'$, i.e., that $\psi'$ is satisfiable if and only if the original $\mathtext{QBF}$-instance $\psi$ evaluates to true. For this, we first give a characterization of models satisfying $\psi'$, which establishes that models for this formula are indeed ``flat versions of quantifier-trees.''

        Hence assume that $S$ is a structure that satisfies $\psi'$ in a state $s_m$. We prove by induction over $n$ that there are natural numbers $0=a_0<\dots<a_{3(2^k)}\leq m+1$ and for every $q\in I_{\exists}$ a function $\sigma_q:\{0,1\}^{q-1}\rightarrow\{0,1\}$ such that $S$ satisfies the following property: if $m-a_i<j\leq m-a_{i-1}$, then it holds for all $h$ that


        \begin{enumerate}[(1)]
\item\label{xp} $S,s_j\vDash x_{p_h}$  iff  $\lceil\frac{i}{3(2^{k-h})}\rceil$ is even,
          \item\label{xq} $S,s_j\vDash x_{q_h}$  iff  $\sigma_{q_h}(a_1\dots,a_{q_h-1})=1$ where $a_d=1$ if $x_d\in\xi(s_j)$ and $a_d=0$ otherwise,
\item $S,s_j\vDash t_0$      iff  $i=3(2^k)$,
          \item $S,s_j\vDash t_{p_h}$  iff  $i=c\cdot 3(2^{k-h})$ for some $c\in\mathbb{N}$,\
          \item $S,s_j\vDash t_{q_h}$  iff  $S,s_j\vDash t_{p_h-1}$,
          \item $S,s_j\vDash u_0$      iff  $i=1$,
          \item $S,s_j\vDash u_{p_h}$  iff  $i=c\cdot 3(2^{k-h})+1$ for some $c\in\mathbb{N}$,
          \item $S,s_j\vDash u_{q_h}$  iff  $S,s_j\vDash u_{p_h-1}$.
        \end{enumerate}

	Note that due to point \ref{xp} for every possible assignment $\pi$ to $\{x_{p_1},\dots,x_{p_k}\}$ there is a $j\in\{m\!-\!a_{3(2^k)}\!+\!1,\dots,m\}$ such that $S,s_j\vDash x_{p_i}$ if and only if $\pi(x_{p_i})=1$. This is the main feature of the construction. The other variables $t_i$ and $u_i$ are necessary to ensure this condition. Figure~\ref{figure: structure} depicts the buildup of structures resulting from these eight properties. The states shown are necessary in a model for $\psi'$, however there can be more states in between but those have the same assignment as one of the displayed states. The assignment for the $\forall$-quantified variables $x_{p_1},\dots,x_{p_k}$ is given for all states and one can see that all possible assignments are present. Assignments to the $\exists$-quantified variables are not displayed because they can differ from structure to structure. The variables $u_i,t_i$ label all states which set them to true.

\begin{figure}
\begin{center}
  \ifpdf
    \includegraphics{Graphics/grafik_ilka.pdf}
  \else
    \includegraphics{Graphics/grafik_ilka.1}
  \fi
\caption{Structure of models of $\psi'$ in the proof of Theorem~\ref{theorem:SAT(S;BF) SAT(S|U;S1) PSPACE-h}}\label{figure: structure}
\end{center}
\end{figure}



        For $n=0$ it holds that $\psi'=\alpha\wedge(\varphi\S \,t_0)$. Since $\alpha$ satisfies the prerequisites of the claim above, there exist natural numbers $0=a_0<a_1<a_2<a_3\leq m+1$ such that
        \begin{enumerate}[$\bullet$]
\item $m-a_1< j\leq m-a_0$ implies $S,s_j\vDash u_0\wedge\overline{t_0}$
          \item $m-a_2< j\leq m-a_1$ implies $S,s_j\vDash \overline{u_0}\wedge\overline{t_0}$
	  \item $m-a_3< j\leq m-a_2$ implies $S,s_j\vDash \overline{u_0}\wedge t_0$
        \end{enumerate}
        The only occurring variables are $u_0$ and $t_0$ and it is easy to see that the above property of $S$ holds for both.

        For the induction step assume that $n>1$ and the claim holds for $n-1$. There are two cases to consider:

        \begin{EcoEnum}
          \item[\textbf{Case 1:}] $Q_n=\forall$. That means
            \begin{equation*}\begin{split}
              \psi'=& \ \alpha\wedge\bigwedge_{i\in I_{\forall}\setminus\{n\}}((\beta^1[i]\wedge\beta^2[i])\S\, t_0)\wedge \bigwedge_{i\in I_{\exists}}((\gamma^1[i]\vee\gamma^2[i])\S\, t_0)\wedge (\varphi \S\, t_0) \\ &\wedge ((\beta^1[n]\wedge\beta^2[n]) \S\, t_0)
            \end{split}\end{equation*}
            It follows that there are natural numbers $0=a_0<\dots<a_{3(2^{k-1})}\leq m+1$ and for every $q\in I_{\exists}$ a function $\sigma_q:\{0,1\}^{q-1}\rightarrow\{0,1\}$ such that $S$ fulfills the properties of the claim (note that the subformula $(\varphi\S \,t_0)$ is not necessary for our argument). Since $S,s_m\vDash (\beta^1[n]\wedge\beta^2[n]) \S \,t_0$ and for $m-a_{3(2^{k-1})}< j\leq m$ it holds that $S,s_j\vDash t_0$ if and only if $j\leq m-a_{3(2^{k-1})-1}$, we have $S,s_j\vDash \beta^1[n]\wedge\beta^2[n]$ for every $m-a_{3(2^{k-1})-1}< j\leq m$. Let $i=c\cdot3$ for some $c\in\mathbb{N}$, then it holds that $m-a_{i+1}<j\leq m-a_{i}$ implies $S,s_j\vDash u_{n-1}$ which means that for these states $s_j$ it holds that $S,s_j\vDash u_{n-1}\wedge \overline{t_{n-1}}\wedge u_n\wedge \overline{t_n} \wedge x_n$. Due to our proposition there are natural numbers $0=b^i_0<b^i_1<\dots<b^i_6\leq a_i+1$ such that
            \begin{enumerate}[$\bullet$]
\item $a_{i}-b^i_1< j\leq a_{i}-b^i_0$ implies $S,s_j\vDash u_{n-1}\wedge\overline{t_{n-1}}\wedge u_n\wedge\overline{t_n}\wedge \overline{x_n}$
              \item $a_{i}-b^i_2< j\leq a_{i}-b^i_1$ implies $S,s_j\vDash \overline{u_{n-1}}\wedge\overline{t_{n-1}}\wedge\overline{u_n}\wedge\overline{t_n}\wedge \overline{x_n}$
	      \item $a_{i}-b^i_3< j\leq a_{i}-b^i_2$ implies $S,s_j\vDash \overline{u_{n-1}}\wedge\overline{t_{n-1}}\wedge\overline{u_n}\wedge t_n\wedge \overline{x_n}$
              \item $a_{i}-b^i_4< j\leq a_{i}-b^i_3$ implies $S,s_j\vDash \overline{u_{n-1}}\wedge\overline{t_{n-1}}\wedge u_n\wedge\overline{t_n}\wedge x_n$
              \item $a_{i}-b^i_5< j\leq a_{i}-b^i_4$ implies $S,s_j\vDash \overline{u_{n-1}}\wedge\overline{t_{n-1}}\wedge\overline{u_n}\wedge\overline{t_n}\wedge x_n$
              \item $a_{i}-b^i_6< j\leq a_{i}-b^i_5$ implies $S,s_j\vDash \overline{u_{n-1}}\wedge t_{n-1}\wedge\overline{u_n}\wedge t_n\wedge x_n$
            \end{enumerate}
            The nearest state before $s_{m-a_{i}}$ that satisfies $\overline{u_{n-1}}$ is $s_{m-a_{i+1}}$ and the nearest state before $s_{m-a_{i}}$ that satisfies $t_{n-1}$ is $s_{m-a_{i+2}}$, therefore it holds that $b^i_1=a_{i+1}-a_{i}$ and $b^i_5=a_{i+2}-a_{i}$. By denoting $b^i_j+a_i$ with $c_{2i+j}$ we define natural numbers $c_0,\dots,c_{3(2^k)}$ for which it can be verified that they fulfill the claim.

          \item[\textbf{Case 2:}] $Q_n=\exists$. In this case we have
            \begin{equation*}\begin{split}
              \psi'=& \ \alpha\wedge\bigwedge_{i\in I_{\forall}}((\beta^1[i]\wedge\beta^2[i])\S\, t_0)\wedge \bigwedge_{i\in I_{\exists}\setminus\{n\}}((\gamma^1[i]\vee\gamma^2[i])\S\, t_0)\wedge (\varphi \S\, t_0) \\ &\wedge ((\gamma^1[n]\vee\gamma^2[n]) \S\, t_0).
            \end{split}\end{equation*}
          Because of the induction hypothesis there are natural numbers $0=a_0<a_1<\dots<a_{3(2^k)}\leq m+1$ such that the required properties are satisfied. Analogously to the first case $S,s_j\vDash\gamma^1[n]\vee\gamma^2[n]$ is true for every $m-a_{3(2^k)}<j\leq m$. Let $i=c\cdot3$, then for $m-a_{i+1}<j\leq m-a_i$ it holds that $S,s_j\vDash u_{n-1}\wedge\overline{t_{n-1}}\wedge u_n\wedge\overline{t_n}\wedge x_n$ or $S,s_j\vDash u_{n-1}\wedge\overline{t_{n-1}}\wedge u_n\wedge\overline{t_n}\wedge \overline{x_n}$, because $S,s_j\vDash u_{n-1}$. For $m-a_{i+2}<j\leq m-a_{i+1}$ we have that $S,s_j\vDash\overline{u_{n-1}}\wedge\overline{t_{n-1}}\wedge\overline{u_n}\wedge\overline{t_n}\wedge x_n$ or $S,s_j\vDash\overline{u_{i-n}}\wedge\overline{t_{i-n}}\wedge\overline{u_n}\wedge\overline{t_n}\wedge\overline{x_n}$ and for $m-a_{i+3}<j\leq m-a_{i+2}$ it must hold $S,s_j\vDash\overline{u_{n-1}}\wedge t_{n-1}\wedge\overline{u_n}\wedge t_n\wedge x_n$ or $S,s_j\vDash\overline{u_{n-1}}\wedge t_{n-1}\wedge\overline{u_n}\wedge t_n \wedge\overline{x_n}$. If $S,s_{a_i}\vDash\gamma^1[n]$, then in all these states $\overline{x_n}$ is satisfied; if $S,s_{a_i}\vDash\gamma^2[n]$, then $x_n$ is. Therefore with $\sigma_n$ defined by $\sigma_n(d_1,\dots, d_{n-1})=1$ if and only if $S,s_{3(d_12^{n-2}+\dots+d_{n-1}2^0)}\vDash\gamma^2[n]$, the induction is complete, because the binary numbers correspond to the assignments to the $\forall$-quantified variables.
        \end{EcoEnum}
        Note that for a structure that satisfies $\psi'$ with the above notation, $S,s_j\vDash\varphi$ holds for every $m-a_{3(2^k)}<j\leq m$, since $\varphi\S\, t_0$ is a conjunct of $\psi'$.

        Now assume that $\psi'$ is satisfiable in a state $s_m$ of a structure $S$. This is if and only if for every $q\in I_{\exists}$ there is a function $\sigma_{q}:\{0,1\}^{q-1}\rightarrow\{0,1\}$ such that $S$ fulfills the above property. Hence each possible assignment $J$ to the $\forall$-quantified variables $\{x_{p_1},\dots,x_{p_k}\}$ can be extended to an assignment to $\{x_1,\dots,x_n\}$ by $J(x_{q_i})=\sigma_{q_i}(J(x_1),\dots,J(x_{q_i-1}))$ which is equivalent to the validity of $\psi$.
        We can prove \PSPACE-hardness for $\tsat{\{\U\},B}$ with an analogous construction.\qed
