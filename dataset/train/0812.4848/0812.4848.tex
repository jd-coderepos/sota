\subsection{Hard cases}

  The following lemma gives our general upper bounds for various combinations of temporal operators.
It establishes that the known upper complexity bounds for the case
  where only the propositional operators \AND, \OR, and negation are
  allowed to appear in the formulae still hold for the more general
  cases that we consider. This does not follow trivially, since there
  is no obvious strategy that converts every -formula into a
  formula using only the standard connectives without leading to an
  exponential increase in formula length. The issues here are similar
  to the ``succinctness gap'' between the logics LTL+Past and LTL
  discussed in~\cite{mar04}. The proof of Parts (1) and (2) of the
  following lemma is a variation of the proof for Theorem 3.4 in
  \cite{bhss06}, where, using a similar reduction, an analogous result
  for circuits was proved.



  \begin{lemma}\label{lemma:PSPACE_ub_all}
  Let  be a finite set of Boolean functions. Then the following holds:
  \begin{enumerate}[\em(1)]
   \item If , then  is in \PSPACE,
   \item if , then  is in \NP, and
   \item if , then  is also in \NP.
  \end{enumerate}
  \end{lemma}

  \proof For (1), we will show that  and
    for (2), we will show that  The complexity result for these cases then follows from Theorem~\ref{theorem:sicl85}.
    \ifreport
    \else
      The proof for case (3) is omitted and given in \cite{bsssv06}.
    \fi

    The construction for (1) and (2) is nearly identical: Let  be a formula with arbitrary temporal operators and Boolean functions from .
    We recursively transform the formula to a new formula using only the Boolean operators , , and , and the temporal
    operators \U, \S, and \X\ for the first case
    and the temporal operator \F\ for the second case\ifreport s\fi.
    For this we construct several formulae, which will be connected via conjunction. Let  be the number of subformulae of .
Accordingly let  be those subformulae with . Let  be new
    variables, \ie, distinct from the input variables of . For all  from 1 to  we make the following case
    distinction:

    \begin{enumerate}[]
    \item If  for a variable , then let .
    \item If , then let .
    \item If , then let .
    \item If , then let .
    \item If , then let .
    \item If , then let .
    \item If  for some ,
          then let ,
          where  is a formula using only , , and , representing the function .
    \end{enumerate}

    Such a formula  always exists with constant length, because the set  is fixed and does not depend on the input.
    Now let  for case (1) and
     for case (2). The part  makes sure that  holds in every
    future state of the structure and  does the same for the past states of the structure.
    Additionally we consider  as a shorthand for .
For case (1) we consider  as a shorthand for  and 
    as a shorthand for , and for case (2) we consider  as a shorthand for . Thus we have that  is from  in case (1) and from  in case (2).
Furthermore  is computable in logarithmic space, because the length of  is polynomial
    and neither  nor the formulae  occur nested. In order to show that
     is the reduction we are looking for, we still need to prove that  is satisfiable if and only if
     is satisfiable. Assume an arbitrary structure , such that
     for some . We first prove by induction on the structure of the formula that  holds if and only if  holds
    in every state  of  (for (1)) respectively in every state which lies in the future of  (for (2)).
    Therefore for (1) let  be an arbitrary state and for (2) let  be an arbitrary state in the future of . Thus by construction
    of  the formulae  hold at  for all . Then the following holds:

    \begin{enumerate}[]
    \item
      If  for a variable , then  and trivially 
      iff .
    \item
      If , then . Thus  iff for the
      successor state  of , we have . By induction this is equivalent to  and
      therefore  iff .
    \item
      The cases for the temporal operator  or  work analogously.
    \item
      If , then . Thus 
      iff there exists a state  in the future of , such that  and in all states  in between
      (including ) . By induction this is equivalent to  and for all states
      in between  and therefore  iff .
    \item  \ifreport
      If , then . Thus 
      iff there exists a state  in the past of , such that  and in all states  in between
      (including ) . By induction this is equivalent to  and for all states
      in between  and therefore  iff .
     \else The case for the temporal operator \S\ works analogously to \U. \fi
    \item
      If , then ,
      where  is a formula using only , , and , representing the function . Thus
       iff . Let  be the subset of , such that
       for all  and  for all .
      By induction  for all  and  for all
       and therefore . Since
       represents the function , we have that  iff .
    \end{enumerate}

    Now, assume that  is satisfiable. Then there exists a structure
     and thus . Since in every state  holds if and only if  holds,
    we have that . For the other direction, assume that  is satisfiable.
    Then there exists a structure . Now we can extend  by adding new variables
     in such a way, that  holds in a state  from  if and only if  holds in that state.
    Call this new structure . Then by construction of , we have , since in every state
     holds if and only if  holds.
    \ifreport
      This concludes the proof of the first two cases.

      \bigskip\noindent
    \newcommand{\xdepth}[1]{\mathtext{depth}_{\X}\!\left(#1\right)}We now show (3). For a formula  in which  is the only temporal operator, let  denote the maximal nesting degree of the -operator in  which we call the -depth of  It is obvious that this number is linear in the length of  Therefore, to show that the problem can be solved in \NP, it suffices to prove the following:

    \begin{enumerate}[(a)]
	\item Such a formula  is satisfiable if and only if there is a structure  with the sequence  such that for every  every variable in  is false, and 
        \item Given the assignments to the variables in the first  states in the structure above, it can be verified in polynomial time if 
    \end{enumerate}

    These claims immediately imply the complexity result. For the first point, it obviously suffices to show one direction. Therefore, let  be an arbitrary structure with sequence  such that  and let  be the structure with sequence  obtained from  as follows: For  the assignment of the variables in the state  is the same as in  For  every variable is false in  To prove claim (a) above, it suffices to prove that 

    To show this, we prove that for every subformula  of  and every  if  then  if and only if  For  and  this implies the desired result 

    We show the claim by induction on the formula  If  is a variable, then, by construction,  if and only if  since the truth assignments of  and  are identical. Now let  be of the form  for an -ary function  In this case, it immediately follows that   Because of the prerequisites,  and hence we know that for each  it holds that  Therefore, we can apply the induction hypothesis to all of the  and we know that  if and only if  This immediately implies that  if and only if  since  is a Boolean function.

    Finally, let  be of the form  for some formula  Hence,  Since  this implies that  Hence, we can apply the induction hypothesis, and conclude that  if and only if  This immediately implies that  if and only if  and hence concludes the induction and the proof of claim (a).

    For claim (b), assume that  and the truth assignments for the first  states in the structure  are given, where all variables are assumed to be false in all further states. We can now, for each subformula  of  mark those states  (for ) in which  holds. Starting with  consider the subformulae of -depth  The question if a formula of -depth  holds at a given state can easily be decided when this is known for all formulae of lower -depth. For  this can be decided easily, since the subformulae of -depth  are exactly the propositional subformulae, and for these, each state can be considered separately. Additionally, observe that in the structure  all states beyond the first  states satisfy exactly the same set of subformulae of  hence only  many states need to be considered.\qed
    \fi


  The following two theorems show that the case in which our Boolean operators are able to express the function 
  leads to \PSPACE-complete problems in the same cases as for the full set of Boolean operators. This function
  already played an important role in the classification result from \cite{lew79}, where it also marked the
  ``jump'' in complexity from polynomial time to \NP-complete.

  \begin{theorem}\label{theorem:SAT(G,X|F,X|U,X;S1) PSPACE-h}
    Let  be a finite set of Boolean functions such that . Then  and  are \PSPACE-complete.
  \end{theorem}

  \proof {Since it is possible to express \F\ using \G\ and negation, Theorem~\ref{theorem:sicl85}
implies that  and  are \PSPACE-hard.
Now, let  be
      a formula in which only temporal operators  and , or  and , and
the Boolean connectives  and   appear.
Let . The complete structure of Post's lattice~\cite{bcrv03} shows that .
      Now we can rewrite  as a -formula with the
      same temporal operators appearing. Due to Lemma~\ref{lemma:short formulae}, we can express the crucial operators
       with short -formulae, \ie, formulae in which every relevant variable occurs only once.
      Therefore, this transformation can be performed in polynomial time. Now, in the -representation of , we
      exchange every occurrence of  with a new variable , and call the result , which is a -formula. It
      is obvious that  is satisfiable if and only if the -formula  is. Since
      , we can express the occurring conjunctions using operators from  (since these are a constant
      number of conjunctions, we do not need to worry about needing long -formulae to express conjunction). This finishes
      the proof for . For the problem , observe that the function
       generates the clone , and therefore there is some -formula equivalent to .
      Now observe that the formula  is equivalent to  Since this formula is independent of the input formula , this can be computed in polynomial time, and therefore this formula can be used to express  in the same way as in the first case. Additionally, observe that if the operator  appears in the original formula  then a subformula  can be expressed as  Hence we conclude from Theorem~\ref{theorem:sicl85}\ref{theorem:sicl85:SAT(F,X|U|U,S,X;BF) PSPACE-c} that  is \PSPACE-complete.\qed
      }




The construction in the proof of Theorem~\ref{theorem:SAT(G,X|F,X|U,X;S1) PSPACE-h} does not seem to be applicable to the languages
with \U\ and/or \S, as it requires a way to express  using these operators. Hence, proving the desired completeness result
requires significantly more work.
Note that the case where  contains the usual operators \AND, \OR, and negation, has already been proved in~\cite{mar04}. Our construction shows that hardness already holds for a class of propositional operators with less expressive power.



\begin{theorem}\label{theorem:SAT(S;BF) SAT(S|U;S1) PSPACE-h}
Let  be a finite set of Boolean functions with .
        Then  and  are \PSPACE-complete.
\end{theorem}


  \proof Since membership for \PSPACE\ is shown in Lemma~\ref{lemma:PSPACE_ub_all} we only need to show hardness. To do this, we give a reduction from . The main idea is to construct a temporal -formula that requires satisfying models to simulate, in a linear structure, the quantifier evaluation tree of a quantified Boolean formula. Once we have ensured that models for the formula in fact are of this structure, we can prove that the quantified formula evaluation problem reduces to .

    First we prove an auxiliary proposition for formulae of a special form which we use as building blocks in the construction. Intuitively the claim states that, given some propositional formulae  that are pairwise contradictory, we can express that a model has a subsequence of states such that  holds in the -th of these states.

    We cannot enforce that the -th state always satisfies the -th formula, since the truth of an LTL-formula using only  as a temporal operator is invariant under transformations of models that simply repeat a state finitely many times in the sequence.

        \begin{claim}
          Let  be satisfiable propositional formulae such that  is valid for all  with . Then the formula
          
          is satisfiable and every structure  that satisfies  in a state  fulfills the following property: there exist natural numbers  such that  implies  for every .
        \end{claim}
        \begin{proof}
          Clearly  is satisfiable: since all formulae  are satisfiable we can find a structure  such that  for all . One can verify that  satisfies  in .

          Let  be a structure that satisfies  in a state . Since  is valid for all  with , in every state only one of the formulae  can be satisfied by . Therefore and since  holds, there are natural numbers  such that  implies  for every . Since , it holds that . Because  we conclude that , which proves the claim.
        \end{proof}
        Now we give the reduction from , which is \PSPACE-complete due to~\cite{sto77}, to . Let

        for some propositional -formula  with variables  and for quantifiers .

        Let  and
         such that  and .

        We construct a temporal formula  such that  is valid if and only if  is satisfiable. Let  be new variables. We start with defining some subformulae using propositional operators from , then we combine them to obtain , and afterwards turn  into a temporal -formula.

        \begin{small}
          ~\par\vspace{-1.3\baselineskip}
          
          ~\par\vspace{-1.0\baselineskip}
          \begin{minipage}{.52\textwidth}
            
          \end{minipage}
          \hspace{-.12\textwidth}
          \begin{minipage}{.53\textwidth}
            
          \end{minipage}
          ~\par\vspace{.3\baselineskip}
          \begin{minipage}{.48\textwidth}
            
          \end{minipage}
          \hspace{-.02\textwidth}
          \begin{minipage}{.48\textwidth}
            
          \end{minipage}
        \par\bigskip
        \end{small}

        The formula  initializes a model as follows: it sets  in the current state and requires that in the past there is a state with  and all states in between satisfy . We will use  and  for -quantified variables  to partition the states such that  is true in one partition and false in the other. Finally, we need  and  to set the values for the -quantified variables.

        We now define the formula , which constitutes the reduction.

        

        The formula  as defined above is specified as a formula using the connectives \AND, \OR, and \NOT. Before proving the correctness of the reduction, we show how  can be rewritten using only the available connectives from . Due to the prerequisites, we know that . From the complete structure of Post's lattice~\cite{bcrv03}, it follows that . Let  denote the set . Since, due to Lemma~\ref{lemma:short formulae}, conjunction, disjunction, and negation can be written as -formulae such that every relevant variable appears only once, we can rewrite  into a temporal -formula with the result growing only polynomially in size (and the transformation can be carried out in polynomial time). Hence we can regard  as a temporal -formula. Now, since , and the \AND-function is an element of , there is a -formula  which is equivalent to  (but both  and  might occur more than once in ). Now consider the propositional conjunctions of up to  literals occurring in the subformulae , , and  of , and recall that in the above step, we have rewritten these into formulae that only use connectives from  and the constant . For each such conjunction , let  be the formula obtained from  by exchanging each occurrence of the constant  with the new variable . Now the formula  is equivalent to . We can therefore replace all formulae  with , and obtain a formula which is equivalent to , but additionally forces the new variable  to true in all the affected states. The remaining conjunctions occurring in the subformula  can simply be rewritten using the -formula---there is only a constant number of these, hence this rewriting can be done in polynomial time.

        It remains to deal with conjunctions on the outmost level of , i.e., the three conjunctions connecting the different parts of the formula and the conjunctions over all  and . We first re-arrange these conjunctions as a formula which is a binary tree of logarithmic depth. Then each conjunction can be replaced by using the formula  defined above. Since the nesting degree of the conjunction (and hence of applications of ) is only logarithmic, this transformation leads to a formula which is polynomial in the length of the original representation of , and can be carried out in polynomial time.

        The result of these transformations is a temporal -formula which is equivalent to , apart from forcing the newly-introduced variable  to true in all worlds in all models of  that lie in the scope of the relevant temporal operators. In particular, this formula is satisfiability-equivalent to . Hence it suffices to prove that the reduction is correct with respect to , i.e., that  is satisfiable if and only if the original -instance  evaluates to true. For this, we first give a characterization of models satisfying , which establishes that models for this formula are indeed ``flat versions of quantifier-trees.''

        Hence assume that  is a structure that satisfies  in a state . We prove by induction over  that there are natural numbers  and for every  a function  such that  satisfies the following property: if , then it holds for all  that


        \begin{enumerate}[(1)]
\item\label{xp}   iff   is even,
          \item\label{xq}   iff   where  if  and  otherwise,
\item       iff  ,
          \item   iff   for some ,\
          \item   iff  ,
          \item       iff  ,
          \item   iff   for some ,
          \item   iff  .
        \end{enumerate}

	Note that due to point \ref{xp} for every possible assignment  to  there is a  such that  if and only if . This is the main feature of the construction. The other variables  and  are necessary to ensure this condition. Figure~\ref{figure: structure} depicts the buildup of structures resulting from these eight properties. The states shown are necessary in a model for , however there can be more states in between but those have the same assignment as one of the displayed states. The assignment for the -quantified variables  is given for all states and one can see that all possible assignments are present. Assignments to the -quantified variables are not displayed because they can differ from structure to structure. The variables  label all states which set them to true.

\begin{figure}
\begin{center}
  \ifpdf
    \includegraphics{Graphics/grafik_ilka.pdf}
  \else
    \includegraphics{Graphics/grafik_ilka.1}
  \fi
\caption{Structure of models of  in the proof of Theorem~\ref{theorem:SAT(S;BF) SAT(S|U;S1) PSPACE-h}}\label{figure: structure}
\end{center}
\end{figure}



        For  it holds that . Since  satisfies the prerequisites of the claim above, there exist natural numbers  such that
        \begin{enumerate}[]
\item  implies 
          \item  implies 
	  \item  implies 
        \end{enumerate}
        The only occurring variables are  and  and it is easy to see that the above property of  holds for both.

        For the induction step assume that  and the claim holds for . There are two cases to consider:

        \begin{EcoEnum}
          \item[\textbf{Case 1:}] . That means
            
            It follows that there are natural numbers  and for every  a function  such that  fulfills the properties of the claim (note that the subformula  is not necessary for our argument). Since  and for  it holds that  if and only if , we have  for every . Let  for some , then it holds that  implies  which means that for these states  it holds that . Due to our proposition there are natural numbers  such that
            \begin{enumerate}[]
\item  implies 
              \item  implies 
	      \item  implies 
              \item  implies 
              \item  implies 
              \item  implies 
            \end{enumerate}
            The nearest state before  that satisfies  is  and the nearest state before  that satisfies  is , therefore it holds that  and . By denoting  with  we define natural numbers  for which it can be verified that they fulfill the claim.

          \item[\textbf{Case 2:}] . In this case we have
            
          Because of the induction hypothesis there are natural numbers  such that the required properties are satisfied. Analogously to the first case  is true for every . Let , then for  it holds that  or , because . For  we have that  or  and for  it must hold  or . If , then in all these states  is satisfied; if , then  is. Therefore with  defined by  if and only if , the induction is complete, because the binary numbers correspond to the assignments to the -quantified variables.
        \end{EcoEnum}
        Note that for a structure that satisfies  with the above notation,  holds for every , since  is a conjunct of .

        Now assume that  is satisfiable in a state  of a structure . This is if and only if for every  there is a function  such that  fulfills the above property. Hence each possible assignment  to the -quantified variables  can be extended to an assignment to  by  which is equivalent to the validity of .
        We can prove \PSPACE-hardness for  with an analogous construction.\qed
