In the previous section we have seen that the property  
fails dramatically for weakly orthogonal TRSs when collapsing rules are present,
and hence also . 
Now we show that \woTRS{s} without collapsing 
rules are infinitary confluent (),
and as a consequence also have the property .

We adapt the projection of parallel steps in \woTRS{s}
from~\cite[Section~8.8.4.]{terese:2003} to infinite terms.
The basic idea is to orthogonalize the parallel steps,
and then project the orthogonalized steps.
The orthogonalization uses that overlapping redexes 
have the same effect and hence can be replaced by each other.
In case of overlaps we replace the outermost redex by the innermost one.
This is possible for infinitary parallel steps since there
can never be infinite chains of overlapping, nested redexes (see Figure~\ref{fig:chain}).
For a treatment of infinitary developments where such chains can occur,
we refer to Section~\ref{sec:diamond}.
See further~\cite[Proposition~8.8.23]{terese:2003} for orthogonalization 
in the finitary case.

\begin{proposition}
  Let ,  be parallel steps in a \woTRS{}.
  Then there exists an \emph{orthogonalization  of  and },
  that is, a pair of orthogonal parallel steps such that , .
\end{proposition}

\begin{proof}
In case of overlaps between  and , 
then for every overlap we replace the outermost redex by the innermost one
(if there are multiple inner redexes overlapping, then we choose the left-most among the top-most redexes).
If there are two redexes at the same position but with respect to different rules,
then we replace the redex in  with the one in .
See also Figure~\ref{fig:orthogonalization:parallel}.
\begin{figure}[hpt!]
\begin{center}
  \includegraphics{figs/parallel}
\end{center}\vspace{-3ex}
\caption{\textit{Orthogonalization of parallel steps; the arrow indicates replacement.}}
\label{fig:orthogonalization:parallel}
\end{figure}
\end{proof}

\begin{definition}
Let ,  be parallel steps in a \woTRS{}.
The \emph{weakly orthogonal projection  of  over }
is defined as the orthogonal projection 
where  is the orthogonalization of  and .
\end{definition}

\begin{remark}
The weakly orthogonal projection does not give rise to a residual system
in the sense of~\cite{terese:2003}.
The projection fulfils the three identities
  ,
  , and
  ,
but not the \emph{cube identity}
  ,
depicted in Figure~\ref{fig:cube}.

\begin{figure}[hpt!]
\begin{center}
\begin{tikzpicture}
  [inner sep=1mm,
   node distance=30mm,
   terminal/.style={
     rectangle,rounded corners=2.25mm,minimum size=4mm,
     very thick,draw=black!50,top color=white,bottom color=black!20,
   },
   >=stealth,
   red/.style={very thick,->},
   behind/.style={gray,thick},
   diag/.style={inner sep=.5mm}
  ]

  \node (s111) {};
  \node (s211) [right of=s111] {};
  \node (s121) [above of=s111] {};
  \node (s221) [above of=s211] {};
  \begin{scope}[node distance=22mm]
    \node (s112) [above right of=s111] {};
    \node (s212) [above right of=s211] {};
    \node (s122) [above right of=s121] {};
    \node (s222) [above right of=s221] {};
  \end{scope}

  \draw [red,behind] (s111) -- (s112); \node at () [anchor=south east,diag] {};
  \draw [red,behind,dotted] (s112) -- (s212); \draw [red,behind,dotted] (s112) -- (s122); 

  \draw [red] (s111) -- (s211); \node at () [anchor=north] {};
  \draw [red] (s111) -- (s121); \node at () [anchor=east] {};

  \draw [red] (s211) -- (s212); \node at () [anchor=north west,diag] {};
  \draw [red] (s211) -- (s221); \node at () [anchor=west] {};
  \draw [red] (s121) -- (s221); \node at () [anchor=south] {};
  \draw [red] (s121) -- (s122); \node at () [anchor=south east,diag] {};

  \draw [red,dashed] (s221) -- (s222)
    node [midway,sloped,above] {}
    node [midway,sloped,below] {};

  \draw [red,dotted,thin,shorten >= 6mm] (s122) -- (s222);
  \draw [red,dotted,thin,shorten >= 6mm] (s212) -- (s222);
\end{tikzpicture}
\end{center}\vspace{-3ex}
\caption{\textit{Cube identity 
   .}}
\label{fig:cube}
\end{figure}
\end{remark}

\begin{lemma}\label{lem:proj:lift}
  Let ,  be parallel steps in a \woTRS{} .
  Let  and  be the minimal depth of a step in  and , respectively.
  Then the minimal depth of the weakly orthogonal projections
   and 
  is greater or equal .
  If  contains no collapsing rules
  then the minimal depth of  and  is greater or equal 
  and , respectively.
\end{lemma}

\begin{proof}
  Immediate from the definition of the orthogonalization (for overlaps the innermost redex is chosen)
  and the fact that in the orthogonal projection
  a non-collapsing rule applied at depth 
  can lift nested redexes at most to depth  (but not above).
\end{proof}

\begin{lemma}[Strip/Lift Lemma]\label{lem:strip}
Let  be a \woTRS{}, 
 a rewrite sequence,
and  a parallel rewrite step.
Let  and  be the minimal depth of a step in 
and , respectively.
Then there exist a term , a rewrite sequence 
and a parallel step 
such that the minimal depth of the rewrite steps in  and  is ;
see Figure~\ref{fig:strip:collapse}.

\begin{figure}[hpt!]
\begin{center}\hspace{1cm}
\begin{tikzpicture}
  [inner sep=1mm,
   node distance=15mm,
   terminal/.style={
     rectangle,rounded corners=2.25mm,minimum size=4mm,
     very thick,draw=black!50,top color=white,bottom color=black!20,
   },
   >=stealth,
   red/.style={thick,->>},
   infred/.style={
     thick,
     shorten >= 1mm,
     decoration={
       markings,
       mark=at position -3mm with {\arrow{stealth}},
       mark=at position -1.5mm with {\arrow{stealth}},
       mark=at position 1 with {\arrow{stealth}}
     },
     postaction={decorate}
   }]

  \node (s) {};
  \node (t1) [node distance=40mm,right of=s] {};
  \node (t2) [below of=s] {};
  \node (u) [below of=t1] {};

  \draw [infred] (s) -- (t1);
  \draw [red,->] (s) -- (t2);
  \draw [infred,dashed] (t2) -- (u);
  \draw [red,->] (t1) -- (u);

  \draw [thick] () -- ();
  \draw [thick] () -- ();
  \draw [thick] () -- ();
  \draw [thick] () -- ();

  \node at () [anchor=south] {};
  \node at () [anchor=east,xshift=-.7ex] {};
  \node at () [anchor=north] {};
  \node at () [anchor=west,xshift=.7ex] {};
\end{tikzpicture}
\end{center}\vspace{-3ex}
\caption{\textit{Strip/Lift Lemma with collapsing rules.}}
\label{fig:strip:collapse}
\end{figure}

If additionally  contains no collapsing rules, then
the minimal depth of a step in  and 
is  and , respectively. See also Figure~\ref{fig:strip}.

\begin{figure}[hpt!]
\begin{center}\hspace{2cm}
\begin{tikzpicture}
  [inner sep=1mm,
   node distance=15mm,
   terminal/.style={
     rectangle,rounded corners=2.25mm,minimum size=4mm,
     very thick,draw=black!50,top color=white,bottom color=black!20,
   },
   >=stealth,
   red/.style={thick,->>},
   infred/.style={
     thick,
     shorten >= 1mm,
     decoration={
       markings,
       mark=at position -3mm with {\arrow{stealth}},
       mark=at position -1.5mm with {\arrow{stealth}},
       mark=at position 1 with {\arrow{stealth}}
     },
     postaction={decorate}
   }]

  \node (s) {};
  \node (t1) [node distance=40mm,right of=s] {};
  \node (t2) [below of=s] {};
  \node (u) [below of=t1] {};

  \draw [infred] (s) -- (t1);
  \draw [red,->] (s) -- (t2);
  \draw [infred,dashed] (t2) -- (u);
  \draw [red,->] (t1) -- (u);

  \draw [thick] () -- ();
  \draw [thick] () -- ();
  \draw [thick] () -- ();
  \draw [thick] () -- ();

  \node at () [anchor=south] {};
  \node at () [anchor=east,xshift=-.7ex] {};
  \node at () [anchor=north] {};
  \node at () [anchor=west,xshift=.7ex] {};
\end{tikzpicture}
\end{center}\vspace{-3ex}
\caption{\textit{Strip/Lift Lemma without collapsing rules.}}
\label{fig:strip}
\end{figure}
\end{lemma}

\begin{proof}
  By compression we may assume  in  
  (note that, the minimal depth  is preserved by compression).
  Let ,
  and define .
  Furthermore, let  denote the prefix of  of length , that is, 
  and let  denote the suffix  of .
  We employ the projection of parallel steps to
  close the elementary diagrams with top  and left ,
  that is, 
  we construct the projections  (right)
  and  (bottom).
Then by induction on  using Lemma~\ref{lem:proj:lift} there exists
  for every 
  a term , and parallel steps  and .
  See Figure~\ref{fig:strip:proof} for an overview.

  \begin{figure}[hpt!]
  \begin{center}
  \begin{tikzpicture}
    [inner sep=1mm,
    node distance=25mm,
    terminal/.style={
      rectangle,rounded corners=2.25mm,minimum size=4mm,
      very thick,draw=black!50,top color=white,bottom color=black!20,
    },
    >=stealth,
    red/.style={thick,->>},
    infred/.style={
      thick,
      shorten >= 1mm,
      decoration={
        markings,
        mark=at position -3mm with {\arrow{stealth}},
        mark=at position -1.5mm with {\arrow{stealth}},
        mark=at position 1 with {\arrow{stealth}}
      },
      postaction={decorate}
    }]
  
    \node (s) {};
    \node (s1) [node distance=13mm,right of=s] {};
    \node (s2) [node distance=10mm,right of=s1] {};
    \node (sn0) [node distance=10mm,right of=s2] {};
    \node (sm0) [node distance=30mm,right of=sn0] {};
    \node (t1) [node distance=30mm,right of=sm0] {};
    \node (t2) [below of=s] {};
    \node (s1') [below of=s1] {};
    \node (s2') [below of=s2] {};
    \node (sn0') [below of=sn0] {};
    \node (sm0') [below of=sm0] {};
    \node (u) [below of=t1] {};

    \node (sn0'') [node distance=12mm,below of=sn0] {};
    \node (sm0'') [node distance=12mm,below of=sm0] {};
    \node (t1'') [node distance=12mm,below of=t1] {};

    \draw [red,->] (s) -- (s1);
    \draw [red,->] (s1) -- (s2);
    \draw [red,->] (s2) -- (sn0);
    \draw [red,->>] (sn0) -- (sm0) node [midway,above] {};
    \draw [infred] (sm0) -- (t1) node [midway,above] {};

    \draw [red,->] (s) -- (t2) node [midway,left,xshift=-.5ex] {};
    \draw [thick] () -- ();
    \draw [thick] () -- ();

    \draw [red,->] (s1) -- (s1') node [midway,right,xshift=.5ex] {};
    \draw [thick] () -- ();
    \draw [thick] () -- ();

    \draw [red,->] (sn0) -- (sn0'') node [midway,right,xshift=.5ex] {};
    \draw [thick] () -- ();
    \draw [thick] () -- ();
    \draw [red,->] (sn0'') -- (sn0') node [midway,right,xshift=.5ex] {};
    \draw [thick] () -- ();
    \draw [thick] () -- ();

    \draw [red,->] (sm0) -- (sm0'') node [midway,right,xshift=.5ex] {};
    \draw [thick] () -- ();
    \draw [thick] () -- ();
    \draw [red,->] (sm0'') -- (sm0') node [midway,right,xshift=.5ex] {};
    \draw [thick] () -- ();
    \draw [thick] () -- ();

    \draw [red,->] (t1) -- (t1'') node [midway,right,xshift=.5ex] {};
    \draw [thick] () -- ();
    \draw [thick] () -- ();
    \draw [red,->,dashed] (t1'') -- (u) node [midway,right,xshift=.5ex] {};
    \draw [thick] () -- ();
    \draw [thick] () -- ();

    \draw [red,->] (t2) -- (s1');
    \draw [red,->] (s1') -- (s2');
    \draw [red,->] (s2') -- (sn0');
    \draw [infred,dashed] (sn0') -- (sm0');
    \draw [infred,dashed] (sm0'') -- (t1'')  node [midway,above] {};
    \draw [infred,dashed] (sm0') -- (u)  node [midway,above] {};
  \end{tikzpicture}
  \end{center}\vspace{-3ex}
  \caption{\textit{Strip/Lift Lemma, proof overview.}}
  \label{fig:strip:proof}
  \end{figure}

  We show that the rewrite sequence constructed at the bottom 
  of Figures~\ref{fig:strip:collapse} and~\ref{fig:strip} is strongly convergent,
  and that the parallel steps  have a limit for 
  (parallel steps are always strongly convergent).

  Let  be arbitrary.
  By strong convergence of  there exists  such that
  all steps in  are at depth .
  Since  is a parallel step there are only finitely many
  redexes  in  rooted above depth .
  By projection of  along 
  no fresh redexes above depth  can be created.
  The steps in  may be cancelled out due to overlaps,
  nevertheless, for all  the set of steps above depth  in 
  is a subset of .

  Let  be the maximal depth of a left-hand side of a rule applied in .
  By strong convergence of  there exists  such that
  all steps in  are at depth .
  As a consequence the steps  in  rooted above depth 
  will stay fixed throughout the remainder of the projection.
  Then for all  the parallel step 
  can be split into 
  where  consists of the steps of  at depth .
  Since  was arbitrary, it follows that projection of  over  has a limit.
  Moreover the steps of the projection of  over 
  are at depth  since rules with pattern depth  can lift steps by at most by .
  Again, since  was arbitrary, it follows that the projection of  over  is strongly convergent.

  Finally, both constructed rewrite sequences (bottom and right) converge towards the same limit 
  since all terms  coincide up to depth 
  (the terms  coincide up to depth  and the lifting effect of the steps  is limited by ).
\end{proof}

\begin{theorem}\label{thm:cr}
Every \woTRS{} without collapsing rules is infinitary confluent.
\end{theorem}

\begin{figure}[hpt!]
\begin{center}
\begin{tikzpicture}
  [inner sep=1mm,
   node distance=30mm,
   terminal/.style={
     rectangle,rounded corners=2.25mm,minimum size=4mm,
     very thick,draw=black!50,top color=white,bottom color=black!20,
   },
   >=stealth,
   red/.style={thick,->>},
   infred/.style={
     thick,
     shorten >= 1mm,
     decoration={
       markings,
       mark=at position -3mm with {\arrow{stealth}},
       mark=at position -1.5mm with {\arrow{stealth}},
       mark=at position 1 with {\arrow{stealth}}
     },
     postaction={decorate}
   }]

  \node (s) {};
  \node (s1) [node distance=40mm,right of=s] {};
  \node (t1) [node distance=60mm,right of=s1] {};
  \node (s2) [below of=s] {};
  \node (t2) [below of=s2] {};
  \node (s') [below of=s1] {};
  \node (t1') [below of=t1] {};
  \node (t2') [below of=s'] {};
  \node (u) [below of=t1'] {};

  \draw [red] (s) -- (s1);
  \draw [infred] (s1) -- (t1);
  \draw [red] (s) -- (s2);
  \draw [infred] (s2) -- (t2);

  \draw [red,dashed] (s1) -- (s');
  \draw [thick] () -- ();
  \draw [thick] () -- ();

  \draw [red,dashed] (s2) -- (s');
  \draw [thick] () -- ();
  \draw [thick] () -- ();


  \draw [infred,dashed] (s') -- (t1');
  \draw [infred,dashed] (t1) -- (t1');
  \draw [infred,dashed] (s') -- (t2');
  \draw [infred,dashed] (t2) -- (t2');
  \draw [infred,dashed] (t1') -- (u);
  \draw [infred,dashed] (t2') -- (u);

  \node at () [anchor=south] {};
  \node at () [anchor=east] {};
  \node at () [anchor=south] {};
  \node at () [anchor=east] {};
  \node at () [anchor=south,yshift=.5ex] {};
  \node at () [anchor=west,xshift=.5ex] {};

  \node at () [anchor=south] {};
  \node at () [anchor=west] {};

  \node at () [anchor=south] {};
  \node at () [anchor=west] {};

  \node at () {finitary diagram};
  \node at () {strip lemma};
  \node at () {strip lemma};
  \node at () {\parbox{3cm}{\centerline{coinduction/}\centerline{repeat construction}\centerline{with }}};
\end{tikzpicture}
\end{center}\vspace{-3ex}
\caption{\textit{Infinitary confluence.}}
\label{fig:confluence}
\end{figure}

\begin{proof}
  An overview of the proof is given in Figure~\ref{fig:confluence}.
  Let  and  be two rewrite sequences.
  By compression we may assume  and .
  Let  be the minimal depth of any rewrite step in  and .
  Then  and  are of the form
  
  and
  
  such that all steps in  and  at depth .
  
  Then  and  can be joined by finitary diagram completion employing the diamond property for parallel steps.
  If follows that there exists a term  and finite sequences of (possibly infinite) parallel steps  and 
  all steps of which are at depth  (Lemma~\ref{lem:proj:lift}).
  We project 
   over  
   over  
  by repeated application of the Lemma~\ref{lem:strip},
  obtaining rewrite sequences 
  ,
  ,
  , and
   with depth , , , and , respectively.
  As a consequence we have ,  and  coincide up to (including) depth .
  Recursively applying the construction to the rewrite sequences  and 
  yields strongly convergent rewrite sequences 
   and 
  where the terms  and  coincide up to depth .
  Thus these rewrite sequences converge towards the same limit .
\end{proof}

We consider an example to illustrate that non-collapsingness is a necessary condition for Theorem~\ref{thm:cr}.
\begin{example}\label{ex:collapse}
  Let  be a TRS over the signature  consisting of the rule:
  
  Then, using a self-explaining recursive notation, 
  the term  rewrites in  many steps to 
   as well as  which have no common reduct.
  The TRS  is weakly orthogonal (even orthogonal) but not confluent.
\end{example}
