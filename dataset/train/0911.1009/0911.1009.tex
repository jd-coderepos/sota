In the previous section we have seen that the property $\UNinf$ 
fails dramatically for weakly orthogonal TRSs when collapsing rules are present,
and hence also $\CRinf$. 
Now we show that \woTRS{s} without collapsing 
rules are infinitary confluent ($\CRinf$),
and as a consequence also have the property $\UNinf$.

We adapt the projection of parallel steps in \woTRS{s}
from~\cite[Section~8.8.4.]{terese:2003} to infinite terms.
The basic idea is to orthogonalize the parallel steps,
and then project the orthogonalized steps.
The orthogonalization uses that overlapping redexes 
have the same effect and hence can be replaced by each other.
In case of overlaps we replace the outermost redex by the innermost one.
This is possible for infinitary parallel steps since there
can never be infinite chains of overlapping, nested redexes (see Figure~\ref{fig:chain}).
For a treatment of infinitary developments where such chains can occur,
we refer to Section~\ref{sec:diamond}.
See further~\cite[Proposition~8.8.23]{terese:2003} for orthogonalization 
in the finitary case.

\begin{proposition}
  Let $\pstep{\astep}{s}{t_1}$, $\pstep{\bstep}{s}{t_2}$ be parallel steps in a \woTRS{}.
  Then there exists an \emph{orthogonalization $\pair{\astep'}{\bstep'}$ of $\astep$ and $\bstep$},
  that is, a pair of orthogonal parallel steps such that $\pstep{\astep'}{s}{t_1}$, $\pstep{\bstep'}{s}{t_2}$.
\end{proposition}

\begin{proof}
In case of overlaps between $\astep$ and $\bstep$, 
then for every overlap we replace the outermost redex by the innermost one
(if there are multiple inner redexes overlapping, then we choose the left-most among the top-most redexes).
If there are two redexes at the same position but with respect to different rules,
then we replace the redex in $\bstep$ with the one in $\astep$.
See also Figure~\ref{fig:orthogonalization:parallel}.
\begin{figure}[hpt!]
\begin{center}
  \includegraphics{figs/parallel}
\end{center}\vspace{-3ex}
\caption{\textit{Orthogonalization of parallel steps; the arrow indicates replacement.}}
\label{fig:orthogonalization:parallel}
\end{figure}
\end{proof}

\begin{definition}
Let $\pstep{\astep}{s}{t_1}$, $\pstep{\bstep}{s}{t_2}$ be parallel steps in a \woTRS{}.
The \emph{weakly orthogonal projection $\project{\astep}{\bstep}$ of $\astep$ over $\bstep$}
is defined as the orthogonal projection $\project{\astep'}{\bstep'}$
where $\pair{\astep'}{\bstep'}$ is the orthogonalization of $\astep$ and $\bstep$.
\end{definition}

\begin{remark}
The weakly orthogonal projection does not give rise to a residual system
in the sense of~\cite{terese:2003}.
The projection fulfils the three identities
  $\project{\astep}{\astep} \approx \unit$,
  $\project{\astep}{\unit} \approx \astep$, and
  $\project{\unit}{\astep} \approx \unit$,
but not the \emph{cube identity}
  $\project{(\project{\astep}{\bstep})}{(\project{\cstep}{\bstep})} \approx
    \project{(\project{\astep}{\cstep})}{(\project{\bstep}{\cstep})}$,
depicted in Figure~\ref{fig:cube}.

\begin{figure}[hpt!]
\begin{center}
\begin{tikzpicture}
  [inner sep=1mm,
   node distance=30mm,
   terminal/.style={
     rectangle,rounded corners=2.25mm,minimum size=4mm,
     very thick,draw=black!50,top color=white,bottom color=black!20,
   },
   >=stealth,
   red/.style={very thick,->},
   behind/.style={gray,thick},
   diag/.style={inner sep=.5mm}
  ]

  \node (s111) {};
  \node (s211) [right of=s111] {};
  \node (s121) [above of=s111] {};
  \node (s221) [above of=s211] {};
  \begin{scope}[node distance=22mm]
    \node (s112) [above right of=s111] {};
    \node (s212) [above right of=s211] {};
    \node (s122) [above right of=s121] {};
    \node (s222) [above right of=s221] {};
  \end{scope}

  \draw [red,behind] (s111) -- (s112); \node at ($(s111)!.5!(s112)$) [anchor=south east,diag] {$\astep$};
  \draw [red,behind,dotted] (s112) -- (s212); \draw [red,behind,dotted] (s112) -- (s122); 

  \draw [red] (s111) -- (s211); \node at ($(s111)!.5!(s211)$) [anchor=north] {$\cstep$};
  \draw [red] (s111) -- (s121); \node at ($(s111)!.5!(s121)$) [anchor=east] {$\bstep$};

  \draw [red] (s211) -- (s212); \node at ($(s211)!.5!(s212)$) [anchor=north west,diag] {$\project{\astep}{\cstep}$};
  \draw [red] (s211) -- (s221); \node at ($(s211)!.4!(s221)$) [anchor=west] {$\project{\bstep}{\cstep}$};
  \draw [red] (s121) -- (s221); \node at ($(s121)!.35!(s221)$) [anchor=south] {$\project{\cstep}{\bstep}$};
  \draw [red] (s121) -- (s122); \node at ($(s121)!.5!(s122)$) [anchor=south east,diag] {$\project{\astep}{\bstep}$};

  \draw [red,dashed] (s221) -- (s222)
    node [midway,sloped,above] {$\project{(\project{\astep}{\bstep})}{(\project{\cstep}{\bstep})}$}
    node [midway,sloped,below] {$\project{(\project{\astep}{\cstep})}{(\project{\bstep}{\cstep})}$};

  \draw [red,dotted,thin,shorten >= 6mm] (s122) -- (s222);
  \draw [red,dotted,thin,shorten >= 6mm] (s212) -- (s222);
\end{tikzpicture}
\end{center}\vspace{-3ex}
\caption{\textit{Cube identity 
   $\project{(\project{\astep}{\bstep})}{(\project{\cstep}{\bstep})} \approx
    \project{(\project{\astep}{\cstep})}{(\project{\bstep}{\cstep})}$.}}
\label{fig:cube}
\end{figure}
\end{remark}

\begin{lemma}\label{lem:proj:lift}
  Let $\pstep{\astep}{s}{t_1}$, $\pstep{\bstep}{s}{t_2}$ be parallel steps in a \woTRS{} $\atrs$.
  Let $d_\astep$ and $d_\bstep$ be the minimal depth of a step in $\astep$ and $\bstep$, respectively.
  Then the minimal depth of the weakly orthogonal projections
  $\project{\astep}{\bstep}$ and $\project{\bstep}{\astep}$
  is greater or equal $\bfunap{\min}{d_\astep}{d_\bstep}$.
  If $\atrs$ contains no collapsing rules
  then the minimal depth of $\project{\astep}{\bstep}$ and $\project{\bstep}{\astep}$ is greater or equal $\bfunap{\min}{d_\astep}{d_\bstep+1}$
  and $\bfunap{\min}{d_\bstep}{d_\astep+1}$, respectively.
\end{lemma}

\begin{proof}
  Immediate from the definition of the orthogonalization (for overlaps the innermost redex is chosen)
  and the fact that in the orthogonal projection
  a non-collapsing rule applied at depth $d$
  can lift nested redexes at most to depth $d+1$ (but not above).
\end{proof}

\begin{lemma}[Strip/Lift Lemma]\label{lem:strip}
Let $\atrs$ be a \woTRS{}, 
$\aseq \funin s \to^\alpha t_1$ a rewrite sequence,
and $\pstep{\astep}{s}{t_2}$ a parallel rewrite step.
Let $d_\aseq$ and $d_\bseq$ be the minimal depth of a step in $\aseq$
and $\astep$, respectively.
Then there exist a term $u$, a rewrite sequence $\bseq \funin t_2 \to^{\le \omega} u$
and a parallel step $\pstep{\bstep}{t_1}{u}$
such that the minimal depth of the rewrite steps in $\bseq$ and $\bstep$ is $\bfunap{\min}{d_\aseq}{d_\bseq}$;
see Figure~\ref{fig:strip:collapse}.

\begin{figure}[hpt!]
\begin{center}\hspace{1cm}
\begin{tikzpicture}
  [inner sep=1mm,
   node distance=15mm,
   terminal/.style={
     rectangle,rounded corners=2.25mm,minimum size=4mm,
     very thick,draw=black!50,top color=white,bottom color=black!20,
   },
   >=stealth,
   red/.style={thick,->>},
   infred/.style={
     thick,
     shorten >= 1mm,
     decoration={
       markings,
       mark=at position -3mm with {\arrow{stealth}},
       mark=at position -1.5mm with {\arrow{stealth}},
       mark=at position 1 with {\arrow{stealth}}
     },
     postaction={decorate}
   }]

  \node (s) {$s$};
  \node (t1) [node distance=40mm,right of=s] {$t_1$};
  \node (t2) [below of=s] {$t_2$};
  \node (u) [below of=t1] {$u$};

  \draw [infred] (s) -- (t1);
  \draw [red,->] (s) -- (t2);
  \draw [infred,dashed] (t2) -- (u);
  \draw [red,->] (t1) -- (u);

  \draw [thick] ($(s)!.5!(t2) + (-.75ex,-.20ex)$) -- ($(s)!.5!(t2) + (.75ex,-.20ex)$);
  \draw [thick] ($(s)!.5!(t2) + (-.75ex,.20ex)$) -- ($(s)!.5!(t2) + (.75ex,.20ex)$);
  \draw [thick] ($(t1)!.5!(u) + (-.75ex,-.20ex)$) -- ($(t1)!.5!(u) + (.75ex,-.20ex)$);
  \draw [thick] ($(t1)!.5!(u) + (-.75ex,.20ex)$) -- ($(t1)!.5!(u) + (.75ex,.20ex)$);

  \node at ($(s)!.5!(t1)$) [anchor=south] {$\ge d_\aseq$};
  \node at ($(s)!.5!(t2)$) [anchor=east,xshift=-.7ex] {$\ge d_\bseq$};
  \node at ($(t2)!.5!(u)$) [anchor=north] {$\ge \bfunap{\min}{d_\aseq}{d_\bseq}$};
  \node at ($(t1)!.5!(u)$) [anchor=west,xshift=.7ex] {$\ge \bfunap{\min}{d_\aseq}{d_\bseq}$};
\end{tikzpicture}
\end{center}\vspace{-3ex}
\caption{\textit{Strip/Lift Lemma with collapsing rules.}}
\label{fig:strip:collapse}
\end{figure}

If additionally $\atrs$ contains no collapsing rules, then
the minimal depth of a step in $\bseq$ and $\bstep$
is $\bfunap{\min}{d_\aseq}{d_\bseq+1}$ and $\bfunap{\min}{d_\bseq}{d_\aseq+1}$, respectively. See also Figure~\ref{fig:strip}.

\begin{figure}[hpt!]
\begin{center}\hspace{2cm}
\begin{tikzpicture}
  [inner sep=1mm,
   node distance=15mm,
   terminal/.style={
     rectangle,rounded corners=2.25mm,minimum size=4mm,
     very thick,draw=black!50,top color=white,bottom color=black!20,
   },
   >=stealth,
   red/.style={thick,->>},
   infred/.style={
     thick,
     shorten >= 1mm,
     decoration={
       markings,
       mark=at position -3mm with {\arrow{stealth}},
       mark=at position -1.5mm with {\arrow{stealth}},
       mark=at position 1 with {\arrow{stealth}}
     },
     postaction={decorate}
   }]

  \node (s) {$s$};
  \node (t1) [node distance=40mm,right of=s] {$t_1$};
  \node (t2) [below of=s] {$t_2$};
  \node (u) [below of=t1] {$u$};

  \draw [infred] (s) -- (t1);
  \draw [red,->] (s) -- (t2);
  \draw [infred,dashed] (t2) -- (u);
  \draw [red,->] (t1) -- (u);

  \draw [thick] ($(s)!.5!(t2) + (-.75ex,-.20ex)$) -- ($(s)!.5!(t2) + (.75ex,-.20ex)$);
  \draw [thick] ($(s)!.5!(t2) + (-.75ex,.20ex)$) -- ($(s)!.5!(t2) + (.75ex,.20ex)$);
  \draw [thick] ($(t1)!.5!(u) + (-.75ex,-.20ex)$) -- ($(t1)!.5!(u) + (.75ex,-.20ex)$);
  \draw [thick] ($(t1)!.5!(u) + (-.75ex,.20ex)$) -- ($(t1)!.5!(u) + (.75ex,.20ex)$);

  \node at ($(s)!.5!(t1)$) [anchor=south] {$\ge d_\aseq$};
  \node at ($(s)!.5!(t2)$) [anchor=east,xshift=-.7ex] {$\ge d_\bseq$};
  \node at ($(t2)!.5!(u)$) [anchor=north] {$\ge \bfunap{\min}{d_\aseq}{d_\bseq+1}$};
  \node at ($(t1)!.5!(u)$) [anchor=west,xshift=.7ex] {$\ge \bfunap{\min}{d_\bseq}{d_\aseq+1}$};
\end{tikzpicture}
\end{center}\vspace{-3ex}
\caption{\textit{Strip/Lift Lemma without collapsing rules.}}
\label{fig:strip}
\end{figure}
\end{lemma}

\begin{proof}
  By compression we may assume $\alpha \le \omega$ in $\aseq \funin s \to^{\le \omega} t_1$ 
  (note that, the minimal depth $d$ is preserved by compression).
  Let $\aseq \funin s \equiv s_0 \to s_1 \to s_2 \to \ldots$,
  and define $\bstep_0 = \bstep$.
  Furthermore, let $\seqpref{\aseq}{n}$ denote the prefix of $\aseq$ of length $n$, that is, $s_0 \to \ldots \to s_n$
  and let $\seqsuf{\aseq}{n}$ denote the suffix $s_n \to s_{n+1} \to \ldots$ of $\aseq$.
  We employ the projection of parallel steps to
  close the elementary diagrams with top $s_n \to s_{n+1}$ and left $\pstep{\bstep_n}{s_n}{s_n'}$,
  that is, 
  we construct the projections $\bstep_{i+1} = \project{\bstep_i}{(s_i \to s_{i+1})}$ (right)
  and $\project{(s_i \to s_{i+1})}{\bstep_i}$ (bottom).
Then by induction on $n$ using Lemma~\ref{lem:proj:lift} there exists
  for every $1 \le n \le \alpha$
  a term $s_n'$, and parallel steps $\pstep{\astep_n}{s_n}{s_n'}$ and $s_{n-1}' \pred s_n'$.
  See Figure~\ref{fig:strip:proof} for an overview.

  \begin{figure}[hpt!]
  \begin{center}
  \begin{tikzpicture}
    [inner sep=1mm,
    node distance=25mm,
    terminal/.style={
      rectangle,rounded corners=2.25mm,minimum size=4mm,
      very thick,draw=black!50,top color=white,bottom color=black!20,
    },
    >=stealth,
    red/.style={thick,->>},
    infred/.style={
      thick,
      shorten >= 1mm,
      decoration={
        markings,
        mark=at position -3mm with {\arrow{stealth}},
        mark=at position -1.5mm with {\arrow{stealth}},
        mark=at position 1 with {\arrow{stealth}}
      },
      postaction={decorate}
    }]
  
    \node (s) {$s \equiv s_0$};
    \node (s1) [node distance=13mm,right of=s] {$s_1$};
    \node (s2) [node distance=10mm,right of=s1] {$\ldots$};
    \node (sn0) [node distance=10mm,right of=s2] {$s_{n_0}$};
    \node (sm0) [node distance=30mm,right of=sn0] {$s_{m_0}$};
    \node (t1) [node distance=30mm,right of=sm0] {$t_1$};
    \node (t2) [below of=s] {$t_2 \equiv s_0'$};
    \node (s1') [below of=s1] {$s_1'$};
    \node (s2') [below of=s2] {$\ldots$};
    \node (sn0') [below of=sn0] {$s_{n_0}'$};
    \node (sm0') [below of=sm0] {$s_{m_0}'$};
    \node (u) [below of=t1] {$u$};

    \node (sn0'') [node distance=12mm,below of=sn0] {$s_{n_0}''$};
    \node (sm0'') [node distance=12mm,below of=sm0] {$s_{m_0}''$};
    \node (t1'') [node distance=12mm,below of=t1] {$t_1''$};

    \draw [red,->] (s) -- (s1);
    \draw [red,->] (s1) -- (s2);
    \draw [red,->] (s2) -- (sn0);
    \draw [red,->>] (sn0) -- (sm0) node [midway,above] {$\ge d$};
    \draw [infred] (sm0) -- (t1) node [midway,above] {$\ge d+p$};

    \draw [red,->] (s) -- (t2) node [midway,left,xshift=-.5ex] {$\astep = \astep_0$};
    \draw [thick] ($(s)!.5!(t2) + (-.75ex,-.20ex)$) -- ($(s)!.5!(t2) + (.75ex,-.20ex)$);
    \draw [thick] ($(s)!.5!(t2) + (-.75ex,.20ex)$) -- ($(s)!.5!(t2) + (.75ex,.20ex)$);

    \draw [red,->] (s1) -- (s1') node [midway,right,xshift=.5ex] {$\astep_1$};
    \draw [thick] ($(s1)!.5!(s1') + (-.75ex,-.20ex)$) -- ($(s1)!.5!(s1') + (.75ex,-.20ex)$);
    \draw [thick] ($(s1)!.5!(s1') + (-.75ex,.20ex)$) -- ($(s1)!.5!(s1') + (.75ex,.20ex)$);

    \draw [red,->] (sn0) -- (sn0'') node [midway,right,xshift=.5ex] {$\astep_{n_0,<d}$};
    \draw [thick] ($(sn0)!.5!(sn0'') + (-.75ex,-.20ex)$) -- ($(sn0)!.5!(sn0'') + (.75ex,-.20ex)$);
    \draw [thick] ($(sn0)!.5!(sn0'') + (-.75ex,.20ex)$) -- ($(sn0)!.5!(sn0'') + (.75ex,.20ex)$);
    \draw [red,->] (sn0'') -- (sn0') node [midway,right,xshift=.5ex] {$\astep_{n_0,\ge d}$};
    \draw [thick] ($(sn0'')!.5!(sn0') + (-.75ex,-.20ex)$) -- ($(sn0'')!.5!(sn0') + (.75ex,-.20ex)$);
    \draw [thick] ($(sn0'')!.5!(sn0') + (-.75ex,.20ex)$) -- ($(sn0'')!.5!(sn0') + (.75ex,.20ex)$);

    \draw [red,->] (sm0) -- (sm0'') node [midway,right,xshift=.5ex] {$\bstep \subseteq \astep_{n_0,<d}$};
    \draw [thick] ($(sm0)!.5!(sm0'') + (-.75ex,-.20ex)$) -- ($(sm0)!.5!(sm0'') + (.75ex,-.20ex)$);
    \draw [thick] ($(sm0)!.5!(sm0'') + (-.75ex,.20ex)$) -- ($(sm0)!.5!(sm0'') + (.75ex,.20ex)$);
    \draw [red,->] (sm0'') -- (sm0') node [midway,right,xshift=.5ex] {$\astep_{m_0,\ge d}$};
    \draw [thick] ($(sm0'')!.5!(sm0') + (-.75ex,-.20ex)$) -- ($(sm0'')!.5!(sm0') + (.75ex,-.20ex)$);
    \draw [thick] ($(sm0'')!.5!(sm0') + (-.75ex,.20ex)$) -- ($(sm0'')!.5!(sm0') + (.75ex,.20ex)$);

    \draw [red,->] (t1) -- (t1'') node [midway,right,xshift=.5ex] {$\bstep$};
    \draw [thick] ($(t1)!.5!(t1'') + (-.75ex,-.20ex)$) -- ($(t1)!.5!(t1'') + (.75ex,-.20ex)$);
    \draw [thick] ($(t1)!.5!(t1'') + (-.75ex,.20ex)$) -- ($(t1)!.5!(t1'') + (.75ex,.20ex)$);
    \draw [red,->,dashed] (t1'') -- (u) node [midway,right,xshift=.5ex] {$\ge d$};
    \draw [thick] ($(t1'')!.5!(u) + (-.75ex,-.20ex)$) -- ($(t1'')!.5!(u) + (.75ex,-.20ex)$);
    \draw [thick] ($(t1'')!.5!(u) + (-.75ex,.20ex)$) -- ($(t1'')!.5!(u) + (.75ex,.20ex)$);

    \draw [red,->] (t2) -- (s1');
    \draw [red,->] (s1') -- (s2');
    \draw [red,->] (s2') -- (sn0');
    \draw [infred,dashed] (sn0') -- (sm0');
    \draw [infred,dashed] (sm0'') -- (t1'')  node [midway,above] {$\ge d$};
    \draw [infred,dashed] (sm0') -- (u)  node [midway,above] {$\ge d$};
  \end{tikzpicture}
  \end{center}\vspace{-3ex}
  \caption{\textit{Strip/Lift Lemma, proof overview.}}
  \label{fig:strip:proof}
  \end{figure}

  We show that the rewrite sequence constructed at the bottom $s_0' \pred s_1' \pred \ldots$
  of Figures~\ref{fig:strip:collapse} and~\ref{fig:strip} is strongly convergent,
  and that the parallel steps $\astep_i$ have a limit for $i \to \infty$
  (parallel steps are always strongly convergent).

  Let $d \in \nat$ be arbitrary.
  By strong convergence of $\aseq$ there exists $n_0 \in \nat$ such that
  all steps in $\seqsuf{\aseq}{n_0}$ are at depth $\ge d$.
  Since $\astep_{n_0}$ is a parallel step there are only finitely many
  redexes $\astep_{n_0,<d} \subseteq \astep_{n_0}$ in $\astep_{n_0}$ rooted above depth $d$.
  By projection of $\astep_{n_0}$ along $\seqsuf{\aseq}{n_0}$
  no fresh redexes above depth $d$ can be created.
  The steps in $\astep_{n_0,<d}$ may be cancelled out due to overlaps,
  nevertheless, for all $m \ge n_0$ the set of steps above depth $d$ in $\astep_m$
  is a subset of $\astep_{n_0,<d}$.

  Let $p$ be the maximal depth of a left-hand side of a rule applied in $\astep_{n_0,<d}$.
  By strong convergence of $\aseq$ there exists $m_0 \ge n_0 \in \nat$ such that
  all steps in $\seqsuf{\aseq}{n_0}$ are at depth $\ge d+p$.
  As a consequence the steps $\bstep$ in $\astep_{m_0}$ rooted above depth $d$
  will stay fixed throughout the remainder of the projection.
  Then for all $m \ge m_0$ the parallel step $\astep_m$
  can be split into $\astep_m = s_m \pred_\bstep s_m'' \pred_{\astep_{m,\ge d}} s_m'$
  where $\astep_{m,\ge d}$ consists of the steps of $\astep_m$ at depth $\ge d$.
  Since $d$ was arbitrary, it follows that projection of $\astep$ over $\aseq$ has a limit.
  Moreover the steps of the projection of $\seqsuf{\aseq}{m_0}$ over $\astep_{m_0}$
  are at depth $\ge d + p - p = d$ since rules with pattern depth $\le p$ can lift steps by at most by $p$.
  Again, since $d$ was arbitrary, it follows that the projection of $\aseq$ over $\astep$ is strongly convergent.

  Finally, both constructed rewrite sequences (bottom and right) converge towards the same limit $u$
  since all terms $\{s_m', s_m'' \where m \ge m_0\}$ coincide up to depth $d-1$
  (the terms $\{s_m \where m \ge m_0\}$ coincide up to depth $d + p -1$ and the lifting effect of the steps $\astep_m$ is limited by $p$).
\end{proof}

\begin{theorem}\label{thm:cr}
Every \woTRS{} without collapsing rules is infinitary confluent.
\end{theorem}

\begin{figure}[hpt!]
\begin{center}
\begin{tikzpicture}
  [inner sep=1mm,
   node distance=30mm,
   terminal/.style={
     rectangle,rounded corners=2.25mm,minimum size=4mm,
     very thick,draw=black!50,top color=white,bottom color=black!20,
   },
   >=stealth,
   red/.style={thick,->>},
   infred/.style={
     thick,
     shorten >= 1mm,
     decoration={
       markings,
       mark=at position -3mm with {\arrow{stealth}},
       mark=at position -1.5mm with {\arrow{stealth}},
       mark=at position 1 with {\arrow{stealth}}
     },
     postaction={decorate}
   }]

  \node (s) {$s$};
  \node (s1) [node distance=40mm,right of=s] {$s_1$};
  \node (t1) [node distance=60mm,right of=s1] {$t_1$};
  \node (s2) [below of=s] {$s_2$};
  \node (t2) [below of=s2] {$t_2$};
  \node (s') [below of=s1] {$s'$};
  \node (t1') [below of=t1] {$t_1'$};
  \node (t2') [below of=s'] {$t_2'$};
  \node (u) [below of=t1'] {$u$};

  \draw [red] (s) -- (s1);
  \draw [infred] (s1) -- (t1);
  \draw [red] (s) -- (s2);
  \draw [infred] (s2) -- (t2);

  \draw [red,dashed] (s1) -- (s');
  \draw [thick] ($(s1)!.5!(s') + (-.75ex,-.20ex)$) -- ($(s1)!.5!(s') + (.75ex,-.20ex)$);
  \draw [thick] ($(s1)!.5!(s') + (-.75ex,.20ex)$) -- ($(s1)!.5!(s') + (.75ex,.20ex)$);

  \draw [red,dashed] (s2) -- (s');
  \draw [thick] ($(s2)!.5!(s') + (-.20ex,-.75ex)$) -- ($(s2)!.5!(s') + (-.20ex,.75ex)$);
  \draw [thick] ($(s2)!.5!(s') + (.20ex,-.75ex)$) -- ($(s2)!.5!(s') + (.20ex,.75ex)$);


  \draw [infred,dashed] (s') -- (t1');
  \draw [infred,dashed] (t1) -- (t1');
  \draw [infred,dashed] (s') -- (t2');
  \draw [infred,dashed] (t2) -- (t2');
  \draw [infred,dashed] (t1') -- (u);
  \draw [infred,dashed] (t2') -- (u);

  \node at ($(s)!.5!(s1)$) [anchor=south] {$\ge d$};
  \node at ($(s)!.5!(s2)$) [anchor=east] {$\ge d$};
  \node at ($(s1)!.5!(t1)$) [anchor=south] {$>d$};
  \node at ($(s2)!.5!(t2)$) [anchor=east] {$>d$};
  \node at ($(s2)!.5!(s')$) [anchor=south,yshift=.5ex] {$\ge d$};
  \node at ($(s1)!.5!(s')$) [anchor=west,xshift=.5ex] {$\ge d$};

  \node at ($(s')!.5!(t1')$) [anchor=south] {$> d$};
  \node at ($(s')!.5!(t2')$) [anchor=west] {$> d$};

  \node at ($(t2)!.5!(t2')$) [anchor=south] {$\ge d$};
  \node at ($(t1)!.5!(t1')$) [anchor=west] {$\ge d$};

  \node at ($(s)!.5!(s')$) {finitary diagram};
  \node at ($(s')!.5!(t1)$) {strip lemma};
  \node at ($(s')!.5!(t2)$) {strip lemma};
  \node at ($(s')!.5!(u)$) {\parbox{3cm}{\centerline{coinduction/}\centerline{repeat construction}\centerline{with $d+1$}}};
\end{tikzpicture}
\end{center}\vspace{-3ex}
\caption{\textit{Infinitary confluence.}}
\label{fig:confluence}
\end{figure}

\begin{proof}
  An overview of the proof is given in Figure~\ref{fig:confluence}.
  Let $\aseq \funin s \to^\alpha t_1$ and $\bseq \funin s \to^\beta t_2$ be two rewrite sequences.
  By compression we may assume $\alpha \le \omega$ and $\beta \le \omega$.
  Let $d$ be the minimal depth of any rewrite step in $\aseq$ and $\bstep$.
  Then $\aseq$ and $\bseq$ are of the form
  $\aseq \funin s \to^* s_1 \to^{\le\omega} t_1$
  and
  $\bseq \funin s \to^* s_2 \to^{\le\omega} t_2$
  such that all steps in $s_1 \to^{\le\omega} t_1$ and $s_2 \to^{\le\omega} t_2$ at depth $> d$.
  
  Then $s \to^* s_1$ and $s \to^* s_2$ can be joined by finitary diagram completion employing the diamond property for parallel steps.
  If follows that there exists a term $s'$ and finite sequences of (possibly infinite) parallel steps $s_1 \pred^* s'$ and $s_2 \pred^* s'$
  all steps of which are at depth $\ge d$ (Lemma~\ref{lem:proj:lift}).
  We project 
  $s_1 \to^{\le\omega} t_1$ over $s_1 \pred^* s'$ 
  $s_2 \to^{\le\omega} t_2$ over $s_2 \pred^* s'$ 
  by repeated application of the Lemma~\ref{lem:strip},
  obtaining rewrite sequences 
  $t_1 \ired t_1'$,
  $s' \ired t_1'$,
  $t_2 \ired t_2'$, and
  $s' \ired t_2'$ with depth $\ge d$, $> d$, $\ge d$, and $> d$, respectively.
  As a consequence we have $t_1'$, $s'$ and $t_2'$ coincide up to (including) depth $d$.
  Recursively applying the construction to the rewrite sequences $s' \ired t_1'$ and $s' \ired t_2'$
  yields strongly convergent rewrite sequences 
  $t_2 \ired t_2' \ired t_2'' \ired \ldots$ and $t_1 \ired t_1' \ired t_1'' \ired \ldots$
  where the terms $t_1^{(n)}$ and $t_2^{(n)}$ coincide up to depth $d + n -1$.
  Thus these rewrite sequences converge towards the same limit $u$.
\end{proof}

We consider an example to illustrate that non-collapsingness is a necessary condition for Theorem~\ref{thm:cr}.
\begin{example}\label{ex:collapse}
  Let $\atrs$ be a TRS over the signature $\{f,a,b\}$ consisting of the rule:
  \begin{align*}
    \bfunap{f}{x}{y} &\to x
  \end{align*}
  Then, using a self-explaining recursive notation, 
  the term $s = \bfunap{f}{\bfunap{f}{s}{b}}{a}$ rewrites in $\omega$ many steps to 
  $t_1 = \bfunap{f}{t_1}{a}$ as well as $t_2 = \bfunap{f}{t_2}{b}$ which have no common reduct.
  The TRS $\atrs$ is weakly orthogonal (even orthogonal) but not confluent.
\end{example}
