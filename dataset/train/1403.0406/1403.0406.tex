\documentclass{tlp}

\submitted{24 September 2014}
\revised{9 February 2015}
\accepted{12 March 2015}

\usepackage{stmaryrd}
\usepackage{calc}
\usepackage{amssymb}
\usepackage{amsmath}
\usepackage{tikz}
\usetikzlibrary{shapes}
\usepackage{xspace}
\usepackage{url}
\usepackage{hyphenat}

\ifx\APPENDIX\undefined
\else
	\usepackage{xr}
	\externaldocument{article}
\fi

\newcommand{\qed}{\hspace*{1em}\hbox{\proofbox}}

\newtheorem{theorem}{Theorem}[section]
\newtheorem{corollary}[theorem]{Corollary}
\newtheorem{lemma}[theorem]{Lemma}
\newtheorem{definition}[theorem]{Definition}
\newtheorem{example}[theorem]{Example}
\newtheorem{remark}[theorem]{Remark}

\newcommand{\comment}[1]{\begin{center} \texttt{#1} \end{center}}
\newcommand{\SW}[1]{{\textcolor{red}{#1}}}
\newcommand\TTTT{\textsf{T\kern-0.2em\raisebox{-0.3em}T\kern-0.2emT\kern-0.2em\raisebox{-0.3em}2}\xspace }
\newcommand\mkbtt{\textsf{mkbtt}\xspace}
\newcommand{\eval}[1]{\mathbf{#1}}
\newcommand{\tf}[1]{{\triangledown_{\!#1}}}
\newcommand{\tfq}[1]{{\bigtriangledown_{\!#1}'}}
\newcommand{\embsm}[1]{\vartriangleright^{\!#1}_{\!\mathsf{emb}}}
\newcommand{\m}[1]{\mathsf{#1}}
\newcommand{\mc}[1]{\mathcal{#1}}
\newcommand{\mr}[1]{\mathrm{#1}}
\newcommand{\rt}{\m{root}}
\newcommand{\Wt}{\m{w,root}}
\renewcommand{\Wt}{\m{kv}}
\newcommand{\kvc}{{\m{kv'}}}
\newcommand{\lex}{\m{lex}}
\newcommand{\mul}{\m{mul}}
\newcommand{\pair}{\mathsf{mp}}
\newcommand{\FF}{\mc{F}}
\newcommand{\VV}{\mc{V}}
\newcommand{\TT}{\mc{T}}
\newcommand{\Var}{\VV\mr{ar}}
\newcommand{\AC}{\mr{\m{AC}}}
\newcommand{\ackbo}{\mr{\m{ACKBO}}}
\newcommand{\scoeff}{\mathit{sc}}
\newcommand{\actkbo}{\mr{\m{ACKBO}}^\scoeff}
\newcommand{\steinbach}{\mr{\m{S}}}
\newcommand{\acrpo}{\mr{\m{ACRPO}}}
\newcommand{\acrpoo}{\mr{\m{ACRPO'}}}
\newcommand{\KV}{\mr{\m{KV}}}
\newcommand{\KVC}{\mr{\m{KV'}}}
\newcommand{\seq}[2][n]{{#2_1},\dots,{#2_{#1}}}
\newcommand{\rr}[3][f]{{#2}{\restriction}^{#3}_{#1}}
\newcommand{\rrs}[3][f]{{#2}{\restriction}^{\smash{#3}}_{#1}}
\newcommand{\RR}{\mc{R}}
\newcommand{\Nat}{\mathbb{N}}
\newcommand{\vcoeff}{\m{vc}}
\renewcommand{\AA}{\mc A}
\newcommand{\p}[1]{\textsf{#1}}

\newcommand{\proper}{proper}
\renewcommand{\proper}{strict}

\newcommand{\High}{\m{a}}
\newcommand{\Low}{\m{b}}
\newcommand{\Bot}{\m{c}}
\newcommand{\Top}{\m{d}}
\newcommand{\ToP}{\m{e}}
\newcommand{\OO}{\mc{O}}

\newcommand{\GT}{\mathrel{\succ}}
\newcommand{\GS}{\mathrel{\succsim}}
\newcommand{\NGT}{\mathrel{\nsucc}}
\newcommand{\REL}{\mathrel{R}}




\author[Akihisa Yamada et al.]{AKIHISA YAMADA\\
Research Institute for Secure Systems, AIST, Japan
\and SARAH WINKLER\\
Institute of Computer Science, University of Innsbruck, Austria
\and NAO HIROKAWA\\
School of Information Science, JAIST, Japan 
\and AART MIDDELDORP\\
Institute of Computer Science, University of Innsbruck, Austria
}

\ifx\APPENDIX\undefined
	\title[AC-KBO Revisited]{AC-KBO Revisited\thanks{The research described in this paper is supported by
	the Austrian Science Fund (FWF) international project I963,
	the bilateral programs of the Japan Society for the Promotion of Science
	and the KAKENHI Grant No.\ 25730004.}
	\thanks{This is an extended version of a paper presented at the Twelfth 
	International Symposium on Functional and Logic Programming (FLOPS 2014), 
	invited as a rapid publication in TPLP. The authors acknowledge the assistance 
	of the conference chairs Michael Codish and Eijiro Sumii.}}
\else
	\title[Online appendix]{{\large\textnormal{Online appendix for the paper}}   \\
	AC-KBO revisited
	\\
	{\large\textnormal{published in Theory and Practice of Logic Programming}}
	}
\fi

\begin{document}
\maketitle
\ifx\APPENDIX\undefined
\ifx\ARTICLE\undefined
\noindent
{\bf Note:} This article has been accepted for publication in
\emph{Theory and Practice of Logic Programming}, \copyright\,Cambridge
University Press.\\
\enlargethispage{8ex}
\fi
\fi

\ifx\APPENDIX\undefined
\begin{abstract}
Equational theories that contain axioms expressing associativity and
commutativity (AC) of certain operators are ubiquitous. Theorem proving
methods in such theories rely on well-founded orders that are compatible
with the AC axioms.
In this paper we
consider various definitions of AC-compatible Knuth-Bendix orders.
The orders of Steinbach and of Korovin and Voronkov are revisited. The
former is enhanced to a more powerful version, and we modify
the latter to amend its lack of monotonicity on non-ground terms.
We further present new complexity results.
An extension reflecting the recent proposal of subterm coefficients
in standard Knuth-Bendix orders is also given.
The various orders are
compared on problems in termination and completion.
\end{abstract}

\begin{keywords}
Term Rewriting,
Termination,
Associative-Commutative Theory,
Knuth-Bendix Order
\end{keywords}

\section{Introduction}

Associative and commutative (AC) operators appear in many applications,
e.g.\ in automated reasoning with respect to algebraic structures 
such as commutative groups or rings.
We are interested in proving termination of term rewrite systems with AC
symbols. AC termination is important when deciding validity in equational
theories with AC operators by means of completion.

Several termination methods for plain rewriting have been
extended to deal with AC symbols. \citeN{BL87}
presented a characterization of polynomial interpretations that ensures
compatibility with the AC axioms. There have been numerous
papers on extending the recursive path order (RPO) of
\citeN{D82} to deal with AC symbols,
starting with the associative path order
of \citeN{BP85} and
culminating in the fully syntactic
AC-RPO of \citeN{R02}.
Several authors~\cite{KT01,MU04,GK01,ALM10} adapted
the influential dependency pair
method of \citeN{AG00} to AC rewriting.

We are aware of only two papers on AC extensions of the order (KBO) of 
\citeN{KB70}.
In this paper we revisit these orders and
present yet another AC-compatible KBO.
\citeN{S90} presented
a first version, which comes with the restriction that AC symbols are
minimal in the precedence. By incorporating ideas of \cite{R02},
\citeN{KV03b} presented a version without this restriction.
Actually, they present two versions. One is defined on ground terms
and another one on arbitrary terms. For (automatically) proving
AC termination of rewrite systems,
an AC-compatible order on arbitrary terms is required.\footnote{Any AC-compatible reduction order  on ground
terms can trivially be extended to
arbitrary terms by defining  if and only if
 for all grounding substitutions .
This is, however, only of (mild) theoretical interest.}
We show that the second order of \citeANP{KV03b} lacks the
monotonicity property which is required by the definition of
simplification orders.
Nevertheless we prove that the order is sound for
proving termination by extending it to an AC-compatible
simplification order.
We furthermore present a simpler variant of this latter order which
properly extends the order of~\citeN{S90}.
In particular, Steinbach's order is a correct
AC-compatible simplification order, contrary to what is claimed
in~\cite{KV03b}.
We also present new complexity results which confirm that
AC rewriting is much more involved than plain rewriting.
Apart from these theoretical contributions, we implemented the
various AC-compatible KBOs to compare them also experimentally.

The remainder of this paper is organized as follows. After
recalling basic concepts of rewriting modulo AC and orders,
we revisit Steinbach's order in Section~\ref{Steinbach}.
Section~\ref{Korovin and Voronkov} is devoted to the two orders of
Korovin and Voronkov. We present a first version of our AC-compatible KBO
in Section~\ref{AC-KBO}, 
also giving the non-trivial proof that
it has the required properties. (The proofs in~\cite{KV03b} are limited
to the order on ground terms.)
In Section~\ref{complexity} we consider the complexity of the
membership and orientation decision problems for the various orders.
In Section~\ref{AC-RPO} we compare AC-KBO with AC-RPO.
In Section~\ref{subterm coefficients} our order is strengthened
with subterm coefficients.
In order to show effectiveness of these orders
experimental data is provided in Section~\ref{experiments}.
The paper is concluded in Section~\ref{conclusion}.

This article is an updated and extended version of~\cite{YWHM14}.
Our earlier results on complexity are extended by showing that
the orientability problems for different versions of AC-KBO
are in NP. Moreover, we include a comparison with AC-RPO, which
we present in a slightly simplified manner compared to \cite{R02}.
Due to space limitations, some proofs can be found in the online appendix.

\section{Preliminaries}

We assume familiarity with rewriting and termination. Throughout this
paper we deal with rewrite systems over
a set  of variables and a \emph{finite} signature 
together with a designated subset  of binary AC symbols. The
congruence relation induced by the equations  and
 for all  is denoted by
. A term rewrite system (TRS for short)  is AC terminating if
the relation  is well-founded. In
this paper AC termination is established by
\emph{AC-compatible simplification orders} , which are
\proper\ orders (i.e., irreflexive and transitive relations)
closed under contexts and substitutions that have the subterm property
 for all  and satisfy
.
A \proper\ order  is \emph{AC-total} if
,  or , for all ground terms  and .
A pair  consisting of a preorder  and a \proper\
order  is said to be an \emph{order pair} if the \emph{compatibility}
condition  holds.


\begin{definition}
\label{lex and mul}
Let  be a \proper\ order and  be a preorder on a set .
The \emph{lexicographic extensions}  and 
are defined as follows:
\begin{itemize}
\item
 if 
for some ,
\item
\smallskip
 if 
for some .
\end{itemize}
Here , , and
 denotes the following condition:
 for all  and either 
 and  or .
The \emph{multiset extensions}  and  are
defined as follows:
\begin{itemize}
\item
 if  for some
,
\item
\smallskip
 if  for some
.
\end{itemize}
Here  if  and  consist of  
and  respectively such that  for all 
, and for every  there is some
 with .
\end{definition}

Note that these extended relations depend on both  and .
The following result is folklore;
a recent formalization of multiset
extensions in Isabelle/HOL is presented in \cite{TAN12}.

\begin{theorem}
\label{thm:order pair}
If  is an order pair then 
and  are order pairs.
\qed
\end{theorem}

\section{Steinbach's Order}
\label{Steinbach}

In this section we recall the AC-compatible KBO  of
\citeN{S90}, which reduces to the standard KBO if AC symbols are
absent.\footnote{The version in \cite{S90} is slightly more general, since non-AC
function symbols can have arbitrary status. To simplify the discussion, we
do not consider status in this paper.}
The order  depends on a precedence
and an admissible weight function.
A \emph{precedence}  is a \proper\ order on .
A \emph{weight function}  for a signature  consists of
a mapping  and a constant  such that 
 for every constant .
The \emph{weight} of a term  is recursively computed as follows:


\bigskip

\noindent
A weight function  is \emph{admissible} for  if 
every unary  with  satisfies  for all
function symbols  different from .
Throughout this paper we assume admissibility.



The \emph{top\hyp flattening} \cite{R02}
of a term  with respect to an AC symbol 
is the multiset  defined inductively as follows:


\begin{definition}
\label{def:Steinbach}
Let  be a precedence and  a weight function.
The order  is inductively defined
as follows:  if
 for all  and either , or
 and one of the following alternatives holds:
\begin{enumerate}
\setcounter{enumi}{-1}
\item
 and  for some ,
\smallskip
\item
, , and ,
\smallskip
\item
, , ,
,
\smallskip
\item
, , ,
and .
\end{enumerate}
\smallskip
The relation  is used as preorder in
 and .
\end{definition}

Cases~0--2 are the same as in the standard Knuth-Bendix order.
In case~3 terms rooted by the same AC symbol  are treated by
comparing their top\hyp flattenings in the multiset extension of
.

\begin{example}
\label{example steinbach}
Consider the signature  with
, precedence
 
and admissible weight function  with 
and . Let  be the following ground TRS:
\
\m{f}(\m{a} + \m{a}) &\to \m{f}(\m{a}) + \m{f}(\m{a})
 \tag{1}

\m{a} + \m{f}(\m{f}(\m{a})) &\to \m{f}(\m{a}) + \m{f}(\m{a})
 \tag{2}

T{\restriction}_\VV &= \{ x \in T \mid x \in \VV \} &
\rr{T}{R} &= \{ t \in T \setminus \VV \mid \rt(t) \mathrel{R} f \}

\rrs{S}{\nless}\,\REL^\mul\,\rrs{T}{\nless}
\uplus T{\restriction}_\VV - S{\restriction}_\VV

S &= \tf{+}(s) = \{ \m{f}(x), y \} &
T &= \tf{+}(t) = \{ x, y \}

\ell = \m{g}(\m{f}(\m{c}),x) >_\KV \m{g}(\m{c},\m{f}(\m{c})) = r

S &= \tf{+}(s) = \{ \ell, \m{c} \} &
T &= \tf{+}(t) = \{ r, \m{c} \}

\m{f} > {+} > \m{g} > \m{a} > \m{b} > \m{h}

w({+}) &= w(\m{h}) = 0&
w(\m{f}) &= w(\m{a}) = w(\m{b}) = w_0 = 1&
w(\m{g}) &= 2
.5ex]
 &
Example~\ref{KV does not subsume S}
\
\rrs{S}{\nless} \REL^\mul \rrs{T}{\nless} \uplus U

\rrs{S}{\nless}\sigma\, \REL^\mul\, 
\rrs{T}{\nless}\sigma \uplus U\sigma

|T'| = |T{\restriction}_\FF \sigma| + |\tf{f}(U\sigma)| +
|\tf{f}(X \sigma)| = |T| - |X| + |\tf{f}(X \sigma)|

\rrs{S'}{<} = \rrs{S}{<}\sigma \uplus \rrs{S{\restriction}_\VV\sigma}{<}

\rrs{T'}{<} = \rrs{T}{<}\sigma \uplus \rrs{T{\restriction}_\VV\sigma}{<}

w(\m{f}) &= w({\circ}) = w({\bullet}) = 0 &
w_0 &= w(\m{a}) = w(\m{c}) = 1 &
w(\m{b}) &= 2

s &= s_1 \circ \dots \circ s_n \circ \m{c} &
t &= t_1 \circ \dots \circ t_m \circ \m{d} \circ \m{d}

s_j(l) = \begin{cases}
y_j & \text{if } \\
\m{a} & \text{otherwise}
\end{cases}

t_i^+ &= \m{f}(x_i, s_1(x_i), \dots, s_m(x_i)) &
t_i^- &= \m{f}(x_i, s_1(\neg x_i), \dots, s_m(\neg x_i))
 \{ S[s_j]_i[\,]_j \} \sqsupset_{k-1}^\mul 
\{ T[\,]_j \}

\High(\Bot + \Bot) \to \High(\Bot) + \Bot \qquad
\Low(\Bot) + \Bot \to \Low(\Bot + \Bot) \qquad
\High(\Low(\Low(\Bot))) \to \Low(\High(\High(\Bot))) \\
\High(p_1(\Bot)) \to \Low(p_2(\Bot)) \qquad
\cdots \qquad
\High(p_m(\Bot)) \to \Low(\High(\Bot)) \qquad
\High(\High(\Bot)) \to \Low(p_1(\Bot))

\ell_1 &=
  \Low(\Low(\Bot + \Bot)) +
  x(\Low(\Top_i)) +
  \High(\ToP_1^0(\ToP_1^1(\Bot))) +
  y(\ToP_1^1(\ToP_1^2(\Bot))) +
  z(\ToP_1^2(\ToP_1^0(\Bot))) \\
r_1 &=
  \Low(\Bot) + \Low(\Bot) +
  \Low(x(\Top_i)) +
  \High(\ToP_1^0(\ToP_1^0(\Bot))) +
  y(\ToP_1^1(\ToP_1^1(\Bot))) +
  z(\ToP_1^2(\ToP_1^2(\Bot)))

\rrs[+]{\tf{+}(\ell_1)}{>} &~=~ \{ x(\Low(\Top_1)),
 \High(\ToP_1^0(\ToP_1^1(\Bot))),
  y(\ToP_1^1(\ToP_1^2(\Bot))),
  z(\ToP_1^2(\ToP_1^0(\Bot))) \} \\
\rrs[+]{\tf{+}(r_1)}{>} &~=~ \{ \phantom{x(\Low(\Top_1)),{}}
 \High(\ToP_1^0(\ToP_1^0(\Bot))),
  y(\ToP_1^1(\ToP_1^1(\Bot))),
  z(\ToP_1^2(\ToP_1^2(\Bot))) \}
\intertext{On the other hand, if  (i.e.,  is falsified)
then  is not satisfiable;
no matter how we assign weights to , , and ,
a term in  has the maximum weight, where
}
\rrs[+]{\tf{+}(\ell_1)}{>} &~=~ \{
  \High(\ToP_1^0(\ToP_1^1(\Bot))),
  y(\ToP_1^1(\ToP_1^2(\Bot))),
  z(\ToP_1^2(\ToP_1^0(\Bot))) \} \\
\rrs[+]{\tf{+}(r_1)}{>} &~=~ \{
  \High(\ToP_1^0(\ToP_1^0(\Bot))),
  y(\ToP_1^1(\ToP_1^1(\Bot))),
  z(\ToP_1^2(\ToP_1^2(\Bot))) \}
\intertext{However, if  (i.e.\  is falsified) then
 can be satisfied by choosing  large
enough, where
}
\rrs[+]{\tf{+}(\ell_1)}{>} &~=~ \{
  \High(\ToP_1^0(\ToP_1^1(\Bot))),
  y(\ToP_1^1(\ToP_1^2(\Bot))) \} \\
\rrs[+]{\tf{+}(r_1)}{>} &~=~ \{
  \High(\ToP_1^0(\ToP_1^0(\Bot))),
  y(\ToP_1^1(\ToP_1^1(\Bot))) \}
\intertext{Similarly, if  then  can be satisfied by
choosing  large enough, where
}
\rrs[+]{\tf{+}(\ell_1)}{>} &~=~ \{ \High(\ToP_1^0(\ToP_1^1(\Bot))) \} \\
\rrs[+]{\tf{+}(r_1)}{>} &~=~ \{ \High(\ToP_1^0(\ToP_1^0(\Bot))) \}

w(\Top_i) = 1 + 2 \cdot \max\,\{ w(\ToP_i^0), \dots, w(\ToP_i^l) \}

w(\ToP_i^m) = 1 + 2 \cdot \max\,\{ w(\ToP_i^j) \mid j \neq m \}

w(t) =
\begin{cases}
w_0 & \text{if } \\
\displaystyle w(f) + \smash[b]{\sum_{1 \leqslant i \leqslant n}}
\scoeff(f,i) \cdot w(t_i) & \text{if }
\end{cases}

\vcoeff(x,t) = \begin{cases}
1 & \text{if } \\
0 & \text{if } \\
\displaystyle \smash[b]{\sum_{1 \leqslant i \leqslant n}}
\scoeff(f,i) \cdot \vcoeff(x,t_i) & \text{if }
\end{cases}

w(t \circ (u \circ v))
&= 2 \cdot w(\circ) + m \cdot w(t) + mn \cdot w(u) + n^2 \cdot w(v)
\\
w((t \circ u) \circ v)
&= 2 \cdot w(\circ) + m^2 \cdot w(t) + mn \cdot w(u) + n \cdot w(v)
\abovedisplayskip]
\begin{minipage}{.45\textwidth}
\vspace{-\abovedisplayskip}

\end{minipage}
\hfill
\begin{minipage}{.5\textwidth}
\vspace{-\abovedisplayskip}

\end{minipage}
\-2.2ex]
membership    & P & P & P & P & NP-complete & NP-hard \
U >_\ackbo^\mul V
\quad\text{or}\quad
V >_\ackbo^\mul U
\quad\text{or}\quad 
U =_\AC^\mul V

T' - S'~ &\subseteq~
T{\restriction}_\FF\sigma \uplus U\sigma - S{\restriction}_\FF\sigma \\
&=~
T{\restriction}_\FF\sigma \uplus U\sigma \uplus X\sigma -
(S{\restriction}_\FF \uplus X\sigma) \\
&\subseteq~
T\sigma - S\sigma

S' - T'~ &\supseteq~
S{\restriction}_\FF\sigma - T{\restriction}_\FF\sigma - U\sigma \\
&=~
S{\restriction}_\FF\sigma \uplus X\sigma -
(T{\restriction}_\FF \uplus U\sigma \uplus X\sigma) \\
&\supseteq~
S\sigma - T\sigma

\tf{f}(s) - \tf{f}(t) >_\KVC^\mul \tf{f}(t) - \tf{f}(s)

\High(p_1(\Bot)) \to p_1(\High(\Bot))
\qquad\cdots\qquad
\High(p_m(\Bot)) \to p_m(\High(\Bot))

together with a rule

for each clause  that contains a negative literal.
The next property is immediate.

\begin{lemma}
\label{lem:base2}
If  then
 for all  and
 for all .
\qed
\end{lemma}

The TRS 
is denoted by .

\begin{lemma}
\label{lem:base3}
Suppose 
and the consequence of Lemma~\ref{lem:base2} holds.
Then 
for some  if and only if for every 
there is some  such that  with  or
 with .
\end{lemma}
\begin{proof}
The ``if'' direction is analogous to Lemma~\ref{lem:clause}.
Let us prove the ``only if'' direction by contradiction.
Suppose  for all ,
 for all ,
and .
As discussed in the proof of Lemma~\ref{lem:clause},
for the multisets  and  
on page~\pageref{encoding}
we obtain  and
all terms in  and  have the same weight.
With the help of Lemma~\ref{lem:base2} we infer that
 is greater than every
other term in  and .
This contradicts .
\end{proof}

Using Lemmata~\ref{lem:base2} and~\ref{lem:base3},
Theorem~\ref{ACKBO orientation NP-hard} can now be proved
in the same way as Theorem~\ref{KV orientation NP-hard}.

\subsection{AC-RPO}

\begin{proof*}[Proof of Lemma \ref{lem:acrpo}]
Because of totality of the precedence,  is
identified with  in the sequel.
First suppose  holds by case~4.
We may assume that  and  coincide on smaller
terms.
The conditions on  are obviously the same. We distinguish
which case applies.
\begin{enumerate} 
\item[4(a)]
We have 
 and thus both
 and
. So case~4(a) is applicable.
\smallskip
\item[4(b)]
We have  and , i.e.,
, and in particular 
. Thus 
 holds. 
Since  and
 imply , case~4(b) applies.
\smallskip
\item[4(c)]
We obtain  as in case~4(b).
Together with  this implies .
As 
and similar for , we obtain  from the
assumption . Hence case~4(c)
is applicable.
\end{enumerate} 
Now let  by case~4.
Again we assume that  and  coincide on smaller
terms. We have
 ().
\begin{enumerate} 
\item[4(a)]
We have . 
Suppose  , i.e., 
 does not hold.
This is only possible if there is some variable
 for which there is no
term  with . This however contradicts
(), so  holds and case~4(a)
applies.
\smallskip
\item[4(b)]
If  holds then case~4(a)
applies by
the reasoning in case~4(a).
Otherwise, due to () we must have 
. Since  implies , case~4(b)
applies.
\smallskip
\item[4(c)]
If  is satisfied we argue as in the preceding case.
Otherwise  and . This implies
both  and .
We obtain  as in case~4(b).
From the assumption  we infer
 and thus case~4(c)
applies.
\qed
\end{enumerate}
\end{proof*}
\fi

\end{document}
