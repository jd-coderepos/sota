\documentclass[11pt, oneside]{article}

\usepackage{latexsym}
\usepackage{amssymb}
\usepackage{amsmath}
\usepackage{amsthm}
\usepackage{graphicx}
\usepackage{fullpage}


\DeclareMathOperator{\arccot}{arccot}
\def\wedge{\mathcal{W}}

\newcommand*\samethanks[1][\value{footnote}]{\footnotemark[#1]}


\begin{document}
\newtheorem{problem}{Problem}
\newtheorem{definition}[problem]{Definition}
\newtheorem{proposition}[problem]{Proposition}
\newtheorem{corollary}[problem]{Corollary}
\newtheorem{remark}[problem]{Remark}
\newtheorem{lemma}[problem]{Lemma}
\newtheorem{algorithm}[problem]{Algorithm}




\title{Minimum-Area Enclosing Triangle with a Fixed Angle}
\author{Prosenjit Bose\thanks{Computational Geometry Lab, School of Computer Science, Carleton University.} \thanks{This research was partially supported by NSERC (Natural Sciences and Engineering
Research Council of Canada).}
\and Jean-Lou De Carufel\samethanks[1] \thanks{This research was partially supported by FQRNT (Fonds qu\'{e}b\'{e}cois de la recherche sur la
nature et les technologies).} }


\maketitle

\begin{abstract}
Given a set  of  points in the plane and a fixed angle , 
we show how to find in  time
all triangles of minimum area with one angle  
that enclose .
We prove that in general,
the solution cannot be written without cubic roots.
We also prove an  lower bound for this problem
in the algebraic computation tree model.
If the input is a convex -gon,
our algorithm takes  time.
\end{abstract}




\section{Introduction}
\label{section introduction}


In geometric optimization, 
the goal is often to find an optimal object 
or optimal placement of an object 
subject to a number of geometric constraints.
Examples include finding the smallest circle 
enclosing a point set~\cite{DBLP:journals/siamcomp/Megiddo83a,Welzl91smallestenclosing}
or finding the smallest circle
enclosing at least  points of a point set of  points
()~\cite{DBLP:journals/comgeo/EfratSZ94,DBLP:journals/ipl/Matousek95}.
In our setting, 
we study the following problem: 
given a set  of  points in the plane, 
find all the triangles of minimum area 
with a fixed angle , 
, 
that enclose .
When no constraint is put on the angles,
Klee and Laskowski~\cite{DBLP:journals/jal/KleeL85} 
gave an  time algorithm 
for finding the minimum-area enclosing triangle. 
This was later improved to  
by O'Rourke et al.~\cite{DBLP:journals/jal/ORourkeAMB86}.
Bose et al.~\cite{DBLP:journals/ijcga/BoseMSS11} 
provided optimal algorithms for the setting where 
one wishes to find the minimum-area isosceles triangles. 
The setting we explore here is in between the two. 
We place a restriction on the angle 
but do not insist on the triangle to be isosceles. 
Our solution, 
which we outline below,
uses ideas from the solutions of Klee and Laskowski
and Bose et al.

The five main steps of the algorithm
are presented in Sections~\ref{section overview preliminaries}, 
\ref{section optimal solution given omega wedge}, 
\ref{section walking around omega cloud},
\ref{section optimal solution apex on circular arc}
and \ref{section putting all together}.
Each section outlines one step,
proves the mathematical formulas involved
and gives its time complexity.
As we present in Section~\ref{section optimal solution apex on circular arc},
at some point,
the algorithm needs to calculate the roots of a fourth degree polynomial.
This is unfortunate when it comes to numerical robustness.
However,
in Section~\ref{section complexity solution},
we show that we cannot dodge
such algebraic expressions.
Finally,
we prove an  lower bound for this problem
in the algebraic computation tree model.



\section{Overview and Preliminaries}
\label{section overview preliminaries}






Since the solution to the general problem 
only needs to consider the vertices
of the convex hull of , 
the first step is to compute the convex hull.
In the remainder of the paper,
we assume that the input is a convex -gon
with vertices given in clockwise order. 

Let  be a convex -gon
(refer to Figure~\ref{figure algo-step0}).
\begin{figure}
\centering
\includegraphics[scale=1]{algo-step0.pdf}
\caption{In this example,
 is a quadrilateral
and .\label{figure algo-step0}}
\end{figure}
We denote the edges and the vertices of 
in clockwise order 
by  and  for 
(all index manipulation is modulo ).
Here and in the following sections,
as we present the algorithm,
we trace each step through the example of Figure~\ref{figure algo-step0}.

We begin with two definitions.
\begin{definition}[-wedge]
Let  be a point in the plane
and  be an angle 
().
Let  and  be two rays emanating from 
such that the angle between  and  is .
We say that the closed set formed by , 
, 
 
and the points between  and 
creates an \emph{-wedge},
denoted .
The point  is called the \emph{apex} of the -wedge.
An -wedge  \emph{touches} a polygon 
when  and both
 and  touch ,
i.e. 
and 
(refer to Figure~\ref{figure algo-step0}).
\end{definition}

For the rest of the paper,
when looking at an -wedge facing down,
 represents the left ray and  represents the right ray
(refer to Figure~\ref{figure algo-step0}).
Also,
when we make the -wedge turn around ,
we do it clockwise.

\begin{definition}[-cloud]
Let  be a convex -gon
and  be an angle 
().
By rotating an -wedge around 
while continually touching ,
the apex traces a sequence of circular arcs
that we call an \emph{-cloud}
(refer to Figure~\ref{figure algo-step1}).
\end{definition}

There are many technical details involved 
in the solution to this problem.
Before getting too caught up in these details,
let us first review the general approach to our solution.

Since we only consider enclosing triangles 
with an angle of , 
each optimal triangle can be constructed 
from an -wedge that touches .
Therefore, 
we consider all possible -wedges that touch .
The apices of these -wedges
lie on an -cloud  which consists 
of a linear number of pieces of circular arcs 
(refer to~\cite{DBLP:journals/ijcga/BoseMSS11}).
Then, 
for each of these -wedges,
it is possible to find the minimal triangle
by identifying a third side.
For the triangle to be optimal,
the midpoint of this third side
has to touch 
(see Proposition~\ref{proposition n-gon wedge minimal triangle} 
and Corollary~\ref{corollary n-gon wedge minimal triangle}).
Hence,
for each -wedge touching ,
there is one and only one triangle to consider for optimality.

Moreover,
when the apex  of an -wedge 
moves clockwise along the -cloud ,
the midpoint  of the third side of the optimal triangle
moves clockwise along 
(see Lemma~\ref{lemma q turns clockwise and so does m}).
As  moves,
we note the positions of 
where  leaves an edge of 
and where  enters a new edge of .
These positions of  are important event points.
They divide  into a linear number of components
(see Section~\ref{section walking around omega cloud}).
Let  be one of these components.
We prove that the minimum-area triangle having a vertex 
(hence an angle ) on 
can be computed in constant time
(see Lemmas~\ref{lemma arc fixed point minimal triangle},
\ref{lemma arc fixed line minimal triangle},
\ref{lemma vertex fixed point minimal triangle}
and~\ref{lemma vertex fixed line minimal triangle}).
We then have a linear number of candidates
(one for each piece of )
to consider rather than infinitely many
if we had to take all possible 
-wedges touching  into account.
What remains is to identify
the optimal ones
from these linear candidates.
With this in mind,
the technical details will fall into place.

\begin{description}
\item[Step 1] Compute the -cloud around 
and denote it by .
\end{description}
The -cloud  consists of  circular arcs
that we denote in clockwise order by  for .
The intersection point of  and  
is denoted by  for 
(refer to Figure~\ref{figure algo-step1}).
\begin{figure}
\centering
\includegraphics[scale=1]{algo-step1.pdf}
\caption{{\bf Step 1}:  is the -cloud of  ().\label{figure algo-step1}}
\end{figure}
We also refer to  by the closed set 
containing all the points of .
{\bf Step 1} takes  time
(refer to~\cite{DBLP:journals/ijcga/BoseMSS11}).




\section{Optimal Solution Given a Fixed -Wedge}
\label{section optimal solution given omega wedge}




In this section,
we present a routine that computes the minimum-area enclosing triangle
from a fixed -wedge touching 
(see Algorithm~\ref{algorithm minimum enclosing triangle fixed wedge}).
Then we describe {\bf Step 2}
that uses this routine.

Take any -wedge  
that touches .
Find 
(respectively )
such that  is enclosed in 
and the midpoint  of  is on .
We claim that
\begin{enumerate}
\item it is always possible to find  and  satisfying these properties,

\item  is the minimum-area triangle enclosing 
that can be constructed with ,
 
\item it takes  time to compute .
\end{enumerate}
These claims are proven 
in Proposition~\ref{proposition n-gon wedge minimal triangle},
Algorithm~\ref{algorithm minimum enclosing triangle fixed wedge},
Lemma~\ref{lemma fixed wedge minimal triangle}
and Corollary~\ref{corollary n-gon wedge minimal triangle}.

We first address the case where  is outside of .
\begin{proposition}
\label{proposition n-gon wedge minimal triangle}
Let  be a convex -gon
and  be a point outside of .
Let  be the triangle that encloses  
such that the midpoint  of  lies on .
The following is true:
\begin{enumerate}
\item\label{proposition n-gon wedge minimal triangle item exists unique}
 exists and is unique
(hence it is well-defined).

\item\label{proposition n-gon wedge minimal triangle item minimality}
Among all triangles that enclose 
and have  as a vertex,
 is the unique one of minimum area.
\end{enumerate}
\end{proposition}

Proposition~\ref{proposition n-gon wedge minimal triangle}
was first proven by Klee and Laskowski
(refer to~\cite[Lemma~1.2]{DBLP:journals/jal/KleeL85}).
We propose a slightly different proof.
Among other things,
we stress the fact that  always exists.
\proof\
\begin{enumerate}
\item Take  and  on the rays that support  from 
such that an edge  of  is flush with 
(refer to Figure~\ref{figure KleeLaskowski-1}).
\begin{figure}
\centering
\includegraphics[scale=1]{KleeLaskowski-1.pdf}
\caption{ exists.\label{figure KleeLaskowski-1}}
\end{figure}
If , 
then we are done.
If not,
suppose without loss of generality
that  is between  and .
Move  along its ray such that
 gets farther from  
and  stays tangent to .
Therefore  gets closer to .
Since  moves continuously
as  moves along its ray,
 will eventually touch an edge of .
This continuity argument implies the existence of .

Suppose there are two triangles,
namely  and 
with midpoints  and  respectively.
We show that these two triangles are equal.
There are three cases to consider:
(1) ,
(2)  and both belong to the same edge of 
or (3)  and both belong to different edges of .

\begin{enumerate}
\item[(1)]
If  is not on a vertex of ,
then it lies in the interior of an edge of 
(refer to Figure~\ref{figure KleeLaskowski-2}).
\begin{figure}
\centering
\includegraphics[scale=1]{KleeLaskowski-2.pdf}
\caption{ is not a vertex of .\label{figure KleeLaskowski-2}}
\end{figure}
Hence this aforementioned edge is in .
This implies that  and .

Suppose  is on a vertex of .
Triangles  and  are congruent by the following:
(refer to Figure~\ref{figure KleeLaskowski-3}).
\begin{figure}
\centering
\includegraphics[scale=1]{KleeLaskowski-3.pdf}
\caption{ is a vertex of .\label{figure KleeLaskowski-3}}
\end{figure}
\begin{itemize}
\item  since  is the midpoint of .

\item  since they are vertical angles.

\item  since  is the midpoint of .
\end{itemize}
Therefore 
and .
Hence we have the following:

where .
So 
because  and  are supplementaries.
Therefore 
,
which is impossible unless .
We conclude that  and .

\item[(2)] Suppose that  and they belong to the same edge of .
If neither  nor  is a vertex,
then this situation is similar to the one in Figure~\ref{figure KleeLaskowski-2},
hence ,  and .
This is a contradiction so this situation is impossible.

Without loss of generality,
suppose that  is a vertex of 
and that .
Let  be the point on the line through 
such that the line segments  and  are parallel
(refer to Figure~\ref{figure KleeLaskowski-4}).
\begin{figure}
\centering
\includegraphics[scale=1]{KleeLaskowski-4.pdf}
\caption{ is a vertex of .\label{figure KleeLaskowski-4}}
\end{figure}
Triangles  and  are similar by the following:
\begin{itemize}
\item  since they are vertical angles.

\item  since they are alternate angles.
\end{itemize}
Therefore,
since ,
then .
However,
since  is the midpoint of ,
.
This is a contradiction so this situation is impossible.

\item[(3)] Suppose  do not belong to the same edge of .
This case is similar to the case where
 and both points 
are on the same edge of .
\end{enumerate}

\item Let  be a triangle 
of minimum area that circumbscribes 
and has  as a vertex
(refer to Figure~\ref{figure KleeLaskowski-5}).
\begin{figure}
\centering
\includegraphics[scale=1]{KleeLaskowski-5.pdf}
\caption{ has minimum area.\label{figure KleeLaskowski-5}}
\end{figure}
Suppose for a contradiction
that the midpoint  of 
does not lie on one of the edges of 
(refer to Figure~\ref{figure KleeLaskowski-5}).

Since  is convex
and  circumbscribes ,
the following construction is possible.
Let  be the vertex of  closest to .
Without loss of generality, 
suppose .
Let  be a line segment that goes through 
and such that 
and  lies on the line through  and .
Let .
We have  since they are vertical angles.
If  is sufficently small,
then 
and  circumbscribes .

Therefore

which contradicts the fact that  has minimum area.
So  lies on one of the edges of 
and  is a local minimum
among triangles circumbscribing 
and having  as a vertex.
We proved in Point~\ref{proposition n-gon wedge minimal triangle item exists unique}
that there exists one and only one such triangle.
\qed
\end{enumerate}

Given the setting of Proposition~\ref{proposition n-gon wedge minimal triangle},
we show how to compute  in  time.
In Algorithm~\ref{algorithm minimum enclosing triangle fixed wedge},
the edges of  are considered in clockwise order.
The variable  indicates
the edge of 
that is currently being considered
and  represents the edge next to it in clockwise order.
Algorithm~\ref{algorithm minimum enclosing triangle fixed wedge} uses
Lemma~\ref{lemma fixed wedge minimal triangle}
that is presented
after this routine.
\begin{algorithm}(Minimum-Area Enclosing Triangle with a Fixed -Wedge)
\label{algorithm minimum enclosing triangle fixed wedge}
\begin{itemize}
\item INPUT: A convex -gon  
and an -wedge  touching 
which supports  at vertices  and 
(refer to Figure~\ref{figure KleeLaskowski-6}).
\begin{figure}
\centering
\includegraphics[scale=1]{KleeLaskowski-6.pdf}
\caption{Computing  takes  time.\label{figure KleeLaskowski-6}}
\end{figure}

\item OUTPUT: The minimum-area triangle enclosing 
that can be constructed with  .
\end{itemize}
\begin{enumerate}
\item .

\item\label{step edge intersects wedge} 
If the line through  does not intersect 
or ,
\begin{itemize}
\item .

\item Go to~\ref{step edge intersects wedge}.
\end{itemize}

\item\label{step midpoint on P}
Let  (respectively )
be the intersection point
of  (respectively )
and the line through .
If the midpoint of 
is on , 
return .

\item If the midpoint of  is between  and ,
\begin{itemize}
\item .

\item Go to~\ref{step midpoint on P}.
\end{itemize}

\item If the computation reaches this step,
it means that the midpoint of  is between  and .
\begin{itemize}
\item Place the Cartesian coordinate system on 
such that 
and  is the positive -axis
(refer to Figure~\ref{figure KleeLaskowski-6}).

\item Let .

\item Take  and 
(refer to the proof of Lemma~\ref{lemma fixed wedge minimal triangle}).

\item Return .
\end{itemize}
\end{enumerate}
\end{algorithm}
In the worst case,
Algorithm~\ref{algorithm minimum enclosing triangle fixed wedge} considers all the edges of .
Since it spends  time per edge,
it takes  time total.

\begin{lemma}
\label{lemma fixed wedge minimal triangle}
Let  and  be two lines
intersecting at 
and let  ()
be the angle between  and .
Let  be a point in 
( and ).
In  time,
we can compute 
and  such that
,  and  lie on a single line
and  is the midpoint of .
\end{lemma}

\proof
Without loss of generality,
suppose  is the -axis,
 is the line\footnote{
In what follows,
several algebraic expressions are written using 
where it seems that they could be simplified by using  instead.
However,
we write  in order to 
properly deal with the angle .} 

and let 
(refer to Figure~\ref{fig alpha-wedge}).
\begin{figure}
\centering
\includegraphics[scale=1]{alpha-wedge.pdf}
\caption{Proof of Lemma~\ref{lemma fixed wedge minimal triangle}.\label{fig alpha-wedge}}
\end{figure}
We have  and 
since  and ,
respectively.

Let  be a line containing  such that
 is not parallel to 
and  is not parallel to .
The line  is the one containing the points
 and  we are looking for.
There are two cases to consider:
(1)  is not vertical
or (2)  is vertical.

\begin{enumerate}
\item[(1)] Suppose  is not vertical.
Let  be the intersection point of  and ,
and  the intersection point of  and .
The general equation of  is .
Note that  since  is not parallel to 
and  since  is not parallel to .
Calculating the intersection point of  and ,
and of  and ,
we find that the general coordinates of  and  are

and .
We are looking for  such that ,
so we need to isolate  in

which solves to .
Therefore
 and .

\item[(2)] Suppose  is vertical.
Therefore 
,
otherwise  is not the expected line.
Using the notation of the previous case,
we get that the equation of  is .
Moreover,
 and .
We want ,
which means
 
So this situation occurs 
if and only if 
and the solution is  
and .
\end{enumerate}

Hence the global solution is
 
and 
in all cases.
\qed

Note that what we obtained
is more general than our claims.
Denote by  the triangle
computed by Proposition~\ref{proposition n-gon wedge minimal triangle}
together with Algorithm~\ref{algorithm minimum enclosing triangle fixed wedge}
and Lemma~\ref{lemma fixed wedge minimal triangle}.
Because  is outside of 
and  touches ,
not only is  the minimum-area triangle
enclosing  
that can be constructed with ,
it is also the minimum-area triangle
enclosing 
and having  as a vertex.


When  is on ,
we can compute
the minimum-area triangle
enclosing  
that can be constructed with  in the same way.
However,
in general,
it does not correspond to the minimum-area triangle
enclosing 
and having  as a vertex
(see Figure~\ref{fig algo-step2-SpecialCase}
and the discussion at the beginning of Section~\ref{section walking around omega cloud}).
\begin{corollary}
\label{corollary n-gon wedge minimal triangle}
Consider an -wedge 
touching 
such that  is on .
Then  is on one of the vertices of 
and the minimum-area triangle enclosing  
that can be constructed with 
can be computed in  time. 
\end{corollary}

Therefore,
given any fixed -wedge  touching ,
we can compute the minimum-area triangle enclosing 
that can be constructed with .
The midpoint  of the third side 
of this optimal triangle
has to be on .
In {\bf Step 3}
(see Section~\ref{section walking around omega cloud}),
we need to know the exact position of  
for one fixed -wedge touching .
However,
it does not matter
for what -wedge touching 
we know the position of ,
as long as we know it for one.
Therefore, 
in {\bf Step 2},
we fix an arbitrary -wedge  touching 
and we compute the position of .

\begin{description}
\item[Step 2] Let  be such that .
Consider the -wedge  
that touches 
(refer to Figure~\ref{figure algo-step2}).
\begin{figure}
\centering
\includegraphics[scale=1]{algo-step2.pdf}
\caption{{\bf Step 2}: 
 is constructed with 
where .\label{figure algo-step2}}
\end{figure}
Apply Algorithm~\ref{algorithm minimum enclosing triangle fixed wedge}
with  and .
\end{description}

Consider Figure~\ref{figure algo-step1}.
Any -wedge touching  
and having its vertex in 
is such that 

and .
This property easily generalizes to the case 
where the apex of the -wedge is on  or on .
Therefore,
when we specify  and ,
we imply that

and ,
and hence the -wedge is unique.
However,
in the example of Figure~\ref{figure algo-step2},
there are infinitely many other -wedges touching 
and having  as a vertex to consider
because  is on a vertex of 
(indeed ).
The algorithm will consider these other -wedges later
(see Section~\ref{section walking around omega cloud}).

From Proposition~\ref{proposition n-gon wedge minimal triangle},
Algorithm~\ref{algorithm minimum enclosing triangle fixed wedge},
Lemma~\ref{lemma fixed wedge minimal triangle}
and Corollary~\ref{corollary n-gon wedge minimal triangle},
{\bf Step 2} takes  time.




\section{Walking Around the -Cloud}
\label{section walking around omega cloud}




In the previous section,
we computed the minimum-area triangle
enclosing 
that can be constructed with a given -wedge
.
The next step is to consider
all possible -wedges touching 
and having their apex on .
For each of these -wedges,
we can compute the minimum-area triangle enclosing .
The final solution is the minimum among all these triangles.
Of course,
there are infinitely many such triangles.
However,
we show that we need to consider 
only  of these triangles.

In what follows,
we consider the points of  
and the related -wedges
in clockwise order.
When thinking of all possible -wedges touching 
and having their apex on ,
one must pay attention to the following case.
If  is the endpoint of one of the circular arcs  of ,
and if at the same time this endpoint is one of the vertices  of ,
then there are infinitely many -wedges to consider
(refer to Figure~\ref{fig algo-step2-SpecialCase}).
\begin{figure}
\centering
\includegraphics[scale=1]{algo-step2-SpecialCase.pdf}
\caption{-wedge turning clockwise around a corner.\label{fig algo-step2-SpecialCase}}
\end{figure}
This situation occurs when  is greater 
than the angle between the two edges  and .
Then the -wedge turns
around the corner created by  and 
while its apex  stays on the vertex .

The following lemma
describes the behaviour of 
(the midpoint of  constructed in {\bf Step 2})
as  moves clockwise along .
Some hypotheses of this lemma
are written using the expression ``close enough''.
Its meaning is twofold.
First,
as an -wedge  touching 
moves continuously
such that its apex  stays on ,
the midpoint  of  
also moves continuously.
Secondly,
let  be an edge of .
If ,
it is possible to move  
such that  stays on .
Such a displacement of  has to be 
of less than  for an 
and such an  always exists.

What needs to be shown
is that  moves clockwise along .
\begin{lemma}
\label{lemma q turns clockwise and so does m}
As an -wedge  touching  moves clockwise
such that its apex  stays on ,
the midpoint  of  
---the third edge of the minimum-area triangle enclosing 
that can be constructed with ---
moves clockwise along .

Specifically,
take 
with  clockwise from .
Denote by 
(respectively by )
the midpoint of the third side of the minimum-area triangle
associated with the -wedge touching  with apex 
(respectively with apex ).
\begin{enumerate}
\item\label{lemma q turns clockwise and so does m item q' neq q''} 
If  (),
,
and  and  are close enough to each other,
then we have the following:
\begin{enumerate}
\item\label{lemma q turns clockwise and so does m item q' neq q'' item m' neq m''}  
If  (),
then .

\item If  (),
then 
(possibly ).
\end{enumerate}

\item\label{lemma q turns clockwise and so does m item q' = q'' = ui}
Suppose  ()
and  ().
Let  and  be two different -wedges
touching  with  as an apex.
If  is clockwise from 
(refer to Figure~\ref{fig algo-step2-SpecialCase})
and  and  are close enough to each other,
then we have the following:
\begin{enumerate}
\item If  (),
then .

\item If  (),
then 
(possibly ).
\end{enumerate} 
\end{enumerate}
\end{lemma}

\proof Without loss of generality,
 is on the -axis
(refer to Figure~\ref{figure clockwise}).
\begin{figure}
\centering
\includegraphics[scale=1]{clockwise.pdf}
\caption{Proof of Lemma~\ref{lemma q turns clockwise and so does m}.\label{figure clockwise}}
\end{figure}
\begin{enumerate}
\item
\begin{enumerate}
\item
If  and  are close enough to each other,
then .
Therefore,
,
,

and ,
for 
such that .
Hence,
,
so .

\item
If  and  are close enough to each other,
then the only situation to discard is .
Thus,
suppose  for a contradiction.
Therefore,
 and  both belong to .
Hence,
an argument similar to the one of the previous case leads to , 
which is a contradiction.
\end{enumerate}

\item The proof is similar to the one of Point \ref{lemma q turns clockwise and so does m item q' neq q''}.
\qed
\end{enumerate}

Therefore,
the midpoint  of the third side of the triangle
is either a vertex of 
or on an edge of .
The goal of {\bf Step 3} is to identify 
the sections of 
where the midpoint is a vertex
and the sections of 
where the midpoint is on an edge of .

\begin{description}
\item[Step 3] Move the apex  of the -wedge
clockwise along .
Maintain ,  and 
as defined in {\bf Step 2}
(refer to Section~\ref{section optimal solution given omega wedge}).
Collect all of the following three types of event points
(refer to Figure~\ref{figure algo-step3}).
\begin{figure}
\centering
\includegraphics[scale=1]{algo-step3.pdf}
\caption{{\bf Step 3}:
 is such that .
Note that  gives birth to two different event points,
namely  and .
They correspond to the -wedge touching 
and having  as a vertex,
and the -wedge touching 
and having  as a vertex.
 means that ,

and .
It is an event point of the first type.
At such a place, 
 does not move even though  does.
 means that
 and .
It is an event point of the second type.
 means that
 and .
It is an event point of the third type.\label{figure algo-step3}}
\end{figure}
\begin{description}
\item[Type 1]
 is on the intersection point of two consecutive circular arcs of  
for an  with .
Formally,
 for an  with .

\item[Type 2]
 is such that the third side  of the triangle
is on an edge  of 
(for an  with )
and the midpoint  of 
is on the first vertex  of 
(when the vertices of  are considered in clockwise order).
Formally,
 is such that  and  
for an  with .

\item[Type 3]
 is such that the third side  of the triangle
is on an edge  of 
(for an  with )
and the midpoint  of 
is on the last vertex  of 
(when the vertices of  are considered in clockwise order).
Formally,
 is such that  and  
for an  with .
\end{description}

For each event point ,
save its type together with the location of ,
the midpoint of the third side of the triangle.
\end{description}

It is easy to find the event points of the first type
and the following two lemmas
show how to find the event points 
of the second and third type.
Lemma~\ref{lemma arc fixed midpoint on a line}
helps identifying event points related
to Case~\ref{lemma q turns clockwise and so does m item q' neq q''}
of Lemma~\ref{lemma q turns clockwise and so does m}.
As for Lemma~\ref{lemma vertex fixed midpoint on a segment},
it helps identifying event points related
to Case~\ref{lemma q turns clockwise and so does m item q' = q'' = ui}
of Lemma~\ref{lemma q turns clockwise and so does m}.
\begin{lemma}
\label{lemma arc fixed midpoint on a line}
Let  be an arc of a circle,
 be a line
and  be a point.
It is possible to find a triangle  
such that ,

and  is the midpoint of 
(or decide that there is no such point)
in  time
(refer to Figure~\ref{arc-circle-segment-point-proof}).
\end{lemma}

\proof
Without loss of generality,
 is the locus of the point 
such that .
Hence we can take
,
 with 
and ,
where  is the radius of .
Let ,

and  be any point of 
(refer to Figure~\ref{arc-circle-segment-point-proof}).
\begin{figure}
\centering
\includegraphics[scale=1]{arc-circle-segment-point-proof.pdf}
\caption{Proof of Lemma~\ref{lemma arc fixed midpoint on a line}.\label{arc-circle-segment-point-proof}}
\end{figure}
There are three cases to consider:
(1) either  has strictly negative slope,
(2)  has non-negative slope
(3) or  is vertical.

\begin{enumerate}
\item[(1)] Suppose  has strictly negative slope.
The general equation for  is  ()
and hence .
Note .
Also let  
(respectively )
be the line through  and 
(respectively through  and ).
Let  
(respectively )
be the intersection point of  and 
(respectively of  and ).

Trivially ,
but there are other restrictions.
The point  must be strictly to the right of ,
so .
Also,  must be strictly to the left of , 
so .
Finally,
 must not reach the point
where the line through  and 
is parallel to .
Hence .
To summarize,
we have


By elementary trigonometry,
we get

Then,
finding the general equation of the line through  and ,
and of the line through  and ,
we can calculate the general coordinates of  and :

Therefore we want to find  such that
 is the midpoint of ,
which leads to the equation

or

Since this is a quadratic equation in ,
it is solvable in constant time 
for  satisfying (\ref{lemma arc fixed midpoint on a line item constraint 1})
and (\ref{lemma arc fixed midpoint on a line item constraint 2})
(or it is possible to decide 
that there is no such solution in constant time).

\item[(2)] The case where  has non-negative slope
is similar to the case where  has strictly negative slope.

\item[(3)] Suppose  is vertical.
Therefore the general equation for  is .
We must have
,
otherwise there is no solution.
There are three subcases to consider:
(3.1) ,
(3.2) 
(3.3) or .

\begin{enumerate}
\item[(3.1)] Suppose  is vertical and .
Trivially ,
but there are other restrictions.
With the notation of the previous cases,
 must be between  and ,
therefore .
The point  must be strictly above 
so .
Also,
 must be strictly under ,
so .
It all sums up to


By elementary trigonometry,
we get

Then,
finding the general equation of the line through  and ,
and of the line through  and ,
we can calculate the general coordinates of  and :

Therefore we want to find  such that
 is the midpoint of ,
which leads to the equation

or

Since this is a quadratic equation in ,
it is solvable in constant time 
for  satisfying (\ref{lemma arc fixed midpoint on a line item constraint 5})
and (\ref{lemma arc fixed midpoint on a line item constraint 6})
(or it is possible to decide 
that there is no such solution in constant time).

\item[(3.2)] Suppose  is vertical and .
In this case there is no solution.

\item[(3.3)] The case where  is vertical and 
is similar to the case where  is vertical and .
\qed
\end{enumerate}
\end{enumerate}


\begin{lemma}
\label{lemma vertex fixed midpoint on a segment}
Let  be a point,
 be a line
and  be a point.
It is possible to find the triangle 
such that ,

and  is the midpoint of 
in  time
(refer to Figure~\ref{sommet-segment-point-proof}).
\end{lemma}

\proof
Without loss of generality,
 is the -axis,
,
 with 
and  is to the left of .
Denote by 
the center of the circumcircle of .

Suppose .
By elementary geometry,
 lies on the line segment bisector of 
and 
(refer to Figure~\ref{sommet-segment-point-proof}).
\begin{figure}
\centering
\includegraphics[scale=1]{sommet-segment-point-proof.pdf}
\caption{Proof of Lemma~\ref{lemma vertex fixed midpoint on a segment}.\label{sommet-segment-point-proof}}
\end{figure}
Hence  for an .
Therefore,
the equation of the circumcircle is\footnote{We need
the assumption  here,
otherwise  is undefined.}

Since  is on this circle,
then

so

from which


If ,
then  and the equation of the circumcircle is .
Therefore,

\qed

There are  event points of the first type.
There are at most  event points of the second type
because there are  edges.
For the same reason,
there are at most  event points of the third type.
Therefore,
{\bf Step 3} 
(computing the event points)
takes  time.




\section{Finding the Optimal Solution when the Apex is on a Circular Arc}
\label{section optimal solution apex on circular arc}




Between any two consecutive event points,
there is a single arc of a circle
(that might be reduced to a single point
if one of the event points is on one of the vertices of ).
Consider such an arc.
As  moves along it,
the minimum-area triangle enclosing  changes.
In this section,
we show how to compute the optimal triangle
for a fixed arc of a circle
between two consecutive event points.
Four different cases can occur:
\begin{itemize}
\item either the circular arc is reduced to a single point or not.
If the former is true,
then the apex  is forced to stay on one of the vertices of .

\item Either the midpoint  is forced to stay on a vertex of  or not.
\end{itemize}
The following lemmas describe how to compute 
the minimum-area triangle enclosing 
in these four different situations.

Lemma~\ref{lemma arc fixed point minimal triangle}
describes how to compute the minimum-area enclosing triangle
when  moves along an arc of a circle
and  is forced to stay on one of the vertices of .
\begin{lemma}
\label{lemma arc fixed point minimal triangle}
Let  be an arc of a circle
and  be a point.
It is possible to find the triangle 
of minimum area such that ,
,  and  lie on the same line,
,  and  lie on the same line,
and 
in  time
(refer to Figure~\ref{arc-circle-point-minima-proof}).
\end{lemma}

\proof
The strategy is to first fix a point  on 
and then follow the proof of Lemma~\ref{lemma fixed wedge minimal triangle}
in order to construct the minimum-area triangle for this fixed .
Then we move  along 
while maintaining the minimum-area triangle.
The smallest one among all these minimal triangles is optimal.

Without loss of generality,
 is the locus of the point 
such that .
Hence we can take
 and 
where  is the radius of .
Let  be any point on 
(refer to Figure~\ref{arc-circle-point-minima-proof}).
\begin{figure}
\centering
\includegraphics[scale=1]{arc-circle-point-minima-proof.pdf}
\caption{Proof of Lemma~\ref{lemma arc fixed point minimal triangle}.\label{arc-circle-point-minima-proof}}
\end{figure}
Note ,
,

(),

( and )
and .
Let  be the orthogonal projection of 
onto the line through  and .
Let  be the orthogonal projection of 
onto the line through  and .
Let  be the tangent to  at .
Finally, 
note  and .

By elementary trigonometry,
we get

By elementary trigonometry and geometry,
we get


With respect to the proof of Lemma~\ref{lemma fixed wedge minimal triangle},
let  (respectively )
be the point on the line 
(respectively on the line )
such that 
(respectively ).
Therefore  
is of minimum area for this fixed 
(and therefore for this fixed )
by Lemma~\ref{lemma fixed wedge minimal triangle}.

We need to study how the area of  evolves 
as  moves along .
Trivially ,
but there are other restrictions.
Since  must not be on the line through ,
.
Also,
 must not be on the line through ,
so .
Finally,
we must have  and ,
so 
and .
It all sums up to

By the proof of Lemma~\ref{lemma fixed wedge minimal triangle},
the area of  
as a function of  is

With the sine law applied on ,
one gets

Therefore,


We have

In order to find the candidates for minimum and maximum of ,
we need to solve

for  satisfying (\ref{lemma arc fixed point minimal triangle constraint 1})
and (\ref{lemma arc fixed point minimal triangle constraint 2}).

We look for solutions to 
for .
At first sight,
we have ,
but we need to study this general solution
in order for it to satisfy (\ref{lemma arc fixed point minimal triangle constraint 1})
and (\ref{lemma arc fixed point minimal triangle constraint 2}).
Putting this solution against these constraints,
we get

which simplify to

If ,
(\ref{lemma arc fixed point minimal triangle constraint 8}) and (\ref{lemma arc fixed point minimal triangle constraint 7})
lead to ,
which is a contradiction.
If ,
(\ref{lemma arc fixed point minimal triangle constraint 8}) and (\ref{lemma arc fixed point minimal triangle constraint 7})
lead to ,
which is a contradiction.
If ,
then 
is a valid solution provided that


In this case,
 is a maximum as we show below.

and

so  is a maximum.

Otherwise,
(\ref{lemma arc fixed point minimal triangle equation 1}) has no solution,
therefore  is monotonic.
The conclusion is that
the minimum of 
is at one (or both) of the extremities
of the domain of .
\qed

Looking at the proof of Lemma~\ref{lemma arc fixed point minimal triangle},
it suggests that the minimum-area triangle might not exist.
Indeed,
if 
and ,
then 
by (\ref{lemma arc fixed point minimal triangle constraint 1})
and (\ref{lemma arc fixed point minimal triangle constraint 2}).
But since the minimum of 
is at one (or both) of the extremities
of the domain of ,
then it does not exist.
However,
in the setting of the general problem,
this does not occur.
By Proposition~\ref{proposition n-gon wedge minimal triangle},
the minimum-area triangle always exists
for a given -wedge
so  will always vary inside an interval
that includes its extremities.

Lemma~\ref{lemma arc fixed line minimal triangle}
describes how to compute the minimum-area enclosing triangle
when  moves along an arc of a circle
and  is forced to stay on one of the edges of .
\begin{lemma}
\label{lemma arc fixed line minimal triangle}
Let  be an arc of a circle
and  be a line.
It is possible to find the point 
such that the line through ,
the line through 
and 
form a triangle of minimum area
in  time
(refer to Figure~\ref{arc-circle-segment-minima-proof}).
\begin{figure}
\centering
\includegraphics[scale=1]{arc-circle-segment-minima-proof.pdf}
\caption{Proof of Lemma~\ref{lemma arc fixed line minimal triangle}.\label{arc-circle-segment-minima-proof}}
\end{figure}
\end{lemma}

The proof of Lemma~\ref{lemma arc fixed line minimal triangle}
contains several technical details.
It is presented in~\ref{appendix proof lemma}.
Lemma~\ref{lemma vertex fixed point minimal triangle}
describes how to compute the minimum-area triangle enclosing 
when  is forced to stay on one of the vertices of 
and  is forced to stay on another one of the vertices of .
\begin{lemma}
\label{lemma vertex fixed point minimal triangle}
Let  and  be two points.
It is possible to find the triangle 
of minimum area such that 
 is the midpoint of 
and 
in  time.
\end{lemma}

\proof Let  be a point such that 
and ,  and  are aligned
(refer to Figure~\ref{sommet-point-minima-proof-2}).
\begin{figure}
\centering
\includegraphics[scale=1]{sommet-point-minima-proof-2.pdf}
\caption{Proof of Lemma~\ref{lemma vertex fixed point minimal triangle}.\label{sommet-point-minima-proof-2}}
\end{figure}
Let  and  be any points such that 
 and ,  and  are aligned.
By construction,
the quadrilateral  is a parallelogram,
,

and the areas of  and  are equal.
Moreover,
 is on a circular arc
that is the locus of the point 
such that .

Since  is fixed and the areas of  and  are equal,
it suffices to minimize the height of  relative to .
The solution is to take  as close as possible to .
\qed

Lemma~\ref{lemma vertex fixed line minimal triangle}
describes how to compute the minimum-area enclosing triangle
when  is forced to stay on one of the vertices of 
and  is forced to stay on one of the edges of .
\begin{lemma}
\label{lemma vertex fixed line minimal triangle}
Let  be a point
and  be a line.
The triangle 
of minimum area such that 

and 
is isosceles
and can be found
in  time.
\end{lemma}


\proof
Let  be any triangle such that ,

and  is to the left of .
We prove that the area of  is bigger than or equal to the area of the isosceles triangle .
Without loss of generality,
assume that  is to the left of .

Let  be the projection of  onto .
From elementary geometry,
as a point  gets closer to ,
 decreases.
Conversely,
as  gets farther from ,
 increases.
Therefore,
since the area of  is ,
we may assume that 
(refer to Figure~\ref{sommet-droite-minima-proof-2}).
\begin{figure}
\centering
\includegraphics[scale=1]{sommet-droite-minima-proof-2.pdf}
\caption{Proof of Lemma~\ref{lemma vertex fixed line minimal triangle}.\label{sommet-droite-minima-proof-2}}
\end{figure}
Note .
Therefore,

and .
Thus,
it is sufficient to prove that the area of 
is bigger than or equal to the area of .
We have

which completes the proof.
\qed


\begin{description}
\item[Step 4] For each pair of consecutive event points,
compute the minimum-area triangle enclosing 
and having a vertex on the corresponding arc of a circle.
\end{description}

Since there are  event points,
{\bf Step 4} takes  time.




\section{Putting it All Together}
\label{section putting all together}




The vertex that subtends the angle  
of the optimal triangle enclosing 
has to be on the -cloud .
 was computed in {\bf Step 1}
in  time
(refer to Section~\ref{section overview preliminaries}).
In {\bf Step 2}
(refer to Section~\ref{section optimal solution given omega wedge}),
we fixed an -wedge 
and computed the minimum-area triangle enclosing 
that can be constructed with 
in  time.
This triangle is such that
the midpoint  of its third side is on 
(refer to Proposition~\ref{proposition n-gon wedge minimal triangle} 
and Corollary~\ref{corollary n-gon wedge minimal triangle}).
Then we divided  into a linear number of pieces
in {\bf Step 3}
(refer to Section~\ref{section walking around omega cloud}).
Within each of these pieces,
an optimal triangle can be computed in  time.
This was done in {\bf Step 4}
(refer to Section~\ref{section optimal solution apex on circular arc}).

\begin{description}
\item[Step 5] Find the smallest area triangles among those calculated 
in {\bf Step 4}.
\end{description}

Since there are  event points,
{\bf Step 5} takes  time.
Since each step takes no longer than  time,
the algorithm takes  time.
If the input is a set of points,
compute the convex hull
and then apply this algorithm.
In this situation,
the computation takes  time
because of the computation of the convex hull.

We summarize the final algorithm.
\begin{algorithm}(Minimum-Area Triangle with a Fixed Angle Enclosing )
\label{algorithm minimum enclosing triangle fixed angle}
\begin{itemize}
\item INPUT: A finite set  of points in the plane
and an angle .

\item OUTPUT: All triangles with minimum area 
having an angle of 
that enclose .
\end{itemize}
\begin{enumerate}
\item[0.] Compute the convex hull of  and denote it by .
We denote the edges and the vertices of 
in clockwise order 
by  and  for 
(all index manipulation is modulo ).

\item Compute the -cloud around 
and denote it by 
(refer to Section~\ref{section overview preliminaries}).
 consists of  circular arcs
that we denote in clockwise order by  for .
The intersection point of  and  
is denoted by  for .

\item\label{step minimum enclosing triangle fixed omega-wedge}  
Let  be such that .
Consider the -wedge  
that touches .
Apply Algorithm~\ref{algorithm minimum enclosing triangle fixed wedge}
with  and 
(refer to Section~\ref{section optimal solution given omega wedge}).
Let  be the output 
of Algorithm~\ref{algorithm minimum enclosing triangle fixed wedge}
and denote by  the midpoint of segment .

\item\label{step event points} 
Move  clockwise along 
and maintain , ,  and 
as defined in~\ref{step minimum enclosing triangle fixed omega-wedge}.
Collect all of the following three types of event points
(see Section~\ref{section walking around omega cloud}
for formal definition):
\begin{description}
\item[Type 1]
 is on the intersection point of two consecutive circular arcs of  
for an  with .

\item[Type 2]
 is such that the third side  of the triangle
is on an edge  of 
(for an  with )
and the midpoint  of 
is on the first vertex  of 
(when the vertices of  are considered in clockwise order).

\item[Type 3]
 is such that the third side  of the triangle
is on an edge  of 
(for an  with )
and the midpoint  of 
is on the last vertex  of 
(when the vertices of  are considered in clockwise order).
\end{description}

\item\label{step minimum enclosing triangle between two event points}
For each pair of consecutive event points
computed in~\ref{step event points},
which we denote by ,
there is a single arc of a circle.
This circular arc 
might be reduced to a single point
if two consecutive event points 
are on one of the vertices of .
When  moves on such an arc,
either  is forced to stay 
on one of the vertices of 
or  is forced to stay on one of the edges of 
(see Sections~\ref{section overview preliminaries}
and~\ref{section walking around omega cloud}
for complete discussion).
For each pair  of consecutive event points,
\begin{itemize}
\item if the circular arc between  and  
is not reduced to a single point
and if  is forced to stay 
on one of the vertices of ,
store the triangle defined by Lemma~\ref{lemma arc fixed point minimal triangle}
(refer to Section~\ref{section optimal solution apex on circular arc}).

\item If the circular arc between  and  
is not reduced to a single point
and if  is forced to stay 
on one of the edges of ,
store the triangle defined by Lemma~\ref{lemma arc fixed line minimal triangle}
(refer to Section~\ref{section optimal solution apex on circular arc}).

\item If the circular arc between  and  
is reduced to a single point
and if  is forced to stay 
on one of the vertices of ,
store the triangle defined by Lemma~\ref{lemma vertex fixed point minimal triangle}
(refer to Section~\ref{section optimal solution apex on circular arc}).

\item If the circular arc between  and  
is reduced to a single point
and if  is forced to stay 
on one of the edges of ,
store the triangle defined by Lemma~\ref{lemma vertex fixed line minimal triangle}
(refer to Section~\ref{section optimal solution apex on circular arc}).
\end{itemize}

\item Return the smallest area triangles among those stored
in~\ref{step minimum enclosing triangle between two event points}.
\end{enumerate}
\end{algorithm}




\section{A Note on the Complexity of the Solution}
\label{section complexity solution}



Notice that one of the cases 
in Section~\ref{section optimal solution apex on circular arc} 
required us to find the roots of a fourth degree polynomial
(refer to Lemma~\ref{lemma arc fixed line minimal triangle}). 
One may ask whether or not 
this is necessary or if one can avoid 
finding the roots of such a polynomial
to solve this problem.
In this section, 
we show that it is unavoidable in certain situations, 
by providing an example of a set of points 
where the optimal solution lies on the root 
of an irreducible fourth degree polynomial.

Consider the quadrilateral  where
,
,

and ,
and take 
(hence we are looking for the minimum-area enclosing right triangle).
It turns out that the optimal right triangle
is such that the right angle is on 
(refer to Figure~\ref{figure quartic})
\begin{figure}
\centering
\includegraphics[scale=1]{quartic.pdf}
\caption{Example where the optimal solution involves the roots of a quartic equation.\label{figure quartic}}
\end{figure}
and the hypotenuse is on .
Therefore, 
we need to solve (\ref{lemma arc fixed line minimal triangle equation 2})
with ,

and .
It leads to the following quartic equation.

This equation admits two real solutions  and .
Note 

then

Solving 
and ,
then testing against 
(refer to the proof of Lemma~\ref{lemma arc fixed line minimal triangle}),
one finds that 
minimizes .

Since  and its resolvent cubic

are irreducible over ,
the algebraic expressions for the roots of  must be written 
with square roots and cubic roots.
Moreover,
they cannot be simplified (see~\cite{dummit}).

Therefore, 
in general, 
we cannot avoid square roots
nor cubic roots in
the computation of the minimum-area enclosing triangle
with a fixed angle.

Finally,
we prove an  time lower bound for this problem
in the algebraic computation tree model.
We use a reduction to Max-Gap problem for points in the first quadrant
of the unit circle (see~\cite{DBLP:journals/algorithmica/LeeW86}).
Our proof is constructed upon the proof of Theorem 5 in~\cite{DBLP:journals/ijcga/BoseMSS11}.
We first need the following lemma.
\begin{lemma}
\label{lemma smallest triangle circle}
The smallest area triangle enclosing a circle is an equilateral triangle.
\end{lemma}

\proof Let  be a triangle
and let  be the radius of its incircle.
Then the area of  is

which is minimized when .
\qed

\begin{proposition}
Given a set  of  points and an angle ,
computing the triangle of minimum area with angle  
that encloses 
requires  operations in the algebraic computation tree model.
\end{proposition}

\proof
As in the proof of Theorem 5 in~\cite{DBLP:journals/ijcga/BoseMSS11},
let  be an instance of the Max-Gap
problem for points in the first quadrant of the unit circle centered at the origin of
the coordinates system,
where ,
for .
We also define the set of points 
as in~\cite{DBLP:journals/ijcga/BoseMSS11}.

Let  be the equilateral triangle in the proof of Theorem 5 in~\cite{DBLP:journals/ijcga/BoseMSS11}.
We prove that 
is the minimum-area triangle that encloses .
Then,
the result follows since
the triangle of minimum area with angle 
that encloses  must be .

Consider the convex hull of .
By construction,
the smallest distance from the origin to an edge is
the distance between the origin and the edge that represents the maximum gap.
Therefore,
the incircle of  is also enclosed in the convex hull of .
Let  be any other triangle enclosing .
Therefore,
it encloses the convex hull of  and hence,
the incircle of .
By Lemma~\ref{lemma smallest triangle circle},
the area of  cannot be smaller than the area of .
\qed

With a similar argument,
we can prove the following lemma.
\begin{lemma}
\label{lemma lower bound ORourke}
Given a set  of  points,
computing the triangle of minimum area that encloses 
requires  operations in the algebraic computation tree model.
\end{lemma}

In~\cite{DBLP:journals/jal/ORourkeAMB86},
O'Rourke et al. show how to compute in  time
the minimum-area triangle enclosing a convex -gon,
which is optimal.
If the input is a set  of  points,
then the following strategy is optimal
by Lemma~\ref{lemma lower bound ORourke}.
Compute the convex hull  of 
and then apply O'Rourke et al.'s algorithm on .



\section{Conclusion}
\label{section conclusion}


We have shown how to compute 
all triangles of minimum area 
with a fixed angle  
that enclose a point set.
It would be interesting to see 
if some of these techniques generalize 
to other settings or other optimization criteria.
For example,
finding the smallest area tetrahedron with a fixed solid angle
of a set of points in three dimensions.



\bibliographystyle{alpha}
\bibliography{MinimumEnclosingAreaTriangleFixedAngle}

\appendix

\section{Proof of Lemma~\ref{lemma arc fixed line minimal triangle}}
\label{appendix proof lemma}

In this section,
we prove Lemma~\ref{lemma arc fixed line minimal triangle}.

\proof
Without loss of generality,
 is the locus of the point 
such that .
Hence we can take
,
and 
where  is the radius of .
Let  be any point on .
There are three cases to consider:
(1)  has strictly negative slope,
(2)  has non-negative slope
or (3)  is vertical.

\begin{enumerate}
\item[(1)] Suppose  has strictly negative slope.
The general equation for  is  ().
Note .
Also note  
(respectively )
the line through  and 
(respectively through  and ).
Note  
(respectively )
the intersection point of  and 
(respectively of  and ).

Trivially ,
but there is one more restriction.
The point  must be between  and ,
hence .
It all sums up to


By elementary trigonometry,
we get

Then,
finding the general equation of the line through  and ,
and of the line through  and ,
we can calculate the general coordinates of  and :

Let 
(or  for short).
Therefore,

Here is the area  of :

The reason why the absolute value  disappeared
is twofold.
Firstly,
by (\ref{lemma arc fixed line minimal triangle constraint 2}),
,
so .
Secondly,
by (\ref{lemma arc fixed line minimal triangle constraint 2}),
,
therefore



Now we need to find for what values
of  is  minimum,
which means that we need to find for what 
does .

Therefore  if and only if

or

The first equation is quadratic in ,
therefore it can be solved in constant time for 
and then it remains to solve for .
The second equation is quartic in ,
therefore it can be solved\footnote{By Galois theory,
polynomials of degree  can be solved exactly in constant time.
Refer to~\cite{dummit}} 
in constant time for 
and then it remains to solve for .

Actually,
what we are interested in is .
However,

and .
Finally,
the extremities of the domain of 
are also candidates.

Overall, 
we get at most eight candidates for the minimum of 
(at most two from the quadratic equation, 
at most four from the quartic equation
and the extremities of the domain of ).
So it can be solved exactly in constant time 
by taking the smallest of the eight candidates.

\item[(2)] The case where  has non-negative slope
is similar to the case where  has strictly negative slope.

\item[(3)] Suppose  is vertical.
The general equation of  is 
for .

otherwise there is no solution.
Using the notation of the previous cases,
there are three subcases to consider:
(3.1) ,
(3.2) 
or (3.3) .

\begin{enumerate}
\item[(3.1)] Suppose  is vertical and .
Trivially ,
but there is one more restriction.
The point  must be between  and ,
hence .
It all sums up to


By elementary trigonometry,
we get

Then,
finding the general equation of the line through  and ,
and of the line through  and ,
we can calculate the general coordinates of  and :

We note 
(or  for short).
Therefore,

Here is the area  of .


The reason why the absolute value  disappeared
is twofold.
Firstly,
by the hypothesis,
,
so .
Secondly,
by (\ref{lemma arc fixed line minimal triangle constraint 8}),
,
therefore


Now we need to find for what values
of  where  is minimum,
which means that we need to find for which 
does .

Therefore  if and only if

or

The first equation is quadratic in ,
therefore it can be solved in constant time for 
and then it remains to solve for .
The second equation is quartic in ,
therefore it can be solved
in constant time for 
and then it remains to solve for .

Actually,
what we are interested in is .
However,

and .
Finally,
the extremities of the domain of 
are also candidates.

Overall, 
we get at most eight candidates for the minimum of 
(at most two from the quadratic equation, 
at most four from the quartic equation
and the extremities of the domain of ).
So it can be solved exactly in constant time 
by taking the smallest of the eight candidates.

\item[(3.2)] Suppose  is vertical and .
In this case there is no solution.

\item[(3.3)] The case where  is vertical and 
is similar to the case where  is vertical and .
\qed
\end{enumerate}
\end{enumerate}


\end{document}
