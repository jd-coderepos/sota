\pdfoutput = 1
\documentclass[11pt]{article}
\usepackage{url}
\usepackage{color}
\usepackage{graphicx}
\usepackage{amsfonts}
\usepackage{amsmath}
\usepackage{amssymb}
\usepackage{latexsym}
\usepackage{rotating}

\usepackage{colortbl}
\usepackage[table]{xcolor}
\usepackage{multirow}
\newcommand\open{\textcolor{blue}{{\bf \sc Open}}}

\definecolor{cyell}{rgb}{.9,0,9,0}
\newcommand{\yell}{\cellcolor{cyell}}

\definecolor{kgray}{rgb}{224,224,224}

\newcommand{\hide}[1]{}
\newcommand{\squeezelist}{\setlength{\itemsep}{0pt}}

\topmargin -0.2in                  \headheight 0pt                 \headsep 0in                    \textheight 9.2in                 \textwidth 6.5in
\oddsidemargin 0in              \evensidemargin 0in



\newcommand\Temp{\mathcal{T}}
\newcommand\A{\mathcal{A}}
\newcommand\cone{{C}}
\newcommand\red[1]{\textcolor{red}{#1}}
\newcommand\blue[1]{\textcolor{blue}{#1}}

\newcommand\Pt{\mathcal P}
\newcommand\yp{{\tt yao}}
\newcommand\tp{{\tt theta}}
\newcommand\pyy{p_{{\tt YY}}}
\newcommand\ang[1]{\widehat{#1}}

\newcommand{\arr}[1]{\overrightarrow{#1}}

\newtheorem{theorem}{{\bf Theorem}}
\newtheorem{corollary}[theorem]{Corollary}
\newtheorem{lemma}[theorem]{Lemma}
\newtheorem{claim}[theorem]{Claim}
\newtheorem{proposition}[theorem]{Proposition}
\newtheorem{conjecture}[theorem]{Conjecture}
\newtheorem{definition}[theorem]{Definition}
\newtheorem{openquestion}[theorem]{Open question}

\newcommand{\qed}{\rule{0.5em}{1.5ex}}
\newcommand{\fqed}{{\hfill~\qed}}
\newenvironment{proof}{{\noindent \bf Proof.}}
                      {{\hfill \fqed} \vspace{1em}}

\newcommand{\eproof}{{\hfill~\fqed} \vspace{1em}}
\newcommand{\notyet}[1]{}

\begin{document}

\title{An Infinite Class of Sparse-Yao Spanners}
\author{Matthew Bauer
    \thanks{ Department of Computing Sciences, Villanova University, Villanova, PA, USA. \protect\url{mbauer03@villanova.edu}.}
\and
Mirela Damian
    \thanks{ Department of Computing Sciences, Villanova University, Villanova, PA, USA. \protect\url{mirela.damian@villanova.edu}.}
}

\date{}

\maketitle

\begin{abstract}
We show that, for any integer , the Sparse-Yao graph  (also known as Yao-Yao) is a spanner with stretch factor . The stretch factor drops down to  for .
\end{abstract}

\section{Introduction}
Let  be a finite set of points in the plane. The Yao graph and the Theta graph for  are directed geometric graphs with vertex set  and directed edges defined by an integer parameter  as follows. Fix a coordinate system and consider the rays obtained by a counterclockwise rotation of the positive -axis about the origin by angles of , for integer . Each pair of successive rays defines a cone whose apex is the origin, for a total of  cones. Translate these cones to each node , then connect  to a ``nearest neighbor'' in each of the  cones using directed edges rooted at . This yields an out-degree of at most . The Yao and Theta graphs differ in the way the ``nearest neighbor'' is defined.
In the case of Yao graphs, the neighbor nearest to  in a cone  is a point  that lies in  and minimizes the Euclidean distance  between  and ; ties are broken arbitrarily.
In the case of Theta graphs, the neighbor nearest to  in a cone  is a point  that lies in  and minimizes the Euclidean distance between  and the orthogonal projection of  onto the bisector of ; ties are broken in favor of a neighbor  that minimizes , and in the case of two such neighbors, one is arbitrarily selected.
Henceforth, we will refer to the Yao graph as  and the Theta graph as .
\begin{figure}[htpb]
\centering
\includegraphics[width=0.8\linewidth]{yaothetaex}
\caption{(a)  (b) Theta graph  (c) Yao graph  (d) Sparse-Yao graph }
\label{fig:yaothetaex}
\end{figure}
Fig.~\ref{fig:yaothetaex} shows a simple example with four points .
The nonempty cones at each point, to be used in constructing  and  (for fixed ),
are delineated in Fig.~\ref{fig:yaothetaex}a. In the cone  with apex  containing both points
 and , note that  (because  lies strictly outside the circle centered at
 of radius ), but the orthogonal projection of  on the bisector of  is closer to 
compared to the orthogonal projection of  on the bisector of . Consequently,  selects
 in  (see Fig.~\ref{fig:yaothetaex}b), whereas  selects  in  (see Fig.~\ref{fig:yaothetaex}c).
Similarly,  favors  over  in the cone with apex  containing both  and ,
whereas  selects .

Interest in Theta graphs and Yao graphs has increased with the advancement of wireless network technologies and the need for efficient communication. Among other properties, communication graphs are required to include short paths between any pair of nodes to enable efficient routing, and to have low degree to reduce MAC-level contention and interference~\cite{DegreeMac06}. It turns out that both  and  obey the first requirement (for any  and other specific values of , as detailed in Table~\ref{tab:yao}), but fail to satisfy the second one. Imagine for example the simple scenario in which  consists of  nodes placed on the circumference of a circle with center node . Then, for , each of  and  will have an edge directed from each of the  nodes towards , because  is ``nearest'' in one of their cones. So each of  and  has out-degree , but in-degree . To overcome the problem of potential high in-degree at a node, the Sparse-Yao graph , also known as the Yao-Yao graph, has been introduced. The graph  is obtained by applying a second Yao step to the set of incoming Yao edges: for each node  and each cone rooted at  containing two or more incoming edges, retain a shortest incoming edge and discard the rest; ties are broken arbitrarily. Fig.~\ref{fig:yaothetaex}d shows that Sparse-Yao graph  corresponding to the Yao graph  from Fig.~\ref{fig:yaothetaex}c: because  and  lie in one same cone with apex , and because ,  keeps  and discards ; similarly, because ,  keeps  and discards .
The degree of  is bounded above by  (at most  outgoing edges and at most  incoming edges at each node). Although not as popular, the Sparse-Theta graph can be defined analogously.

Ignore for the moment the direction of the edges in ,  and , and view these graphs as \emph{undirected} graphs.
We present some interesting properties of these graphs, along with our main result, after a few brief definitions. Let  be an undirected graph with vertex set . The \emph{length} of a path in  is the sum of the Euclidean lengths of its constituent edges. For a fixed real , we say that  is a -\emph{spanner} for  if, for each pair of points , there is a path in  whose length is at most ; the value  is called the \emph{stretch factor} of . The graphs  and  are known to be spanners for any ; the stretch factors for specific ranges of  are listed in Table~\ref{tab:yao}. Very little in comparison is known about Sparse-Yao graphs. The only existing results are negative and show that , for , are not -spanners for any constant real value . For a comprehensive discussion of spanners, we refer the reader to the books by Peleg~\cite{Peleg00} and Narasimhan and Smid~\cite{ns-gsn-07}.

\begin{table}[htpb]
\begin{center}
\scriptsize{
\begin{tabular}{|c|c|c|c|} \hline
  & Graph  & \raisebox{-2pt}{Graph } & Graph  \2pt]
\hline\hline
\raisebox{-2pt}{} & \multicolumn{3}{c|}{\raisebox{-2pt}{NO~\cite{MollaThesis09}}} \2pt]
\hline
\raisebox{-2pt}{} & \multicolumn{3}{c|}{\raisebox{-2pt}{\open}} \2pt]
\hline
\raisebox{-6pt}{} & \multirow{2}{*}{\raisebox{-1em}{}} &
\raisebox{-6pt}{~\cite{BDD+10}} &
\multirow{2}{*}{\raisebox{-1em}{\normalsize{\red{~~[this paper]}}}} \6pt]
\hline\hline
\end{tabular}
} \vspace{-1em}
\end{center}
\caption{Spanning properties of Theta and Yao-based graphs.}
\label{tab:yao}
\end{table}

In this paper we take a first step towards proving that  is a spanner, for sufficiently large .
Our main result is that  is a -spanner, for any  and . As far as we know, this is the first positive result regarding the spanning property of Sparse-Yao graphs. This result relies on a recent result by Bonichon et al.~\cite{Bon+10}, who prove that  is a -spanner. Our main contribution is showing that   contains a short path between the endpoints of each edge in . More precisely, we show that
corresponding to each edge , there is a path between  and  in  no longer than , for  and . Combined with the fact that  is a -spanner, this yields an upper bound of  on the stretch factor of . This result also shows that the class of Sparse-Yao spanners is infinite.

\subsection{Notation and Definitions}
Throughout the rest of the paper we work with the graphs ,  and  defined for a fixed point set  and for positive integer . We view paths in these graphs as \emph{undirected}, and refer to the direction of an edge only when necessary to establish certain graph properties.
All three graphs ,  and  use a first ray emanating from the origin of the coordinate system in the direction of the positive -axis; each successive ray is obtained by a counter-clockwise rotation of the previous ray by angle  about the origin ( in the case of , and  in the case of  and ). A \emph{cone} is a region between two successive rays. Starting from the positive -axis, the cones encountered in counter-clockwise order are  in the case of  (see Fig.~\ref{fig:thetacross}a), and  in the case of . Note that the subscripts  and  are used to differentiate between the cones used in constructing  and those used in constructing  and ; and the numerical subscripts are used to identify a particular cone from among all cones with the same apex.
Each cone  is \emph{half-open} and \emph{half-closed} in the sense that it includes the ray clockwise from  bounding , but excludes the ray counter-clockwise from  bounding .

For any point  and fixed cone , let  denote the copy of  translated so that its apex coincides with .
For any two points , we use  to refer to the cone with apex  that contains , used in constructing .
We define  to be the open equilateral triangle with two of its sides along the
bounding rays for , and the third side passing through  (see, for example, the large shaded
triangle from Fig.~\ref{fig:proof1}a).

We say that two edges \emph{intersect} each other if they share a common point. If the common point is not an endpoint, the edges \emph{cross} each other. Fig.~\ref{fig:thetacross}b shows a pair of crossing edges; compare it to the two pairs of intersecting but non-crossing edges from Fig.~\ref{fig:thetacross}c. Throughout the paper,  is used to denote the
concatenation operator. A path in a graph between two points  and  is denoted by . To avoid confusion, we attach to the path notation one of the subscripts ,  and , depending on whether the path is in ,  or . For example,  refers to a path in  from  to .

The rest of the paper is organized as follows. Sec.~\ref{sec:basic} introduces a few isolated lemmas that are used in our main proof. The proofs of these lemmas are rather involved, and for this reason we defer them until Sec.~\ref{sec:proofs}, by which point their use is the main proof should be clearly understood. Sec.~\ref{sec:main} presents our main result. We wrap up with some conclusions and future work in Sec.~\ref{sec:conclusions}.

\section{Preliminaries}
\label{sec:basic}
In this section we provide a few isolated lemmas that will be used in the main proof. We defer the proofs of these lemmas until after the proof of the main theorem (Thm.~\ref{thm:maintheta}), so that the flow of ideas can be followed without interruption. We encourage the reader to skip ahead to \S\ref{sec:main}, and refer back to these lemmas from the context of Thm.~\ref{thm:maintheta}, where their usefulness will become evident.

\medskip
\noindent
We begin this section with the statement of a result established in~\cite{Bon+10}.

\begin{theorem}
For any pair of points , there is a path in 
whose total length is bounded above by .
~\emph{\cite{Bon+10}}
\label{thm:theta6}
\end{theorem}
The bound  on the stretch factor of  is tight~\cite{Bon+10}. The key ingredient in the result of Thm.~\ref{thm:theta6} is a specific subgraph of , called \emph{half-}. This graph preserves only half of the edges of , those belonging to non
consecutive cones. Bonichon et al.~\cite{Bon+10} show that half- is a
\emph{triangular-distance}\footnote{The \emph{triangular distance} from a point  to a point 
is the side length of the smallest equilateral triangle centered at  and touching .}
\emph{Delaunay triangulation}, computed as the dual of the Voronoi diagram based on
the triangular distance function.
This result, combined with Chew's result from~\cite{Chew89}, showing that any triangular-distance Delaunay triangulation is a -spanner, yields the result of Thm.~\ref{thm:theta6}. For details, we refer the reader
to~\cite{Bon+10}.


\begin{figure}[htpb]
\centering
\includegraphics[width=0.5\linewidth]{thetacross}
\caption{(a) Cones numbering (b) Crossing edges (c) Intersecting, non-crossing edge pairs.}
\label{fig:thetacross}
\end{figure}


\noindent
Lem.~\ref{lem:thetapath} below will play a central role in the proofs of Lemmas~\ref{lem:thetapathbb} through~\ref{lem:thetapathaa3}.
\begin{lemma}
Let  and let  and  be the other two vertices of . Let  be the point on  such that  is parallel to . Let  be a shortest path in  from  to . If  is empty of points in , then . Moreover, each edge of  is no longer than . \emph{[Refer to Fig.~\ref{fig:S-trapezoid}a.]}
\label{lem:thetapath}
\end{lemma}
Note that Lem.~\ref{lem:thetapath} does not specify which of the two sides  and  lies clockwise from , so the lemma applies in both situations.

\begin{figure}[htpb]
\centering
\includegraphics[width=0.9\linewidth]{S-trapezoid}
\caption{(a) Lem.~\ref{lem:thetapath}: 
(b) Lem.~\ref{lem:thetapathbb}: , , . }
\label{fig:S-trapezoid}
\end{figure}


\medskip
\noindent
Lemmas~\ref{lem:thetapathbb} through~\ref{lem:thetapathaa3} isolate specific situations that will arise in the analysis of our main result. These lemmas can be stated and investigated independently.

\begin{lemma}
Fix an integer  and let ,  and  be distinct points in  such that  lies in  below the bisector of ,
and  lies in . Let  be the point on  such that  is perpendicular on . Then there is a path  in  of length

Furthermore, each edge of  is strictly shorter than . \emph{[Refer to Fig.~\ref{fig:S-trapezoid}b.]}
\label{lem:thetapathbb}
\end{lemma}

\begin{lemma}
Fix an integer  and angle . Let ,  and  be distinct points in  such that (i)  lies in ,
and (ii)  lies in , such that . Then there is a path  in  of length

\label{lem:thetapathaa1}
\end{lemma}
\vspace{-1.5em}
Intuitively, the distance  is the context of Lem.~\ref{lem:thetapathaa1} is fairly small. For this reason,
Lem.~\ref{lem:thetapathaa1} claims an upper bound on the length of  that is good enough for our purposes (in the sense that it is superseded by the upper bounds derived in the companion lemmas), but is not necessarily tight.

\begin{figure}[htpb]
\centering
\includegraphics[width=0.5\linewidth]{S-thetapathaa1}
\caption{Lem.~\ref{lem:thetapathaa1}: , , , .}
\label{fig:thetapathaa1}
\end{figure}


\vspace{-1em}
\begin{lemma}
Fix an integer  and angle . Let , ,  and  be distinct points in  that satisfy the following properties: (i)  lies in  below the bisector of , (ii)  lies in , and (iii)  lies in  such that . Then there is a path  in  of length

\noindent
Furthermore, each edge of  is strictly shorter than .
\emph{[Refer to Fig.~\ref{fig:S-thetapathaa2}a])}
\label{lem:thetapathaa2}
\end{lemma}

\begin{figure}[htpb]
\centering
\includegraphics[width=\linewidth]{S-thetapathaa2}
\caption{, ,  (a) Lem.~\ref{lem:thetapathaa2}:  (b) Lem.~\ref{lem:thetapathaa3}: .}
\label{fig:S-thetapathaa2}
\end{figure}


\noindent
Lemma~\ref{lem:thetapathaa3} below complements Lem.~\ref{lem:thetapathaa2} regarding the relative position of  and . The upper bound derived in this lemma may look somewhat unpolished, however it is intentionally left in a form that is most useful to the main theorem (Thm.~\ref{thm:maintheta}) from \S\ref{sec:main}.
\begin{lemma}
Fix an integer  and angle . Let , ,  and  be distinct points in  that satisfy the following properties: (i)  lies in , below the bisector of  (ii)  lies in ,
and (iii)  lies in  below  such that .
Let  be the point on  such that  is perpendicular on . Then there is a path  in  of length

Furthermore, each edge of  is strictly shorter than .
\emph{[Refer to Fig.~\ref{fig:S-thetapathaa2}b.]}
\label{lem:thetapathaa3}
\end{lemma}


\section{ is a Spanner}
\label{sec:main}
This section contains our main result, which shows that there is an infinite class of sparse Yao graphs that are spanners. In particular, we show that  is a -spanner, for  and . Our approach takes advantage of the empty triangular area embedding each edge in , and establishes ``short'' paths in  between the endpoints of each edge in . This, combined with the result of Thm.~\ref{thm:theta6}, yields our main result.

\begin{theorem}
For each edge , there is a path  in  of length
, for any  and .
\label{thm:maintheta}
\end{theorem}
\begin{proof}
Let . The proof is by induction on the length of the edges in . The base case corresponds to a shortest edge .
In this case we show that  and . Assume to the contrary that
, and let , with , be the edge that lies in the cone  with apex  containing . The Law of Cosines applied on , along with the fact that , yields . Now note that for any ,  and , therefore . This along with Thm.~\ref{thm:theta6} shows that  contains a path . This implies that all edges on  are strictly shorter than , contradicting our assumption that  is a shortest edge in . Similar arguments show that , so the theorem holds for the base case.


For the inductive step, pick an arbitrary edge , and assume that the theorem holds for all edges in  strictly shorter than . We now seek a path  in  that satisfies the conditions of the theorem. We assume without loss of generality that  lies in the first cone ; if this is not the case, then we can always rotate the point set  about  by a multiple of  so that our assumption holds. (Note that the edge sets for  and  remain unaltered by this rotation.) We can also assume that  lies along or below the bisector of ; the situation in which  lies above the bisector of  is symmetric with respect to the bisector of .
Let  be the Yao cone with apex  containing .

We begin our analysis by considering the most complex case, in which a relevant edge from  is not in , and a relevant edge from  is not in . We will see later that all other cases are particular instances of this complex case. Thus we start with the assumption that  and hence  must contain an edge  with the property . We proceed further through the complex case with the assumption that . Let  be the Yao cone with apex  containing . Then  must contain an edge  such that . Because ,  is empty of points in , therefore  must lie outside of . One immediate observation here is that  must also lie outside of ; otherwise , contradicting our assumption that .

We first determine a ``short'' path from  to  in . Let  be the foot of the perpendicular from  on  (see Fig.~\ref{fig:proof1}a). By Lem.~\ref{lem:thetapathbb}, there is a path  in  of length . Also according to Lem.~\ref{lem:thetapathbb}, each edge of  is strictly shorter than . This enables us to apply the inductive hypothesis on each edge , and claim the existence of a path  in  of length . Concatenating these paths and summing up the inequalities for all edges on  yields a path  in  of length

We now express  in terms of  and angle  (this latter inequality holds because  and  are in the same half-open Yao cone  of angle ). The fact  implies  , or in a simpler form . We substitute this inequality in~(\ref{eq:case0-1}) to obtain

Next we focus our attention on determining a ``short'' path  from  to  in . Given this, we can subsequently define a path  from  to  as

Depending on the relative position of  and , we must consider four possible cases. First note that  implies that , and therefore  must lie in the open half-plane that contains  and is delimited by the perpendicular on  through . Because , this half-plane shares no points with , however it may share points with any other -cone apexed at .
Thus . We consider each of these situations in turn.

\begin{figure}[htpb]
\centering
\includegraphics[width=\linewidth]{proof1}
\caption{Thm.~\ref{thm:maintheta} (a) Case 1:  (b) Case 2: .}
\label{fig:proof1}
\end{figure}


\paragraph{Case 1:} . (Refer to Fig.~\ref{fig:proof1}a.) This situation meets the conditions of Lem.~\ref{lem:thetapathaa2}, which tells us that  contains a path  from  to  of length

Lem.~\ref{lem:thetapathaa2} also tells us that each edge of  is strictly shorter than . This enables us to use the inductive hypothesis and claim the existence of a path  in  of length

Ignoring the direction of the edges,  is a path in  from  to .
Substituting inequalities~(\ref{eq:case1-2}) and~(\ref{eq:case0-2}) in~(\ref{eq:pab}), and using the fact that
, we derive an upper bound for the length of the path  as

To prove the inductive step, we need to show that the right side of the inequality~(\ref{eq:case1-3}) does not exceed , which (after eliminating the term ) holds if

Two conditions must be met in order to satisfy inequality~(\ref{eq:case1-4}):

We note that the term  defined in~(\ref{eq:case1-1}) decreases as  decreases, and consequently the term on the left hand side of the first inequality above increases as  decreases. This property helps in verifying that the two inequalities above hold for any  () and . We also note that smaller  values imply smaller ; for example, for  (), the constraint on  is .

\paragraph{Case 2:} . (Refer to Fig.~\ref{fig:proof1}b.) This situation meets the conditions of Lem.~\ref{lem:thetapathaa1}, which tells us that  contains a path  from  to  of length . Note that for the values of  imposed by the lemma,  and the term . Because the entire path  is strictly shorter than , each edge of  is also strictly shorter than , so we can use the inductive hypothesis to claim the existence of a path  from  to  in  of length

Substituting inequalities~(\ref{eq:case2-1}) and~(\ref{eq:case0-2}) in~(\ref{eq:pab}), along with ,  yields 
To prove the inductive step, we need to show that the right side of the inequality~(\ref{eq:case2-2}) does not exceed , which (after eliminating the term ) holds if

Two conditions must be met in order to satisfy inequality~(\ref{eq:case2-3}):

Again, note that the term on the left hand side of the first inequality increases as  decreases. This property helps in verifying that the two inequalities above hold for any  () and . The lower bound on  drops down to  for , and lowers to  for .

\paragraph{Case 3:} . This case is depicted in Fig.~\ref{fig:proof2}a. The result of  Lem.~\ref{lem:thetapathaa1} and the arguments used for Case 2 above apply here as well, yielding the same lower bounds for  and  (and upper bound for ) as as in Case 2.

\begin{figure}[htpb]
\centering
\includegraphics[width=\linewidth]{proof2}
\caption{Thm.~\ref{thm:maintheta} (a) Case 3:  (b) Case 4: .}
\label{fig:proof2}
\end{figure}


\paragraph{Case 4:} . This case is depicted in Fig.~\ref{fig:proof2}b.
Let  be the foot of the perpendicular from  on . This context matches the one of Lem.~\ref{lem:thetapathaa3}, so we can use it to claim the existence of a path  from  to  of length . Also according to Lem.~\ref{lem:thetapathaa3}, each edge of  is strictly shorter than . This enables us to use the inductive hypothesis and claim the existence of a path  in  of length

Ignoring the direction of the edges,  is a path from  to  in . By inequalities~(\ref{eq:case4-1}) and~(\ref{eq:case0-1}), an upper bound for the length of the path  defined in~(\ref{eq:pab}) is

(In deriving the right side term above, we used the fact that .)
Now note that the ray with origin  parallel to  lies inside , therefore the angle formed by this ray with  is smaller than . It follows that . This in turn implies that
, or equivalently . Similarly,
. Substituting these two latter inequalities in~(\ref{eq:mainpab1}), and using
the fact that , yields the upper bound

We now express  in terms of  and . We have already established that
, therefore , or equivalently . Similarly,
. Summing up these two inequalities yields . It follows that . Substituting this inequality in~(\ref{eq:mainpab2}) yields

To prove the inductive step, we need to show that , which according to inequality~(\ref{eq:mainpab3}) holds if (after eliminating the term )

Two conditions must be met in order to satisfy the inequality above:

We note that the term  increases as  decreases. This property helps in verifying that the two inequalities above hold for any  () and . The lower bound on  drops down to  for .

\medskip
\noindent
It remains to discuss the simpler cases in which  or  (or both). We show that these are special cases of the above. Consider first the case in which . If  as well, then  and the theorem holds. Otherwise, we let  and define ; if  as in case 4 above, we also let , so that the analysis for case 4 applies here as well; the other cases (1, 2 and 3) need no special adjustments. Similarly, if , we let  and define ; then the analysis for case 2 above settles this entire case.  Now note that the upper bound for  yielded by the above analysis for these special cases is slightly smaller that the one obtained for the general case (because it does not include one of the strictly positive terms  or ), therefore the spanning condition  holds for the same values of  and .
\end{proof}

\medskip
\noindent
Thms.~\ref{thm:theta6} and~\ref{thm:maintheta} together yield the main result of this paper, stated in Thm.~\ref{thm:main} below.
\begin{theorem}
For any ,  is a -spanner, with .
\label{thm:main}
\end{theorem}

\section{Proofs of Lemmas from \S\ref{sec:basic}}
\label{sec:proofs}

\subsection{Proof of Lem.~\ref{lem:thetapath}}
The proof is by induction on the pairwise distances between the points in .
\begin{figure}[htpb]
\centering
\includegraphics[width=0.85\linewidth]{trapezoid1}
\caption{Lemma~\ref{lem:thetapath}: (a) Open region empty of points in  (b,c)  does not intersect .}
\label{fig:trapezoid1}
\end{figure}
The base case corresponds to a closest pair of points . In this case, the circle centered at  of radius  is empty of points in , and similarly for the circle centered at  of radius . Any equilateral triangle with vertex  and  on its boundary fits inside the union of these two circles (see Fig.~\ref{fig:trapezoid1}a). This along with the fact that cones are half-open and half-closed implies that  and the lemma holds for this case.

For the inductive case, pick an arbitrary pair of points , and assume that the lemma holds for any pair of points at distance less than .
Let , and let  be the edge in  incident to  (note that  exists, because  contains  and therefore is non-empty).
First note that  may not cross over to the other side of , because in that case the projection of  on the bisector of  would be farther from  than the projection of  on the bisector of , contradicting . This along with the fact that  is empty of points in  shows that  lies in the closed region . Next we focus on determining a ``short'' path  from  to . Given this, we can subsequently define a path

Let  be the intersection point between  and the line parallel to  passing through . Let  be the point on  such that  is parallel to . By the triangle inequality

We distinguish two cases, depending on whether  intersects  or not.
Assume first that  does not intersect . Then . We have already established that  lies in the closed region , so in this case  lies either interior to , or on . Consider first the situation in which  is interior to , depicted in Fig.~\ref{fig:trapezoid1}b. In this case . Let  and  be the intersection points between the bounding rays of  and  and , respectively. Note that the equilateral triangle obtained by removing the trapezoid  from  lies inside , which is empty of points in  (by the lemma statement). Thus the inductive hypothesis applies here to show that there is a path  from  to  of length

Substituting inequalities~(\ref{eq:pcb}) and~(\ref{eq:actri}) in~(\ref{eq:aub}) yields . (Here we used the fact that  and .) So the first claim of the lemma holds in this case. The second claim of the lemma follows immediately from the fact that  and the inductive hypothesis, by which each edge of  is no longer than .

If  lies on the line segment  (as in Fig.~\ref{fig:trapezoid1}c), then the trapezoid  from Fig.~\ref{fig:trapezoid1}b degenerates to the line segment . The equilateral triangle with side  that lies in  coincides with one of  or , and is empty of points in . So the induction hypothesis applies again to show that  contains a path between  and  no longer than . This along with inequality~(\ref{eq:actri}) shows that the path  is no longer than , so the lemma holds. (Note that the second claim of the lemma follows immediately from the fact that each of  and  is no longer than .)

\begin{figure}[htpb]
\centering
\includegraphics[width=0.9\linewidth]{trapezoid2}
\caption{Lemma~\ref{lem:thetapath}:  intersects  (a)  interior to  (b)  on  (c)  on .}
\label{fig:trapezoid2}
\end{figure}


Assume now that  intersects . Recall that  must lie in the closed region , so in this case  lies in the closed equilateral triangle .
Consider first the situation in which  lies strictly interior to , as depicted in  Fig.~\ref{fig:trapezoid2}a. Let  and  be points on  and  respectively, such that  is parallel to  and  is parallel to . Note that the equilateral triangle obtained by removing  from  is empty of points in , because it lies inside , which contains no points in .
This along with the fact that  enables us to use the inductive hypothesis to claim the existence of a path  from  to  of length

Because we seek undirected paths, we ignore the direction of the edges and let . Substituting inequalities~(\ref{eq:pcb2}) and~(\ref{eq:actri}) in~(\ref{eq:aub}) yields
. (Here we used the fact that , and .) Also note that , and each edge of  is no longer than  (by the inductive hypothesis). So the lemma holds for this case as well.

If  lies on the line segment , then the equilateral triangle with side  lying inside  coincides with one of  or . The first situation reduces to a special instance of the case depicted in Fig.~\ref{fig:trapezoid1}b, in which  and  coincide (so the trapezoid  is really a triangle); the second situation reduces to a special instance of the case depicted in Fig.~\ref{fig:trapezoid2}a, in which  and  coincide. So the lemma holds for this case.

It remains to discuss the situation in which  lies on the line segment , as depicted in Fig.~\ref{fig:trapezoid2}b. This situation occurs when  and  are both candidates for the  edge selected in the cone , and ties are broken in favor of . Because ,  is empty of points in . In particular, the equilateral triangle with side  that lies inside  (see the small shaded triangle in Fig.~\ref{fig:trapezoid2}b) is empty of points in . This triangle coincides with one of  or , so this is again a degenerate case in which the trapezoid   reduces to the line segment . The inductive hypothesis applies here to show that  contains a path between  and  no longer than . This along with~(\ref{eq:actri}) and the fact that  shows that the path  is no longer than . Also note that each of  and  is no longer than , so the second claim of the lemma holds. This completes the proof.
\eproof

\subsection{Proof of Lem.~\ref{lem:thetapathbb}}


First observe that , because , and  and  lie in the same Yao cone . This implies that  lies on the line segment  (otherwise  would be longer than , a contradiction). Also note that  implies that the interior of  is empty of points in , so  must lie to the right of  and above . (This latter claim follows from the fact that
 is acute because . Also note that  cannot lie on the boundary of , because ties in  are broken in favor of the edge of shorter Euclidean length.) It can be easily verified that .
\begin{figure}[htpb]
\centering
\includegraphics[width=0.45\linewidth]{thetapathbb}
\caption{Lem.~\ref{lem:thetapathbb}: , , .}
\label{fig:thetapathbb}
\end{figure}


Let  be at the intersection between the line through  parallel to the right side of  and the horizontal line though . Let  be at the intersection between the right side of  and the left ray of . Note that the equilateral triangle obtained after removing  from  lies inside  and therefore it is empty of points in .
This places us in the context of Lem.~\ref{lem:thetapath} so we can claim the existence of a path  in  of length

Next we establish an upper bound on  and  in terms of . Let  be the intersection point between the horizontal through  and the vertical through . Then , therefore , and
. Summing up these two inequalities yields
. This along with the fact that  (the two terms are equal when  is horizontal) yields the upper bound stated by the lemma.

We now turn to the second claim of the lemma. By Lem.~\ref{lem:thetapath}, each edge of  is no longer than . Now note that  and . It follows that . This along with the Law of Sines  shows that  is strictly shorter than , which in turn is no longer than . This completes the proof.
\eproof

\subsection{Proof of Lem.~\ref{lem:thetapathaa1}}
Here we discuss only the case  depicted in Fig.~\ref{fig:thetapathaa1}; the case  is symmetric.
By Thm.~\ref{thm:theta6},  contains a path  no longer than . Thus we focus on bounding .
First note that  (because  and  are in the same cone of angle ), and  (because  includes the entire -cone ).
It follows that
, and . Substituting these inequalities in the Law of Sines applied on triangle  yields

This shows that  is an upper bound for , therefore  is an upper bound for .
\eproof


\subsection{Proof of Lem.~\ref{lem:thetapathaa2}}
Recall that  implies that  is empty of points in . This along with the fact that  implies that  crosses the left side of . Refer to Fig.~\ref{fig:thetapathaa2} throughout this proof.
\begin{figure}[htpb]
\centering
\includegraphics[width=\linewidth]{thetapathaa2}
\caption{Lem.~\ref{lem:thetapathaa2}: ; ; and . (a)  \emph{below} the bisector of  (b)  \emph{above} the bisector of .}
\label{fig:thetapathaa2}
\end{figure}
Let  be at the intersection between the line through  parallel to the left side of  and the horizontal line though . By the lemma statement , therefore . This implies that the right ray bounding  intersects the left side of ; we call the intersection point .
(Note that if  lies on the boundary of , then the isosceles trapezoid  degenerates to a line segment ; our arguments below apply to this scenario as well.)
The triangle obtained after removing  from  lies inside , therefore it is empty of points in . This enables us to use
Lem.~\ref{lem:thetapath} and claim the existence of a path  in  of length

Define three points ,  and  as follows:  is the foot of the perpendicular from  on ;  is the intersection point between the horizontal through  and ; and  is the intersection point between  and . (If  is on the boundary of , then we let ,  and .) Note that  must lie on the line segment  -- otherwise,  would be longer than , contradicting the fact that . We further expand the right side of~(\ref{eq:paa1}) into

We discuss two cases, depending on whether  lies below or above the bisector of .

\paragraph{Case 1:}  lies along or below the bisector of . (Refer to Fig.~\ref{fig:thetapathaa2}a). In this case, observe that
 sits along or clockwise from  with respect to , because the line supporting  is orthogonal to the bisector of , and  sits along or clockwise from the bisector with respect to . Also note that . This along with the Law of Sines  implies . This enables us to further expand the term on the right side of inequality~(\ref{eq:paa1-1}) as follows:

A similar analysis performed on triangle  shows that , which along with the Law of Sines  implies that

Next we bound the term  from inequality~(\ref{eq:paa2}) in terms of  as well. Note that , which is orthogonal to , lies to the right of the vertical through , which in turn forms a  angle with . It follows that  and . This along with inequality~(\ref{eq:paa-ij}) and the Law of Cosines 
implies that

Inequalities~(\ref{eq:paa2}),~(\ref{eq:paa-ij}) and~(\ref{eq:paa-a'j}) together yield

Now note that , which implies , or equivalently . This along with inequality~(\ref{eq:paa3}) yields the upper bound

This upper bound matches the one claimed by the lemma when the  operator yields .

\paragraph{Case 2:}  lies above the bisector of . (Refer to Fig.~\ref{fig:thetapathaa2}b;
we note that the relative position of  and  on the upper ray of  is irrelevant to this case, so  is absent in Fig.~\ref{fig:thetapathaa2}b.)
Because  is below the bisector of  and  is above the bisector, and because , the inequalities  hold. Consequently,
. This along with the Law of Sines  yields

We sum up  on both sides of the inequality above and substitute the result in~(\ref{eq:paa1-1}) to derive an upper bound

(The latter inequality above uses the fact that  and , for any .)
Next we turn our attention to the triangle , to derive upper bounds for  and . We have already established that . This along with the fact that  implies that , which in turn implies that  (recall that ). This along with the Law of Sines
 implies that  and
.
Substituting these inequalities in~(\ref{eq:pcc2}) yields

Because , the inequality  holds. Substituting this in~(\ref{eq:pcc3}), and combining the result with the upper bound from~(\ref{eq:ub1}) obtained for the first case,
yields the upper bound stated by the lemma.

We now turn to the second claim of the lemma. By Lem.~\ref{lem:thetapath}, no edge of  exceeds . Let  be at the intersection between  and the upper ray of , and note that
 is strictly shorter than , which in turn is strictly shorter than  (because  is obtuse). This, along with , yields the second claim of the lemma.
\eproof

\subsection{Proof of Lem.~\ref{lem:thetapathaa3}}
Recall that  implies that  is empty of points in . This along with the fact that  lies below  implies that  crosses the bottom side of , as well as . Refer to Fig.~\ref{fig:thetapathaa3} throughout this proof.
\begin{figure}[htpb]
\centering
\includegraphics[width=0.5\linewidth]{thetapathaa3}
\caption{Lem.~\ref{lem:thetapathaa3}: , , , ,  is small.}
\label{fig:thetapathaa3}
\end{figure}
Because ,  is acute, forcing  to sit above . We now show that  must lie on the line segment . To see this, consider the foot  of the perpendicular from  on  (not marked in Fig.~\ref{fig:thetapathaa3}, to avoid excessive labeling). The point  must lie on the line segment  --  otherwise  would be longer than , contradicting . This, together with the fact that  sits clockwise from   with respect to  (because  sits clockwise from  with respect to ), shows that  must lie to the right of  as well.

Let  be the intersection point between the horizontal through  and the left bounding ray of . Note that , because  (by the lema statement). This implies that the upper bounding ray of  intersects the bottom side of ; we label the intersection point . Note that the triangle obtained after removing  from  lies inside  and therefore is empty of points in . This enables us to apply
Lem.~\ref{lem:thetapath} and claim the existence of a path  in  of length

Let  be the intersection point between  and the line supporting . (Note that  lies to the left of , because  is orthogonal to the bisector of , which lies counterclockwise from  with respect to .) Clearly . Now observe that ; this is because the angle formed by  with the vertical through  is precisely , and  lies along or to the left of the vertical through . This implies that , or equivalently . This along with the fact that  implies

The triangle inequality applied on  tells us that , which substituted in  yields . This along with the triangle inequality  yields . Summing up this latter inequality with~(\ref{eq:a'x'2}), and substituting the result in~(\ref{eq:paa0-1}), yields the upper bound stated by the lemma.

Let  be the intersection point between  and . For the second claim of the lemma, we use the fact that no edge of  is longer than  (by Lem.~\ref{lem:thetapath}), which in turn is strictly shorter than  where .
\eproof

\section{Conclusions}
\label{sec:conclusions}
This paper establishes the first positive result regarding the spanning property of Sparse-Yao graphs (also known as Yao-Yao graphs). We show that, for any , the Sparse-Yao graph  is a spanner with stretch factor ; the stretch factor drops down to  for . We leave open our conjecture that  has constant stretch factor for any  larger than a specific threshold value (no less than ).

\medskip


\def\cprime{}
\begin{thebibliography}{10}

\bibitem{Bon+10}
N.~Bonichon, C.~Gavoille, N.~Hanusse, and D.~Ilcinkas.
\newblock Connections between {T}heta-graphs, {D}elaunay triangulations, and
  orthogonal surfaces.
\newblock In {\em Proc. of the 36th international conference on Graph-theoretic
  concepts in computer science}, WG'10, pages 266--278, Berlin, Heidelberg,
  2010. Springer-Verlag.

\bibitem{BDD+10}
P.~Bose, M.~Damian, K.~Dou\"{\i}eb, J.~O'Rourke, B.~Seamone, M.~Smid, and
  S.~Wuhrer.
\newblock Pi/2-angle {Y}ao graphs are spanners.
\newblock {\em CoRR}, abs/1001.2913, 2010.

\bibitem{bmnsz-agbsp-03}
P.~Bose, A.~Maheshwari, G.~Narasimhan, M.~Smid, and N.~Zeh.
\newblock Approximating geometric bottleneck shortest paths.
\newblock {\em Computational Geometry: Theory and Applications},
  29(3):233--249, 2004.

\bibitem{Chew89}
L.~Paul Chew.
\newblock There are planar graphs almost as good as the complete graph.
\newblock {\em Journal of Computer and System Sciences}, 39:205--219, 1989.

\bibitem{DMP09}
M.~Damian, N.~Molla, and V.~Pinciu.
\newblock Spanner properties of -angle {Y}ao graphs.
\newblock In {\em Proc. of the 25th European Workshop on Computational
  Geometry}, pages 21--24, March 2009.

\bibitem{DR10}
M.~Damian and K.~Raudonis.
\newblock Yao graphs span {T}heta graphs.
\newblock In {\em Proc. of the 4th International Conference on Combinatorial
  Optimization and Applications - Volume Part II}, COCOA'10, pages 181--194,
  Berlin, Heidelberg, December 2010. Springer-Verlag.

\bibitem{DegreeMac06}
B.~Hamdaoui and P.~Ramanathan.
\newblock Energy-efficient and {MAC}-aware routing for data aggregation in
  sensor networks.
\newblock In {\em Sensor Network Operations}, chapter 5.3, pages 291--308.
  Wiley-IEEE Press, March 2006.

\bibitem{MollaThesis09}
N.~Molla.
\newblock Yao spanners for wireless ad hoc networks.
\newblock Technical report, M.S. Thesis, Department of Computer Science,
  Villanova University, December 2009.

\bibitem{ns-gsn-07}
G.~Narasimhan and M.~Smid.
\newblock {\em Geometric Spanner Networks}.
\newblock Cambridge University Press, New York, NY, USA, 2007.

\bibitem{Peleg00}
D.~Peleg.
\newblock {\em Distributed computing: a locality-sensitive approach}.
\newblock Society for Industrial and Applied Mathematics, Philadelphia, PA,
  USA, 2000.

\end{thebibliography}

\end{document}
