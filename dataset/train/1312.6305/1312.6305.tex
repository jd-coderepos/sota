\documentclass[11pt]{article}

\usepackage{latexsym,color,amsmath,amssymb,amsfonts,mathrsfs,euscript}
\usepackage{enumerate}
\usepackage{color}
\usepackage{graphicx}
\usepackage{theorem}
\usepackage{amstext}
\usepackage{amssymb}
\usepackage{url}

\setlength{\textwidth}{6.0in}
\setlength{\evensidemargin}{0.25in}
\setlength{\oddsidemargin}{0.25in}
\setlength{\textheight}{9.0in}
\setlength{\topmargin}{-0.5in}
\setlength{\parskip}{2mm}
\setlength{\baselineskip}{1.7\baselineskip}

\newtheorem{theorem}{Theorem}[section]
\newtheorem{lemma}[theorem]{Lemma}
\newtheorem{corollary}[theorem]{Corollary}
\newtheorem{proposition}[theorem]{Proposition}
{\theorembodyfont{\rm} \newtheorem{defn}[theorem]{Definition}}

\newcommand{\Tree}{\EuScript{T}}
\newcounter{itemcounter}

\def\reals{\mathbb R}
\def\peps{(p,\varepsilon)}
\def\eps{\varepsilon}
\def\R{\cal{R}}
\def\r{\frak{r}}
\def\meas#1#2{\overline{#1}(#2)}
\def\ol#1{\overline{#1}}


\title{Output-Sensitive Tools for Range Searching in Higher Dimensions\thanks{Work by Micha Sharir was supported by NSF Grant CCF-08-30272, by Grants
2006/194 and 2012/229 from the U.S.-Israel Binational Science Foundation,
by Grants 338/09 and 892/13 from the Israel Science Fund, by the Israeli
Centers for Research Excellence (I-CORE) program (Center no. ~4/11), and
by the Hermann Minkowski--MINERVA Center for Geometry at Tel Aviv University.
The paper is part of the second author's M.Sc. dissertation, prepared under the supervision of the first author.
}}

\author{ Micha Sharir\thanks{School of Computer Science,
Tel Aviv University, Tel Aviv 69978, Israel, and
Courant Institute of Mathematical Sciences,
New York University, New York, NY 10012, USA;
\texttt{michas@tau.ac.il}.}
\and
Shai Zaban\thanks{School of Computer Science,
Tel Aviv University, Tel Aviv 69978, Israel;
\texttt{shaiza@bezeqint.net}.}}

\begin{document}
\maketitle

\begin{abstract}
Let  be a set of  points in . A point  is \emph{-shallow} if it lies in a halfspace which contains at most  points of 
(including ). We show that if all points of  are -shallow, then  can be partitioned into  subsets, so that any hyperplane crosses
at most  subsets. Given such a partition, we can apply the standard construction of a spanning tree with small
crossing number within each subset, to obtain a spanning tree for the point set , with crossing number . This allows us to extend the construction of Har-Peled and Sharir \cite{hs11} to three and higher dimensions, to obtain, for any set
of  points in  (without the shallowness assumption), a spanning tree  with {\em small relative crossing number}. That is, any hyperplane
which contains  points of  on one side, crosses  edges of . Using a similar mechanism,
we also obtain a data structure for halfspace range counting, which uses  space (and somewhat higher preprocessing cost), and answers a
query in time , where  is the output size.
\end{abstract}


\section{Introduction}

\subsection{Background}
One of the central themes in computational geometry is the design of efficient range searching algorithms. Typically, in such a problem one is given an
input set  of  points in  dimensions, and the goal is to preprocess  into a data structure, so that for any query {\em range} of some type (say,
a halfspace ), one can efficiently count, report, or test for emptiness the set . A significant component of most of these algorithms involves
space decomposition techniques, which partition the input set into subsets with some useful structure. Apart from range searching, partitions are also used
in constructions of spanning trees with small crossing number, approximations, -nets, and in several other computational geometry problems.

In a typical approach (see, e.g., \cite{mat92a}), given a point set  in , one partitions  into subsets of approximately equal sizes, so
that each of them is contained within some region of constant description complexity (e.g., simplices). We remark that the subsets are pairwise disjoint,
while the containing regions might have nonempty intersections. We seek partitions with {\em small crossing number}, meaning that the maximum number of
enclosing regions crossed by the boundary of a range (typically, a hyperplane) is small.

In this paper, we consider a variant of the partition paradigm, referred to as a partition of a {\em shallow point set}, which is applied to a set  of
{\em shallow} points in , where a point  is -shallow if there exists a hyperplane  that contains at most  points of ,
including , on one of its sides. We will construct a partition whose crossing number, namely, the maximum number of its simplices that can be crossed by
a hyperplane, depends, in addition to , also on the shallowness of , i.e., on the maximum shallowness  of its points.

As is typical in computational geometry, we consider the dimension  as a fixed (small) constant, thus factors depending only on  will be regarded as constants.


\paragraph{Cuttings.}
One of the major ingredients of our partitioning technique is {\em cuttings} of arrangements of hyperplanes. A cutting is a collection of (possibly
unbounded) -dimensional closed cells with constant description complexity (e.g., simplices) with pairwise disjoint interiors, which cover the entire
 (or some specified portion thereof). Let  be a collection of  hyperplanes in  and let  be a cutting of the arrangement
. For each simplex , let  denote the collection of hyperplanes intersecting the interior of . The
cutting  is called a -{\em cutting} for  if  for every simplex .

It will sometimes be convenient to work with weighted collections of hyperplanes, where such a collection is a pair (), where  is a collection of
hyperplanes, and  is a weight function on . For each , we write  for . The notions
introduced for unweighted collections of hyperplanes can usually be generalized for weighted collections in an obvious way. For example, a cutting  is
a -cutting for () if for every simplex , the collection  has total weight at most .


\paragraph{Partitions: related work.} Our study extends to higher dimensions the planar construction of a partition of a set of shallow points, recently
developed by Har-Peled and Sharir \cite{hs11}. This technique partitions  into  subsets, each containing  points and enclosed in a
(possible unbounded) triangle, so that the triangles are pairwise disjoint, and the crossing number of this partition (by any line) is only .

Har-Peled and Sharir also constructed a set  of  -shallow points in  (in fact,  is a set in convex position, so all its points are
-shallow), so that the (maximum) crossing number of any partition of  into subsets of size  is . See below for a
further discussion of this phenomenon, which was the starting point and the motivation for the study presented in this paper.

We next review the more standard partitioning techniques of Matou\v{s}ek \cite{mat92a,mat92b}. There  is a set of  points in , ,
which do not have to be shallow. Again, we partition  into  pairwise disjoint subsets, each having between  and  points (here  is an arbitrary parameter), enclosed by some simplex, where these simplices might have a nonempty intersection \footnote{A recent refined construction
by Chan[XX] constructs simplices with pairwise disjoint simplices.}. The partition, referred to as a \emph{simplicial partition}, has the property, that any
hyperplane  crosses at most  simplices.

Let ,  and  be as above. Matou\v{s}ek \cite{mat92b} has also developed a variant of the partitioning scheme described above, yielding a partition
of  subsets of size between  and , such that any -{\em shallow} hyperplane  (that is, a hyperplane which contains at most 
points of  on one side) crosses at most  simplices of the partition.

(In contrast, the construction in \cite{hs11}, and the one presented in this paper, deal with sets whose points are shallow, but we seek a small crossing
number with respect to every hyperplane.)


\paragraph{Applications: related work.} Let us now review some applications based on the various kinds of partitions mentioned above.

\paragraph{Halfspace range searching.} Matou\v{s}ek \cite{mat92a} uses the standard partition recursively to construct an efficient {\em partition tree},
whose root stores the entire input set  and a simplicial partition thereof. Each node  of the tree stores a subset  of , and a simplicial
partition of , and each subset of the partition corresponds to a distinct child of . Overall, the resulting partition tree has linear size, can be
constructed in  deterministic time, such that, given a query halfspace , one can count the number of points in  in
 time.

In a variant of this approach, Matou\v{s}ek \cite{mat92b} exploits the partition machinery with respect to shallow hyperplanes to efficiently solve the
halfspace range reporting (or emptiness) problem. More specifically, one can construct a data structure of size , in  time,
such that, given a query halfspace , one can report the points in  in  time,
where .

Another useful application of partitions of sets of shallow points, as described above, is the construction of spanning trees with small {\em relative}
crossing number. We recall the standard result, due to Chazelle and Welzl \cite{cw89} (see also \cite{wel92}), on the existence of spanning trees with small
crossing number. That is, given a set  of  points in , there exists a straight-edge spanning tree  on , such any hyperplane
crosses at most  edges of . Har-Peled and Sharir have refined this construction in the plane, to obtain a spanning tree  that
has the following property. Define the {\em weight}  of a line  to be the smaller of the two numbers of points of  on each side of
. Then  crosses only  edges of ; see \cite{hs11} for more details. In this paper, we follow the same machinery to
extend the above construction to higher dimensions.

\paragraph{Relative -approximations.} The existence of a spanning tree with small relative crossing number, for a point set  in the plane, as just
reviewed, facilitates the construction of {\em relative -approximations} for , with respect to halfplane ranges, whose size is smaller than the
one guaranteed by the general theory of Li et al. \cite{lls01} (see \cite{hs11}). For given , a relative -approximation for  is a
subset  with the property that, for each halfspace ,

provided that . Given such a spanning tree, we convert it into a spanning path with the same asymptotic crossing number, and
generate a matching out of this path by selecting every other edge along it, and apply a ``halving technique'' on the resulting matching, in which one point
of each pair of the matching is drawn independently at random, thus getting rid of half of the points. By the standard theory of discrepancy (see
\cite{cha01}), the small crossing number of the matching implies a correspondingly low discrepancy of the halving. We repeat this halving procedure until
the combined discrepancy first exceeds a given prescribed parameter, and then return the remaining points as the desired approximation. Har-Peled and Sharir
\cite{hs11} apply this construction to a planar point set , and obtain a relative -approximation for , of size

For most values of  and , this is better than the general bound  given in \cite{lls01}. See Section \ref{subsection_approximations} below for more details. In this paper, we extend this construction to higher dimensions, using the extension of spanning trees with small relative crossing number to higher dimensions, as mentioned above.


\paragraph{Our results.} We introduce a new variant of the partitioning machinery, based on the general approach that was introduced in \cite{mat92a} and
\cite{mat92b}, and obtain a partition of a set  of  -shallow points in  into  subsets, each of size between  and ,
so that each subset is contained within a -dimensional simplex and so that the crossing number of the partition (the maximum number of simplices crossed
by a hyperplane) is at most  for  and  for , where 
is the near-constant inverse Ackermann function. Note that the size of the sets in the partition is roughly the same as the shallowness parameter , and
that the exponent in the bound on the crossing number is smaller than the one provided for the general setup in \cite{mat92a}.

We use this partition, as in \cite{hs11}, to construct a spanning tree with small relative, output-sensitive, crossing number.
Specifically, any hyperplane  of weight  (the number of input points on its ``lighter" side) crosses at most  edges of the resulting spanning tree for , and  edges for .

We then extend the planar construction of relative -approximations for points and halfplanes introduced in \cite{hs11} to higher dimensions. We again
follow a similar machinery as the one in \cite{hs11}, based on the classical halving technique, to convert a spanning tree with small relative crossing
number to a relative -approximation. The properties of the relative -approximation and the exact result are detailed in Section
\ref{subsection_approximations} below.

Another application of the shallow-points partition is an exact output-sensitive range counting algorithm for point sets and halfspace ranges in
. We combine the range reporting algorithm of Matou\v{s}ek \cite{mat92b} with our shallow-points partition and the standard simplex range
counting algorithm of Matou\v{s}ek \cite{mat92a}, to obtain an improved, output-sensitive range counting algorithm, whose running time is , for , where  is the output size. This is an improvement over the previous bound 
of \cite{mat92a} when  (although when  is very small, the range reporting of \cite{mat92b} will be faster). See Section
\ref{subsection_range_countning} for a more precise statement of this result. (One weak aspect of our structure is that, for  we do not know how
to construct it in near-linear time. The current bound on the preprocessing cost is roughly  for some exponent  that depends on  and
satisfies . See Theorem \ref{thm_range_counting} for the precise statement.)

\subsection{Known notions of approximations} \label{subsection_approximations}
We recall the result of Li et al. \cite{lls01}, and two useful extensions of -approximations and -nets derived from it, which we will exploit later in this paper.

Let  be a \emph{range space}, where  is a set of  objects and  is a collection of subsets of , called \emph{ranges}. The \emph{measure} of a range  in  is the quantity

Assume that  has {\em finite VC-dimension} , which is a constant independent of ; see \cite{mat02} for more details.

Let  be two given parameters. Consider the distance function

A subset  is called a {\em -sample} for , if for each  we have


\begin{theorem} \label{thm_nu_alpha_sample} {\emph{Li et al.} \bf \cite{lls01}}
A random sample of  of size

is a -sample for  with probability at least , with an appropriate choice of an absolute constant of proportionality.
\end{theorem}

Har-Peled and Sharir \cite{hs11} show that, by appropriately choosing  and , various standard constructs, such as -nets and -approximations, are special cases of -samples; thus bounds on the sample sizes that guarantee that the sample will be one of these constructs with high probability are immediately obtained from Theorem \ref{thm_nu_alpha_sample}. Two other, less standard special cases of -samples are relative -approximations \cite{hs11} and {\em shallow -nets} \cite{shsh11}; we review them next.

\paragraph{Relative -approximations.}

\begin{defn} \label{definition_peps}
{\em Relative -Approximation}. Let  be a range space of finite VC-dimension . For given parameters , a subset  is a \emph{relative -approximation} for  if, for each , we have
\begin{center}
\begin{tabular}{lp{1\linewidth}}
(i) , \quad & if   \
\frac{c_{1}}{\eps^{2}p}\left(\delta \log \frac{1}{p} + \log \frac{1}{q} \right)

\frac{c'_{1}}{\eps} \left( \delta \log\frac{1}{\eps} +
\log\frac{1}{q} \right)

|Y| \geq \frac{n}{8k}, \quad \text{and} \quad \frac{n}{2k} \leq |Z| \leq \frac{2n}{k}.
 \label{eq_chernoff1}
{\bf Pr}\left\{X > (1+\delta)\mu \right\} < \left(\frac{e^{\delta}}{(1+\delta)^{(1+\delta)}}\right)^{\mu} \equiv \eps(\mu,1+\delta).
 \label{eq_chernoff2}
{\bf Pr}\left\{X > t \right\} < \left(\frac{e^{1-\frac{\mu}{t}}\mu}{t} \right)^{t} = \eps(\mu, t/\mu),
 \label{eq_chernoff3}
{\bf Pr}\left\{X < (1 - \delta)\mu \right\} < e^{-\frac{\mu\delta^{2}}{2}},

S_{> y} = \sum_{t > y \ \mid \ t \in \mathbb N} t \cdot {\bf Pr}\left\{X = t\right\}, \quad\quad
S_{\geq y} = \sum_{t \geq y \ \mid \ t \in \mathbb N} t \cdot {\bf Pr}\left\{X = t\right\}.

S_{> \frac{\beta n}{k}} \leq \eps(n/k, \beta) \cdot \left(\frac{\beta n}{k} + 1 + \frac{\beta}{\beta - e^{1-1/\beta}} \right).

S_{\geq t_{0}} = \sum_{t \geq t_{0}} t \cdot {\bf Pr}\left\{X = t \right\} =
\sum_{t \geq t_{0}} t \left({\bf Pr} \left\{X \geq t \right\} - {\bf Pr}\left\{X \geq t + 1 \right\} \right) =

t_{0} \cdot {\bf Pr}\left\{X \geq t_{0}\right\} + \sum_{t = t_{0}+1}^{n} {\bf Pr}\left\{X \geq t\right\} = t_{0} \cdot {\bf Pr}\left\{X \geq t_{0} \right\} + \sum_{t = t_{0}}^{n-1} {\bf Pr}\left\{X > t\right\}.

S_{> \frac{\beta n}{k}} = S_{\geq \lfloor \frac{\beta n}{k}+1 \rfloor} =
\left\lfloor \frac{\beta n}{k}+1 \right\rfloor \cdot {\bf Pr}\left\{X \geq \left\lfloor \frac{\beta n}{k}+1 \right\rfloor \right\} + \sum_{t = \lfloor \frac{\beta n}{k}+1 \rfloor}^{n-1} {\bf Pr}\left\{X > t\right\} \leq

\left(\frac{\beta n}{k} + 1\right) {\bf Pr}\left\{X > \frac{\beta n}{k}\right\} + \sum_{t = \lfloor \frac{\beta n}{k} + 1 \rfloor}^{n-1}{\bf Pr}\left\{X > t \right\} \leq

\left(\frac{\beta n}{k} + 1\right)\cdot\eps(n/k, \beta) + \sum_{t = \lfloor \frac{\beta n}{k} + 1 \rfloor}^{n-1}{\bf Pr}\left\{X > t \right\}.

\sum_{t \geq y}{\bf Pr}\left\{X > t\right\} \leq \sum_{t \geq y}\left(\frac{e^{1-\frac{\mu}{t}} \mu}{t}\right)^{t} \leq \sum_{t \geq y}\left(\frac{e^{1-\frac{\mu}{y}}\mu}{y}\right)^{t} <

\left(\frac{e^{1-\frac{\mu}{y}}\mu}{y}\right)^{y}\left(\frac{y}{y - e^{1-\frac{\mu}{y}}\mu}\right) = \eps(\mu, y/\mu) \cdot \left(\frac{y}{y - e^{1-\frac{\mu}{y}}\mu}\right).

\sum_{t = \lfloor \frac{\beta n}{k} + 1 \rfloor}^{n-1}{\bf Pr}\left\{X > t \right\} \leq \sum_{t \geq \frac{\beta n}{k}}{\bf Pr}\left\{X > t \right\} \leq \eps(n/k,\beta) \cdot \left(\frac{\beta}{\beta - e^{1-1/\beta}}\right),
 \label{eq1}
{\bf Pr}\left\{|Y| > \frac{\beta n}{k}\right\} \leq  {\bf Pr}\left\{|Z| > \frac{\beta n}{k}\right\} < \eps(n/k, \beta).
 {\bf Pr}\left\{a \in Y \right\} \geq \frac{1}{k}\left(1-\frac{1}{k}\right)^{k-1} \geq \frac{1}{ke}. \label{eq2}
{\bf E}\{|Y|\} \geq \frac{|P|}{ke} = \frac{n}{ke}.

\frac{n}{ke} \leq {\bf E}\{|Y|\} =& \sum_{t=0}^n t\cdot {\bf Pr}\left\{|Y|=t\right\} = \\
&\sum_{t < \frac{\alpha n}{k}} t\cdot {\bf Pr}\left\{|Y|=t \right\} +
\sum_{\frac{\alpha n}{k} \leq t \leq \frac{\beta n}{k}} t\cdot {\bf Pr}\left\{|Y|=t\right\} + \\
&\sum_{t > \frac{\beta n}{k}} t\cdot {\bf Pr}\left\{|Y|=t \right\} \leq
q\frac{\alpha n}{k} + (1-q-\eps)\frac{\beta n}{k} + s.

q \leq \frac{\beta(1-\eps) - e^{-1} + s\frac{k}{n}}{\beta-\alpha} \leq
\frac{\beta - e^{-1} + \eps\frac{k}{n}\left(1 + \frac{\beta}{\beta - e^{1-1/\beta}}\right)}{\beta-\alpha},
 \label{eq_sample_k}
1-q \geq 1 - \frac{\beta - e^{-1} + \eps\frac{k}{n}\left(1 + \frac{\beta}{\beta - e^{1-1/\beta}}\right)}{\beta-\alpha}.

{\bf Pr}\left\{|Z| > \frac{\gamma n}{k} \right\} < \eps(n/k,\gamma),

{\bf Pr}\left\{|Z| < \frac{n}{\nu k} \right\} < e^{-\frac{n}{k}\frac{(1-1/\nu)^{2}}{2}}.

{\bf Pr}\left\{\frac{n}{2k} \leq |Z| \leq \frac{2n}{k} \right\} > 0.999.

{\bf Pr}\left\{\left(|Y| \geq \frac{n}{8k}\right) \wedge \left(\frac{n}{2k} \leq |Z| \leq \frac{2n}{k} \right) \right\} > 0.22,

1-q \geq 1 - \frac{\beta - e^{-1} + \eps\frac{k}{n}\left(1 + \frac{\beta}{\beta - e^{1-1/\beta}}\right)}{\beta-\alpha}
\approx 1 - \left(1 - e^{-1} + \eps\frac{k}{n}\left(1 + \frac{\beta}{\beta - e^{1-1/\beta}}\right)\right).

1 - (1 - e^{-1}) = e^{-1} \approx 0.368.

q' = \left(1 - \left(1-\frac{1}{k}\right)^{k-1}\right)
\left(1 - \left(1-\frac{1}{k}\right)^{ck}\right)^2 \approx
\left(1-e^{-1}\right) \left(1-e^{-c}\right)^2 .

1 - q' = 1 - \left(1-e^{-1}\right) \left(1-e^{-c}\right)^2 ,

1 - \left(1-e^{-1}\right) = e^{-1},

\frac{c_{1}}{\eps^{2}p}\left(\delta \log \frac{1}{p} + \log \frac{1}{q} \right)  \leq \frac{n}{2k},

\frac{c_{1}}{\eps^{2}p}\left(\delta \log \frac{1}{p} + \log \frac{1}{q} \right) \leq
\frac{4 c_{1} \delta}{c_{2}} \left(1 - \frac{\log \log \frac{n}{c_{2}k}}{\log \frac{n}{c_{2} k}}\right) \cdot \frac{n}{k} + \frac{4 c_{1}}{c_{2}} \cdot \frac{\log \frac{1}{q}}{\log \frac{n}{c_{2} k}} \cdot \frac{n}{k} \leq

\frac{4 c_{1}(\delta + \log \frac{1}{q})}{c_{2}} \cdot \frac{n}{k} \leq \frac{n}{2k},
n_{\sigma}(p, t) = O(2^{-t}n_{\sigma}(p/t, 1)),
|CT_{\sigma}(R)| =
\begin{cases}
    O(|R|^{d-1}\log(|R|+1)), & \mbox{for } \\
    O(|R|\alpha(|R|+1)), & \mbox{for }.
\end{cases}

n_{\sigma}(p,1) = {\bf E}\{|CT_{\sigma}(R)|\} = {\bf E} \left\{O\left(|R|^{d-1}\log(|R|+1)\right)\right\} =
\label{eq_CT_R}
O(({\bf E}\{|R|\})^{d-1}\log({\bf E}\{|R|\})) = O((np)^{d-1}\log (np)).

S \leq \sum_{t=1}^{\infty}t^{d}n_{\sigma}(p,t).

S \leq \sum_{t=1}^{\infty}t^{d}\cdot c 2^{-t}n_{\sigma}(p/t,1) \leq c' \sum_{t=1}^{\infty}t^{d}2^{-t}\left(\frac{r}{t}\right)^{d-1}\log \left(\frac{r}{t}\right) \leq

c' \left(\sum_{t=1}^{\infty}t2^{-t}\right)r^{d-1}\log r = O(r^{d-1}\log r).
\Pi = \biggl\{(P_{1},\Delta_{1}), \ldots,(P_{m},\Delta_{m})\biggr\}, 
&O\left((n/k)^{1-1/(d-1)} \log^{2/(d-1)}(n/k) + \log |Q|\right), \quad \mbox{{\em for} , {\em and}} \\
&O\left(\alpha(n/k)\log^{2}(n/k) + \log |Q|\right), \quad \mbox{\emph{for} }.

t_{i} = c_{4}\left(\frac{n_{i}/k}{\log^{2}(n_{i}/k)}\right)^{1/(d-1)},

|\Xi_{i}| \leq c_{3}t_{i}^{d-1}\log t_{i} \leq
c_{3}c_{4}^{d-1} \cdot \frac{n_{i}}{k} \cdot \frac{(d-1)\log c_{4} + \log\frac{n_{i}}{k} - 2 \log \log \frac{n_{i}}{k}}{(d-1) \log^{2}\frac{n_{i}}{k}} \leq  \frac{c_{3}c_{4}^{d-1}}{d-1}\cdot \frac{n_{i}}{k} \cdot \frac{1}{\log \frac{n_{i}}{k}}.

\overline{Z_{i}}(\tau) = \frac{|Z_{i} \cap \tau|}{|Z_{i}|} \geq \frac{|Y_{i} \cap \tau|}{|Z_{i}|} \geq {\frac{|Y_{i}|}{|\Xi_{i}|}} \cdot \frac{1}{|Z_{i}|} \geq
\frac{n_{i}}{8k} \cdot \frac{k}{2n_{i}} \cdot \frac{1}{|\Xi_{i}|} \geq
\frac{k}{n_{i}} \cdot \frac{(d-1)}{16 c_{3}c_{4}^{d-1}} \cdot \log \frac{n_{i}}{k} = A \frac{k}{n_{i}},
 \label{eq_Z_measure}
A = \frac{d-1}{16 c_{3}c_{4}^{d-1}} \log \frac{n_{i}}{k}.

|P_{i}' \cap \tau | \geq p n_{i} = c_{2}k \log \frac{n_{i}}{c_{2}k} \geq c_{2}k \geq \max \{d+1,2k\}.

\overline{P_{i}'}(\tau) \geq \overline{Z_{i}}(\tau) - \eps p \geq
\frac{k}{n_{i}} \left(A - \frac{c_{2}}{2}\log \frac{n_{i}}{c_{2}k}\right),

|P_{i}' \cap \tau | \geq k \left(A - \frac{c_{2}}{2}\log \frac{n_{i}}{c_{2}k}\right).

\frac{d-1}{16 c_{3}c_{4}^{d-1}} \log \frac{n_{i}}{k} \geq D + \frac{c_{2}}{2}\log \frac{n_{i}}{c_{2}k},

\left(\frac{d-1}{16 c_{3}c_{4}^{d-1}} - \frac{c_{2}}{2}\right) \log \frac{n_{i}}{k} \geq D - \frac{c_2}{2}\log c_{2}.
 \label{eq_partition_constants}
\frac{d-1}{2 c_{3}c_{4}^{d-1}} \geq 4c_{2} + D - \frac{c_{2}}{2}\log c_{2}.

w_{i+1}(Q) &\leq w_{i}(Q) + \frac{w_{i}(Q)}{t_{i}} = w_{i}(Q)\left(1 + \frac{1}{t_{i}} \right) \\
&\leq w_{i}(Q)\left(1 + \frac{1}{c_{4}} \left(\frac{\log^{2}(n_{i}/k)}{n_{i}/k} \right)^{1/(d-1)} \right).

w_{q}(Q) \leq |Q|\prod_{i=0}^{q-1}\left(1 + \frac{1}{t_{i}} \right) \leq |Q|\prod_{i=0}^{q-1} \left(1 + \frac{1}{c_{4}} \left(\frac{\log^{2}(r-i)}{r-i}\right)^{1/(d-1)} \right).

\log w_{q}(Q) \leq \log |Q| + \frac{1}{c_{4}\ln 2} \sum_{i=0}^{q-1}\left(\frac{\log^{2}(r-i)}{r-i}\right)^{1/(d-1)}
 \label{eq_partition_step}
\leq \log |Q| + \frac{1}{c_{4}^{'}} \sum_{j=1}^{r}\left(\frac{\log^{2} j}{j}\right)^{1/(d-1)},

\kappa & \leq \log w_{q}(Q) \leq \log|Q| + \frac{1}{c_{4}^{'}}\log^{2/(d-1)}r \sum_{j=1}^{r}\left(\frac{1}{j}\right)^{1/(d-1)} \\
& = O\left(\log|Q| + r^{1-1/(d-1)} \log^{2/(d-1)}r\right) \\
& = O \left(\log |Q| + (n/k)^{1 - 1/(d-1)} \log^{2/(d-1)}(n/k)\right),

t_{i} = c_{4} \cdot \frac{n_{i}/k}{\alpha(n_{i}/k) \log (n_{i}/k)},

|\Xi_{i}| \leq c_{3} t_{i}\alpha(t_{i}) \leq c_{3}c_{4} \cdot \frac{n_{i}}{k} \cdot \frac{1}{\log\frac{n_{i}}{k}},

\kappa \leq \log w_{q}(Q) \leq \log |Q| + \frac{1}{c_{4}^{'}} \sum_{j=1}^{r}\frac{\alpha(j) \log j }{j} \leq \log |Q| + \frac{1}{c_{4}^{'}} \alpha(r) \log r \sum_{j=1}^{r} \frac{1}{j} =

O\left(\log |Q| + \alpha(r) \log^{2}r \right) = O\left(\log |Q| + \alpha(n/k) \log^{2}(n/k) \right),

(d + 1)\kappa_{0} + O\left((n/k)^{1-1/(d-1)} \right),

O\left(\log |Q| + (n/k)^{1 - 1/(d-1)} \log^{2/(d-1)}(n/k)\right) = O\left((n/k)^{1 - 1/(d-1)} \log^{2/(d-1)}(n/k)\right)

(d+1) \cdot O\left((n/k)^{1 - 1/(d-1)} \log^{2/(d-1)}(n/k)\right) + O\left((n/k)^{1-1/(d-1)} \right)

= O\left((n/k)^{1-1/(d-1)} \log^{2/(d-1)}(n/k)\right),

O\left(\sum_{j \geq 0} \left(\left(\tfrac{3}{4}\right)^{j} \tfrac{n}{k} \right)^{1-1/(d-1)} \log^{2/(d-1)} \left(\left(\tfrac{3}{4} \right)^{j} \tfrac{n}{k} \right)\right)
= O\left((n/k)^{1-1/(d-1)} \log^{2/(d-1)}(n/k)\right).

O\left((n/k)^{1-1/(d-1)} \log^{2/(d-1)}(n/k) \cdot k^{1-1/d}\right) = O\left(n^{1-1/(d-1)} k^{1/d(d-1)} \log^{2/(d-1)}(n/k)\right)

O\left(\sum_{j \geq 0}\alpha \left(\left(\tfrac{3}{4}\right)^{j} \tfrac{n}{k} \right)
\log^{2}\left(\left(\tfrac{3}{4}\right)^{j}\tfrac{n}{k}\right) \right)
& = O\left(\alpha(n/k) \sum_{j \geq 0}\left(\log(n/k) - j \right)^{2}\right) \\
& = O\left(\alpha(n/k) \log^{3}(n/k) \right).

O\left(\alpha(n/k) \log^{2}(n/k) (\sqrt{k} + \log(n/k))\right),

& \sum_{i=1}^{U} O\left(n^{1-1/(d-1)}(2^{i})^{1/d(d-1)} \log^{2/(d-1)}(n/2^{i})\right) \\
& = O\left(n^{1-1/(d-1)}\sum_{i=1}^{U}\left(2^{1/d(d-1)}\right)^{i}(\log n - i)^{2/(d-1)}\right) \\
& = O\left(n^{1-1/d}\sum_{j=\log n - U}^{\log n - 1}\frac{j^{2/(d-1)}}{\left(2^{1/d(d-1)}\right)^{j}}\right).

\int_{a}^{\infty} \frac{x^{2/(d-1)}}{(2^{1/d(d-1)})^{x}} dx.
 \label{eq_integral}
\int_{a}^{\infty}\frac{x^{u}}{w^{x}}dx \leq \frac{1}{\ln w} \cdot \frac{a^{u}}{w^{a}} + \frac{u}{\ln^{2}w} \cdot \frac{1}{w^{a}} = O\left(\frac{a^{u}}{w^{a}}\right),

O\left(n^{1-1/d}(k/n)^{1/d(d-1)} \log^{2/(d-1)}(n/k) \right) = O\left(n^{1-1/(d-1)}k^{1/d(d-1)} \log^{2/(d-1)}(n/k) \right),

\sum_{i=1}^{U}O\left(\alpha(n/2^{i}) \log^{2}(n/2^{i}) \left(\sqrt{2^{i}} + \log(n/2^{i})\right) \right),

I_{1} = \sum_{i=1}^{U} O\left(\sqrt{2^{i}} \alpha(n/2^{i}) \log^{2}(n/2^{i})\right), \quad \mbox{and} \quad I_{2} = \sum_{i=1}^{U} O\left(\alpha(n/2^{i})\log^{3}(n/2^{i})\right).
 \label{eq_alpha_sum}
\sum_{i=1}^{U}O\left(\sqrt{2^{i}}(\log n - i)^{2}\alpha(n/2^{i})\right) =
O\left(\sqrt n \sum_{j=\log n - U}^{\log n - 1}\frac{j^{2}\alpha(2^{j})}{\sqrt{2^{j}}}\right).

\frac{\alpha(2x)}{\alpha(x)} \leq 1 + \frac{1}{\alpha(x)} \leq \frac{4}{3}.

S \leq \alpha(n/k) \sum_{j=\log n - U}^{\log n - 1} \frac{j^{2} \left(\tfrac{4}{3}\right)^{j - \log n + U}}{\sqrt{2^{j}}} \leq \alpha(n/k) \left(\tfrac{3}{4}\right)^{\log n - U} \sum_{j=\log n - U}^{\log n - 1} j^{2} \left(\tfrac{4}{3\sqrt{2}}\right)^{j}.

S \leq c_{1} \cdot \alpha(n/k) \left(\tfrac{3}{4}\right)^{\log n - U} \log^{2}(n/k) \left(\tfrac{4}{3\sqrt{2}}\right)^{\log n - U} = O\left(\sqrt{\frac{k}{n}} \alpha(n/k) \log^{2}(n/k)\right),

I_{1} = O\left(\sqrt{k} \alpha(n/k) \log^{2}(n/k)\right).

I_{2} = \sum_{i=1}^{U} O\left(\alpha(n/2^{i})\log^{3}(n/2^{i})\right) = O\left(\alpha(n)\log^{4}n \right).

O\left(\sqrt{k}\alpha(n/k)\log^{2}(n/k) + \alpha(n)\log^{4}n \right).

\chi(P \cap H) = \sum_{p \in P \cap H}\chi(p) = O\left(n^{(d-2)/2(d-1)} |P \cap H|^{1/2d(d-1)} \log^{(d+1)/2(d-1)}n \right).

\gamma = \frac{2d(d-1)-1}{(d-1)(d+1)}, \quad \quad \mu = \frac{2d}{d+1}, \quad \text{and} \quad \quad \eta = \frac{d}{d-1}.

O\left(\frac{1}{\eps^{\mu} p^{\gamma}} \log^{\eta} \frac{1}{\eps p}\right).

\frac{1}{\eps^{\mu}p^{\gamma}} < \frac{1}{\eps^{2}p}, \quad\quad \text{or} \quad\quad \eps < p^{\frac{\gamma-1}{2-\mu}}.

\eps < p^{\frac{d(d-2)}{2(d-1)}},

\left| |P_{i} \cap H| - |P_{i}' \cap H|\right| & \leq c_{1} \cdot n_{i-1}^{(d-2)/2(d-1)} \cdot |P_{i-1} \cap H|^{1/2d(d-1)} \cdot \log^{\frac{d+1}{2(d-1)}}n_{i-1} \\
& \leq c_{2} \cdot n_{i}^{(d-2)/2(d-1)} \cdot |P_{i-1} \cap H|^{1/2d(d-1)} \cdot \log^{\frac{d+1}{2(d-1)}}n_{i},

\meas{P_{i}}{H} = \frac{|P_{i} \cap H|}{|P_{i}|} \quad \text{and} \quad \meas{P_{i}'}{H} = \frac{|P_{i}' \cap H|}{|P_{i}'|},

|\meas{P_{i}}{H} - \meas{P_{i}'}{H}| \leq c_{2} \frac{|P_{i-1} \cap H|^{1/2d(d-1)}}{n_{i}^{1 - (d-2)/2(d-1)}} \log^{\frac{d+1}{2(d-1)}} n_{i}= c_{3} \frac{ \meas{P_{i-1}}{H}^{1/2d(d-1)}}{n_{i}^{(d+1)/2d}} \log ^{\frac{d+1}{2(d-1)}}n_{i},

\meas{P_{i-1}}{H} = \frac{|P_{i-1} \cap H|}{|P_{i-1}|} = \frac{|P_{i} \cap H|}{2|P_{i}|} + \frac{|P_{i}' \cap H|}{2|P_{i}'|} = \frac{1}{2}\left(\meas{P_{i}}{H} + \meas{P_{i}'}{H}    \right).

|\meas{P_{i-1}}{H} - \meas{P_{i}}{H}| = \frac{|\meas{P_{i}}{H} - \meas{P_{i}'}{H}|}{2} \leq
c_{4} \frac{\meas{P_{i-1}}{H}^{1/2d(d-1)}}{n_{i}^{(d+1)/2d}} \log^{\frac{d+1}{2(d-1)}}n_{i},

\frac{c_{4} \cdot \log^{\frac{d+1}{2(d-1)}}n_{i}}{n_{i}^{(d+1)/2d}} \cdot \frac{\meas{P_{i-1}}{H} + p }{p^{1-1/2d(d-1)}} \leq \frac{c_{4} \cdot \log^{\frac{d+1}{2(d-1)}}n_{i}}{n_{i}^{(d+1)/2d} p^{1-1/2d(d-1)}} \left(\meas{P_{i-1}}{H} + \meas{P_{i}}{H} + p \right).

d_{p}(\meas{P_{i-1}}{H}, \meas{P_{i}}{H}) \leq \frac{c_{4} \cdot \log^{\frac{d+1}{2(d-1)}} n_{i}}{n_{i}^{(d+1)/2d} p^{1-1/2d(d-1)}}.

d_{p}(\meas{P}{H}, \meas{P_{i}}{H}) &\leq \sum_{k=1}^{i}d_{p}(\meas{P_{k-1}}{H}, \meas{P_{k}}{H}) \leq \frac{c_{4}}{p^{1-1/2d(d-1)}}\sum_{k=1}^{i}\frac{\log^{\frac{d+1}{2(d-1)}} n_{k}}{n_{k}^{(d+1)/2d}} \\
&= O\left(\frac{\log^{\frac{d+1}{2(d-1)}}n_{i}}{p^{1-1/2d(d-1)} n_{i}^{(d+1)/2d}}\right).

\max \left\{ O\left(\log^{\frac{3d+1}{d+1}} n \right) , \;
O\left(\frac{ \log^{\frac{3d+1}{d+1}} \frac{1}{\eps p}}
{ p^{\frac{d+\lfloor d/2 \rfloor}{d+1}} \eps^{\frac{2d}{d+1}} }
\right) \right\} ,
 \label{eq_range_counting_time}
Q(n) =
\begin{cases}
    O\left(n^{1-\lfloor d/2 \rfloor} \log^{O(1)}n + w_{H}\right), & \mbox{for }, \\
    O\left(n^{1-1/(d-1)} w_{H}^{1/d(d-1)} \log^{O(1)}n \right), & \mbox{otherwise},
\end{cases}

T(n) =
\begin{cases}
    O\left(n^{1 + \frac{(d-2)(d-1)}{2(d(d-1)-1)}} \log^{O(1)}n  \right), & \mbox{for  even}, \\
    O\left(n^{1 + \frac{(d-3)(d-1)}{2(d(d-1)-1)}}\log^{O(1)}n  \right), & \mbox{for  odd},
\end{cases}
 \label{eq_range_counting_k1}
k_{1} = c_{1} n^{1 - \frac{d-1}{d(d-1)-1}},

O\left(n^{1-1/ \lfloor d/2 \rfloor} \log^{O(1)}n + |P \cap H|\right),

O\left((n/k_{1})^{1-1/(d-1)} \cdot k_{1}^{1-1/d} \log^{O(1)}n \right) = O\left(n^{1-1/(d-1)} k_{1}^{1/d(d-1)} \log^{O(1)}n \right).

n^{1-1/\lfloor d/2 \rfloor} \log^{O(1)}n < k_{1} = \Theta\left(n^{1-1/(d-1)}k_{1}^{1/d(d-1)}\right),

O\left(\sum_{i=1}^{U}(n/k_{i})^{1-1/(d-1)} k_{i}^{1-1/d} \log^{O(1)}n \right) & = O\left(\sum_{i=1}^{U}n^{1-1/(d-1)} k_{i}^{1/d(d-1)} \log^{O(1)}n \right) \\
& = O\left(n^{1-1/(d-1)} w_{H}^{1/d(d-1)} \log^{O(1)}n \right).

k > k_{1} = c_{1} n^{1 - (d-1)/(d(d-1)-1)},
 \label{eq_range_counting_preproc}
\frac{n}{k} = O\left(n^{(d-1)/(d(d-1)-1)}\right),

k_{j-1}+1 > 2^{j-2}k_{1} \geq 2^{(\lceil \log (w_{H}/k_{1}) \rceil + 2) - 2} \cdot k_{1} \geq w_{H}.

u &= O\left(|R|^{\lfloor d/2 \rfloor} t^{\lceil d/2 \rceil} \right) = O\left(|R|^{\lfloor d/2 \rfloor} \log^{O(1)}n \right) \\
&= O\left((r\log r)^{\lfloor d/2 \rfloor} \log^{O(1)}n \right) = O\left((n/k)^{\lfloor d/2 \rfloor} \log^{O(1)}n \right).

(n/k)^{\lfloor d/2 \rfloor} = O\left(n^{\frac{(d-1)\lfloor d/2 \rfloor}{d(d-1)-1}}\right),
 \label{eq_range_counting_t_level}
u =
\begin{cases}
    O\left(n^{\frac{1}{2}\left(1 + \frac{1}{d(d-1)-1}\right)} \log^{O(1)}n \right), & \mbox{for  even}, \\
    O\left(n^{\frac{1}{2}\left(1 - \frac{d-2}{d(d-1)-1}\right)} \log^{O(1)}n \right), & \mbox{for  odd},
\end{cases}

O\left(n \cdot u^{1-1/\lfloor d/2 \rfloor} \log^{O(1)}u \right).

O\left(n^{1 + \frac{(d-2)(d-1)}{2(d(d-1)-1)}} \log^{O(1)}n \right)

O\left(n^{1 + \frac{(d-3)(d-1)}{2(d(d-1)-1)}} \log^{O(1)}n \right)

expected time. Having a logarithmic number of steps, we construct all the sets  within the same asymptotic expected bound. We note that the overall time consumed by the construction of the other data structure components in each step is smaller than the bound above. We omit the description of these other steps of the construction, as they are identical to those given in \cite{mat92a, mat92b}. This completes the proof of Theorem \ref{thm_range_counting}. 


\begin{thebibliography}{}

\bibitem[AdBMS98]{adbms98}
P. K. Agarwal, M. de Berg, J. Matou\v{s}ek, and O. Schwarzkopf,
Constructing levels in arrangements and higher order Voronoi diagrams,
{\em SIAM J. Comput.}, 27(3):657--667, 1998.

\bibitem[AM95]{am95}
P. K. Agarwal and J. Matou\v{s}ek,
Dynamic half-space range reporting and its applications,
{\em Algorithmica}, 13:325--345, 1995.

\bibitem[AMS98]{ams98}
P. K. Agarwal, J. Matou\v{s}ek, and O. Schwarzkopf,
Computing many faces in arrangements of lines and segments,
{\em SIAM J. Comput.}, 27(2):491--505, 1998.

\bibitem[APS93]{aps93}
B. Aronov, M. Pellegrini, and M. Sharir,
On the zone of a surface in a hyperplane arrangement,
{\em Discrete Comput. Geom.}, 9(2):177--186, 1993.

\bibitem[AS92]{as92}
N. Alon and J. H. Spencer,
{\em The Probabilistic Method},
Publisher, City, 1992. Appendix A: {\em Bounding of Large Deviations}, pp. 233--240.

\bibitem[Ch12]{ch12}
T. M. Chan,
Optimal partition trees,
{\em Discrete Comput. Geom.}

\bibitem[CF90]{cf90}
B. Chazelle and J. Friedman,
A deterministic view of random sampling and its use in geometry,
{\em Combinatorica} 10 (3):229--249, 1990.

\bibitem[Cha01]{cha01}
B. Chazelle,
{\em The Discrepancy Method: Randomness and Complexity},
Cambridge University Press, New York, 2001.

\bibitem[Cha93]{cha93}
B. Chazelle,
An optimal convex hull algorithm in any fixed dimension,
{\em Discrete Comput. Geom.}, 10:377--409, 1993.

\bibitem[CW89]{cw89}
B. Chazelle and E. Welzl,
Quasi-optimal range searching in spaces with finite VC-dimension,
{\em Discrete Comput. Geom.}, 4:387--421, 1989.

\bibitem[CS89]{cs89}
K. L. Clarkson and P. W. Shor,
Applications of random sampling in computational geometry, II,
{\em Discrete Comput. Geom.}, 4:467--490, 1989.

\bibitem[Ed87]{ed87}
H. Edelsbrunner,
{\em Algorithms in Combinatorial Geometry},
Springer-Verlag, Heidelberg, 1987.

\bibitem[Ez14]{ez14}
E. Ezra,
A size-sensitive discrepancy bound for set systems of bounded primal
shatter dimension,
{\em Proc. ACM--SIAM Annu. Sympos. Discrete Algorithms}, 2014, to appear.
Also available at \texttt{http://www.cims.nyu.edu/~esther/Publications/discrepancy.pdf}

\bibitem[HS11]{hs11}
S. Har-Peled and M. Sharir,
Relative -approximations in geometry,
{\em Discrete Comput. Geom.} 45:462--496, 2011.

\bibitem[HW87]{hw87}
D. Haussler and E. Welzl,
Epsilon-nets and simplex range queries,
{\em Discrete Comput. Geom.}, 2:127--151, 1987.

\bibitem[LLS01]{lls01}
Y. Li, P. M. Long, and A. Srinivasan,
Improved bounds on the sample complexity of learning,
{\em J. Comput. Syst. Sci.}, 62:516--527, 2001.

\bibitem[Mat91a]{mat91a}
J. Matou\v{s}ek,
Cutting hyperplane arrangements,
{\em Discrete Comput. Geom.}, 6(5):385--406, 1991.

\bibitem[Mat91b]{mat91b}
J. Matou\v{s}ek,
Computing the center of planar point sets,
in {\em Computational Geometry: Papars from the DIMACS Special Year}
(J. E. Goodman, R. Pollack and W. Steiger, eds.),
AMS, Providence, RI, pp. 221--230, 1991.

\bibitem[Mat92a]{mat92a}
J. Matou\v{s}ek,
Efficient partition trees,
{\em Discrete Comput. Geom.}, 8(3):315--334, 1992.

\bibitem[Mat92b]{mat92b}
J. Matou\v{s}ek,
Reporting points in halfspaces,
{\em Comput. Geom. Theory Appl.}, 2(3):169--186, 1992.

\bibitem[Mat02]{mat02}
J. Matou\v{s}ek,
{\em Lectures on Discrete Geometry}, Springer-Verlag, Berlin, 2002;
Centerpoint theorem: pp. 14--15, Cuttings: pp. 152--163, VC-Dimension: pp. 237--243.

\bibitem[Mat99]{mat99}
J. Matou\v{s}ek,
{\em Geometric Discrepancy}, Algorithms and Combinatorics,
Vol. 18, Springer Verlag, Heidelberg, 1999.

\bibitem[ShSh11]{shsh11}
M. Sharir and H. Shaul,
Semi-algebraic range reporting and emptiness searching with applications,
{\em Siam J. Comput.}

\bibitem[Wel92]{wel92}
E. Welzl, On spanning trees with low crossing numbers, in
{\em Data Structures and Efficient Algorithms, Final Report on the DFG Special Joint Initiative},
volume 594 of {\em Lect. Notes in Comp. Sci.}, Springer-Verlag, Heidelberg, pp. 233--249, 1992.

\end{thebibliography}

\end{document}
