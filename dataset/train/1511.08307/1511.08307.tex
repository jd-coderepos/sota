\documentclass[preprint]{sigplanconf}




\usepackage{amsmath}
\usepackage{graphicx}
\usepackage{latexsym}
\usepackage{url}

\usepackage{framed}

\usepackage{color}
\definecolor{darkblue}{rgb}{0.0,0,0.5} \definecolor{byzantium}{rgb}{0.44, 0.16, 0.39}
\newcommand{\kw}[1]{{\tt {\color{darkblue} #1}}}
\newcommand{\nt}[1]{{\tt #1}}

\newcommand{\TODO}[1]{{\color{blue} [TODO: #1]}}
\newcommand{\FIXME}[1]{{\color{red} {\bf FIXME}: #1}}



\begin{document}

\special{papersize=8.5in,11in}
\setlength{\pdfpageheight}{\paperheight}
\setlength{\pdfpagewidth}{\paperwidth}

\conferenceinfo{CONF 'yy}{Month d--d, 20yy, City, ST, Country} 
\copyrightyear{20yy} 
\copyrightdata{978-1-nnnn-nnnn-n/yy/mm} 
\doi{nnnnnnn.nnnnnnn}





\permissiontopublish             

\preprintfooter{Unpublished, draft paper to be revised. Comments and suggestions are welcome}   

\title{Nez: Practical Open Grammar Language}


\authorinfo{Kimio Kuramitsu}
         {Yokohama National University, JAPAN}
         {kimio@ynu.ac.jp\\http://nez-peg.github.io/}



\maketitle

\begin{abstract}

Nez is a PEG(Parsing Expressing Grammar)-based open grammar language that allows us to describe complex syntax constructs without action code. Since open grammars are declarative and free from a host programming language of parsers, software engineering tools and other parser applications can reuse once-defined grammars across programming languages. 

A key challenge to achieve practical open grammars is the expressiveness of syntax constructs and the resulting parser performance, as the traditional action code approach has provided very pragmatic solutions to these two issues. In Nez, we extend the symbol-based state management to recognize context-sensitive language syntax, which often appears in major programming languages. In addition, the Abstract Syntax Tree constructor allows us to make flexible tree structures, including the left-associative pair of trees. Due to these extensions, we have demonstrated that Nez can parse not all but many grammars.

Nez can generate various types of parsers since all Nez operations are independent of a specific parser language. To highlight this feature, we have implemented Nez with dynamic parsing, which allows users to integrate a Nez parser as a parser library that loads a grammar at runtime. To achieve its practical performance, Nez operators are assembled into low-level virtual machine instructions, including automated state modifications when backtracking, transactional controls of AST construction, and efficient memoization in packrat parsing. We demonstrate that Nez dynamic parsers achieve very competitive performance compared to existing efficient parser generators. 

\end{abstract}

\category{D.3.1}{Programming Languages}{Formal Definitions and Theory -- Syntax}
\category{D.3.4}{Programming Languages}{Processors -- Parsing}

\terms
Languages, Algorithms

\keywords
Parsing expression grammars, Context-sensitive grammars, Packrat parsing

\section{Introduction}

A parser generator is a standard approach to reliable parser development in programming languages and many parser applications. Developers use a formal grammar, such as LALR(), LL(), or PEG, to specify programming language syntax. Based on the grammar specification, a parser generator tool, such as Yacc\cite{Yacc}, ANTLR3/4\cite{PLDI11_Antlr}, or Rats\cite{PLDI06_Rats}, produces an efficient parser code that can be integrated with the host language of the compiler and interpreter. This generative approach, obviously, enables developers to avoid error-prone coding for lexical and syntactic analysis. 

Traditional parser generators, however, are not entirely free from coding. One particular reason is that the formal grammars used today lack sufficient expressiveness for many of the popular programming language syntaxes, such as typedef in C/C++ and other context-sensitive syntaxes\cite{POPL04_PEG,PLDI06_Rats,ONWARD15_Iguana}. In addition, a formal grammar itself is a syntactic specification of the input language, not a schematic specification for tree representations that are needed in semantic analysis. To make up for these limitations, most parser generators take an ad hoc approach called {\em semantic action}, a fragment of code embedded in a formal grammar specification. The embedded code is written in a host language and combined in a way that it is hooked at a parsing context that requires extended recognition or tree constructions. 

The problem with arbitrary action code is that they decrease the reusability of the grammar specification\cite{ICPC08_SemanticActions}. For example, Yacc cannot generate the Java version of a parser simply because Yacc uses C-based action code. There are many Yacc ports to other languages, but they do not reuse the original Yacc grammar without porting of action code. This is an undesirable situation because the well-defined grammar is demanded everywhere in IDEs and other software engineering tools\cite{OOPSLA11_Spoofax}. 

Nez is a grammar language designed for open grammars. In other words, our aim is that once we write a grammar specification, anyone can use the grammar in any programming language. To achieve the openness of the grammar specification, Nez eliminates any action code from the beginning, and provides a declarative but small set of operators that allow the context-sensitive parsing and flexible Abstract Syntax Tree construction that are needed for popular programming language syntaxes. This paper presents the implementation status of Nez with the experience of our grammar development.

The Nez grammar language is based on {\em parsing expression grammars (PEGs)}, a popular syntactic foundation formalized by Ford\cite{POPL04_PEG}. PEGs are simple and portable and have many desirable properties for defining modern programming language syntax, but they still have several limitations in terms of defining popular programming languages. Typical limitations include {\tt typedef}-defined name in C/C++ \cite{POPL04_PEG,PLDI06_Rats}, delimiting identifiers (such as \verb|<<END|) of the Here document in Perl and Ruby, and indentation-based code layout in Python and Haskell \cite{POPL13_Indentation}. 

In the first contribution of Nez, we model a single extended parser state, called a {\em symbol table}, to describe a variety of context-sensitive patterns appearing in programming language constructs. 
The key idea behind the symbol table is simple. We allow any parsed substrings to be handled as {\em symbols}, a specialized word in another context. The symbol table is a runtime, growing, and recursive storage for such symbols. If we handle typedef-defined names as symbols for example, we can realize different parsing behavior through referencing the symbol table. 
Nez provides a small set of symbol operators, including matching, containment testing, and scoping. 
More importantly, since the symbol table is a single, unified data structure for managing various types of symbols, we can easily trace state changes and then automate state management when backtracking. In addition, the symbol table itself is simply a stack-based data structure;  we can translate it into any programming language. 

Another important role of the traditional action code is the construction of ASTs. Basically, PEGs are just a syntactic specification, not a schematic specification for tree structures that represent AST nodes. In particular, it is hard to derive the left-associative pair of trees due to the limited left recursion in PEGs. In Nez, we introduce an operator to specify tree structures, modeled on capturing in perl compatible regular expressions (PCREs), and extend tree manipulations including tagging for typing a tree, labeling for sub-nodes, and left-folding for the left-associative structure. Due to these extensions, a Nez parser produces arbitrary representations of ASTs, which would leads to less impedance mismatching in semantic analysis.
 
To evaluate the expressiveness of Nez, we have performed extensive case studies to specify various popular programming languages, including C, C\#, Java8, JavaScript, Python, Ruby, and Lua. Surprisingly, the introduction of a simple symbol table improves the declarative expressiveness of PEGs in a way that Nez can recognize almost all of our studied languages and then produce a practical representation of ASTs.  
Our case studies are not complete, but they indicate that the Nez approach is promising for a practical grammar specification without any action code. 

At last but not least, parsing performance is an important factor since the acceptance of a parser generator relies definitively on practical performance. In this light, Nez can generate three types of differently implemented parsers, based on the traditional source generation, the grammar translation, and the parsing machine\cite{LPeg}. From the perspective of open grammars, we highlight the latter parsing machine that enables a PCRE-style parser library that loads a grammar at runtime. To achieve practical performance,
Nez operators are assembled into low-level virtual instructions, including automated state modifications when backtracking, transactional controls of AST construction, and efficient memoization in packrat parsing. We will demonstrate that the resulting parsers are portable and achieve very competitive performance compared to other existing standard parsers for Java, JavaScript, and XML. 

The remainder of this paper proceeds as follows. 
Section \ref{sec:nez} is an informal introduction to the Nez grammar specification language. 
Section \ref{sec:design} is a formal definition of Nez. 
Section \ref{sec:condition} describes eliminating parsing conditions from Nez.
Section \ref{sec:impl} describes parser runtime and implementation.
Section \ref{sec:casestudies} reports a summary of our case studies.
Section \ref{sec:perf} studies the parser performance.
Section \ref{sec:relatedwork} briefly reviews related work.
Section \ref{sec:conclusion} concludes the paper.
The tools and grammars presented in this paper 
are available online at \url{http://nez-peg.github.io/}

\section{A Taste of Nez} \label{sec:nez}

This section is an informal introduction to the Nez grammar specification language.

\begin{table}[bt]
\begin{center}
\begin{tabular}{llll} \hline
PEG  & Type & Proc. & Description\\ \hline
\verb|' '| & Primary & 5 & Matches text\\
 & Primary & 5 & Matches character class \\
 & Primary & 5 & Any character\\
 & Primary & 5 & Non-terminal application\\
 & Primary & 5 & Grouping\\
 & Unary suffix & 4 & Option\\
 & Unary suffix & 4 & Zero-or-more repetitions\\
 & Unary suffix & 4 & One-or-more repetitions\\
 & Unary prefix & 3 & And-predicate\\
 & Unary prefix & 3 & Negation\\
 & Binary & 2 & Sequencing\\
 & Binary & 1 & Prioritized Choice\\ \hline

AST  &  &  & \\ \hline

 & Construction & 5 & Constructor\\
(e)\{\ & Construction & 5 & Left-folding\\ 
 & Construction & 5 & Tagging \\
\verb|` `| & Construction & 5 & Replaces a string \\ \hline

Symbols  &  &  & \\ \hline

 & Action & 5 & Symbolize Nonterminal  \\ 
 & Predicate & 5 & Exists symbols \\ 
 & Predicate & 5 & Exists  symbol \\ 
 & Predicate & 5 & Matches symbol \\ 
 & Predicate & 5 & Equals symbol \\ 
 & Predicate & 5 & Contains symbol \\ 
 & Action & 5 & Nested scope for  \\ 
 & Action & 5 & Isolated local scope for  \\ \hline 

Conditional  &  &  & \\ \hline

 & Action & 5 &  on c is defined \\ 
 & Action & 5 &  on c is undefined \\ 
 & Predicate & 5 & If c is defined  \\ 
 & Predicate & 5 & If c is undefined  \\ \hline
\end{tabular}

\caption{Nez operators: "Action" stands for symbol mutators and "predicate" stands for symbol predicates. } 
\label{table:nez}

\end{center}
\end{table}

\subsection{Nez and Parsing Expression}

Nez is a PEG-based grammar specification language. The basic constructs come from those of PEGs, such as production rules and parsing expressions. A Nez grammar is a set of production rules, each of which is defined by a mapping from a nonterminal  to a parsing expression :



Table \ref{table:nez} shows a list of Nez operators that constitute the parsing expressions. All PEG operators inherit the formal interpretation of PEGs\cite{POPL04_PEG}. That is, the string \verb|'abc'| exactly matches the same input, while \verb|[abc]| matches one of these characters. The . operator matches any single character. The , , and  expressions behave as in common regular expressions, except that they are greedy and match until the longest position. The  attempts two expressions  and  sequentially, backtracking to the starting position if either expression fails.  The choice  first attempt  and then attempt  if  fails. The expression  attempts  without any character consuming. The expression  fails if  succeeds, but fails if  succeeds. Furthermore information on PEG operators is detailed in \cite{POPL04_PEG}.

The expressiveness of PEGs is almost similar to that of {\em deterministic} context-free grammars (such as LALR and LL grammars). In general, PEGs are said to express all LALR grammar languages, which are widely used in a standard parser generator such as Lex/Yacc\cite{Yacc}. In addition, PEGs have many desirable properties that are characterized by:

\begin{itemize}
\item {\em deterministic behavior}, avoiding the dangling if-else problem
\item {\em left recursion-free}, preserving operator precedence
\item {\em scanner-less parsing}, avoiding the extensive use of lexer hacks such as in C++ grammars, and 
\item {\em unlimited lookahead}, recognizing highly nested structures such as        . 
\end{itemize}

Nez inherits all these properties from PEGs, and has extended features based on AST operators, symbol operators and conditional parsing, as listed in Table \ref{table:nez}. 

\subsection{AST Operators}

The first extension of parsing expressions in Nez is a flexible construction of AST representations. Each node of an AST contains a substring extracted from the input characters. Nez has adopted an PCRE-like capturing operator, denoted by , to capture the substring. The productions {\tt Int} and {\tt Long} below are examples of capturing a sequence of digits (as defined in {\tt NUM}.)

\begin{verbatim}
  NUM = [0-9]+
  Int = { NUM }
  Long = { NUM } [Ll]
\end{verbatim}

A string to be captured is the exact one matched with {\tt NUM}. That is, {\tt Long} accepts the input \verb|0L| but captures the only substring \verb|0|. An empty string may be captured by an empty expression. 

Tagging is introduced to distinguish the type of nodes. We can just add a \verb|#|-prefixed tag to each of the nodes. 

\begin{verbatim}
  Int = { NUM #Int }
  Long = { NUM #Long } [Ll]
\end{verbatim}

A backquoat operator \verb|` `| is a string operator that replaces the captured string with the specified string. Using the empty capture, we are allowed to create an arbitrary node at any point of a parsing expression.  

\begin{verbatim}
  DefaultValue = { `0` #Int }
\end{verbatim}

The connector \verb|| is used to specify a child node and append it to a node that is constructed on  the left-hand side of \verb|(Int) ('+' (Int) ('+'  ...}| that allows a left-hand-constructed node to be contained in a new right-hand node. Accordingly, we can make a precedence-preserving form of ASTs by folding from the repetition. 

\begin{verbatim}
  Add = Int {(int) #Add}* 
\end{verbatim}

The connector and the left-folding operator can associate an optional label for the connected child node. The associated label follows the \verb|left '+' left ('+' #Add / '-' #Sub) left ('*' #Mul / '/' #Div) AAA\verb|<exists|~A \verb|>|A\verb|<exists|~A~s \verb|>|As\verb|<match|~A \verb|>|A\verb|<is|~A \verb|>|AA\verb|<isa|~A \verb|>|AA\verb|<match|~A \verb|>|\verb|<is|~A \verb|>|\verb|<block|~e\verb|>|\verb|<block|~e\verb|>|\verb|<local|~A~e~\verb|>|Aec\verb|<if| ~c\verb|>| ~ eec!c\verb|<if| ~!c\verb|>|!\verb|<if| ~c\verb|>|e_1e_2\verb|<if|~c\verb|>| ~ e_1 ~/~ \verb|<if|~!c\verb|>| ~ e2\verb|<on| ~c~e\verb|>|ec\verb|<on| ~!c~e\verb|>|ec\verb|<on| ~{\tt IgnoreNewLine} ~e\verb|>|CC = \verb|''|\verb|<on|~c~e~\verb|>|\verb|<block <symbol|~C~ \verb|>|~e~ \verb|>|\verb|<if|~c~\verb|>|\verb|<exists|~C~\verb|''>|A,B,C \in Na,b \in \Sigmax,y,z \in \Sigma^{*}xyeA=eT[A, x]AxS=[A,x][A', y]\epsilonv\#tT\texttt{new}(x)xvv = v'v'v\text{tag}(v, \#t)v\#t\texttt{replace}(v, x)vx\text{link}(v, v')v'v[A, x]xAS[A,x][B,y] ... S_1 = [A,x][B,y]S_2 = [A,x][B,y][A,z]\epsilon[A,x] \in S[A,x]SS [A,z][A, z]SS S'SS'S_1 [A,z] = S_2S_1 \subset S_2Stop(S, A)Atop(S_1, A) = xtop(S_2, A) = zdel(S, A)ASG = (N, \Sigma, P, e_s, \mathcal{T})N\SigmaPe_s\mathcal{T}e\epsilonAe ~ e' e ~ / ~ e' e*ee\{~e~\}\{\ &  : left-folding  \\
& \verb#|  # & (e)\#t`x`\langle symbol~A\rangle\langle exists~A\langle exists~A~x\rangle\langle match~A\rangle\langle is~A\rangle\langle isa~A\rangle\langle block~e\rangle\langle local~A~e\ranglep \in PAeA=eA \in Nee?e / \epsilone*A' = e A' / \epsilone+e e*(S, xy, v) \xrightarrow{e} (S', x, v')Sxyyeexvv'(S, x, v) \xrightarrow{e} (S S', y, v')S'(S, x, v) \xrightarrow{e} (S', y, v')v = v'\bullet\& eS \mapsto S S'v \mapsto v'\epsilonaA = ee~e'e/e'\&e!e\{ ~e~ \}~e~\}} (SS', y, \texttt{new}(x))}

\frac{~~ (S, xy, v) \xrightarrow{e} (S', y, v')}
{(S, xy, v) \xrightarrow{\\{\} \vspace{-3mm}


\fbox{} \vspace{-3mm}


\fbox{} \vspace{-3mm}


\end{small}

\caption{Semantics of AST operators}
\label{fig:semast}
\end{figure}

\begin{figure}[bt]
\begin{small}
\fbox{} \vspace{-3mm}


\fbox{} \vspace{-3mm}


\fbox{}

\fbox{} \vspace{-3mm}


\fbox{} \vspace{-3mm}


\fbox{} \vspace{-3mm}


\fbox{} \vspace{-3mm}


\fbox{} \vspace{-3mm}

\end{small}
\caption{Semantics of Nez's symbol operators}
\label{fig:semsym}
\end{figure}

Figure \ref{fig:semast} and Figure \ref{fig:semsym} show the semantics of AST operators and symbol operators. The recognition of Nez is AST-independent, because any values  are not premise for the transitions. Accordingly, an AST-eliminated grammar accepts the same input with the original Nez grammar. 

Formally, the language generated by the expression  is , and the language of grammar L(G) is . Note that the semantic value is unnecessary in the language definition. As suggested in the study of predicated grammars\cite{PLDI11_Antlr}, the language class of  is considered to be {\em context-sensitive} because predicates can check both the left and right context. 

\section{Elimination of Parsing Condition} \label{sec:condition}



In the previous section, we described the formal definition of Nez grammar language without conditional parsing. One reason for this is that conditional parsing can be replaced with symbol operators and an empty symbol.  In addition, we can eliminate conditions from a Nez grammar, and Nez parsers usually run based on the eliminated grammar. This section describes how to eliminate the conditions.

For simplicity, we consider the a single parsing condition, labeled . Let  be a Boolean variable such as , where  stand for not , and  be a conversion function of an expression  into the eliminated one. 

Here we write  for the eliminated grammar. Eliminating -condition from  is a conversion from  into , and defined: for each production  in , two new productions  and  are added into , as follows:



Now, we define the conversion function  recursively:

\begin{itemize}
\item  =  if , or  if 
\item  = 
\item  = 
\item  = 
\item  = 
\item  =  if ,  if 
\item  =  if ,  if 
\item  = 
\item  =  
\item  =  if  is none of the above
\end{itemize}

The number of productions in  is twice as many as that of . That means that a grammar in which  conditions are eliminated results in  productions in worst cases. 
In practice, we can make some unification for the same production such that , which may considerably reduce the number of productions in . Our empirical study (as described in Section \ref{sec:casestudies}) suggests that a single grammar involves not so many conditions that it would cause such an exponential increase of productions.

Note that the significant reason why we eliminate conditions is that condition operators may cause the serious invalidation of packrat parsing. In general, packrat parsing works on the assumption of stateless parsing\cite{ICFP02_PackratParsing}, while Nez adds new states such as ASTs, the symbol table, and parsing conditions. As we showed in Section \ref{sec:design}, the state of  ASTs does not influence any parsing behavior. The symbol table usually causes state changes with some character consumption, resulting in a fact that nonterminals rarely produce different results in the same position. As Grimm pointed out in \cite{PLDI06_Rats}, the {\em flow-forward} state change is not problematic in packrat parsing.  However, conditional parsing always causes state changes without any character consumption. This makes it easier to make a situation of different results of the same nonterminal in the same position. In fact, packrat parsing does not work without eliminating conditions from a Nez grammar. 

\section{Parser Runtime and Implementation} \label{sec:impl}

One of the advantages in open grammar is the freedom from parser implementation methods. In fact, the Nez parser tool, which we have developed with the Nez language, can generate three types of differently implemented parsers. First, as well as traditional parser generators, Nez produces parser source code written in the target language of the parser application. Second, Nez provides a grammar translator into other existing PEG-based grammars, (such as Rats, PEGjs, or PEGTL). Third, Nez itself works as an efficient parser interpreter that loads a grammar at runtime. In this section, we briefly describe these three implementations. 

\subsection{Parser Runtime and Parser Generation} 

The parser runtime required in Nez is fundamentally lightweight and portable. Nez parsers, as well as PEG parsers, can be implemented with recursive descent parsing with backtracking. All productions are simply implemented with parse functions that compute on the input characters, and then nonterminal calls are computed by function calls. Backtracking is simply implemented over the call stacks. 

Notoriously, backtracking might cause an exponential time parsing in the worst cases, while packrat parsing\cite{ICFP02_PackratParsing} is well established to guarantee the linear time parsing with PEGs. The idea behind packrat parsing is a memoization whereby  all results of nonterminal calls are stored in the memoization table to avoid redundant nonterminal calls. Although the memoization table may require some complexity for its efficiency,  we use a simple and constant-memory memoization table, presented in \cite{PRO101}. 

In addition to the standard PEG parser runtime, Nez parsers require two additional state management to handle ASTs and symbols. Note that we assume that conditions are eliminated upfront, as described in Section \ref{sec:condition}. 

The AST construction runtime provides a parser with APIs that are based on AST operators in Nez. Importantly, Nez parsers are speculative paring in a way that some of the AST operations may be discarded when backtracking. To handle the consistency of ASTs, we support  transactional operations (such as {\sf commit} and {\sf abort}) and all operations are stored as logs to be committed. This transactional structure can be easily implemented with stack-based logging, and the AST construction is partially aborted at the backtracking time. Note that efficient packrat parsing with ASTs requires additional transactional management for ASTs. A detailed mechanism is reported in \cite{ASTMachine}.

The symbol table requires another state management. As with the operation logs in the AST constructor, we use a stack-based structure to control both the symbol scoping and backtracking consistency. The symbol table runtime provides a parser with APIs that enable adding symbols, eliminating symbols, and testing symbols to match the input.

Originally, we write the AST runtime and the symbol table runtime in Java. There is no use of functional data structure; instead, both operation logs and symbols are stored in a linked list, which is available in any programming language. Indeed, we have already ported Nez runtime into several languages, including C and JavaScript. The C version of the AST runtime is at most 500 lines in code and the symbol table runtime is at most 200 lines in code, suggesting its high portability. 

\subsection{Grammar Translation}

A Nez grammar is specified with a declarative form of the AST operators and symbol operators, which are performed as a kind of action in the parsing context. These actions are limited in number and can be statically translated into code fragments written in any programming language. This means that a Nez grammar is convertible into PEGs 
with embedded semantic action code.  

Grammar translation is another approach to the Nez parser implementation. The advantage is that we enjoy optimized parsers that existing PEG tools generate.  On the other hand, impedance mismatch occurs between two grammars. In particular, many existing PEG-based parser generators have their own supports for AST representations. For example, PEGjs\cite{PEGjs} produces a JSON object as a semantic value of all nonterminal calls. In those cases, the translation of AST operators is unnecessary in practice. The symbol operators are always translatable if the symbol runtime that run on a host language is readily available.  

Figure \ref{fig:action} shows an example of converting symbol operators into  semantic actions \verb| { ... } | and semantic predicates \verb| &{ ... } |. The rollback of the symbol table at the backtracking time is automated before attempting alternatives. As shown, a Nez grammar is recursively convertible into PEGs with action code using the Nez runtime.

\begin{figure}[tb]
\begin{small}
\begin{tabular}{ll}
 &  \\
 &   \\
&   \\

 &
 \\

 &
 \\
 &  \\
 &  \\

 &
 \\
 &  \\
 &  \\

 &
 \\

 &
 \\

 &
 \\

 &
 \\

 &
 \\

\end{tabular}
\end{small}

\caption{Implementing Nez symbol operators with action code}
\label{fig:action}
\end{figure}

Currently, the Nez tool provides the grammar translation into PEGjs, PEGTL, and LPeg, although the translation of the AST construction might be restricted due to the impedance mismatch. Note that translating a Nez grammar into a LALR or LL grammar is an interesting challenge, though beyond the scope of this paper.

\subsection{Virtual Parsing Machine}

The simplicity of PEGs makes it easier to achieve dynamic parsing in a way that a parser interpreter loads a grammar  at runtime to parse the input. Since dynamic parsing requires no source compilation process, dynamic parsing makes it easier to import the Nez parser functionality as a parser library into parser applications. We consider that dynamic parsing is better suitable for many use cases of open grammars. Accordingly, the standard Nez parser is based on dynamic parsing and then implemented on top of a virtual parsing machine, an efficient implementation of dynamic parsing. 

A Nez parsing machine is a stack-based virtual machine that runs with a set of bytecode instructions, specialized for PEG, AST, symbol, and memoization operators. Table \ref{table:insts} is a summary of the bytecode instructions. A parsing expression  in Nez is converted to bytecode by using a compile function , where  is a label representing the next code point. Figure \ref{fig:compile} shows an excerpt of the definition of . The byte compilation is simply inductive, while Nez supports additional super instructions to generate optimized bytecode.  

Currently, the Nez parser tool includes a bytecode compiler and a parsing machine, both written in Java. In addition, the parsing machine is ported into C and JavaScript. Especially, the C version of the Nez parsing machine is highly optimized with indirect thread code and SEE4 instructions. The performance evaluation is studied in Section \ref{sec:perf}.

\begin{table}[tb]
\begin{tabular}{lp{6cm}}
PEG &
\textsf{
\noindent
nop fail alt succ jump call ret pos back skip byte any 
} \\
AST &
\textsf{
\noindent
tpush tpop tleftfold tnew tcapture ttag treplace 
tstart tcommit tabort 
} \\
Symbol &
\textsf{
\noindent
sopen sclose smask symbol exists isdef match is isa 
} \\
Memo &
\textsf{
\noindent
lookup memo memofail tlookup tmemo
} \\

\end{tabular}

\caption{Instructions of the Nez parsing machine}
\label{table:insts}

\end{table}

\begin{figure}[tb]

\begin{small}
\begin{center}

\begin{tabular}{lrrl}
      &  \verb|      = | &  &   \\
&  &  &   \\
&  &  &   \\
      &  \verb|      = | & &   \\
      &  \verb|      = | & &   \\
      &  \verb|      = | & &   \\

 &  \verb|      = | & &   \\
 &  &  &   \\

 &  \verb|      = | & &   \\
& & &   \\
&  & &   \\
&  &  &   \\

 &  \verb|      = | & &   \\
& & &   \\
&  &  &   \\

 &  \verb|      = | & &   \\
& & &   \\
&  &   &   \\
&  &  &   \\ \hline

 &  \verb|      = | & &   \\
& & &   \\
&  &   &   \\

(e),\: L)\kw{tpush}\tau(e,\: L')L'\kw{tlink}\kw{tpop}\tau(\#t,\: L)\kw{ttag}~\#t\tau(\langle block~e\rangle,\: L)\kw{sopen}\tau(e,\: L')L'\kw{sclose}\tau(\langle local~A~e\rangle,\: L)\kw{sopen}\kw{mask}~A\tau(e,\: L')L'\kw{sclose}\tau(\langle symbol~A\rangle,\: L)\kw{pos}\tau(A,\: L')L'\kw{symbol}\tau(\langle exists~A\rangle,\: L)\kw{exists}~A\tau(\langle match~A\rangle,\: L)\kw{match}~A\tau(\langle is~A\rangle,\: L)\kw{pos}\tau(A,\: L')L'\kw{is}~Ae!k!!!kkk!e~U~e!$.  The Nez virtual machine is said to be an extended version of the LPeg machine. Notably, the Nez machine has instruction supports for packrat parsing \cite{ICFP02_PackratParsing}, which is a major lack in the LPeg machine. The originality of Nez parsers is that we combine these established techniques in a way that the parser runtime is still simple and portable.  



\section{Conclusion} \label{sec:conclusion}

Nez is a simple, portable, and declarative grammar specification language. As an alternative to action code, Nez provides a small set of extended PEG operators that allow the grammar developer to make AST constructions, as well as context-sensitive and conditional parsing. Due to these extensions, Nez can recognize a language that cannot be well expressed by pure PEGs. In addition, the Nez parser runtime is also simple and portable. Based on the Java-based implementation, we have developed C and JavaScript ports, suggesting that the Nez parser runtime is implementable in any modern programming languages. More importantly, Nez parsers achieve practical and competitive performance in each of the ported environments.

This paper is an initial report on Nez and the idea of open grammars. Further evaluations are obviously necessary on the expressiveness of Nez with extensive case studies. To achieve the aim of open grammars, a lot of interesting challenges remain. Future work that we will investigate includes grammar modularity and inference, a tree checker for better connectability of parser applications, and a DFA-based  machine for more efficient parsing. In the end, we hope that Nez parsers will be available in many programming languages. Our developed tools and grammars are available online at \url{http://nez-peg.github.io/}. 
   
\acks
The C version of Nez parsers (cnez and virtual parsing machines) have been developed by Masahiro Ide and Shun Honda.The JavaScript version has been developed by Takeru Sudo. Finally, the anonymous reviewers are acknowledged for their helpful suggestions for improving an earlier draft of this paper.



\bibliographystyle{abbrvnat}
\bibliography{../bib/parser,../bib/mypaper,../bib/url,../bib/data}  

\end{document}
