\documentclass{fundam}
\usepackage{latexsym,amssymb,amsmath,upgreek}
\usepackage{xspace}
\usepackage{enumerate}
\usepackage{mathrsfs}
\usepackage{setspace}
\usepackage{color}

\bibliographystyle{plain}
\pagestyle{empty}

\long\def\COMMENT#1{}

\newcommand{\Implies}{\supset}
\renewcommand{\And}{\wedge}
\newcommand{\Or}{\vee}
\newcommand{\compl}[1]{\overline{#1}}

\newcommand{\syntree}[1]{\mathit{S}_{#1}}

\newcommand{\SmallDomain}{\mathit{SmallDomain}}
\newcommand{\false}{\mathbf{false}}
\newcommand{\true}{\mathbf{true}}
\newcommand{\free}{\mathit{Free}}
\newcommand{\dss}{\mydelta^{*^*}}
\newcommand{\bde}{\mydelta^{\mathbf{sk}}}
\newcommand{\de}{\mydelta^{\epsilon}}
\newcommand{\bss}{b^{*^*}}
\newcommand{\tss}{{\cal T}^{*^*}}
\newcommand{\ds}{\mydelta^*}
\newcommand{\bs}{b^*}
\newcommand{\myts}{{\cal T}^*}
\newcommand{\comment}[1]{}
\newcommand{\vect}{\overrightarrow}
\newcommand{\arity}{\mathrm{arity}}
\newcommand{\var}{\mathit{Var}}
\newcommand{\defAs}
   {\mbox{}}


\newcommand{\Key}{\mathit{Key}}
\newcommand{\QKey}{\mathit{QKey}}
\newcommand{\SubstKey}{\hbox{\it SubstKey}}
\newcommand{\Free}{\mathit{Free}}
\newcommand{\Bound}{\mathit{Bound}}
\newcommand{\RelF}{\mathit{RelF}}
\newcommand{\rel}{\mathit{Rel}}
\newcommand{\negative}{\complement}

\newcommand{\Hf}{H_{\varphi}}

\newcommand{\myspace}{~~~}


\def\eod {{\unskip\nobreak\hfil\penalty50
\hskip2em\hbox{}\nobreak\hfil 
\parfillskip=0pt \finalhyphendemerits=0 \par \medskip}}

\def\qed {{\unskip\nobreak\hfil\penalty50
\hskip2em\hbox{}\nobreak\hfil \rule{2mm}{2mm}
\parfillskip=0pt \finalhyphendemerits=0 \par \medskip}}


\newcommand{\sko}{\ensuremath{\mathbf{sko}}\xspace}
\newcommand{\rank}{\ensuremath{\mathit{rank}}\xspace}
\newcommand{\vectB}{\ensuremath{\vect{\mbox{\boldmath }}}\xspace}
\newcommand{\boldC}{\ensuremath{\mbox{\boldmath }}\xspace}
\newcommand{\boldD}{\ensuremath{\mbox{\boldmath }}\xspace}
\newcommand{\myalpha}{\ensuremath{\mbox{\boldmath }}\xspace}
\newcommand{\mybeta}{\ensuremath{\mbox{\boldmath }}\xspace}
\newcommand{\mygamma}{\ensuremath{\mbox{\boldmath }}\xspace}
\newcommand{\mydelta}{\ensuremath{\mbox{\boldmath }}\xspace}
\newcommand{\mysmdelta}{\ensuremath{\mbox{\boldmath }}\xspace}
\newcommand{\model}{\ensuremath{\mbox{\boldmath }}\xspace}
\newcommand{\domain}{\ensuremath{\mbox{\boldmath }}\xspace}
\newcommand{\prestr}{\ensuremath{\mbox{\boldmath }}\xspace}
\newcommand{\intrp}{\ensuremath{\mbox{\boldmath }}\xspace}
\newcommand{\assign}{\ensuremath{\mbox{\boldmath }}\xspace}
\newcommand{\smintrp}{\ensuremath{\mbox{\boldmath }}\xspace}
\newcommand{\smassign}{\ensuremath{\mbox{\boldmath }}\xspace}
\newcommand{\eval}{\ensuremath{\mbox{\boldmath }}\xspace}
\newcommand{\ekey}{\ensuremath{\mbox{\boldmath }}\xspace}\newcommand{\smeval}{\ensuremath{\mbox{\boldmath }}\xspace}

\newcommand{\pow}{\mbox{\rm pow}}
\newcommand{\Rel}{\mbox{\rm Rel}}

\newcommand{\myvect}[1]{\vect{\raisebox{0pt}[9pt][\depth]{}}}
\newcommand{\mmyvect}[2]{\vect{\raisebox{0pt}[#1pt][\depth]{}}}

\newcommand{\duno}{\delta^{a}}
\newcommand{\ddue}{\delta^{s}}
\newcommand{\dprime}{\xi}

\newcommand{\tab}{\mathit{Tab}}
\newcommand{\terms}{\mathit{Terms}}
\newcommand{\SubF}{\mathit{SubF}}
\newcommand{\Kr}{\mathit{K}}
\newcommand{\K}{\mathsf{K}}
\newcommand{\T}{\mathsf{T}}
\newcommand{\Ac}{\mathbf{5}}
\newcommand{\B}{\mathsf{B}}
\newcommand{\Aq}{\mathbf{4}}
\newcommand{\D}{\mathsf{D}}
\newcommand{\Sc}{\mathsf{S5}}
\newcommand{\Kqc}{\mathsf{K45}}

\newcommand{\TSc}{\tau_{\Sc}}
\newcommand{\TScuno}{\tau_{\Sc}^{1}}
\newcommand{\TScdue}{\tau_{\Sc}^{2}}

\newcommand{\TKqc}{\tau_{\Kqc}}
\newcommand{\TKqcuno}{\tau_{\Kqc}^{1}}
\newcommand{\TKqcdue}{\tau_{\Kqc}^{2}}

\newcommand{\condition}[1]{\textbf{C}\ref{#1}\xspace}
\newcommand{\cnd}[1]{condition \condition{#1}}
\newcommand{\cnds}[2]{conditions \condition{#1} and \condition{#2}}

\newcommand{\QLQSR}{\ensuremath{\mbox{}}\xspace}
\newcommand{\TLQSR}{\ensuremath{\mbox{}}\xspace}
\newcommand{\QLQS}{\ensuremath{\mbox{}}\xspace}
\newcommand{\MLS}{\ensuremath{\mbox{}}\xspace}

\newcommand{\dlmlsscart}{\ensuremath{\mathcal{DL\langle}\mathsf{MLSS}_{2,m}^{\times}\mathcal{\rangle}}\xspace}

\newtheorem{myprop}{Property}

\pagestyle{empty}

\begin{document}

\title{On the satisfiability problem for a 4-level quantified syllogistic and some applications to modal logic (extended version)}

\author{Domenico Cantone \\
Dipartimento di Matematica e Informatica, Universit\`a di
Catania\\
      Viale A. Doria 6, I-95125 Catania, Italy \\
cantone@dmi.unict.it\\
\and Marianna Nicolosi Asmundo \\
Dipartimento di Matematica e Informatica, Universit\`a di
Catania\\
      Viale A. Doria 6, I-95125 Catania, Italy \\
      nicolosi@dmi.unict.it}
\maketitle




\begin{abstract}
We introduce a multi-sorted stratified syllogistic, called ,
admitting variables of four sorts and a restricted form of
quantification over variables of the first three sorts, and prove that
it has a solvable satisfiability problem by showing that it enjoys a
small model property.  Then, we consider the fragments  of
, consisting of -formulae whose quantifier prefixes
have length bounded by  and satisfying certain syntactic
constraints, and prove that each of them has an \textsf{NP}-complete
satisfiability problem.  Finally we show that the modal logic 
can be expressed in .
\end{abstract}

\section{Introduction}
Most of the decidability results in computable set theory concern
one-sorted multi-level syllogistics, namely collections of formulae
admitting variables of one sort only, which range over the von Neumann
universe of sets (see \cite{CFO89,COO01} for a thorough account of the
state-of-art until 2001).
Only a few stratified syllogistics, where
variables of different sorts are allowed, have been investigated,
despite the fact that in many fields of computer science and
mathematics often one has to deal with multi-sorted languages.\footnote{The locutions `multi-level syllogistic' and `stratified syllogistic' were chosen by Jack Schwartz to
name many decidable fragments of computable set theory because he saw them as generalizations of Aristotelian syllogistics.}
For instance, in modal logics, one has to consider entities of different
types, namely worlds, formulae, and accessibility relations.

In \cite{FerOm1978} an efficient decision procedure was presented for
the satisfiability of the Two-Level Syllogistic language ().
 has variables of two sorts and admits propositional connectives
together with the basic set-theoretic operators , and the predicate symbols , and .  Then,
in \cite{CanCut90}, it was shown that the extension of  with the
singleton operator and the Cartesian product operator is decidable.
Tarski's and Presburger's arithmetics extended with sets have been
analyzed in \cite{CCS90}.  Subsequently, in \cite{CanCut93}, a
three-sorted language  (Three-Level Syllogistic with
Singleton, Powerset and general Union) has been proved decidable.
Recently, in \cite{CanNic08}, it was shown that the language 
(Three-Level Quantified Syllogistic with Restricted quantifiers) has a
decidable satisfiability problem.   admits variables of three
sorts and a restricted form of quantification.  Its vocabulary
contains only the predicate symbols  and .  In spite of that,
 allows one to express several constructs of set theory.
Among them, the most comprehensive one is the set-formation operator, which in
turn enables one to express other operators like the powerset
operator, the singleton operator, and so on.  In \cite{CanNic08} it is
also shown that the modal logic  can be expressed in a
fragment of , whose satisfiability problem is
\textsf{NP}-complete.

In this paper we present a decidability result for the satisfiability
problem of the set-theoretic language  (Four-Level Quantified
Syllogistic with Restricted quantifiers).   is an extension of
 admitting variables of four sorts and a restricted form of
quantification over variables of the first three sorts.  In addition
to the predicate symbols  and , its vocabulary contains also the
pairing operator .

We will prove that the theory  enjoys a small model property
by showing how one can extract, out of a given model satisfying a
-formula , another model of  but of bounded finite
cardinality.  The construction of the finite model extends the
decision algorithm described in \cite{CanNic08}.  Concerning complexity
issues, we will show that the satisfiability problem for each of the
fragments  of , whose formulae are restricted to
have their quantifier prefixes of length at most  and must
satisfy certain additional syntactic constraints to be seen later, is
\textsf{NP}-complete.

In addition to the modal logic , already expressible in the
language , it turns out that in  one can also
formalize several properties of binary relations (needed to define
accessibility relations of well-known modal logics) and some Boolean
operations over relations and the inverse operation over binary
relations.  We will also show that the modal logic  can be
formalized in the fragment .  As is well-known, the
satisfiability problem for  is \textsf{NP}-complete; thus our
alternative decision procedure for  can be considered optimal in
terms of its computational complexity.


\section{The language }\label{language}
Before defining the language  of our interest, it is
convenient to present the syntax and the semantics of a more general,
unrestricted four-level quantified fragment, denoted .
Subsequently, we will introduce suitable restrictions over the
formulae of  to characterize the sublanguage .


\subsection{The unrestricted language }\label{genericlanguage}

\paragraph{Syntax of .}
The four-level quantified language  involves the four collections
, , , and  of variables. Each 
contains variables of \emph{sort i}, denoted by .
When we refer to variables of sort 0 we prefer to write  instead of .
In addition to the variables in , terms of sort 2
include also \emph{pair terms} of the form , for
.\
M\langle x,y\rangle \defAs \{\{Mx\},\{Mx,My\}\}\,.

    \label{condition}
\neg \varphi_0 \rightarrow \bigwedge_{i=1}^n \bigwedge_{j=1}^m  z_i \in Z_j^1

D^* :=\{Mx : x \hbox{ in } {\cal V}_0'\} \cup \Delta\,.

\Phi \defAs \{ \varphi_{i,k_j} : 1 \leq j  \leq \ell_i \hbox{ and }
 1 \leq i \leq r  \}.
M[z_1/u_1,\ldots,z_n/u_n]
(\varphi_{0})_{X_{h_1}^{1},\ldots,X_{h_m}^{1}}^{Z_1^{1}\;\,,\ldots,\;Z_m^{1}}
=\false\,,
    \label{DStar}
    |D^*| \leq |{\cal V}_0'| + 4|{\cal V}_1'| + 16|{\cal V}_2'| +
    \left((|{\cal V}_1'| + 4|{\cal V}_2'| - 1)^{L_{m}} L_{n}\right)|\Phi| - 5\,.

    M^{*}x & = & \left\{
    \begin{array}{ll}
        Mx\,, & \mbox{if }  \\
        d^{*}\,, & \mbox{otherwise}\, ,
    \end{array}
    \right.
    \\
    M^{*}X^1 & = & MX^1 \cap D^{*}\, ,
    \\
   M^{*}X^2  &=&
    \left((MX^2 \cap \pow(D^{*})) \setminus \{M^{*}X^1: X^1 \in
    ({\cal V}'_{1}\cup {\cal V}_1^F)\}\right)
    \\
     & & \qquad\cup \{M^{*}X^1: X^1 \in ({\cal V}'_{1} \cup {\cal V}_1^F),~MX^1 \in MX^2\}\, ,\\
M^{*}X^3  &=&
    \left((MX^3 \cap \pow(\pow(D^{*}))) \setminus \{M^{*}X^2: X^2 \in
    {\cal V}'_{2}\}\right)
     \\
     & & \qquad\cup \{M^{*}X^2: X^2 \in {\cal V}'_{2},~MX^2 \in MX^3\}\,.

    \model^{*,z} & = & \model^{*}[z_{1}/u_{1},\ldots,z_{n}/u_{n}], \\
    \model^{*,Z^1} & = & \model^{*}[Z_{1}^{1}/U_{1}^{1},\ldots,Z_{m}^{1}/U_{m}^{1}], \\
    \model^{*,Z^2} & = & \model^{*}[Z_{1}^{2}/U_{1}^{2},\ldots,Z_{p}^{2}/U_{p}^{2}],

    \model^{z,*} & = & \Rel(\model^{z},D^{*}, {\cal V}'_{1}, {\cal V}_{1}^{F}, {\cal V}'_{2}),\\
    \model^{Z^1,*} & = & \Rel(\model^{Z^{1}},D^{*},{\cal V}'_{1} \cup
                         \{Z_{1}^{1},\ldots,Z_{m}^{1}\}, {\cal V}_{1}^{F}, {\cal V}'_{2}),\\
    \model^{Z^2,*} & = & \Rel(\model^{Z^{2}},D^{*},\mathcal{F}^{*},{\cal V}'_{1}, {\cal V}_{1}^{F}, {\cal V}'_{2}\cup
                         \{Z_{1}^{2},\ldots,Z_{p}^{2}\}).

        M^{*,z}X^2 & = & M^{*}X^2 =
                        ((MX^2 \cap \pow(D^{*})) \setminus \{M^{*}X^1: X^1 \in ({\cal V}'_{1}\cup {\cal V}_{1}^{F})\})\\
                 &   &  \phantom{M^{*}X^2 =} \qquad \cup \{M^{*}X^1: X^1 \in ({\cal V}'_{1}\cup {\cal V}_{1}^{F}),~MX^1 \in MX^2\} \,  \\
                 &  & \phantom{M^{*}X^2} = ((M^{z}X^2 \cap \pow(D^*)) \setminus \{M^{z,*}X^1 : X^1 \in ({\cal V}_{1}'\cup {\cal V}_{1}^{F})\}) \\
                 &   &  \phantom{M^{*}X^2 =} \qquad \cup \{M^{z,*}X^1 : X^1 \in ({\cal V}_{1}'\cup {\cal V}_{1}^{F}), M^{z}X^1 \in M^{z}X^2\} \\
                 &  & \phantom{M^{*}X^2} =  M^{z,*}X^2 \,.

        M^{*,z}X^3 & = & M^{*}X^3 =
                        ((MX^3 \cap \pow(\pow(D^*))) \setminus \{M^{*}X^2 : X^2 \in {\cal V}_{2}'\})\\
                 &   &  \phantom{M^{*}X^2 =} \qquad\cup \{M^{*}X^2 : X^2 \in {\cal V}_{2}', MX^2 \in MX^3\}  \\
                 &  & \phantom{M^{*}X^2} = ((M^{z}X^3 \cap \pow(\pow(D^*))) \setminus \{M^{z,*}X^2 : X^2 \in {\cal V}_{2}'\}) \\
                 &   &  \phantom{M^{*}X^2 =} \qquad\cup \{M^{z,*}X^2 : X^2 \in {\cal V}_{2}', M^{z}X^2 \in M^{z}X^3\} \\
                 &  & \phantom{M^{*}X^2} =  M^{z,*}X^3 \,.

    M^{Z^1,*}X^1 = M^{Z^1}X^1 \cap D^{*} = MX^1 \cap D^{*} = M^{*}X^1 = M^{*,Z^1}X^1\,.
    
    M^{Z^1,*}Z_{j}^1 = M^{Z^1}Z_{j}^1 \cap D^{*} = U_{j}^1 \cap D^{*} = U_{j}^1
      =  M^{*,Z^1}Z_{j}^1\,.
    
        M^{*,Z^1}X^2 & = & M^{*}X^2 =
                        ((MX^2 \cap \pow(D^{*})) \setminus \{M^{*}X^1:
			X^1 \in ({\cal V}'_{1}\cup {\cal
			V}_{1}^{F})\}) \notag\\
                 &   &  \phantom{M^{*}A =} \qquad\cup \{M^{*}X^1: X^1
         \in ({\cal V}'_{1}\cup {\cal V}_{1}^{F}),~MX^1 \in
         MX^2\} \, , \label{a1}\

    By putting
    
    then by (\ref{a1}) and (\ref{a2}) can be rewritten as
    
    Moreover, since, as can easily verified, we have
    
    then
    
    Therefore, (\ref{a3}) and (\ref{a4}) readily imply .
\item
    Let , then  and
    
    Since , the thesis follows.
\end{enumerate}
\end{proof}


\begin{lemma}
\label{le_eqM*ZMZ*2}
Let 
and .  Then the -interpretations
 and  coincide.\end{lemma}
\begin{proof}
We show that  and 
coincide by proving that they agree over variables of all sorts.
\begin{enumerate}
\item Plainly , for every .

\item Let . Then .

\item  Let  such that . Then

and

Since  the thesis follows, at least in
the case in which .
On the other
hand, if , say , then
, and

Clearly the thesis follows also in this case.

\item  Let . Then we have
    \hspace*{-2cm}
    

    By putting
    
    then (\ref{a5}) and (\ref{a6})
    can be respectively rewritten as
    
    Moreover, it is easy to verify that the following relations hold:
    
    so that
    
    Therefore, in view of (\ref{a7}) and (\ref{a8}) above,
    (\ref{a9})  yields .
\end{enumerate}
\end{proof}

The following lemma proves that satisfiability is preserved in the case of purely universal formulae.
\begin{lemma}\label{quantifiedform}
Let , , and  be conjuncts of .  Then
\begin{itemize}
\item [(i)] if , then ;

\item [(ii)] if , then ;
\item [(iii)] if , then .
\end{itemize}
\end{lemma}
\begin{proof}
\begin{itemize}
\item [(i)]
Assume by contradiction that there exist 
such that .
Then, there must be an atomic formula  in  that
is interpreted differently in  and in .
Recalling that  is a propositional combination of
quantifier-free atomic formulae of any level, let us first suppose
that  is  and, without loss of generality,
assume that .  Then , so that, by Lemma \ref{le_M*zMz*}, .  Then, Lemma \ref{le_basic} yields ,
a contradiction.  The other cases are proved in an analogous way.

\item [(ii)] This case can be proved much along the same lines as the
proof of case (ii) of Lemma 4 in \cite{CanNic08}.  Here, one has to take
care of the fact that  may contain purely universal
formulae of level 1 occurring only positively in  and not
satisfying Restriction \ref{restriction1} of Section
\ref{restrictionquant}.  This is handled similarly to case (i)
of this lemma.  Another issue that has to be considered is the fact
that the collection of relevant variables of sort 1 for  are not
just the variables occurring free in , namely the ones in , but also the variables in , introduced to
denote the elements distinguishing the sets , for .


\item [(iii)]
Assume, by way of contradiction, that
,
but
. Hence there exist  such that .

Without loss of generality, assume that , for  and where , and
that , for  and
, for some .

Let  be the formula obtained by simultaneously
substituting  with  in
, and let .  Further, let
 be a -interpretation differing from
 only in the evaluation of , with
.

We distinguish the following two cases:

\begin{description}
    \item[Case :]
If , then  and  coincide and a
contradiction can be obtained by showing that the implications

hold, since these together with the fact that  would yield , contradicting our initial
hypothesis.  The first implication,
, is plainly derived from the definition of
.  The second one, , can
be proved as follows.  For every purely universal formula either of
level 1 or of level 2, , occurring only positively
in , it follows that  by
reasoning as in case (i) or in case (ii) of the present lemma, respectively.  For
each other atomic formula  occurring in
 we have to show that  and 
evaluate  in the same manner.  If
 is a quantifier-free atomic formula, the proof
follows directly from Lemma \ref{le_basic}.  If  is
an atomic formula of level 1, it can only be of type , where  is any variable in
.
Reasoning analogously to case (i) of the present lemma, it follows that .  Next,
let us prove that .  Assume by contradiction that .  That is, .  Then, there are 
such that .  By the
construction in Section \ref{decisionproc}, all these s are in
,  and thus we finally obtain that

contradicting our hypothesis.

Finally, , follows from the definition of
 and of .


\item[Case :] In this case, the schema of the proof is
analogous to the one in the previous case.  However, since
 and  do not coincide, the single steps
are carried out in a slightly different manner.  Thus, for the sake of
clarity we report below the details of the proof.

In order to obtain a contradiction we prove that the following
implications hold


The first implication, , can be
immediately deduced from the definition of  and of
.  The second implication,
, can be proved as shown
next.  If  is a purely universal formula either of
level 1 or of level 2 occurring only positively in ,
we have  and,
since  and
 coincide (by Lemma \ref{le_eqM*ZMZ*2}), we obtain
.  Then, reasoning as in case
(i) (if  is of level 1) or in case (ii) (if
 is of level 2) of the present lemma, it follows
that .  If
 is a quantifier-free atomic formula occurring in
, we prove that  in
 is interpreted in  and in
 in the same way, using Lemmas \ref{le_eqM*ZMZ*2} and
\ref{le_basic}.

If  is a purely universal formula of level 1, it
must have the form

where  is any variable in .  In this case the
proof is carried out as shown next.  Reasoning as in case (i), we have , and by Lemma \ref{le_eqM*ZMZ*2}, that
. Proceeding as in the first case of this item of
the present lemma,
we obtain that
 and, by Lemma \ref{le_eqM*ZMZ*2}, that
, contradicting our hypothesis.

Finally, the third implication, 
follows directly from the definition of  and of
.
\end{description}
\end{itemize}
\end{proof}
We can now state and prove our main result.
\begin{theorem}\label{correctness}
Let  be a -interpretation satisfying a normalized
-conjunction .  Then , where
 is the relativized interpretation of  with respect
to a domain  satisfying (\ref{DStar}).
\end{theorem}
\begin{proof}
We only have to prove that , for each conjunct
 occurring in .  Each such  must be of one of the
types (1)--(3) enumerated in Section \ref{normal3LQS}.  By applying
either Lemma \ref{le_basic} or Lemma \ref{quantifiedform} to each
 (according to its type) we obtain the thesis.
\end{proof}
From the above reduction and relativization steps, the following
result follows easily:
\begin{corollary}
    The fragment  enjoys a small model property (and
    therefore it has a solvable satisfiability problem). \end{corollary}





\section{Expressiveness of the language }\label{sec:applications}
\label{expressiveness}
Much as shown in \cite{CanNic08}, the language  can express a
restricted variant of the set-formation operator, which in turn allows one to
express other significant set operators such as binary union,
intersection, set difference, the singleton operator, the powerset
operator (over subsets of the universe only), etc.  More specifically,
atomic formulae of type , for , can be expressed in  by
the formulae

provided that the syntactic constraints of  are satisfied.


Since  is a superlanguage of , the language
 can express the syllogistic  (cf.\ \cite{FerOm1978}) and
the sublanguage  of  not involving the set-theoretic
construct of general union, since these are expressible in
, as shown in \cite{CanNic08}.  We recall that
 admits variables of three sorts and, besides the usual
set-theoretical constructs, it involves the `singleton set' operator
, the powerset operator , and the general union
operator .
 can plainly be decided by the decision procedure presented in
\cite{CanCut93} for the whole fragment .

Among the other constructs of set theory which are expressible in the
language  (cf.\ \cite{CanNic08}), we cite:
\begin{itemize}

\item literals of the form , where  denotes the collection of subsets of  with less than 
elements;

\item the unordered Cartesian product , where  denotes the collection ;

\item literals of the form , where
 is the variant of the powerset
introduced in \cite{Can91} which denotes the collection

\end{itemize}
For instance, a literal of the form , with , can be expressed by the -formula

as can be easily verified.

\subsection{Other applications of }
Within the  language it is also possible to define binary
relations over elements of a domain together with several conditions
on them which characterize accessibility relations of well-known modal
logics.  These formalizations are illustrated in Table
\ref{tab:accesrel}.


\begin{table}[tb]
\begin{tabular}{|l|l|}
  \hline
Binary relation & \\\hline\hline
  Reflexive & \\\hline
  Symmetric & \\\hline
  Transitive & \\\hline
  Euclidean  &  \\ \hline
  Weakly-connected &   \\
                   & \hfill \\ \hline
  Irreflexive & \\ \hline
  Intransitive & \\\hline
  Antisymmetric & \\\hline
  Asymmetric & \\
  \hline
\end{tabular}
\caption{\label{tab:accesrel}  formalization of conditions of accessibility relations}
\end{table}

Usual Boolean operations over relations can be defined as shown in
Table \ref{tab:Bolop}.
\begin{table}[tb]
\begin{center}
\begin{tabular}{|l|l|l|}
  \hline
Intersection &  & \\\hline
Union &  & \\\hline
Complement &  &  \\\hline
Set difference &  & \\\hline
Set inclusion &   & \\
 \hline
\end{tabular}
\caption{\label{tab:Bolop}  formalization of Boolean operations over relations}
\end{center}
\end{table}
The language  allows one also to express the inverse
 of a given binary relation  (namely, to express
the literal ) by means of the
-formula .

In the next section we will present an application of the decision
procedure for -formulae to modal logic.
For this purpose we introduce below a family  of fragments of , each of which has an
\textsf{NP}-complete satisfiability problem, and then show, in the
next section, that the modal logic  can be formalized in
 in a succint way, thus rediscovering the
\textsf{NP}-completeness of the decision problem for  (cf.\
\cite{Lad77}).

Formulae in  must satisfy various syntactic constraints.
First of all, all quantifier prefixes occurring in a formula in
 must have their length bounded by the constant .
Thus, given a satisfiable -formula  and a
-model  for it, from
Theorem~\ref{correctness} it follows that  is satisfied by
the relativized interpretation  of 
with respect to a domain  whose size is bounded by the
expression in (\ref{DStar}).  But since in this case 
and , where  and  are defined as in Step 4
of the construction of  (cf.\ Section~\ref{ssseUniv}), it
follows that the bound in (\ref{DStar}) is quadratic in the size of
.  The remaining syntactic constraints on
-formulae will allow us to deduce that , for any free variable  of sort 2 in
, and , for
any free variable  of sort 3 in , so that the model
 can be guessed in nondeterministic polynomial time in the
size of , and one can check in deterministic polynomial time
that  actually satisfies , proving that the
satisfiability problem for -formulae is in \textsf{NP}.
As the satisfiability problem \textsf{SAT} for propositional logic can
be readily reduced to that for -formulae, the
\textsf{NP}-completeness of the latter problem follows.

\begin{definition}[-formulae]\label{def:hlang}
Let  be a -formula involving the designated free
variables , , and  (of sort 1, 2, and 3,
respectively).  Let  be the free variables of sort
2 occurring in , distinct from .  Likewise, let
 be the free variables of sort 3 occurring in
, distinct from . Then  is a
-formula, with , if it has the form (up to the
order of the conjuncts)

where
\begin{enumerate}
\item ,

\emph{i.e.,  is the (nonempty) universe of discourse};

\item ,

\emph{i.e.,  (together with
formula )};

\item 

      ,

\emph{i.e., 
(together with formulae  and )};

\item either  or , for ,

\emph{so that, , for 
(together with formulae  and )};

\item , for ,

\emph{i.e., , for  (together with formulae , , and
)};

\item  is a propositional combination of
\begin{enumerate}
    \item quantifier-free atomic formulae of any level,

    \item\label{levelOne} purely universal formulae of level 1 of the form
    
    with ,

    \item\label{levelTwo} purely universal formulae of level 2 of the form
    
    where  and  is a propositional combination of
    quantifier-free atomic formulae and of purely universal formulae
    of level 1 satisfying (\ref{levelOne}) above,

    \item purely universal formulae of level 3 of the form
    
    where  and  is a propositional combination of
    quantifier-free atomic formulae, and of purely universal formulae
    of level 1 and of level 2 satisfying (\ref{levelOne}) and
    (\ref{levelTwo}) above.
\end{enumerate}
\end{enumerate}

\end{definition}

Having defined the fragments , for , next we
prove that each of them has an \textsf{NP}-complete satisfiability
problem.
\begin{theorem}
The satisfiability problem for  is
\textsf{NP}-complete, for any .
\end{theorem}
\begin{proof}
The satisfiability problem \textsf{SAT} for propositional logic can be
readily reduced to the one for -formulae, for any , as follows. Given a formula , we
construct a quantifier-free -formula 
by replacing each propositional letter  in  by the
quantifier-free formula , where  is a fixed
variable of sort 1 and the s are distinct variables of sort
0 in a one-one correspondence with the distinct propositional
letters in . Plainly,  is propositionally satisfiable if
and only if  is satisfiable by a -model.
Therefore the \textsf{NP}-hardness of the satisfiability problem for
-formulae follows.

To prove that our problem is in \textsf{NP}, we reason as follows.
Let

be a satisfiable -formula, and let  be a set of
formulae constructed as follows.  Initially, we put

and then, we modify  according to the following six rules, until
no rule can be further applied:\footnote{We recall that an
implication  has to be regarded as a shorthand for
the disjunction .}
\begin{enumerate}[~~R1:]
\item if  is in , then ,

\item if  (resp., ) is in  (i.e.,  is a conjunctive formula), then
we put 
(resp., ),

\item if  (resp., ) is in  (i.e.,  is a disjunctive formula), then
we choose a , , such that  (resp., )
is satisfiable and put  (resp., ),

\item if 
is in , then ,
where  are newly introduced variables of
sort 0,

\item if 
is in , then ,
where  are fresh variables of sort 1,

\item if  is in , then ,
where  are newly introduced variables
of sort 2.
\end{enumerate}
Plainly, the above construction terminates in 
steps and if we put , it turns
out that
\begin{enumerate}[~~~~(a)]
\item  is a satisfiable -formula,

\item , and

\item  is a valid -formula.
\end{enumerate}
In view of (a)--(c) above, to prove that our problem is in
\textsf{NP}, it is enough to construct in nondeterministic polynomial
time a -interpretation and show that we can check in polynomial
time that it actually satisfies .

Let  be a -model for  and let  be the relativized interpretation of  with respect
to a domain  satisfying (\ref{DStar}), hence such that
, since  is a
-formula (cf.\ Theorem \ref{correctness} and the
construction described in Sections \ref{ssseUniv} and \ref{machinery}).

In view of the remarks just before Definition~\ref{def:hlang}, to
complete our proof it is enough to check that
\begin{itemize}
\item , for any free variable 
of sort 2 in  (which entails that ),

\item , for any free
variable  of sort 3 in  (which entails that ), and

\item  can be verified in deterministic
polynomial time.
\end{itemize}

To prove that , for any free
variable  in , we reason as follows.  Let  be a
variable of sort 2 occurring free in .  From Definition
\ref{relintrp}, we recall that


Observe that

Indeed, if the variable  coincides with , then
(\ref{MXTwo}) follows from the fact that  contains the conjunct
.  On the other hand, if  is distinct from , then  contains either the conjunct  or the conjunct . In the first case,
 together with the conjunct
, implies again (\ref{MXTwo}).
From (\ref{MStarXTwo}) and (\ref{MXTwo}), we get . The other case is handled in a similar way.

Checking that , for any
free variable  of sort 3 in , can be carried out much as
was done for free variables of sort 2.

From what we have shown so far, it follows that
in nondeterministic polynomial time one can construct
\begin{itemize}
    \item the -formula , as a result of
    applications of rules R1--R6 to the initial set 
    (corresponding to the input formula ) until saturation is
    reached,

    \item the -interpretation  (of
    ).
\end{itemize}
By the soundness of rules R1--R6, it follows that the -formula
 is valid.  Thus, we obtain a succint
certificate of the satisfiability of  if we show that it is
possible to check in polynomial time that 
holds.  This is equivalent to show that we can check in polynomial
time that , for every conjunct  of .
We distinguish the following cases.
\begin{description}
\item [ is a quantifier-free atomic formula:] Since all variables
in  are interpreted by  with sets of polynomial size,
the task of checking memberships and equalities among such sets can be
performed in polynomial time.

\begin{sloppypar}
\item [ is a purely universal formula of level 1 , with :] We have that
 if and
only if , for
every .  From the previous case, for any
, one can compute in polynomial time whether
.  Since the
collection of such -tuples  has polynomial
size in , it turns out that one can check that  in polynomial time.
\end{sloppypar}


\item [ is a purely universal formula of level 2:]  If

in order to verify that , it is enough to check
that , which can be
clearly done in polynomial time.

If , with  a free variable of sort 2, then in
order to verify that  it is enough to check
whether , which again can
be done in polynomial time.

Finally, if  where  and  is a propositional
combination of quantifier-free atomic formulae and of purely universal formulae of level 1 of the form , with  (cf.\
Definition~\ref{def:hlang}(\ref{levelTwo})), then  if and only if , for every .  Again,
the latter task can be accomplished in polynomial time, since, in view
of the previous two cases  can be checked in polynomial time, for each
-tuple , and the number of
such -tuples is polynomial.



\item [ is a purely universal formula of level 3:]
This case can be handled much along the same lines of the previous
case.
\end{description}
Summing up, we have shown that the satisfiability problem for
-formulae is in \textsf{NP}.  This, together with its
\textsf{NP}-hardness, which was shown before, implies the
\textsf{NP}-completeness of our problem.
\end{proof}

In the next section we show how the fragment  can be used
to formalize the modal logic .

\subsection{Applying  to modal logic}
The \emph{modal language}  is based on a countably
infinite set of propositional letters , the classical propositional connectives `',
`' , and `', the modal operators `', `'
(and the parentheses).   is the smallest set such that
, and such that if
, then , , , , .  Lower case letters like  denote elements of
 and Greek letters like  and  represent
formulae of .  Given a formula  of
, we indicate with  the collection of
the subformulae of .  The \emph{modal depth} of a formula
 is the maximum nesting depth of modalities occurring in
. In the rest of the paper we also make use of the propositional connective `' defined in terms of
`' and `' as: .



A \emph{normal modal logic}  is any subset of 
which contains all the tautologies and the axiom

and which is closed with respect to the following rules:
\begin{description}
\item [(Modus Ponens):] if , then ,
\item [(Necessitation):] if , then ,
\item [(Substitution):] if , then ,
\end{description}
where , and the formula  is the result of uniformly substituting in  propositional letters with formulae (the reader may consult a text on modal logic like \cite{ModLog01} for more details).

A \emph{Kripke frame} is a pair  such that  is
a nonempty set of possible worlds and  is a binary relation on 
called \emph{accessibility relation}.  If  holds, we say that
the world  is accessible from the world .  A \emph{Kripke model}
is a triple , where  is a
Kripke frame and  is a function mapping propositional letters into
subsets of .  Thus,  is the set of all the worlds in which
 is true.

Let  be a Kripke model and let  be a
world in . Then, for every  and for every
, the satisfaction relation
 is defined as follows:
\begin{itemize}
\item  ~iff~ ;

\item  ~iff~  or ;

\item  ~iff~  and ;

\item  ~iff~ ;

\item  ~iff~ ,
for every  such that ;

\item  ~iff~ there is a  such that  and .
\end{itemize}
A formula  is said to be \emph{satisfied} at  in  if
;  is said to be \emph{valid} in
 (and we write ), if , for every .

The smallest normal modal logic is , which contains only the modal
axiom  and whose accessibility relation  can be any binary
relation.  The other normal modal logics admit together with 
other modal axioms drawn from the
ones in Table \ref{tab:modalax}.

Translation of a normal modal logic into the  language is
based on the semantics of propositional and modal operators. For any normal modal logic, the formalization of
the semantics of modal operators depends on the axioms that characterize the logic.

In the case of the logic , whose decision problem has been shown
to be -complete in \cite{Lad77}, the modal formulae
 and  can be expressed in the
 language and thus the logic  can be entirely translated
into the  fragment.  This is shown in what follows.


\begin{table}[tb]
\begin{center}
\begin{tabular}{|l|l|l|}
  \hline
Axiom & Schema & Condition on  (see Table \ref{tab:accesrel}) \\\hline
  &  & Reflexive \\
 &  & Euclidean \\
  &  & Symmetric \\
 &  & Transitive \\
  &  & Serial: \\
\hline
\end{tabular}
\caption{\label{tab:modalax} Axioms of normal modal logics}
\end{center}
\end{table}

\subsubsection{The logic }
The normal modal logic  is obtained from the logic  by
adding to  the axioms
 and  listed in Table \ref{tab:modalax}.  Semantics of the
modal operators  and  for the logic  can be
described as follows.  Given a formula  of  and a
Kripke model , we put:
\begin{itemize}
\item  ~iff~ , for every  s.t. there is a  with ,
\item  ~iff~ , for some  s.t. there is a  with .
\end{itemize}
This formulation allows one to express a formula  of 
into the  fragment.  In order to simplify the definition of
the translation function  introduced below, we give the notion
of the ``empty formula'', to be denoted by , and which will
not be interpreted in any particular way.  The only requirement on
 needed for the definitions to be given below is that
 and  must be regarded as
syntactic variations of , for any -formula .

Intuitively, the translation function  associates to each formula 
of  a -formula defining a variable  of sort 1, which denotes
the subset  of  such that  if and only if , for every Kripke model
. We proceed as follows.

For every propositional letter , let ,
with , and let  be the function defined recursively as follows:

\begin{itemize}
\item ,

\item ,

\item ,

\item ,

\item 

\hfill ,

\item 

\hfill ,
\end{itemize}
where  is the empty formula, , and .

Finally, for every  in , if  is a
propositional letter in  we put , otherwise .
Next, by means of the following formulae, we characterize a variable
 of sort 3, intended to denote the accessibility relation 
of the logic :
\begin{itemize}
\item ,

\item ,

\item ,

\item ,

\item .
\end{itemize}

Correctness of the translation is stated by the following lemma.
\begin{lemma}\label{leK45}
For every formula  of the logic ,  is
satisfiable in a model  if and only if
there is a
-interpretation satisfying .\end{lemma}
\begin{proof}
Let  be a world in .  We construct a
-interpretation  as follows:
\begin{itemize}
\item ,

\item , where  is a propositional letter and
,

\item , for every subformula  of
,
distinct from a propositional letter.
\end{itemize}
To prove the lemma, it would be enough to show that  ~iff~ .  However, it
is more convenient to prove the following more general property:\\
\indent \emph{Given a  and a  such that , we have
}
We proceed by structural induction on  by considering for
simplicity only the relevant cases in which 
and .
\begin{itemize}
\item Let  and assume that .  Let  be a world of  such that  for some , and let 
be such that  and .  We have that  and, by inductive hypothesis, .
Since , then .  Hence  and thus .  Since , by modus
ponens we have the thesis.  The thesis follows also in the case in
which there is no  such that .  In
fact, in that case  holds for any .

Consider next the case in which .
Then, there must be a  such that , for some , and .  Let
 be such that  and .
Then, by inductive hypothesis, .

By definition of , we have .  By the above instantiations and by the
hypotheses, we have that  and .  Thus, by modus ponens, we
obtain the thesis.

\item Let  and assume that .  Then there are  such that  and .  Let 
be such that  and .  Then, by inductive
hypothesis, .  Since , it follows that .  By the hypotheses and the variable instantiations above
it follows that  and .
Finally, by an application of modus ponens the thesis follows.

On the other hand, if , then for
every , either there is no  such that , or .  Let  be such that  and .  If , by inductive hypothesis, we have that .

Since , by the hypotheses and
by the variable instantiations above we get  and .  Finally,
by modus ponens we infer the thesis.
\end{itemize}
\end{proof}
It can be easily verified that  is polynomial in the
size of  and that its satisfiability can be checked in
nondeterministic polynomial time since the formula

belongs to .\footnote{ is intended to
characterize a nonempty set of possible worlds.} Thus, the decision
algorithm for  we have presented and the translation function
described above yield a nondeterministic polynomial decision procedure
for testing the satisfiability of any formula  of .


\section{Conclusions and future work}
We have presented a decidability result for the satisfiability problem
for the fragment  of multi-sorted stratified syllogistic
embodying variables of four sorts and a restricted form of
quantification.  As the semantics of the modal formulae  and  in the modal logic 
can be easily formalized in a fragment of , admitting a
nondeterministic polynomial decision procedure, we obtained an
alternative proof of the \textsf{NP}-completeness of . The results
reported in the paper offer numerous hints of future work, some of which are discussed in what follows.

 Recently, we have analyzed several fragments of elementary set theory. It
will be interesting to ameliorate existing techniques to verify in a formal way the truth
of expressivity results that for the moment we have only conjectured. Moreover, we plan to
find complexity results for the fragments  (cfr. \cite{CanNic08}) and , and for some of their sublanguages
like, for instance, the sublanguages of  characterized by the fact that quantifier prefixes have
length bounded by a constant. According to the construction of Section \ref{ssseUniv} small models for
formulae of these sublanguages have a finite domain  that is polynomial in the size of the formula. However, their
formulae are not subject to the syntactical constraints characterizing formulae of the  languages and allowing
the satisfiability problem for the  fragments to be \textsf{NP}-complete.

As we mentioned in the Introduction, stratified syllogistics have been studied less than one sorted multi-level ones.
Thus, a comparison of the results obtained in this paper with the results regarding one sorted multi-level
set theoretic decidability is in order.

Formalizations of modal logics in set theory have already been
provided within the framework of hyperset theory \cite{BaMo96} and of
weak set theories \cite{DMP95}, without the extensionality and
foundation axioms.

We intend to continue our study, started with \cite{CanNic08},
concerning the limits and possibilities of expressing modal, and
more generally, non-classical logics in the context of stratified
syllogistics.  Currently, in the case of modal logics characterized by
a liberal accessibility relation like , we are not able to
translate the modal formulae  and 
in . We plan to verify if  allows one to express modal logics with nesting of modal operators of bounded length.
We also intend to investigate extensions of  which
allow one to express suitably constrained occurrences of the
composition operator on binary relations and of the set-theoretic
operator of general union. We expect that these extensions will make it possible to express
all the normal modal logic systems and several multi-modal logics.
Finally, since within  we are able to express Boolean
operations on relations, we plan to investigate the possibility of
translating fragments of Boolean modal logic and expressive description logics
admitting boolean constructors over roles.




\bibliographystyle{plain}
\begin{thebibliography}{10}

\bibitem{BaMo96} J. Barwise and L. Moss.
\newblock{\em Vicious circles}.
\newblock Vol. 60 of {\em CSLI Lecture notes.} CSLI, Stanford, CA, 1996.

\bibitem{ModLog01}
P. Blackburn, M. de Rijke, and Y. Venema.
\newblock{\em Modal Logic}.
\newblock Cambridge University Press, Cambridge Tracts in Theoretical Computer Science, 2001.


\bibitem{Can91}
D.~Cantone.
\newblock Decision procedures for elementary sublanguages of set theory: X. Multilevel syllogistic extended by the singleton and powerset operators.
\newblock {\em Journal of Automated Reasoning}, volume 7, number 2, pages 193--230, Kluwer Academic Publishers,
Hingham, MA, USA, 1991.


\bibitem{CanCut90}
D.~Cantone and V.~Cutello.
\newblock A decidable fragment of the elementary theory of relations and some
  applications.
\newblock In {\em ISSAC '90: Proceedings of the International Symposium on
  Symbolic and Algebraic Computation}, pages 24--29, New York, NY, USA, 1990.
  ACM Press.

\bibitem{CanCut93}
D.~Cantone and V.~Cutello.
\newblock {D}ecision procedures for stratified set-theoretic syllogistics.
\newblock In Manuel Bronstein, editor, {\em Proceedings of the 1993
  International Symposium on Symbolic and Algebraic Computation, ISSAC'93
  (Kiev, Ukraine, July 6-8, 1993)}, pages 105--110, New York, 1993. ACM Press.

\bibitem{CCS90}
D.~Cantone, V.~Cutello, and J.~T. Schwartz.  \newblock Decision
problems for Tarski and Presburger arithmetics extended with sets.
\newblock In {\em CSL '90: Proceedings of the 4th Workshop on Computer
Science Logic}, pages 95--109, London, UK, 1991.  Springer-Verlag.

\bibitem{CanFer1995}
D.~Cantone and A.~Ferro
\newblock Techniques of computable set theory with applications to proof verification.
\newblock {\em Comm. Pure Appl. Math.}, pages 901--945, vol. XLVIII, 1995. Wiley.


\bibitem{CFO89}
D.~Cantone, A.~Ferro, and E.~Omodeo.
\newblock {\em Computable set theory}.
\newblock Clarendon Press, New York, NY, USA, 1989.



\bibitem{CanNic08}
D.~Cantone and M.~Nicolosi Asmundo.
\newblock On the satisfiability problem for a 3-level
quantified syllogistic with an application to modal logic.
\newblock {\em Submitted}, 2012.
Available at: http://www.dmi.unict.it/nicolosi/3LQSRmodal.pdf

\bibitem{COO01}
D.~Cantone, E.~Omodeo, and A.~Policriti.
\newblock {\em Set Theory for Computing - From decision procedures to
  declarative programming with sets}.
\newblock Springer-Verlag, Texts and Monographs in Computer Science, 2001.

\bibitem{DMP95}
G. D'Agostino, A. Montanari, and A. Policriti.
\newblock A set-theoretic translation method for polimodal
logics.
\newblock {\em Journal of Automated Reasoning}, 3(15): 317--337, 1995.

\bibitem{DJ90}
N.~Dershowitz and J.~P.~Jouannaud.
\newblock Rewrite Systems.
\newblock {\em Handbook of Theoretical Computer Science, Volume B: Formal Models and Sematics (B)},
    243--320, 1990.

\bibitem{FerOm1978}
    A.~Ferro and E.G.~Omodeo.
    \newblock {\em An efficient validity test for formulae in extensional two-level syllogistic}.
    Le Matematiche, 33:130--137, 1978.

\bibitem{FoOmPo04}
A.~Formisano, E.~Omodeo, and A.~Policriti.
\newblock Three-variable statements of set-pairing.
\newblock {\em Theoretical Computer Science}, 322(1), 147--173, 2004.


\bibitem{Lad77}
R. Ladner.
\newblock The computational complexity of provability in systems of modal propositional logic.
\newblock{\em SIAM Journal of Computing,} 6: 467-480, 1977.


\end{thebibliography}



\end{document} 