\documentclass[runningheads]{llncs}
\usepackage{amssymb}
\setcounter{tocdepth}{3}
\usepackage{graphicx}
\usepackage{amsmath}
\usepackage{epsfig}
\usepackage{url}

\urldef{\mailsa}\path|{vamsik}@engr.uconn.edu|
\newcommand{\keywords}[1]{\par\addvspace\baselineskip
\noindent\keywordname\enspace\ignorespaces#1}

\begin{document}

\title{A Simplified Proof For The Application Of Freivalds' 
Technique to Verify Matrix Multiplication}

\titlerunning{Simplified proof to verify matrix multiplication}

\author{Vamsi Kundeti}
\institute{Department of Computer Science and Engineering\\
University of Connecticut\\
Storrs, CT 06269, USA\\
\mailsa\\
} \maketitle

\begin{abstract}
{\em Fingerprinting} is a well known technique, which is often used in
designing Monte Carlo algorithms for verifying identities involving matrices,
integers and polynomials. The book by Motwani and Raghavan~\cite{motwani1995}
shows how this technique can be applied to check the correctness of matrix 
multiplication -- check if  where  and  are three  matrices. 
The result is a Monte Carlo algorithm running in time  with an exponentially 
decreasing error probability after each independent iteration. In this paper we 
give a simple alternate proof addressing the same problem. We also give further 
generalizations and relax various assumptions made in the proof.
\end{abstract}

\section{Introduction}
{\em Fingerprinting} or {\em Freivalds' technique} is a standard method which is often
employed in designing Monte Carlo algorithms. Let  be a large universe/set of elements,
given any  our goal is to check if  and  are the same. Since we need 
 bits to represent any , this means checking if  
deterministically would need  time. The basic idea behind finger printing
is create a random mapping  such that , and verify if .
However it should be clear that  does not necessarily mean  -- in fact the goal is
to find a  such the {\em error probability}  is very small. Once we prove
that our {\em error probability} is bounded by some constant, a Monte Carlo algorithm is clearly
immediate. Motwani and Raghavan~\cite{motwani1995} applied this technique to check the correctness
of matrix multiplication, we state the as follows. Given three  matrices  and  
check if . Clearly a simple deterministic algorithm takes  time. Firstly In 
this paper we give a simple alternate proof for the Theorem- presented in~\cite{motwani1995},
secondly we relax various constraints and give a much general proof.

\section{Our Proofs}
We first give a simple and alternative proof for Theorem- in ~\cite{motwani1995}. Later in Theorem~\ref{thm2}
we show that the assumption on the {\em uniformness} is not necessary. 
\begin{theorem}
\label{thm1}
Let , and  be three  matrices such that . Let 
is a random vector from a uniform distribution. Then 
\end{theorem}
\begin{proof}
Let  be a  matrix and  be the column vectors of .
Then . This means that multiplying a vector with a matrix is linear
combination of the columns, the coefficient  is the  component of . Since 
is a boolean and  acts as an indicator variable on the selection of column . So if 
is chosen from a uniform distribution . 

Now let  and  be the column vectors of , similarly 
let  be the column vectors of . Let
, clearly  since . Then

since .  Intuitively this means we select our random vector  such that
 for all , such a selection will always ensure  even though
.
\end{proof}

\begin{theorem}
\label{thm2}
Let , and  be three  matrices. Let  any vector
with each component  is a  random variable . Then 
. Where  is an arbitrary probability density/distribution
function.
\end{theorem}
\begin{proof}
Continuing with the proof of Theorem-~\ref{thm1} , .
\end{proof}

\begin{corollary}
There always exists an  time Monte Carlo algorithm with exponentially decreasing error probability, for the
problem to check if . 
\end{corollary}

\section{Conclusions}
We give a simple and alternate proof for the proof given by Motwani~\cite{motwani1995}, to verify if 
using a Monte Carlo algorithm. We also relax uniformness assumption made by the proof.

\bibliographystyle{splncs}
\bibliography{RAND-2009.bib}

\end{document}
