\documentclass{sig-alt-full}

\overfullrule=5mm
\usepackage{amssymb,amsmath}
\usepackage{theorem}
\usepackage{graphics}
\usepackage{amsfonts}
\usepackage{mathrsfs}
\usepackage{amscd}
\usepackage{color}
\usepackage{url}
\usepackage[plainpages=false,pdfpagelabels,colorlinks=true,citecolor=blue,hypertexnames=false]{hyperref}

\hyphenation{bi-de-gree}

\newcommand{\bigO}{{\mathcal{O}}}
\newcommand{\bigOsoft}{\tilde{\mathcal{O}}}
\newcommand{\lc}{\operatorname{lc}}
\newcommand{\diag}{\operatorname{diag}}
\newcommand{\ev}{\operatorname{\sf Ev}}
\newcommand{\bideg}{\operatorname{bideg}}
\newcommand{\HO}{\mathcal H}
\newcommand{\bN}{\mathbb{N}}
\newcommand{\cM}{\mathcal M}
\newcommand{\cP}{\mathcal P}
\newcommand{\cQ}{\mathcal Q}

\def\gathen#1{{#1}}
\def\hoeven#1{{#1}}

\newtheorem{lemma}{Lemma}
\newtheorem{theorem}[lemma]{Theorem}
\newtheorem{fact}[lemma]{Fact}
\newtheorem{cor}[lemma]{Corollary}

\newtheorem{problem}{Problem} \renewcommand{\theproblem}{}
\newtheorem{hypothesis}{Hypothesis} \renewcommand{\thehypothesis}{}
\newtheorem{notation}{Notation} \renewcommand{\thenotation}{}
\newtheorem{definition}{Definition} \renewcommand{\thedefinition}{}

\arraycolsep2pt

\begin{document}
\conferenceinfo{ISSAC XXXX,}{}
\CopyrightYear{2010}
\crdata{}

\title{Complexity of Creative Telescoping\\ for Bivariate Rational Functions\titlenote{\small We warmly thank the referees for their very helpful comments.
---
AB and FC were supported in part
by the Microsoft Research\,--\,Inria Joint Centre,
and SC and ZL by a grant of the
National Natural Science Foundation of China (No.~60821002).\vspace{-25pt}}}
\newfont{\authfntsmall}{phvr at 11pt}
\newfont{\eaddfntsmall}{phvr at 9pt}
\def\more-auths{\end{tabular}
\begin{tabular}{c}}
\numberofauthors{2}
\author{
\alignauthor {\authfntsmall Alin Bostan, Shaoshi Chen, Fr\'ed\'eric Chyzak}\\
\affaddr{Algorithms Project-Team, INRIA Paris-Rocquencourt}\\
\affaddr{78153 Le Chesnay (France)}\\
\email{\eaddfntsmall{\{alin.bostan,shaoshi.chen,frederic.chyzak\}@inria.fr}}
\alignauthor {\authfntsmall Ziming Li}\\
\affaddr{Key Laboratory of Mathematics Mechanization, Academy of Mathematics and System Sciences}\\
\affaddr{100190 Beijing (China)}\\
\email{\eaddfntsmall{zmli@mmrc.iss.ac.cn}}}

\maketitle
\begin{abstract}
The long-term goal initiated in this work is to obtain fast
algorithms and implementations for definite integration in
Almkvist and Zeilberger's framework of (differential)
creative telescoping.
Our complexity-driven approach is to
obtain tight degree bounds on the various expressions
involved in the method. To make the
problem more tractable, we restrict
to \emph{bivariate rational\/} functions. By considering this
constrained class of inputs, we are able to blend the general method
of creative telescoping with the well-known Hermite reduction. We
then use our new method to compute diagonals of rational power series
arising from combinatorics.
\end{abstract}

\vspace{1mm}
\noindent
{\bf Categories and Subject Descriptors:} \\
\noindent I.1.2 [{\bf Computing Methodologies}]: Symbolic and
Algebraic Manipulations --- \emph{Algebraic Algorithms}

\vspace{1mm}
\noindent {\bf General Terms:} Algorithms, Theory.

\vspace{1mm}
\noindent {\bf Keywords:} Hermite reduction, creative telescoping.


\begin{section}{Introduction}
The long-term goal of the research initiated in the present work is to
obtain fast algorithms and implementations for the definite integration of
general special functions, in a complexity-driven perspective.

As most special-function integrals cannot be expressed in closed form,
their evaluation cannot be based on table look-ups only, and even when
closed forms are available, they may prove to be intractable in
further manipulations.  In both cases, the difficulty can be mitigated
by representing functions by annihilating differential operators.
This motivated Zeilberger to introduce a method now known as
\emph{creative telescoping\/}~\cite{Zeilberger1990}, which
applies to a large class of special functions:
the D-finite functions~\cite{Lipshitz1989} defined by
sets of linear differential equations of any order, with polynomial
coefficients.  Zeilberger's method applies in general to multiple
integrals and sums.

\begin{figure*} \begin{center} \renewcommand{\arraystretch}{1.2}
\tabcolsep4pt
\begin{tabular}{l|l|ll|ll|ll}
  \hline
  & Method &  &  &  &  & Complexity & \\ \hline
  Minimal & Hermite reduction (new) &  &  &  &   &  & Las Vegas \\ \cline{2-8}
  Telescoper & Almkvist and Zeilberger &  &  &  &  &  & Las Vegas \\ \hline
  Nonminimal& Lipshitz elimination &  &  &  &  &  & deterministic \\\cline{2-8}
  Telescoper & Cubic size &  &  &  &  &  & deterministic \\\hline
\end{tabular}
\caption{Complexity of creative telescoping methods (under Hyp.~(H')), together with bounds on output}\label{fig:complexity}
\end{center}
\vskip-15pt
\end{figure*}




A sketch of Zeilberger's method is as follows.
Given a D-finite function~ of the variables
 and~, the definite integral

is D-finite, and a linear differential equation satisfied by~ can
be constructed~\cite{Zeilberger1990}.
To explain this, let  be
a field of characteristic zero,  and~ be the usual
derivations on the rational-function field ,
both restricting to zero on~,
and let  be the ring of linear differential
operators over .
The heart of the method is to solve the
\emph{differential telescoping equation}
\eqref{eq:CT} below
for  and  for
some .
The operator~ is called a
\emph{telescoper\/} for~, and ~a \emph{certificate\/} of~ for~.
Under the assumption

 is then proved to be an annihilator of .

The main emphasis in works since the 1990's has been on finding
telescopers of order minimal over all telescopers for~, which are
called \emph{minimal telescopers}.
(Two minimal telescopers differ by a multiplicative factor in~.)
In view of the computational
difficulty of solving~\eqref{eq:CT}, there has been special
attention to subclasses of inputs. Of particular importance is the
case of hyperexponential functions, defined by first-order
differential equations, studied by Almkvist and Zeilberger
in~\cite{Almkvist1990}. Their method is a direct differential
analogue of Zeilberger's algorithm for the recurrence
case~\cite{Zeilberger1991}.

On the other hand, very little is known about the complexity of
creative telescoping: the only related result seems to be an
analysis in~\cite{Gerhard2004} of an algorithm for hyperexponential
indefinite integration.
In order to get complexity estimates, we simplify the
problem by restricting to a smaller class of inputs, namely that of
bivariate rational functions.
Although restricted, this class already has many applications,
for instance in combinatorics,
where many nontrivial problems are encoded as diagonals of rational
formal power series, themselves expressible as integrals.
Our goal thus reads as follows.

\begin{problem}
Given , find a pair  with
 in  and  in  such that

\end{problem}

By considering this more constrained class of inputs, we are indeed
able to blend the general method of creative telescoping with the
well-known Hermite reduction~\cite{Hermite1872}.

Essentially two algorithms for minimal telescopers can be
found in the literature:
The classical way~\cite{Almkvist1990}
is to apply a differential analogue of Gosper's indefinite summation
algorithm, which reduces the problem to solving an auxiliary linear
differential equation for polynomial solutions.
An algorithm developed later in~\cite{GeddesLe2002} (see
also~\cite{Le2000}) performs Hermite reduction on~ to get an
additive decomposition of the form , where~the  and  are in  and the~
are squarefree. Then, the algorithm in~\cite{Almkvist1990} is
applied to each~ to get a telescoper~ minimal for it.
The least common left multiple of the 's is then proved to be a
minimal telescoper for~. This algorithm performs well only for
specific inputs (both in practice and from the complexity
viewpoint), but it inspired our Lemma~\ref{le:dGf-for-F'/F}
via~\cite{Le2000}.

As a first contribution in this article, we present a new, provably
faster algorithm for computing minimal telescopers for bivariate
rational functions.
Instead of a single use of Hermite reduction as in~\cite{Le2000}, we apply
Hermite reduction to the 's, iteratively for~,
which yields

for some factor~ of the squarefree part of the denominator of~.
If  are not all zero and such that
, then the operator~ is a telescoper for~, and more specifically, the first nontrivial linear
relation obtained in this way yields a minimal telescoper for~.

As a second contribution, we give the first proof of a polynomial
complexity for creative telescoping on a specific class of inputs,
namely on bivariate rational functions.
For \emph{minimal\/} telescopers, only a polynomial bound on~ (but none
on~) was given for special inputs in~\cite{GeddesLe2002};
more specifically, we derive complexity estimates for all
mentioned methods (see Fig.~\ref{fig:complexity}), showing that
our approach is faster.
Furthermore, we analyse the bidegrees of \emph{non minimal\/}
telescopers generated by other approaches:
Lipshitz' work~\cite{Lipshitz1988} can be
rephrased into an existence theorem for telescopers with
polynomial size; the approach followed in the recent work on
algebraic functions~\cite{BCLSS2007} leads to
telescopers of smaller degree sizes.
These are new instances of the philosophy, promoted in~\cite{BCLSS2007},
that relaxing minimality
can produce smaller outputs.

A third contribution is a fast Maple implementation~\cite{OurSoft},
incorporating a
careful implementation of the original Hermite reduction algorithm,
making use of the special form of~ in~\eqref{EQ:incremental}
and of usual modular techniques (probabilistic rank estimate) to
determine when to invoke the solver for linear algebraic equations.
Experimental results indicate that our implementation
outperforms Maple's core routine.

Note that for the fastest method we propose, denoted by~\verb+H1+ in
Tables~\ref{tab:random}--\ref{tab:algos}, we chose to output the
certificate as a mere sum of (small) rational functions, without any
form of normalisation. This choice seems to be uncommon for
creative-telescoping algorithms, but a motivation is how the
certificate is used in practice: Very often, like for applications
to diagonals in \S\,\ref{sec:implementation}, the certificate is
actually not needed. In other applications, the next step of the
method of creative telescoping is to integrate~\eqref{eq:CT} between
 and~, leading to . Therefore, only evaluations of the certificate are really
needed, and normalisation can be postponed to after specialising at
 and~.

The end of this section, \S\,\ref{sec:background}, provides classical
complexity results, notation, and hypotheses that will be used
throughout.  We then study Hermite reduction over~ in
\S\,\ref{sec:Hermite-reduction}, proving output degree bounds and a
low-complexity algorithm.  This is then applied in
\S\,\ref{sec:minimal-order} to derive our new algorithm for creative
telescoping, and to compare its complexity with that of Almkvist and
Zeilberger's approach.
For nonminimal telescopers, we show the existence
of some of lower arithmetic size in \S\,\ref{sec:nonminimal-order}:
cubic for nonminimal order instead of quartic for minimal order.
See the summary in Figure~\ref{fig:complexity}, where the low
complexity of algorithms for minimal telescopers relies on Storjohann
and Villard's algorithms~\cite{Storjohann2005}, thus inducing a
\emph{certified\/} probabilistic feature.
We apply our results to the
calculation of diagonals in \S\,\ref{sec:implementation}, and describe our
implementation and comment on execution timings
in~\S\,\ref{sec:implementation}.

\begin{subsection}{Background on complexity --- Notation}
\label{sec:background} We recall basic notation and complexity facts
for later use. Let  be again a field of characteristic zero.
Unless otherwise specified, all complexity estimates are given in
terms of arithmetical operations in , which we denote by ``ops''.
Let  be the set of  matrices
with coefficients in~ of degree at most~. Let
 be a feasible exponent of matrix multiplication, so
that two matrices from~ can be multiplied using
 ops.
Facts \ref{EvaInter} and~\ref{le:polymatrix} below
show the complexity of multipoint evaluation, rational
interpolation,
and algebraic operations on polynomial matrices using fast
arithmetic, where the notation  indicates
cost estimates with hidden logarithmic
factors~\cite[Def.~25.8]{MCA2003}.

\begin{fact}\label{EvaInter}
For  of degree less than~, pairwise
distinct  in~, and ,
we have:
\vspace{-0.15cm}
\begin{enumerate}
\item[(i)] Evaluating  at the 's takes  ops.
\vspace{-0.2cm}
\item[(ii)] For , constructing
 with  and  such that
 and  for  takes
~ops.
\end{enumerate}
\end{fact}

\begin{fact}\label{le:polymatrix}
For  in , \ , we have:
\vspace{-0.15cm}
\begin{enumerate}
  \item[(i)] If  is an invertible  matrix with ,
where  and , then the degree of  is at most
.
\vspace{-0.2cm}
  \item[(ii)] If  is not of full rank and with , where  and , then there exists a nonzero  with coefficients of degree at most~ such that .
\vspace{-0.2cm}
  \item[(iii)]The rank  and a basis of the null space of  can be computed using
  ~ops.
\end{enumerate}
\end{fact}

\noindent (For proofs, see \cite[Cor.~10.8, 5.18, 11.6]{MCA2003}
and~\cite[Th.~7.3]{Storjohann2005}.)

\smallskip

We call \emph{squarefree factorisation\/} of~
w.r.t.~ the unique product 
equal to~ for  and 
satisfying~ and such that the~'s are primitive,
squarefree, and pairwise coprime. The \emph{squarefree part~
of~\/} w.r.t.~ is the product .
Let  denote the polynomial~, and  the leading
coefficient of~ w.r.t.~.
The following two formulas about , , and  can be proved
by mere calculations.

\begin{fact}\label{prop:deflation}
Let  denote . Then we have
\vspace{-0.15cm}
\begin{enumerate}
 \item[(i)]  ;
\vspace{-0.2cm}
 \item[(ii)] .
\end{enumerate}
\end{fact}

Let  be a nonzero element in~, where~ are two
coprime polynomials in~. The degree of~ in~ is
defined to be~, and denoted
by~. The degree of~ in~ is defined similarly. The
\emph{bidegree\/} of~ is the pair~, which
is denoted by~. The bidegree~of~ is said to be
\emph{bounded (above) by~}, written
, when~
and~.

We say that~ is \emph{proper\/} if the degree of~ in~ is
less than that of~.
For creative telescoping, we may always assume w.l.o.g.\ that
~is proper. If not, rewrite  with  and ~proper. A telescoper~ for~ with
certificate~ is a telescoper for~ with certificate~.

\medskip\noindent{\bf Hypothesis (H)} \ \emph{From now on,  and
 are assumed to be nonzero polynomials in  such that
, \ , and ~is primitive
w.r.t.\ .}

\medskip\noindent{\bf Notation} \ \emph{From now on, we write , , and  for the bidegrees of
, , and , respectively.}

\medskip\noindent The following hypothesis makes our estimates
concise.

\medskip\noindent{\bf Hypothesis (H')} \ \emph{Occasionally, we shall
require the ex\-tend\-ed hypothesis: Hypothesis~(H) and
.}
\end{subsection}

\end{section}

\begin{section}{Hermite reduction}\label{sec:Hermite-reduction}
Let  be a field of characteristic zero, either  or~ in
what follows. Let  be the field of rational functions in~
over~, and  be the usual derivation on it. For a rational
function~, \emph{Hermite reduction\/}~\cite{Hermite1872}
computes rational functions  and   in  satisfying

Horowitz and Ostrogradsky's method~\cite{Ostrogradsky1845,
Horowitz1971} computes the same decomposition as in~\eqref{eq:ADP}
by solving a linear system. For the details of those methods,
see~\cite[Chapter 2]{BronsteinBook}.
\begin{lemma}\label{le:unique}
If ~is proper, a pair  satisfying~\eqref{eq:ADP} for
proper~ is unique.
\end{lemma}
\begin{proof}
This is a consequence of~\cite[Theorem~2.10]{Horowitz1971} after
writing~ as a sum~ and integrating.
\end{proof}
\begin{lemma}\label{le:ComplexityUHR}
Let~ be a nonzero rational function in~ of degree at most~ in~,
then Hermite reduction on~ can
be performed using~ operations in~.
\end{lemma}
\begin{proof}
See~\cite[Theorem 22.7]{MCA2003}.
\end{proof}
In contrast, the method of Horowitz and Ostrogradsky takes
 operations in~~\cite[\S\,22.2]{MCA2003}. Thus,
Hermite's method is quasi-optimal and asymptotically faster than the
former.

From now on, we fix  and analyse the complexity
of
Hermite reduction over~ in terms of
operations in~.
To this end, we use an evaluation-interpolation approach.

\begin{subsection}{Output size estimates}
We derive an upper bound on the bidegrees of  and~
satisfying~\eqref{eq:ADP} by studying the linear system
in~\cite{Horowitz1971}.

Analysing Hermite reduction (under~(H))
shows the existence of 
with , ~and

In order to bound the bidegrees of  and~, we
reformulate~\eqref{eq:HOansatz} into the equivalent form

where  is a polynomial in  of bidegree at most  by Fact~\ref{prop:deflation}.
Viewing  and~ as polynomials in  with undetermined
coefficients, we form the following linear system, equivalent to~\eqref{eq:HO},

where , , and , , and
 are the coefficient vectors of , , and  with sizes
, , and , respectively.
Under the constraint of properness of  and~,  is
unique by Lemma~\ref{le:unique}.
Then~\eqref{eq:HOsystem} has a unique solution, which leads to
the following lemma.
\begin{lemma}\label{le:HOsystem}
The matrix  is invertible over .
\end{lemma}
As the matrix  is
uniquely defined by~, we call it the matrix \emph{associated\/}
with~, denoted by . Let  be its determinant, so
that  by
Fact~\ref{le:polymatrix}\emph{(i)}.
For later use, we also define ~as the determinant
of~, so that  by
Fact~\ref{le:polymatrix}\emph{(i)} and since .
\begin{lemma} \label{le:HRsize}
There exist  with
 and~, and such that:
\vspace{-0.15cm}
\begin{enumerate}
\item[(i)] ;
\vspace{-0.2cm}
\item[(ii)] 
and~.
\end{enumerate}
\end{lemma}
\begin{proof}
Applying Cramer's rule to~\eqref{eq:HOsystem}
leads to~\emph{(i)}.
Assertion~\emph{(ii)\/} next follows by determinant expansions.
\end{proof}

In what follows, we shall encounter proper rational functions
with denominator~ satisfying~.
The following lemma is an easy corollary of Lemma~\ref{le:HRsize} for
such functions.
\begin{cor} \label{cor:specialQ}
Assuming  in addition to Hypothesis~(H), there exist  with  and~ less than~, and
such that
\begin{enumerate}
\item[(i)] ;
\item[(ii)]  and~ are bounded
by .
\end{enumerate}
\end{cor}
\end{subsection}
\begin{subsection}{Algorithm by evaluation and interpolation}
We observe that an asymptotically optimal complexity can be achieved
by evaluation and interpolation at each step of Hermite reduction
over . This inspires us to adapt Gerhard's modular
method~\cite{Gerhard2001, Gerhard2004} to~.
Recall that, by Hyp.~(H), ~is nonzero and primitive over~.


\medskip\noindent {\bf Definition}
\emph{An element  is \emph{lucky\/} if\/
 and .}
\begin{lemma}\label{le:unlucky}
There are at most  unlucky points.
\end{lemma}
\begin{proof}
Let  be the th subresultant w.r.t.\  of 
and~.
By~\cite[Corollary 5.5]{Gerhard2004}, all
unlucky points are in the set .
By~\cite[Corollary 3.2\emph{(ii)}]{Gerhard2004}, .
\end{proof}
\begin{lemma}\label{le:commutative}
Let , , and  be the same
as in Lemma~\ref{le:HRsize}, and let  be lucky.
Then  and 
is the unique pair such that

\end{lemma}
\begin{proof}
By the luckiness of~,  and
, so .
This implies , which, by
Lemma~\ref{le:HOsystem}, is invertible over~.
Hence~,
and the evaluation at~ of the equality in
Lemma~\ref{le:HRsize}\emph{(i)\/} is well-defined.
Thus,  is a solution of~\eqref{eq:ADk}.
Uniqueness follows from Lemma~\ref{le:unique}.
\end{proof}
\begin{theorem}\label{th:BHR}
Algorithm \textsf{HermiteEvalInterp} in Figure~\ref{fig:HREvaInter} is
correct and takes~~ops.
\end{theorem}
\begin{proof}
Set  to~. Lemma~\ref{le:unlucky} implies that
the  lucky points found in Step~3 are all less
than~.
By Lemmas \ref{le:unique} and~\ref{le:HRsize}\emph{(i)}, 
and~.
By Lemma~\ref{le:commutative},  and
.
By Lemma~\ref{le:HRsize}\emph{(ii)\/} and
since~, it suffices to rationally interpolate
 and~ from values at  lucky points.
This shows the correctness.
The dominant computation in Step~1 is the gcd, which takes
~ops by \cite[Cor.~11.9]{MCA2003}.
For each integer , testing luckiness amounts to evaluations at~ and
computing , which takes
 ops by Fact~\ref{EvaInter}\emph{(i)\/}
and~\cite[Cor.~11.6]{MCA2003}. Then,
generating~ in Step~3 costs  ops.
By Fact~\ref{EvaInter}\emph{(i)}, evaluations in Step~4 take
 ops. For each , the cost of
the Hermite reduction in Step~4 is  ops by
Lemma~\ref{le:ComplexityUHR}. Thus, the total cost of Step~4 is
 ops. By
Fact~\ref{EvaInter}\emph{(ii)}, Step~5
takes  ops.
Since  and ,
the total cost is as announced.
\end{proof}
\begin{figure}
\framebox[8.4cm]{
\begin{minipage}{8.1cm}
\rule[.3cm]{0cm}{0cm}
{\rm {Algorithm \textsf{HermiteEvalInterp}()}

\smallskip

\noindent \quad{\sc Input}:~ satisfying Hypothesis~(H).

\noindent \quad{\sc Output}:~
solving~\eqref{eq:HOansatz}.
\begin{enumerate}
\item Compute  and ;
\item Set ;
\item Set  to the set of  smallest nonnegative integers that are lucky for~;
\item For each , compute  such that

using Hermite reduction over ;
\item Compute  by rational interpolation and return this pair.
\end{enumerate}}
\end{minipage}}
\caption{Hermite reduction over  via evaluation and
interpolation.} \label{fig:HREvaInter}
\vskip-10pt
\end{figure}

\end{subsection}

\vspace{-0.14cm}
As the generic output size of Hermite reduction is proportional
to~, which is~, Algorithm
\textsf{HermiteEvalInterp} has quasi-optimal complexity.
\end{section}

\begin{section}{Minimal telescopers}\label{sec:minimal-order}
We analyse two algorithms for constructing minimal telescopers for
bivariate rational functions and their certificates.
\begin{subsection}{Hermite reduction approach}\label{HRA}
We design a new algorithm, presented in Figure~\ref{fig:HRTelescoping},
to compute minimal telescopers for rational functions by basing on
Hermite reduction. For~ and~, Hermite reduction
decomposes~ into

where~ are proper. Since the squarefree
part of the denominator of~ divides~, so does the
denominator of~.
The following lemma shows that \eqref{eq:ithHR}~recombines into
telescopers and certificates;
next, Lemma~\ref{le:minimaltele} implies that the first pair obtained in
this way by Algorithm \textsf{HermiteTelescoping}
in Figure~\ref{fig:HRTelescoping}
yields a minimal telescoper.

\begin{lemma}\label{le:bound}
The rational functions  are linearly dependent
over~.
\end{lemma}

\begin{proof}
The constraints on~ imply
 for all , from which
follows the existence of a nontrivial linear dependence among the
's over .
\end{proof}

\begin{lemma}\label{le:minimaltele}
An integer~ is minimal such that 
for  not all zero
if and only if
 is a minimal telescoper for~ with
certificate .
\end{lemma}

\begin{proof}
Multiplying~\eqref{eq:ithHR} by  before summing yields

where the first two sums are proper.
Thus, by Lemma~\ref{le:unique}, ~is a telescoper of order~ for~
with certificate 
if and only if  with~.
The lemma follows.
\end{proof}

\begin{subsubsection}{Order bounds for minimal telescopers}\label{se:lowerbound}
Lemmas \ref{le:bound} and~\ref{le:minimaltele} combine into an upper bound
on the order of minimal telescopers for~.
\begin{cor}\label{cor:upperbound}
Minimal telescopers have order at most~.
\end{cor}
The bound~ is shown in~\cite{BCLSS2007}
for rational functions of the form  with .
Apagodu and Zeilberger~\cite{Apagodu2006} obtain a similar bound for a
class of nonrational hyperexponential functions, but their proof does
not seem to apply to rational functions, as it heavily relies on the
presence of a nontrivial exponential part.

We also derive a lower bound on the order of the minimal telescoper,
to be used as an optimisation at the end of \S\,\ref{sec:compl-estim}:
choosing a lucky~,
next applying Hermite reduction in~ to~, yields

where  are proper and the denominator
of~ divides~.
Let  be the smallest integer such that  are linearly dependent over~.
\begin{lemma}\label{le:lowerbound}
A minimal telescoper has order at least .
\end{lemma}
\begin{proof}
We first claim that , for  as
in~\eqref{eq:ithHR}. Note that the squarefree part w.r.t.~ of the
denominator of  divides  for all .
By~\cite[Cor.~5.5]{Gerhard2004},  is lucky for the
denominator of  for all .
Then, the claim on~ follows from Lemma~\ref{le:commutative}
applied to~.
Let  be the
minimal order of a telescoper, then  are
linearly dependent over~ by Lemma~\ref{le:minimaltele}. Thus
 are linearly dependent over~,
which implies .
\end{proof}
\end{subsubsection}

\begin{subsubsection}{Degree bounds for minimal telescopers}
To derive degree bounds for  and 
in~\eqref{eq:ithHR},
let , , , and  be defined
as before Lemma~\ref{le:HRsize},
and set .

\begin{lemma} \label{le:special}
Let~ be in  with . Then, for all , there exist~ with both  and
 bounded by~, such that

\end{lemma}

\begin{proof}
A straightforward calculation leads to

where~.
By Corollary~\ref{cor:specialQ},
there exist  such that

with  and  bounded by
 .
Setting~
ends the proof.
\end{proof}

\begin{lemma} \label{le:ithsize}
For~, there exist~ such that

Moreover,  and .
\end{lemma}
\begin{proof}
We proceed by induction on~.
For~, the claim follows from
Lemma~\ref{le:HRsize}.
Assume that~ and that the claim holds
for the values less than~.
For brevity, we set ,
\ ,
and .
The induction hypothesis implies

with bidegree bounds on  and~.
Fact~\ref{prop:deflation}\emph{(i)\/} implies that  is in , with
.
Hence .
This observation and an easy calculation imply that

where~ and~.
Furthermore, by Lemma~\ref{le:special}
there are~ with bidegrees at most
, such that

Setting~ and~, we
arrive at~\eqref{EQ:ten}.
It remains to verify the degree bounds.
The induction hypothesis implies that
both  and~ are bounded by~.
It follows that  is
bounded by~.
Similarly,~ is bounded by~, and so is~.
The bounds on degrees in~ are obvious.
\end{proof}

We next derive degree bounds for the minimal telescopers
obtained at an intermediate stage of \textsf{HermiteTelescoping};
refined bounds on the output will be given by Theorem~\ref{th:AZtelesize}.

\begin{lemma}
Under~(H'), Step~2(c) of Algorithm \textsf{Hermite\-Telescoping} computes
a minimal telescoper  with order~
and a certificate  for  with  and .
\end{lemma}

\begin{proof}
By Lemma~\ref{le:minimaltele}, we exhibit a minimal telescoper
by considering the first nontrivial linear dependence among the
's in~\eqref{EQ:ten}.
Let  be the coefficient matrix of the system in~
obtained from .
By Lemma~\ref{le:ithsize}, ~is of size at most
 and with coefficients of degree at most
 in .
Hence, there exists a solution
 of degree at
most~ in~ by
Fact~\ref{le:polymatrix}\emph{(ii)}.
Since  and , the degree estimates of
 and  are as announced.
\end{proof}
\end{subsubsection}

\begin{figure}
\framebox[8.4cm]{
\begin{minipage}{8.1cm}
\rule[.3cm]{0cm}{0cm}
{\rm {Algorithm \textsf{HermiteTelescoping}()}

\smallskip

\noindent \quad{\sc Input}: 
satisfying Hypothesis~(H).

\noindent \quad{\sc Output}:
A minimal telescoper  with certificate .
\begin{enumerate}
\item Apply~\textsf{HermiteEvalInterp} to  to get  such that .
If , return .
\item For  from 1 to  do
\begin{enumerate}
\item Apply~\textsf{HermiteEvalInterp} to 
  to express it as .
\item Set  and .
\item Solve  for 
  using \cite{Storjohann2005}.
If there exists a
nontrivial solution, then set , and break.
\end{enumerate}
\item Compute the content~ of~ and return
.
\end{enumerate}}
\end{minipage}}
\caption{Creative telescoping by Hermite reduction}
\label{fig:HRTelescoping}
\vskip-10pt
\end{figure}

\begin{subsubsection}{Complexity estimates}\label{sec:compl-estim}
We proceed to analyse the complexity of the algorithm in
Figure~\ref{fig:HRTelescoping} and of an optimisation.

\begin{theorem}
Under Hyp.~(H'), Algorithm
\textsf{HermiteTelescoping} in Figure~\ref{fig:HRTelescoping} is correct
and takes  ops,
where  is the order of the minimal telescoper.
\end{theorem}

\begin{proof}
The formulas in Step~2(a) create the loop invariant .
Correctness then follows from Lemmas~\ref{le:bound} and~\ref{le:costlowerbound}.
Step~1 takes  ops by Theorem~\ref{th:BHR} under~(H').
By Lemma~\ref{le:ithsize}, .
So the cost for performing Hermite reduction on
 in Step~2(a) is~
ops by Theorem~\ref{th:BHR}. The bidegrees of  and  in
Step~2(b) are in  by
Lemma~\ref{le:ithsize}. Since adding and differentiating have linear
complexity, Step~2(b) takes  ops.
For each~, the coefficient matrix of 
in Step~2(c) is of size at most  and with
coefficients of degree at most .
Moreover, the rank of this matrix is either  or~.
Then, Step~2(c) takes  ops by
Fact~\ref{le:polymatrix}\emph{(iii)}.
Computing the content and divisions in Step~3 has
complexity .
If the algorithm returns when
, then the total cost is in

which is as announced.
\end{proof}

An optimisation, based on Lemma~\ref{le:lowerbound},
consists in guessing the order~ so as to perform
Step~2(c) a few times only:
As a preprocessing step, choose  lucky for~, then detect
linear dependence of~ in~\eqref{eq:ithUHR}.
The minimal~ for dependence is a lower bound~ on~.
So Step~2(c) is then performed only when~.
In practice, the lower bound~ computed in this way almost always
coincides with the actual order~.
So normalising the 's becomes the dominant step, as observed
in experiments.
We analyse this optimisation by first estimating the cost for
computing~.

\begin{lemma}\label{le:costlowerbound}
Under Hypothesis~(H'), computing a lower order bound~ for minimal
telescopers takes ~ops.
\end{lemma}

\begin{proof}
Since differentiating has linear complexity, the derivative
 takes  ops.
By Fact~\ref{EvaInter}\emph{(i)}, the evaluation 
takes as much.
The cost of Hermite reduction
on  is  ops by
Lemma~\ref{le:ComplexityUHR}.
By Fact~\ref{le:polymatrix}\emph{(iii)\/} with~,
computing the rank of the coefficient
matrix of~, with  as
in~\eqref{eq:ithUHR}, takes  ops.
Thus, the total cost for computing a lower bound on~
is ~ops.
\end{proof}

\begin{cor}
For runs such that , the previous optimisation
of~\textsf{HermiteTelescoping} takes
~ops.
\end{cor}
\begin{proof}
In view of Lemma~\ref{le:costlowerbound},
the estimate \eqref{eq:analyis-without-rho0} becomes
,
which is ~ops,
whence the result.
\end{proof}

\end{subsubsection}
\end{subsection}

\begin{subsection}{Almkvist and Zeilberger's approach}\label{AZA}
We analyse the complexity of Almkvist and Zeilberger's
algorithm~\cite{Almkvist1990} when restricted to bivariate rational
functions.
In order to get a telescoper whose order~ is minimal, the resulting
algorithm, denoted \textsf{RatAZ}, solves~\eqref{eq:CT} for
increasing, prescribed values of~
until it gets a solution  with the 's not all zero.
For the analysis, we start by studying the parameterisation of the
differential Gosper algorithm of~\cite{Almkvist1990}
under the same restriction to~.

\begin{definition}[\cite{Gerhard2004}]
Let  be a field and  be non\-zero polynomials. A
triple  is said to be a \emph{differential Gosper
form\/} of the rational function  if

\end{definition}

For hyperexponential~, a key step in~\cite{Almkvist1990} is to
\emph{compute\/} a differential Gosper form of the logarithmic derivative of
, where the~'s are
undetermined from~.
In the analogue \textsf{RatAZ}, this form is \emph{predicted\/} by
Lemma~\ref{le:dGf-for-F'/F} below, which is a technical
generalisation of a result by Le~\cite{Le2000} on~ when~ has a
squarefree denominator.

Write , splitting content and primitive part
w.r.t.~.
By an easy induction,  for .
For this section, set , , and
.

\begin{lemma}\label{le:dGf-for-F'/F}
If ~is nonzero,
the triple  is a differential Gosper form of .
\end{lemma}

\begin{proof}
First, observe  and~.
Next,  is
.
There remains to prove , for
any~.
Recall that the squarefree part~ of~ is the product~
and that ~denotes~.
By Fact~\ref{prop:deflation}\emph{(ii)},

If ~divides~, ~reduces to~ modulo~.
If not, it reduces to ,
which rewrites to~ modulo~.
In both cases, ~is coprime with~, as , \ ,
and~.
\end{proof}

By another induction, we observe
,
so that
.

The next step in \textsf{RatAZ} is, for fixed~, to reduce~\eqref{eq:CT}
by the change of unknown , so as to determine all
 for which the differential equation in~

has a polynomial solution in .
For later use, we recall the following consequence
of~\cite[Corollary 9.6]{Gerhard2004}.

\begin{lemma}\label{le:polysol}
Let  be such that  is a
nonnegative integer and .
Let  be such that .
If ~is a polynomial solution of , then .
\end{lemma}
The following lemma generalises~\cite[Lemma~2]{Le2000}
to present a degree bound for~.



\begin{figure}
\framebox[8.4cm]{
\begin{minipage}{8.1cm}
\rule[.3cm]{0cm}{0cm}
{\rm {Algorithm \textsf{RatAZ}()}

\smallskip

\noindent \quad{\sc Input}:
 satisfying Hypothesis~(H).

\noindent \quad{\sc Output}:
A minimal telescoper  with certificate .
\begin{enumerate}
\item Compute , ,
and ,  primitive parts of ,  w.r.t.~,
respectively;
\item Set  to ;
\item For  do
\vspace{-0.2cm}
\begin{enumerate}
\item Set  to~, extract the linear system

from~\eqref{eq:pdga} (for~) and
compute a basis~ of the null space of~ by~\cite{Storjohann2005}.
\item If  contains a solution  such that  are not all
nonzero, then set , and go to Step~4;
\item Update , \ , \ , and .
\end{enumerate}
\item Compute the content~ of~ and return
.
\end{enumerate}
}
\end{minipage}}
\caption{Improved Almkvist--Zeilberger algorithm}
\label{fig:RatAZ}
\vskip-10pt
\end{figure}

\begin{lemma}\label{le:degb}
If  is a solution of~\eqref{eq:pdga} for , then\/  is bounded by .
\end{lemma}
\begin{proof}
Let  and .
By the definition of~,
.
Fact~\ref{prop:deflation}\emph{(i)\/} implies that
.
Therefore, .
As  and , \ .
The lemma holds by Lemma~\ref{le:polysol}.
\end{proof}

We end the present section using the approach of Almkvist and Zeilberger
to provide tight degree bounds on the outputs from Algorithms
\textsf{HermiteTelescoping} and \textsf{RatAZ}.

\begin{theorem}\label{th:AZtelesize}
Under Hypothesis~(H'), there exists a minimal telescoper  with certificate  with
 and .
\end{theorem}

\begin{proof}
By Corollary~\ref{cor:upperbound}, there exists a
smallest  at most , for which \eqref{eq:CT}~has a
solution with the 's not all zero.
For this , we estimate the size of the polynomial
matrix~ derived from~\eqref{eq:pdga} by undetermined
coefficients.
By the remark on~ after
Lemma~\ref{le:dGf-for-F'/F}, we have  where
 and
.
The matrix~ contains two blocks
 and , where  is the
same as in Lemma~\ref{le:degb}.
By the minimality of , the dimension of the null space of~
is~1. So there exists  with coefficients of
degree at most  in~
such that , which implies degree bounds in  for  and~.
The degree bound in  for  is
obvious.
\end{proof}

We now analyse the complexity of the algorithm in
Fig.~\ref{fig:RatAZ}.

\begin{theorem}
Under Hypothesis~(H'),
Algorithm~\textsf{RatAZ} in Figure~\ref{fig:RatAZ} is correct and takes
 ops, where  is the
order of the minimal telescoper.
\end{theorem}
\begin{proof}
By the existence of a telescoper, Corollary~\ref{cor:upperbound}, and
Lemma~\ref{le:degb}, the algorithm
always terminates and returns a minimal telescoper~,
of order~ at most~.
Gcd computations dominate the cost of Steps 1 and~2, which
take  ops.
For each , the dominating cost in Step~3 is
computing the null space of~.
Let  and . By the same argument as
in the proof of Theorem~\ref{th:AZtelesize}, the matrix  is of
size at most  and with coefficients of
degree at most~.
Let  be the rank of~, which is either  or
 by construction.
Thus, a basis of the null space of~ can
be computed within  ops
by Fact~\ref{le:polymatrix}\emph{(iii)}. Since ,  is included in
.
Since Step~3
terminates at , the total cost of the algorithm is
 ops.
This is within the announced complexity,
~ops.
\end{proof}

\begin{cor}
Algorithms \textsf{HermiteTelescoping} and \textsf{RatAZ} in
Fig.\ \ref{fig:HRTelescoping} and~\ref{fig:RatAZ} both output the
primitive minimal telescoper~ together with its certificate~,
which satisfy ,
\ , and
.
\end{cor}

\begin{proof}
Both algorithms output the primitive minimal telescoper, as they
compute a minimal telescoper at an intermediate step, and
owing to their last step of content removal.
Bounds follow from Corollary~\ref{cor:upperbound} and
Theorem~\ref{th:AZtelesize}.
\end{proof}

\end{subsection}

\end{section}

\begin{section}{Nonminimal telescopers}\label{sec:nonminimal-order}
Here, we discard Hypothesis~(H) and trade the minimality of telescopers for smaller total output sizes.
To this end, we adapt and slightly extend the arguments
in~\cite{Lipshitz1988} and~\cite[\S\,3]{BCLSS2007}.

\medskip
Given  of bidegree~, our goal is to
find a (possibly nonminimal) telescoper for~.
It is sufficient to
find a nonzero differential operator  that annihilates~.
Indeed, any  such that
 can be written , where  is
nonzero in  and .
If , then clearly  is a telescoper for ; otherwise,  yields

for some ,
which implies that  is again a telescoper for~. Moreover,
in both cases,  and .
Furthermore, for any , a direct calculation yields

where  and  and .  From these inequalities, we derive
the size and complexity estimates in Figure~\ref{fig:complexity} (bottom half), using
two different filtrations of .

\medskip\noindent{\bf Lipshitz's filtration (\cite{Lipshitz1988}).}
Let  be the -vector space
of dimension 
spanned by .
By~\eqref{eq:nonminimal},  is contained
in the vector space of dimension

spanned by
.
Choosing
 yields ;
therefore, there exists  in  with total degree at most
 in ,
, and  that annihilates~.
Moreover, ~is found by
linear algebra in dimension
.

\smallskip\noindent{\bf A better filtration (\cite{BCLSS2007}).}
Instead of taking total
degree, set  to the -vector space of dimension

generated by
.
By~\eqref{eq:nonminimal},   is contained
in the vector space of dimension
\hbox{ spanned by}


\noindent
.
Choosing  and  results in .
This implies the existence of  in  with total degree at most  in  and 
and degree at most  in~ that annihilates .
Again,  is found by linear algebra over~, but in smaller dimension
.

\end{section}

\begin{section}{Implementation and timings}\label{sec:implementation}

We implemented in Maple~13 all the algorithms described;
as we used Maple's generic solver \verb+SolveTools:-Linear+,
all of our implementations are deterministic.

The evaluation-interpolation algorithm \textsf{HermiteEvalInterp}
for Hermite reduction (Fig.~\ref{fig:HREvaInter}) does not
perform well, mainly because Maple's rational interpolation routines
are far too slow.
We thus implemented Algorithm~\textsf{HermiteReduce} (original
version) in~\cite[\S\,2.2]{BronsteinBook} (carefully avoiding
redundant extended gcd calculations), and noted that it performs
better.

We then implemented a variant of Algorithm \textsf{HermiteTelescoping}
in Figure~\ref{fig:HRTelescoping}, using \textsf{HermiteReduce} in
place of \textsf{HermiteEvalInterp}, and including the optimisation at
the end of \S\,\ref{sec:compl-estim},
refined by additional modular calculations.

For a rational function,
Algorithm \textsf{HermiteTelescoping} returns the minimal telescoper~ and the certificate~.
The algorithm separates the computation for~
from that for~.  Indeed,  is formed by
the coefficients of~,~, the  and their derivatives given in Figure~\ref{fig:HRTelescoping}.
This feature enables us to either return the certificate~ as a sum of unnormalised rational functions,
or a normalised rational function.

A selection of timings by this implementation and others are given in
Table~\ref{tab:random};
our code, the full table, as well as the random inputs
are given in~\cite{OurSoft}.
For our experiments, we exhaustively
considered all 49 bidegree patterns in factorisations of denominators
 () that add up to bidegree~(5,5), and
generated corresponding random denominators, imposing the
integers of the expanded forms to have around 26 digits.
Numerators were generated as random bidegree-(5,5) polynomials with
coefficients of 26 digits.

\begin{table}
\begin{scriptsize}
\tabcolsep2pt
\begin{center}
\begin{tabular}{r|rrrrrrrr}
No.& \tt AZ & \tt Abr & \tt RAZ & \tt H1 & \tt H2 & \tt HO & \tt EI & \tt MG \\
\hline
29 & 44 & 72 & 32 & 28 & 36 & 20 & 608 & 528 \\
43 & 52 & 76 & 36 & 20 & 24 & 32 & 652 & 584 \\
46 & 4268 & 1436 & 784 & 492 & 1288 & 752 & 343413 & 18945 \\
49 & 474269 & 34694 & 20977 & 10336 & 36254 & 22417 &  & 652968
\end{tabular}
\end{center}
\end{scriptsize}
\vskip-12pt
\caption{Creative telescoping on random instances}\label{tab:random}
\begin{small}
Timings in ms for algorithms in Table~\ref{tab:algos} (stopped after 30~min).
\end{small}
\vskip-10pt
\end{table}


\smallskip\noindent {\bf Application to diagonals.}
The diagonal of a formal power series  in  is defined to be the power series .
For a D-finite power series~, it is known to be
D-finite~\cite{Lipshitz1988}, and it is even algebraic
for a bivariate rational function ~\cite[\S\,6.3]{Stanley1999}.
A linear differential operator  that
annihilates~ can then be computed via rational-function
telescoping, owing to the following classical lemma from~\cite{Lipshitz1988}.

\vspace{-0.2cm}
\begin{lemma}
Any telescoper for  annihilates .
\end{lemma}
By this lemma, it suffices to compute a telescoper without its certificate to get an annihilator.
Algorithm \textsf{HermiteTelescoping} is suitable for
this task, since it separates computation of telescopers and
certificates.
Alternatively, for , we can compute an annihilator of
 either as the differential resolvent of the resultant
, or simply \emph{guess\/} it from the
first terms of the series expansion of~.

We compare the various algorithms
on an example borrowed from~\cite{Flaxman-2004-SMM}
(timings of execution are given in Table~\ref{tab:diags}):



All computer calculations have been performed on a Quad-Core Intel
Xeon X5482 processor at 3.20GHz, with 3GB of RAM, using up to 6.5GB of
memory allocated by Maple.

\begin{table}
\begin{scriptsize}
\tabcolsep2pt
\begin{center}
\begin{tabular}{r|rrrrrrrrrr}
 & \tt AZ & \tt Abr & \tt RAZ & \tt H1 & \tt H2 & \tt HO & \tt RR & \tt GHP \\
\hline
4 & 176 & 136 & 100 & 116 & 208 & 108 & 220 & 956 \\
8 & 3032 & 4244 & 4380 & 1976 & 5344 & 4396 & 10336 & 154409 \\
10 & 11740 & 12816 & 7108 & 7448 & 24565 & 7076 & 46882 & 1118313 \\
\hline
4 & 184 & 168 & 120 & 120 & 220 & 116 & 224 & 1340 \\
8 & 3540 & 3704 & 2540 & 2092 & 6976 & 2516 & 10348 & 271480 \\
10 & 16817 & 17013 & 9200 & 8068 & 32218 & 9092 & 46750 & 
\end{tabular}
\end{center}
\end{scriptsize}
\vskip-12pt
\caption{Computation of the diagonals of~\eqref{eq:diags}}\label{tab:diags}
\begin{small}
Timings in ms by creative telescoping of  (upper half) or
 (second half).  Algorithms listed in Table~\ref{tab:algos}.
\end{small}
\end{table}

\begin{table}
\vskip-3pt
\begin{scriptsize}
\begin{list}{}{\itemsep-3pt}
\item[\tt AZ] \verb+DETools[Zeilberger]+
\item[\tt Abr] \verb+AZ+ with Abramov's denominator bound by option \verb+gosper_free+
\item[\tt RAZ] Algorithm \textsf{RatAZ} of Fig.~\ref{fig:RatAZ}, with
  lower-bound prediction
\item[\tt H1] our Hermite-based approach, without certificate normalisation
\item[\tt H2] \verb+H1+, but with normalised certificate
\item[\tt HO] \textsf{RAZ}, solving~\eqref{eq:CT} by Horowitz--Ostrogradsky
\item[\tt EI] \verb+H1+ with evaluation and interpolation for calculations over~
\item[\tt MG] \verb+Mgfun+'s creative telescoping for general D-finite functions
\item[\tt RR] telescoper computation by resultant and differential resolvent
\item[\tt GHP] telescoper guessing by diagonal expansion and
  Hermite--Pad\'e
\end{list}
\end{scriptsize}
\vskip-15pt
\caption{List of the algorithms for the experiments}\label{tab:algos}
\vskip-10pt
\end{table}

\end{section}


{\scriptsize


\bibliographystyle{abbrv}
\begin{thebibliography}{10}

\bibitem{Almkvist1990}
G.~Almkvist and D.~Zeilberger.
\newblock The method of differentiating under the integral sign.
\newblock {\em J. Symb. Comput.}, 10:571--591, 1990.

\bibitem{Apagodu2006}
M.~Apagodu and D.~Zeilberger.
\newblock Multi-variable {Z}eilberger and {A}lmkvist-{Z}eilberger algorithms
  and the sharpening of {W}ilf-{} {Z}eilberger theory.
\newblock {\em Adv. in Appl. Math.}, 37(2):139--152, 2006.

\bibitem{BCLSS2007}
A.~Bostan, F.~Chyzak, B.~Salvy, G.~Lecerf, and {\'E}.~Schost.
\newblock Differential equations for algebraic functions.
\newblock In {\em I{SSAC}'07}, pages 25--32. ACM, New York, 2007.

\bibitem{BronsteinBook}
M.~Bronstein.
\newblock {\em {S}ymbolic Integration {I}: {T}ranscendental functions},
  volume~1 of {\em Algorithms and Computation in Mathematics}.
\newblock Springer-Verlag, Berlin, second edition, 2005.

\bibitem{Flaxman-2004-SMM}
A.~Flaxman, A.~W. Harrow, and G.~B. Sorkin.
\newblock Strings with maximally many distinct subsequences and substrings.
\newblock {\em Electron. J. Combin.}, 11(1):R8, 10 pp., 2004.

\bibitem{MCA2003}
J.~\gathen{von zur} Gathen and J.~Gerhard.
\newblock {\em Modern Computer Algebra}.
\newblock Cambridge University Press, Cambridge, second edition, 2003.

\bibitem{GeddesLe2002}
K.~O. Geddes and H.~Q. Le.
\newblock An algorithm to compute the minimal telescopers for rational
  functions (differential-integral case).
\newblock In {\em Mathematical Software}, pages 453--463. WSP, 2002.

\bibitem{Gerhard2001}
J.~Gerhard.
\newblock Fast modular algorithms for squarefree factorization and {H}ermite
  integration.
\newblock {\em Appl. Algebra Engrg. Comm. Comput.}, 11(3):203--226, 2001.

\bibitem{Gerhard2004}
J.~Gerhard.
\newblock {\em Modular Algorithms in Symbolic Summation and Symbolic
  Integration (LNCS)}.
\newblock SpringerVerlag, 2004.

\bibitem{Hermite1872}
C.~Hermite.
\newblock Sur l'int\'egration des fractions rationnelles.
\newblock {\em Ann. Sci. \'Ecole Norm. Sup. (2)}, 1:215--218, 1872.

\bibitem{Horowitz1971}
E.~Horowitz.
\newblock Algorithms for partial fraction decomposition and rational function
  integration.
\newblock In {\em SYMSAC'71}, pages 441--457, New York, USA, 1971. ACM.

\bibitem{Le2000}
H.~Q. Le.
\newblock On the differential-integral analogue of {Z}eilberger's algorithm to
  rational functions.
\newblock In {\em Proc.\ of the 2000 Asian Symposium on Computer Mathematics},
  pages 204--213, 2000.

\bibitem{Lipshitz1988}
L.~Lipshitz.
\newblock The diagonal of a {D}-finite power series is {D}-finite.
\newblock {\em J. Algebra}, 113(2):373--378, 1988.

\bibitem{Lipshitz1989}
L.~Lipshitz.
\newblock {D}-finite power series.
\newblock {\em J. Algebra}, 122(2):353--373, 1989.

\bibitem{Ostrogradsky1845}
M.~Ostrogradsky.
\newblock De l'int{\'e}gration des fractions rationnelles.
\newblock {\em Bull.\ de la classe physico-math{\'e}matique de l'Acad.\
  Imp{\'e}riale des Sciences de Saint-P{\'e}tersbourg}, 4:145--167, 286--300,
  1845.

\bibitem{Risch1969}
R.~H. Risch.
\newblock The problem of integration in finite terms.
\newblock {\em Trans. Amer. Math. Soc.}, 139:167--189, 1969.

\bibitem{Risch1970}
R.~H. Risch.
\newblock The solution of the problem of integration in finite terms.
\newblock {\em Bull. Amer. Math. Soc.}, 76:605--608, 1970.

\bibitem{Stanley1999}
R.~P. Stanley.
\newblock {\em Enumerative Combinatorics. {V}ol. 2}, volume~62 of {\em
  Cambridge Studies in Advanced Mathematics}.
\newblock CUP, 1999.

\bibitem{Storjohann2005}
A.~Storjohann and G.~Villard.
\newblock Computing the rank and a small nullspace basis of a polynomial
  matrix.
\newblock In {\em I{SSAC}'05}, pages 309--316. ACM, New York, 2005.

\bibitem{Zeilberger1990}
D.~Zeilberger.
\newblock A holonomic systems approach to special functions identities.
\newblock {\em J. Comput. Appl. Math.}, 32:321--368, 1990.

\bibitem{Zeilberger1991}
D.~Zeilberger.
\newblock The method of creative telescoping.
\newblock {\em J. Symbolic Comput.}, 11(3):195--204, 1991.

\bibitem{OurSoft}
\url{http://algo.inria.fr/chen/BivRatCT/}, 2010.

\end{thebibliography}
}


\end{document}
