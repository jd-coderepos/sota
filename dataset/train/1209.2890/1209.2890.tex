\documentclass{LMCS}
\pdfoutput=1

\def\doi{8(3:3)2012}
\lmcsheading {\doi}
{1--44}
{}
{}
{Feb.~29, 2012}
{Oct.~10, 2012}
{}
 


\usepackage{amssymb}
\usepackage{amsmath}
\usepackage{stmaryrd}
\usepackage{bbm}
\usepackage{amsfonts}
\usepackage{amsthm}
\usepackage{enumerate}
\usepackage{hyperref}
\usepackage{upgreek} 
\usepackage{cmll}
\usepackage{float}
  \floatstyle{boxed} \restylefloat{figure}
\usepackage{subfigure}\usepackage{proof}
\usepackage{tikz}
\usepackage[all]{xy}
\usepackage{enumerate,hyperref}
\newcommand{\rR}{\ \texttt{R}\ }
\newcommand{\rel}[1]{\ \texttt{#1}\ }
\newcommand{\nak}[1]{\tilde{#1}}
\newcommand{\ls}[2]{\langle #2 / #1\rangle} \newcommand{\cs}[2]{\{ #2 / #1\}} \newcommand{\Bag}[1]{[#1]} \newcommand{\Proj}[1]{\pi_{#1}}
\newcommand{\eval}{\mathrm{ev}}
\newcommand{\restr}{\hspace{-4pt}\upharpoonright}\newcommand{\dom}{\mathrm{dom}}
\newcommand{\curry}{\Uplambda}
\newcommand{\size}[1]{\mathrm{size}(#1)}
\newcommand\msize[1]{\mathrm{size_m}(#1)}
\newcommand{\mge}{>_{\mathrm{m}}}
\newcommand{\nat}{\mathbf{N}}
\newcommand{\natp}{\nat^+}
\newcommand{\one}{\mathbf{1}}
\newcommand{\bool}{\mathbf{2}}
\newcommand{\st}{ \mid }
\newcommand{\perm}[1]{\mathfrak{S}_{#1}}
\newcommand{\card}[1]{\# #1}
\newcommand{\bag}[1]{[#1]}
\newcommand{\Omegatuple}[1]{\Mfin{#1}^{(\omega)}}
\newcommand{\MRel}{\bold{M\hspace{-1pt}Rel}}
\newcommand{\Pow}[1]{\cP(#1)}
\newcommand{\Powf}[1]{\cP_{\mathrm{f}}(#1)}
\newcommand{\Id}[1]{\mathrm{Id}_{#1}}
\newcommand{\comp}{\circ}
\newcommand{\With}[2]{{#1}\with{#2}}
\newcommand{\Termobj}{\mathbbm{1}}
\newcommand{\App}{\mathrm{Ap}}
\newcommand{\Abs}{\uplambda}
\newcommand{\Funint}[2]{[{#1}\!\imp\!{#2}]}

\newcommand{\full}{\gto{\bang}}
\newcommand{\dlam}{\ensuremath{\partial\lambda}}
\newcommand{\dzlam}{\ensuremath{\partial_0\lambda}}
\newcommand{\lam}{\ensuremath{\lambda}}
\newcommand{\bang}{\oc}
\newcommand{\Var}{\mathrm{Var}}
\renewcommand{\hole}[1]{\llparenthesis #1\rrparenthesis}
\newcommand{\paral}{\vert}
\newcommand{\FV}{\mathrm{FV}}
\newcommand{\Set}[1]{\Lambda^{#1}}
\newcommand{\FSet}[1]{\Lambda^{#1}_{\bang}}
\newcommand{\ContSet}{\Set{\gt}_{\hole{\cdot}}}
\newcommand{\FContSet}{\Set{\gt\bang}_{\hole{\cdot}}}
\newcommand{\sums}[1]{\bool\langle\Set{#1}\rangle}
\newcommand{\Fsums}[1]{\bool\langle\FSet{#1}\rangle}
\newcommand{\msto}{\twoheadrightarrow}
\newcommand{\toh}{\to_{h}}\newcommand{\mstoh}{\msto_{h}}\newcommand{\lsubst}[2]{\langle #2 / #1 \rangle}	\newcommand{\llsubst}[2]{\langle\!\langle #2 / #1 \rangle\!\rangle}
\newcommand{\subst}[2]{\{ #2 / #1 \}}	\newcommand{\dg}[2]{\mathrm{deg}_{#1}(#2)} \newcommand{\obsle}{\sqsubseteq_{\mathcal{O}}}
\newcommand{\obseq}{\approx_{\mathcal{O}}}
\newcommand{\Tobsle}{\sqsubseteq^{\gt}_{\mathcal{O}}}
\newcommand{\Tobseq}{\approx^{\gt}_{\mathcal{O}}}
\newcommand{\Fobsle}{\sqsubseteq^{\gt\bang}_{\mathcal{O}}}
\newcommand{\Fobseq}{\approx^{\gt\bang}_{\mathcal{O}}}
\newcommand{\TE}[1]{#1^{\circ}} \newcommand{\at}{\!::\!}
\newcommand{\rank}{\mathrm{rk}}
\newcommand{\len}{\ell}
\newcommand{\Int}[1]{\llbracket #1\rrbracket} \newcommand{\trm}[1]{#1^{\textrm{--}}}
\newcommand{\cont}[2]{#1^{+}\hole{#2}}
\newcommand{\Mfin}[1]{\mathcal{M}_{\mathrm{f}}(#1)}
\newcommand{\mcup}{\uplus}
\newcommand{\mmcup}{\bar{\mcup}}
\newcommand{\Pair}[2]{\langle{#1},{#2}\rangle}

\newcommand{\gramm}{\mathrel{::=}}
\newcommand{\ass}{:=}
\renewcommand{\iff}{\Leftrightarrow}
\newcommand{\seq}[1]{\vec{#1}}
\newcommand{\imp}{\Rightarrow}
\newcommand{\Apex}[1]{^{\: #1}}

\newcommand{\sA}{\mathbb{A}}
\newcommand{\sB}{\mathbb{B}}
\newcommand{\sC}{\mathbb{C}}
\newcommand{\sD}{\mathbb{D}}
\newcommand{\sM}{\mathbb{M}}
\newcommand{\sN}{\mathbb{N}}
\newcommand{\sL}{\mathbb{L}}
\newcommand{\sH}{\mathbb{H}}
\newcommand{\sP}{\mathbb{P}}
\newcommand{\sQ}{\mathbb{Q}}
\newcommand{\sR}{\mathbb{R}}
\newcommand{\sV}{\mathbb{V}}
\newcommand{\sW}{\mathbb{W}}


\newcommand{\bA}{\mathbf{A}}
\newcommand{\bB}{\mathbf{B}}
\newcommand{\bC}{\mathbf{C}}
\newcommand{\bD}{\mathbf{D}}
\newcommand{\bE}{\mathbf{E}}
\newcommand{\bF}{\mathbf{F}}
\newcommand{\bG}{\mathbf{G}}
\newcommand{\bH}{\mathbf{H}}
\newcommand{\bI}{\mathbf{I}}
\newcommand{\bJ}{\mathbf{J}}
\newcommand{\bK}{\mathbf{K}}
\newcommand{\bL}{\mathbf{L}}
\newcommand{\bM}{\mathbf{M}}
\newcommand{\bN}{\mathbf{N}}
\newcommand{\bO}{\mathbf{O}}
\newcommand{\bP}{\mathbf{P}}
\newcommand{\bQ}{\mathbf{Q}}
\newcommand{\bR}{\mathbf{R}}
\newcommand{\bS}{\mathbf{S}}
\newcommand{\bT}{\mathbf{T}}
\newcommand{\bU}{\mathbf{U}}
\newcommand{\bV}{\mathbf{V}}
\newcommand{\bW}{\mathbf{W}}
\newcommand{\bX}{\mathbf{X}}
\newcommand{\bY}{\mathbf{Y}}
\newcommand{\bZ}{\mathbf{Z}}

\newcommand{\mbbA}{\mathbb{A}}
\newcommand{\mbbB}{\mathbb{B}}
\newcommand{\mbbC}{\mathbb{C}}
\newcommand{\mbbD}{\mathbb{D}}
\newcommand{\mbbE}{\mathbb{E}}
\newcommand{\mbbF}{\mathbb{F}}
\newcommand{\mbbG}{\mathbb{G}}
\newcommand{\mbbH}{\mathbb{H}}
\newcommand{\mbbI}{\mathbb{I}}
\newcommand{\mbbL}{\mathbb{L}}
\newcommand{\mbbM}{\mathbb{M}}
\newcommand{\mbbN}{\mathbb{N}}
\newcommand{\mbbW}{\mathbb{W}}
\newcommand{\mbbY}{\mathbb{Y}}
\newcommand{\mbbX}{\mathbb{X}}
\newcommand{\mbbZ}{\mathbb{Z}}

\newcommand{\ga}{\alpha}
\newcommand{\gb}{\beta}
\newcommand{\gc}{\gamma}
\newcommand{\gd}{\delta}
\newcommand{\gep}{\varepsilon}
\newcommand{\gz}{\zeta}
\newcommand{\geta}{\eta}
\newcommand{\gth}{\theta}
\newcommand{\gi}{\iota}
\newcommand{\gv}{\nu}
\newcommand{\gk}{\kappa}
\newcommand{\gl}{\lambda}
\newcommand{\gm}{\mu}
\newcommand{\gn}{\nu}
\newcommand{\gx}{\xi}
\newcommand{\gp}{\pi}
\newcommand{\gr}{\rho}
\newcommand{\gs}{\sigma}
\newcommand{\gt}{\ensuremath{\tau}}
\newcommand{\gu}{\upsilon}
\newcommand{\gph}{\varphi}
\newcommand{\gch}{\chi}
\newcommand{\gps}{\psi}
\newcommand{\go}{\omega}
\newcommand{\gto}{\ensuremath{\bar\tau}}

\newcommand{\gG}{\Gamma}
\newcommand{\gF}{\Phi}
\newcommand{\gD}{\Delta}
\newcommand{\gT}{\Theta}
\newcommand{\gP}{\Pi}
\newcommand{\gX}{\Xi}
\newcommand{\gS}{\Sigma}
\newcommand{\gO}{\Omega}
\newcommand{\gL}{\Lambda}

\newcommand\rA{{\mathrm{A}}}
\newcommand\rB{{\mathrm{B}}}
\newcommand\rC{{\mathrm{C}}}
\newcommand\rD{{\mathrm{D}}}
\newcommand\rE{{\mathrm{E}}}
\newcommand{\rF}{\mathrm{F}}
\newcommand\rG{{\mathrm{G}}}
\newcommand\rH{{\mathrm{H}}}
\newcommand\rI{{\mathrm{I}}}
\newcommand\rL{{\mathrm{L}}}
 
\newcommand\rb{{\mathrm{b}}}
\newcommand\rc{{\mathrm{c}}}
\newcommand\rd{{\mathrm{d}}}
\newcommand\re{{\mathrm{e}}}
\newcommand{\rf}{\mathrm{f}}
\newcommand\rg{{\mathrm{g}}}
\newcommand\rh{{\mathrm{h}}}
\newcommand\ri{{\mathrm{i}}}
\newcommand\rl{{\mathrm{l}}}
\newcommand\mrm{{\mathrm{m}}}
\newcommand\rn{{\mathrm{n}}}
\newcommand\ro{{\mathrm{o}}}
\newcommand\rp{{\mathrm{p}}}
\newcommand\rr{{\mathrm{r}}}
\newcommand\rs{{\mathrm{s}}}
\newcommand\rt{{\mathrm{t}}}
\newcommand\lab{\mathrm{lab}}

\newcommand{\cA}{\mathcal{A}}
\newcommand{\cB}{\mathcal{B}}
\newcommand{\scC}{\mathcal{C}}
\newcommand{\cD}{\mathcal{D}}
\newcommand{\cE}{\mathcal{E}}
\newcommand{\cF}{\mathcal{F}}
\newcommand{\cG}{\mathcal{G}}
\newcommand{\cH}{\mathcal{H}}
\newcommand{\cI}{\mathcal{I}}
\newcommand{\cJ}{\mathcal{J}}
\newcommand{\cK}{\mathcal{K}}
\newcommand{\cL}{\mathcal{L}}
\newcommand{\cM}{\mathcal{M}}
\newcommand{\cN}{\mathcal{N}}
\newcommand{\cO}{\mathcal{O}}
\newcommand{\cP}{\mathcal{P}}
\newcommand{\cQ}{\mathcal{Q}}
\newcommand{\cR}{\mathcal{R}}
\newcommand{\cS}{\mathcal{S}}
\newcommand{\cT}{\mathcal{T}}
\newcommand{\cU}{\mathcal{U}}
\newcommand{\cV}{\mathcal{V}}
\newcommand{\cW}{\mathcal{W}}
\newcommand{\cX}{\mathcal{X}}
\newcommand{\cY}{\mathcal{Y}}
\newcommand{\cZ}{\mathcal{Z}}

\newcommand{\pts}{.\,.\,}
\newcommand{\cOnv}[1]{#1\!\downarrow} \renewcommand{\div}[1]{#1\!\uparrow} \newcommand{\module}[1]{\bool\langle #1 \rangle}
 
\def\eg{{\em e.g.}}
\def\cf{{\em cf.}}

\begin{document}

\title[Full Abstraction for Resource Calculi]{Full Abstraction for the Resource Lambda Calculus with Tests, through Taylor Expansion\rsuper*}

\author[A.~Bucciarelli]{Antonio Bucciarelli\rsuper a}	\address{{\lsuper a}Univ Paris Diderot, Sorbonne Paris Cit\'e, PPS, UMR 7126, CNRS, F-75205 Paris, France}	\email{antonio.bucciarelli@pps.jussieu.fr}  

\author[A.~Carrano]{Alberto Carraro\rsuper b}	\address{{\lsuper b}Universit\`a Ca'Foscari,  Via Torino 155, 30172 Mestre, Venice, Italia}	\email{acarraro@dsi.unive.it}  

\author[T.~Ehrhard]{Thomas Ehrhard\rsuper c}	\address{{\lsuper c}CNRS, PPS, UMR 7126, Univ Paris Diderot, Sorbonne Paris Cit\'e, F-75205 Paris, France}	\email{thomas.ehrhard@pps.jussieu.fr}  

\author[G.~Manzonetto]{Giulio Manzonetto\rsuper d}	\address{{\lsuper d}Univ Paris 13, Sorbonne Paris Cit\'e, LIPN, UMR 7030, CNRS, F-93430 Villetaneuse, France}	\email{giulio.manzonetto@lipn.univ-paris13.fr}  



\keywords{Resource lambda calculus, relational semantics, full abstraction, differential linear logic}
\subjclass{F.4.1}
\titlecomment{{\lsuper*}This paper is an extended version of ``Full Abstraction for Resource Calculus with Tests'' appeared in the proceedings of the 20\textsuperscript{th} Conference on Computer Science Logic (CSL'11) \cite{BucciarelliCEM11}.
Work partially supported by the international ANR-NSFC Project \emph{Locali}.}




\begin{abstract}
 \noindent  We study the semantics of a resource-sensitive extension of the lambda calculus in a canonical reflexive object of a category of sets and relations, a relational version of Scott's
original model of the pure lambda calculus. This calculus is related to Boudol's resource calculus and is derived from Ehrhard and Regnier's differential extension of Linear Logic and of the lambda calculus.  We extend it with new constructions, to be understood as implementing a very simple exception mechanism, and with a ``must'' parallel composition. These new operations allow to associate a context of this calculus with any point of the model and to prove full abstraction for the finite sub-calculus where ordinary lambda calculus application is not allowed. The result is then extended to the full calculus by means of a Taylor Expansion formula.
As an intermediate result we prove that the exception mechanism is not essential in the finite sub-calculus. \end{abstract}

\maketitle


\section{Introduction}


In concurrent calculi like CCS \cite{Milner80}, guarded processes are resources
that can be used only once by other processes. This fundamental 
linearity of resources leads naturally to non-determinism,
since several agents  (senders and receivers) 
can interact on the same channel. In general, 
various synchronization scenarios are possible, giving  rise to 
different behaviours. On the other hand in the -calculus \cite{Bare}, a
function (receiver) can duplicate its argument (sender) arbitrarily. 
Thanks to this asymmetry, the -calculus enjoys a
strong determinism (namely, the Church-Rosser Theorem), but for the same reason it 
lacks any form of control on resource handling. 
\medskip

{\bf Resource Lambda Calculi.} Resource -calculi stem from an attempt to combine 
the functionality of the -calculus and 
the resource-sensitivity of process calculi.
Boudol has been the first to design a resource-conscious functional programming language, 
the \emph{resource -calculus}  \cite{Boudol93}, extending the usual one 
along two directions. 
First, a function is not necessarily applied to a single argument, but can also be
applied to a multiset of arguments called \emph{resources}.
Second, a resource can be either linear (it must be used exactly once) or reusable (it can be used \emph{ad libitum}). 
In this context, the evaluation of a function applied to a multiset of resources gives rise to several possible choices, 
corresponding to the different possibilities of distributing the resources in the multiset among the occurrences of the formal parameter.
From the viewpoint of concurrent programming, this  was a natural step to take since one
of the main features of this programming setting is the consumption of
resources which cannot be copied. Milner's
-calculus \cite{Milner93} features this phenomenon in great generality, and Boudol's
calculus keeps track of it in a functional setting.

Together with Regnier, Ehrhard observed that this idea of resource consumption can be
understood as resulting from a \emph{differential} extension of
-calculus (and of Linear Logic)~\cite{EhrhardR03}. 
Instead of considering two kinds of resources, they defined two kinds of applications: the \emph{ordinary} application and a \emph{linear} one.
In a simply typed setting, linear application of a term  to a multiset made of  terms
, combined with ordinary application to a term ,
corresponds to computing , where  is the
-th derivative of  which is of type  and associates a
symmetric -linear map with any element of . The symmetry of this
multilinear map corresponds to Schwarz's Theorem and
is implemented in the resource -calculus by the use of multisets for
representing linear applications. 
A notable advantage of this approach is that it allows to apply powerful methods from differential calculus in the context of \lam-calculus.
For instance, iterated differentiation yields very naturally a Taylor expansion formula, which consists in expanding the ordinary application into
several linear applications of the differential \lam-calculus. 
More precisely, if  and  are \lam-terms, then the Taylor expansion of  is given by

in analogy with the standard Taylor formula of the entire functions.
The Taylor expansion has been studied in \cite{EhrhardR08} where the authors relate it to the B\"ohm tree of a \lam-term, giving the intuition that the former is a resource conscious improvement of the latter.

The main difference between Boudol's resource -calculus and Ehrhard and Regnier's
differential -calculus is that the first is lazy --- this means that in many cases linear substitutions must be delayed. 
To that effect, the calculus features a linear explicit substitution mechanism. 
Moreover, it implements a fixed reduction strategy similar to linear head reduction.
Therefore, Boudol's calculus is not an extension of the ordinary 
\lam-calculus.
Also, the resource \lam-calculus is rather affine than linear, since depletable resources cannot be duplicated but can be erased.
Another difference lies in the respective origins of these calculi: 
the resource \lam-calculus originates from syntactical considerations related to the theory 
of concurrent processes, while the differential one arises from denotational models of linear logic where the existence of differential
operations has been observed. These models are based on the well-known relational model
of Linear Logic \cite{Girard88}, and the interpretation of the new differential constructions
is as natural and simple as the interpretation of the ordinary Linear Logic
constructions. 

In this paper we work with a resource-sensitive \lam-calculus because our techniques depend on the linear logic structure underlying the calculus and on the presence
 of a Taylor expansion formula. Two main syntaxes have been proposed for the differential -calculus: Ehrhard and Regnier's original one \cite{EhrhardR03},
 simplified by Vaux in~\cite{Vaux07}, and Tranquilli's \emph{resource calculus} of \cite{Tranquilli10} whose syntax  is close to Boudol's one.
 These calculi share a common semantical backbone as well as similar connections with differential Linear Logic and proof nets. 
We adopt roughly Tranquilli's syntax and call our calculus  \emph{\dlam-calculus}. To avoid the problem of handling the coefficients introduced by the Taylor formula 
 we conveniently suppose that the formal sum in the calculus is idempotent; this amounts to saying that we only check whether a term appears in a result, not how many
 times it appears. This is very reasonable when studying convergency properties since  converges exactly when  does.


\medskip
{\bf Full Abstraction.} 
A natural problem when a new calculus is introduced is to characterize when two programs are operationally equivalent, 
namely when one can be replaced by the other in every context without noticing any difference with respect to a given observational equivalence. 
In this paper we prove a full abstraction result (a semantical characterization of operational equivalence)
for the \dlam-calculus in the spirit of~\cite{BoudolCL99}. As in that paper, we extend the 
language with a convergence testing mechanism. 
Implicitly, this extension already appears in~\cite{EhrhardL10}, in a differential linear logic setting: it corresponds to the
-ary tensor and par cells. To implement the corresponding extension of the 
-calculus, we
introduce two sorts of expressions: the \emph{terms} (variable, application,
abstraction, ``throw''  where   is a test) and the \emph{tests}
(empty test, parallel composition of tests and 
``catch''  where   is a term).
Parallel composition allows to combine tests in such a way that
the combination succeeds if and only if each test succeeds. 
Outcomes of tests (convergence or divergence) are the only observations
allowed in our calculus, and the corresponding contextual equivalence and preorder on terms
constitute our main object of study.


This extended \dlam-calculus, that we call \emph{\dlam-calculus with tests}, has a natural denotational
interpretation in a model of the pure -calculus introduced by
Bucciarelli, Ehrhard and Manzonetto in~\cite{BucciarelliEM07},
which is indeed a denotational model of the differential pure nets
of~\cite{EhrhardL10} as one can check easily. This model is a reflexive
object    in the Kleisli category of the linear logic model of sets and relations where 
is the set of all finite multisets over . An element of  can be described as a finite
tree which alternates two kinds of layers: \emph{multiplicative layers} where
subtrees are indexed by natural numbers and \emph{exponential layers} where
subtrees are organized as non-empty multisets. To be more precise, 
(negative) pairs of layers alternate with  (positive) pairs,
respecting a strict polarity discipline very much in the spirit of
Ludics~\cite{Girard03}. The empty positive multiplicative tree corresponds to the
empty tensor cell and the negative one to the empty par cell.  The
corresponding constructions ,  are therefore quite easy to
interpret.


We use this logical interpretation to turn the elements of  into 
\dlam-calculus terms with tests. 
More precisely, with each element  of , we
associate a test  with a hole  for a term, and we show that
 belongs to the interpretation of a (closed) term  iff the test
 converges. From this fact, we derive  a full abstraction
result for the fragment of the \dlam-calculus with tests in
which all ordinary applications are trivial, that we call \dzlam-calculus with tests.
To extend this
result to the \dlam-calculus with tests, we
use the Taylor formula introduced in~\cite{EhrhardR03} which
allows to turn any ordinary application into a sum of infinitely many linear
applications of all possible arities. One exploits then the fact that the
Taylor formula holds in the model, as well as a simulation lemma
which relates the head reduction of a term with the head reduction of its
Taylor expansion.

\medskip \textbf{Contributions.}
In Section~\ref{sec:thomas} we provide the abstract categorical framework which is 
needed to interpret the -calculus and its extension with tests.
The syntax and operational semantics of the -calculus with tests (which is the promotion-free fragment)
are presented in Section~\ref{sec:dzlam-calculus+tests}, while its relational model  is described concretely in Section~\ref{sec:rela-sem}.
The definability of the elements of  in the \dzlam-calculus with tests is
the main conceptual contribution of this paper --- it shows that, in this setting, the standard syntax \emph{versus} semantics dichotomy is
essentially meaningless.  
From definability it follows easily that the relational model is fully abstract for the -calculus with tests, as shown in Section~\ref{sec:FA-dzlam-with-Tests}.
This result is analyzed further in Section~\ref{sec:FA-notests-nobang}, where it is proved that in the absence of promotion the test operators do not add any discriminatory 
power to the contexts, thus showing that  is also fully abstract for the -calculus without tests. 

We then focus on the full -calculus with tests.
Section~\ref{sec:FullRC} is devoted to present its syntax, operational
semantics and relational semantics.  In Section~\ref{sec:TaylorExp} we
consider the use of Taylor expansions to reduce the full abstraction
problem for  to its ``\dzlam'' version, thus introducing an
original and promising reduction technique.
 


\renewcommand{\phi}{\varphi}
\renewcommand\epsilon{\varepsilon}
\renewcommand\sharp{\#}
\newcommand\Eqref[1]{(\ref{#1})}

\newcommand{\Iff}{\quad\hbox{iff}\quad}
\newcommand{\Implies}{\Rightarrow}
\newcommand\Equiv{\Leftrightarrow}
\newcommand{\St}{\mid}

\newcommand{\Ro}{\circ}
\newcommand{\Inf}{\bigwedge}
\newcommand{\Infi}{\wedge}
\newcommand{\Sup}{\bigvee}
\newcommand{\Supi}{\vee}


\newcommand{\arrow}{\rightarrow}
\renewcommand{\Bot}{{\mathord{\perp}}}
\newcommand{\Top}{\top}

\newcommand\Seqempty{\langle\rangle}

\newcommand\Fini{{\mathrm{fin}}}



\def\frownsmile{\mathrel{\vbox{\hbox{}\vspace{-2ex}\hbox{}\vspace{-.5ex}}}}
\def\smilefrown{\mathrel{\vbox{\hbox{}\vspace{-2ex}\hbox{}\vspace{-.5ex}}}}

\newcommand\Part[1]{{\cal P}({#1})}

\newcommand\Union{\bigcup}

\newcommand{\Linarrow}{\multimap}

\def\frownsmile{\mathrel{\vbox{\hbox{}\vspace{-2ex}\hbox{}\vspace{-.5ex}}}}
\def\smilefrown{\mathrel{\vbox{\hbox{}\vspace{-2ex}\hbox{}\vspace{-.5ex}}}}

\newcommand\CScoh[3]{{#2}\mathrel{\frownsmile_{{#1}}}{#3}}
\newcommand\CScohs[3]{{#2}\mathrel{{\frown}_{#1}}{#3}}
\newcommand\CScohstr[3]{\CScohs{#1}{#2}{#3}}
\newcommand\CSincoh[3]{{#2}\mathrel{\smilefrown_{{#1}}}{#3}}
\newcommand\CSincohs[3]{{#2}\mathrel{{\smile}_{#1}}{#3}}
\newcommand\CSeq[3]{{#2}\mathrel{{=}_{#1}}{#3}}

\newcommand\Myleft{}
\newcommand\Myright{}

\newcommand\Web[1]{\Myleft|{#1}\Myright|}
\newcommand\Supp[1]{\Myleft|{#1}\Myright|}

\newcommand\Emptymset{[]}
\newcommand\Mset[1]{[{#1}]}
\newcommand\Sset[1]{\langle{#1}\rangle}


\newcommand\MonP{P}
\newcommand\MonPZ{\MonP_0}

\newcommand\Cl[1]{\mbox{\textrm{Cl}}({#1})}
\newcommand\ClP[1]{\mbox{\textrm{Cl}}_{\MonP}({#1})}

\newcommand\Star{\star}
\newcommand\CohName{\mbox{\textbf{Coh}}}
\newcommand\COH[1]{\CohName(#1)}
\newcommand\COHR[2]{\CohName(#1,#2)}

\newcommand\Par[2]{{#1}\parr{#2}}
\newcommand\ITens{\mathrel\otimes}
\newcommand\Tens[2]{{#1}\ITens{#2}}
\newcommand\Tensp[2]{({#1}\ITens{#2})}
\newcommand\Tensexp[2]{{#1}^{\mathord\otimes{#2}}}
\newcommand\IWith{\mathrel{\&}}
\newcommand\BWith{\bigwith}
\newcommand\IPlus{\oplus}
\newcommand\Plus[2]{{#1}\IPlus{#2}}
\newcommand\ILfun{\multimap}
\newcommand\Lfun[2]{{#1}\ILfun{#2}}
\newcommand\Orth[1]{{#1}^\Bot}
\newcommand\Orthr[2]{{#1}^{\Bot(#2)}}
\newcommand\Biorthr[3]{{#1}^{\Bot(#2)\Bot(#3)}}
\newcommand\Triorthr[4]{{#1}^{\Bot(#2)\Bot(#3)\Bot(#4)}}
\newcommand\Orthw[1]{{#1}^\Bot}


\newcommand\Scal[2]{\langle{#1}\mid{#2}\rangle}





\newcommand\Inj{\lambda}

\newcommand\GlobalIndex{I}
\newcommand\Index{\GlobalIndex}
\newcommand\Relbot{\Bot}
\newcommand\MonPZI{{\MonPZ}^\Index}
\newcommand\ClPI[1]{{\ClP{{#1}}}^\Index}
\newcommand\WebI[1]{\Web{#1}^\Index}


\newcommand\Scalb[2]{\Scal{\Bar{#1}}{\Bar{#2}}}

\newcommand\Ortho[2]{{#1}\mathrel{\bot}{#2}}
\newcommand\Orthob[2]{\Bar{#1}\mathrel{\bot}\Bar{#2}}

\newcommand\Biorth[1]{{#1}^{\Bot\Bot}}
\newcommand\Triorth[1]{{#1}^{\Bot\Bot\Bot}}

\newcommand\Relpretens[2]{\cR_{#1}\bar\ITens\cR_{#2}}
\newcommand\Relpreplus[2]{\cR_{#1}\bar\IPlus\cR_{#2}}



\newcommand\RelP[1]{\widetilde{#1}}

\newcommand\Eqw[2]{\delta({#1},{#2})}
\newcommand\Eqwb[2]{\Eqw{\Bar{#1}}{\Bar{#2}}}

\newcommand\PFacts[1]{\cF({#1})}


\newcommand\Facts{\cF(\MonPZI)}


\newcommand\RelL[1]{\overline{#1}}

\newcommand\PRel[1]{R_{#1}}

\newcommand\PFamb[2]{[\Bar{#1},\Bar{#2}]}

\newcommand\Fplus[2]{\Bar{#1}+\Bar{#2}}

\newcommand\Char[1]{\epsilon_{#1}}

\newcommand\Fproj[2]{\pi_{#1}(\Bar{#2})}

\newcommand\One{1}


\newcommand\Pbot[1]{\Bot_{#1}}
\newcommand\PBot[1]{\Bot_{#1}}
\newcommand\PRBot[1]{\Bot_{#1}}
\newcommand\PROne[1]{1_{#1}}

\newcommand\Pproj[1]{\pi_{#1}}

\newcommand\Zext[1]{\zeta_{\Bar{#1}}}
\newcommand\Aext[1]{\bar\zeta_{\Bar{#1}}}

\newcommand\Mall{\hbox{\textsf{MALL}}}
\newcommand\RMall{\Mall(\Index)}
\newcommand\RMallr[1]{\Mall({#1})}
\newcommand\FDom[1]{d({#1})}

\newcommand\RBot[1]{\Bot_{#1}}
\newcommand\ROne[1]{\One_{#1}}

\newcommand\Seq[1]{\vdash{#1}}
\newcommand\Seqpolr[1]{\vdash_\POLS{#1}}
\newcommand\Seqv[1]{\Seq{#1}}
\newcommand\Seqt[3]{\vdash_{\mathsf{t}}{#1}:{#2}\mid{#3}}
\newcommand\Seqn[2]{\vdash_{\mathsf{n}}{#1\mid#2}}
\newcommand\Seqnfree[2]{\vdash_{\mathsf{1}}{#1\mid#2}}
\newcommand\Seql[2]{\vdash_{\mathsf{0}}{#1\mid#2}}
\newcommand\Seqnp[3]{\vdash_{\mathsf{p}}{#1:#2\mid#3}}
\newcommand\Seqtfree[1]{\vdash_1{#1}}
\newcommand\Ccup{,}

\newcommand\RSeq[2]{\vdash_{#1}{#2}}

\newcommand\Restr[2]{{#1}|_{#2}}
\newcommand\FRestr[2]{{#1}|_{#2}}

\newcommand\FSem[1]{{#1}^{*}}
\newcommand\PSem[1]{{#1}^{*}}

\newcommand\FFamb[2]{{#1}\langle\Bar{#2}\rangle}

\newcommand\Premskip{\hskip1cm}

\newcommand\Forg[1]{\underline{#1}}

\newcommand\Phase[1]{{#1}^\bullet}

\newcommand\Punit[1]{1^{#1}}

\newcommand\Reg[1]{R_{#1}}

\newcommand\Cont[1]{{#1}^\circ}

\newcommand\Neutral{e}
\newcommand\RNeutral[1]{\Neutral_{#1}}

\newcommand\POne{1}

\newcommand\relstack[2]{\underset{#2}{#1}}
\newcommand\RPlus[4]{{#3}\relstack\IPlus{{#1},{#2}}{#4}}
\newcommand\RWith[4]{{#3}\relstack\IWith{{#1},{#2}}{#4}}

\newcommand\PReq[1]{=_{#1}}

\newcommand\Fam[2]{\hbox{\textrm{Fam}}_{#1}(#2)}
\newcommand\Famfunc[1]{\hbox{\textrm{Fam}}_{#1}}

\newcommand\Pval[1]{\rho_{#1}}

\newcommand\Fcoh{\mathsf C}
\newcommand\Fincoh{\overline\Fcoh}
\newcommand\Feq{\mathsf E}

\newcommand\RZero{\Zero}
\newcommand\RTop{\Top}
\newcommand\EmptyFam{\emptyset}
\newcommand\Partial[1]{\Omega_{#1}}

\newcommand\PMall{\hbox{\textsf{MALL}}_\Omega}
\newcommand\PRMall{\Mall_\Omega(\Index)}
\newcommand\Homo[1]{H(#1)}
\newcommand\RProd[2]{\Lambda_{#1}{({#2})}}
\newcommand\FProd[1]{\widetilde{#1}}

\newcommand\CltoPCl[1]{{#1}^P}

\newcommand\Partfin[1]{{\cP_\Fini}({#1})}




\newcommand\RLL{\LL(\Index)}
\newcommand\RLLP{\LL^+(\Index)}
\newcommand\RLLext{\LL^\mathrm{ext}(\Index)}

\newcommand\IExcl{{\mathord{!}}}


\newcommand\RExcl[1]{\IExcl_{#1}}
\newcommand\RInt[1]{\IInt_{#1}}


\newcommand\Fempty[1]{0_{#1}}

\newcommand\FAct[1]{{#1}_*}

\newcommand\Excl[1]{\IExcl{#1}}
\newcommand\Excls[1]{\IExcl_{\mathsf s}{#1}}


\newcommand\Noindex[1]{\langle{#1}\rangle}

\newcommand\Card[1]{\#{#1}}
\newcommand\Multi[1]{\mathsf m({#1})}

\newcommand\FamRestr[2]{{#1}|_{#2}}

\newcommand\Pact[1]{{#1}_*}

\newcommand\Pinj[1]{\zeta_{#1}}

\newcommand\Locun[1]{1^J}

\newcommand\Isom\simeq

\newcommand\FGraph[1]{\mathsf{g}({#1})}
\newcommand\GPFact[1]{\mathsf f_0({#1})}
\newcommand\GFact[1]{\mathsf f({#1})}

\newcommand\NUCS{non-uniform coherence space}
\newcommand\NUCSb{non-uniform coherence space }
\newcommand\NUCSs{non-uniform coherence spaces}
\newcommand\NUCSsb{non-uniform coherence spaces }
\newcommand\Nucs{\mathbf{nuCS}}

\newcommand\Comp{\mathrel\circ}

\newcommand\Funinv[1]{#1^{-1}}

\newcommand\Reindex[1]{{#1}^*}

\newcommand\Locbot[1]{\Bot({#1})}
\newcommand\Locone[1]{1({#1})}
\newcommand\LocBot{\Locbot}
\newcommand\LocOne{\Locone}

\newcommand\INJ[1]{\mathcal I({#1})}
\newcommand\COHP[1]{\mathcal C({#1})}
\newcommand\FuncFam[1]{\mathrm{Fam}_{#1}}
\newcommand\SET{\mathbf{Set}}
\newcommand\Locmod[2]{{#1}({#2})}
\newcommand\FuncPhase[1]{\mathcal{F}_{#1}}
\newcommand\Trans[2][]{{\widehat{#2}}_{#1}}
\newcommand\Useful[1]{\Web{#1}_{\mathrm u}}

\newcommand\Limpl[2]{{#1}\Linarrow{#2}}

\newcommand\ConstFam[2]{{#1}^{#2}}



\newcommand\Center[1]{\mathcal C({#1})}

\newcommand\Derel[1]{\mathrm{d}_{#1}}

\newcommand\Myid{\operatorname{Id}}

\newcommand\Mspace[2]{{#2}^*_{#1}}

\newcommand\CStoMS[1]{{#1}^+}
\newcommand\MStoCS[1]{{#1}^-}

\newcommand\PhaseCoh[1]{{\mathbf{Coh}}_{#1}}

\newcommand\Nat{{\mathbb{N}}}
\newcommand\Natnz{{\Nat^+}}

\newcommand\Rien{\Omega}

\newcommand\DenEq[2]{{#1}\sim{#2}}

\newcommand\VMon[2]{\langle{#1},{#2}\rangle}
\newcommand\VMonc[3]{\langle{#1},{#2},{#3}\rangle}

\newcommand\Multind[1]{\begin{substack}#1\end{substack}}
\newcommand\Biind[2]{\genfrac{}{}{0pt}{1}{#1}{#2}}

\newcommand\Adapt[1]{\cA_{#1}}

\newcommand\Rllp[1]{\widetilde{#1}}

\newcommand\Emptyfun[1]{0_{#1}}

\newcommand\Multisetfin[1]{\mathcal{M}_{\mathrm{fin}}({#1})}

\newcommand\Myparag[1]{\noindent\textbf{#1.}\ }

\newcommand\Webinf[1]{\Web{#1}_\infty}
\newcommand\Fin[1]{\mathsf{F}({#1})}


\newcommand\Phasefin{\mathbb{F}}

\newcommand\Fspace[1]{\mathrm f{#1}}
\newcommand\Space[1]{\mathrm m{#1}}

\newcommand\Prom[1]{{#1}^!}
\newcommand\Promp[1]{{(#1)}^!}
\newcommand\Promm[1]{{#1}^{!!}}

\newcommand\Ssupp[1]{\Supp{#1}}

\newcommand\FIN{\mathbf{Fin}}
\newcommand\FINH[2]{\FIN({#1},{#2})}
\newcommand\FINF[1]{\FIN[{#1}]}
\newcommand\SETINJ{\mathbf{Inj}}
\newcommand\EMB{\FIN_{\mathrm e}}

\newcommand\SNat{\mathsf N}
\newcommand\Snat{\mathsf N}

\newcommand\Iter[1]{\mathsf{It}_{#1}}
\newcommand\Piter[2]{\Iter{#1}^{(#2)}}
\newcommand\Case[1]{\mathsf{Case}_{#1}}
\newcommand\Fix[1]{\mathsf{Y}_{#1}}
\newcommand\Pfix[2]{\mathsf{Y}_{#1}^{(#2)}}
\newcommand\Ifthenelse[3]{\mathtt{if}\,{#1}\,\mathtt{then}\,{#2}\,\mathtt{else}\,{#3}}

\newcommand\Trace[1]{\mathsf{Tr}{#1}}
\newcommand\Tracelin[1]{\mathsf{tr}{#1}}
\newcommand\Ptrace[1]{\|{#1}\|}
\newcommand\Enat[1]{\overline{#1}}
\newcommand\FA[1]{\forall{#1}}
\newcommand\Trcl[2]{\langle {#1},{#2}\rangle}
\newcommand\Faproj[1]{\epsilon^{#1}}
\newcommand\Faintro[3]{\Lambda^{{#1},{#2}}({#3})}

\newcommand\Bool{\mathbf{Bool}}
\newcommand\True{\mathbf t}
\newcommand\False{\mathbf f}


\newcommand\Tarrow{\arrow}
\newcommand\Diffsymb{\mathsf D}

\newcommand\Diff[3]{\mathrm D_{#1}{#2}\cdot{#3}}
\newcommand\Diffexp[4]{\mathrm D_{#1}^{#2}{#3}\cdot{#4}}
\newcommand\Diffvar[3]{\mathrm D_{#1}{#2}\cdot{#3}}
\newcommand\Diffterm[3]{\mathrm D_{#1}{#2}\cdot{#3}}
\newcommand\Zero{0}
\newcommand\Appv[2]{\App{#1}{\Tofval{#2}}}
\newcommand\Appp[2]{({#1}\,{#2})}
\newcommand\Apppv[2]{\Appp{#1}{\Tofval{#2}}}
\newcommand\Applp[2]{({#1}){#2}}
\newcommand\Applpv[2]{\Applp{#1}{\Tofval{#2}}}
\newcommand\Apprp[2]{{#1}({#2})}
\newcommand\Apprpv[2]{\Apprp{#1}{\Tofval{#2}}}
\newcommand\Applrp[2]{({#1})({#2})}
\newcommand\Diffp[3]{\frac{\partial{#1}}{\partial{#2}}\cdot{#3}}
\newcommand\Derp[3]{\frac{\partial{#1}}{\partial{#2}}\cdot{#3}}
\newcommand\Derplist[4]{\frac{\partial^{#4}{#1}}{\partial#2_1\cdots\partial#2_{#4}}\cdot\left(#3_1,\dots,#3_{#4}\right)}
\newcommand\Derplistexpl[6]{\frac{\partial^{#6}{#1}}{\partial{#2}\cdots\partial{#3}}\cdot\left({#4},\dots,{#5}\right)}
\newcommand\Derpmult[4]{\frac{\partial^{#4}{#1}}{\partial{#2}^{#4}}\cdot\left(#3_1,\dots,#3_{#4}\right)}
\newcommand\Derpm[4]{\frac{\partial^{#4}{#1}}{\partial{#2}^{#4}}\cdot\left({#3}\right)}
\newcommand\Derpmultgros[4]{\frac{\partial^{#4}}{\partial{#2}^{#4}}\left({#1}\right)\cdot\left(#3_1,\dots,#3_{#4}\right)}
\newcommand\Derpmultbis[4]{\frac{\partial^{#4}{#1}}{\partial{#2}^{#4}}\cdot{#3}}
\newcommand\Derpmultbisgros[4]{\frac{\partial^{#4}}{\partial{#2}^{#4}}\left({#1}\right)\cdot{#3}}
\newcommand\Derppar[3]{\left(\Derp{#1}{#2}{#3}\right)}
\newcommand\Derpgros[3]{\frac{\partial}{\partial{#2}}\Bigl({#1}\Bigr)\cdot{#3}}
\newcommand\Derpdeux[4]{\frac{\partial^2{#1}}{\partial{#2}\partial{#3}}\cdot{#4}}
\newcommand\Derpdeuxbis[5]{\frac{\partial^{#5}{#1}}{\partial{#2}\partial{#3}}\cdot{#4}}
\newcommand\Diffpv[3]{\frac{\partial}{\partial{#2}}{#1}\cdot{#3}}
\newcommand\Diffpd[5]{\frac{\partial^2{#1}}{\partial{#2}\partial{#3}}\cdot\left({#4},{#5}\right)}

\newcommand\Paragraph[1]{\smallbreak\noindent\textbf{#1}}

\newcommand\Redamone{\mathrel{\beta^{0,1}}}
\newcommand\Redonecan{\mathrel{{\bar\beta}^1_{\mathrm D}}}
\newcommand\Redpar{\mathrel{\rho}}
\newcommand\Redparcan{\mathrel{\bar\rho}}

\newcommand\Sat[1]{#1^*}

\newcommand\Can[1]{\left\langle{#1}\right\rangle}
\newcommand\Candiff[3]{\Delta_{#1}{#2}\cdot{#3}}



\newcommand\List[3]{#1_{#2},\dots,#1_{#3}}
\newcommand\Listbis[3]{#1_{#2}\dots #1_{#3}}
\newcommand\Listc[4]{{#1}_{#2},\dots,{#4},\dots,{#1}_{#3}}
\newcommand\Listbisc[4]{{#1}_{#2}\dots{#4}\dots{#1}_{#3}}
\newcommand\Absent[1]{\widehat{#1}}
\newcommand\Kronecker[2]{\delta_{{#1},{#2}}}

\newcommand\Eqindent{\quad}

\newcommand\Subst[3]{{#1}\left[{#2}/{#3}\right]}
\newcommand\Substz[2]{{#1}\left[0/{#2}\right]}
\newcommand\Substpar[3]{\left({#1}\right)\left[{#2}/{#3}\right]}
\newcommand\Substbis[2]{{#1}\left[{#2}\right]}

\newcommand\Span[1]{\overline{#1}}
\newcommand\SN{\mathcal{N}}
\newcommand\WN{\mathcal{N}}
\newcommand\Extred[1]{\mathop{\mathrm R}^{\textrm{ext}}{(#1)}}
\newcommand\Onered[1]{\mathop{\mathrm R}^1{(#1)}}
\newcommand\NO[1]{\mathop{\mathrm N}({#1})}
\newcommand\NOD[1]{\mathop{\mathrm N_0}({#1})}



\newcommand\Freemod[2]{{#1}\left\langle{#2}\right\rangle}
\newcommand\Mofl[1]{{#1}^\#}
\newcommand\Mlext[1]{\widetilde{#1}}
\newcommand\Difflamb{\Lambda_D}
\newcommand\Terms{\Lambda_{\mathrm D}}
\newcommand\Diffmod{\Freemod R\Difflamb}

\newcommand\Factor[1]{{#1}!}
\newcommand\Binom[2]{\left({{#1}\atop{#2}}\right)}
\newcommand\Multinom[2]{\left[{{#1}\atop{#2}}\right]}
\newcommand\Suite[1]{\bar#1}
\newcommand\Head[1]{\mathrel{\tau^{#1}}}

\newcommand\Headlen[2]{{\mathrm{L}}({#1},{#2})}
\newcommand\Betaeq{\mathrel{\mathord\simeq_\beta}}
\newcommand\Betadeq{\mathrel{\mathord\simeq_{\beta_{\mathrm D}}}}

\newcommand\Vspace[1]{E_{#1}}
\newcommand\Real{\mathbf{R}}

\newcommand\Ring{R}
\newcommand\Linapp[2]{\left\langle{#1}\right\rangle{#2}}
\newcommand\Fmod[2]{{#1}\langle{#2}\rangle}
\newcommand\Fmodr[1]{{\Ring}\langle{#1}\rangle}
\newcommand\Imod[2]{{#1}\langle{#2}\rangle_\infty}
\newcommand\Imodr[1]{{\Ring}\langle{#1}\rangle_\infty}
\newcommand\Res[2]{\langle{#1},{#2}\rangle}
\newcommand\Funofmat[1]{\widehat{#1}}
\newcommand\Transp[1]{{}^{\mathrm{t}}\!{#1}}
\newcommand\Idmat{\operatorname{I}}
\newcommand\FINMOD[1]{\operatorname{\mathbf{Fin}\,}({#1})}
\newcommand\Bcanon[1]{e_{#1}}

\newcommand\Mfinc[2]{\mathcal M_{#1}({#2})}
\newcommand\Expt[2]{\operatorname{exp}_{#1}({#2})}
\newcommand\Dexpt[2]{\operatorname{exp}'_{#1}({#2})}

\newcommand\Semtype[1]{{#1}^*}
\newcommand\Semterm[2]{{#1}^*_{#2}}

\newcommand\Elem[1]{\operatorname{Elem}({#1})}
\newcommand\Fcard[1]{\operatorname{Card}({#1})}
\newcommand\Linhom[2]{\cL({#1},{#2})}
\newcommand\Linhombil[2]{\cL^2({#1},{#2})}
\newcommand\Linhommulti[3]{\cL^{#1}({#2},{#3})}
\newcommand\Hom[2]{\operatorname{Hom}(#1,#2)}
\newcommand\Compana{\circ_\cA}

\newcommand\Der[1]{\mathsf d_{#1}}
\newcommand\Dernat{\mathsf d}
\newcommand\Derm[2]{\mathsf d_{#1}^{(#2)}}
\newcommand\Digg[1]{\mathsf p_{#1}}
\newcommand\Diggnat{\mathsf p}
\newcommand\Coder[1]{{\overline{\mathsf d}}_{#1}}
\newcommand\Codernat{{\overline{\mathsf d}}}
\newcommand\Coderm[2]{{\overline{\mathsf d}}_{#1}^{(#2)}}
\newcommand\Contr[1]{\mathsf c_{#1}}
\newcommand\Contrnat{\mathsf c}
\newcommand\Contrm[2]{\mathsf c_{#1}^{(#2)}}
\newcommand\Weakm[2]{\mathsf w_{#1}^{(#2)}}
\newcommand\Weak[1]{\mathsf w_{#1}}
\newcommand\Weaknat{\mathsf w}
\newcommand\Cocontr[1]{{\overline{\mathsf{c}}}_{#1}}
\newcommand\Cocontrnat{{\overline{\mathsf{c}}}}
\newcommand\Cocontrm[2]{{\overline{\mathsf{c}}}_{#1}^{(#2)}}
\newcommand\Coweak[1]{{\overline{\mathsf{w}}}_{#1}}
\newcommand\Coweaknat{{\overline{\mathsf{w}}}}



\newcommand\Conv[1]{\operatorname{c}^{#1}}
\newcommand\IConv{\mathrel{*}}
\newcommand\Exclun[1]{\operatorname{u}^{#1}}
\newcommand\Derzero[1]{\partial^{#1}_0}
\newcommand\Dermorph[1]{\partial^{#1}}
\newcommand\Dirac[1]{\delta_{#1}}



\newcommand\Lintop[1]{\lambda({#1})}
\newcommand\Neigh[1]{\mathsf V(#1)}
\newcommand\Bnd[1]{\mathsf D(#1)}
\newcommand\Matrix[1]{\mathsf{M}({#1})}

\newcommand\Ev{\operatorname{ev}}
\newcommand\Evlin{\mathsf{ev}}
\newcommand\REL{\operatorname{\mathbf{Rel}}}
\newcommand\RELI{\REL^{\mathord\subseteq}}
\newcommand\POLR{\operatorname{\mathbf{Pol}}}
\newcommand\POLRI{\POLR^{\mathord\subseteq}}
\newcommand\POP{\operatorname{\mathbf{Pop}}}
\newcommand\POPI{\POP^{\mathord\subseteq}}
\newcommand\SCOTTLIN{\mathbf{ScottL}}
\newcommand\RELS{\mathsf R}
\newcommand\POLS{\mathsf S}
\newcommand\POPS{\mathsf P}



\newcommand\Diag[1]{\Delta_{#1}}
\newcommand\Codiag[1]{\overline\Delta_{#1}}
\newcommand\Final[1]{\mathsf t_{#1}}
\newcommand\Initial[1]{\overline{\mathsf t}_{#1}}

\newcommand\Norm[1]{\|{#1}\|}

\newcommand\Tpower[2]{{#1}^{\otimes{#2}}}

\newcommand\Termty[3]{{#1}\vdash{#2}:{#3}}
\newcommand\Polyty[3]{{#1}\vdash_!{#2}:{#3}}

\newcommand\Ruleskip{\quad\quad\quad\quad}

\newcommand\Sterms{\Delta}
\newcommand\Pterms{\Delta^!}
\newcommand\Nsterms{\Delta_0}

\newcommand\Relspan[1]{\overline{#1}}
\newcommand\Rel[1]{\mathrel{#1}}

\newcommand\Redone{\leadsto^1}
\newcommand\Redzone{\leadsto}
\newcommand\Red{\leadsto}
\newcommand\Redalt{\leadsto^*_0}
\newcommand\Redgen{\leadsto_{\mathsf{g}}}
\newcommand\Redpart{\leadsto_{\mathsf{w}}}


\newcommand\Multn{\mathrm{m}}
\newcommand\Shape[1]{\mathcal{T}(#1)}
\newcommand\Tay[1]{{#1}^*}

\newcommand\Deg[1]{\mathrm{deg}_{#1}}
\newcommand\Linearize[2]{\mathcal L^{#1}_{#2}}
\newcommand\Fmodrel{\Fmod}

\newcommand\Redeq{=_\Delta}
\newcommand\Kriv{\mathsf K}
\newcommand\Dom{\operatorname{\mathsf{D}}}
\newcommand\Domp[1]{\Dom(#1)}

\newcommand\Codom{\operatorname{\mathsf{Codom}}}
\newcommand\Cons{::}
\newcommand\Addtofun[3]{{#1}[{#2}\mapsto{#3}]}
\newcommand\Tofclos{\mathsf T}
\newcommand\Tofstate{\mathsf T}
\newcommand\BT{\operatorname{\mathsf{BT}}}
\newcommand\NF{\operatorname{\mathsf{NF}}}

\newcommand\Msubst[3]{\partial_{#3}(#1,#2)}
\newcommand\Symgrp[1]{\frak S_{#1}}

\newcommand\Tcoh{\mathrel{\frownsmile}}
\newcommand\Tcohs{\mathrel{\frown}}
\newcommand\Size[1]{|{#1}|}
\newcommand\Psize[1]{|{#1}|_{\mathrm P}}
\newcommand\Ssize[1]{|{#1}|_{\mathrm S}}

\newcommand\Symbofcell[1]{\mathsf l({#1})}
\newcommand\Portsofwire{\partial}
\newcommand\V{\vdots}

\newcommand\Oppofwire[1]{{#1}^*}

\newcommand\Exclo[1]{\Int{\Orth A}}
\newcommand\Typeq[1]{\cE_{#1}}

\newcommand\Orthat[1]{\bar{#1}}

\newcommand\Dedeq{\mathrel{\mathord{\vdash}_{\operatorname{eq}}}}

\newcommand\Typing[1]{\vdash{#1}\mid}

\newcommand\Eq{\simeq}

\newcommand\Grsw[2]{\cG({#1},{#2})}

\newcommand\Struct[1]{\mathsf{S}({#1})}

\newcommand\Meas[1]{\mathsf{M}({#1})}

\newcommand\In{\iota}
\newcommand\Out{o}
\newcommand\Diffn[3]{{#1}^*_{#2,#3}}

\newcommand\Exp[1]{\exp_{#1}}

\newcommand\Coname[1]{\overline{#1}}
\newcommand\Namesof[1]{\textsf N({#1})}
\newcommand\ProcEmpty{*}
\newcommand\ProcPar[2]{{#1}\mid{#2}}
\newcommand\ProcParI{\mid}
\newcommand\ProcNu[2]{\nu{#1}\cdot{#2}}
\newcommand\ProcIn[3]{{#1}({#2})\cdot{#3}}
\newcommand\ProcInLab[4]{{#1}^{#4}({#2})\cdot{#3}}
\newcommand\ProcOut[3]{\overline{#1}\langle{#2}\rangle\cdot{#3}}
\newcommand\ProcOutLab[4]{\overline{{#1}^{#4}}\langle{#2}\rangle\cdot{#3}}
\newcommand\SoloOut[2]{\overline{#1}\langle{#2}\rangle}
\newcommand\FreeNames{\operatorname{\mathsf{FV}}}
\newcommand\ProcCong{\sim}
\newcommand\ProcRed{\mathrel\leadsto}
\newcommand\StateRed{\mathrel\leadsto}
\newcommand\NetRed{\leadsto}
\newcommand\StateRedCan{\leadsto_{\mathsf{can}}}
\newcommand\StateCan{\operatorname{\mathsf{Can}}}
\newcommand\ProcTrad[2]{[{#1}]_{#2}}
\newcommand\ClosTrad[2]{[{#1}]_{#2}}
\newcommand\SoupTrad[2]{[{#1}]_{#2}}
\newcommand\StateTrad[3]{[{#1,#2}]_{#3}}
\newcommand\StateTradOpen[2]{[{#1}]^o_{#2}}
\newcommand\StateOfProc{\operatorname{\mathsf{St}}}
\newcommand\ProcOfState{\operatorname{\mathsf{Pr}}}


\newcommand\PortIn[1]{#1^+}
\newcommand\PortOut[1]{#1^-}
\newcommand\PortPair[1]{\PortIn{#1},\PortOut{#1}}
\newcommand\NetBroadcast[1]{\mathsf{Br}_{#1}}

\newcommand\Into[1]{\Int{\Orth{#1}}}

\newcommand\FreePortP[1]{{#1}^+}
\newcommand\FreePortN[1]{{#1}^-}

\newcommand\Soup{\operatorname{\mathsf{Soup}}}
\newcommand\Private{\operatorname{\mathsf{Priv}}}

\newcommand\Labels{\cL}
\newcommand\NameOfLabel{\operatorname{\mathsf{Name}}}
\newcommand\LabelsOfState{\operatorname{\Labels}}
\newcommand\LabelsOfProc{\operatorname{\Labels}}
\newcommand\StateLTS{\mathbb{S}_{\Labels}}
\newcommand\StateTrans[2]{\mathrel{\mathop\rightarrow^{#1/\overline{#2}}}}

\newcommand\LabelsOfNet{\operatorname{\Labels}}
\newcommand\NetLTS{\mathbb{D}_{\Labels}}
\newcommand\NetTrans[2]{\mathrel{\mathop\rightarrow^{#1/\overline{#2}}}}
\newcommand\NetCom{\operatorname{\mathsf{Com}}}

\newcommand\LTStrad{\Phi}
\newcommand\RedCom[2]{D_{#1,#2}}

\newcommand\PCNF{\Delta}
\newcommand\PCNFreducible[2]{\PCNF_{#1,#2}}
\newcommand\PCNFreduced[2]{\operatorname{\mathsf{red}}_{#1,#2}}

\newcommand\NetNorm[1]{\|{#1}\|}

\newcommand\Num{\operatorname{\mathsf{N}}}
\newcommand\StrNum[1]{\Num_{\mathrm{str}}({#1})}
\newcommand\NDNum[1]{\Num_{\mathrm{ND}}({#1})}
\newcommand\SNDNum[1]{\Num({#1})}

\newcommand\FormSem[2]{[{#1}]_{#2}}
\newcommand\FormPureSem[1]{[{#1}]}
\newcommand\ProofSem[2]{[{#1}]_{#2}}

\newcommand\MsetAdd{+}

\newcommand\NetOfProof[2]{{#1}^*_{#2}}
\newcommand\Experiments[2]{\operatorname{\mathsf{exper}}_{#2}({#1})}
\newcommand\ExpRes[1]{\operatorname{\mathsf{res}}({#1})}
\newcommand\NetSem[3]{[{#1}]_{#2}^{#3}}
\newcommand\ProcSem[2]{[{#1}]_{#2}}

\newcommand\Trees{\mathsf{D}}

\newcommand\NetCombine[2]{{#1}\cdot{#2}}
\newcommand\NetDual[2]{{#1}\perp{#2}}



\newcommand\NetRedEq{\mathrel{\mathord\leadsto^{\mathord=}}}
\newcommand\NetRedC{\mathord\leadsto_{\mathrm{c}}}
\newcommand\NetRedND{\mathord\leadsto_{\mathrm{nd}}}
\newcommand\NetRedS{\mathord\leadsto_{\mathrm{s}}}
\newcommand\NetRedSND{\mathord\leadsto_{\mathrm{snd}}}
\newcommand\NetRedSNDW{\mathord\leadsto_{\mathrm{sndw}}}
\newcommand\NetRedNSNDW{\mathord\leadsto_{\mathrm{nw}}}
\newcommand\NetRedSNDMax{\mathord\leadsto^{\mathrm{max}}_{\mathrm{snd}}}
\newcommand\NetRedStr{\mathord\leadsto^{\mathord+}}
\newcommand\NetRedN{\mathord\leadsto_{\mathrm{n}}}
\newcommand\NetRedCan{\mathord\leadsto_{\mathrm{Can}}}

\newcommand\NetRedM{\mathord\leadsto_{\mathrm{m}}}
\newcommand\NetRedA{\mathord\leadsto_{\mathrm{a}}}
\newcommand\NetRedBox{\mathord\leadsto_{\mathrm{bx}}}
\newcommand\NetRedBB{\mathord\leadsto_{\mathrm{bb}}}



\newcommand\NetEqAC{\mathrel{\mathord\sim_{\mathrm{ac}}}}

\newcommand\GenContrCell{?^*}
\newcommand\GenCocontrCell{!^*}

\newcommand\ComRedPart[3]{\operatorname{\textsf{Red}}_{#1,#2}({#3})}
\newcommand\ComReduced[3]{\operatorname{\textsf{red}}_{#1,#2}({#3})}

\newcommand\TransCl[1]{#1^*}
\newcommand\ReflCl[1]{#1^-}
\newcommand\TransClStr[1]{#1^+}

\newcommand\Atoms{\cA}
\newcommand\Atorth[1]{\overline{#1}}
\newcommand\Forall[2]{\forall{#1}\,{#2}}
\newcommand\Exists[2]{\exists{#1}\,{#2}}

\newcommand\Rulename[1]{\quad(}
\newcommand\Proofseparation{\quad\quad\quad\quad}



\newcommand\Subproof[2]{\noLine \AxiomC{\raggedleft{\hbox{\begin{tabular}{cc}&{#1}\hspace{-2em}\ \end{tabular}}}}\UnaryInfC{#2}}

\newcommand\Psupp[2]{\textsf{supp}^{#2}({#1})}
\newcommand\Quant[1]{\mathcal{T}{#1}}
\newcommand\Symgroup[1]{\mathsf{S}_{#1}}
\newcommand\Symgroupf[1]{\mathsf{S}^{\mathrm{fin}}_{#1}}
\newcommand\Orbit[3]{(#1)^{#2}_{#3}}
\newcommand\Vvec[1]{\vec{#1}}

\newcommand\Satovt[1]{\widehat{#1}}

\newcommand\Trovt[1]{\Lambda({#1})}
\newcommand\Absso[1]{\Lambda({#1})}
\newcommand\Soproj[1]{\epsilon^{#1}}
\newcommand\Projso[1]{\epsilon^{#1}}
\newcommand\Coprojso[1]{{\bar\epsilon}^{#1}}

\newcommand\RELSO[1]{\mathbf{Rel}^{(#1)}}
\newcommand\RELACTION[1]{\mathbf{Rel}[{#1}]}
\newcommand\FINSO[1]{\FIN^{(n)}}

\newcommand\Substso[1]{{#1}^*}

\newcommand\Tupleid[1]{\vec{#1}}

\newcommand\Invar[2]{{#1}:{#2}}
\newcommand\Moved[2]{{#1}_{#2}}

\newcommand\Melange[1]{\eta^{#1}}
\newcommand\Act[2]{{#1}\cdot{#2}}
\newcommand\Forallf[1]{\forall\,{#1}}
\newcommand\Existsf[1]{\exists\,{#1}}
\newcommand\Tsem[2]{[{#1}]^{#2}}
\newcommand\Tsempol[2]{[{#1}]^{#2}_\POLS}
\newcommand\Tsempop[2]{[{#1}]^{#2}_\POPS}
\newcommand\Tsemrel[2]{[{#1}]^{#2}_\RELS}
\newcommand\Psem[2]{[{#1}]^{#2}}

\newcommand\Psemp[1]{[{#1}]}

\newcommand\Sirpring{\mathsf S}

\newcommand\Sig{\Sigma}
\newcommand\Arity{\mathop{\mathsf{ar}}}
\newcommand\Celltype{\mathop{\mathsf{symb}}}
\newcommand\Wireports{\partial}

\newcommand\Plusl{\mathop{\mathord\oplus\mathsf l}}
\newcommand\Plusr{\mathop{\mathord\oplus\mathsf r}}

\newcommand\Putfig[2]{\begin{minipage}[c]{#1\textwidth}\input{#2}\end{minipage}}

\newcommand\Putfigs[3]{\begin{minipage}[c]{#1\textwidth}\scalebox{#2}{\input{#3}}\end{minipage}}


\newcommand\Figspace[1]{\hspace{#1em}}

\newcommand\Emptynet{\epsilon}

\newcommand\ARG{s}

\newcommand\Mark{\vbox{\hbox{}\vspace{-4mm}\hbox{}}}

\newcommand\Forallcell[1]{\forall{#1}}
\newcommand\Existscell[2]{\exists{#1},{#2}}
\newcommand\Promcell[2]{{#1}^!,{#2}}
\newcommand\Wirerev[1]{{#1}^*}

\newcommand\Sexcl{\scriptstyle{\mathord!}}
\newcommand\Sint{\scriptstyle{\mathord?}}

\newcommand\Finsub{\sqsubseteq}

\newcommand\Field{\mathbf k}
\newcommand\Simple[1]{\hat{#1}}

\newcommand\Cut[2]{\langle{#1}\mid{#2}\rangle}



\newcommand\CONTR{\mathsf c}
\newcommand\DER{\mathsf d}
\newcommand\WEAK{\mathsf w}

\newcommand\Ttens[2]{\otimes(#1,#2)}
\newcommand\Tpar[2]{\IPar(#1,#2)}
\newcommand\Tcocontr[2]{\CONTR_\oc(#1,#2)}
\newcommand\Tcontr[2]{\CONTR_\wn(#1,#2)}
\newcommand\Tcoder[1]{\DER_\oc(#1)}
\newcommand\Tder[1]{\DER_\wn(#1)}
\newcommand\Tweak{\WEAK_\wn}
\newcommand\Tcoweak{\WEAK_\oc}
\newcommand\Tprom[1]{{#1}^\oc}
\newcommand\Tpromn[2]{{#1}^\oc_{#2}}
\newcommand\Netofseq[1]{{#1}^\bullet}

\newcommand\Net[2]{[#1,#2]}

\newcommand\Vecb[2]{\vec{#1}_{<#2}}
\newcommand\Veca[2]{\vec{#1}_{>#2}}

\newcommand\Ann{\mathop{\mathsf{ann}}}
\newcommand\Canbasis[1]{\mathsf e_{#1}}

\newcommand\Op[1]{{#1}^{\mathsf{op}}}

\newcommand\Derc[1]{\partial_{#1}}
\newcommand\Dercm[2]{\partial_{#1}^{#2}}
\newcommand\Coderc[1]{\overline\partial_{#1}}
\newcommand\Codercm[2]{\overline\partial_{#1}^{#2}}
\newcommand\Partexcl[2]{\oc_{#1}#2}

\newcommand\Shuffle{\mathsf{Shuffle}}
\newcommand\Funofmatm{\theta}

\newcommand\Primint[1]{I_{#1}}
\newcommand\Stermsh[2]{\Sterms_{#1}^{(#2)}}

\newcommand\Primmor[1]{I_{#1}}

\newcommand\Curlin[1]{\mathsf{cur}(#1)}

\newcommand\Taym[2]{\mathrm T^{#2}_{#1}}

\newcommand\Named[2]{#1\cdot#2}
\newcommand\Names{\cN}
\newcommand\Truenames{\cN}
\newcommand\Noname{\tau}
\newcommand\Interface[1]{\mathsf{Int}(#1)}
\newcommand\Freeinterface[1]{\mathsf{Int}^-(#1)}
\newcommand\Foliage[1]{\mathsf{Fol}(#1)}
\newcommand\Freefoliage[1]{\mathsf{Fol}^{-}(#1)}
\newcommand\Freecontext[1]{{#1}^-}
\newcommand\Freeset[1]{{#1}^-}
\newcommand\Cutnet[1]{{#1}_{\textsf c}}
\newcommand\Freenet[1]{{#1}_{\textsf f}}

\newcommand\Rolam{\mathsf{lam}}
\newcommand\Roapp{\mathsf{app}}

\newcommand\Tenspow[2]{{#1}^{\mathord\otimes #2}}
\newcommand\Complin{\,}
\newcommand\Compmat{\,}
\newcommand\Compl{\Complin}
\newcommand\Beta{\beta}
\newcommand\Dbeta{\delta}

\newcommand\Dapp[2]{\Diffsymb{#1}\cdot{#2}}
\newcommand\Dappm[3]{\Diffsymb^{#1}{#2}\cdot{(#3)}}
\newcommand\Dsubst[3]{\frac{\partial #1}{\partial #3}\cdot{#2}} 
\newcommand\Combapp[4]{\App{\Dappm{#1}{#2}{#3}}{#4}} 

\newcommand\FINV{\FIN(\Field)}
\newcommand\FINVK{\FIN_!(\Field)}
\newcommand\Fun[1]{\mathsf{Fun}(#1)}

\newcommand\Monoidal{\mu}
\newcommand\Monoidaln[1]{\mu^{(#1)}}
\newcommand\Seely{\mathsf m}

\newcommand\Polyhom[2]{\mathbf{Pol}_\Field(#1,#2)}
\newcommand\Anahom[2]{\widetilde{\mathbf{Pol}}_\Field(#1,#2)}

\newcommand\LL{\textsf{LL}}
\newcommand\DILL{\textsf{DiLL}}
\newcommand\MLL{\textsf{MLL}}
\newcommand\MALL{\textsf{MALL}}

\newcommand\Excllab{\scriptstyle\oc}
\newcommand\Intlab{\scriptscriptstyle\wn}
\newcommand\Parlab{\scriptscriptstyle\IPar}
\newcommand\Tenslab{\scriptscriptstyle\ITens}

\newcommand\Catnet[2]{[#1,#2]}

\newcommand\Qcoeff{\mathbb B}

\newcommand\Kl[1]{#1_{\oc}}

\newcommand\Sem[1]{[#1]}

\newcommand\CBVLAM{call-by-value lambda-calculus}
\newcommand\CBV{call-by-value}
\newcommand\Let[3]{\App{\Abs{#1}{#3}}{\Tofval{#2}}}
\newcommand\Letbis[3]{\textsf{let }#1=#2\textsf{ in }#3}

\newcommand\Valsymb{\mathsf V}

\newcommand\Redv{\beta_{\Valsymb}}
\newcommand\Redvz{{\hat\beta}_{\Valsymb}}
\newcommand\Redvtr{\beta^*_{\Valsymb}}
\newcommand\Redvztr{{\hat\beta}^*_{\Valsymb}}
\newcommand\Redrv{\delta}
\newcommand\Rednsrv{\tilde\delta}
\newcommand\Redrvtr{\beta^*_{\Valsymb}}

\newcommand\Parv{\rho_{\Valsymb}}
\newcommand\Parvtr{\rho^*_{\Valsymb}}
\newcommand\Lamval{\cV}
\newcommand\Lamvall{\cV_\lambda}

\newcommand\Supertens[2]{#1^{\mathord\otimes #2}}
\newcommand\Supertensup[2]{#1^{\mathord\otimes(#2)}}
\newcommand\Supertensp[2]{(#1)^{\mathord\otimes #2}}
\newcommand\Supertenspup[2]{(#1)^{\mathord\otimes(#2)}}

\newcommand\Tofval[1]{\left\langle#1\right\rangle}
\newcommand\Tofvalr[1]{\left\langle#1\right\rangle}
\newcommand\Tofvalrs[1]{\left\langle#1\right\rangle}
\newcommand\Tredv{\cN}
\newcommand\Simpl[2]{{#1}\multimap{#2}}

\newcommand\Softype[1]{#1^\bullet}

\newcommand\Cow{1}
\newcommand\Cbunch[2]{[#1]\cdot#2}
\newcommand\Cbunchp[2]{([#1]\cdot#2)}

\newcommand\Init[1]{\cI(#1)}
\newcommand\Downcl[1]{\mathord\downarrow#1}
\newcommand\Upcl[1]{\mathord\uparrow#1}
\newcommand\Downclr[2]{\mathord\downarrow_{#1}#2}
\newcommand\Upclr[2]{\mathord\uparrow_{#1}#2}

\newcommand\Substr{\sqsubseteq}

\newcommand\ScottU{U_{\mathsf S}}

\newcommand\Intint[1]{\underline{#1}}

\newcommand\Dsubstv[3]{\partial_{#3}(#1;#2)} 

\newcommand\Proofvskip{\
\xymatrix @R=1.5em @C=3pc
  {\Tens{\Excl X}{\Excl Y}
    \ar[r]^{\Seely_{X,Y}}
    \ar[dd]^{\Tens{\Digg X}{\Digg Y}}
    & \Excl{(\With XY)}\ar[d]^{\Digg{\With XY}}\\
    &\Excl{\Excl{(\With XY)}}
    \ar[d]^{\Excl{\Pair{\Excl{\Proj 1}}{\Excl{\Proj 2}}}}\\
    \Tens{\Excl{\Excl X}}{\Excl{\Excl Y}}\ar[r]^{\Seely_{\Excl X,\Excl Y}}
    &\Excl{(\With{\Excl X}{\Excl Y})}
  }

  \xymatrix @R=0.6em @C=4pc
  {\Supertensp{\Excl X}{k}\ar[r]^-{\Supertensp{\Contrm Xn}k}
    & \Supertensp{\Supertensp{\Excl X}{n}}k\ar[r]^-{\sigma}
    & \Supertensp{\Supertensp{\Excl X}{k}}n
  }

  \xymatrix @R=0.8em @C=3pc
  {\Supertensp{\Excl X}{k}\ar[r]^-{\Supertensp{\Weak X}k}
    & \Supertensp{\One}k\ar[r]^-\lambda
    & \One
  }

  \xymatrix @R=0.6em @C=3pc
  {\Supertensp{\Excl X}k\ar[r]^{\Supertensp{\Digg X}k}
    & \Supertensp{\Excl{\Excl X}}{k}\ar[r]^-{\Monoidaln k_X}
    & \Excl{(\Supertensp{\Excl X}{k})}\ar[r]^-{\Excl f}
    & \Excl Y
  }  
\label{eq:der-coder-monoidal}
\vcenter{\hbox{\xymatrix @R=3em @C=3pc
{
  \Tens{X}{\Excl Y}\ar[r]^{\Tens{\Coder X}{\Excl Y}}\ar[d]^{\Tens{X}{\Der Y}}
  & \Tens{\Excl X}{\Excl Y}\ar[d]^{\Monoidal_{X,Y}}\\
  \Tens XY\ar[r]^{\Coder{\Tens XY}}
  & \Excl{(\Tens XY)}
}}}
\label{eq:coder-digg}
\vcenter{\hbox{\xymatrix @R=3em @C=3pc
{
  X\ar[r]^{\Coder X}\ar[d]_{\Tens{(\Coder{\Excl X}\Compl\Coder
    X)}{(\Excl{\Coweak X}\Compl\Monoidal_\One)}}
  & \Excl X\ar[d]^{\Digg X}\\
  \Tens{\Excl{\Excl X}}{\Excl{\Excl X}}\ar[r]^{\Cocontr{\Excl X}}
  & \Excl{\Excl X}
}}}

  \Excl f=\sum_{n=0}^\infty \Coderm Yn\Compl\Supertens fn\Compl\Derm Xn

    \Mset{\List f1n,\Prom f}=
    \Cocontr U\Comp\Tensp{\Mset{\List f1n}}{\Prom f}\Comp\Contrm
    U{k,2}:\Supertens{(\Excl U)}k\to\Excl U\,.
  
    (h_1\mid\cdots\mid h_n)=\Mixn n\Comp(h_1\ITens\cdots\ITens h_n)\Comp\Contrm
    U{k,n}:\Supertens{(\Excl U)}k\to\Bot\,.
  2ex]
\subfigure[Grammar of terms, bags, tests, expressions, sums.]{\label{fig:grammar}
    \begin{tabular}{@{}p{.1\linewidth}@{}p{.125\linewidth}@{}p{.525\linewidth}@{}p{.15\linewidth}@{}}
    :       &         &        &\hfill terms\3pt]    
    :       &         &        &\hfill tests\3pt]     
    \\   
    \hline
    \\
    \multicolumn{3}{p{.80\linewidth}}{
    \hspace{-6pt}} &sums of terms\3pt]    
    \multicolumn{3}{p{.80\linewidth}}{
    \hspace{-6pt}} &\hfill sums of tests\3pt]        
    \end{tabular}
}
\4pt]
\subfigure[Notation on parallel composition of tests.]{
\label{fig:notpar}
\begin{minipage}{\linewidth}

\vspace{-3pt}
\end{minipage}
}
\
	\bold{I}\ass \lam x.x,\quad \bold{T}\ass \lam xy.x,\quad \bold{F} \ass \lam xy.y,\quad \bold{D}:=\lam x.x[x],

	\Xi_{n_1,\ldots,n_m} \ass \lam x_1\dots x_m.\bold{I}[x_1]^{\sim n_1}\cdots [x_m]^{\sim n_m},\textrm{ for all }n_1,\dots,n_m\in\nat,
2ex]
\subfigure[Notation on .]{
\label{fig:notsums}
\begin{minipage}{\linewidth}
2pt]
	\textstyle\gto(\sum_i V_i) = \sum_i \gto(V_i)\qquad
	\textstyle(\sum_i P_i)\mcup \sP = \sum_i P_i\mcup \sP\qquad
	\textstyle[\sum_i L_i]= \sum_i [L_i]\-6pt]
3pt]
\hrulefill\2ex]
\subfigure[Definition of linear substitution. In the abstraction case we assume wlog .]{\label{fig:linsubst}
\begin{minipage}{\linewidth}
2ex]
\bag{L_1,\ldots,L_k}\lsubst{x}{N} &=&\Sigma_{i = 1}^k [L_1,\ldots,L_i\lsubst{x}{N}\ldots,L_k],\1ex]
(MP)\lsubst{x}{N} &=& M\lsubst{x}{N}P + M(P\lsubst{x}{N}),\1ex]
(\lam y.M)\lsubst{x}{N} &=& \lam y.M\lsubst{x}{N}.\
\end{minipage}
}
\caption{\footnotesize Notations on sums and definition of linear substitution.}
\label{fig:statics2}
\end{figure} 
\begin{conv} As a syntactic sugar -- and \emph{not} as actual syntax --
we extend all the constructors to sums by multilinearity, setting for instance
 
in such a way that the equations in Figure~\ref{fig:notsums} hold.
\end{conv}

This kind of meta-syntactic notation is discussed thoroughly in \cite{EhrhardR08}. 

\begin{rem}
In the  particular case of empty sums, we get
 , , , , , ,  and .
 Therefore  annihilates any term, bag or test (but not the sums).
\end{rem}


We now give some examples of this extended (meta-)syntax.

\begin{exa} We have:
\begin{enumerate}[1.] 
\item ,
\item  by sum idempotency,
\item ,
\item , therefore:
\item .
\end{enumerate}
\end{exa}

In the following two definitions we make an essential use of the extended syntax.
We recall that an operator  is \emph{extended by linearity} by setting .

\begin{defi}[Substitution] Let  and .
The \emph{(capture-free) substitution of  for  in }, denoted by , is defined as usual.
Accordingly,   denotes an expression of the extended  syntax.
Finally, we extend this operation to sums as in  by linearity in .
\end{defi}



\begin{defi}[Linear Substitution] The \emph{linear (capture-free) substitution of  for  in }, denoted by , is defined 
in Figure~\ref{fig:linsubst}.
The expression  belongs to the extended syntax. We extend this operation 
to sums as in  by linearity in , as we did for usual substitution.
\end{defi}

Roughly speaking, the linear substitution  replaces 
\emph{exactly one} free occurrence of  in  with the term .
If there is no occurrence of  in  then the result is 0.
In presence of multiple occurrences, all possible choices are made and the result is the sum of terms corresponding to them.

\begin{rem}
Observe that   is linear in  and in , 
whereas   is linear in  but not in  .
\end{rem}

We now give some examples of linear and classic substitution.

\begin{exa} Let  and .
\begin{enumerate}[1.]
\item If  is closed, then ,
\item ,
\item ,
\item ,
\item .
\end{enumerate}
\end{exa}

Linear substitutions commute in the sense expressed by the next theorem, 
whose proof is rather classic and thus omitted.

\begin{thm}[Schwarz's Theorem, cf.\ \cite{EhrhardR03}]\label{thm:Schwarz}
For ,  and  we have:

In particular, if  the two substitutions commute.\qed
\end{thm}

\begin{nota}\
\begin{iteMize}{}
\item Given a bag  and  we set .

\item Given bags  and  we set .
\end{iteMize}
In particular, .
\end{nota}

The above notation  makes sense because, by Theorem~\ref{thm:Schwarz}, the expression  
 is actually independent from the enumeration of  in . Moreover recall that we use -equivalence, so that bound variables can 
 be renamed in order to avoid capture of free variables during substitution.

\subsection{The Operational Semantics}\label{subs:OpSem}

In this section we are going to introduce the reduction rules defining the operational semantics of the \dzlam-calculus with tests.

\begin{defi}\label{def:reductionrules} The \emph{reduction semantics} of the \dzlam-calculus with tests is generated by the 
rules in Figure~\ref{fig:dzlamopsem}. \begin{figure}[t]
\centering
\textbf{Reduction Semantics}\3pt]
\hrulefill\2ex]
\subfigure[Context closure of a relation .]{\label{fig:context_closure}
\begin{minipage}{\linewidth}
-1ex]
 M\lsubst{x}{L_1}\cdots\lsubst{x}{L_k}\subst{x}{0} = \Sigma_{\sigma \in \perm{k}} M\{L_{\sigma(1)}/x^1,\ldots,L_{\sigma(k)}/x^k\}; 
	D\hole{\cdot}\gramm \hole{\cdot} \mid \lam x.D \mid DP \mid M[D,\seq L] \mid \gto(C)\\
	C\hole{\cdot}\gramm \gt[D,\seq L]

	M\Tobsle N \iff \forall C\hole{\cdot}\in\ContSet\textrm{ closing  }(C\hole{M}\msto \varepsilon\ \imp C\hole{N}\msto \varepsilon).

    \begin{array}{ll}
    t\comp s=\{(a,\beta)\quad \st&\exists k\in\nat,\ \exists (a_1,\alpha_1),\dots,(a_k,\alpha_k)\in s\textrm{ such that } \\
					       &a = a_1\mcup\dots\mcup a_k\ \text{and}\ ([\alpha_1,\dots,\alpha_k],\beta)\in t\qquad \}.\\
    \end{array}
     
	 \Proj{1}=\{([(1,\ga)],\ga)\st \ga\in S \} : \With{S}{T}\to S,\\
	 \Proj{2}=\{([(2,\ga)],\ga)\st \ga\in T \} : \With{S}{T} \to T.

  \Pair st=\{(a,(1,\ga))\st(a,\ga)\in s\}\cup\{(b,(2,\gb))\st(b,\gb)\in t\}\,.

\eval_{ST} =\{(([(a,\gb)],a),\gb)\st a\in\Mfin S\ \text{and}\ \gb\in T\} : \With{\Funint ST}S\to T\,.

  \eval_{ST}\comp (\curry(s)\times \Id{S}) = s.

\Int{M}_{\seq x}\subseteq \Int{N}_{\seq x} \iff M\Tobsle N
 (M\seq P)\lsubst{\seq x}{\seq P'}\subst{\seq x}{0}\msto_\beta \bold{I} + \sM,\textrm{ for some }\sM\in\sums{r}. 
	H = \lam z_1\dots z_q.\Xi_{k_1,\ldots,k_q}[z_1\seq P_{1,1},\dots,z_1\seq P_{1,k_1}]\cdots[z_q\seq P_{q,1},\dots,z_q\seq P_{q,k_q}].

\begin{array}{ll}
Q'_k = \mcup^q_{i = 1} \mcup^{k_i}_{j = 1} P'_{i,j,k}&\textrm{ for all }1\le k \le m\\
Q''_k = \mcup^q_{i = 1} \mcup^{k_i}_{j = 1} P''_{i,j,k}&\textrm{ for all }1\le k \le n\\
\end{array}

\begin{array}{l}
(MQ'_1\cdots Q'_{h-1}(Q'_{h}\mcup [H])Q'_{h+1} \cdots Q'_m) \lsubst{\seq x}{\seq Q''}\subst{\seq x}{0}\msto_\gb\\
((\lam y_1\dots y_m.y Q_1\cdots Q_q)Q'_1\cdots Q'_{h-1}(Q'_{h}\mcup [H])Q'_{h+1} \cdots Q'_m) \lsubst{\seq x}{\seq Q''}\subst{\seq x}{0} + \sM_1\msto_\gb\\
(HQ_1\cdots Q_q)\lsubst{\seq y}{\seq Q'}\lsubst{\seq x}{\seq Q''}\subst{\seq y,\seq x}{0} + \sM_2\msto_\gb\\
(\Xi [M_{1,1}\seq P_{1,1},\dots,M_{1,k_1}\seq P_{1,k_1}]\cdots[M_{q,1}\seq P_{q,1},\dots,M_{q,k_q}\seq P_{q,k_q}])\lsubst{\seq y}{\seq Q'}\lsubst{\seq x}{\seq Q''}\subst{\seq y,\seq x}{0} + \sM_3\msto_\gb\\
\Xi [M_{1,1}\seq P_{1,1}\sigma_{1,1},\dots,M_{1,k_1}\seq P_{1,k_1}\sigma_{1,k_1}]\cdots[M_{q,1}\seq P_{q,1}\sigma_{q,1},\dots,M_{q,k_q}\seq P_{q,k_q}\sigma_{q,k_q}] + \sM_4\\
\end{array}

	M \obsle N \iff \forall D\hole{\cdot}\in\Set{r}_{\hole{\cdot}} \textrm{ closing } (D\hole{M} \mbox{ is solvable } \imp D\hole{N} \mbox{ is solvable}). 

x^\ell = x,\qquad (\lam x.M)^\ell = \lam x.M^\ell,\qquad (MP)^\ell = M^\ell P^\ell,\\
[L_1,\dots,L_k]^\ell = [L_1^\ell,\dots,L_k^\ell],\qquad
\gto_i(V)^\ell = \lam x_1\ldots x_{\ell(i)}.V^\ell\textrm{ where },\\
(\gt[(L_1)_{i_1},\dots,(L_k)_{i_k}])^\ell =  \lam x.x[L_1^\ell[]^{\sim\ell(i_1)},\dots,L_k^\ell[]^{\sim\ell(i_k)}]\textrm{ where }.

\Int{M}_{\seq x}\subseteq \Int{N}_{\seq x} \iff M\obsle N
2ex]
\subfigure[Grammar of terms, bags, tests, expressions, sums.]{\label{fig:Fgrammar}
    \begin{tabular}{@{}p{.1\linewidth}@{}p{.125\linewidth}@{}p{.525\linewidth}@{}p{.15\linewidth}@{}}
    :       &         &        &\hfill terms\3pt]    
    :       &         &        &\hfill tests\3pt]     
    \\   
    \hline
    \\
    \multicolumn{3}{p{.80\linewidth}}{
    \hspace{-6pt}} &sums of terms\3pt]    
    \multicolumn{3}{p{.80\linewidth}}{
    \hspace{-6pt}} &\hfill sums of tests\3pt]        
    \end{tabular}
}
\4pt]
\subfigure[Linear Substitution.]{
\label{fig:LinSubstBang}
\begin{minipage}{\linewidth}


\vspace{-3pt}
\end{minipage}
}
\2ex]
\subfigure[Reduction rules. In the  rule we assume wlog .]{
\label{fig:newredrules}
\begin{minipage}{\linewidth}
\centering
\textbf{Reduction Semantics (New Rules)}
1ex]
\gto(V)[L_1,\ldots,L_k,\sN^{\bang}]\to_{\gto} \left\{
\begin{array}{ll}
\gto(V)  & \textrm{if ,} \\
0  & \textrm{otherwise.} \

\end{minipage}
}
\4pt]
\centering
\textbf{Context Closure (New Rule)}\
\infer[\mathrm{bangres}]{\bag{(M+\sN)^\oc} \mcup \sP \rel{R} \bag{(\sM+\sN)^\oc}\mcup \sP}{M\rel{R} \sM}

	P \gramm [L_1,\ldots,L_k,\sN^\bang] \hspace{250pt}\text{bags}
[L_1,\ldots,L_k,\sN\lsubst{x}{N},\sN^\bang]=0,
	D\hole{\cdot}\gramm \hole{\cdot} \mid \lam x.D \mid DP  \mid M[D,\seq L,\sN^\bang]\mid M[\seq L,(D + \sN)^{\bang}] \mid \gto(C)\\
	C\hole{\cdot}\gramm \gt[D,\seq L]

	M\Fobsle N \iff \forall C\hole{\cdot}\in\FContSet\,\textrm{ closing  }(\cOnv{C\hole{M}}\ \imp \cOnv{C\hole{N}}).

\hspace{-1pt}
\begin{array}{l}
\Int{[L_1,\dots,L_k,\sN^{\bang}]}_{\seq x} = \{(\mcup_{r=1}^{k+m} \seq a_r,[\beta_1,\ldots,\beta_{k + m}]) \st (\seq a_j,\beta_j)\in\Int{L_j}_{\seq x},1\le j\le k \textrm{ and } \\
\qquad\qquad\qquad\qquad\qquad\qquad\qquad\qquad\qquad\qquad\quad\ \!  (\seq a_i,\beta_i)\in \Int{\sN}_{\seq x},\ k < i\le k + m,\ m\geq 0\}.
\end{array}

	\TE{(MN)} =\sum_{n=0}^\infty \frac{1}{n!} M[\underbrace{N,\ldots,N}_{n\textrm{ times}}]

\begin{array}{rcl}
\TE{x} &=& \{x\}, \\
\TE{(\lam x.M)} &=& \{\lam x.M'\st M'\in\TE{M}\},\\
\TE{(MP)} &=& \{M'P'\st M'\in\TE{M},\ P'\in\TE{P} \},\\
\TE{(\gto(V))} &=& \{ \gto(V')\st V'\in\TE{V}\}\\
\TE{(\gt[M_1,\ldots,M_k])} &=& \{ \gt[M_1',\ldots,M_k']\st M_i'\in\TE{M_i},\textrm{ for } 1 \leq i \leq k\},\\
\TE{[L_1,\ldots,L_k,\sN^\bang]} &=& \{[L_1',\ldots,L_k']\mcup P \st L'_i\in\TE{L_i},\textrm{ for } 1\le i\le k,\ P\in \Mfin{\TE{\sN}} \},\\
\TE{(\Sigma_{i=1}^k A_i)} &=& \cup_{i=1}^k \TE{A_i}.\\
\end{array}

\Int{M}_{\seq x}\subseteq \Int{N}_{\seq x} \iff M\Fobsle N.
(N_0[N_1,\ldots,N_h])\lsubst{y}{P} = \Sigma_{\seq {P'} \in
  \mathfrak{P}}
N_0\lsubst{y}{P_0'}[N_1\lsubst{y}{P_1'},\ldots,N_h\lsubst{y}{P_h'}].

\begin{array}{rl}
\sV^{(\ell + k)} = &\lam z.z[(\lam x_1\ldots x_{k}.\lam y.y[\seq M^{(\ell +k)}[]^{\sim\ell(\seq m) + k}]) []^{\sim k},\seq L^{(\ell +k)}[]^{\sim\ell(\seq r) +k}]\\
\msto &\lam z.z[\lam y.y[\seq M^{(\ell +k)}[]^{\sim\ell(\seq m) + k}],\seq L^{(\ell +k)}[]^{\sim\ell(\seq r) +k}]\\
\end{array}
 
	\sV'^{(\ell'+k)} = \lam z.z[(\lam x_1\ldots x_{\ell'(j) + k}. V^{(\ell'+k)})\seq P^{(\ell'+k)}[]^{\sim\ell'(i)+k},\seq L^{(\ell'+k)}[]^{\sim\ell'(\seq r)+k}]\textrm{ is solvable.}

	\begin{array}{rl}
	\sV^{(\ell+k)} = &\lam z.z[(\lam x_1\ldots x_{\ell'(j) + k + 1}.V^{(\ell+k)})[]\seq P^{(\ell+k)}[]^{\sim\ell(i)+k},\seq L^{(\ell+k)}[]^{\ell(\seq r)+k}]\\
	\to_\beta&\lam z.z[(\lam x_2\ldots x_{\ell'(j) + k + 1}. V^{(\ell+k)})\seq P^{(\ell+k)}[]^{\sim\ell(i)+k},\seq L^{(\ell+k)}[]^{\ell(\seq r)+k}]\\
	=_{\alpha}&\lam z.z[(\lam x_1\ldots x_{\ell'(j) + k}. V^{(\ell+k)})\seq P^{(\ell+k)}[]^{\sim\ell(i)+k},\seq L^{(\ell+k)}[]^{\ell(\seq r)+k}]\\	
	\end{array}

	\sV^{(\ell+k)} = \lam z.z[(\lam x.M^{(\ell+k)})Q^{(\ell+k)}\seq P^{(\ell+k)}[]^{\sim\ell(i)+k},\seq L^{(\ell+k)}[]^{\sim\ell(\seq r)+k}] \to_\beta \sV'^{(\ell+k)}.

	\sV^{(\ell + k)} = \lam z.z[(\lam x_1\ldots x_{\ell(j) +k}.\lam y.y[\seq M^{(\ell +k)}[]^{\sim\ell(\seq m)+k}]) []^{\sim \ell(i) + k},\seq L^{(\ell +k)}[]^{\sim\ell(\seq r) +k}]\msto_\beta 0.

\begin{array}{rl}
\sV^{(\ell+k')} = &\lam z.z[(\lam x. M^{(\ell + k')})[]^{\sim\ell(i)+k'}, \seq L^{(\ell+k')}[]^{\sim \ell(\seq r) + k'}]\\
\to_\gb &\lam z.z[(M\subst{x}{0})^{(\ell+k')}[]^{\sim\ell(i)+k}, \seq L^{(\ell+k')}[]^{\sim \ell(\seq r) + k'}].\\
\end{array}

\begin{array}{rl}
\sV^{(\ell+k)} = &\lam z.z[(\lam x_1\dots x_{\ell(j)+k}.V^{(\ell+k)})P^{(\ell+k)}\seq P^{(\ell+k)}[]^{\sim \ell(i)+k},\seq L^{(\ell+k)}[]^{\sim\ell(\seq r)+k}]\\
\msto_\beta&\lam z.z[(\lam x_2\dots x_{\ell(j)+k}.V^{(\ell+k)}\lsubst{x}{P^{(\ell+k)}}\subst{x}{0})\seq P^{(\ell+k)}[]^{\sim \ell(i)+k},\seq L^{(\ell+k)}[]^{\sim\ell(\seq r)+k}].\\
\end{array}
 
	\sV'^{(\ell+k)} = \lam z.z[(\lam x_1\ldots x_{\ell(j) + k}. V^{(\ell +k)})\seq P^{(\ell+k)}[]^{\sim\ell(i)+k},\seq L^{(\ell+k)}]\msto_\beta 0.

	\begin{array}{rl}
	\sV^{(\ell + k')} = &\lam z.z[(\lam x_1\ldots x_{\ell(j) + k'}.V^{(\ell+k')})[]\seq P^{(\ell+k')}[]^{\sim\ell(i)+k'},\seq L^{(\ell+k')}[]^{\sim\ell(\seq r)+k'}]\\
	\msto_\beta&\lam z.z[(\lam x_2\ldots x_{\ell(j) + k'}. V^{(\ell+k')})\seq P^{(\ell+k')}[]^{\sim\ell(i)+k'},\seq L^{(\ell+k')}[]^{\sim\ell(\seq r)+k'}]\\
	=&\lam z.z[(\lam x_1\ldots x_{\ell'(j) + k}. V^{(\ell'+k)})\seq P^{(\ell'+k)}[]^{\sim\ell'(i)+k},\seq L^{(\ell'+k)}[]^{\sim\ell'(\seq r)+k}]\\
	\end{array}

\sV'^{(\ell+k)} = \lam z.z[M^{(\ell+k)}\lsubst{x}{P^{(\ell+k)}}\subst{x}{0}\seq P'^{(\ell+k)}[]^{\sim\ell(i)+k},\seq L^{(\ell+k)}[]^{\sim\ell(\seq r)+k}]\msto_\beta 0.
 \TE{A} = \cup_{P' \in \Mfin{\TE{\sN}}} \TE{M}([\TE{\seq {L}}] \mcup P') 
\hspace{-2pt}\begin{array}{ll}
& \ \ \ \ \TE{(A\lsubst{x}{N})} \\
& = \TE{(M\lsubst{x}{N}[\seq L,\sN^\bang])} \cup \\
& \ \ \ \ \cup_{i=1}^{k} \TE{(M[L_1,\pts,L_i\lsubst{x}{N},\pts,L_k,\sN^\bang])} \cup \\
& \ \ \ \ \cup\ \TE{(M[\seq L,\sN\lsubst{x}{N},\sN^\bang])} \\
& = \cup_{P\in \Mfin{\TE{\sN}}} \TE{(M\lsubst{x}{N})}([\TE{\seq {L}}]\mcup P) \cup \\
& \ \ \ \ \cup_{P'\in \Mfin{\TE{\sN}}} \cup_{i=1}^{k} \TE{M}([\TE{L_1},\pts,\TE{(L_i\lsubst{x}{N})},\pts,\TE{L_k}] \mcup P') \cup \\
& \ \ \ \ \cup_{P''\in \Mfin{\TE{\sN}}} \TE{M}([\TE{\seq {L}},\TE{(\sN\lsubst{x}{N})}] \mcup P'') \\
& = \cup_{P\in \Mfin{\TE{\sN}}} \TE{M}\lsubst{x}{\TE{N}}([\TE{\seq {L}}] \mcup P) \cup \\
& \ \ \ \ \cup_{P'\in \Mfin{\TE{\sN}}} \cup_{i=1}^{k} \TE{M}([\TE{L_1},\pts,\TE{L_i}\lsubst{x}{\TE{N}},\pts,\TE{L_k}] \mcup P') \cup \\
& \ \ \ \ \cup_{P''\in \Mfin{\TE{\sN}}} \TE{M}([\TE{\seq {L}},\TE{\sN}\lsubst{x}{\TE{N}}]\mcup P'') \\
& \ \ \ \ \textrm{by induction hypothesis,} \\
& = \cup_{P\in \Mfin{\TE{\sN}}} (\TE{M}([\TE{\seq {L}}] \mcup P))\lsubst{x}{\TE{N}} \\
& = \TE{A}\lsubst{x}{\TE{N}} \\
\end{array}

\hspace{-2pt}
\begin{array}{ll}
& \ \ \ \ \cup_{P\in\Mfin{\TE{\sN}}} \TE{A}\lsubst{x}{P}\subst{x}{0} \\
& = \cup_{P' \in \Mfin{\TE{\sM}}} \cup_{P\in\Mfin{\TE{\sN}}} (\TE{M}([\seq {\TE{L}}] \mcup P'))\lsubst{x}{P}\subst{x}{0} \\
& = \cup_{P' \in \Mfin{\TE{\sM}}}\cup_{P_0,P_1,P_2 \in\Mfin{\TE{\sN}}} \\
& \ \ \ \ \TE{M}\lsubst{x}{P_0}\subst{x}{0}([\TE{\seq {L}}]\lsubst{x}{P_1}\subst{x}{0} \mcup P'\lsubst{x}{P_{2}}\subst{x}{0}) \\
& = \cup_{P' \in \Mfin{\TE{(\sM\subst{x}{\sN})}}} \TE{(M\subst{x}{\sN})}(\TE{([\seq {L}]\subst{x}{\sN})} \mcup P') \\
& \ \ \ \ \textrm{by induction hypothesis, using the fact that 	} \\
& \ \ \ \	\cup_{P' \in \Mfin{\TE{\sM}}}\cup_{P_2 \in\Mfin{\TE{\sN}}}P'\lsubst{x}{P_{2}}\subst{x}{0}\textrm{ is equal to }\\
& \ \ \ \	\Mfin{\cup_{P\in\Mfin{\TE{\sN}}}\TE{\sM}\lsubst{x}{P}\subst{x}{0}}\\
& = \TE{(M\subst{x}{\sN}[\seq {L}\subst{x}{\sN},\sM\subst{x}{\sN}^\bang])} \\
& = \TE{(A\subst{x}{\sN})}\rlap{\hbox to279 pt{\hfill\qEd}}
\end{array}

\end{enumerate}

 
\end{document}
