

\documentclass[a4paper]{llncs}

\usepackage{color}  \usepackage{makeidx}  

\usepackage[utf8]{inputenc}
\usepackage[T1]{fontenc}
\usepackage{amsfonts}
\usepackage{amssymb}
\usepackage{graphicx}
\usepackage[polish,english]{babel}
\newtheorem{fact}{Fact}
\parindent0pt
\begin{document}

\title{Hierarchy of persistence \\with respect to the length of action's disability}

\author{Kamila Barylska\thanks{The study is founded 
by the Polish National Science Center
(grant No.2013/09/D/ST6/03928).}\inst{1}\and Edward Ochma\'{n}ski\inst{2,3}\\
\email\{{khama,edoch\}{@}mat.umk.pl}}

\institute{Faculty of Mathematics and Computer Science, Nicolaus Copernicus University,\\ Chopina 12/18, 87-100 Toru\'{n}, Poland
\and Faculty of Mathematics and Computer Science, Nicolaus Copernicus University,\\ Chopina 12/18, 87-100 Toru\'{n}, Poland - retired
\and Institute of Computer Science, Polish Academy of Sciences,\\Jana Kazimierza 5, 01-248 Warszawa, Poland - retired
}

\maketitle

\begin{abstract}
The notion of persistence, based on the rule "no action can disable another one" is one of the classical notions in concurrency theory. It is also one of the issues discussed  in the Petri net theory. We recall two ways of generalization of this notion: the first is "no action can kill another one" (called l/l-persistence) and the second "no action can kill another enabled one" (called the delayed persistence, or shortly e/l-persistence). Afterwards we introduce a more precise notion, called e/l-k-persistence, in which one action disables another one for no longer than a specified number k of single sequential steps. Then we consider an infinite hie\-rarchy of such e/l-k persistencies. We prove that if an action is disabled, and not killed, by another one, it can not be postponed indefinitely.
Afterwards, we investigate the set of markings in which two actions are enabled simultaneously, and also the set of reachable markings with that feature. We show that the minimum of the latter is finite and effectively computable.
Finally we deal with decision problems about e/l-k persistencies. We show that all the kinds of e/l-k persistencies are decidable with respect to steps, markings and nets.


\begin{keywords}
Petri nets, concurrency, persistence, decision problems
\end{keywords}

\end{abstract}

\section{Introduction}

Petri nets constitute a very useful and suitable tool for concurrent systems modeling. Thanks to them, we can not only model real systems, but also analyze their properties and design systems which fulfill given criteria. For many years, concurrent systems have been examined in the context of their compliance with certain desirable properties, which fits in with the trend of the so-called ethics of concurrent computations. One of the commonly found undesirable properties of concurrent systems is the presence of conflicts, and thus, one of the most desirable properties of them is conflict-freeness. The notion of persistence, proposed by Karp/Miller \cite{KarpMiller} is one of the the most important notions in concurrency theory. It is based on the behaviourally oriented rule "no action can disable another one", and generalizes the structurally defined notion of conflict-freeness.
\\ \\
The notion of persistence is one of the issues frequently discussed in the Petri net theory - \cite{BestDar,Grabowski,Hack,LandRob,Mayr,Koutny} and many others. It is being studied not only in terms of theoretical properties, and also as a useful tool for designing and analyzing software environments \cite{BestEsparza}. In engineering, persistence is a~highly desirable property, especially in a case of designing systems to be implemented in hardware. Many systems can not work properly without satisfying this property.
\\ \\
We say that an action of a processing system is persistent if, whenever it becomes enabled, it remains enabled until executed. A system is said to be persistent if all its actions are persistent. This classical notion has been introduced by Karp/Miller \cite{KarpMiller}. In section~2.6, we show two generalizations of the classical notion (defined in \cite{BarOch}): l/l-persistence and e/l-persistence which form the following hierarchy: . An action is said to be l/l-persistent if it remains live until executed, and is e/l-persistent if, whenever it is enabled, it cannot be killed by another action. For uniformity, we name the traditional persistence notion e/e-persistence. Next, we recall that those kinds of persistence are decidable in place/transition nets.
\\ \\
In section 3.1, we extend the hierarchy mentioned above with an infinite hierarchy of e/l-persistent steps. A step  is said to be \emph{e/l-k-persistent} for some k~ if the execution of an action  pushes the execution of any other enabled action away for at most k steps (more precise: if the execution of an action  stops the enabledness of any other action, then the enabledness is restored not later than after k steps). 
\\ \\
In section 3.2 we study decision problems related to the notion of e/l-k-persistence. These problems include EL-k Step Persistence Problem and EL-k Marking Persistence Problem. We show that both problems are decidable (Theorem~\ref{t45} and Theorem \ref{t46}). 
\\ \\
The next problem we want to focus on is EL-k Net Persistence Problem. In order to check the decidability of the problem we need to take advantage of additional tools and facts.
That is why we investigate the set of markings in which two actions are enabled simultaneously, and also the set of reachable markings with that feature. We show that the minimum of the latter is finite and effectively computable. We also prove that if some action pushes the enabledness of another one away for more than k steps, then it also needs to happen in some minimal reachable marking enabling these two actions. In our proofs we use the decidability of the Set Reachability Problem (from \cite{BarOch}) and also we make use of the theory of residual sets of Valk/Jantzen \cite{ValkJantzen}. Finally, we show that e/l-k-persistence is decidable with respect to nets (Theorem \ref{t418}).


\mbox{ }\\
We also prove (section 3.3) that if an action of an arbitrary p/t-net is disabled (but not killed) by another one, it can not be postponed indefinitely. We show that if a p/t-net is e/l-persistent, then it is e/l-k-persistent for some k  (Theorem~\ref{t410}), and such a number k can be effectively found (Theorem~\ref{t420}). We also point, that the above-cited result does not hold for nets which do not have the monotonicity property (i.e. it is not true that the action enabled in some marking  is also enabled in any marking  greater than ), for example for inhibitor nets.
\\ \\
The concluding section contains some questions and plans for further investigations.
\\ \\
A preliminary version of the paper was presented on the International Workshop on Petri Nets and Software Engineering (Hamburg, Germany, June 25-26, 2012) with electronical proceedings available online at CEUR-WS.org as Volume 851. The present paper is an improved and extended version of it.

\section{Basic Notions}

\subsection{Denotations}
The set of non-negative integers is denoted by . Given a set , the cardinality (number of elements) of  is denoted by , the powerset (set of all subsets) by , the cardinality of the powerset is . Multisets over  are members of  , i.e. functions from  into .

\subsection{Petri Nets and Their Computations}

The definitions concerning Petri nets are mostly based on \cite{DeselReisig}.

\begin{definition}[Nets]
\label{d21}
\emph{Net} is a triple , where:
\begin{itemize}
\item  and  are finite disjoint sets, of \emph{places} and \emph{transitions}, respectively;
\item  is a relation, called the \emph{flow relation}.
\end{itemize}
\end{definition}\mbox{ }\\
\indent For all  we denote:\\
\indent  -  the set of entries to \\
\indent  -  the set of exits from .\\
\\ \\
\newpage
Petri nets admit a natural graphical representation. Nodes represent places and transitions, arcs represent the flow relation. Places are indicated by circles, and transitions by boxes.
\\ \\
The set of all finite strings of transitions is denoted by , the length of  is denoted by , number of occurrences of a transition  in a string  is denoted by , two strings  such that   are said to be \emph{Parikh equivalent}, which is denoted by .
\\
\begin{definition}[Place/Transition Nets]
\label{d22}
\emph{Place/transition net} (shortly, \emph{p/t-net}) is a quadruple , where:
\begin{itemize}
\item  is a net, as defined above;
\item  is a multiset of places, named the \emph{initial marking}; it is marked by \emph{tokens} inside the circles, capacity of places is unlimited.
\end{itemize}
\end{definition}\mbox{ }\\
Multisets of places are named \emph{markings}. In the context of p/t-nets, they are mostly represented by nonnegative integer vectors of dimension , assuming that  is strictly ordered. The natural generalizations, for vectors, of arithmetic operations  and , as well as the partial oder , all defined componentwise, are well known and their formal definitions are omitted here.
\\ \\
In this context, by  we understand a vector of dimension  which has 1 in every coordinate corresponding to a place that is an entry to (an exit from, respectively)  and 0 in other coordinates.
\\ \\
A transition  is \emph{enabled} in a marking  whenever  (all its entries are marked). If  is enabled in , then it can be executed, but the execution is not forced. The execution of a transition  changes the current marking  to the new marking  (tokens are removed from entries, then put to exits). The execution of an action  in a marking  we call a (sequential) step. We shall denote  for the predicate "a is enabled in " and  for the predicate " is enabled in  and  is the resulting marking".
\\ \\
This notions and predicates we extend, in a natural way, to strings of transitions:  for any marking , and   () iff  and  .
\\ \\
\textbf{Remark:} Wherever this will not lead to confusion, we will use a notation  to denote the fact that the action  is enabled in a marking  and a marking  is the result of the execution of action  in a marking .
\\ \\
If , for some , then  is said to be \emph{reachable from }; the set of all markings reachable from  is denoted by  . Given a p/t-net , the set  of markings reachable from the initial marking  is called the \emph{reachability set} of S, and markings in   are said to be \emph{reachable} in S.

\mbox{ }\\
A transition  is said to be \emph{live in a marking } if there is a string  such that  is enabled in . A transition  that is not live in a~marking  is said to be \emph{dead in a marking }. Let  be a marking such that  for some , then if a transition  is enabled (live) in  and not enabled (not live) in , we say that (the execution of)  \emph{disables} (\emph{kills})  in a marking . We also say that an action   \emph{disables} (\emph{kills})  (in a net ) if  \emph{disables} (\emph{kills})  in some reachable marking .
\\ 
\begin{definition} [Inhibitor nets ]
\label{d23}
\emph{Inhibitor net}  is a quintuple , where:
\begin{itemize}
\item  is a p/t-net, as defined above;
\item  is the set of inhibitor arcs (depicted by edges ended with a small empty circle). Sets of entries and exits are denoted by  and , as in p/t-nets; the set of \emph{inhibitor entries} to  is denoted by .
\end{itemize}
\end{definition}
A transition  (of an inhibitor net) is enabled in a marking  whenever  (all its entries are marked) and   - all inhibitor entries to  are empty. The execution of  leads to the resulting marking .
\\ \\
The following well-known fact follows easily from Definitions \ref{d21} and \ref{d22}.
\begin{fact} [Diamond and big diamond properties]
\label{f24}
Any place/transition net possesses the following property:

\emph{Big Diamond Property}:

If  (Parikh equivalence), then .

Its special case with  is called the \emph{Diamond Property}:

If , then .
\end{fact}

\begin{definition}
\label{d261}
We say that a Petri net  has the \emph{monotonicity property} if and only if .
\end{definition}

\begin{fact}
\label{f27}
P/t-nets have the monotonicity property.
\end{fact}

\begin{proof}
Obvious, since in p/t-nets the tokens of  can be regarded as frozen (disactive) tokens. 	
\end{proof}
	
\begin{fact}
\label{f28}
Inhibitor nets do not have the monotonicity property.
\end{fact}

\begin{proof}
Let us look at the example of Fig. \ref{Fig1}. It can be easily seen that .  holds but  doesn't hold.	
\end{proof}

\begin{figure}[h]
\centering
\includegraphics[width=0.2\textwidth]{fig1.eps}
\caption{Non-monotonic inhibitor net}
\label{Fig1}
\end{figure}


\subsection{Monoid }
\begin{definition}[Monoid , rational operations, rational subsets]
\label{def_mon}\\
The monoid  is the set of -dimensional non-negative integer vectors with the componentwise addition .
\\
If  then  and the star operation is defined as , where  and . The partial order  is understood componentwise, and  means   and  . Rational subsets of  are subsets built from finite subsets with finitely many operations of union  , addition  and star .
\end{definition}
\begin{theorem}[Ginsburg/Spanier\cite{GinsburgSpanier}]
\label{twGins}
Rational subsets of  form an effective boolean algebra (i.e. are closed under union, intersection and difference).

\end{theorem}
\begin{definition} [-extension]
\label{d25}
Let , where   is a new symbol (denoting infinity). We extend, in a natural way, arithmetic operations: \\, , and the order: . \\The set of k-dimensional vectors over  we shall denote by , and its elements we shall call \emph{-vectors}. Operations  and the order  in  are componentwise.
\\ 
For , we denote by Min() the set of all minimal (wrt ) members of , and by Max() the set of all maximal (wrt ) members of . Let  be -vectors such that , then we say that \emph{ covers } (\emph{ is covered by }) .
\end{definition}
Let us recall the well known important fact known as the Dickson's Lemma.

\begin{lemma} [\cite{Dickson}]
\label{l26}
Any subset of incomparable elements of   is finite.
\end{lemma}

\begin{definition}[Closures, convex sets]
\label{def_clos}
\begin{itemize}
\item Let  and . We denote:  , \\,  , , and call the sets  and  of , respectively;
\item A set  such that  () is said to be -(-) ;
\item A set  such that   is said to be .
\end{itemize}


\end{definition}
We also recall a fact proved in \cite{BarOch}:

\begin{proposition}
\label{p26}
Any convex subset of  is rational.
\end{proposition}

\subsection{Reachability graph/tree and coverability graph}

Let us recall the notions of  \emph{a reachability graph/tree} and \emph{a coverability graph}. Their definitions can be also found in any monograph or survey about Petri nets (see \cite{DeselReisig,Starke} or arbitrary else). Reachability graphs/trees are used for studying complete behaviors of nets, but they are usually infinite, whichmakes an accurate analysis of them difficult. That is why we study coverability graphs, which represent the behaviours of nets only partially, but are always finite.
\\ \\
\emph{The reachability graph} of a p/t-net  is a couple  where .
\\ \\
The reachability graph  represents graphically the behaviour of the net . Vertices of the graph are reachable markings from the set , while edges are ordered pairs of reachable states, labeled by actions. More precisely: the edge  iff  is a state reachable from the initial marking , an action  (the label of the edge ) is enabled in a state  and . The existence of an edge  in the reachability graph of the net  indicates that the marking  is reachable in , the action  is enabled in  and after the execution ot the action  in the marking , the net  reaches the state .
\\ \\
Sometimes it is more convenient to use a special graph structure for listing all reachable markings of a given p/t-net, namely a tree structure. Such a tree is called \emph{a~reachability tree}. 
\\ \\For a given net  we construct its reachability tree  proceeding as follows:
\begin{itemize}
\item We start with the initial marking  which is the root vertex of the reachability tree.
\item For each action  enabled in the initial marking of the net, we create a new vertex , such that , and an edge  leading from  to  labelled by . 
\item We repeat the procedure for all the newly created vertices (markings).
\end{itemize}\mbox{ }\\
\textbf{Remark:} The construction of a reachability tree is a process potentially endless, as the structure is infinite in many cases.\\
\begin{definition}
Let  be a reachability tree of a net . The \emph{k-component} of the reachability tree  is the initial part of the tree of the depth~k (all vertices at depth lower than or equal to k).
\end{definition}\mbox{ }\\
In the case of a coverability tree it is convenient to present a constructional definition. That is why we introduce:
\\ \\
\textbf{Algorithm of the construction of a coverability graph}
\\ \\
We create a coverability graph for a p/t-net 
\begin{itemize}
\item Step 0. \textit{An initial vertex}\\
We set  \textbf{blue} for a start.\\
GOTO Step 1.\\ 
\item Step 1. \textit{Generating of new working vertices}\\
If there is no \textbf{blue} vertices then STOP.\\
We take an arbitrary \textbf{blue} vertex  and draw from it all the arcs of the form  for all  enabled in , where . If the vertex  already exists (in any colour), then the newly created arc leads to the existing vertex (we do not create a new one). New vertices are set \textbf{yellow}. After drawing all such arcs we set the vertex   \textbf{grey} (\textit{a final node}).\\
GOTO Step 2.\\ 
\item Step 2. \textit{Coverability checking}\\
If there is no \textbf{yellow} vertices GOTO Step 1.\\
We take an arbitrary \textbf{yellow} vertex  and check for any of the paths from  to  whether a vertex  such that  lies on the path. If such a vertex exists then every coordinate of the marking  greater than the corresponding coordinate of the marking  changes to . Finally we set the vertex  \textbf{blue}.\\
GOTO Step 2.\\ 
\end{itemize}
\begin{example}
\begin{figure}[h]
\centering
\includegraphics[width=1\textwidth]{fig_c.eps}
\caption{A p/t-net and its coverability graph}
\label{FigC}
\end{figure}
\label{e_cov}
Let us look at the Example of Figure \ref{FigC}. A p/t-net and stages of the construction of its coverabilty graph are presented there.
\end{example}
\mbox{ }\\
\textbf{Remark:} A coverability graph is always finite. The proof is based on two facts: the Dickson's Lemma (Lemma~\ref{l26}) and the monotonicity property (Fact \ref{f27}).
\\ 
\subsection{Reachability and Coverability Problems}


Let us now recall very famous decision problems concerning Petri nets, namely the Reachability Problem and the Coverability Problem.
\\ \\
\textbf{Reachability Problem}
\\
\indent\textbf{\emph{Instance:}} P/t-net , and a marking .\\
\indent\textbf{\emph{Question:}} Is  reachable in S?
\\ \\
\textbf{Coverability Problem}
\\
\indent\textbf{\emph{Instance:}} P/t-net , and a marking .\\
\indent\textbf{\emph{Question:}} Is  coverable in S?
\\ \\
\textbf{Remark:} It is well known that the above problems are decidable (coverability: Karp/Miller \cite{KarpMiller}, Hack \cite{Hack}; reachability: Mayr \cite{Mayr}, Kosaraju \cite{Kosaraju}). 


\subsection{Three Kinds of Persistence}


The notion of persistence is one of the classical notions in concurrency theory. The notion is recalled in \cite{BarOch} (named in the sequel e/e-persistence). Some of its generalizations: l/l-persistence and e/l-persistence are also introduced there.
\\ \\
\textbf{Note on terminology}\\
The notion of persistence in its classical meaning is a property of nets. The definition of \cite{LandRob} involves the entire concurrent system. 
If we choose to define the concept of persistence starting from actions by markings, ending with whole nets, the classic definition can be interpreted in two ways. Namely, one can consider concepts of \emph{persistence} and \emph{nonviolence}. An extensive discussion on the links between persistence and nonviolence can be found in \cite{Koutny}. In the context of \cite{BarMikOch} and  \cite{Koutny} it seems that it would be more appropriate to use the notion of nonviolence instead of using the concept of persistence. However, because our paper is an extension of \cite{BarOch}, we decided to stick to the concept of persistence.


\newpage
Let us sketch the notions of e/e-persistence, l/l-persistence and e/l-persistence informally. The classical e/e-persistence means "no action can disable another one", the l/l-persistence means "no action can kill another one" and the e/l-persistence means "no action can kill another enabled one". Let us go on to formal definitions.

\begin{definition}[Three kinds of persistence]
\label{d31}
Let  be a place/transition net. \\
If  
\begin{itemize}
\item , then  is said to be \emph{e/e-persistent};
\item , then  is said to be \emph{l/l-persistent};
\item , then  is said to be \emph{e/l-persistent}.
\end{itemize}\mbox{}\\
The classes of e/e-persistent (l/l-persistent, e/l-persistent) p/t-nets will be denoted by ,  and , respectively.
\end{definition}
In \cite{BarOch} one can find a proof of the following theorem:
\begin{theorem}\mbox{}\\
\label{t_hier}
The three classes of persistent place/transition nets
form an increasing hierarchy: .
\end{theorem}
\begin{figure}[h]
\centering
\includegraphics[width=0.1\textwidth]{fig_a.eps}
\caption{A hierarchy of persistent nets}
\label{FigA}
\end{figure}
\begin{example}
\label{ex_fig_e_d}
To see the strictness of the above inclusion, let us look at the Examples of Figure \ref{Fig_E} and \ref{Fig_D} (derived from \cite{BarOch}).
\end{example}
\begin{figure}[h]
\centering
\includegraphics[width=0.2\textwidth]{fig_e.eps}
\caption{The net is live, so , but not  }
\label{Fig_E}
\end{figure}
\begin{figure}[h]
\centering
\includegraphics[width=0.7\textwidth]{fig_d.eps}
\caption{The transition  kills  undirectly (because in the current marking, the transition  is not enabled), so the net is , but not  }
\label{Fig_D}
\end{figure}
It is also shown in \cite{BarOch} that the following decision problems are decidable:
\\
\\
\textbf{\emph{Instance:}} A p/t-net  \\
\textbf{\emph{Questions:}}\\
\indent\textbf{EE Net Persistence Problem:} Is the net S e/e-persistent?\\
\indent\textbf{LL Net Persistence Problem:} Is the net S l/l-persistent?\\
\indent\textbf{EL Net Persistence Problem:} Is the net S e/l-persistent?\\
\\ 
\newpage
The proofs of decidability of the above problems need to put into work a very efficient result of Valk/Jantzen \cite{ValkJantzen} and benefit from the decidability of reachability problem (more specifically - decidability of the Set Reachability Problem for rational convex sets).
\\
The alternative proof of Theorem \ref{t418} uses exactly the same proving technique as the proofs of decidability of the persistence problems mentioned above.

\section{Properties of e/l-persistence}

\subsection{Hierarchy of e/l-persistence}

In the previous section we defined three kinds of persistence. Now, we extend the hierarchy mentioned above with an infinite hierarchy of e/l-persistent steps.

\begin{definition}[E/l-persistent steps - an infinite hierarchy]
\label{d41}\\
Let  be a p/t-net, let  be a marking.
We call a step :

\begin{itemize}
\item \emph{e/l-0-persistent} iff it is \emph{e/e-persistent} (the execution of an action a does not disable any other action);
\item \emph{e/l-1-persistent} iff  (the execution of an action a pushes the execution of any other enabled action away for at most 1 step);
\item \emph{e/l-2-persistent} iff  (the execution of an action a pushes the execution of any other enabled action away for at most 2 steps);
\\
\indent\ldots

\item \emph{e/l-k-persistent} for some  iff  (the execution of an action a pushes the execution of any other enabled action away for at most  steps);
\\
\indent\ldots
\item \emph{e/l--persistent} iff  (the execution of an action a pushes the execution of any other enabled action away).
\end{itemize}
\end{definition}

\textbf{Remark:} Note that e/l--persistent steps are exactly e/l-persistent steps.
\\ \\
Directly from Definition \ref{d41} we get the
\begin{fact}
\label{f42}
Let  be a p/t-net, let  be a marking. If the step  is e/l--persistent for some , then it is also e/l--persistent for every .
\end{fact}

\begin{definition}
\label{d43}
Let  be a p/t-net,  be a marking and .
Marking  is \emph{e/l-k-persistent} iff for every action  that is enabled in  the step  is \emph{e/l-k-persistent}.
P/t-net  is \emph{e/l-k-persistent} iff every marking reachable in S is \emph{e/l-k-persistent}.
We denote the class of e/l--persistent p/t-nets by .
\end{definition}

\begin{example}
\label{e_f}
\begin{figure}[h]
\centering
\includegraphics[width=0.2\textwidth]{fig_f.eps}
\caption{A p/t-net (for Ex.\ref{e_f}) that is not e/l-k-persistent for any }
\label{Fig_F}
\end{figure}

Let us look at the example of Fig. \ref{Fig_F}. Both actions  and  are enabled in the initial marking. After the execution of the action , the action  is never enabled again, and after the execution of the action , the action  is never enabled again. So the net can not be e/l-k-persistent for any natural number k.
\end{example}

\begin{example}
Let us look at the example of Fig. \ref{Fig_E}. The net is not e/l-0-persistent but it is e/l-1-persistent.
\end{example}
\begin{example}
\label{e411}

\begin{figure}[h]
\centering
\includegraphics[width=0.4\textwidth]{fig2.eps}
\caption{A p/t-net (for Ex.\ref{e411}) that is e/l-3 persistent but not e/l-2 persistent}
\label{Fig2}
\end{figure}

Let us look at the example of Fig. \ref{Fig2}. The only possible situation for temporary disabling an action by another one is the execution of  that disables . And then  could be enabled again after the execution of the sequence , so after 3 steps. Hence, the net is e/l-3-persistent, and obviously not e/l-2-persistent.
\end{example}
The direct conclusion from Fact \ref{f42} and Definition \ref{d43} is as follows:
\begin{fact}
\label{f44}
Let  be a p/t-net,  be a marking, and .
If the marking  is e/l--persistent, then it is also e/l--persistent for every .
If the net S is e/l--persistent, then it is also e/l--persistent for every .
\end{fact}
\mbox{ }\\
\textbf{Remark:} Based on this Fact \ref{f44} we can extend the existing hierarchy of persistent nets as shown in Figure \ref{FigB}.
\begin{figure}[h]
\centering
\includegraphics[width=0.8\textwidth]{fig_b.eps}
\caption{A hierarchy of persistent nets - an extension}
\label{FigB}
\end{figure}

\subsection{Related decision problems}

\subsubsection{\textbullet\ EL-k Step Persistence Problem \\and EL-k Marking Persistence Problem}\mbox{ }\\ \\
Let  be a fixed natural number. Now we can formulate basic problems regarding the concept of e/l-k-persistence. \\ \\
\newpage

\mbox{ }\\
The first problem is as follows:
\\ \\
\textbf{EL-k Step Persistence Problem}
\\
\indent\textbf{\emph{Instance:}} P/t-net S, marking , action  enabled in .\\
\indent\textbf{\emph{Question:}}Is the step  e/l-k-persistent?

\begin{theorem}
\label{t45}
The EL-k Step Persistence Problem is decidable (for any ).
\end{theorem}

\begin{proof}
An algorithm to check if a step  is e/l-k-persistent (for some ) for a given net :\\
Let us build the part of the depth of k+1 (we call it the (k+1)-component) of the reachability tree of , where  is a marking obtained from  by execution of . The step  is e/l-k-persistent if for every action , such that  and  is enabled in , there is a path in the (k+1)-component of the reachability tree of  containing an arc labeled by . 	
\end{proof}\mbox{ }\\
Let us introduce another problem:
\\ \\
\textbf{EL-k Marking Persistence Problem}
\\
\indent\textbf{\emph{Instance:}}P/t-net , marking .\\
\indent\textbf{\emph{Question:}}Is the marking  e/l-k-persistent?

\begin{theorem}
\label{t46}
The EL-k Marking Persistence Problem is decidable \\
\indent\indent\indent (for any .
\end{theorem}

\begin{proof}
For every action  that is enabled in a marking , we check if a step  is e/l-k-persistent (for some ) for a given net , using the algorithm of Theorem \ref{t45}. 	
\end{proof}



\subsubsection{\textbullet \ EL-k Net Persistence Problem}\mbox{ }\\ 
\\
Let us consider the following problem:
\\ \\
\textbf{EL-k Net Persistence Problem}
\\
\indent\textbf{\emph{Instance:}}P/t-net .\\
\indent\textbf{\emph{Question:}}Is the net S e/l-k-persistent?
\\
\\To solve this problem we must prove a set of auxiliary facts.
\\ \\
From this moment, let  be an arbitrary p/t-net.
\\ \\
Let us define the following set of markings:\\
- the set of markings enabling actions  and  simultaneously. \\ 
\\
Let us define , the minimum marking enabling actions  and  simultaneously: if  then  else  (for .\\
Note that .

\subsubsection{\textbullet \ Mutual Enabledness Reachability Problem}\mbox{ }\\
\\
Let us formulate an auxiliary problem:
\\ \\
\textbf{Mutual Enabledness Reachability Problem}
\\
\indent\textbf{\emph{Instance:}}P/t-net , actions .\\
\indent\textbf{\emph{Question:}}Is there a marking  such that  and  ? \\
\indent(Is there a reachable marking  such that \\ \indent actions  and  are both enabled in ?)
\\

\begin{theorem}
\label{t413}
The Mutual Enabledness Reachability Problem is decidable.
\end{theorem}

\begin{proof}
Let . We build a coverability graph for the p/t-net S. We check whether in the graph exists a vertex corresponding to an -marking  such that  covers . If so, then actions  and  are simultaneously enabled in some reachable marking of the net S. Otherwise, those transitions are never enabled at the same time. 	
\end{proof}
Let  be the set of minimal (wrt ) reachable markings of the net S. As members of  are incomparable, the set   is finite, by Lemma \ref{l26}.\\ \\
Le us denote by  the set of all reachable markings of the net S enabling actions  and  simultaneously: . \\ \\
Let   be a set of all minimal reachable markings of the net S enabling action  and  simultaneously.


\subsubsection{\textbullet \ Results of Valk and Jantzen}\mbox{ }\\
\\
In order to construct the set , we put into work the theory of residue sets of Valk/Jantzen \cite{ValkJantzen}.

\begin{definition}[Valk/Jantzen \cite{ValkJantzen}]
A subset  has property   if and only if the problem "Does   intersect ?" is decidable for any  -vector .
\end{definition}

\begin{theorem}[Valk/Jantzen \cite{ValkJantzen}]
\label{twValk}
Let  be a right-closed set. Then the set  is effectively computable if and only if  has property .
\end{theorem}

\subsubsection{\textbullet \ Set Reachability Problem}\mbox{ }\\
\\
We also use the fact of decidability of the Set Reachability Problem for rational convex sets (Def. \ref{def_mon},\ref{def_clos}), proved in  \cite{BarOch}.
\\ \\
\textbf{Set Reachability Problem}
\\
\indent\textbf{\emph{Instance:}}P/t-net  and a set .\\
\indent\textbf{\emph{Question:}}Is there a marking , reachable in S? \\

\begin{theorem}
\label{tplus}
The Set Reachability Problem is decidable for rational convex sets in p/t-nets.
\end{theorem}
The Set Reachability Problem is a generalization of the classical Marking Reachability Problem. The proof uses decidability of the Reachability Problem.
\subsubsection{\textbullet \ Minimal reachable markings enabling two actions simultaneously}\mbox{ }\\ 
\\
Now we are ready to prove:

\begin{proposition}
\label {p414}
The set  can be effectively constructed for a given net  .
\end{proposition}

\begin{proof}
Let us take the right closure  of the set .
\\Note that . To show that the set of minimal elements of the set  is effectively computable, it is enough to demonstrate that the set   has the property RES (i.e. for any  -vector  the problem "?" is decidable) and apply Theorem \ref{twValk}. \\
Let , where . Let us notice, that   is a convex set, hence rational (Proposition \ref{p26}). The set  is also a rational convex set. As an intersection of convex rational sets, the set  is convex and rational (Theorem \ref{twGins}) as well.\\
Hence, putting into work decidability of the Set Reachability Problem for rational convex sets (Theorem \ref{tplus}) we decide whether any marking from the set  is reachable in S. Therefore, we can decide whether the set  is nonempty. (It is the case when at least one marking from the set  is reachable in S.) Let us notice that the set   is nonempty if and only if the set  is nonempty. That is why the set   has the property RES, and consequently the set  is effectively computable by Theorem \ref{twValk}. 	
\end{proof}

\begin{example}
\label{e415}

\begin{figure}[h]
\centering
\includegraphics[width=0.4\textwidth]{fig4.eps}
\caption{A p/t-net for Ex.\ref{e415}.}
\label{Fig4}
\end{figure}


The set of all minimal reachable markings of the net depicted in Figure \ref{Fig4} enabling action  and  simultaneously, is .
\end{example}

\begin{proposition}
\label{p416}
If there exists a marking   such that the execution of an action  in  pushes the execution of an action  away for more than  steps (for some ), then there exists some minimal marking   such that the execution of an action  in  pushes the execution of an action  away for more than  steps, too.
\end{proposition}

\begin{proof}
Let  be a marking, such that the execution of an action  in  pushes the execution of an action  away for more than k steps (for some ). Let  such that . Such a marking has to exist. Suppose that there is a string ,  such that . Then obviously also  (from the monotonicity property - Fact \ref{f27}). We obtain a contradiction. Hence, the execution of an action  in  postpones the execution of  for more than k steps. 	
\end{proof}

\subsubsection{\textbullet \ EL-k Transition Persistence Problem and EL-k Net Persistence Problem}\mbox{ }\\
\\
Now, we are ready to introduce the following problem:
\\ \\
\textbf{EL-k Transition Persistence Problem}
\\
\indent\textbf{\emph{Instance:}}P/t-net , ordered pair , .\\
\indent\textbf{\emph{Question:}}Is there a reachable marking  such that  \\
\indent\indent \indent\indent?
\\
\indent\indent \indent\indent(Does  postpone  for more than k steps?)

\begin{theorem}
\label{t417}
The EL-k Transition Persistence Problem is decidable.
\end{theorem}

\begin{proof}
We introduce an algorithm of deciding if an action  pushes the execution of an action  away for more than k steps in some reachable marking .
\begin{enumerate}
\item We check whether both actions  and  are enabled in some reachable marking (using decidability of Mutual Enabledness Reachability Problem).
\begin{enumerate}
\item If not, we answer NO.
\item Otherwise:
\begin{enumerate}
\item We build the set . This set can be effectively computed by Proposition \ref{p414} using Valk/Jantzen algorithm.
\item For all markings :\\
. \\
We build an initial part of the depth of k+1 (the (k+1)-component) of the reachability tree of . If the piece has an edge labeled by , we answer NO. Otherwise we answer YES. 	
\end{enumerate}
\end{enumerate}
\end{enumerate}
\end{proof}
And now the proof of decidability of the EL-k Net Persistence Problem is ready.

\begin{theorem}
\label{t418}
The EL-k Net Persistence Problem is decidable (for any ).
\end{theorem}

\begin{proof}
S is e/l-k-persistent iff the algorithm solving EL-k Transition Persistence Problem answers NO for all ordered pairs , .
\end{proof}
\vspace*{-1cm}
\begin{example}
\label{e_fig_g}
\begin{figure}[h]
\centering
\includegraphics[width=0.4\textwidth]{fig_g.eps}
\caption{2-component of the reachability tree of the net of Figure \ref{Fig_E}.}
\label{Fig_G}
\end{figure}

Let us check whether the action  of the net S of Figure \ref{Fig_E} postpones the action  for more than 1 step.\\
Actions  and  are both enabled in the initial marking. \\
The set  consists of a single marking . 
We take . We build a 2-component of the reachability tree of the net . The tree is depicted in Figure \ref{Fig_G}. The tree has an edge labeled by  so the action  does not postpone the action  for more than 1 step.
\end{example}


\subsubsection{\textbullet \ EL-k Transition Persistence Problem - an alternative approach}\mbox{ }\\
\\
In order to show decidability of the EL-k Net Persistence Problem we can use the technique used for proving decidability of LL Net Persistence Problem and EL Net Persistence Problem, presented in \cite{BarOch}.
\\ \\
Again, we deal with the EL-k Transition Persistence Problem, crucial for the proof. We show an alternative proof of decidability of that problem.


\mbox{ }\\
\textbf{EL-k Transition Persistence Problem}
\\
\indent\textbf{\emph{Instance:}}P/t-net , ordered pair , .\\
\indent\textbf{\emph{Question:}}Is there a reachable marking  such that  \\
\indent\indent \indent\indent?
\\
\\
Let us define, in order to reformulate the problem above, the following sets of markings:\\
 - markings enabling  \\
 - markings enabling  \\
 - markings enabling a such that after the execution of a the action b is potentially enabled after at most k steps.
\\
\\
Now we can reformulate the question of the  problem above:
\\
\indent\textbf{\emph{Question:}}Is the set
 reachable in ?
\\
\\
Let us look again at\\ \\
\textbf{Theorem \ref{t418}}. 
\textit{The EL-k Net Persistence Problem is decidable (for any ).}

\begin{proof}
First note that, by the monotonicity property (Fact \ref{f27}), the set \newline is convex, thus rational (by Proposition \ref{p26}). The rational expressions for  and  are   and . Clearly, the set  is right-closed, by the monotonicity property. We shall prove that it has the property RES. Namely,  ()intersects  if and only if  (i.e.  is enabled in ) and there is a path in the reachability tree, limited to (k+1) first levels, of the net , where  is an  -marking obtained from  by the execution of , containing an arc labelled by . It is obviously decidable. Hence, the set  has the property RES, thus (by Theorem \ref{twValk}) the set  is effectively computable. As  is right-closed, we get the rational expression for it: . Finally, using Theorem \ref{twGins} of Ginsburg/Spanier \cite{GinsburgSpanier}, we compute rational expression for  and Theorem \ref{twValk} yields decidability of the problem. 	
\end{proof}

\subsection{Collapsing of the hierarchy of e/l-persistence}


\subsubsection{\textbullet \ k-enabledness}\mbox{ }\\
\\
Let us recall the well-known fact, that follows from the Dickson's Lemma (Lemma~\ref{l26}).

\begin{fact}
\label{f47}
Every infinite sequence of markings contains an infinite increasing (not necessarily strictly) subsequence of markings.
\end{fact}
Recall also that p/t-nets have the monotonicity property - Fact \ref{f27}.

\mbox{ }\\
Let us define the notion of k-enabledness.
\begin{definition}[k-enabledness]
\label{d48}
Let  be a p/t-net, let  be a marking. For  we say that the action  is \emph{k-enabled} in the marking  if and only if , such that .
\end{definition}
\newpage
Now, we can show:

\begin{lemma}
\label{l49}
Let  be a p/t-net. For an arbitrary  there exists a natural number , such that in every marking  the transition  is -enabled or it is dead.
\end{lemma}
\begin{proof}
Suppose that the lemma does not hold for some action . It means that for each  there is a marking  such that  is not k-enabled but not dead. This means that  is -enabled for some . Thus, there exists  an infinite sets of markings  and integers , such that the action  is live in each marking  and it is not -enabled in  for all .
Let us choose (by Fact \ref{f47}) an infinite increasing sequence of markings .
Since the action  is live in , it is k-enabled in , for some . As the strictly increasing sequence  is infinite,  for some j. By the monotonicity property (Fact \ref{f27}), the action  is k-enabled, hence  -enabled in the marking . Contradiction.	
\end{proof}
\mbox{ }\\
\textbf{Remark:} Note that the proof of Lemma \ref{l49} is purely existential, it does not present any algorithm for finding k.
\\
\\
Now, we are ready to formulate the main theorem of the section:

\begin{theorem}
\label{t410}
If a p/t-net is e/l-persistent, then it is e/l--persistent for some .\\
In words: Whenever an action is disabled by another one, it is pushed away for not more than -steps.
\end{theorem}

\begin{proof}
If the net is e/l-persistent, then no action kills another enabled one. From the Lemma \ref{l49} we know, that if an action  is not dead then it is -enabled.
Let us take , for the numbers  from the Lemma \ref{l49}. One can see that every action in the net that is not dead, is K-enabled. Thus, the execution of any action may postpone the execution of an action  for at most K steps. 	
So we have the implication: if a p/t-net is e/l-persistent, then it is e/l-K-persistent, for K defined above. 	
\end{proof}
\mbox{ }\\
\textbf{Remark:} As the proof of Lemma \ref{l49} explicitly uses the monotonicity property of p/t-nets, the Theorem \ref{t410} holds only for nets satisfying this property. The following example shows that Theorem \ref{t410} does not hold for nets without the monotonicity property (for instance, inhibitor nets).

\begin{example}
\label{e412}

\begin{figure}[h]
\centering
\includegraphics[width=0.5\textwidth]{fig3.eps}
\caption{An inhibitor net for Ex.\ref{e412}}
\label{Fig3}
\end{figure}

Let us look at the example of Fig. \ref{Fig3}. We can see an inhibitor net and its computation such that for every  one can push an action away for a distance greater than k steps.\\
This net is live, hence it is e/l-persistent, but it is not e/l-k-persistent for any .\\
In the infinite computation  the first  pushes  away for 1 step, the second - for 2 steps and every k-th  - for k steps.
\end{example}

\subsubsection{\textbullet \ Collapsing of the hierarchy - an effective proof}\mbox{ }\\
\\
Finally, let us recall other decision results of \cite{BarOch}:
\\ \\
\textbf{Transitions Persistence Problems}
\\
\indent\textbf{\emph{Instance:}}P/t-net , and transitions .\\
\indent\textbf{\emph{Questions (informally):}}\\
\indent \indent\indent EE-Persistence Problem: Does  disable an enabled ?\\
\indent \indent\indent LL-Persistence Problem: Does  kill a live ?\\
\indent \indent \indent EL-Persistence Problem: Does  kill an enabled ?\\
\\
From \cite{BarOch} we know that the problems are decidable.

\begin{theorem}
\label{t419}
For a given p/t-net  and a pair of transitions  one can calculate a minimum number  such that  postpones an enabled~ for at most  steps (if such a number exists).
\end{theorem}

\begin{proof}\
\begin{itemize}
\item We check whether both actions  and  are enabled in some reachable marking (using decidability of Mutual Enabledness Reachability Problem). If not,  does not exist (actions  and  are never enabled at the same time). \
Otherwise:
\begin{itemize}
\item We ask whether  kills an enabled  (EL-Persistence Problem).\\
If YES then  does not exist ( kills )\\
else:\
\begin{itemize}
\item We compute the set .This set can be effectively computed by Proposition \ref{p414} using Valk/Jantzen algorithm.\
\item We build the initial part of reachability tree of the net  as long as from every  we get a marking  with the property that a path leads to a~vertex  (it can be an empty path) such that . Clearly, such part of the tree is finite, as we get the whole  and for any  a finite path leading from  to a vertex  such that . The maximum length of such paths is the desired number . 	
\end{itemize}
\end{itemize}
\end{itemize}
\end{proof}

\begin{theorem}
\label{t420}
If a p/t-net  is e/l-persistent, then it is e/l--persistent for some  and such a  can be effectively computed.
\end{theorem}

\begin{proof}
For every pair  of transitions we find  defined above. The number we are looking for is . 	
\end{proof}
\mbox{ }\\ \\
We established that an action can not postpone another action (without killing it) indefinitely (Theorem \ref{t410}). We proved, that if a p/t-net is e/l-persistent, then it is e/l-k-persistent for some . We showed that such a k exists and we present any algorithm for finding this k. \\ \\

\section{Conclusions}

It is shown in \cite{BarMikOch} that if we change the firing rule in the following way: only e/e-persistent computations are permitted, then we get a new class of nets (we call them \emph{nonviolence nets}) which are computationally equivalent to Turing machines.
We plan to investigate net classes, with firing rules changed (only e/l-k-persistent computations are allowed) and answer the question:
\\ \\
\textbf{Question 1:}\\
What is the computational power of nets created this way?
\\ \\
In this paper, we have investigated the hierarchy of persistence in p/t-nets. We would like to study the hierarchy of e/l-k-persistence in some extensions of p/t-nets, for instance nets with read arcs and reset nets. 
All results of the paper hold for nets with read arcs (\cite{MontanariRossi}), as they can be simulated by classical Petri nets with self-loops with the same reachability set (but with distinct step semantics).
On the contrary, only Lemma \ref{l49} and Theorem \ref{t410} hold (with the same proof) for other extended Petri nets posessing the monotonicity property (e.g. reset, double, transfer nets), but the results supported with the fact of decidability of the Reachability Problem (Proposition \ref{p414}, Theorem \ref{t418}, Theorem \ref{t419}) cannot be applied to those nets, because of undecidability of the Reachability Problem in them (see \cite{Dufourd}).


\bibliography{TCS}{}
\bibliographystyle{plain}

\end{document}
