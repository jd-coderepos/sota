\documentclass[copyright,creativecommons]{eptcs}
\usepackage{breakurl}               \usepackage{amssymb}
\usepackage{graphicx}
\usepackage{extarrows}
\usepackage{theorem}
\usepackage[all]{xy}

\newcommand{\eop}{\hspace*{\fill}}

\newtheorem{definition}{Definition}
\newtheorem{proposition}[definition]{Proposition}
\newtheorem{lemma}[definition]{Lemma}
\newtheorem{theorem}[definition]{Theorem}
{\theorembodyfont{\rmfamily}
  \newtheorem{example}[definition]{\it Example}
  \newtheorem{myproof}{\it Proof.}
}
\renewcommand{\themyproof}{}
\newenvironment{proof}{\begin{myproof}}{\eop\end{myproof}}

\newcommand{\cA}{\mathcal{A}}
\newcommand{\cF}{\mathcal{F}}
\newcommand{\cG}{\mathcal{G}}
\newcommand{\cI}{\mathcal{I}}
\newcommand{\cP}{\mathcal{P}}
\newcommand{\cS}{\mathcal{S}}
\newcommand{\cT}{\mathcal{T}}
\newcommand{\cV}{\mathcal{V}}

\newcommand{\Arity}{\mathit{arity}}
\newcommand{\Root}{\mathit{root}}
\newcommand{\Var}{{\mathcal{V}ar}}
\newcommand{\FVar}{{\mathit{FV}}}
\newcommand{\Pos}{{\mathcal{P}os}}
\newcommand{\posvar}[1]{{\langle}#1{\rangle}}
\newcommand{\Ran}{{\mathcal{R}an}}
\newcommand{\Terms}{{\mathcal{T}}}

\newcommand{\Constrained}[1]{[\![ #1 ]\!]}
\newcommand{\TwC}[2]{{#1}_{[#2]}}

\newcommand{\Fol}{{\mathcal{F}\!\mathit{ol}}}
\newcommand{\Suffix}{{\mathcal{S}\!\mbox{\it uffixes}}_{<}}
\newcommand{\Eval}{\mathsf{eval}}

\newcommand{\Bothsquares}{\scalebox{.5}[.8]{}}
\newcommand{\Replace}[1]{{#1}|_{\Bothsquares}}
\newcommand{\Subpatterns}{{\mathit{Patterns}_{\lhd}}}
\newcommand{\Inst}{{\mathit{LessGeneralized}}}
\newcommand{\Maximal}[1][]{{\mathit{Least}_{\sqsubseteq #1}}}
\newcommand{\Labels}{{\mathit{Labels}}}
\newcommand{\Initials}{\widetilde{\Labels}}

\newcommand{\EQsymbol}{\approx}
\renewcommand{\equiv}{=}

\title{On Constructing Constrained Tree Automata Recognizing Ground Instances of Constrained Terms}
\def\titlerunning{On Constructing CTAs Recognizing Ground Instances of
Constrained Terms}
\author{
Naoki Nishida
\institute{Graduate School of Information Science, Nagoya University\\
Furo-cho, Chikusa-ku, 4648603 Nagoya, Japan}
\email{nishida@is.nagoya-u.ac.jp}
\and
Masahiko Sakai
\institute{Graduate School of Information Science, Nagoya University\\
Furo-cho, Chikusa-ku, 4648603 Nagoya, Japan}
\email{sakai@is.nagoya-u.ac.jp}
\and
Yasuhiro Nakano
\institute{Graduate School of Information Science, Nagoya University\\
Furo-cho, Chikusa-ku, 4648603 Nagoya, Japan}
\email{ynakano@trs.cm.is.nagoya-u.ac.jp}
}
\def\authorrunning{N.~Nishida, M.~Sakai, Y.~Nakano}

\begin{document}

\maketitle


\begin{abstract}
An inductive theorem proving method for constrained term rewriting
systems, which is based on rewriting induction, needs a decision
procedure for reduction-completeness of constrained terms. In addition,
the sufficient complete property 
of constrained term rewriting systems enables us
to relax the side conditions of some inference rules in the proving
method. These two properties can be reduced to intersection emptiness
problems related to sets of ground instances for constrained terms. 
This paper proposes a method to construct deterministic, complete,
and constraint-complete constrained tree automata recognizing ground
instances of constrained terms.
\end{abstract}


\section{Introduction}
\label{sec:intro}

The \emph{constrained rewriting} in this paper is a computation model
that rewrites a term by applying a constrained rewriting rule if the
term satisfies the constraint on interpretable domains attached to the
rule~\cite{furuichi,BJ08,sakata,SNS11,BJ12}. 
Proposed a \emph{rewriting induction} method on the constrained
rewriting~\cite{sakata,NNKSS11}, a decision procedure of 
\emph{reduction-completeness} of terms must be extended for
\emph{constrained terms}, terms admitting constraints attached, 
in order to apply the method to mechanical inductive theorem proving,
where a term is said to be reduction-complete~\cite{Fri89} 
if any ground instance of the term is a redex. 
If a constrained term rewriting system is terminating and all terms are
reduction-complete, the rewrite system is said to be sufficient complete,
which is useful to relax the application conditions of inference rules
in the above method~\cite{sakata}.
These properties are proved by using tree automata with constraints,
whose rules contain constraints on interpretable subterms.
More precisely, the properties are reducible to the intersection
emptiness problem of ground instances of terms satisfying constraints
attached to the terms. 

This paper proposes a construction method of \emph{constrained tree automata}
that accept ground instances of constrained term (in
Section~\ref{sec:construction}).
Moreover the obtained tree automata have nice properties:
the \emph{constraint-completeness}~\cite{CTA}, completeness and
determinacy,
where the first property is necessary for proving correctness of the
constructed tree automata, and the next two properties contribute avoiding 
size explosion at the construction of product automata. 

\section{Preliminaries}
\label{sec:preliminaries}

In this section, we briefly recall the basic notions of
terms~\cite{TATA}, constraints over predicate logic~\cite{Logic}, and
constrained tree automata~\cite{CTA}. 



Throughout the paper, we use  as a countably infinite set of
\emph{variables}.
For a \emph{signature}  (a finite set of function symbols with fixed
arities), the set of \emph{terms} over  and    is
denoted by .  
The notation  represents the function symbol  with the arity .
The set  of \emph{ground terms} is abbreviated to
.
The set of variables appearing in term  is denoted by .
A term is called \emph{linear} if any variable appears in the term at
most once. 
The set of \emph{positions} for term  is denoted by :
\begin{itemize}
 \item    for   , and
 \item    for  
       .
\end{itemize}
For terms  and a position  of , the notation  denotes
the term obtained from  by replacing the \emph{subterm  of 
at } with .
For a \emph{substitution} , we denote the \emph{range} of
 by . 



Let  be a signature and  be a set of \emph{predicate symbols}
with fixed arities.
The notation  represents the predicate symbol  with the arity
. 
First-order (quantifier-free) \emph{formulas}  over , , 
and  are defined in BNF as follows:\footnote{ It is possible to allow to introduce quantifiers.
To be more precise, introduction of quantifiers to our setting does not
affect the results. 
Though, for the sake of readability, we do not introduce them here. }

where    and   .
We may omit brackets ``'', ``'' from formulas as usual. 
The set of first-order formulas over , , and 
is denoted by .
For a formula , the notation  represents the set of
variables in .  
A variable-free formula is called \emph{closed}.
A \emph{structure} for  and  
is a tuple    such that the \emph{universe}
 is a non-empty set of concrete values,  and  
with types fnA and 
are interpretations for  and , resp.:
\begin{itemize}
 \item    for   , and
 \item    for   .
\end{itemize}
The interpretation of formulas  w.r.t.\ , denoted by , is defined as usual.
We say that a formula  \emph{holds} w.r.t.\  if .
Formulas in  interpreted by  are called
\emph{constraints} (\emph{w.r.t.\ }).




In the following, we use  and  for signatures,  for a set of
 predicate symbols, and    for a structure
 for  and , without notice. 
Before formalizing constrained tree automata, 
we generalize the interpretation of constraints under terms.
For a sequence  of natural numbers, the notation  denotes
the special variable related to .
We denote the set of such variables by :
  
 .
A formula  in  \emph{holds}
 w.r.t.\ a ground term    if , where  is inductively defined as follows: 
 \begin{itemize}
  \item ,
  \item ,
  \item  if  
	 and    for all variables
	  , and
	 
	, where    and  is the
	substitution ,
  \item otherwise, , and
  \item the relation  is defined as usual for any Boolean
	connective. \end{itemize}
We may omit  from ``'' if it is clear in context.
Note that    does not coincide with 
  for every ground term  and constraint 
(see~\cite{CTA}).


By generalizing \emph{AWEDCs}~\cite{TATA}, \emph{constrained tree
automata} are defined as follows~\cite{CTA}~---~note that
\emph{AWEDCs}~\cite{TATA} are CTAs.
\begin{definition}A \emph{constrained tree automata} (CTA) over ,
 , , and  is a tuple    such that
\begin{itemize}
  \item  is a finite set of \emph{states} (unary symbols), 
  \item  is a finite set of \emph{final} states (i.e., 
	 ), and 
  \item  is a finite set of \emph{constrained transition
	rules}\,\footnote{ We consider transition rules  and
  to be equivalent if  
  and  is semantically equivalent to  (i.e.,
  is valid w.r.t.\ ).}
 of the form
 
   where   
 ,    and  
 .  
 \end{itemize}
 We often omit the arguments of states by writing  instead of
 , and then transition rules are written in the form
 . 
 We also may omit  from . 

The \emph{move relation}  is defined as follows:
   iff  has no nest of state symbols,  
  ,   ,
   ,  
 
 , , 
and   .
\end{definition}
The terminologies of CTAs are defined analogously to those of \emph{tree
 automata}, except for determinism and completeness~---~ is called
 \begin{itemize}
  \item \emph{deterministic} if for every ground term
       , there is at most one state    such that 
        , and
  \item \emph{complete} if for every ground term ,
	there is at least one state    such that 
	 .
 \end{itemize}
Note that the above definition of determinism and completeness for CTAs
is the same as the definition of the properties for AWEDCs (see
\cite{TATA}). 

\begin{example}\label{ex:CTA}
Let   ,  
 ,   , and  be the structure
  for  and  such that   , 
  ,   , and
  are interpreted over integers as usual.
 Consider a CTA   
 over , , , and  where 
 
 The term
  is accepted
 by ~---~we have that
   since both 
 
 
 
 and  hold.
 The term 
 also transitions into , and thus,  is not
 deterministic. 
 On the other hand, the term
  is not accepted by
  since
 . 
 Note that  recognizes the set of instances of the
 \emph{constrained terms} ,
 , and 
 ,  
i.e.,  
   
  (see
 Definition~\ref{def:constrained-term}).

 Consider the CTA    obtained from 
  by replacing 
 with :
 
 The constraints  and  are 
 semantically equivalent, i.e., for all terms  
 , 
 iff . 
 However, this is not the case for similar constraints over fixed terms,
 e.g.,  does not hold, but
  holds.
 Thus,  is accepted by
 , and hence   . 
\end{example}

A CTA    is called
\emph{constraint-complete}~\cite{CTA} if for every ground term  
 and all transition rules 
   with  
  , 
we have that    and   
for all variables  in .
Note that every CTA can be transformed into an equivalent,
deterministic, complete, and constraint-complete CTA~\cite{CTA}. 
Note also that completeness and constraint-completeness are different notions. 

\section{Recognizing Ground Instances of Constrained Terms}
\label{sec:construction}

This section defines \emph{constrained terms} and their instances,
and then proposes a method for constructing a CTA recognizing the set of
ground instances for a given set of constrained terms. 
The method is based on the construction of a \emph{tree automaton}
recognizing the set of ground instances for unconstrained terms, which
is complete and deterministic~\cite[Exercise~1.9]{TATA}.
Accessibility of states is in general undecidable, and thus, it is
difficult to get rid of inaccessible states which affect
the \emph{intersection emptiness problem}. 
In this sense, it is worth developing a construction method that
introduces inaccessible states as little as possible. 
\begin{definition}\label{def:constrained-term}
A \emph{constrained term} is a pair , written as
, of a linear term  
   and a formula  
 . 
The set of \emph{ground instances} of a constrained term 
, denoted by , is defined
 as follows:
 
.
The argument of  is naturally extended to sets 
. \end{definition}

To deal with constrained terms, we consider \emph{constrained
patterns}. 
We introduce wildcard symbols  and 
to denote arbitrary interpretable and un-interpretable terms
resp.
In the following, we denote
 (the set of patterns)
by , 
 (the set of interpretable patterns) by
, and

(the set of un-interpretable patterns) by . 
\begin{definition}
A pair of a ground term  
  and a formula  is called a \emph{constrained
 pattern} if  and    for
 each variable  that occurs in . 

 For a constrained term , 
 denotes the set of constrained terms  where
\begin{itemize}
\item   v\square\blacksquare, and
 \item   .\footnote{  is unique since  is linear. }
\end{itemize}
We extend the domain of  to sets of
 constrained terms:
 
 .
\end{definition}
Roughly speaking, a constrained pattern 
 represents a set of terms obtained by replacing  and
  in  by interpretable and un-interpretable terms,
 resp., such that the constraint obtained by the corresponding
 replacement holds.
\begin{example}
 \label{ex:replace}
Let   ,   , 
and   .
Let symbols , , , and  be 
interpreted by  as zero function and successor function, 
less-or-equal relation, and greater-or-equal relation, resp.
 For   ,

\end{example}

Next, we define a function to augment their subterms to constrained patterns. 
\begin{definition}
For a set  of constrained patterns, 
the set  of proper subterms for constrained patterns in
  is defined as follows: 

\end{definition}
\begin{example}
For  in Example~\ref{ex:replace}, 

 .
\end{example}

We define a quasi-order over constrained patterns 
that represents an approximation relation.
\begin{definition}
 \label{sqsubseteq}
 A quasi-order  over terms in
  is inductively defined as
 follows: 
 \begin{itemize}
  \item    for  
	,
  \item    for  
	, and
  \item    if 
         for all     .
 \end{itemize}


Abusing notations, we also define a quasi-order  over
 formulas in  as follows:
  
iff    and  is valid w.r.t.\ .


A quasi-order  over constrained patterns is defined as
 follows:
   
 iff    and   .
For constrained patterns  and , we say
 that  is \emph{more general} than 
 ( is \emph{less general} than ) if
   . 

We use  for equality part of , and
 for strict part of .
\end{definition}

Next, to compute more concrete patterns, we define an operation 
for constrained patterns.
\begin{definition}
We define a binary operator  over
  inductively as follows:
\begin{itemize}
 \item          for 
        , 
 \item        
        for   , and
 \item     . 
\end{itemize}
\end{definition}

In the following, we define the set of constrained patterns used for
labels of states.
\begin{definition}
For a set  of constrained patterns,  is the smallest set
 satisfying the following:
 \begin{itemize}
  \item 
	 ,
  \item \phi'\cS
	 ,
	and
  \item  if
	,   
	and  is satisfiable w.r.t.\ .
 \end{itemize}
Note that we do not distinguish terms that enjoy , and that
  is finite up to . 
\end{definition}
\begin{example}
 \label{ex:Inst}
Consider  and  in Example~\ref{ex:replace}.
The set 
 contains the following in addition to
 : 

The relation between the constrained patterns in 
 w.r.t.\  is illustrated as follows:

where  denotes .
\end{example}

Finally, we show a construction of a CTA recognizing .
We prepare two kinds of states  and 
for each pattern to distinguish whether the term with the state
satisfies the constraint in the corresponding constrained term.
In the following, we use  to denote  or , and,
given an unconstrained pattern  and a set  of constrained
patterns, we denote the set of \emph{least general} constrained patterns
in  which are more general than  by :

Now we give an intuitive outline of the construction.
\begin{description}
 \item[Final states]
	    For a pattern  capturing an instance of 
	    in ,
	    we add the state  to  if  also captures
	    an instance of a non-variable proper subterm of
	     in ;
	    otherwise, we add  and
	     where  is in
	     and
	    , resp.
 \item[States]
	    In addition to , for a pattern  capturing an
	    instance of a non-variable proper subterm of
	     in , we add  to ;
	    for arbitrary terms in  and
	    , we add  and
	     to , resp.
 \item[Transition rules]
	     transitions to 
	    a state by considering the following properties of the least
	    general pattern  which is more general than
	    : 
	    \begin{enumerate}
	     \renewcommand{\labelenumi}{(\alph{enumi})}
	     \renewcommand{\theenumi}{(\alph{enumi})}
	     \item\label{condition-1}
		  whether the pattern  also matches a non-variable
		  \emph{proper} subterm of some initial constrained term 
		    ,
	     \item\label{condition-2}
		  whether the pattern  also matches an instance of
		   in , and
	     \item\label{condition-3} 
		  whether  is in . 
	    \end{enumerate}
	    The state 
	    transitions to is decided as follows:
	    \begin{center}
	     \begin{tabular}{c|ccc|c}
	      state
	      & \ref{condition-1} 
	      & \ref{condition-2} 
	      & \ref{condition-3}
	      & the transition rule is in \\
	      \hline
 & yes & no & --- &  \\
 & yes & yes & --- &  \\
 & no & no & yes &  \\
 & no & yes & yes &
	       \\
 & no & no & no &  \\
 & no & yes & no &
	       \\
	     \end{tabular}
	    \end{center}
\end{description}
This approach to the construction is formalized as follows.
\begin{definition}
 \label{def:G(T)}
 For a finite set  of constrained terms, 
 we prepare the set  of constrained patterns whose term parts
 are used as labels for states:
 
 The subset  of constrained patterns that match
 elements of  
 is defined as follows:
 
We prepare the set  of candidates for states as follows:

 Then, we define a CTA    such that
5pt]
  Q & = & 
  Q_f 
  \cup
  \{ q_u \in Q_0 \mid \tilde{q}_u \not\in Q_f \}
  \cup 
  \{ q_\square, q_\blacksquare\}
  \
where

\end{definition}
Note that for any pattern   , 
   iff    or   , and
thus,    which makes \ref{condition-1} true implicitly
holds in the conditions of  and
.
Note also that the constructed transition rules are not always
optimized via the construction in Definition~\ref{def:G(T)}.\footnote{ As in the case of unconstrained tree automata~\cite{TATA}, 
a state  is not accessible if there is no transition rule of the
form .
Thus, such a state  and all the transition rules having  can be
removed from .}
\begin{theorem}
 \label{th:correctness}
The CTA  constructed in Definition~\ref{def:G(T)} is a
 deterministic, complete, and constraint-complete CTA such that 
  .
\end{theorem}
\begin{proof}
Due to the construction of constrained patterns used for states, 
for   , 
all of the following hold:
\begin{itemize}
 \item    and   
       for any variable   , 
 \item  is
       valid w.r.t.\ , and
 \item if  is a least general in , then
        is not satisfiable w.r.t.\  for any
       least general constrained pattern  in 
       such that   .
 \end{itemize}
It follows from both the first property above and the construction of
 transition rules that  is constraint-complete.  
It follows from the second and third properties above that  is
 complete and deterministic. 
From these properties, every given term transitions to a unique state
 that keeps the structure of the given term as much as possible in the
 sense of the patterns to be considered. 
 In constructing transition rules, we add constraints to transition
 rules so as to transition to a final state if a given initial ground
 term is an instance of a constrained term in . 
 For these reasons, it is clear that   .
\end{proof}

\begin{example} 
 \label{ex:complete}
Consider ,  and  in Example~\ref{ex:replace},
  , and
  .
Then, we have that 
5pt]
   \Subpatterns(\Replace{T})
   =
   \{ \mathsf{s}(\square) \} \12pt]
   \Initials(T)
   =
   \Replace{T}
   \cup
   \{
   \TwC{\mathsf{f}(\mathsf{s}(\square))}{\posvar{1}\leq\mathsf{0}\wedge\posvar{1.1}\geq\mathsf{0}},
       \TwC{\mathsf{f}(\mathsf{s}(\square))}{\posvar{1}\leq\mathsf{0}\wedge\neg
       (\posvar{1.1}\geq\mathsf{0})},
       \TwC{\mathsf{f}(\mathsf{s}(\square))}{\neg
       (\posvar{1}\leq\mathsf{0}) \wedge \posvar{1.1}\geq\mathsf{0}}
       \} \\
 \end{array}

 \Delta =
 \left \{
 \begin{array}{l@{~~~~}l@{~~~~}l@{~~~~}l}
  \mathsf{0} \to q_\square,
   &
  \mathsf{s}(q_\square) \to q_{\mathsf{s}(\square)},
   &
   \mathsf{f}(q_\square) \xrightarrow[]{\posvar{1} \leq \mathsf{0}} 
   \tilde{q}_\blacksquare,
   &
   \mathsf{f}(q_{\mathsf{s}(\square)}) \xrightarrow[]{\posvar{1}
   \leq \mathsf{0} \wedge \posvar{1.1} \geq \mathsf{0}} 
   \tilde{q}_\blacksquare,
   \\
  & 
  \mathsf{s}(q_\blacksquare) \to q_\blacksquare,
   &
   \mathsf{f}(q_\square) \xrightarrow[]{\neg (\posvar{1} \leq \mathsf{0})} 
   q_{\blacksquare},
   &
   \mathsf{f}(q_{\mathsf{s}(\square)}) \xrightarrow[]{\posvar{1}
   \leq \mathsf{0} \wedge \neg (\posvar{1.1} \geq \mathsf{0})} 
   \tilde{q}_\blacksquare,
   \\
  &
  \mathsf{s}(q_{\mathsf{s}(\square)}) \to q_{\mathsf{s}(\square)},
   &
   \mathsf{f}(q_\blacksquare) \to q_\blacksquare,
   &
   \mathsf{f}(q_{\mathsf{s}(\square)}) \xrightarrow[]{\neg
   (\posvar{1} \leq \mathsf{0}) \wedge \posvar{1.1} \geq \mathsf{0}} 
   \tilde{q}_\blacksquare,
   \\
  &
   \mathsf{s}(\tilde{q}_\blacksquare) 
   \to q_\blacksquare,
   &
   \mathsf{f}(\tilde{q}_\blacksquare) \to q_\blacksquare,
   &
   \mathsf{f}(q_{\mathsf{s}(\square)}) \xrightarrow[]{\neg
   (\posvar{1} \leq \mathsf{0}) \wedge \neg (\posvar{1.1} \geq \mathsf{0})} 
   q_{\blacksquare}
   \\ 
 \end{array}
 \right \}
 
\end{example}

\section{Conclusion}
\label{sec:conclusion}

In this paper, we proposed a construction method of deterministic,
complete, and constraint-complete CTAs recognizing ground instances of
constrained terms. 
For the lack of space, we did not describe how to apply it to the
verification of reduction-completeness and sufficient completeness,
while we have already worked for some examples. 

Unlike the case of tree automata, for a state, it is in general
undecidable whether there exists a term reachable to the state, and
thus, the intersection emptiness problem of CTAs is
undecidable in general (see the case of
AWEDC~\cite[Theorem~4.2.10]{TATA}).
For this reason, we will use a trivial sufficient condition that the set
of final states of product automata is empty. 
Surprisingly, this is sometimes useful for product automata. 
To make this approach more powerful, we need to develop a method to
find states that are not reachable from any ground term, e.g., there is
a possibility to detect a transition rule that is never used:
for  in
	 Example~\ref{ex:complete},
we know that in applying this rule, the first argument of 
is always an interpretable term of the form , and thus,
 in the constraint can be replaced by
;
then we can notice that the constraint is unsatisfiable, and thus, this
transition rule is never used. 
Formalizing this observation is one of our future work. 

\bibliographystyle{eptcs}
\begin{thebibliography}{10}
\providecommand{\bibitemdeclare}[2]{}
\providecommand{\surnamestart}{}
\providecommand{\surnameend}{}
\providecommand{\urlprefix}{Available at }
\providecommand{\url}[1]{\texttt{#1}}
\providecommand{\href}[2]{\texttt{#2}}
\providecommand{\urlalt}[2]{\href{#1}{#2}}
\providecommand{\doi}[1]{doi:\urlalt{http://dx.doi.org/#1}{#1}}
\providecommand{\bibinfo}[2]{#2}

\bibitemdeclare{inproceedings}{BJ08}
\bibitem{BJ08}
\bibinfo{author}{Adel \surnamestart Bouhoula\surnameend} \&
  \bibinfo{author}{Florent \surnamestart Jacquemard\surnameend}
  (\bibinfo{year}{2008}): \emph{\bibinfo{title}{Automated Induction with
  Constrained Tree Automata}}.
\newblock In \bibinfo{editor}{Alessandro \surnamestart Armando\surnameend},
  \bibinfo{editor}{Peter \surnamestart Baumgartner\surnameend} \&
  \bibinfo{editor}{Gilles \surnamestart Dowek\surnameend}, editors: {\sl
  \bibinfo{booktitle}{Proceedings of the 4th International Joint Conference on
  Automated Reasoning}}, {\sl \bibinfo{series}{Lecture Notes in
  Computer Science}} \bibinfo{volume}{5195}, \bibinfo{publisher}{Springer}, pp.
  \bibinfo{pages}{539--554}, \doi{10.1007/978-3-540-71070-7\_44}.

\bibitemdeclare{article}{BJ12}
\bibitem{BJ12}
\bibinfo{author}{Adel \surnamestart Bouhoula\surnameend} \&
  \bibinfo{author}{Florent \surnamestart Jacquemard\surnameend}
  (\bibinfo{year}{2012}): \emph{\bibinfo{title}{Sufficient completeness
  verification for conditional and constrained {TRS}}}.
\newblock {\sl \bibinfo{journal}{Journal of Applied Logic}}
  \bibinfo{volume}{10}(\bibinfo{number}{1}), pp. \bibinfo{pages}{127--143},
  \doi{10.1016/j.jal.2011.09.001}.

\bibitemdeclare{misc}{TATA}
\bibitem{TATA}
\bibinfo{author}{Hubert \surnamestart Comon\surnameend}, \bibinfo{author}{Max
  \surnamestart Dauchet\surnameend}, \bibinfo{author}{R\'emi \surnamestart
  Gilleron\surnameend}, \bibinfo{author}{Florent \surnamestart
  Jacquemard\surnameend}, \bibinfo{author}{Denis \surnamestart
  Lugiez\surnameend}, \bibinfo{author}{Christof \surnamestart
  L\"oding\surnameend}, \bibinfo{author}{Sophie \surnamestart Tison\surnameend}
  \& \bibinfo{author}{Marc \surnamestart Tommasi\surnameend}
  (\bibinfo{year}{2007}): \emph{\bibinfo{title}{Tree Automata Techniques and
  Applications}}.
\newblock \bibinfo{howpublished}{Available on:
  \url{http://www.grappa.univ-lille3.fr/tata}}.
\newblock \bibinfo{note}{Release October, 12th 2007}.

\bibitemdeclare{article}{Fri89}
\bibitem{Fri89}
\bibinfo{author}{Laurent \surnamestart Fribourg\surnameend}
  (\bibinfo{year}{1989}): \emph{\bibinfo{title}{A Strong Restriction of the
  Inductive Completion Procedure}}.
\newblock {\sl \bibinfo{journal}{Journal of Symbolic Computation}}
  \bibinfo{volume}{8}(\bibinfo{number}{3}), pp. \bibinfo{pages}{253--276},
  \doi{10.1016/S0747-7171(89)80069-0}.

\bibitemdeclare{article}{furuichi}
\bibitem{furuichi}
\bibinfo{author}{Yuki \surnamestart Furuichi\surnameend},
  \bibinfo{author}{Naoki \surnamestart Nishida\surnameend},
  \bibinfo{author}{Masahiko \surnamestart Sakai\surnameend},
  \bibinfo{author}{Keiichirou \surnamestart Kusakari\surnameend} \&
  \bibinfo{author}{Toshiki \surnamestart Sakabe\surnameend}
  (\bibinfo{year}{2008}): \emph{\bibinfo{title}{Approach to Procedural-program
  Verification Based on Implicit Induction of Constrained Term Rewriting
  Systems}}.
\newblock {\sl \bibinfo{journal}{IPSJ Transactions on Programming}}
  \bibinfo{volume}{1}(\bibinfo{number}{2}), pp. \bibinfo{pages}{100--121}.
\newblock \bibinfo{note}{In Japanese}.

\bibitemdeclare{book}{Logic}
\bibitem{Logic}
\bibinfo{author}{Michael \surnamestart Huth\surnameend} \&
  \bibinfo{author}{Mark \surnamestart Ryan\surnameend} (\bibinfo{year}{2000}):
  \emph{\bibinfo{title}{Logic in Computer Science: Modelling and Reasoning
  about Systems}}.
\newblock \bibinfo{publisher}{Cambridge University Press}.

\bibitemdeclare{article}{NNKSS11}
\bibitem{NNKSS11}
\bibinfo{author}{Naoki \surnamestart Nakabayashi\surnameend},
  \bibinfo{author}{Naoki \surnamestart Nishida\surnameend},
  \bibinfo{author}{Keiichirou \surnamestart Kusakari\surnameend},
  \bibinfo{author}{Toshiki \surnamestart Sakabe\surnameend} \&
  \bibinfo{author}{Masahiko \surnamestart Sakai\surnameend}
  (\bibinfo{year}{2011}): \emph{\bibinfo{title}{Lemma Generation Method in
  Rewriting Induction for Constrained Term Rewriting Systems}}.
\newblock {\sl \bibinfo{journal}{Computer Software}}
  \bibinfo{volume}{28}(\bibinfo{number}{1}), pp. \bibinfo{pages}{173--189},
  \doi{10.11309/jssst.28.1\_173}.
\newblock \bibinfo{note}{In Japanese}.

\bibitemdeclare{inproceedings}{CTA}
\bibitem{CTA}
\bibinfo{author}{Naoki \surnamestart Nishida\surnameend},
  \bibinfo{author}{Futoshi \surnamestart Nomura\surnameend},
  \bibinfo{author}{Katsuhisa \surnamestart Kurahashi\surnameend} \&
  \bibinfo{author}{Masahiko \surnamestart Sakai\surnameend}
  (\bibinfo{year}{2012}): \emph{\bibinfo{title}{Constrained Tree Automata and
  their Closure Properties}}.
\newblock In \bibinfo{editor}{Keisuke \surnamestart Nakano\surnameend} \&
  \bibinfo{editor}{Hiroyuki \surnamestart Seki\surnameend}, editors: {\sl
  \bibinfo{booktitle}{Proceedings of the 1st International Workshop on Trends
  in Tree Automata and Tree Transducers}}, pp.
  \bibinfo{pages}{24--34}.

\bibitemdeclare{inproceedings}{SNS11}
\bibitem{SNS11}
\bibinfo{author}{Tsubasa \surnamestart Sakata\surnameend},
  \bibinfo{author}{Naoki \surnamestart Nishida\surnameend} \&
  \bibinfo{author}{Toshiki \surnamestart Sakabe\surnameend}
  (\bibinfo{year}{2011}): \emph{\bibinfo{title}{On Proving Termination of
  Constrained Term Rewriting Systems by Eliminating Edges from Dependency
  Graphs}}.
\newblock In \bibinfo{editor}{Herbert \surnamestart Kuchen\surnameend}, editor:
  {\sl \bibinfo{booktitle}{Proceedings of the 20th International Workshop on
  Functional and (Constraint) Logic Programming (WFLP 2011)}}, {\sl
  \bibinfo{series}{Lecture Notes in Computer Science}} \bibinfo{volume}{6816},
  \bibinfo{publisher}{Springer}, pp. \bibinfo{pages}{138--155},
  \doi{10.1007/978-3-642-22531-4\_9}.

\bibitemdeclare{article}{sakata}
\bibitem{sakata}
\bibinfo{author}{Tsubasa \surnamestart Sakata\surnameend},
  \bibinfo{author}{Naoki \surnamestart Nishida\surnameend},
  \bibinfo{author}{Toshiki \surnamestart Sakabe\surnameend},
  \bibinfo{author}{Masahiko \surnamestart Sakai\surnameend} \&
  \bibinfo{author}{Keiichirou \surnamestart Kusakari\surnameend}
  (\bibinfo{year}{2009}): \emph{\bibinfo{title}{Rewriting Induction for
  Constrained Term Rewriting Systems}}.
\newblock {\sl \bibinfo{journal}{IPSJ Transactions on Programming}}
  \bibinfo{volume}{2}(\bibinfo{number}{2}), pp. \bibinfo{pages}{80--96}.
\newblock \bibinfo{note}{In Japanese}.

\end{thebibliography}

\end{document}
