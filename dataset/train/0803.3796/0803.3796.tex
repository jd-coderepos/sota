\documentclass{LMCS}

\usepackage{amssymb}
\usepackage[all]{xy}
\usepackage{listings}
\usepackage{stmaryrd}
\usepackage{changebar,xspace,hyperref,enumerate}
\mathcode`<="4268
\mathcode`>="5269
\mathchardef\ls="213C
\mathchardef\gr="213E

\newcommand{\ndi}{\mathbin{\raisebox{0.15ex}{-\hspace{-0.2em}-\hspace{-0.2em}-\hspace{-0.2em}-\hspace{-0.2em}-\hspace{-0.2em}-}\hspace{-0.7em}\succcurlyeq}}

\newcommand{\gfp}{\mbox{\rm gfp}}

\newcommand{\interpretation}[1]{\llbracket #1 \rrbracket_{\delta}}
\newcommand{\unterpretation}[1]{\llbracket #1 \rrbracket_{1}}

\newcommand{\comment}[1]{\hspace{1em}\mbox{[{\small #1}]}}

\newcommand{\bottom}{\perp}

\newcommand{\modality}{\lozenge}

\newcommand{\bigmid}{\; \bigg| \,}

\def\doi{4 (2:2) 2008}
\lmcsheading {\doi}
{1--23}
{}
{}
{Sep.~20, 2007}
{Apr.~09, 2008}
{}   

\begin{document}

\title[Approximating a Behavioural Pseudometric]{Approximating a
Behavioural Pseudometric without Discount for Probabilistic
Systems\rsuper *}
\author[F.~van Breugel]{Franck van Breugel\rsuper a}
\address{{\lsuper a}York University,
         4700 Keele Street, Toronto, M3J 1P3, Canada}
\email{franck@cse.yorku.ca}
\thanks{{\lsuper{a,b}}Supported by NSERC}
\author[B.~Sharma]{Babita Sharma\rsuper b}
\address{{\lsuper b}IBM Toronto Lab, 8200 Warden Avenue, Markham, L6G
1C7, Canada}
\email{babitas@ca.ibm.com}
\author[J.~Worrell]{James Worrell\rsuper c}
\address{{\lsuper c}Oxford University Computing Laboratory,
         Parks Road, Oxford, OX1 3QD, England}
\email{jbw@comlab.ox.ac.uk}
\keywords{probabilistic transition system, behavioural pseudometric, probabilistic bisimilarity, approximation algorithm}
\subjclass{F.3.1, F.3.2}
\titlecomment{{\lsuper*}An extended abstract of this paper has appeared as \cite{BSW07:fossacs}}

\begin{abstract}
\noindent
Desharnais, Gupta, Jagadeesan and Panangaden introduced a family of behavioural
pseudometrics for probabilistic transition systems.  These pseudometrics are a
quantitative analogue of probabilistic bisimilarity.  Distance zero captures
probabilistic bisimilarity.  Each pseudometric has a discount factor, a real 
number in the interval .  The smaller the discount factor, the more 
the future is discounted.  If the discount factor is one, then the future is 
not discounted at all.  Desharnais et al.\ showed that the behavioural 
distances can be calculated up to any desired degree of accuracy if the 
discount factor is smaller than one.  In this paper, we show that the 
distances can also be approximated if the future is not discounted.  A key 
ingredient of our algorithm is Tarski's decision procedure for the first order 
theory over real closed fields.  By exploiting the Kantorovich-Rubinstein 
duality theorem we can restrict to the existential fragment for which more 
efficient decision procedures exist.
\end{abstract}

\maketitle

\section{Introduction}

For systems that contain quantitative information, like, for example, 
probabilities, time and costs, several {\em behavioural pseudometrics\/} 
(and closely related notions) have been introduced (see, for example, 
\cite{BW05:tcs,B98:dc,CB02:emsoft,AHM03:icalp,DCPP05:qapl,DGJP04:tcs,DHW03:concur,GJS90:ifip,GP05a:dcec,MRST06:tcs,Y02:tcs}).  
In this paper, we focus on {\em probabilistic transition systems},
which are a variant of Markov chains.
Desharnais, Gupta, Jagadeesan and Panangaden \cite{DGJP04:tcs}
introduced a family of behavioural pseudometrics for these systems.  
These pseudometrics assign a distance, a real number in the interval ,
to each pair of states of the probabilistic transition system.
The distance captures the behavioural similarity of the states.
The smaller the distance, the more alike the states behave.  The
distance is zero if and only if the states are {\em probabilistic bisimilar}, a
behavioural equivalence introduced by Larsen and Skou \cite{LS91:ic}.

The pseudometrics of Desharnais et al.\ are defined via real-valued 
interpretations of Larsen and Skou's probabilistic modal logic.  Formulae 
assume truth values in the interval .  Conjunction and disjunction 
are interpreted using the lattice structure of the unit interval.  The 
modality  is interpreted arithmetically by integration.  The behavioural 
distance between states  and  is then defined as the supremum over 
all formulae  of the difference in the truth value of  in 
 and in .\footnote{More generally, de Alfaro \cite{A03:concur} and McIver and
Morgan \cite{MM:tocl} 
have given real-valued interpretations to the modal mu-calculus following 
this pattern.  Moreover, de Alfaro has shown that the behavioural 
pseudometrics induced by mu-calculus formulae agree with those of 
\cite{DGJP04:tcs}.}

The definition of the behavioural pseudometrics of Desharnais et al.\
is para\-metrized by a {\em discount factor\/} , a real number in the
interval .  The smaller the discount factor, the more (behavioural
differences in) the future are discounted.  In the case that 
equals one, the future is not discounted.  All differences in behaviour,
whether in the near or far future, contribute alike to the distance.
For systems that (in principle) run forever, we may be interested in all
these differences and, hence, in the pseudometric that does not discount
the future.

In \cite{DGJP99:concur}, Desharnais et al.\ presented an {\em algorithm\/} to
{\em approximate\/} the behavioural distances for  smaller than one.
The first and third author \cite{BW06:tcs} presented also an
approximation algorithm for  smaller than one.

There is a fundamental difference between pseudometrics that discount the
future and the one that does not.  This is, for example, reflected by the
fact that all pseudometrics that discount the future give rise to the same
topology, whereas the pseudometric that does not discount the future gives
rise to a different topology (see, for example, \cite[page~350]{DGJP04:tcs}).  
As a consequence,
it may not be surprising that neither approximation algorithm mentioned
in the previous paragraph can be modified in an obvious way to handle the
case that  equals one.

The main contribution of this paper is an algorithm that approximates 
behavioural distances in case the discount factor~ equals one. 
Starting from the \emph{logical} definition of the pseudometric by 
Desharnais et al., we first give a characterisation of the pseudometric 
as the greatest (post-)fixed point of a functional on a complete lattice 
, where  is the set of states of the probabilistic 
transition system in question.  This functional is closely related to the 
Kantorovich metric \cite{K42:dan} on probability measures.  Next, we dualize 
this characterization exploiting the Kantorovich-Rubinstein
duality theorem \cite{KR58:vestni}.  Subsequently, we show, exploiting
the dual characterization, that a pseudometric being a post-fixed
point can be expressed in the existential fragment of the first order
theory over real closed fields.  Based on the fact that this
first order theory is decidable, a result due to Tarski \cite{T51},
we show how to approximate the behavioural distances. 
Finally, we discuss an implementation of our algorithm in
Mathematica.

Exploiting the techniques put forward in this paper, we have also
developed an algorithm to approximate the behavioural pseudometric that
is presented in \cite{B05:concur}.  The other algorithm 
can be found in \cite{S06:york}.

\section{Systems and pseudometrics}

Some basic notions that will play a role in the rest of this paper are
presented below.  First we introduce the systems of interest:
probabilistic transition systems.

\begin{defi}
A {\sl probabilistic transition system\/} is a tuple  
consisting of
\begin{enumerate}[]
\item
a finite set  of states and
\item
a function  satisfying
.
\end{enumerate}
We write  if  and
 if .
\end{defi}

For states~ and ,  is the probability of making a
transition to state~ given that the system is in state~.
Each state~ either has no outgoing transitions ()
or a transition is taken with probability 1 ().
To simplify the presentation, we do not consider the case that a 
state~ may refuse to make a transition with some probability, 
that is, .  However, all our 
results can easily be generalized to handle that case as well
(see \cite{S06:york}).  We also do not consider transitions that are 
labelled with actions.  All our results can also easily be modified to 
handle labelled transitions (see \cite{S06:york}).  In the labelled case, 
the definition of probabilistic transition system is a mild generalisation 
of the notion of Markov chain.

We restrict to rational transition probabilities in order that
probabilistic transitions systems be finitely representable.  Here we
assume that rational numbers are represented as pairs of integers in
binary. We believe that the algorithm presented in this paper could be
adapted to accommodate transition probabilities that are algebraic
numbers, but we do not pursue this question here.

In the rest of this paper, we will use the following probabilistic
transition system as our running example.

\begin{exa}
\label{example:1}
We consider a probabilistic transition system with five states:
, , ,  and .  The following table contains
the transition probabilities and, hence, captures .
\begin{center}
\begin{tabular}{l|lllll}
      &  &  &  &  & \\
\hline
 & 0     &  &  & 0     & 0    \\
 &  & 0     & 0     &  &  \\
 & 0     & 0     & 1     & 0     & 0    \\
 & 0     & 0     & 0     & 0     & 0    \\
 & 0     & 0     & 0     & 0     & 1    
\end{tabular}
\end{center}
The probabilistic transition system can be depicted as the following
graph.

\end{exa}

We consider states of a probabilistic transition system behaviourally
equivalent if they are probabilistic bisimilar \cite{LS91:ic}.

\begin{defi}
Let  be a probabilistic transition system. An equivalence relation  
on the set of states  is a {\sl probabilistic bisimulation\/} if 
 implies  
for all -equivalence classes .  States  and  are 
{\sl probabilistic bisimilar\/}, denoted , if  
for some probabilistic bisimulation .
\end{defi}

Note that probabilistic bisimilar states  and  have the same probability of 
transitioning to an equivalence class  of probabilistic bisimilar states.

\begin{exa}
Consider the probabilistic transition system of Example~\ref{example:1}.
The smallest equivalence relation containing  is a probabilistic bisimulation.  
Hence, the states  and  are probabilistic bisimilar.
\end{exa}

The behavioural pseudometrics that we study in this paper yield pseudometric spaces
on the state space of probabilistic transition systems.

\begin{defi}
\label{definition:3}
A {\sl 1-bounded pseudometric space\/} is a pair  consisting of a
set  and a distance function  satisfying
\begin{enumerate}[(1)]
\item
for all , ,
\item
for all , , , and
\item
for all , , , .
\end{enumerate}
Instead of  we often write  and we denote the distance function
of a metric space  by .
\end{defi}

\begin{exa}
Let  be a set.  The discrete metric 
is defined by

\end{exa}

A (1-bounded) pseudometric space differs from a (1-bounded) metric space
in that different points may have distance zero in the former and not in
the latter.  Since different states of a system may behave the same, 
such states will have distance zero in our behavioural pseudometrics.

In the characterization of a behavioural pseudometric in 
Section~\ref{section:4} nonexpansive functions play a key role.

\begin{defi}
Let  be a 1-bounded pseudometric space.  A function 
is {\sl nonexpansive\/} if for all , ,

The set of nonexpansive functions from  to  is denoted by
.
\end{defi}

\begin{exa}
If the set  is endowed with the discrete metric, then every function
from  to  is nonexpansive.
\end{exa}

\section{Behavioural pseudometrics}
\label{section:3}

Desharnais, Gupta, Jagadeesan and Panangaden \cite{DGJP04:tcs} introduced 
a family of behavioural pseudometrics for probabilistic transitions systems.  
Below, we will briefly review the key ingredients of their definition.  

To define their behavioural pseudometrics, Desharnais et al.\ defined a 
real-valued semantics of a variant of Larsen and Skou's probabilistic
modal logic \cite{LS91:ic}.  We describe this variant, adapted to the case 
of unlabelled transition systems, in Definition~\ref{definition:5}.

\begin{defi}
\label{definition:5}
The logic  is defined by

where .
\end{defi}

The main difference between
the above logic and the one of Larsen and Skou is that we have 
 and  whereas they combine the 
operators  and  into one.  Since they consider
labelled transitions, they use the notation  for this
combined operator.

Desharnais et al.\ provided a family of real-valued interpretations
of the logic.  That is, given a probabilistic transition system and
a discount factor , the interpretation gives a quantitative
measure of the validity of a formula~ of the logic in a 
state~ of the system.  The interpretation 
is a real number in the interval .  It measures the validity
of the formula~ in the state~.  This real number can roughly
be thought of as the probability that  is true in .

\begin{defi}
Given a probabilistic transition system  and 
a discount factor , for each , 
the function  is defined by

\end{defi}

\begin{exa}
Consider the probabilistic transition system of Example~\ref{example:1}.
For this system, 
and .
\end{exa}

Given a discount factor , the behavioural pseudometric
 assigns a distance, a real number in the interval ,
to every pair of states of a probabilistic transition system.  The distance
is defined in terms of the logical formulae and their interpretation.
Roughly speaking, the distance is captured by the logical formula that
distinguishes the states the most.

\begin{defi}
Given a probabilistic transition system  and 
a discount factor ,
the distance function  is defined by

\end{defi}

\begin{exa}
Consider the probabilistic transition system of Example~\ref{example:1}.
For example, the states  and  are  apart.  This
distance is witnessed by the formula .
The distances\footnote{These distances were obtained by ad-hoc methods including
  Proposition \ref{proposition:7a} and checked for numerous different
  discount factors using the algorithm described in \cite{BW06:tcs}.}
are collected in the following table.  Since a distance function is
symmetric and the distance from a state to itself is zero, we do not
give all the entries.
\begin{center}
\begin{tabular}{l|p{8em}p{8em}p{8em}p{8em}}
      &     &     &     & \\
\hline
 &  \\
 &  &  & \\
 &  &  &  & \\
 &  &  & 0        & 
\end{tabular}
\end{center}
\end{exa}

\begin{prop}[{\cite[Theorem~5.2]{DGJP04:tcs}}]
\label{proposition:1}
 is a 1-bounded pseudometric space.
\end{prop}
\begin{proof}
First, observe that

As a consequence, we can replace 
 in
the definition of  with
.
Checking now that  satisfies the three conditions of
Definition~\ref{definition:3} is straightforward.
\end{proof}

States having distance zero defines an equivalence relation.  That is,
for a pseudometric  on states, the relation  on states 
defined by

is an equivalence relation.  We denote the equivalence class that
contains the state  by , that is,


Each behavioural pseudometric  is a quantitative analogue
of probabilistic bisimilarity.  This behavioural equivalence is
exactly captured by those states that have distance zero.

\begin{prop}[{\cite[Theorem~4.10]{DGJP04:tcs}}]
\label{proposition:2}
Given a probabilistic transition system  and
a discount factor , 

\end{prop}
\proof We split the proof in two parts.
\begin{enumerate}[]
\item
Assume that .  It suffices to show that

for all .  We can prove this by structural induction
on .  We focus here on the only nontrivial case: 
.
Let  be the -equivalence
classes.  Assume that  is an element of .
By induction, the function 
restricted to  is constant.  Hence,

\item
We show that the relation  
is a probabilistic bisimulation.  Obviously,  is
an equivalence relation.  Assume that .
That is, .  Let  be an -equivalence
class.  Without loss of any generality, we may assume that  is of
the form .  From the definition of  we can infer that
all states in  assign
the same value to each formula.  For each state 
there exists a formula  such that 
.
Without loss of any generality, we may assume that 
.
Hence, there exists a rational  in  such that
 and
. 
Now consider the formula

Then  iff .
As a consequence,

Therefore, 
and, hence,  is a probabilistic bisimulation.\qed
\end{enumerate}


In \cite{DGJP99:concur}, Desharnais et al.\ present a decision procedure
for the behavioural pseudometric  when  is 
smaller than one.  Let us briefly sketch their algorithm.  They
define the depth of a logical formula as follows.

One can easily verify that 

for each .
This suggests that one can compute  to any desired degree of 
accuracy by restricting attention to formulae  of a fixed modal 
depth. Clearly, there exist infinitely many formulae of each fixed modal 
depth.  Nevertheless, Desharnais et al.\ show how to construct a finite 
subset  of the logical formulae of at most depth  such 
that

In this way,  can be approximated up to arbitrary accuracy
\emph{provided}  is smaller than one.

\section{A fixed point characterization and its dual}
\label{section:4}

For the rest of this paper, we focus on the behavioural pseudometric
that does not discount the future.  That is, we concentrate on the
pseudometric~.  Below, we present an alternative characterization
of this pseudometric.  In particular, we characterize  as the
greatest (post-)fixed point of a function  from a complete
lattice to itself.  This characterization can be viewed as a
quantitative analogue of the greatest fixed point characterization of
bisimilarity \cite{P81:tcs}.

We also dualize the definition of  exploiting the Kantorovich-Rubinstein
duality theorem \cite{KR58:vestni}.  As we will see in Section~\ref{section:6},
this dual characterization will allow us to define  as the solution to a minimization
problem rather than a maximization problem, as above.  In turn this will allow us to capture
the fact that a pseudometric is a post-fixed point of  in the existential fragment
of the first order theory over real closed fields.

For the rest of this paper, we fix a probabilistic transition
system .  We endow the set of pseudometrics on  with
the following order.

\begin{defi}
The relation  on 1-bounded pseudometrics on  is defined by

\end{defi}

Note the reverse direction of  and  in the above
definition.  We decided to make this reversal so that  is a
greatest fixed point, in analogy with the characterization of bisimilarity,
rather than a least fixed point.  This choice has no impact on any
results in this paper.

\begin{prop}[{\cite[Lemma~3.2]{DGJP02:lics}}]
\label{proposition:3}
The set of 1-bounded pseudometrics on  endowed with the order
 forms a complete lattice.
\end{prop}
\begin{proof}
Obviously,  is a partial order.  The top element is
the 1-bounded pseudometric  defined by

The bottom element is the 1-bounded pseudometric  defined by

Let  be a nonempty set of 1-bounded pseudometrics on .  The
meet of  is the 1-bounded pseudometric  defined by

The join of  can be expressed in terms of the meet of 
(see, for example, \cite[Lemma~2.15]{DP90}).
\end{proof}

Whereas meets of pseudometrics are computed pointwise using the supremum
on [0,1], joins of pseudometrics are not.

Next, we introduce a function from this complete lattice to itself
of which the behavioural pseudometric  is the greatest fixed point.

\begin{defi}
\label{definition:9}
Let  be a 1-bounded pseudometric on .  The
distance function  is defined by

if  and , and
s_1 \not\rightarrows_2 \not\rightarrow 
\end{defi}

Note that we can write  above rather than 
since , being a closed subset of the product space
, is compact.

The functional  is closely related to the Kantorovich metric 
\cite{K42:dan} on probability measures.  In the definition of that
metric, nonexpansive functions play a key role.\footnote{The Kantorovich metric is the smallest distance function
on probability measures for which integration of nonexpansive
functions is nonexpansive.}

\begin{prop}
 is a 1-bounded pseudometric on .
\end{prop}
\begin{proof}
Note that  implies .
Furthermore, if  and  then

As a consequence, if  and  then

Now that we have this alternative representation of , 
checking that it satisfies the three conditions of Definition~\ref{definition:3}
is straightforward.
\end{proof}

\begin{prop}[{\cite[Proposition~38]{BHMW06:tcs}}]
\label{proposition:5}
 is order-preserving.
\end{prop}
\begin{proof}
Let  and  be 1-bounded pseudometrics on  with .
Note that any function  that is nonexpansive with
respect to  is also nonexpansive with respect to .  Therefore
 for all  since the latter involves taking the  over a larger set.
\end{proof}

Since  is a 1-bounded pseudometric on  and  is 
order-preserving, we can conclude from Tarski's fixed point theorem 
\cite[Theorem~1]{T55:pjm} that  has a greatest fixed point.
We denote the greatest fixed point of  by .
This greatest fixed point of  is also the greatest post-fixed
point of  (see, for example, \cite[Theorem~4.11]{DP90}\footnote{ is a {\em post-fixed point\/} of  if .
In \cite[page~94]{DP90}, such a  is called a pre-fixpoint.}).

\begin{thm}
\label{proposition:6}
.
\end{thm}

\begin{proof}
We first prove that  is a post-fixed point of .  
That is, we show that .  
To prove this, we distinguish the following three cases.
\begin{enumerate}[]
\item
If  and  then 
the property is vacuously true.
\item
If  and , or
 and , then
the formula  witnesses that the
states  and  have distance one.
\item
Assume that  and .
According to \cite[Proposition~39]{BW05:tcs}, the set

is dense in , that is,
each  can be approximated up to
arbitrary accuracy by some .  As a consequence,

\end{enumerate}

Next we prove that  is the greatest post-fixed point of .
Assume that  is a post-fixed point of .  We have to show that
.  That is, . 
We restrict our attention to the case that
 and .  It suffices to show that

for all .  This can be proved by structural induction on
.  We consider only the nontrivial case: .

\end{proof}

A similar result can be obtained by combining Theorem~40 and 44 of \cite{BHMW06:tcs}.

Let us recall (a minor variation of) the Kantorovich-Rubinstein duality theorem.  Let  be a 
1-bounded compact pseudometric space.  Let  and  be Borel probability measures on .  
We denote the set of Borel probability measures on the product space with marginals 
 and , that is, the Borel probability measures  on  such that 
for all Borel subsets  of ,

by .  The Kantorovich-Rubinstein duality theorem tells us


The following proposition, which is a consequence of the Kantorovich-Rubinstein 
duality theorem, defines  as a minimum as opposed to the maximum in 
Definition~\ref{definition:9}.

\begin{prop}[{\cite[Corollary~19]{BW06:tcs}}]
\label{proposition:7}
Let  be a 1-bounded pseudometric on .  Let , 
such that  and .  Then

where  if

\end{prop}
\begin{proof}
Since the set  is finite, the space  is compact.  The probability
distributions  and  define Borel
probability measures on .  Applying the Kantorovich-Rubinstein
gives us the desired result.
\end{proof}

\section{The algorithm}
\label{section:6}

Before we present our algorithm, we first show that the fact that a pseudometric
is a post-fixed point of  can be expressed in (the existential 
fragment of) the first order theory over real closed fields.
This will allow us to exploit Tarski's decision procedure to
approximate the behavioural pseudometric.  

For the rest of this paper, we assume that the probabilistic transition system 
has  states , , \ldots, .  Instead of  
we will write .  We represent a 1-bounded
pseudometric on the set  of states of the probabilistic transition
system, as (the values of) a collection of real valued variables . 

The fact that  is a 1-bounded pseudometric can now be captured as follows.

\begin{defi}
The predicate  is defined by

\end{defi}

Furthermore, the fact that  is a post-fixed point of  can be captured as follows.

\begin{defi}
The predicate  is defined by

where

\end{defi}

Now we are ready to present our algorithm.  Consider the states
 and .  We restrict our attention to the case 
that  and .  In the
other cases the computation of the distance is trivial.

In our algorithm, we use the algorithm {\tt tarski} that takes 
as input a sentence of the first order theory of real closed fields 
and decides the truth or falsity of the given sentence.  The fact 
that there exists such an algorithm was first proved by Tarski \cite{T51}.

Let  be the desired accuracy.  That is, we want to find an 
interval  such that  
and .
The algorithm {\tt approximate} takes as input an interval
 such that 
and returns the desired result.  As a consequence, {\tt approximate}(0, 1)
returns an approximation of  with accuracy .

\begin{quote}
\lstset{basicstyle=\ttfamily,basewidth={0.5em,0.1em},breaklines=true,mathescape=true}
\begin{lstlisting}
approximate(, ):
   if 
      return 
   else
      
      if tarski()
         return approximate(, )
      else
         return approximate(, )
\end{lstlisting}
\end{quote}

Note that the argument of {\tt tarski} is a sentence that is part of
the existential fragment of the first order theory over real closed fields.
For this fragment there are more efficient decision procedures than
for the general theory (see, for example, \cite{SPR96:jacm}).  

Let us sketch a correctness proof of our algorithm.  Assume that
.  We distinguish the following
three cases.
\begin{enumerate}[]
\item
If , then the algorithm obviously returns the
desired result.
\item
Assume that  and suppose that
{\tt tarski}
returns true.  Then there exists a 1-bounded pseudometric  that is 
a post-fixed point of  and .  Since 
is the greatest post-fixed point of , we have that .
Hence, .  By assumption
, therefore .
\item
Assume that  and suppose that {\tt tarski}
returns false.  Then  for every 1-bounded
pseudometric  that is a post-fixed point of .  Since 
is a post-fixed point of , we have that .  By assumption ,
therefore, .
\end{enumerate}
Obviously, the algorithm terminates.

\section{Conclusion}

This paper combines a number of ingredients, known already for a long time, 
including the Kantorovich-Rubinstein duality theorem of the fifties, 
Tarski's fixed point theorem of the forties and Tarski's decision procedure 
for the first order theory of real closed fields of the thirties.
We show that the behavioural pseudometric , which does not discount 
the future, can be approximated up to an arbitrary accuracy.  While the 
combination of the above results into a decision procedure for the 
pseudometric is not technically difficult, we do solve a problem that has 
been open since 1999.  Most of the results in Section~\ref{section:3} and 
\ref{section:4} are (variations on) known results.  As far as we know, 
the results in Section~\ref{section:6} and Appendix~\ref{section:8} are new.
The techniques exploited in this paper have also been used to
approximate other behavioural pseudometrics that do not discount the
future such as, for example, the one presented in \cite{B05:concur}.
Furthermore, our algorithm can easily be adjusted to the discounted
case.  

Since the satisfiability
problem for the existential fragment of the first order theory of
real closed fields is in \textsc{PSPACE}, it is not surprising that
our algorithm can only handle small examples as we have shown in
Appendix~\ref{section:8}.  As a consequence, the quest for practical
algorithms to approximate  is still open.  Since the closure
ordinal of  is , as proved in
Appendix~\ref{section:added}, an iterative algorithm might be feasible.

As future work, we plan to apply our techniques to obtain
approximation algorithms for other behavioural pseudometrics such as,
for example, the one for systems that combine nondeterminism and
probability presented in \cite{DCPP05:qapl} and the pseudometric for
weak probabilistic bisimilarity in \cite{DGJP02:lics}.  In the latter
case the pseudometric can be characterized as the fixed point of a
functional based on the Kantorovich and Hausdorff metrics.  These can
easily be encoded in the first-order theory of the reals.  However, the
need to consider the transitive closure of the silent transition relation
suggests that some non-trivial extension of the work presented here is
called for.

\section*{Acknowledgement}

The authors would like to thank Christel Baier for providing
some pointers to the literature and Jeff Edmonds, Parke Godfrey
and the referees for their constructive feedback.

\bibliographystyle{plain}
\bibliography{submission}

\appendix

\section{Closure ordinal of }
\label{section:added}

The greatest fixed point of an order-preserving function on a complete lattice
can be obtained by iteration (see, for example, \cite[Exercise~4.13]{DP90}).

\begin{defi}
For each ordinal , the 1-bounded pseudometric  on 
is defined by

\end{defi}

As we will see in the next example, for some systems we need at least  iterations
to reach the greatest fixed point of .  

\begin{exa}
Consider the system of Example~\ref{example:1}.  For all ,

Hence, for this system we need  iterations.
\end{exa}

In the rest of this appendix, we prove that we need at most
 iterations for any system.  This tells us that the 
closure ordinal of  is
, that is, .
As a consequence,  is the greatest fixed point of
 (see, for example, \cite[Example~4.13]{DP90}).  As we will 
see below, the fact that  is a fixed point of  follows
from the facts that  is order-preserving 
(Proposition~\ref{proposition:5}) and Lipschitz 
(Proposition~\ref{proposition:d}).

In \cite[page~418]{DGJP02:lics}, Desharnais et al.\ state that
a functional similar to  has closure ordinal .

Recall that for a pseudometric , the equivalence relation  
relates states that have distance zero.  From each equivalence class 
 we pick a designated state which we denote by .  
Hence,  and also .

\begin{prop}
\label{proposition:a}
For all , ,

\end{prop}
\begin{proof}

\end{proof}

Let .  The ratio 
of  and  is defined by

Note that we never divide by zero since 
and, hence, .  

Below, we will use the convention that the minimum of 
the empty set is one and the maximum of the empty set is zero.

Given pseudometrics  and  such that  and
given an , we next show that there
exists a  that is nonexpansive.

\begin{prop}
\label{proposition:b}
Let  and .
Let  be defined by

Then .
\end{prop}
\begin{proof}
Let , .  We have to show that

We distinguish two cases.  If  then
 and, hence, .
Therefore  and, hence, the property is vacuously
true.  Let .  According to 
Proposition~\ref{proposition:a}, .
Also  since , and

\end{proof}

Next, we bound  from above.

\begin{prop}
\label{proposition:c}
Let  and .
Let .
Then

for all .
\end{prop}
\begin{proof}
Let . Then

Furthermore,

and

\end{proof}

Now we can prove that  is Lipschitz, that is,

for some constant .

\begin{prop}
\label{proposition:d}
Let .
For all , ,

\end{prop}
\begin{proof}
Let , .  Then

\end{proof}

Finally, we prove that the closure ordinal of  is .

\begin{prop}
\label{proposition:6a}
.
\end{prop}
\begin{proof}
First, we show that .
By definition,  for
all .  Since  is order-preserving,
 for all .
Obviously, .  Therefore,
 is a lower bound of .
Since  is the greatest lower bound by definition,
.

We have left to show that , that is,

for all , .  Let , .  Let .  It 
suffices to show that there exists an  such that
. 
Let .
Since the set  is finite, for every  there exists an 
such that for all , ,

Here we pick  to be .
From Proposition~\ref{proposition:d} we can conclude that

\end{proof}

\section{An implementation in Mathematica}
\label{section:8}

A decision procedure for the first order theory of real closed fields based 
on quantifier elimination was first given by Tarski \cite{T51}.  A number of 
algorithms have been developed thereafter for the theory (see, for example,
\cite{SPR96:jacm,C75:atfl,H05}).  Collin's algorithm is implemented in 
the tool Mathematica and can be used for solving our formulae.  However, it 
works for very small examples and therefore it is essential to simplify the 
formula and reduce its size to make it solvable.  To simplify the formula,
we first compute some of the distances using the following results.

\begin{prop}
\label{proposition:A}
\mbox{}
\begin{enumerate}[]
\item
If  and  then .
\item
If  and , or  and 
 then .
\end{enumerate}
\end{prop}
\begin{proof}
We only consider the first case.  The second one can be proved similarly.
If  and  then
.
\end{proof}

\begin{exa}
Consider the probabilistic transition system of Example~\ref{example:1}.
State  has distance one to all other states.
\end{exa}

Next, we present a simple characterization of the distance between a state 
that never terminates (that is, the probability of reaching a state with no 
outgoing transitions is zero) and another state.

Given a state  and ,  is the probability
of terminating in less than  transitions when started in .

\begin{defi}
\label{definition:termination}
For each , the function 
is defined by

\end{defi}

\begin{exa}
Consider the probabilistic transition system of Example~\ref{example:1}.
Then we have that , 
, ,
 and .
\end{exa}

Obviously, for a state  without outgoing transitions, we have
that .  For a state  that cannot reach
any state without outgoing transitions, we have that .
For the remaining states, we can compute the probability of termination
using standard techniques as described in, for example, 
\cite[Section~11.2]{GS97}.

\begin{prop}
\label{proposition:7a}
If  then .
\end{prop}
\proof Assume that .  We prove that for all ,

by induction on .
\begin{enumerate}[]
\item
Obviously, .
\item
We have to prove that .  We
distinguish the following two cases.
\begin{enumerate}[]
\item
If 
then .
\item
Now let us assume that .  First we show that
 as a function from  to  is nonexpansive.
For all , ,

Since


Let .  For all ,

As a consequence,

Since  was chosen arbitrarily, we can conclude that

\item
Finally,

\end{enumerate}
From Theorem~\ref{proposition:6} and Proposition~\ref{proposition:6a} we can conclude 
that .\qed
\end{enumerate}


\begin{exa}
Consider the probabilistic transition system of Example~\ref{example:1}.
From Proposition~\ref{proposition:7a} we can conclude that
, ,
 and .
\end{exa}

Given a probabilistic bisimulation , we can quotient the probabilistic
transition system  as follows.

\begin{defi}
Let  be a probabilistic bisimulation.  The probabilistic transition
system  consists of
\begin{enumerate}[]
\item
the set  of -equivalence classes and 
\item
the function  defined by

\end{enumerate}
\end{defi}

Note that the function  is well-defined since  
is a probabilistic bisimulation.  We will apply the above quotient 
construction for probabilistic bisimilarity (which can be computed in
polynomial time \cite{BEM00:jcss}).

\begin{exa}
Consider the probabilistic transition system of Example~\ref{example:1}.
The smallest equivalence relation containing  is a 
probabilistic bisimulation.  The resulting quotient can be depicted as

\end{exa}

By quotienting, the number of states that need to be considered and, hence,
the number of variables in the formula may be reduced.  However, we still
have to check that the quotiented system gives rise to the same distances.
Next we relate the behavioural pseudometric  of the original system 
 with the behavioural pseudometric  of the 
quotiented system .

\begin{prop}
\label{proposition:B}
For all , , .
\end{prop}
\proof First all, note that

As a consequence, we have left to consider the case 
and .  We prove that for all ,
 by induction on .
We distinguish the following three cases.
\begin{enumerate}[]
\item
If  then the property is vacuously true.
\item
Assume that  for all , .
Let , .  We have to prove that .
In the proof of this case, we make use of the following two observations.  
For each , there exists a 
such that  for all , since
 
Similarly, we can show that for each , there exists
 such that  for all .
Note that if states  and  are probabilistic bisimilar then 
 and, hence,  and, therefore, ,
since  is nonexpansive.

\item
Furthermore,

\end{enumerate}

To simplify the formula even further, we exploit the following three observations.
\begin{enumerate}[]
\item
Since  is a pseudometric,  and 
.  Therefore,
in  we can replace
all 's with zero and all 's where  with
's.  As a consequence, we only need to consider 's with .
This reduces the number of variables in the formula considerably.
\item
Let  be the set of pairs of states for which the distances have already
been computed.  Then 

is equivalent to

since  is the greatest post-fixed point.  As a consequence, we can
replace all 's where  with their already computed
distances .  Again, the number of variables may be reduced.
\item
If , we can infer that  for all . 
As a consequence, we can replace the occurrences of all those 's 
with 0.  Symmetrically, if  we can simplify the formula
similarly.  Also this simplification may reduce the number of variables.
\end{enumerate}

We have implemented these simplifications in the form of a Java program that takes
as input the probability matrix  and that produces as output the simplified
formula in a format that can be fed to Mathematica.\footnote{The code and documentation is available at the URL\ 
{\tt www.cse.yorku.ca/franck/research/pm2m}.}

\begin{exa}
Consider the probabilistic transition system of Example~\ref{example:1}.
The simplified formula for this system is given below.
\lstset{basicstyle=\ttfamily,basewidth={0.5em,0.1em},breaklines=true,mathescape=true}

\lstset{basicstyle=\ttfamily,
        breaklines=true,
        basewidth={0.5em,0.1em},
        numbers=left, 
        numberstyle=\tiny}
\begin{small}
\begin{lstlisting}
Reduce[
  Exists[d12,
    (0 <= d12 <= 1) && (0.11112 <= d12 + 0.27778) && (d12 <= 0.38889) && 
    Exists[{u12,u13,u32,u42,u43,u33}, 
      (0 <= u12 <= 1) && (0 <= u13 <= 1) && (0 <= u32 <= 1) && 
      (0 <= u42 <= 1) && (0 <= u43 <= 1) && 
      (u12 + u32 + u42 == 0.4) && (u13 + u43 + u33 == 0.6) && 
      (u12 + u13 == 0.7) && (u32 + u33 == 0.1) && (u42 + u43 == 0.2) && 
      (d12 * u12 + 0.11112 * u13 + 0.27778 * u32 + u42 + u43 <= d12)] && 
    Exists[{u21,u23,u24,u31,u33, u34}, 
      (0 <= u21 <= 1) && (0 <= u23 <= 1) && (0 <= u24 <= 1) && 
      (0 <= u31 <= 1) && (0 <= u34 <= 1) && 
      (u21 + u31 == 0.7) && (u23 + u33 == 0.1) && (u24 + u34 == 0.2) && 
      (u21 + u23 + u24 == 0.4) && (u31 + u33 + u34 == 0.6) && 
      (d12 * u21 + 0.27778 * u23 + u24 + 0.11112 * u31 + u34 <= d12)] && 
    (0 <= d12 <= 0.5)]]
\end{lstlisting}
\end{small}
Line~3 correspond to , line 4--9 correspond to
 and line 10--15 correspond to
.  
The formula was reduced to true by Mathematica in 8.2 seconds on a 3GHz machine with 1GB RAM.
When feeding Mathematica the formula that has not been simplified, it runs out of
memory after some time.

We also attempted to solve this example with a solver called QEPCAD~B \cite{C03:jsac} but
the performance of Mathematica on this example was better.
\end{exa}


\end{document}
