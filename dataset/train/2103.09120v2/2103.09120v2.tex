\pdfoutput=1


\documentclass[11pt]{article}

\usepackage{emnlp2021}

\usepackage{times}
\usepackage{latexsym}
\usepackage{graphicx}
\renewcommand{\UrlFont}{\ttfamily\small}
\usepackage{microtype}
\usepackage{booktabs}
\usepackage{multirow}
\usepackage{amsmath}
\usepackage{amssymb}
\usepackage{array}
\usepackage{xspace}
\usepackage{bm}

\usepackage{color, colortbl}
\definecolor{Gray}{gray}{0.93}
\usepackage{array}
\newcolumntype{P}[1]{>{\raggedright\arraybackslash}p{#1}}
\newcolumntype{N}[1]{>{\raggedright\arraybackslash\columncolor{Gray}}p{#1}}

\usepackage[T1]{fontenc}


\usepackage[utf8]{inputenc}

\usepackage{microtype}

\newcommand\ftnote[1]{\footnote{\raggedright#1}}

\newcommand\BibTeX{B\textsc{ib}\TeX}
\newcommand{\graphadapter}{{\small\textsc{StructAdapt}}\xspace}
\newcommand{\graphadaptergcn}{{\small\textsc{StructAdapt-gcn}}\xspace}
\newcommand{\graphadapterrgcn}{{\small\textsc{StructAdapt-rgcn}}\xspace}
\newcommand{\vanilladapter}{{\small\textsc{Adapt}}\xspace}
\newcommand{\finetune}{{\small\textsc{Fine-tune}}\xspace}
\newcommand{\finetunetop}{{\small\textsc{FT-top2}}\xspace}
\newcommand{\finetunebottom}{{\small\textsc{FT-bottom2}}\xspace}
\newcolumntype{P}[1]{>{\centering\arraybackslash}p{#1}}
\newcolumntype{M}[1]{>{\centering\arraybackslash}m{#1}}

\title{Structural Adapters in Pretrained Language Models \\ for AMR-to-Text Generation}

\author{Leonardo F. R. Ribeiro, Yue Zhang and Iryna Gurevych \vspace{1mm} \\
\rule{0pt}{2.5ex}
  Ubiquitous Knowledge Processing Lab, Technical University of Darmstadt\\
  School of Engineering, Westlake University \\
 \texttt{ribeiro@aiphes.tu-darmstadt.de}
}

\begin{document}
\maketitle
\begin{abstract}

Pretrained language models (PLM) have recently advanced graph-to-text generation, where the input graph is linearized into a sequence and fed into the PLM to obtain its representation. However, efficiently encoding the graph structure in PLMs is challenging because such models were pretrained on natural language, and modeling structured data may lead to catastrophic forgetting of distributional knowledge. In this paper, we propose \graphadapter, an adapter method to encode graph structure into PLMs. Contrary to prior work, \graphadapter effectively models interactions among the nodes based on the graph connectivity, only training graph structure-aware adapter parameters. In this way, we incorporate task-specific knowledge while maintaining the topological structure of the graph. We empirically show the benefits of explicitly encoding graph structure into PLMs using \graphadapter, outperforming the state of the art on two AMR-to-text datasets, training only 5.1\% of the PLM parameters.\ftnote{Our code and checkpoints are available at \href{https://github.com/UKPLab/StructAdapt}{https://github.com/UKPLab/StructAdapt}.}
\end{abstract}

\section{Introduction}

Data-to-text tasks aim to generate meaningful and coherent natural language text that faithfully conveys \emph{structured data}. Some examples of structured information include tables \cite{parikh-etal-2020-totto}, Knowledge Graphs (KGs) \cite{gardent-etal-2017-webnlg, VOUGIOUKLIS20181} and Abstract Meaning Representation (AMR) \cite{banarescu-etal-2013-abstract}. In this work, we focus on AMR-to-text generation where the goal is to generate a fluent and grammatical sentence that is faithful to a given AMR graph (See Figure~\ref{fig:amrexamples}a). AMR is a semantic formalism that has received much research interest \cite{song-etal-2018-graph, doi:10.116200269, ribeiro-etal-2019-enhancing, opitz-etal-2020-amr, bamboo, fu-etal-2021-end} and has been shown to benefit downstream tasks such as text summarization \cite{liao-etal-2018-abstract} and machine translation \cite{doi:10.116200252}.  Both statistical \cite{flanigan-etal-2016-generation, pourdamghani-etal-2016-generating} and neural methods \cite{bai-etal-2020-online,cai-lam-2020-graph} have been investigated for AMR-to-text generation, and dominant methods make use of Graph Neural Networks (GNNs) \cite{Kipf:2016tc} or Transformers \cite{NIPS2017_7181} for representing the input graph.

\begin{figure}[t]
    \centering
    \includegraphics[width=.47\textwidth]{images/figure1-adapter-paper.pdf}
    \caption{(a) AMR for the sentence \textit{More power to her for her achievements}. While in (b) the pretrained model gets as input the graph linearization, in (c) it additionally receives the graph connectivity information.}
    \label{fig:amrexamples}
\end{figure}

\emph{Pretrained language models} (PLMs) \cite{devlin-etal-2019-bert, liu2020roberta, radford2019language, lewis2019bart} have been shown useful as a general text representation method, giving much improved results on a wide range of tasks \cite{wang-etal-2018-glue, NIPS2019_8589}. However, they cannot be directly leveraged to benefit AMR-to-text generation, and more generally graph-to-text generation, due to the structural nature of the input. One solution is to transform the structured input into a sequence, which can be directly fed into PLMs (See Figure~\ref{fig:amrexamples}b). Recent studies \cite{mager2020gpttoo, harkous2020text,ribeiro2020investigating,ribeiro2021smelting} transform AMRs into sequences by top-down linearization \cite{konstas-etal-2017-neural}. It has been shown that such linearized graph representation can be used to fine-tune a PLM and improve graph-to-text generation performances \cite{kale2020texttotext}.

The above methods, however, suffer from two salient limitations. First, linearized graph structures are different in nature from natural language. As a result, knowledge from large-scale pretraining intuitively cannot be fully transferred, and fine-tuning a sentence representation using linearized graphs can lead to catastrophic forgetting of such distributional knowledge \cite{goodfellow2013an, Kirkpatrick3521}. Second, a linearized representation weakens structural information in the original graphs by diluting the explicit connectivity information (i.e., which nodes are connected to each other), and PLMs must infer how edge connections are specified in the sequence. This fact was also observed by \citet{song-etal-2018-graph}, \citet{beck-etal-2018-graph} and \citet{ribeiro-etal-2019-enhancing}, who show that GNN encoders outperform sequential encoders for AMR-to-text generation without pretraining.

To mitigate the issues, we aim to explicitly encode the graph data into a PLM without contaminating its original distributional knowledge. To this end, we propose \graphadapter, a novel structure-aware adapter that effectively allows leveraging the input graph structure into PLMs (See Figure~\ref{fig:amrexamples}c). The main idea is to add layer-wise modules, which extract information from the pretrained layers and make use of it in a graph-structure encoding. As shown in Figure~\ref{fig:adapterarc}, \graphadapter employs a \emph{graph convolution} in order to learn representations built upon the graph connectivity over the PLM encoder. Because \graphadapter is added to each encoder layer, deep integration of linguistic knowledge and graph knowledge can be achieved. During fine-tuning, only the adapter parameters are trained, whereas the PLM parameters remain unchanged, in contrast to previous methods based on the graph linearizations that fine-tune all model parameters.

Empirically we show that \graphadapter significantly outperforms linearized fine-tuning baselines and naive sequential adapters \cite{pmlr-v97-houlsby19a}. Moreover, \graphadapter is more robust to different graph linearizations, better treats reentrancies (nodes with more than one entering edge) and long-range node dependencies. Our proposed models, based on \graphadapter, surpass the current state of the art on LDC2017T10 and LDC2020T02 datasets by up to 3.1 BLEU points, training only 5.1\% of the original PLM parameters.

\section{Related Work}

\paragraph{Fine-tuning for Graph-to-text Generation.} While previous approaches \cite{song-etal-2018-graph,ribeiro-etal-2019-enhancing, cai-lam-2020-graph, schmitt2020modeling, zhang-etal-2020-lightweight} have shown that explicitly encoding the graph structure is beneficial, fine-tuning PLMs on linearized structured data has established a new level of performance in data-to-text generation \cite{radev2020dart, kale2020texttotext, ribeiro2021smelting}. Our work can be seen as integrating the advantage of both graph structure encoding and PLMs, using a novel adapter module. 

\citet{mager2020gpttoo} employ cycle consistency to improve the adequacy of generated texts from AMRs using GPT-2 \cite{radford2019language}, whereas \citet{harkous2020text} train a classifier to rank candidate generations based on the semantic fidelity. \citet{ribeiro2020investigating} investigate encoder-decoder PLMs for graph-to-text generation, and show that task-specific pretraining can lead to notable improvements and that PLMs benefit much more from the graph structure of AMRs than of KGs. \citet{hoyle2020promoting} explore the extent to which PLMs are invariant to graph linearization, finding that models trained on canonical linearizations fail to generalize to meaning-preserving alternatives. Compared to this line of work, which tunes all PLM parameters, our method obtains a further 19x reduction in task-specific parameters, tuning only 5.1\% of the parameters while achieving state-of-the-art performance, being more robust to permutations of the graph representation and better encoding larger graphs.

\paragraph{Lightweight Fine-tuning.}
Recently, different approaches have emerged as an alternative training strategy in order to avoid fine-tuning all parameters of a PLM. \citet{sidetuning2019} train a lightweight ``side'' network that is fused with the pretrained model via summation. \citet{li2021prefixtuning} propose to prepend a trainable continuous prefix as an alternative to adapters, maintaining comparable performance in data-to-text tasks using fewer trained parameters. \citet{DBLP:journals/corr/abs-2103-10385} develop a method to automatically search prompts in the continuous space and evaluate it in few-shot NLU tasks. \citet{hambardzumyan-etal-2021-warp} propose adversarial reprogramming attempts to learn task-specific word embeddings to customize the language model for the downstream task.

Adapter-based approaches \cite{pmlr-v97-houlsby19a, NIPS2017_e7b24b11,lauscher-etal-2020-common,pfeiffer-etal-2020-adapterhub, pfeiffer2021adapterfusion} introduce a small number of task specific parameters, keeping the underlying pretrained model fixed. \citet{pfeiffer-etal-2020-mad} propose an adapter method to arbitrary tasks and languages by learning modular language and task representations. The above works are related to \graphadapter as it trains much fewer parameters, but also different because they do not explicitly encode the input structure, whereas \graphadapter directly aims to encode it.

\section{Graph-to-Text Model}

Let  denote a rooted and directed AMR graph with a node set  and labeled edges , where  and  is a relation type. An example of an AMR graph and its corresponding sentence is shown in Figure~\ref{fig:amrexamples}a.

\subsection{Encoder-Decoder Architecture}
\label{sec:baselinemodel}
Consider a conditional generation task where the input is a context  and the output  is a sequence of tokens. In AMR-to-text generation, the context  is the AMR graph and  is the sentence that describes the AMR graph in natural language. 

Let  denote a PLM parametrized by , where  is encoded by a bidirectional encoder, and the decoder predicts  autoregressively, conditioned on the encoded  and its left context. We focus on PLMs based on the Transformer encoder-decoder architecture \cite{NIPS2017_7181}, as they are suitable for conditional text generation. We define , where  is a function that linearizes  into a sequence of tokens.\footnote{The variable of a re-entrant node -- node with more than one incoming edge -- is replaced with its co-referring concept.} Following~\citet{damonte-cohen-2019-structural}, as shown in Figure~\ref{fig:amrexamples}b, we linearize the AMR into a sequence of nodes and edges using the depth-first traversal of the canonical human-created AMR.\footnote{Other AMR linearizations are discussed in \S\ref{sec:amrinputrep}.} In a nutshell, the hidden representation , for all , is computed by the encoder layer , where  is the hidden dimension. The decoder hidden representation  is computed by the layer  of the autoregressive decoder at time step .


\subsection{Fine-tuning}
The model is initialized with pretrained parameters  (e.g. using T5, \citeauthor{2019t5}, \citeyear{2019t5}) and fine-tuned to optimize the following log-likelihood objective over each gold instance :


\begin{figure}[t]
    \centering
    \includegraphics[width=.4\textwidth]{images/model-and-adapter.pdf}
    \caption{Integration of the adapter modules with the (a) encoder and (b) decoder layers of the Transformer; layer normalization and residual connections are omitted for clarification. (c) \vanilladapter with two feed-forwards layers. (d) \graphadapter encodes the graph structure using a \emph{graph convolutional layer}.}
    \label{fig:adapterarc}
\end{figure}

\subsection{Baseline Adapter} We employ an adapter module after the feed-forward sub-layer of each layer on both encoder (Figure~\ref{fig:adapterarc}a) and decoder (Figure~\ref{fig:adapterarc}b) of the PLM. We modify the adapter architecture from \citet{pmlr-v97-houlsby19a}, computing the adapter representation at each layer , given the encoder layer representation  (or  in the decoder), as follows:

where  is the activation function and  denotes layer normalization.  and  are adapter parameters, and  is the hidden dimension of the adapter.
Figure~\ref{fig:adapterarc}c illustrates the baseline adapter module, which we call \vanilladapter. 

\paragraph{Training.} Let the set of adapters' parameters for the encoder and decoder layers be parametrized by . The training objective is the same as Equation (\ref{eqn:loss}), but the set of trainable parameters changes: the PLM parameters  are frozen and the adapter parameters  are the only trainable parameters. In contrast to fine-tuning, adapters substantially reduce the number of trainable parameters that are used to adapt the PLM to the downstream task.

\subsection{Limitation}
\label{sec:intuition}

Intuitively, the connection between nodes in the input graph can influence the encoding of  by guiding what to extract from  in order to generate . Note that in both fine-tuning and \vanilladapter approaches, the self-attention mechanisms of the encoder layers treat the sequence of nodes and edges  essentially as a fully connected graph, greatly diluting the original graph structure. In this way, the model has to retrieve the original connectivity of the graph from . For example, the AMR linearization in Figure~\ref{fig:amrexamples}b has two mentions of the node \emph{she}, and the model should capture that both mentions belong to the same node in the original graph. 

\section{Structural Adapter}

We propose \graphadapter, a lightweight alternative to injecting \emph{structural inductive bias}\footnote{The model architecture explicitly encodes the graph structure, i.e., which nodes are connected to each other.} into PLMs.

We first describe the intuition in \S\ref{sec:intuition} and define our method formally in \S\ref{sec:method}.


\subsection{Intuition}
\label{sec:intuition}

Injecting \emph{graph structural bias} into graph-to-text models trained from scratch improves the performance compared to linearized approaches \cite{damonte-cohen-2019-structural, ribeiro-etal-2019-enhancing}. However, it is not straightforward how to effectively model the input graph structure when fine-tuning PLMs, which usually are pretrained using natural language and not structured data. 

Our key idea is modeling the graph connectivity in the encoder utilizing an adapter module, using information flows between adjacent nodes in a message-passing update, employing a \emph{graph convolution} (see Figure~\ref{fig:adapterarc}d). In this way, the graph structure substantially impacts the node representations, better encoding the input graph without impacting the knowledge learned during pretraining. This can lead to more efficient and better AMR-to-text generation as we will show in \S\ref{sec:exps} and \S\ref{sec:graphrepeval}. Moreover, different adapters for distinct graph domains can be used with the same PLM, yielding a high degree of parameter sharing for graph-to-text tasks.


\subsection{Graph Representation}
\label{sec:graphrep}
We convert each  into a bipartite graph , replacing each labeled edge  with two unlabeled edges  and . Similar to~\citet{beck-etal-2018-graph}, this process converts the graph into its unlabeled version. Figure~\ref{fig:amrreps} shows an (a) AMR subgraph and (b) its unlabeled representation.

Note that PLMs typically use a vocabulary with subword units \cite{sennrich-etal-2016-neural}. This presents a challenge in how to represent such a graph using subword tokens. Inspired by~\citet{ribeiro2020modeling}, we transform each  into a new token graph , where each token of a node in  becomes a node . We convert each edge  into a set of edges and connect every token of  to every token of . That is, an edge  will belong to  if and only if there exists an edge  such that  and , where  and  are seen as sets of tokens. Figure~\ref{fig:amrreps}c shows an example of the token graph.

\begin{figure}[t]
    \centering
    \includegraphics[width=.45\textwidth]{images/examples-amr-3.pdf}
    \caption{An example of (a) an AMR graph structure, (b) its unlabeled version and three different subword representations: (c) \emph{rep1}, (d) \emph{rep2} and (e) \emph{rep3}.}
    \label{fig:amrreps}
\end{figure}



\subsection{Method}
\label{sec:method}

\graphadapter employs a two-layer architecture in order to re-purpose the PLM for the graph-to-text task using a small number of new parameters. Formally, for each node , given the hidden representation  from the encoder layer , \graphadapter computes:

where  is the immediate neighborhood of  in .  is the graph convolution that computes the node representation based on the \emph{local neighborhood} of , and  is a parameter. Figure~\ref{fig:adapterarc}d illustrates \graphadapter.\footnote{Preliminary experiments with other architecture configurations led to worse or similar performance.}

\paragraph{Graph Convolution.} The graph convolutional layer allows exploration of distinct strategies for neighborhood aggregation in order to model structural information of the input graph. Different GNN architectures \cite{velickovic2018graph, xu2018how} can be employed as the graph convolution. Moreover, in this way, we avoid changing the self-attention mechanism of the current pretrained encoder, allowing to also capture \emph{global information} based on the pretrained knowledge.

Our graph convolution is based on the Graph Convolutional Network (GCN) proposed by~\citet{Kipf:2016tc}. At each layer , we compute the representation of a node  as follows:  


where  is a set of nodes with incoming edges to  and  itself,  is the degree of , and  is a parameter. 

We also consider the variant relational GCN (RGCN)~\cite{Schlichtkrull2018ModelingRD} as graph convolution. RGCN allows capturing the reverse edge direction so that we can consider the differences in the incoming and outgoing relations, which has shown to be beneficial~\cite{beck-etal-2018-graph}. In particular, the node representation is computed as:

where  denotes the set of relations, i.e., the edge types \emph{default} and \emph{reverse},  denotes the set of neighbors under relation , and  encodes the edge type between the nodes  and . 


Note that \graphadapter computes the refined structural node representation  based on the local node context, using as input the global representation  generated by the current PLM encoder layer. In this way, the model is able to capture both the global context based on the PLM linguistic knowledge and the local context based on the graph knowledge. Finally, we employ \vanilladapter into the decoder in order to adapt the language model to the graph-to-text task.
 


\section{Experiments}
\label{sec:exps}

Our models are initialized with pre-trained T5 \hbox{\cite{2019t5}}, but our approach can be combined with other PLMs such as BART \cite{lewis2019bart}. Our implementation is based on Hugging Face Transformer models \citep{wolf2019huggingfaces}. We use T5\textsubscript{base} for all experiments and report results with T5\textsubscript{large} for the test sets.\footnote{Hyperparameter details are in the appendix~\ref{appe:hyperparameters}.} We use the Adam optimizer \cite{kingma:adam} and employ a linearly decreasing learning rate schedule without warm-up. BLEU is used for the stopping criterion. Following recent work~\cite{mager2020gpttoo,zhang-etal-2020-lightweight}, we evaluate our proposed models on LDC2017T10 and LDC2020T02 corpora. 

\paragraph{Evaluation.} We evaluate the results with BLEU \cite{papineni-etal-2002-bleu} and chrF++ \cite{popovic-2015-chrf} metrics. We also report the meaning () component of the -score \cite{opitz-frank-2021-towards}, which measures how well the source AMR graph can be reconstructed from the generated sentence. We use BERTScore~\cite{bert-score} allowing a semantic evaluation that depends less on the surface forms. Finally, we also perform a human evaluation (\S\ref{sec:humanEval}).

\begin{table}[t]
\centering
\resizebox{\columnwidth}{!}{
{\renewcommand{\arraystretch}{0.9}

\begin{tabular}{@{\hspace*{1mm}}l@{\hspace*{3mm}}l@{\hspace*{3mm}}l@{\hspace*{3mm}}l@{\hspace*{3mm}}l@{\hspace*{1mm}}}
\toprule
 &\textbf{BLEU} &\textbf{chrF++}  & \,\,\,  & \textbf{BERT}  \\
 \midrule
 
\citet{mager2020gpttoo} & 33.0 & 63.9 & \,\,\,\,\,\,\,- & \,\,\,\,\,\,\,-\\
\citet{zhang-etal-2020-lightweight} & 33.6 & 63.2 & \,\,\,\,\,\,\,- & \,\,\,\,\,\,\,-\\
\citet{harkous2020text} & 37.7 & \,\,\,\,\,- & \,\,\,\,\,\,\,- & \,\,\,\,\,\,\,-\\
\citet{hoyle2020promoting} & 44.9 & \,\,\,\,\,- & 76.54 & \,\,\,\,\,\,\,-\\
\citet{ribeiro2020investigating} & 45.8 & 72.5 & \,\,\,\,\,\,\,- & \,\,\,\,\,\,\,-\0.15cm]
\toprule
 \multicolumn{5}{c}{T5\textsubscript{base}}   \\
 \midrule
\finetune & 38.3{\small 0.3} & 68.6{\small 0.1} & 77.8{\small 0.3} & 95.5{\small 0.1}\\
\finetunetop{\small(14.8\%)} & 29.9{\small 0.1} & 63.0{\small 0.1} & 74.1{\small 0.2} & 94.4{\small 0.2}\\
\finetunebottom{\small(14.8\%)} & 35.9{\small 0.3} & 67.0{\small 0.2} & 76.9{\small 0.1} & 95.3{\small 0.1}\\
\vanilladapter{\small(8.5\%)} & 38.7{\small 0.4} & 69.2{\small 0.2} & 78.3{\small 0.1} & 95.6{\small 0.1}\\
\graphadapter{\small-GCN(2.1\%)} & 39.0{\small 0.3} & 69.1{\small 0.2} & 78.4{\small 0.2} & 95.7{\small 0.2}\\
\graphadapter{\small-GCN(8.5\%)} & 41.0{\small 0.5} & 70.0{\small 0.2} & 78.4{\small 0.1} & 95.7{\small 0.1}\\
\graphadapter{\small-RGCN(6.3\%)} & \textbf{44.0}{\small 0.3} &
\textbf{71.2}{\small 0.2} & \textbf{79.4}{\small 0.1} &
\textbf{95.9}{\small 0.2}\\

\midrule
\multicolumn{5}{c}{T5\textsubscript{large}}   \\

\midrule
\finetune & 41.2{\small 0.5} & 70.2{\small 0.2}& 78.0{\small 0.1} &95.8{\small 0.2}\\
\finetunetop{\small(7.9\%)} & 28.8{\small 0.4} & 61.8{\small 0.5} & 73.9{\small 0.2} & 94.1{\small 0.2}\\
\finetunebottom{\small(7.9\%)} & 37.6{\small 0.3} & 68.0{\small 0.2} & 77.2{\small 0.2} & 95.5{\small 0.1}\\
\vanilladapter{\small(6.8\%)} & 42.9{\small 0.3} & 71.6{\small 0.2} & 78.9{\small 0.1} & 96.1{\small 0.1}\\
\graphadapter{\small-GCN(1.7\%)} & 44.1{\small 0.4} & 71.8{\small 0.3} & 79.1{\small 0.1} & 96.1{\small 0.2}\\
\graphadapter{\small-GCN(6.8\%)} & 45.8{\small 0.2} & 72.5{\small 0.1} & 79.3{\small 0.2} & 96.2{\small 0.1}\\
\graphadapter{\small-RGCN(5.1\%)} & \textbf{46.6}{\small 0.3} & \textbf{72.9}{\small 0.2} &\textbf{79.6}{\small 0.1} & \textbf{96.3}{\small 0.1}\\
\bottomrule
\end{tabular}}}
\caption{Results on the LDC2017T10 test set. Mean (s.d.) over 4 seeds.}
\label{tab:testsetresults}
\end{table}


\subsection{Main Results}

We compare \graphadapter with four methods: fine-tuning (\finetune), fine-tuning only the top or bottom 2 layers (\finetunetop, \finetunebottom) and \vanilladapter. All models use the same graph linearization generated by the depth-first traversal. We also report recent state-of-the-art results on both datasets. Tables~\ref{tab:testsetresults} and \ref{tab:testsetresults-ldc2020} show the results.
 

We find that training only 5.1\% task-specific parameters, \graphadapterrgcn achieves a BLEU score of 46.6 in LDC2017T10, substantially improving over \finetune and other lightweight baselines (\vanilladapter, \finetunetop, \finetunebottom), and outperforming \citet{ribeiro2020investigating} and \citet{hoyle2020promoting} which fine-tune T5 updating significantly more parameters. \graphadapter also achieves state-of-the-art performance on LDC2020T02, considerably improving over \citet{Micheleamr}, which implicitly models the graph structure information using linearization techniques.

In general, \graphadapter is better than \vanilladapter when training the same number of parameters, and slightly better even when training only 1.7\% of the parameters for both datasets. This highlights that the gains not only come from using an adapter architecture, but from considering the graph connectivity. \graphadapterrgcn is more effective than \graphadaptergcn using fewer parameters, demonstrating that considering reverse relations is advantageous. \vanilladapter is consistently better than \finetune, agreeing with our intuition of catastrophic forgetting when fine-tuning. Interestingly, in contrast to popular strategies that focus on upper layers in fine-tuning~\cite{howard-ruder-2018-universal, pmlr-v97-houlsby19a, li2021prefixtuning}, \finetunebottom's performance is better than \finetunetop's, suggesting that lower layers have a significant impact in adapting the PLM to structured data.

\begin{table}[t]
\centering
\resizebox{\columnwidth}{!}{
{\renewcommand{\arraystretch}{0.9}

\begin{tabular}{@{\hspace*{1mm}}l@{\hspace*{3mm}}l@{\hspace*{3mm}}l@{\hspace*{3mm}}l@{\hspace*{3mm}}l@{\hspace*{1mm}}}
\toprule
 &\textbf{BLEU} &\textbf{chrF++}  & \,\,\,  &\textbf{BERT}  \\
 \midrule

\citet{zhang-etal-2020-lightweight} & 34.3 & 63.7 & \,\,\,\,\,\,\,- & \,\,\,\,\,\,\,-\\
\citet{Micheleamr} & 44.9 & 72.9 & \,\,\,\,\,\,\,- & \,\,\,\,\,\,\,-\\

\midrule
\multicolumn{4}{c}{T5\textsubscript{large}}   \\
\midrule
\finetune & 41.6{\small 0.6} & 70.4{\small 0.5} & 78.5{\small 0.2} & 96.0{\small 0.1}\\
\finetunetop{\small(7.9\%)} & 33.4{\small 0.5} &  63.5{\small 0.3} & 73.4{\small 0.4}& 94.3{\small 0.1}\\
\finetunebottom{\small(7.9\%)} & 38.2{\small 0.2} &  68.3{\small 0.1} & 78.1{\small 0.2}& 95.6{\small 0.1}\\
\vanilladapter{\small(6.8\%)} & 43.0{\small 0.2} & 71.3{\small 0.2} & 79.3{\small 0.1} & 96.2{\small 0.1}\\
\graphadapter{\small-GCN(1.7\%)} & 46.2{\small 0.2} & 71.8{\small 0.2} & 79.4{\small 0.3}  & 96.0{\small 0.2}\\
\graphadapter{\small-GCN(6.8\%)}& 47.1{\small 0.4} & 72.5{\small 0.1}  & 79.7{\small 0.2} & 96.2{\small 0.1} \\
\graphadapter{\small-RGCN(5.1\%)} & \textbf{48.0}{\small 0.2} & \textbf{73.2}{\small 0.1} & \textbf{80.1}{\small 0.3} & \textbf{96.3}{\small 0.1}\\
\bottomrule
\end{tabular}}}
\caption{Results on the LDC2020T02 test set.}
\label{tab:testsetresults-ldc2020}
\end{table}

Different from our work, both \citet{mager2020gpttoo} and \citet{ribeiro2020investigating} use the {\small\textsc{penman}} notation which makes the input much longer (containing more tokens), and demonstrate that this representation is able to achieve strong results -- this is orthogonal to our \graphadapter representation and can be incorporated in future work. 

\begin{table}[t]
\small
\centering
{\renewcommand{\arraystretch}{0.9}
\setlength\tabcolsep{2pt}
\setlength{\belowrulesep}{1pt}
\setlength{\aboverulesep}{1pt}
\begin{tabular}{c c c c} 
\toprule
\textbf{Graph Size} & \multicolumn{1}{c}{\vanilladapter} & \multicolumn{1}{c}{\graphadapterrgcn} &  \\
\midrule
 {\small All} &   &  &\\
\midrule
01-30 &  &  &\\
31-60 &  &  &\\
 &  &  &\\
\bottomrule
\end{tabular}}


\caption{Meaning similarity obtained in the human evaluation. The ranking was determined by Mann-Whitney tests with . Difference between systems which have a letter in common is not statistically significant.}
\label{tab:humanevevaluation}
\end{table}

Overall, the results indicate that explicitly considering the graph structure using an adapter mechanism is effective for AMR-to-text generation, significantly reducing the number of trained parameters while improving generation quality. 



\subsection{Human Evaluation}
\label{sec:humanEval}
To further assess the quality of the generated texts by the adapter-based models in LDC2020T02, we conduct a human evaluation via crowdsourcing using Amazon Mechanical Turk.  We follow previous work~\cite{ribeiro-etal-2019-enhancing,castro-ferreira-etal-2019-neural} and evaluate the \emph{meaning similarity}, i.e., how close in meaning is the generated text to the reference sentence.\footnote{We also assessed the \emph{fluency} of the texts and the differences between the models were not statistically significant.} We divide the datapoints into 3 different sets by by the graph size, i.e., the number of nodes, after converting edges into nodes (cf.\ \S\ref{sec:graphrep}). This setting allows us to evaluate the performance of the models based on the complexity of the AMR graph.

We randomly select 100 generated texts for each set and each model (total of 600), which annotators then rate on a 1-7 Likert scale. For each text we collect scores from 3 annotators and use MACE~\cite{hovy-etal-2013-learning}, a Bayesian model that incorporates the reliability of individual workers, to merge sentence-level labels.\footnote{Refer to Appendix~\ref{appe:humaneval} for a detailed description of the human evaluation.} Table~\ref{tab:humanevevaluation} shows that \graphadapter improves the meaning similarity over \vanilladapter with statistically significant margins (). Note that the gains mainly come from datapoints with  nodes, indicating that \graphadapter is better when encoding larger graphs.



\begin{figure}[t]
\includegraphics[width=0.48\textwidth]{images/graphs_nopar3.pdf}
    \caption{(a) Impact (measure with BLEU) of the number of parameters in the LDC2017T10 dev set. (b) Performance in the LDC2017T10 test set when experimenting with different amounts of training data.}
    \label{fig:hidden}
\end{figure}

\subsection{Detailed Discussion}

\paragraph{Parameter/Performance Trade-off.}

We investigate how the number of parameters affects the models. A higher hidden dimensionality means more trainable parameters, and smaller adapters introduce fewer parameters at a possible cost to performance. That is, the adapter size controls the parameter efficiency. Figure~\ref{fig:hidden}a shows the effect of the number of trained parameters in the performance measured using BLEU. Each point in the \vanilladapter and \graphadapter curves represents a hidden dimension in the range . \graphadaptergcn is consistently better than \vanilladapter over all model capacities, even though both approaches train the same number of parameters. \graphadapterrgcn achieves similar performance than \finetune when training only 0.8\% of the parameters whereas \vanilladapter achieves similar performance to 8.5\%, demonstrating the effectiveness of injecting the graph structure into the PLM.

\paragraph{Low-data Setting.}
Previous work \cite{li2021prefixtuning} has shown that lightweight fine-tuning has an advantage in some generation tasks when the training size is smaller. Therefore, we investigate how \graphadapter behaves in a low-data setting. We subsample the LDC2017T10 training set to analyze different smaller training sets. For each size, we sample 5 different datasets and average over 2 training random seeds. Thus, we average over 10 models to get an estimate for each low-data setting.\footnote{We use the LDC2017T10 dev set to choose hyperparameters and do early stopping.} Figure~\ref{fig:hidden}b shows the results. First note that both adapter-based approaches improve over \finetune. When training with only 1000 datapoints, \graphadapter outperforms \finetune by 8.2 BLEU points. Also note that the gap between \vanilladapter and \finetune decreases when the size of the training set increases. In general, \graphadapter outperforms \finetune and \vanilladapter in low-resource scenarios by 7.3 and 4.8 BLEU points on average, respectively, whereas requiring much fewer trained parameters than \finetune and fewer number of parameters than \vanilladapter. 

\begin{table}[!t]

	\small
	\centering
	\setlength{\tabcolsep}{3pt}
	\setlength\extrarowheight{-4pt}
	\begin{tabular}{p{7.5cm}}
		\toprule
		(b / break-up-08 \\
      \quad\quad :ARG1 {\color{red}\textbf{(i / i)}} \\
      \quad\quad :ARG3 (p / person \\
            \quad\quad\quad\quad :ARG0-of (h / have-rel-role-91 \\
                  \quad\quad\quad\quad\quad\quad :ARG1 (p2 / person \\
                        \quad\quad\quad\quad\quad\quad\quad\quad:ARG0-of (h2 / have-rel-role-91 \\
                              \quad\quad\quad\quad\quad\quad\quad\quad\quad\quad:ARG1 {\color{red}\textbf{i}} \\
                              \quad\quad\quad\quad\quad\quad\quad\quad\quad\quad :ARG2 (s3 / son))) \\
                  \quad\quad\quad\quad\quad\quad :ARG2 (f / father))) \\
      \quad\quad :time (s2 / since \\
            \quad\quad\quad\quad :op1 (d / date-entity :month 8))) \\
		\midrule
		{\small\textsc{reference}}: Me and my son's father have been broken up since August.\\
		\midrule
		\finetune{\small\textsc{-2000}}: I've broken up with my son and father since August.\\
		\midrule
		\finetune: I've been with my son's father since August.\\
		\midrule
		\graphadapter{\small\textsc{-2000}}: Since August 8 I have broken up with my son's father. \\
		\midrule
		\graphadapter: I've been breaking up with my son's father since August. \\
		\bottomrule
	\end{tabular}

	\caption{An example of an AMR graph and generated sentences by different models trained on full data and on a low-data setting with 2000 datapoints.}
	\label{tab:sampleamr}
\end{table}

\paragraph{Case Study.}

We perform a case study to provide a better understanding of the \graphadapter's performance. Table~\ref{tab:sampleamr} shows an AMR graph in {\small\textsc{penman}} notation containing reentrancies (marked in bold) and sentences generated by \finetune and \graphadapter trained on the LDC2017T10 full training set and in a low-data setting where the models are trained with 2000 data points. \finetune fails in generating a sentence with the correct concept \emph{break-up} whereas \graphadapter correctly generates a sentence that describes the input graph. The incorrect verb tense is due to lack of tense information in AMR. \finetune{\small\textsc{-2000}} mixes the semantic relation between \emph{I} and \emph{son} (i.e., mistranslation of the edges in the graph) whereas \graphadapter{\small\textsc{-2000}} generates a correct sentence (except by generating the number 8).
Overall, \graphadapter produces a more accurate text output than \finetune by generating correct pronouns and mentions when control verbs and reentrancies are involved, in both full and low-data scenarios.

\begin{table}[t]
\small
\centering
{\renewcommand{\arraystretch}{0.9}

\begin{tabular}{@{\hspace*{1mm}}l@{\hspace*{4mm}}c@{\hspace*{3mm}}c@{\hspace*{1mm}}} 
\toprule
 &  \textbf{BLEU} & \textbf{BERT} \\
\midrule
\finetune & 38.5 &  95.6\\
\midrule
\vanilladapter {\small\xspace\textsc{only enc}}  & 38.5 &  95.7\\
\vanilladapter {\small\xspace\textsc{only dec}}  & 11.6 & 90.3\\
\vanilladapter {\small\xspace\textsc{enc + dec}}  & 38.6 & 95.6\\
\midrule
\graphadaptergcn {\small\xspace\textsc{only enc}}  & 40.3 & 95.9\\
\graphadaptergcn {\small\xspace\textsc{enc + dec}}  & 41.7 & 96.0\\
\bottomrule
\end{tabular}}
\caption{Impact of the adapter modules in the encoder or decoder in the LDC2017T10 dev set. All adapter-based models have the same number of parameters.}
\label{tab:adaptersparameters}
\end{table}

\paragraph{Model Variations.}

In Table~\ref{tab:adaptersparameters}, we report an ablation study on the impact of distinct adapter components, using adapters only in the encoder or decoder. We evaluate different architecture configurations keeping the same number of parameters for a fair comparison. We find that only training adapters in the decoder is not sufficient for a good performance, even having the same number of parameters. This suggests that adapting the PLM encoder to handle graph structures is key in AMR-to-text tasks. Interestingly, the model that only employs \graphadapter in the encoder (i.e., no \vanilladapter is used in the decoder) has a better performance ( BLEU) than using \vanilladapter in both encoder and decoder, highlighting \graphadapter's strong graph encoding abilities. Finally, the best performance is achieved when we employ \graphadapter in the encoder and \vanilladapter in the decoder, reaching 41.7 BLEU points.

\section{Graph Representation Evaluation}
\label{sec:graphrepeval}

In this section, we explore how different graph properties impact the models' abilities to encode the input graph structure.

\subsection{Impact of the Graph Representation}
\label{sec:amrinputrep}
 Inspired by~\citet{damonte-cohen-2019-structural}, we investigate two different approaches when linearizing the AMR: (i) only nodes have explicit representations, whereas edge relations are represented by the adapter parameters using the RGCN;\footnote{We use regularization based on the basis decomposition for relation weights \cite{Schlichtkrull2018ModelingRD} since AMR can contain around 150 different edge types.} and (ii) the sequence of nodes and edges using depth-first traversal of the graph. 
 
 We also propose and evaluate three different graph structures based on subwords (cf.\ \S\ref{sec:graphrep}): \emph{rep1}: for each edge, we connect every token from the source node to every token of the target node; \emph{rep2}: we connect the last token of the source node to the first token of the target node and connect the tokens of a node sequentially; \emph{rep3}: we connect the first token of the source node to the first token of the target node and connect the token of a node sequentially. Figure~\ref{fig:amrreps} shows an example of the three representations for an AMR graph structure. Additionally, we also investigate a fully connected graph structure (\emph{complete graph}), that is, similarly to the self-attention mechanism in Transformers, all nodes and edges are connected.
 
 
 \begin{table}[t]
\small
\centering
{\renewcommand{\arraystretch}{0.9}

\begin{tabular}{l@{\hspace*{1mm}}M{2.2cm}@{\hspace*{2mm}}M{0.8cm}M{0.8cm}}  
\toprule
\textbf{Linearization} & \textbf{Graph Representation} & \textbf{BLEU} & \textbf{BERT}  \\
\midrule
\multirow{ 3}{*}{(i) only nodes} & \emph{rep1} & 39.1 & 95.8\\
 & \emph{rep2} & 38.5 & 95.6\\
 & \emph{rep3} & 38.9 & 95.7\\
\midrule
\multirow{ 3}{*}{(ii) nodes and edges} & \emph{rep1} & 41.7 & 96.0\\
& \emph{rep2} & 40.4 & 95.8\\
& \emph{rep3} & 40.8 & 95.9\\
& \emph{complete graph} & 39.4 & 95.8\\

\bottomrule
\end{tabular}}
\caption{Performance on the LDC2017T10 dev set when using different graph representation strategies.}
\label{tab:graphrep}
\end{table}



As shown in Table~\ref{tab:graphrep}, explicitly considering nodes and edges in the graph linearization is beneficial. This approach has the advantage of allowing the model to handle new edge relations during inference, as they are not encoded as model parameters. Note that the \emph{complete graph} representation has relatively inferior performance, again demonstrating the advantage of explicitly encoding the input graph connectivity. 

Finally, we observe that the best configuration is using nodes and edges with \emph{rep1} (see an example in Figure~\ref{fig:amrreps}c). We believe that this is because \emph{rep1} allows direct interactions between all source and target tokens, making all token representations of an AMR node directly influenced by the neighbouring tokens. 


\subsection{Robustness to Graph Linearization}
A critical advantage of modeling the graph structure is to be less dependent on linearization strategies because the graph connectivity is invariant to the graph linearization. We thus are interested in measuring the impact of the graph linearization in the models. 

Following \citet{hoyle2020promoting}, we investigate three different graph linearizations: (i)  {\small\textsc{canon}}: the original order of the canonical human-created linearizations in AMR corpora; (ii) {\small\textsc{reconf}}: the order from the canonical graph linearization is ignored, except for the top node;\footnote{{\small\textsc{reconf}} can significantly modify the linearization, including shifting edge labels (e.g., poss to poss-of).} and (iii) {\small\textsc{random}}: constructs a linearization from a random node in the graph, disregarding all order information from the canonical format, but it remains a valid traversal of the graph. All linearizations are converted to a sequence of node and edge labels using depth-first traversal and used for both training and evaluation. Examples of such graph linearizations are shown in Appendix~\ref{appe:exgraphline}. 

Table~\ref{tab:differentgraphlinearizations} presents the results. Note that while {\small\textsc{reconf}} has a negative impact on all models, \graphadapter has the best performance. \vanilladapter has similar performance gains over \finetune in all graph linearizations. Finally, note that for {\small\textsc{random}}, there is a drastic performance drop in \finetune and the gap between \graphadapter and \finetune is widest ( BLEU), demonstrating that explicitly encoding the graph structure is beneficial and that \graphadapter is much less impacted by different graph linearizations.

\begin{table}[t]
\small
\centering
{\renewcommand{\arraystretch}{0.9}

\begin{tabular}{l@{\hspace*{3mm}}c@{\hspace*{2.6mm}}c@{\hspace*{2.6mm}}c} 
\toprule
 & {\small\textsc{canon}} & {\small\textsc{reconf}} & {\small\textsc{random}}  \\
\midrule
\finetune & 38.0 & 35.6 & 31.3 \\
\vanilladapter  & +0.9 & +0.8 & +0.9\\
\graphadapterrgcn  & \textbf{+4.1}  & \textbf{+3.6} & \textbf{+5.9} \\
\bottomrule
\end{tabular}}
\caption{Differences, with respect to \finetune, in the BLEU score of the LDC2017T10 test set as a function of different graph linearizations.}
\label{tab:differentgraphlinearizations}
\end{table}


\subsection{Graph Properties}

Table~\ref{tab:reentrances} shows the effects of the graph size, graph diameter and reentrancies in the performance. First, note that the BLEU scores decrease as the graph size increases since larger graphs often are more complex. The performance gap between \graphadapter and \finetune becomes larger for relatively larger graphs, showing that \graphadapter is able to better encode complex graphs. As \vanilladapter is not aware of the graph connectivity, it has much worse scores compared to \graphadapter, especially for larger graphs. 

It is expected that the benefit of the \graphadapter will be more evident for AMR graphs containing larger diameter as the encoder is aware of the input graph structure. As seen in Table~\ref{tab:reentrances}, similarly to the graph size, the scores decrease as the graph diameter increases. \graphadapter achieves a clear improvement when handling graphs with  diameter, with a improvement of  BLEU points over \finetune. 

Previous work \cite{damonte-cohen-2019-structural,szubert-etal-2020-role} showed that reentrancies (nodes with multiple parents) pose difficulties in encoding AMRs correctly. Because \graphadapter is the only approach to model reentrancies explicitly, we expect it to deal better with these structures. The gap between \graphadapter and the other models is widest for examples with more reentrancies, confirming our hypothesis. In particular, when graphs contain  reentrancies, \graphadapter has an improvement of  BLEU points compared to \vanilladapter.

\begin{table}[t]
\small
\centering
{\renewcommand{\arraystretch}{0.9}

\begin{tabular}{lccc} 
\toprule
 \textbf{graph size}& 1-30 & 31-60 &   \\
 \textbf{ datapoints} & 548 & 537 & 286  \\
\midrule
\finetune & 40.6 & 37.3 & 38.1 \\
\vanilladapter  & +0.5 & +1.4 & +1.1 \\
\graphadapterrgcn  & \textbf{+2.3}  & \textbf{+4.0} & \textbf{+4.6} \\
\toprule
 \textbf{graph diameter}& 1-10 & 11-20 &   \\
 \textbf{ datapoints} & 384 & 769 & 218  \\
\midrule
\finetune & 43.3 & 37.6 & 38.5  \\
\vanilladapter  & -0.1 & +1.7 & +0.3 \\
\graphadapterrgcn  & \textbf{+0.5}  & \textbf{+4.3} & \textbf{+4.2} \\
\toprule
 \textbf{ reentrancies}& 0 & 1-3 & 4-20  \\
 \textbf{ datapoints} & 619 & 664 & 88  \\
\midrule
\finetune & 42.9 & 38.0 & 31.3 \\
\vanilladapter  & +0.2 & +1.7 & +0.8 \\
\graphadapterrgcn  & \textbf{+3.4}  & \textbf{+4.4} & \textbf{+4.4} \\
\bottomrule
\end{tabular}}
\caption{Differences, with respect to  \finetune, in the BLEU score of the LDC2017T10 test set as a function of the graph size, graph diameter and number of reentrancies.}
\label{tab:reentrances}

\end{table}


\section{Conclusion}

We presented \graphadapter, a novel adapter architecture to explicitly model graph structures into pretrained language models, providing an extensive evaluation of our approach and showing that it achieves state-of-the-art results on two AMR-to-text benchmarks, training much fewer parameters. We also found that \graphadapter is more effective when encoding complex graphs, when trained on fewer datapoints, and is more robust to different graph linearizations and reentrancies. In future work, we plan to consider other graph-to-text tasks, such as those based on Knowledge Graphs.

\section*{Acknowledgments}
We thank our anonymous reviewers for their thoughtful comments. We also would like to thank Jonas Pfeiffer, Jorge Cardona, Juri Opitz, Kevin Stowe, Thy Tran, Tilman Beck and Tim Baumg{\"a}rtner for their feedback on this work. This work has been supported by the German Research Foundation (DFG) as part of the Research Training Group ``Adaptive Preparation of Information form Heterogeneous Sources'' (AIPHES, GRK 1994/1) and as part of the DFG funded project UKP-SQuARE with the number GU 798/29-1. 

\bibliography{anthology,custom}
\bibliographystyle{acl_natbib}

\clearpage
\appendix

\section*{Appendices}

In this supplementary material, we detail experiments' settings and additional information about the human evaluation and graph representations.

\section{Details of Models and Hyperparameters} \label{appe:hyperparameters}
The experiments were executed using the version  of the \emph{transformers} library released by Hugging Face \citep{wolf2019huggingfaces}. In Table \ref{tab:hyper}, we report the hyperparameters used to train the models presented in this paper. We train until the development set BLEU has not improved for 5 epochs.

\begin{table}[!h]
\centering
\resizebox{\columnwidth}{!}{
\begin{tabular}{lccc}
\toprule
         & \textbf{learning rate} & \textbf{batch size} & \textbf{beam search size} \\
\midrule

\finetune & 3e-05  & 4 & 5   \\
\finetunetop & 1e-04  & 4 & 5   \\
\finetunebottom & 1e-04  & 4 & 5   \\
\vanilladapter & 1e-04  & 4 & 5   \\
\graphadapter & 1e-04  & 4 & 5   \\
\bottomrule
\end{tabular}}
\caption{Hyperparameter settings for our methods. }
\label{tab:hyper}
\vspace{-4mm}
\end{table}


\section{Details on the Human Evaluation}
\label{appe:humaneval}
The human evaluation was conducted via Amazon Mechanical Turk. We randomly select 100 generated texts for each of the 3 sets and each adapter model (\vanilladapter, \graphadaptergcn), with a total of 600 texts to be evaluated. The annotators then rate the meaning similarity on a 1-7 Likert scale. For each text, we collect scores from 3 annotators. We use MACE~\cite{hovy-etal-2013-learning} to further improve upon these raw answers by unsupervised estimation of worker trustworthiness and subsequent recovery of the most likely score. Models are ranked according to the mean of sentence-level scores. We defined a filter for all our evaluations, allowing to participate only workers who have more than 5000 HITs approved and with an acceptance rate of 95\% or higher. The task took workers a median time of 1.6 minutes per pair of sentences. We apply a quality control step filtering workers who do not score some faked and known sentences properly or did the experiment in a very short time. 

\section{Example of Graph Linearizations}
\label{appe:exgraphline}

In Table~\ref{tab:sampleamrappendix}, we present three different linearizations for the same AMR graph and its corresponding reference sentence. Figure~\ref{fig:amrexamplesappendix} shows the two possible graphs that are represented by the linearizations. In particular, Figure~\ref{fig:amrexamplesappendix}a shows a graph that is represented by {\small\textsc{canon}} and {\small\textsc{reconf}} linearizations and Figure~\ref{fig:amrexamplesappendix}b shows a graph that is represented by {\small\textsc{random}}. Note that whereas the linearizations can greatly differ from each other, the graph structure for all linearizations remains very similar.

\begin{figure}[h]
    \centering
    \includegraphics[width=.45\textwidth]{images/amr-rec-random.pdf}
    \caption{Two AMR graphs with the same meaning.}
    \label{fig:amrexamplesappendix}
    \vspace{-4mm}
\end{figure}


\begin{table}[!h]

	\small
	\centering
	\setlength{\tabcolsep}{3pt}
	\setlength\extrarowheight{-4pt}
	\begin{tabular}{p{7.5cm}}
		\toprule
		{\small\textsc{canon}} \\
		\\
		    (s / subsidize-01 \\
      \quad\quad :ARG1 (u / utility \\
            \quad\quad\quad\quad:poss (s2 / she) \\
            \quad\quad\quad\quad:mod (a / all))) \\
		\midrule
				{\small\textsc{reconf}} \\
		\\
		    (s / subsidize-01 \\
      \quad\quad :ARG1 (u / utility \\
            \quad\quad\quad\quad :mod (a / all) \\
            \quad\quad\quad\quad :poss (s2 / she))) \\
            
		\midrule
						{\small\textsc{random}} \\
		\\
		    (s2 / she \\
      \quad\quad :poss-of (u / utility \\
            \quad\quad\quad\quad:ARG1-of (s / subsidize-01) \\
            \quad\quad\quad\quad:mod (a / all))) \\
		\midrule
		{\small\textsc{sentence}}: Her utilities are all subsidized.\\
		\bottomrule
	\end{tabular}
\caption{Different linearizations for an AMR graph.}
	
	\label{tab:sampleamrappendix}
	\vspace{-4mm}
\end{table}



\end{document}
