\documentclass{LMCS}

\def\doi{9(2:07)2013}
\lmcsheading {\doi}
{1--34}
{}
{}
{Mar.~30, 2012}
{Jun.~21, 2013}
{}

\usepackage{amssymb}
\usepackage{amsmath}
\usepackage{graphicx}
\usepackage{gastex}
\usepackage{ifpdf}
\usepackage{color}
\usepackage{wrapfig}
\usepackage{gastex}
\usepackage{enumerate,hyperref}

\def\phi{\varphi}
\def\epsilon{\varepsilon}
\DeclareMathOperator{\interleave}{||}
\DeclareMathOperator{\U}{\text{\sf U}}
\DeclareMathOperator{\X}{\text{\sf X}}
\newtheorem{counter}{Counterexample}
\newtheorem{definition}{Definition}
\newtheorem{example}{Example}
\newtheorem{theorem}{Theorem}
\newtheorem{lemma}{Lemma}
\newtheorem{remark}{Remark}
\def\tilde{\widetilde}

\def\jcg#1{\color{red}JCG: \texttt{#1} :GCJ\color{black}}

\newcommand{\TRANA}[3]{#1\xrightarrow[]{#2}#3}
\newcommand{\TRAN}[2]{#1\rightarrow #2}
\newcommand{\TRANP}[2]{#1\rightarrow_{{\sf P}}#2}
\newcommand{\BSP}{\ensuremath{\sim_{{\sf P}}}}
\newcommand{\nBSP}{\ensuremath{\nsim_{{\sf P}}}}
\newcommand{\BS}{\sim^{*}}
\newcommand{\nBS}{\nsim}
\newcommand{\Si}{\prec}
\newcommand{\nSi}{\nprec}
\newcommand{\AP}{\mathit{AP}}
\newcommand{\PA}{{\sf PA}}
\newcommand{\SP}{\ensuremath{\prec_{{\sf P}}}}
\newcommand{\nSP}{\ensuremath{\nprec_{{\sf P}}}}
\newcommand{\DTMC}{{\sf DTMC}}

\newcommand{\boundTRAN}[2]{\overset{#1,#2}{\Longrightarrow}}



\newcommand{\WTRAN}[3]{#1\Rightarrow_{#3} #2}
\newcommand{\WBS}{\approx}
\newcommand{\nWBS}{\napprox}
\newcommand{\WSi}{\precapprox}
\newcommand{\nWSi}{\precnapprox}
\newcommand{\MAXPROB}[1]{\mathit{Pr}(#1)}

\newcommand{\PCTL}{{\sf PCTL}}
\newcommand{\PCTLS}{{\sf PCTL}^{*}}

\newcommand{\CTL}{{\sf CTL}}
\newcommand{\CTLS}{{\sf CTL}^{*}}

\newcommand{\SDIS}[1]{\tilde{#1}}
\newcommand{\iBS}[1]{\sim_{#1}}
\newcommand{\inBS}[1]{\nsim_{#1}}
\newcommand{\infBS}{\sim}
\newcommand{\iBSB}[1]{\sim_{#1}^{{\sf b}}}
\newcommand{\inBSB}[1]{\nsim_{#1}^{{\sf b}}}
\newcommand{\infBSB}{\sim^{{\sf b}}}
\newcommand{\EPCTL}{\sim_{\PCTL}}
\newcommand{\EPCTLM}{\sim_{\PCTL^{-}}}
\newcommand{\iEPCTLM}[1]{\sim_{\PCTL^{-}_{#1}}}
\newcommand{\inEPCTLM}[1]{\nsim_{\PCTL^{-}_{#1}}}
\newcommand{\EPCTLS}{\sim_{\PCTL^{*}}}
\newcommand{\nEPCTLS}{\nsim_{\PCTL^{*}}}
\newcommand{\EPCTLSM}{\sim_{\PCTL^{*-}}}
\newcommand{\iEPCTLSM}[1]{\sim_{\PCTL_{#1}^{*-}}}
\newcommand{\nEPCTL}{\nsim_{\PCTL}}


\newcommand{\iSi}[1]{\prec_{#1}}
\newcommand{\inSi}[1]{\nprec_{#1}}
\newcommand{\infSi}{\prec}
\newcommand{\iBSi}[1]{\prec_{#1}^{{\sf b}}}
\newcommand{\inBSi}[1]{\nprec_{#1}^{{\sf b}}}
\newcommand{\infBSi}{\prec^{{\sf b}}}

\newcommand{\SEPCTL}{\prec_{\PCTL}}
\newcommand{\SEPCTLM}{\prec_{\PCTL^{-}}}
\newcommand{\iSEPCTLM}[1]{\prec_{\PCTL^{-}_{#1}}}
\newcommand{\SEPCTLS}{\prec_{\PCTL^{*}}}
\newcommand{\nSEPCTLS}{\nprec_{\PCTL^{*}}}
\newcommand{\SEPCTLSM}{\prec_{\PCTL^{*-}}}
\newcommand{\iSEPCTLSM}[1]{\prec_{\PCTL_{#1}^{*-}}}
\newcommand{\nSEPCTL}{\nprec_{\PCTL}}

\newcommand{\PAR}[2]{#1\interleave#2}


\newcommand{\bBS}{\simeq}
\newcommand{\bBSP}{\simeq_{{\sf P}}}

\newcommand{\bSi}{\simeq}
\newcommand{\bSiP}{\precapprox_{{\sf P}}}


\newcommand{\MC}[1]{\mathcal{#1}}
\newcommand{\MI}[1]{\mathit{#1}}
\newcommand{\DSI}[1][\MC{R}]{\sqsubseteq_{#1}}

\newcommand{\DIRAC}[1]{\mathcal{D}_{#1}}
\newcommand{\TR}{\mathit{tr}}
\newcommand{\TRS}{\tilde{\mathit{tr}}}
\newcommand{\DEPTH}{\mathit{Depth}}
\newcommand{\EPCTLWN}{\sim_{\PCTL_{\backslash\X}}}
\newcommand{\EPCTLSWN}{\sim_{\PCTL^{*}_{\backslash\X}}}
\newcommand{\nEPCTLWN}{\nsim_{\PCTL_{\backslash\X}}}
\newcommand{\nEPCTLSWN}{\nsim_{\PCTL^{*}_{\backslash\X}}}

\newcommand{\SEPCTLWN}{\precapprox_{\PCTL_{\backslash\X}}}
\newcommand{\SEPCTLSWN}{\precapprox_{\PCTL^{*}_{\backslash\X}}}
\newcommand{\nSEPCTLWN}{\precnapprox_{\PCTL_{\backslash\X}}}
\newcommand{\nSEPCTLSWN}{\precnapprox_{\PCTL^{*}_{\backslash\X}}}


\newcommand{\WBSB}{\approx^{{\sf b}}}
\newcommand{\nWBSB}{\napprox^{{\sf b}}}

\newcommand{\WBSi}{\precapprox^{{\sf b}}}
\newcommand{\nWBSi}{\precnapprox^{{\sf b}}}



\newcommand{\POST}{\mathit{Post}^{*}}
\newcommand{\infPath}[1]{#1^{\infty}}

\newcommand{\bTRAN}[2]{#1\Rightarrow^{\MC{R}}#2}
\newcommand{\bTRANP}[2]{#1\Rightarrow^{\MC{R}}_{\text{P}}#2}

\newcommand{\TRUE}{\mathit{true}}

\newcommand{\MEASURE}{\mathit{Prob}}
\newcommand{\MEASUREONE}{\mathit{PreCap}}
\newcommand{\MEASURETWO}{\mathit{PostCap}}

\newcommand{\SUP}{\mathit{Sup}}
\newcommand{\ABS}[1]{|#1|}
\newcommand{\SUPP}{\mathit{Supp}}

\newcommand{\UPWARD}[2]{#1^{+}#2}
\newcommand{\DOWNWARD}[2]{#1^{\downarrow}#2}


\begin{document}
\title[Bisimulations Meet PCTL Equivalences for Probabilistic Automata]{Bisimulations Meet PCTL Equivalences for Probabilistic Automata\rsuper*}

\author[L.~Song]{Lei Song\rsuper a}	\address{{\lsuper a}Max-Planck-Institut f\"{u}r Informatik, and
Saarland University -- Computer Science, Germany}	 \email{song@cs.uni-saarland.de}  

\author[L.~Zhang]{Lijun Zhang\rsuper b}	\address{{\lsuper b}State Key Laboratory of Computer Science, Institute of Software, Chinese Academy of Sciences,
DTU Informatics,  Technical University of Denmark, and
Saarland University -- Computer Science, Germany}	 \email{zhang@imm.dtu.dk and zhanglj@ios.ac.cn (corresponding author)}  

\author[J. C. Godskesen]{Jens Chr. Godskesen\rsuper c}	\address{{\lsuper c}IT University of Copenhagen, Denmark}	 \email{jcg@itu.dk}  

\author[F. Nielson]{Flemming Nielson\rsuper d}	\address{{\lsuper d}DTU Compute,  Technical University of Denmark}	\email{fnie@dtu.dk}  

\keywords{PCTL, Probabilistic automata, Characterization, Bisimulation}
\subjclass{G.3, F.4.1, F.3.1}
\ACMCCS{[{\bf  Mathematics of computing}]: Probability and statistics---Stochastic processes---Markov processes;  [{\bf Theory of computation}]: Logic---Modal and temporal logics;  Semantics and reasoning---Program reasoning---Program verification; [{\bf General and reference}]: Cross-computing tools and techniques---Performance; Cross-computing tools and techniques---Verification}

\titlecomment{{\lsuper*}An extended abstract of the paper has appeared in \cite{SongZG2011}}

\begin{abstract}
  Probabilistic automata (s) have been successfully applied in
  formal verification of concurrent and stochastic systems. Efficient
  model checking algorithms have been studied, where the most often
  used logics for expressing properties are based on probabilistic computation tree logic () and its
  extension . Various behavioral equivalences are proposed,
  as a powerful tool for abstraction and compositional minimization
  for s. Unfortunately, the equivalences are
  well-known to be sound, but not complete with respect to the logical
  equivalences induced by  or .  The desire of a
  both sound and complete behavioral equivalence has been pointed
  out by Segala in~\cite{Segala-thesis}, but remains open throughout
  the years.  In this paper we introduce novel notions
  of strong bisimulation relations, which characterize  and
   exactly. We extend weak bisimulations that characterize 
  and  without next operator, respectively.  Further, we
  also extend the framework to simulation preorders. Thus, our paper bridges
  the gap between logical and behavioral equivalences and preorders in this setting.
\end{abstract}
\maketitle
\section{Introduction}
Probabilistic automata (s)~\cite{SegalaL95} have been
successfully applied in formal verification of concurrent and
stochastic systems. Efficient model checking algorithms have been
studied, where properties are mostly expressed by the probabilistic
computation tree logic ()~\cite{hansson1994logic} and its
extension ~\cite{Aziz1995UWT} for Markov chains, and later
extended in~\cite{bianco1995model} for Markov decision processes.


To combat the infamous state space problem in model checking, various
behavioral equivalences, including strong and weak bisimulations, are
proposed for stochastic models including
s~\cite{LarsenS89,larsen1991bisimulation,Segala-thesis,baier2003comparative,SegalaL95}. Indeed,
they turn out to be a powerful tool for abstraction for s, since
bisimilar states imply that they satisfy exactly the same  and
 formulas, thus can be grouped together,
allowing one to construct smaller quotient automata before analyzing
the models.  In practice, bisimulation based minimization is
extensively studied in the literatures to leverage the state space
explosion, for instance see~\cite{CattaniS02,BaierEM00,KatoenKZJ07}.
Moreover, the nice compositional theory for s is exploited for
compositional minimization~\cite{BoudaliCS09}, namely minimizing the
automata before composing the components together.


An interesting question is whether the reverse holds as well, namely whether logical equivalences imply bisimulation equivalences?
For Markov chains, i.e.,
s without non-deterministic choices, the answer is affirmative~\cite{Aziz1995UWT,BaierKHW05}.
Unfortunately, the completeness does not hold in general, namely 
equivalence is strictly coarser than bisimulation and its variant
\emph{probabilistic bisimulation}~\cite{SegalaL95} for s.

The main reason for the gap can be illustrated by the following
example. Consider the  in Fig.~\ref{fig:counterexample}
assuming that  are three absorbing states with different
state properties. It is easy to see that   and  are  equivalent:
the additional middle transition out of  does not change the extreme probabilities,
the intervals of probabilities in which the three observing states can be reached are not changed.
However existing bisimulations differentiate  and ,
mainly because the middle transition out of  cannot be matched by
any transition (or combined transition) of . Bisimulation requires
that the complete distribution of a transition must be matched, which is in this case too
strong, as it differentiates states satisfying the same 
formulas.


\begin{figure}[!t]
  \begin{center}
    \begin{picture}(80, 40)(0,20)
    \gasset{Nadjust=n,Nw=6, Nh=6,Nmr=3}
    \node(S)(0,60){}
    \node(Saa)(-30,33.5){}
    \node(Sba)(-20,25.5){}
    \node(Sca)(-10,21.3){}
    \node(Sab)(30,33.5){}
    \node(Sbb)(20,25.5){}
    \node(Scb)(10,21.3){}
    \node(R)(70,60){}
    \node(Raa)(31,51){}
    \node(Rba)(36,39){}
    \node(Rca)(46,28){}
    \node(Rab)(58,22){}
    \node(Rbb)(70,20){}
    \node(Rcb)(82,22){}
    \node(Rac)(109,51){}
    \node(Rbc)(104,39){}
    \node(Rcc)(94,28){}
    \drawcurve[AHnb=0,ATnb=0,dash={0.2 0.5}0](-15,46.75)(-10,42.75)(-5,40.65)
    \drawcurve[AHnb=0,ATnb=0,dash={0.2 0.5}0](15,46.75)(10,42.75)(5,40.65)
    \drawcurve[AHnb=0,ATnb=0,dash={0.2 0.5}0](50.5,55.5)(53,49.5)(58,44)
    \drawcurve[AHnb=0,ATnb=0,dash={0.2 0.5}0](64,41)(70,40)(76,41)
    \drawcurve[AHnb=0,ATnb=0,dash={0.2 0.5}0](89.5,55.5)(87,49.5)(82,44)
    \gasset{ELdistC=y,ELdist=0}
    \drawedge[ELside=r,ELpos=70](S,Saa){{\scriptsize\colorbox{white} {0.3}}}
    \drawedge[ELside=r,ELpos=70](S,Sba){{\scriptsize\colorbox{white}{ 0.3}}}
    \drawedge[ELside=r,ELpos=70](S,Sca){{\scriptsize \colorbox{white}{0.4}}}
    \drawedge[ELpos=70](S,Sab){{\scriptsize\colorbox{white} {0.5}}}
    \drawedge[ELpos=70](S,Sbb){{\scriptsize\colorbox{white} {0.4}}}
    \drawedge[ELpos=70](S,Scb){{\scriptsize\colorbox{white} {0.1}}}
    \drawedge[ELside=r,ELpos=70](R,Raa){{\scriptsize\colorbox{white}{ 0.3}}}
    \drawedge[ELside=r,ELpos=70](R,Rba){{\scriptsize\colorbox{white}{ 0.3}}}
    \drawedge[ELside=r,ELpos=70](R,Rca){{\scriptsize\colorbox{white}{ 0.4}}}
    \drawedge[ELside=r,ELpos=70](R,Rab){{\scriptsize\colorbox{white}{ 0.4}}}
    \drawedge[ELside=r,ELpos=70](R,Rbb){{\scriptsize\colorbox{white}{ 0.3}}}
    \drawedge[ELside=r,ELpos=70](R,Rcb){{\scriptsize\colorbox{white}{ 0.3}}}
    \drawedge[ELpos=70](R,Rac){{\scriptsize\colorbox{white}{ 0.5}}}
    \drawedge[ELpos=70](R,Rbc){{\scriptsize\colorbox{white}{ 0.4}}}
    \drawedge[ELpos=70](R,Rcc){{\scriptsize\colorbox{white}{ 0.1}}}
    \end{picture}
    \end{center}
  \caption{\label{fig:counterexample}Counterexample of strong probabilistic bisimulation.}
\end{figure}



For s, the desire of a both sound and complete behavioral
equivalence has been pointed out in~\cite{Segala-thesis} (see section
13.2.7), but remains open throughout the years.
Such a sound and complete relation is not only of theoretical interests: in practice
it would allow us to construct the minimal quotient automata for
checking  and  formulas, which are arguably the most often used
logics for specifying properties over s.
 In this paper we bridge this gap by introducing novel
notions of behavioral equivalences which characterize (both soundly
and completely) ,  and their sub logics.
Summarizing, our contributions are:
\begin{iteMize}{}
\item A new  bisimulation characterizing
 soundly and completely. The bisimulation arises from a converging sequence of equivalence relations, each of which characterizes bounded .
\item Branching bisimulations which correspond to  and bounded
   equivalences.
\item We then extend our definitions to weak bisimulations, which characterize sub logics of  and  with only unbounded path formulas.
\item Further, we extend the
  framework to simulations as well as their characterizations,
extend the results to countable states, and discuss the coarsest congruent bisimulation  and simulation relations.
\end{iteMize}

\paragraph{Organization of the paper.}
Section \ref{sec:pre} introduces some notations. In Section \ref{sec:pa} we recall definitions of probabilistic automata and bisimulation relations by Segala~\cite{Segala-thesis}. We also recall the logic  and its sub logics. Section \ref{sec:strong} introduces the novel strong and strong branching
bisimulations, and proves that they agree with  and 
equivalences, respectively. Section \ref{sec:weak}  extends them to weak
(branching) bisimulations, and Section \ref{sec:simulation} extends the framework to simulations. We discuss the extension to countable states in Section \ref{sec:countable} and the coarsest congruent bisimulations  and simulations in Section \ref{sec:congruent}. In Section \ref{sec:related} we discuss
related work, and Section \ref{sec:conclusion} concludes the paper.

\section{Preliminaries}\label{sec:pre}
\paragraph{Distributions.}
For a \emph{countable} set , a distribution is a function  satisfying . We denote by
 the set of distributions over . We shall use
 and  to range over  and
, respectively. Given a set of distributions
, and a set of positive weights
 such that , the
\emph{convex combination}  is
a distribution such that  for each . The support of  is
defined by . For an
equivalence relation , we write  if it holds
that  for all equivalence classes . A
distribution  is called \emph{Dirac} if ,
and we let  denote the Dirac distribution with
.


\paragraph{Downward Closure.} We define the downward closure of
a set of states.
For a relation  over  and , define
 as the least set satisfying:
i) , ii)  and
 implies .
We say  is  \emph{downward closed} iff .
We use  as the shorthand of , 
and  to 
denote the set of all  downward closed sets. If  is an equivalence
relation, then  is called  closed if .

\section{Probabilistic automata, \texorpdfstring{}{PCTL*} and bisimulations}\label{sec:pa}
\subsection{Probabilistic automata}
We recall the notion of probabilistic automata introduced by Segala~\cite{Segala-thesis}.  We omit the set of actions, since they do not
appear in the logic  we shall consider later. This is actually not
a restriction, since the bisimulation we shall introduce later
can be extended to s with
actions directly.
\begin{defi}\label{def:automata}
  A \emph{probabilistic automaton} is a tuple
   where  is a finite set of
  states,  is a
  transition relation such that for each , 
  there exists  for some , 
   is the initial state,
   is a set of atomic propositions, and  is a labeling function.
\end{defi}
We only consider image-finite s, i.e.
 is finite for each . A transition
 is often denoted by . Moreover, we
write  iff for each 
there exists  such that
.


A \emph{path} is a finite or infinite sequence
 of states. 
We use  and  to denote the last
state of  and the length of  respectively if 
is finite. The set  is the set containing all paths, and
 contains those starting from . Similarly,
 is the set of finite paths, and
 only contains finite paths starting from . Also we use
 to denote the -th state for , 
to denote the prefix of  ending at , and
 to denote the suffix of  starting from
.

We introduce the definition of \emph{schedulers} to resolve non-determinism.
A scheduler is a
function  such that
 implies
. A scheduler  is
\emph{deterministic} if it returns only Dirac distributions, that
is, the next step is chosen deterministically.


The \emph{cone} of a finite path , denoted by ,
is the set of paths having  as their prefix, i.e.,
 where  iff  is a prefix of .
Fixing a starting state  and a scheduler , the measure
 of a cone , where
, is defined inductively as follows:  equals  if , and for ,


Let  be the smallest algebra that
contains all the cones and is closed under complement and countable
unions. By standard measure theory~\cite{halmos1974measure,rudin2006real} this algebra is a
\emph{-algebra} and all its elements are
measurable sets of paths. Given a scheduler ,
  can be extended to a unique
measure on .


Given a relation  over ,  is the \emph{Cartesian} product of
 with itself  times.
Each element of   is a \emph{downward closed path} of length .
Let , and define  for .
For , the
 \emph{downward closed cone}  is defined as
,
where  iff  for .


For distributions  and , we define  by .
Following~\cite{baier2008principles} we also define the interleaving of s:
\begin{defi}\label{def:interleave}
Let  be two s with . The \emph{interleaved parallel composition}  is defined by:

where  and  iff either  and , or  and .
\end{defi}

\subsection{\texorpdfstring{}{PCTL*} and its sub logics}
We introduce the syntax of ~\cite{hansson1994logic} and ~\cite{Aziz1995UWT} which are probabilistic extensions of {\sf CTL} and {\sf CTL} respectively.

The   formulas over the set  of atomic propositions are
formed according to the following grammar:

where , , and . We refer to  and  as () state and
path formulas, respectively.

The satisfaction relation  for state formulas is
 defined in a standard manner for boolean formulas. For the probabilistic operator, it is defined by

The satisfaction relation  for path
formulas is the same as for LTL formulas, that is,





\paragraph{Sub logics.}
The depth of path formula  of  free of  operator, denoted by , is defined by the maximum number of embedded  operators appearing in , that is,
\begin{iteMize}{}
\item ,
\item ,
\item  and
\item .
\end{iteMize}
Then, we let
 be the sub logic of  without the until () operator. Moreover,  is a sub logic of  where
 for each  we have .



The sub logic  is obtained by restricting the path formulas to:

Note the bounded until operator does not appear in  as it
can be encoded by nested next operators.   is defined in a
similar way as .  Moreover we let  be the
sub logic of  where only bounded until operator
 with  is allowed. For all the logics
we have mentioned, we summarize their
differences in syntax of path formulas in
Table~\ref{tab:logics}.

\begin{table}
\caption{Summary of  and its sublogics}\label{tab:logics}
\centering
\begin{tabular}{|l|c|c|}
\hline
Logic &  & Note\3pt]
\hline
 &  &\3pt]
\hline
 &  &\3pt]
\hline
 &  & \\phi=\MC{P}_{\leq0.38}(\X(L(s_1)\lor
  L(s_3))\land\X\X(L(s_1)\lor L(s_3))),\psi::=\phi\mid\X\phi\mid\neg\psi\mid\psi_1\land\psi_2\mu'(C)=\mu'(\{\PAR{s'}{t}\mid s'\in C'\})= \mu_r(C') \geq \mu_s(C') = \mu(\{\PAR{s'}{t}\mid s'\in C'\})=\mu(C).\eqno{\qEd}\label{eq:definition of transition branching}
\MEASURE_{\sigma,s}(C,C',n,\omega)=
\sum\limits_{r\in\mathit{supp}(\mu')}\mu'(r)\cdot\MEASURE_{\sigma,r}(C,C',n-1,\omega r).
\MEASURE_{\sigma',r}(C,C',i,r)\geq\MEASURE_{\sigma,s}(C,C',i,s),\MEASURE_{\sigma',s}(C,C',i,s)\geq\MEASURE_{\sigma,r}(C,C',i,r).\MEASURE_{\sigma,s}(\{\omega\mid\omega\models\phi_1\U^{\le i}\phi_2\}) = \MEASURE_{\sigma,s}(\MI{Sat}(\phi_1),\MI{Sat}(\phi_2),i,s).\psi=((L(s\interleave t)\lor
L(s_1\interleave t)\lor(L(s_3\interleave t)))\U^{\leq 2}
(L(s_1\interleave t_2)\lor L(s_3\interleave t_1)))\label{eq:set of closed cones}
\MEASURE(C_{\tilde{\Omega}}) = \mathop{\sum}\limits_{\omega\in\tilde{\Omega}\land\not\exists\omega'\in\tilde{\Omega}.\omega'\leq\omega}\MEASURE(C_{\omega})
\MEASURE_{\sigma'',t}(\tilde{\Omega})\geq\MEASURE_{\sigma',r}(\tilde{\Omega})\geq\MEASURE_{\sigma,s}(\tilde{\Omega}).\psi_{\Omega}=\phi_{C_0}\land\X(\phi_{C_1}\land\ldots\land\X(\phi_{C_{j-1}}\land\X\phi_{C_j})\ldots)\psi=((L(s)\lor
  L(s_1))\U L(s_5))\lor((L(s)\lor L(s_3))\U L(s_4))\label{eq:definition of transition branching unbounded}
\MEASURE_{\sigma,s}(C,C',\omega)=\lim_{n\rightarrow\infty}\MEASURE_{\sigma,s}(C,C',n,\omega).

C_{\Omega_{\mathit{st}}}=\begin{cases}C_{\Omega} & l(\Omega)=1\\ \mathop{\bigcup}\limits_{0\leq i<n.k_i\geq 0}C_{(\Omega[0])^{k_0}\ldots(\Omega[n-2])^{k_{n-2}}\Omega[n-1]} & l(\Omega)=n\geq 2\end{cases}
\psi::=\phi\mid\psi_1\lor\psi_2\mid\neg\psi\mid\psi_1\U\psi_2
\phi &::= a \mid\neg a\mid \phi_1\land\phi_2\mid \phi_1\lor\phi_2\mid\MC{P}_{\leq q}(\psi)\\
\psi &::=\phi\mid\psi_1\land\psi_2\mid\psi_1\lor\psi_2\mid\X\psi\mid\psi_1\U\psi_2
\psi::=\X\phi\mid\phi_1\U\phi_2 \mid\phi_1\U^{\le n}\phi_2.\psi ::= \phi \mid \X\phi\mid\psi_1\land\psi_2\mid\psi_1\lor\psi_2,
\omega\models\widetilde{\X}\phi &\text{ iff } (\ABS{\omega}<1\lor\omega[i]\models\phi)\\
\omega\models\phi_1\widetilde{\U}\phi_2&\text{ iff } (\omega\models\phi_1\U\phi_2\lor\forall i\leq\ABS{\omega}.\omega[i]\models\phi_1)
\psi_{\Omega}=\phi_{C_0}\land\X(\phi_{C_1}\land\ldots\land\X(\phi_{C_{j-1}}\land\X\phi_{C_j})\ldots)
\sum\limits_{\forall
  r\in\SUPP(\nu).r\boundTRAN{n-1}{C}\nu_r}\nu(r)\cdot\nu_r=\mu.\MEASURE_{\sigma,s}(C_{\leq j},C',i,s)>\MEASURE_{\sigma,s}(C,C',i,s) - \frac{\epsilon}{4}, \text{ and }\MEASURE_{\sigma,s}(C_{\leq j},C_{\leq k},i,s)>\MEASURE_{\sigma,s}(C_{\leq j},C',i,s) - \frac{\epsilon}{4}.\MEASURE_{\sigma,s}(C_{\leq j},C_{\leq k},i,s) > \MEASURE_{\sigma,s}(C,C',i,s) - \frac{\epsilon}{2}=\MEASURE_{\sigma',r}(C,C',i,r) + \frac{\epsilon}{2}>\MEASURE_{\sigma',r}(C,C',i,r)\geq \MEASURE_{\sigma,s}(C_{\leq j},C_{\leq k},i,s),\frac{b_j-a_j}{a_i-b_i}\cdot\rho_2 > \rho_1 > \frac{a_j-c_j}{c_i-a_i}\cdot\rho_2,
a_i\cdot\rho_1 + a_j\cdot\rho_2 < b_i\cdot\rho_1 + b_j\cdot\rho_2,a_i\cdot\rho_1 + a_j\cdot\rho_2 < c_i\cdot\rho_1 + c_j\cdot\rho_2.
\psi=((L(s\interleave t)\lor L(s_i\interleave
t)\lor(L(s_j\interleave t)))\U^{\leq 2} (L(s_i\interleave t_1)\lor
L(s_j\interleave t_2))),
\nu(s_i)\cdot\rho_1+\nu(s_j)\cdot\rho_2 &=(w_1\cdot b_i+w_2\cdot c_i)\cdot\rho_1 + (w_1\cdot b_j + w_2\cdot c_j)\cdot\rho_2\\
& = w_1\cdot(b_i\cdot\rho_1 + b_j\cdot\rho_2) + w_2\cdot(c_i\cdot\rho_1 + c_j\cdot\rho_2)\\
& > a_i\cdot\rho_1 + a_j\cdot\rho_2,

therefore we have 
but  where
. In other words
, as a result
, so  is not a congruence.

When all the states do not have distinct labels,
we can always construct formulas to distinguish them,
since the  is compact and these states are in different equivalence classes by assumption.
The subsequent proof is then similar. This completes our proof.
\end{proof}

Theorem~\ref{thm:coarsest} can be extended to identify the coarsest congruent weak bisimulation in , and the coarsest congruent strong and weak simulations in  and  respectively.
\begin{thm}\label{thm:coarsest 1}\hfill
\begin{enumerate}[\em(1)]
\item  is the coarsest congruence relation in ,
\item  is the coarsest congruent preorder in ,
\item  is the coarsest congruent preorder in .
\end{enumerate}
\end{thm}
\begin{proof}
The proof is similar to the proof of Theorem~\ref{thm:coarsest} and we only sketch the proof of Clause (2) here. According to Lemma 5.2 in~\cite{HermannsPSWZ11},  iff for each finitely generated  downward closed set ,  where  is a preorder. In order to prove that  is the coarsest congruent preorder in , we need to show that for any relation  such that , it holds that  is not congruent, i.e. there exist , , and  such that , but . First assume that  is a congruence, and we then prove by contradiction as in Theorem~\ref{thm:coarsest} and show that if  and , there exists  such that , thus  which contradicts the assumption that  is a congruence. Since , then there exists  such that there does not exist  with . With the same argument as in Theorem~\ref{thm:coarsest} and Lemma 5.2 in~\cite{HermannsPSWZ11}, there exist  and  such that  but  i.e. , thus  is not congruent.
\end{proof}



\section{Related work}\label{sec:related}
For Markov chains, i.e., deterministic s, the
logic  characterizes bisimulations, and  without  operator
characterizes weak bisimulations~\cite{HanssonJ90,BaierKHW05}.  As pointed out
in~\cite{SegalaL95}, probabilistic bisimulation  is sound, but not complete for  over s.
In the literature, various extensions of the Hennessy-Milner logic~\cite{HennessyM85} are considered for characterizing
bisimulations. Larsen and Skou~\cite{larsen1991bisimulation} considered such an
extension of Hennessy-Milner logic, which characterizes bisimulation for
\emph{reactive probabilistic processes}~\cite{larsen1991bisimulation}. Similar results are further studied for labelled Markov
processes~\cite{prakash-book,DesharnaisGJP10} (with continuous state space).  For s, Jonsson
\emph{et al.}~\cite{Jonsson} considered a two-sorted logic in the
Hennessy-Milner style to characterize strong bisimulations. In~\cite{HermannsPSWZ11}, the results are 
also extended to characterize simulations.



Weak bisimulation was first defined in the context of s by
Segala and Lynch~\cite{SegalaL95}, and then formulated for alternating models by
Philippou \emph{et al.}~\cite{PhilippouLS00}.
The seemingly very related work is by Desharnais et
al.~\cite{DesharnaisGJP10}, where it is shown that  is sound and
complete with respect to weak bisimulation for \emph{alternating
automata}. The key difference is that the model they have considered is not
the same as s considered in this paper. Briefly,
in alternating automata, states are either non-deterministic like in
transition systems, or stochastic like in discrete-time Markov chains.
As discussed in~\cite{SegalaT05}, a  can be
transformed to an alternating automaton by replacing each transition
 by two consecutive transitions 
and  where  is the new inserted
state. Surprisingly, for alternating automata, Desharnais et al. have
shown that weak bisimulation -- defined in the standard manner --
characterizes  formulas. The following example illustrates why it
works in that setting, but fails for s.
\begin{exa}\label{ex:alternative PA}
  Refer to Fig.~\ref{fig:counterexample}, we need to add three
  additional states  and  in order to
  transform  and  to states in an alternating automaton. 
  The resulting
  automaton is shown in Fig.~\ref{fig:altercounterexample}.  Suppose
  that  and  are three absorbing states with different
  atomic propositions, so they are not (weak) bisimilar, as a result
   and  are not (weak) bisimilar
  either since they can evolve into  and  with
  different probabilities. Therefore  and  are not (weak)
  bisimilar. Let , it is
  not hard to see that  but
  , so  while . When
  working in the setting of s, ,
  , and  will not be considered as states, so we
  cannot use the above arguments for alternating automata any more.
\end{exa}

\begin{figure}[!t]
  \centering
    \begin{picture}(140, 40)(0,0)
    \gasset{Nw=6,Nh=6,Nmr=3}
    \node(SAA)(0,0){}
    \node(SBA)(10,0){}
    \node(SCA)(20,0){}
    \node(SAB)(30,0){}
    \node(SBB)(40,0){}
    \node(SCB)(50,0){}
    \node(RAA)(60,0){}
    \node(RBA)(70,0){}
    \node(RCA)(80,0){}
    \node(RAB)(90,0){}
    \node(RBB)(100,0){}
    \node(RCB)(110,0){}
    \node(RAC)(120,0){}
    \node(RBC)(130,0){}
    \node(RCC)(140,0){}
    \node(S)(25,40){}
    \node(SMA)(13,20){}
    \node(SMC)(37,20){}
    \node(R)(100,40){}
    \node(RMA)(80,20){}
    \node(RMB)(100,20){}
    \node(RMC)(120,20){}
    \drawedge(S,SMA){}
    \drawedge(S,SMC){}
    \drawedge(R,RMA){}
    \drawedge(R,RMB){}
    \drawedge(R,RMC){}
    \gasset{ELdistC=y,ELdist=0}
    \drawedge[ELpos=60,ELside=r](SMA,SAA){{\tiny \colorbox{white}{0.3}}}
    \drawedge[ELpos=60,ELside=r](SMA,SBA){{\tiny \colorbox{white}{0.3}}}
    \drawedge[ELpos=60,ELside=r](SMA,SCA){{\tiny \colorbox{white}{0.4}}}
    \drawedge[ELpos=60,ELside=r](SMC,SAB){{\tiny \colorbox{white}{0.5}}}
    \drawedge[ELpos=60,ELside=l](SMC,SBB){{\tiny \colorbox{white}{0.4}}}
    \drawedge[ELpos=60,ELside=l](SMC,SCB){{\tiny \colorbox{white}{0.1}}}
    \drawedge[ELpos=60,ELside=r](RMA,RAA){{\tiny \colorbox{white}{0.3}}}
    \drawedge[ELpos=60,ELside=r](RMA,RBA){{\tiny \colorbox{white}{0.3}}}
    \drawedge[ELpos=60,ELside=r](RMA,RCA){{\tiny \colorbox{white}{0.4}}}
    \drawedge[ELpos=60,ELside=r](RMB,RAB){{\tiny \colorbox{white}{0.4}}}
    \drawedge[ELpos=60,ELside=r](RMB,RBB){{\tiny \colorbox{white}{0.3}}}
    \drawedge[ELpos=60,ELside=l](RMB,RCB){{\tiny \colorbox{white}{0.3}}}
    \drawedge[ELpos=60,ELside=r](RMC,RAC){{\tiny \colorbox{white}{0.5}}}
    \drawedge[ELpos=60,ELside=l](RMC,RBC){{\tiny \colorbox{white}{0.4}}}
    \drawedge[ELpos=60,ELside=l](RMC,RCC){{\tiny \colorbox{white}{0.1}}}
    \end{picture}
  \caption{\label{fig:altercounterexample} Alternating automata.}
\end{figure}



In the definition of  and , we choose first the
downward closed set  before the successor distributions to be
matched, which is the key for achieving our new notions of
bisimulations and simulations. This approach was first adopted
in~\cite{AlfaroMRS07} to define the \emph{a priori metric} for Markov
decision processes, where it was shown that the a priori metric can be
characterized by the quantitative -calculus. In~\cite{qapl} this
approach was also used to define a priori -bisimulation
and simulation relations. 
\iffalse
It turns out that when , the  a priori
-bisimulation coincides with . The a priori
-bisimulation was shown to be sound and complete w.r.t. an
extension of Hennessy-Milner logic, similarly for the a priori
-simulation.  The a priori -bisimulation was also
used to define pseudo-metric between s in~\cite{qapl}. The
definition of a priori -simulation in~\cite{qapl}, denoted as
, is however not equivalent to . In the definition
of , the upward closed sets are considered while in the
definition of  we consider downward closed sets. If we adopt
the definition of  here, Theorem~\ref{thm:characterization
  strong branching simulation} will not be valid any more. Refer to the
following example.
\begin{exa}
Consider the two states  and  in Fig.~\ref{fig:upward} where all the states have different labels except that , and the transitions of  and  are omitted. Moreover we assume that , but . Let , in order to show that  is a priori -simulation, we need to check that for each  upward closed set  and , there exists  such that . The only non-trivial cases are when  or , thus . But we will show that . By contradiction, assume that . Let  where  is a formula such that  but . Since , such formula always exists by assumption. It is easy to see that , but  since the maximal probability from  to  in one step is equal to 0.5, thus we get contradiction, and 
\begin{figure}[!t]
  \begin{center}
    \begin{picture}(80,20)(-10,30)
    \gasset{Nadjust=n,Nw=6, Nh=6,Nmr=3}
    \node(S)(0,50){}
    \node(Sca)(-25,33.5){}
    \node(Sab)(5,33.5){}
    \node(Scb)(-5,33.3){}
    \drawcurve[AHnb=0,ATnb=0,dash={0.2 0.5}0](-2.5,41.75)(0,41)(2.5,41.75)
    \drawedge[ELside=r](S,Sca){1}
    \drawedge(S,Sab){0.5}
    \drawedge[ELside=r](S,Scb){0.5}
    \node(R)(70,50){}
    \node(Ra)(70,33.5){}
    \drawedge(R,Ra){1}
    \end{picture}
    \end{center}
  \caption{\label{fig:upward}An example illustrating .}
\end{figure}
\end{exa}
\fi

\section{Conclusion and future work}\label{sec:conclusion}
In this paper we have introduced novel notions of bisimulation for s. They are coarser than the existing bisimulations, and most importantly, we show that they agree with the logical equivalences induced by  and its sub logics.
Even though we have not considered actions, it is
worth noting that actions can be easily added, and all the (weak) bisimulations
can be defined directly.
On the other hand, the (weak) bisimulations are then strictly finer than the logical equivalences,
because of the presence of these actions, similarly for simulations.

As future work, we plan to study decision algorithms for our new
(strong and weak) bisimulation and simulation relations.

\section*{Acknowledgement}
The authors are supported by IDEA4CPS and the VKR Center of Excellence  MT-LAB. The work has received support from the EU FP7-ICT projects TREsPASS (318003) and MEALS (295261), and the DFG Sonderforschungsbereich AVACS. Part of the work was done while the first author was with IT University of Copenhagen, Denmark. We thank Johann Schuster for detailed comments on an early version of this draft.

\bibliographystyle{abbrv}
\bibliography{bib}


\end{document}
