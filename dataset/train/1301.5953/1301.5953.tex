\documentclass{llncs}

\pagestyle{plain}

\usepackage{amsmath,amssymb}
\usepackage[linesnumbered]{algorithm2e}
\usepackage{graphicx}
\usepackage{enumerate}

\DeclareMathOperator{\scat}{{\rm sc}}
\DeclareMathOperator{\comp}{{\rm comp}}
\DeclareMathOperator{\merg}{{\rm merge}}

\newcommand{\cP}{{\mathcal P}}
\newcommand{\cQ}{{\mathcal Q}}
\renewcommand{\P}{{\mathsf P}}
\newcommand{\NP}{{\mathsf{NP}}}
\newcommand{\coNP}{{\mathsf{coNP}}}
\newcommand\pcase[1] {\medskip\par\noindent \textbf{#1}}


\begin{document}
\title{Linear-Time Algorithms for Scattering Number and Hamilton-Connectivity of Interval Graphs
\thanks{Paper supported by Royal Society Joint Project Grant JP090172.}}


\author
{Hajo Broersma\inst{1} 
\and 
Ji\v{r}\'{\i} Fiala\inst{2}\thanks{Author supported by the GraDR-EuroGIGA project GIG/11/E023 and by the project Kontakt LH12095.} 
\and
Petr A. Golovach\inst{3}  
\and 
Tom\'a\v{s} Kaiser\inst{4}\thanks{Author supported by project P202/11/0196 of the Czech Science Foundation.}  
\and 
Dani\"el Paulusma\inst{5}\thanks{Author supported by EPSRC (EP/G043434/1).}  
\and Andrzej Proskurowski\inst{6}
}

\institute{
Faculty of EEMCS, University of Twente, Enschede, {\tt h.j.broersma@utwente.nl} 
\and
Department of Applied Mathematics, Charles University, Prague, {\tt fiala@kam.mff.cuni.cz}
\and
Institute of Computer Science, University of Bergen, {\tt petr.golovach@ii.uib.no}
\and
Department of Mathematics, University of West Bohemia, Plze\v{n}, {\tt kaisert@kma.zcu.cz}
\and
School of Engineering and Computing Sciences, Durham University, {\tt daniel.paulusma@durham.ac.uk}
\and
Department of Computer Science, University of Oregon, Eugene, {\tt andrzej@cs.uoregon.edu}
}

\maketitle


\begin{abstract}\noindent
Hung and Chang showed that for all  an interval graph has a path cover of size at most  if and only if its scattering number is at 
most . They also showed that an interval graph has a Hamilton cycle if and only if its scattering number is at most .
We complete this characterization by proving that for all  an interval graph is -Hamilton-connected if and only if its scattering number is at most . 
We also give an  time algorithm for computing the scattering number of an interval graph with  vertices an  edges, which improves the  time bound of Kratsch, Kloks and M\"uller.  
As a consequence of our two results the maximum  for  which an interval graph is -Hamilton-connected can be computed in  time. 
\end{abstract}

\section{Introduction}\label{s:intro}

The {\sc Hamilton Cycle} problem is that of 
testing whether a given graph has a Hamilton cycle, i.e., a cycle passing through all the vertices.
This problem
 is one of the most notorious -complete problems within Theoretical Computer Science.
It remains -complete on many graph classes such as the classes of planar cubic 3-connected graphs~\cite{GJT76},
chordal bipartite graphs~\cite{Mu96},  and strongly chordal split graphs~\cite{Mu96}. In contrast, for interval graphs, Keil~\cite{Ke85} showed in 1985 that {\sc Hamilton Cycle} can be solved in  time, thereby
strengthening an earlier result of Bertossi~\cite{Be83} for proper interval graphs.
Bertossi and Bonucelli~\cite{BB86} proved that {\sc Hamilton Cycle} is -complete for undirected path graphs, double interval graphs and rectangle graphs, all three of which are classes of intersection graphs that contain the class of interval graphs.
We examine whether the linear-time result of Keil~\cite{Ke85}  can be strengthened on interval graphs to hold for other connectivity properties, which are -complete to verify in general.
This line of research is well embedded in the literature. 
Before surveying existing work and presenting our new results, we first give the necessary terminology.

\subsection{Terminology}\label{s-term}

We only consider undirected finite graphs with no self-loops and no multiple edges. We refer to the textbook of Bondy and Murty~\cite{BM08} for any undefined graph terminology.
Throughout the paper we let  and  denote the number of vertices and edges, respectively, of the input graph.

Let  be a graph. If  has a {\em Hamilton cycle\/}, i.e., a cycle containing all the vertices of , then  is  {\em hamiltonian}.
Recall that the corresponding -complete decision problem is called {\sc Hamilton Cycle}.
If  contains a {\em Hamilton path}, i.e., a path containing all the vertices of , then  is  {\em traceable}. In this case, the corresponding decision problem is called the {\sc Hamilton Path} problem,
which is also well known to be -complete (cf.~\cite{GJ79}).
The problems {\sc 1-Hamilton Path} and {\sc 2-Hamilton Path} are those of testing whether a given graph has a Hamilton path that starts in some given vertex  or that is between two given vertices  and , respectively. Both problems are -complete by a straightforward reduction from {\sc Hamilton Path}. The {\sc Longest Path} problem is to compute the maximum length
of a path in a given graph. This problem is -hard by a reduction from {\sc Hamilton Path} as well.

Let  be a graph.
If for each two distinct vertices  there exists a Hamilton path with end-vertices  and , then
 is {\em Hamilton-connected\/}. If  is Hamilton-connected for every set  with  for some integer , then
 is {\em -Hamilton-connected\/}. Note that a graph is Hamilton-connected if and only if it 0-Hamilton-connected. The  {\sc Hamilton Connectivity} problem is that of computing the
maximum value of  for which a given graph is -Hamilton-connected.  Dean~\cite{De93} showed that already deciding whether  is -complete.
Ku\v{z}el,  Ryj{\'a}\v{c}ek and Vr{\'a}na~\cite{KRV12} proved this for . 
A straightforward generalization of the latter result yields the same for any integer .
As an aside, the {\sc Hamilton Connectivity} problem has recently been studied by Ku\v{z}el, Ryj\'a\v{c}ek and Vr\'ana~\cite{KRV12}, who showed that -completeness of the case 
for line graphs would disprove the conjecture of Thomassen that every 4-connected line graph is hamiltonian, unless . 

A {\it path cover} of a graph  is a set of mutually vertex-disjoint paths
  with  . The size of a smallest path cover is denoted by .
The {\sc Path Cover} problem is to compute this number, whereas the 
{\sc 1-Path Cover} problem is to compute the size of a smallest path cover that contains a path in which some given vertex  is an end-vertex.
Because a Hamilton path of a graph is a path cover of size 1, 
{\sc Path Cover} and {\sc 1-Path Cover} are -hard via a reduction from {\sc Hamilton Path} and {\sc 1-Hamilton Path}, respectively.

We denote the number of connected components of a graph  by . 
A subset  is a
{\em vertex cut\/} of  if , and  is called {\it -connected} if the size of a smallest vertex cut of  is at least .
We say that  is \mbox{{\em{}-tough}} if  for every vertex cut  of  . The {\em toughness\/}  of a graph  was defined by Chv\'atal~\cite{Ch73} as

where we set  if  is a complete graph.
Note that  if  is hamiltonian; the reverse statement does not hold in general (see~\cite{BM08}).
The {\sc Toughness} problem is to compute  for a graph .
Bauer, Hakimi and Schmeichel~\cite{BHS90} showed that already deciding whether  is -complete.

The \emph{scattering number} of a graph  was defined by Jung~\cite{Ju78} as
 
where we set  if  is a complete graph.
We call a set  on which  is attained a \emph{scattering set}.
Note that  if  is hamiltonian.
Shih, Chern and Hsu~\cite{SCH92} show  that
 for all graphs . Hence,  if  is traceable.
The {\sc Scattering Number} problem is to compute  for a graph .
The observation that  if and only if  combined with the aforementioned result of Bauer, Hakimi and Schmeichel~\cite{BHS90}  implies that already deciding whether 
 is -complete.

A graph  is an {\it interval graph} if it is the intersection graph of a set of closed intervals on the real line, i.e., the vertices of  correspond to the intervals and two vertices are adjacent in  if and only if their intervals have at least one point in common. An interval graph is {\it proper} if it has a closed interval representation in which no interval is properly contained in some other interval. 

\subsection{Known Results}\label{s-known}

We first discuss the results on testing hamiltonicity properties for proper interval graphs. Besides giving a linear-time algorithm for solving {\sc Hamilton Cycle} on proper interval graphs, Bertossi~\cite{Be83} also showed that a proper interval graph is traceable if and only if it is connected~\cite{Be83}. His work was extended by Chen, Chang and Chang~\cite{CCC97} who showed that
a proper interval graph is hamiltonian if and only if it is 2-connected, and that a proper interval graph is Hamilton-connected if and only if it is 3-connected. 
In addition, Chen and Chang~\cite{CC96} showed that a proper interval graph has scattering number at most  if and only if it is -connected.

Below we survey the results on testing hamiltonicity properties for interval graphs that appeared after the aforementioned result of Keil~\cite{Ke85} on solving {\sc Hamilton Cycle} for interval graphs  in   time.

\medskip
\noindent
{\it Testing for Hamilton cycles and Hamilton paths.} 
The  time algorithm of Keil~\cite{Ke85}
makes use of an interval representation. One can find such a representation by executing the  time interval recognition algorithm of 
Booth and Leuker~\cite{BL76}. If an interval representation is already given,
Manacher, Mankus and Smith~\cite{MMS90} showed that {\sc Hamilton Cycle} and {\sc Hamilton Path} can be solved in  time. In the same
paper, they ask whether the time bound for these two problems can be improved to  time if a so-called sorted interval representation is given.
Chang, Peng and Liaw~\cite{CPL99} answered this question in the affirmative. They showed that this even holds for {\sc Path Cover}.

\medskip
\noindent
{\it When no Hamilton path exists.}
In this case, {\sc Longest Path} and {\sc Path Cover} are natural problems to consider.
Ioannidou, Mertzios and Nikolopoulos~\cite{IMN11} gave an  algorithm for solving {\sc Longest Path} on interval graphs.
Arikati and Pandu Rangan~\cite{AP90} and also Damaschke~\cite{Da93}
showed that {\sc Path Cover} can be solved in  time on interval graphs. 
Damaschke~\cite{Da93}  
posed the complexity status of {\sc 1-Hamilton Path} and {\sc 2-Hamilton Path} on interval graphs as open questions. 
The latter question is still open, but Asdre and Nikolopolous~\cite{AN10} answered the former 
question  by presenting an  time algorithm that solves {\sc 1-Path Cover}, and hence {\sc 1-Hamilton Path}.
Li and Wu~\cite{LW} announced an  time algorithm for {\sc 1-Path Cover} on interval graphs.
Although Hung and Chang~\cite{HC11} do not mention the scattering number explicitly, they show that for all 
an interval graph has a path cover of size at most  if and only if its scattering number is at 
most . Moreover, they give an  time algorithm that finds a scattering set of an interval graph  with .
They also prove that an interval graph  is hamiltonian if and only if . Recall that the latter condition is equivalent to . 
As such, their second result is claimed~\cite{CJKL98,KKS07} to be implicit already in Keil's algorithm~\cite{Ke85}.


  \subsection{Our Results}\label{s-ours}
 
 {\it When a Hamilton path does exist.} 
 In this case, {\sc Hamilton Connectivity} is a natural problem to consider.
 Isaak~\cite{Is98} used a closely related variant of toughness called -path toughness to characterize interval graphs that contain the th power of a Hamiltonian path. However, the aforementioned results of Hung and Chang~\cite{HC11} suggest that trying to characterize -Hamilton-connectivity in terms
 of the scattering number of an interval graph may be more appropriate than doing this in terms of its toughness. We confirm this by showing that for all  an interval graph is -Hamilton-connected if and only if its scattering number is at most . Together with the results of Hung and Chang~\cite{HC11} this leads to the following theorem.
  
\begin{theorem}\label{t-char}
Let  be an interval graph. Then  if and only if\
  C_{j,k} = (\bigcup_{i=j}^kC_i).

c(G-S)-|S| < c(G-(S-\{u_1\})) - |S-\{u_1\}|,

  |S| \leq (p-1) + (k-1) = k+p-2

\begin{array}{lcl}
1 &\geq &\scat(G')\3pt]
   &=&c(G-T-U)-|T|-|U|+|T|\3pt]
   &= &\scat(G)+|T|\
implying that , as required.

\begin{figure}
\begin{center}
\includegraphics{3hamconn.pdf}
\end{center}
\caption{The essential cases in the proof of Theorem~\ref{t-ham} for .}\label{fig:3hamconn}
\end{figure}

Now suppose that . 
First let .
By Theorem~\ref{thm:syst}, there exists a spanning 3-stave  between  and . 
Let  be an arbitrary pair of vertices  of . We distinguish four cases in order to find a Hamilton path between  and ; see Fig.~\ref{fig:3hamconn} for an illustration.

\pcase{Case 1:}  and . In this case,  is the desired Hamilton path.

\pcase{Case 2:}  and . Assume without loss of generality that .
We split  before  into the subpaths  and , i.e.,  becomes the first vertex of  
and it does not belong to .
Then  is the desired path. The case with  and  is symmetric.

\pcase{Case 3:}  and  belong to different paths, say  and . We split 
 after  into  and , and we also split  before , as above.
Then  is the desired path.

\pcase{Case 4:}  and  belong to the same path, say . Without loss of generality, assume that both  and  appear in this order on . 
We split  after  and before  
into three subpaths . If  and  are consecutive on , i.e., when  is empty, then 
 is the desired path.
Otherwise, let  be any vertex on  that is a neighbor of the first vertex of . Such  exists since
the path  dominates . 
We split  after  into  and . By the choice of ,  and  can be combined through  into a valid path  containing exactly the same vertices as  and  and starting at .   
Then we choose .

\medskip
\noindent
Now let .
Let  be a set of vertices with . We need to show that  is Hamilton-connected. Let  be a scattering set of  and let .
Because  is a scattering set of  , we find that  is a vertex cut of . We use this to derive that
3pt]
&= &c(G-S^*)-|S^*|+|S^*|-|T|\3pt]
&\leq &-1.
\end{array}

Then, by returning to the case  with  instead of , we find that  is Hamilton-connected, as required. This completes the proof of Theorem~\ref{t-ham}.\qed
\end{proof}


\section{Future Work}\label{s-con}

We conclude our paper by posing a number of open problems. We start with recalling two open problems posed in the literature.

First of all, Damaschke's question~\cite{Da93} on the complexity status of 2-{\sc Hamilton Path} is still open. Our results imply 
that we may restrict ourselves to interval graphs with scattering number equal to zero or one. This can be seen as follows. Let  be an interval graph that together with two of its vertices  and  forms an instance
of 2-{\sc Hamilton Path}. We apply Corollary~\ref{c-scat} to compute  in  time. If , then  is Hamilton-connected by Theorem~\ref{t-char}. Then, by definition, there exists a Hamilton path between
 and . If , then  is not hamiltonian, also due to Theorem~\ref{t-char}. Hence, there exists no Hamilton path between  and .

Second, Asdre and Nikolopoulos~\cite{AN10} asked about the complexity status of the -{\sc Path Cover} problem on interval graphs.
This problem generalizes -{\sc Path Cover} and is to determine the size of a smallest path cover of a graph  subject to the additional condition that every vertex of a given set  of size  is 
an end-vertex of a path in the path cover. The same authors show that both {\sc -Path Cover} and -{\sc Hamilton Path} can be solved in  time on proper interval graphs~\cite{AN10a}. 

The {\sc Spanning Stave} problem is that of computing the minimum value of  for which a given graph has a spanning -stave.
Because a Hamilton path of a graph is a  spanning -stave and {\sc Hamilton Path} is -complete, this problem is -hard. What is the computational complexity of
{\sc Spanning Stave} on interval graphs? The following example shows that we cannot generalize Lemma~\ref{lem:eat} and apply Algorithm~\ref{alg:maxkstave} as an attempt to solve this problem.
Take the graph with four vertices , , ,  and edges , , , , . The resulting graph is interval.  
However, we only have a spanning -stave between  and  (as their degrees are ) but there is a spanning -stave between  and , namely
. 

Chen et al.~\cite{CCLLW} define the {\em spanning connectivity\/} of a Hamilton-connected graph  as the largest integer  such that  has a spanning -stave between any two vertices of  for all integers . So, for instance, the complete graph on  vertices has spanning connectivity , and a graph has spanning connectivity at least  if and only if it is Hamilton-connected.
By the latter statement, the corresponding optimization problem {\sc Spanning Connectivity} is -hard.
What is the computational complexity of {\sc Spanning Connectivity} on interval graphs or even proper interval graphs?

Kratsch, Kloks and M\"uller~\cite{KKM94} gave an  time algorithm for solving {\sc Toughness} on interval graphs. Is it possible to improve this bound to linear on interval graphs just as we did for 
{\sc Scattering Number}?

Finally, can we extend our  time algorithms for {\sc Hamilton Connectivity} and {\sc Scattering Number} to superclasses of interval graphs such as circular-arc graphs and cocomparability graphs? 
The complexity status of {\sc Hamilton Connectivity} is still open for both graph classes, although {\sc Hamilton Cycle} can be solved in  time on circular-arc graphs~\cite{SCH92} and in  time on cocomparability graphs~\cite{DS94}. It is known~\cite{KKM94} that {\sc Scattering Number} can be solved in  time on circular-arc graphs and in polynomial time on cocomparability graphs of bounded dimension.


\begin{thebibliography}{10}

\bibitem{AP90}
S.R. Arikati and C. Pandu Rangan, Linear algorithm for optimal path cover problem on interval graphs, Information Processing Letters 35 (1990) 149--153.

\bibitem{AN10a}
K. Asdre and S.D. Nikolopoulos, A polynomial solution to the k-fixed-endpoint path cover problem on proper interval graphs, Theor. Comput. Sci. 411 (2010) 967--975.

\bibitem{AN10}
K. Asdre and S.D. Nikolopoulos, The 1-fixed-endpoint path cover problem is polynomial on interval graphs, Algorithmica 58 (2010) 679--710.

\bibitem{BHS90}
D. Bauer, S.L. Hakimi, and E. Schmeichel,
Recognizing tough graphs is -hard,
Discrete Applied Mathematics 28 (1990) 191--195.

\bibitem{Be83}
A.A. Bertossi, Finding hamiltonian circuits in proper interval graphs, Information Processing Letters 17 (1983) 97--101.

\bibitem{BB86}
A.A. Bertossi and M.A. Bonucelli,
Hamilton circuits in interval graph generalizations,
Information Processing Letters 23 (1986) 195-200.

\bibitem{BM08}
J.A. Bondy and U.S.R. Murty, Graph Theory, vol. 244 of Graduate Texts in Mathematics, Springer Verlag, 2008.

\bibitem{BL76}
K.S. Booth and G.S. Lueker, Testing for the consecutive ones property, interval graphs, and graph planarity using pq-tree algorithms, J. Comput. Syst. Sci. 13 (1976) 335--379.

\bibitem{CPL99}
M.-S. Chang, S.-L. Peng, and J.-L. Liaw,
Deferred-query: An efficient approach for some problems on interval graphs, Networks 34 (1999) 1--10.

\bibitem{CC96}
C. Chen and C.-C. Chang, Connected proper interval graphs and the guard problem in spiral polygons,  Combinatorics and Computer Science, Lecture Notes in Computer Science Volume 1120 (1996) 39--47.

\bibitem{CCC97}
C. Chen, C.-C. Chang, and G.J. Chang, Proper interval graphs and the guard problem, Discrete Mathematics 170 (1997) 223--230.

\bibitem{CCLLW}
Y. Chen, Z.-H. Chen, H.-J. Lai, P. Li, and E. Wei, On spanning disjoint paths in line graphs, Graph and Combinatorics, to appear.

\bibitem{CJKL98}
G. Chen, M.S. Jacobson, A.E. K\'ezdy and J. Lehel,  Tough enough chordal graphs are hamiltonian, Networks 31 (1998) 29--38.

\bibitem{Ch73}
V. Chv\'atal, Tough graphs and hamiltonian circuits, Discrete Mathematics 5 (1973) 215--228.

\bibitem{Da93}
P. Damaschke, Paths in interval graphs and circular arc graphs, Discrete Mathematics 112 (1993) 49--64.

\bibitem{De93}
A.M. Dean, The computational complexity of deciding hamiltonian-connectedness, Congr. Num. 93 (1993) 209--214.

\bibitem{DS94}
J.S. Deogun and G. Steiner, Polynomial algorithms for hamiltonian cycle in cocomparability graphs, SIAM Journal on Computing 23 (1994) 520--552.

\bibitem{GJ79}
M.R. Garey and D.S. Johnson, Computers and Intractability: A Guide to the Theory of NP-Completeness, W. H. Freeman \& Co Ltd, 1979.

\bibitem{GJT76}
M.R. Garey, D.S. Johnson, and R.E. Tarjan, The planar hamiltonian circuit problem is NP-complete, SIAM Journal on Computing 5 (1976) 704--714.

\bibitem{Hs94}
D. Hsu, On container width and length in graphs, groups, and networks, IEICE Trans. Fund, E77-A (1994) 668--680.

\bibitem{HC11}
R.-W. Hung and M.-S. Chang, Linear-time certifying algorithms for the path cover and hamiltonian cycle problems on interval graphs, Appl. Math. Lett. 24 (2011) 648--652.

\bibitem{IMN11}
K. Ioannidou, G.B. Mertzios, and S.D. Nikolopoulos,
The longest path problem has a polynomial solution on interval graphs,
Algorithmica 61 (2011) 320--341.

\bibitem{Is98}
G. Isaak, Powers of hamiltonian paths in interval graphs,
Journal of Graph Theory 28 (1998) 31--38.

\bibitem{Ju78}
H.A. Jung, On a class of posets and the corresponding comparability graphs, Journal of Combinatorial Theory, Series B 24 (1978) 125--133.
 
\bibitem{KKS07}
T. Kaiser, D. Kr\'al', and L. Stacho, Tough spiders, Journal of Graph Theory 56 (2007) 23--40.

\bibitem{Ke85}
J.M. Keil,  Finding hamiltonian circuits in interval graphs, Information Processing Letters 20 (1985) 201--206.

\bibitem{KKM94}
D. Kratsch, T. Kloks, and H. M\"uller, Computing the toughness and the scattering number for interval and other graphs, Tech. Rep. Rapports de recherche no. 2237, INRIA Rennes, 1994.

\bibitem{KRV12}
R. Ku\v{z}el, Z. Ryj{\'a}\v{c}ek, and P. Vr{\'a}na,
Thomassen's conjecture implies polynomiality of -hamilton-connectedness in line graphs, Journal of Graph Theory 69 (2012) 241--250.

\bibitem{LW}
P. Li and Y. Wu, A linear time algorithm for solving the 1-fixed-endpoint path cover problem on interval graphs, draft, cited in
http://math.sjtu.edu.cn/faculty/ykwu/paths-in-interval-graphs.pdf.

\bibitem{LHH07}
C.-K. Lin, H.-M. Huanga, and L.-H. Hsu,
On the spanning connectivity of graphs,
Discrete Mathematics 307 (2007) 285--289.

\bibitem{MMS90}
G.K. Manacher, T.A. Mankus, and C.J. Smith, An optimum  algorithm for finding a canonical hamiltonian path and a canonical hamiltonian circuit in a set of intervals, Information Processing Letters 35 (1990) 205--211.

\bibitem{Mu96}
H. M\"uller, Hamiltonian circuits in chordal bipartite graphs, Discrete Mathematics 156 (1996) 291--298.

\bibitem{SCH92}
W.K. Shih, T.C. Chern, and W.L. Hsu, An  time algorithm for the hamiltonian cycle problem on circular-arc graphs, SIAM Journal on Computing 21 (1992) 1026--1046.

\end{thebibliography}
\end{document}