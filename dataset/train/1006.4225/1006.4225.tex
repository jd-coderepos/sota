\documentclass[twocolumn,10pt]{IEEEtran}
\IEEEoverridecommandlockouts
\usepackage{amsmath}
\usepackage{amsfonts}
\usepackage{amssymb}
\usepackage{amsthm}
\usepackage{graphicx}
\usepackage{psfrag}
\usepackage{cite}
\usepackage{algorithm}
\usepackage{algorithmic}
\usepackage{color}
\usepackage{float}
\usepackage{epsfig,array,multicol,verbatim}
\usepackage{subfigure}
\usepackage{cases}
\usepackage{setspace}
\usepackage{url}

\usepackage[top=0.8in, bottom=0.8in, left=0.8in, right=0.8in]{geometry}

\theoremstyle{plain} \newtheorem{theorem}{Theorem}
\theoremstyle{plain} \newtheorem{proposition}{Proposition}
\theoremstyle{plain} \newtheorem{corollary}{Corollary}
\theoremstyle{remark} \newtheorem{remark}{Remark}
\theoremstyle{remark} \newtheorem{lemma}{Lemma}
\theoremstyle{plain} \newtheorem{definition}{Definition}
\theoremstyle{plain} \newtheorem{assumption}{Assumption}
\theoremstyle{plain} \newtheorem{fact}{Fact}

\renewcommand{\algorithmicrequire}{\textbf{Input:}}
\renewcommand{\algorithmicensure}{\textbf{Output:}}

\begin{document}
\title{Optimal Spectrum Sharing in MIMO Cognitive Radio Networks via Semidefinite Programming}
\author{Ying~Jun~(Angela)~Zhang,~\IEEEmembership{Member,~IEEE} and Anthony~Man-Cho~So
\thanks{This work was supported in part by the Competitive Earmarked
Research Grant (Project Number 418707) established under the
University Grant Committee of Hong Kong, the Direct Grant for Research (Project Numbers 2050401 and 2050439) established by The Chinese University of Hong Kong, and Project \#MMT-p2-09 of the Shun Hing Institute of Advanced Engineering, The Chinese University of Hong Kong.}
\thanks{Y. J. Zhang is with the Department of Information Engineering and the Shun Hing Institute of Advanced Engineering, The Chinese University of Hong Kong, Hong Kong (email: yjzhang@ie.cuhk.edu.hk).}
\thanks{A. M.-C. So is with the Department of Systems Engineering and Engineering Management and the Shun Hing Institute of Advanced Engineering, The Chinese University of Hong Kong, Hong Kong (email: manchoso@se.cuhk.edu.hk)}
}

\maketitle

\begin{abstract}
In cognitive radio (CR) networks with multiple-input multiple-output (MIMO) links, secondary users (SUs) can exploit ``spectrum holes" in the space domain to access the spectrum allocated to a primary system.  However, they need to suppress the interference caused to primary users (PUs), as the secondary system should be transparent to the primary system. In this paper, we study the optimal secondary-link beamforming pattern that balances between the SU's throughput and the interference it causes to PUs. In particular, we aim to maximize the throughput of the SU, while keeping the interference temperature at the primary receivers below a certain threshold.

Unlike traditional MIMO systems, SUs may not have the luxury of knowing the channel state information (CSI) on the links to PUs. This presents a key challenge for a secondary transmitter to steer interference away from primary receivers. In this paper, we consider three scenarios, namely when the secondary transmitter has complete, partial, or no knowledge about the channels to the primary receivers. In particular, when complete CSI is not available, the interference-temperature constraints are to be satisfied with high probability, thus resulting in chance constraints that are typically hard to deal with. Our contribution is fourfold. First, by analyzing the distributional characteristics of MIMO channels, we propose a unified homogeneous quadratically constrained quadratic program (QCQP) formulation that can be applied to all three scenarios, in which different levels of CSI knowledge give rise to either deterministic or probabilistic interference-temperature constraints. The homogeneous QCQP formulation, though non-convex, is amenable to semidefinite programming (SDP) relaxation methods. Secondly, we show that the SDP relaxation admits no gap when the number of primary links is no larger than two. A polynomial-time algorithm is presented to compute the optimal solution to the QCQP problem efficiently. Thirdly, we propose a randomized polynomial-time algorithm for constructing a near-optimal solution to the QCQP problem when there are more than two primary links. Finally, we show that when the secondary transmitter has no CSI on the links to primary receivers, the optimal solution to the QCQP problem can be found by a simple matrix eigenvalue-eigenvector computation, which can be done much more efficiently than solving the QCQP directly.

\end{abstract}

\begin{IEEEkeywords}
Cognitive radio networks, MIMO, Semidefinite programming
\end{IEEEkeywords}



\section{Introduction}
Cognitive radio (CR), which allows secondary users (SUs) to opportunistically utilize the frequency spectrum originally assigned to licensed primary users (PUs), is a promising approach to alleviate spectrum scarcity \cite{H05}. In CR networks with single-antenna nodes, SUs can transmit only when it detects a spectrum hole in either time or frequency domain, so as to avoid causing harmful interference to PUs \cite{P09,TMS09}. Such schemes, however, only work when the primary system severely underutilizes the assigned spectrum. Otherwise, the secondary system would not have adequate chances to access the wireless channel.

Recent development in multiple-input multiple-output (MIMO) antenna techniques opens up a new dimension, namely space, for co-channel users to coexist without causing severe interference to each other \cite{GSS+03}. Indeed, in CR networks where stations are equipped with multiple antennae, SUs can transmit at the same time as the PUs through space-domain signal processing. The nature of CR networks gives rise to several challenging issues that do not exist in traditional MIMO systems. First, SUs are solely responsible for suppressing the interference they cause to PU receivers, as the primary system should not be aware of the existence of the secondary system. That is, we cannot rely on the PUs to do receiver-side interference cancellation.  Secondly, SUs may not have the luxury of knowing the channel state information (CSI) on the links to PUs, as the primary system would not deliberately provide their channel estimation to the secondary system. This imposes difficulty on transmitter-side pre-interference cancellation at SU transmitters. It is therefore necessary to revisit space-domain signal processing in the context of MIMO CR networks. In particular,  SUs need to configure their beamforming patterns in a way that balances between their own throughput and the interference they cause to PUs.

Multi-antenna CR networks were recently studied in \cite{KG08,ZL08,SF09}.
Assuming that CSI on all links is perfectly known to the SUs,  \cite{KG08} formulates the SU beamforming problem as a non-convex optimization problem. A semi-distributed algorithm is proposed to obtain a local optimal solution to the problem. On the other hand, \cite{SF09} assumes that the PU can act as a scheduler for SU transmissions. Under this idealistic assumption, an opportunistic orthogonalization scheme is proposed. In \cite{ZL08}, Zhang and Liang studied capacity-achieving transmit spatial spectrum for a single SU, assuming that the SU has full CSI and there is no interference from PUs to the SU. Insightful solution methods are proposed to provide better intuition that may not be obtainable from a numerical optimization perspective. The issue of imperfect CSI estimation is considered in \cite{ZLC08, GLM09, PVS+09} for multiple-input single-output (MISO) CR networks. Therein, robust optimization problems are formulated to ensure the service qualities for both SUs and PUs are satisfactory in the worst case.

In this paper, we study the problem of optimal secondary-link beamforming. Specifically, we aim to maximize the throughput of the SU under the constraint that the interference to PU receivers is below a certain threshold. In contrast to the previous work \cite{KG08,ZL08,SF09}, we consider three practical scenarios: (1) when the SU transmitter knows both the channel matrices from it to PU receivers and the beamforming patterns at PU receivers; (2) when the SU transmitter does not know the beamformer at PU receivers; (3) when the secondary transmitter knows neither the channel matrices to PU receivers nor the beamforming patterns at PU receivers. Note that the deterministic interference-temperature constraints could be too stringent if the SU does not have full CSI (which is the case in the second and third scenarios), as the SU will need to consider the worst-case channel realization when configuring the beamformer. Fortunately, many wireless applications can tolerate occasional dips in the service quality. In order to have a more efficient utilization of the spectrum, we can take advantage of this opportunity and replace the deterministic constraints by probabilistic interference constraints (also referred to as chance constraints). Chance constraints, however, are typically tougher to deal with than deterministic constraints.
The contribution of this paper is fourfold:
\begin{itemize}
  \item We propose a unified homogeneous quadratically constrained quadratic program (QCQP) formulation that can be applied to all three scenarios mentioned above. In particular, the homogeneous QCQP formulation can accommodate both deterministic and probabilistic interference-temperature constraints.
  \item The QCQP formulation, though non-convex, is amenable to semidefinite programming (SDP) relaxation methods. We show that the SDP relaxation admits no gap with the true optimal solution when the number of PUs is no larger than two. A polynomial-time algorithm is presented to compute the optimal solution efficiently.
  \item When there are more than two PUs, we propose a randomized polynomial-time algorithm that can produce a provably near-optimal solution. Numerical results show that the solution produced by our algorithm almost achieves the optimal value.
  \item In the third scenario where the SU transmitter knows neither the channel matrices nor the beamformer at PU receivers, we show that the optimal solution can be obtained very efficiently through a simple matrix eigenvalue-eigenvector computation. That is, there is no need to solve the QCQP problem in this case.
\end{itemize}

We should emphasize that the incomplete CSI here is not to be confused with that in \cite{ZLC08,GLM09,PVS+09}, where it is assumed that the SU knows the CSI on all links, except that there may be uncertainty in the channel estimation. Robust optimization techniques are employed in these papers to deal with the worst-case channel realization.  The case where only channel statistics is known to the SU transmitter is also studied in \cite{PVS+09}.  This is similar to a special case of the third scenario considered in our paper, when the receivers have only one antenna.

The rest of this paper is organized as follows. The system model is described in Section \ref{sec:2}. In Section \ref{section:3}, we formulate the SU beamforming problem as a series of homogeneous QCQP problems for different scenarios. The SDP relaxations of these homogeneous QCQP problems are then introduced in Section \ref{section:4}.  In Section \ref{sec:exact}, a polynomial-time algorithm for finding an optimal solution to the QCQP problem when the number of primary users is no larger than two is presented. In Section \ref{sec:largeK}, we propose a randomized polynomial-time algorithm to find a near-optimal beamforming solution when there are more than two PUs.  In Section \ref{sec:simulation}, the performance of the proposed schemes is evaluated via simulations. Finally, the paper is concluded in Section \ref{section:conclusions}.

\section{System Model}\label{sec:2}
\subsection{System Setup}
In this paper, we consider a CR network in which a secondary user intends to share the spectrum with a primary system consisting of  primary links.  We shall discuss the possibility of extending the proposed approach to a multiple-secondary-link scenario in Section \ref{section:conclusions}. In the sequel, we use the subscript  to denote the secondary link and the subscript  to denote the  primary link. Let  (or ) and  (or ) denote the number of transmit (or receive) antennae of the secondary and primary links, respectively.  We use  to denote the channel matrix from the secondary transmitter to the secondary receiver, and ,  and  to denote the channel matrices from the secondary transmitter to the  primary receiver, from the  primary transmitter to the secondary receiver, and from the  primary transmitter to the  primary receiver, respectively. We assume a Rayleigh fading and rich scattering environment, so that the entries of the channel matrices are independently and identically distributed (i.i.d.) complex Gaussian random variables with zero mean and unit variance. As pointed out in \cite{B03, MZSY08}, in an interference-limited environment, each active link should only transmit one data stream at a time to avoid excessive interference to other links. In this case, we use scalars  and  to denote the transmitted signals by the secondary transmitter and the  primary transmitter, respectively. Without loss of generality, suppose that  and .

Let  and  denote the beamforming vectors at the secondary transmitter and receiver, respectively. Likewise, let  and  be the beamforming vectors at the  primary transmitter and receiver. In particular, we have  and , where  and  are the transmit power of the secondary and the  primary links, respectively. Without loss of generality, we normalize the receive beamforming vectors so that  and .  Then, the received signal at the secondary receiver after receive beamforming is

where  and  denote the path losses from the secondary transmitter to the secondary receiver and from the  primary transmitter to the secondary receiver, respectively, and  denotes a circular complex additive Gaussian noise vector at the secondary receiver. As such, the signal to interference and noise ratio (SINR) on the secondary link is given by

The secondary link's transmission causes an interference signal  at the output of the  primary receiver, resulting in an interference power of ,
where  is the path loss from the secondary transmitter to the  primary receiver.

Before leaving this subsection, we emphasize that channel matrices on different links are independent. Furthermore, the beamforming vectors  and  are solely determined by the channels between the nodes in the primary system, as the secondary system is transparent to the primary system. Therefore,  and  are independent of , , and .

\subsection{Objective and Assumptions}

In this paper, we aim to find for the secondary link the optimal beamforming vectors  and  so that the SINR on the secondary link is maximized, while the interference to primary link  is below a tolerable threshold . Mathematically, this can be formulated as the following optimization problem:

\max_{\mathbf{t}_S, \mathbf{r}_S}&& \gamma_S \\
\text{s.t. }&&\alpha_{k,S}\big| \mathbf{r}_k^H \mathbf{H}_{k,S}\mathbf{t}_S\big|^2 \leq \epsilon_k \qquad\forall k=1,\ldots,K, \label{constraint:10b}\\
&&\|\mathbf{t}_S\|_2^2\leq P_{S,max},

where  is the maximum transmission power of the secondary link.
Throughout this paper, we assume that the path losses , , , and  do not vary significantly within the time period of interest, and hence are known to all stations. It is also reasonable to assume that the secondary user knows its own channel  at both transmitter and receiver sides. Moreover, the secondary receiver can estimate  for all  by overhearing the transmission of primary transmitters.

In practice, however, the secondary transmitter may not know the CSI on the links between primary receivers, because primary receivers would not purposely provide CSI to the secondary system. In this paper, we are interested in the following three different scenarios:

\noindent \emph{\textbf{Scenario 1:}} The secondary transmitter has perfect knowledge of the vector .

This scenario corresponds to a time division duplex (TDD) system in which channels are reciprocal and the primary receivers use the same beamforming vectors for both reception and transmission. In this case, the secondary transmitter can estimate  by overhearing the transmission of primary receivers.

\noindent \emph{\textbf{Scenario 2:}} The secondary transmitter knows  but not .

This scenario corresponds to a TDD system in which primary receivers do not use the beamforming vector  for transmission.

\noindent \emph{\textbf{Scenario 3:}} The secondary transmitter has no knowledge about  and .

This scenario corresponds to the case where the secondary link has no way to estimate the channel from the primary receiver.

Note that in both Scenarios 2 and 3, constraints \eqref{constraint:10b} are no longer well defined. Indeed, for any given , the value of the left-hand-side is uncertain to the secondary transmitter, as the realizations of  and/or  are unknown. Therefore, a revision is necessary. One way is to guarantee that the constraints are always satisfied regardless of the realizations of  and . Then, \eqref{constraint:10b} can be replaced by

where the maximization is taken over  for Scenario 2, and over  and  for Scenario 3.

Besides the worst-case guarantee, in practical applications, we may allow the interference to exceed a certain threshold  with a small outage probability . In this case, \eqref{constraint:10b} can be replaced by

where the probability is taken over  for Scenario 2, and over  and  for Scenario 3.  Note that \eqref{eqn:chance} is equivalent to \eqref{eqn:worstcase} when  for all .

\subsection{Distribution of }
In this subsection, we discuss the probability distribution of the beamforming vector  at PU receivers. The results will be useful later when we address the chance constraints in \eqref{eqn:chance}.

As mentioned,  is solely determined by the channels between primary nodes. Consider the matrix

where  is the path loss from the  PU transmitter to the  PU receiver, and  are the columns of , which are independent of each other.  Let  be the  matrix obtained by deleting the  column of . Then, in general, the vector  takes the form

where  is a normalization factor that ensures , and  is a random Hermitian matrix that is independent of . In particular, we have  for the matched-filter (MF) receiver,

for the zero-forcing (ZF) receiver, and

for the minimum-mean-squared-error (MMSE) receiver. Before proceeding, let us state a definition and introduce two assumptions.
\begin{definition}
A random vector  is called a normalized complex Gaussian vector if , where . In particular, . A normalized complex Gaussian vector is an isotropically distributed unit vector\footnote{A unit vector is said to be isotropically distributed if it is equally likely to point in any direction in the complex space. In other words, the vector is uniformly distributed on a complex unit sphere.}.
\end{definition}

\begin{assumption}\label{ass:1}
The  column of , i.e., , has the same distribution as , where  is a normalized complex Gaussian vector, and  is a scaling factor.
\end{assumption}
Assumption \ref{ass:1} is valid for most practical MIMO systems. Consider the singular value decomposition (SVD) of , where   and  are unitary matrices containing the left and right singular vectors, respectively, and  is a diagonal matrix containing the singular values. It is known that the columns of  and  have the same distribution as a normalized complex Gaussian vector \cite{MH99}. In the case of single-user precoding, it is optimal to set  to be proportional to , the right singular vector corresponding to the maximum singular value  \cite{RC98}. As a result,  is proportional to , where  is the left singular vector corresponding to the maximum singular value. Hence, Assumption \ref{ass:1} is valid. It can also be shown that the assumption is valid when linear multiuser precoding is deployed.

\begin{assumption}\label{ass:2}
The entries of  are independent complex Gaussian random variables.
\end{assumption}
Assumption \ref{ass:2} is in general valid.  Indeed, it is obvious that the columns of  are independent, as they are related to channel matrices on different links. Moreover, for a given , we have , thus implying that  has independent entries. Being the transmit precoding vector of the  primary link,   is typically a function of  and is independent of . This justifies the validity of Assumption \ref{ass:2}.


By Assumption \ref{ass:2}, we can write  as , where

and  is an  matrix with independent standard complex Gaussian entries. Let  be the SVD of . It is known that  and  are isotropically distributed unitary matrices\footnote{A unitary matrix is said to be isotropically distributed if its probability density is unchanged when premultiplied by a deterministic unitary matrix.}, and that  are independent \cite{MH99}.

\begin{proposition} \label{pro:1}
 has the same distribution as a normalized complex Gaussian vector as long as the random Hermitian matrix  is a unitarily invariant matrix\footnote{A random Hermitian matrix  is called unitarily invariant if the joint distribution of its entries equals that of  for any unitary matrix  independent of .}
\end{proposition} 

\begin{proof}
Being a unitarily invariant matrix,  can be decomposed as , where  is an isotropically distributed unitary matrix independent of the diagonal matrix  \cite{TV04}. Thus, . Since the distribution of  is rotationally invariant and  is independent of ,  has the same distribution as . It follows that  has the same distribution as

Upon conditioning on  and ,  is a deterministic unit vector. Since multiplying any deterministic unit vector by an isotropically distributed unitary matrix results in an isotropically distributed unit vector \cite{TV04}, the unit vector  is isotropically distributed for the given  and . Since this holds for any realizations of  and , it follows that  is isotropically distributed, and therefore has the same distribution as a normalized complex Gaussian vector.
\end{proof}

\begin{corollary} \label{cor:1}
 has the same distribution as a normalized complex Gaussian Gaussian vector when a MF, ZF, or MMSE receiver is deployed.
\end{corollary}
\begin{proof}
To prove Corollary \ref{cor:1},  all we need is to show that   is unitarily invariant for MF, ZF, and MMSE receivers. As this is trivial in the MF case, where , we will focus on the cases with ZF and MMSE receivers in the following.

For ZF receivers, substituting   to \eqref{eqn:ZF}, we have . Since  is an isotropically distributed unitary matrix, it has the same distribution as  for any unitary matrix  that is independent of . Therefore,  is unitarily invariant.

In the case of MMSE, \eqref{eqn:MMSE} can be written as , where  is a Hermitian matrix. Since  is an isotropically distributed unitary matrix,  is unitarily invariant.

\end{proof}


\section{Optimal Beamforming as Homogeneous QCQP}\label{section:3}
In this section, we show that the optimal SU beamforming problems that arise in the three scenarios discussed in Section \ref{sec:2} can all be formulated as quadratically constrained quadratic programming (QCQP) problems.  To begin, let us simplify Problem (\ref{eqn:generalform}) by exploiting the properties of its optimal receive beamforming solution .  Observe that the variable  appears only in the objective function of Problem \eqref{eqn:generalform}.  Thus, for any given , the optimal  that maximizes  is simply an MMSE receiver \cite{JD93} given by

where , and  is a normalization factor that ensures .

Upon substituting  into  (see (\ref{eq:SINR})), we obtain , where .  In particular, we can eliminate the variable  from Problem (\ref{eqn:generalform}) and replace the objective function with the quadratic form .

\subsection{Homogeneous QCQP Formulation in Scenario 1}\label{subsection:3-2}

Now, recall that the secondary transmitter has perfect knowledge of  in Scenario 1. With the optimal  given in \eqref{eqn:rS}, Problem \eqref{eqn:generalform} becomes

\max_{\mathbf{t}_S} && \mathbf{t}_S^H\mathbf{A}\mathbf{t}_S \label{eqn:bf-obj} \\
\text{s.t. } && \mathbf{t}_S^H\mathbf{Q}_k^1\mathbf{t}_S \leq 1 \qquad \forall k=1,\ldots,K, \label{constraint14b}\\
&& \mathbf{t}_S^H\mathbf{t}_S \leq P_{S,max}, \label{eqn:bf-power}

where

are Hermitian positive semidefinite matrices. Problem \eqref{eqn:Scenario1} is a homogeneous QCQP, where both the objective function and inequality constraints are quadratic without linear terms.

In subsequent subsections, we will show that similar homogeneous QCQP problems can be formulated for Scenarios 2 and 3, with  in \eqref{constraint14b} replaced by some suitable matrices  and , respectively.

\subsection{Homogeneous QCQP Formulation in Scenario 2}\label{subsection:3-3}

In Scenario 2, the realization of  is not known to the secondary transmitter.  In order to have a more efficient utilization of the spectrum, we can exploit the distribution of  and consider the probabilistic interference constraints

To tackle the constraints in (\ref{eqn:chance2}), we need the following lemma:
\begin{lemma}\label{lem:scn2-F}
   Let  be a normalized complex Gaussian vector, i.e., , where  is a standard complex Gaussian vector.  Let  be an arbitrary vector, and let  be arbitrary scalars.  Then, we have
   
\end{lemma}
\noindent{\emph{Proof:}}  Since the distribution of  is rotationally invariant, we may assume without loss of generality that . Then, we have

where  denotes the  entry of a vector. Note that  has the same distribution as , where  is the standard real chi-square random variable with  degrees of freedom.  Moreover, it is independent of , which has the same distribution as .  Hence, we have

Now, let

be the regularized incomplete beta function.  It is known that the random variable

follows the so-called -distribution with  degrees of freedom, whose cumulative distribution function (CDF) is given by .  Upon working out the summation in \eqref{eqn:incomplete}, we obtain

Thus, we conclude that

which completes the proof.


By Lemma \ref{lem:scn2-F}, we can rewrite the chance constraints (\ref{eqn:chance2}) as

which of course are equivalent to

where

In particular, the optimal beamforming vector  in Scenario 2 can be found by solving the QCQP problem (\ref{eqn:bf-obj}), (\ref{eq:scn2-qc}) and (\ref{eqn:bf-power}).

As mentioned earlier, when , the chance constraints \eqref{eqn:chance2} are equivalent to the worst-case constraints \eqref{eqn:worstcase}. In this case, we have .

\subsection{Homogeneous QCQP Formulation and Closed-Form Solution in Scenario 3}\label{subsection:3-4}

\subsubsection{\textbf{Homogeneous QCQP}}

In Scenario 3, both  and  are unknown to the secondary transmitter. Similar to Scenario 2, we consider the following probabilistic interference constraints:

Since  and  are independent, upon conditioning on , we see that  is a complex Gaussian random variable with mean 0 and variance . Note that the conditional distribution of  is independent of . It follows that unconditionally, we have .  In particular, the random variable  follows an exponential distribution with parameter .  Note that this result holds as long as  is a unit-length vector, regardless of its distribution.

Since

it follows that the chance constraints in \eqref{eqn:chance3} can be written as

or equivalently,

where .  Thus, the optimal beamforming vector  in Scenario 3 can be found by solving the QCQP problem (\ref{eqn:bf-obj}), (\ref{eq:scn3-qc}) and (\ref{eqn:bf-power}).

We would like to emphasize that if the worst-case interference temperature constraints \eqref{eqn:worstcase} are to be satisfied in Scenario 3, i.e., , then the only feasible solution is . That is, the secondary transmitter can never transmit under the overly stringent constraints. On the other hand, by allowing a small outage probability , the secondary link can transmit on the same spectrum of the primary system.

\subsubsection{\textbf{Closed-Form Solution}}

Here, we show that the optimal  in Scenario 3 can be found very efficiently by a simple eigenvalue-eigenvector computation.  Indeed, observe that the constraints (\ref{eq:scn3-tnorm}) and (\ref{eqn:bf-power}) can be combined to yield the single constraint

Thus, the optimal beamforming problem in Scenario 3, which is given by

is reduced to the problem of finding the largest eigenvalue of  and its associated eigenvector.  Specifically, let  be the eigenvector of  corresponding to the largest eigenvalue.  Then, the optimal solution to \eqref{eqn:eigenvalueproblem} is simply .  Note that there is no need to solve any QCQP in this case.

\section{SDP Relaxation}\label{section:4}
We have shown in the last section that the optimal beamforming solution can be efficiently obtained by a simple eigenvalue-eigenvector computation in Scenario 3. However, to obtain the optimal solutions for Scenarios 1 and 2, homogeneous QCQP problems of the following form have to be solved:

\max_{\mathbf{t}_S} && \mathbf{t}_S^H\mathbf{A}\mathbf{t}_S \\
\text{s.t. } && \mathbf{t}_S^H\mathbf{Q}_k\mathbf{t}_S \leq 1 \qquad\forall k=1,\ldots,K, \label{sdp-constraint14b}\\
&&\mathbf{t}_S^H\mathbf{t}_S \leq P_{S,max}\label{sdp-constraint14c}.

Here,  is equal to  and  in Scenarios 1 and 2, respectively.  Unfortunately, since Problem (\ref{eqn:QCQP}) involves \emph{maximizing} a convex function over an intersection of  ellipsoids, it is NP-hard in general \cite{NRT99}.  In this section, we show how Problem (\ref{eqn:QCQP}) can be tackled using semidefinite programming (SDP) relaxation methods.

To begin, observe that  for any matrix , where  is a rank one Hermitian positive semidefinite matrix.  Thus, by relaxing the rank constraint , we obtain the following SDP relaxation of Problem (\ref{eqn:QCQP}):

\max_{\mathbf{X}\succeq\mathbf{0}}&& \mathrm{tr}(\mathbf{A}\mathbf{X})\\
\text{s.t. }&&\mathrm{tr}(\mathbf{Q}_k\mathbf{X}) \leq 1 \qquad\forall k=1,\ldots,K, \label{eqn:SDP-constraint1}\\
&&\mathrm{tr}(\mathbf{X}) \leq P_{S,max} \label{eqn:SDP-constraint2}.

The dual of (\ref{eqn:SDP1}) is given by

\min_{y_1,\ldots,y_K,y_{K+1}}&& \sum_{k=1}^{K}y_k+P_{S,max}y_{K+1}, \\
\text{s.t. } && \sum_{k=1}^{K} y_k\mathbf{Q}_k+y_{K+1}\mathbf{I}-\mathbf{A}\succeq\mathbf{0}, \label{eqn:SDP-dual-constraint1}\\
&& y_k\geq0 \qquad\forall k=1,\ldots,K+1.

It is known that SDP problems are convex and can be solved in polynomial time using standard interior-point methods \cite{VB96}.  Moreover, if we can find a rank-one optimal solution to (\ref{eqn:SDP1}), then we can extract from it an optimal solution to the original QCQP problem (\ref{eqn:QCQP}).  In this case, there is no gap between the optimal value of (\ref{eqn:QCQP}) and that of (\ref{eqn:SDP1}), and Problems (\ref{eqn:QCQP}) and (\ref{eqn:SDP1}) are equivalent.  Of course, the SDP relaxation \eqref{eqn:SDP1} is in general not equivalent to the QCQP problem (\ref{eqn:QCQP}), as we have discarded the rank constraint .  In the next two sections, we will discuss how to recover a rank-one solution from an optimal solution to \eqref{eqn:SDP1}.  Before proceeding, however, let us introduce the following lemma, which will be useful for our later discussions.

\begin{lemma}\label{lem:2}
Both \eqref{eqn:SDP1} and its dual \eqref{eqn:SDP-dual} satisfy the Slater condition, i.e., they are strictly feasible.
\end{lemma}
We omit the proof here due to page limit.
\begin{remark}
Since both \eqref{eqn:SDP1} and \eqref{eqn:SDP-dual} are strictly feasible, a pair of primal and dual feasible solutions  to \eqref{eqn:SDP1} and \eqref{eqn:SDP-dual} is optimal if and only if the following complementary conditions hold:

&&\mathrm{tr}\left(\mathbf{X}^*\left(\sum_{k=1}^{K} y_k^*\mathbf{Q}_k+y_{K+1}\mathbf{I}-\mathbf{A}\right)\right)=0,\label{eqn:comp1}\\
&&y_k^*\left(\mathrm{tr}(\mathbf{Q}_k\mathbf{X^*})-1\right)=0 \qquad\forall k=1,\ldots,K, \label{eqn:comp2} \\
&&y_{K+1}^*\left(\mathrm{tr}(\mathbf{X^*})-P_{S,max}\right)=0.

\end{remark}

\section{Optimal Rank-One Solution when }\label{sec:exact}
As it turns out, when there are no more than two primary links (i.e., when ), there is no gap between the optimal value of the SDP relaxation \eqref{eqn:SDP1} and that of the original QCQP problem \eqref{eqn:QCQP}.  Moreover, a rank-one optimal solution to (\ref{eqn:SDP1}), and hence an optimal solution to (\ref{eqn:QCQP}), can be found in polynomial time.  Specifically, we have the following proposition, which follows directly from the results of Huang et al.~\cite{HZ07,HdMZ10}:
\begin{proposition}\label{pro:nogap}
The homogeneous QCQP problem \eqref{eqn:QCQP} can be solved exactly in polynomial time when the number of primary links  is at most 2. In particular, an optimal solution to (\ref{eqn:QCQP}) can be constructed from an optimal solution of \eqref{eqn:SDP1} in polynomial time.
\end{proposition}
Proposition \ref{pro:nogap} establishes the existence of polynomial-time algorithms for constructing optimal solutions to \eqref{eqn:QCQP} when .  In subsections \ref{subsection:4-1} and \ref{subsection:4-2}, we describe two such algorithms---one for the case where , and the other for the case where .  Both of them are based on the following decomposition theorem of Huang et al.~\cite{HZ07,HdMZ10}:
\begin{theorem}\label{thm:decompose}
Suppose that  is a Hermitian positive semidefinite matrix of rank R, and  and  are two given Hermitian matrices. Then, there is a rank-one decomposition of , namely, , such that  and  for all .  Moreover, such a decomposition can be found in polynomial time.
\end{theorem}
We refer the interested readers to \cite{HdMZ10} for the proof. To be self-contained, the algorithm for computing the decomposition guaranteed by Theorem \ref{thm:decompose}, which runs in polynomial time, is given in Algorithm \ref{alg:1}.
\begin{algorithm}
\caption{Algorithm for computing the decomposition guaranteed by Theorem \ref{thm:decompose}}\label{alg:1}
\begin{algorithmic}[1]
\REQUIRE Hermitian matrices   and , and Hermitian positive semidefinite matrix  with .
\ENSURE , a rank-one decomposition of  such that , , for .
\STATE Compute an arbitrary rank-one decomposition  such that  using, for example, Cholesky factorization.
\STATE Let .
\REPEAT
\IF { for all }
\STATE .
\ELSE
\STATE Let  be such that .
\STATE Determine  such that .
\STATE , and set 
\ENDIF
\IF {}
\STATE .
\ENDIF
\STATE .
\UNTIL  \COMMENT{Comment: We have now found a decomposition  such that .}
\STATE Let .
\REPEAT
\IF { for all }
\STATE .
\ELSE
\STATE Let  be such that .
\STATE Compute the arguments  and  and the modulus , and determine  such that

\STATE Set .
\STATE , and set 
\ENDIF
\IF {}
\STATE .
\ENDIF
\UNTIL 
\end{algorithmic}
\end{algorithm}

\subsection{Optimal Rank-One Solution when }\label{subsection:4-1}
When , there are two quadratic constraints in \eqref{eqn:QCQP}, namely,  and . Let  be an arbitrary optimal solution to \eqref{eqn:SDP1}, which can be obtained in polynomial time by standard interior-point algorithms. We now show how to construct a rank-one solution to \eqref{eqn:SDP1} from .  By Theorem \ref{thm:decompose}, we can find a rank-one decomposition  such that

where  is the rank of . Now, choose an arbitrary , say , and let . From \eqref{X-decompose} and the fact that  is a feasible for \eqref{eqn:SDP1}, we have

Thus, we conclude from \eqref{eqn:feasible1} that  is a rank-one feasible solution to \eqref{eqn:SDP1}.

Now, from the complementary conditions in \eqref{eqn:comp}, we have

Since  by \eqref{eqn:SDP-dual-constraint1}, we have  for .  This implies that .  In a similar fashion, since  and , we have  and .  This, together with the feasibility conditions in \eqref{eqn:feasible1}, leads to the conclusion that the rank-one matrix  is an optimal solution to \eqref{eqn:SDP1}, and  is an optimal solution to \eqref{eqn:QCQP}.

Note that  can be computed in polynomial time, as both  and the decomposition  can be computed in polynomial time.

\subsection{Optimal Rank-One Solution when }\label{subsection:4-2}
When , there are three quadratic constraints in \eqref{eqn:QCQP}, namely, ,  and . In the following, we show how to construct a rank-one solution from an arbitrary optimal solution  to \eqref{eqn:SDP1} in two different cases.

\subsubsection{\textbf{At least one constraint in (\ref{eqn:SDP1}) is non-binding at optimality}}
Without loss of generality, suppose that  is the non-binding constraint, while  and  can be either binding or non-binding. Due to the complementary conditions \eqref{eqn:comp2}, we must have .  Now, construct, via Theorem \ref{thm:decompose}, a rank-one decomposition  such that  and  for . Since , there must exist an  such that . Without loss of generality, assume that  and let . By the same argument as in the preceding subsection, the following feasible conditions and complementary conditions hold:

Moreover, we have  and  since .  Hence,  is an optimal rank-one solution to \eqref{eqn:SDP1} and  is an optimal solution to \eqref{eqn:QCQP}.

\subsubsection{\textbf{All constraints in (\ref{eqn:SDP1}) are binding at optimality}}
When all constraints are binding, we have  and . Then, we have .  In this case, we construct, again via Theorem \ref{thm:decompose}, a rank-one decomposition  such that

and

for .  Since , there must exist an , say , such that . Let , and consequently we have .  This, together with \eqref{eqn:equal1} and \eqref{eqn:equal2}, leads to  and . Hence,  is a feasible solution, and the complementary conditions in \eqref{eqn:comp2} are satisfied. Furthermore, since , we see that .  Hence, we conclude that  is an optimal rank-one solution to \eqref{eqn:SDP1}, and  is an optimal solution to the QCQP problem \eqref{eqn:QCQP}. Moreover,  can be computed in polynomial time.

\section{Rank-One Solution when }\label{sec:largeK}
When , there may not exist any rank-one optimal solution to the SDP \eqref{eqn:SDP1}.  Moreover, the QCQP problem (\ref{eqn:QCQP}) is NP-hard in general, and hence it is unlikely that we can extract, in polynomial time, an optimal solution to it from an optimal solution to the SDP (\ref{eqn:SDP1}).  However, as we shall see, we can generate a provably near-optimal solution to the QCQP problem (\ref{eqn:QCQP}) from an optimal solution to the SDP (\ref{eqn:SDP1}) using a very simple randomized procedure.

To begin, let  be an optimal solution to the SDP \eqref{eqn:SDP1}.  Define , so that constraint \eqref{eqn:bf-power} is equivalent to .  Consider the randomized procedure outlined in Algorithm \ref{alg:2} for generating a feasible solution to (\ref{eqn:QCQP}) from .  Algorithm \ref{alg:2} can be viewed as a generalization of the procedure developed by Nemirovski et al.~\cite{NRT99} for handling {\it real} homogeneous QCQP problems.  Our goal now is to show that Algorithm \ref{alg:2} indeed returns a feasible solution to (\ref{eqn:QCQP}).  In fact, we will prove in Theorem \ref{thm:random} that the solution returned by Algorithm \ref{alg:2} is not only feasible to (\ref{eqn:QCQP}), but is also likely to be a good one, in the sense that it has an objective value that is close to the optimal value of the QCQP problem (\ref{eqn:QCQP}).  We note that such a phenomenon can also be observed from our simulations, as will be explained in the next section.

\begin{algorithm}
\caption{Generate a feasible solution to (\ref{eqn:QCQP}) from an optimal solution  to (\ref{eqn:SDP1})}\label{alg:2}
\begin{algorithmic}[1]
\REQUIRE An optimal solution  to the SDP (\ref{eqn:SDP1}).
\ENSURE A feasible solution  to \eqref{eqn:QCQP}.

\STATE Decompose , where . Let  and , where .  It can be shown that  and .

\STATE Find an unitary matrix  that diagonalizes , i.e.,  is a diagonal matrix. Set .

\STATE Let  be an  random vector whose entries are independently and uniformly distributed on the unit circle in the complex plane. In other words, we have , where  is uniformly distributed between 0 and .

\STATE Return  as the solution.
\end{algorithmic}
\end{algorithm}



Before we introduce and prove Theorem \ref{thm:random}, let us note the following facts:
\begin{fact}\label{fact:rank}
\cite{HZ07} There exists an optimal solution to Problem \eqref{eqn:SDP1} with rank , where  is the number of quadratic constraints. Moreover, such an optimal solution can be found in polynomial time.
\end{fact}

\begin{fact}\label{fact:Qk}
We have  for .  In particular, we can decompose  as  for some .
\end{fact}
In order to study the quality of the solution returned by Algorithm \ref{alg:2}, we need the following lemmata:
\begin{lemma}\label{lem:A}
Let  be given.  Consider the events

where  is obtained from the rank-one decomposition of  (see Fact \ref{fact:Qk}). Then, we have

\end{lemma}
\noindent{\emph{Proof:}} If  does not take place, then we have  for all  and .  This implies that

Note that .  Hence, \eqref{eqn:lema1} implies that  when  does not take place.  This completes the proof. 
\begin{lemma} \label{lem:Hoeffding}
(Hoeffding's Inequality, Complex Version) Let  be independent complex-valued random variables with  and  for .  Then, for any , we have

where  denotes the -norm of the vector .
\end{lemma}
\noindent{\emph{Proof:}} Let  and  be the real and imaginary parts of , respectively.  Then, we have ,  and  for , and

The desired result then follows from an application of the real version of the Hoeffding inequality \cite{H63}.


We are now ready to present Theorem \ref{thm:random}.  It extends Nemirovski et al's result in \cite{NRT99}, which is concerned with real homogeneous QCQP problems, to the case of \emph{complex} homogeneous QCQP problems.

\begin{theorem}\label{thm:random}
The vector  returned by Algorithm \ref{alg:2} is well defined and is a feasible solution to Problem \eqref{eqn:QCQP}. Moreover, for any , we have

where .  In particular, with probability at least , the objective value of the solution returned by Algorithm \ref{alg:2} is at least  times the optimal value of the QCQP problem (\ref{eqn:QCQP}).
\end{theorem}
\noindent\emph{Proof:} We first prove that  is well defined, i.e., . To see this, note that , where .  Since , it follows that , and hence , must be strictly larger than zero.

Now, observe that

for .  It follows that  is a feasible solution to \eqref{eqn:QCQP}.

Next, we compute

where the second equality is due to the fact that  is a diagonal matrix and  for .  Hence, to prove the bound in \eqref{eqn:prob}, it suffices to show that

Now, by Lemma \ref{lem:A}, we have .  Moreover, since  and  for , we have

by Lemma \ref{lem:Hoeffding}.  This establishes \eqref{eqn:prob1} and hence the bound in \eqref{eqn:prob}.

Finally, the last statement in the theorem follows from the observation that  is an upper bound on the optimal value of the QCQP problem \eqref{eqn:QCQP}, as (\ref{eqn:SDP1}) is a relaxation of (\ref{eqn:QCQP}).  This completes the proof of Theorem \ref{thm:random}.






Before leaving this section, we emphasize that the optimal beamforming vector  can always be found efficiently in Scenario 3 through a matrix eigenvalue-eigenvector computation, regardless of the number of primary links. In Scenarios 1 and 2, however, the optimal solution can be obtained in polynomial time only when  is no larger than two. Otherwise, we can only find an approximate solution in polynomial time via Algorithm \ref{alg:2}. Fortunately, as we will show in the next section, the approximate solution is nearly optimal most of the time.

\section{Numerical Simulations}\label{sec:simulation}
In this section, the performance of the proposed algorithms are evaluated through simulations. Throughout this section, we assume that all stations are equipped with 4 antennae. The wireless fading channel is Rayleigh distributed, and path loss exponent equals 4. The length of the secondary link is 10 meters, and  is chosen in such a way that the average SNR received by each antenna at the secondary receiver is 10dB, if there is no interference. We also assume that all primary users transmit at power . Likewise, the transmit beamforming vector  of primary user  are set to be the dominant right singular vector of  . Meanwhile, the primary receivers use MMSE beamforming vectors, as given in Subsection II-C. Unless otherwise stated,  is set to  for all  in Scenarios 2 and 3. Each point in the figures is an average of 50000 independent simulation runs.

\subsection{}

We first investigate a network with one secondary link and two primary links. The primary links are placed such that the distances between the secondary transmitter and the two primary receivers are 15 and 13 meters, respectively, while the distances between the secondary receiver and the two primary transmitters are 12.4 and 12.7 meters. As discussed in previous sections, the optimal beamforming solution  can be found in polynomial time in this case.

In Fig. \ref{fig:K2SINR}, the optimal SINR  (in dB scale) is plotted against , when  varies from 0 to 10 dB for all . It is not surprising to see that  increases with the increase of the tolerable interference  at the primary receivers. Meanwhile, the more channel information at the secondary transmitter, the higher the SINR at the secondary receiver, especially when  is low. Noticeably, the SINR gap between the three scenarios narrows when  increases. This is because when the primary users can tolerate higher interference levels, the secondary user can spend less effort in eliminating interference to the primary users. Hence, the advantage of knowing  and  becomes less obvious.

Fig. \ref{fig:K2outage} illustrates the tradeoff between the optimal SINR  of the secondary link and the outage probability  of the primary links. It is not surprising that in both Scenarios 2 and 3, the secondary link can achieve a higher SINR when the primary links can tolerate a higher outage probability.

\begin{figure}[!ht]
\centering
\includegraphics[width=0.5\textwidth]{K=2_4antennas.eps}
\caption{SINR at the secondary receiver vs.  when .}\label{fig:K2SINR}
\end{figure}

\begin{figure}[!ht]
\centering
\includegraphics[width=0.5\textwidth]{K2_outage_tradeoff.eps}
\caption{Tradeoff between SINR and outage probability  when  and .}\label{fig:K2outage}
\end{figure}

\subsection{}
We now simulate a network with one secondary link and four primary links. The distance between the secondary transmitter and the four primary receivers are 20, 18, 15 and 13 meters, while that between the secondary receiver and the four primary transmitters are 16, 14, 12.4 and 13.2 meters, respectively. With four primary links, only approximate solutions can be obtained in polynomial time in Scenarios 1 and 2, as discussed in Section \ref{sec:largeK}.

In Fig. \ref{fig:K4SINR}, the randomized algorithm in Algorithm \ref{alg:2} is carried out to obtain the beamforming vector in Scenarios 1 and 2. The optimal beamforming vector in Scenario 3 is obtained through an eigenvalue-eigenvector computation. For comparison, we also plot the optimal value of the SDP relaxation \eqref{eqn:SDP1}, which is an upper bound on the maximum achievable SINR. As the figure shows, the randomized algorithm performs very close to the optimum. The achieved SINR almost overlaps with the upper bound of the optimal SINR. Meanwhile, similar conclusions drawn from Fig. \ref{fig:K2SINR} also apply here.

The tradeoff between  and  in the four primary link case is illustrated in Fig. \ref{fig:K4outage}. Recall that in Scenario 3, the only feasible solution when  is . Fortunately, the achievable SINR in Scenario 3 increases rapidly with  as long as , as shown in both Fig. \ref{fig:K2SINR} and \ref{fig:K4SINR}.

\begin{figure}[!ht]
\centering
\includegraphics[width=0.5\textwidth]{K=4_4antennas.eps}
\caption{SINR at the secondary receiver vs.  when .}\label{fig:K4SINR}
\end{figure}

\begin{figure}[!ht]
\centering
\includegraphics[width=0.5\textwidth]{K=4_4antennas_outage.eps}
\caption{Tradeoff between SINR and outage probability  when  and .}\label{fig:K4outage}
\end{figure}

\subsection{A Grid Network with 9 Primary Links}
In this subsection, we consider a network with 9 primary links arranged in a 70-by-40 meter grid as shown in Fig. \ref{fig:grid}. The lengths of all link are equal to 10 meters. The secondary link is randomly placed in the area. In Fig. \ref{fig:K9SINR}, SINR at the secondary receiver is plotted against . Each point in the curve is an average of 20000 independent secondary-link placements.

Again, the figure shows that Algorithm \ref{alg:2} performs very close to the optimum. The achieved SINR almost overlaps with the upper bound of its optimal value. With full CSI, Scenario 1 can achieve a much higher SINR than Scenarios 2 and 3, especially when  is small. The better performance, however, comes at a price. In practical systems where full CSI is not available, one has to resort to the schemes developed for Scenarios 2 and 3 to achieve the maximum SINR.

\begin{figure}[!ht]
\centering
\includegraphics[width=0.5\textwidth]{Grid.eps}
\caption{Placement of 9 primary links.}\label{fig:grid}
\end{figure}

\begin{figure}[!ht]
\centering
\includegraphics[width=0.5\textwidth]{K=9_4antennas.eps}
\caption{SINR at the secondary receiver vs.  when .}\label{fig:K9SINR}
\end{figure}

\section{Conclusions and Discussions}\label{section:conclusions}
In this paper, we considered optimal secondary-link beamforming in MIMO CR networks when the secondary transmitter has complete, partial, or no channel knowledge on the links to primary receivers. We proposed a unified homogeneous QCQP formulation for all three scenarios with either deterministic or probabilistic interference-temperature constraints. In Scenario 3, the QCQP problem reduces to a matrix eigenvalue-eigenvector computation problem, which can be solved very efficiently. For Scenarios 1 and 2, we approached the QCQP problem by SDP relaxation. Notably, the SDP relaxation admits no gap with the true optimal value when there are no more than two primary links. In this case, the optimal beamforming solution can be computed in polynomial time. When the number of primary users exceeds two, we proposed a randomized polynomial-time algorithm that can construct a provably near-optimal solution to the QCQP problem from an optimal solution to the SDP.

The reader may notice that there is a gap between the theoretical performance of the randomized polynomial-time algorithm (Algorithm \ref{alg:2}) as established in Section \ref{sec:largeK} and its practical performance as demonstrated in Section \ref{sec:simulation}.  This can be attributed to the fact that the main theoretical result in Section \ref{sec:largeK}, namely Theorem \ref{thm:random}, only provides a {\it worst-case} guarantee on the performance of Algorithm \ref{alg:2}.  In other words, the guarantee is valid regardless of the distribution of the input data.  However, in the setting of MIMO CR networks, the input data follow a specific probability distribution, and the worst-case instance may not arise too frequently.  It would be interesting to see whether one can obtain better theoretical guarantees by performing a {\it probabilistic analysis} of the performance of Algorithm \ref{alg:2} (see \cite{SY10} and the references therein for related work).

So far, we have considered one secondary link only. However, the proposed schemes can be easily extended to a multiple-secondary-link system with the aid of medium-access-control (MAC). Suppose that there is a narrowband busy-tone channel in addition to the data-transmission channel. When a secondary link wishes to transmit a packet, it first senses the channel to see whether there is another secondary link transmitting. If not, it sends a short busy tone on the busy-tone channel to reserve the airtime. Other secondary links, upon hearing the busy tone, will keep silent during the airtime reserved by the transmitting link. Having successfully reserved the airtime, the link will then start to transmit its data packet on the data-transmission channel. In case more than one secondary transmitter sends busy tones at the same time, a collision has occurred on the busy-tone channel and the secondary transmitters will each wait for a random time period before attempting again. By doing so, it is guaranteed that there is only one secondary link transmitting data packets at a time, and the proposed optimal beamforming methods can be applied. For practical implementation, we can adopt the random-access protocols in IEEE 802.11 wireless local area networks (WLANs), such as RTS/CTS DCF, to coordinate the contention on the busy-tone channel.

Note that multiple secondary links can also coexist without the aid of a MAC protocol by properly configuring their beamforming vectors, preferably in a distributed manner. In this case, secondary links interfere with each other, and thus the optimal beamforming problem becomes much more challenging. We will address this problem in our future research.

\bibliographystyle{IEEEbib}
\bibliography{sdpbib-jsac}






\end{document} 