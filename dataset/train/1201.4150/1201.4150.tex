
\documentclass[11pt]{article}
\hyphenation{op-tical net-works semi-conduc-tor IEEEtran}
\usepackage{amsbsy}
\usepackage{amssymb,amsmath}
\usepackage{graphicx,latexsym,longtable}

\begin{document}


\title{Optimal Threshold Control by the Robots of Web Search Engines with Obsolescence of Documents}

\author{Konstantin Avrachenkov\thanks{INRIA Sophia Antipolis, France, K.Avrachenkov@sophia.inria.fr},
Alexander Dudin\thanks{Belarusian State University, Belarus, dudin@bsu.by},
Valentina Klimenok\thanks{Belarusian State University, Belarus, klimenok@bsu.by},\\
Philippe Nain\thanks{INRIA Sophia Antipolis, France, nain@sophia.inria.fr},
Olga Semenova\thanks{Belarusian State University, Belarus, olgasmnv@tut.by}
}


\date{}

\maketitle

\begin{abstract}
A typical Web Search Engine consists of three principal parts: crawling engine,
indexing engine, and searching engine. The present work aims to optimize the
performance of the crawling engine. The crawling engine finds new Web pages and
updates Web pages existing in the database of the Web Search Engine. The crawling
engine has several robots collecting information from the Internet. We first
calculate various performance measures of the system (e.g., probability of
arbitrary page loss due to the buffer overflow, probability of starvation of
the system, the average time waiting in the buffer). Intuitively, we would like
to avoid system starvation and at the same time to minimize the information loss.
We formulate the problem as a multi-criteria optimization problem and attributing
a weight to each criterion we solve it in the class of threshold policies.
We consider a very general Web page arrival process modeled by Batch Marked
Markov Arrival Process and a very general service time modeled by Phase-type
distribution. The model has been applied to the performance evaluation and
optimization of the crawler designed by INRIA Maestro team in the framework
of the RIAM INRIA-Canon research project.
\end{abstract}






\section{Introduction}
The problem of control by the robots (crawlers) that traverse the Web
and bring Web pages to the indexing engine that updates the data
base of a Web Search Engine is formulated and analyzed in
\cite{tlnc}. This problem is formulated in \cite{tlnc} as the
controlled queueing system. The system has a single server with the
exponential service time distribution, finite buffer of capacity
 There are  available robots and each of these
robots, when activated, brings pages to the server in a Poisson
stream at fixed rate.  These  stationary Poisson processes are
mutually independent and independent of service times.

The number of active robots may be modified at any arrival or
departure epoch. When an arrival occurs, the incoming robot is
de-activated at once; the controller may then decide to keep it idle
or to activate it. When a departure occurs the controller may either
decide to activate one additional robot, if one is available, or to
do nothing (i.e. the number of active robots is left unchanged).

In \cite{tlnc}, the problem of finding a policy that minimizes a
weighted sum of the loss rate and starvation probability
(probability of the empty system) is considered. It is solved by
means of the tools of the Markov Decision Problems theory.

As the possible generalizations of the model, which are certainly
worthwhile analyzing, the following ones are mentioned in
\cite{tlnc}:
\begin{itemize}
\item[] More general input processes, e.g., a  ({\it
Markov Modulated Poisson Process}) should be considered so as to
reflect more accurately ``traveling times'' of robots in the network;
\item[] Because of the obsolescence of stored documents
issue, the waiting time should be bounded, even if the buffer size is
effectively infinite;
\item[] Other cost functions could be investigated, for
instance, cost functions including response times.
\end{itemize}

In this paper, we made all the mentioned and some further
generalizations.

We assume that, under the fixed number of currently active robots,
the arrival process is of the  type. The  is a more
general process comparing to the  and allows delivering of a
batch of Web pages to be indexed while the  assumes that the
pages are delivered one-by-one. It is very typical for a computer
system to operate in batch mode.

We assume that the service time distribution is of the  (Phase) 
type which is much more general comparing to the exponential distribution
assumed in \cite{tlnc}. The class of phase type distributions is
dense in the field of all positive-valued distributions and practically
we can deal with any real distribution \cite{ANO96}.

Since web pages can become obsolete, we bound the waiting time stochastically. 
Waiting time of each web page in a buffer is restricted by a random variable 
having  distribution identical and mutually independent for all Web pages.
The phase type distribution has been used to model obsolescence times
for instance in \cite{W87}.

We suppose that the cost function can have a more general form 
than in \cite{tlnc} and include the obsolescence probability and 
response time.

In the next section we formulate the model and optimization problem.
Section 3 contains the steady-state analysis of the
multi-dimensional Markov chain which defines dynamics of the system
under the fixed values of the parameters defining the strategy of
control. In Section 4, main performance measures of the system are
computed. In Section 5, the conditional sojourn time distributions
are calculated. In Section 6, the case of ordinary arrivals is touched in brief.
In Section 7, the theoretical results are illustrated by numerical examples.
In particular, the mathematical model is applied to the performance evaluation
and optimization of the robot designed by INRIA Maestro team in the
framework of the RIAM INRIA-Canon research project.
Section 8 concludes the paper.



\section{Mathematical Model}

We consider a single server system with the finite buffer of
capacity  So, the total number of Web pages which
can stay in the system is restricted by the number  Web pages
are served by a server in order of their arrivals.

Service times of Web pages are independent identically distributed
random variables having  distribution with irreducible
representation  It means the following.
Service of a Web page is defined as a time until the continuous-time
Markov chain  having the states  as
the transient and state  as absorbing one reaches the absorbing
state. An initial state of the chain is selected in a random way,
according to the probability distribution defined by the row-vector
, where  is the
stochastic row vector of dimension . Transitions of the Markov
chain , are described by the generator 
where the matrix  is a sub-generator and the column
vector  is defined by  and has all
non-negative and at least one positive components,  is
the column vector of dimension  consisting of all 1's.
 The average service time  is given by 
For more details about the  type distribution, its properties,
special cases and applications see \cite{n81,pp}. 

Web pages can be delivered into the system by  available robots.
The number of active robots varies in the set  We
assume that the process of Web pages delivering under  active robots is described as follows. Let
 be an irreducible continuous time Markov chain
having finite state space . Sojourn time of the
chain , in the state  has exponential
distribution with a parameter . After this time
expires, with probability  the chain jumps into
the state  without generation of Web pages and with
probability  the chain jumps into the state
 and a batch consisting of  Web pages  is generated, .  The introduced probabilities satisfy conditions:



The parameters defining this flow are kept in the square matrices
 of size  defined by their entries:
 Denote


The matrix  is the infinitesimal generator of
the process  under the fixed number  of
active robots. The stationary distribution vector  of this process satisfies the equations
   
 Here and in the sequel,  is the zero row vector.
 The average intensity  (fundamental rate) of the  under the fixed number  of
active robots is defined by

 and the intensity  of group arrivals is defined
 by 
 The variance  of  intervals between group arrivals is calculated as
 while the correlation coefficient
 of intervals between successive group arrivals is
given by



The introduced representation of the arrival process via the matrices
 unifies several
possible interpretations of the process of Web pages delivered by
the fixed number of active robots:
\begin{itemize}
\item[1.] The processes of Web pages delivering by all robots are
independent  processes. Let the process of Web pages
delivering by the th robot be the  which is governed by the
continuous time Markov chain   having
finite state space  and defined by the matrices
 of size   See \cite{l} for
more details about the , its properties and special cases. We
denote 

Let us assume that the robots are arranged in such a way that the
first robot is always active, then, when a queue decreases, the
second robot can be activated, etc, the th robot is the most rare
activated.

The matrices  
of size  defined by
formulae (1) are expressed via the matrices 
of size  describing the s in the
following way:

 Here  and  denote Kronecker
product and sum of matrices correspondingly (see, e.g.,
\cite{grah}),  denotes identity matrix of size  If the size
of the matrix is clear from context the suffix can be omitted.


\item[2.] The common process of Web pages delivered by all robots 
together is the  process
directed by the   continuous time Markov chain 
having finite state space  and defined by the
matrices  of size  Some set of thinning
probabilities   is
fixed. When  robots are active, procedure of thinning the 
process with the thinning probability  is applied. It means
that an arbitrary arriving batch is accepted with probability 
and is rejected with the complimentary probability  We
denote 

The matrices  
defined by formulae (1) are expressed via the matrices   describing the common  and via the thinning probability
 in the following way:


\item[3.] Let  the process of Web pages delivering by robots is described by
a  The  is directed   by continuous time Markov chain
 having finite state space .
Sojourn time of the chain , in the state 
has exponential distribution with a parameter . After
this time expires, with probability  the chain jumps
into the state  without generation of Web pages and with
probability  the chain jumps into the state
 and a batch consisting of  Web pages  are
delivered by the th robot.  The introduced probabilities satisfy
conditions:


The matrices  
defined by formulae (1) are expressed via the matrices
 formed by the
probabilities  and  in the
following way:

\item[4.] The  process of Web pages delivery by all robots is the  process
directed by the   continuous time Markov chain 
having finite state space  and defined by the
transition intensity matrices  depending on the number of active robots.
\end{itemize}
Interpretation 3 seems be the most attractive because it assumes
that the work of the robots can be dependent, which is quite
realistic. Because the total amount of Web servers from which new
pages should be brought is more or less constant, reduction of the
number of active robots causes the increase of field in Internet,
which is scrawled by each robot, and corresponding change of travel
time. So, the  looks to be the most realistic model of Web pages
delivery.

If a batch of delivered Web pages meets free server one
Web page starts the service immediately while the rest moves to the
buffer. If the server is busy at an arrival epoch, all Web pages of
the batch are placed into the buffer if there is enough free space
in the buffer. If the number of free places in the buffer is less
than the number of Web pages in the batch, the corresponding number
of Web pages is lost. It means that we consider so called partial
admission strategy. The alternative strategies of complete rejection
or complete admission can be investigated in analogous way.

For each Web page placed into the buffer, the waiting time is
 restricted by the random
variable (so called obsolescence time) having  distribution with
irreducible representation  It
means the following. Available waiting time of the th Web page in
the buffer is defined as a time   until the continuous-time Markov
chain  having the states  as the
transient and state  as absorbing one, reaches the absorbing
state. Transition of this process into the absorbing state means
that this Web page gets out of date (obsolescence or dashout
occurs). An initial state of the chain is selected in a random way,
according to the probability distribution defined by the row-vector
, where  is the
stochastic row vector of dimension . Transitions of the Markov
chain , are described by the generator 
 where the matrix  is sub-generator and the column
vector  is defined by 
 The average  time until obsolescence  is given by 

If the obsolescence time expires before a Web page is picked-up from
the buffer to the server, it is assumed  that this Web page
immediately leaves the buffer and is lost. The obsolescence times of
different Web pages are independent of each other and identically
distributed. It is worth to note that the analysis presented below
could be drastically simplified if we suggest that the obsolescence
time is exponentially distributed. However, this suggestion rarely
holds true in the real world systems because this suggestion means
that, with high probability, information obsoletes very quickly.

Reasonable class of strategies of control by robots is the class of
the threshold strategies defined as follows. Integers
 are fixed such as  If the number  of Web pages in the system
satisfies inequality  then  robots
are active and  robots are de-activated, 

Note that the described threshold strategies are popular in
literature in controlled queues, see, e.g., \cite{aod,c,d,dk,t}.
For some systems, it is proven that the optimal strategy in the
class of all Markovian strategies belongs to the class of threshold
strategies. For some other systems such a result is not proven, but
the optimal strategy is sought in the class of threshold strategies.
Advantage of such strategies is their intuitive justification and
relative simplicity of implementation in real-life systems.
Numerical examples presented in the paper \cite{tlnc} for a partial
case of our model confirm that the threshold strategies are optimal
in the class of all Markovian strategies, although authors cannot
prove this fact. Our system is much more complicated and we also 
cannot prove optimality of the optimal threshold strategy in wider
classes of strategies. We just try to find an optimal threshold
strategy and believe that it is optimal or sub-optimal in wider
classes as well.

We also mention that the description of the given above threshold strategy
suits only for the case when  While the numerical examples
presented in \cite{tlnc} address, e.g., the case  and .
However, if we look at the optimal strategy given by Figure 2 in
\cite{tlnc}, we see that the optimal number of the active robots
varies between 4 and 14 with 4 switching points where two or three
robots are activated or de-activated.  Because, in contrast to
\cite{tlnc}, we do not assume that each active robot generates a
stationary Poisson process of arrivals at a fixed rate but we assume
that each robot generates a batch Markovian arrival process,
we see that our strategy suits for the case  as well. We could achieve this formally by allowing non-strict inequalities:
  in fixing the
 thresholds.
 So, speaking below about a robot we may think about a virtual robot
 as a group of several
 available real robots which are activated and de-activated
 simultaneously.

We will solve the problem of choosing the optimal threshold
strategy. The cost function is assumed to be of the following form:

where  is the mean number of pages delivered into the
system by robots during a unit of time (fundamental rate of the
arrival process),  is probability of arbitrary page loss
due to the buffer overflow,  is probability of arbitrary
page obsolescence during waiting in a queue,  is
probability of starvation of the system,   is the
response time (average sojourn time of Web pages which are not lost
or deleted due to obsolescence),  is the average number 
of active robots,  are the
corresponding non-negative cost coefficients. The values of the cost
coefficients can be set up by experts in the domain. Alternatively,
the cost coefficients can be viewed as Lagrange multipliers in the
constrained problem and can be found from the dual problem 
formulation.

It is clear that if the average number of active robots is
increasing then the three first summands in (2) are increasing while
the last summand, charge for starvation of the system, is
decreasing. The last charge is important because starvation means
that the indexing machine is idle and so freshness of the
data base suffers.

The problem of minimization of the cost criterion (2) is not
trivial. To solve this problem, we will use so called direct
approach. To this end, we will calculate the stationary distribution
of the system state under an arbitrary fixed set of  thresholds
 It will allow us to calculate the main
performance measures of the system and the value  of the cost
criterion as a function  Problem of finding
the optimal set  is then easy solved on
computer, e.g., by enumeration.


\section{Stationary distribution of the number of Web pages in the system}

Let some set of  thresholds  be fixed. We are
interested in the stationary distribution of the process  where  is the number of Web pages in the system
at the epoch  This process is
non-Markovian. To investigate this process, we will consider the
following multi-dimensional continuous time Markov chain

where  is the state of the directing process of arrival
process at epoch    is the state
of the process which directs a service at epoch 
  is the state of the process, which
directs a obsolescence of the th Web page in a queue at epoch
 

We assume that  the Web pages in the buffer are numerated in order
of their arrival into the system. If a batch of Web pages arrives
  to the system, the  accepted Web pages are numerated in a uniform
random manner. When a Web page is picked up to the service or is
deleted from the queue because its admissible waiting time expires,
the rest of Web pages is immediately  enumerated correspondingly.

Denote


 Because
the state space of the the Markov chain  is finite
and due to assumption about irreducibility of the processes defining
arrival, service and obsolescence processes, limits (3) exist.

Enumerate the states of the Markov chain  in the
lexicographic order and form the probability row vectors  of probabilities corresponding to the state 
of the first component of the process . Denote also
.

Let  be the infinitesimal generator of the Markov chain  and  be the block of the generator 
consisting of enumerated in lexicographic order intensities of
transition of the Markov chain  from the states
with the value  of the component  to the states with the
value  of this component,  Dimension of the block
 is defined by  where


 {\it Lemma.} ~Non-zero blocks  of the infinitesimal generator  of the Markov chain  are defined by





 Here



and the value  which corresponds to the number of active
robots when  Web pages stay in the system, is calculated by
 where the value  is defined by relations


 Proof of the lemma is implemented by analyzing the probabilities of
transition of the multi-dimensional Markov chain 
under the fixed set of  thresholds  during an
interval of infinitesimal length. It is rather transparent and is
omitted.

It is well-known that the vector  of stationary
probabilities satisfies the following equilibrium equations


Dimension of the vector  is equal to  It can be rather high. For instance, if
 this dimension is equal to  For 
this equals to 2046. So, direct solution of equations (4) by "brute
force" can be very time and computer memory demanding.

 Effective and
numerically stable algorithm for computing the blocks  of the vector , which exploits a
structure of generator  and is presented in \cite{kkod}, consists
of the following steps:
\begin{itemize}
\item[1.] Compute the matrices  from the
backward recursion

 
\item[2.] Compute the matrices  from the
backward recursions


\item[3.]
Compute the matrices  by recursion

\item[4.] Compute the vector  as the unique solution to the following system
of linear algebraic equations:

\item[5.] Compute the vectors  by
formula

\end{itemize}
Thus, the problem of computing the stationary distribution  of the considered queueing system under
the arbitrary fixed set of the   thresholds 
can be considered solved.


\section{Main Performance Measures of the System}

As the main performance measures of the system we consider the
values 
which appear in cost criterion (2).

Calculation of the values  is the easiest.

{\it Theorem 1.} Probability  of the system starvation
(idle state of the indexing machine) is calculated by

Average number  of active robots is calculated by

where probability  that  robots are active at an
arbitrary epoch is computed by

{\it Theorem 2.} Probability  of an arbitrary Web page
loss due to the buffer overflow is calculated by

where the average intensity  of the input flow is computed
by

Proof. It follows from the formula of total probability that

where  is probability that there are  Web pages in the
system at an epoch of arrival of a batch consisting of 
Web pages,  is probability that arbitrary Web page arrives in
the batch consisting of  Web pages,  is probability
that the arbitrary Web page will be accepted into the system
conditional that it arrives in the batch consisting of  Web pages
and  Web pages present in the system at the epoch of arrival.

It can be shown that the listed probabilities are calculated by




By substituting expressions (11) - (13) into formula (10) we get
formulae (8), (9). The theorem is proven.

{\it Theorem 3.} Probability  of an arbitrary Web page
obsolescence is calculated by

Probability  of an arbitrary Web page successful
service in the system is calculated by

The statement of the theorem is clear because the right hand side of
(14) represents the ratio of the obsolescence rate and arrival rate
into the system. The right hand side of (15) represents the ratio of
the  rate of  successfully  served in the system Web pages and
arrival rate into the system.

\section{Sojourn Time Distribution}

Let   and  be distribution functions of
sojourn time of an arbitrary Web page in the system under study, an
arbitrary Web page, which will get successful service, and arbitrary
Web page, which will be deleted from the system due to its
obsolescence, and  and  be the
corresponding Laplace-Stieltjes transforms:



{\it Theorem 4.}  Laplace-Stieltjes transforms 
and  are calculated by

 

where

the column vectors  are computed by


 the vectors  are computed
recursively by



where

 is Kronecker delta: 

Proof. To prove the theorem, we use the method of collective marks
(method of catastrophes). We interpret the parameter   as an
intensity of some imaginary stationary Poisson flow of catastrophes.
So,  has the meaning of probability that no catastrophe
arrives during the sojourn time of an arbitrary Web page,
  is probability that no catastrophe arrives during
the sojourn time of an arbitrary Web page conditional  that this
Web page will get service successfully, and  is
probability that no catastrophe arrives during the sojourn time
of an arbitrary Web page conditional  that this Web page will be
deleted from the system due to its obsolescence.

So, formula (16) evidently stems from the formula of total
probability.

It is obvious that sojourn time of the arbitrary (tagged) Web page
does not depend on the arrival process after the tagged Web page
arrival epoch. Thus,  in analysis we can ignore transitions of the
directing process of the arrival process after the epoch of the
tagged Web page arrival.

Formulae (17) and (18) also follow from the formula of total
probability. Here row vector  defines probability
of an arbitrary Web page arrival at the moment when there are 
Web pages in the system in a batch of size  and probability
distribution of the directing processes of service and obsolescence
at this moment. Column vector  defines
probability of no catastrophe arrival during the conditional
sojourn time of a tagged Web page who arrives at the moment when
there are  Web pages in the system in a batch of size  under
the fixed value of the directing processes of service and
obsolescence at the arrival epoch. For  condition is that the
tagged Web page will get service successfully. For  condition
is that the tagged Web page will be deleted from the system due to
its obsolescence.

Relation (19) is based again on the formula of total probability.
The matrix  defines probability distribution of
installation, upon the arrival epoch of a tagged Web page,  of the
initial states of the directing process of service (if , i.e.,
the system was empty at the arrival epoch) and of the directing
processes of obsolescence of the Web pages, which arrive at the same
batch as the tagged Web page and are placed in a buffer before the
tagged Web page, and  of this Web page. Recall that we assume that
 if the batch consists of  Web pages then the
tagged Web page will be the th in the batch, 
with probability 

The vector  defines probability of no
catastrophe arrival during the conditional sojourn time of a tagged
Web page who sees  Web pages in the system before him in a queue
 and the corresponding states of the
directing processes of service and obsolescence after the moment of
arrival.

Recurrent formulae (20) for the vector Laplace-Stieltjes transforms
 is clear if we take into account that: (i)  is the vector Laplace-Stieltjes transform of
the service time distribution, (ii) 
defines probability that no catastrophe arrives during the time
interval between the epoch of a tagged Web page arrival, at which
the number of Web pages in a queue before the tagged Web page is
equal to , and the epoch when the number of Web pages in a queue
before the tagged Web page is decreased to , (iii) the matrix
 defines transition of the directing processes
of service and obsolescence of the Web pages at the epoch of
decreasing.

Formulae (21) for the vector Laplace-Stieltjes transforms  takes into account reasonings (ii),(iii) presented
above as well as consideration that no catastrophe should arrive
until the obsolescence moment of a tagged Web page and that  Web pages can depart from the system until the
obsolescence moment. The theorem is proven.

{\it Corollary 1.}  Average sojourn time (response time)  of
an arbitrary Web page, average sojourn time   of an
arbitrary Web page who will get successful service, and average
sojourn time   of an arbitrary Web page who will be
deleted from the system due to its obsolescence are computed by



where the column vectors  are computed by

 the vectors  are
computed  by




  

Proof of corollary evidently follows from the well-known expression
for the mean value of a random variable via the derivative of the
Laplace-Stieltjes transform of its distribution function.

Note that expressions for higher order moments and variance on
sojourn time distribution can be also easily derived based on
equations (16)-(18).

\section{Case of the Ordinary Arrival Process}
Consider the special case when Web pages arrive not in batches, but
one-by-one. It means that  In this case  the generator  of the Markov
chain  is the three block diagonal matrix which in turn
means that this chain is a finite space
Quasi-Birth-and-Death-Process. Thus, in this case the algorithm for
solving equilibrium equations (4) for the vector  of
stationary probabilities and formulae for some performance measures
simplify. The algorithm has the form
\begin{itemize}
\item[1.] Compute the matrices  from the
backward recursion

with the terminal condition

\item[2.] Compute the matrices  by recursion

\item[3.] Compute the vector  as the unique solution to the following system
of linear algebraic equations:

\item[4.] Compute the vectors  by
formula

\end{itemize}
Formula for loss probability  is given by

where

Laplace-Stieltjes transforms  and  are
computed by


Average sojourn times    and  are
computed by


\section{Numerical example}

To demonstrate feasibility of the developed algorithms for
calculating the stationary state distribution of the system under
the fixed parameters of the control strategy and calculating the
optimal set of these parameters, let us consider numerical examples.
First we suppose that the system can have any number of active
robots between one and four at any time moment () and the
buffer capacity is equal to  ().

We assume that the arrival process is formed according to Model~4
from Section~2. Namely, when  robots are activated the
-input is described by the matrices ,
, , given by



The intensities  of the  when  robots are
active,  are computed by

The  coefficients of correlation  and the intensities
of  batches arrival  are the following:
, ,
, ,
, ,
, .

Let the  distribution of service time be defined by the row
vector  and sub-generator
  The mean service
time is equal to 0.657.

Let the  distribution of a Web page obsolescence time be
described by the row vector  and  sub-generator
  The mean time
until obsolescence is equal to 5.

Let the cost  coefficients be fixed by , ,
, ,  We have chosen the cost coefficients
in this way to obtain commensurable optimal values in the optimal solution
and non-trivial optimal policy.

Note that in this example we fixed the cost coefficients based on
some heuristic reasonings or common sense. In general, the right
choice of the cost coefficients is the important and difficult task.
It requires good knowledge of the real world system, which is
described by the mathematical model under study, and clear
understanding what is the most undesirable for the concrete system
(loss of the delivered Web pages, obsolescence of a page, starvation
of the system, long waiting in the queue, keeping to many robots be
active, etc), what is less important. So, the help of experts is
required for the right choice of the cost coefficients. If such a
choice does not seem be possible, some alternative formulation of
optimization problem, e.g., multi-criteria problem or problem with
constraints may be considered. In the latter approach the cost
coefficients are Lagrange multipliers in the constrained problem
and can be found from the dual problem formulation. 

Let us find the optimal strategy of control by the system under the
fixed above values of the system parameters and the cost
coefficients. The thresholds  in the problem
formulation are fixed such as 
i.e., the thresholds cannot coincide and the use of  all  modes
of operation (the number of the mode is characterized by the number
of the active robots) is mandatory. It is intuitively clear that
actually it can happen that the optimal strategy does not need to
use some modes of operation at all. So, to find the optimal
strategy, we have to compare the values of the optimal values of the
cost criterion when all  modes are used, when  modes are
used while 1 mode is ignored, , two modes are used while
 modes are ignored, when only one mode is used (i.e., the
number of the active robots is not varied).

Let us denote by  the value of the cost criterion when exactly
 robots are always active,  The values ,
 are given by , ,
, . So, if there is no possibility to
control the number of robots, and one has to decide how many robots
should permanently work, the best choice is  to have permanently
three active robots.

Next we consider the threshold type strategies for controlling the
number of active robots. Table 1 contains the optimal value of the
cost criterion for various combinations of the used modes and the
optimal threshold strategy for each such a combination.
\begin{center}
Table 1: The Value of the Optimal Cost Criterion for various
threshold strategies
\begin{tabular}{|c|c|c|}
\hline possible numbers of active robots & \begin{tabular}{c}
  Optimal \\
  thresholds \\
\end{tabular} & \begin{tabular}{c}
  Optimal value \\
of the cost \\
  criterion \\
\end{tabular}
\\\hline
1&--&149.91\\ 2&--&110.0\\3&--&89.40\\4&--&130.31\\
  2 or 1&2& 103.54\\
 \textbf{3 or 1} & \textbf{2} &\textbf{63.54}\\
  4 or 1&1&74.47\\
  3 or 2&2&76.21\\
  4 or 2&1 &86.13 \\
   4 or 3&0 &94.14 \\
     3 or 2 or 1&2,2 &63.54 \\
 4 or 2 or 1&1,2 &73.69 \\
  4 or 3 or 1& 0,2&80.50 \\
4 or 3 or 2&0,2 &80.50\\
 4 or 3 or 2 or 1&0,2,2 &67.52 \\
  \hline
  \end{tabular}
\end{center}

\vspace{4mm}

Let us explain the entries of the table. For instance, the
highlighted line corresponds to the control with one threshold at
two. When the queue length does not exceed two, there are three
active robots and when the queue length exceeds two, the number of
active robots decreases to one. As another example, the control
corresponding to the last line of the table has the following
structure: when the system is empty, four robots are active; when
the queue length is greater than zero and smaller than three, three
robots are active; when the queue length exceeds two, only one robot
is active.

As it is seen from Table 1, the optimal strategy assumes that only
 two among four available operation
modes (modes with one and three active robots) should be used. The
optimal value  of the cost criterion is equal to 63.54. It is
evident that  gives the relative profit  more than 28\%
comparing to the case without control. The value of  is computed
as


The dependence of the cost criterion on the threshold when the
strategy of control  uses only two modes, for all possible
combinations of the modes, is shown on Figure 1.

\begin{figure}[htb]
\centering \includegraphics[scale=0.6]{figure1.eps}\\
  \caption{Dependence of the cost criterion on the threshold}
\end{figure}


The obtained results illustrate the necessity of the input control
and possibility to reduce the cost of the system operation by means
of the threshold type control.

Now let  us vary the buffer size   from 1 to 100. Table 2
contains the optimal criterion value for the cases with and without
control ( and , ), the optimal threshold
 and the relative profit  for different buffer capacity .
Note that in all cases the optimal control only uses the modes with
one and three active robots. The results for  are
approximately the same as for .

\newpage
\begin{center}
Table 2: Variable Buffer Size 
\\ \vspace{2mm}
\begin{tabular}{|c|c|c|c|c|c|c|c|}
  \hline
 &  &  &  &  &  &  & , \% \\
  \hline
  1 & 0 & 147.5& 244.7 & 233.4 & 187.2 & 258.8 & 21.0 \\
  2 & 1 & 96.8 & 199.2 & 174.0 & 128.8 & 194.4 & 24.8 \\
  3 & 2 & 79.1 & 172.6 & 140.3 & 105.4 & 160.0 & 24.9 \\
  4 & 2 & 68.3 & 158.1 & 121.7 & 94.7 & 140.6 & 27.85\\
  5 & 2 & 63.5 & 149.9 & 110.0 & 89.4 & 130.3 & 28.9 \\
  6 & 3 & 60.8 & 144.7 & 102.3 & 86.7 & 124.1 & 29.8 \\
  7 & 3 & 59.3 & 141.6 & 97.2 & 85.5 & 120.5 & 30.6 \\
  8 & 3 & 58.4 & 138.5 & 93.7 & 85.0 & 118.3 & 31.2 \\
  9 & 3 & 57.8 & 137.9 & 91.3 & 84.9 & 117.1 & 31.8\\
10 & 3  & 57.5 & 137.0 & 89.7 & 85.0 & 116.5 & 32.3 \\
20 & 3  & 57.2 & 137.0 & 86.3 & 87.2 & 120.0 & 33.7\\
30 & 3  & 57.2 & 137.0 & 86.3 & 87.4 & 123.6 & 33.7\\
  \hline
\end{tabular}\end{center}

\vspace{4mm}


Let the mean service time  be varied by means of the matrix 
multiplication by the value  which varies from 0.1 to 15. Table 3
contains the value , the mean service time , the optimal set
of possible numbers of active robots, the optimal value of the
threshold, the optimal cost criterion value and the value of the
cost criterion when only one of the modes is in use (when the number
of active robots is fixed), and the relative profit . Note that
when  is equal to or grater than 15, i.e.  is less than
0.04, no dynamic control is required.

Now we vary the mean time  until obsolescence by the same way
as it was done for the mean service time. Table 4 shows obtained
results. Note that in the optimal set of modes only one and three
active robots are used in all considered cases.

Consider the effect of the service time variation on the cost
criterion value given that the mean service time  is constant
and equals to 0.657. Let the matrix  of service time distribution
have the form

and vector .

The variance of service time is calculated as

To maintain the mean service time , the entries  and
 of matrix  must be related through the formula
 as

Note that  should be grater than 1.369 to keep 
positive.

Let us vary the value of  from 1.521 to 9. The service
time variation  takes the values from 0.431 to 5.798. In the
case  service time distribution is exponential one.
In the optimal set of operation modes only one and three active
robots are used. Figure 2 shows dependence of the cost criterion
value on the threshold under different values of service time
variation ().

\begin{figure}[htb]
\centering \includegraphics[scale=0.6]{service_dispersion2.eps}\\
  \caption{Dependence of the cost criterion on service time variation}
\end{figure}

But when  becomes larger than 4 (), not the
modes with one and three active robots  but the modes with one and
four active robots are the optimal set of exploited modes. Figure 3
shows the dependence of the cost criterion when only optimal set of
modes is in use. In the figure, two lower curves correspond to the
modes with one and three active robots and other curves correspond
to the modes with one and four active robots.

\begin{figure}[htb]
\centering \includegraphics[scale=0.6]{service_dispersion3.eps}\\
  \caption{Dependence of the cost criterion on service time variation}
\end{figure}

Now let us vary the variance of time until obsolescence in the same
way. Let the matrix  have the form

and  To maintain the average time until
obsolescence , the value  is related to 
as

We vary  from 0.2 to 7. Thus the variation takes the
values from 25 to 449. Note that in the case , we get
the time until obsolescence distributed exponentially. The optimal
strategy consists of modes with one and three active robots in all
considered cases. Figure 4 shows dependence of the cost criterion
value on variation of time until obsolescence (). Note that as the variation grows the optimal
threshold decreases from 2 to 0.

\begin{figure}[htb]
\centering \includegraphics[scale=0.6]{waiting_dispersion.eps}\\
  \caption{Dependence of the cost criterion on variation of time
until obsolescence}\label{fig:3.3}
\end{figure}

Now let us consider the example based on real data obtained from
the robot designed by INRIA Maestro team in the framework of RIAM
INRIA-Canon Research Project. Data about the information
delivery process by the robot to the data base was presented
in the form of the text file which contained more than 65 000
timestamps defining epochs of information delivery.

This data was
processed in the following way.
\begin{itemize}
\item[]
Inter-arrival times were computed.
\item[]
The obtained sample was censored: very long intervals, which
actually correspond to the periods when the crawling process was
stopped due to some reason, are deleted from the sample.
\item[]
The obtained sample was transformed into two samples in such a way
as the very short inter-arrival times were deleted from the initial
sample and the corresponding information about the number of
successively deleted intervals was placed into another sample. As
the result of these manipulations, we stated that the arrival
process is the batch arrival process. One sample defines the
intervals between the epochs of batches arrival, the second sample
defines the number of the information units in the corresponding
batch. By information units we mean either the principal part of
a Web page (e.g., Web page main html file) or its embedded resources
(e.g., image and audio files).
\item[]
Based on the second sample, distribution of the number of the
information units in a batch was computed as follows. The size of a
batch varies in the interval  and probabilities  that
the batch size is equal to  are the following:
, , ,
, , , , . The mean batch size
is equal to
 3.263389905223716.
 \item[]
 Based on the first sample, we computed estimation of the mean value of an interval between
 arrivals of batches as 212.03,
estimation of the variance of such an interval as 51352.38 and the
estimations of lag- correlation of inter-arrival times for 
equal to  are given by  0.622;  0.574; 0.555; 0.537;
0.523; 0.507. Thus, the flow defined by a sample under study has
slowly decreasing correlation. In this situation, it is reasonable
to apply method by Diamond and Alfa \cite{da} oriented to such flows.
 \item[]
 As the result, the process of the arrival of the batches was
 defined by the  which is characterized by the matrices
 
\item[]
Based on this  of batches and information about the batch size
distribution, the  of Web pages is constructed. It is defined
by the matrices  and 
\end{itemize}

To estimate a Web page service time distribution, the following
information  about the size of an arbitrary information unit
delivered by a crawler (content size) in the used data base was
exploited: the mean content size is equal to 49207.0356 bytes, the
mean squared content size is equal to 2.0527E+11, and the mean cubed
content size is equal to   2.5773E+18. Based on this information,
the service time distribution of a Web page was described, up to
some normalizing constant defined by the content processing rate
(in our application the processing rate is constant), 
by the hyper-exponential distribution which is the partial case of the
 distribution defined by the vector  and by sub-generator
 The mean service
time is~8.2. The squared coefficient of variation of the service
time distribution is equal to 86.03. Because the exponential
distribution has the squared coefficient of variation equal to 1, is
is clear that this distribution can not be considered as a good
approximation of the service time distribution.

We assume that the system has four available operation modes. The
buffer capacity is .

When the  robots are activated, the -input is described by
the matrices ,

where the matrices  are defined above.
The intensities  of the  when  robots are
used,  are as follows: ,
, ,
.

The distribution of a Web page obsolescence time is exponentially
distributed with parameter 0.0005, i.e., the mean time until
obsolescence is equal to 2000.

The cost criterion coefficients are taken as ,
, , , 

The cost criterion values   when  
robots are always activated are given by , ,
 and . When all four modes of operations are
exploited the optimal cost criterion value is 563.51 and the optimal
set of thresholds is [2,2,2], i.e., the optimal strategy assumes
that four robots should be active until the number of Web pages in
the system does not exceed 2. When the number of Web pages in the
system  exceeds 2, three robots should be deactivated and only one
robot should be active.

Table  contains the optimal values of the cost criterion for the
different combinations of operation modes.

\begin{center}
Table : The Values of the Optimal Cost Criterion for the Fixed
Combination of Operation Modes
\begin{tabular}{|c|c|c|}
\hline possible numbers of active robots & \begin{tabular}{c}
  Optimal \\
  thresholds \\
\end{tabular} & \begin{tabular}{c}
  Optimal value \\
of the cost \\
  criterion \\
\end{tabular}
\\\hline
1&--&666.28\\ 2&--&657.07\\3&--&639.03\\4&--&621.25\\
  1,2&0& 624.97\\
 {1,3} & {0} &{591.72}\\
  \textbf{1,4}&\textbf{2}&\textbf{563.51}\\
  2,3&1&622.81\\
  2,4&2 &593.29 \\
   3,4&3 &609.66 \\
     1,2,3&0,0 &591.72 \\
 1,2,4&2,2 &563.51 \\  1,3,4& 2,2&563.51 \\
2,3,4&2,2 &593.29\\
 1,2,3,4&2,2,2 &563.51 \\
  \hline
  \end{tabular}
\end{center}
The relative profit of operation mode control exceeds 9\% comparing
to the case when no control is applied and all four robots are
always active.


\section{Conclusion}

In this paper we provide performance evaluation and optimization
of the crawling part of a Web search engine. We model the crawler 
with a finite buffer, monotonically controlled arrival rate
(controlled number of crawling robots), and with stochastically 
bounded waiting time. The system is considered under rather general 
assumptions about the arrival process, service and obsolescence 
time distributions. Stationary distribution of the system state, 
sojourn time distribution and main performance measures of the system 
are calculated under any fixed set of thresholds defining the control 
strategy. This allows us to reduce the problem of optimal control 
to minimization of a known function of several integer variables.

Numerical results are presented. They show that the dynamic input
control can give essential profit. Effects of buffer size changes
and changes of average service and obsolescence times and their
variances are investigated. In particular, we illustrate that the
assumption about the exponential distribution of service and
obsolescence times can give poor estimation of the system performance 
measures and optimal values of the thresholds when actually these times 
have high variation.

The model has been applied to the performance evaluation and
optimization of the crawler designed by INRIA Maestro team in the framework
of the RIAM INRIA-Canon research project.


\section*{Acknowledgment}
The authors would like to thank Ministry of Science of France for
financial support of this research via ECO-NET programme and RIAM
INRIA-Canon Research Project.

\begin{thebibliography}{10}

\bibitem{aod}

J. Artalejo, D.S. Orlovsky, and A.N. Dudin, "Multiserver retrial model
with variable number of active servers", \emph{Computers and
Industrial Engineering},  vol. 28, pp. 273-288, 2005.

\bibitem{ANO96}

S. Asmussen, O. Nerman and M. Olsson, "Fitting Phase-Type Distributions 
via the EM Algorithm", {\it Scandinavian Journal of Statistics},
vol. 23, pp. 419-441, 1996.  

\bibitem{c}

T.~Crabill, "Optimal control of a service facility with variable
expenential service times and constant arrival rate",
\emph{Management Science}, vol. 9, pp. 560-566, 1972.

\bibitem{da}

J.E.~Diamond and A.S.~Alfa, "On approximating higher order s
with s of order two", \emph{Queueing Systems},  vol. 34, pp.
269-288, 2000.

\bibitem{d}

A.~Dudin, "Optimal multi-threshold control for a  queue
with  service modes", \emph{Queueing Systems},  vol. 30, pp.
273-287, 1998.

\bibitem{dk}

 A.N.~Dudin and V.I.~Klimenok,  "Optimal admission control in a queueing system
with heterogeneous traffic", \emph{Operations Research Letters},
 vol. 28, pp. 108-118, 2003.

\bibitem{grah}

A.~Graham, \emph{Kronecker Products and Matrix Calculus with
Applications}, \hskip 1em plus
  0.5em minus 0.4em\relax Chichester: Ellis Horwood, 1981.

\bibitem{kkbd}

 C.S. Kim,  V.I. Klimenok, A.A. Birukov and A.N. Dudin,  "Optimal
   multi-threshold control by the BMAP|SM|1 retrial system", \emph{Annals of
   Operations Research}, vol. 141,  pp. 193--210, 2006.

\bibitem{kkod}
 V.I.~Klimenok,  C.S.~Kim,  D.S.~Orlovsky and A.N.~Dudin,
"Lack of invariant property of  Erlang   model",
\emph{Queueing Systems},  vol. 49, pp. 187-213, 2005.

\bibitem{l}
 D.~M. Lucantoni, "New results on the single server queue with a
batch Markovian arrival process," \emph{ Communications in
Statistics-Stochastic Models},  vol.  7, pp. 1-46, 1991.

\bibitem{n81}
M.~Neuts, \emph{Matrix-geometric solutions in stochastic models},
\hskip 1em plus
  0.5em minus 0.4em\relax Baltimore, USA: The Johns Hopkins University Press, 1981.

  \bibitem{pp}
A.  Pattavina and A. Parini A., "Modelling voice call inter-arrival
and holding time distributions in mobile networks", in: Performance
Challenges for Efficient Next Generation Networks - Proc.of 19th
International Teletraffic Congress, Aug.-Sept. 2005, pp. 729-738.

  \bibitem{tlnc}
J.~Talim, Z.~Liu, Ph.~Nain and E.~G.~Coffman, Jr., "Controlling the
Robots of Web Search Engines", \emph{Performance Evaluation Review},
 vol. 29. No 1, pp. 236-244, 2001.

 \bibitem{t}
 H.~Tijms, "On the optimality of a swith-over policy for controlling
 the queue size in a  queue with variable service rates",
 \emph{Lecture Notes in Computer Sciences}, vol. 40. pp. 236-242,
 1976.
 
 \bibitem{W87}
 C.E. Wells, "Determining the Future Costs of Lifetime Warranties",
  {\it IIE Transactions}, vol. 19, pp. 178-181, 1987.
 
\end{thebibliography}

\onecolumn

\begin{center}
Table 3: Variable Mean Service Time
\\ \vspace{2mm}
\begin{tabular}{|c|c|c|c|c|c|c|c|c|c|}
  \hline
&   &     & &  &    &    & & & , \%\\
\hline
0.1&    6.57&   1,3&  0&  53.1&   58.58&  80.67&  104.74& 131.84& 9.35\\
0.2&    3.29&   1,3&  1&  45.08&  59.45&  72.77&  94.16&  122.04& 24.17\\
0.3&    2.19&   1,3&  1&  42.93&  68.51&  72.02&  89.42&  118.55& 37.34\\
0.4&    1.64&   1,3&  1&  43.33&  80.36&  74.27&  86.7&   117.45& 41.66\\
0.5&    1.31&   1,3&  1&  45.28&  93.09&  78.34&  85.15&  117.77& 42.2\\
0.7&    0.94&   1,3&  2&  51.17&  117.98& 89.66&  84.65&  121.17& 39.55\\
0.9&    0.73&   1,3&  2&  58.88&  140.09& 103.05& 87.14&  126.89& 32.43\\
1&  0.66&   1,3&  2&  63.54&  149.91& 110&    89.4&   130.31& 28.93\\
3&  0.22&   1,3&  3&  160.48& 244.04& 211.44& 182.29& 204.54& 11.96\\
5&  0.13&   1,4&  1&  217.02& 272.19& 254.87& 243.01& 251.25& 10.7\\
7&  0.09&   1,4&  1&  250.83& 285.26& 276.93& 274.4&  279.31& 8.59\\
9&  0.07&   1,4&  0&  272.76& 292.74& 290.05& 292.8&  297.61& 5.96\\
11& 0.06&   1,4&  0&  288.23& 297.59& 298.7&  304.77& 310.39& 3.15\\
13& 0.05&   1,4&  0&  299.92& 300.97& 304.82& 313.15& 319.78& 0.35\\
15& 0.04&   1&  --& 309.061&    303.47& 309.37& 319.33& 326.96& 0\\
\hline
\end{tabular}
\end{center}

\newpage
\begin{center}
Table 4: Variable Mean Time until Obsolescence
\\ \vspace{2mm}
\begin{tabular}{|c|c|c|c|c|c|c|c|c|}
  \hline
&   &    &  &    &    & & & , \%\\
\hline 0.01 & 500 & 3 & 52.07 & 130.02 & 92.47 & 81.28 & 118.93 &
37.16 \\
0.1&    50& 3&  52.41&  132.22& 94.24&  81.99&  119.99& 36.08\\
0.2&    25& 2&  53.83&  134.55& 96.17&  82.79&  121.17& 34.98\\
0.3&    16.7&   2&  55.1&   136.78& 98.05&  83.6&   122.34& 34.09\\
0.4&    12.5&   2&  56.36&  138.9&  99.88&  84.41&  123.51& 33.23\\
0.5&    10& 2&  57.6&   140.93& 101.67& 85.24&  124.67& 32.43\\
0.7&    7.1&    2&  60.02&  144.74& 105.12& 86.9&   126.95& 30.93\\
0.9&    5.6&    2&  62.38&  148.25& 108.42& 88.57&  129.21& 29.57\\
1&  5&  2&  63.54&  149.91& 110&    89.4&   130.31& 28.93\\
3&  1.67&   1&  83.6&   173.68& 135.36& 105&    149.91& 20.38\\
5&  1&  1&  95.64&  187.76& 152.48& 117.52& 165.21& 18.62\\
10& 0.5&    1&  116.47& 206.98& 178.08& 138.25& 191.33& 15.75\\
20& 0.25&   0&  131.1&  223.17& 201.46& 158.66& 218.9&  17.37\\
30& 0.17&   0&  136.67& 230.45& 212.45& 168.67& 233.25& 18.97\\
40& 0.13&   0&  139.97& 234.54& 218.83& 174.63& 242.04& 19.85\\
50& 0.1&    0&  142.15& 237.18& 222.99& 178.58& 247.97& 20.4\\
60& 0.083&  0&  143.72& 239.03& 225.91& 181.39& 252.25& 20.77\\
70& 0.071&  0&  144.9&  240.38& 228.08& 183.5&  255.47& 21.04\\
80& 0.063&  0&  145.81& 241.42& 229.75& 185.13& 257.99& 21.24\\
90& 0.056&  0&  146.55& 242.24& 231.08& 186.44& 260.01& 21.4\\
100&    0.05&   0&  147.16& 242.91& 232.15& 187.51& 261.67& 21.52\\
200&    0.025&  0&  150.08& 245.97& 237.19& 192.58& 269.62& 22.07\\
\hline
\end{tabular}
\end{center}]







\end{document}
