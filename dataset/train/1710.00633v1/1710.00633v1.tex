\documentclass[letterpaper]{article}
\usepackage{amsmath,graphicx,mlspconf}
\usepackage{multirow}
\usepackage{color}
\usepackage{amssymb}
\usepackage{url}
\usepackage{setspace}








\copyrightnotice{978-1-5090-6341-3/17/\^{1}^{1,2}^{1}^1^2<586\%3 \times 32 \times 2\omega\sigmafW=\omega f/2L = \lfloor2W \rfloor- 1\lfloor x \rfloorxx = \log(x) + 1s[0,1]16\mathrm{c}3 \times 31\mathrm{m}2 \times 22\mathrm{fcr}\mathrm{fcs}\mathcal{D}=\{\mathbf{x}_n, t_n\}_{n=1}^NP\mathbf{x}t \in \{1,\dots, C\}\hat{t} = f(\mathbf{x})j\L| x |x\hat{s}^{(j)}30\omega=3.0f=2W=3L=5\sigma=0.67224 \times 224s=150[0, 1]41510^{-5}0.90.9993589\%44\%89-97\%85\%75-89\%44\%12\%6\%5\%1\%6\%4\%4\%3\%3\%_1030[-60,0][30,90]<$1.5 Hz) across the whole image, which might correspond to the network identifying spindles and K-complexes, respectively, which are characteristic features of the N2 stage. Slow wave activity (0.5-3 Hz) seems to be present in the N3 sensitivity map (Fig.~\ref{fig:class_sm}-d), with highest impact in the current and succeeding epochs. Finally, wide power band spanning from approximately 0.5 to 9 Hz is present in Fig.~\ref{fig:class_sm}-e, probably accounting for the mixed-frequency signal distinctive of R stage. 







\section{Conclusions}
\label{sec:conclusions}
We have demonstrated that classification of sleep stages can be effectively framed as a visual task by first creating natural colour like images using multitaper spectral estimation and then applying recent achievements in the object recognition field to obtain state-of-the-art classification accuracy. Moreover, this approach greatly enhances the interaction with the domain expert by providing interpretable patterns to make sense of as well as a framework based on sensitivity analysis to easily inspect the network's reasoning. We think that the tools presented here can transcend EEG sleep scoring and be applied to other tasks within EEG analysis or, more generally, to other biological domains (e.g., EMG) where time-frequency signals are recorded. Further improvement of the method includes better hyperparameter optimisation when generating the spectral images. A thorough study of the obtained VGGNet layers might also be of interest to gain a deeper understanding of the internal structure of the network.





\begin{spacing}{0.965}
\bibliographystyle{IEEEbib}
\bibliography{references}
\end{spacing}

\end{document}
