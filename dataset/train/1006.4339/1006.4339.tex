\subsection{Submodular prize-collecting clustering}\label{sec:submod-cluster}

First we present and analyze a primal-dual algorithm for 
\spcsf, and later 
we see how this algorithm can be used to achieve the goal of
identifying and removing certain demands from the optimal solution
such that the additional penalty is negligible.

Consider an instance $(G(V,E), \D, \pi)$ of the \spcsf.
A set $S \subseteq V$ is said to \emph{cut} a demand $d=\{s,t\}$
if and only if $|S \cap d| = 1$.  We denote this by the short-hand $d \odot S$,
and say the demand $d$ \emph{crosses} the set $S$.
In the linear program~\eqref{eqn:s-lp:1}--\eqref{eqn:s-lp:3},
there is a variable $y_{S,d}$ for any $S\subseteq V$, $d\in\D$ such
that $d \odot S$.
Conveniently, we use the short-hands $y_S := \sum_{d\in\D} y_{S,d}$
and $y_d := \sum_{S\subseteq V} y_{S,d}$.



\begin{lp}
& \sum_{S: e\in\delta(S)}\: y_S \leq c_e &\hspace{3cm}&&
       \forall e\in E \label{eqn:s-lp:1}\\
& \sum_{d\in D} y_d \leq \pi(D) &&&\forall D\subseteq \D  \label{eqn:s-lp:2}\\
& y_{S,d} \geq 0 &&&\forall d\in\D, S\subseteq V, d\odot S. \label{eqn:s-lp:3}
\end{lp}










We produce a solution to the above LP.
Theorem~\ref{thm:reduce} is proved via some properties of this solution.
 These constraints look like the dual of a natural linear program for 
\spcsf.
For the sake of convenience, we use the notation $y(D) := \sum_{d\in D} y_d$ for any $D \subseteq \D$.




\begin{lemma}\label{lem:submod-growth}
Given an instance $(G, \D, \pi)$ of \spcsf,
we produce in polynomial time a forest $F$ and a subset $\dunsat \subseteq \D$ of demands, 
along with a feasible vector $y$ for the above LP such that
\begin{enumerate}
 \item $y(\dunsat) = \pi(\dunsat)$;
 \item $F$ satisfies any demand in $\dsat := \D \setminus \dunsat$;
and
 \item $\len(F) \leq 2 y(\D)$.
\end{enumerate}
\end{lemma}










 

The solution is built up in two stages.
First we perform an \emph{submodular growth} to find a forest $F_1$ and a corresponding $y$ vector.
This is different from the usual growth phase of \cite{GW95, AKR91} in that
the penalty function may go tight for a set of vertices that are not currently connected.
In the second stage, we \emph{prune} some  edges of $F_1$ to obtain another forest $F_2$.
Below we describe the two phases of Algorithm~\ref{alg:submod-pc-cluster} (\algo{Submodular-PC-Clustering}).



\paragraph{Growth}
 We begin with a zero vector $y$, and an empty set $F_1$.
 A demand $d\in\D$ is said to be \emph{live} if and only if
$x(D) < \pi(D)$ for any $D \subseteq \D$ that $d\in D$.  
If a demand is not live, it is \emph{dead}.
 During the execution of the algorithm \algo{Submodular-PC-Clustering}, 
we maintain a partition $\C$ of vertices $V$ into clusters; 
it initially consists of singleton sets.
 Each cluster is either \emph{active} or \emph{inactive};
  the cluster $C\in\C$ is \emph{active} if and only if there is a live demand $d: d\odot C$.
We simultaneously \emph{grow} all the active clusters by $\eta$.
 In particular, if there are $\kappa(C)>0$ live demands crossing an active cluster $C$, 
we increase $y_{C,d}$ by $\eta/\kappa(C)$ for each live demand $d: d\odot C$.
 Hence, $y_{C}$ is increased by $\eta$ for every active cluster $C$.
 We pick the largest value for $\eta$ that does not violate any of the constraints in \eqref{eqn:s-lp:1} or \eqref{eqn:s-lp:2}.
 Obviously, $\eta$ is finite in each iteration because the 
values of these variables cannot be larger than $\pi(\D)$.
 Hence, at least one such constraint goes tight after each growth step.
 If this happens for an edge constraint for $e=(u,v)$,
 then there are two clusters $C_u\ni u$ and $C_v\ni v$ in $\C$,
 at least one of which is growing.
 We merge the two clusters into $C = C_u \cup C_v$ by adding the 
edge $e$ to $F_1$, remove the old clusters and add the new one to $\C$.
 Nothing needs to be done if a constraint \eqref{eqn:s-lp:2} becomes tight.
The number of iterations is at most $2|V|$ because at each 
event either a demand dies, or the size of $\C$ decreases.

Computing $\eta$ is nontrivial here.
In particular, we have to solve an auxiliary linear program to find its value.
New variables $y^\ast_{S,d}$ denote the value of vector $y$ after a growth of
size $\eta$.  All the constraints are written for the new variables.
There are exponentially many constraints in this LP, however,
it admits a separation oracle and thus can be optimized.\footnote{
Notice that there are only a polynomial number of non-zero variables at each step
since $y_{S,d}$ may be non-zero only for clusters $S$, and these clusters form a
laminar family in our algorithm.
Verifying constraints \eqref{eqn:eta:1}-\eqref{eqn:eta:3} and \eqref{eqn:eta:5} is very simple.
Verifying constraints \eqref{eqn:eta:4} is equivalent to finding 
$\min_{D\subseteq \D} \pi(D) - y^\ast(D)$ and checking that it is non-negative.
The function to minimize is submodular and thus can be minimized in polynomial time~\cite{IFF00}.
A standard argument shows that the values of these variables have polynomial size.
We defer to the full version of the paper the detailed discussion of
how the LP can be approximated. 
} \begin{lp}
\maximize &\eta \label{eqn:eta:0}\\
\subject& y^\ast_{S,d} = y_{S,d} + \frac{\eta}{\kappa(S)} &&&\forall d\in \D, S\subseteq V, d\odot S, \kappa(S) > 0 \label{eqn:eta:1}\\
& y^\ast_{S,d} = y_{S,d} &&&\forall d\in \D, S\subseteq V, d\odot S, \kappa(S) = 0 \label{eqn:eta:2}\\
&\sum_{S: e\in\delta(S)}\: y^\ast_S \leq c_e &\hspace{0cm}&&
       \forall e\in E \label{eqn:eta:3}\\
& \sum_{d\in D} y^\ast_d \leq \pi(D) &&&\forall D\subseteq \D  \label{eqn:eta:4}\\
& y^\ast_{S,d} \geq 0 &&&\forall d\in\D, S\subseteq V, d\odot S. \label{eqn:eta:5}
\end{lp}




\xxx{maybe one intuitive par on the growth process}
\xxx{TODO:mention at some point why the submodular algorithm is so strange.  many simpler methods fail!}







\paragraph{Pruning}
Let $\eS$ denote the set of all clusters formed during the execution
of the growth step.  It can be easily observed that the clusters $\eS$
are laminar and the maximal clusters are the clusters of $\C$. 
In addition, notice that $F_1[C]$ is connected for each $C\in\eS$.
 
 Let $\B\subseteq\eS$ be the set of all clusters $C$ that do not
cut any live demand.  
Notice that a demand $d$ may still be live at the end of the growth stage
if it is satisfied; roughly speaking, the demand is satisfied before it exhausts
its \emph{potential}.
In the pruning stage, we
iteratively remove edges from $F_1$ to obtain $F_2$.  More
specifically, we first initialize $F_2$ with $F_1$.  Then, as long as
there is a cluster $S\in\B$ such that $F_2\cap\delta(S) = \{e\}$, we remove the edge $e$ from
$F_2$. 

A cluster $C$ is called a \emph{pruned cluster} if it is pruned in the
second stage in which case, $\delta(C)\cap F_2 = \emptyset$.  Hence, a
pruned cluster cannot have non-empty and proper intersection with a
connected component of $F_2$.


\begin{algorithm}
\caption{\algo{Submodular-PC-Clustering}\label{alg:submod-pc-cluster}}
\textbf{Input:} Instance $(G(V,E), \D, \pi)$ of \prob{Generalized prize-collecting Steiner forest}\\
\textbf{Output:} Forest $F$, subset of demands $\dunsat$ and fractional solution $y$.

\begin{algorithmic}[1]
\STATE Let $F_1\gets\emptyset$.
 \STATE Let $y_{S,d}\gets 0$ for any $d\in\D, S\subseteq V, d\odot S$.
 \STATE Let $\eS\gets\C\gets\left\{ \{v\}: v\in V^\ast\right\}$.
\WHILE {there is a live demand}
\STATE Compute $\eta$ via LP~\eqref{eqn:eta:0}:
the largest possible value such that simultaneously 
increasing  $y_{C}$ by $\eta$ for all active clusters $C \in \C$ does not violate Constraints~\eqref{eqn:s-lp:1}-\eqref{eqn:s-lp:3}.
\STATE Let $y_{C,d}\gets y_{C,d}+\frac{\eta}{\kappa(C)}$ for all live demands $d$ crossing clusters $C \in \C$, i.e., $d \odot C$.
  \IF {$\exists e\in E$ that is tight and connects two clusters $C_1$ and $C_2$}
   \STATE Pick one such edge $e=(u,v)$.
   \STATE Let $F_1\gets F_1\cup \{e\}$.
   \STATE Let $C \gets C_1 \cup C_2$.
   \STATE Let $\C\gets \C\cup \{C\} \setminus \{C_1, C_2\}$.
   \STATE Let $\eS \gets \eS \cup \{C\}$.
  \ENDIF
\ENDWHILE
\STATE Let $F_2 \gets F_1$.
\STATE Let $\B$ be the set of all clusters $S\in\eS$ that
do not cut any live demands.
\WHILE {$\exists S\in\B$ such that $F_2\cap\delta(S)=\{e\}$ for an edge $e$}
  \STATE Let $F_2 \gets F_2 \setminus \{e\}$.
\ENDWHILE
\STATE Let $\dunsat$ denote the set of dead demands.
\STATE Output $F := F_2$, $\dunsat$ and $y$.
\end{algorithmic}
\end{algorithm}
















We first bound the length of the forest $F$.
The following lemma is similar to the analysis of the algorithm in~\cite{GW95}.
However, we do not have a primal LP to give a bound on the dual.
Rather, the upper bound for the length is $\pi(\D)$.
In addition, we bound the cost of a forest $F$ that may have
more than one connected component, whereas the prize-collecting Steiner tree
algorithm of~\cite{GW95} finds a connected graph at the end.
\begin{lemma}\label{lem:submod-growth:cost}
 The cost of $F_2$ is at most $2y(\D)$.
\end{lemma}

\begin{proof}
Recall that the growth phase has several events corresponding to an edge or set constraint going tight.
We first break apart $y$ variables by epoch.  Let $t_j$ be the
  time at which the $j^{\rm th}$ event point occurs in the growth phase ($0=t_0\leq t_1 \leq t_2 \leq \cdots$), so the $j^{\rm th}$ epoch is the interval of time from $t_{j-1}$ to $t_j$.  For each cluster $C$, let $y_{C}^{(j)}$  be the amount by which $y_C$ grew during epoch $j$, which is  $t_j-t_{j-1}$ if it was active during this epoch, and zero otherwise.  Thus, $y_C = \sum_j y_{C}^{(j)}$.
Because each edge $e$ of $F_2$
  was added at some point by the growth stage when its edge packing constraint \eqref{eqn:s-lp:1} became tight, we can exactly apportion the cost
  $c_e$ amongst the collection of clusters $\{C : e\in \delta(C)\}$ whose
  variables ``pay for'' the edge, and can divide this up further
  by epoch.  In other words, $c_e = \sum_j \sum_{C:e\in \delta(C)}
  y_{C}^{(j)}$.  We will now prove that the total edge cost from $F_2$
  that is apportioned to epoch $j$ is at most $2 \sum_{C} y_{C}^{(j)}$.  In other words, during each epoch,
  the total rate at which edges of $F_2$ are paid for by all active
  clusters is at most twice the number of active clusters.
  Summing over the epochs yields the desired conclusion.

  We now analyze an arbitrary epoch $j$.  Let $\C_j$ denote the set
  of clusters that existed during epoch $j$.
Consider the graph $F_2$, and then collapse each cluster
  $C \in \C_j$ into a supernode.  Call the resulting graph $H$.
Although the nodes of
  $H$ are identified with clusters in $\C_j$, we will continue to refer
  to them as clusters, in order to to avoid confusion with the nodes of
  the original graph.  Some of the clusters are active and some may be
  inactive.  Let us denote the active and inactive clusters in $\C_j$
  by $\C_{act}$ and $\C_{dead}$, respectively.
  The edges of $F_2$ that are being partially paid for during epoch $j$
  are exactly those edges of $H$ that are incident to an active cluster,
  and the total amount of these edges that is paid off during epoch
  $j$ is $(t_j-t_{j-1}) \sum_{C \in \C_{act}} \deg_H(C)$.
  Since every active cluster grows by exactly $t_j-t_{j-1}$ in epoch
  $j$, we have $\sum_{C} y_{C}^{(j)} \geq \sum_{C \in\C_j}y_{C}^{(j)} = (t_j-t_{j-1}) |\C_{act}|$. Thus, it suffices to show that $\sum_{C
    \in \C_{act}} \deg_H(C) \leq 2 |\C_{act}|$.





First we
  must make some simple observations about $H$.  Since $F_2$ is a subset of the edges in $F_1$, and each cluster represents a
  disjoint induced connected subtree of $F_1$, the contraction to $H$ introduces  no cycles.  Thus, $H$ is a forest.
  All the leaves of $H$ must
  be live clusters because otherwise the corresponding cluster $C$ would be
  in $\B$ and  hence would have been pruned away.


  With this information about $H$, it is easy to bound $\sum_{C \in
    \C_{act}} \deg_H(C)$.
  The total degree in $H$ is at most $2(|\C_{act}|+|\C_{dead}|)$.
  Noticing that the degree of dead clusters is at least two,
  we get $\sum_{C\in\C_{act}}\deg_H(C) \leq 2(|\C_{act}|+|\C_{dead}|) - 2|\C_{dead}| = 2|\C_{act}|$ as desired.
\end{proof}


Now we can prove Lemma~\ref{lem:submod-growth}
that characterizes the output of \algo{Submodular-PC-Clustering}.

\begin{proof}[\proofname\ of Lemma~\ref{lem:submod-growth}]
For every demand $d \in \dunsat$ we have a set $D \ni d$
such that $y(D) = \pi(D)$.  The definition of $\dunsat$ guarantees
$D \subseteq \dunsat$.  Therefore, we have sets $D_1, D_2, \dots, D_l$
that are all tight (i.e., $y(D_i) = \pi(D_i)$) and they span $\dunsat$
(i.e., $\dunsat = \cup_i D_i$).
To prove $y(\dunsat) = \pi(\dunsat)$, we use induction
and combine $D_i$'s two at a time.
For any two tight sets $A$ and $B$ we have
$y(A \cup B) = y(A) + y(B) - y(A\cap B) = \pi(A) + \pi(B) - y(A\cap B)
\geq \pi(A) + \pi(B) -\pi(A\cap B) \geq \pi(A \cup B)$,
where the second equation follows from tightness of $A$ and $B$,
the third step is a result of Constraint~\eqref{eqn:s-lp:2},
and the last step follows from submodularity.
Constraint~\eqref{eqn:s-lp:2} has it that $\pi(A\cup B) \geq y(A \cup B)$,
therefore, it has to hold with equality.

Clearly, at the end of execution of \algo{Submodular-PC-Clustering},
any live demand is already satisfied.  Notice that such demands are not
affected in the pruning stage.  Hence, only dead demands may be not satisfied.
This guarantees the second condition.  The third condition follows from
Lemma~\ref{lem:submod-growth:cost}.
\end{proof}






\subsection{Restricting the demands}\label{sec:reduce:1}\label{sec:rest-prize}

We prove Theorem~\ref{thm:reduce} in this section.
First, we obtain a constant-factor approximate solution $F^+$---this can be done, e.g., via the $3$-approximation algorithm for general graphs \cite{HKKN10}.
Let $\D^+$ denote the demands satisfied by $F^+$.
 We denote by $T^+_j$ the connected components of $F^+$.
 For each demand $d=\{s,t\}\in\D^+$ we clearly have $\{s,t\} \subseteq V(T^+_j)$
for some $j$.  However, for an unsatisfied demand $d'=\{s',t'\} \in \D \setminus \D^+$,
the vertices $s'$ and $t'$ belong to  two different components of $F^+$.
Construct $G^\ast$ from $G$ by reducing the length of edges of $F^+$ to zero.
The new penalty function $\pi^\ast$ is defined as follows:
\begin{align}\label{eqn:new-pi}
\pi^\ast(D) &:= \eps^{-1} \pi(D) &\text{for } D \subseteq \D.
\end{align}
 Finally we run \algo{Submodular-PC-Clustering} on $(G^\ast, \D, \pi^\ast)$; see Algorithm~\ref{alg:prize-restrict}. 















\begin{algorithm}
\caption{\algo{Restrict-Demands}\label{alg:prize-restrict}}
\textbf{Input:} Instance $(G, \D, \pi)$ of \prob{Submodular Prize-Collecting Steiner Forest}\\
\textbf{Output:} Forest $F$ and $\dunsat$.

\begin{algorithmic}[1]
 \STATE Use the algorithm of Hajiaghayi et al.~\cite{HKKN10} to find a $3$-approximate 
solution:  a forest $F^+$ satisfying subset $\D^+$ of demands.
\STATE Construct $G^\ast(V, E^\ast)$ in which $E^\ast$ is the same as $E$ except that
the edges of $F^+$ have length zero in $E^\ast$. \STATE Define $\pi^\ast$ as Equation~\eqref{eqn:new-pi}.
\STATE Call \algo{Submodular-PC-Clustering} on $(G^\ast, \D, \pi^\ast)$ to obtain the result $F$, $\dunsat$ and $y$.
\STATE Output $F$ and $\dunsat$.
\end{algorithmic}
\end{algorithm}


Now we show that the algorithm \algo{Restrict-Demands} outlined above  satisfies the requirements
of Theorem~\ref{thm:reduce}.  Before doing so, we show how the cost of
a forest can be compared to the values of the output vector $y$.
\begin{lemma}\label{lem:coloring}
If a graph $F$ satisfies a set $\dsat$ of demands,
then $\len(F) \geq \sum_{d\in\dsat} y_d$.
\end{lemma}
This is quite intuitive.
Recall that the $y$ variables color the edges of the graph.
Consider a segment on edges corresponding to cluster $S$ with color $d$.
At least one edge of $F$ \emph{passes through} the cut $(S,\bar{S})$.
Thus, a portion of the cost of $F$ can be charged to $y_{S,d}$.
Hence, the total cost of the graph $F$ is at least as large as 
the total amount of colors paid for by $\dsat$.
We now provide a formal proof.
\begin{proof}
The length of the graph $F$ is
\begin{align*}
 \sum_{e\in F} c_e
  &\geq  \sum_{e\in F}\sum_{S: e\in\delta(S)} y_S &\text{by \eqref{eqn:s-lp:1}}\\
  &=     \sum_S |F \cap \delta(S)|y_S \\
  &\geq  \sum_{S: F \cap \delta(S)\neq\emptyset} y_S \\
  &=     \sum_{S: F\cap\delta(S)\neq\emptyset}\sum_{d: d\odot S} y_{S,d} \\
  &=     \sum_{d}\sum_{\substack{S: d\odot S\\F\cap\delta(S)\neq\emptyset}} y_{S,d} \\
  &\geq  \sum_{d\in \dsat}\sum_{\substack{S: d\odot S\\F\cap\delta(S)\neq\emptyset}} y_{S,d} \\
  &=     \sum_{d\in \dsat}\sum_{S: d\odot S} y_{S,d}, \\\intertext{because $y_{S,d}=0$ if $d\in \dsat$ and $F\cap\delta(S)=\emptyset$,}
  &=     \sum_{d\in \dsat} y_d  &\qedhere
\end{align*}
\end{proof}





\begin{proof}[\proofname\ of Theorem~\ref{thm:reduce}]
  We know that $\len(F^+) + \pi(\D \setminus \D^+) \leq 3\OPT$ because we
start with a $3$-approximate solution.
For any demand $d=(s,t)$, we know that $y_d$ is not more than the distance of $s, t$ in $G^\ast$.
Since distance between endpoints of $d$ is zero if it is satisfied in $\D^+$,
$y_d$ is non-zero only if $d\in\D\setminus\D^+$, we have $y(\D) = y(\D\setminus\D^+)\leq \pi^\ast(\D\setminus\D^+)$ by constraint~\eqref{eqn:s-lp:2}.
  Lemma~\ref{lem:submod-growth} gives $\len(F)$ in $G^\ast$, denoted by $\len_{G^\ast}(F)$,
is at most $2 y(\D) \leq 2\pi^\ast(\D\setminus\D^+) = 2\eps^{-1}\pi(\D\setminus\D^+) \leq 6\eps^{-1}\OPT$.
  Therefore, $\len(F) = \len(F^+) + \len_{G^\ast}(F) \leq (6\eps^{-1} + 3)\OPT$. 

To establish the second condition of the theorem, take an optimal forest $F'$:
$F'$ satisfies demands $\D^\OPT$, and we have $\len(F') + \pi(\D \setminus \D^\OPT) = \OPT$.
Define $A := \D^\OPT \setminus \dsat$ and $B := \dunsat \setminus A$.
The penalty of $F'$ under $\pi'$ is $\pi((\D\setminus\D^\OPT) \cup \dunsat) = \pi((\dsat\setminus\D^\OPT) \cup A \cup B)$.
Hence, the increase in penalty of $F'$ due to changing from $\pi$ to $\pi'$
is $\pi((\dsat\setminus\D^\OPT) \cup A \cup B) - \pi((\dsat\setminus\D^\OPT)  \cup B) \leq \pi(A\cup B) - \pi(B)$ due to the decreasing marginal cost property of submodular functions. 
We have $y(A \cup B) = \pi^\ast(A \cup B) = \eps^{-1}\pi(A\cup B)$ because $A\cup B = \dunsat$ is the set of dead demands of \algo{Submodular-PC-Clustering}; see the first condition of Lemma~\ref{lem:submod-growth}.
We also have $\eps^{-1}\pi(B) = \pi^\ast(B) \geq y(B)$ because of Constraint \eqref{eqn:s-lp:2}.
Therefore, the additional penalty is at most $\eps [ y(A \cup B) - y(B) ] = \eps y(A)$.
Since $F'$ satisfies the demands $A$,
we have $y(A) \leq \len(F') \leq \OPT$ from Lemma~\ref{lem:coloring}.
Therefore, the additional penalty is at most $\eps \OPT$.


The extension to \spctsp and \spcs is straight-forward once we observe
that the cost of building a tour or a stroll on a subset $S$ of vertices
is at least the cost of constructing a Steiner tree on the same set.
Hence, there algorithm pretends it has an \prob{SPCST} instance, and restricts
the demand set accordingly.  However, the extra penalty due to the
ignored demands $\dunsat$ is charged to the Steiner tree cost which
is no more than the TSP or stroll length.
\end{proof}





















