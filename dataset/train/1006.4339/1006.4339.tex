\subsection{Submodular prize-collecting clustering}\label{sec:submod-cluster}

First we present and analyze a primal-dual algorithm for 
\spcsf, and later 
we see how this algorithm can be used to achieve the goal of
identifying and removing certain demands from the optimal solution
such that the additional penalty is negligible.

Consider an instance  of the \spcsf.
A set  is said to \emph{cut} a demand 
if and only if .  We denote this by the short-hand ,
and say the demand  \emph{crosses} the set .
In the linear program~\eqref{eqn:s-lp:1}--\eqref{eqn:s-lp:3},
there is a variable  for any ,  such
that .
Conveniently, we use the short-hands 
and .



\begin{lp}
& \sum_{S: e\in\delta(S)}\: y_S \leq c_e &\hspace{3cm}&&
       \forall e\in E \label{eqn:s-lp:1}\\
& \sum_{d\in D} y_d \leq \pi(D) &&&\forall D\subseteq \D  \label{eqn:s-lp:2}\\
& y_{S,d} \geq 0 &&&\forall d\in\D, S\subseteq V, d\odot S. \label{eqn:s-lp:3}
\end{lp}










We produce a solution to the above LP.
Theorem~\ref{thm:reduce} is proved via some properties of this solution.
 These constraints look like the dual of a natural linear program for 
\spcsf.
For the sake of convenience, we use the notation  for any .




\begin{lemma}\label{lem:submod-growth}
Given an instance  of \spcsf,
we produce in polynomial time a forest  and a subset  of demands, 
along with a feasible vector  for the above LP such that
\begin{enumerate}
 \item ;
 \item  satisfies any demand in ;
and
 \item .
\end{enumerate}
\end{lemma}










 

The solution is built up in two stages.
First we perform an \emph{submodular growth} to find a forest  and a corresponding  vector.
This is different from the usual growth phase of \cite{GW95, AKR91} in that
the penalty function may go tight for a set of vertices that are not currently connected.
In the second stage, we \emph{prune} some  edges of  to obtain another forest .
Below we describe the two phases of Algorithm~\ref{alg:submod-pc-cluster} (\algo{Submodular-PC-Clustering}).



\paragraph{Growth}
 We begin with a zero vector , and an empty set .
 A demand  is said to be \emph{live} if and only if
 for any  that .  
If a demand is not live, it is \emph{dead}.
 During the execution of the algorithm \algo{Submodular-PC-Clustering}, 
we maintain a partition  of vertices  into clusters; 
it initially consists of singleton sets.
 Each cluster is either \emph{active} or \emph{inactive};
  the cluster  is \emph{active} if and only if there is a live demand .
We simultaneously \emph{grow} all the active clusters by .
 In particular, if there are  live demands crossing an active cluster , 
we increase  by  for each live demand .
 Hence,  is increased by  for every active cluster .
 We pick the largest value for  that does not violate any of the constraints in \eqref{eqn:s-lp:1} or \eqref{eqn:s-lp:2}.
 Obviously,  is finite in each iteration because the 
values of these variables cannot be larger than .
 Hence, at least one such constraint goes tight after each growth step.
 If this happens for an edge constraint for ,
 then there are two clusters  and  in ,
 at least one of which is growing.
 We merge the two clusters into  by adding the 
edge  to , remove the old clusters and add the new one to .
 Nothing needs to be done if a constraint \eqref{eqn:s-lp:2} becomes tight.
The number of iterations is at most  because at each 
event either a demand dies, or the size of  decreases.

Computing  is nontrivial here.
In particular, we have to solve an auxiliary linear program to find its value.
New variables  denote the value of vector  after a growth of
size .  All the constraints are written for the new variables.
There are exponentially many constraints in this LP, however,
it admits a separation oracle and thus can be optimized.\footnote{
Notice that there are only a polynomial number of non-zero variables at each step
since  may be non-zero only for clusters , and these clusters form a
laminar family in our algorithm.
Verifying constraints \eqref{eqn:eta:1}-\eqref{eqn:eta:3} and \eqref{eqn:eta:5} is very simple.
Verifying constraints \eqref{eqn:eta:4} is equivalent to finding 
 and checking that it is non-negative.
The function to minimize is submodular and thus can be minimized in polynomial time~\cite{IFF00}.
A standard argument shows that the values of these variables have polynomial size.
We defer to the full version of the paper the detailed discussion of
how the LP can be approximated. 
} \begin{lp}
\maximize &\eta \label{eqn:eta:0}\\
\subject& y^\ast_{S,d} = y_{S,d} + \frac{\eta}{\kappa(S)} &&&\forall d\in \D, S\subseteq V, d\odot S, \kappa(S) > 0 \label{eqn:eta:1}\\
& y^\ast_{S,d} = y_{S,d} &&&\forall d\in \D, S\subseteq V, d\odot S, \kappa(S) = 0 \label{eqn:eta:2}\\
&\sum_{S: e\in\delta(S)}\: y^\ast_S \leq c_e &\hspace{0cm}&&
       \forall e\in E \label{eqn:eta:3}\\
& \sum_{d\in D} y^\ast_d \leq \pi(D) &&&\forall D\subseteq \D  \label{eqn:eta:4}\\
& y^\ast_{S,d} \geq 0 &&&\forall d\in\D, S\subseteq V, d\odot S. \label{eqn:eta:5}
\end{lp}




\xxx{maybe one intuitive par on the growth process}
\xxx{TODO:mention at some point why the submodular algorithm is so strange.  many simpler methods fail!}







\paragraph{Pruning}
Let  denote the set of all clusters formed during the execution
of the growth step.  It can be easily observed that the clusters 
are laminar and the maximal clusters are the clusters of . 
In addition, notice that  is connected for each .
 
 Let  be the set of all clusters  that do not
cut any live demand.  
Notice that a demand  may still be live at the end of the growth stage
if it is satisfied; roughly speaking, the demand is satisfied before it exhausts
its \emph{potential}.
In the pruning stage, we
iteratively remove edges from  to obtain .  More
specifically, we first initialize  with .  Then, as long as
there is a cluster  such that , we remove the edge  from
. 

A cluster  is called a \emph{pruned cluster} if it is pruned in the
second stage in which case, .  Hence, a
pruned cluster cannot have non-empty and proper intersection with a
connected component of .


\begin{algorithm}
\caption{\algo{Submodular-PC-Clustering}\label{alg:submod-pc-cluster}}
\textbf{Input:} Instance  of \prob{Generalized prize-collecting Steiner forest}\\
\textbf{Output:} Forest , subset of demands  and fractional solution .

\begin{algorithmic}[1]
\STATE Let .
 \STATE Let  for any .
 \STATE Let .
\WHILE {there is a live demand}
\STATE Compute  via LP~\eqref{eqn:eta:0}:
the largest possible value such that simultaneously 
increasing   by  for all active clusters  does not violate Constraints~\eqref{eqn:s-lp:1}-\eqref{eqn:s-lp:3}.
\STATE Let  for all live demands  crossing clusters , i.e., .
  \IF { that is tight and connects two clusters  and }
   \STATE Pick one such edge .
   \STATE Let .
   \STATE Let .
   \STATE Let .
   \STATE Let .
  \ENDIF
\ENDWHILE
\STATE Let .
\STATE Let  be the set of all clusters  that
do not cut any live demands.
\WHILE { such that  for an edge }
  \STATE Let .
\ENDWHILE
\STATE Let  denote the set of dead demands.
\STATE Output ,  and .
\end{algorithmic}
\end{algorithm}
















We first bound the length of the forest .
The following lemma is similar to the analysis of the algorithm in~\cite{GW95}.
However, we do not have a primal LP to give a bound on the dual.
Rather, the upper bound for the length is .
In addition, we bound the cost of a forest  that may have
more than one connected component, whereas the prize-collecting Steiner tree
algorithm of~\cite{GW95} finds a connected graph at the end.
\begin{lemma}\label{lem:submod-growth:cost}
 The cost of  is at most .
\end{lemma}

\begin{proof}
Recall that the growth phase has several events corresponding to an edge or set constraint going tight.
We first break apart  variables by epoch.  Let  be the
  time at which the  event point occurs in the growth phase (), so the  epoch is the interval of time from  to .  For each cluster , let   be the amount by which  grew during epoch , which is   if it was active during this epoch, and zero otherwise.  Thus, .
Because each edge  of 
  was added at some point by the growth stage when its edge packing constraint \eqref{eqn:s-lp:1} became tight, we can exactly apportion the cost
   amongst the collection of clusters  whose
  variables ``pay for'' the edge, and can divide this up further
  by epoch.  In other words, .  We will now prove that the total edge cost from 
  that is apportioned to epoch  is at most .  In other words, during each epoch,
  the total rate at which edges of  are paid for by all active
  clusters is at most twice the number of active clusters.
  Summing over the epochs yields the desired conclusion.

  We now analyze an arbitrary epoch .  Let  denote the set
  of clusters that existed during epoch .
Consider the graph , and then collapse each cluster
   into a supernode.  Call the resulting graph .
Although the nodes of
   are identified with clusters in , we will continue to refer
  to them as clusters, in order to to avoid confusion with the nodes of
  the original graph.  Some of the clusters are active and some may be
  inactive.  Let us denote the active and inactive clusters in 
  by  and , respectively.
  The edges of  that are being partially paid for during epoch 
  are exactly those edges of  that are incident to an active cluster,
  and the total amount of these edges that is paid off during epoch
   is .
  Since every active cluster grows by exactly  in epoch
  , we have . Thus, it suffices to show that .





First we
  must make some simple observations about .  Since  is a subset of the edges in , and each cluster represents a
  disjoint induced connected subtree of , the contraction to  introduces  no cycles.  Thus,  is a forest.
  All the leaves of  must
  be live clusters because otherwise the corresponding cluster  would be
  in  and  hence would have been pruned away.


  With this information about , it is easy to bound .
  The total degree in  is at most .
  Noticing that the degree of dead clusters is at least two,
  we get  as desired.
\end{proof}


Now we can prove Lemma~\ref{lem:submod-growth}
that characterizes the output of \algo{Submodular-PC-Clustering}.

\begin{proof}[\proofname\ of Lemma~\ref{lem:submod-growth}]
For every demand  we have a set 
such that .  The definition of  guarantees
.  Therefore, we have sets 
that are all tight (i.e., ) and they span 
(i.e., ).
To prove , we use induction
and combine 's two at a time.
For any two tight sets  and  we have
,
where the second equation follows from tightness of  and ,
the third step is a result of Constraint~\eqref{eqn:s-lp:2},
and the last step follows from submodularity.
Constraint~\eqref{eqn:s-lp:2} has it that ,
therefore, it has to hold with equality.

Clearly, at the end of execution of \algo{Submodular-PC-Clustering},
any live demand is already satisfied.  Notice that such demands are not
affected in the pruning stage.  Hence, only dead demands may be not satisfied.
This guarantees the second condition.  The third condition follows from
Lemma~\ref{lem:submod-growth:cost}.
\end{proof}






\subsection{Restricting the demands}\label{sec:reduce:1}\label{sec:rest-prize}

We prove Theorem~\ref{thm:reduce} in this section.
First, we obtain a constant-factor approximate solution ---this can be done, e.g., via the -approximation algorithm for general graphs \cite{HKKN10}.
Let  denote the demands satisfied by .
 We denote by  the connected components of .
 For each demand  we clearly have 
for some .  However, for an unsatisfied demand ,
the vertices  and  belong to  two different components of .
Construct  from  by reducing the length of edges of  to zero.
The new penalty function  is defined as follows:

 Finally we run \algo{Submodular-PC-Clustering} on ; see Algorithm~\ref{alg:prize-restrict}. 















\begin{algorithm}
\caption{\algo{Restrict-Demands}\label{alg:prize-restrict}}
\textbf{Input:} Instance  of \prob{Submodular Prize-Collecting Steiner Forest}\\
\textbf{Output:} Forest  and .

\begin{algorithmic}[1]
 \STATE Use the algorithm of Hajiaghayi et al.~\cite{HKKN10} to find a -approximate 
solution:  a forest  satisfying subset  of demands.
\STATE Construct  in which  is the same as  except that
the edges of  have length zero in . \STATE Define  as Equation~\eqref{eqn:new-pi}.
\STATE Call \algo{Submodular-PC-Clustering} on  to obtain the result ,  and .
\STATE Output  and .
\end{algorithmic}
\end{algorithm}


Now we show that the algorithm \algo{Restrict-Demands} outlined above  satisfies the requirements
of Theorem~\ref{thm:reduce}.  Before doing so, we show how the cost of
a forest can be compared to the values of the output vector .
\begin{lemma}\label{lem:coloring}
If a graph  satisfies a set  of demands,
then .
\end{lemma}
This is quite intuitive.
Recall that the  variables color the edges of the graph.
Consider a segment on edges corresponding to cluster  with color .
At least one edge of  \emph{passes through} the cut .
Thus, a portion of the cost of  can be charged to .
Hence, the total cost of the graph  is at least as large as 
the total amount of colors paid for by .
We now provide a formal proof.
\begin{proof}
The length of the graph  is

\end{proof}





\begin{proof}[\proofname\ of Theorem~\ref{thm:reduce}]
  We know that  because we
start with a -approximate solution.
For any demand , we know that  is not more than the distance of  in .
Since distance between endpoints of  is zero if it is satisfied in ,
 is non-zero only if , we have  by constraint~\eqref{eqn:s-lp:2}.
  Lemma~\ref{lem:submod-growth} gives  in , denoted by ,
is at most .
  Therefore, . 

To establish the second condition of the theorem, take an optimal forest :
 satisfies demands , and we have .
Define  and .
The penalty of  under  is .
Hence, the increase in penalty of  due to changing from  to 
is  due to the decreasing marginal cost property of submodular functions. 
We have  because  is the set of dead demands of \algo{Submodular-PC-Clustering}; see the first condition of Lemma~\ref{lem:submod-growth}.
We also have  because of Constraint \eqref{eqn:s-lp:2}.
Therefore, the additional penalty is at most .
Since  satisfies the demands ,
we have  from Lemma~\ref{lem:coloring}.
Therefore, the additional penalty is at most .


The extension to \spctsp and \spcs is straight-forward once we observe
that the cost of building a tour or a stroll on a subset  of vertices
is at least the cost of constructing a Steiner tree on the same set.
Hence, there algorithm pretends it has an \prob{SPCST} instance, and restricts
the demand set accordingly.  However, the extra penalty due to the
ignored demands  is charged to the Steiner tree cost which
is no more than the TSP or stroll length.
\end{proof}





















