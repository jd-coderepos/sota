\documentclass{elsart}
\usepackage{graphics}
\usepackage{epsfig}
\usepackage{color}
\usepackage{enumerate}
\usepackage{ulem}




\def\para{\vspace{3mm}}




\def\deg{{\rm deg}}
\def\tdeg{{\rm tdeg}}
\def\card{{\rm Card}}

\def\qed{\hfill  \framebox(5,5){}}
\def\lcoeff{{\rm lcoeff}}
\def\coeff{{\rm coeff}}
\def\mcd{{\rm gcd}}
\def\StHa{{\rm StHa}}
\def\stha{{\rm stha}}
\def\Sres{{\rm Sres}}
\def\Res{{\rm Res}}
\def\ord{{\rm ord}}
\def\pp{{\mathcal  P}}
\def\cc{{\mathcal  C}}
\def\oo{{\mathcal  O}_d({\mathcal C})}
\def\sig{{\rm sign}}

\expandafter\chardef\csname pre amssym.def at\endcsname=\the\catcode`\@ \catcode`\@=11

\def\undefine#1{\let#1\undefined}
\def\newsymbol#1#2#3#4#5{\let\next@\relax
 \ifnum#2=\@ne\let\next@\msafam@\else
 \ifnum#2=\tw@\let\next@\msbfam@\fi\fi
 \mathchardef#1="#3\next@#4#5}
\def\mathhexbox@#1#2#3{\relax
 \ifmmode\mathpalette{}{\m@th\mathchar"#1#2#3}\else\leavevmode\hbox{}\fi}
\def\hexnumber@#1{\ifcase#1 0\or 1\or 2\or 3\or 4\or 5\or 6\or 7\or 8\or
 9\or A\or B\or C\or D\or E\or F\fi}

\font\tenmsa=msam10 \font\sevenmsa=msam7 \font\fivemsa=msam5
\newfam\msafam
\textfont\msafam=\tenmsa \scriptfont\msafam=\sevenmsa \scriptscriptfont\msafam=\fivemsa \edef\msafam@{\hexnumber@\msafam}
\mathchardef\dabar@"0\msafam@39
\def\dashrightarrow{\mathrel{\dabar@\dabar@\mathchar"0\msafam@4B}}
\def\dashleftarrow{\mathrel{\mathchar"0\msafam@4C\dabar@\dabar@}}
\let\dasharrow\dashrightarrow
        \font\tenmsb=msbm10
\font\sevenmsb=msbm7 \font\fivemsb=msbm5
\newfam\msbfam
\textfont\msbfam=\tenmsb \scriptfont\msbfam=\sevenmsb \scriptscriptfont\msbfam=\fivemsb \edef\msbfam@{\hexnumber@\msbfam}
\def\Bbb#1{\fam\msbfam\relax#1}

\newtheorem{theorem}{{\bf Theorem}}
\newtheorem{remark}{{\bf Remark}}
\newtheorem{definition}[theorem]{{\bf Definition}}
\newtheorem{corollary}[theorem]{{\bf Corollary}}
\newtheorem{proposition}[theorem]{{\bf Proposition}}
\newtheorem{lemma}[theorem]{{\bf Lemma}}
\newtheorem{example}{{\bf Example}}




\begin{document}
\begin{frontmatter}






\title{On the Shape of Curves that are Rational in Polar Coordinates}



\author[a]{Juan Gerardo Alc\'azar\thanksref{proy1}},
\ead{juange.alcazar@uah.es}
\author[b]{Gema Mar\'{\i}a D\'{\i}az-Toca\thanksref{proy2}},
\ead{gemadiaz@um.es}




\address[a]{Departamento de Matem\'aticas, Universidad de Alcal\'a,
E-28871 Madrid, Spain}
\address[b]{Departamento de Matem\'atica Aplicada, Universidad de
Murcia,  30100 Murcia, Spain}




\thanks[proy1]{Supported by the Spanish ``Ministerio de
Ciencia e Innovacion" under the Project MTM2011-25816-C02-01. Member of the Research Group {\sc asynacs} (Ref. {\sc ccee2011/r34})}

\thanks[proy2]{Supported by the Spanish ``Ministerio de
Ciencia e Innovacion" under the Project MTM2011-25816-C02-02.}



\begin{abstract}
In this paper we provide a computational approach to the shape
of curves which are rational in polar coordinates, i.e. which are defined by means of a parametrization
 where both  are rational functions. Our study includes theoretical aspects on the shape of these curves, and algorithmic results which eventually lead to an algorithm for plotting the ``interesting parts" of the curve, i.e. the parts showing the main geometrical features.
\end{abstract}
\end{frontmatter}

\section{Introduction}\label{section-introduction}

Plotting and correct visualization of algebraic curves, both in the case when they are implicitly defined by means of a polynomial , or by a parametrization  with  being rational, have received a great deal of attention in the literature on scientific computation (see for example \cite{JGDT}, \cite{Cheng}, \cite{Eigen}, \cite{Emel}, \cite{Lalo}, \cite{Hong}, \cite{seidel}). With this paper, we want to initiate a similar study for curves which are written in polar coordinates. Such curves may appear in Engineering and Physics, in particular in Mechanics (very specially in Celestial Mechanics), and also in Cartography. In this sense, here we address those curves which are rational when considered in polar form, i.e. curves defined by means of a parametrization  where both  are rational functions. Arquimedes' spiral, Cote's spiral under certain conditions, Fermat's spiral or Lituus' spiral (some of these can be found in www.mathematische-basteleien.de/spiral.htm), for example, belong to this class; also, in Cartography these curves may appear when using for example conic, pseudo-conic or polyconic projections (\cite{Buga}, \cite{Sny}).

It is quite natural to start wondering if this kind of curves can be algebraic, when considered in cartesian coordinates. However, one may prove (see Theorem \ref{th-non-alg} in Subsection \ref{subsec-prelim}) that, with the exceptions of lines and circles, this cannot happen. As a consequence they can exhibit properties that algebraic curves cannot have. Here we analyze some of these which are relevant from the point of view of plotting, in particular the existence or not of infinitely many self-intersections (see Subsection \ref{subsec-self-int}), the appearance of limit points, limit circles and spiral branches, all of them introduced in Subsection \ref{subsec-limit}, and the relationship between both phenomena.

The possibility of detecting in advance the above phenomena is very useful to improve the plotting of these curves, which can be really ``devilish". To give an idea, one may use the computer algebra system Maple, where the instruction {\tt polarplot} is available (we have also tried other packages like Maxima, SAGE and Mathematica; however, either no similar instruction was available, or the corresponding command behaved in a very similar way to {\tt polarplot}). This instruction allows to draw curves defined in polar coordinates either by means of an equation , or by a parametrization , and has a nice performance for simple curves, but may not be enough for illustrating the behavior of a more complicated curve. As an example, one may consider the following parametrizations in polar coordinates: (1) ; (2) ; (3) . If one uses {\tt polarplot} to visualize these curves, one obtains the outputs in Fig. 1: here the curves (1), (2), (3) are displayed from left to right; in (1) and (2) we have asked Maple to plot the curve for , and in the case of (3) we have chosen  (because  are both non-bounded for ). However, in the three cases it is clear that the output does not really help to understand the behavior of the curve.

\begin{figure}[ht]
\begin{center}
\centerline{}
\end{center}
\caption{Some complicated curves plotted with {\tt polarplot}}
\end{figure}

By using our ideas, one can obtain an algorithm which provides, for a given curve of the considered kind, information and several plottings corresponding to the more interesting parts of it. We have implemented this algorithm in Maple 15. Our algorithm analyzes the curve, detects its main features, and uses the command {\tt polarplot} for plotting the curve over several intervals (generated by our algorithm), each one showing certain features of the curve; together with the provided information, these plottings help to clarify the behavior of the curve. The outputs of our algorithm for curves in Fig. 1 can be checked in Subsection 3.2  and Subsection 3.4.

The paper is structured as follows: in Section 2, the reader will find the theoretical results on the shape of these curves. In Section 3, we provide some details on the algorithm, together with several examples of outputs. Finally, in Section 4 we present some conclusions and suggest future lines of research.










\section{Geometrical Properties of Curves Rational in Polar Coordinates}\label{sec-background}

\subsection{Preliminaries and First Properties}\label{subsec-prelim}

Along the paper, given a point  we denote its polar coordinates by , and its cartesian coordinates by .
Notice that the polar coordinates of  are not unique, since ,  with , or
 with , define the same point.


We will analyze the geometry of a planar curve
 which is {\it rational} when parametrized in polar coordinates. In other words, there exist four real
polynomials , , ,  not all of them constant, with , such that

parametrizes  in polar coordinates. We will also consider the curve , i.e. the rational planar curve parametrized by  over the --plane; for example, if  (which parametrizes an Arquimedes spiral, then  is a line over the  plane. Notice that our goal is not to describe , but still this curve is useful at several steps for proving certain facts on . 

Furthermore we assume that , as a parametrization of , is {\it proper}, i.e. that it is
injective for almost all values of 
(equivalently, that almost all points of  are generated by just one -value). Observe that if  is not proper then it can always be properly reparametrized by applying the algorithm
in Chapter 6 of \cite{SWPD}. Also, we will say that a point  is {\it reached} by the parametrization if there exists some
 such that . It can be proven (see Proposition 4.2 in \cite{Andradas})
 that the only point of  that may not be reached by the parametrization is



whenever it exists. Furthermore, if  is proper then the real points of  that are reached by complex, but not real, values of the
parameter are, except perhaps for , isolated points of  (see also Proposition 4.2 in \cite{Andradas}). Therefore, the
real part of  can be expressed as


In the sequel we will discard the isolated points of  and we will focus on the remaining part of . Since polar coordinates provide a natural mapping  (not 1:1) from  onto , we will say that a point of  is {\it reached} by  if it has the form , where  has been reached by . In this language, notice that we are identifying  where  is the real part of , discarding isolated singularities.

 Now if  is either a circle centered at the origin or a line passing through the origin, then it is algebraic in algebraic and polar form at the same time (in fact,  is represented by an equation of the type  or ). The next theorem proves that circles and lines are the only cases when this phenomenon happens. We acknowledge here the help of Fernando San Segundo for proving the theorem. Here we use the expression {\it real curve} to mean a curve with infinitely many real points.






\begin{theorem} \label{th-non-alg}  Let  be a real algebraic curve. Then it is rational in polar coordinates if and only if
it is either a real circle centered at the origin or a real line passing through the origin.
\end{theorem}

 {\bf Proof.}  The implication  is clear; so, let us focus on . For this purpose, let  be the implicit equation of . Now since  is rational, then it is irreducible, and so also . If  depends only on , taking into account that  is irreducible and that
  is a real curve (because  are real functions),  must be linear. So,  can be either a circle or a point (the origin, if ); but in this last case  is not a real curve, and hence it must be a circle. If  depends only on , then for similar reasons  must be a line. Thus, assume that  depends on both , and let us see that in this case  cannot be algebraic. Indeed, if  is algebraic, let  be its implicit equation and let . Notice that every zero of  is also a zero of .
In this situation, consider
the resultant , where  is the function obtained when substituting formally
 and  in .  Since  is irreducible and 
depends on both , does not have any factor only depending on , and therefore  and  cannot
have any factor in common; hence,  cannot be identically .
Observe also that  must depend explicitly on , i.e. . Now by Lemma 4.3.1 in \cite{Wi96} we have that the resultant  specializes properly when
 and , i.e.

Since  depends on both  we can find an open interval  such that for every , the equation  and therefore also , have at least one real solution. Hence, by well-known properties of resultants, for all  it
holds that  is a zero of , i.e.  vanishes over . Hence, by the
Identity Theorem (see page 81 in \cite{Jong}), and taking into account that  is analytic, it holds that  is identically . Since  is a polynomial in
, ,  where  is present (i.e. ), this implies that , , , are algebraically dependent. However, this cannot happen
because ,  are trascendental functions. \qed



Hereafter we exclude the cases when either  or  are constant.

\subsection{Self-intersections}\label{subsec-self-int}

A point generated by  corresponds to a self-intersection of
 if it exists , , and , such that  is solution of some of the following two
systems:

Moreover, when  exists, we also have to consider the self-intersections involving
, i.e. the solutions of

So, in order to study the self-intersections of  we have to study the
above systems. Additionally, whenever the equation  has more than one solution, or one solution and , the origin is also a self-intersection.


In the algebraic case, Bezout's theorem forces every algebraic curve to have
finitely many self-intersections. However, in our case this does not necessarily hold.  Since  is rational, it cannot happen that a same point of  is crossed by infinitely many branches of . But there can be infinitely many {\it different} self-intersections, which can be detected by analyzing  and . We start with . For this purpose, we write







We denote the numerator of  by , and the numerator of  by .

\begin{lemma} \label{lem-inst-1}
There do not exist  and  (in fact, ) such that  simultaneously divides  and
.
\end{lemma}

{\bf Proof.} Assume that  we have an irreducible polynomial  and  such that  divides both  and .  We need to show that . Suppose that  divides  (resp. ). Since it also divides , it divides , and therefore it also divides  (resp. ).
On the other hand,  by hypothesis. That shows that .
If  divides  (resp. ) we argue similarly with . \qed


Then we are ready to proceed with the following theorem, that will have important consequences on the study of .

\begin{theorem} \label{th-1}
 For any , , (in fact, ) the system  has finitely many solutions.
\end{theorem}

{\bf Proof.}  In order to prove the statement, we need to show that for , , .
 For this purpose, assume by contradiction
that there exists some , , such that . Then  defines an algebraic curve  over ; furthermore, since by hypothesis ,  cannot be . So, there are infinitely many points  with . Now by Lemma \ref{lem-inst-1} only finitely many of them fulfill . So, the functions , ,  and
 are well-defined at almost all the points , and , . Since for a fixed  there can only be finitely many values
of  such that  (because  is rational) we deduce then that there are infinitely many points 
such that  too. In other words, the curves defined over the -plane by  and  have infinitely many points in common, and since  is irreducible and both have the same degree, then they must
define . Now let ; then
 is a zero of , and since  also defines  then it
is also a point of . Following the same reasoning, we conclude that
 also belongs to , and in fact that  for all . Since  we have that these are all different points of . Hence, we have that  intersects the line  at infinitely many points. But this is impossible because  is
algebraic. \qed

We can obtain similar results for . For this purpose, we denote the numerator of  by , and the numerator of
 by . Then the following lemma, analogous to Lemma \ref{lem-inst-1}, holds.

\begin{lemma} \label{lem-inst-2}
There does not exist  and  (in fact, ) such that  simultaneously divides  and .
\end{lemma}

The following theorem, very similar to Theorem \ref{th-1}, holds.

\begin{theorem} \label{th-2}
 For any  (in fact, ), it holds that .
\end{theorem}

{\bf Proof.} Arguing by contradiction as in the proof of Theorem \ref{th-1}, we conclude that  and  define the
same curve (namely, ). So, let  where  does not divide . Then
, and for the same reason  too. Proceeding
this way we get that all the points of the form , with , belong to  . For any value  all these points are different; so, we get that the intersection of  with the line  consists of infinitely
many different points. But this cannot happen because  is algebraic. \qed


Hence, by Theorem \ref{th-1} and Theorem \ref{th-2}, we obtain the following result on the existence of infinitely many
self-intersections of .
\begin{corollary} \label{corol-3}
 has infinitely many self-intersections if and only if  or  have solutions for infinitely many values .
\end{corollary}

Based on Corollary \ref{corol-3}, we have the following result which provides a sufficient condition for  to have finitely many
self-intersections.

\begin{theorem} \label{th-3}
If  is bounded, then there are at most finitely many self-intersections.
\end{theorem}

{\bf Proof.}  If  is bounded, the number of integer values of  satisfying the second equation of  or  for some  is necessarily finite. Then the result follows from Corollary \ref{corol-3}.  \qed

The converse of Theorem \ref{th-3} is not necessarily true (see Example 2). So, we still need a characterization for the existence of infinitely
many self-intersections.
\begin{lemma} \label{need}
The polynomial  cannot be constant. More precisely, it has positive degree in .\end{lemma}


{\bf Proof.}  By Theorem \ref{th-1},  cannot be identically . Moreover, by writing explicitly the system ,

we observe that the points , , are solutions of . So, for any ,  vanishes at  and therefore it cannot be constant. For the same reason,   cannot be an univariate polynomial in . In fact,  is a divisor of  . \qed


\begin{lemma} \label{need2}
The polynomial  cannot be constant. Moreover,  if  has solutions for infinitely many values of , then  has positive degree in .
\end{lemma}

{\bf Proof.} By Theorem \ref{th-2},  cannot be identically . Moreover, by writing explicitly the system ,

we observe that if  with  then there exists  such that the point  is a solution of . If , then the points  are solutions of  for any . As a consequence,   the polynomial  cannot be a constant.

Now assume that  has infinitely many solutions. That implies that there are infinitely   solutions of  with , ,  and   . Due to Theorem \ref{th-2} and the linearity of  in ,the polynomial   has positive degree in both  and .
\qed

Next we denote by  (resp. ) the result of taking out
from the square-free part of  (resp. ) the univariate factors in
.


\begin{theorem} \label{th-4}
  has infinitely many self-intersections if and only if either  or 
is an algebraic curve (in the
-plane), non-bounded in .
\end{theorem}

{\bf Proof.}
If  has infinitely many self-intersections,  by Corollary \ref{corol-3} one of the systems ,  has solutions for infinitely many values . By reasoning over , , one may see that each value  or  gives rise to just finitely many solutions of the system (otherwise, we would reach the conclusion that ). Consequently, if some system ,  has solutions for infinitely many values ,  or  has infinitely many solutions too. Since resultants are  combinations of the polynomials which define the systems  and  and they are non-zero by Lemma \ref{need} and Lemma \ref{need2}, either  or  is non-bounded in .

Conversely,  assume that  is
non-bounded in  (we would argue in a similar way with ). Since it has by definition no univariate factors depending on , there are at most finitely many points  with  where the leading coefficient of  with respect to ,  and  vanish. By the Extension Theorem of resultants, any
other point of  corresponds to a solution of . Since  is non-bounded in , observe that it contains infinitely many points with  and so with . Therefore   has infinitely many self-intersections. \qed


The above results are illustrated in the following examples.

{\bf Example 1.} Let  be parametrized in polar coordinates by . The plotting of this curve for  is shown in Figure \ref{ej1}. One may see that  is not bounded; so, there might be infinitely many self-intersections. In this case we get  and .
Finally, it is easy to see that  is bounded in , but  is not. So, from Theorem \ref{th-4} we conclude that
 has infinitely many self-intersections. In fact, one may see that all these self-intersections lay on the -axis.

\begin{figure}[ht]
\begin{center}
\centerline{ \psfig{figure=PolarCurvesplot2d1.eps,width=5cm,height=5cm} }
\end{center}
\caption{:  non-bounded, infinitely many self-intersections}\label{ej1}
\end{figure}


{\bf Example 2.} Let  be parametrized in polar coordinates by . Again
 is not bounded. However, the plotting of this curve for , shown in Figure
\ref{ej2}, suggests that the curve has not infinitely many self-intersections (in fact, it has no self-intersections at all). Let us check this by using
Theorem \ref{th-4}. We get
 and . Both curves are clearly bounded in ; therefore, there are at most finitely many
self-intersections.

\begin{figure}[ht]
\begin{center}
\centerline{ \psfig{figure=PolarCurvesplot2d5.eps,width=5cm,height=5cm} }
\end{center}
\caption{:  non-bounded, finitely many self-intersections}\label{ej2}
\end{figure}

{\bf Example 3.} Let  be parametrized by . Since
 is bounded, from Theorem \ref{th-3} it follows that there are at most finitely many
self-intersections. Furthermore,   and
, both bounded in . So, by Theorem \ref{th-4} we derive the same
conclusion. The plotting of the curve for  is shown in Figure \ref{ej3}.

\begin{figure}[ht]
\begin{center}
\centerline{ \psfig{figure=PolarCurvesplot2d4.eps,width=5cm,height=5cm} }
\end{center}
\caption{:  bounded.}\label{ej3}
\end{figure}

\subsection{Limit Circles, Limit Points and Spiral Branches}\label{subsec-limit}

Figure \ref{clpl} shows the plotting of the curves defined by  for  (left), and  for  (right). In the first case, one sees that the curve winds infinitely around
the circle , coming closer and closer to it. In the second case, the curve winds
infinitely around the origin of coordinates, somehow ``converging" to it. Finally,  Figure
\ref{ej2} shows a curve that winds infinitely around the origin, but getting further and further from it.  We will refer to the first situation by saying that here the curve exhibits a {\sf limit circle} (
in this example).
In the second case, we will say that the origin is a {\sf limit point}; finally, we will refer to the third situation by saying
that the curve presents a {\sf spiral branch}. Along this subsection, we address these phenomena from a theoretical point of view and relate them to the appearance of infinitely many self-intersections.

\begin{figure}[ht]
\begin{center}
\centerline{}
\end{center}
\caption{Example of Limit Circle (left) and Limit Point  (right)}\label{clpl}
\end{figure}



\begin{definition} \label{def-limit-spiral}
Let . We say that  exhibits, for :
\begin{enumerate}
\item  A {\sf limit circle} if , , and .
\item  A {\sf limit point} at the origin if , and .
\item   A {\sf spiral branch}t if , and 
\end{enumerate}
In each case, we will say that  {\it generates} the limit circle, the limit point or the spiral branch.
\end{definition}

Notice that limit points are degenerated cases of limit circles, namely when . One might wonder if limit circles can be centered
at a point different from the origin, or if there can be limit points other than the origin. The answer, negative in both cases, is given by the
following theorem.

\begin{theorem} \label{no-limit}
 A curve rational in polar coordinates cannot have a limit circle centered at a point different from the origin, or a limit point at a point
different from the origin.
\end{theorem}

{\bf Proof.} If  had a limit circle centered at , or a limit point , we would have infinitely
many local maxima and minima of . However, this cannot happen because  is a rational function,
and therefore the number of -values fulfilling  is finite. \qed



\begin{remark} \label{rem-2}  The same argument of Theorem \ref{no-limit} proves that
a curve rational in polar coordinates cannot have any other ``attractor"
different from the origin, or a circle centered at the origin.
\end{remark}

 Note that if  has infinitely many self-intersections, by Theorem \ref{th-3} the function  is not bounded and so there must be limit points, limit circles or spiral branches. More concretely, we have the following result.  


\begin{proposition} \label{limit-self-int}
Let  be a subset of  (not necessarily an interval) with infinitely many -values generating self-intersections of
. Then the closure of  contains some -value giving rise to either a limit point, or a limit circle, or a spiral branch of .
\end{proposition}



The converse of Proposition \ref{limit-self-int} is not true; for instance, in Example 2 we have a curve with a spiral branch for
 without self-intersections. Now we want to characterize the situation when a limit point, a limit circle or a spiral branch have infinitely many close self-intersections. First we need the following definition.

\begin{definition} \label{close-inf}
We say that a limit circle, a limit point or a spiral branch generated by
 has {\sf infinitely many close self-intersections}, if there exists some real interval  verifying:
\begin{enumerate}
\item ;
\item   does not contain any other -value generating a limit circle, limit point or spiral;
\item  infinitely many  generate self-intersections of .
\end{enumerate}
\end{definition}

Definition \ref{close-inf} is illustrated in Figure \ref{imcs}; the thin lines correspond to the branch generated by the interval  appearing in Definition \ref{close-inf}.

\begin{figure}[ht]
\begin{center}
\centerline{}
\end{center}
\caption{Infinitely many close self-intersections}\label{imcs}
\end{figure}




\begin{theorem} \label{th-self-assoc}
Let  generating a limit circle, a limit point or a spiral branch. Then
\begin{enumerate}
\item If , then  has infinitely many close self-intersections if and only if  is a horizontal asymptote of some of the curves  or .
\item  If , then it has infinitely many close self-intersections if and only if some of the curves  or  exhibits an infinite branch as  which is not  a horizontal
asymptote (i.e. there exists a sequence of real points  of the curve with  and ).
\end{enumerate}
\end{theorem}

{\bf Proof.} We prove (1); the proof of (2) is similar. Assume that there exists an interval 
satisfying Definition \ref{close-inf}.
Then the set of points of either  or  with  must be non-bounded in .
This can only happen if there exists  such that  is an asymptote of either  or . However, if 
is an asymptote then for any interval  containing , and not containing any other -value where  is infinite, we can find
a non-bounded portion of either  or , therefore giving rise to infinitely many self-intersections of ,
with -values in . So, from Proposition \ref{limit-self-int} we have that . Using this same argument we can prove the converse
statement, i.e. if  is an asymptote then any interval containing it has the desired properties. \qed

The detection of asymptotes of an algebraic curve is addressed for example in \cite{Zeng}.

\begin{corollary} \label{corol-self-assoc}
If  has infinitely many close self-intersections, then any interval containing  generates infinitely many self-intersections
of .
\end{corollary}

The above results are illustrated in the following examples.

{\bf Example 1 (cont.):} Recall that  and . The second one has an infinite
branch  as ; so, from Theorem \ref{th-self-assoc} we deduce that every -interval of the form  or 
contains infinitely many -values generating self-intersections of .

{\bf Example 4.} Consider the curve defined by . This
curve has a spiral branch for  and a limit point for . Observe
Figure \ref{la7} where the curve is plotted for different values of . A direct computation yields  and




The curve  is
empty over the reals. However a factor of  corresponds to a hyperbola whose asymptotes are  and . Hence, from
statement (1) of Theorem \ref{th-self-assoc} we deduce that the spiral branch
generated by  has infinitely many self-intersections.



\begin{figure}[ht]
\begin{center}
\centerline{}
\end{center}
\caption{ }\label{la7}
\end{figure}





\subsection{Asymptotes}\label{as}

It is classical (see for example \cite{DD}) that asymptotes in polar form correspond to values  such that:
\begin{itemize}
\item .
\item .
\item .
\end{itemize}
If the above conditions hold, then the asymptote is the line parallel to
the line with slope , at distance  of the origin.

{\bf Example 5.} Consider the curve . Then we have that ,  and . So, the line passing through
the origin and with slope 1, i.e. , is an asymptote of the curve for  (see Figure \ref{f8}).

\begin{figure}[ht]
\begin{center}
\centerline{ \psfig{figure=Asymptoteplot2d1.eps,width=5cm,height=5cm} }
\end{center}
\caption{An asymptote}\label{f8}
\end{figure}

\section{Visualizing Curves that are Rational in Polar Coordinates}

The goal of this section is to introduce an algorithm,  {\tt polares},  for visualizing the ``interesting" part of  , providing as well relevant information about the curve. We have implemented it in Maple 15 and tested it over several examples, some of which are presented in this section.

Next we describe the algorithm in the following different cases:
\begin{enumerate}
  \item  and  are both bounded
  \item  is bounded and  is not.
  \item   is bounded and  is not.
  \item   and  are both unbounded.
\end{enumerate}

Obviously, the first step of the algorithm is to detect the case we are in; then the algorithm proceeds accordingly.

\subsection{Algorithm {\tt polares} when  and  bounded}

{\tt Input:} A proper polar parametrization . \newline
{\tt Output:}
 \begin{enumerate}
  \item Information about the existence of .
  \item Information about the self-intersections.
  \item Plot  of  for  in  using the Maple command \texttt{polarplot}.
\end{enumerate}

Examples:

\begin{itemize}
  \item 
\end{itemize}

\texttt{\textcolor{red}{polares([]);}}\newline
\texttt{\textcolor{blue}{
 r and theta both bounded \newline
 Real point at the infinity such that (r, theta)=[0, 1] and the point is [0, 0] \newline
The point at infinity (0,0) is reached 1 times in R, so self-intersection at the origen}}

\begin{figure}[ht]
\begin{center}
\centerline{ \psfig{figure=ej1_todoacotado.eps,width=5cm,height=5cm} }
\end{center}
\caption{}\end{figure}
\begin{itemize}
  \item 
  \end{itemize}


  \texttt{\textcolor{red}{polares([]);}}\newline
\texttt{\textcolor{blue}{  r and theta both bounded \newline
Real point at the infinity such that (r, theta)=[0, 1] and the point is [0, 0] \newline
The point at infinity (0,0) is reached 1 times in R, so self-intersection at the origen \newline
System (1) gives self-intersections for  k in [[ -2,2]], k0\newline
System (2) gives self-intersections for k in [[ -2,1]]}}


  \begin{figure}[ht]
\begin{center}
\centerline{ \psfig{figure=ej2_todoacotado.eps,width=5cm,height=5cm} }
\end{center}
\caption{}
\end{figure}



\subsection{Algorithm {\tt polares} when  is bounded and  not}\label{3.2}

{\tt Input:} A proper polar parametrization . \newline
{\tt Output:}
 \begin{enumerate}
  \item Information about the existence of .
  \item Information about the self-intersections.
  \item Information about the existence of asymptotes.
  \item \label{plot}Plot of  using the Maple command \texttt{polarplot}:
  \begin{enumerate}
  \item If there are not values of  generating asymptotes, then we plot the curve for ;

  \item Otherwise, let  be the set of values of  generating asymptotes, let  be the set of real values of  such that  and, if  for , let  be the real values of  generating the maximum and minimum of .
\newline
Let   with .
  \newline
Now,

\begin{enumerate}
  \item If , then  and its plot may be of interest. So we add
  and  to the set . 
 \newline 
For every , we plot the curve   for   . For   in  , the color of the plot will be red and for  the color will be blue. 
  \newline
As for  and  , we plot  in the ranges  and  .
  \newline
  As for  and  , we plot  in the ranges  and  .


\item  If , then for every , we plot the curve  in the range  . For  in  , the color of the plot will be red and for  the color will be blue.
\newline
As for  and  , we plot   for the ranges  and  .
\newline
Finally, as for , we plot   for the ranges   and  .



\end{enumerate}





\end{enumerate}
Let us point out that we border the values of  generating the asymtotes for plotting.

\end{enumerate}

Examples:

\begin{itemize}
  \item 
\end{itemize}

\texttt{\textcolor{red}{polares([]);}}\newline
\texttt{\textcolor{blue}{
r unbounded and theta bounded \newline
Real point at the infinity such that (r, theta)=[1, 1] and the point is [cos(1), sin(1)]\newline
Point at infinity is not reached with k=0\newline
Point at infinity is not reached with k0\newline
System (1) gives self-intersections for  k in [[ -2,2]], k0\newline
The values of t generating asymptotes are  \newline
Values of t considered in the plot }}




 \begin{figure}[ht]
 
 { \psfig{figure=ej1_1_tethaacot.eps,width=5cm,height=4cm} }&
 { \psfig{figure=ej1_2_tethaacot.eps,width=5cm,height=4cm} }&
 { \psfig{figure=ej1_3_tethaacot.eps,width=5cm,height=4cm} }\\
 { \psfig{figure=ej1_4_tethaacot.eps,width=5cm,height=4cm} }&
 { \psfig{figure=ej1_5_tethaacot.eps,width=5cm,height=4cm} }&
  \end{array}
\begin{array}{ccc}
 { \psfig{figure=ej2_1_tethaacot.eps,width=5.5cm,height=4cm} }&
 { \psfig{figure=ej2_2_tethaacot.eps,width=5.5cm,height=4cm} }&
 { \psfig{figure=ej2_3_tethaacot.eps,width=5.5cm,height=4cm} }
\end{array}

 \begin{array}{ccc}
 { \psfig{figure=ej1_1_racot.eps,width=5.5cm,height=4cm} }&
 { \psfig{figure=ej1_2_racot.eps,width=5.5cm,height=4cm} }&
 { \psfig{figure=ej1_3_racot.eps,width=5.5cm,height=4cm} }
 \end{array}
 
 \begin{array}{ccc}

\caption{ }
\label{15}
\end{figure}
\section{Conclusions/Further Work}

In this paper we have presented several theoretical results and an algorithm for properly plotting curves parametrized by rational functions in polar form. Our results allow to algorithmically identify phenomena which are typical of these curves, like the existence of infinitely many self-intersections, spiral branches, limit points or limit circles. Furthermore, the algorithm
has been implemented in Maple 15, and provides good results. Natural extensions of the study here are space curves which are rational in spherical or cylindrical coordinates, curves which are algebraic, although non-necessarily rational, in polar coordinates (i.e. fulfilling , with  algebraic), and similar phenomena for the case of surfaces. It would be also interesting to analyze the curves defined by (implicit) expressions
of the type , where  is algebraic, since this class contains, and in fact extends, the class of algebraic curves; also, it includes the important subclass of curves defined by equations , with  a rational function, which often appear in Geometry and Physics. Some of these questions will be explored in the future.






\begin{thebibliography}{56}













\bibitem{JGDT} Alcazar J.G., Diaz-Toca G. (2010) {\it Topology of 2D and 3D Rational Curves}, Computer Aided Geometric Design, vol. 27, pp.
483-502.



\bibitem{Andradas} Andradas C., Recio T. (2007) {\it Missing points and branches of real parametric curves}, Applicable Algebra in
Engineering and Computing 18 (1-2), pp. 107-126





\bibitem{Buga} Bugayevskiy L.M., Snyder J. (1995). {\it Map Projections: A Reference Manual}, CRC Press, Taylor and Francis Group.





\bibitem{Cheng} Cheng J., Lazard S., Pearanda L., Pouget M., Rouillier F., Tsigaridas E. (2010), {\it On the topology of real algebraic plane curves}, Mathematics and Computer Science Volume 4 (1), pp. 113-137.



\bibitem{DD} Doneddu A. {\it Analyse et Geometrie Diferentiele}, Dunod Ed. (1970).



\bibitem{Eigen} Eigenwilling A., Kerber M., Wolpert N. (2007) {\it Fast and Exact Geometric Analysis of Real Algebraic Plane Curves}, in C.W. Brown, editor, Proc. Int. Symp. Symbolic and Algebraic Computation, pp. 151-158, Waterloo, Canada. ACM.

    \bibitem{Emel} Emeliyanenko P., Berberich E., Sagraloff M. (2009), {\it Visualizing Arcs of Implicit Algebraic Curves, Exactly and Fast}, Proceedings of
Advances in Visual Computing : 5th International Symposium, ISVC 2009, Lecture Notes in Computer Science, pp. 608-619.










\bibitem{Lalo} Gonz\'alez-Vega L., Necula I. (2002).
{\it Efficient topology determination of implicitly defined
algebraic plane curves}, Computer Aided Geometric Design, vol. 19
pp. 719-743.

\bibitem{Hong} Hong H. (1996). {\it An effective
method for analyzing the topology of plane real algebraic curves},
Math. Comput. Simulation 42 pp. 571-582 


\bibitem{Jong} Jong T., Pfister G. (2000). {\it Local Analytic Geometry}, Ed. Vieweg, Advanced Lectures on Mathematics.












\bibitem{seidel} Seidel R., Wolpert N. (2005) {\it On the Exact Computation of the
Topology of Real Algebraic Curves}. Proc. of the 21st Ann. ACM Symp. on Comp. Geom. (SCG 2005). ACM, 2005 107--115.









\bibitem{SWPD} Sendra J.R., Winkler F., P\'erez-D\'{\i}az P. (2008). {\it Rational Algebraic Curves}, Springer-Verlag.

\bibitem{Sny} Snyder, J. P. (1997). {\it Flattening the earth: two thousand years of map projections}, University of Chicago Press





\bibitem{Wi96} Winkler F. (1996), {\it Polynomial Algorithms in Computer Algebra}. Springer Verlag, ACM Press.

\bibitem{Zeng} Zeng G. (2007), {\it Computing the asymptotes for a real plane algebraic curve},
Journal of Algebra, Vol. 316, Issue 2, pp. 680-705.

\end{thebibliography}





\end{document}
