\documentclass[journal]{IEEEtran}
\usepackage{graphicx}
\graphicspath{{figures/}}
\usepackage{balance}
\usepackage{subfigure}
\usepackage{amssymb}
\usepackage{epsfig}
\newcommand{\dd}[1]{{\tfrac{\partial}{\partial #1}}}
\newcommand{\dfd}[2]{\frac{\partial #1}{\partial #2}}
\newcommand{\dtwod}[1]{\frac{\partial^2}{\partial #1^2}}
\newcommand{\dtwodd}[2]{\frac{\partial^2}{\partial #1 \partial #2}}
\newcommand{\dndd}[3]{\frac{\partial^{#3}}{\partial {#1} \cdots \partial {#2}}}
\newcommand{\dtwofdd}[3]{\frac{\partial^2 #1}{\partial #2 \partial #3}}
\newcommand{\Var}[2]{\mbox{Var}_{#1}\left[ {#2} \right]}
\newcommand{\Cov}[1]{\mbox{Cov}\left( {#1} \right)}
\newcommand{\E}[2]{E_{#1}\left[ {#2} \right]}
\newcommand{\I}[1]{\mathrm{I}_{#1}}
\newcommand{\PR}[1]{P\left[{#1}\right]}
\newcommand{\ie}{{\it i.e.}}
\newcommand{\eg}{{\it e.g.}}
\newcommand{\argmin}[1]{\mathop{\mbox{argmin}}_{#1}}
\newcommand{\argmax}[1]{\mathop{\mbox{argmax}}_{#1}}
\newcommand{\argminX}{\argmin{ X: \; X^TX=I_D}}
\newcommand{\Order}[1]{\mathcal{O}\left( {#1} \right)}
\newcommand{\ds}[1]{\mathrm{d}{#1}}
\newcommand{\dv}[1]{\mathrm{d}\Vc{#1}}
\newcommand{\ol}[1]{\overline{#1}}
\newcommand{\tr}{\mbox{tr}}
\newcommand{\me}{\mathrm{e}}
\newcommand{\mR}{\mathbb{R}}
\newcommand{\vone}{\Vc{1}}
\newcommand{\vzero}{\Vc{0}}
\newcommand{\vI}{\Vc{I}}
\newcommand{\vbeta}{\boldsymbol{\beta}}
\newcommand{\mX}{\mathcal{X}}
\newcommand{\mY}{\mathcal{Y}}
\newcommand{\mM}{\mathcal{M}}
\newcommand{\diag}[1]{{\mbox{diag}\left\{{#1}\right\}}}
\newcommand{\bb}[0]{\bar{b}}
\newcommand{\bs}[0]{\bar{\sigma}}
\newcommand{\mbA}[0]{\mathbf{A}}
\newcommand{\mbB}[0]{\mathbf{B}}
\newcommand{\mbC}[0]{\mathbf{C}}
\newcommand{\mbD}[0]{\mathbf{D}}
\newcommand{\mbE}[0]{\mathbf{E}}
\newcommand{\mbF}[0]{\mathbf{F}}
\newcommand{\mbG}[0]{\mathbf{G}}
\newcommand{\mbH}[0]{\mathbf{H}}
\newcommand{\mbI}[0]{\mathbf{I}}
\newcommand{\mbJ}[0]{\mathbf{J}}
\newcommand{\mbK}[0]{\mathbf{K}}
\newcommand{\mbL}[0]{\mathbf{L}}
\newcommand{\mbM}[0]{\mathbf{M}}
\newcommand{\mbN}[0]{\mathbf{N}}
\newcommand{\mbO}[0]{\mathbf{O}}
\newcommand{\mbP}[0]{\mathbf{P}}
\newcommand{\mbQ}[0]{\mathbf{Q}}
\newcommand{\mbR}[0]{\mathbf{R}}
\newcommand{\mbS}[0]{\mathbf{S}}
\newcommand{\mbT}[0]{\mathbf{T}}
\newcommand{\mbU}[0]{\mathbf{U}}
\newcommand{\mbV}[0]{\mathbf{V}}
\newcommand{\mbW}[0]{\mathbf{W}}
\newcommand{\mbX}[0]{\mathbf{X}}
\newcommand{\mbY}[0]{\mathbf{Y}}
\newcommand{\mbZ}[0]{\mathbf{Z}}
\newcommand{\mba}[0]{\mathbf{a}}
\newcommand{\mbb}[0]{\mathbf{b}}
\newcommand{\mbc}[0]{\mathbf{c}}
\newcommand{\mbd}[0]{\mathbf{d}}
\newcommand{\mbe}[0]{\mathbf{e}}
\newcommand{\mbf}[0]{\mathbf{f}}
\newcommand{\mbg}[0]{\mathbf{g}}
\newcommand{\mbh}[0]{\mathbf{h}}
\newcommand{\mbi}[0]{\mathbf{i}}
\newcommand{\mbj}[0]{\mathbf{j}}
\newcommand{\mbk}[0]{\mathbf{k}}
\newcommand{\mbl}[0]{\mathbf{l}}
\newcommand{\mbm}[0]{\mathbf{m}}
\newcommand{\mbn}[0]{\mathbf{n}}
\newcommand{\mbo}[0]{\mathbf{o}}
\newcommand{\mbp}[0]{\mathbf{p}}
\newcommand{\mbq}[0]{\mathbf{q}}
\newcommand{\mbr}[0]{\mathbf{r}}
\newcommand{\mbs}[0]{\mathbf{s}}
\newcommand{\mbt}[0]{\mathbf{t}}
\newcommand{\mbu}[0]{\mathbf{u}}
\newcommand{\mbv}[0]{\mathbf{v}}
\newcommand{\mbtv}[0]{\mathbf{\tilde{v}}}
\newcommand{\mbw}[0]{\mathbf{w}}
\newcommand{\mbx}[0]{\mathbf{x}}
\newcommand{\mby}[0]{\mathbf{y}}
\newcommand{\mbty}[0]{\mathbf{\tilde{y}}}
\newcommand{\mbz}[0]{\mathbf{z}}
\newcommand{\mbzero}[0]{\mathbf{0}}
\newcommand{\mbalpha}[0]{{\boldsymbol{\alpha}}}
\newcommand{\mbdelta}[0]{{\boldsymbol{\delta}}}
\newcommand{\mbepsilon}[0]{{\boldsymbol{\epsilon}}}
\newcommand{\mbmu}[0]{{\boldsymbol{\mu}}}
\newcommand{\mbPi}[0]{{\boldsymbol{\Pi}}}
\newcommand{\mbpi}[0]{{\boldsymbol{\pi}}}
\newcommand{\mbtheta}[0]{{\boldsymbol{\theta}}}
\newcommand{\mbphi}[0]{{\boldsymbol{\phi}}}


\newcommand{\Definition}[2]{\vspace{0.1in} \noindent    {\begin{minipage}{\linewidth}\underline{\bf Def'n:}  {\it #1}\\
#2\end{minipage}}\vspace{0.1in} \\ }
\newcommand{\Lemma}[2]{\vspace{0.1in}\\ \noindent  {\begin{minipage}{\linewidth}\underline{\bf Lemma:}  {#1}\\
{\bf Proof:} #2 \end{minipage}}\vspace{0.1in} \\ }
\newcommand{\Theorem}[2]{\vspace{0.1in} \noindent \begin{minipage}{\linewidth}\underline{\bf Theorem:}
{#1} \\ {\bf Proof:} {#2} \end{minipage}\vspace{0.1in} \\ }
\newcommand{\StartOf}[1]{{\noindent \rule{\linewidth}{1pt}\\ \noindent \large \bf #1}\\}
\newcommand{\Today}[1]{{\noindent \large Today: #1}\\}
\newcommand{\Example}[1]{{\vspace{0.15in} \noindent \bf Example: #1}\\}
\newcommand{\Note}[1]{{\vspace{0.05in} \noindent {\bf Note:} #1}\\}
\newcommand{\IF}[1]{\,\,\,\, \mbox{if} \,\,\, #1}
\newcommand{\pdfarray}[2]{{ \left\{\begin{array}{ll} {#1}, & {#2} \\ 0, & o.w. \end{array}\right. }}
\newcommand{\pdfarrays}[4]{{ \left\{\begin{array}{ll} {#1}, & {#2} \\ {#3}, & {#4} \\ 0, & o.w. \end{array}\right. }}
\newcommand{\twooptions}[4]{{ \left\{\begin{array}{ll} {#1}, & {#2} \\ {#3}, & {#4} \end{array}\right. }}
\newcommand{\CDFarrays}[4]{{ \left\{\begin{array}{ll}
   0,    & {#3} < {#2} \\
   {#1}, & {#2} \le {#3} < {#4} \\
   1, & {#3} \ge {#4} \end{array}\right. }}
\newcommand{\CDFarray}[3]{{ \left\{\begin{array}{ll}
   0,    & {#3} < {#2} \\
   {#1}, & {#3} \ge {#2} \end{array}\right. }}
\newcommand{\CCDFarrays}[4]{{ \left\{\begin{array}{ll}
   1,    & {#3} < {#2} \\
   {#1}, & {#2} \le {#3} < {#4} \\
   0, & {#3} \ge {#4} \end{array}\right. }}
\newcommand{\CCDFarray}[3]{{ \left\{\begin{array}{ll}
   1,    & {#3} < {#2} \\
   {#1}, & {#3} \ge {#2} \end{array}\right. }}
\newcommand{\Vector}[1]{\left[ \begin{array}{c} #1 \end{array} \right]}
\newcommand{\MatTwoCols}[1]{\left[ \begin{array}{cc} #1 \end{array} \right]}
\newcommand{\MatThreeCols}[1]{\left[ \begin{array}{ccc} #1 \end{array} \right]}
\newcommand{\PP}[0]{\vspace{0.11in}\noindent}
\newcommand{\Fourier}[1]{{\mathfrak{F}\left\{ {#1}\right\}} }
\newcommand{\IFourier}[1]{{\mathfrak{F}^{-1}\left\{ {#1}\right\}} }
\newcommand{\DTFT}[1]{{\mathrm{DTFT}\left\{ {#1}\right\}} }
\newcommand{\IDTFT}[1]{{\mathrm{DTFT}^{-1}\left\{ {#1}\right\}} }
\newcommand{\rect}[0]{\mbox{rect}}
\newcommand{\sinc}[0]{\mbox{sinc}}
\newcommand{\Summary}[1]{{\vspace{0.15in} \noindent \bf Summary of today's lecture}: #1\\}
\newcommand{\rank}[0]{\mbox{rank}}
\newcommand{\atantwo}[2]{{\mbox{atan2} \left({#1}, {#2} \right) }}
\newcommand{\erf}[0]{\mbox{erf}}
\newcommand{\erfc}[0]{\mbox{erfc}}

\newcommand{\En}[0]{\mathcal{E}}
\newcommand{\Ebno}[0]{\frac{\mathcal{E}_b}{N_0}}
\newcommand{\Question}[0]{{\vspace{0.15in} \noindent \bf Question:  }}
\newcommand{\dB}[0]{\mbox{(dB)}}
\newcommand{\dBm}[0]{\mbox{(dBm)}}
\newcommand{\dBW}[0]{\mbox{(dBW)}}
\newcommand{\intinfty}[2]{{\int_{#1 = -\infty}^{\infty} #2  d{#1} }}
\newcommand{\decision}[2]{{{#1 \atop >} \atop {< \atop #2}}}
\newcommand{\Q}[1]{{\mbox{Q}\left( {#1} \right)}}
\newcommand{\Qinv}[1]{{\mbox{Q}^{-1}\left( {#1} \right)}}
\newcommand{\sgn}[1]{{\mbox{sgn}\left\{ {#1} \right\}}}
\newcommand{\entropy}[1]{H\left[ {#1} \right]}
\newcommand{\stdPM}[0]{(P,\mathcal{F}, \Omega)}
\newcommand{\apriori}[0]{\emph{a priori} }
\newcommand{\nnn}[0]{\nonumber \\ }
\newcommand{\nn}[0]{\nonumber }
\newcommand{\sifi}[0]{-field}
\newcommand{\communicateswith}[0]{{\leftrightarrow }}
\newcommand{\MatFourCols}[1]{\left[ \begin{array}{cccc} #1 \end{array} \right]}
\newcommand{\MatFiveCols}[1]{\left[ \begin{array}{ccccc} #1 \end{array} \right]}
\newcommand{\MatSixCols}[1]{\left[ \begin{array}{cccccc} #1 \end{array} \right]}
\newcommand{\MatSevenCols}[1]{\left[ \begin{array}{ccccccc} #1 \end{array} \right]}
\newcommand{\MatEightCols}[1]{\left[ \begin{array}{cccccccc} #1 \end{array} \right]}
\newcommand{\MatNineCols}[1]{\left[ \begin{array}{ccccccccc} #1 \end{array} \right]}
\newcommand{\limin}[1]{\lim_{{#1}\rightarrow \infty}}

\newcommand{\multiplefloor}[2]{{\left\lfloor {#1} \right\rfloor_{#2} }}
\newcommand{\multipleceil}[2]{{\left\lceil {#1} \right\rceil_{#2} }}
 
\title{Through-Wall Motion Tracking Using Variance-Based Radio Tomography Networks}
\author{Joey Wilson and Neal Patwari\\
Sensing and Processing Across Networks (SPAN) Lab\\
University of Utah\\
Salt Lake City, UT, USA\\
joey.wilson@utah.edu,  npatwari@ece.utah.edu}


\begin{document}

\maketitle

\begin{abstract}
This paper presents a new method for imaging, localizing, and tracking motion behind walls in real-time. The method takes advantage of the motion-induced variance of received signal strength measurements made in a wireless peer-to-peer network. Using a multipath channel model, we show that the signal strength on a wireless link is largely dependent on the power contained in multipath components that travel through space containing moving objects. A statistical model relating variance to spatial locations of movement is presented and used as a framework for the estimation of a motion image. From the motion image, the Kalman filter is applied to recursively track the coordinates of a moving target. Experimental results for a 34-node through-wall imaging and tracking system over a 780 square foot area are presented.
\end{abstract}



\section{Introduction}

This paper explores a method for tracking the location of people and objects moving behind walls, without the need for an electronic device to be carried by or attached to the target. The technology is an extension of ``radio tomographic imaging'' \cite{Wilson09a}, which is so-called because of its analogy to medical tomographic imaging methods.  We call this extension \textit{variance-based radio tomographic imaging (VRTI)}, since it uses the signal strength variance caused by moving objects within a wireless network.  The general field of locating people or objects when they don't carry a device is also called ``device-free passive localization'' \cite{Youssef07} in contrast to technologies like active radio frequency identification (RFID) which only locate objects that carry a radio transmitter.

In a mission-critical application, we envision a building imaging scenario similar to the following.  Emergency responders, miltary forces, or police arrive at a scene where entry into a building is potentially dangerous.  They deploy radio sensors around (and potentially on top of) the building area, either by throwing or launching them, or dropping them while moving around the building.  The nodes immediately form a network and self-localize, perhaps using information about the size and shape of the building from a database (\eg, Google maps) and some known-location coordinates (\eg, using GPS).  Then, nodes begin to transmit, making signal strength measurements on links which cross the building or area of interest.  The RSS measurements of each link are transmitted back to a base station and used to estimate the positions of moving people and objects within the building.

Radio tomography provides life-saving benefits for emergency responders, police, and military personnel arriving at potentially dangerous situations. Many correctional and law enforcement officers are injured each year because they lack the ability to detect and track offenders through building walls \cite{Hunt01}.  By showing the locations of people within a building during hostage situations, building fires, or other emergencies, radio tomography can help law enforcement and emergency responders to know where they should focus their attention.

This paper explores the use of radio tomography in highly obstructed areas for the purpose of tracking moving objects through walls. First, a review of previous work and related research is summarized in Section \ref{section.related}. In Section \ref{section.VRTI}, we address a fundamentally different method for the use of RSS measurements which we call \textit{variance-based radio tomography (VRTI)}.  As the name implies, rather than use measurements of the change in mean of a link's RSS, measurements of the variance of the link's RSS values are used.  When a moving object affects the amplitude or phase of one or more multipath components over time, the phasor sum of all multipath at the receiver experiences changes, and higher RSS variance is observed. The amount of RSS variance relates to the physical location of motion, and an image representing motion is estimated using measurements from many links in the wireless network.

We briefly review the Kalman filter and apply it in Section \ref{section.kalman} to track the location of a moving object or person. In Section \ref{section.experiment}, experimental results demonstrate the use of RSS variance to locate a moving object on the inside of a building. This section also quantifies the accuracy of localization by comparing known movement paths with those estimated by the VRTI tracking system. We show that the VRTI system can track the location of an experimenter behind walls with approximately three feet average error for this experiment.

Finally, Section \ref{section.futureResearch} discusses some possibilities for future research. Advances in wireless protocols, antenna design, and physical layer modeling will bring improvements to VRTI through-wall tracking.

\section{Related Research}\label{section.related}

Previous work shows that changes in link path losses can be used to accurately estimate an image of the attenuation field, that is, a spatial plot of attenuation per unit area \cite{Wilson09a}.  Experimental tests show that in an unobstructed area surrounded by a network of nodes, the estimated image displayed the positions of people in the area.

Indoor radio channel characterization research demonstrates that objects moving near wireless communication links cause variance in RSS measurements \cite{Bultitude87}. This knowledge has been applied to detect and characterize motion of network nodes and moving objects in the network environment \cite{Woyach06}. Polarization techniques have also been used to detect motion \cite{Pratt08}. These studies focus mostly on detection and velocity characterization of movement, but do not attempt to localize the movement as the work presented in this paper does.

Youssef, Mah, and Agrawala \cite{Youssef07} demonstrated that variance of RSS on a number of WiFi links in an indoor WLAN can be used to (1) detect if motion is occuring within a wireless network, and (2) localize the moving object based on a manually trained lookup. In most emergency situations, however, manual training is not possible since it can take a significant amount of time and access to the area being tracked is restricted.

Real-time location systems (RTLS) are based on a technology that uses electronic tags for locating objects. For logistics purposes in large facilities, commercial real-time location systems are deployed by installing infrastructure in the building and attaching active radio frequency identificaiton (RFID) tags to each object to be tracked. RTLS systems are not useful in most emergency operations, however, since they require setup inside of a building prior to system use.  Further, RTLS systems cannot locate people or objects which do not have an RFID tag.  In emergencies, an operation cannot rely on an adversary wearing a tag to be located.  Thus, tag-based localization methods are insufficient for most emergency operations.

An alternate tag-free localization technology is ultra-wideband (UWB) through-wall imaging (TWI) (also called through-the-wall surveillance).  In radar-based TWI, a wideband phased array steers a beam across space and measures the delay of the reflection response, estimating a bearing and distance to each target. Through-wall radar imaging has garnered significant interest in recent years  \cite{Aryanfar04,Lin05,Lin06,Song05,Vertiy04}, for both static imaging and motion detection. Commercial products include Cambridge Consultants' Prism 200 \cite{Cambridge} and Camero Tech's Xaver800 \cite{Camero}, and are prohibitively expensive for most applications, on the order of US \d1/d^41/d^2\Order{N^2}N\nuV_i\Phi_iiR_{dB} = 10 \log_{10} \| \tilde{V}\|^2i\mathcal{S}_i\mathcal{S}_i\mbz \in \mathcal{S}_i\Phi_i\mathcal{S}_i\mbz \in \mathcal{S}_i\Phi_i[0, 2\pi)\mathcal{Z}i\mbz \in \mathcal{S}_i\mbz \in \mathcal{Z}i\mbz \notin \mathcal{S}_i\mbz \in \mathcal{Z}\mathcal{T} = \{i:\mbz \notin \mathcal{S}_i \forall \mbz \in \mathcal{Z}\}\bar{V} = \left|\sum_{i \in \mathcal{T}} V_i \exp\left( j\Phi_i\right) \right|\bar{\Phi} = \angle \sum_{i \in \mathcal{T}} V_i \exp\left( j\Phi_i\right)\mathcal{T}V_{ns,I}V_{ns,Q}V_{ns,I}V_{ns,Q}\sigma^2_{\nu}\sigma^2\tilde{V}R = |\tilde{V}|RI_0(\cdot)R^2R_{dB} = 20 \log_{10} RR_{dB}c = \frac{\log 10}{20}R_{dB}KK_{dB}KK\sigma^2\bar{V}^2K_{dB}K_{dB}K_{dB}K_{dB}KK_{dB}<-2KKK_{dB}>10^2K_{dB}=16KM\textbf{x}N\textbf{x}jnw_{j}j\textbf{s}M\times1\textbf{W}N \times M\textbf{n}M\times1\textbf{x}N \times 1w_j\textbf{W}d_{lj}(1) + d_{lj}(2) < d_l + \lambdad_ld_{lj}(1)d_{lj}(2)jl\psi\lambdaT_slR_{dB_l}(kT_s)kT_slN_B\hat{s}_ll\textbf{Q}\alpha\textbf{D}_x\textbf{D}_y\hat{\textbf{s}}\Pi\hat{\textbf{s}}\upsilon^2_m\upsilon^2_n\textbf{c}\textbf{z}\bar{\textbf{P}}\textbf{P}\textbf{G}\textbf{c}=(0,0)\textbf{P}=\textbf{I}_2\textbf{I}_2\bar{\textbf{P}} = \textbf{P} + \upsilon^2_m \textbf{I}_2\textbf{G}= \bar{\textbf{P}}(\bar{\textbf{P}}+ \upsilon^2_n \textbf{I}_2)^{-1}\textbf{z}\textbf{c} = \textbf{c} + \textbf{G}(\textbf{z}-\textbf{c})\textbf{P} = (\textbf{I}_2-\textbf{G}) \bar{\textbf{P}}\Delta_p\lambda\delta_c\sigma_x^2(dB)^2\alpha\psi(dB)^2N_B(17.6,21.3)\upsilon^2_m = .01\upsilon^2_n = 5\epsilon = 3.37\upsilon^2_m = .0001\upsilon^2_n = 5\epsilon = 6.53\upsilon^2_m = .01\upsilon^2_n = 5\epsilon = 3.37\upsilon^2_m = .0001\upsilon^2_n = 5\epsilon = 6.53\upsilon_m\upsilon^2_n = 5Lz_x[k]z_y[k]kp_x[k]p_y[k]\upsilon^2_m = .01\upsilon^2_n = 53.37\upsilon_m\upsilon^2_m=.01\upsilon^2_n = 5\zeta = 1.46\epsilon_pp\upsilon^2_m=.01\upsilon^2_n = 51.46$ feet.

\section{Future Research}\label{section.futureResearch}

Many areas of future research are possible to improve VRTI through-wall tracking technology. First, large and scalable VRTI networks capable of tracking entire homes and buildings need to be explored. This will require advanced wireless networking protocols that can measure the RSS of each link quickly when the number of nodes is high. Perhaps frequency hopping and grouping of nodes will allow a VRTI system to measure each link's RSS while maintaining a low delay in delivering the measurements to a base station.

Advancements on the physical layer modeling will allow VRTI systems to track movement more accurately, and with less nodes.  In this paper, an ellipsoid model is used to relate RSS variance on a link to the locations of movements.  This is certainly an approximation, and future work will require the refinement of the variance weighting model, thus leading to more accurate motion images and coordinate tracking. Other regularization and image estimation techniques may also improve through-wall tracking.

Radio devices could be designed specifically for VRTI tracking applications. The affect of overall node transmission power on imaging performance is an important area to be investigated. Directional and dual-polarized antenna designs would most likely improve images in a through-wall VRTI system. Radio devices capable of sticking to an exterior wall and directively transmitting power into the structure would be extremely useful in emergency deployment and multi-story VRTI.

Finally, localization of nodes plays a significant role in tracking of motion with VRTI networks. In an emergency, rescue or enforcement teams will not have time to survey a location. With automatic node self-localization techniques, the nodes could be thrown or randomly placed around an area and locate themselves without human moderation, saving valuable time.

\section{Conclusion}
Locating interior movement from outside of a building is extremely valuable in emergency situations, enabling police, military forces, and rescue teams to safely locate people prior to entering.  Variance-based radio tomography is a powerful new method for through-wall imaging that can be used to track the coordinates of moving objects. The cost of VRTI hardware is very low in comparison to existing through-wall imaging systems, and a single network is capable of tracking large areas. These features may enable many new applications that are otherwise impractical.

This paper discusses how RSS variance relates to the power contained in multipath components affected by moving objects. The variance of RSS is related to the location of movement relative to node locations, and this paper provides a formulation to estimate a motion image based on variance measurements. The Kalman filter is applied as a mechanism for tracking movement coordinates from image data. A 34-node VRTI experiment is shown to be capable of tracking a moving object through typical home exterior walls with an approximately 3ft average error. An object moving in place can be located with approximately 1.5ft average error.

The experiments presented in this paper demonstrate the theoretical and practical capabilities of VRTI for tracking motion behind walls. Many avenues for future research are presented which may improve image accuracy and enable larger and faster VRTI networks. These future research areas include wireless protocols, antenna design, radio channel modeling, localization, and image reconstruction.

\section*{Acknowledgements}
This material is based upon work supported by the National Science Foundation under the Early Career Faculty Development (CAREER) Grant No. ECCS-0748206. Any opinions, findings, and conclusions or recommendations expressed in this material are those of the authors and do not necessarily reflect the views of the National Science Foundation.


\bibliographystyle{ieeetr}
\bibliography{refs}

\end{document}
