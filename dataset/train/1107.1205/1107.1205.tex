\documentclass[copyright,creativecommons,sharealike]{eptcs}
\providecommand{\event}{QAPL 2011} \usepackage{breakurl}             

\usepackage{cite}
\usepackage{mparhack}
\usepackage[latin1]{inputenc}
\usepackage[T1]{fontenc}
\usepackage{amsmath,amsxtra,amssymb,stmaryrd}
\usepackage{bbm,mathtools}
\usepackage{xspace}

\usepackage{tikz}
\usetikzlibrary{decorations}
\usetikzlibrary{automata}
\usetikzlibrary{shapes.multipart}

\usepackage{complexity}
\usepackage[normalem]{ulem}
\usepackage{url}

\usepackage[amsmath,amsthm,thmmarks,thref]{ntheorem}
\theoremstyle{plain} 
\newtheorem{theorem}{Theorem}
\newtheorem{definition}[theorem]{Definition}
\newtheorem{proposition}[theorem]{Proposition}
\newtheorem{lemma}[theorem]{Lemma}
\newtheorem{conjecture}[theorem]{Conjecture}
\newtheorem{corollary}[theorem]{Corollary} 
\theorembodyfont{\upshape}

\newcommand*\Strat{\Theta}
\newcommand*\Stratblind{\tilde \Strat}
\newcommand*\stratA{\Strat_1}
\newcommand*\stratB{\Strat_2}
\renewcommand*\K{\mathbbm{K}}
\newcommand*\Real{\mathbbm{R}}
\DeclareMathOperator{\Tr}{Tr} \DeclareMathOperator{\Pa}{Pa} \DeclareMathOperator{\fPa}{fPa} \DeclareMathOperator{\tr}{tr} \DeclareMathOperator{\len}{len} \newcommand*\bigmid{\mathrel{\big|}}
\newcommand*\tto[1]{\xrightarrow{#1}}
\newcommand*\ie{\textit{i.e.}} 
\newcommand*\eg{\textit{e.g.}}
\newcommand*\cf{\textit{cf.}}
\newcommand*\vblind{\tilde v}
\newcommand*\last{\textup{last}} \newcommand{\Round}[1]{\ensuremath{\textup{Round}_{(#1)}}\xspace} 



\begin{document}



\title{Distances for Weighted Transition Systems: \\ Games and
  Properties}

\author{Uli Fahrenberg 
  \institute{IRISA/INRIA\thanks{Most of this work was conducted while this author was still at
      Aalborg University.} \\
    Rennes Cedex \\ France}
  \email{ulrich.fahrenberg@irisa.fr}
  \and
  Claus Thrane \qquad\qquad Kim G.~Larsen
  \institute{
    Department of Computer Science \\ Aalborg University \\
    Denmark}
  \email{\{crt,kgl\}@cs.aau.dk}
 }

\def\titlerunning{Distances for Weighted Transition Systems}
\def\authorrunning{Fahrenberg, Thrane \& Larsen}

\maketitle

\begin{abstract}
  We develop a general framework for reasoning about distances between
  transition systems with quantitative information.  Taking as
  starting point an arbitrary distance on system traces, we show how
  this leads to natural definitions of a linear and a branching
  distance on states of such a transition system.  We show that our
  framework generalizes and unifies a large variety of previously
  considered system distances, and we develop some general properties
  of our distances.  We also show that if the trace distance admits a
  recursive characterization, then the corresponding branching
  distance can be obtained as a least fixed point to a similar
  recursive characterization.  The central tool in our work is a
  theory of infinite path-building games with quantitative objectives.
\end{abstract}

\section{Introduction}

In verification of concurrent and reactive systems, one generally
seeks to assert properties of systems expressed in terms of \emph{sets
  of traces} (or languages) or in terms of \emph{computation trees}.
The language point of view leads to what is generally called
\emph{linear} semantics, whereas the tree point of view leads to
\emph{branching} semantics.  These semantics are the extreme points in
a spectrum containing a number of other useful notions;
see~\cite{Glabbeek01-lbs} for an overview.

As emphasized in \cite{DBLP:conf/fm/HenzingerS06}, working with applications in
complex reactive systems or in embedded systems means that classical
notions of linear and branching equivalence (or inclusion) of
processes often need to be extended to accommodate \emph{quantitative}
information. This can be in relation to real-time behavior, resource
usage, or can be probabilistic or stochastic information.
In such a quantitative setting, equivalences and inclusions are
replaced by symmetric or asymmetric \emph{distances} between systems.

This approach of \emph{quantitative analysis} has been taken in
numerous papers by multiple authors, both in the real-time (or
hybrid), in the probabilistic, and in general quantitative settings,
see~\cite{DBLP:conf/qest/DesharnaisLT08,conf/concur/CernyHR10,conf/csl/ChatterjeeDH08,conf/icalp/AlfaroFS04,conf/icalp/AlfaroHM03,journals/tcs/DesharnaisGJP04,FahrenbergLT10,Giacalone90,conf/formats/2005/HenzM05,journals/jlap/ThraneFL10,DBLP:conf/stoc/Kozen83,axiomat}
for a (non-exhaustive) choice of references.  Indeed, the quantitative
approach is also useful in settings \emph{without} quantitative
information in the models, \eg~in~\cite{conf/concur/CernyHR10} various
distances related to implementation correctness of discrete systems
are considered.

The above-mentioned dichotomy between languages and trees persists in
the quantitative setting, where one hence encounters both notions of
\emph{linear} and of \emph{branching} distances.  To the best of our
knowledge, the treatment of those distances, and of the relations
between them, has so far been somewhat ad hoc.  Indeed, the general
approach appears to be to introduce some particular distances which
are relevant for a particular application and then show some useful
properties; in this paper, we try to unify and generalize some of
these approaches.

The present paper is in a sense a follow-up to previous
papers~\cite{FahrenbergLT10,journals/jlap/ThraneFL10} by the same
authors.  In those papers, we introduce and investigate three
different linear and branching distances.  A paper similar in spirit
to these is~\cite{conf/icalp/AlfaroFS04}, which analyses properties of
what we later will call the \emph{point-wise} distance for weighted
Kripke structures.  The starting point for the present paper is then
the observation of similarities between the constructions for
different types of distances, which we here generalize to encompass
all of them and to construct a coherent framework.

In this paper, we take the view that in practical applications, say in
reactive systems, the \emph{system distance} which measures adherence
to the property which we want to verify, will be specific to the
concrete domain of the application. Hence in a general framework like
the one proposed here, its description must be given as an
\emph{input}. A method to obtain the actual system distances, for some
desired level of interaction, is then prescribed by the framework.

In this paper we assume that this system distance input is given as a
\emph{distance on traces}: Given two sequences of executions, one
needs only to define what it means for these sequences to be closely
related to each other.  We show that such a \emph{trace distance}
always gives rise to natural notions both of linear and of branching
distance.

To relate linear and branching distances, we introduce a general
notion of \emph{simulation game with quantitative objectives}.  The
idea of using games for linear and branching equivalences is not
new~\cite{DBLP:conf/banff/Stirling95} and has been used in a
quantitative setting
\eg~in~\cite{conf/csl/ChatterjeeDH08,DBLP:conf/qest/DesharnaisLT08},
but here we explore this idea in its full generality.

One interesting result which we can show in our general framework is
that for all interesting trace distances, the corresponding linear and
branching distance are \emph{topologically inequivalent}.  From an
application point of view this means that corresponding linear and
branching distances (essentially) always measure very different things
and that results about one of them cannot generally be transferred to
the other.  This result -- and indeed also its proof -- is a
generalization of the well-known fact that language inclusion does not
imply simulation to a quantitative setting.

We also show that for the common special case that the trace distance
has a recursive characterization, the associated branching distance
can be obtained as a least fixed point to a similar recursive
characterization.  This is again a generalization of some standard
facts about simulation, but shows that for a large class of branching
distances, characterizations as least fixed points are available.

\subsection*{Acknowledgment}

The authors acknowledge interesting and fruitful discussions on the
topic of this work with Tom Henzinger, Pavol {\v C}ern{\'y} and Arjun
Radhakrishna of IST Austria.

\section{From Trace Distances to System Distances}

Our object of study in this work are general -weighted transition
systems (to be defined below), where  is some set of weights.  For
applications,  may be further specified and admit some extra
structure, but below we just assume  to be some finite or infinite
set.

\begin{definition}
  A \emph{trace} is an infinite sequence  of elements in .  The set of all such traces is
  denoted .
\end{definition}

Note that we confine our study to \emph{infinite} traces; this is
mostly for convenience, to avoid issues with finite traces of
different length.  All our results are valid when also finite traces
are allowed and the definitions changed accordingly.  We write
 for the th element in a trace , and 
for the trace obtained from  by deleting elements 
up to .

\begin{definition}
  A \emph{-weighted transition system} (WTS) is a pair 
  of sets  with .
\end{definition}

We use the familiar notation  to indicate that .  Note that  and  may indeed be infinite, also
infinite branching.  For simplicity's sake we shall follow the common
assumption that all our WTS are \emph{non-blocking}, \ie~that for any
state  there is a transition  in .

A \emph{path from } in a WTS  is an infinite
sequence  of
transitions in .  The set of such is denoted .  We will
in some places also need \emph{finite} paths, \ie~finite sequences
 of transitions; the set
of finite paths from  is denoted .  For a finite path
 as above, we let  denote its length and  its last state.  We write  for the th state and  for the th weight in a finite
or infinite path.

A path  gives
rise to a trace .  The set of
(infinite) \emph{traces from } is denoted .

\subsection{Interlude: Hemimetrics}
\label{se:metrics}

Before we proceed, we recall some of the notions regarding asymmetric
metrics which we will be using.  First, a \emph{hemimetric} on a set
 is a function  which satisfies  for all  and the triangle inequality  for all .

We will have reason to consider two different notions of equivalence
of hemimetrics. Two hemimetrics ,  on  are said to be
\emph{Lipschitz equivalent} if there are constants 
such that

for all .  Lipschitz equivalent hemimetrics are hence
dependent on each other; intuitively, a property using one hemimetric
can always be approximated using the other.

Another, weaker, notion of equivalence of hemimetrics is the
following: Two hemimetrics ,  on  are said to be
\emph{topologically equivalent} if the topologies on  generated by
the open balls , for , , and , coincide.  Topological equivalence hence
preserves topological notions such as convergence of sequences: If a
sequence  of points in  converges in one hemimetric, then
it also converges in the other.

It is a standard fact that Lipschitz equivalence implies topological
equivalence.  From an application point-of-view, topological
equivalence is interesting for showing \emph{negative} results;
proving that two hemimetrics are not topologically equivalent can be
comparatively easy, and implies that intuitively, the two hemimetrics
measure very different properties.

\subsection{Examples of Trace Distances}
\label{sec:trace_distances}

The framework we are proposing in this article takes as starting point
a \emph{trace distance} defined on executions of a weighted automaton,
\ie~a hemimetric .
In this section we introduce a number of different such trace
distances, to show that the framework is applicable to a variety of
interesting examples.

\paragraph{Discrete trace distances.}
\label{ex:lts}

The discrete trace distance on  is defined by  if  and 
otherwise.  Hence only equality or inequality of traces is measured;
we shall see below that this distance exactly recovers the usual
Boolean framework of trace inclusion and simulation.

If  comes equipped with a preorder  indicating that a label  may be replaced by any
 with , as \eg~in~\cite{Thomsen87}, then we
may refine the above distance by instead letting  if  for all  and  otherwise.  We will see later that using this trace
distance, we exactly recover the \emph{extended simulation}
of~\cite{Thomsen87}; note that something similar is done
in~\cite{DBLP:journals/dke/MedeirosAW08}. 

\paragraph{Hamming distance.}

If one defines a metric  on  by  if  and  otherwise, then the sum  for
any pair of finite traces ,  of equal length is
precisely the well-known Hamming distance~\cite{Hamming50}.  For
infinite traces, some technique can be used for providing finite
values for infinite sums; two such techniques are to use \emph{limit
  average} or \emph{discounting}.  We can hence define the
limit-average Hamming distance by , and for a fixed
discounting factor , the discounted Hamming distance
by .

Note that this approach can easily be generalized to other
(hemi)metrics  on ; indeed the discrete trace distances from
above can be recovered using  if  and  otherwise.

\paragraph{Labeled weighted transition systems.}
\label{ex:wlts}

A common example of weighted
systems~\cite{DBLP:conf/formats/BouyerFLMS08,conf/concur/CernyHR10,conf/csl/ChatterjeeDH08,journals/jlap/ThraneFL10,DBLP:conf/concur/Breugel05}
has  where  is a discrete set of
labels.  Hence  has  as
discrete component and  as real weight.  A useful trace
distance for this type of systems is the \emph{point-wise distance},
see~\cite{conf/icalp/AlfaroFS04,journals/jlap/ThraneFL10}, given by
 if  for all  and  otherwise.
This measures the biggest individual difference between  and
.

Another interesting trace distance in this setting is the
\emph{accumulated distance}~\cite{journals/jlap/ThraneFL10}, where
individual differences in weights are added up.  Again one can use
limit average or discounting for infinite sums; limit-average
accumulating distance is defined by

and discounted accumulating distance, for a fixed , by

This is indeed a generalization of the Hamming distance above, setting
 if  and  otherwise.

Also of interest is the \emph{maximum-lead distance}
from~\cite{conf/formats/2005/HenzM05}, where the individual weights
are added up and one is concerned with the maximal difference between
the accumulated weights.  The definition is


\subsection{Simulation Games}
\label{se:wsg}

In this central section we introduce the game which we will use to
define both the linear and the branching distances.  We shall use some
standard terminology and constructions from game theory here; for a
good introduction to the subject see \eg~\cite{Ferguson}.

Let  be a weighted transition system with  and  a trace
distance. Using  as a game graph, the \emph{simulation game} played
on  from  is an infinite turn-based two-player game,
where we denote the strategy space of Player~ by  and the
utility function of Player~1 by .  As usual  determines the pay-off of Player~1; we will
not use pay-offs for Player~2 here.

The game moves along transitions in  while building a pair of paths
extending from  and , according to the strategies of the
players.  In the terminology
of~\cite{DBLP:journals/tcs/Bodlaender93,DBLP:journals/tcs/FraenkelS93}
we are playing a \emph{partisan path-forming game}.

A \emph{configuration} of the game is a pair of finite paths  (\ie~the history) which are
consecutively updated by the players as the game progresses.
The players must play according to a strategy of the following types: 
\begin{itemize}
\item , the set of
  mappings from pairs of finite paths to transitions, with the
  additional requirement that for all  and ,  for some , .  This
  is the set of Player-1 strategies which observe the complete
  configuration.
\item Similarly,  with the
  additional requirement that for all  and ,  for some , .
\item , the set of \emph{blind}
  Player-1 strategies which cannot observe the moves of Player~2.
  (Blind Player-2 strategies can be defined similarly, but we will not
  need those here.)  It is convenient to identify  with
  the subset of  of all strategies  which satisfy
   for all
  , .
\item In the proof of Proposition~\ref{prop:hemi} we will also need
  Player-2 strategies with additional memory.  Such a strategy is a
  mapping , where 
  is a set used as memory.
\end{itemize}

Given a game with configuration , a \emph{round} is
played, according to a \emph{strategy profile} (\ie~a pair of
strategies) , by
first updating  according to  and then updating the
resulting configuration according to .  Hence we define

where  denotes sequence concatenation.

A strategy profile 
inductively determines an infinite sequence  of configurations given by  and  for .  The
paths in this sequence satisfy ,
where  denotes prefix ordering, hence the limits ,

exist (as infinite paths).
We define the utility function  as

This determines the pay-off to Player~1 when the game is played
according to strategies , .  Note again that the
utility function for Player~2 is left undefined; especially we make no
claim as to the game being zero-sum.

The \emph{value} of game on  from  is defined to be
the optimal Player-1 pay-off

Observe that the game is \emph{asymmetric}; in general, .  

A strategy  for Player~1 is said to be \emph{optimal}
if it realizes the supremum above, \ie~whenever . The strategy is
called \emph{-optimal} for some  provided that
.  Note that -optimal strategies always
exist for any , whereas optimal strategies may not.

We also recall that the game is said to be \emph{determined} if the
sup and inf above can be interchanged, \ie~if .  Intuitively, the game is determined if there also exist
-optimal Player-2 strategies for any  which
realize the value of the game (up to ) independent of the
strategy Player~1 might choose.

The \emph{1-blind value} of the game is defined to be


\subsection{Example: Discrete Trace Distance}
\label{se:discrete}

It may be instructive to apply the above simulation game in the
context of the discrete trace distance  if
,  otherwise, from
Section~\ref{sec:trace_distances}.  In this case, the game has value
 if and only if, for every  there
exist a  which in each round  of the
game, in configuration (), maps
 whenever
.  Otherwise,
.

Hence we have  if  simulates  in the sense
of~\cite{milner89}, and  otherwise.  In other
words, for discrete trace distance the game reduces to the standard
simulation game of~\cite{DBLP:conf/banff/Stirling95}.

Likewise, the blind value  if and only if every
 and corresponding path  has a
match  where configuration (),
of round  facilitates  whenever .  Hence  if ; we recover standard trace inclusion.

\subsection{Linear and Branching Distance}

We can now use the game introduced in Section~\ref{se:wsg} to define
linear and branching distance:

\begin{definition}
  \label{de:branch}
  Let  be a WTS and .
  \begin{itemize}
  \item The \emph{linear distance} from  to  is the 1-blind
    value .
  \item The \emph{branching distance} from  to  is the value
    .
  \end{itemize}
\end{definition}

We proceed to show that the distances so defined are hemimetrics on
, \cf~the proof of Theorem~1 in~\cite{conf/concur/CernyHR10}.
For linear distance, this also follows from
Theorem~\ref{thm:dl-formula} below, and we only include the proof for
reasons of exposition.  For branching distance, we have to assume in
the proof below that the simulation game is \emph{determined};
currently we do not know whether this assumption can be lifted.

\begin{proposition}
  \label{prop:hemi}
  Linear distance  is a hemimetric on , and if the simulation
  game is determined, so is .
\end{proposition}

\begin{proof}
  Non-negativity of  and  follow directly from the
  non-negativity of .  To prove that 
  for all , given any strategy , we
  construct a Player-2 strategy  that mimics
   and attains the game value of , as follows:
  
  It can be seen easily that the paths constructed by both players are
  the same, \ie~ for some . Therefore,  as  is a hemimetric,
  whence .

  We are left with showing that  and  obey the triangle
  inequality.  For linear distance, let  and write
   () for the set of (blind)
  Player- strategies in the simulation game computing , for  and .  Let
  .
  It might be beneficial to look at Figure~\ref{fig:strategychase} to
  see the ``chase of strategies'' we will be conducting.

  \begin{figure}[tp]
    \centering
    \begin{tikzpicture}[shorten >=1pt, auto, initial text=, scale=.2]
      \node (s1) {\normalsize };
      \node (pi113) [above of=s1] {};
      \node (pi112) [below left of=s1] {};
      \node (s2) [below right of=s1, xshift=3em, yshift=-3em]
      {\normalsize };
      \node (pi212) [below left of=s2] {};
      \node (pi123) [below right of=s2] {};
      \node (s3) [above right of=s2, xshift=3em, yshift=3em]
      {\normalsize };
      \node (pi213) [above of=s3] {};
      \node (pi223) [below right of=s3] {};
      \draw (s1) -- (s2);
      \draw (s1) -- (s3);
      \draw (s2) -- (s3);
      \draw [->] (pi113) to [out=210, in=90] (pi112);
      \draw [->] (pi112) to [out=270, in=170] (pi212);
      \draw [->] (pi212) to [out=350, in=190] (pi123);
      \draw [->] (pi123) to [out=10, in=270] (pi223);
      \draw [->] (pi223) to [out=90, in=330] (pi213);
    \end{tikzpicture}
    \caption{\label{fig:strategychase} Construction of strategies in the
      proof of Proposition~\ref{prop:hemi}}
  \end{figure}

  Choose a blind Player-1 strategy .  Blind strategies correspond to choosing a
  path, so let  be the path chosen by . This path in turn corresponds to a blind Player-1 strategy
  .

  Let  be a Player-2 strategy
  for which .  Write  for the path
  constructed by the strategy profile , and let  be a
  blind Player-1 strategy which constructs .

  Let  ensure .  Write
   for the path constructed by the strategy
  profile , and let  be a strategy which constructs .
  For the strategy profile  in
  , the paths constructed are  and
  .  Hence we have
  

  As  was chosen arbitrarily,
  we have
  
  and as also  was chosen arbitrarily, .

  \smallskip For branching distance, we cannot construct the paths in a one-shot
  manner as above, as the transitions chosen by Player~1 may depend on
  the history of the play.  Let again ; assuming that the
  simulation game is determined, we can choose Player-2 strategies
  ,  for which  and
  .  Intuitively, we will use
  these strategies to allow Player~2 to find replying moves to
  Player-1 moves in the game computing  by using the
  reply given by  to the reply given by .  Hence we still follow the proof strategy depicted in
  Figure~\ref{fig:strategychase}, but now for individual moves.

  The strategy  uses a finite path  as memory and is defined by
  
  with memory update .  The initial memory for 
  is set to be the empty path, hence as the game progresses, a path
   is constructed.  

  Now choose some , and let
   and  be the paths
  constructed by the strategy profile .  If  is the corresponding memory
  path, then the pair  is constructed by the strategy
  profile  and the pair  by the profile .  Hence we can use the exact same reasoning as
  in~\eqref{eq:triangle} to conclude that
  
  and hence .
\end{proof}

\subsection{Properties}

The following general result confirms that, regardless of the trace
distance chosen, the linear distance is always bounded above by the
branching distance.  In the context of the \emph{discrete} trace
distance from Section~\ref{sec:trace_distances}, this specializes to
the well-known fact that simulation implies language inclusion.

\begin{theorem}
  \label{thm:bound}
  For any , we have .
\end{theorem}

\begin{proof}
  Any Player-1 strategy in  is also in , hence
  
\end{proof}

The game definition of linear distance yields the following explicit
formula.  Note the resemblance of this to the well-known Hausdorff
construction for lifting a metric on a set to its set of subsets.

\begin{theorem}
  \label{thm:dl-formula}
  For all  we have
  
\end{theorem}

\begin{proof}
  The definition of  immediately entails the fact that
  for any  there exists  such that
  . It remains to show
  that .
  By blindness, any  produces a unique path
   independent of the opponent strategy
  .  Hence we need only consider strategies
   which define a single path  from , and the
  result follows.
\end{proof}

We finish this section by exposing two properties regarding
\emph{equivalence} of the introduced hemimetrics.  Transferring
(in)equivalence of distances from one setting to another is an
important subject, and we hope to show other results of the below
kind, especially relating trace distance to branching distance, in
future work.

\begin{proposition}[\cf~{\cite[Thm.~3.87]{aliprantis2007infinite}}]
  If trace distances  and  are Lipschitz equivalent,
  then the corresponding linear distances  and  are
  topologically equivalent.
\end{proposition}

\begin{proof}
  This follows immediately from Theorem~\ref{thm:dl-formula} and
  Theorem~3.87 in~\cite{aliprantis2007infinite}.  Note that the
  theorem in~\cite{aliprantis2007infinite} actually is stronger; it is
  enough to assume  and  to be \emph{uniformly
    equivalent}.
\end{proof}

The next theorem shows that if a trace distance can be used for
measuring trace differences beyond the first symbol (which will be the
case except for some especially trivial trace distances), then the
corresponding linear and branching distances are topologically
inequivalent.  The proof is an easy adaption of the standard argument
for the fact that language inclusion does not imply simulation.

\begin{definition}
  A trace distance  is said to be
  \emph{one-step indiscriminate} if  implies  for all .
\end{definition}

\begin{proposition}
  \label{pr:wts-ineq}
  If  is not one-step indiscriminate, then there exists a
  weighted transition system  on which the corresponding distances
   and  are topologically inequivalent.
\end{proposition}

\begin{proof}
  Let  such that , , and .   is depicted in
  Figure~\ref{fi:wts-ineq}.

  \begin{figure}[tp]
    \centering
    \begin{tikzpicture}[shorten >=1pt, auto, initial text=, scale=.2]
      \begin{scope}[outer sep=1pt,minimum size=5pt,inner sep=2pt, node
        distance=2.25cm]
        \node (v0) {}; 
        \node (v1) [below of=v0, yshift=.6cm] {};
        \node (v2) [below right of=v1] {};
        \node (v3) [below left of=v1] {};
        \node (v4) [below of=v3] {};
        \node (v5) [below of=v2] {};
      \end{scope}
      \begin{scope}[->]
        \draw (v0) -- node[left]  {} (v1);
        \draw (v1) -- node[above right, xshift=-.1cm] {} (v2);
        \draw (v1) -- node[above left, xshift=.2cm]  {} (v3);
        \draw[dashed] (v3) -- node[left] {} (v4);
        \draw[dashed] (v2) -- node[right] {}(v5);
      \end{scope}
    \end{tikzpicture}     
    \qquad
    \begin{tikzpicture}[->,shorten >=1pt, auto, node distance=.75cm,
      initial text=]
      \begin{scope}[outer sep=1pt,minimum
        size=5pt,inner sep=2pt, node distance=2.25cm] 
        \node (u0) {};        
        \node (u1) [below left of=u0]{};
        \node (u2) [below right of=u0]{};
        \node (u3m) [below of=u1,yshift=.6cm] {}; \node (u3) [below of=u3m] {};
        \node (u4m) [below of=u2,yshift=.6cm] {}; \node (u4) [below of=u4m] {};    
      \end{scope}
      \begin{scope}
        \draw (u0) -- node[above,xshift=-.2cm]  {} (u1);
        \draw (u0) -- node[above,xshift=.2cm] {} (u2);
        \draw (u2) -- node[right] {} (u4m);
        \draw (u1) -- node[left] {} (u3m);
        \draw (u3m)[dashed] -- node[left] {} (u3);
        \draw (u4m)[dashed] -- node[right] {} (u4);
      \end{scope}
    \end{tikzpicture}
    \caption{\label{fi:wts-ineq}Weighted transition system for the proof of
      Proposition~\ref{pr:wts-ineq}}
  \end{figure}
  
  We have , hence .  On the other
  hand, .  As two equivalent hemimetrics have value  at
  the same set of pairs of points, this finishes the proof.
\end{proof}

Also note that if  and  are cyclic, the construction can
be adapted to yield a \emph{finite} WTS .

\section{Recursively Defined Distances}
\label{sec:special}

The game definition of branching distance in
Definition~\ref{de:branch} gives a useful framework, but it is not
very operational.  In this section we show that if the given trace
distance has a recursive characterization, then the corresponding
branching distance can be obtained as the least fixed point of a
similar recursive formula.

We give the fixed-point theorem first and show in
Section~\ref{se:applications} below that the theorem covers all
examples of distances introduced earlier.

\begin{theorem}
  \label{th:F-recursion}
  Let  be a complete lattice and , ,  such that
  ,  is monotone,  is
  monotone for all , and
  
  for all .  Define  by
  
  Then  has a least fixed point , and .
\end{theorem}

Let us give some intuition about the theorem before we prove it.  Note
first the composition , where  maps pairs of traces
to the lattice  which will act as \emph{memory} in the applications
below.  Equation~\eqref{eq:F-recursion} then expresses that  acts
as a \emph{distance iterator function} which, within the lattice
domain, computes the trace distance by looking at the first elements
in the traces and then iterating over the rest of the trace.  Under
the premises of the theorem then, branching distance is the projection
by  of the least fixed of a similar recursive function involving
.

\begin{proof}
  It is not difficult to show that  indeed has a least fixed point:
  The lattice  with partial order defined point-wise
  by  iff  for all 
  is complete, and  is monotone because of the monotonicity
  condition on , hence Tarski's Fixed-point Theorem can be applied.

  To show that , we pull back  along : With
  the notation for the simulation game from Section~\ref{se:wsg},
  define
  
  We have  by monotonicity of , so it will suffice
  to show that .

  Let us first prove that  is a fixed point for : Let , then
  
  (the last step by the monotonicity assumption on ; note that in
  the second sup-inf pair, strategies from  and  are
  considered).  By the recursion formula~\eqref{eq:F-recursion} for
  , we end up with
  
  Now because of independence of choices, we can rewrite this to
  
  and collapsing the sup-sup and inf-inf into one sup and inf,
  respectively, conclude .

  To show that  is the least fixed point for , let  such that ; we want to prove
  .  Note first that for all  and all , there is  such that .  Now fix  and let ; we will be done once we can construct a Player-2 strategy
   for which .

  We have to define  for configurations  in which  and .  Write ; by the note above, we can choose a transition  for which ,
  so we let .  The
  so-defined strategy has
  
\end{proof}

\subsection{Applications}
\label{se:applications}

We reconsider here the example trace distances from
Section~\ref{sec:trace_distances} and exhibit the corresponding linear
and branching distances.

\paragraph{Discrete trace distances.}

For the discrete trace distance on  given by  if  and 
otherwise, we saw already in Section~\ref{se:discrete} that we recover
ordinary trace inclusion and simulation.  For linear distance, we can
also use Theorem~\ref{thm:dl-formula} to show that  if
 and  otherwise.

For the branching distance, we can now also apply
Theorem~\ref{th:F-recursion} with ,  the identity
mapping, and  if , 
otherwise.  Then the branching distance is the least fixed point of
the equations , hence  if  simulates  in the standard
sense~\cite{milner89}, and  otherwise.

For the refined discrete trace distance  if
 for all , 
otherwise, we analogously get  if all  can be refined by a  (\ie~ for all ) and  otherwise.  Also,  if there is a relation  for which  and whenever  and , then
also  with  and  (the
\emph{extended simulation} of~\cite{Thomsen87}), and  otherwise.

\paragraph{Hamming distance.}

For Hamming distance induced by the metric  if  and
 otherwise on , linear distance is given by  if and only if any trace  can be matched
by a trace  with Hamming distance at most , both
for the limit-average and the discounting interpretation.  The
branching distance associated with the discounting version is
precisely the (discounted) \emph{correctness distance}
of~\cite{conf/concur/CernyHR10}:  measures ``how often
[the system starting in ] can be forced to cheat'', \ie~to take a
transition different from the one the system starting in  takes.

\paragraph{Labeled weighted transition systems.}

For the trace distances on labeled weighted transition systems, let us
for simplicity assume that , hence .  For the
point-wise trace distance  we can derive a recursive formula for the corresponding
branching distance by applying Theorem~\ref{th:F-recursion} with ,  the identity mapping, and .  Then  is the least fixed point to the equations


For discounted accumulated trace distance , we can similarly let , then the corresponding branching distance is the
least fixed point to the equations

Note that these two branching distances are exactly the ones the
authors define in~\cite{journals/jlap/ThraneFL10}.

For the maximum-lead distance , we need to do
more work.  Intuitively, a recursive formulation needs to keep track
of the accumulated delay, hence needs (infinite) memory.  This can be
accomplished by letting ; the set
of functions from leads to distances.  We can then define

and . Now with , we indeed have that , hence we can
apply Theorem~\ref{th:F-recursion} to conclude that , where  is the least fixed point to the equations

This is precisely the formulation of branching maximum-lead distance
given in~\cite{conf/formats/2005/HenzM05}.

\section{Conclusion and Future Work}

We have shown that simulation games with quantitative objectives
provide a general framework for studying linear and branching
distances for quantitative systems.  Specifically, that our framework covers and unifies a number of
previously distinct approaches, and that certain common special cases
lead to useful recursive characterizations of branching distance.

Already we have seen that one very general property, topological
inequivalence of linear and branching distance, follows almost
immediately from the game characterization. Also this general approach
permits the conclusion that independent of the trace distance, the
branching distance provides an upper bound on the linear distance, a
property which is useful for applications such as analysis of
real-time systems, where linear distances are known to be
uncomputable~\cite{journals/jlap/ThraneFL10}.

It seems likely that by permitting a broader range of strategies, we
may encompass more advanced levels of system interaction and
observations, and hence capture quantitative extensions of other
well-known system relations such as 2-nested
simulation~\cite{DBLP:journals/iandc/GrooteV92,DBLP:conf/stacs/AcetoFI01}
or bisimulation~\cite{milner89}.  Thus, we expect our framework to be
of great use for reasoning about, and applying quantitative
verification.

The game perspective on linear and branching distances also suggests
that several interesting results and properties of games with
quantitative objectives are transferable to our setting.  As an
example, one may consider computability and complexity results: For a
concrete setting such as finite weighted labeled automata, discounted
or limit average accumulating distances can be computed using
discounted and mean-payoff games, respectively.  Hence the complexity
of computing these branching distances is in
.
Similarly, results concerning strategy iteration or value iteration
for games with quantitative objectives may be transferred to the
distance setting.

\bibliographystyle{eptcs}
\bibliography{linearbranchingbib}

\end{document}
