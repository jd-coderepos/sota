\documentclass{article}

\PassOptionsToPackage{numbers, compress}{natbib}




\usepackage[final]{neurips_2020}






\usepackage[utf8]{inputenc} \usepackage[T1]{fontenc}    \usepackage[colorlinks=true,citecolor=blue]{hyperref}
\usepackage{url}            \usepackage{booktabs}       \usepackage{amsfonts}       \usepackage{nicefrac}       \usepackage{microtype}      

\usepackage{graphicx}
\usepackage{subfigure}
\usepackage{amsthm}

\newcommand{\theHalgorithm}{\arabic{algorithm}}

\usepackage{amsmath}
\usepackage{amssymb}
\usepackage{xcolor}
\usepackage{mathtools}
\usepackage{sidecap}
\usepackage{wrapfig}
\usepackage[ruled,vlined,noend]{algorithm2e}

\usepackage{enumitem}
\setlist[itemize,enumerate]{leftmargin=*}

\usepackage{mdframed}
\definecolor{theoremcolor}{rgb}{0.94, 0.94, 0.94}
\definecolor{examplecolor}{rgb}{1, 1, 1.0}
\mdfsetup{
    backgroundcolor=theoremcolor,
    linewidth=0pt,
}

\usepackage{xfrac}
\usepackage{comment}

\def\solution{\mathop{\rm{solution}}}

\newmdtheoremenv[linewidth=0pt,innerleftmargin=4pt,innerrightmargin=4pt]{definition}{Definition}
\newmdtheoremenv[linewidth=0pt,innerleftmargin=4pt,innerrightmargin=4pt]{proposition}{Proposition}
\newmdtheoremenv[linewidth=0pt,innerleftmargin=0pt,innerrightmargin=0pt,backgroundcolor=examplecolor]{example}{Example}
\newmdtheoremenv{corollary}{Corollary}
\newmdtheoremenv{theorem}{Theorem}
\newmdtheoremenv{lemma}{Lemma}

\DeclareMathOperator*{\argmax}{arg\,max}
\DeclareMathOperator*{\argmin}{arg\,min}



\newcommand{\andre}[1]{{\textcolor{blue}{\bf [{\sc Andre:} #1]}}}


\newcommand{\remove}[1]{}



\title{Sparse and Continuous Attention Mechanisms}



\usepackage{soul}
\usepackage{marvosym}
\newcommand\markUnbabel{\Cancer}
\newcommand\markIT{\Leo}
\newcommand\markISR{\Jupiter}
\newcommand\markLUMLIS{\Saturn}
\newcommand\markUvA{\Virgo}

\author{Andr\'e F.~T.~Martins\textsuperscript{\markIT,\markLUMLIS,\markUnbabel} 
  \And
  Ant\'onio Farinhas\textsuperscript{\markIT} 
  \And
  Marcos Treviso\textsuperscript{\markIT} 
  \AND
  Vlad Niculae\textsuperscript{\markUvA,\markIT}\And
  Pedro M.~Q.~Aguiar\textsuperscript{\markISR,\markLUMLIS} 
  \And
  M\'ario A.~T.~Figueiredo\textsuperscript{\markIT,\markLUMLIS} 
  \AND \2ex]
  \textsuperscript{\markIT{}}Instituto de Telecomunica\c{c}\~oes,  Instituto Superior Técnico, Lisbon, Portugal \\
  \textsuperscript{\markISR{}}Instituto de Sistemas e Rob\'otica, Instituto Superior Técnico, Lisbon, Portugal \\
  \textsuperscript{\markLUMLIS{}}LUMLIS (Lisbon ELLIS Unit), Lisbon, Portugal \\
  \textsuperscript{\markUvA{}}Informatics Institute, University of Amsterdam, The Netherlands \\
  \textsuperscript{\markUnbabel{}}Unbabel, Lisbon, Portugal
}




\begin{document}
\maketitle
\begin{abstract}
Exponential families are widely used in machine learning; they include many distributions in continuous and discrete domains (\textit{e.g.}, Gaussian, Dirichlet, Poisson, and categorical distributions via the softmax transformation). Distributions in each of these families have fixed support. In contrast, for finite domains, there has been recent work on sparse alternatives to softmax (\textit{e.g.} sparsemax and -entmax), which have varying support, being able to assign zero probability to irrelevant categories.
This paper expands that work in two directions: first, we extend -entmax\remove{and Fenchel-Young (FY) losses} to continuous domains, revealing a link with Tsallis statistics and deformed exponential families. 
Second, we introduce continuous-domain attention mechanisms, deriving efficient gradient backpropagation algorithms for .
Experiments on  attention-based text classification, machine translation, and visual question answering  illustrate the use of continuous attention in 1D and 2D, showing that it allows attending to time intervals and compact regions.\remove{and FY-based interval regression on weather data.}
\end{abstract}

\section{Introduction}

Exponential families are ubiquitous in statistics and machine learning \citep{brown1986fundamentals,barndorff2014information}. They enjoy many useful properties, such as the existence of conjugate priors (crucial in Bayesian inference) and 
the classical Pitman-Koopman-Darmois theorem \citep{pitman1936sufficient,darmois1935lois,koopman1936distributions}, which states that,
among families with {\bf fixed support} (independent of the parameters), exponential families are the only having sufficient statistics of fixed dimension for any number of i.i.d.\ samples.

Departing from exponential families, there has been recent work on discrete, finite-domain distributions with {\bf varying and sparse support},  via the \textit{sparsemax} and the \textit{entmax} transformations \citep{Martins2016ICML,blondel2020learning,peters2019sparse}. 
Those approaches drop the link to exponential families of categorical distributions provided by the softmax transformation, which always yields dense probability mass functions. In contrast, sparsemax and entmax can lead to sparse distributions, whose support is not constant throughout the family. This property has been used \remove{to obtain new loss functions for multi-label classification and sparse sequence-to-sequence problems, as well as} to design sparse attention mechanisms with improved interpretability \citep{peters2019sparse,correia2019adaptively}.

However, {sparsemax} and {entmax} are so far limited to discrete domains. Can a similar approach be extended to continuous domains? This paper provides that extension and pinpoints a connection with ``deformed exponential families'' \citep{naudts2009q,sears2010generalized,ding2010t} and Tsallis statistics \citep{Tsallis1988}, leading to {\bf -sparse families} (\S\ref{sec:sparse_families}). 
We use this construction to obtain new density families with varying support, 
including the {\it truncated parabola} and {\it paraboloid} distributions (2-sparse counterpart of the Gaussian, \S\ref{sec:sparsemax} and Fig.~\ref{fig:gaussian_paraboloid}). 

Softmax and its variants are widely used in \textit{attention mechanisms}, an important component of neural networks \citep{bahdanau2014neural}. Attention-based neural networks can ``attend'' to finite sets of objects and identify relevant features. 
We use our extension above to devise new {\bf continuous attention mechanisms} (\S\ref{sec:attention}), which can attend to continuous data streams and to domains that are inherently continuous, such as images. Unlike traditional attention mechanisms, ours are suitable for selecting compact regions, such as 1D-segments or 2D-ellipses. We show that the Jacobian of these transformations are generalized covariances, and we use this fact to obtain efficient backpropagation algorithms (\S\ref{sec:jacobian}). 

As a proof of concept, we apply our models with continuous attention to text classification, machine translation, and visual question answering tasks, with encouraging results (\S\ref{sec:experiments}).

\remove{We also generalize the recent Fenchel-Young (FY) losses \citep{blondel2020learning} 
to arbitrary domains, illustrating their usefulness by estimating sparse continuous densities for regression problems under bounded noise \citep{d2013bounded}. We show that properties of FY losses hold for general -sparse families, including convexity and closed-form gradient w.r.t.\ their canonical parameters. We use this to perform interval regression, which returns mean estimates and intervals, based on the support of their distribution.} 

\begin{comment}
Summing up, the main contributions of this paper are:
\begin{itemize}[nosep]
    \item We extend sparsemax, regularized prediction maps, and FY losses to arbitrary (possibly continuous) densities.
    \item We present -sparse families. For , the densities in these families have varying, sparse supports. Some new distributions are constructed, namely the \textit{truncated parabola} (2-sparse counterpart of the Gaussian).
    \item We develop new attention mechanisms for continuous domains, revealing a connection between their Jacobian and generalized covariances. We use it to obtain an efficient backpropagation algorithm, applying it to document classification and machine translation tasks.
    \item We show that properties of FY losses hold for general -sparse families, including convexity and closed-form gradient w.r.t.\ their canonical parameters. We use this to perform interval regression, which returns mean estimates and intervals, based on the support of their distribution.
\end{itemize}
\end{comment}

\paragraph{Notation.}
Let  be a measure space, 
where  is a set,  is a -algebra, and  is a measure. 
We denote by  the set of -absolutely continuous probability measures. From the Radon-Nikodym theorem \citep[\S31]{halmos2013measure}, each element of  is identified (up to equivalence within measure zero) with a probability density function , with . 
For convenience, we often drop  from the integral. We denote the measure of  as
, and  
the support of a density  as . 
Given , we write expectations 
as
.  Finally, we define . 

\begin{comment}
\paragraph{Notation.}
Let  be a measure space. 
If  is finite, we take   and  the counting measure. If ,  is the Borel -algebra and  the Lebesgue measure. The set of -absolutely continuous probability measures is denoted as , which is a convex set. From the Radon-Nikodym theorem \citep[\S31]{halmos2013measure}, each element of  is identified (up to equivalence within measure zero) with a probability density function , with . We write, with some abuse of notation, .
If the base measure is clear from the context, we often drop  from the integral, writing  instead of , for some .
We denote the measure of  as
, and  
the support of a density  by . 
Given functions  and , we write expectations and covariances  as
 and
.
Finally, we define  if , and , if . 
\end{comment}





\begin{figure*}[t]
\centering
\includegraphics[width=.27\textwidth]{figures/sparse_distributions}\quad
\includegraphics[width=.33\textwidth]{figures/gaussian-crop}\includegraphics[width=.33\textwidth]{figures/paraboloid-crop}\caption{{\bf 1D and 2D distributions generated by the -RPM for .} Left: Univariate location-scale families, including Gaussian and truncated parabola (top) and Laplace and triangular (bottom). Middle and right: Bivariate Gaussian  and truncated paraboloid . \label{fig:gaussian_paraboloid}}
\end{figure*}

\remove{
\begin{figure*}[t]
\centering \includegraphics[width=.8\textwidth]{figures/sparse_distributions}
\caption{{\bf Location-scale families with different -Tsallis regularizations.} Left: Gaussian and truncated parabola distributions with zero mean and different variances. Middle: same for Laplace and triangular distributions. Right: Distributions  for varying . \label{fig:gaussian_laplacian_distributions}}
\end{figure*}
}


\section{Sparse Families}\label{sec:sparse_families}

In this section, we provide background on exponential families and its generalization through Tsallis statistics. We link these concepts, studied in statistical physics, 
to sparse alternatives to softmax recently 
proposed in the machine learning literature \citep{Martins2016ICML,peters2019sparse}, extending 
the latter to continuous domains. 

\subsection{Regularized prediction maps (-RPM)}

Our starting point is the notion of -regularized prediction maps, introduced by \citet{blondel2020learning} for finite domains . 
This is a general framework for mapping vectors in  (\textit{e.g.}, label scores computed by a neural network) into probability vectors in  (the simplex), with a regularizer  encouraging uniform distributions. Particular choices of  recover argmax, softmax \citep{bridle1990probabilistic}, and sparsemax \citep{Martins2016ICML}. Our definition below extends this framework to arbitrary measure spaces , where we assume   is a lower semi-continuous,  proper, and strictly convex function.

\vspace{0.1cm}
\begin{definition}
\label{def:regularized_prediction}
The -\textbf{regularized prediction map} (-RPM)  is defined as

where  is the set of functions for which the maximizer above exists and is unique.
\end{definition}

It is often convenient to consider a ``temperature parameter'' , absorbed into  via . If  has a unique global maximizer , the low-temperature limit yields ,
a Dirac delta distribution at the maximizer of . 
For finite , this is the {\it argmax} transformation shown in \citep{blondel2020learning}. 
Other interesting examples of regularization functionals are shown in the next subsections.

\subsection{Shannon's negentropy and exponential families}

A natural choice of regularizer is the {Shannon's negentropy}, . 
In this case, if we interpret  as an energy function, the -RPM corresponds to the well-known {\it free energy variational principle}, leading to Boltzmann-Gibbs distributions (\citep{cover2012elements}; see App.~\ref{sec:diff_ent_exp_family}): 

where 

is the log-partition function. 
If  is finite and  is the counting measure, the integral in \eqref{eq:boltzmann} is a summation and we can write  as a vector . In this case, the -RPM is the {\it softmax transformation},

If ,  is the Lebesgue measure, and  for  and  (i.e.,  is a positive definite matrix), 
we obtain a {\it multivariate Gaussian}, . 
\remove{with Shannon negentropy .} 
This becomes a univariate Gaussian  if . 
For  and defining , with   and , we get a {\it Laplace} density, . \remove{, with Shannon negentropy .}




\paragraph{Exponential families.} 
Let , where  is a vector of \textit{statistics} and  is a vector of \textit{canonical parameters}. 
A family of the form \eqref{eq:boltzmann} parametrized by  is called an {\it exponential family} \citep{barndorff2014information}. 
Exponential families 
have many appealing properties, such as the existence of conjugate priors and sufficient statistics, and a dually flat geometric structure \citep{amari2016information}. Many well-known distributions are exponential families, including the categorical and Gaussian distributions above, and Laplace distributions with a fixed . 
A key property of exponential families is that the support is constant within the same family and dictated by the base measure : this follows immediately from the positiveness of the  function in \eqref{eq:boltzmann}. We abandon this property in the sequel.



\subsection{Tsallis' entropies and -sparse families}

Motivated by applications in statistical physics,
\citet{Tsallis1988} proposed a generalization of Shannon's negentropy. 
This generalization is rooted on the notions of  -logarithm,  (not to be confused with base- logarithm), and -exponential, : 

Note that , , and  for any . 
Another important concept is that of ``-escort distribution''  \citep{Tsallis1988}: this is the distribution  given by

Note that we have . 

The {\bf -Tsallis negentropy}  \citep{havrda1967quantification,Tsallis1988} is defined as:\footnote{This entropy is normally defined up to a constant, often presented without the  factor. We use the same definition as \citet[\S 4.3]{blondel2020learning} for convenience.} 
Note that , for any , 
with  recovering Shannon's negentropy (proof in App.~\ref{sec:gini_ent_sparse_family}). 
Another notable case is ,
the negative of which is called the Gini-Simpson index \citep{Jost2006,Rao1982}. 
We come back to the  case in \S\ref{sec:sparsemax}.

\remove{
This family is continuous in , \textit{i.e.}, , for any , with  recovering Shannon's negentropy (proof in App.~\ref{sec:gini_ent_sparse_family}). 
Another notable case is ,
the negative of which has several names, {\it e.g.}, Gini-Simpson index \citep{Jost2006} or Rao's quadratic entropy \citep{Rao1982}. 
We will come back to the  case in \S\ref{sec:sparsemax}.
}

For ,  is strictly convex, hence it can be plugged in as the regularizer in  Def.~\ref{def:regularized_prediction}. 
The next proposition (\citep{naudts2009q}; proof in  App.~\ref{sec:gini_ent_sparse_family}) provides an expression for -RPM using the -exponential  \eqref{eq:beta_log_exp}: 





\begin{comment}
\vspace{0.15cm}
\begin{proposition}\label{prop:solution_rpm_tsallis}
For  and , 

where  is a coefficient that ensures normalization.
\end{proposition}
\end{comment}

\vspace{0.15cm}
\begin{proposition}\label{prop:solution_rpm_tsallis}
For  and , 

where 
 is a normalizing function:

\end{proposition}

Let us contrast \eqref{eq:entmax} with Boltzmann-Gibbs distributions \eqref{eq:boltzmann}, recovered with . One key thing to note is that the -exponential, for , can return zero values. Therefore, the distribution  in \eqref{eq:entmax} may not have full support, \textit{i.e.}, we may have .  
We say that  has {\it sparse support} if .\footnote{This should not be confused with sparsity-inducing distributions \citep{FigueiredoNIPS2001,TippingJMLR2001}.} This generalizes the notion of sparse vectors.
 

\paragraph{Relation to sparsemax and entmax.} 
\citet{blondel2020learning}  showed that, for finite ,  -RPM is the {\bf sparsemax} transformation, . 
\remove{(Euclidean projection of   onto the -dimensional probability simplex ).} 
Other values of  were studied by \citet{peters2019sparse}, under the name {\bf -entmax} transformation. For , these transformations have a propensity for returning sparse distributions, where several entries have zero probability. 
Proposition~\ref{prop:solution_rpm_tsallis} shows that similar properties can be obtained when  is continuous. 


\paragraph{Deformed exponential families.} 
With a linear parametrization , 
distributions with the form \eqref{eq:entmax} are called \textit{deformed exponential} or \textit{-exponential families} \citep{naudts2009q,sears2010generalized,ding2010t,matsuzoe2012geometry}. 
\remove{
also referred to as a \textit{-exponential family} \citep{ding2010t} and a \textit{-exponential family} \citep{matsuzoe2012geometry}. 
}
The geometry of these families induced by the Tsallis -entropy was studied by \citet[\S 4.3]{amari2016information}.\footnote{Unfortunately, the literature is inconsistent in defining these coefficients. Our  matches that of \citet{blondel2020learning}; Tsallis'  equals ; this family is also related to Amari's -divergences, but their . Inconsistent definitions have also been proposed for -exponential families regarding how they are normalized; for example, the Tsallis maxent principle leads to a different definition. See App.~\ref{sec:tsallis_maxent} for a detailed discussion.} 
Unlike those prior works, we are interested in the sparse, light tail scenario (), not in heavy tails. For , we call these {\bf -sparse families.} 
When , -sparse families become exponential families and they cease to be ``sparse'', in the sense that all distributions in the same family have the same support.

\begin{comment}
In particular, a 2-sparse family takes the form:

where 
.
Note that  and that
 is the log-partition function (see \eqref{eq:boltzmann}).
\end{comment}








A relevant problem is that of characterizing . When , 
 is the log-partition function (see \eqref{eq:boltzmann}), and  its first and higher order derivatives are equal to the moments of the sufficient statistics. The following proposition (stated as \citet[Theorem~5]{amari2011geometry}, and proved in our App.~\ref{sec:proof_gradient_A}) 
characterizes  for  in terms of an expectation under the -escort distribution for  (see \eqref{eq:escort}). This proposition will be used later to derive the Jacobian of entmax attention mechanisms. \remove{and the gradient and Hessian of Fenchel-Young losses.}


\vspace{0.3cm}

\begin{proposition}\label{prop:gradient_A}
 is a convex function and its gradient is given by

\end{proposition}







\subsection{The 2-Tsallis entropy: sparsemax}\label{sec:sparsemax}

In this paper, we focus on the case . For finite , this corresponds to the sparsemax transfomation proposed by \citet{Martins2016ICML}, 
which has appealing  theoretical and computational properties. 
In the general case, 
plugging  in \eqref{eq:entmax} leads to the -RPM,

\textit{i.e.},  is obtained from  by subtracting a constant (which may be negative) and truncating, where that constant  must be such that . 
\remove{For the discrete case, this constant has been called the \textit{threshold function} by \citet{Martins2016ICML}.}

If  is continuous and  the Lebesgue measure, we call -RPM the {\bf continuous sparsemax} transformation. Examples follow, some of which correspond to  novel distributions.

\paragraph{Truncated parabola.} 
If , we obtain the continuous sparsemax counterpart of a Gaussian, which we dub a ``truncated parabola'':

where  (see App.~\ref{sec:proof_truncated_parabola}). 
\remove{
, 
 and .
}
This function, depicted in Fig.~\ref{fig:gaussian_paraboloid} (top left), is widely used in density estimation. For  and , it is known as the Epanechnikov kernel \citep{epanechnikov1969non}.


\paragraph{Truncated paraboloid.} 
The previous example can be generalized to , with , where , leading to a ``multivariate truncated paraboloid,'' the sparsemax counterpart of the multivariate Gaussian (see middle and rightmost plots in Fig.~\ref{fig:gaussian_paraboloid}):

The expression above, derived in App.~\ref{sec:proof_truncated_paraboloid}, reduces to \eqref{eq:truncated_parabola} for .
Notice that (unlike in the Gaussian case) a diagonal  does not lead to a product of independent truncated parabolas.


\paragraph{Triangular.} 
Setting , with ,  yields the triangular distribution 

where  (see App.~\ref{sec:proof_triangular}). 
\remove{
, and 
}
Fig.~\ref{fig:gaussian_paraboloid} (bottom left) depicts this distribution alongside Laplace.





\paragraph{Location-scale families.} 
More generally, let
 for a location   and a scale  , where
 is convex and continuously differentiable.
Then, we have

where 
and  is the solution of the equation  (a sufficient condition for such solution to exist is  being strongly convex; see App.~\ref{sec:proof_location_scale} for a proof).  
\remove{
The support of this distribution is
. 
}
This example subsumes the truncated parabola () and the triangular distribution (). 


\paragraph{2-sparse families.} 
Truncated parabola and paraboloid distributions form a 2-sparse family, with statistics  and  canonical parameters . 
\remove{
From (\ref{eq:sparsemax}--\ref{eq:truncated_parabola}):

Contrast this expression with the one for Gaussian distributions, which form an exponential family with the same sufficient statistics and canonical parameters, for which .
}
Gaussian distributions form an exponential family with the same sufficient statistics and canonical parameters. 
In 1D, truncated parabola and Gaussians are both particular cases of the so-called ``-Gaussian'' \citep[\S 4.1]{naudts2009q}, for . 
\remove{Fig.~\ref{fig:gaussian_laplacian_distributions} (right) shows -Gaussian distributions for .}
Triangular distributions with a fixed location  and varying scale  also form a 2-sparse family (similarly to Laplace distributions with fixed location being exponential families).

\begin{comment}
\begin{wrapfigure}{r}{0.45\columnwidth}
\centering
\includegraphics[width=.8\textwidth]{figures/sparse_distributions}
\caption{{\bf Location-scale families with different -Tsallis regularizations.} Left: Gaussian and truncated parabola distributions with zero mean and different variances. Middle: same for Laplace and triangular distributions. Right: Distributions  for varying . \label{fig:gaussian_laplacian_distributions}}
\end{wrapfigure}
\end{comment}


\begin{comment}
\subsection{Deformed exponential families and sparse families}\label{sec:deformed_sparse}
Def.~\ref{def:regularized_prediction} is fully general concerning the class of functions  from which  can be chosen. In practice, it is often useful to consider finite-dimensional parametrized function classes. The simplest way to do this is via linear functions , where  is a vector of \textit{statistics} and  a vector of \textit{canonical parameters}. This construction underlies exponential families  \citep{barndorff2014information}. In this section, we present \andre{rephrase?} {\bf sparse families} as an alternative to exponential families. Our definition is related to \andre{rephrase} that of ``deformed exponential families'' \citep{naudts2009q,sears2010generalized}, 
also referred to as -exponential families \citep{ding2010t}, and -exponential families \citep{matsuzoe2012geometry}. 
The geometry of these families induced by the Tsallis -entropy is studied by \citet[\S 4.3]{amari2016information}.\footnote{Unfortunately, there is some inconsistency in the literature about the value of these coefficients. Our  matches that of \citet{peters2019sparse}; Tsallis'  equals ; this family is also related to Amari's -divergences, but their .}
Unlike those prior works, we are interested in the sparse, light tail scenario (), not in heavy tails.


\begin{definition}Let  be a feature mapping. For , a family of probability distributions parametrized by  is called an {\bf -sparse family} if its members can be written in the form

where  is a normalizing function (see App.~\ref{sec:A_alpha}):

\end{definition}

In particular, a 2-sparse family takes the form:

where 
.
Note that  and that
 is the log-partition function (see \eqref{eq:boltzmann}).



This construction is similar to the one in the previous section for a linear function  and .
When , -sparse families become exponential families and they cease to be ``sparse'', in the sense that all distributions in the same family have the same support.


A relevant problem is that of characterizing . When  (the log-partition function) its first and higher order derivatives are equal to the moments of the sufficient statistics. The following proposition\footnote{\label{footnote:escort}The statement involves the so-called ``-escort distribution''  \citep{Tsallis1988}, which clearly satisfies , given by
} (stated as \citet[Theorem~5]{amari2011geometry}, and proved in our App.~\ref{sec:proof_gradient_A}) characterizes  for ; it will be used later to derive the Jacobian of entmax transformations and the gradient and Hessian of Fenchel-Young losses. \andre{remove if we don't do FY losses}


\vspace{0.3cm}

\begin{proposition}\label{prop:gradient_A}
 is a convex function and its gradient is given by

\end{proposition}

\andre{To do:  add the convexity proof. Our result is a bit more general: they assume theta is a minimal parametrization and the map btw theta and eta is one-to-one (their sec 3.1). We don't make this assumption (in fact, our discrete examples use overcomplete parameterizations).}

The case  recovers a well-known result: the gradient of the log-partition function is the expectation of the sufficient statistics.



\paragraph{Examples of -sparse families.}
Truncated parabola distributions, for varying location  and scale , are a 2-sparse family. Take  and define the canonical parameters .
From \eqref{eq:lambda_gaussian}, we obtain

Likewise, Gaussians are an exponential family with the same sufficient statistics and canonical parameters, and .
Truncated parabola and Gaussians are both particular cases of the so-called ``-Gaussian'' \citep[\S 4.1]{naudts2009q}, for . 
\andre{consider replacing by \citep[\S 6]{naudts2010q}}
Fig.~\ref{fig:gaussian_laplacian_distributions} (right) shows -Gaussian distributions for . 

Triangular distributions, with a fixed location  and varying scale , also form a 2-sparse family (similarly to Laplace distributions with fixed location being exponential families).
\end{comment}



\section{Continuous Attention}\label{sec:attention}




Attention mechanisms have become a key component of  neural networks \citep{bahdanau2014neural,sukhbaatar2015end,vaswani2017attention}. They dynamically detect and extract relevant input features (such as words in a text or regions of an image). So far, attention has only been applied to discrete domains; we generalize it to {\it continuous} spaces.

\paragraph{Discrete attention.}
Assume an input object split in  pieces, \textit{e.g.}, a document with  words or an image with  regions. 
A vanilla attention mechanism works as follows: each piece has a -dimensional representation (\textit{e.g.}, coming from an RNN or a CNN), yielding a matrix . These representations are compared against a query vector (\textit{e.g.}, using an additive model \citep{bahdanau2014neural}), leading to a score vector . 
Intuitively, the relevant pieces that need attention should be assigned high scores. Then, a transformation  (\textit{e.g.}, softmax or sparsemax) is applied to the score vector to produce a probability vector .
We may see this as an -RPM. The probability vector  is then used to compute a weighted average of the input representations, via . This context vector  is finally used to produce the network's decision.

To learn via the backpropagation algorithm, the Jacobian of the transformation , , is needed. 
\citet{Martins2016ICML} gave  expressions for softmax and sparsemax,

where , and  is a binary vector whose \textsuperscript{th} entry is 1 iff . 

\subsection{The continuous case: score and value functions}
\label{subsec:continuous_attention}

Our extension of -RPMs to arbitrary domains (Def.~\ref{def:regularized_prediction}) opens the door for constructing {\bf continuous attention mechanisms}. The idea is simple: instead of splitting the input object into a finite set of pieces, we assume an underlying continuous domain: \textit{e.g.}, text may be represented as a function  that maps points in the real line (, continuous time) onto a -dimensional vector representation, representing the ``semantics'' of the text evolving over time; images may be regarded as a smooth function in 2D (), instead of being split into regions in a grid. 

Instead of scores , we now have a {\bf score function} , which we map to a probability density . 
This density is used in tandem with the  value mapping  to obtain a context vector . 
Since  may be infinite dimensional, we need to parametrize  , , and  to be able to compute in a finite-dimensional parametric space. 


\paragraph{Building attention mechanisms.}
We represent  and  using basis functions, 
 and , 
defining 
and , 
where  and . The score function  is mapped into a probability density , 
from which we compute the context vector as ,  with . 
\remove{
from which the expectation
 is then obtained.
Given , a context vector is computed as ; from the definition of , this is equivalent to writing . 
}
Summing up yields the  following:
\vspace{0.1cm}
\begin{comment}
\begin{definition}\label{def:attention_mechanism}
Let  be a tuple where  and  are basis functions,
and  is a regularization functional.
An \textbf{attention mechanism} is a mapping  from an input parameter vector  to a vector ,

with  and .
If , we  call this \textbf{entmax} attention, denoted as . The values  and  lead to softmax and sparsemax attention, respectively.
\end{definition}
\end{comment}
\begin{definition}\label{def:attention_mechanism}
Let  be a tuple with  , , and .
An \textbf{attention mechanism} is a mapping , defined as:

with  and .
If , we  call this \textbf{entmax} attention, denoted as . The values  and  lead to softmax and sparsemax attention, respectively.
\end{definition}

Note that, if  and  (Euclidean canonical basis), we recover the discrete attention of  \citet{bahdanau2014neural}.  
Still in the finite case, if  and  are key and value vectors and  is a query vector, this recovers the key-value attention of \citet{vaswani2017attention}.  

On the other hand, for  and , we obtain new attention mechanisms (assessed experimentally for the 1D and 2D cases 
in \S\ref{sec:experiments}): for , the underlying density  is Gaussian, and for , it is a truncated paraboloid (see \S\ref{sec:sparsemax}). In both cases, we show (App.~\ref{sec:gaussian_basis}) that the expectation \eqref{eq:attention_expectation} is tractable (1D) or simple to approximate numerically (2D) if 
 are Gaussian RBFs, and we use this fact in \S\ref{sec:experiments} (see  Alg.~\ref{algo:forward_backward_gaussian} for pseudo-code for the case ).



\begin{algorithm}[t]
\small
\SetAlgoLined
\SetKwInput{KwInput}{Parameters}
\SetKwFunction{FRegression}{Regression}
\SetKwFunction{FForward}{Forward}
\SetKwFunction{FBackward}{Backward}
\def\algspace{.5\baselineskip}
\SetKwProg{Fn}{Function}{:}{}
\KwInput{Gaussian RBFs , basis functions , value function  with , score function  with }
\vspace{\algspace}
\Fn{\FForward{}}{
    \hfill \tcp*[h]{Eqs.\,\ref{eq:attention_expectation}, \ref{eq:continuous_softmax_forward_pass}}\\\KwRet{ (context vector)}}
\vspace{\algspace}
\vspace{\algspace}
\Fn{\FBackward{}}{
\For{ \KwTo }{
        \\
        \hfill \tcp*[h]{Eqs.\,\ref{eq:jacob}, \ref{eq:continuous_softmax_backward_pass_01}--\ref{eq:continuous_softmax_backward_pass_02}}\\        
    }
\KwRet{}
}
\begin{comment}
\vspace{\algspace}
\KwInput{\begin{itemize}
\item Representation for discrete parts 
\item Corresponding locations 
\item Basis functions 
\item Ridge penalty 
\end{itemize}
}
\KwOutput{Continuous representation  with }
\vspace{\algspace}
\Fn{\FRegression{}}{
    \\
    \\
    \\
    \KwRet{\hfill \tcp*[h]{Eq.\,\eqref{eq:gamma}}}
}
\end{comment}
\caption{Continuous softmax attention with , , and Gaussian RBFs.\label{algo:forward_backward_gaussian}}
\end{algorithm}

 
\begin{comment}
On the other hand, for  and , we obtain new attention mechanisms (assessed experimentally in \S\ref{sec:experiments}): for , the underlying density  is Gaussian, and for , it is a truncated parabola (see \S\ref{sec:sparsemax}). In both cases, we show (App.~\ref{sec:gaussian_basis}) that the expectation \eqref{eq:attention_expectation} is tractable if 
 are Gaussian RBFs, and we use this fact in \S\ref{sec:experiments}. 
\andre{say something about the 2D case} 
\end{comment}

\paragraph{Defining the value function .}
In many problems, the input is a discrete sequence of observations (\textit{e.g.}, text) or it was discretized (\textit{e.g.}, images), at locations . To turn it into a continuous signal, we need to smooth and interpolate these observations. If we start with a  discrete encoder representing the input as a matrix , one way of obtaining a value mapping  is by ``approximating''  with {\it multivariate ridge regression}. With  , and packing the basis vectors  as columns of matrix , we obtain: 
where  is the Frobenius norm, and
the  matrix  depends only on the values of the basis functions at discrete time steps and can be obtained off-line for different input lenghts .
The result is an expression for  with  coefficients, cheaper than  if . 


\subsection{Gradient backpropagation with continuous attention}\label{sec:jacobian}


The next proposition, based on Proposition~\ref{prop:gradient_A} and proved in App.~\ref{sec:proof_jacobian_entmax}, allows backpropagating over continuous entmax attention mechanisms. We define, for , a {\it generalized -covariance},

where  is the -escort distribution in \eqref{eq:escort}.
For , we have the usual covariance; for , we get a covariance taken w.r.t.\ a uniform density on the support of ,  scaled by .


\vspace{0.1cm}
\begin{proposition}\label{prop:jacobian_entmax}
Let  with .
The Jacobian of the -entmax is:
    
\end{proposition}

Note that in the finite case, \eqref{eq:jacob} reduces to the expressions in \eqref{eq:jacobians_discrete} for softmax and sparsemax. 

\paragraph{Example: Gaussian RBFs.} 
As before, let , , and . 
For , we obtain closed-form expressions for the expectation \eqref{eq:attention_expectation} and the Jacobian \eqref{eq:jacob}, for any : 
 is a Gaussian, the expectation 
\eqref{eq:attention_expectation} is the integral of a product of Gaussians, and the covariance \eqref{eq:jacob} involves first- and second-order Gaussian moments. Pseudo-code for the case  is shown as Alg.~\ref{algo:forward_backward_gaussian}. 
For ,  is a truncated paraboloid. 
In the 1D case, both \eqref{eq:attention_expectation} and \eqref{eq:jacob} can be expressed in closed form in terms of the  function. 
In the 2D case, we can reduce the problem to 1D integration using the change of variable formula and working with polar coordinates. See App.~\ref{sec:gaussian_basis} for details. 


We use the facts above in the experimental section (\S\ref{sec:experiments}), where we experiment with continuous variants of softmax and sparsemax attentions in natural language processing and vision applications.


\remove{
\section{Continuous Fenchel-Young Losses}\label{sec:fy_losses}

We saw above how to construct distributions  from  via the -RPM (\S\ref{sec:sparse_families}), and how to used them to build attention mechanisms (\S\ref{sec:attention}). What about the reverse: can we estimate  from an empirical data distribution ? 
For finite , \citet{blondel2020learning} introduced the notion of {\bf Fenchel-Young loss}. Here, we extend that notion to arbitrary domains.\footnote{The construction hinges on the notion of Fenchel dual, denoted , of an l.s.c. proper convex function  \citep{Bauschke_Combettes2011}:
}

\begin{definition}
Given an l.s.c., proper, strictly convex function , the \textbf{Fenchel-Young loss} (FY loss) , defined as 

\end{definition}

One target use of FY losses is to estimate , given some empirical data distribution , by minimizing . The following proposition (that stems directly from the definition of  in \eqref{eq:Omega_star2}) motivates this use of the FY loss.
\vspace{0.1cm}
\begin{proposition}\label{prop:fy_properties1}
For any l.s.c., proper, strictly convex function , we have that  and   almost everywhere.
\end{proposition}

For the parametric case, , the following proposition (proved in App.~\ref{sec:proof_fy_properties}) extends the result of \citet{blondel2020learning} 
to arbitrary domains, creating the path to estimating  from data and showing the existence of fixed-dimensional sufficient statistics for -sparse families. 
\vspace{0.05cm}
\begin{proposition}\label{prop:fy_properties}
Let . Then, 
\vspace{-0.2cm}
\begin{enumerate}[nosep]
\item .
    \item  is convex w.r.t.\ . 
\item ,
    where .
\end{enumerate}
\end{proposition}
In point 3, if , an empirical data distribution,  is the empirical mean of the statistics;  the statement shows that estimating  only depends on 
 through , which generalizes the concept of sufficient statistics from exponential families.  


\begin{example}\label{ex:fy_gaussians} For  (Shannon entropy), the Fenchel dual is the log-partition function  \eqref{eq:boltzmann}, , and the FY loss recovers the  Kullback-Leibler divergence (KLD).  
For  and , 

since the negentropy of a Gaussian is as given in Example~\ref{ex:Gaussian}. This is the well-known KLD between two Gaussians.
\end{example}






As shown in App.~\ref{sec:gini_ent_sparse_family}, 
the conjugate of  is

from which the following examples of FY losses can be obtained (full derivations are in App.~\ref{sec:fy_example_derivations}).



\begin{example} For  and , we have (see \eqref{eq:truncated_parabola})  and  as given by \eqref{eq:lambda_gaussian}. Plugging these in \eqref{eq:omega_conjugate} yields 
If  is another truncated parabola, the FY loss between  and  becomes (see App.~\ref{sec:fy_example_derivations}):

If ,   and \eqref{eq:kl_gaussians} are both equal to .

\end{example}



\begin{example}\label{ex:fy_triangular}
If   and 

\eqref{eq:triangular}, we have

\end{example}

We use the FY expressions from examples~\ref{ex:fy_gaussians}-\ref{ex:fy_triangular} for our interval regression experiments in \S\ref{sec:experiments}.
}


\section{Experiments}\label{sec:experiments}

As proof of concept, we test our continuous attention mechanisms on three tasks: document classification, machine translation, and visual question answering (more experimental details in App.~\ref{sec:model_hyperparams}). 


\remove{
\subsection{Continuous attention mechanisms}
}



\paragraph{Document classification.} 

Although textual data is fundamentally discrete, modeling long documents as a continuous signal may be advantageous, due to smoothness and independence of length. To test this hypothesis, we use the IMDB movie review dataset \citep{maas2011learning}, 
whose inputs are documents (280 words on average) and  outputs are sentiment labels (positive/negative). Our baseline is a biLSTM with discrete attention. 
For our continuous attention models, we normalize the document length  into the unit interval , and use  as the score function, leading to a 1D Gaussian () or  truncated parabola () as the attention density.  We compare three attention variants: 
{\bf discrete attention} with softmax \citep{bahdanau2014neural} and sparsemax \citep{Martins2016ICML};  
{\bf continuous attention},  
where a CNN and max-pooling  
yield a document representation  from which 
we compute  and ;   
and {\bf combined attention}, which obtains  from discrete attention, computes  and , applies the continuous attention, and sums the two context vectors (this model has the same number of parameters as the discrete attention baseline). 

\begin{table}[t]
    \caption{Results on IMDB in terms of accuracy (\%). 
    For the continuous attentions, we used  Gaussian RBFs 
, with  linearly spaced in  and . 
} 
    \label{table:results_doc_classification}
    \vspace{-0.1cm}
    \begin{scriptsize}
    \begin{center}
\begin{comment}
\begin{tabular}{lclccclccc}
        \toprule
        \sc Discrete &  & \sc Continuous &  &  &  & \sc Disc. + Cont. &  &  & \\
        \midrule 
        softmax    		& 90.78		& 
        softmax			& 90.20		& 90.68		& 90.52    &
        softmax		& 90.98		& 90.69		& 89.62   	\\
        sparsemax	 		& 90.58		& 
        sparsemax	 	& 90.52		& 89.63		& 90.90   	&
        sparsemax	 	& \bf 91.10		& \bf 91.18		& \bf 90.98    	\\
        \bottomrule
    \end{tabular}
\end{comment}
\begin{tabular}{lc}
        \toprule
        \sc Attention &  \\
        \midrule 
        Discrete softmax    		& 90.78 	\\
        Discrete sparsemax	 		& 90.58 	\\
       \bottomrule
 \end{tabular}
 \qquad
\begin{tabular}{lccc}
        \toprule
        \sc Attention &  &  &  \\
       \midrule
        Continuous softmax			& 90.20		& 90.68		& 90.52    	\\
        Continuous sparsemax	 	& 90.52		& 89.63		& 90.90   	\\
        Disc. + Cont. softmax		& 90.98		& 90.69		& 89.62   	\\
        Disc. + Cont. sparsemax	 	& \bf 91.10		& \bf 91.18		& \bf 90.98    	\\
        \bottomrule
    \end{tabular}
\begin{comment}
\begin{tabular}{lccc}
        \toprule
        \sc Attention &  &  &  \\
        \midrule 
        Discrete softmax    		& 90.78		& 90.78		& 90.78    	\\
        Discrete sparsemax	 		& 90.58		& 90.58		& 90.58   	\\
       \midrule
        Continuous softmax			& 90.20		& 90.68		& 90.52    	\\
        Continuous sparsemax	 	& 90.52		& 89.63		& 90.90   	\\
        Disc. + Cont. softmax		& 90.98		& 90.69		& 89.62   	\\
        Disc. + Cont. sparsemax	 	& \bf 91.10		& \bf 91.18		& \bf 90.98    	\\
        \bottomrule
    \end{tabular}
\end{comment}
\end{center}
    \end{scriptsize}
\end{table}

Table~\ref{table:results_doc_classification} 
shows accuracies for different numbers  of Gaussian RBFs. The accuracies of the individual models are similar, suggesting that continuous attention is as effective as its discrete counterpart, despite having fewer basis functions than words, \textit{i.e.}, . Among the continuous variants, the sparsemax outperforms the softmax, except for . We also see that a large  is not necessary to obtain good results, which is encouraging for tasks with long sequences. Finally, combining discrete and continuous sparsemax produced the best results, without increasing the number of parameters.




\begin{figure*}[t]
\centering
\includegraphics[width=0.3\textwidth]{figures/attention_baseline_23.pdf}
\includegraphics[width=0.3\textwidth]{figures/attention_continuous_softmax_23.pdf}
\includegraphics[width=0.3\textwidth]{figures/attention_continuous_sparsemax_23.pdf}
\caption{\label{fig:attention_maps}Attention maps in machine translation: discrete (left), continuous softmax (middle), and continuous sparsemax (right), for a sentence in the De-En IWSLT17 validation set. In the rightmost plot, the selected words are the ones with positive density. In the test set, these models attained BLEU scores of 23.92 (discrete), 24.00 (continuous softmax), and 24.25 (continuous sparsemax).}
\end{figure*}


\paragraph{Machine translation.} 

We use the DeEn IWSLT 2017 dataset \citep{cettolo2017overview}, and a biLSTM model with discrete softmax attention as a baseline. 
For the continuous attention models, we use the combined attention setting described above, with 30 Gaussian RBFs and  linearly spaced in  and . 
The results (caption of Fig.~\ref{fig:attention_maps}) show a slight benefit in the combined  attention over discrete attention only, without any additional parameters. Fig.~\ref{fig:attention_maps} shows heatmaps for the different attention mechanisms on a DeEn sentence. The continuous mechanism tends to have attention means close to the diagonal,   adjusting the variances based on alignment confidence or when a larger context is needed ({\it e.g.}, a peaked density for the target word ``sea'', and a flat one for ``of'').


\begin{comment}
\begin{SCtable}[][t]
    \caption{BLEU scores for the DeEn machine translation task.} 
    \label{table:results_nmt}
    \begin{small}
\begin{tabular}{lc}
        \toprule
        {\sc Attention}     		& BLEU (\%)		 	\\
        \midrule
        Discrete     		& 23.92		 	\\
        \midrule
        Disc. + Cont. softmax		& 24.00		  	\\
        Disc. + Cont. sparsemax	 	& \bf 24.25		  	\\
        \bottomrule
    \end{tabular}
\end{small}
\end{SCtable}
\end{comment}


\paragraph{Visual QA.} 



Finally, we report experiments with 2D continuous attention on visual question answering, using the VQA-v2 dataset \cite{Goyal2019} and a modular co-attention network as a baseline \cite{Yu2019}.\footnote{Software code is available at \url{https://github.com/deep-spin/mcan-vqa-continuous-attention}.} The discrete attention model attends over a  1414 grid.\footnote{An alternative would be bounding box features from an external object detector \citep{anderson2018bottom}. We opted for grid regions to check if continuous attention has the ability to detect relevant objects on its own. However, our method can handle bounding boxes too, if  the  coordinates in the  regression
\eqref{eq:B_regression} are placed on those regions.} For continuous attention, we normalize the image size into the unit square . We fit a 2D Gaussian () or truncated paraboloid () as the attention density; both correspond  to  , with . We use the mean and variance according to the discrete attention probabilities and obtain  and  with moment matching. We use  Gaussian RBFs, with  linearly spaced in   and . Overall, the number of neural network parameters is the same as in discrete attention. 

The results in Table~\ref{table:results_vqa} show similar accuracies for all attention models, with a slight advantage for continuous softmax. Figure~\ref{fig:examples_vqa} shows an example (see App.~\ref{sec:model_hyperparams} for more examples and some failure cases): in the baseline model, the discrete attention is too scattered, possibly mistaking the lamp with a TV screen.  The continuous attention models focus on the right region and answer the question correctly, with continuous sparsemax enclosing all the relevant information in its supporting ellipse.


\begin{figure*}[t]
\centering
\includegraphics[width=0.21\textwidth]{figures/question}
\includegraphics[width=0.21\textwidth]{figures/discrete}
\includegraphics[width=0.21\textwidth]{figures/softmax}
\includegraphics[width=0.21\textwidth]{figures/sparsemax}
\caption{\label{fig:examples_vqa}Attention maps for an example in VQA-v2: original image, discrete attention, continuous softmax, and continuous sparsemax. The latter encloses all probability mass within the outer ellipse.}
\end{figure*}

\begin{table}[t]
    \caption{Accuracies of different models on the \textit{test-dev} and \textit{test-standard} splits of VQA-v2.} 
    \label{table:results_vqa}
    \vspace{-.1cm}
    \begin{small}
    \begin{center}
\begin{tabular}{l@{\hspace{10pt}}c@{\hspace{10pt}}c@{\hspace{10pt}}c@{\hspace{10pt}}c@{\hspace{10pt}}c@{\hspace{10pt}}c@{\hspace{10pt}}c@{\hspace{10pt}}c}
        \toprule
        \sc Attention & \multicolumn{4}{c}{Test-Dev}  & \multicolumn{4}{c}{Test-Standard} \\
        {} & Yes/No & Number & Other & Overall & Yes/No & Number & Other & Overall \\
        \midrule 
        Discrete softmax    		& 83.40 & 43.59	& 55.91 & 65.83 & 83.47 & 42.99 & 56.33 & 66.13    	\\
        \midrule
        2D continuous softmax			& 83.40	& 44.80	& 55.88 & \textbf{65.96} & 83.79 & 44.33 & 56.04 & \textbf{66.27}    	\\
        2D continuous sparsemax	 	& 83.10 & 44.12 & 55.95 & 65.79 & 83.38 & 43.91 & 56.14 & 66.10  	\\
        \bottomrule
    \end{tabular}
\end{center}
    \end{small}
\end{table}




\remove{
\subsection{Interval regression}

\andre{Indeed, this is not a censored regression task with prescribed intervals. Interval-valued regression is studied in symbolic data analysis (see \citet{billard2000regression} and \citet{maia2008forecasting}. It is the regression counterpart of multi-label classification. Using two networks is an interesting suggestion (similar to the midpoint and range method of \citet{neto2010constrained}, we will add a comparison.}

Finally, we perform
regression
experiments to assess the continuous FY losses from \S\ref{sec:fy_losses}. Regression typically predicts a scalar outcome; however, when predicting stochastic phenomena, one is often interested in more information than just a point prediction. \textit{E.g.}, interval regression predicts an interval ; intervals are intuitively interpretable and, in some applications, they are available as supervision.

We address temperature prediction as interval
regression, using 2015--2019 data from the U.S.~Climate Reference Network \citep{diamond2013us} for Ithaca, NY, 
predicting minimum and maximum daily temperatures. Let  denote the prediction function, computed in some  parametric form by an NN. Given an interval
, we posit a  target distribution ,
and minimize the total loss

The input features  are minimum and maximum temperatures in the previous 2 days, and an index  of the target day ( corresponds
to 2015-01-01 and  to 2019-12-31). We remove windows with missing data and leave out 99
validation and 99 test days with non-overlapping windows.
We consider the following models for : \textbf{(i)} Gaussian with ;
\textbf{(ii)} Gaussian with 95\% CI equal to ; \textbf{(iii)} truncated parabola of support ; \textbf{(iv)} triangular of support . These choices correspond to assumptions about daily temperature distributions, given
min and max values; in all cases, we fit a distribution of the same type as .

We train a 2-hidden-layer NN, outputting a location
parameter via an affine map, and a scale using a softplus-affine map. We tune the hidden dimension in , the Adam learning rate in , and the dropout
rate in . 
The baseline predicts the mid-point of the interval using mean
squared error and sets the interval width to the mean interval length in the training data. Tab.~\ref{tab:weather} shows that a truncated parabola leads to the best fit
in terms of both mean and support, suggesting sparse
distributions as a promising tool for interval regression.



\begin{figure}[h]\centering
\includegraphics[width=.5\textwidth]{figures/autoregressive.pdf}
\vspace{-0.5cm}
\caption{Auto-regressive temperature interval predictions (C).}
\end{figure}


\begin{table}[t]
    \caption{\label{tab:weather}Temperature interval prediction: mean squared error (w.r.t.\ the interval midpoint) and Jaccard similarity w.r.t.\ the gold interval.}
    \vskip 0.15in
    \begin{small}
    \begin{center}
    \begin{tabular}{lcc}
    \toprule
    & MSE & {\sc Jaccard} \\
    \midrule
    Baseline
    & 0.0112 & 0.4701 \\
    Gaussian-
    & 0.0111 & 0.4679 \\
    Gaussian-
    & 0.0109 & 0.4709 \\
    Truncated parabola
    & \textbf{0.0108} & \textbf{0.4807} \\
    Triangular
    & 0.0110 & 0.4734 \\
    \bottomrule
    \end{tabular}
    \end{center}
    \end{small}
    \vskip -0.2in
\end{table}
}

\section{Related Work}\label{sec:related}
\paragraph{Relation to the Tsallis maxent principle.}
Our paper unifies two lines of work: deformed exponential families from statistical physics \citep{Tsallis1988,naudts2009q,amari2011geometry}, and sparse alternatives to softmax recently proposed in the machine learning literature \citep{Martins2016ICML,peters2019sparse,blondel2020learning}, herein extended to continuous domains. This link may be fruitful for future research in both fields. 
While most prior work is focused on heavy-tailed distributions (), we focus instead on light-tailed, sparse distributions, the other side of the spectrum (). See App.~\ref{sec:tsallis_maxent} for the relation to the Tsallis maxent principle.  

\paragraph{Continuity in other architectures and dimensions.} 
In our paper, we consider attention networks exhibiting temporal/spatial continuity in the input data, be it text (1D) or images (2D). Recent work propose continuous-domain CNNs for 3D structures like point clouds and molecules \citep{wang2018deep,schutt2017schnet}. The dynamics of continuous-time RNNs have been studied in \citep{funahashi1993approximation}, and similar ideas have been applied to irregularly sampled time series \citep{rubanova2019latent}. 
Other recently proposed frameworks produce continuous variants in other dimensions, such as network depth \citep{chen2018neural}, or in the target domain for machine translation tasks \citep{kumar2018mises}.
Our continuous attention networks can be used in tandem with these frameworks.

\paragraph{Gaussian attention probabilities.} 
\citet{cordonnier2020relationship} analyze the relationship between (discrete) attention and convolutional layers, and consider spherical Gaussian attention probabilities as relative positional encodings. By contrast, our approach removes the need for positional encodings: by converting the input to a function on a predefined continuous space, positions are encoded {\it implicitly}, not requiring explicit positional encoding. 
Gaussian attention has also been hard-coded as input-agnostic self-attention layers in transformers for machine translation tasks by 
\citet{you-etal-2020-hard}. 
Finally, in their DRAW architecture for image generation, \citet[\S 3.1]{gregor2015draw} propose a selective attention component which is parametrized by a spherical Gaussian distribution. 


 

\section{Conclusions and Future Work}

We proposed extensions to regularized prediction maps,\remove{and FY losses,} originally defined on finite domains, to arbitrary measure spaces (\S\ref{sec:sparse_families}). 
With Tsallis -entropies for , we obtain sparse families, whose members can have zero tails, such as triangular or truncated parabola distributions. 
We then used these distributions to construct continuous attention mechanisms (\S\ref{sec:attention}). We derived their Jacobians in terms of generalized covariances (Proposition~\ref{prop:jacobian_entmax}), allowing for efficient forward and backward propagation.  Experiments for 1D and 2D cases were shown on attention-based text classification, machine translation, and visual question answering (\S\ref{sec:experiments}), with encouraging results. \remove{and FY-based interval regression.}

There are many avenues for future work. 
As a first step, we  considered unimodal distributions only (Gaussian, truncated paraboloid), for which we show that the forward and backpropagation steps have closed form or can be reduced to 1D integration. However, there are applications in which  multiple attention modes are desirable. This can be done by considering mixtures of distributions,  multiple attention heads, or sequential attention steps. 
Another direction concerns combining our continuous attention models with 
other spatial/temporal continuous architectures for CNNs and RNNs \citep{wang2018deep,schutt2017schnet,funahashi1993approximation} or 
with continuity in other dimensions, such as depth \citep{chen2018neural} or output space \citep{kumar2018mises}.






\section*{Broader Impact}

We discuss the broader impact of our work, including ethical aspects and future societal consequences. Given the early stage of our work and its predominantly theoretical nature, the discussion is mostly speculative.

The continuous attention models developed in our work can be used in a very wide range of applications, including natural language processing, computer vision, and others. For many of these applications, current state-of-the-art models use discrete softmax attention, whose interpretation capabilities have been questioned in prior work \citep{jain2019attention,serrano2019attention,wiegreffe2019attention}. Our models can potentially lead to more interpretable decisions, since they lead to less scattered attention maps (as shown in our Figures~\ref{fig:attention_maps}--\ref{fig:examples_vqa}) and are able to select contiguous text segments or image regions. 
As such, they may provide better inductive bias for interpretation.


In addition, our attention densities using Gaussian and truncated paraboloids include a variance term, being potentially useful as a measure of confidence---for example, a large ellipse in an image may indicate that the model had little confidence about where it should attend to answer a question, while a small ellipse may denote high confidence on a particular object. 

We also see opportunities for research connecting our work with other continuous models \citep{wang2018deep,schutt2017schnet,chen2018neural} leading to end-to-end continuous models which, by avoiding discretization, have the potential to be less susceptible to adversarial attacks via input perturbations. 
Outside the machine learning field, the links drawn in \S\ref{sec:sparse_families} between sparse alternatives to softmax and models used in non-extensive (Tsallis) statistical physics suggest a potential benefit in that field too. 



Note, however, that our work is a first step into all these directions, and as such further investigation will be needed to better understand the potential benefits. We strongly recommend carrying out user studies before deploying any such system, to better understand the benefits and risks. Some of the examples in App.~\ref{sec:model_hyperparams} may help understand potential failure modes. 



We should also take into account that, for any computer vision model, there are important societal risks related to privacy-violating surveillance applications. Continuous attention holds the promise to scale to larger and multi-resolution images, which may, in the longer term, be deployed in such undesirable domains. 
Ethical concerns hold for natural language applications such as machine translation, where 
biases present in data can be arbitrarily augmented or hidden by machine learning systems.
For example, our natural language processing experiments mostly use English datasets (as a
target language in machine translation, and in document classification). 
Further work is needed to understand if our findings generalize to other languages. Likewise, in the vision experiments, the VQA-v2 dataset uses COCO images, which have documented biases \citep{wang2019balanced}. 
In line with the fundamental scope and early stage of this line of research, we deliberately choose applications on standard benchmark datasets, in an attempt to put as much distance as possible from malevolent applications. 
Finally, although we chose the most widely used evaluation metrics for each task (accuracy for document classification and visual question answering, BLEU for machine translation), these metrics do not always capture performance quality---for example, BLEU in machine translation is far from being a perfect metric.

The data, memory, and computation requirements for training systems with continuous attention do not seem considerably higher than the ones which use discrete attention. On the other hand, for NLP applications, our approach has the potential to better compress sequential data, by using fewer basis functions than the sequence length (as suggested by our document classification experiments). While there is nothing specific about our research that poses environmental concerns  or that promises to alleviate such concerns, our models share the same problematic property as other neural network models in terms of their energy consumption to train models and tune hyperparameters \citep{strubell2019energy}.


\begin{ack}
This work was supported by the European Research Council (ERC StG DeepSPIN 758969),
by the P2020 program MAIA (contract 045909), and by the Funda\c{c}\~ao para a Ci\^encia e Tecnologia 
through contract UIDB/50008/2020. 
We would like to thank Pedro Martins, Zita Marinho, and the anonymous reviewers for their helpful feedback.




\end{ack}



\bibliography{neurips_2020}
\bibliographystyle{unsrtnat}


\newpage
\onecolumn

\appendix

\bigskip

\begin{center}
\LARGE{\bf Supplemental Material}
\end{center}


\begin{comment}
\section{Existence and Uniqueness of the Regularized Prediction Map}
\label{sec:appendix_existence}

We show here that the -RPM in \eqref{eq:reg_prediction} is well defined, i.e., the arg-max contains one and only one density.

The constraint set  is convex, because for any  and any , we have that . The first term in \eqref{eq:reg_prediction} is clearly continuous (thus also lower semi-continuous), since it is linear w.r.t. . For the same reason, it is concave (and also convex); in fact, for any , \andre{improve the writing here}

Since  is proper, a sufficient condition for the existence of maximizer(s) in \eqref{eq:reg_prediction} is thus that  is also proper \citep{Bauschke_Combettes2011}, \andre{point to section?} \textit{i.e.}, that  is never equal to , which is guaranteed by the definition of  in \eqref{eq:proper}.

Finally, since  is lower semi-continuous, proper, and convex, a sufficient condition for the uniqueness of the arg-max in \eqref{eq:reg_prediction} is that  be strictly convex, in addition to being convex and lower semi-continuous. \andre{improve the writing; R3 said: It seems to me that the proof can be constructed by saying that the expectation term varies linearly in p (and is bounded), and the regularizer is strictly convex (so (-regularizer) is strictly concave). Then Linear + Concave = Concave, and has a global maximum. It is good if you have rigorous proofs, but adding more accessible intuitions would increase the accessibility of the work.}
\end{comment}


\section{Differential Negentropy and Boltzmann-Gibbs distributions}\label{sec:diff_ent_exp_family}

We adapt a proof from \citet{cover2012elements}.
Let  be the Shannon negentropy, which is proper, lower semi-continuous, and strictly convex \citep[example~9.41]{Bauschke_Combettes2011}, and let   be the Kullback-Leibler divergence between distributions  and  (which is always non-negative and equals  iff ).
Take  as in \eqref{eq:boltzmann}, where  is the log-partition function.

We have, for any :

Therefore, we have, for any  , that

with equality if and only if . Since the right hand side is constant with respect to , we have that the posited  must be the maximizer of \eqref{eq:reg_prediction}.





\section{Tsallis Negentropy and  Sparse Distributions}\label{sec:gini_ent_sparse_family}



\subsection{Shannon as a limit case of Tsallis when }

We show that  for any . 
From \eqref{eq:tsallis}, it suffices to show that  for any . 
Let , and
. Observe that 
so we are in an indeterminate case.
We take the derivatives of  and : 

and . 
From l'H\^{o}pital's rule,



\subsection{Proof of Proposition~\ref{prop:solution_rpm_tsallis}}

The proof of Proposition~\ref{prop:solution_rpm_tsallis} is  similar to the one in \S\ref{sec:diff_ent_exp_family}, replacing the KL divergence by the Bregman divergence induced by , and using an additional bound.
Let  be the (functional) Bregman divergence between distributions  and  induced by , and let  
Note that, from \eqref{eq:tsallis},  
From the non-negativity of the Bregman divergence \cite{bregman1967relaxation}, we have, for any :


Therefore, we have, for any ,

with equality iff , which leads to zero Bregman divergence ({\it i.e.}, a tight inequality ) and to 
 ({\it i.e.}, a tight inequality ).

We can use the equality above to obtain an expression for the Fenchel conjugate  ({\it i.e.}, the value of the maximum in \eqref{eq:reg_prediction} and the right hand side in \eqref{eq:variational_proof_tsallis}):


\subsection{Normalizing function }\label{sec:A_alpha}

Let . 
The expression for  in Prop.~\ref{prop:solution_rpm_tsallis} is
obtained by inverting \eqref{eq:entmax}, yielding
,  and integrating with respect to  , leading to:

from which the desired expression follows.


\begin{comment}
\section{Normalizing Function for -Sparse Families}\label{sec:A_alpha}

The expression for  in \eqref{eq:A_alpha} is
obtained by inverting \eqref{eq:alpha_sparse_family}, yielding
,  and integrating with respect to  , leading to:

\end{comment}



\section{Relation to the Tsallis Maxent Principle}\label{sec:tsallis_maxent}

We discuss here the relation between the -exponential family of distributions as presented in Prop.~\ref{prop:solution_rpm_tsallis} and the distributions arising from the Tsallis maxent principle \citep{Tsallis1988}. We put in perspective the related work in statistical physics \citep{abe2003geometry,naudts2009q}, information geometry \citep{amari2011geometry,amari2016information}, and the discrete case presented in the machine learning literature \citep{blondel2020learning,peters2019sparse}. 

We start by noting that our  parameter matches the  used in prior machine learning literature related to the ``-entmax transformation'' \citep{blondel2020learning,peters2019sparse}. In the definition of Tsallis entropies  \eqref{eq:tsallis}, our  corresponds to the entropic index  defined by \citet{Tsallis1988}. 
However, our -exponential families correspond to the -exponential families as defined by \citet{naudts2009q}, and to the -exponential families described by \citet{ding2010t} (which include the -Student distribution). The family of Amari's -divergences relates to this  as  \citep[\S4.3]{amari2016information}. 

These differences in notation have historical reasons, and they are explained by the different ways in which Tsallis entropies relate to -exponential families. 
In fact, the physics literature has defined -exponential distributions in two distinct ways, as we next describe. 

Note first that the -RPM in our  Def.~\ref{def:regularized_prediction} is a generalization of the free energy variational principle, if we see  as an energy function and  the entropy scaled by a temperature. 
Let  be the Tsallis -entropy. 
An equivalent constrained version of this problem is the maximum entropy ({\it maxent}) principle \citep{jaynes1957information}:

The solution of this problem corresponds to a distribution in the -exponential family \eqref{eq:entmax}:

for some Lagrange multiplier . 

However, this construction differs from the one by \citet{Tsallis1988} and others, who use {\it escort distributions} (Eq.~\ref{eq:escort}) in the expectation constraints. Namely, instead of \eqref{eq:constrained_maxent}, they consider the problem:

The solution of \eqref{eq:constrained_maxent_tsallis} is of the form

where  is again a Lagrange multiplier. This is derived, for example, in \citep[Eq.~15]{abe2003geometry}. 
There are two main differences between \eqref{eq:sol_maxent} and \eqref{eq:sol_maxent_tsallis}:
\begin{itemize}
    \item While \eqref{eq:sol_maxent} involves the -exponential, \eqref{eq:sol_maxent_tsallis} involves the -exponential.
    \item In \eqref{eq:sol_maxent}, the normalizing term  is {\it inside} the -exponential. In \eqref{eq:sol_maxent_tsallis}, there is an normalizing factor  {\it outside} the -exponential. 
\end{itemize}
Naturally, when , these two problems become equivalent, since an additive term inside the exponential is equivalent to a multiplicative term outside. However, this does {\it not} happen with -exponentials ( in general, for ), and therefore these two alternative paths lead to two different definitions of -exponential families. Unfortunately, both have been considered in the physics literature, under the same name, and  
this has been subject of debate. Quoting \citet[\S 1]{naudts2009q}: 

\begin{quote}
{\it ``An important question is then whether in the modification the normalization should stand in front of the deformed exponential function, or whether it should be included as  inside. From the general formalism mentioned above it follows
that the latter is the right way to go.''}
\end{quote}

Throughout our paper, we use the definition of \citep{naudts2009q,amari2011geometry}, equivalent to the maxent problem \eqref{eq:constrained_maxent}. 

\section{Proof of Proposition~\ref{prop:gradient_A}}\label{sec:proof_gradient_A}

We adapt the proof from \citet[Theorem 5]{amari2011geometry}. 
Note first that, for ,

and


\begin{comment}
and


\end{comment}



Therefore we have:

from which we obtain


\begin{comment}

\end{comment}


To prove that  is convex, we will show that its Hessian is positive semidefinite. Note that

hence, for ,

where we used the fact that  for  and that integrals of positive semidefinite functions and positive semidefinite.

\section{Normalization Constants for Continuous Sparsemax Distributions}\label{sec:normalization_constants}

\subsection{Truncated parabola}\label{sec:proof_truncated_parabola}

Let  as in \eqref{eq:truncated_parabola}.
Let us determine the constant  that ensures this distribution normalizes to 1. Note that  does not depend on the location parameter , hence we can assume  without loss of generality. We must have
 and , hence
, which finally gives:


\remove{
The Gini negentropy of this distribution is

}

\subsection{Multivariate truncated paraboloid}\label{sec:proof_truncated_paraboloid}

Let  as in \eqref{eq:truncated_paraboloid}.
Let us determine the constant  that ensures this distribution normalizes to 1, where we assume again  without loss of generality. To obtain , we start by invoking the formula for computing the volume of an ellipsoid defined
by the equation :

where  is the Gamma function.
Since each slice of a paraboloid is an ellipsoid, we can apply Cavalieri's principle to obtain the volume of a paraboloid  of height  as follows:

Equating the volume to 1, we obtain  as:


\subsection{Triangular}\label{sec:proof_triangular}

Let  as in \eqref{eq:triangular}.
Let us determine the constant  that ensures this distribution normalizes to 1.
Assuming again  without loss of generality, we must have
 and , hence
, which finally gives
.

\remove{
The negentropy of this distribution is

}

\subsection{Location-scale families}\label{sec:proof_location_scale}

We first show that  is the solution of the equation .
From symmetry around , we must have

where we made a variable substitution ,
which proves the desired result.
Now we show that a solution always exists if  is strongly convex, {\it i.e.}, if there is some  such that
 for any .
Let . We want to show that the equation  has a solution. Since  is continuously differentiable,  is continuous.
From the strong convexity of , we have that  for any , which implies that .
Therefore, since , we have by the intermediate value theorem that there must be some  such that .








\section{Proof of Proposition~\ref{prop:jacobian_entmax}}\label{sec:proof_jacobian_entmax}




We have

Using the expression for  from Proposition~\ref{prop:gradient_A} yields the desired result.


\section{Continuous Attention with Gaussian RBFs}\label{sec:gaussian_basis}

We derive expressions for the evaluation and gradient computation of  continuous attention mechanisms where  are Gaussian radial basis functions, both for the softmax () and sparsemax () cases. 
For softmax, we show closed-form expressions for any number of dimensions (including the 1D and 2D cases). 
For sparsemax, we derive closed-form expressions for the 1D case, and we reduce the 2D case to a univariate integral on an interval, easy to compute numerically. 

This makes it possible to plug both continuous attention mechanisms in neural networks and learn them end-to-end with the gradient backpropagation algorithm.


\subsection{Continuous softmax ()}

We derive expressions for continuous softmax for multivariate Gaussians in .  
This includes the 1D and 2D cases, where . 

If , for , the distribution , with , is a multivariate Gaussian where the mean  and the covariance matrix  are related to the canonical parameters as . 

We derive closed form expressions for the attention mechanism output 
 in \eqref{eq:attention_expectation} and for its Jacobian  in \eqref{eq:jacob},  when  are Gaussian RBFs, i.e., each  is of the form . 

\paragraph{Forward pass.}

Each coordinate of the attention mechanism output becomes the integral of a product of Gaussians,


We use the fact that the product of two Gaussians is a scaled Gaussian:

where 

Therefore, the forward pass can be computed as:


\paragraph{Backward pass.}
To compute the backward pass, we have that each row of the Jacobian  becomes a first or second moment under the resulting Gaussian:
 and, noting that ,



\subsection{Continuous sparsemax in 1D (, )}

With , the distribution , with , becomes a truncated parabola where   and   are related to the canonical parameters as above, i.e., .

We  derive closed form expressions for the attention mechanism output  in \eqref{eq:attention_expectation} and its Jacobian  in \eqref{eq:jacob} when  and Gaussian RBFs, {\it i.e.}, each  is of the form .

\paragraph{Forward pass.}
Each coordinate of the attention mechanism output becomes:

where  and , as stated in \eqref{eq:lambda_gaussian_proof}, and we made the substitution
. 
We use the fact that, for any  such that :

from which the expectation \eqref{eq:expectation_continuous_sparsemax_rbf} can be computed directly.

\paragraph{Backward pass.}
Since , we have from \eqref{eq:beta_covariance} and \eqref{eq:erf_expr123} that each row of the Jacobian  becomes:

and




\subsection{Continuous sparsemax in 2D (, )}

Let us now consider the case where . 
For , the distribution , with , becomes a bivariate truncated paraboloid where  and   are related to the canonical parameters as before, . We obtain expressions for the attention mechanism output  and its Jacobian  that include 1D integrals (simple to integrate numerically), when  are Gaussian RBFs, {\it i.e.}, when each  is of the form .

We start with the following lemma:

\smallskip

\begin{lemma}\label{lemma:affine_transform_gaussian}
Let  be a -dimensional multivariate Gaussian, 
Let  be a full column rank matrix (with ), and . 
Then we have  with:

If , then  is invertible and the expressions above can be simplified to:

\end{lemma}

\begin{proof}
The result can be derived by writing  and splitting the exponential of the sum as a product of exponentials.
\end{proof}

\paragraph{Forward pass.}
For the forward pass, we need to compute

with

and (from \eqref{eq:truncated_paraboloid})

Using Lemma~\ref{lemma:affine_transform_gaussian} and  the change of variable formula (which makes the determinants cancel), we can reparametrize  and write this as an integral over the unit circle:

with , 
. 
We now do a change to polar coordinates, , where . The integral becomes:

where in the second line we applied again Lemma~\ref{lemma:affine_transform_gaussian}, resulting in 

Applying Fubini's theorem, 
we fix  and integrate with respect to . We use the formulas \eqref{eq:erf_expr123} and the fact that, for any  such that :

We obtain a closed from expression for the inner integral:

The desired integral can then be expressed in a single dimension as 

which may be integrated numerically.



\paragraph{Backward pass.}

For the backward pass we need to solve
and 
 where  denotes the support of the density , a region bounded by an ellipse. 
Note that these expressions include integrals of vector-valued functions and that \eqref{eq:1-2_row_J} and \eqref{eq:3-6_row_J} correspond to the first to second and the third to sixth row of the Jacobian, respectively. The integrals that do not include Gaussians have closed form expressions and can be computed as

and
 where  is the area of the region  given by



All the other integrals are solved using the same affine transformation and change to polar coordinates as in the forward pass. Given this, , , ,  and  are the same as before. 
To solve \eqref{eq:1-2_row_J} we write

in polar coordinates,

which can be then expressed in a single dimension as 

with

We do the same for

which can then be expressed in a single dimension as 
with

Finally, to solve \eqref{eq:3-6_row_J} we simplify the integral

with



The integral can then be expressed in a single dimension as 

with

where









\remove{
\section{Proof of Proposition~\ref{prop:fy_properties} }\label{sec:proof_fy_properties}



The proof adapts that of \citet{Blondel2019AISTATS} 
\andre{maybe replace by \citep{blondel2020learning}} 
when  Fenchel duality is now taken in the infinite-dimensional set , which endowed with the inner product  forms a Hilbert space \citep{Bauschke_Combettes2011}. \andre{point to section?} 
The non-negativity of  stems from the Fenchel-Young inequality in Hilbert spaces.
The loss is zero iff  is a dual pair, i.e., if .
The gradient of  is

where we used the fact that
.
This proves point 1.
To prove point 2, note that we have

where we used the result of Prop.~\ref{prop:jacobian_entmax}.
Since for any probability distribution the covariance operation  leads to a positive semi-definite matrix, so does  . Therefore, the Hessian of  with respect to  is positive semi-definite, i.e.,  is convex on the canonical parameters, which proves point 2.
Finally, point 3 is an immediate consequence of points 1 and 2: Since
 is convex, any stationary point is a global minimum,
and, from point 1, any stationary point  must satisfy
.

One interpretation of point 3 is that we may fit an -sparse
density to an empirical distribution  by matching the expected statistics.
Point 3 guarantees that the result is optimal in the FY loss sense, analogous to
how, for exponential families, moment matching amounts to maximum likelihood.
Figure~\ref{fig:misfit} illustrates the result of fitting a Gaussian, truncated
parabola, and triangular distribution to samples drawn from each of the three
types of distribution. Kolmogorov-Smirnov test results, reported for each plot,
successfully reject all mismatched fits, confirming adequate fitting.

\begin{figure*}[t]\centering\includegraphics[width=.9\textwidth]{figures/misfit.pdf}
\caption{\label{fig:misfit}Fitting a Gaussian, truncated parabola, and triangular distribution
(rows, ) to 10,000 samples from a Gaussian, truncated parabola, and
triangular (columns, ). For each fit, we report the
Kolmogorov-Smirnov distance  and corresponding -value .}
\end{figure*}


\section{Derivations of Fenchel-Young Losses}\label{sec:fy_example_derivations}

We derive in this section closed-form expressions for Fenchel-Young losses between truncated parabolas, triangular distributions, and between Gaussians and truncated parabolas and uniform distributions.

\subsection{FY loss between truncated parabolas}

If ,
we have , so this yields:

Let
 be another truncated parabola,
with .
Doing the change of variable ,
the Fenchel-Young loss becomes:

which becomes 0 (and is minimized) when  and .
Note that when  (\textit{i.e.}, if the two distributions have the same variance), the sparsemax loss between truncated parabolas and the KL divergence between Gaussians are both equal to .

\subsection{FY loss between triangular distributions}

If ,
we have 
and , so \eqref{eq:omega_conjugate} yields:

Let
 be another triangular distribution,
with .
We can assume without loss of generality that .
To obtain , we need to consider four cases:
\begin{itemize}
\item \framebox{}
In this case,  (after some math).
\item \framebox{}
By symmetry, .
\item \framebox{}
We need to split the integral in three intervals , , and , giving (after some math):

\item \framebox{}
By symmetry:

\end{itemize}
Putting everything together, we obtain:

and
the Fenchel-Young loss becomes:







\subsection{KL  between as uniform and a Gaussian}

Let now  be a uniform density, and let  with .
The log-partition function of , expressed in terms of  and , is

The Shannon negentropy of a uniform distribution is

The expectation  is

Putting all together, we get:



\subsection{Sparsemax loss between uniform and truncated parabola}

Let now  be a uniform density, and let  with .
We have

The Gini negentropy of a uniform distribution is

The expectation  is the same as \eqref{eq:expectation_under_uniform}.
Putting all together, we get:

}



\section{Experimental Details and Model Hyperparameters}\label{sec:model_hyperparams}

\subsection{Document classification}

We used the IMDB movie review dataset \citep{maas2011learning},\footnote{\url{https://ai.stanford.edu/~amaas/data/sentiment}} which consist of user-written text reviews with binary labels (positive/negative). 
Following \citep{jain2019attention}, we used 25K training documents, 10\% of which for validation, and 25K for testing. The training and test sets are perfectly balanced: 12.5K negative and 12.5K positive examples. 
The documents have 280 words on average.  

Our architecture is the same as \citep{maas2011learning}, a BiLSTM with attention. 
We used pretrained GloVe embeddings from the 840B release,\footnote{\url{http://nlp.stanford.edu/data/glove.840B.300d.zip}} kept frozen. 
We tuned three hyperparameters using the discrete softmax attention baseline:
learning rate within ;  within ; number of epochs within . 
We picked the best configuration by doing a grid search and by taking into consideration the accuracy on the validation set (selected values in bold).
Table~\ref{tab:table_all_hyperparams} shows the hyperparameters and model configurations used for all document classification experiments.

\begin{comment}
For the continuous attention models, we normalize the document length  into the unit interval , with the  word positioned at , and we use  as the score function, leading to a 1D Gaussian () or a truncated parabola () as the attention density.  We compare three attention variants: 
{\bf discrete attention} \citep{bahdanau2014neural}, where  is obtained from a discrete softmax or sparsemax transformation and leads to a context vector ; 
{\bf continuous attention}, 
where a CNN and max-pooling over time 
yields a document representation  from which 
we compute  and , 
leading to a density  (continuous softmax) or 
 (continuous sparsemax), 
and a context vector , where  are Gaussian RBFs.  
Finally, {\bf combined attention} first uses discrete attention to compute  and  and then computes  and , proceeding with the continuous attention as before to get , and combining the two vectors as . Note that this combination does not increase the number of neural network parameters, compared to discrete attention. 
\end{comment}

\begin{table}[t]
    \caption{Hyperparmeters for document classification.}
    \label{tab:table_all_hyperparams}
    \begin{small}
    \begin{center}
    \begin{tabular}{llllll}
        \toprule
        \sc Hyperparameter & \sc Value  \\
        \midrule
        Batch size                  & 16    \\
        Word embeddings size        & 300     \\
        BiLSTM hidden size          & 128     \\
        Merge BiLSTM states         & Concat     \\
        Attention scorer            & \citep{bahdanau2014neural}     \\
        Conv filters                & 128   \\
        Conv kernel size            & 3     \\
        Early stopping patience     & 5     \\
        Number of epochs            & 10     \\
        Optimizer                   & Adam      \\
         regularization     & 0.0001     \\
        Learning rate               & 0.001     \\
        \bottomrule
    \end{tabular}
    \end{center}
    \end{small}
    \vskip -0.1in
\end{table}

\subsection{Machine translation}

We used the DeEn dataset from the IWSLT 2017 evaluation campaign \citep{cettolo2017overview}, with the standard splits (206K, 9K, and 2K sentence pairs for train/dev/test).\footnote{\url{https://wit3.fbk.eu/mt.php?release=2017-01-trnted}} We used BPE \citep{sennrich2016neural} with 32K merges to reduce the vocabulary size.
Our implementation is based on Joey-NMT \citep{kreutzer2019joey} and we used the provided configuration script for the baseline, a BiLSTM model with discrete softmax attention\footnote{\url{https://github.com/joeynmt/joeynmt/blob/master/configs/iwslt14_deen_bpe.yaml}} with the  hyperpameters in Table~\ref{tab:table_all_hyperparams_nmt}. 

\begin{comment}
For the continuous attention models, we used the combined attention setting described above, with 30 Gaussian RBFs and  linearly spaced in  and .
\end{comment}

\begin{table}[t]
    \caption{Hyperparmeters for neural machine translation.}
    \label{tab:table_all_hyperparams_nmt}
    \begin{small}
    \begin{center}
    \begin{tabular}{llllll}
        \toprule
        \sc Hyperparameter & \sc Value  \\
        \midrule
        Batch size                  & 80    \\
        Word embeddings size        & 620     \\
        BiLSTM hidden size          & 1000     \\
        Attention scorer            & \citep{bahdanau2014neural}     \\
        Early stopping patience & 8 \\
        Number of epochs            & 100     \\
        Optimizer                   & Adam      \\
         regularization     & 0     \\
        Dropout & 0.0 \\
        Hidden dropout & 0.2 \\
        Learning rate               & 0.0002     \\
        Scheduling & Plateau \\
        Decrease factor & 0.7 \\
        Lower case & True \\
        Normalization & Tokens \\
        Maximum output length & 80 \\
        Beam size & 5 \\
        RNN type & GRU \\
        RNN layers & 1 \\
        Input feeding & True \\
        Init. hidden & Bridge \\
        \bottomrule
    \end{tabular}
    \end{center}
    \end{small}
    \vskip -0.1in
\end{table}


\subsection{Visual question answering}

We used the VQA-v2 dataset \cite{Goyal2019} with the standard splits (443K, 214K, and 453K question-image pairs for train/dev/test, the latter subdivided into  test-dev, test-standard, test-challenge and test-reserve). We adapted the implementation of \cite{Yu2019},\footnote{\url{https://github.com/MILVLG/mcan-vqa}} consisting of a Modular Co-Attention Network (MCAN). Our architecture is the same as \cite{Yu2019} except that we represent the image input with grid features generated by a ResNet \cite{He2016} pretrained on ImageNet \cite{Russakovsky2015}, instead of bounding-box features \cite{Anderson2018}. 
The images are resized to  before going through the ResNet that outputs a feature map of size . To represent the input question words we use 300-dimensional GloVe word embeddings \cite{pennington2014glove}, yielding a question feature matrix representation. Table~\ref{tab:table_hyperparams_VQA} shows the hyperparameters used for all the VQA experiments presented.

All the models we experimented with use the same features and were trained only on the train set without data augmentation.

\paragraph{Examples.} 
Figure~\ref{fig:examples_vqa_skate} illustrates the difficulties that continuous attention models may face when trying to focus on objects that are too far from each other or that seem to have different relative importance to answer the question. Intuitively, in VQA, this becomes a problem when counting objects in those conditions. On the other side, in counting questions that require the understanding of a contiguous region of the image only, continuous attention may perform better (see Figure~\ref{fig:examples_vqa_2birds}).

Figures~\ref{fig:examples_vqa_hat} and \ref{fig:examples_vqa_soccer} show other examples where continuous attention focus on the right region of the image and answers the question correctly. For these cases, discrete attention is more diffuse than its continuous counterpart: in both examples, it attends to two different regions in the image, leading to incorrect answers.

\begin{comment}
For the continuous attention models, the image size is normalized into the unit square  with each coordinate  positioned at  for , creating a meshgrid. We fit a bivariate Gaussian () or a bivariate truncated paraboloid () as the attention density; both correspond  to a score function , where   and  and  are location and scale parameters. We first use the discrete attention weights to compute  and use moment matching to compute  and , yielding a density  (continuous softmax) or 
 (continuous sparsemax), 
and a context vector , where  are  Gaussian RBFs 
, with  linearly spaced in a square grid  and , where  is the  identity matrix. Again, this does not increase the number of neural network parameters, compared to discrete attention. 
\end{comment}


\begin{table}[t]
    \caption{Hyperparmeters for VQA.}
    \label{tab:table_hyperparams_VQA}
    \begin{small}
    \begin{center}
    \begin{tabular}{llllll}
        \toprule
        \sc Hyperparameter & \sc Value  \\
        \midrule
        Batch size                  & 64    \\
        Word embeddings size        & 300     \\
        Input image features size   & 2048 \\
        Input question features size & 512 \\
        Fused multimodal features size & 1024 \\
        Multi-head attention hidden size        & 512 \\
Number of MCA layers        & 6 \\
        Number of attention heads   & 8 \\
        Dropout rate                & 0.1 \\
        MLP size in flatten layers  & 512 \\
        Optimizer                   & Adam \\
        Base learning rate at epoch  starting from 1 & \\
        Learning rate decay ratio at epoch   & 0.2 \\
        Number of epochs            & 13 \\
        
        \bottomrule
    \end{tabular}
    \end{center}
    \end{small}
\end{table}


\begin{figure*}[t]
\centering
\includegraphics[width=0.24\textwidth]{figures/skate-question}
\includegraphics[width=0.24\textwidth]{figures/skate-discrete}
\includegraphics[width=0.24\textwidth]{figures/skate-softmax}
\includegraphics[width=0.24\textwidth]{figures/skate-sparsemax}
\caption{\label{fig:examples_vqa_skate}Attention maps for an example in VQA-v2: original image, discrete attention, continuous softmax, and continuous sparsemax.}
\end{figure*}


\begin{figure*}[t]
\centering
\includegraphics[width=0.24\textwidth]{figures/2birds-question}
\includegraphics[width=0.24\textwidth]{figures/2birds-discrete}
\includegraphics[width=0.24\textwidth]{figures/2birds-softmax}
\includegraphics[width=0.24\textwidth]{figures/2birds-sparsemax}
\caption{\label{fig:examples_vqa_2birds}Attention maps for an example in VQA-v2: original image, discrete attention, continuous softmax, and continuous sparsemax.}
\end{figure*}



\begin{figure*}[t]
\centering
\includegraphics[width=0.24\textwidth]{figures/hat-question}
\includegraphics[width=0.24\textwidth]{figures/hat-discrete}
\includegraphics[width=0.24\textwidth]{figures/hat-softmax}
\includegraphics[width=0.24\textwidth]{figures/hat-sparsemax}
\caption{\label{fig:examples_vqa_hat}Attention maps for an example in VQA-v2: original image, discrete attention, continuous softmax, and continuous sparsemax.}
\end{figure*}



\begin{figure*}[t]
\centering
\includegraphics[width=0.24\textwidth]{figures/soccer-question}
\includegraphics[width=0.24\textwidth]{figures/soccer-discrete}
\includegraphics[width=0.24\textwidth]{figures/soccer-softmax}
\includegraphics[width=0.24\textwidth]{figures/soccer-sparsemax}
\caption{\label{fig:examples_vqa_soccer}Attention maps for an example in VQA-v2: original image, discrete attention, continuous softmax, and continuous sparsemax.}
\end{figure*}






    
\remove{
\subsection{Interval regression.}

The dataset may be downloaded from
\url{https://www.ncdc.noaa.gov/crn/}.
\begin{table}[!htb]
    \caption{Hyperparmeters for interval regression.}
    \label{tab:table_reg_hyperparams}
    \vskip 0.15in
    \begin{small}
    \begin{center}
    \begin{tabular}{llllll}
        \toprule
        \sc Hyperparameter
& \sc MSE
& \sc Gaussian-
& \sc Gaussian-
& \sc Truncated parabola
& \sc Triangular \\
\midrule
\emph{(tuned)} & & & & &\\
Learning rate &
0.002 &
0.001 &
0.0005 &
0.001 &
0.001 \\
Hidden layer size &
64 &
256 &
32 &
256 &
256 \\
Dropout probability &
0.2 &
0.1 &
0.2 &
0.1 &
0.05 \\
\emph{(fixed)} & & & & &\\
Optimizer & Adam & & & &\\
Batch& full dataset & & & &\\
N.~epochs& 50,000 & & & &\\
\bottomrule
    \end{tabular}
    \end{center}
    \end{small}
    \vskip -0.1in
\end{table}
} 
\end{document}
