We prove, in a constructive way, that the type inference problem for
\ETAS{} is decidable. Namely a type inference algorithm is
designed such that, for every lambda term $M$ it produces
a \emph{principal typing} from which all and only its typings can be
obtained by a suitable substitution. The substitution is a partial
function, defined if it satisfies a set of linear constraints.  If
there is not a substitution defined on its principal typing, then $M$
is not typable.  We will also prove that the computational complexity
of the type inference procedure is of the same order as the type
inference for simple type assignment system.

The design of the algorithm is based on the following Generation Lemma.
\begin{lem}[Generation Lemma] 
Let $\Gamma \mid \Delta \mid \Theta \vdash M:A$.
 \begin{enumerate}[\em i)]
\item Let $M\equiv x.$ If $A$ is linear, then either $\{x:A\} \subseteq \Gamma$ or $\{x:A\} 
\subseteq \Theta$. 
Otherwise, $\{x:A\}\in \Delta$.
\item Let $M \equiv \lambda x.N$. Then $A$ is of the shape $\underbrace{!...!}_{n}(B \lin C) 
\ (n \geq 0)   $.
\begin{enumerate}[$\bullet$]
\item Let $n=0$. If $B$ is linear then $\Gamma,x:B \mid \Delta \mid \Theta \vdash N:C$, else
$\Gamma \mid \Delta,x:B \mid \Theta \vdash N:C.$ 
\item Let $n >0$. Then $\emptyset \mid \Delta \mid \emptyset \vdash M:A$ and
$\Gamma' \mid \Delta' \mid \Theta' \vdash N:C$, where 
$\Delta = \underbrace{!...!}_{n}((\Gamma' \cup \Delta' \cup \Theta')-\{x:B\}  )   $.   
\end{enumerate}
\item Let $M \equiv PQ$. Then $A$ is of the shape $\underbrace{!...!}_{n}B  \ (n \geq 0)   $,
$\Gamma_{1}  \mid \Delta' \mid \Theta' \vdash P: C \lin \underbrace{!...!}_{m}B$ and  
$\Gamma_{2}  \mid \Delta' \mid \Theta' \vdash Q:C$, for some $m \leq n$.
\begin{enumerate}[\em(a)]
\item If $n-m=0$, then $\Gamma=\Gamma_{1}\cup \Gamma_{2}  $, $\Delta=\Delta'$ and $\Theta =\Theta'$.
\item If $n-m>0$, then $\emptyset \mid \Delta \mid \emptyset \vdash M:A$, where
$\Delta = \underbrace{!...!}_{n-m}( \Gamma_{1}\cup \Gamma_{2}\cup \Delta' \cup \Theta'  )  $.      
\end{enumerate}
 \end{enumerate}
 \end{lem}  
\noindent The principal typing is described through the notion of a
 \emph{type scheme}, which is an extension of that one used in~\cite{Coppola03tlca} in
 the context of $\LambdaEA$ and \NEAL.  Roughly speaking, a type scheme
 describes a class of types, i.e.  it can be transformed into a type
 through a suitable notion of a substitution.

\begin{defi}\hfill
\begin{enumerate}[i)]
\item\emph{Linear schemes} and \emph{schemes} are
  respectively defined by the grammars
\begin{align*}
  \mu & ::=\alpha \mid \sigma \linear \sigma\\
  \sigma &::= \mu \mid !^p\mu.
\end{align*}
where $\alpha$ belongs to a countable set of scheme variables and the \emph{exponential} $p$  
is defined by the grammar
\begin{displaymath}
 p::= a \mid p+p  
  \end{displaymath}
  where $a$ ranges over a countable set of \emph{literals}.
 Linear schemes are
  ranged over by $\mu, \nu$, schemes are ranged over by $\sigma,
  \tau, \rho$, exponentials are ranged over by $p,q,r$ and literals are ranged over by $a,b$.
  
 Note that the grammar does not generate nesting exponentials, i.e., $!^{p}!^{q}\alpha$ is not a correct scheme, while
 $!^{p+q}\alpha$ is correct. 
\item A \emph{modality set} is a set of linear constraints in the
form either $p=q$ or $p>0$ or $p=0$, where $p$ and $q$ are exponentials. Modality sets are ranged over by C.
\item A \emph{type scheme} is denoted by $\sigma\restriction_{C}$, where $\sigma$ is a scheme 
and $C$ is
a modality set. Type schemes will be ranged over by $\zeta, \theta$. Let $\mathcal{T}$ 
denote the set of type schemes.
  \item $\overline{\sigma \restriction_{C}}$ denotes the simple type skeleton underlying the 
 type scheme $\sigma\restriction_{C}$, and it is defined as follows:
  \begin{align*}
\overline{\alpha\restriction_{C}}=\alpha;\\
\overline{\sigma \lin \tau\restriction_{C}} =\overline{\sigma\restriction_{C}}\lin \overline{\tau\restriction_{C}}\\
 \overline{!^{p}\sigma\restriction_{C}} = \overline{\sigma\restriction_{C}}\\
\end{align*}
\item A \emph{scheme substitution} $S$ is a partial function from type schemes
  to types, replacing scheme variables by types and literals by
  natural numbers, in such a way that constraints in $C$ are satisfied. If $p$ is an exponential,
  let $S(p)$ be the result of applying the scheme substitution $S$ on all the literals in $p$, e.g. 
 if $p$ coincides with $a_{1}+...+a_{n}$, then $S(p)$ is $S(a_1)+...+S(a_n)$. 
 $C$ is satisfied by $S$ if, for every constraint
 $p=q$ ($p>0, p=0$) in $C$, $S(p)=S(q)$ ($S(p)>0, S(p)=0$)
Clearly the solvability of a
set of linear constraints is a decidable problem. 
The application of a substitution 
  $S$ satisfying $C$ to a type scheme $\sigma\restriction_C$ is defined inductively as follows:
  \begin{align*}
    S (\alpha\restriction_C) &=A &\texttt{ if } [\alpha \mapsto A]\in S;\\
    S (\sigma \linear \tau\restriction_C)& =S (\sigma\restriction_C) \linear S (\tau\restriction_C) &;\\
    S (!^{p}\mu\restriction_C)&=\underbrace{!...!}_n S (\mu\restriction_C) & 
    \texttt{ if } p=a_{1}+...+a_{m}\\
    & & n=S(a_1)+...+S(a_m).
  \end{align*}
   If $C$ is not satisfied by $S$, then $S(\sigma\restriction_C)$ is undefined.
\end{enumerate}
\end{defi}

Binary relation $\equiv$ is extended to denote the syntactical
identity between both types, schemes and type schemes.  Making clear what we
said before, a type scheme can be seen as a description of the set of
all types that can be obtained from it through a scheme
substitution defined on it. 
\medskip

A substitution is a total function from type schemes to type schemes mapping scheme variables 
to schemes, and generating some constraints. 
A substitution is denoted by a pair 
$\langle s,C\rangle$, where $s$ is a function from scheme variables to schemes and $C$ is a modality set. 
Substitutions will be ranged over by $t$. 
The behaviour of $\langle s,C\rangle$ is defined as follows.
\begin{align*}
    \langle s,C\rangle(\alpha\restriction_{C'}) &=\sigma\restriction_{C \cup C'} &\mbox{    if    } [\alpha \mapsto \sigma]\in s;\\
    \langle s,C\rangle (\sigma \linear \tau\restriction_{C'})& =\sigma' \linear \tau'\restriction_{C''} &\mbox{    if    } 
    \langle s,C\rangle(\sigma\restriction_{C'})=
    \sigma'\restriction_{C'''} \\
    & &\mbox{    and    } \langle s,C'''\rangle(\tau\restriction_{C'})=
    \tau'\restriction_{C''};\\
    \langle s,C\rangle (!^{p}\mu\restriction_{C'})&=!^{p}\nu\restriction_{C''} & \mbox{    if    }\langle s,C\rangle
    (\mu\restriction_{C'})=\nu\restriction_{C''};\\
    \langle s,C\rangle (!^{p}\mu\restriction_{C'})&=!^{r}\nu\restriction_{C''\cup \{r=p+q\}} & \mbox{    if    }
    \langle s,C\rangle(\mu\restriction_{C'})=!^{q}\nu\restriction_{C''} .
  \end{align*}
  Note that the last rule is necessary in order to preserve the 
  correct syntax of schemes, where the nesting of exponentials is not allowed. 

  In order to define the principal typing, we will use a unification
algorithm for type schemes, which is a variant of that defined in
\cite{Coppola03tlca}.  Let $=_e$ be the relation between type schemes
such that $\sigma\restriction_{C} =_{e}\tau\restriction_{C'}$ if $\overline{\sigma\restriction_{C}}\equiv \overline{\tau\restriction_{C}}$.

The unification algorithm, which we will present in Table
\ref{U-unification}, in Structured Operational Semantics style, is a function $U$ from $\mathcal{T}\times\mathcal{T}$
to substitutions.





\begin{table*}
\begin{center}
\fbox{
\begin{minipage}{.9\textwidth}
\begin{displaymath}
  \infer[(U1)]{U(\alpha\restriction_C, \alpha\restriction_{C'})=\langle [],\emptyset\rangle }{}
\end{displaymath}
\vspace{1mm}   
\begin{displaymath}
\infer[(U2)]{U(\alpha\restriction_C,\sigma\restriction_{C'})=\langle [\alpha\mapsto\sigma],\emptyset
    \rangle }{\alpha \textrm{
      does not occur in } \sigma }
      \end{displaymath}
\vspace{1mm}
\begin{displaymath}
\infer[(U3)]{U(\sigma\restriction_{C},\alpha\restriction_{C'})=
\langle [\alpha\mapsto\sigma],\emptyset\rangle }{\alpha \text{
      does not occur in } \sigma}
\end{displaymath}
\vspace{1mm}   
\begin{displaymath}
\infer[(U4)]{U(\sigma\linear\tau\restriction_{C},!^{q}\nu\restriction_{C'})=\langle s,C''\cup \{q = 0\}, \rangle}{U(\sigma\linear\tau\restriction_{C},\nu\restriction_{C'})=\langle s,C''\rangle}
\end{displaymath}
\vspace{1mm}
\begin{displaymath}
\infer[(U5)]{U(!^{p} \mu\restriction_{C} ,\sigma\linear\tau\restriction_{C'})=
\langle s,C''\cup \{p = 0\}\rangle}{U(\mu\restriction_{C}, \sigma\linear\tau\restriction_{C'})=\langle s,C''\rangle}
\end{displaymath}
\vspace{1mm}   
\begin{displaymath}
\infer[(U6)]{U(!^{p} \mu \restriction_{C} ,!^{q}\nu\restriction_{C'} )=
\langle s, C'' \cup \{p =q  \}\rangle }
{U(\mu\restriction_{C} ,\nu\restriction_{C'})=\langle s,C''\rangle }
\end{displaymath}
\vspace{1mm}   
\begin{displaymath}
\infer[(U7)]{U(\sigma_1 \linear \sigma_2\restriction_{C} ,\tau_1 \linear
  \tau_2\restriction_{C'})=\langle s_1,C_{1}\rangle \circ 
  \langle s_2,C_{2}\rangle}
  {\begin{array}{l}
  U(\sigma_1\restriction_{C} ,\tau_1\restriction_{C'} )=
  \langle s_{1},C_1\rangle\\
   U( \langle s_1,C_{1}\rangle(\sigma_2\restriction_{C}) , 
  \langle s_1,C_{1}\rangle(\tau_2\restriction_{C'}))=\langle
  s_2,C_{2}\rangle
  \end{array} }
\end{displaymath}

\vspace{1mm}  
$\langle s_1,C_{1}\rangle \circ 
  \langle s_2,C_{2}\rangle $ is the substitution such that \\
$\langle s_1,C_{1}\rangle \circ 
  \langle s_2,C_{2}\rangle(\sigma\restriction_{C})=\langle s_2,C_{2}\rangle(\langle s_1,C_{1}\rangle
  (\sigma\restriction_{C}))$.
\end{minipage}}
\end{center}
\caption{The unification algorithm $U$}\label{U-unification}
\end{table*}

The following lemma  proves that the function $U$ supplies a more general unifier for two type schemes,
in two steps: the substitution it generates is the most general unifier with respect to
the relation $=_{e}$, and moreover there is a most general scheme substitution
unifying the two type schemes modulo the syntactic equivalence $\equiv$.

\begin{lem}\label{e-unif}\hfill
\begin{enumerate}[\em i)]
\item (correctness) $U(\sigma\restriction_{C}, \tau\restriction_{C'})=\langle s,C''\rangle $ implies
  $\langle s,C''\rangle(\sigma\restriction_{C})=\sigma'\restriction_{C'''}$,   
  $\langle s,C''\rangle(\tau\restriction_{C'})=\tau'\restriction_{C''''}$ where $\sigma'\restriction_{C'''}=_{e}\tau'\restriction_{C''''}$.
  Moreover for every scheme substitution $S$, defined on both the type schemes,
  $S(\sigma'\restriction_{C'''})\equiv  S(\tau'\restriction_{C''''})$.
\item (completeness) $S(\sigma\restriction_{C})\equiv S(\tau\restriction_{C'})$ implies
  $U(\sigma\restriction_{C},\tau\restriction_{C'})=\langle s,C''\rangle $ 
and  there is $S'$ such that, for every type scheme $\zeta$, 
  $S(\zeta)=S'(\langle s,C''\rangle(\zeta))$, for some $S'$.
\end{enumerate}
\end{lem}

\proof\hfill
\begin{enumerate}[i)]
\item Easy, by induction on the rules defining $U(\sigma\restriction_{C}, \tau\restriction_{C'})$.
\item By induction on the pair ($n,m$), where $n$ and $m$ are respectively the number of scheme variables occurring in
both $\sigma$ and $\tau$ and the total number of symbols of $\sigma$ and $\tau$.\\
Let $\sigma \equiv \alpha$, and let $S(\sigma\restriction_{C})\equiv S(\tau\restriction_{C'})\equiv A$; clearly 
either $\tau \equiv \alpha$ or $\alpha$ cannot occur as proper subterm of $\tau$. 
In the first case $U(\alpha\restriction_{C},\alpha\restriction_{C'})=\langle [], \emptyset\rangle $, and
$S =S'$. In the second case
$U(\sigma\restriction_{C},\tau\restriction_{C'})=\langle [\alpha \mapsto \tau], \emptyset\rangle $. Then every 
scheme substitution $S'$, solving both $C$ and $C'$ and acting as $S$ on all the scheme 
variables occurring in $\sigma$ and $\tau$ but $\alpha$
does the desired job.\\
Let $\sigma \equiv \sigma_{1}\linear \sigma_{2}$ and $\tau \equiv \tau_{1}\linear \tau_{2}$.
So $S(\sigma_{1}\linear \sigma_{2}\restriction_{C})\equiv S(\sigma_{1}\restriction_{C})\linear S(\sigma_{2}\restriction_{C})$,
and $S(\tau_{1}\linear \tau_{2}\restriction_{C'})\equiv S(\tau_{1}\restriction_{C'})\linear S(\tau_{2}\restriction_{C'})$.
So by induction 
$U(\sigma_{1}\restriction_{C},\tau_{1}\restriction_{C'})=\langle s_{1}, C_{1}\rangle $,
$\langle s_{1}, C_{1}\rangle (\sigma_{1}\restriction_{C})= \sigma'\restriction _{C'_{1}}$
and $\langle s_{1}, C_{1}\rangle (\tau_{1}\restriction_{C'})= \tau'\restriction _{C''_{1}}$
where $\sigma'\restriction _{C'_{1}}=_{e}\tau'\restriction _{C''_{1}}$. 
Moreover there is $S_{1}$
such that $S(\sigma_{2}\restriction_{C}) \equiv S_{1}(\langle s_{1}, C_{1}\rangle
(\sigma_{2}\restriction _{C}))$ and
$S(\tau_{2}\restriction_{C'})\equiv S_{1}(\langle s_{1}, C_{1}\rangle
(\tau_{2}\restriction_{C'}))$.
In case $s_{1}$ is not empty, the number 
of scheme variables in both the type schemes is less than
in $\sigma $ and $\tau$; otherwise their total number of symbols is less than the one in 
$\sigma$ and $\tau$.
In both cases we can apply the induction hypothesis and conclude the proof.\\
All the other cases follow directly from the induction hypothesis.\qed
\end{enumerate}

The principal type scheme of a term is a pair in the form
$\langle\Sigma;\zeta\rangle$, where $\Sigma$ is a scheme context (i.e., a set of assignments
of the shape $x:\theta$, where no variable is repeated), and $\zeta$ is a
type scheme. 

In order to simplify the text of the algorithm, we will use the following conventions:
\begin{enumerate}[$\bullet$]
\item Let $\sigma $ be a scheme. $!^{p}\sigma $ denotes $!^{p}\mu$ in case $\sigma \equiv \mu$,
$!^{r}\mu$, where $r=p+q$ in case $\sigma \equiv !^{q}\mu$; if $\zeta \equiv \sigma \restriction_{C}$, then
$!^{p}\zeta $ denotes $!^{p}\sigma \restriction_{C}$;
\item Let $\Sigma $ be a scheme context. $!^{p}\Sigma$ denotes the scheme context 
$\{x:!^{p}\zeta \mid x:\zeta \in \Sigma \}. $
\end{enumerate}

The principal type scheme algorithm is defined in
Table~\ref{table-def-type-inf}.

\begin{table*}
\begin{center}
\scalebox{.9}{\framebox{\begin{minipage}{.9\textwidth}
\begin{itemize}
\item $ \PT(x)=\langle\{x:!^a \alpha\restriction_{\emptyset}\},
!^a \alpha\restriction_{\emptyset}\rangle$
\item \begin{tabbing}
    $\PT(\lambda x.M)=$ \= \texttt{let} \=
    $\PT(M)=\langle\Sigma,\sigma\restriction_{C}\rangle$ \texttt{in}\\ 
    \>\> \texttt{if} \=$x$ doesn't occur free in $M$\\
     \>\>$\langle!^b\Sigma, !^b(!^{a}\alpha\linear\sigma)\restriction_{C}\rangle$\\
    \>\> \texttt{else}\\
    \>\>\> \texttt{let} \=$\Sigma = \{x:\tau\restriction_{C'}\}\cup\Sigma'$ \texttt{in}\\
    \>\>\>\> \texttt{if} \= $x$ occurs multiple times in $M$\\
    \>\>\>\>\> \texttt{let} \=$U(\tau\restriction_{C'},!^{a}\alpha\restriction_{\{a>0\}})=\langle s_{1},C_1\rangle$ 
    \texttt{in}\\
\>\>\>\>\>\> \texttt{let} \=$\langle s_{1},C_1\rangle(\tau\restriction_{C'})=\tau'\restriction_{C''} $  \texttt{and}\\
  \>\>\>\>\>\>\>  $\langle s_{1},C_1\rangle(\sigma\restriction_{C})=\sigma'\restriction_{C'''} $ 
  \texttt{in}\\
    \>\>\>\>\>\>\> $\langle \langle s_{1},C_1\rangle(!^{b}\Sigma'), 
    !^{b}(\tau' \lin \sigma')\restriction_{C'' \cup C'''}\rangle$\\
    \>\>\>\> \texttt{else}\\
    \>\>\>\>\> $\langle !^{a}\Sigma', !^a(\tau\linear \sigma)\restriction_{C\cup C'} \rangle$
  \end{tabbing}
\item \begin{tabbing}
   $\PT(M_1\ M_2)=$ \= \texttt{let} \=
     $\PT(M_1) = \langle \Sigma_1, \sigma_1\restriction_{C_1} \rangle$ \texttt{,}
     $\PT(M_2) = \langle \Sigma_2, \sigma_2\restriction_{C_2}\rangle$ \\
 \>\>   \texttt{and let they are disjoint in}\\
 \>\> \texttt{let} \= $\{x_1,\ldots,x_k\} = dom(\Sigma_1)\cap
dom(\Sigma_2)$ \texttt{and} \\
\>\>\> $\forall 1\le i\le k\ \forall 1\le j\le 2\
x_i:\zeta_j^i\in\Sigma_j$ \texttt{and}\\
\>\>\> $U(\sigma_1\restriction_{C_1},(\sigma_2\linear!^{a}\alpha)\restriction_{C_2}) = \langle s',C'\rangle$
\texttt{and}\\
\>\>\> $U(\zeta_1^1,\zeta_2^1) = t_{1}$ \texttt{and}\\
\>\>\> $U(t_{i-1}(\zeta_1^i),t_{i-1}(\zeta_2^i)) = t_{i}$ \texttt{and} \\
\>\>\>\texttt{let} $t=t_{1}\circ t_{2}\circ...\circ t_{k}$ \texttt{in}\\
\>\>\>$\langle !^{b}(t(\Sigma_{1})\cup t(\Sigma_{2})), !^{b}(t(!^{a}\alpha)\restriction_{C'})\rangle$
  \end{tabbing}
\end{itemize}
$\alpha, a,b$ are fresh variables.
\end{minipage}
}}
\caption{The type inference algorithm $\mathit{PT}$.} \label{table-def-type-inf}
\end{center}
\end{table*}





\begin{thm}[Type Inference]\hfill 
\begin{enumerate}[\em i)] 
\item (correctness) If $\PT(M) = \langle \Sigma, \zeta\rangle$ then
for every scheme substitution
$S$ defined on all the type schemes occurring in $PT(M)$, $\exists \Gamma,\Delta,\Theta$ partitioning
$S(\Sigma)$ s.t. $\Gamma \mid \Delta \mid \Theta \vdash M: S(\zeta)$.
\item (completeness)  If $\Gamma \mid \Delta \mid \Theta \vdash M: A$ then
 $\PT(M)=\langle \Sigma,\zeta\rangle $ and there exists a scheme
 substitution $S$ defined on all the type schemes occurring in $PT(M)$ such that 
 $S(\Sigma) \subseteq \Gamma \cup \Delta \cup \Theta $ and $A=S(\zeta)$.
\end{enumerate} 
\end{thm}  

\proof\hfill
\begin{enumerate}[i)] 
\item By induction on $M$.
\begin{enumerate}[$\bullet$]
\item $M \equiv x$. Then $PT(M)=\langle\{x:!^a \alpha\restriction_{\emptyset}\};
!^a \alpha\restriction_{\emptyset}\rangle$.
Every scheme substitution $S$ satisfies the empty set of constraints. 

If 
$S(!^a \alpha\restriction_{\emptyset})$ is linear, then
take $\Delta=\emptyset$ and either $\Gamma =\{x:S(!^a \alpha\restriction_{\emptyset})\}$ 
and $\Theta = \emptyset$ 
or $\Theta = \{x:S(!^a \alpha\restriction_{\emptyset})\}$ and $\Gamma =\emptyset$. 
Otherwise choose 
$\Gamma =\Theta =\emptyset$, and $\Delta=\{x:S(!^a \alpha\restriction_{\emptyset})\}$.
\item $M \equiv \lambda x.P$. This case follows directly by induction.
\item $M\equiv M_{1}M_{2}$. 
Then $\PT(M_1) = \langle \Sigma_1, \theta_{1}\rangle$, 
$\PT(M_2) = \langle \Sigma_2, \theta_{2}\rangle$, and 
$PT(M)=\langle \Sigma',\theta' \rangle $,
where $\theta' $ is defined as in Table \ref{table-def-type-inf}.

Let $S$ be a scheme substitution defined on all the type schemes occurring in $PT(M)$:
note that, by the definition of the function $U$ this implies that $S$ is defined on both $PT(M_{1})$
and $\PT(M_2)$. Moreover, by construction of $PT$, $x:\zeta_{i} \in \Sigma_{i}$ ($1 \leq i \leq 2$)
implies $U(\zeta_{1},\zeta_{2})$ is defined, and $S(\zeta_{1})\equiv S(\zeta_{2})$,
by Lemma \ref{e-unif}.\\
By induction 
$\Gamma_{i} \mid \Delta_{i} \mid \Theta_{i}\vdash M_{i}:S(\theta_{i})$
($1 \leq i \leq 2$). Note that every type derivation for $M$ ends with an application of the rule $(E^{\lin})$, followed by a sequence, 
may be be empty, of rule ($!$). Since in rule $(E^{\lin})$ the two linear contexts are disjoint, we can build the partition of the contexts in
the following way:\\
$\Gamma_{1}=\{x:S(\zeta) \mid (\{x:S(\zeta)\}\subseteq S(\Sigma_{1})) \wedge
(x\in FV(M_{1}) \wedge x\not \in FV(M_{2})) \wedge
(S(\zeta) \mbox{ is linear })\}$,\\
$\Gamma_{2}=\{x:S(\zeta) \mid \{x:S(\zeta)\}\subseteq S(\Sigma_{2}) 
\wedge (x\in FV(M_{2}) \wedge x\not \in FV(M_{1})) \wedge
(S(\zeta) \mbox{ is linear })\}$,\\
$\Delta_{i}=\{x:S(\zeta) \mid (\{x:S(\zeta)\}\subseteq S(\Sigma_{i}))  \wedge
(S(\zeta) \mbox{ is modal })\}$ and\\
$\Theta_{i}=\Sigma_{i}-(\Gamma_{i} \cup \Delta_{i})  $ ($1 \leq i \leq 2$).\\
 By the weakening Lemma, we have that 
$\Gamma_{i} \mid \Delta_{1},\Delta_{2} \mid \Theta_{1},\Theta_{2}
\vdash M_{i}:S(\theta_{i})$ ($1 \leq i \leq 2$). 
Since $S(\theta_{1})\equiv S(\theta_{2})\lin
S(!^{a}\alpha\restriction_{C_{2}}))$ (by Lemma \ref{e-unif}.i), 
the proof follows by rule ($E_{\lin}$).
\end{enumerate}
\item By induction on the derivation proving $\Gamma \mid \Delta \mid \Theta \vdash M: A$.

Let the last used rule be ($A^{L}$). Then $M \equiv x$, and $\{x:A\} \subseteq\Gamma $. By definition, 
$ \PT(x)=\langle\{x:!^a \alpha\restriction_{\emptyset}\};!^a \alpha\restriction_{\emptyset}
\rangle$, and the proof follows easily.

The case ($A^{P}$) is similar.

The cases $(I^{L}_{\lin})$ and $(I^{I}_{\lin})$ both follow by induction and weakening lemma.

Let us consider the case when the last used rule is $(E_{\lin})$. 
Then $M \equiv M_{1}M_{2}$, and 
$\Gamma_{1}\mid \Delta \mid \Theta \vdash M_{1}:B \lin A $ and 
$\Gamma_{2}\mid \Delta \mid \Theta \vdash M_{2}:B $, for some $B$,
where $\Gamma_{1}$ and $\Gamma_{2}$ are disjoint.
By induction $\PT(M_i) = \langle \Sigma_i, \zeta_{i}\rangle$ and 
there is $S_{i}$ defined on all type schemes in $PT(M_i)$ such that $S_{1}(\zeta_{1})\equiv B\lin A$,
and $S_{2}(\zeta_{2})\equiv B $.\\
Moreover 
$S_{i}(\Sigma_{i})\subseteq \Gamma_{i}\cup \Delta \cup \Theta$ ($1 \leq i \leq 2$). 
Since by construction $PT(M_{1})$ and $PT(M_{2})$ are disjoint, there is a well defined scheme substitution
$S$ acting as both $S_{1}$ and $S_{2}$ and such that 
$S(!^a\alpha\restriction_{\emptyset})\equiv A$, for $\alpha, a$ fresh. So
if $\zeta_{2}\equiv \sigma\restriction_{C}$,
$S(\zeta_{1})\equiv S(\sigma\lin !^{p_{1}}\alpha\restriction_{C})$, 
and $S$ is defined on all the type schemes in
$PT(M_i)$ ($1\leq i \leq 2$).
Then 
by Lemma \ref{e-unif}.ii), 
$U(\zeta_{1},\sigma\lin !^{p_{1}}\alpha\restriction_{C})=\langle s',C' \rangle$.
Since $S$ satisfies all the constraints in $PT(M_{i})$ ($1\leq i \leq 2$), if 
$x:\theta_{1}$ and $x:\theta_{2}$ belong to $\Sigma_{1}$ and $\Sigma_{2}$
respectively, then $S(\theta_{1})\equiv S(\theta_{2})$, and so, by Lemma \ref{e-unif}.ii), 
they can be unified.
So $PT(M_{1}M_{2})$ is defined, and by 
induction it enjoys the desired properties.

Let the last used rule be $(!)$. Then $A \equiv !A'$ and the premise of the rule is 
$\Gamma' \mid \Delta' \mid \Theta' \vdash M:A'$, where 
$!\Gamma' \cup !\Delta' \cup ! \Theta' \subseteq \Delta $.
By induction $PT(M)=\langle \Sigma, \zeta \rangle$ and there is a scheme substitution $S'$
such that $S'(\Sigma)\subseteq \Gamma' \cup \Delta' \cup \Theta' $.
Let $S$ be such that $S$ is defined on all the type schemes in $PT(M)$, and
such that $S(\zeta)\equiv !S'(\zeta)$, for all 
type scheme $\zeta$: it is easy to check that, if $S'$ is defined, i.e., it satisfies all the constraints in $PT(M)$, than $S$ is 
well defined too, and so it does the right job.\qed
\end{enumerate}



\noindent The complexity of the type inference algorithm $PT$ is of
the same order as the type inference algorithm for simple types. In
order to prove this, we need some notations.  If $A(n)$ is an
algorithm running on a datum $n$, let us denote by $|A(n)|$ its
asymptotic complexity.  Moreover, if $\sigma$ is a scheme, let
$|\sigma | $ be the number of symbols in it.  Let $\mathit{TA}(M)$ be
the type inference algorithm for simple types running on a term
$M$. By abuse of notation, we assume that, for every type scheme
$\sigma$, $\overline{\sigma}$ denotes a simple type: in fact the
syntax of type schemes, when erasing exponentials and constraints,
coincides with that of simple types.
\begin{thm}[Complexity]
$PT(M) \in \mathit{O}(L(M)+|\mathit{TA}(M)|)$.
\end{thm}
\proof
First of all, let us observe that the the unification algorithm $U$ coincide with the Robinson unification, when it is applied to two type schemes
whose set of constraints is empty, so, if $RU$ denotes the Robinson's unification, 
$|U(\sigma \restriction_{\emptyset},
\tau \restriction_{\emptyset} )|=|RU (\overline{\sigma},\overline{\tau})|$. 
Then 
$|U( \sigma\restriction_{C} , 
  \tau\restriction_{C'})|\leq |RU(\overline{\sigma},\overline{\tau})| +|\sigma | + |\tau |$.
Remember that Robinson unification is equivalent to the principal simple type assignment.
Moreover it is easy to see that, if $PT(M)=\langle \Sigma, \sigma \restriction_{C} \rangle$, then 
$\mathit{TA}(M)=\langle\overline{\Sigma}, \overline{\sigma}\rangle$, when $\overline{\Sigma}$ denotes the context obtained 
from $\Sigma$ by applying the function $\overline{[.]}$ to all types in it.

The proof is by induction on $M$. If $M $ is a constant the proof is trivial.
If $M \equiv \lambda x. N$, then $|PT(M)|= |PT(N)| + |U(\tau\restriction_{C}, !^{a}\alpha \restriction_{a>0})| =
|PT(N)| + k$, for a constant $k$, since $\alpha$ is a scheme variable. Then the result follows by induction.
Let $M\equiv PQ$. Then $|PT(PQ)| = |PT(P)| + |PT(Q)| + |U(\sigma\restriction_{C}, 
  (\tau\linear !^{a}\alpha)\restriction_{C'})| + |U(\Sigma_{1},\Sigma_{2})|$ 
if $PT(P)=\langle \Sigma_{1},\sigma\restriction_{C} \rangle$ and 
  $PT(Q)=\langle \Sigma_{2},\tau\restriction_{C'}\rangle $. $U(\Sigma_{1},\Sigma_{2})$ is an abbreviation for the unification of the 
  two scheme contexts $\Sigma_{1}$ and $\Sigma_{2}$, as specified in the algorithm. 
 By induction $PT(PQ)= |\mathit{TA}(P)| +k_{1} + |\mathit{TA}(Q)| + k_{2} + |RU(\overline{\sigma},\overline{\tau})| +|\sigma| +|\tau|+
 |RU(\overline{\Sigma_{1}},\overline{\Sigma_{2}}) + |\Sigma_{1}| + |\Sigma_{2}|$.
 Remembering that $|\mathit{TA}(PQ)|= |\mathit{TA}(P)| + |\mathit{TA}(Q)| + |RU(\overline{\sigma},\overline{\tau})| +
 |RU(\overline{\Sigma_{1}},\overline{\Sigma_{2}})|$, the result follows.\qed
 






\begin{exa}
  Let us illustrate the application of the type inference algorithm to
  the term:
$$\underline{2}\mbox{ }\underline{3}\equiv (\lambda xy. x(xy))\lambda
xy.x(x(xy)).
$$ 
  Let $a,b,c,d,e,f,g,h,k,i,m,n,p,q,r$ be literals and $\alpha,
  \beta,\gamma, \delta,\epsilon$ be scheme variables.  First we will
  build PT(\underline{2}). Starting from
$$PT(x)=\langle \{x:!^{a}\alpha\restriction_{\emptyset}\},!^{a}\alpha\restriction_{\emptyset} \rangle 
\mbox{ and } PT(y)=\langle
\{y:!^{b}\beta\restriction_{\emptyset}\},!^{b}\beta\restriction_{\emptyset}
\rangle,
$$
  due to $U(!^{a}\alpha\restriction_{\emptyset},!^{b}\beta \linear
  !^{c}\gamma\restriction_{\emptyset})=\langle [\alpha \mapsto
  !^{b}\beta \linear !^{c}\gamma ], \{a=0\} \rangle$, we obtain:
$$PT(xy)=\langle \{x:!^{d+a}(!^{b}\beta \linear !^{c}\gamma)\restriction_{C_{0}}, y:!^{d+b}\beta\restriction_{C_{0}}\}, 
!^{c+d}\gamma\restriction_{C_{0}}\rangle
$$ 
  where $C_{0}=\{a=0\}$.

  Now a fresh version of $PT(x)$ is needed, so consider $\langle
  \{x:!^{a_{1}}\alpha_{1}\restriction_{\emptyset}\},!^{a_{1}}\alpha_{1}
  \restriction_{\emptyset} \rangle$.  The rule for application allows
  us to perform certain unifications.  First we obtain
$$U(!^{a_{1}}\alpha_{1}\restriction_{\emptyset}, (!^{c+d}\gamma \linear !^{e} \delta)\restriction_{C_{0}})=
\langle [\alpha_{1}\mapsto !^{c+d}\gamma \linear !^{e} \delta],
C_{1}\rangle
$$ 
  where $C_{1}=\{a_{1}=0 \}$.  A second unification is
  necessary for unifying the two premises on $x$ in the first
  component of $PT(xy)$ and $PT(x)$, respectively:
$$U(!^{a_{1}}\alpha_{1}\restriction_{\emptyset}, !^{d+a}(!^{b}\beta \linear !^{c}\gamma)\restriction_{C_{0}})=
\langle [\alpha_{1}\mapsto !^{b}\beta \linear !^{c}\gamma], C_{2}
\rangle
$$ 
  where $C_{2}= \{a_{1}=a+d\} $.  By composing the two substitutions,
  we have $\langle [\gamma \mapsto \beta, \gamma \mapsto \delta],
  \{c+d=b, e=c \}\rangle$.  So 
$$PT(x(xy))=\langle
  \{x:!^{h+a_{1}}(!^{b }\beta \linear !^{e}\beta)\restriction_{C_{3}},
  y:!^{h+b+d}\beta\restriction_{C_{3}}\}, !^{h+e }\beta
  \restriction_{C_{3}} \rangle
$$
  where $C_{3}=C_{0}\cup C_{1}\cup C_{2}\cup \{c+d=b, c=e\}.$ By
  applying the rule for the abstraction, we obtain:
$$PT(\lambda y.x(xy))=\langle \{x:!^{f}(!^{b }\beta \linear !^{e}\beta)\restriction_{C_{3}}\}, !^{k}(!^{h+b+d}\beta \linear
!^{h+e} \beta)\restriction_{C_{3}}   \rangle
$$
  and
$$PT(\lambda xy.x(xy)) = \langle\emptyset; !^{i}(!^{f}(!^{b}\beta \linear !^{e}\beta)\linear !^{k}(!^{h+b+d}\beta \linear
!^{h+e} \beta))\restriction_{C_{4}}   \rangle
$$
  where $C_{4}=C_{3}\cup \{f=k+h+a_{1}, f>0\}$.

  Due to the particular form of $PT(\lambda xy.x(xy))$, we can deduce
  that the term $\lambda xy.x(xy)$ can be assigned, among others,
  the following types
$$!(A \lin A)\lin !(A\lin A), !(A\lin A)\lin !A \lin !A, !!(A\lin
A)\lin !(!A \lin !A).
$$
  In particular, the scheme substitution that replaces $\beta$ by $!(A
  \lin A)$, furthermore $b$, $e$, $h$, $d$ and $a_1$ by $0$, and both
  $k$ and $f$ by $1$, satisfies the constraints and generates the
  typing $\emptyset \mid \emptyset \mid \emptyset \vdash
  \underline{2}:!(A \lin A) \lin !(A \lin A)$, whose derivation is
  shown in Example \ref{exa:terms}.

  In order to build the principal type scheme of $\underline{3}$, we
  need to start from two fresh copies of $PT(x)$ and $PT(x(xy))$, let
$$\langle \{x:!^{n} \epsilon \restriction_{\emptyset}\}, !^{n} \epsilon \restriction_{\emptyset}\rangle \mbox{   and   }
\langle \{x:!^{h'+a'_{1}}(!^{b'}\alpha \linear !^{e'}\alpha)\restriction_{C'_{3}}, y:!^{h'+b'+d'}\alpha\restriction_{C'_{3}}\},
!^{h'+e' }\alpha \restriction_{C'_{3}} \rangle
$$
  where $C'_{3}=\{a'=0,a'_{1}=0,a'_{1}=d'+a', b'=c'+d', e'=c' \}$.

  By applying the rule for application, we obtain, for the first
  unification:
$$U(!^{n} \epsilon\restriction_{\emptyset},!^{h'+e' }\alpha \linear !^{p}\eta\restriction_{C'_{3}} )= 
\langle [\epsilon \mapsto !^{h'+e' }\alpha \linear !^{p}\eta ],
C_{5}\rangle
$$
  where $C_{5}=\{n=0\}.$ By unifying the two premises on $x$, we have
$$U(!^{n} \epsilon\restriction_{\emptyset},
!^{h'+a'_{1}}(!^{b '}\alpha \linear !^{e'}\alpha)\restriction_{C'_{3}} )=\langle [\epsilon \mapsto !^{b '}\alpha \linear !^{e'}\alpha] 
, C_{6} \rangle
$$ 
  where $C_{6 }= \{n=h'+a'_{1}\}$. So, composing the two substitutions:
$$PT(x(x(xy)))=\langle \{x:!^{q+h'+a'_{1}}(!^{b'}\alpha \linear !^{e'}\alpha)\restriction_{C_{7}}, 
y:!^{q+h'+b'+d'} \alpha\restriction_{C_{7}} \},
!^{p+q}\alpha\restriction_{C_{7}} \rangle$$
  where $C_{7}=C'_{3}\cup C_{5}\cup C_{6}\cup \{b'=h'+e',e'=p  \}$.
  So, by applying the rules for abstraction, we have:
$$PT(\lambda y.x(x(xy)) )=\langle \{x:!^{r+q+h'+a'_{1}}(!^{b'}\alpha
\linear !^{e'}\alpha)\restriction_{C_{7}} \}, !^{r}(!^{q+h'+b'+d'}
\alpha \linear !^{p+q}\alpha)\restriction_{C_{7}}
\rangle
$$
  and
$$PT(\lambda x y.x(x(xy)) )=\langle \emptyset,!^{s}(!^{g}(!^{b'}\alpha
\linear !^{e'}\alpha) \linear !^{r}(!^{q+h'+b'+d'} \alpha \linear
!^{p+q}\alpha))\restriction_{C_{8}} \rangle
$$
  where $C_{8}=C_{7}\cup \{g=r+q+h'+a'_{1}, g>0   \}$.

  It is easy, but boring, to check that the typings for
  $\underline{3}$ are the same that the ones for $\underline{2}$, by
  inspecting the modality set.  Now, in order to build
  $PT(\underline{2}\mbox{ }\underline{3})$, we need to unify the two
  type schemes:
$$\sigma \equiv !^{i}(!^{k+h+a_{1}}(!^{b }\beta \linear
!^{e}\beta)\linear !^{k}(!^{h+b+d}\beta \linear !^{h+e}
\beta))\restriction_{C_{4}}
$$
  and 
$$\tau \equiv !^{s}(!^{g}(!^{b'}\alpha \linear !^{e'}\alpha) \linear
!^{r}(!^{q+h'+b'+d'} \alpha \linear !^{p+q}\alpha))\linear
!^{t}\gamma)\restriction_{C_{8}}
$$ 
  obtaining:
$$U(\sigma, \tau)=\langle [\beta \mapsto !^{b'}\alpha\linear !^{e'}\alpha, \gamma \mapsto !^{h+b+d}\beta \linear !^{h+e}\beta] , C_{9}  
\rangle
$$
  where $C_{9}=\{ i=0, k+h+a_{1}=s,t=k,s=k+h+a_{1},b=g, e=r,
  b'=q+h'+b'+d',e=p+q \}$. So
$$PT(\underline{2}\mbox{ }\underline{3})=\langle \emptyset, !^{t}(!^{h+b+d}(!^{b'}\alpha \linear !^{e'}\alpha)\linear !^{h+e}(!^{b'}\alpha
\linear !^{e'}\alpha))\restriction_{C_{10}}  \rangle
$$
  where $C_{10}= C_{8}\cup C_{9}$.

  Finally, the scheme substitution that replaces $\alpha$ by $A$,
  furthermore $b$, $e$, $b'$ and $e'$ by $0$, and both $t$ and
  $h$ by $1$, satisfies the constraints and generates the typing
  $\emptyset \mid \emptyset \mid \emptyset \vdash\underline{2}\mbox{ }
  \underline{3}:!(!(A\lin A)\lin!(A \lin A))$, whose derivation is
  shown in Example \ref{exa:terms}.
\end{exa}

