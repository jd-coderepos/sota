\documentclass[11pt,a4paper]{article}
\usepackage[hyperref]{acl2020}
\usepackage{times}
\usepackage{latexsym} 
\usepackage{amssymb}
\usepackage{amsmath}
\usepackage{booktabs}
\usepackage{url} 
\usepackage{enumitem}
\usepackage{mathtools}
\usepackage{fontawesome} 
\usepackage{contour}
\usepackage{subcaption}
\usepackage{fdsymbol}





\usepackage{pifont}\newcommand{\cmark}{\ding{51}}\newcommand{\xmark}{\ding{55}}

\newcommand{\ul}[1]{\uline{\phantom{#1}}\llap{\contour{white}{#1}}
}

\renewcommand{\UrlFont}{\ttfamily\small}
\setitemize{noitemsep,topsep=0pt,parsep=0pt,partopsep=0pt}

\usepackage{microtype}
\usepackage{color, colortbl}

\definecolor{Gray}{gray}{0.85}
\definecolor{LightCyan}{rgb}{0.88,1,1}
\aclfinalcopy \def\aclpaperid{1530} 


\newcommand{\xiang}[1]{{\color{red}(Xiang: #1)}}
\newcommand{\yuchen}[1]{{\color{red}Yuchen: #1}}
\newcommand{\chandra}[1]{{\color{brown}Chandra: #1}}
\newcommand{\yejin}[1]{{\color{magenta}Yejin: #1}}

\newcommand\BibTeX{B\textsc{ib}\TeX}

\title{\vspace*{-0.5in}
{{\small \hfill EMNLP-Findings'20}\\
\vspace*{.25in}}  {C}ommon{G}en: {A} {C}onstrained {T}ext {G}eneration {C}hallenge \\ for {G}enerative {C}ommonsense {R}easoning}
 

 
\author{ 
Bill Yuchen Lin\textsuperscript{{$\varheartsuit$}}\quad  Wangchunshu Zhou\textsuperscript{{$\varheartsuit$}}\quad
Ming Shen\textsuperscript{{$\varheartsuit$}} \quad 
Pei Zhou\textsuperscript{{$\varheartsuit$}} \quad 
\\ \textbf{Chandra Bhagavatula\textsuperscript{{$\spadesuit$}} \quad Yejin Choi\textsuperscript{{$\spadesuit \vardiamondsuit$}}
\quad Xiang Ren\textsuperscript{{$\varheartsuit$}}
}
\\
{{\textsuperscript{$\varheartsuit$}University of Southern California}} \quad {\textsuperscript{$\spadesuit$}Allen Institute for Artificial Intelligence} \\ 
{\textsuperscript{$\vardiamondsuit$}Paul G. Allen School of Computer Science \& Engineering, University of Washington}\\
{
\texttt{\{yuchen.lin,xiangren\}@usc.edu,\{chandrab,yejinc\}@allenai.org}} 
} 
 
\renewcommand\footnotemark{}

\begin{document} 
\maketitle



\begin{abstract}




Recently, large-scale pretrained language models have demonstrated impressive performance on several commonsense-reasoning benchmark datasets. However, building machines with commonsense to compose realistically plausible sentences remains challenging.
In this paper, we present a constrained text generation task, \textsc{CommonGen} associated with a benchmark dataset, to explicitly test machines for the ability of \textit{generative commonsense reasoning}.
Given a set of common concepts (e.g., \{dog, frisbee, catch, throw\}); the task is to generate a coherent sentence describing an everyday scenario using these concepts (e.g., ``a man throws a frisbee and his dog catches it'').

The \textsc{CommonGen} task is challenging because it inherently requires 1) \textit{relational reasoning} with background  commonsense knowledge, and 2) \textit{compositional generalization} ability to work on unseen concept combinations.
Our dataset, constructed through a combination of crowdsourced and existing caption corpora, consists of 
79k commonsense descriptions over 35k unique concept-sets.
Experiments show that there is  a large gap between state-of-the-art text generation models (e.g., T5) and human performance.
Furthermore, we demonstrate that the learned generative commonsense reasoning capability can be transferred to improve downstream tasks 
by generating additional context.
\end{abstract}

  \section{Introduction}
\label{sec:intro}

\begin{figure}[ht!]
	\centering
	\includegraphics[width=1.\linewidth]{intro}
	\caption{\small{\textbf{An example of the dataset of \textsc{CommonGen}.} GPT-2, UniLM, BART and T5 are large pre-trained text generation models, \textit{fine-tuned} on the proposed task.}
	\vspace{-1em}
	}
	\label{fig:intro}
\end{figure}

Commonsense reasoning, 
the ability to make acceptable and logical assumptions about ordinary scenes in our daily life,
has long been acknowledged as a critical bottleneck of artificial intelligence and natural language processing~\cite{davis2015commonsense}.
Most recent commonsense reasoning challenges, such as CommonsenseQA~\cite{Talmor2018CommonsenseQAAQ}, SocialIQA~\cite{sap-etal-2019-social}, WinoGrande~\cite{Sakaguchi2019WINOGRANDEAA} and HellaSwag~\cite{Zellers2019HellaSwagCA}, have been framed as \textit{discriminative} tasks -- i.e. AI systems are required to \textit{choose} the correct option from a set of choices based on a given context. 
While significant progress has been made on these discriminative tasks, 
we argue that commonsense reasoning in text generation poses a distinct complementary challenge.
In this paper, we advance machine commonsense towards \textit{generative} reasoning ability.


	\begin{figure*}[h!]
		\centering
		\includegraphics[width=1\linewidth]{challenges.pdf}
		\caption{\small{Two \textbf{key challenges} of \textsc{CommonGen}:  \textit{relational reasoning} with underlying commonsense knowledge about given concepts (left), and \textit{compositional generalization} for  unseen combinations of concepts (right).}}
		\label{fig:challenges}
	\end{figure*}
Humans acquire the ability to compose sentences by learning to understand and use common concepts that they recognize in their surrounding environment~\cite{tincoff1999some}. The acquisition of such an ability is regarded as a significant milestone of human development~\cite{moore2013development}.  \textit{Can machines acquire such generative commonsense reasoning ability?} 
To initiate the investigation, we present \textsc{CommonGen}\footnote{\url{http://inklab.usc.edu/CommonGen/}.} -- a novel constrained generation task that requires machines to generate a sentence describing a day-to-day scene using concepts from a given \textit{concept-set}. 
For example, in Figure~\ref{fig:intro}, given a set of concepts: \{\textit{dog}, \textit{frisbee}, \textit{catch}, \textit{throw}\}, machines are required to generate a sentence such as ``a man \textit{throws} a {frisbee} and his \textit{dog} \textit{catches} it in the air.''


To successfully solve the task, models need to incorporate two key capabilities: a) \textit{relational reasoning}, and b) \textit{compositional generalization}. 
Grammatically sound sentences may not always be realistic as they might violate our commonsense (e.g., \textit{``a dog throws a frisbee ..."}). In order to compose a plausible sentence that describes an everyday scenario, models need to construct a grammatical sentence while adhering to and reasoning over the commonsense relations between the given concepts.
Models additionally need \textit{compositional generalization} ability to infer about unseen concept compounds. This encourages models to reason about a potentially infinite number of novel combinations of familiar concepts -- an ability believed to be a limitation of current AI systems~\cite{lake2018generalization, keysers2020measuring}.



























































Therefore, in support of the \textsc{CommonGen} task, we present a dataset consisting of 35,141 concept-sets associated with 77,449 sentences.
We explicitly design our dataset collection process to capture the key challenges of relational reasoning and compositional generalization described above, through an actively controlled crowd-sourcing process.
We establish comprehensive baseline performance for state-of-the-art language generation models with both extensive automatic evaluation and manual comparisons. The best model, based on T5 \cite{raffel2019exploring}, achieves 28.86\% with significant gap compared to human performance of 52.43\% in the \texttt{SPICE} metric -- demonstrating the difficulty of the task.
Our analysis shows that state-of-the-art models struggle at the task, generating implausible sentences -- e.g. ``dog throws a frisbee ..." , ``giving massage to a table", etc. 
Additionally,
we show that successful \textsc{CommonGen} models can benefit downstream tasks (e.g., commonsense-centric question answering) via generating useful context as background scenarios.
We believe these findings point to interesting future research directions for the community of commonsense reasoning. 






	 
	
	\section{Task Formulation and Key Challenges}
	\label{sec:formualtion}
	We formulate the proposed \textsc{CommonGen} task with mathematical notations and discuss its inherent challenges with concrete examples. 
	The input is an unordered set of $k$ concepts $x=\{c_1,c_2,\dots,c_k\}\in \mathcal{X}$ (i.e. a concept-set), where each concept $c_i\in \mathcal{C}$ is a common object (noun) or action (verb). 
	We use $\mathcal{X}$ to denote the space of all possible concept-sets and use $\mathcal{C}$ to denote the concept vocabulary (a subset of ConceptNet's unigram concepts).
	The expected output is a simple, grammatical sentence $y\in\mathcal{Y}$ that describes a common scenario in our daily life, using  all given concepts in $x$ (morphological inflections are allowed).
	A scenario can depict either a static situation or a short series of actions.
The \textsc{CommonGen} task is to learn a function $f:
	\mathcal{X} \rightarrow \mathcal{Y}$, which maps a concept-set $x$ to a sentence $y$. 
The unique challenges of this task come from two aspects:
	


	\smallskip
	\noindent
	\textbf{Relational Reasoning with Commonsense.}
	Expected generative reasoners should prioritize the most plausible scenarios over many other less realistic ones.
	As shown in Figure~\ref{fig:challenges},
	models need to recall necessary relational commonsense facts that are relevant to the given concepts, and then reason an optimal composition of them for generating a desired sentence.
In order to complete a scenario,  generative commonsense reasoners also need to reasonably associate additional concepts (e.g., `woman',  `gym') as agents or background environments for completing a coherent scenario. 


	This not only requires understanding underlying commonsense relations between concepts, 
	but also incrementally composing them towards a globally optimal scenario.
	The underlying reasoning chains are  
	inherently based on a variety of background knowledge such as spatial relations,  object properties, physical rules, temporal event knowledge, social conventions, etc. 
	However, they may not be recorded in any existing knowledge bases.




	
	\smallskip
	\noindent
	\textbf{Compositional Generalization.}
Humans can compose a sentence to describe a scenario about the concepts they may never seen them co-occurring.
	For example, in Figure~\ref{fig:challenges}, there is a testing concept-set $\hat{x}=$\{\textit{pear}, \textit{basket}, \textit{pick}, \textit{put}, \textit{tree}\}.
	The concept `pear' never appear in the training data, and `pick' never co-occurs with `basket'. 
We, humans, can generalize from these seen scenarios in the training data and infer that a plausible output: $\hat{y}=$\textit{``a girl picks some pears from a tree and put them into her basket.''} 
This compositionally generalization ability via analogy, i.e., to make ``infinite use of finite means''~\cite{chomsky1965aspects}, is challenging for machines.
This analogical challenge not only requires inference about similar concepts (e.g., `apple' $\rightarrow$ `pear') but also their latent associations.



	 	
	\section{Dataset Construction and Analysis}
	\label{sec:dataset}
Figure~\ref{fig:builddata}
illustrates the overall workflow of our data construction for the proposed \textsc{CommonGen} task.	
We utilize several existing caption corpora for sampling frequent concept-sets (Sec.~\ref{ssec:conceptset}) for reflecting common scenarios.
	We employ AMT crowd workers for collecting human-written sentences (Sec.~\ref{ssec:amt}) for the development and test set, while we carefully monitor the quality of crowd workers and refine them dynamically.
	Finally, we present the statistics of the \textsc{CommonGen} dataset, and the analysis on the challenges  (Sec.~\ref{ssec:analysis}).
	
	
\begin{figure}[t!]
		\centering
		\includegraphics[width=0.9\linewidth]{builddata.pdf}
		\caption{ \textbf{Dataset construction workflow overview.} 
		\vspace{-1em}
		}
		\label{fig:builddata}
	\end{figure}

	\subsection{Collecting Concept-Sets from Captions}
	\label{ssec:conceptset}
It can be unreasonable to present any \emph{arbitrary} set of concepts (e.g., $x=$\{\textit{apple, fold, rope}\}) and ask a reasoner to generate a commonsense scenario, since such an arbitrary set of concepts can be too unrelated. Therefore, our concept-sets are supposed to reflect reasonable concept co-occurrences in everyday situations.
As web images and video clips capture diverse everyday scenarios, 
    we use their caption text as a natural resource for collecting concept-sets and their corresponding descriptions of commonsense scenarios.
More specifically, we collect visually-grounded sentences from several existing caption datasets, including {image captioning} datasets, such as  {Flickr30k}~\cite{young-etal-2014-image}, {MSCOCO}~\cite{Lin2014MicrosoftCC}, {Conceptual Captions}~\cite{Sharma2018ConceptualCA}, as well as {video captioning} datasets including {LSMDC}~\cite{lsmdc}, {ActivityNet}~\cite{krishna2017dense}, and {VATEX}~\cite{Wang_2019_ICCV}.
	
We first conduct part-of-speech tagging over all sentences in the corpora such that words in sentences can be matched to the concept vocabulary of ConceptNet.
	Then, we compute the sentence frequency of 
	concept-sets consisting of  3$\sim$5 concepts.
	That is, for each combination of three/four/five concepts in the vocabulary, we know how many sentences are in the corpora covering all concepts.
	
Ideally, we want the selected concept-sets in our dataset to reflect the natural distribution of concept-sets in the real world. 
At first glance, a reasonable solution may seem to sample from the distribution of the concept-sets based on their frequencies in the source datasets. 
However, we find that this method leads to a rather unnaturally skewed collection of concept-sets, due to the inherent data biases from the source datasets.   
We therefore design a function to score a concept-set $x$ based on  \textit{scene diversity} and  \textit{inverse frequency penalty}.
	We denote $S(x)$ as the set of unique sentences that contain all given concepts $\{c_1, c_2, \dots, c_k\}$, and then we have  
	{{$$ \texttt{score}(x) =|S(x)|   \frac{|\bigcup_{s_i \in S(x)}\{w | w \in s_i\}|}{\sum_{s_i \in S(x)} \text{len}(s_i)} \rho(x),$$}}where $\rho(x) = \frac{|\mathcal{X}|}{\max_{c_i\in x}|\{x' ~|~ c_i \in x' ~~\text{and}~~ x' \in \mathcal{X}\}|}.$
	The first term in $\texttt{score}$ is the number of unique sentences covering all given concepts in $x$, and
	the second term is to represent the diversity of the scenes described in these sentences.
	Th last term $\rho(x)$ is the penalty of inverse frequency.
	Specifically, we find the concept in $x$ that has the maximum ``set frequency'' (i.e., the number of unique concept-sets containing a particular concept), then we take the inverse with the number of all concept-sets for normalization. 
	This penalty based on inverse set-frequency effectively controls the bias towards highly frequent concepts.
	With the distribution of such scores of concept-sets, we sample our candidate examples for the next steps.


	\begin{table}[t]
	\small
		\centering
		\scalebox{0.94
		}{
			\begin{tabular}{@{}l|ccc@{}}
				\toprule
			\textbf{Statistics} & \textbf{Train} & \textbf{Dev} & \textbf{Test} \\ \midrule
        \textbf{\# Concept-Sets} & \textbf{32,651} & \textbf{993} & \textbf{1,497} \\
        \qquad {-Size} = 3 & 25,020 & 493 & - \\
        \qquad {-Size} = 4 & 4,240 & 250 & 747 \\
        \qquad {-Size} = 5 & 3,391 & 250 & 750 \\
        
        \midrule
        {\textbf{\# Sentences }} & {67,389} & {4,018} & {7,644} \\ 
        {\textbf{per Concept-Set }} & 2.06 & 4.04 & \textbf{5.11} \\ 
        {\textbf{Average Length}} & 10.54 & 11.55 & 13.28 \\ 
        \midrule 
        \midrule 
        \textbf{\# Unique Concepts} & 4,697 & 766 & 1,248 \\
        \textbf{\# Unique Concept-Pairs } & 59,125 & 3,926 & 8,777 \\
        \textbf{\# Unique Concept-Triples } & 50,713 & 3,766 & 9,920 \\
        \midrule
\textbf{\% Unseen Concepts} & - & 6.53\% & 8.97\% \\
\textbf{\% Unseen Concept-Pairs} & - & 96.31\% & 100.00\% \\
\textbf{\% Unseen Concept-Triples} & - & 99.60\% & 100.00\% \\
        
				\bottomrule                        
			\end{tabular}} 
		\caption{The \textbf{basic statistics} of the \textsc{CommonGen} data. We highlight the ratios of concept compositions that are unseen in training data, which assures the challenge in compositional generalization ability.}
		\label{tab:basicstat}
	\end{table}
	
	
	\subsection{Crowd-Sourcing References via AMT}
	\label{ssec:amt}
In order to ensure the best quality,
     the references of the evaluation examples are crowdsourced from crowd workers on \textit{Amazon Mechanical Turk},
     which amounts to \textbf{10,060} references over 2.5k distinct concept-sets.
     Note that these newly collected references for dev and test examples can ensure that we can do a fair comparisons targeting generalization, considering potential data-leak (i.e., recent pre-trained language models might have seen the caption datasets). 
    Each concept-set was assigned to at least 3  workers. 
  	In addition to references about given concept-sets, we also ask the workers to provide rationale sentences to explain what commonsense facts they have used, for ensuring that the described scenarios are common in daily life (example rationales are shown in Fig~\ref{fig:morecase}).


    We control the quality by actively filtering workers who produced low-quality references, then removing their annotations, and finally re-opening the slots only for quality workers. {There were 1,492 accepted workers in total and 171 disqualified workers in the end after the active filtering.}
    There are three criteria for efficiently narrowing down candidates for us to further manually remove out low-quality workers: 1) coverage via part-of-speech tagging, 2) especially high perplexity via GPT-2, and 3) length of the rationales.
    Meanwhile, we also dynamically replaced the concept-sets that majority of the references do not make sense to ensure the final quality.
    
    \subsection{Permutation-Invariant Annotating}
    We have to present every input as a a string for annotators to read, which means we take a random permutation of the concept-set as a linear sequence. Therefore we may wonder if annotators will make flexible adjustment on the concept order when creating references for CommonGen.
	To address this concern, 
	we first study the correlation between the input concept-order and the reference concept-order (i.e., the order of the given concepts in the human annotations).
	We find that 96.97\% of the references, of which the concept-order is different from the order shown when they are annotating.
	
    More specifically, 
    we use Spearmans's rank correlation coefficient to understand the correlation between input concept-order and reference concept-order.
    It turns out that the mean correlation over all input-reference pairs on test examples is -0.031, which suggests that different permutation of the input concept-order do not have notable influence on the order of concept in the human references, thus being permutation-invariant.
	
	\subsection{Finalizing Adequate References}
	As there may be more than one acceptable scenes for each input concept-set, we would like to check if our human references are enough before we finalizing our dataset.
	Thus, we took one more round of crowd-sourcing to add one more reference for each concept-set by \textit{new} annotators.
	Then, we compute the inter-annotator agreement (IAA) by using the cosine similarity between all pairs of human references, based on SentenceBERT~\cite{reimers-gurevych-2019-sentence} (fine-tuned for semantic similarity analysis).
	Note that if we have $k$ human references for an example in the end, then we will have $k(k-1)/2$ different pairs of references, each of which has a cosine similarity between their sentence embeddings.
	Then, we take the \textit{median} of these similarity scores as a proxy to understand if we have collect adequate human references.
	
	The underlying rationale here is that if there are more references that are very similar to each other yet from different annotators, then it is likely that current references are adequate for this example.
	As shown in Figure~\ref{fig:iaacurve}, we simulated different sizes of number of references per example. We find that the IAA will be saturated when we have the fifth ones, and thus we believe references are adequate. 
	Also, from the \textit{std} of these IAA scores, we find that the diversity of the references, and it also saturate when there are five references.




	


    


      
    
	\subsection{Down-Sampling Training Examples }
	
    In order to evaluate the \textit{compositional generalization} ability, 
    we down-sample the remaining candidate concept-sets to construct a distantly supervised training dataset (i.e., using caption sentences as the human references). 
    We explicitly control the overlap of the concept-sets between training examples and dev and test examples.
	The {basic statistics }of the final dataset is shown in Table~\ref{tab:basicstat}.
	There are on average four sentences for each example in dev and test sets, which provide a richer and more diverse test-bed for automatic and manual evaluation.
	Table~\ref{tab:basicstat} also shows the ratio of \textit{unseen concept compositions} (i.e., concept, concept-pair, and concept-triple) in the dev and test. Notably, all pairs of concepts in every test concept-set are unseen in training data and thus pose a challenge for compositional generalization.

    
        
	


\begin{figure}[t!]
\centering
\hspace{-0.4em}
\begin{subfigure}[b]{0.47\textwidth}
   \includegraphics[width=1\linewidth]{trend_std.pdf}
   \label{fig:trend_std} 
\end{subfigure}

\begin{subfigure}[b]{0.482\textwidth}
    \vspace{-2.4em}
    \hspace{-0.4em}
   \includegraphics[width=1\linewidth]{trend_median.pdf}
   \label{fig:trend_median}
\end{subfigure}
\vspace{-2em}
\caption[Two numerical solutions]{The curve of inter-annotator agreement (IAA) in terms of their std (up) and median (bottom) when average number of references increase.}
\label{fig:iaacurve}
\end{figure}


	
    \begin{figure}[t!]
		\centering
		\includegraphics[width=0.98\linewidth]{hopdist.pdf}
		\caption{\small \textbf{Connectivity analysis} in 5-size concept-sets in the test set, each of which consists of 10 concept pairs. {For example, 12.0 in blue means: there are 12\% concept-sets that have 3 concept pairs with one-hop connections on ConceptNet.}}
		\label{fig:connectivity}
	\end{figure}

	
	
	\subsection{Analysis of Underlying Common Sense}
	\label{ssec:analysis}
We here introduce deeper analysis of the dataset by utilizing the largest commonsense knowledge graph (KG), ConceptNet~\cite{Speer2017ConceptNet5A}, as an tool to study connectivity and relation types.
    
    \smallskip
    \noindent
    \textbf{Connectivity Distribution.~} 
    If the concepts inside a given concept-set is more densely connected with each other on the KG, then it is likely to be easier to write a scenario about them.
In each 5-size concept-set (i.e. a concept-set consists of five concepts), 
    there are 10 unique pairs of concepts, the connections of which we are interested in. 
    As shown in Figure~\ref{fig:connectivity}, if we look at the one-hop links on the KG, about 60\% of the 5-size concept-set have less than one link among all concept-pairs.
    On the other hand, if we consider two-hop links, then nearly 50\% of them are almost fully connected (i.e. each pair of concepts has connections).
    These two observations together suggest that the \textsc{CommonGen} has a reasonable difficulty: the concepts are not too distant or too close, and thus the inputs are neither too difficult nor too trivial. 
    
    \smallskip
    \noindent
    \textbf{Relation Distribution.~}
    Furthermore, the relation types of such connections can also tell us what kinds of commonsense knowledge are potentially useful for relational reasoning towards generation.
    We report the frequency of different relation types\footnote{ Relation definitions are at \url{https://github.com/commonsense/conceptnet5/wiki/Relations}.} of the one/two-hop connections among concept-pairs in the dev and test examples in Fig.~\ref{fig:relationdist}.
    To better summarize the distributions, 
    we categorize these relations into five major types and present their distribution in Table~\ref{tab:cate}, respectively for one/two-hop connections between concept pairs.
    
\begin{figure}[!t]
	\centering
	\includegraphics[width=1\linewidth]{relationtypes.pdf}
	\captionof{table}{The \textbf{distributions of the relation categories} on one/two-hop connections.}
	\label{tab:cate}
\end{figure}



	
	
				
%
 	














\section{Methods}
\label{sec:baseline}
We briefly introduce the  baseline methods that are tested on the \textsc{CommonGen} task.



\smallskip
\noindent
\textbf{Encoder-Decoder Models.}
Bidirectional RNNs and Transformers~\cite{Vaswani2017AttentionIA} are two most popular architectures for seq2seq learning.
We use them with the addition of attention mechanism~\cite{Luong2015EffectiveAT} with copying ability~\cite{gu-etal-2016-incorporating}, which are based on an open-source framework OpenNMT-py~\cite{Klein2017OpenNMTOT}.
We use \texttt{bRNN-CopyNet} and \texttt{Trans-CopyNet} denote them respectively.
To alleviate the influence from the concept ordering in such sequential learning methods, 
we randomly permute them multiple times  for training and decoding and then get their average performance.
To explicitly eliminate the order-sensitivity of inputs,
we replace the encoder with a mean pooling-based MLP network (\texttt{MeanPooling-CopyNet}).


\smallskip
\noindent
\textbf{Non-autoregressive generation.} 
Recent advances~\cite{Lee2018DeterministicNN, Stern2019InsertionTF} in conditional sentence generation have an emerging interest on (edit-based) non-autoregressive generation models, which iteratively refine generated sequences.
We assume that these models potentially would have better performance because of their explicit modeling on iterative refinements, and thus study the most recent such model {Levenshtein Transformer} (\texttt{LevenTrans}) by~\citeauthor{Gu2019LevenshteinT} (\citeyear{Gu2019LevenshteinT}). 
We also include a recent enhanced version, \texttt{ConstLeven}~\cite{Susanto2020LexicallyCN}, which incorporates lexical constraints in \texttt{LevenTrans}.




\begin{table*}[th!]
	\centering
	\scalebox{0.88
	}{
		\begin{tabular}{@{}c|cc|cc|c|cc|c@{}}
		
\textbf{Model $\backslash$ Metrics}             & \multicolumn{2}{c}{\texttt{ROUGE-2/L}} & \multicolumn{2}{c}{\texttt{\textbf{BLEU}-3/\textbf{4}}}  & \texttt{{METEOR}} & \texttt{\textbf{CIDEr}} & \texttt{\textbf{SPICE}} &  \texttt{Coverage}  \\ \midrule 
bRNN-CopyNet~\cite{gu-etal-2016-incorporating} & 7.67 & 27.77 & 12.58 & 7.06 & 16.38 & 5.06 & 13.39 &  51.15 \\ 
Trans-CopyNet & 8.64 & 27.85 & 12.47 & 7.56 & 15.91 & 4.65 & 12.85 & 49.06 \\
MeanPooling-CopyNet & 9.65 & 31.15 & 12.29 & 7.08 & 17.10 & 5.18 & 15.18 & 55.70 \\ 
LevenTrans.~\cite{Gu2019LevenshteinT} & 10.61 & 31.87 & 21.51 & 12.65 & 20.50 & 7.45 & 16.84 & 63.81 \\ 
ConstLeven.~\cite{Susanto2020LexicallyCN} & 11.77 & 33.04 & 20.87 & 11.26 & 25.23 & 10.80 & 20.05 & 94.51 \\ 
\midrule   
GPT-2~\cite{radford2019language} & 16.85 & 39.01 & 33.92 & 23.73 & 26.83 & 12.19 & 23.57 & 79.09  \\  
BERT-Gen~\cite{bao2020unilmv2}  & 17.78 & 40.21 & 33.29 & 23.47 & 28.25 & 12.61 & 24.82 & 86.06  \\
UniLM~\cite{Dong2019UnifiedLM} & 21.20 & \textbf{43.60} & \underline{41.82} & \underline{30.73} & 30.62 & \underline{14.89} & 27.43 & 89.19 \\ 
UniLM-v2~\cite{bao2020unilmv2} & 18.11 & 40.51 & 34.31 & 24.53 & 29.04 & 13.19  & 25.52 & 89.13  \\  
BART~\cite{Lewis2019BARTDS} & \textbf{22.02} & 41.78 & 39.52 & 29.01 & \textbf{31.83} & 13.98 & \underline{28.00}  & \textbf{97.35}  \\ 
 T5-Base~\cite{raffel2019exploring} & 14.63 & 34.56 &  28.76 &  18.54 & 23.94 & 9.40 & 19.87 &  76.67  \\
 T5-Large~\cite{raffel2019exploring} & \underline{21.74} & \underline{42.75} &  \textbf{43.01} &  \textbf{31.96} & \underline{31.12} & \textbf{15.13} & \textbf{28.86} &  \underline{95.29}  \\  

\midrule
Human Performance (Upper Bound) & 36.72 & 53.45 & 52.55 & 46.49 & 38.79 & 37.64 & 52.43 & 99.33  \\ 
\bottomrule

\end{tabular}
		
	} 
	\caption{\textbf{Experimental results} of different baseline methods on the \textsc{CommonGen} test set (v1.1). The first group of models are non-pretrained models, while the second group is large pretrained models that we have fine-tuned. The best models are \textbf{bold} and second best ones are \underline{underlined} within each metric. We highlight the metrics that we used in our official leaderboard. (Results on dev set are at  Table.~\ref{tab:devexp}.)}
	\label{tab:exp}
\end{table*}














\smallskip
\noindent
\textbf{Pre-trained Language Generation Models.}
We also employ various pre-trained language generation models, including \texttt{GPT-2}~\cite{radford2019language}, \texttt{UniLM}~\cite{Dong2019UnifiedLM},
\texttt{UniLM-v2}~\cite{bao2020unilmv2},
\texttt{BERT-Gen}~\cite{bao2020unilmv2}, 
\texttt{BART}~\cite{Lewis2019BARTDS}, and \texttt{T5}~\cite{raffel2019exploring}, to tackle this task and test their generative commonsense reasoning ability.
We fine-tuned all the above models on our training data with a seq2seq format.

Specifically, to use \texttt{GPT-2} for this sequence-to-sequence task, we condition the language model
on the format ``\textit{$c_1~c_2~\dots~c_k = y$}'' during fine-tuning, where $c_i$ is a concept in the given concept-set and connects with other concepts with a blank; $y$ is a target sentence.
For inference, we sample from the fine-tuned
\texttt{GPT-2} model after a prompt of ``\textit{$c_1~c_2~\dots~c_k =$}'' with beam search and use the first generated
sentence as the output sentence. 
For \texttt{BERT-Gen}, we use the \texttt{s2s-ft} package\footnote{\small{\url{https://github.com/microsoft/unilm}}} to fine-tune them in a sequence-to-sequence fashion that is similar to the LM objective employed by UniLM.

As for \texttt{T5}, the state-of-the-art text-to-text pre-trained model which is pre-trained with a multi-task objective by prepending a task description before the input text, we prepend the input concept set with a simple prompt: ``\texttt{generate a sentence with:}'' and fine-tune the model with the source sentence on the format ``\textit{generate a sentence with $c_1~c_2~\dots~c_k$}.'' For decoding, we employ the standard beam search with a beam size of 5 for all compared models.
We also report their results with a lexically-constrained decoding method, dynamic beam allocation (DBA)~\cite{post-vilar-2018-fast}, which do not show improvement over conventional beam searching.
\footnote{The used hyper-parameters are reported in the appendix. }


















 	

	\section{Evaluation}
	\label{sec:evaluation}

	We first introduce the automatic evaluation metrics, then present main experimental results with manual analysis, and finally introduce the potential application in transferring CommonGen-trained models for other downstream tasks.




	 	
	
	\subsection{Metrics}
    Following other conventional generation tasks, we use several widely-used automatic metrics to automatically assess the performance, such as \texttt{BLEU}~\cite{Papineni2001BleuAM}, \texttt{ROUGE}~\cite{Lin2004ROUGEAP},  \texttt{METEOR}~\cite{Banerjee2005METEORAA}, which mainly focus on measuring surface similarities.  We  report the concept \texttt{Coverage}, which is the average percentage of input concepts that are present in lemmatizatized outputs.
    
    In addition, we argue that it is more suitable to use evaluation metrics specially design for captioning task, such as \texttt{CIDEr}~\cite{Vedantam2014CIDErCI} and \texttt{SPICE}~\cite{Anderson2016SPICESP}.
    They usually assume system generations and human references use similar concepts, and thus focus on evaluate the associations between mentioned concepts instead of n-gram overlap.
    For example, the \texttt{SPICE} metric uses dependency parse trees as proxy of scene graphs to measure the similarity of scenarios.\footnote{We also tried recent metrics such as BERTScore~\cite{Zhang2020BERTScore}, but we find that they 
    overly focus on lexical semantics instead of dependencies between words, thus resulting low correlation with the manual evaluation results.}
    
    


    




	
	To estimate \textit{human performance} within each metric, we treat each reference sentence in dev/test data as a ``system prediction''  to be compared with all other references, which is equivalent to compute inter-annotator agreement within each metric. 
	Thus, systems that have better generative ability than average crowd-workers should exceed this.
	
	
	
	\subsection{Experimental Results}
		\smallskip
	\noindent
	\textbf{Automatic Evaluation.} 
	Table~\ref{tab:exp} presents the experimental results
in a variety of metrics.
We can see that all fine-tuned pre-trained models (the lower group) outperform  non-pretrained models (the upper group) with a significant margin.
This is not surprising because their pretraining objectives, including masked language modeling, word ordering, and text infilling which predicts missing words or text spans, are relevant to our task. 
	On the other hand, we find that the key disadvantage of non-pretrained models with CopyNet still falls in the failure of using all given concepts (i.e., low coverage), which results in worse results.
	
	
	Among them, UniLM, BART, and T5 performs the best, which may be due to its inherent sequence-to-sequence pre-training framework. 	
	We found that \texttt{BART} has the best concept coverage, which is probably due to its comprehensive pre-training tasks that aim to recover text with noise.
	The results suggest that further modifying  pre-trained models is a promising direction for generative commonsense. 
	








	

















	


	
 
	\begin{table}[t!]
	\small
		\centering
		\scalebox{0.86
		}{	 \begin{tabular}{@{}c|c|c|c|c|c|c}
			 	\toprule
      & C.Leven & GPT & BERT-G. & UniLM & BART  & T5    \\ \midrule
Hit@1 & 3.2         & 21.5  & 22.3     & 21.0  & \underline{26.3} & \textbf{26.8} \\
Hit@3 & 18.2        & 63.0  & 59.5     & \underline{69.0}  &\underline{69.0} & \textbf{70.3 }\\
Hit@5 & 51.4        & 95.5  & 95.3     & \underline{96.8} & 96.3 & \textbf{97.8} \\
\bottomrule
\end{tabular}
			} 
		\caption{\small{\textbf{Manual Evaluation via Pair-wise Comparisons for Ranking.} Numbers are hit rates (\%) at top 1/3/5.}
}
		\label{tab:maneval}
	\end{table}
	\smallskip
\noindent
	\textbf{Manual Evaluation.} 
We conduct manual evaluation with a focus on \textit{commonsense plausibility} for comparing the 6 best-performing models in Table~\ref{tab:maneval}.
We ask five graduate students to compare 1,500 pairs of model-generated sentences respectively, for ranking the models within 100 concept-sets that are covered by all the models. 
The final average ranked results are shown in Table~\ref{tab:maneval} and their inter-annotator agreement is 0.85 in \textit{Kendall's rank correlation  coefficient}.

Note that the coverage-weighted hit@1 rate correlates with the \texttt{SPICE} metric the most, i.e.,\textbf{ 0.94} in \textit{Spearman's} $\rho$ for model ranks, CIDEr for 0.91, while METEOR and ROUGE-2 are both 0.88 and BLEU-4 is 0.78.
	
	
	\begin{figure}[t!]
		\centering
		\includegraphics[width=1\linewidth]{casestudy.pdf}
		\caption{A case study with a concept-set \{\textit{hand}, \textit{sink}, \textit{wash}, \textit{soap}\} for qualitative analysis of machine generations. 
		Human references are collected from AMT.}
		\label{fig:casestudy}
	\end{figure}
	
	\smallskip
	\noindent
	\textbf{Case study.}	
    Fig.~\ref{fig:casestudy} shows the top generations of different models and human references about an input concept-set:  \{\textit{hand}, \textit{sink}, \textit{soup}, \textit{wash}\} (more cases are shown in Fig.~\ref{fig:morecase} in the appendix). 
We find that non-pretrained seq2seq models (e.g., bRNN, MeanPooling, ConstLeven) can successfully use part of given concepts, while the generated sentences are less meaningful and coherent.
On the contrary, the outputs of fine-tuned pre-trained language models are significantly more commonsensical.
Most of them use all given concepts in their outputs.
    ConstLeven tends to make use of frequent patterns to compose a non-sense sentence but uses all concepts.
    GPT-2 and UniLM incorrectly compose the dependency among \textit{hand}, \textit{wash}, and \textit{soap}.
    The phrase `a sink of soaps' in BERT-gen's output makes itself less common.
    BART and T5 generate relatively reasonable scenarios, but both are not as natural as human references; BART's contains repetitive content while T5's lacks a human agent.
    
    \begin{figure}[t!]
		\centering
		\includegraphics[width=0.98\linewidth]{transfer.pdf}
		\caption{{\textbf{Learning curve for the transferring study.} We use several trained \textsc{CommonGen} (GG) models to generate choice-specific context for the CSQA task.  Detailed numbers are shown in Tab.~\ref{tab:detailtransfer} in the appendix.
}}
		\label{fig:learncurve}
	\end{figure}
	
	
 
\begin{table*}[th!]
	\centering
	\scalebox{0.9
	}{
		\begin{tabular}{@{}c|cc|cc|c|cc|c@{}}
		
\textbf{Model $\backslash$ Metrics}             & \multicolumn{2}{c}{\texttt{ROUGE-2/L}} & \multicolumn{2}{c}{\texttt{BLEU-3/4}}  & \texttt{METEOR} & \texttt{CIDEr} & \texttt{SPICE} &  \texttt{Coverage}  \\ \midrule T5-large+DBA & 16.8    & 36.71   & 27.3   & 18.7   & 25.3   & 8.62  & 24.3  & 83.98    \\
T5-base+DBA  & 15.07   & 34.82   & 24.8   & 16     & 23.5   & 9.31  & 21.3  & 76.81    \\
GPT-2+DBA    & 17.56   & 39.45   & 29.4   & 20.6   & 24.9   & 10.85 & 26.8  & 79.51    \\
BART+DBA     & 18.15   & 37.02   & 28.3   & 19.1   & 25.5   & 9.82  & 25.1  & 84.78   \\  
 
\bottomrule

\end{tabular}
		
	} 
	\caption{Experimental results of models with DBA decoding method on the test set.}
	\label{tab:dbaexp}
\end{table*} 


\noindent
 \textbf{Influence of Dynamic Beam Allocation.~}
Considering that all tested models decode sentences with beam searching, one may wonder what if we use a decoding method specially designed for constrained decoding.
 Thus, we employed dynamic beam allocation (DBA)~\cite{post-vilar-2018-fast}.
 The results are shown in Table~\ref{tab:dbaexp}.
 Note that the models are the same as in Table~\ref{tab:exp} while only the decoding method is changed to DBA.
 We can see that all methods are negatively impacted by the decoding method.
 This suggests that for the \textsc{CommonGen} task and pre-trained language models, we may need to focus on knowledge-based decoding or re-ranking as future directions.
	
    \subsection{Transferring CommonGen Models}
    \label{ssec:transfer}
    One may wonder how fine-tuned \textsc{CommonGen} models can benefit commonsense-centric downstream tasks such as Commonsense Question Answering~\cite{Talmor2018CommonsenseQAAQ} (CSQA) with their generative commonsense reasoning ability.
    To this end, we use the models trained with the \textsc{CommonGen} dataset for generating useful context. 


    We extract the nouns and verbs in questions and all choices respectively, and combine the concepts of the question $q$ and each choice $c_i$ to build five concept-sets. 
    Then, we use these concept-sets as inputs to a trained \textsc{CommonGen} model (e.g., T5) for generating scenario a sentence $g_i$ for each as choice-specific contexts.
    Finally, we prepend the outputs in front of the questions, i.e., ``\textless s\textgreater G: $g_i$ $|$ Q: $q$ \textless /s\textgreater~C: $c_i$ \textless /s\textgreater ''. Note that the  state-of-the-art RoBERTa-based models for CSQA uses the same form without ``G: $g_i|$'' in fine-tuning.
   
    
    We show the learning-efficiency curve in Fig.~\ref{fig:learncurve}, where $y$ is the accuracy on the official dev set and $x$ is the number of training steps.
    The details of the experiments are shown in the appendix.


    We highlight the performance of original RoBERTa-Large as the baseline.
    We find that some CommonGen models further improves the performance by a large margin, e.g., $76.9 \xrightarrow{\texttt{UniLM}} 78.4$ and they converge at better accuracy in the end. Note that BERT-gen and ConstLeven cause negative transfer due to the low quality of generated context. 
    Particularly, we find that the context generated by the T5-based CommonGen model (CG-T5) helps speed up training about 2 times, if we look at 550th steps of CG-T5 (74.85\%) and 1,250th steps of original RoBERTa (74.77\%).
	
     
    Through manual analysis, we find that the successful \textsc{CommonGen} models can generate more reasonable and natural sentence for correct choices while noisy sentences for wrong choices.
    For example with CG (T5), $q$=``\textit{What do \underline{people} \underline{aim} to do at \underline{work}?}'',  $c_i$=`\underline{complete} \underline{job}' (\cmark) with $g_i$=``\textit{people work to complete a job aimed at achieving a certain goal.}''; 
    $c_j$=`\underline{wear} \underline{hats}' (\xmark) $g_j$=``\textit{people wearing hats aim their guns at each other while working on a construction site.}''
	The used question concepts and choice concepts are {underlined}.
			 
	






	


	
	






 	\section{Related Work}
	\label{sec:relatedwork}
	\noindent
	\textbf{Commonsense benchmark datasets.~}
There are many emerging datasets for testing machine commonsense from different angles,
	such as commonsense extraction~\cite{Xu2018AutomaticEO,Li2016CommonsenseKB}, next situation prediction ({SWAG}~\cite{Zellers2018SWAGAL}, CODAH~\cite{Chen2019CODAHAA}, HellaSWAG~\cite{Zellers2019HellaSwagCA}), cultural and social understanding~\cite{Lin2018MiningCD, sap2018atomic, sap-etal-2019-social}, visual scene comprehension~\cite{Zellers2019FromRT}, and general commonsense question answering~\cite{Talmor2018CommonsenseQAAQ, huang-etal-2019-cosmos, wang-etal-2019-make,wang-etal-2020-semeval}. 
	   However, the success of fine-tuning pre-trained language models for these tasks does not necessarily mean machines can produce novel assumptions in a more open, realistic, generative setting.
	We see \textsc{CommonGen} as a novel, complementary  commonsense reasoning benchmark task for advancing machine commonsense in NLG.


    
   
	
	
	\smallskip
	\noindent
	\textbf{Constrained Text Generation.~}
Constrained text generation aims to decode sentences with expected attributes such as sentiment~\cite{Luo2019TowardsFT, Hu2017TowardCG}, tense~\cite{Hu2017TowardCG}, template~\cite{Zhu2019TextI, j-kurisinkel-chen-2019-set}, style~\cite{fu2018style, Luo2019ADR, Li2018DeleteRG}, topics~\cite{Feng2018TopictoEssayGW}, etc.
	Two related scenarios with our task is lexically constrained decoding and word ordering~\cite{Zhang2015DiscriminativeSW, Hasler2018NeuralMT, Dinu2019TrainingNM, Hokamp2017LexicallyCD, puduppully-etal-2017-transition,Miao2018CGMHCS}. 
	However, they are not easily adopted by the recent pre-trained language models and thus not directly useful for our task.
Topical story generation~\cite{Fan2018HierarchicalNS, yao2019plan} is also a related direction, while it targets generating longer, creative stories around the given topics, making it hard to directly adopt them to our task.
	Additionally, the \textsc{CommonGen} task brings some more challenges mentioned in Section~\ref{sec:formualtion}.
	Prior constrained generation methods cannot address these issues together in a unified model.


	\smallskip
	\noindent
	\textbf{Incorporating Commonsense for NLG.~}
There are a few recent works that incorporate commonsense knowledge in language generation tasks such as essay generation~\cite{Guan2018StoryEG, Yang2019EnhancingTG}, image captioning~\cite{DBLP:conf/cvpr/LuYBP18}, video storytelling~\cite{Yang2019KnowledgeableSA}, and conversational systems~\cite{Zhang2019ConversationGW}.
	These works suggest that generative commonsense reasoning has a great potential to benefit downstream applications.
    Our proposed \textsc{CommonGen}, to the best of our knowledge, is the very first constrained sentence generation dataset for assessing and conferring generative machine commonsense and we hope it can benefit such applications.
    Our transferring study in Sec.~\ref{ssec:transfer} also shows the potential benefits of CommonGen-generated contexts.

 	
	\section{Conclusion}
	\label{sec:conclusion} 
	Our major contribution in this paper are threefold:
	\begin{itemize}
	    \item  we present \textsc{CommonGen}, a novel constrained text generation task for generative commonsense reasoning, with a large dataset;
	    \item we carefully analyze the inherent challenges of the proposed task, i.e.,  a) relational reasoning with latent commonsense knowledge, and b) compositional generalization.
	    \item our extensive experiments systematically examine recent pre-trained language generation models (e.g., UniLM, BART, T5) on the task , and find that their performance is still far from humans, generating grammatically sound yet realistically implausible sentences.
	\end{itemize}
Our study points to interesting future research directions on modeling commonsense knowledge in language generation process, towards conferring machines with generative commonsense reasoning ability. 
	We hope \textsc{CommonGen} would also benefit downstream NLG applications such as conversational systems and storytelling models.


	
	

 	
		\section*{Acknowledgements}
This research is based upon work supported in part by the Office of the Director of National Intelligence (ODNI), Intelligence Advanced Research Projects Activity (IARPA), via Contract No. 2019-19051600007, the DARPA MCS program under Contract No. N660011924033 with the United States Office Of Naval Research, the Defense Advanced Research Projects Agency with award W911NF-19-20271, and NSF SMA 18-29268. 
The views and conclusions contained herein are those of the authors and should not be interpreted as necessarily representing the official policies, either expressed or implied, of ODNI, IARPA, or the U.S. Government. We would like to thank all the collaborators in USC INK research lab for their constructive feedback on the work.
	
	
	
	\bibliography{acl2020}
	\bibliographystyle{acl_natbib}
	
	\clearpage
	\appendix
\section{Supplementary Figures and Tables}
We include additional figures and tables that we mentioned in the main content here.
\begin{itemize}
    \item Figure~\ref{fig:relationdist} shows the detailed \textbf{distribution of the commonsense relations} between given concepts, the summary of which was shown in Table~\ref{tab:cate} of the main content.
    \item  Figure~\ref{fig:morecase} presents 4 more \textbf{case studies} with human rationales which we asked our crowd workers to provide.  
    \item  Figure~\ref{fig:amt} shows instructions and AMT interface for crowd-sourcing human references.
    \item Table~\ref{tab:devexp} shows the model performances on the dev set of \textsc{CommonGen}, as a reference for future development.
    \item Table~\ref{tab:detailtransfer} is the full results of the learning curve in Figure~\ref{fig:casestudy}. We highlight the highest checkpoints and the speed-up by the CG-T5, which are discussed in Section~\ref{ssec:transfer}.
\end{itemize}


\section{Experimental Details}
\textbf{Main experiments.}\quad 
We present some implementation details in training and testing the baseline models in Table~\ref{tab:model_details}.
The detailed instructions for installing dependencies and all necessary training command-lines are shown in the instruction `\textbf{readme.md}' files.
The number of trainable model parameters are directly induced from either output of the frameworks or the original papers.
We show some key hyper-parameters that we manually tuned on top of the development set. 





All key hyper-parameters were initialized by the default values as suggested by the original authors of the frameworks.
The bound of our manual tuning is done by iterating the magnitudes or the neighboring choices, for example, the learning rates (`lr') of the last seven models  are selected from $\{1e-3,\dots,1e-4, \dots, 1e-5\}$.
Then, similarly, the batch size (bsz) is first maximized by making full use of the GPU memory.
Note that the first three models are implemented with the OpenNMT-py framework~\footnote{\url{https://github.com/OpenNMT/OpenNMT-py}}.
The \texttt{LevenTrans}\footnote{\url{https://github.com/pytorch/fairseq/blob/master/examples/nonautoregressive_translation/README.md}}, \texttt{ConstLeven}\footnote{\url{https://github.com/raymondhs/constrained-levt}}, and \texttt{BART}\footnote{\url{https://github.com/pytorch/fairseq/tree/master/examples/bart}} are adopted by the official authors' release.
The \texttt{BERT-gen}, \texttt{UniLM}, \texttt{UniLMv2} are all based on their official source code\footnote{\url{https://github.com/microsoft/unilm}}.
The \texttt{GPT-2} and \texttt{T5} are both adopted by the \texttt{huggingface} transformers\footnote{\url{https://github.com/huggingface/transformers}} framework~\cite{Wolf2019HuggingFacesTS}.
All models use beam searching as their decoding algorithms and beam-size are mostly 5, which is selected from $\{5,10,20\}$.
All our models were trained on Quadro RTX 6000 GPUs.
The training time of X-CopyNet and LevenTrans models are less than 12 hours with a single GPU.
The second group of models are trained between 12 and 24 hours, expect for T5-large, which we used 3 GPUs and fine-tuned about 48 hours.
\textit{Note that all the above methods are \underline{self-contained} in our submitted code as long as users follow the associated readme instructions.}
\begin{figure}[t!]
	\centering
	\includegraphics[width=1\linewidth]{model_details.pdf}
	\captionof{table}{The paths to the instruction files in our submitted code zip file (under the `\textit{methods/}' folder), and their numbers of parameters and key hyper-parameters.}
\label{tab:model_details}
\end{figure}

\textbf{Transferring study experiments.}\quad 
 We use the same hyper-parameters which are searched over the baseline RoBERTa-Large model for these experiments. 
 The best hyper-parameter\footnote{We follow the hps selected by 100 trials of tuning in \url{https://github.com/pytorch/fairseq/tree/master/examples/roberta/commonsense_qa}.} of RoBERTa-Large for \texttt{CommonsenseQA}\footnote{\url{https://www.tau-nlp.org/commonsenseqa}}: 
 \begin{itemize}
     \item batch size = 16, learning rate = 1e-5,
     \item maximum updates = 3,000 ($\sim$5 epochs) \item warmup steps=150, dropout rate=0.1
     \item weight decay = 0.01, adam\_epsilon = 1e-6
 \end{itemize}
 
 











	
\begin{table*}[th!]
	\centering
	\scalebox{0.85
	}{
		\begin{tabular}{@{}c|cc|cc|c|cc|c@{}}
		
\textbf{Model $\backslash$ Metrics}             & \multicolumn{2}{c}{\texttt{ROUGE-2/L}} & \multicolumn{2}{c}{\texttt{BLEU-3/4}}  & \texttt{METEOR} & \texttt{CIDEr} & \texttt{SPICE} &  \texttt{Coverage}  \\ \midrule 
bRNN-CopyNet~\cite{gu-etal-2016-incorporating} & 9.23 & 30.57 & 13.60 & 7.80 & 17.40 & 6.04 & 16.90 &  58.95 \\ 
Trans-CopyNet & 11.08 & 32.57 & 17.20 & 10.60 & 18.80 & 7.02 & 18.00 & 62.16 \\
MeanPooling-CopyNet & 11.36 & 34.63 & 14.80 & 8.90 & 19.20 & 7.17 & 20.20 & 68.32 \\ 
LevenTrans.~\cite{Gu2019LevenshteinT} & 12.22 & 35.42 & 23.10 & 15.00 & 22.10 & 8.94 & 21.40 & 71.83 \\ 
ConstLeven.~\cite{Susanto2020LexicallyCN} & 13.47 & 35.19 & 21.30 & 12.30 & 25.00 & 11.06 & 23.20 & 96.87 \\ 
\midrule   
GPT-2~\cite{radford2019language} & 17.74 & 41.24 & 32.70 & 23.30 & 27.50 & 13.26 & 27.60 & 85.46  \\  
BERT-Gen~\cite{bao2020unilmv2}  & 18.73 & 42.36 & 33.00 & 23.70 & 29.10 & 13.34 & 28.70 & 91.71  \\
UniLM~\cite{Dong2019UnifiedLM} & 21.68 & \textbf{45.66} & \underline{40.40} & \underline{30.40} & \textbf{31.00} & \underline{15.72} & \underline{31.40} & 92.41 \\ 
UniLM-v2~\cite{bao2020unilmv2} & 19.24 & 43.01 & 33.40 & 24.20 & 29.20 & 13.65  & 29.30 & 93.57  \\  
BART~\cite{Lewis2019BARTDS} & \textbf{22.13} & 43.02 & 37.00 & 27.50 & \textbf{31.00} & 14.12 & 30.00  & \textbf{97.56}  \\  
T5-Base~\cite{raffel2019exploring} & 15.33 & 36.20 &  28.10 &  18.00 & 24.60 & 9.73 & 23.40 &  83.77  \\
 T5-Large~\cite{raffel2019exploring} & \underline{21.98} & \underline{44.41} &  \textbf{40.80} &  \textbf{30.60} & \textbf{31.00} & \textbf{15.84} & \textbf{31.80} &  \underline{97.04}  \\  

\bottomrule

\end{tabular}
		
	} 
	\caption{Experimental results of different baseline methods on the \textsc{CommonGen} dev set. The first group of models are non-pretrained models, while the second group is large pretrained models that we have fine-tuned. The best models are \textbf{bold} and second best ones are \underline{underlined} within each metric.}
	\label{tab:devexp}
\end{table*}

 We tried 10 random seeds and use the best one (42).
 Then, we follow the steps described in Sec.~\ref{ssec:transfer} to run other CG-enhanced models with the same hps.
 This suggests that further searching for them may have even better performance.
 
	








    
 	\begin{figure*}[t!]
		\centering
		\includegraphics[width=1\linewidth]{amt.png}
		\caption{Our annotation interface on the AMT platform. The upper part is the instruction for the annotators and we provide an example for them. Note that we give the part-of-speech hints (from the captain corpora) to boost the speed of annotation, but we do not remove sentences with wrong part-of-speech as long as they also make sense.}
		\label{fig:amt}
	\end{figure*}
	
	
	\begin{figure*}[t!]
		\centering
		\includegraphics[width=1\linewidth]{reldist.pdf}
		\caption{One/two-hop relation frequency in the \textsc{CommonGen} dev.\&test sets on ConceptNet.}
		\label{fig:relationdist}
	\end{figure*}

		\begin{figure*}[t!]
		\centering
		\includegraphics[width=1.0\linewidth]{morecase.pdf}
		\caption{Four cases for qualitative analysis of machine generations. 
		References are collected from AMT crowd-workers and they are required to provide rationales. Note that the third one is a positive case showing that some models can successfully generate reasonable scenarios. However, most models perform poorly on the other cases.}
		\label{fig:morecase}
	\end{figure*}
	
    
	
	

\begin{table*}[th!]
	\centering
	\scalebox{0.8
	}{
\begin{tabular}{c|cccccc@{}}
\toprule
Training Steps & RoBERTa-Large & w/CG(BART) & w/CG(T5) & w/CG(UniLM) & w/CG(BERT-Gen) & w/CG(ConstLeven) \\
\midrule
50             & 0.2252  & 0.1884     & 0.2506   & 0.2244      & 0.2007         & 0.2162            \\
100            & 0.3088  & 0.2703     & 0.3587   & 0.3153      & 0.2924         & 0.2809            \\
150            & 0.5053  & 0.2973     & 0.5643   & 0.1851      & 0.3391         & 0.3653            \\
200            & 0.5717  & 0.4439     & 0.6650   & 0.3833      & 0.5274         & 0.5324            \\
250            & 0.6020  & 0.5242     & 0.6937   & 0.5348      & 0.5839         & 0.6396            \\
300            & 0.6388  & 0.6601     & 0.7117   & 0.6323      & 0.6274         & 0.6634            \\
350            & 0.6675  & 0.6814     & 0.7150   & 0.6503      & 0.6626         & 0.6740            \\
400            & 0.6830  & 0.6830     & 0.7215   & 0.6847      & 0.6781         & 0.6773            \\
450            & 0.7027  & 0.7068     & 0.7338   & 0.6921      & 0.7068         & 0.6962            \\
500            & 0.7019  & 0.7076     & 0.7428   & 0.7011      & 0.6929         & 0.7052            \\
550            & 0.6978  & 0.7248     & \underline{0.7486}   & 0.7256      & 0.7068         & 0.6904            \\
600            & 0.6790  & 0.7232     & 0.7494   & 0.7338      & 0.7248         & 0.7068            \\
650            & 0.7150  & 0.7289     & 0.7428   & 0.7469      & 0.7101         & 0.7117            \\
700            & 0.7142  & 0.7453     & 0.7477   & 0.7387      & 0.7305         & 0.7183            \\
750            & 0.7027  & 0.7453     & 0.7314   & 0.7527      & 0.7166         & 0.7183            \\
800            & 0.7158  & 0.7355     & 0.7437   & 0.7371      & 0.7281         & 0.7240            \\
850            & 0.7174  & 0.7445     & 0.7625   & 0.7420      & 0.7379         & 0.7322            \\
900            & 0.7191  & 0.7543     & 0.7559   & 0.7502      & 0.7477         & 0.7338            \\
950            & 0.7355  & 0.7486     & 0.7477   & 0.7387      & 0.7428         & 0.7404            \\
1000           & 0.7477  & 0.7510     & 0.7461   & 0.7486      & 0.7428         & 0.7363            \\
1050           & 0.7346  & 0.7502     & 0.7568   & 0.7469      & 0.7412         & 0.7297            \\
1100           & \underline{0.7428}  & 0.7527     & 0.7551   & 0.7494      & 0.7363         & 0.7420            \\
1150           & 0.7379  & 0.7609     & 0.7576   & 0.7641      & 0.7453         & 0.7437            \\
1200           & 0.7469  & 0.7477     & 0.7502   & 0.7461      & 0.7420         & 0.7477            \\
1250           & 0.7477  & 0.7412     & 0.7592   & 0.7518      & 0.7273         & 0.7371            \\
1300           & 0.7502  & 0.7518     & 0.7617   & 0.7666      & 0.7518         & 0.7412            \\
1350           & 0.7469  & 0.7502     & 0.7551   & 0.7568      & 0.7437         & 0.7404            \\
1400           & 0.7420  & 0.7494     & 0.7641   & 0.7559      & 0.7494         & 0.7428            \\
1450           & 0.7510  & 0.7584     & 0.7625   & 0.7461      & 0.7461         & 0.7461            \\
1500           & 0.7535  & 0.7674     & 0.7690   & 0.7551      & 0.7412         & 0.7428            \\
1550           & 0.7461  & 0.7559     & 0.7674   & 0.7510      & 0.7445         & 0.7412            \\
1600           & 0.7437  & 0.7584     & 0.7584   & 0.7543      & 0.7445         & 0.7420            \\
1650           & 0.7568  & 0.7609     & 0.7633   & 0.7543      & 0.7494         & 0.7428            \\
1700           & 0.7551  & 0.7584     & 0.7633   & 0.7625      & 0.7535         & 0.7396            \\
1750           & 0.7600  & 0.7568     & 0.7699   & 0.7740      & 0.7551         & \textbf{0.7518}            \\
1800           & 0.7617  & 0.7559     & 0.7731   & 0.7740      & 0.7527         & 0.7486            \\
1850           & \textbf{0.7690}  & 0.7584     & 0.7772   & 0.7707      & \textbf{0.7617}         & 0.7461            \\
1900           & 0.7658  & 0.7592     & \textbf{0.7805}   & \textbf{0.7838}      & 0.7486         & 0.7445            \\
1950           & 0.7584  & 0.7617     & 0.7715   & 0.7715      & 0.7510         & 0.7396            \\
2000           & 0.7510  & 0.7617     & 0.7690   & 0.7715      & 0.7445         & 0.7355            \\
2050           & 0.7551  & 0.7641     & 0.7731   & 0.7649      & 0.7559         & 0.7477            \\
2100           & 0.7641  & 0.7617     & 0.7641   & 0.7625      & 0.7559         & 0.7412            \\
2150           & 0.7584  & 0.7543     & 0.7658   & 0.7641      & 0.7527         & 0.7461            \\
2200           & 0.7584  & 0.7477     & 0.7649   & 0.7633      & 0.7453         & 0.7371            \\
2250           & 0.7551  & 0.7559     & 0.7641   & 0.7609      & 0.7461         & 0.7363            \\
2300           & 0.7535  & 0.7600     & 0.7699   & 0.7674      & 0.7412         & 0.7420            \\
2350           & 0.7551  & 0.7617     & 0.7682   & 0.7625      & 0.7502         & 0.7412            \\
2400           & 0.7559  & 0.7649     & 0.7699   & 0.7625      & 0.7559         & 0.7387            \\
2450           & 0.7584  & 0.7674     & 0.7707   & 0.7658      & 0.7477         & 0.7387            \\
2500           & 0.7551  & 0.7649     & 0.7600   & 0.7633      & 0.7502         & 0.7363            \\
2550           & 0.7592  & 0.7658     & 0.7731   & 0.7658      & 0.7518         & 0.7387            \\
2600           & 0.7559  & 0.7658     & 0.7715   & 0.7600      & 0.7420         & 0.7371            \\
2650           & 0.7576  & 0.7674     & 0.7690   & 0.7600      & 0.7494         & 0.7420            \\
2700           & 0.7568  & \textbf{0.7707}    & 0.7690   & 0.7600      & 0.7461         & 0.7379            \\
2750           & 0.7568  & 0.7699     & 0.7674   & 0.7649      & 0.7445         & 0.7437            \\
2800           & 0.7592  & 0.7682     & 0.7690   & 0.7617      & 0.7445         & 0.7453            \\
2850           & 0.7592  & 0.7641     & 0.7707   & 0.7649      & 0.7461         & 0.7445            \\
2900           & 0.7609  & 0.7649     & 0.7740   & 0.7658      & 0.7477         & 0.7437            \\
2950           & 0.7617  & 0.7649     & 0.7740   & 0.7658      & 0.7469         & 0.7437            \\
3000           & 0.7600  & 0.7658     & 0.7731   & 0.7658      & 0.7437         & 0.7420     \\   
\bottomrule
\end{tabular}
		
	} 
	\caption{Experimental results of the transferring study on CommonsenseQA dev set.}
	\label{tab:detailtransfer}
\end{table*}

 
	
\end{document}
