\documentclass[a4paper,twoside,11pt]{article}
\usepackage[margin=2.9cm]{geometry}
\usepackage[affil-it]{authblk}
\usepackage{amsmath,amsfonts,amsthm,amssymb}
\usepackage[font={small}]{caption}
\usepackage[font={small}]{subfig}
\usepackage{graphicx}
\usepackage{paralist}
\usepackage{times}
\usepackage{xspace}

\graphicspath{{./img/}}

\title{Planar Octilinear Drawings with One Bend Per Edge}

\author[1]{Michael A. Bekos\thanks{Electronic address: \texttt{bekos@informatik.uni-tuebingen.de}}}

\author[2]{Martin~Gronemann\thanks{Electronic address: \texttt{gronemann@informatik.uni-koeln.de}}}

\author[1]{Michael~Kaufmann\thanks{Electronic address: \texttt{mk@informatik.uni-tuebingen.de}}}

\author[1]{Robert~Krug\thanks{Electronic address: \texttt{krug@informatik.uni-tuebingen.de}}}

\affil[1]{Wilhelm-Schickhard-Institut f\"ur Informatik, Universit\"at T\"ubingen, Germany}
\affil[2]{Institut f\"ur Informatik, Universit\"at zu K\"oln, Germany}

\date{}

\newtheorem{lemma}{Lemma}
\newtheorem{theorem}{Theorem}
\newtheorem{definition}{Definition}
\newcommand{\skel}[1]{G^\textit{skel}_{#1}}
\newcommand{\Vpert}[1]{V^\textit{pert}_{#1}}
\newcommand{\Epert}[1]{E^\textit{pert}_{#1}}
\newcommand{\Vskel}[1]{V^\textit{skel}_{#1}}
\newcommand{\Eskel}[1]{E^\textit{skel}_{#1}}
\newcommand{\pert}[1]{G^\textit{pert}_{#1}}
\newcommand{\gdeg}[1]{\textit{deg}(#1)}
\newcommand{\pdeg}[2]{\textit{deg}^{\textit{pert}}_{#1}(#2)}
\newcommand{\sdeg}[2]{\textit{deg}^{\textit{skel}}_{#1}(#2)}
\newcommand{\redge}[1]{\textit{ref}(#1)}
\newcommand{\poles}[1]{\mathcal{P}_{#1}}
\newcommand{\IPGeo}{IP-\ref{ip:1}\xspace}
\newcommand{\IPFix}{IP-\ref{ip:2}\xspace}
\newcommand{\IPPort}{IP-\ref{ip:3}\xspace}


\begin{document}
\maketitle

\begin{abstract}
In \emph{octilinear drawings} of planar graphs, every edge is drawn
as an alternating sequence of horizontal, vertical and diagonal
() line-segments. In this paper, we study octilinear
drawings of low edge complexity, i.e., with few bends per edge. A
-planar graph is a planar graph in which each vertex has degree
less or equal to . In particular, we prove that every 4-planar
graph admits a planar octilinear drawing with at most one bend per
edge on an integer grid of size . For 5-planar
graphs, we prove that one bend per edge still suffices in order to
construct planar octilinear drawings, but in super-polynomial area.
However, for 6-planar graphs we give a class of graphs whose planar
octilinear drawings require at least two bends per edge.
\end{abstract}


\section{Motivation}
\label{sec:introduction}



Drawing edges as octilinear paths plays a central role in the design
of metro-maps (see e.g., \cite{HMN06,NW11,SROW11}), which dates back
to the 1930's when Henry Beck, an engineering draftsman, designed
the first schematic map of London Underground using mostly
horizontal, vertical and diagonal segments; see
Fig.\ref{fig:beck1933}. Laying out networks in such a way is called
\emph{octilinear graph drawing}. More precisely, an \emph{octilinear
drawing} of a (planar) graph  of maximum degree eight is a
(planar) drawing  of  in which each vertex occupies a
point on the integer grid and each edge is drawn as a sequence of
alternating horizontal, vertical and diagonal ()
line-segments. For an example, see Fig.\ref{fig:4p_example_large} in
Section~\ref{sec:sample}.

\begin{figure}[t]
    \centering
    \includegraphics[width=.7\textwidth]{beck1933}
    \caption{Henry~Beck Tube Map (first edition), 1933. Printed at Waterlow \& Sons Ltd., London.}
    \label{fig:beck1933}
\end{figure}

For drawings of (planar) graphs to be readable, special care is
needed to keep the number of bends small. However, the problem of
determining whether a given embedded 8-planar graph (that is, a
planar graph of maximum degree eight with given combinatorial
embedding) admits a bendless octilinear drawing is NP-hard
\cite{Noellenburg05}. This negative result motivated us to study
octilinear drawings of low \emph{edge complexity}, that is, with few
bends per edge. Surprisingly enough, very few results relevant to
this problem were known, even if the octilinear model has been
well-studied in the context of metro-map visualization and map
schematization (see e.g.~\cite{Wolff13}). As an immediate byproduct
of a result of Keszegh et al.~\cite{KPP13}, it turns out that every
-planar graph, with , admits a planar octilinear
drawing with at most two bends per edge; see
Section~\ref{sec:relatedwork}. On the other hand, every 3-planar
graph on five or more vertices admits a planar octilinear drawing in
which all edges are bendless~\cite{GLM14,Kant92}.

In this paper, we bridge the gap between the two aforementioned
results. In particular, we prove that every 4-planar graph admits a
planar octilinear drawing with at most one bend per edge in cubic
area (see Section~\ref{sec:4planar}). We further show that every
5-planar graph also admits a planar octilinear drawing with at most
one bend per edge, but our construction may require super-polynomial
area (see Section~\ref{sec:5planar}). Hence, we have made no effort
in proving a concrete bound. We complement our results by
demonstrating an infinite class of 6-planar graphs whose planar
octilinear drawings require at least two bends per edge (see
Section~\ref{sec:6planar}).

\section{Related Work}
\label{sec:relatedwork}


The research on the (planar) \emph{slope number of graphs} focuses
on minimizing the number of used slopes (see e.g.,
\cite{JJ13,KPP13,KPPT08,LLMN13,MP11}). Octilinear drawings can be
seen as a special case thereof, since only four slopes (horizontal,
vertical and the two diagonals) are used. In a related work, Keszegh
et al.~\cite{KPP13} showed that any -planar graph admits a planar
drawing with one bend per edge, in which all edge-segments have at
most  different slopes. So, for  and for , we
significantly reduce the number of different slopes from  and
, resp., to . They also proved that -planar graphs, with
, admit planar drawings with two bends per edge that
require at most  different slopes. It is
not difficult to transfer this technique to the octilinear model and
show that any -planar graph, with , admits a
planar octilinear drawing with two bends per edge. However, for
, Di~Giacomo et al.~\cite{GLM14} recently proved that any
3-planar graph with  vertices has a bendless planar
drawing with at most  different slopes and angular resolution
 (see also~\cite{Kant92}); their approach also yields
octilinear drawings.

Octilinear drawings can be considered as an extension of
\emph{orthogonal drawings}, which allow only horizontal and vertical
segments (i.e., graphs of maximum degree  admit such drawings).
Tamassia~\cite{Tamassia87} showed that one can minimize the total
number of bends in orthogonal drawings of embedded 4-planar graphs.
However, minimizing the number of bends over all embeddings of a
4-planar graph is NP-hard~\cite{GT01}. Note that the core of
Tamassia's approach is a min-cost flow algorithm that first
specifies the angles and the bends of the drawing, producing an
\emph{orthogonal representation}, and then based on this
representation computes the actual drawing by specifying the exact
coordinates for the vertices and the bends of the edges. It is known
that Tamassia's algorithm can be employed to produce a bend-minimum
octilinear representation for any given embedded 8-planar graph.
However, a bend-minimum octilinear representation may not be
realizable by a corresponding planar octilinear drawing \cite{BT04}.
Furthermore, the number of bends on a single edge might be very
high, but can easily be bounded by applying appropriate capacity
constraints to the flow-network.

Biedl and Kant~\cite{BK94} showed that any 4-planar graph except the
octahedron admits a planar orthogonal drawing with at most two bends
per edge on an  integer grid. Hence, the octilinear drawing
model allows us to reduce the number of bends per edge at the cost
of an increased area. On the other hand, not all 4-planar graphs
admit orthogonal drawings with one bend per edge; however, testing
whether a 4-planar graph admits such a drawing can be done in
polynomial time~\cite{BKRW14}. In the context of metro-map
visualization, several approaches have been proposed to produce
metro-maps using octilinear or nearly-octilinear polylines, such as
force-driven algorithms~\cite{HMN06}, hill climbing multi-criteria
optimization techniques~\cite{SROW11} and mixed-integer linear
programs~\cite{NW11}. However, the problem of laying out a metro-map
in an octilinear fashion is significantly more difficult than the
octilinear graph drawing problem, as several metro-lines may connect
the same pair of stations and the positions of the vertices have to
reflect geographical coordinates of the stations.


\section{Preliminaries}
\label{sec:preliminaries}


In our algorithms, we incrementally construct the drawings similar
to the method of Kant~\cite{Kant92b}. We first employ the canonical
order to cope with triconnected graphs. Then, we extend them to
biconnected graphs using the SPQR-tree~\cite{BT90} and to simply connected
graphs using the BC-tree. In this section we briefly recall them;
however we assume basic familiarity.


\begin{definition}[Canonical order~\cite{Kant92b}]
For a given triconnected plane graph  let  be a partition of  into paths such that ,  and  is a path on the outer face of . For  let
 be the subgraph induced by  and assume it 
inherits its embedding from . Partition  is a canonical
order of  if for each  the following hold:
\begin{inparaenum}[(i)]
\item  is biconnected,
\item all neighbors of  in  are on the outer face,
of 
\item all vertices of  have at least one neighbor in 
for some .
\end{inparaenum}
 is called a singleton if  and a chain otherwise.
\end{definition}

\begin{definition}[BC-tree]
The \emph{BC-tree}  of a connected graph  has a
B-node for each biconnected component of  and a C-node for each
cutvertex of . Each B-node is connected with the C-nodes that are
part of its biconnected component.
\end{definition}

An SPQR-tree~\cite{BT90} provides information about the
decomposition of a biconnected graph into its triconnected
components. It can be computed in linear time and space~\cite{GM00}.
Every triconnected component is associated with a node
 in the SPQR-tree . The triconnected component
itself is referred to as the \emph{skeleton} of , denoted by
. We refer to the degree of
a vertex  in  as . We
say that  is an \emph{R-node}, if  is a simple
triconnected graph. A bundle of at least three parallel edges
classifies  as a \emph{P-node}, while a simple cycle of length
at least three classifies  as an \emph{S-node}. By construction
R-nodes are the only nodes of the same type that are allowed to be
adjacent in . The leaves of  are formed by
the \emph{Q-nodes}. Their skeleton consists of two parallel edges;
one of them corresponds to an edge of  and is referred to as
\emph{real edge}. The skeleton edges that are not real are referred
to as \emph{virtual edges}. A virtual edge  in 
corresponds to a tree node  that is adjacent to  in
, more exactly, to another virtual edge  in
. We assume that  is rooted at a Q-node.
Hence, every skeleton (except the one of the root) contains exactly
one virtual edge  that has a counterpart in the skeleton
of the parent node. We call this edge the \emph{reference edge} of
 denoted by . Its endpoints,  and , are
named the \emph{poles} of  denoted by . Every subtree rooted at a node  of 
induces a subgraph of  called the \emph{pertinent graph} of 
that we denote by . We
abbreviate the degree of a node  in  with
. The pertinent graph is the subgraph of  for
which the subtree describes the decomposition.

Now, assume that  be a simple, biconnected -planar graph,
whose SPQR-tree  is given. Additionally, we may assume
that  is rooted at a Q-node that is adjacent to an S-
or R-node. Notice that at least one such node exists since the graph
does not contain any multi-edges, which would be the case if only a
P-node existed. Biconnectivity and maximum degree of  yield basic
bounds for the graph degree of a node , i.e., . By construction the pertinent graph of a tree node
 is a (connected) subgraph of ; thus . For the degrees in a skeleton graph , we
obtain bounds based on the type of the corresponding node. Skeletons
of Q-nodes are cycles of length two, whereas S-nodes are by
definition simple cycles of length at least three; hence,
. For P- and R-nodes the degree can be bounded by
, since the skeleton of the former is
at least a bundle of three parallel virtual edges and the latter's
skeleton is triconnected by definition. The upper bound is derived
from the relation between skeleton and graph degrees: A virtual edge
 hides at least one incident edge of each node (not
necessarily an -edge). This observation can be easily proven
by induction on the tree. Hence, .

Next, we use this observation to derive bounds for the pertinent
degree by distinguishing two cases depending on whether  is a
pole or not. Recall that  is a subgraph of  that is
obtained by recursively replacing virtual edges by the skeletons of
the corresponding children. In the first case when  is an
internal node in , i.e., ,  is
not incident to the reference edge in . Thus, every edge
of  hidden by a virtual edge in  is in .
Hence, . In the other case,
i.e., , at least one edge that is hidden by
the reference edge, is not part of , thus,
. Notice that the
lower bounds depend on the skeleton degree which in turn depends on
the type of node, unlike the upper bounds that hold for all tree
nodes. The next lemma tightens these bounds based on the type of the
parent node.

\begin{lemma}
Let  be a tree node that is not the root in the SPQR-tree
 of a simple, biconnected, -planar graph  and
 its parent in . For , it
holds that , if  is a P- or an R-node
and  otherwise, i.e.  is an S- or a
Q-node. \label{lem:pdeg_bounds}
\end{lemma}
\begin{proof}
Since the case where  is an S- or a Q-node follows from the
fact that  is k-planar and the reference edge hides at least
one edge that is not in , we restrict ourselves to the
more interesting cases where  is either a P- or an R-node.
From our previous observations we know that . Each of the at least three edges in  hides at
least one edge of  that is incident to . However, the total
number of edges is at most  due to the degree restriction. Hence,
we are left with the problem of  edges of  being hidden by at
least three virtual edges, each hiding at least one. As a result the
virtual edge that corresponds to  cannot contribute more than
 edges to its pertinent graph .
\end{proof}

\begin{lemma}
In the SPQR-tree  of a planar biconnected graph  with  for every , there exists at
least one Q-node that is adjacent to a P- or an R-node.
\label{lem:pr_node}
\end{lemma}
\begin{proof}
Assume to the contrary that all Q-nodes are adjacent to S-nodes
only. We pick such a Q-node and root  at it. Let 
be an S-node (possibly the root itself) with poles  such that there is no other S-node in the subtree of
. By definition of an S-node,  has at least two children.
If all of them were Q-nodes then there exists a 
with  and ; a contradiction. Hence,
there is at least one child  that is a P- or an R-node.
However, since the leaves of  are Q-nodes and those are
not allowed to be children of P- and R-nodes by our assumption,
there exists at least one other S-node in the subtree of  and
therefore in the subtree of  which contradicts our choice of
.
\end{proof}


\section{Octilinear Drawings of 4-Planar Graphs}
\label{sec:4planar}


In this section, we focus on planar octilinear drawings of 4-planar
graphs. We first consider the case of triconnected 4-planar graphs
and then we extend our approach first to biconnected and then to
simply connected graphs. Central in our approach is the port
assignment; by the \emph{port} of a vertex we refer to the side of
the vertex an edge is connected to. The different ports on a vertex
are distinguished by the cardinal directions.

\subsection{The Triconnected Case}
\label{sec:4tricon}


Let  be a triconnected 4-planar graph and  be a canonical order of . We momentarily neglect
the edge  of the first partition  of  and we
start by placing the second partition, say a chain , on a horizontal line from left to right.
Since by definition of ,  and  are adjacent
to the two vertices,  and , of the first partition ,
we place  to the left of  and  to the right of
. So, they form a single chain where all edges are
drawn using horizontal line-segments that are attached to the east
and west port at their endpoints. The case where  is a
singleton is analogous. Having laid out the base of our drawing, we
now place in an incremental manner the remaining partitions. Assume
that we have already constructed a drawing for  and we now
have to place , for some .

In case where  is a chain of  vertices, we draw them from left to right along a horizontal line
one unit above . Since  and  are the only
vertices that are adjacent to vertices in , both only to
one, we place the chain between those two as in
Fig.\ref{fig:4p_chain}. The port used at the endpoints of  in
 depends on the following rule: Let  (,
resp.) be the neighbor of  (, resp.) in . If the
edge  (, resp.) is the last to be attached
to vertex  (, resp.), i.e., there is no vertex  in
,  such that  (,
resp.), then we use the northern port of  (, resp.).
Otherwise, we choose the north-east port for  or the
north-west port for .

In case of a singleton , we can apply the previous
rule if the singleton is of degree three, as the third neighbor of
 should belong to a partition  for some .
However, in case where  is of degree four we may have to deal with
an additional third edge  that connects  with
. By the placement so far, we may assume that  lies
between the other two endpoints, thus, we place  such that
. This enables us to draw  as a vertical
line-segment; see Fig.\ref{fig:4p_singleton}.

The above procedure is able to handle all chains and singletons
except the last partition , because  may have  edges
pointing downwards. One of these edges is , by
definition of . We exclude  and draw  as an
ordinary singleton. Then, we shift  to the left and up as in
Fig.\ref{fig:4p_final}. This enables us to draw  as a
horizontal-vertical segment combination. For , we move
 one unit to the right and down. We free the west port of 
by redrawing its incident edges as in Fig.\ref{fig:4p_final} and
attach  to it. Edge  will be drawn as a
diagonal segment with positive slope connected to  and a
horizontal segment connected to , which requires one bend. Let
 be the other incomplete edge according to Figure
\ref{fig:4p_final}. It will be drawn using a diagonal segment with
positive slope connected to  and a horizontal segment connected
to , again requiring one bend.

So far, we have specified a valid port assignment and the
y-coordinates of the vertices. However, we have not fully specified
their x-coordinates. Notice that by construction every edge, except
the ones drawn as vertical line-segments, contains exactly one
horizontal segment. This enables us to stretch the drawing
horizontally by employing appropriate cuts. A \emph{cut}, for us, is
a -monotone continuous curve that crosses only horizontal
segments and divides the current drawing into a left and a right
part. It is not difficult to see that we can shift the right part of
the drawing that is defined by the cut further to the right while
keeping the left part of the drawing on place and the result remains
a valid octilinear drawing.

To compute the x-coordinates, we proceed as follows. We first assign
consecutive x-coordinates to the first two partitions and from there
on we may have to stretch the drawing in two cases. The first one
appears when we introduce a chain, say , as it may not fit into
the gap defined by its two adjacent vertices in . In this
case, we horizontally stretch the drawing between its two adjacent
vertices in  to ensure that their horizontal distance is at
least . The other case appears when an edge that
contains a diagonal segment is to be drawn. Such an edge requires a
horizontal distance between its endpoints that is at least the
height it bridges. We also have to prevent it from intersecting any
horizontal-vertical combinations in the face below it. We can cope
with both cases by horizontally stretching the drawing by a factor
that is bounded by the current height of the drawing. Since the
height of the resulting drawing is bounded by , it
follows that in the worst case its width is . We are now
ready to state the main theorem of this subsection.

\begin{figure}[t]
    \centering
    \begin{minipage}[b]{.32\textwidth}
        \centering
        \subfloat[\label{fig:4p_chain}{}]
        {\includegraphics[width=.9\textwidth]{4p_chain}}
    \end{minipage}
    \begin{minipage}[b]{.32\textwidth}
        \centering
        \subfloat[\label{fig:4p_singleton}{}]
        {\includegraphics[width=.9\textwidth]{4p_singleton}}
    \end{minipage}
    \begin{minipage}[b]{.32\textwidth}
        \centering
        \subfloat[\label{fig:4p_final}{}]
        {\includegraphics[width=\textwidth,page=3]{4p_final}}
    \end{minipage}
    \caption{
    (a)~Horizontal placement of a chain .
    (b)~Placement of a singleton  with degree four.
    (c)~Final layout after repositioning  and  (the shape of the dotted edges can be obtained by extending the stubs until they intersect).}
    \label{fig:4p_canonical}
\end{figure}

\begin{theorem}
Given a triconnected 4-planar graph , we can compute in 
time an octilinear drawing of  with at most one bend per edge on
an  integer grid.
\end{theorem}
\begin{proof}
In order to keep the time complexity of our algorithm linear, we
employ a simple trick. We assume that any two adjacent points of the
underlying integer grid are by  units apart in the horizontal
direction and by one unit in the vertical direction. This a priori
ensures that all edges that contain a diagonal segment will not be
involved in crossings and simultaneously does not affect the total
area of the drawing, which asymptotically remains cubic. On the
other hand, the advantage of this approach is that we can use the
shifting method of Kant~\cite{Kant92b} to cope with the introduction
of chains in the drawing, that needs  time in total by keeping
relative coordinates that can be efficiently updated and computing
the absolute values only at the last step.
\end{proof}

Note that our algorithm produces drawings that have a linear number
of bends in total (in particular, exactly  bends). In
the following, we prove that this bound is asymptotically tight.

\begin{theorem}
There exists an infinite class of 4-planar graphs which do not admit
bendless octilinear drawings and if they are drawn with at most one
bend per edge, then a linear number of bends is required.
\end{theorem}
\begin{proof}
Based on the simple fact that in an orthogonal drawing a triangle
requires at least one bend, we describe an example that translates
this idea to the octilinear model (see Fig.\ref{fig:4p_lowerbound}).
While a triangle can easily be drawn bendless with the additional
ports available, we will occupy those to enforce the creation of a
bend as in the orthogonal model. Furthermore, the example is
triconnected. Hence, its embedding is fixed up to the choice of the
outer face. Our construction is heavily based on the so called
\emph{separating triangle}, i.e., a three-cycle whose removal
disconnects the graph. Each vertex of such a triangle has degree
four. Any triangle which is drawn bendless has a  angle
inside. But since the triangles are nested and have incident edges
going inside of the triangles, this is impossible.
\end{proof}

\begin{figure}[t]
    \centering
    \includegraphics[width=.4\textwidth]{4-planar_sep_tri_lb}
    \caption{Nested separating triangles each requiring one bend.}
    \label{fig:4p_lowerbound}
\end{figure}

\subsection{The Biconnected Case}
\label{sec:4bicon}


Following standard practice, we employ a rooted SPQR-tree and assume
for a tree node that the pertinent graphs of its children are drawn
in a pre-specified way. Consider a node  in  with
poles . In the drawing of
,  should be located at the upper-left and  at the
lower-right corner of the drawing's bounding box with a port
assignment as in Fig.\ref{fig:4planar_layout_spec}. In general, we
assume that the edges incident to  (, resp.) use the western
(eastern, resp.) port at their other endpoint, except of the
northern (southern, resp.) most edge which may use the north (south,
resp.) port instead. In that case we refer to  and  as
\emph{fixed}; see  in
Fig.\ref{fig:4planar_layout_spec}. More specifically, we maintain
the following invariants:

\begin{enumerate}[{I}P-1:]
\item \label{ip:1} The width (height) of the drawing of  is
quadratic (linear) in the size of .  is located at
the upper-left;  at the lower-right corner of the drawing's
bounding box.
\item \label{ip:2} If ,  is fixed;
 is fixed if  and 's parent is not the
root.
\item \label{ip:3} The edges that are incident at  and  in
 use the south, south-east and east ports at  and the
north, north-west and west port at , resp. If  or  is not
fixed, incident edges are attached at their other endpoints via the
west and east port, respectively. If  or  is fixed, the
northern-most edge at  and the southern-most edge at  may use
the north (south, resp.) port at its other endpoint.
\end{enumerate}

Notice that the port assignment, i.e. \IPPort, guarantees the
ability to stretch the drawing horizontally even in the case where
both poles are fixed. Furthermore, \IPFix is \emph{interchangeable}
in the following sense: If  and , then  is fixed but  is not. But, if we relabel  and 
such that  and , then  and
. By \IPFix, we can create a drawing where both
 and  are not fixed and located in the upper-left and
lower-right corner of the drawing's bounding box. Afterwards, we
mirror the resulting layout vertically and horizontally to obtain
one where  and  are in their respective corners and not fixed.
Notice that in general the property of being fixed is not symmetric,
e.g., when  and  holds, 
remains fixed while  becomes fixed as well. For a non-fixed
vertex, we introduce an operation that is referred to as forming or
creating a \emph{nose}; see Fig.\ref{fig:4planar_nose_example},
where  has been moved downwards at the cost of a bend. As a
result, the west port of  is no longer occupied.

\begin{figure}[t]
    \centering
    \begin{minipage}[b]{.24\textwidth}
        \centering
        \subfloat[\label{fig:4planar_layout_spec}{}]
        {\includegraphics[width=.95\textwidth]{4-planar_layout_spec}}
    \end{minipage}
    \hfill
    \begin{minipage}[b]{.24\textwidth}
        \centering
        \subfloat[\label{fig:4planar_nose_example}{}]
        {\includegraphics[width=.95\textwidth,page=1]{4p_nose_new}}
    \end{minipage}
    \hfill
    \begin{minipage}[b]{.24\textwidth}
        \centering
        \subfloat[\label{fig:4p_P_case_1}{}]
        {\includegraphics[width=.95\textwidth,page=1]{4p_P_case}}
    \end{minipage}
    \hfill
    \begin{minipage}[b]{.24\textwidth}
        \centering
        \subfloat[\label{fig:4p_P_case_2}{}]
        {\includegraphics[width=.95\textwidth,page=2]{4p_P_case}}
    \end{minipage}
    \caption{
    (a)~Schematic view of the layout requirements.
    (b)~Creating a nose at .
    (c)~First P-node subcase without an -edge but  might be fixed in a child .
    (d)~Second P-node subcase with an -edge where  might get fixed in a child .}
    \label{fig:4p_requirements}
\end{figure}

\begin{description}
\item[P-node case:] Let  be a P-node. By Lemma~\ref{lem:pdeg_bounds},
for a child  of , it holds that 
and . So,  can form a nose in ,
while  might be fixed in the case where .
Notice that there exists at most one such child due to the degree
restriction. We distinguish two cases based on the existence of an
-edge.

In the first case, assume that there is no -edge, i.e., there
is no child that is a Q-node. We draw the children of  from top
to bottom such that a possible child in which  is fixed, is drawn
topmost (see  in Fig.\ref{fig:4p_P_case_1}). In the second
case, we draw the -edge at the top and afterwards the
remaining children (see Fig.\ref{fig:4p_P_case_2}). Of course, this
works only if  is not fixed in any of the other children. Let
 be such a potential child where  is fixed, i.e.,
, and thus, the only child that remains to be
drawn. Here, we use the property of interchangeability to ``unfix"
 in . As a result  can form a nose, whereas  may now
be fixed in  when  holds, as in
Fig.\ref{fig:4p_P_case_2}. However, then 
follows. Notice that the presence of an -edge implies that
the parent of  is not the root of , since this
would induce a pair of parallel edges. Hence, by \IPFix we are
allowed to fix  in . Port assignment and area requirements
comply in both cases with our invariant properties.
\item[S-node case:] We place the drawings of the children,
say , of an S-node  in a ``diagonal
manner'' such that their corners touch as in
Fig.\ref{fig:4p_S_case}. In case of Q-nodes being involved, we draw
their edges as horizontal segments (see, e.g., edge  in
Fig.\ref{fig:4p_S_case} that corresponds to Q-node ). Observe
that  and  inherit their port assignment and pertinent degree
from  and , respectively, i.e.,  and . So, we
may assume that  is fixed in , if  is fixed in .
Similarly,  is fixed in , if  is fixed in . By
\IPFix,  is not allowed to be fixed in the case where the parent
of  is the root of . However, Lemma
\ref{lem:pr_node} states that we can choose the root such that 
is not fixed in that case, and thus, complies with \IPFix. Since we
only concatenated the drawings of the children, \IPGeo and \IPPort
are satisfied.
\item[R-node case:] For the case where  is an R-node with poles
, we follow the basic idea of the
triconnected algorithm of the previous section and describe the
modifications necessary to handle the drawing of the children of
. To do so, we assume the worst case where no child of  is
a Q-node. Let  denote the child that is represented by the
virtual edge . Notice that due to
Lemma~\ref{lem:pdeg_bounds},  and
 holds. Hence, with \IPFix we may assume
that at most one out of  and  is fixed in . We
choose the first partition in the canonical ordering to be  and distinguish again between whether the partition to be
placed next is a chain or a singleton.


\begin{figure}[t]
    \centering
    \begin{minipage}[b]{.36\textwidth}
        \centering
        \subfloat[\label{fig:4p_S_case}{}]
        {\includegraphics[width=\linewidth, page=1]{4p_S_case}}
    \end{minipage}\hspace{.13\textwidth}
    \begin{minipage}[b]{.36\textwidth}
        \centering
        \subfloat[\label{fig:4p_R_chain_case}{}]
        {\includegraphics[width=\linewidth, page=1]{4p_R_case_new}}
    \end{minipage}\\
    \begin{minipage}[b]{.36\textwidth}
        \centering
        \subfloat[\label{fig:4p_R_singleton_case}{}]
        {\includegraphics[width=\linewidth, page=2]{4p_R_case_new}}
    \end{minipage}\hspace{.13\textwidth}
    \begin{minipage}[b]{.36\textwidth}
        \centering
        \subfloat[\label{fig:4p_R_last_vertex_case}{}]
        {\includegraphics[width=\linewidth, page=4]{4p_R_case_new}}
    \end{minipage}
    \caption{
    (a)~S-node with children ;  is a Q-node representing the edge . Optional edges are drawn dotted.
    (b)~Example for a chain  with virtual edges representing  in the R-node case.
    (c)~Singleton  with possibly three incident virtual edges representing .
    (d)~Placing  and moving up  which might be fixed in .}
\end{figure}

In case of a chain, say  with two
neighbors  and  in , we have to replace two
types of edges with the drawings of the corresponding children: the
edges  representing the
children  and  ( resp.) representing  ( resp.). We place
the vertices of  on a horizontal line high enough above
 such that every drawing may fit in-between it and
. Then, we insert the drawings aligned below the horizontal
line and choose for ,  to be the fixed node in
, whereas in  ( resp.), we set 
( resp.) to be fixed. Hence, for ,  may
form a nose in  pointing upwards while  and
 form each one downwards as depicted in
Fig.\ref{fig:4p_R_chain_case}. For the extra height and width, we
stretch the drawing horizontally.

For the case where  and  is a singleton,
we only outline the difference which is a possible third edge  to  representing say . While the other two
involved children, say   and , are handled as in
the chain-case,  requires extra height now and we may
place  such that  fits below  as in
Fig.\ref{fig:4p_R_singleton_case}. Notice that  holds and therefore by \IPFix both  and  are not fixed
in . Hence, forming a nose at  and  as in
Fig.\ref{fig:4p_R_singleton_case} is feasible.

It remains to describe the special case where the last singleton
 is placed. Since , both have not been
fixed yet. We proceed as in the triconnected algorithm and move  above  as depicted in Fig.\ref{fig:4p_R_last_vertex_case},
high enough to accommodate the drawing of the child 
represented by the edge . Since we may require  to
form a nose in  as in Fig.\ref{fig:4p_R_last_vertex_case},
we choose  to be fixed in . However, we are allowed by
\IPFix to fix  since  remains unfixed. For the area
constraints of \IPGeo, we argue as follows: Although some diagonal
segments may force us to stretch the whole drawing by its height,
the height of the drawing has been kept linear in the size of
. Since we increase the width by the height a constant
number of times per step, the resulting width remains quadratic.
\item[Root case:] For the root of  we distinguish two cases:
In the first case, there exists a vertex  with . Then, we choose as root a Q-node  that represents one
of its three incident edges and orient the poles  such that
. Hence, for the child  of  follows
. In the other case, i.e., for every  we have , we choose a Q-node that is not adjacent to
an S-node, whose existence is guaranteed by Lemma~\ref{lem:pr_node}.
In both cases, we may form a nose with  pointing downwards and
draw the edge as in the triconnected algorithm.
\end{description}

\begin{theorem}
Given a biconnected 4-planar graph , we can compute in 
time an octilinear drawing of  with at most one bend per edge on
an  integer grid.
\end{theorem}
\begin{proof}
The SPQR-tree  can be computed in -time and its
size is linear to the size of  \cite{GM00}. The pertinent degrees
of the poles at every node can be pre-computed by a bottom-up
traversal of . Drawing a P-node requires constant time;
S- and R-nodes require time linear to the size of the skeleton.
However, the sum over all skeleton edges is linear, as every virtual
edge corresponds to a tree node.
\end{proof}

\subsection{The Simply Connected Case}
\label{sec:4con}
After having shown that we can cope with biconnected 4-planar
graphs, we turn our attention to the connected case. We start by
computing the BC-tree of  and root it at some arbitrary B-node.
Every B-node, except the root, contains a designated cut vertex that
links it to the parent. A \emph{bridge} for a biconnected component
consists only of a single edge. Similar to the biconnected case, we
define an invariant for the drawing of a subtree: The cut vertex
that links the subtree to the parent is located in the upper left
corner of the drawing's bounding box.

\begin{figure}[t]
    \centering
    \begin{minipage}[b]{.24\textwidth}
        \centering
        \subfloat[\label{fig:4p_bc_root}{}]
        {\includegraphics[page=3]{4p_bc}}
    \end{minipage}
    \begin{minipage}[b]{.24\textwidth}
        \centering
        \subfloat[\label{fig:4p_bc_Snode}{}]
        {\includegraphics[page=4]{4p_bc}}
    \end{minipage}
    \begin{minipage}[b]{.24\textwidth}
        \centering
        \subfloat[\label{fig:4p_bc_Rnode}{}]
        {\includegraphics[page=5]{4p_bc}}
    \end{minipage}
    \begin{minipage}[b]{.24\textwidth}
        \centering
        \subfloat[\label{fig:4p_bc_bridges}{}]
        {\includegraphics[page=1]{4p_bc}}
    \end{minipage}
    \caption{
    (a)~Rooting the SPQR-tree such that  is in the upper-left corner.
    (b)~All possible situations at an S-node . For attaching  to , the layout had to be modified.
    (c)~Attaching a subtree via a bridge to a cut vertex  in an R-node. The dashed edge  may only be present if .
    (d)~A cut vertex where all of its children are attached via bridges.}
\end{figure}

Any subgraph, say , induced by a non-bridge biconnected
component can be laid out using the biconnected algorithm. However,
to construct a drawing that satisfies our invariant we have to take
care of two problems. First, the cut vertex, say , that links
 to the parent, has to be drawn in the upper-left corner of the
subtrees drawing. Second, there may be other cut vertices of  in
 to which we have to attach their corresponding subtrees.

For the first problem we describe how to root the SPQR-tree
 for  so that  is located in the upper-left
corner. There are at least two Q-nodes having  as a pole (as
 is biconnected) and the degree of  in  is at most
. In the biconnected case, we distinguished for the root of the
tree between whether there exists  with  or
not. Hence, we may choose for the root of  a Q-node
having  as a pole and orient it such that , thus,
satisfying . Then, we flip the final drawing of
 such that  is in the upper left corner (see
Fig.\ref{fig:4p_bc_root}).

Next, we address the second problem. Let  be a cut vertex in
 that is not the link to the parent. If  has degree ,
then it may occur in the pertinent graph of every node. However, in
this case we only have to attach a subtree of the BC-tree that is
connected via a bridge. This poses no problem, as there are enough
free ports available at  and we can afford a bend at the
bridge. We only consider S- and R- nodes here since the poles of
P-nodes occur in the pertinent graphs of the first two. For R-nodes
we assume that the south east port at  is free. So, we attach
the drawing via the bridge by creating a bend as in
Fig.\ref{fig:4p_bc_Rnode}. In the diagonal drawing of an S-node, the
north-east port is free. So, we can proceed similar; see
Fig.\ref{fig:4p_bc_Snode}.

If  has degree  in , it only occurs in the pertinent
graph of an S-node; see  in Fig.\ref{fig:4p_bc_Snode}. However,
we may no longer assume that the bridge is available. As a result,
we cannot afford a bend and have to deal with two incident edges
instead of one. We modify the drawing by exploiting the two real
edges incident to  in the S-nodes layout to free the east and
south east port; see  in Fig.\ref{fig:4p_bc_Snode}. This
enables us to attach the subtrees drawing without modifying it. We
finish this section by dealing with the most simple case where there
are only bridges attached to a cut vertex. The idea is illustrated
in Fig.\ref{fig:4p_bc_bridges} and matches our layout specification.

\begin{theorem}
Given a connected 4-planar graph , we can compute in  time
an octilinear drawing of  with at most one bend per edge on an
 integer grid. \label{thm:4planarconnected}
\end{theorem}
\begin{proof}
Decomposing a connected graph into its biconnected components takes
linear time. It remains the area property. Inserting a subtree with
 vertices and the given dimensions into the drawing of an R- or
S-node clearly increases the width of the drawing by at most
 and the height by at most . Hence, the total drawing
area is cubic, as desired.
\end{proof}


\section{Octilinear Drawings of 5-Planar Graphs}
\label{sec:5planar}


In this section, we focus on planar octilinear drawings of 5-planar
graphs. As in Section~\ref{sec:4planar}, we first consider the case
of triconnected 5-planar graphs and then we extend our approach
first to biconnected and then to the simply connected graphs.

\subsection{The Triconnected Case}
\label{sec:5tricon}


Let  be a triconnected 5-planar graph and  be a canonical order of . We place the first two
partitions  and  of , similar to the case of 4-planar
graphs. Again, we assume that we have already constructed a drawing
for  and now we have to place , for some
. We further assume that the - and
-coordinates are computed simultaneously so that the drawing of
 is planar and horizontally stretchable in the following
sense: If  is an edge incident to the outer face
of , then there is always a cut which crosses  and can
be utilized to horizontally stretch the drawing of . This
is guaranteed by our construction which makes sure that in each step
the edges incident to the outer face have a horizontal segment. In
other words, one can define a cut through every edge incident to the
outer face of  (\emph{stretchability-invariant}).

\begin{figure}[t]
    \centering
    \begin{minipage}[b]{.32\textwidth}
        \centering
        \subfloat[\label{fig:5p_chain}{}]
        {\includegraphics[width=.9\textwidth]{5p_chain}}
    \end{minipage}
    \begin{minipage}[b]{.32\textwidth}
        \centering
        \subfloat[\label{fig:5p_singleton}{}]
        {\includegraphics[width=.9\textwidth]{5p_singleton}}
    \end{minipage}
    \begin{minipage}[b]{.32\textwidth}
        \centering
        \subfloat[\label{fig:5p_before_final}{}]
        {\includegraphics[width=.9\textwidth,page=1]{5p_final}}
    \end{minipage}
    \caption{
    (a)~Horizontal placement of a chain .
    (b)~Placement of a singleton  of degree five.
    (c)~Final layout (the shape of the dotted edges can be obtained by extending the stubs until they intersect).}
    \label{fig:4p_canonical}
\end{figure}

If  is a chain, it is placed
exactly as in the case of 4-planar graphs, but with different port
assignment. Recall that by  (, resp.) we denote the
neighbor of  (, resp.) in . Among the northern
available ports of vertex  (, resp.), edge  (, resp.) uses the eastern-most unoccupied port
of  (western-most unoccupied port of , resp.); see
Fig.\ref{fig:5p_chain}. If  does not fit into the gap between
its two adjacent vertices  and  in , then
we horizontally stretch  between  and  to
ensure that the horizontal distance between  and  is
at least . This can always be accomplished due to the
stretchability-invariant, as both  and  are on the
outer face of . Potential crossings introduced by edges of
 containing diagonal segments can be eliminated by employing
similar cuts to the ones presented in the case of 4-planar graphs.
So, we may assume that  is plane. Also,  complies with
the stretchability-invariant, as one can define a cut that crosses
any of the newly inserted edges of  and then follows one of the
cuts of  that crosses an edge between  and
.

In case of a singleton  of degree  or , our
approach is very similar to the one of the case of 4-planar graphs.
Here, we mostly focus on the case where  is of degree five. In
this case, we have to deal with two additional edges (called
\emph{nested}) that connect  with , say  and
; see Fig.\ref{fig:5p_singleton}. Such a pair of edges
does not always allow vertex  to be placed along the next
available horizontal grid line; 's position is more or less
prescribed, as each of  and  may have only one northern  port
unoccupied. However, a careful case analysis on the type of ports
(i.e., north-west, north or north-east) that are unoccupied at 
and  in conjunction with the fact that  is horizontally
stretchable shows that we can always find a feasible placement for
 (usually far apart from ). Potential crossings due to
the remaining edges incident to  are eliminated by employing
similar cuts to the ones presented in the case of 4-planar graphs.
So, we may assume that  is planar. Similar to the case of a
chain, we prove that  complies with the
stretchability-invariant. In this case special attention should be
paid to avoid crossings with the nested edges of , as a nested
edge may contain no horizontal segment. Note that the case of the
last partition  is treated in the same way, even if
 is potentially incident to three nested edges; see
Fig.\ref{fig:5p_before_final}.

To complete the description of our approach it remains to describe
how edge  is drawn. By construction both  and 
are along a common horizontal line. So,  can be drawn
using two diagonal segments that form a bend pointing downwards; see
Fig.\ref{fig:5p_before_final}.

\begin{theorem}
Given a triconnected 5-planar graph , we can compute in 
time an octilinear drawing of  with at most one bend per edge.
\end{theorem}
\begin{proof}
Unfortunately, we can no longer use the shifting method of
Kant~\cite{Kant92b}, since the - and -coordinates are not
independent. However, the computation of each cut can be done in
linear time, which implies that our drawing algorithm needs 
time in total.
\end{proof}

Recall that when placing a singleton  that
has four edges to , the height of  is determined by
the horizontal distance of its neighbors along the outer face of
, which is bounded by the actual width of the drawing of
. On the other hand, when placing a chain  the amount
of horizontal stretching required in order to avoid potential
crossings is delimited by the height of the drawing of .
Unfortunately, this connection implies that for some input
triconnected 5-planar graphs our drawing algorithm may result in
drawings of super-polynomial area, as the following theorem
suggests.

\begin{figure}[t]
    \centering
    \includegraphics[width=.3\textwidth]{5p_exparea}
    \caption{A recursive construction of an infinite class of 5-planar graphs requiring super-polynomial drawing area.}
    \label{fig:5p_exparea}
\end{figure}

\begin{theorem}
There exist infinitely many triconnected 5-planar graphs for which
our drawing algorithm produces drawings of super-polynomial area.
\label{thm:5planarExp}
\end{theorem}
\begin{proof}
Fig.\ref{fig:5p_exparea} illustrates a recursive construction of an
infinite class of 5-planar triconnected graphs with this property.
The base of the construction is a ``long chain'' connecting 
and  (refer to the bold drawn edges of
Fig.\ref{fig:5p_exparea}). Each next member, say , of
this class is constructed by adding a constant number of vertices
(colored black in Fig.\ref{fig:5p_exparea}) to its immediate
predecessor member, say , of this class, as illustrated in
Fig.\ref{fig:5p_exparea}. If  and  is the width and the
height of , respectively, then it is not difficult to show that
 and , which implies that the
required area is asymptotically exponential.
\end{proof}

\subsection{The Biconnected Case}
\label{sec:bicon}


For the 4-planar case we defined several invariants in order to keep
the area of the resulting drawings polynomial. Since we drop this
requirement now we can define a (simpler) new invariant for the
biconnected 5-planar case. When considering a node  in
 and its poles , then in
the drawing of ,  and  are horizontally aligned at
the bottom of the drawing's bounding box as in
Fig.\ref{fig:5p_layout}. If an -edge is present, it can be
drawn at the bottom. An -edge only occurs in the pertinent
graph of a P-node (and Q-node). Again, we use the term \emph{fixed}
for a pole-node that is not allowed to form a nose. We maintain the
following properties through the recursive construction process: In
S- and R- nodes,  and  are not fixed. In P- and Q-nodes, only
one of them is fixed, say . But similar to the 4-planar
biconnected case, we may swap their roles.

\begin{figure}[t]
    \centering
    \begin{minipage}[b]{.24\textwidth}
        \centering
        \subfloat[\label{fig:5p_layout}{}]
        {\includegraphics[width=.95\textwidth,page=1]{5p_bicon_new}}
    \end{minipage}
    \begin{minipage}[b]{.24\textwidth}
        \centering
        \subfloat[\label{fig:5p_pnode}{}]
        {\includegraphics[width=.95\textwidth,page=3]{5p_bicon_new}}
    \end{minipage}
    \begin{minipage}[b]{.45\textwidth}
        \centering
        \subfloat[\label{fig:5p_snode}{}]
        {\includegraphics[width=.95\textwidth,page=2]{5p_bicon_new}}
    \end{minipage}
    \caption{
    (a)~Layout specification;  and  are located at the bottom.
    (b)~P-node with an -edge from a Q-node .  and  form a nose in .
    (c)~S-node example with four children .}
    \label{fig:5p_bicon}
\end{figure}

\begin{description}
\item[P-node case:] Let  be a P-node. It is not difficult to see
that  has at most  children; one of them might be a Q-node,
i.e., an -edge, which can be drawn at the bottom as a
horizontal segment. Since P-nodes are not adjacent to each other in
, the remaining children are S- or R-nodes. By our
invariant we may form noses enabling us to stack them as in
Fig.\ref{fig:5p_pnode}, as  and  are not fixed in them.
\item[S-node case:] Let  be an S-node with children . Instead of the diagonal layout used earlier, we now
align the drawings horizontally; see Fig.\ref{fig:5p_snode}. In the
S-node case, the poles inherit their pertinent degree from the
children and the same holds for the property of being fixed.
However, by our new invariant this is forbidden, as it clearly
states that  and  are not fixed. It is easy to see that when
 is a P-node,  is fixed by the invariant in . In
this case, we swap the roles of the poles in  such that 
is not fixed. However, the other pole of , say , is
fixed now. Since the skeleton of an S-node is a cycle of length at
least three,  holds. As a result, both  and  are
not fixed in the resulting drawing.
\item[R-node case:] To compute a layout of an R-node, we employ the
triconnected algorithm (with  and ). So, let 
be an R-node and  a child of  that corresponds to
the virtual edge  in . Then,  and  holds. When inserting the
drawing of , we require at most three consecutive
ports at  and  for the additional edges. As the triconnected
algorithm assigns ports in a consecutive manner based on the
relative position of the endpoints, we modify the port assignment so
that an edge may have more than one port assigned. To do so, we
assign each edge  in  a pair
 that
reflects the number of ports required by this edge at its endpoints.
Then, we extend the triconnected algorithm such that when a port of
 is assigned to an edge , 
additional consecutive ports in clockwise or counterclockwise order
are reserved. The direction depends on the different types of edges
that we will discuss next.

\begin{figure}[t]
    \centering
     \begin{minipage}[b]{.16\textwidth}
        \centering
        \subfloat[\label{fig:5p_port_reserv_chain}{}]
        {\includegraphics[width=.9\textwidth,page=6]{5p_bicon_new}}
    \end{minipage}
    \begin{minipage}[b]{.16\textwidth}
        \centering
        \subfloat[\label{fig:5p_port_reserv_chain_inserted}{}]
        {\includegraphics[width=.9\textwidth,page=7]{5p_bicon_new}}
    \end{minipage}
    \begin{minipage}[b]{.16\textwidth}
        \centering
        \subfloat[\label{fig:5p_port_reserv_leg}{}]
        {\includegraphics[width=.9\textwidth,page=4]{5p_bicon_new}}
    \end{minipage}
    \begin{minipage}[b]{.16\textwidth}
        \centering
        \subfloat[\label{fig:5p_port_reserv_leg_inserted}{}]
        {\includegraphics[width=.9\textwidth,page=5]{5p_bicon_new}}
    \end{minipage}
    \begin{minipage}[b]{.16\textwidth}
        \centering
        \subfloat[\label{fig:5p_port_reserv_singleton}{}]
        {\includegraphics[width=.9\textwidth,page=8]{5p_bicon_new}}
    \end{minipage}
    \begin{minipage}[b]{.16\textwidth}
        \centering
        \subfloat[\label{fig:5p_port_reserv_singleton_inserted}{}]
        {\includegraphics[width=.9\textwidth,page=9]{5p_bicon_new}}
    \end{minipage}
    \caption{
    (a)~Virtual edge  connecting two consecutive vertices of a chain. At both endpoints the drawing of  requires two ports.
    (b)~Replacing  in (a) with the corresponding drawing of the child .
    (c)~Example of an edge  that requires three ports at  and two at  .
    (d)~Inserting the drawing of  into (c) with  being fixed and  forming a nose.
    (e)~Reserving ports for the nested edges. A single port for a real edge is reserved and then two ports for the virtual edge e = .
    (f)~Final layout after inserting the drawing of .}
    \label{fig:5p_bicon_R}
\end{figure}

The simplest type of edges are the ones among consecutive vertices
 of a chain. Recall that  is a
special case and the edge  is drawn differently. Also,
the edges from  to  are drawn as horizontal segments; see
Fig.\ref{fig:5p_before_final}. For each such edge we reserve the
additional ports at  in counter-clockwise order and at
 in clockwise order; see
Fig.\ref{fig:5p_port_reserv_chain}. So, we can later plug the
drawing of the children into the layout as in
Fig.\ref{fig:5p_port_reserv_chain_inserted} without forming noses.
The second type of edges are the ones that connect  to   and  in
. No matter if  is a singleton or a chain, we proceed
by reserving the ports as in the previous case, i.e., at 
clockwise, ( counter-clockwise, resp.) and at 
counter-clockwise ( clockwise); see
Fig.\ref{fig:5p_port_reserv_leg}. In case where  or
 is a virtual edge, we choose the poles such that
 ( resp.) is fixed in  ( resp.). Thus, we can create a nose with  (
resp.). Having exactly the ports required at both endpoints, we
insert the drawing by replacing the bend with a nose as in
Fig.\ref{fig:5p_port_reserv_leg_inserted}. The remaining edges from
 to  in case of a singleton  can be
handled similarly; see Fig.\ref{fig:5p_bicon_R}. Notice that during
the replacement of the edges, the fixed vertex is always the upper
one. The only exception are the horizontal drawn edges of a chain.
There, it does not matter which one is fixed, as none of the poles
has to form a nose.
\item[Root case:] We root  at an arbitrarily chosen
Q-node representing a real edge . By our invariant we may
construct a drawing with  and  at the bottom of the drawing's
bounding box, hence, we draw the edge  below the bounding box
with a ninety degree bend using the south east port at  and south
west port at .
\end{description}

\begin{theorem}
Given a biconnected 5-planar graph , we can compute in 
time an octilinear drawing of  with at most one bend per edge.
\end{theorem}
\begin{proof}
We have shown that the ability to rotate and scale suffices to
extend the result from 4-planar to 5-planar at the expense of the
area. Similar to the 4-planar case, computing  takes
linear time. Hence, the overall runtime is governed by the
triconnected algorithm.
\end{proof}

\subsection{The Simply Connected Case}


In the following, we only outline the differences in comparison with
the corresponding 4-planar case. As an invariant, the drawing of
every subtree should conform to the layout depicted in
Fig.\ref{fig:5p_bc_tree_layout}. For a single biconnected component
, let  refer to the cut vertex linking it to the parent. As
root for the SPQR-tree  of , we again choose a
Q-node  whose real edge is incident to ; see
Fig.\ref{fig:5p_bc_t_root}. Hence, the layout generated by the
biconnected approach matches this scheme.

\begin{figure}[htb]
    \centering
     \begin{minipage}[b]{.24\textwidth}
        \centering
        \subfloat[\label{fig:5p_bc_tree_layout}{}]
        {\includegraphics[width=.9\textwidth,page=1]{5p_bc}}
    \end{minipage}
     \begin{minipage}[b]{.24\textwidth}
        \centering
        \subfloat[\label{fig:5p_bc_t_root}{}]
        {\includegraphics[width=.9\textwidth,page=2]{5p_bc}}
    \end{minipage}
     \begin{minipage}[b]{.24\textwidth}
        \centering
        \subfloat[\label{fig:5p_bc_tree_s_ports}{}]
        {\includegraphics[width=.9\textwidth,page=3]{5p_bc}}
    \end{minipage}
      \begin{minipage}[b]{.24\textwidth}
        \centering
        \subfloat[\label{fig:5p_bc_tree_single_ports}{}]
        {\includegraphics[width=.9\textwidth,page=4]{5p_bc}}
    \end{minipage}
    \caption{
    (a)~Layout scheme for a BC-subtree rooted at .
    (b)~Rooting  at a Q-node .
    (c)~Attaching a subtree at a chain and in (d)~at a singleton inside an R-node.}
    \label{fig:5p_bc_tree}
\end{figure}

It remains to show that we can attach the children. Since we are
able to scale and rotate, we keep things simple and look for
suitable spots to attach them. Notice that in the drawings of
S-nodes and chains in R-nodes all southern ports are free. Hence, we
may rotate the drawings of the subtrees and attach the at most three
(two for a chain) edges to  there (refer to
Fig.\ref{fig:5p_bc_tree_s_ports} for an example of a chain). The
only exception are the singletons. Assume that  is an ordinary
singleton that has one nested edge attached. Hence, it has degree
four, leaving us with a single bridge to attach the component;
Fig.\ref{fig:5p_bc_tree_single_ports}. However, this does not hold
in case . Consider the case where  has a nested edge
and we have to attach a subtree that requires two ports. As a result
 has degree  in  and, thus, all northern ports are
free.

\begin{theorem}
Given a connected 5-planar graph , we can compute in 
time an octilinear drawing of  with at most one bend per edge.
\label{thm:5planarconnected}
\end{theorem}
\begin{proof}
We described how to attach any subtree to cut vertices inside a
biconnected component. Furthermore, the component itself complies
with the layout scheme. In addition, this scheme enables us to
compose such drawings at a cut vertex using rotations.
\end{proof}

\section{A Note on Octilinear Drawings of 6-Planar Graphs}
\label{sec:6planar}
In this section, we show that it is not always possible to construct
a planar octilinear drawing of a given -planar graph with at most
one bend per edge. In particular, we present an infinite class of
6-planar graphs, which do not admit planar octilinear drawings with
at most one bend per edge.

\begin{theorem}
There exists an infinite class of 6-planar graphs which do not admit
planar octilinear drawings with at most one bend per edge.
\label{thm:6planar}
\end{theorem}
\begin{proof}
Our proof is heavily based on the following simple observation: If
the outer face  of a given planar octilinear
drawing  consists of exactly three vertices, say 
and , that have the so-called \emph{outerdegree-property},
i.e.,  and , then it
is not feasible to draw all edges delimiting
 with at most one bend per edge; one of them 
has to be drawn with (at least) two bends in . Next, we
construct a specific maximal 6-planar graph, in which each face has
at most one vertex of degree  and at least two vertices of degree
; see Fig.\ref{fig:6p_twobends}. This specific graph does not
admit a planar octilinear drawing with at most one bend, as its
outerface is always bounded by three vertices that have the
outerdegree-property.

To obtain an infinite class of 6-planar graphs with this property,
we give the following recursive construction. Let  and  be
two copies of the graph of Fig.\ref{fig:6p_twobends}. Let also 
be a bounded face of , . We proceed to subdivide each
edge of  by introducing a new vertex on it. We further assume
that the new vertices of  are pairwise adjacent (see the top
part of Fig.\ref{fig:6p_construction}). Hence, they form a
triangular face, say , in the augmented graph, say
, constructed in this manner. Up to now, each of the
newly introduced vertices is of degree four. Now, assume that
 is drawn on the plane so that  is a bounded face
in , and  is drawn such that  is
the unbounded face in . By choosing  as the
outer face in  we make sure that we can connect
the three degree four vertices of  to the three degree four
vertices of  in the following way: We appropriately scale down
 and proceed to draw it in the interior of 
without introducing any crossings (see the small gray-colored
triangle of the bottom drawing of Fig.\ref{fig:6p_construction}). If
we connect the vertices of  and  in an ``octahedron-like
manner'', then all vertices of  and  are of degree 
and the resulting graph, say , is
maximal -planar and has the outerdegree-property.
\end{proof}

\begin{figure}[t]
    \centering
    \begin{minipage}[b]{.48\textwidth}
        \centering
        \subfloat[\label{fig:6p_twobends}{}]
        {\includegraphics[width=\textwidth,page=1]{6p_twobends}}
    \end{minipage}
    \begin{minipage}[b]{.48\textwidth}
        \centering
        \subfloat[\label{fig:6p_construction}{}]
        {\includegraphics[width=\textwidth,page=2]{6p_twobends}}
    \end{minipage}
    \caption{(a)~A maximal -planar graph in which each face has at most one vertex of degree  (black-colored vertices) and at least two vertices of degree  (gray-colored vertices).
    From Euler's formula for maximal planar graphs, it follows that any graph with this property must have at least  vertices of degree .
    Hence, this is the smallest graph with this property.
    (b)~Illustration of the recursive construction.}
    \label{fig:6p_twobends_construction}
\end{figure}

\section{A Sample Octilinear Drawing with at most 1 bend per edge}
\label{sec:sample}


\begin{figure}[h!]
    \centering
    \includegraphics[width=\textwidth]{4p_example_large}
    \caption{Example layout of a biconnected 4-planar graph. Vertices are labeled by their indices. The
    corresponding SPQR-tree  has been rooted at a Q-node
    representing the edge  with the only child being an
    S-node whose skeleton is the simple cycle .
    It has two R-nodes as children, a smaller in the upper left (with
    poles ) and a larger one (with poles ) occupying most of the drawing area. The latter
    one contains two smaller S-nodes (with poles 
    and ) and a P-node (with poles ) that has two children. One of them being an
    -edge, the other one an S-node.}
    \label{fig:4p_example_large}
\end{figure}

\section{Conclusions}
\label{sec:conclusions}


Motivated by the fundamental role of planar octilinear
drawings in map schematization, we presented algorithms for their construction
with at most one bend per edge for 4- and 5-planar graphs.
We also improved the known bounds on the required number of slopes
for - and -planar drawings from  and , resp.
(\cite{KPP13}) to . Our work raises several open problems:

\begin{itemize}
\item Is it possible to construct planar octilinear drawings of
4-planar (5-planar, resp.) graphs with at most one bend per edge in
 (polynomial, resp.) area?
\item Does any triangle-free 6-planar graph admit a planar
octilinear drawing with at most one bend per edge?
\item What is the complexity to determine whether a -planar graph
admits a planar octilinear drawing with at most one bend per edge?
\item What is the number of necessary slopes for bendless drawings
of -planar graphs?
\end{itemize}

\bibliographystyle{abbrv}
\bibliography{references}


\end{document}
