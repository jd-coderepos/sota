\subsection{Proof of Theorem~\ref{Thm:GESimulation}}

If an xor-constraint conjunction  has a {\it \UP{}-propagation table} for
the set of variables , we denote this by
. 



\begin{lemma}
\label{Lem:PTableProp}
Let  be a satisfiable conjunction of xor-constraints such that  for some , and  ``alias'' variables for the subsets , respectively, and .
It holds that .
\end{lemma}


\begin{proof}
We prove the lemma by induction on the sequence .
The induction hypothesis is that Lemma~\ref{Lem:PTableProp} holds for the
case .

Base case: .
The claim holds trivially, because .

Induction step for . 
By the property PT1, the ``alias'' variable  for the set of variables  is present in  and the xor-constraint  is in .
By the induction hypothesis, it holds that .
By the property PT2, it holds that the xor-constraint  is in .
It follows that .

\end{proof}

\begin{lemma}
\label{Lem:PTableProp2}
Let  be a conjunction of xor-constraints such that  for some 
of variables in , and  be a satisfiable conjunction of
xor-constraints in  such that .
If  for some , then
it holds for the ``alias'' variable  for the subset  that .
\end{lemma}

\begin{proof}
By Lemma~\ref{Lemma:LinearCombs}, there is a subset  of xor-constraints in  such that .
By the property PT1, it holds that the the ``alias'' variable  for the set of variables  is present in  and the xor-constraint  is in .
Also by the property PT1, it holds for each xor-constraint  in  that the ``alias'' variable  for the set of
variables  is present in  and the xor-constraint
 and by the property PT3 the xor-constraint  is in .
It holds by Lemma~\ref{Lem:PTableProp}, that .
\end{proof}


\begin{lemma}
\label{Lem:PropTableImpl}
If  is an xor-constraint conjunction and , then  .
\end{lemma}

\begin{proof}
Consider the pseudo code for the algorithm  in Fig.~\ref{Fig:PropTable}.
The variable  takes the value of each subset of  in the loop in 
lines 1-5, and as the result  has a variable  for each non-empty
subset  of  such that  is in . The property PT1 is satisfied by the lines 2-3 and the property PT3 by the lines 4-5.

In the loop in lines 6-11 is iterated for every pair of subsets  such that . It holds for the smallest-indexed
variables  such that 
 the xor-constraints ,
, and  that the xor-constraint 
is in . This satisfies PT2.
\end{proof}

\begin{lemma}
\label{Lem:GEConjunctions}
Given an xor-constraint conjunction  
and an elimination order
 for the variables of  for the algorithm
 where , it holds that there is a sequence of xor-constraint conjunctions  in  and a sequence of sets of variables
 such that
it holds for each triple :
\begin{itemize}
\item 
\item ,
\item 
\item 
\end{itemize}
\end{lemma}
\begin{proof}
Assume an xor-constraint conjunction  and an elimination
order  for the variables 
for the algorithm  where .
The translation  in Figure~\ref{Fig:GETrans} is initialized with  and . The loop in lines 1-5 is run  times and 
takes the values . In the first iteration of the loop, all
variables of  are in the set . Then for each successive
iteration  it holds that  because
 is removed from the set  in the line 4. 
We now argue that the xor-constraints in the conjunction   are in . 
After choosing to ``eliminate'' the variable  in the line 3, the
xor-constraint conjunction  is augmented with ,
    so .
It is clear that , so  is identical to
the xor-constraint conjunction  after the th iteration of the
loop.
Upon th iteration of the loop in the lines 1-5,       
the translation  in Figure~\ref{Fig:PropTable} is initialized
with  and . 
After all the  iterations are done it is clear that .
By Lemma~\ref{Lem:PropTableImpl} 
it holds that 
and also .
because adding xor-constraints cannot break any conditions
of the UP-propragation table.
\end{proof}

\begin{lemma}
\label{Lem:GEPropagation}
Given a satisfiable xor-constraint conjunction  in an
xor-constraint conjunction  and an elimination
order
 for the variables of  for the algorithm
 where , it holds that there is a sequence of xor-constraint conjunctions  in 
such that for each  in  it holds that
\begin{itemize}
\item given literals  such that , it holds that
    .
\end{itemize}
\end{lemma}

\begin{proof}
Assume a satisfiable xor-constraint clause conjunction  in an
xor-constraint conjunction  and an elimination
order  for the variables of 
for the algorithm .

By Lemma~\ref{Lem:GEConjunctions}, it holds that
there is a sequence of xor-constraint conjunctions  in  and a sequence of sets of variables
 such that
it holds for each triple :
\begin{itemize}
\item 
\item ,
\item 
\end{itemize}

Let  be the set of variables the ``disappear'' in the
normal form of the linear combination of the xor-constraints in .

We define a corresponding sequence of  tuples  as follows:
\begin{itemize}
\item 
, and
\item 
be the set of variables have occurrences in the xor-constraints of the variable 
and also remain in the normal form of the linear combination of , and
\item 
be the set of variables remain in the normal form of the linear combination of the xor-constraints of the variable  that also disappear in the normal form of the linear combination of , and
\item  is a variable such that the xor-constraint  is in  (it exists because  and ), and
\item if  or , then , otherwise
\begin{itemize}
\item if  is in ,
    then , otherwise
\item 
.
\end{itemize}
\end{itemize}

We prove the lemma by induction on the structure of the xor-constraint conjunction
sequence .

The induction hypothesis is that the lemma holds for the xor-constraint conjunction
sequence .

Base case: . 
Assume any literals  such that
. 
It holds that , so . 
It clearly holds that .

Induction step: .
Assume any literals  such that  and . 
If , then it holds by the induction hypothesis that .


We have two cases to consider:
\begin{itemize}
\item Case I: . 
It holds that 
  and ,
  so by induction hypothesis it holds that . 
It holds that . 
By Lemma~\ref{Lem:PTableProp2}, it holds that .

\item Case II: . 
It holds that .
By Lemma~\ref{Lem:PTableProp2}, it holds that . 
It holds that , so .
It holds by induction hypothesis that .
\end{itemize}
\end{proof}

The following lemma states that  translation refutes any
unsatisfiable xor-constraint conjunctions.
\begin{lemma}
\label{Lem:GEUnsat}
If  is an unsatisfiable xor-constraint conjunction, then
 where
.
\end{lemma}

\begin{proof}
Assume an unsatisfiable xor-constraint conjunction .
By Lemma~\ref{Lemma:LinearCombs}, there is a subset  of xor-constraints in  
such that .
Let .  It holds that . Thus,  is satisfiable. 
Let .
By
Lemma~\ref{Lem:GEConjunctions}, it holds that:
\begin{itemize}

\item there is a set of variables  such that  and , and

\item there is a variable 
such that the xor-constraint  is in , and

\item the xor-constraint  is in .
\end{itemize}

Because , it holds 
by Lemma~\ref{Lem:GEPropagation} that .
Since  and , it follows
that .
\end{proof}

\begin{lemma} 
\label{Lem:PropTableModels}
The satisfying truth assignments of  are exactly the
ones of  when projected to  where .
\end{lemma}

\begin{proof}
It holds by definition that , so it suffices to show that if  is a satisfying truth
assignment for , it can be extended to a satisfying truth
assignment  for . 
Assume that  is a truth assignment such that .
Let  be a truth assignment identical to  except for the following additions. The translation  in Figure~\ref{Fig:PropTable} adds four kinds of xor-constraints.
\begin{enumerate}
\item  where  is a non-empty subset of 
and  is a new variable. If , add  to , otherwise add  to . It is clear that .

\item  if the xor-constraints , , and  are in  augmented with xor-constraints from the
previous step.
From the previous step it is clear that , , and
. It follows that .

\item  where  such that
the xor-constraints  and  are in  augmented with xor-constraints from the previous step. Since  is an original xor-constraint in ,
    it holds that . It follows that .

\item  where  and  such that
the xor-constraints  and  where  is a non-empty subset of 
are in  augmented with xor-constraints from the previous step. If , then add  to , otherwise add  to . If , then add  to , otherwise add  to .
It follows that .
\end{enumerate}
\end{proof}

\begin{retheorem}{\ref{Thm:GESimulation}} 
If  is an xor-constraint
conjunction, then  is a GE-simulation formula for
 provided that .
\end{retheorem}

\begin{proof}
We first prove that the satisfying truth assignments of  are
exactly the ones of  when
projected to . 
The translation \getransname{} in Figure~\ref{Fig:GETrans} only adds xor-constraint
conjunctions of the type  for some set of variables  and some xor-constraint conjunction 
and by Lemma~\ref{Lem:PropTableModels} the satisfying assignments
of  are exactly the ones of  
when projected to .
It follows by induction that the satisfying truth assignment for 
are exactly to the ones of  when projected to .

Next we show that if  is satisfiable 
and 
 , then  is -derivable from . 
By Lemma~\ref{Lemma:LinearCombs}, there is a subset  of xor-constraints in  such that ,
By Lemma~\ref{Lem:GEPropagation}, it holds that 
, so .

It remains to show that if  is unsatisfiable,
   then . Assume that 
    is unsatisfiable. By Lemma~\ref{Lem:GEUnsat}, it holds
   that .
All the requirements for GE-simulation formula are satisfied, so  is a GE-simulation formula for .
\end{proof}

\subsection{Proof of Theorem~\ref{Thm:XorSimp}}
\begin{retheorem}{\ref{Thm:XorSimp}}
If  is the result of applying
one of the simplification rules to  
and 
, then .  
\end{retheorem}

\begin{proof}
Let  be the result of applying one of the
simplification rules to , , and .
If S1 was the simplification rule used, then it clearly
holds that .

Otherwise, S2 was used to simplify an xor-constraint  in  with
an xor-constraint  in  such that .
It holds that 
and .
It must hold that there is an xor-clause  in  such that for each  it holds that .
We prove by induction  is -derivable from .
The induction hypothesis is that for each  it
holds that .

Base case: . If , then  is in  and . Otherwise, .
Since , it holds that
 for some 
.
The xor-constraint  is in , and the
xor-constraint  .
It clearly holds that .

Induction step: . If , then  is in 
and . Otherwise .
We have two cases to consider:
\begin{itemize}
\item Case 1: .
By induction hypothesis it holds for each 
that .
If there is a variable  such that , then  and then
, so .
Otherwise, it holds , and it clearly holds that .

\item Case 2: .
By induction hypothesis it holds for each 
that .
If there is a variable  such that , then  and then
, so .
Otherwise, it holds , and it clearly holds that .
\end{itemize}
\end{proof}

\subsection{Proof of Theorem~\ref{Thm:XCut}}
\newcommand{\CutAp}{V'_\textup{a}}
\newcommand{\CutBp}{V'_\textup{b}}
\newcommand{\xorpartA}{\xorpart^\textup{a}}
\newcommand{\xorpartB}{\xorpart^\textup{b}}
\newcommand{\Iface}{X'}
\newcommand{\Ifaceparity}{\parity{}'}


\begin{retheorem}{\ref{Thm:XCut}}
Let  be an -cut partition of .
Let , , and .
 Then
it holds that:
\begin{itemize}
\item If  is unsatisfiable, then
\begin{enumerate}
\item  
or  is unsatisfiable; or 
\item  and  for some 
and .
\end{enumerate}
\item 
If  is satisfiable
and  for some , , and  where  and , then
\begin{enumerate}
\item 
or ;
or
\item  and  for some ,
, , and .
\end{enumerate}
\end{itemize}
\end{retheorem}
\begin{proof}
  Let  be an -cut partition of
  
  with
  ,
  ,
  , and
  .
Such partition exists because the xor-assumption literals 
  are unit xor-constraints.

  \begin{itemize}
\item  Case I:  is unsatisfiable.
By Lemma~\ref{Lemma:LinearCombs},
  there is a subset  of xor-constraints
  in 
  such that .
Observe that
  .
If ,
  then
   is also unsatisfiable.
Similarly,
  if ,
  then  is unsatisfiable.
Otherwise,
  it must be that
  
  and
  
  with 
  because
  ,
   and
  .
Thus
   and
  .
  
 \item Case II:
   is satisfiable
  and
  .
By Lemma~\ref{Lemma:LinearCombs},
 there is a subset  of xor-constraints
  in 
  such that .
Again,
  observe that
  .
Assume that  and ; the other case is symmetric.
Then we simplify the equation
 
by (i) substituting  with  and (ii) evaluating .
This gives two cases:
\begin{enumerate}
\item evaluating  gives
an empty expression and the simplified equation is then , so it follows that
  .

\item evaluating  gives
an xor-constraint  for some  and  because 
  ,
   and
  .
The simplified equation is then ,
so it follows that
  .
\end{enumerate}
\end{itemize}
\end{proof}

\subsection{Proof of Theorem~\ref{Thm:TreeDecomposition}}

\begin{lemma}
\label{Lem:PCProp}
If  is a satisfiable conjunction in  such that
, , ,
and 
where 
and , 
then 
where  are the ``alias'' variables for the sets , respectively.
\end{lemma}

\begin{proof}
By Lemma~\ref{Lemma:LinearCombs}, there is a subset  of xor-constraints in  such that .
By the property PT1, it holds for each xor-constraint  in  
that that the corresponding ``alias'' variable  for the set of
variables  is present in 
and by the property PT3 the xor-constraint  is in .
It holds by Lemma~\ref{Lem:PTableProp}, that .
\end{proof}


\begin{retheorem}{\ref{Thm:TreeDecomposition}}
If  is the family of variable sets in the tree
decomposition of the primal graph of an xor-constraint conjunction  and 
 is a sequence of xor-constraint conjunctions
such that  and  for , then  is a GE-simulation formula for
 with  xor-constraints, where
.  
\end{retheorem}

\begin{figure}[ht]
\centering
\begin{tabular}{c@{\qquad}c}
\includegraphics[scale=0.4]{Figures/tw_ex}
&
\includegraphics[scale=0.4]{Figures/tw_ex2}
\\
(a) a constraint graph
&
(b) subgraph of the constraint graph
\\
\\
\end{tabular}
\caption{(a) A constraint graph for an instance , (b) subgraph of the constraint graph illustrating that }
\label{Fig:TWEx}
\end{figure}

\begin{figure}[ht]
\centering
\includegraphics[width=0.5\textwidth]{Figures/tw_primal}
\caption{Primal graph for the instance whose constraint graph is shown in Fig.~\ref{Fig:TWEx}(a)}
\label{Fig:TWExp}
\end{figure}

\begin{figure}[ht]
\centering
\includegraphics[width=0.5\textwidth]{Figures/tw_tree}
\caption{Tree decomposition of the primal graph in Fig.~\ref{Fig:TWExp}. Assume
that , , , , , , and
, and . It holds that
        . The \UP{} system can deduce , i.e.  by ``propagating'' intermediate linear combinations
        starting from the leaves of the tree decomposition towards the root
        node (the node with the set of variables ). Since  is in
         it holds that . And in a similar way because , ,and  are in , then
        ,
    , and . By combining these
        intermediate results, it holds that  and finally }
        \label{Fig:TWTree}
\end{figure}






\begin{proof}
The construction is illustrated in Figures~\ref{Fig:TWEx}, \ref{Fig:TWExp}, \ref{Fig:TWTree}.

Let .
We first prove that the satisfying truth assignments of  are
exactly the ones of  when projected to
. 
By Lemma~\ref{Lem:PropTableModels}, the satisfying truth assignments of  are exactly the ones of  when projected
to , so by induction the satisfying truth assignments of  are exactly the ones of  when projected
to .
The number of xor-constraints in  is , so the number of xor-constraints in  is .

It holds for each  by
Lemma~\ref{Lem:PropTableImpl} that .
Next we show that if  is satisfiable and , then  is -derivable from . 
Assume that  is satisfiable and .
\newcommand{\subtree}{\ensuremath{T'}}
We prove by induction on the structure of the tree decomposition that the
following property holds for each subtree  of the tree decomposition having the set of variables  and the root node of
 with the set of variables :
\begin{itemize}
\item If  is a satisfiable conjunction in  such that
,
and  
where  and for each  there is a
 for which it holds that 
and
, 
then 
where  are the ``alias'' variables for the variable sets
, respectively.
\end{itemize}
The induction hypothesis is that the property holds for each proper subtree of
.

Base case:  has only one node. The property holds by Lemma~\ref{Lem:PCProp}.


Induction step:  has more than one node.
Let . 
The idea is to remove xor-constraints involving variables other than in
 from  and add additional xor-constraints of the type involving variables only in .
This is done by considering each direct child node of the root node of  at
a time possibly rewriting  by substituting a sub-conjunction of  with at most one xor-constraint having only variables in .
Let  be the subtree induced by one direct child node of the root node
having the set of variables . The per-child substitution operation of  is defined as follows.
\newcommand{\phiA}{\phi^{\textup{a}}}
\newcommand{\phiB}{\phi^{\textup{b}}}
Let  be the maximal conjunction of xor-constraints in  such that
, and  be the conjunction of
xor-constraints in  but not in .
If  is empty, then nothing needs to be removed from .
Otherwise,  is non-empty and there is an -cut partition
 of  such that ,  and .
By Theorem~\ref{Thm:XCut}, it holds that
\begin{enumerate}
\item 
or ;
or
\item  and  for some ,
, , and .
\end{enumerate}

We analyze the cases:
\begin{itemize}
\item[] Case 1: 
or . Since , it must be that . In this case, set .
\item[] Case 2:  and  for some ,
, , and . Again since , it must be that . In this case, set .
\end{itemize} 

After each child node has been processed in this way, it holds that  and .
It also holds by the induction hypothesis for each xor-constraint  in the
sequence  of
added xor-constraints that the corresponding ``alias'' variables  for , respectively, that,
      since , then it holds
      that . 
Now , ,
and each xor-constraint  in  has its corresponding ``alias'' variable  implied by unit propagation, that is, .
By induction it follows that .

It remains to show that if  is unsatisfiable,
then . Assume that 
 is unsatisfiable. 
By Lemma~\ref{Lemma:LinearCombs}, there is a minimal subset  of
xor-constraints in 
such that . 
Now, let 
be a subset of  identical to  except that one xor-constraint  in  is removed.
It clearly holds that  is satisfiable and . 
There is a node in  that has the variables  such that .
It holds that , so by Lemma~\ref{Lem:PTableProp2} it holds for the ``alias'' variable  for  that 
.
Repeat the proof as above for the satisfiable case and for the subset  showing
that .
It follows that .
All the requirements for GE-simulation formula are satisfied, so 
 is a GE-simulation formula for .
\end{proof}


