\subsection{Proof of Theorem~\ref{Thm:GESimulation}}

If an xor-constraint conjunction $\psi$ has a {\it \UP{}-propagation table} for
the set of variables $Y \subseteq \VarsOf{\psi}$, we denote this by
$\HasPropTable{Y}{\psi}$. 



\begin{lemma}
\label{Lem:PTableProp}
Let $ \phi $ be a satisfiable conjunction of xor-constraints such that $
\HasPropTable{Y}{\phi} $ for some $ Y \subseteq
\VarsOf{\phi} $, and $ a, a_1, \dots, a_n
\in \VarsOf{\phi} $ ``alias'' variables for the subsets $Y', Y_1, \dots,
    Y_n \subseteq Y$, respectively, and $ Y' = Y_1 \oplus \dots \oplus Y_n $.
It holds that $ \phi \wedge (a_1 \equiv \parity{1}) \wedge \dots \wedge
(a_n \equiv \parity{n}) \UPderiv (a \equiv \parity{1} \oplus \dots \oplus
    \parity{n})$.
\end{lemma}


\begin{proof}
We prove the lemma by induction on the sequence $ a_1, \dots, a_n $.
The induction hypothesis is that Lemma~\ref{Lem:PTableProp} holds for the
case $a_1, \dots, a_{n-1} $.

Base case: $n = 1$.
The claim holds trivially, because $a = a_1$.

Induction step for $ n > 1 $. 
By the property PT1, the ``alias'' variable $a'$ for the set of variables $(Y_1 \oplus \dots \oplus Y_{n-1})$ is present in $\VarsOf{\phi}$ and the xor-constraint $ (a' \oplus Y_1 \oplus
        \dots \oplus Y_{n-1} \equiv \bot) $ is in $\phi$.
By the induction hypothesis, it holds that $ \phi \wedge (a_1 \equiv
        \parity{1}) \wedge \dots \wedge (a_{n-1} \equiv \parity{n-1}) \UPderiv
(a' \equiv \parity{1} \oplus \dots \oplus \parity{n-1}) $.
By the property PT2, it holds that the xor-constraint $ (a \oplus a_{n-1}
\oplus a' \equiv \bot) $ is in $ \phi$.
It follows that $ \phi \wedge (a_1 \equiv \parity{1}) \wedge \dots \wedge (a_n \equiv \parity{n}) \UPderiv (a \equiv \parity{1} \oplus \dots \oplus \parity{n}) $.

\end{proof}

\begin{lemma}
\label{Lem:PTableProp2}
Let $ \phi $ be a conjunction of xor-constraints such that $
\HasPropTable{Y}{\phi} $ for some $ Y \subseteq \VarsOf{\phi} $
of variables in $\phi$, and $ \phi' $ be a satisfiable conjunction of
xor-constraints in $\phi$ such that $ \VarsOf{\phi'} \subseteq Y$.
If $ \phi' \Models (Y' \equiv \parity{}) $ for some $Y' \subseteq Y$, then
it holds for the ``alias'' variable $ a \in \VarsOf{\phi} $ for the subset $Y'$ that $ \phi \UPderiv (a \equiv \parity{}) $.
\end{lemma}

\begin{proof}
By Lemma~\ref{Lemma:LinearCombs}, there is a subset $S = (Y_1 \equiv \parity{1}) \wedge \dots \wedge (Y_n \equiv \parity{n}) $ of xor-constraints in $\phi'$ such that $ \SetLC{S} = (Y' \equiv \parity{}) $.
By the property PT1, it holds that the the ``alias'' variable $a$ for the set of variables $Y'$ is present in $ \VarsOf{\phi} $ and the xor-constraint $ (a \oplus Y' \equiv \bot) $ is in $\phi$.
Also by the property PT1, it holds for each xor-constraint $ (Y_i \equiv
\parity{i}) $ in $S$ that the ``alias'' variable $ a_i $ for the set of
variables $Y_i$ is present in $\VarsOf{\phi}$ and the xor-constraint
$ (a_i \oplus Y_i \equiv
\bot) $ and by the property PT3 the xor-constraint $ (a_i \equiv
    \parity{i}) $ is in $ \phi$.
It holds by Lemma~\ref{Lem:PTableProp}, that $ \phi \wedge (a_1 \wedge
\parity{1}) \wedge \dots \wedge (a_n \equiv \parity{n}) \UPderiv
(a \equiv \parity{}) $.
\end{proof}


\begin{lemma}
\label{Lem:PropTableImpl}
If $ \xorclauses$ is an xor-constraint conjunction and $ Y \subseteq
\VarsOf{\xorclauses}$, then $ \HasPropTable{Y}{\xorclauses \wedge \PropTable{Y,
    \xorclauses, |Y|}}$ .
\end{lemma}

\begin{proof}
Consider the pseudo code for the algorithm $\PropTableName$ in Fig.~\ref{Fig:PropTable}.
The variable $ Y' $ takes the value of each subset of $ Y $ in the loop in 
lines 1-5, and as the result $ \xorclauses'$ has a variable $a$ for each non-empty
subset $Y'$ of $ Y $ such that $ (a \oplus Y' \equiv \parity{}) $ is in $ \xorclauses'$. The property PT1 is satisfied by the lines 2-3 and the property PT3 by the lines 4-5.

In the loop in lines 6-11 is iterated for every pair of subsets $ Y_1, Y_2
\subset Y $ such that $ Y_1 \not = Y_2$. It holds for the smallest-indexed
variables $a_1,a_2,a_3 \in \VarsOf{\xorclauses}$ such that 
 the xor-constraints $ (a_1 \oplus Y_1 \equiv  \bot) $,
$ (a_2 \oplus Y_2 \equiv \bot)$, and $ (a_3 \oplus (Y_1 \oplus Y_2) \equiv
    \bot)$ that the xor-constraint $ (a_1 \oplus a_2 \oplus a_3 \equiv \bot)$
is in $ \xorclauses' $. This satisfies PT2.
\end{proof}

\begin{lemma}
\label{Lem:GEConjunctions}
Given an xor-constraint conjunction $\phi_0$ 
and an elimination order
$\Tuple{x_1,\dots,x_n}$ for the variables of $\phi_0$ for the algorithm
$\getransname$ where $k=|\VarsOf{\phi_0}|$, it holds that there is a sequence of xor-constraint conjunctions $
\Tuple{\phi_1,\dots,\phi_n}$ in $ \psi = \phi_0 \wedge \getrans{\phi_0} $ and a sequence of sets of variables
$\Tuple{Y_1,\dots,Y_n}$ such that
it holds for each triple $ \Tuple{x_i,Y_i,\phi_i}$:
\begin{itemize}
\item $ Y_i = \VarsOf{\ClausesOf{x_i}{\phi_{i-1}}} \cap \Set{x_i,\dots,x_n} $
\item $ \HasPropTable{Y_i}{\psi}$,
\item $\phi_i = \phi_{i-1} \wedge \PropTable{Y_i,\phi_{i-1}, k}$
\item $\phi_n = \phi_0 \wedge \getrans{\phi_0}$
\end{itemize}
\end{lemma}
\begin{proof}
Assume an xor-constraint conjunction $ \phi_0 $ and an elimination
order $ \Tuple{x_1,\dots,x_n}$ for the variables $ \phi_0$
for the algorithm $ \getransname$ where $k=|\VarsOf{\phi_0}|$.
The translation $ \getrans{\phi_0} $ in Figure~\ref{Fig:GETrans} is initialized with $ \xorclauses' = \phi_0
$ and $ V = \VarsOf{\phi_0} $. The loop in lines 1-5 is run $n$ times and $V$
takes the values $ V_1,\dots,V_n$. In the first iteration of the loop, all
variables of $ \phi_0 $ are in the set $ V_1 = V$. Then for each successive
iteration $ i $ it holds that $ V_i = V_{i-1} \setminus \Set{x_{i-1}} $ because
$ x_i $ is removed from the set $ V$ in the line 4. 
We now argue that the xor-constraints in the conjunction  $ \phi_1 \wedge \dots \wedge \phi_n$ are in $ \psi = \phi_0 \wedge \getrans{\phi_0} $. 
After choosing to ``eliminate'' the variable $ x_i$ in the line 3, the
xor-constraint conjunction $ \xorclauses' $ is augmented with $
\PropTable{\VarsOf{\ClausesOf{x_i}{\xorclauses'}} \cap V_i, \xorclauses', k}$,
    so $\phi_i = \phi_{i-1} \wedge \PropTable{Y_i,\phi_{i-1}, k}$.
It is clear that $ V_i = \Set{x_i, \dots, x_n}$, so $ \phi_i $ is identical to
the xor-constraint conjunction $ \xorclauses'$ after the $i$th iteration of the
loop.
Upon $i$th iteration of the loop in the lines 1-5,       
the translation $ \PropTableName $ in Figure~\ref{Fig:PropTable} is initialized
with $Y = Y_i $ and $ \xorclauses' = \phi_{i-1}$. 
After all the $n$ iterations are done it is clear that $ \phi_n = \phi_0
\wedge \getrans{\phi_0}$.
By Lemma~\ref{Lem:PropTableImpl} 
it holds that $ \HasPropTable{Y_i}{\xorclauses'}$
and also $ \HasPropTable{Y_i}{\psi}$.
because adding xor-constraints cannot break any conditions
of the UP-propragation table.
\end{proof}

\begin{lemma}
\label{Lem:GEPropagation}
Given a satisfiable xor-constraint conjunction $ \phi_0' $ in an
xor-constraint conjunction $ \phi_0$ and an elimination
order
$\Tuple{x_1,\dots,x_n}$ for the variables of $\phi_0$ for the algorithm
$\getransname$ where $k=|\VarsOf{\phi_0}|$, it holds that there is a sequence of xor-constraint conjunctions $
\Tuple{\phi_1',\dots,\phi_n'}$ in $ \psi = \phi_0 \wedge \getrans{\phi_0} $
such that for each $ \phi_i'$ in $\Tuple{\phi_0',\dots,\phi_n'}$ it holds that
\begin{itemize}
\item given literals $ \AL_1, \dots, \AL_k, \IL$ such that $ (\SetLC{\phi_i'}) \wedge \AL_1 \wedge \dots \wedge \AL_k \Models \IL $, it holds that
    $ \psi \wedge \AL_1 \wedge \dots \wedge \AL_k \UPderiv \IL $.
\end{itemize}
\end{lemma}

\begin{proof}
Assume a satisfiable xor-constraint clause conjunction $ \phi_0' $ in an
xor-constraint conjunction $ \phi_0 $ and an elimination
order $ \Tuple{x_1, \dots, x_n}$ for the variables of $ \phi_0$
for the algorithm $ \getransname $.

By Lemma~\ref{Lem:GEConjunctions}, it holds that
there is a sequence of xor-constraint conjunctions $
\Tuple{\phi_1,\dots,\phi_n}$ in $ \psi = \phi_0 \wedge \getrans{\phi_0} $ and a sequence of sets of variables
$\Tuple{Y_1,\dots,Y_n}$ such that
it holds for each triple $ \Tuple{x_i,Y_i,\phi_i}$:
\begin{itemize}
\item $ Y_i = \VarsOf{\ClausesOf{x_i}{\phi_{i-1}}} \cap \Set{x_i,\dots,x_n} $
\item $ \HasPropTable{Y_i}{\psi}$,
\item $\phi_i = \phi_{i-1} \wedge \PropTable{Y_i,\phi_{i-1},k}$
\end{itemize}

Let $ \LCVarsGone{\phi_{i-1}'} = \VarsOf{\phi_{i-1}'} \setminus
\VarsOf{\SetLC{\phi_{i-1}'}}$ be the set of variables the ``disappear'' in the
normal form of the linear combination of the xor-constraints in $ \phi_{i-1}'$.

We define a corresponding sequence of $n$ tuples $\Tuple{Y_i', X_i, V_i, a_i,
    \phi_i'} $ as follows:
\begin{itemize}
\item $Y_i' = \VarsOf{\ClausesOf{x_i}{\phi_{i-1}'}} \cap \Set{x_i,\dots,x_n} $
, and
\item $X_i = \VarsOf{\SetLC{\phi_{i-1}'}} \cap \VarsOf{\ClausesOf{x_{i}}{\phi_i'}}$
be the set of variables have occurrences in the xor-constraints of the variable $x_i$
and also remain in the normal form of the linear combination of $ \phi_{i-1}'$, and
\item $V_i = \VarsOf{\SetLC{\ClausesOf{x_i}{ \phi_{i-1}'}}} \cap \LCVarsGone{\phi_{i-1}'} $
be the set of variables remain in the normal form of the linear combination of the xor-constraints of the variable $x_i$ that also disappear in the normal form of the linear combination of $ \phi_{i-1}'$, and
\item $a_i$ is a variable such that the xor-constraint $(a_i \oplus V_i \equiv \parity{i})$ is in $ \phi_i \wedge (\bot \equiv \bot) $ (it exists because $V_i \subseteq Y_i$ and $ \HasPropTable{Y_i}{\phi_i}$), and
\item if $ x_i \not \in \VarsOf{\phi_i'} $ or $ V_i = \emptyset $, then $ \phi_i' = \phi_{i-1}' $, otherwise
\begin{itemize}
\item if $ (a_i \oplus V_i \equiv \parity{i} \oplus \parity{i}') $ is in $ \phi_{i-1}'$,
    then $ \phi_i' = \phi_{i-1}' \setminus \ClausesOf{x_i}{\phi_{i-1}'}$, otherwise
\item 
$ \phi_i' = \phi_{i-1}' \setminus \ClausesOf{x_i}{\phi_{i-1}'} \wedge (a_i \oplus V_i \equiv \parity{i} \oplus \parity{i}')$.
\end{itemize}
\end{itemize}

We prove the lemma by induction on the structure of the xor-constraint conjunction
sequence $ \Tuple{\phi_0', \dots, \phi_n'} $.

The induction hypothesis is that the lemma holds for the xor-constraint conjunction
sequence $ \Tuple{\phi_{i}',\dots,\phi_n'} $.

Base case: $i = n$. 
Assume any literals $ \AL_1,\dots,\AL_k $ such that
$\SetLC{\phi_i'} \wedge \AL_1 \wedge \dots \wedge \AL_k \Models \IL $. 
It holds that $ \VarsOf{\phi_i'} = \emptyset $, so $ \VarsOf{\IL} \in \VarsOf{\AL_1, \dots, \AL_k}$. 
It clearly holds that $ \phi_0 \wedge \AL_1 \wedge \dots \wedge
\AL_k \UPderiv \IL $.

Induction step: $ 0 \leq i-1 < n $.
Assume any literals $ \AL_1,\dots,\AL_k $ such that $
\VarsOf{\AL_1,\dots,\AL_k,\IL} \subseteq \VarsOf{\phi_0} $ and $
\SetLC{\phi_{i-1}'} \wedge \AL_1 \wedge \dots \wedge \AL_k \Models \IL $. 
If $ \phi_{i-1}' = \phi_{i}'$, then it holds by the induction hypothesis that $ \phi_0 \wedge \AL_1 \wedge \dots \wedge \AL_k \UPderiv \IL $.


We have two cases to consider:
\begin{itemize}
\item Case I: $\VarsOf{\IL} \in X_i$. 
It holds that $ \VarsOf{\SetLC{\phi_{i}'}} 
  \subseteq \VarsOf{\AL_1,\dots,\AL_k} \cup \Set{a_{i}}$
  and $ \SetLC{\phi_i'} \wedge \AL_1 \dots \wedge \AL_k \Models (V_i \equiv \parity{i}) $,
  so by induction hypothesis it holds that $ \phi_0 \wedge \AL_1 \wedge \dots \wedge \AL_k \UPderiv (a_i \equiv \parity{i}') $. 
It holds that $ \SetLC{\ClausesOf{x_i}{\phi_{i-1}'}} \wedge \AL_1 \wedge \dots \wedge \AL_k \wedge (a_i \equiv \parity{i}') \Models \IL $. 
By Lemma~\ref{Lem:PTableProp2}, it holds that $ \phi_i \wedge \AL_1 \wedge \dots
\AL_k \wedge (a_i \equiv \parity{i}') \UPderiv \IL $.

\item Case II: $\VarsOf{\IL} \not \in X_i$. 
It holds that $ \SetLC{\ClausesOf{x_i}{\phi_{i-1}'}} \wedge \AL_1 \wedge \dots
\wedge \AL_k \Models (V_i \equiv \parity{i}) $.
By Lemma~\ref{Lem:PTableProp2}, it holds that $ \phi_i \wedge \AL_1 \wedge \dots \wedge \AL_k \UPderiv (a_i \equiv \parity{i}') $. 
It holds that $ \VarsOf{\SetLC{\phi_{i}'}} \subseteq \VarsOf{\AL_1, \dots, \AL_k, \IL} \cup \Set{a_i}$, so $\SetLC{\phi_{i}'} \wedge \AL_1 \wedge \dots \wedge \AL_k \wedge (a_i \equiv \parity{i}') \Models \IL$.
It holds by induction hypothesis that $ \phi_0 \wedge \AL_1 \wedge \dots \wedge \AL_k \wedge (a_i \equiv \parity{i}') \UPderiv \IL$.
\end{itemize}
\end{proof}

The following lemma states that $\getransname$ translation refutes any
unsatisfiable xor-constraint conjunctions.
\begin{lemma}
\label{Lem:GEUnsat}
If $\xorclauses$ is an unsatisfiable xor-constraint conjunction, then
$\xorclauses \wedge \getrans{\xorclauses} \UPderiv (\bot \equiv \top) $ where
$k=|\VarsOf{\xorclauses}|$.
\end{lemma}

\begin{proof}
Assume an unsatisfiable xor-constraint conjunction $ \xorclauses$.
By Lemma~\ref{Lemma:LinearCombs}, there is a subset $S = C_1 \wedge \dots \wedge C_m $ of xor-constraints in $\xorclauses$ 
such that $\SetLC{S} = (\bot \equiv \top) $.
Let $ (X_m \equiv \parity{m}) = C_m$.  It holds that $ \SetLC{C_1 \wedge \dots
\wedge C_{m-1}} = (X_m \equiv \parity{m} \oplus \top) $. Thus, $ \psi = C_1
\wedge \dots \wedge C_{m-1} $ is satisfiable. 
Let $\psi = \xorclauses \wedge \getrans{\xorclauses}$.
By
Lemma~\ref{Lem:GEConjunctions}, it holds that:
\begin{itemize}

\item there is a set of variables $ Y \subseteq
\VarsOf{\psi} $ such that $ \VarsOf{C_m}
\subseteq Y $ and $ \HasPropTable{Y}{\psi} $, and

\item there is a variable $ y \in \VarsOf{\psi}$
such that the xor-constraint $ (y \oplus X_m \equiv
    \parity{m} \oplus \parity{m}') $ is in $\psi$, and

\item the xor-constraint $ (y \equiv \parity{m}') $ is in $ \psi$.
\end{itemize}

Because $ \psi \Models (y \equiv \parity{m}' \oplus \top) $, it holds 
by Lemma~\ref{Lem:GEPropagation} that $\psi \UPderiv (y \equiv \parity{m}' \oplus \top) $.
Since $ \psi \UPderiv (y \equiv \parity{m}) $ and $ \psi \wedge
\getrans{\xorclauses} \UPderiv (y \equiv \parity{m} \oplus \top) $, it follows
that $ \psi \UPderiv (\bot \equiv \top) $.
\end{proof}

\begin{lemma} 
\label{Lem:PropTableModels}
The satisfying truth assignments of $\xorclauses$ are exactly the
ones of $ \xorclauses \wedge \PropTable{Y, \xorclauses, k}$ when projected to $
\VarsOf{\xorclauses}$ where $ Y \subseteq \VarsOf{\xorclauses}$.
\end{lemma}

\begin{proof}
It holds by definition that $ \xorclauses \wedge \PropTable{Y, \xorclauses,k} \Models
\xorclauses$, so it suffices to show that if $\TA$ is a satisfying truth
assignment for $\xorclauses$, it can be extended to a satisfying truth
assignment $\TA'$ for $\PropTable{Y, \xorclauses, k}$. 
Assume that $\TA$ is a truth assignment such that $\TA \Models \xorclauses$.
Let $\TA'$ be a truth assignment identical to $\TA$ except for the following additions. The translation $\PropTable{Y,\phi,k}$ in Figure~\ref{Fig:PropTable} adds four kinds of xor-constraints.
\begin{enumerate}
\item $ (y \oplus Y' \equiv \bot) $ where $ Y' $ is a non-empty subset of $Y$
and $y$ is a new variable. If $ \tau \Models (Y' \equiv \top) $, add $y$ to $\tau'$, otherwise add $\neg y$ to $\tau'$. It is clear that $ \tau' \Models (y \oplus Y' \equiv \bot) $.

\item $ (a_1 \oplus a_2 \oplus a_3 \equiv \parity{1} \oplus \parity{2} \oplus
\parity{3}) $ if the xor-constraints $ (a_1 \oplus Y_1 \equiv \parity{1})
$, $ (a_2 \oplus Y_2 \equiv \parity{2})$, and $ (a_3 \oplus (Y_1 \oplus Y_2)
\equiv \parity{3}) $ are in $\phi$ augmented with xor-constraints from the
previous step.
From the previous step it is clear that $ \tau' \Models (a_1 \oplus Y_1 \equiv
\parity{1}) $, $\tau' \Models (a_2 \oplus Y_2 \equiv \parity{2}) $, and
$\tau' \Models (a_3 \oplus Y_3 \equiv \parity{3}) $. It follows that $\tau'
\Models (a_1 \oplus a_2 \oplus a_3 \equiv \parity{1} \oplus \parity{2} \oplus
\parity{3})$.

\item $ (y \equiv \parity{}') $ where $ y \in \VarsOf{\phi} $ such that
the xor-constraints $ (y \oplus Y' \equiv \parity{} \oplus \parity{}') $ and $ (Y' \equiv \parity{}) $ are in $\xorclauses$ augmented with xor-constraints from the previous step. Since $ (Y' \equiv \parity{}) $ is an original xor-constraint in $\phi$,
    it holds that $\tau' \Models (Y' \equiv \parity{}) $. It follows that $\tau' \Models (y \equiv \parity{}') $.

\item $ (y \oplus y' \equiv \parity{} \oplus \parity{}') $ where $ y, y' \in \VarsOf{\phi} $ and $ \parity{}, \parity{}' \in \Set{\top,\bot}$ such that
the xor-constraints $ (y \oplus Y' \equiv \parity{}) $ and $ (y' \oplus Y' \equiv \parity{}') $ where $Y'$ is a non-empty subset of $Y$
are in $\xorclauses$ augmented with xor-constraints from the previous step. If $ \tau' \Models (Y' \equiv \parity{} \oplus \top) $, then add $ y $ to $ \tau'$, otherwise add $ \neg y $ to $\tau'$. If $ \tau' \Models (Y' \equiv \parity'{} \oplus \top)$, then add $ y' $ to $ \tau'$, otherwise add $ \neg y' $ to $\tau'$.
It follows that $ \tau' \Models (y \oplus y' \equiv \parity{} \oplus \parity{}')$.
\end{enumerate}
\end{proof}

\begin{retheorem}{\ref{Thm:GESimulation}} 
If $\xorclauses$ is an xor-constraint
conjunction, then $\getrans{\xorclauses}$ is a GE-simulation formula for
$\xorclauses$ provided that $k=|\VarsOf{\xorclauses}|$.
\end{retheorem}

\begin{proof}
We first prove that the satisfying truth assignments of $\xorclauses$ are
exactly the ones of $\psi = \xorclauses \wedge \getrans{\xorclauses}$ when
projected to $\VarsOf{\xorclauses}$. 
The translation \getransname{} in Figure~\ref{Fig:GETrans} only adds xor-constraint
conjunctions of the type $\PropTable{Y, \phi, k} $ for some set of variables $
Y \subseteq \VarsOf{\xorclauses}$ and some xor-constraint conjunction $\phi$
and by Lemma~\ref{Lem:PropTableModels} the satisfying assignments
of $ \phi$ are exactly the ones of $ \phi \wedge \PropTable{Y, \phi, k}$ 
when projected to $ \VarsOf{\phi} $.
It follows by induction that the satisfying truth assignment for $\xorclauses$
are exactly to the ones of $ \xorclauses \wedge 
\getrans{\xorclauses}$ when projected to $ \VarsOf{\xorclauses}$.

Next we show that if $\xorclauses$ is satisfiable 
and 
 $\xorclauses \wedge \AL_1 \wedge \dots \wedge \AL_k \Models \IL$, then $\IL$ is $\UP{}$-derivable from $ \xorclauses \wedge \getrans{\xorclauses} \wedge \AL_1 \wedge \dots \wedge \AL_k$. 
By Lemma~\ref{Lemma:LinearCombs}, there is a subset $S$ of xor-constraints in $\phi \wedge \AL_1 \wedge \dots \wedge \AL_k $ such that $ \SetLC{S} = \IL $,
By Lemma~\ref{Lem:GEPropagation}, it holds that 
$\SetLC{S} \UPderiv \IL $, so $ \psi \wedge \AL_1 \wedge \dots \wedge \AL_k \UPderiv \IL $.

It remains to show that if $\xorclauses$ is unsatisfiable,
   then $\psi \UPderiv (\bot \equiv \top)$. Assume that 
   $\xorclauses$ is unsatisfiable. By Lemma~\ref{Lem:GEUnsat}, it holds
   that $\psi \UPderiv (\bot \equiv \top)$.
All the requirements for GE-simulation formula are satisfied, so $\getrans{\xorclauses}$ is a GE-simulation formula for $\xorclauses$.
\end{proof}

\subsection{Proof of Theorem~\ref{Thm:XorSimp}}
\begin{retheorem}{\ref{Thm:XorSimp}}
If $ \Tuple{\phi_a', \phi_b'} $ is the result of applying
one of the simplification rules to $ \Tuple{\phi_a, \phi_b} $ 
and 
$\phi_a \wedge \phi_b \wedge \AL_1 \wedge \dots \wedge
\AL_k \UPderiv \IL $, then $\phi_a' \wedge \phi_b' \wedge
\AL_1 \wedge \dots \wedge \AL_k \UPderiv \IL $.  
\end{retheorem}

\begin{proof}
Let $ \Tuple{\phi_a', \phi_b'} $ be the result of applying one of the
simplification rules to $ \Tuple{\phi_a, \phi_b}$, $ \phi_a \wedge \phi_b
\wedge \AL_1 \wedge \dots \wedge \AL_k \UPderiv \IL $, and $ \IL = (x \equiv \parity{})$.
If S1 was the simplification rule used, then it clearly
holds that $ \phi_a' \wedge \phi_b' \wedge \AL_1 \wedge \dots \wedge \AL_k \UPderiv \IL $.

Otherwise, S2 was used to simplify an xor-constraint $ \XC$ in $\phi_a $ with
an xor-constraint $ \XC' $ in $ \phi_b $ such that $ |\VarsOf{\XC} \cap \VarsOf{\XC'}|
\geq |\VarsOf{\XC'}| - 1$.
It holds that $ \phi_a' = \phi_a \setminus \Set{\XC} \cup \Set{\XC \LC \XC'} $
and $ \phi_b' = \phi_b $.
It must hold that there is an xor-clause $ C = (x \oplus y_1 \oplus \dots
\oplus y_n \equiv \parity{} \oplus \parity{1} \oplus \dots \oplus
\parity{n}) $ in $ \psi $ such that for each $y_i \in \Set{y_1, \dots,
y_n} $ it holds that $ \psi \UPderiv (y_i \equiv \parity{i}) $.
We prove by induction $ \IL $ is $\UP$-derivable from $ \psi' = \phi_a' \wedge
\phi_b' \wedge \AL_1 \wedge \dots \wedge \AL_k $.
The induction hypothesis is that for each $y_i \in \Set{y_1, \dots, y_n} $ it
holds that $ \psi \UPderiv (y_i \equiv \parity{i}') $.

Base case: $C = \IL = (x \equiv \parity{})$. If $ C \not = \XC $, then $ (x \equiv \parity{}) $ is in $ \psi' $ and $ \psi' \UPderiv \IL $. Otherwise, $ C = \XC $.
Since $ |\VarsOf{\XC'} \cap \VarsOf{\XC}| \geq |\VarsOf{\XC'}| - 1$, it holds that
$\VarsOf{\XC \LC \XC'} = \Set{x'} $ for some $x' \in \VarsOf{\psi} $
$ \VarsOf{\XC'} = \Set{x, x'} $.
The xor-constraint $ \XC \LC \XC'$ is in $ \phi_a' $, and the
xor-constraint $\XC'$ $\phi_b' $.
It clearly holds that $ \psi' \UPderiv \IL $.

Induction step: $C \not = \IL$. If $ C \not = \XC $, then $ C $ is in $ \psi'$
and $ \psi' \UPderiv \IL$. Otherwise $ C = \XC $.
We have two cases to consider:
\begin{itemize}
\item Case 1: $ x \in \VarsOf{\XC'} $.
By induction hypothesis it holds for each $ y_i \in \Set{y_1, \dots, y_n} $
that $ \psi' \UPderiv (y_i \equiv \parity{i}) $.
If there is a variable $z \in \VarsOf{\XC'}$ such that $ z \not \in
\VarsOf{\XC} $, then $ (\XC \LC \XC') \wedge (y_1 \equiv \parity{1}) \wedge \dots \wedge (y_n \equiv \parity{n}) \UPderiv (z \equiv \parity{}') $ and then
$ \XC \wedge (y_1 \equiv \parity{1}) \wedge \dots \wedge (y_n \equiv \parity{n}) \wedge (z \equiv \parity{}') \UPderiv (x \equiv \parity{}) $, so $ \psi' \UPderiv \IL $.
Otherwise, it holds $ \VarsOf{\XC'} \subseteq \VarsOf{\XC} $, and it clearly holds that $ \psi' \UPderiv \IL $.

\item Case 2: $ x \not \in \VarsOf{\XC'} $.
By induction hypothesis it holds for each $ y_i \in \Set{y_1, \dots, y_n} $
that $ \psi' \UPderiv (y_i \equiv \parity{i}) $.
If there is a variable $z \in \VarsOf{\XC'}$ such that $ z \not \in
\VarsOf{\XC} $, then $ \XC' \wedge (y_1 \equiv \parity{1}) \wedge \dots \wedge (y_n \equiv \parity{n}) \UPderiv (z \equiv \parity{}') $ and then
$ (\XC \LC \XC') \wedge (y_1 \equiv \parity{1}) \wedge \dots \wedge (y_n \equiv \parity{n}) \wedge (z \equiv \parity{}') \UPderiv (x \equiv \parity{}) $, so $ \psi' \UPderiv \IL $.
Otherwise, it holds $ \VarsOf{\XC'} \subseteq \VarsOf{\XC} $, and it clearly holds that $ \psi' \UPderiv \IL $.
\end{itemize}
\end{proof}

\subsection{Proof of Theorem~\ref{Thm:XCut}}
\newcommand{\CutAp}{V'_\textup{a}}
\newcommand{\CutBp}{V'_\textup{b}}
\newcommand{\xorpartA}{\xorpart^\textup{a}}
\newcommand{\xorpartB}{\xorpart^\textup{b}}
\newcommand{\Iface}{X'}
\newcommand{\Ifaceparity}{\parity{}'}


\begin{retheorem}{\ref{Thm:XCut}}
Let $ (\CutA, \CutB) $ be an $\Vars$-cut partition of $\xorclauses$.
Let $ \xorclausesA = \bigwedge_{D \in \CutA} D$, $\xorclausesB = \bigwedge_{D \in \CutB} D$, and $\AL_1,\dots,\AL_k \in \LitsOf{\xorclauses}$.
 Then
it holds that:
\begin{itemize}
\item If $\xorclauses \wedge \AL_1 \wedge \dots \wedge \AL_k $ is unsatisfiable, then
\begin{enumerate}
\item $ \xorclausesA \wedge \AL_1 \wedge \dots \wedge \AL_k $ 
or $ \xorclausesB \wedge \AL_1 \wedge \dots \wedge \AL_k $ is unsatisfiable; or 
\item $\xorclausesA \wedge \AL_1 \wedge \dots \wedge \AL_k \Models 
(X' \equiv \parity{}')$ and $ \xorclausesB \wedge \AL_1 \wedge \dots
\AL_k \Models (X' \equiv \parity{}' \oplus \top) $ for some $ X' \subseteq X $
and $ \parity{}' \in \set{\top, \bot} $.
\end{enumerate}
\item 
If $\xorclauses \wedge \AL_1 \wedge \dots \wedge \AL_k $ is satisfiable
and $ \xorclauses \wedge \AL_1 \wedge \dots \wedge \AL_k \Models (Y \equiv \parity{}) $ for some $Y \subseteq \VarsOf{\xorclauses^\alpha}$, $Y \cap (\VarsOf{\xorclauses^\beta} \setminus \VarsOf{\xorclauses^\alpha}) = \emptyset$, and $\parity{} \in \Set{\top, \bot}$ where $\alpha \in \Set{\textup{a},\textup{b}}$ and $ \beta \in \Set{\textup{a}, \textup{b}} \setminus \Set{\alpha} $, then
\begin{enumerate}
\item $\xorclausesA \wedge \AL_1 \wedge \dots \wedge \AL_k \Models (Y \equiv \parity{}) $
or $ \xorclausesB \wedge \AL_1 \wedge \dots \wedge \AL_k \Models (Y \equiv \parity{}) $;
or
\item $\xorclauses^\alpha \wedge \AL_1 \wedge \dots \wedge \AL_k \Models 
( X' \equiv \parity{}') $ and $ \xorclauses^\beta \wedge \AL_1 \wedge \dots \wedge \AL_k \wedge (X' \equiv \parity{}') \Models (Y \equiv \parity{}) $ for some $X' \subseteq X$,
$\parity{}' \in \set{\top,\bot}$, $\alpha \in \Set{\textup{a},\textup{b}}$, and $\beta \in \Set{\textup{a},\textup{b}}\setminus\Set{\alpha}$.
\end{enumerate}
\end{itemize}
\end{retheorem}
\begin{proof}
  Let $(\CutAp,\CutBp)$ be an $\Vars$-cut partition of
  $\xorpart \land (\AL_1) \land ... \land (\AL_k)$
  with
  $\VarsOf{\CutAp} = \VarsOf{\CutA}$,
  $\VarsOf{\CutBp} = \VarsOf{\CutB}$,
  $\CutA \subseteq \CutAp$, and
  $\CutB\subseteq \CutBp$.
Such partition exists because the xor-assumption literals $\AL_i$
  are unit xor-constraints.

  \begin{itemize}
\item  Case I: $\xorpart \land {\AL_1 \land ... \land \AL_k}$ is unsatisfiable.
By Lemma~\ref{Lemma:LinearCombs},
  there is a subset $S$ of xor-constraints
  in $\xorpart \land (\AL_1) \land ... \land (\AL_k)$
  such that $\BigLinComb_{\XC \in S} \XC = (\F \equiv \T)$.
Observe that
  $\BigLinComb_{\XC \in S} \XC =
   (\BigLinComb_{\XC \in {\CutAp \cap S}} \XC) \LinComb
   (\BigLinComb_{\XC \in {\CutBp \cap S}} \XC)$.
If $\BigLinComb_{\XC \in {\CutAp \cap S}} \XC = (\F \equiv \T)$,
  then
  $\xorpartA \land {\AL_1 \land ... \land \AL_k}$ is also unsatisfiable.
Similarly,
  if $\BigLinComb_{\XC \in {\CutBp \cap S}} \XC = (\F \equiv \T)$,
  then $\xorpartB \land {\AL_1 \land ... \land \AL_k}$ is unsatisfiable.
Otherwise,
  it must be that
  $\BigLinComb_{\XC \in {\CutAp \cap S}} \XC = (\Iface \equiv \Ifaceparity)$
  and
  $\BigLinComb_{\XC \in {\CutBp \cap S}} \XC = (\Iface \equiv \Ifaceparity \X \T)$
  with $\Ifaceparity \in \Set{\F,\T}$
  because
  $\CutAp \cap \CutBp = \emptyset$,
  ${\VarsOf{\CutAp} \cap \VarsOf{\CutBp}} = \Iface$ and
  $(\BigLinComb_{\XC \in {\CutAp \cap S}} \XC) \LinComb
   (\BigLinComb_{\XC \in {\CutBp \cap S}} \XC) = (\F \equiv \T)$.
Thus
  $\xorpartA \land {\AL_1 \land ... \land \AL_k} \Models (\Iface \equiv \Ifaceparity)$ and
  $\xorpartB \land {\AL_1 \land ... \land \AL_k} \Models (\Iface \equiv \Ifaceparity \X \T)$.
  
 \item Case II:
  $\xorpart \land \AL_1 \land ... \land \AL_k$ is satisfiable
  and
  $\xorpart \land \AL_1 \land ... \land \AL_k \Models (Y \equiv \parity{})$.
By Lemma~\ref{Lemma:LinearCombs},
 there is a subset $S$ of xor-constraints
  in $\xorpart \land (\AL_1) \land ... \land (\AL_k)$
  such that ${\BigLinComb_{\XC \in S} \XC} = (Y \equiv \parity{})$.
Again,
  observe that
  $(\BigLinComb_{\XC \in S} \XC) = 
   (\BigLinComb_{\XC \in {\CutAp \cap S}} \XC) \LinComb
   (\BigLinComb_{\XC \in {\CutBp \cap S}} \XC)$.
Assume that $Y \subseteq \VarsOf{\xorclausesB}$ and $ Y \cap (\VarsOf{\xorclausesA} \setminus \VarsOf{\xorclausesB}) = \emptyset $; the other case is symmetric.
Then we simplify the equation
$(\BigLinComb_{\XC \in S} \XC) = 
(\BigLinComb_{\XC \in {\CutAp \cap S}} \XC) \LinComb
(\BigLinComb_{\XC \in {\CutBp \cap S}} \XC)$ 
by (i) substituting $(\BigLinComb_{\XC \in S} \XC)$ with $ (Y \equiv \parity{})$ and (ii) evaluating $ (\BigLinComb_{\XC \in {\CutAp \cap S}} \XC) $.
This gives two cases:
\begin{enumerate}
\item evaluating $ (\BigLinComb_{\XC \in {\CutAp \cap S}} \XC) $ gives
an empty expression and the simplified equation is then $ (Y \equiv \parity{}) = (\BigLinComb_{\XC \in {\CutBp \cap S}} \XC)$, so it follows that
  $\xorpartB \land \AL_1 \land ... \land \AL_k \Models (Y \equiv \parity{})$.

\item evaluating $ (\BigLinComb_{\XC \in {\CutAp \cap S}} \XC) $ gives
an xor-constraint $ (\Iface \equiv \Ifaceparity{}) $ for some $ \Iface \subseteq X $ and $\Ifaceparity{} \in \Set{\top, \bot}$ because 
  $\CutAp \cap \CutBp = \emptyset$,
  ${\VarsOf{\CutAp} \cap \VarsOf{\CutBp}} = \Iface$ and
  $(\BigLinComb_{\XC \in {\CutAp \cap S}} \XC) \LinComb
   (\BigLinComb_{\XC \in {\CutBp \cap S}} \XC) = (Y \equiv \parity{})$.
The simplified equation is then $ (Y \equiv \parity{}) = (\Iface \equiv \Ifaceparity{}) \LinComb (\BigLinComb_{\XC \in {\CutBp \cap S}} \XC)$,
so it follows that
  $\xorpartB \land {\AL_1 \land ... \land \AL_k \land (\Iface \equiv \Ifaceparity)} \Models (Y \equiv \parity{})$.
\end{enumerate}
\end{itemize}
\end{proof}

\subsection{Proof of Theorem~\ref{Thm:TreeDecomposition}}

\begin{lemma}
\label{Lem:PCProp}
If $\phi$ is a satisfiable conjunction in $\xorclauses \wedge \psi$ such that
$\VarsOf{\phi} \subseteq Y$, $Y \subseteq \VarsOf{\xorclauses}$, $\HasPropTable{Y}{\xorclauses \wedge \psi}$,
and $ \phi \wedge (Y_1 \equiv \parity{1}) \wedge \dots \wedge (Y_n \equiv \parity{n}) \Models (Y' \equiv \parity{}')$
where $Y_1,\dots,Y_n,Y' \subseteq Y $
and $\parity{1},\dots,\parity{n},\parity{}' \in \Set{\top,\bot}$, 
then $\xorclauses \wedge \psi \wedge a_1 \equiv \parity{1} \wedge \dots \wedge a_n \equiv
\parity{n} \UPderiv a' \equiv \parity{}'$
where $ a_1,\dots,a_n,a' $ are the ``alias'' variables for the sets $Y_1,\dots,Y_n,Y'$, respectively.
\end{lemma}

\begin{proof}
By Lemma~\ref{Lemma:LinearCombs}, there is a subset $\phi'$ of xor-constraints in $\phi \wedge (Y_1 \equiv \parity{1}) \wedge \dots \wedge (Y_n \equiv \parity{n}) $ such that $ \SetLC{\phi'} = (Y' \equiv \parity{}') $.
By the property PT1, it holds for each xor-constraint $ (Y'' \equiv \parity{}'') $ in $\phi'$ 
that that the corresponding ``alias'' variable $ a'' $ for the set of
variables $Y''$ is present in $\VarsOf{\phi}$
and by the property PT3 the xor-constraint $ (a'' \equiv
    \parity{}'') $ is in $ \xorclauses \wedge \psi$.
It holds by Lemma~\ref{Lem:PTableProp}, that $ \xorclauses \wedge \psi \wedge (a_1 \wedge \parity{1}) \wedge \dots \wedge (a_n \equiv \parity{n}) \UPderiv
(a' \equiv \parity{}') $.
\end{proof}


\begin{retheorem}{\ref{Thm:TreeDecomposition}}
If $\Set{X_1, \dots, X_n} $ is the family of variable sets in the tree
decomposition of the primal graph of an xor-constraint conjunction $
\xorclauses$ and 
$\phi_0, \dots, \phi_n$ is a sequence of xor-constraint conjunctions
such that $ \phi_0 = \xorclauses $ and $ \phi_i = \phi_{i-1} \wedge \PropTable{X_i, \phi_{i-1}, |X_i|} $ for $i \in \Set{1,\dots,n}$, then $ \phi_n \setminus \xorclauses $ is a GE-simulation formula for
$\xorclauses$ with $O(n {2^{2k}}) + |\xorclauses|$ xor-constraints, where
$k = \max(|X_1|, \dots, |X_n|) $.  
\end{retheorem}

\begin{figure}[ht]
\centering
\begin{tabular}{c@{\qquad}c}
\includegraphics[scale=0.4]{Figures/tw_ex}
&
\includegraphics[scale=0.4]{Figures/tw_ex2}
\\
(a) a constraint graph
&
(b) subgraph of the constraint graph
\\
\\
\end{tabular}
\caption{(a) A constraint graph for an instance $\xorclauses$, (b) subgraph of the constraint graph illustrating that $\xorclauses \Models x_1 \equiv \top$}
\label{Fig:TWEx}
\end{figure}

\begin{figure}[ht]
\centering
\includegraphics[width=0.5\textwidth]{Figures/tw_primal}
\caption{Primal graph for the instance whose constraint graph is shown in Fig.~\ref{Fig:TWEx}(a)}
\label{Fig:TWExp}
\end{figure}

\begin{figure}[ht]
\centering
\includegraphics[width=0.5\textwidth]{Figures/tw_tree}
\caption{Tree decomposition of the primal graph in Fig.~\ref{Fig:TWExp}. Assume
that $X_1 = \Set{x_1,x_2,x_3,x_4,x_5},
     X_2 = \Set{x_3,x_4,x_5,x_9},
     X_3 = \Set{x_2,x_4,x_5,x_8},
     X_4 = \Set{x_2,x_3,x_5,x_7},
     X_5 = \Set{x_2,x_3,x_4,x_5,x_6},
     X_6 = \Set{x_4,x_5,x_6,x_{10},x_{11}},  \phi_0=\xorclauses$, $\phi_1 = \phi_0 \wedge
\PropTable{X_1, \phi_0, |X_1|}$, $ \phi_2 = \phi_1
\wedge \PropTable{X_2, \phi_1, |X_2|}$, $\phi_3 = \phi_2
\wedge \PropTable{X_3, \phi_2, |X_3| }$, $ \phi_4 =
\phi_3 \wedge \PropTable{X_4, \phi_3, |X_4|}$, $\phi_5
= \phi_4 \wedge \PropTable{X_5, \phi_4, |X_5|}$, and
$\phi_6 = \phi_5 \wedge \PropTable{X_6, 
    \phi_5, |X_6|} $, and $ \psi = \phi_6 \setminus \xorclauses$. It holds that
        $\HasPropTable{X_1}{\xorclauses \wedge \psi}, \dots,
    \HasPropTable{X_6}{\xorclauses \wedge \psi}$. The \UP{} system can deduce $
        (x_1 \equiv \top)$, i.e. $ \xorclauses \wedge \psi \UPderiv (x_1 \equiv
                \top)$ by ``propagating'' intermediate linear combinations
        starting from the leaves of the tree decomposition towards the root
        node (the node with the set of variables $X_1$). Since $x_6 \equiv \top$ is in
        $\xorclauses$ it holds that $ \xorclauses \wedge \psi \UPderiv
        a_{2,3,4} \equiv \bot$. And in a similar way because $ x_7 \equiv
        \top$, $x_8\equiv\top$,and $x_9 \equiv\top$ are in $\xorclauses$, then
        $ \xorclauses \wedge \psi \UPderiv a_{2,3,5} \equiv \bot $,
    $\xorclauses \wedge \psi \UPderiv a_{2,4,5} \equiv \bot$, and $\xorclauses
        \wedge \psi \UPderiv a_{3,4,5} \equiv \bot$. By combining these
        intermediate results, it holds that $ \xorclauses \wedge \psi \UPderiv
        a_{2,3,4,5} \equiv \bot$ and finally $ \xorclauses \wedge \psi \UPderiv
        x_1 \equiv \top$}
        \label{Fig:TWTree}
\end{figure}






\begin{proof}
The construction is illustrated in Figures~\ref{Fig:TWEx}, \ref{Fig:TWExp}, \ref{Fig:TWTree}.

Let $ \psi = \phi_n \setminus \xorclauses $.
We first prove that the satisfying truth assignments of $\xorclauses$ are
exactly the ones of $\xorclauses \wedge \psi$ when projected to
$\VarsOf{\xorclauses}$. 
By Lemma~\ref{Lem:PropTableModels}, the satisfying truth assignments of $ \phi$ are exactly the ones of $ \phi \wedge \PropTable{Y, \phi, k} $ when projected
to $\VarsOf{\phi}$, so by induction the satisfying truth assignments of $
\xorclauses$ are exactly the ones of $ \xorclauses \wedge \psi$ when projected
to $ \VarsOf{\xorclauses}$.
The number of xor-constraints in $ \PropTable{Y, \phi, k} $ is $O(2^{2k}) +
|\phi|$, so the number of xor-constraints in $ \psi $ is $ O(n 2^{2k}) +
|\xorclauses| $.

It holds for each $ X_i \in \Set{X_1, \dots, X_n} $ by
Lemma~\ref{Lem:PropTableImpl} that $ \HasPropTable{X_i}{\xorclauses \wedge
\psi}$.
Next we show that if $\xorclauses$ is satisfiable and $\xorclauses \wedge \AL_1
\wedge \dots \wedge \AL_k \Models \IL$, then $\IL$ is $\UP{}$-derivable from $
\xorclauses \wedge \psi \wedge \AL_1 \wedge \dots \wedge
\AL_k$. 
Assume that $ \xorclauses$ is satisfiable and $ \xorclauses \wedge \AL_1 \wedge \dots \wedge \AL_k \Models \IL $.
\newcommand{\subtree}{\ensuremath{T'}}
We prove by induction on the structure of the tree decomposition that the
following property holds for each subtree $\subtree$ of the tree decomposition having the set of variables $X_{T'}$ and the root node of
$\subtree$ with the set of variables $X_r$:
\begin{itemize}
\item If $\phi$ is a satisfiable conjunction in $\xorclauses \wedge \psi \wedge \AL_1 \wedge \dots \wedge \AL_k $ such that
$\VarsOf{\phi} \subseteq X_{T'}$,
and $ \phi \wedge (Y_1 \equiv \parity{1}) \wedge
\dots \wedge (Y_m \equiv \parity{m}) \Models (Y' \equiv \parity{}') $ 
where $Y' \subseteq X_r $ and for each $Y_j \in \Set{Y_1,\dots,Y_n}$ there is a
$k \in \Set{1,\dots,n}$ for which it holds that $Y_j \subseteq X_k $
and
$\parity{1},\dots,\parity{m},\parity{}' \in \Set{\top,\bot}$, 
then $\xorclauses \wedge \psi \wedge (a_1 \equiv \parity{1}) \wedge \dots \wedge (a_n \equiv
        \parity{m}) \UPderiv (a' \equiv \parity{}')$
where $a_1,\dots,a_n,a'$ are the ``alias'' variables for the variable sets
$Y_1,\dots,Y_n,Y'$, respectively.
\end{itemize}
The induction hypothesis is that the property holds for each proper subtree of
$T'$.

Base case: $T'$ has only one node. The property holds by Lemma~\ref{Lem:PCProp}.


Induction step: $T'$ has more than one node.
Let $\phi' = \phi \wedge (Y_1 \equiv \parity{1}) \wedge \dots \wedge (Y_m \equiv \parity{m}) $. 
The idea is to remove xor-constraints involving variables other than in
$X_r$ from $\phi'$ and add additional xor-constraints of the type involving variables only in $X_r$.
This is done by considering each direct child node of the root node of $T'$ at
a time possibly rewriting $\phi'$ by substituting a sub-conjunction of $\phi'$ with at most one xor-constraint having only variables in $X_r$.
Let $T''$ be the subtree induced by one direct child node of the root node
having the set of variables $X_{T''}$. The per-child substitution operation of $\phi'$ is defined as follows.
\newcommand{\phiA}{\phi^{\textup{a}}}
\newcommand{\phiB}{\phi^{\textup{b}}}
Let $\phiA $ be the maximal conjunction of xor-constraints in $\phi'$ such that
$\VarsOf{\phiA} \subseteq X_{T''}$, and $\phiB$ be the conjunction of
xor-constraints in $\phi'$ but not in $\phiA$.
If $\phiA$ is empty, then nothing needs to be removed from $\phi'$.
Otherwise, $\phiA$ is non-empty and there is an $\Vars$-cut partition
$(\CutA, \CutB)$ of $\phi'$ such that $ \xorclausesA = \bigwedge_{D \in \CutA} D$, $\xorclausesB = \bigwedge_{D \in \CutB} D$ and $\VarsOf{\phi^a} \cap \VarsOf{\phi^b} = X \subseteq X_r \cap X_c $.
By Theorem~\ref{Thm:XCut}, it holds that
\begin{enumerate}
\item $\phiA \Models (Y' \equiv \parity{}') $
or $ \phiB \Models (Y' \equiv \parity{}') $;
or
\item $\phi^\alpha \Models 
( X'' \equiv \parity{}'') $ and $ \phi^\beta \wedge (X'' \equiv \parity{}'') \Models (Y' \equiv \parity{}') $ for some $X'' \subseteq X$,
$\parity{}' \in \set{\top,\bot}$, $\alpha \in \Set{\textup{a},\textup{b}}$, and $\beta \in \Set{\textup{a},\textup{b}}\setminus\Set{\alpha}$.
\end{enumerate}

We analyze the cases:
\begin{itemize}
\item[] Case 1: $\phiA \Models (Y' \equiv \parity{}') $
or $ \phiB \Models (Y' \equiv \parity{}') $. Since $Y' \subseteq X_r$, it must be that $\phiB \Models (Y' \equiv \parity{}') $. In this case, set $\phi' \leftarrow \phiB $.
\item[] Case 2: $\phi^\alpha \Models 
( X'' \equiv \parity{}'') $ and $ \phi^\beta \wedge (X'' \equiv \parity{}'') \Models (Y' \equiv \parity{}') $ for some $X'' \subseteq X$,
$\parity{}' \in \set{\top,\bot}$, $\alpha \in \Set{\textup{a},\textup{b}}$, and $\beta \in \Set{\textup{a},\textup{b}}\setminus\Set{\alpha}$. Again since $Y' \subseteq X_r$, it must be that $\alpha = \textup{a}$. In this case, set $ \phi' \leftarrow \phiB \wedge (X'' \equiv \parity{}'') $.
\end{itemize} 

After each child node has been processed in this way, it holds that $\phi'
\Models (Y \equiv \parity{}) $ and $\VarsOf{\phi'} \subseteq X_r $.
It also holds by the induction hypothesis for each xor-constraint $ (X_i'' \equiv \parity{i}'') $ in the
sequence $ (X_1'' \equiv \parity{1}''), \dots, (X_q'' \equiv \parity{q}'') $ of
added xor-constraints that the corresponding ``alias'' variables $a_1'', \dots,
      a_q''$ for $X_1'', \dots, X_q''$, respectively, that,
      since $ \phi \wedge (Y_1 \equiv \parity{1}) \wedge \dots \wedge (Y_m \equiv \parity{m}) \wedge (X_{1}'' \equiv \parity{1}'') \wedge \dots \wedge (X_{i-1}'' \equiv \parity{i-1}'') \Models (X_{i}'' \equiv \parity{i}'')$, then it holds
      that $\xorclauses \wedge \psi \wedge \AL_1 \wedge \dots \wedge \AL_k \wedge (a_1 \equiv \parity{1}) \wedge \dots \wedge (a_m \equiv \parity{m}) \wedge (a_1'' \equiv \parity{1}'') \wedge \dots \wedge (a_{i-1}'' \equiv \parity{i-1}'') \UPderiv a_i \equiv \parity{i}''$. 
Now $\phi' \Models (Y \equiv \parity{}) $, $\VarsOf{\phi'} \subseteq X_r$,
and each xor-constraint $ (X'' \equiv \parity{}'') $ in $\phi'$ has its corresponding ``alias'' variable $ a'' $ implied by unit propagation, that is, $\xorclauses \wedge \psi \wedge \AL_1 \wedge \dots \wedge \AL_k \UPderiv (a' \equiv \parity{}') $.
By induction it follows that $\xorclauses \wedge \psi \wedge \AL_1 \wedge \dots \wedge
\AL_k \UPderiv \IL $.

It remains to show that if $\xorclauses$ is unsatisfiable,
then $\psi \UPderiv (\bot \equiv \top)$. Assume that 
$\xorclauses$ is unsatisfiable. 
By Lemma~\ref{Lemma:LinearCombs}, there is a minimal subset $S$ of
xor-constraints in $ \xorclauses \wedge \AL_1 \wedge \dots \wedge \AL_k $
such that $ \SetLC{S} = (\bot \equiv \top) $. 
Now, let $ S' = S \setminus \XC $
be a subset of $S$ identical to $S$ except that one xor-constraint $\XC = (X' \equiv \parity{}) $ in $S$ is removed.
It clearly holds that $S'$ is satisfiable and $ \SetLC{S'} = (X' \equiv \parity{} \oplus \top) $. 
There is a node in $T'$ that has the variables $X_i$ such that $ X' \subseteq X_i$.
It holds that $ \HasPropTable{X_i}{\xorclauses \wedge \psi}$, so by Lemma~\ref{Lem:PTableProp2} it holds for the ``alias'' variable $a' $ for $X'$ that 
$\xorclauses \wedge \psi \wedge \AL_1 \wedge \dots \wedge \AL_k \UPderiv (a' \equiv \parity{} \oplus \top) $.
Repeat the proof as above for the satisfiable case and for the subset $S'$ showing
that $ \xorclauses \wedge \psi \wedge \AL_1 \wedge \dots \wedge \AL_k \UPderiv 
(a' \equiv \parity{}) $.
It follows that $ 
\xorclauses \wedge \psi \wedge \AL_1 \wedge \dots \AL_k \UPderiv (\bot \equiv \top) $.
All the requirements for GE-simulation formula are satisfied, so 
$\psi$ is a GE-simulation formula for $\xorclauses$.
\end{proof}


