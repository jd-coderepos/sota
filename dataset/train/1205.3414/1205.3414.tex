\documentclass[]{sig-alternate}

\usepackage{amsmath,amssymb}
\usepackage{alltt,mathrsfs}
\usepackage{multirow,rotating}


\usepackage{colortbl}
\definecolor{kugray5}{RGB}{224,224,224}
 


\def\myproof{\noindent{\sc Proof.}~}
\def\foorp{\hfill}
\def\val {\ensuremath{\mathrm{val}}}
\def\ord {\ensuremath{\mathrm{ord}}}
\def\char {\ensuremath{\mathrm{char}}}

\def\partial{\delta}

\usepackage[plainpages=false,pdfpagelabels,colorlinks=true,citecolor=blue,hypertexnames=false]{hyperref}

\usepackage{comment}
\makeatletter
\newif\if@restonecol
\makeatother
\let\algorithm\relax
\let\endalgorithm\relax


\usepackage[lined,boxed,ruled]{algorithm2e}
\SetKwInOut{Input}{input}
\SetKwInOut{Output}{output}
\def\RDAC{\mathsf{RDAC}}
\def\DAC{\mathsf{DAC}}
\def\LinAlg{\mathsf{LinSolve}}
\def\Sylvester{\mathsf{Sylvester}}
\def\NewtonAE{\mathsf{NewtonAE}}
\def\DiffSylvester{\mathsf{DiffSylvester}}
\def\DiffSylvesterDifferential{\mathsf{DiffSylvesterDifferential}}
\def\PolCoeffsDE{\mathsf{PolCoeffsDE}}

\def\Spec{\operatorname{Spec}}
\def\card{\operatorname{card}}

\def\C {\ensuremath{\mathbb{C}}}
\def\Q {\ensuremath{\mathbb{Q}}}
\def\N {\ensuremath{\mathbb{N}}}
\def\R {\ensuremath{\mathbb{R}}}
\def\Z {\ensuremath{\mathbb{Z}}}
\def\F {\ensuremath{\mathbb{F}}}
\def\H {\ensuremath{\mathbb{H}}}
\def\K {\ensuremath{\mathbb{K}}}
\def\L {\ensuremath{\mathbb{L}}}
\def\A {\ensuremath{\mathbb{A}}}
\def\M {\ensuremath{\mathsf{M}}}
\def\I {\ensuremath{\int_q}}
\def\Id{\ensuremath{\mathsf{Id}}}
\def\GLn{\operatorname{GL}_n}

\def\mA {\ensuremath{{A}}}
\def\mB {\ensuremath{{B}}}
\def\mC {\ensuremath{{C}}}
\def\mD {\ensuremath{{D}}}
\def\mE {\ensuremath{{E}}}
\def\mF {\ensuremath{{F}}}
\def\mG {\ensuremath{{G}}}
\def\mH {\ensuremath{{H}}}
\def\mI {\ensuremath{{1}}}
\def\mL {\ensuremath{{L}}}
\def\mP {\ensuremath{{P}}}
\def\mQ {\ensuremath{{Q}}}
\def\mR {\ensuremath{{R}}}
\def\mS {\ensuremath{{S}}}
\def\mT {\ensuremath{{T}}}
\def\mU {\ensuremath{{U}}}
\def\mV {\ensuremath{{V}}}
\def\mW {\ensuremath{{W}}}
\def\mY {\ensuremath{{Y}}}
\def\mZ {\ensuremath{{Z}}}
\def\mzero {\ensuremath{{0}}}

\newtheorem{Def}{Definition}
\newtheorem{Coro}{Corollary}
\newtheorem{Theo}{Theorem}
\newtheorem{Prop}{Proposition}
\newtheorem{Lemma}{Lemma}

\def\gathen#1{{#1}}
\def\hoeven#1{{#1}}


\def\TODO#1{{\bf TO BE DONE: #1!}}

\newcommand{\tmop}[1]{\ensuremath{\operatorname{#1}}}
\newcommand{\assign}{:=}

\begin{document}

\def\more-auths{\end{tabular}
\begin{tabular}{c}}
\title{Power Series Solutions of\\ Singular (q)-Differential Equations}
\numberofauthors{3}
\author{
\alignauthor Alin Bostan\2mm]
	\affaddr{Algorithms Project}\\
	\affaddr{Inria (France)}\\
	\email{Alin.Bostan@inria.fr}\\
	\email{Bruno.Salvy@inria.fr}
\alignauthor Muhammad F. I. Chowdhury\2mm]
	\affaddr{ORCCA and CS Department}\\
	\affaddr{University of Western Ontario}\\
	\affaddr{London, ON (Canada)}\\
        \email{mchowdh3@csd.uwo.ca}\\
	\email{eschost@uwo.ca}
\alignauthor Romain Lebreton\\label{eq:maindiff}
x^k \partial(\mF) = \mA \mF + \mC,
f=1+\frac{ab}{c}x+\frac{a(a+1)b(b+1)}{c(c+1)}\frac{x^2}{2!}+\dotsb,\label{eq:hypergeom}
x(x-1)f''+(x(a+b+1)-c)f'+abf=0
\label{eq:hypergeom2}
\frac{g(g-1)}{g'^2}F''+\frac{g'^2(g(a+b+1)-c)+(g-g^2)g''}{g'^3}F'+abF=0
\partial(fg)=f\partial(g)+\partial(f)\sigma(g).\label{eq:main}
x^k \partial(\mF) = \mA \sigma(\mF) + \mC.
\partial(x)=1,\qquad \sigma: x \mapsto qx,\partial(x^i)=\gamma_i x^{i-1}\text{ with }
\gamma_0=0\text{ and }\gamma_i=1+q+\cdots+q^{i-1}\text{ }.\partial(f)=\sum_{i \ge 1} \gamma_if_i x^{i-1}\label{eq:Ri}
\mR_i \mF_i = \Delta_i,  
\begin{cases}
\mR_i = (q^i \mA_0-\gamma_i\Id)&\quad\text{if }\\
\mR_i = q^i\mA_0&\quad\text{if }.
\end{cases}
  \label{eq:bfF}
\mathbf{F}=\varphi_0 + \varphi_1 \mathbf{F}_{j_1}+ \cdots + \varphi_r
\mathbf{F}_{j_r},
\mathbf{F}=\varphi_0 + \varphi_1 \mathbf{F}_{j_1}+ \cdots +
  \varphi_{\mu(i)} \mathbf{F}_{\mu(i)},\partial(\mathbf{F})=\partial(\varphi_0) + \partial(\varphi_1)
\mathbf{F}_{j_1}+ \cdots + \partial(\varphi_r) \mathbf{F}_{j_r}, E(\mathbf{F},\mathbf{C},i) = x^k \partial(\mathbf{F}) - \Big( (q^i
A-\gamma_i x^{k-1} \Id) \sigma(\mathbf{F}) + \mathbf{C}\Big ).x^k\partial (\mathbf{F})=x^k\partial (\mathbf{H})+x^{m+k}\partial (\mathbf{K})+\gamma_mx^{m+k-1}\sigma(\mathbf{K})\label{eq:E}
E(\mathbf{F},\mathbf{C},i) = \left( E(\mathbf{H},\mathbf{C},i) \bmod x^m \right )+ x^m E(\mathbf{K},{\mathbf{D}},i+m).
  \label{eq:Wb}
\mB=\mW^{-1}(x^k \partial(\mW)-\mA \sigma(\mW)) \bmod x^N.
\tag{\ref{eq:main}}
x^k\partial(\mF)=\mA\sigma(\mF) +\mC \bmod x^N
\label{eq:finaleq}
x^k\partial(\mY)=\mB\sigma(\mY)+ \mW^{-1}\mC  \bmod x^N.
x^k \partial(\mF) = x^k \partial(\mW) \sigma(\mY) + x^k \mW \partial(\mY).x^k \partial(\mF)-\mA \sigma(\mF) -\mC = \mW (x^k \partial(\mY) -\mB
\sigma(\mY) -\mW^{-1}\mC) \bmod x^N.\label{eq:W}
x^k \partial(\mW)=\mA \sigma(\mW)-\mW \mB \bmod x^N
x \mW' = \mA\mW-\mW \mA_0 \bmod
x^N.(\mA_0- i \Id)\mW_i -\mW_i \mA_0 = -\sum_{j < i}\mA_{i-j} \mW_j.
  \label{eq:YP}x^k \partial(Y)=P \sigma(Y)+Q,
\gamma_{i-k+1}Y_{i-k+1}=q^iP_0Y_i+\dots+q^{i-k+1}P_{k-1}Y_{i-k+1}+Q_i.B=A\bmod x^k,\quad V=\Id\label{eq:WB}
x^k \partial(\mU) = \mB \sigma(\mU)- \mU\mB + \Gamma \bmod x^N,
\gamma_{i-k+1}\mU_{i-k+1}=q^iB_0\mU_i-\mU_iB_0+C,x^k \partial(\mU^{(i,j)}) = (B^{(i,i)}-B^{(j,j)}) \mU^{(i,j)} + \Gamma^{(i,j)} \bmod x^N.
x^k\partial(H&+HU)-A\sigma(H+HU)+(H+HU)B\\
&=(x^k\partial(H)-A\sigma(H)+HB)(\Id+\sigma(U))\\
&\qquad+H(x^k\partial(U)+UB-B\sigma(U))\\
&=R(\Id+\sigma(U))-R\bmod x^N=R\sigma(U)\bmod x^N.
C(N)=C(m)+O(n^\omega\M(N)+n^\omega\log nN),\quad N>k.x^k\partial(Y)=B\sigma(Y)+\Gamma\bmod x^N.
Lemma~\ref{lemma:PolCoeffsDE} gives us generators of the solution
space of this equation . If it is
inconsistent, we infer that Eq.~\eqref{eq:main} is. Else, from
the generators  obtained in Lemma~\ref{lemma:PolCoeffsDE}, we
deduce that  is a generator of the solution space
of Eq.~\eqref{eq:main} . Since the matrix  has few
columns (at most ), the cost of all these computations is dominated
by that of Proposition~\ref{prop:W}, as reported in
Thm.~\ref{theo:2}.



\section{Implementation}

We implemented the divide-and-conquer and Newton iteration algorithms,
as well as a quadratic time algorithm, on top of
NTL~5.5.2~\cite{NTL}. In our experiments, the base field is
, with  a 28 bit prime; the systems
were drawn at random. Timings are in seconds, averaged over 50 runs;
they are obtained on a single core of a 2 GHz Intel Core 2.


Our implementation uses NTL's built-in \texttt{zz\_pX} polynomial
arithmetic, that is, works with ``small'' prime fields (of size about
 over 32 bit machines, and  over 64 bits machines).
For this data type, NTL's polynomial arithmetic uses a combination of
naive, Karatsuba and FFT arithmetic.

There is no built-in NTL type for polynomial matrices, but a simple
mechanism to write one. Our polynomial matrix product is naive, of
cubic cost. For small sizes such as  or , this is
sufficient; for larger , one should employ improved schemes (such
as Waksman's~\cite{Waksman70}, see also~\cite{DrIsSc11}) or
evaluation-interpolation techniques~\cite{BoSc05}.

Our implementation follows the descriptions given above, up to a few
optimizations for algorithm  (which are all classical
in the context of Newton iteration). For instance, the inverse of
 should not be recomputed at every step, but simply updated; some
products can be computed at a lower precision than it appears (such as
, where  is known to have a high valuation).


In Fig.~\ref{fig:1d}, we give timings for the scalar case, with 
and . Clearly, the quadratic algorithm is outperformed for
almost all values of ; Newton iteration performs better than the
divide-and-conquer approach, and both display a subquadratic behavior.
Fig.~\ref{fig:2d} gives timings when  varies, taking  and
 as before. For larger values of , the divide-and-conquer
approach become much better for this range of values of~, since it
avoids costly polynomial matrix multiplication (see
Thm.~\ref{theo:1}).

\begin{figure}[t]
\centerline{\includegraphics[width=8cm]{q___1_2_10_1000.pdf}}
\vspace{-0.5cm}
\caption{Timings with , , }
\label{fig:1d}
\centerline{\includegraphics[width=8cm]{q_1_1_5_10_1000.pdf}}
\vspace{-1cm}
\caption{Timings with , }
\label{fig:2d}
\end{figure}

Finally, Table~\ref{tab:1} gives timings obtained for , for
larger values of  (in this case, a plot of the results would be
less readable, due to the large gap between the divide-and-conquer
approach and Newton iteration, in favor of the former); DAC stands for
``divide-and-conquer''. In all cases, the experimental results
confirm to a very good extent the theoretical cost analyses.


\begin{table}[t]
  \centerline{
    \begin{tabular}{|c|c||c|c|c|c|c|}
      \hline
      \multicolumn{2}{|c|}{Newton}&\multicolumn{4}{c|}{}\\
      \cline{3-6}
      \multicolumn{2}{|c|}{}&5&9&13&17\\
      \hline
      \multirow{4}{*}{}
                          & 50 & 0.01 & 0.11 & 0.32 & 0.72\\
      \cline{2-6}         & 250 & 0.22 & 1.2 & 3.7 & 8.1\\
      \cline{2-6}         & 450 & 0.50 & 2.8 & 8.3 & 18\\
      \cline{2-6}         & 650 & 0.93 & 5.1 & 16 & 34\\
      \hline
    \end{tabular}
}

  \centerline{
    \begin{tabular}{|c|c||c|c|c|c|c|}
      \hline
      \multicolumn{2}{|c|}{DAC}&\multicolumn{4}{c|}{}\\
      \cline{3-6}
      \multicolumn{2}{|c|}{}&5&9&13&17\\
      \hline
      \multirow{4}{*}{}
                          & 50 & 0.01 & 0.01 & 0.02 & 0.04\\
      \cline{2-6}         & 250 & 0.03 & 0.07 & 0.15 & 0.25\\
      \cline{2-6}         & 450 & 0.06 & 0.16 & 0.32 & 0.52\\
      \cline{2-6}         & 650 & 0.10 & 0.27 & 0.53 & 0.88\\
      \hline
    \end{tabular}
  }
\caption{Timings with , }
\label{tab:1}
\end{table}





\scriptsize
\bibliographystyle{abbrv}
\begin{thebibliography}{10}
\vspace{0.2cm}
\bibitem{Balser00}
W.~Balser.
\newblock {\em Formal power series and linear systems of meromorphic ordinary
  differential equations}.
\newblock Universitext. Springer-Verlag, New York, 2000.

\bibitem{BaBrPf10}
M.~Barkatou, G.~Broughton, and E.~Pfl{\"u}gel.
\newblock A monomial-by-monomial method for computing regular solutions of
  systems of pseudo-linear equations.
\newblock {\em Math. Comput. Sci.}, 4(2-3):267--288, 2010.

\bibitem{BaPf99}
M.~Barkatou and E.~Pfl{\"u}gel.
\newblock An algorithm computing the regular formal solutions of a system of
  linear differential equations.
\newblock {\em J. Symb. Comput.}, 28(4-5):569--587, 1999.

\bibitem{Bernstein1998}
D.~J. Bernstein.
\newblock Composing power series over a finite ring in essentially linear time.
\newblock {\em J. Symb. Comput.}, 26(3):339--341, 1998.

\bibitem{BoChOlSaScSc07}
A.~Bostan, F.~Chyzak, F.~Ollivier, B.~Salvy, S.~Sedoglavic, and {\'E}.~Schost.
\newblock Fast computation of power series solutions of systems of differential
  equations.
\newblock In {\em Symposium on Discrete Algorithms, SODA'07}, pages 1012--1021.
  ACM-SIAM, 2007.

\bibitem{BoGoPeSc05}
A.~Bostan, L.~Gonz{\'a}lez-Vega, H.~Perdry, and {\'E}.~Schost.
\newblock From {N}ewton sums to coefficients: complexity issues in
  characteristic .
\newblock In {\em MEGA'05}, 2005.

\bibitem{BostanMorainSalvySchost2008}
A.~Bostan, F.~Morain, B.~Salvy, and {\'E}.~Schost.
\newblock Fast algorithms for computing isogenies between elliptic curves.
\newblock {\em Math. of Comp.}, 77(263):1755--1778, 2008.

\bibitem{BostanSalvySchost2008}
A.~Bostan, B.~Salvy, and {\'E}.~Schost.
\newblock Power series composition and change of basis.
\newblock In {\em ISSAC'08}, pages 269--276. ACM, 2008.

\bibitem{BoSc05}
A.~Bostan and {\'E}.~Schost.
\newblock Polynomial evaluation and interpolation on special sets of points.
\newblock {\em J. Complexity}, 21(4):420--446, 2005.

\bibitem{BostanSchost2009}
A.~Bostan and {\'E}.~Schost.
\newblock Fast algorithms for differential equations in positive
  characteristic.
\newblock In {\em ISSAC'09}, pages 47--54. ACM, 2009.

\bibitem{BrKu78}
R.~P. Brent and H.~T. Kung.
\newblock Fast algorithms for manipulating formal power series.
\newblock {\em J. ACM}, 25(4):581--595, 1978.

\bibitem{BrTr80}
R.~P. Brent and J.~F. Traub.
\newblock On the complexity of composition and generalized composition of power
  series.
\newblock {\em SIAM J. Comput.}, 9:54--66, 1980.

\bibitem{CaKa91}
D.~G. Cantor and E.~Kaltofen.
\newblock On fast multiplication of polynomials over arbitrary algebras.
\newblock {\em Acta Inform.}, 28(7):693--701, 1991.

\bibitem{CoWi90}
D.~Coppersmith and S.~Winograd.
\newblock Matrix multiplication via arithmetic progressions.
\newblock {\em J. Symb. Comput.}, 9(3):251--280, 1990.

\bibitem{DrIsSc11}
C.-{\'E}. Drevet, M.~Islam, and {\'E}.~Schost.
\newblock Optimization techniques for small matrix multiplication.
\newblock {\em Theoretical Computer Science}, 412:2219--2236, 2011.

\bibitem{FischerStockmeyer1974}
Fischer and Stockmeyer.
\newblock Fast on-line integer multiplication.
\newblock {\em J. of Computer and System Sciences}, 9, 1974.

\bibitem{GaGe99}
J.~von~zur Gathen and J.~Gerhard.
\newblock {\em Modern Computer Algebra}.
\newblock Cambridge University Press, 1999.

\bibitem{GuruswamiRudra2008}
V.~Guruswami and A.~Rudra.
\newblock Explicit codes achieving list decoding capacity: Error-correction
  with optimal redundancy.
\newblock {\em IEEE Trans. on Information Theory}, 54(1):135--150, 2008.

\bibitem{Hoeven2002}
J.~van~der Hoeven.
\newblock Relax, but don't be too lazy.
\newblock {\em J. Symb. Comput.}, 34(6):479--542, 2002.

\bibitem{Hoeven2003}
J.~van~der Hoeven.
\newblock Relaxed multiplication using the middle product.
\newblock In {\em ISSAC'03}, pages 143--147. ACM, 2003.

\bibitem{Hoeven2007}
J.~van~der Hoeven.
\newblock New algorithms for relaxed multiplication.
\newblock {\em J. Symb. Comput.}, 42(8):792--802, 2007.

\bibitem{Hoeven09}
J.~van~der Hoeven.
\newblock Relaxed resolution of implicit equations.
\newblock Technical report, HAL, 2009.
\newblock \texttt{{http://hal.archives-ouvertes.fr/hal-00441977}}.

\bibitem{Hoeven11}
J.~van~der Hoeven.
\newblock From implicit to recursive equations.
\newblock Technical report, HAL, 2011.
\newblock \texttt{{http://hal.archives-ouvertes.fr/hal-00583125}}.

\bibitem{Hoeven12}
J.~van~der Hoeven.
\newblock Faster relaxed multiplication.
\newblock Technical report, HAL, 2012.
\newblock \texttt{{http://hal.archives-ouvertes.fr/hal-00687479}}.

\bibitem{KedlayaUmans2011}
K.~S. Kedlaya and C.~Umans.
\newblock Fast polynomial factorization and modular composition.
\newblock {\em SIAM J. Comput.}, 40(6):1767--1802, 2011.

\bibitem{Kirrinnis01}
P.~Kirrinnis.
\newblock Fast algorithms for the {S}ylvester equation {}.
\newblock {\em Theoretical Computer Science}, 259(1--2):623--638, 2001.

\bibitem{LeSi08}
R.~Lercier and T.~Sirvent.
\newblock On {E}lkies subgroups of -torsion points in elliptic curves
  defined over a finite field.
\newblock {\em Journal de {T}h{\'e}orie des {N}ombres de {B}ordeaux},
  20(3):783--797, 2008.

\bibitem{ScSt71}
A.~Sch{\"o}nhage and V.~Strassen.
\newblock {S}chnelle {M}ultiplikation gro\ss er {Z}ahlen.
\newblock {\em Computing}, 7:281--292, 1971.

\bibitem{NTL}
V.~Shoup.
\newblock {NTL 5.5.2}: A library for doing number theory, 2009.
\newblock \texttt{www.shoup.net/ntl}.

\bibitem{Stothers10}
A.~Stothers.
\newblock {\em On the Complexity of Matrix Multiplication}.
\newblock PhD thesis, University of Edinburgh, 2010.

\bibitem{VassilevskaWilliams11}
V.~{Vassilevska Williams}.
\newblock Breaking the {C}oppersmith-{W}inograd barrier.
\newblock Technical report, 2011.

\bibitem{Waksman70}
A.~Waksman.
\newblock On {W}inograd's algorithm for inner products.
\newblock {\em IEEE Trans. On Computers}, C-19:360--361, 1970.

\bibitem{Wasow65}
W.~Wasow.
\newblock {\em Asymptotic expansions for ordinary differential equations}.
\newblock John Wiley \& Sons, 1965.

\end{thebibliography}

\end{document}
