\begin{table*}[]
\centering
\begin{tabular}{c|c|c|c|c|c|c}
    &\multicolumn{3}{c}{selfie2anime} &\multicolumn{3}{c}{cat2dog} \\
     \hline
    & DFID $\downarrow$ &FID $\downarrow$ & LPIPS $\uparrow$  & DFID $\downarrow$ &FID $\downarrow$ & LPIPS $\uparrow$ \\ \hline
    GNR        & \textbf{35.6} & \textbf{34.4} &\textbf{0.505} & \textbf{26.1}  &\textbf{26.9} & \textbf{0.569} \\ \hline
    DRIT++     & 94.6 & 63.8 & 0.201 & 160.1 &91.5 & 0.231 \\ \hline
    CouncilGAN & 56.2 & 38.1 & 0.430 & 172.5 & 90.8 & 0.298\\ \hline
    StarGANv2  & 83.0 & 59.8 & 0.427 & 53.6 & 44.2 & 0.530\\ 
\end{tabular}
\caption{\textbf{Quantitative Comparisons:} We compare GNR with other SoTA frameworks. DFID compares the distribution of a single image translated with different styles with the distribution of real images, thus focus on output diversity. FID compares general image quality while LPIPS also focuses on output diversity. Our method shows significant improvements across all metrics, especially on the Diversity FID and LPIPS metric which measures diversity.}
\label{tab:fid}
\end{table*}
