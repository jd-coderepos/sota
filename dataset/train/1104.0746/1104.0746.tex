\documentclass[envcountsect]{llncs}
\usepackage{stmaryrd}
\usepackage{amsmath}
\usepackage{amssymb}
\usepackage{newlfont}
\usepackage{graphicx}
\usepackage{caption}
\usepackage{algorithm,setspace}
\usepackage{algorithmic}

\title{Quantifier Elimination over Finite Fields Using Gr\"obner~Bases\thanks{This research was sponsored by National Science
   Foundation under contracts no.~CNS0926181,
   no.~CCF0541245, and no. CNS0931985,
   the SRC under contract
   no.~2005TJ1366, General Motors under contract no.~GMCMUCRLNV301, Air 
Force
   (Vanderbilt University) under contract
   no.~18727S3, the GSRC under contract no. 1041377 
(Princeton University), the Office of Naval Research under award 
no.~N000141010188, and DARPA under contract FA8650-10-C-7077.
}}
\author{Sicun Gao\and Andr\'e Platzer \and Edmund M. Clarke }
\institute{Carnegie Mellon University, Pittsburgh, PA, USA}

\begin{document}

\maketitle
\begin{abstract} 
We give an algebraic quantifier elimination algorithm for the first-order theory over any given finite field using Gr\"obner basis methods. The algorithm relies on the strong Nullstellensatz and properties of elimination ideals over finite fields. We analyze the theoretical complexity of the algorithm and show its application in the formal analysis of a biological controller model.
\end{abstract}

\section{Introduction}

We consider the problem of quantifier elimination of first-order logic formulas in the theory  of arithmetic in any given finite field . Namely, given a quantified formula  in the language, where  is a vector of quantified variables and  a vector of free variables, we describe a procedure that outputs a quantifier-free formula , such that  and  are equivalent in . 

Clearly,  admits quantifier elimination. A naive algorithm is to enumerate the exponentially many assignments to the free variables , and for each assignment , evaluate the truth value of the closed formula  (with a decision procedure). Then the quantifier-free formula equivalent to  is , where . This naive algorithm {\em always} requires exponential time and space, and cannot be used in practice. Note that a quantifier elimination procedure is more general and complex than a decision procedure: Quantifier elimination yields an equivalent quantifier-free formula  while a decision procedure outputs a yes/no answer. For instance, fully quantified formulas over finite fields can be ``bit-blasted'' and encoded as Quantified Boolean Formulas (QBF), whose truth value can, in principle, be determined by QBF decision procedures. However, for formulas with free variables, the use of decision procedures can only serve as an intermediate step in the naive algorithm mentioned above, and does not avoid the exponential enumeration of values for the free variables. We believe there has been no investigation into quantifier elimination procedures that can be practically used for this theory. 

Such procedures are needed, for instance, in the formal verification of cipher programs involving finite field arithmetic~\cite{SmithD08,using} and polynomial dynamical systems over finite fields that arise in systems biology~\cite{virus,bio2,poly}. Take the S2VD virus competition model~\cite{virus} as an example, which we study in detail in Section 6: The dynamics of the system is given by a set of polynomial equations over the field . We can encode image computation and invariant analysis problems as quantified formulas, which are solvable using quantifier elimination. As is mentioned in~\cite{virus}, there exists no verification method suitable for such systems over general finite fields so far.

In this paper we give an algebraic quantifier elimination algorithm for . The algorithm relies on strong Nullstellensatz and Gr\"obner basis methods. We analyze its theoretical complexity, and show its practical application. 

In Section 3, we exploit the strong Nullstellensatz over finite fields and properties of elimination ideals, to show that Gr\"obner basis computation gives a way of eliminating quantifiers in formulas of the form , where the s are atomic formulas and  is a quantifier block. We then show, in Section 4, that the DNF-expansion of formulas can be avoided by using standard ideal operations to ``flatten'' the formulas. Any quantifier-free formula can be transformed into conjunctions of atomic formulas at the cost of introducing existentially quantified variables. This transformation is linear in the size of the formula, and can be seen as a generalization of the {\em Tseitin transformation}. Combining the techniques, we obtain a complete quantifier elimination algorithm. 

In Section 5, we analyze the complexity of our algorithm, which depends on the complexity of Gr\"obner basis computation over finite fields. For ideals in  that contain  for each , Buchberger's Algorithm computes Gr\"obner bases within exponential time and space~\cite{Lakshman}. Using this result, the worst-case time/space complexity of our algorithm is bounded by  when  contains no more than two {\em alternating blocks} of quantifiers, and  for more alternations. Recently a polynomial-space algorithm for Gr\"obner basis computation over finite fields has been proposed in \cite{pspaceGB}, but it remains theoretical so far. If the new algorithm can be practically used, the worst-case complexity of quantifier elimination is  for arbitrary alternations. 

Note that this seemingly high worst-case complexity, as is common for Gr\"obner basis methods, does not prevent the algorithm from being useful on practical problems. This is crucially different from the naive algorithm, which always requires exponential cost, not just in worst cases. In Section 6, we show how the algorithm is successfully applied in the analysis of a controller design in the S2VD virus competition model~\cite{virus}, which is a polynomial dynamical system over finite fields. The authors developed control strategies to ensure a safety property in the model, and used simulations to conclude that the controller is effective. However, using the quantifier elimination algorithm, we found bugs that show inconsistency between specifications of the system and its formal model. This shows how our algorithm can provide a practical way of extending formal verification techniques to models over finite fields.

Throughout the paper, omitted proofs are provided in the Appendix.

\section{Preliminaries}
\subsection{Ideals, Varieties, Nullstellensatz, and Gr\"obner Bases}

Let  be any field and  the polynomial ring over  with indeterminates . An {\em ideal} generated by  is ,  Let  be an arbitrary point, and  be a polynomial. We say that  {\em vanishes} on  if . 
\begin{definition}
For any subset  of , the {\bf affine variety} of  over  is {}
\end{definition}
\begin{definition}
For any subset  of , the {\bf vanishing ideal}
of  is defined as 
\end{definition}
\begin{definition}
Let  be any ideal in , the {\bf radical} of  is defined as 
\end{definition}
 When , we say  is a radical ideal. The celebrated Hilbert Nullstellensatz established the correspondence between radical ideals and varieties:
\begin{theorem}[Strong Nullstellensatz~\cite{Lang}]\label{sN}
For an arbitrary field , let  be an ideal in . We have  where  is the algebraic closure of  and 
\end{theorem}

The method of Gr\"obner bases was introduced by Buchberger~\cite{buchberger76} for the algorithmic solution of various fundamental problems in commutative algebra. For an ideal  in a polynomial ring, Gr\"obner basis computation transforms  to a canonical representation  that has many useful properties. Detailed treatment of the theory can be found in \cite{grobnerbook}. 

\begin{definition}
Let  be the set of monomials in . A {\bf monomial ordering}  on  is a well-ordering on T satisfying \2) For all ,  then .
\end{definition}

We order the monomials appearing in any single polynomial  with respect to . We write  to denote the {\em leading monomial} in  (the maximal monomial under ), and  to denote the {\em leading term} of  ( multiplied by its coefficient). We write  where  is a set of polynomials.

Let  be an ideal in . Fix any monomial order on . The ideal of leading
monomials of , , is the ideal generated by the leading monomials of all polynomials in . Now we are ready to define:
\begin{definition}[Gr\"obner Basis~\cite{grobnerbook}]
A {\bf Gr\"obner basis} for  is a set  satisfying  
\end{definition}

\subsection{The First-order Theory over a Finite Field}

Let  be an arbitrary finite field of size , where  is a prime power. We fix the structure to be  and the signature  (``'' is a logical predicate). For quantified formulas, we write  to emphasize that the  is a vector of quantified variables and  is a vector of free variables.

The standard first-order theory for each  consists of the usual axioms for fields \cite{modelbook} plus , which fixes the size of the domain. We write this theory as . In , we consider all the atomic formulas as polynomial equations . The {\em realization} of a formula is the set of assignments to its free variables that makes the formula true over . Formally:

\begin{definition}[Realization]
Let  be a formula with free variables . The realization of , written as , is inductively defined as:
\begin{itemize}
\item  (in particular, )

\item 

\item 

\item 
\end{itemize}
\end{definition}
\begin{proposition}[Fermat's Little Theorem]
Let  be a finite field. For any , we have . Conversely, .
\end{proposition}
\begin{definition}[Quantifier Elimination]
 admits quantifier elimination if for any formula , where the  variables are quantified and the  variables free, there exists a quantifier-free formula  such that . 
\end{definition}

\subsection{Nullstellensatz in Finite Fields}

The strong Nullstellensatz admits a special form over finite fields. This was proved for prime fields in \cite{germ} and used in~\cite{poly,Marchand98onthe}. Here we give a short proof that the special form holds over arbitrary finite fields, as a corollary of Theorem~\ref{sN}.

\begin{lemma}\label{first-lemma}
For any ideal ,  is radical.
\end{lemma}

\begin{theorem}[Strong Nullstellensatz in Finite Fields]
\label{null}
For an arbitrary finite field , let  be an ideal,
then 
\end{theorem}
\begin{proof}
Apply Theorem~\ref{sN} to  and use Lemma~\ref{first-lemma}. We have 
.
But since , it follows that  Thus we obtain \qed
\end{proof}



\section{Quantifier Elimination Using Gr\"obner Bases}

In this section, we show that the key step in quantifier elimination can be realized by Gr\"obner basis computation. Namely, for any formula  of the form , we can compute a quantifier-free formula  such that . We use the following notational conventions:
\begin{itemize}
\item  is the number of quantified variables and  the number of free variables. We write  and , and call them {\em field polynomials} (following \cite{germ}).

\item We use  to denote the assignment for the  variables, and  for the  variables.  is a complete assignment for all the variables in .

\item When we write  or a formula , we assume that all the  variables do occur in  or . We assume that the  variables always rank higher than the  variables in the lexicographic order. 
\end{itemize}

\subsection{Existential Quantification and Elimination Ideals}

First, we show that eliminating the  variables is equivalent to {\em projecting} the variety  from  to . 

\begin{lemma}\label{formula-ideal}
For , we have .
\end{lemma}

\begin{definition}[Projection]
The -th projection mapping is defined as:  where .
For any set , we write 
\end{definition}

\begin{lemma}\label{just-projection}
.
\end{lemma}

Next, we show that the projection  of the variety  from  to ,  is exactly the variety . 

\begin{definition}[Elimination Ideal \cite{ideals}]
Let  be an ideal. The l-th {\bf elimination ideal} , for , is the ideal of  defined by 
\end{definition}

The following lemma shows that adding field polynomials does not change the realization. For , we have:
\begin{lemma} \label{extend-formulas}

\end{lemma}

Now we can prove the key equivalence between projection operations and elimination ideals. This requires the use of Nullstellensatz for finite fields.

\begin{theorem}\label{key-lemma}
Let  be an ideal which contains the field polynomials for all the variables in . We have 
\end{theorem}
\begin{proof} We show inclusion in both directions.
\begin{itemize}
\item  

For any , there exists  such that . That is,  satisfies all polynomials in ; in particular,  satisfies all polynomials in  that only contain the  variables ( is not assigned to variables). Thus, 

\item  

Let  be a point in  such that . Consider the polynomial 

 vanishes on all the points in , except , since  is excluded in the product for all . In particular,  vanishes on all the points in , because for each ,  must be different from , and  (since there are no  variables). Therefore,  is contained in the vanishing ideal of , i.e., .

Now, Theorem \ref{null} shows .
Since  already contains the field polynomials, we know , and consequently  Since , we must have . But on the other hand, . Hence . But since ,  we know .\qed
\end{itemize}
\end{proof}

\subsection{Quantifier Elimination using Elimination Ideals}

Theorem \ref{key-lemma} shows that to obtain the projection of a variety over , we only need to take the variety of the corresponding elimination ideal. In fact, this can be easily done using the Gr\"obner basis of the original ideal:

\begin{proposition}[cf. \cite{ideals}]\label{GB}
Let  be an ideal and let  be the Gr\"obner basis of  with respect to the lexicographic order . Then for every ,  is a Gr\"obner basis of the -th elimination ideal . That is, 
\end{proposition}

Now, putting all the lemmas together, we arrive at the following theorem:

\begin{theorem}\label{main-theorem}
Let  be  be a formula in , with . Let  be the Gr\"obner basis of . Suppose  then we have 
\end{theorem}

\begin{proof}
We write  for convenience. First, by Lemma \ref{extend-formulas}, adding the polynomials  and  does not change the realization:
{}Next, by Lemma \ref{just-projection}, the quantification on  corresponds to projecting a variety:
{}Using Theorem \ref{key-lemma}, we know that the projection of a variety is equivalent to the variety of the corresponding elimination ideal, i.e., . Now, using the property of Gr\"obner bases in Proposition \ref{GB}, we know the elimination ideal  is generated by :
{}Finally, by Lemma \ref{formula-ideal}, an ideal is equivalent to the conjunction of atomic formulas given by the generators of the ideal: 

Connecting all the equations above, we have shown  Note that  (they do not contain  variables).\qed
\end{proof}

\section{Formula Flattening with Ideal Operations}

If negations on atomic formulas can be eliminated (to be shown in Lemma \ref{no-negation}), Theorem \ref{main-theorem} already gives a direct quantifier elimination algorithm. That is, we can always use duality to make the innermost quantifier block an existential one, and expand the quantifier-free part to DNF. Then the existential block can be distributed over the disjuncts and Theorem \ref{main-theorem} is applied. However, this direct algorithm {\em always} requires exponential blow-up in expanding formulas into DNF. 

We show that the DNF-expansion can be avoided: Any quantifier-free formula can be transformed into an equivalent formula of the form , where  are new variables and s are polynomials. The key is that Boolean conjunctions and disjunctions can both be turned into additions of ideals; in the latter case new variables need be introduced. This transformation can be done in linear time and space, and is a generalization of the {\em Tseitin transformation} from  to general finite fields. 

We use the usual definition of ideal addition and multiplication. Let  and  be ideals, and  be a polynomial. Then  and .

\begin{lemma}[Elimination of Negations]
Suppose  is a quantifier free formula in  in NNF and contains  negative atomic formulas. Then there is a formula , where  contains new variables  but no negative atoms, such that .
\label{no-negation}\end{lemma}

\begin{lemma}[Elimination of Disjunctions]\label{no-disjunction}
Suppose  and  are two formulas in variables , and  and  are ideals in  satisfying  and . Then, using  as a new variable, we have 
\end{lemma}



\begin{theorem}
For any quantifier-free formula  given in NNF, there exists a formula  of the form  such that . Furthermore,  can be generated in time , and also .
\label{flatten-theorem}
\end{theorem}
\begin{proof}
Since  is in NNF, all the negations occur in front of atomic formulas. We first use Lemma \ref{no-negation} to eliminate the negations. Suppose there are  negative atomic formulas in , we obtain . Now  does not contain negations.

We then prove that there exists an ideal  for  satisfying , where  are the introduced variables (which rank higher than the existing variables in the variable ordering, so that the projection  truncates assignments on the  variables).
\begin{itemize}
\item If  is an atomic formula , then ;

\item If  is of the form , then ;

\item If  is of the form , then , where  is new.
\end{itemize}
Note that the new variables are only introduced in the disjunction case, and therefore the number of  variables equals the number of disjunctions. Following Lemma \ref{formula-ideal} and \ref{no-disjunction}, the transformation preserves the realization of the formula in each case. Hence, we have . Writing , we know  Notice that the number of rewriting steps is bounded by the number of logical symbols appearing in . Hence the transformation is done in time linear in the size of the formula. The number of new variables is equal to the number of negations and disjunctions. 
\qed
\end{proof}

\section{Algorithm Description and Complexity Analysis}

We now describe the full algorithm using the following notations:
\begin{itemize}
\item The input formula is given by . Each  represents a {\em quantifier block}, where  is either  or .  and  are different quantifiers. We write .  is a quantifier-free formula in  and  given in NNF, where  are free variables. 
\item We assume the innermost quantifier is existential, . (Otherwise we apply quantifier elimination on the negation of the formula.)
\end{itemize}
\subsection{Algorithm Description}
\begin{algorithm}[h]
\caption{Quantifier Elimination for }
\label{mainalgo}
    \begin{algorithmic}[1]
\STATE {\bf Input:}  where  is the number of quantifier alternations,  is an existential block (), and  is in negation normal form.
\STATE {\bf Output:} A quantifier-free equivalent formula of 
\STATE {\bf \em Procedure} {\bf\em QE()}
\WHILE{}
\STATE  {\bf\em Eliminate\_Negations()}
\STATE  {\bf\em Formula\_Flattening()}
\STATE 
\STATE  = Gr\"obner\_Basis()
\IF{}
\STATE 
\STATE {\bf break}
\ENDIF
\STATE  where 
\STATE  
\FOR{ to }
\STATE {\bf \em QE}
\ENDFOR
\STATE 
\STATE 
\ENDWHILE
\RETURN 
    \end{algorithmic}
\end{algorithm}


Section 3 shows how to eliminate existential quantifiers over conjunctions of positive atomic formulas. Section 4 shows how formulas can be put into conjunctions of positive atoms with new quantified variables. It follows that we can always eliminate the innermost existential quantifiers, and iterate the process by flipping the universal quantifiers into existential ones. We first emphasize some special features of the algorithm:
\begin{itemize}
\item In each elimination step, a full {\em quantifier block} is eliminated. This is desirable in practical problems, which usually contain many variables but few alternating quantifier blocks. For instance, many verification problems are expressible using two blocks of quantifiers (-formulas). 
\item The quantifier elimination step essentially transforms an ideal to another ideal. This corresponds to transforming conjunctions of atomic formulas to conjunctions of new atomic formulas. Therefore, the quantifier elimination steps do not introduce new nesting of Boolean operators.
\item The algorithm always directly outputs CNF formulas.
\end{itemize}

A formal description of the full algorithm is given in Algorithm \ref{mainalgo}. The main steps in the algorithm are explained below. Each loop of the algorithm contains three main steps. In Step 1,  is flattened; in Step 2, the innermost existential quantifier block is eliminated; in Step 3, the next (universal) quantifier block is eliminated and the process loops back to Step 1. The algorithm terminates either after Step 2 or Step 3, when there are no remaining quantifiers to be eliminated.

 {\bf Step 1: (Line 5-7)} 

First, since  is in NNF, we use Theorem \ref{flatten-theorem} to eliminate the negations and disjunctions in  to get , where  are the variables introduced for eliminating negations (Lemma \ref{no-negation}), and  are the variables introduced for eliminating disjunctions (Lemma \ref{no-disjunction}).


 {\bf Step 2: (Line 8-12)}

Since , using Theorem 4.1, we can eliminate the variables  simultaneously by computing  

Now we have 

If there are no more quantifiers, the output is , which is in CNF.


 {\bf Step 3: (Line 13-18)} 

Since , we distribute the block  over the conjuncts:
{}Now we do elimination recursively on  for each , which can be done using only Step 1 and Step 2. We obtain:{

and the formula becomes (note that the extra negation is distributed){

If there are no more quantifiers left, the output formula is , which is in CNF. Otherwise, , and we return to Step 1.



\begin{theorem}[Correctness]
Let  be a formula  where  and  is in NNF. Algorithm \ref{mainalgo} computes a quantifier-free formula , such that  and  is in CNF.
\end{theorem}

\subsection{Complexity Analysis}

The worst-case complexity of Gr\"obner basis computation on ideals in  that contain  for each variable  is known to be single exponential in the number of variables in time and space. This follows from the complexity result for Gr\"obner basis computation of zero-dimensional radical ideals~\cite{Lakshman} (a direct proof can be found in \cite{gao09}). 

\begin{proposition}\label{gb-time}
Let  be an ideal. The time and space complexity of Buchberger's Algorithm is bounded by , assuming that the length of input () is dominated by .
\end{proposition}

Now we are ready to estimate the complexity of our algorithm.

\begin{theorem}[Complexity]\label{complexity}
Let  be the input formula with  quantifier blocks. When , the time/space complexity of Algorithm 1 is bounded by . Otherwise, it is bounded by .
\end{theorem}
\begin{proof}
The complexity is dominated by Gr\"obner basis computation, whose complexity is determined by the number of variables occurring in the ideal. When , the main loop is executed once, and the number of newly introduced variables is bounded by the original length of the input formula. Therefore, Gr\"obner basis computations can be done in single exponential time/space. When , the number of newly introduced variables is bounded by the length of the formula obtained from the previous run of the main loop, which can itself be exponential in the number of the remaining variables.  In that case, Gr\"obner basis computation can take double exponential time/space. 

 {\bf Case :}

In Step 1, the number of the introduced  and  variables equals to the number of negations and disjunctions that appear in the . Hence the total number of variables is bounded by the length of . The flattening takes linear time and space, , as proved in Theorem \ref{flatten-theorem}.

In Step 2, by Proposition \ref{gb-time}, Gr\"obner basis computation takes time/space . 

In Step 3, the variables  have all been eliminated. The length of each  (see Formula (\ref{secondform}) in Step 3) is bounded by the number of monomials consisting of the remaining variables, which is  (because the degree on each variable is lower than ). Following Proposition \ref{gb-time}, Gr\"obner basis computation on each  takes time and space , which is dominated by . Also, since the number  of conjuncts is the number of polynomials in the Gr\"obner basis computed in the previous step, we know  is bounded by . In sum, Step 3 takes  time/space in worst case.

Thus, the algorithm has worst-case time and space complexity  when .

 {\bf Case :}

When , the main loop is iterated for more than one round. The key change in the second round is that, the initial number of conjunctions and disjunctions in each conjunct could both be exponential in the number of the remaining variables (). That means, writing the max of  as  (see Formula~(\ref{firstform}) in Step 3), both  and  can be of order . 

In Step 1 of the second round, the number of the  variables introduced for eliminating the negations is . The number of the  variables introduced for eliminating disjunctions is also . Hence the flattened formula may now contain  variables. 

In Step 2 of the second round, Gr\"obner basis computation takes time and space exponential in the number of variables. Therefore, Step 2 can now take  in time and space. 

In Step 3 of the second round, however, the number of conjuncts  {\em does not} become doubly exponential. This is because  in Step 3 no longer contains the exponentially many introduced variables -- they were already eliminated in the previous step. Thus  is reduced back to single exponential in the number of the remaining variables; i.e., it is bounded by . Similarly, the Gr\"obner basis computation on each , which now contains variables ,  takes time and space . In all, Step 3 takes time and space .

In sum, the second round of the main loop can take time/space . But at the end of the loop, the size of formula is reduced to  after the Gr\"obner basis computations, because it is at most single exponential in the number of the remaining variables. Therefore, the double exponential bound remains for future iterations of the main loop.\qed
\end{proof}

Recently, \cite{pspaceGB} reports a Gr\"obner basis computation algorithm in finite fields using polynomial space. This algorithm is theoretical and cannot be applied yet. Given the analysis above, if such a polynomial-space algorithm for Gr\"obner basis computation can be practically used, the intermediate expressions do not have the double-exponential blow-up. On the other hand, it does not lower the space bound of our algorithm to polynomial space, because during flattening of the disjunctions, the introduced terms are multiplied together. To expand the introduced terms, one may still use exponential space. It remains further work to investigate whether the algorithm can be practically used and how it compares with Buchberger's Algorithm.
\begin{proposition}
If there exists a polynomial-space Gr\"obner basis computation algorithm over finite fields for ideals containing the field polynomials, the time/space complexity of our algorithm is bounded by .
 \label{pspace}
\end{proposition}

\section{Example and Application}

\subsection{A Walk-through Example}

Consider the following formula over :  which has three alternating quantifier blocks and one free variable. We ask for a quantifier-free formula  equivalent to . 

We fix the lexicographic ordering to be  First, we compute the Gr\"obner basis  of the ideal: {}and obtain the Gr\"obner basis of the elimination ideal { }After this,  and  have been eliminated, and we have:
 {}Now we eliminate quantifiers in  again by computing the Gr\"obner basis  of the ideal {} We obtain . Therefore 
{.} (Note that if both  and  are both free variables,  would be the quantifier-free formula containing  that is equivalent to .) 

Next, we introduce  and  to eliminate the negations, and  to eliminate the disjunction: 

We now do a final step of computation of the Gr\"obner basis  of:
{}We obtain . This gives us the result formula  which means that  can take any value in  to make the formula true.
\subsection{Analyzing a Biological Controller Design}

We studied a virus competition model named S2VD~\cite{virus}, which models the dynamics of virus competition as a polynomial system over finite fields. The authors aimed to design a controller to ensure that one virus prevail in the environment. They pointed out that there was no existing method for verifying its correctness. The current design is confirmed effective by computer simulation and lab experiments for a wide range of initializations. We attempted to establish the correctness of the design with formal verification techniques. However, we found bugs in the design. 

All the Gr\"obner basis computations in this section are done using scripts in the SAGE system~\cite{sage}, which uses the underlying Singular implementation~\cite{singular}. All the formulas below are solved within 5 seconds on a Linux machine with 2GHz CPU and 2GB RAM. They involve around 20 variables over , with nonlinear polynomials containing multiplicative products of up to 50 terms.  

\begin{figure}[ht]
\centering
\label{fig}
\includegraphics[width=80pt]{original-hex.png}\hspace{.25in}
\includegraphics[width=70pt]{hex.png}\hspace{.25in}
\includegraphics[width=50pt]{hex-6.png}
\caption{(a) The ten rings of S2VD; (b) Cell x and its neighbor  cells; (c) The counterexample}
\end{figure}

\subsubsection{The S2VD Model}The model consists of a hexagonal grid of cells. Each hexagon represents a cell, and each cell has six neighbors. There are four possible colors for each cell. A green cell is infected with (the good) Virus G, and a red cell is infected with (the bad) Virus R. When the two viruses meet in one cell, Virus G captures Virus R and the cell becomes yellow. A cell not infected by any virus is white. The dynamics of the system is determined by the interaction of the viruses. 

There are ten rings of cells in the model, with a total of 331 cells (Figure 1(a)). In the initial configuration, the cells in Ring 4 to 10 are set to white, and the cells in Ring 1 to 3 can start with arbitrary colors. The aim is to have a controller that satisfies the following safety property: The cells in the outermost ring are either green or white at all times. The proposed controller detects if any cell has been infected by Virus R, and injects cells that are ``one or two rings away'' from it with Virus G. The injected Virus G is used to block the further expansion of Virus R.

Formally, the model is a polynomial system over the finite field , with each element representing one color: . The dynamics is given by the function  For each cell , its dynamics  is determined by the color of its six neighbors , specified by the nonlinear polynomial , where  and . The designed controller is specified by another function : For each cell , with  representing the cells in the two rings surrounding it, we define . More details can be found in \cite{virus}.

\subsubsection{Applying Quantifier Elimination}
We first try checking whether the safety property itself forms an inductive invariant of the system (which is a strong sufficient check). To this end, we check whether the controlled dynamics of the system remain inside the invariant on the boundary (Ring 10) of the system. Let  be a cell in Ring 10 and  be the cells in its immediate two rings. We assume the cells outside Ring 10 () are white. See Figure 1(b) for the coding of the cells. We need to decide the formula:
{\footnotesize}
where (writing ){\footnotesize}After quantifier elimination, Formula (\ref{thirdform}) turns out to be false. In fact, we obtained . Therefore, the safety property itself is not an inductive invariant of the system. We realized that there is an easy counterexample of safety of the proposed controller design: Since the controller is only effective when red cells occur, it does not prevent the yellow cells to expand in all the cells. Although this is already a bug of the system, it may not conflict with the authors' original goal of controlling the red cells. However, a more serious bug is found by solving the following formula:
{\footnotesize 

}Formula (\ref{fourthform}) expresses the desirable property that when none of the neighbor cells of  is red,  never becomes red. However, we found again that , which means in this scenario the  cell can still turn red. Thus, the formal model is inconsistent with the informal specification of the system, which says that non-red cells can never interact to generate red cells. In fact, the authors mentioned that the dynamics  is not verified because of the combinatorial explosion. Finally, to give a counterexample of the design, we solve the formula
{\footnotesize 

}The formula checks whether there exists a configuration of  which are all non-red, such that  becomes red.  evaluates to true. Further, we obtain  as a witness assignment for . This serves as the counterexample (see Figure 1(c)). 

This example shows how our quantifier elimination procedure provides a practical way of verifying and debugging systems over finite fields that were previously not amenable to existing formal methods and cannot be approached by exhaustive enumeration.

\section{Conclusion}
In this paper, we gave a quantifier elimination algorithm for the first-order theory over finite fields based on the Nullstellensatz over finite fields and Gr\"obner basis computation. We exploited special properties of finite fields and showed the correspondence between elimination of quantifiers, projection of varieties, and computing elimination ideals. We also generalized the Tseitin transformation from Boolean formulas to formulas over finite fields using ideal operations. The complexity of our algorithm depends on the complexity of Gr\"obner basis computation. In an application of the algorithm, we successfully found bugs in a biological controller design, where the original authors expressed that no verification methods were able to handle the system. In future work, we expect to use the algorithm to formally analyze more systems with finite field arithmetic. The scalability of the method will benefit from further optimizations on Gr\"obner basis computation over finite fields. It is also interesting to combine Gr\"obner basis methods and other efficient Boolean methods (SAT and QBF solving). See \cite{gao09} for a discussion on how the two methods are complementary to each other.

\section*{Acknowledgement}

The authors are grateful for many important comments from Jeremy Avigad, Helmut Veith, Paolo Zuliani, and the anonymous reviewers. 

\begin{thebibliography}{20}

\bibitem{sage} The SAGE Computer Algebra system, {\tt http://sagemath.org}

\bibitem{singular} The Singular Computer Algebra system, {\tt http://www.singular.uni-kl.de/}

\bibitem{grobnerbook} {Becker, T., Weispfenning, V.}: {Gr\"obner Bases}. Springer, (1998)

\bibitem{poly} {Le Borgne, M., Benveniste, A., Le Guernic, P.}: {Polynomial dynamical systems over finite fields}. In: {Algebraic Computing in Control}, Vol. 165, Springer, (1991)


\bibitem{Marchand98onthe} Marchand, H., Le Borgne, M.: On the Optimal Control of Polynomial Dynamical Systems over . In: 4th International Workshop on Discrete Event Systems, pp. {385--390} (1998)


\bibitem{buchberger76} Buchberger, B.: A Theoretical Basis for the Reduction of Polynomials to Canonical
Forms. In: ACM SIGSAM Bulletin, 10(3), pp.19-29, (1976)

\bibitem{ideals} Cox, D., Little, J., O'Shea, D.: Ideals, Varieties, and Algorithms. Springer (1997)

\bibitem{using} Cox, D., Little, J., O'Shea, D.: Using Algebraic Geometry. Springer (2005)

\bibitem{gao09} Gao, S.: Counting Zeroes over Finite Fields with Gr\"obner Bases. Master Thesis, Carnegie Mellon University (2009)

\bibitem{germ} Germundsson, R.: Basic results on ideals and varieties in Finite Fields. Technical
Report LiTH-ISY-I-1259, Linkoping University, S-581 83 (1991)

\bibitem{virus} Jarrah, A., Vastani, H., Duca, K., and Laubenbacher, R.: An Optimal Control Problem
for in vitro Virus Vompetition. In: 43rd IEEE Conference on Decision and Control (2004)

\bibitem{bio2} Jarrah, A.S., Laubenbacher, R., Stigler, B., and Stillman, M.: Reverse-engineering of
polynomial dynamical systems. In: Advances in Applied Mathematics, vol. 39, pp 477--489 (2007)

\bibitem{Lakshman} Lakshman, Y.N.: On the Complexity of Computing a Gr\"obner Casis for the Radical
of a Zero-dimensional Ideal. In STOC '90, pp 555--563, New York, NY, USA, (1990)

\bibitem{Lang} Lang, S.: Algebra, 3rd Edition. Springer (2005)

\bibitem{modelbook} Marker, D.: Model theory. Springer (2002)

\bibitem{SmithD08} Smith, E.W., Dill, D.L.: Automatic Formal Verification of Block Cipher Implementations.
In: FMCAD, pp. 1--7 (2008)

\bibitem{pspaceGB} Tran, Q.N.: Gr\"obner Bases Computation in Boolean Rings is PSPACE. In: International Journal of Applied Mathematics and Computer Sciences, vol. 5, No. 2, (2009)

\end{thebibliography}

\newpage

\section*{Appendix: Omitted Proofs}

\subsubsection{Proof of Lemma 2.1}

This is a consequence of the Seidenberg's Lemma (Lemma 8.13 in \cite{grobnerbook}). It can also be directly proved as follows.
\begin{proof}
We need to show  Since by definition, any ideal is contained in its
radical, we only need to prove 

Let  denote . Consider an arbitrary polynomial  in the ideal . By definition, for some integer , . Let  and  be the images of,
respectively,  and , in  under the
canonical homomorphism from  to  For brevity we write .

Now we have , and we further need . We prove, by induction on the structure of polynomials, that for any , .  
\begin{itemize}
\item If  (), then

\item If , by inductive hypothesis, , and,
since any element divisible by  is zero in  (), then

\end{itemize}
Hence  for any , without loss of generality
we can assume  in . Then, since , \qed
\end{proof}




\subsubsection{Proof of Lemma \ref{formula-ideal}}
\begin{proof}
Let  be an assignment vector for .

If , then  and . 

If , then  is true and .\qed
\end{proof}

\subsubsection{Proof of Lemma \ref{just-projection}}
\begin{proof} We show set inclusion in both directions.
\begin{itemize}
 \item
For any , by definition, there exists  such that  satisfies . Therefore, , and .
\item 
For any , there exists  such that . By definition, .\qed
\end{itemize}
\end{proof}

\subsubsection{Proof of Lemma 3.3}
\begin{proof}
We have , which follows from Proposition 2.1. \qed
\end{proof}


\subsubsection{Proof of Lemma \ref{no-negation}}
\begin{proof}
Let  denote substitution of  in  by . Suppose the negative atomic formulas in  are . 

We introduce a new variable , and substitute  by . Since the field  does not have zero divisors, all the solutions for  (the Rabinowitsch trick). 

Iterating the procedure, we can use  new variables  so that:

 Since the result formula contains no more negations and the s are new variables, it can be put into prenex form .\qed
\end{proof}

\subsubsection{Proof of Lemma \ref{no-disjunction}}
\begin{proof} follows from the definition of realization. We only need to show the second equality. Let  be a point.

- Suppose . If , then . If , then . In both cases, .

- Suppose . There exists  such that . If , then all the polynomials in  and  need to vanish on ; if  then  vanishes on ; if   then  vanishes on . In all cases, .\qed
\end{proof}



\subsubsection{Proof of Theorem 5.1}
\begin{proof}
We only need to show the intermediate formulas obtained in Step 1-3 are always equivalent to the original formula . In Step 1, the formula is flattened with ideal operations, which preserve the realization of the formula as proved in Theorem \ref{flatten-theorem}. In Step 2, we have (by Theorem \ref{main-theorem})  

Hence the formula obtained in Step 2 is equivalent to . In Step 3, the substitution preserves realization of the formula because

where the second equality is guaranteed by Theorem \ref{main-theorem} again.

The loop terminates either at the end of Step 2 or Step 3. Hence the output quantifier-free formula  is always in conjunctive normal form, which contains only variables , and is equivalent to the original formula .\qed
\end{proof}
\end{document}
