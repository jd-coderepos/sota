

 \documentclass[journal]{IEEEtran}

\usepackage{graphicx}


\usepackage{graphics}
\usepackage{cite}
\usepackage{url}
\usepackage{enumerate}
\usepackage{latexsym}
\usepackage{newlfont}
\usepackage{amsthm}
\usepackage{amsmath}
\usepackage{amsfonts}
\usepackage{amssymb}
\usepackage{dropping}
\usepackage{pifont}
\usepackage[usenames,dvipsnames]{color}





\makeatletter


\newtheorem{theorem}{Theorem}[section]
\newtheorem{lemma}[theorem]{Lemma}
\newtheorem{proposition}[theorem]{Proposition}
\newtheorem{corollary}[theorem]{Corollary}
\newtheorem{definition}[theorem]{Definition}
\newtheorem{remark}[theorem]{Remark}


\begin{document}
\title{THRIVE: Threshold Homomorphic encRyption based secure and privacy preserving bIometric VErification system}

\newcommand{\mehmet}[1]{\authnote{Mehmet}{#1}}
\newcommand{\cagatay}[1]{\authnote{Cagatay}{#1}}
\def\msk#1{\textcolor{red}{[[#1]]}}
\def\ck#1{\textcolor{blue}{[[#1]]}}

\makeatletter \def\@r@al{\raise\arrowlower}
\def\@lrulefill{\cleaders\hbox{-}\hfill}
\def\@lendofrule{\@r@al\hbox{}\mkern-4mu}
\def\@rendofrule{\mkern-4mu\@r@al\hbox{}}
\def\@rafill{\mkern-6mu\@lrulefill\@rendofrule\@r@al\llap{}}
\def\@lafill{\m@th\@r@al\rlap{}\@lendofrule\@lrulefill\mkern-6mu}
\def\@rdfill{\mkern-6mu\@lrulefill\@rendofrule}
\def\@ldfill{\m@th\@lendofrule\@lrulefill\mkern-6mu} \newdimen\minarrowwidth
\minarrowwidth=0pt \newdimen\arrowwidth \newdimen\arrowlower \arrowlower=-4pt
\def\@showmessage#1{\setbox\@tempboxa\hbox{}
\ifdim\wd\@tempboxa<\minarrowwidth \arrowwidth\minarrowwidth \else
\arrowwidth\wd\@tempboxa \fi \hbox{\m@th\@lrulefill}}
\def\sends{\@sends} \def\@sends#1{\@ldfill\@showmessage{#1}\@rafill}
\def\receives{\@receives} \def\@receives#1{\@lafill\@showmessage{#1}\@rdfill}
\makeatother


\newcommand{\ignore}[1]{}
\newcommand{\com}[3]{\textrm{commit}}
\newcommand{\comh}[4]{\textrm{commit}}
\newcommand{\comnor}[2]{\textrm{commit}}
\pagestyle{plain}

\author{\IEEEauthorblockN{
Cagatay~Karabat\IEEEauthorrefmark{1}'\IEEEauthorrefmark{2},
Mehmet Sabir Kiraz\IEEEauthorrefmark{1}, 
Hakan~Erdogan\IEEEauthorrefmark{2},
Erkay~Savas\IEEEauthorrefmark{2}\\
\IEEEauthorblockA{\IEEEauthorrefmark{1}TUBITAK BILGEM UEKAE, Kocaeli, Turkey} \\
\IEEEauthorblockA{\IEEEauthorrefmark{2}Sabanci University, TR-34956, Tuzla, Istanbul, Turkey}\\
\{cagatay.karabat, mehmet.kiraz\}@tubitak.gov.tr, \{haerdogan, erkays\}@sabanciuniv.edu}}




\maketitle

\begin{abstract}


In this paper, we propose a new biometric verification and template protection system which we call the THRIVE system. The system includes novel enrollment and authentication protocols based on threshold homomorphic cryptosystem where the private key is shared between a user and the verifier. In the THRIVE system, only encrypted binary biometric templates are stored in the database and verification is performed via homomorphically randomized templates, thus, original templates are never revealed during the authentication stage. The THRIVE system is designed for the malicious model where  the cheating party may arbitrarily deviate from the protocol specification. Since threshold homomorphic encryption scheme is used, a malicious database owner cannot perform decryption on encrypted templates of the users in the database. Therefore, security of the THRIVE system is enhanced using a two-factor authentication scheme involving the user's private key and the biometric data. We prove security and privacy preservation capability of the proposed system in the simulation-based model with no assumption. The proposed system is suitable for applications where the user does not want to reveal her biometrics to the verifier in plain form but she needs to proof her physical presence by using biometrics. The system can be used with any biometric modality and biometric feature extraction scheme whose output templates can be binarized. The overall connection time for the proposed THRIVE system is estimated to be 336~ms on average for 256-bit biohash vectors on a desktop PC running with quad-core 3.2 GHz CPUs at 10 Mbit/s up/down link connection speed. Consequently, the proposed system can be efficiently used in real life applications.

\end{abstract}


\begin{IEEEkeywords}
Biometric, Security, Privacy, Threshold Cryptography, Homomorphic Encryption, Malicious Attacks
\end{IEEEkeywords}


\IEEEpeerreviewmaketitle


\section{Introduction}

\IEEEPARstart{I}{n} recent years, public and commercial organizations invest on secure electronic authentication (e-authentication) systems to reliably verify identity of individuals. Biometrics is one of the rapidly emerging technologies for e-authentication systems \cite{Jain1}. However, it is impossible to discuss biometrics without security and privacy issues \cite{Prabhakar, Jain2}. Biometrics, which are stored in a smart card or a central database, is under security and privacy risks due to increased number of attacks against identity management systems in recent years \cite{Prabhakar, Ratha, Roberts, Jain2}.


Security and privacy concerns on biometrics limit their widespread usage in real life applications. The initial solution that occurs to mind for security and privacy problems is to use cryptographic primitives. On the other hand, biometric templates cannot be directly used with conventional encryption techniques (i.e. AES, 3DES) since biometric data are inherently noisy \cite{Kevenaar}. In other words, the user is not able to present exactly the same biometric data repeatedly. Namely, when a biometric template is encrypted during the enrollment stage, it should be decrypted to pass the authentication stage for comparison with the presented biometric. This, however, again leads to security and privacy issues for biometric templates at the authentication stage \cite{Kevenaar}.  Another problem with regards to such a solution is the key management, i.e. storage of encryption keys. When a malicious database manager obtains decryption keys, he can perform decryption and obtain biometric templates of all users. Similar problems are valid for cryptographic hashing methods. Since cryptographic hash is a one-way function, when a single bit is changed the hash sum becomes completely different due to the avalanche effect \cite{Feistel}. Thus, successful authentication by exact matching cannot be performed even for legitimate users due to the noisy nature of biometric templates. Therefore, biometric templates cannot directly be used with the cryptographic hashing methods. 


Biometric systems which use error correction methods are proposed to cope with noisy nature of the biometric templates in the literature \cite{Hao, Kanade, Juels2}. In such systems, the biometric data collected at the enrollment stage is exactly the same with the biometric data collected at the authentication stage since they use error correction methods. In other words, these systems can get error-free biometric templates and thus cryptographic primitives (i.e. encryption and hashing) can successfully be employed without suffering from the avalanche effect \cite{Davida, Juels2, Tulyakov, Kevenaar}. However, high error correcting capability requirements make them impractical for real life applications \cite{Sutcu2}. Furthermore, side information (parity bits) is needed for error correction and this may lead to information leakage and even other attacks (i.e., error correcting code statistics, and non-randomness attacks) \cite{Stoianov1}. Zhou \textit{et al.} clearly demonstrate in their works that redundancy in an error correction code causes privacy leakage for biometric systems \cite{Zhou, Zhou2}.


Although biometric template protection methods are proposed to overcome security and privacy problems of biometrics \cite{Jain2, Cimato1, Matsumoto, Putte, Bringer, Barni, Nkumar, Feng, Bui, Juels, Karabat, Bai, Kuan, Rathgeb, Lumini}, recent research shows that security issues are still valid for these schemes \cite{Scheirer, Adler, Boult, Kong, Vielhauer, Cheung, Kummel}. Furthermore, there are a number of works on privacy leakages of biometric applications \cite {Simoens, Ignatenko1},  and yet more biometrics template protection methods \cite{Zhou, Zhou2, Simoens2}. In the literature, Zhou \textit{et al.} propose a framework for security and privacy assessment of biometric template protection methods \cite{Zhou}. In addition, Ignatenko \textit{et al.} analyze the privacy leakage in terms of the mutual information between the public helper data and biometric features in a biometric template protection method. A trade-off between maximum secret key rate and privacy leakage is given in their works \cite{Ignatenko1, Ignatenko2}. 

Recently, homomorphic encryption methods are used with biometric feature extraction methods to perform verification via encrypted biometric templates \cite{Erkin, Sadeghi, Barni, Barni2}. However, these methods offer solutions in the honest-but-curious model where each party is obliged to follow the protocol, but can arbitrarily analyze the knowledge that it learns during the execution of the protocol to obtain some additional information. The existing systems are not designed for the malicious model where each party can arbitrarily deviate from the protocol and may be corrupted. Moreover, they do not take into account security and privacy issues of biometric templates stored in the database \cite{Barni, Barni2}. The authors state that their security model will be improved in the future work by applying encryption methods also on the biometric templates stored in the database. Furthermore, some of these systems are just designed for a single biometric modality or a specific feature extraction method which also limits their application areas \cite{Erkin, Sadeghi}. In addition, an adversary can enroll himself on behalf of any user to their systems since they do not offer any solutions for malicious enrollment. Finally, all these systems suffer from computational complexity.

Biohashing schemes are one of the emerging biometric template protection methods \cite{Karabat, Bai, Kuan, Rathgeb, Lumini}. These schemes offer low error rates and fast verification at the authentication stage. However, they suffer from several attacks reported in the literature \cite{Kong, Vielhauer, Cheung, Kummel}. These schemes should be improved  to be safely used in a wide range of real life applications. In our work, we develop new enrollment and authentication protocols for biometric verification methods. Our goal is to increase security and enhance privacy of the biometric schemes. The THRIVE system can work with any biometric feature extraction scheme whose outputs are binary or can be binarized. Since biohashing schemes can output binary templates called a biohash, they can be successfully used with the proposed system. 


\subsection{Our contributions}

In this paper, we address adversary attacks in case of an active attacker who aims to gain access to the system in the malicious attack model. By taking these adversary attacks into account, we develop a new biometric authentication system based on threshold homomorphic cryptosystem. Our main goal is to increase the security of the system and preserve privacy of biometric templates of the users. The contributions of this work can be summarized as follows:

\begin{itemize}
\item A new biometric authentication system (which we call the THRIVE system) is proposed in the malicious model and the proposed system can be used with any existing biometric modality whose output can be binarized. 

\item Even if an adversary gains an access to the database and steals encrypted biometric templates,  he can neither authenticate himself by using these encrypted biometric templates due to the proposed authentication protocol, nor decrypt these encrypted biometric templates due to the -threshold homomorphic encryption scheme. 

\item Only encrypted binary templates are stored in the database and biometric templates are never released even during authentication. Thus, the proposed system offers a new and advanced biometric template protection method without any helper data. In addition, only legitimate users can enroll in the proposed system since a signature scheme is used with the proposed enrollment protocol.

\item The THRIVE system can be used in the applications where the user does not trust the verifier since she does not need to reveal her biometric template and/or private key to authenticate herself and  the verifier does not need to reveal any data to the user with the proposed authentication protocol. 

\item Even if an adversary intercepts the communication channel between the user and the verifier, he cannot obtain any useful information on the biometric template since all exchanged messages are randomized and/or encrypted and he cannot perform decryption due to the -threshold homomorphic encryption scheme. Furthermore, he cannot use the obtained data from message exchanges in this communication channel since nonce and signature schemes are used together in the authentication.

\item The THRIVE system is a two-factor authentication system (biometric and secret key) and is secure against illegal authentication attempts. In other words, a malicious adversary cannot gain access to the proposed system without having the biometric data and the private key of a legitimate user by performing adversary attacks described in \cite{Ratha2} as well as hill-climbing attacks \cite{Adler2, Galbally, Uhl, Martinez}.

\item In the THRIVE system, the generated protected biometric templates are irreversible since templates are encrypted they are irreversible by definition as soon as decryption key is not stolen. 

\item The THRIVE system can generate a number of protected templates from the same biometric data of a user due to the randomized encryption and biohashing. Thus, it ensures diversity. Besides, they are also cancelable i.e. when they are stolen, they can be re-generated.

\item A THRIVE  authentication protocol run requires only 336 ms on average for 256 bit biohash vectors and 671 ms on average for 512 bit biohash vectors on a desktop PC with quad-core 3.2 GHz CPUs at 10 Mbit/s up/down link connection speed. Therefore, the proposed system is sufficiently efficient to be used in real-world applications.

\end{itemize}

The paper is structured as follows. Related work is addressed in Section 2. Preliminaries are described in Section 3. The proposed biometric authentication system is introduced in Section 4. Security proof of the proposed protocols are given in Section 5. Complexity analysis of the proposed system is discussed in Section 6.  Section 7 concludes the paper.

\section{Related Work}

Biometric template protection schemes are proposed to mitigate the security and privacy problems of biometrics \cite{Jain2, Matsumoto, Putte, Bringer, Barni, Nkumar, Feng, Bui, Juels}. However, various vulnerabilities of these technologies are reported in the literature \cite{Scheirer, Adler, Boult}. Jain \textit{et al.} classify biometric template protection schemes into two main categories \cite{Jain2}: 1) Feature transformation based schemes, 2) Biometric cryptosystems as illustrated in Figure~\ref{categorize}. 

The main idea behind biometric cryptosystems (also known as biometric encryption systems) is either binding a cryptographic key with a biometric template or generating the cryptographic key directly from the biometric template~\cite{Uludag}. Thus, the biometric cryptosystems can be classified into two main categories: 1) Key binding schemes, 2) Key generation schemes. The biometric cryptosystems use helper data, which is public information, about the biometric template for verification. Although helper data are supposed not to leak any critical information about the biometric template, Rathgeb \textit{et al.} show that helper data is vulnerable to statistical attacks \cite{Rathgeb2}. Furthermore, Ignatenko \textit{et al.} show how to compute a bound on possible secret rate and privacy leakage rate for helper data schemes ~\cite{Ignatenko3}. Adler performs hill-climbing attack against biometric encryption systems ~\cite{Adler}. In addition, Stoianov \textit{et al.} propose several attacks (i.e., nearest impostors, error correcting code statistics, and non-randomness attacks) to biometric encryption systems \cite{Stoianov1}. 

In the literature, fuzzy commitment \cite{Juels2} and fuzzy vault schemes \cite{Juels} are categorized under the key binding schemes. These schemes aim to bind a cryptographic key with a biometric template. In ideal conditions, it is infeasible to recover either the biometric template or the random bit string without any knowledge of the user's biometric data. However, this is not the case in reality because biometric templates are not uniformly random. Furthermore, error correction codes (ECC) used in biometric cryptosystems lead to statistical attacks (i.e., running ECC in a soft decoding or erasure mode and ECC Histogram attack) \cite{Stoianov1, Stoianov2}. Ignatenko \textit{et al.} show that fuzzy commitment schemes leak information in cryptographic keys and biometric templates which lead to security flaws and privacy concerns \cite{Ignatenko1, Ignatenko2}. In addition, Zhou \textit{et al.} argue that fuzzy commitment schemes leak private data. Chang \textit{et al.} describe a non-randomness attack against fuzzy vault scheme which causes distinction between the minutiae points and the chaff points \cite{Chang}. Moreover, Kholmatov \textit{et al.} perform a correlation attack against fuzzy vault schemes \cite{berrin}.

Keys are generated from helper data and a given biometric template in key generation schemes \cite{Jain2}. Fuzzy key extraction schemes are classified under key generation schemes and use helper data \cite{Dodis1, Dodis2, Yagiz, Ong, Arakala}. These schemes can be used as an authentication mechanism where a user is verified via her own biometric template as a key. Although fuzzy key extraction schemes provide key generation from biometric templates, repeatability of the generated key (in other words stability) and the randomness of the generated keys (in other words entropy) are two major problems of them \cite{Jain2}. Boyen  \textit{et al.} describe several vulnerabilities (i.e. improper fuzzy sketch constructions may lead information on the secret, biased codes may cause majority vote attack, and permutation leaks) of the fuzzy key extraction schemes from outsider and insider attacker perspectives \cite{Boyen}. Moreover, Li \textit{et al.} mention that when an adversary obtains sketches, they may reveal the identity of the users \cite{Li}.

Biohashing schemes are simple yet powerful biometric template protection methods \cite{Karabat, Bai, Kuan, Rathgeb, Lumini} which can be classified under salting based schemes. It is worth pointing out that biohashing is completely different from cryptographic hashing. Although biohashing schemes are proposed to solve security and privacy issues, there are still security and privacy issues associated with them \cite{Kong, Vielhauer, Cheung, Kummel}. In these works, the authors claim that biohashes can be reversible under certain conditions and an adversary can estimate biometric template of a user from her biohash. Consequently, when biohashes are stored in the databases and/or smart cards in their plain form, they can threaten the security of the system as well as the privacy of the users. Moreover, an adversary can use an obtained biohash to threaten the system security by performing malicious authentication. Furthermore, when the secret key is compromised, an adversary can recover the biometric template since these schemes are generally invertible \cite{Jain2}. 

Non-invertible transform based schemes use a non-invertible transformation function, which is a one-way function, to make the biometric template secure \cite{Sutcu, Jin, Yang}. User's secret key determines the parameters of  non-invertible transformation function and this secret key should be provided at the authentication stage. Even if an adversary obtains the secret key and/or the transformed biometric template, it is computationally hard to recover the original biometric template. On the other hand, these schemes suffer from the trade-off between discriminability and non-invertibility which limits their recognition performance \cite{Jain2}.

Another security approach is the use of cryptographic primitives (i.e. encryption, hashing) to protect biometric templates. These works generally focus on fingerprint-based biometric systems. Tuyls \textit{et al.} propose the fingerprint authentication system which incorporates cryptographic hashes \cite{Tuyls}. They use an error correction scheme to get exactly the same biometric template from the same user in each session which is similar to the fuzzy key extraction schemes. They store cryptographic hashes of biometric templates in the database and make comparison in the hash domain. However, there is no guarantee to get exactly the same biometric templates from the user even if the system incorporates an error correction scheme in real life applications since it is limited with the pre-defined threshold of error correction capacity. They also use helper data which are sent over a public channel and this may lead to security flaws as well. Moreover, an adversary can threaten the security of the system when he performs an attack against the database since he can obtain the user id, helper data and the hashed version of the secret which is generated by the biometric data and the helper data. Although the adversary cannot obtain the biometric data itself in its plain form, he can get all needed credentials (i.e. hash values of the secrets) to gain access to the system.

Kerschbaum \textit {et al.} propose a protocol to compare fingerprint templates without actually exchanging them by using secure multi-party computation in the honest-but-curious model \cite{Kerschbaum}. At the enrollment stage, the user gives her fingerprint template, minutiae pairs and PIN to the system. Thus, the verifier knows the fingerprint templates which are collected at the enrollment stage.  Although the user does not send her biometric data at the authentication, the verifier already has the user's enrolled biometric data and this threatens the privacy of the user in case of a malicious verifier. In addition, a malicious verifier can use these fingerprint templates for malicious authentication. Furthermore, since the fingerprint comparison reveals the  matching scores (i.e. Hamming distance\cite{Hamming}), the attacker can perform a hill climbing attack against this system. Apart from these security and privacy flaws, the authors just focus on secure comparison in their protocol and they do not develop any solutions for the malicious model. 

Erkin \textit {et al.} \cite{Erkin} propose a privacy preserving face recognition system for the eigen-face recognition algorithm \cite{Turk}. They design a protocol that performs operations on encrypted images by using the Pailler homomorphic encryption scheme. Later, Sadeghi \textit{et al.}  improve the efficiency of this system \cite{Sadeghi}. In both works, they use the eigen-face recognition algorithm together with homomorphic encryption schemes. However, they limit the recognition performance of the system with the eigen-face method although there are various feature extraction methods which perform better than it. Unfortunately, their system cannot be used for any other feature extraction method for face images. Moreover, they do not use a threshold cryptosystem which prevents from a malicious party aiming to perform decryption by himself. Storing face images (or corresponding feature vectors) in the database in plain is the most serious security flaw of this system. An adversary, who gains access to the database, can obtain all face images. Therefore, the adversary can perform the sixth attack type - attack against the database which definitely threatens the security of the system and the privacy of the users.

Barni \textit {et al.} \cite{Barni, Barni2} propose a privacy preservation system for fingercode templates by using homomorphic encryption in the honest-but-curious model. They, however, do not propose any security and privacy solutions on the biometric templates stored in the database. This issue is mentioned as a future work in their paper. In addition, they do not use threshold encryption which would prevent from  a malicious party aiming to perform decryption by himself. Therefore, their proposed system is open to adversary attacks against the database as stated in their work. They do not address the malicious enrollment issue as well. Moreover, the user must trust the server in their system. Although they achieve better performance than \cite{Erkin, Sadeghi} in terms of bandwidth saving and time efficiency, they do not address the applications where the user and the verifier do not trust each other (e.g. the malicious model).

There are also some works on secure Hamming distance calculation by using cryptographic primitives \cite{Osadchy, Rane, BringerCP13, Kulkarni}. These papers, however, limit their works only with secure Hamming distance calculation. These methods do not address biometric authentication as a whole and fails to satify security, privacy, template protection at the same time by taking into account computational efficiency which is very critical for real-world applications. Osadchy \textit {et al.} \cite{Osadchy} propose Pailler homomorphic encryption based secure Hamming distance calculation for face biometrics. The system is called SCiFI. Although they claim that SCiFI is computationally efficient, it mostly uses pre-computation techniques. Its pre-computation time includes processing time that must be done locally by each user before using the system each time. They report that SCiFI's online running time takes 0.31 seconds for a face vector of size 900 bits however its offline computation time takes 213 seconds. Since these computations should be done by the user just before each attempt to use the system, the protocol is not that much efficient. Besides, SCiFI is only secure for semi-honest adversaries. Rane \textit {et al.} \cite{Rane} also propose secure Hamming distance calculation for biometric applications. However, their proposed method fails to ensure biometric database security since biometric templates are stored in plain format in the database. Thus, a malicious verifier can threaten a user's security and privacy. Bringer \textit {et al.} \cite{BringerCP13} propose a secure Hamming distance calculation for biometric application. The system is called SHADE and it is based on committed oblivious transfer \cite{KSV06}. However, they also cannot guarantee biometric database security since biometric templates are stored in plain form in the database. Kulkarni \textit {et al.} \cite{Kulkarni} propose a biometric authentication system based on \textit{somewhat} homomorphic encryption scheme of  Boneh \textit{et al.}\cite{BGN06} which allows an arbitrary number of addition of ciphertexts but supports only one multiplication operation between the ciphertexts. Although the values stored on the enrollment server are the XORed values of the biometric template vector with the corresponding user's key, the user first extracts and sends her biometric features to the trusted enrollment server. Again this system uses a trusted enrollment server and fails to protect security and to preserve privacy of a user against a malicious database manager. In addition, the system is not efficient since 58 sec are required for successful authentication of a 2048 bit binary feature vector.


\section{Preliminaries}

\subsection{Threshold Homomorphic Cryptosystem}

In this section, we briefly describe underlying cryptographic primitives of the protocols. Given a public key encryption scheme, let  denote its message or plaintext space,  the ciphertext space, and  its randomness. Let  depict an encryption of  under the public key  where  is a random value. Let  be its corresponding private key, which allows the holder to retrieve a message from a ciphertext. The decryption is done with the private key  as .

In a -threshold cryptosystem, the knowledge of a private key is distributed among parties . Then, at least  of these parties are required for successful decryption. On the other hand, there is a public key to perform encryption. More formally, let  be the participants. We define a -threshold encryption scheme with three phases as follows:

\begin{itemize}
  \item In the \emph{\textbf{key generation}} phase, each participant  receives a pair , where  and  are the \emph{shares} of the public and secret key, respectively. Then, the overall public key  is constructed by collaboratively \emph{combining} the shares. Finally  is broadcast to allow anyone to encrypt messages in . The shares of this public key are also broadcast which allow all parties to check the correctness of the decryption process.
  
  \item The \emph{\textbf{encryption}} phase is done as in any public key encryption cryptosystem. If  is the message, a (secret) random value  from  is chosen and  is broadcast under a public key .
	
  \item In the \emph{\textbf{threshold decryption}} phase, given that  (or more) participants agree to decrypt a ciphertext , they follow two steps. First, each participant produces a decryption share by performing , . After broadcasting , they all can apply a reconstruction function  on these shares so that they can recover the original message by performing  where  represent the group of  participants willing to recover . 
\end{itemize}

In case of a -threshold scheme, the additional requirement is that if less than  parties gather their correct shares of the decryption of a given ciphertext, they will get no information whatsoever about the plaintext. In the proposed system, we use the -threshold cryptosystem between the claimer (the user) and the verifier where both players must cooperate to decrypt. 

A public key encryption scheme is said to be additively homomorphic if given  and  it follows that  where  and . There are various versions of threshold homomorphic cryptosystems. The most widely used are ElGamal \cite{Elgamal} or Paillier \cite{Paillier} cryptosystems. In our proposal, we will use a threshold version of Goldwasser-Micali (GM) encryption scheme (i.e., between a user and a verifier) proposed by Katz and Yung in \cite{KY02}. Note that GM scheme is XOR-homomorphic \cite{GM84}, i.e., given any two bits  in , any random values ,    , and any encryptions , it is easy to compute . 

In the proposed protocol, we use a variant of the threshold decryption protocol which is the so-called private threshold decryption \cite{Damgard}. The requirement of this protocol is that one of the  parties will be the only party who will recover the secret. All  other parties follow the protocol and broadcast their shares to achieve this requirement. The party who will learn the plaintext proceeds with the decryption process privately, collects all decryption shares from the  other parties, and privately reconstructs the message. The remaining parties will not get any information about this message.

\subsubsection{Threshold XOR-Homomorphic Goldwasser-Micali Encryption Scheme}
We next give a brief explanation of (2,2)-GM cryptosystem between two users (in our proposal, between a user and a verifier) using a Trusted Dealer. We note that one can also exclude a trusted dealer using the scheme in \cite{KY02}:\\
\textbf{Key generation:}\\
The trusted dealer first chooses prime numbers  and   such that  and      . The dealer next chooses , , ,    such that  and . He sets  =  and    and sends  to the first party and  to the second party. He finally broadcasts .\\
\textbf{Encryption of a bit   :}\\
Choose    and compute a ciphertext .\\
\textbf{Decryption:}\\
All parties compute the Jacobi symbol . If  then all parties stop because either the encryption algorithm was not run honestly or the ciphertext was corrupted during the transmission. (Note that  is always 1, because  =  = 1 (i.e., either  and  or  and ). If  then the first party broadcasts . The second party (who is going to decrypt) will privately compute  and . Finally, the decrypted bit  is computed as . 

Note that it is easy to see whether  is a quadratic residue by computing   . The reason is briefly as follows. We first note that by Euler's theorem  where   . We also know that  is quadratic residue iff . If the Jacobi symbol    = 1 then by using     we have either  and  or  and . If    (resp. -1) and    (resp. -1) then   1  (resp. ) and     (resp. ). Hence, for both cases   1  and    . By the Chinese Remainder Theorem, we have    . Hence,  is quadratic residue iff .

\subsection{Biometric Verification Scheme}

Biometric verification schemes perform an automatic verification of a user based on her specific biometric data (e.g., face, fingerprint, iris). They have two main stages: 1) Enrollment stage, and 2) Authentication stage. The user is enrolled to the system at the enrollment stage. Then, she again provides her biometric data to the system at the authentication stage to prove her identity. Any biometric scheme, which provides binary outputs or whose outputs can be binarized, can work with the proposed threshold homomorphic cryptosystem. The THRIVE system can work with any biometric feature extraction method which produces fixed size vectors as templates and perform verification with distance calculations between the enrolled and the provided template at the authentication stage. When the output of a biometric feature extraction method is not binary, locality sensitive hashing can be used to binarize the feature vector \cite{gionis}. After binarization, the binary templates can successfully be used with the proposed system. In this paper, we use biohashing as an example algorithm for extracting binary biometric templates. Although biohashing has its own security and privacy preservation mechanism, we do not rely on these for the security or the privacy preservation features. Thus it can be replaced with any other binary feature extraction method.


Biohashing schemes are simple yet powerful biometric template protection methods \cite{Karabat, Bai, Kuan, Rathgeb, Lumini}. Biohash is a binary and pseudo-random representation of a biometric template. Biohashing schemes use two inputs: 1) Biometric template, 2) User's secret key. A biometric feature vector is transformed into a lower dimension sub-space using a pseudo-random set of orthogonal vectors which are generated from the user's secret key. Then, the result is binarized to produce a pseudo-random bit-string which is called the biohash. In an ideal case, the Hamming distance between the biohashes belonging to the biometric templates of the same user is expected to be relatively small. On the other hand, the distance between the biohashes belonging to different users is expected to be sufficiently high to achieve higher recognition rates.  

We descibe the random projection (RP) based biohashing scheme proposed by Ngo \textit{et al.} \cite{Ngo}. In this scheme, there are three main steps: 1) Feature extraction, 2) Random projection, 3) Quantization. These steps are explained for face biometrics.

\subsubsection{Feature Extraction}

The feature extraction is performed on the face images, which are collected at the enrollment stage, belonging to the users,  where  and  denotes number of users,  and  denotes number of training images per user. The face images are lexicographically re-ordered and the training face vectors, , are obtained. Then, Principle Component Analysis (PCA) \cite{Turk} is applied. 


where  is the PCA matrix trained by the face images in the training set, \textit{\textbf{w}} is the mean face vector, and  is vector containing PCA coefficients belonging to the  training image of the  user.

\subsubsection{Random Projection}

At this phase, a RP matrix, , is generated to reduce the dimension of the PCA coefficient vectors. The RP matrix elements are independent and identically distributed (\textit{i.i.d}) and generated from a Gauss distribution with zero mean and unit variance by using a Random Number Generator (RNG) with a seed derived from the user's secret key. The Gram-Schmidt (GS) procedure is applied to obtain an orthonormal projection matrix  to have more distinct projections. Finally, PCA coefficients are projected onto a lower -dimensional subspace.


where  is an intermediate biohash vector belonging to the  training image of the  user.

\subsubsection{Quantization}

At this phase, the intermediate biohash vector  elements are binarized with respect to the threshold.

 
where  denotes biohash vector of the  training image of the  user and  denotes the mean value of the intermediate biohash vector . 

A biohash vector, , for the  user is stored in the database at the enrollment stage for verification purpose during the authentication stage. Note that,  can be any vector among  vectors in a real-world application. For simulation purposes, we take into account all possible biohashes for a user by computing . The user is authenticated if the Hamming distance between  and   is below a threshold .


where  denotes the  bit of ,  denotes the  bit of , and  denotes the binary XOR (exclusive OR) operator. Consequently, the verifier decides whether the claimer is a legitimate user or not according the threshold.

\begin{figure*}[tb]
\centering
\begin{center}
\includegraphics [scale=0.43]{enrollment_fig.jpg}
\end{center}
\caption{Illustration of the THRIVE enrollment stage: the user has control over the biometric sensor, the feature extractor and the biohash generator whereas the verifier has control over the database.}
\label{enrollment_fig}
\end{figure*}

\begin{figure*}[htp]
\centering
\fbox{\begin{tabular}{lcl}
\textbf{User ({\sffamily U})} & \hspace{-1.6 cm} \textbf{Verifier ({\sffamily V})}\\
\textbf{\small{Public:}}  \hspace{4 cm}  &\hspace{-0.4 cm} \textbf{\small{Public:}}  \\
\textbf{\small{Private:}} =    & \hspace{-1.6 cm} \textbf{\small{Private:}} \\\\
Compute   \\

\hspace{5cm} \text{\sf Sign}\\
& \hspace{-2.3cm} Verify \& Store \text{\sf Sign}\\

\textbf{\small{Private:}}  & \hspace{-9.3cm} \textbf{\small{Private:}} \\\\

Choose     \\
Compute \\
\\
\hspace{2cm}nonce \hspace{1cm}\\
\\
& \hspace{-5.1cm} Retrieve \text{\sf Sign}\\
Compute \text{\sf Enc}\\
Compute \text{\sf Dec} \\
\\
\hspace{2cm}\text{\sf Sign}noncenonce\\
\\
& \hspace{-8cm} Verify 
\sum^{n}_{j=1}R^{j}_{i} \oplus T^{j}_{i} \leq \mu
  
\textit{T}_i^j = \textit{r}_i^j\oplus\textit{B}^j_{enroll_i}

\textit{R}_i^j = \textit{r}_i^j\oplus\textit{B}^j_{auth_i}
_{V_i}_{sk_{U_i}}\left(\left\langle \tilde{C}^{j}_{i}:j=1,\ldots,n\right)\right\rangle\tilde{\text{nonce}}_{V_i}\tilde{\textbf{B}}_{auth}_{enroll}_{sk_{U_i}}(<\text{\sf Enc}_{pk_i}(r_i^j),T^{1,j}_i:j =1, \ldots, n>, _{U_i}, \tilde{\text{nonce}}_{V_i})_{sk_{U_i}}(<\text{\sf Enc}_{pk_i}(r_i^j),T^{1,j}_i:j =1, \ldots, n>, _{U_i}, \tilde{\text{nonce}}_{V_i})U_i\tilde{C}^{\prime \prime j}_i = _{pk_i}(r_i^j)\cdot \tilde{C}_i^j\tilde{C}^{\prime \prime j}_{i}sk_i^2U_iT_i^{2,j}\tilde{b}_{0} = [\tilde{C}^{\prime \prime j}_{i}]^{{(N - p_{0} - q_{0} +1)}/4} \mod N\tilde{b}_{1} = [\tilde{C}^{\prime \prime j}_{i}]^{{(-p_{1}-q_{1})}/4} \mod Nsk_i^1p_1, q_1\tilde{b}\tilde{C}^{\prime \prime j}_{i}\tilde{b}_2 \mod N \equiv (1- \tilde{2b})/(\tilde{b}_0 \tilde{b_1}) \mod NT_i^{2,j}\tilde{C}^{\prime \prime j}_{i}T_i^{3,j}p_i^0q_i^0\sum^{k}_{j=1} R_i^j \oplus T_i^jV(sk_i^2)pk_{U_i}pk_i_{sk_{U_i}}(<C_i^j:j=1,\ldots,n>))<C_i^j:j=1,\ldots,n>_{sk_{U_i}}(<C_i^j:j=1,\ldots,n>)\tilde{r}_i^j\tilde{B}^j_{auth_i}  \in_R \{0,1\}\tilde{R}_i^j = \tilde{r}_i^j \oplus \tilde{B}^j_{auth_i}r_i^jB^j_{auth_i}\tilde{R}_i^j_{sk_{U_i}}(<C_i^j:j=1,\ldots,n>)_{sk_{U_i}}(<C_i^j:j=1,\ldots,n>)pk_{U_i}V\tilde{C}^{\prime j}_{i} = \text{\sf Enc}_{pk_i}(\tilde{r}_i^j) \cdot C_i^j\tilde{C}^{\prime j}_{i}\tilde{r}_i^j \oplus B^j_{enroll_i}sk_i^2V\tilde{T}_i^{1,j}\tilde{b}_0= [\tilde{C}^{\prime j}_{i}]^{{(N -p_0 -q_0 +1)}/4} \mod N\tilde{b}_2 = [\tilde{C}^{\prime j}_{i}]^{{(-p_2-q_2)}/4} \mod Nsk_i^2p_2, q_2\tilde{b}\tilde{C}^{\prime j}_{i}\tilde{b}_1 \mod N \equiv (1- \tilde{2b})/(\tilde{b}_0 \tilde{b_2}) \mod NT_i^{1,j}\tilde{C}^{\prime \prime j}_{i}\text{\sf Sign}_{sk_{U_i}}(<\text{\sf Enc}_{pk_i}(r^{j}_{i}),T^{1,j}_{i}:j =1, \ldots, n>, _{U_i}, _{V_i})sk_{U_i}sk_{U_i}U_iV(e,n)(p,q,d)n = pqed \equiv 1 \mod (p-1)(q-1)dd_1d_2cU_iVm\equivc^d\equivc^{d_1+d_2} \mod nU_i(pk_{U_i},sk_{U_i})nnC_i^jj = 1, \ldots nnnnEnc_{pk_i}(r_i^j)\cdot C_i^jj = 1, \ldots nn+22nnEnc_{pk_i}(r_i^j)j = 1, \ldots nn3nn2n+22nn3n3.2(2,2)7^{th}$ Framework Research Programme of the European Union (EU), grant agreement number: 284989. The authors would like to thank the EU for the financial support and the partners within the consortium for a fruitful collaboration. For more information about the BEAT consortium please visit http://www.beat-eu.org.

\bibliographystyle{IEEEtran}
\bibliography{references}

\end{document}
