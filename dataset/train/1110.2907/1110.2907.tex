





\documentclass[conference]{IEEEtran}


\usepackage{graphicx}
\usepackage{algorithm,algorithmic}
















\ifCLASSINFOpdf
\else
\fi















































\hyphenation{op-tical net-works semi-conduc-tor}


\begin{document}
\title{System Identification Using Reweighted Zero Attracting Least Absolute Deviation Algorithms}


\author{\IEEEauthorblockN{Fuxi Wen}
\IEEEauthorblockA{School of Electrical and Electronic Engineering
 \\Nanyang Technological University\\Singapore
\\
Email: wenfuxi@ntu.edu.sg}
}










\maketitle


\begin{abstract}
In this paper,  the  norm penalty on the filter coefficients is incorporated in the least mean absolute deviation (LAD) algorithm to improve the performance of the LAD algorithm.  The performance of LAD, zero-attracting LAD (ZA-LAD)  and reweighted  zero-attracting LAD (RZA-LAD) are evaluated for linear time varying system identification under the non-Gaussian (-stable)  noise environments. Effectiveness of the ZA-LAD type algorithms is demonstrated through computer simulations.
\end{abstract}

\begin{IEEEkeywords}
Least mean absolute deviation (LAD), Zero-attracting LAD, Reweighted  zero-attracting LAD, Sparse system identification,  norm, -stable noise
\end{IEEEkeywords}


\section{Introduction}
\label{S1}
Adaptive  filters have been widely used in active noise control, channel equalization, adaptive inverse control, echo cancellation, noise cancellation, linear prediction and system identification.  Least mean squares (LMS), recursive least squares (RLS) and their variations are the widely used  adaptive algorithms \cite{ref1b}-\cite{ref4b}. The least mean squares (LMS) algorithm adjusts the filter coefficients to minimize the least mean squares of the error signal. Compared to recursive least squares (RLS) algorithm, the LMS algorithm has a slower convergence speed, however it does not involve any matrix operations. Therefore, the LMS algorithm requires fewer computational resources and memory than the RLS algorithm. The implementation of the LMS algorithm also is less complicated than the RLS algorithm. 

In many applications, the unknown system response can be assumed to be sparse, which means only a small fraction of the coefficients are different from zero. For example,  sparse echo cancellation \cite{ref20}, internet  telephony, sparse nonlinear channel estimation \cite{ref33}, sparse system estimation \cite{ref34}-\cite{ref35}. Zero-attracting filters exploiting the sparse nature of the system can improve the  performance of the adaptive filter. A similar idea is used in linear prediction of speech in \cite{ref36} to obtain better coding properties by enhancing the sparsity of residual and predictor.  

Motivated by Lasso \cite{ref10} and recent progress in compressive
sensing \cite{ref11},\cite{ref12}, the zero-attracting LMS (ZA-LMS) algorithm and reweighted  zero-attracting LMS (RZA-LMS) were proposed in \cite{ref24},  for sparse system identification.  ZA-LMS is a combination of LMS algorithm with -norm penalty of the coefficient vector. The reweighted -norm minimization algorithm is first proposed in \cite{refrl} to enhance the sparsity of the system. Convergence analysis of ZA-LMS is given in \cite{ref25}. To reduce the eigenvalue spread of the input signal correlation matrix, ZA-LMS and RZA-LMS are extended to the transform domain in \cite{ref26}. 

The LMS type algorithms generally provide a more accurate solution, with less mis-adjustment when the noise is Gaussian, however, they are very sensitive to the outliers. On the other hand, though the convergence rate of the LAD type algorithms is low,  they are robust to the outliers, such as the -stable noise. The -stable noise model is a more generalized model \cite{ref6lmp}-\cite{ref7lmpsi}, such that the Gaussian model can be seen as an special case of -stable model by setting the characteristic exponent .

In this paper, the performance of the least mean absolute deviation (LAD) \cite{ref6lmp}, zero-attracting LAD (ZA-LAD) \cite{ref32} and reweighted  zero-attracting LAD (RZA-LAD) algorithms are evaluated for linear time varying system identification under the non-Gaussian (-stable)  noise environments. 

The rest of the paper is organized as follows. The LAD, ZA-LAD and RZA-LAD algorithms are given in Section II. A brief discussion about the parameter selection is provided in Section III. In Section IV, we evaluate the performance of the three algorithms for linear time varying system identification under Gaussian and non-Gaussian noise environments.  Conclusions of our work and  some further research directions are provided in Section V.

\section {The LAD, ZA-LAD and RZA-LAD algorithms}
\label{S2}
Notations:   is the system input signal at time ,  is the coefficient vector and   is the desire signal at time , The superscripts  denotes the transpose and  denotes the sign function.  and  denote the absolute value and  -norm, respectively.
\subsection {The LAD algorithm}
\label{S2-1}
The output of the LAD algorithm is given by

The cost function is formulated as
\setlength\arraycolsep{0.1em}

where  is the error signal.

Using the stochastic gradient approach, the filter weights are estimated iteratively by

where 

Substituting (\ref{eq_4}) into (\ref{eq_3}), we obtain the updating equation for the LAD algorithm:

where  is the step size that should be chosen carefully to ensure convergence.

The LAD algorithm is summarized in Algorithm \ref{alg1}.
\begin{algorithm}[!h]
\caption{LAD}
\label{alg1}
\begin{algorithmic}
\STATE   Inputs:  
\STATE  Initialization:  
\FOR{} 
\STATE 
\STATE 
\STATE 
\ENDFOR
\end{algorithmic}
\end{algorithm}
\subsection {The ZA-LAD algorithm}
\label{S2-2}
The output of the ZA-LAD algorithm is 

For the ZA-LAD algorithm,  norm penalty is used to explore the sparse nature of the filter coefficients. The cost function is formulated as
\setlength\arraycolsep{0.1em}

where  is the error signal,  is the regularization parameter.

Using the stochastic gradient approach, the filter weights are estimated iteratively by

where 

Substituting (\ref{eq_9}) into (\ref{eq_8}), we obtain the updating equation for the ZA-LAD algorithm:

where  is the stepsize and .

The ZA-LAD algorithm is summarized in Algorithm \ref{alg2}.
\begin{algorithm}[!h]
\caption{ZA-LAD}
\label{alg2}
\begin{algorithmic}
\STATE   Inputs:  
\STATE  Initialization:  
\FOR{} 
\STATE 
\STATE 
\STATE 
\ENDFOR
\end{algorithmic}
\end{algorithm}
\subsection {The RZA-LAD algorithm}
\label{S2-3}
The output of the RZA-LAD algorithm is 


Compared with  norm, log-sum penalty function is a better approximation for the sparsity of the filter coefficients. For the RZA-LAD algorithm, log-sum penalty function is used as the coefficient penalty function. The cost function is formulated as
\setlength\arraycolsep{0.1em}

where  is the error signal,  is the regularization parameter,  is a positive number and  is the number of elements in weight vector .

Using the stochastic gradient approach, the filter weights are estimated iteratively by

and 

where  and the th element of  is given by

Substituting (\ref{eq_14}) into (\ref{eq_13}), we obtain the updating equation for the RZA-LAD algorithm.
The updating equation is given by

where  is the stepsize and  and  is given in (\ref{eq_15A}).

The RZA-LAD algorithm is summarized in Algorithm \ref{alg3}.
\newpage
\begin{algorithm}[!h]
\caption{RZA-LAD}
\label{alg3}
\begin{algorithmic}
\STATE   Inputs:  
\STATE  Initialization:  
\FOR{} 
\STATE 
\STATE 
\STATE 
\ENDFOR
\end{algorithmic}
\end{algorithm}
\section{Parameter Selection - A Brief Discussion}
\label{S3}
The ZA-LAD and RZA-LAD algorithms are regularization based adaptive algorithms.
For the regularization based adaptive algorithms, the cost function is defined by combining the the  norm of the error signal with the  norm penalty of the coefficient vector. 


\subsection{The choice of  ( or  in Algorithm \ref{alg2} or Algorithm \ref{alg3})}
\label{S3-1}
Regularization plays a fundamental role in adaptive filtering, however the better performance is not obtained if the regularization parameter is not 
chosen properly. According to  (\ref{eq_10}) and (\ref{eq_15}), the parameter  denotes the importance of the  norm term or the intensity of attraction. So a large  results in a faster convergence since the intensity of attraction increases as  increases. However, steady-state misalignment increases as  increases. Therefore, the parameter  are determined by the trade-off between adaptation speed and adaptation quality in particular applications. 

One possible way to find the optimal regularization parameter  for four LMS-type algorithms is given in \cite{ref22}. The performance of the regularization based algorithms may be improved by using an adaptive regularization factor. To verify the performance of the LAD type algorithms, a fixed regularization factor is used.

\subsection{The choice of }
\label{S3-2}
We introduce the parameter  in (\ref{eq_12}) in order to provide stability and
to ensure that a zero-valued component in  does not strictly prohibit a
nonzero estimate at the next step. As empirically demonstrated in Section \ref{S4-1}, the RZA-LAD algorithm is robust to the choice of .
As mentioned in \cite{refrl},  should be set slightly smaller than the expected nonzero magnitudes of .
\section{Simulations}
\label{S4}
In this section, the performance of the proposed method is assessed via computer simulations. Two experiments are designed to demonstrate the steady-state performance, convergence rate, and tracking ability of theLAD, ZA-LAD, RZA-LAD, LMS, ZA-LMS and RZA-LAD algorithms. For comparison purposes, we also implement the LMS, RZA-LMS and ZA-LMS algorithms. Mean square deviation (MSD) is taken as a metric, which is defined as 

where  is the coefficient estimated in the th  independent trial and  is the real coefficient of the system.

The noise is non-Gaussian with an -stable distribution. The Matlab -stable distribution toolbox written by Mark S. Veillette is used to generate the noise with -stable distribution. The characteristic exponent , the symmetry parameter , the location parameter  and the dispersion (also called the scale) . 

The six filters (LAD, ZA-LAD, RZA-LAD, LMS, ZA-LMS and RZA-LAD) are run 500 times independently. The parameters are  given in Table \ref{table1}.
\begin{table}[!h]
\caption{Simulation parameters}
\label{table1}
\centering
\begin{tabular}{ccccc}
\hline
Algorithms &  &  &  \\
\hline
LAD & &  & \\
ZA-LAD &  & & \\
RZA-LAD &&  &\\
LMS & &  & \\
ZA-LMS &  & & \\
RZA-LMS &&  &\\
 \hline
\end{tabular}
\end{table}
\subsection {Example 1: Robustness of parameter }
\label{S4-1}
To verify the effect of parameter , the following experiment is considered.  A linear time invariant system with 16 coefficients is considered. Initially, we set the  tap with value 1 and the others to zero, making the system have a sparsity of 1/16. The performance of LAD, ZA-LAD and RZA-LAD with different    is compared when the input signal is a white Gaussian random sequence with variance
of 1 and the generalized signal-to-noise ratio (GSNR) as defined in \cite{refgsnr} is 10 . The average MSD is shown in Fig.\ref{fig1_r}. 
\begin{figure}[!ht]
\centering
\includegraphics[width=0.5\textwidth]{PSel1}
\caption{Robustness of parameter . The input signal is Gaussian, noise is   stable, characteristic exponent  , GSNR = 10dB, 500 independent trials.}
\label{fig1_r}
\end{figure}

From Fig.\ref{fig1_r}, we can see that, the performance of ZA-LAD and RZA-LAD is similar when  is used. It is because for a small , . The update functions for ZA-LAD (\ref{eq_10}) and RZA-LAD  (\ref{eq_15})  are equivalent, then the RZA-LAD algorithm reduces to the conventional ZA-LAD algorithm.
Compared with LAD and ZA-LAD, better steady-state performance is obtained for the RZA-LAD algorithm  when  or . Another observation is that, the  convergence rate of RZA-LAD is slower when a larger  is used. It is because,  is in the denominator part of the third term of the RZA-LAD update function (\ref{eq_15}). A larger  will make the step-size in the third term of (\ref{eq_15}) smaller. The convergence rate is slower due to the relatively smaller step-size. 
\subsection {Example 2: Gaussian Input}
\label{S4-2}
In the second experiment, similar to the simulation setup used in \cite{ref25}, a linear time varying system with 16 coefficients is considered. Initially, we set the  tap with value 1 and the others to zero, making the system have a sparsity of 1/16. After 3000 iterations, all the odd taps are set to 1, while all the even taps remains
to be zero, i.e., a sparsity of 8/16. After 6000 iterations all the even taps are set with value -1 while all the odd taps are maintained to be
1, leaving a completely non-sparse system. 

The performance of LAD, ZA-LAD, RZA-LAD, LMS, ZA-LMS and RZA-LMS is compared when the input signal is a white Gaussian random sequence with variance
of 1 and the generalized signal-to-noise ratio (GSNR) is 10 . The average MSD is shown in Fig.\ref{fig1_g}. 
\begin{figure}[!ht]
\centering
\includegraphics[width=0.5\textwidth]{NG1-2_500LAD}
\caption{Both of the input signal and noise are Gaussian, GSNR = 10dB, 500 independent trials.}
\label{fig1_g}
\end{figure}

From Fig.\ref{fig1_g}, we can see that, when the system is very sparse (before the 3000 iterations),  RZA-LAD yields fastest convergence rate and best steady-state performance compared with LAD and ZA-LAD. And the MSD of the LAD  and ZA-LAD algorithms are similar. After the 3000 iteration, as the number of non-zero efficients increases to eight, a similar performance is obtained for LAD and ZA-LAD, while the RZA-LAD algorithm maintains the best performance among the three algorithms. After 6000 iterations, the system is completely non-sparse, the performance of all the three algorithms are similar.
Since the impulsive noise is considered, all the three LMS type algorithms are not converged.

\subsection {Example 3: Non-Gaussian Input}
\label{S4-3}
The system in the second experiment is the same as the second one, except the switching times are set to the 20000 iteration and 40000 iteration. 

The performance of LAD, ZA-LAD, RZA-LAD, LMS, ZA-LMS and RZA-LMS is compared when the input signal  is now a correlated signal generated by , where  is a white Gaussian noise. The average MSD is shown in Fig.\ref{fig2_ng}. 
\begin{figure}[!ht]
\centering
\includegraphics[width=0.5\textwidth]{NgIn1_2T100}
\caption{The input signal is Gaussian, noise is   stable, characteristic exponent  , GSNR = 10dB, 500 independent trials.}
\label{fig2_ng}
\end{figure}

From Fig.\ref{fig2_ng} we can see that, when the system is very sparse (before the 20000 iterations), both  ZA-LAD and RZA-LAD yield faster convergence rate and better steady-state performance than LAD. And the MSD of the RZA-LAD algorithm is lower than the ZA-LAD algorithm. After the 20000 iteration, as the number of non-zero efficients increases to eight, the performance of the ZA-LAD deteriorated, a similar performance is obtained for LAD and ZA-LAD, while the RZA-LAD algorithm maintains the best performance among the three algorithms. After th 40000 iteration, the system is completely non-sparse, the performance of the LAD and RZA-LAD algorithms are similar.
The convergence rate of the RZA-LAD algorithm is low compared with LAD.
All the three LMS type algorithms are still not converged due to the impulsive noise.
\section{Conclusion}
In this paper, the performance of LAD,  ZA-LAD and  RZA-LAD are evaluated for linear time varying system identification under the Gaussian and non-Gaussian (-stable)  noise environments. The coefficients are updated using fixed stepsize. 

Better performance is obtained for the ZA-LAD type algorithms by exploiting the sparse nature of the system. RZA-LAD performs best among all the three algorithms, even when the system is non-sparse. Furthermore, the ZA-LAD type algorithms are robust to the non-Gaussian noise with -stable distribution. 

There are two potential further research directions, the first one is evaluating the performance of ZA-LAD type algorithms using variable stepsize and automatically adapted regularization parameters to ensure reliable performance across a broad array of signals.
The second one is extending the ZA-LAD type algorithms to nonlinear system such as nonlinear system identification using Volterra filter.



\begin{thebibliography}{00}


\bibitem{ref1b} Widrow, B. and Stearns, S. D. : Adaptive Signal Processing,  Prentice Hall, Upper Saddle River, NJ (1985)

\bibitem{ref2b} Haykin, S. : Adaptive Filter Theory, 4th ed. Prentice Hall, Upper Saddle River, NJ (2002)

\bibitem{ref3b} Sayed, Ali H. : Fundamentals of Adaptive Filtering, Wiley, New York, NJ (2003)

\bibitem{ref4b} Diniz, Paulo S. R. : Adaptive filtering: algorithms and practical implementations, 3rd.ed. Springer, Boston, MA (2008)

\bibitem{ref20} Arenas-Garcia, J.  and Figueiras-Vidal, A.R. : Adaptive combination of proportionate filters for sparse echo cancellation. IEEE Trans. on Audio, Speech, and Language Processing. 17(6): pp. 1087-1098 (2009)

\bibitem{ref33} Kalouptsidis, N. et al. : Adaptive algorithms for sparse nonlinear channel estimation. in Statistical Signal Processing. In Proc. IEEE Workshop on Statistical Signal Processing, Cardiff, Wales, UK: pp. 221-224 (2009)

\bibitem{ref34} Gu, Y. Jin, J. and Mei, S. :  norm constraint LMS algorithm for sparse system identification. IEEE Signal Process. Lett. 16(9): pp. 774-777 (2009)

\bibitem{ref35} Slavakis, K. Kopsinis, Y. and Theodoridis. S. : Adaptive algorithm for sparse system identification using projections onto weighted  balls. In Proc. IEEE ICASSP, Texas, USA : pp. 3742-3745 (2010)  
 
\bibitem{ref36} Giacobello, D. et al. : Enhancing sparsity in linear prediction of speech by iteratively reweighted -norm minimization. In Proc. IEEE ICASSP,  Texas, USA : pp. 4650-4653 (2010) 

\bibitem{ref10} Tibshirani, R. : Regression shrinkage and selection via the {LASSO}. J.
Royal. Statist. Soc B., vol. 58, pp. 267–288 (1996)

\bibitem{ref11} Candes, E. : Compressive sampling. Int. Congress of Mathematics, vol.
3, pp. 1433–1452 (2006)

\bibitem{ref12} Baraniuk, R. : Compressive sensing. IEEE Signal Process. Mag.,
vol. 25, pp. 21–30 (2007)

\bibitem{ref24}  Chen, Y. Gu, Y. and Hero. A.O. : Sparse LMS for system identification. In Proc. IEEE ICASSP, Taipei, Taiwan : pp. 3125-3128 (2009)

\bibitem{refrl} Candes, E. Wakin, M. and Boyd, S. : Enhancing sparsity by reweighted  minimization. Journal of Fourier Analysis and Applications, 14(5): pp. 877-905 (2008)

\bibitem{ref25} Shi K. and Shi, P. : Convergence analysis of sparse LMS algorithms with -norm penalty based on white input signal. Signal Processing, 90(12): pp. 3289-3293 (2010)

\bibitem{ref26} Shi K. and Ma. X. : Transform domain LMS algorithms for sparse system identification. In Proc. IEEE ICASSP, Texas, USA : pp. 3714-3717 (2010)

\bibitem{ref6lmp} Shao M. and Nikias, C.L. : Signal processing with fractional lower order moments: stable processes and their applications. Proc. IEEE, 81(7): pp. 986-1010 (1993)

\bibitem{ref7lmpsi} Binwei, W.  and Barner, K.E. : Nonlinear system identification in impulsive environments. IEEE Trans. Signal Process. 53(7): pp. 2588-2594 (2005) 
 
\bibitem{ref32} OBrien, M.S. Sinclair, A.N. and Kramer. S.M. : High resolution deconvolution using least-absolute-values minimization. In  Proc. IEEE, Ultrasonics Symposium, Honolulu, HI, USA, vol.2, pp. 1151-1156 (1990)

\bibitem{ref22} Benesty, J. Paleologu, C. and Ciochina, S. : On regularization in adaptive filtering. IEEE Trans. on Audio, Speech,
and Language Processing. 19(6): pp. 1734 - 1742 (2010)

\bibitem{refgsnr} Tsakalides, P. and Nikias, C.L.: Maximum likelihood localization of sources in noise modeled as a stable process. IEEE Trans. Signal Process. 43(11): pp. 2700-2713 (1995) 

\end{thebibliography}

\end{document}