\documentclass[copyright,creativecommons]{eptcs}
\usepackage{breakurl}             \usepackage{framed,amsfonts}
\usepackage{color}
\usepackage{url}
\usepackage{graphicx}
\usepackage{rotating}
\usepackage{amsmath}
\newtheorem{lemma}{Lemma}
\newtheorem{theorem}{Theorem}
\newtheorem{definition}{Definition}
\newtheorem{example}{Example}
\newcommand{\eop}{\hspace*{\fill}}
\newcommand{\sem}[1]{[\![ #1 ]\!]}
\newenvironment{proof}{{\it Proof.}\quad}{\eop\vspace*{4mm}}
\def\To{\Rightarrow}
\newcommand{\angl}[1]{\langle #1 \rangle}
\newcommand{\height}[1]{{\sf height}(#1)}
\newcommand{\dom}[1]{{\sf dom}(#1)}
\newcommand{\ntrfc}{\textsc{n-t}^{\text{R}}_{\text{fc}}}
\newcommand{\trfc}{\textsc{t}^{\text{R}}_{\text{fc}}}
\newcommand{\tr}{\textsc{t}^{\text{R}}}
\newcommand{\ttt}{\textsc{t}}
\newcommand{\mtt}{\textsc{mtt}}
\newcommand{\mttlsi}{\textsc{mtt}_{\text{lsi}}}
\newcommand{\mso}{\textsc{mso}}
\newcommand{\xml}{\textsc{xml}}
\newcommand{\gsm}{\textsc{gsm}}
\newcommand{\dgsm}{\textsc{2dgsm}}
\newcommand{\io}{\textsc{io}}
\newcommand{\oi}{\textsc{oi}}
\newcommand{\dol}{\textsc{d0l}}
\newcommand{\hdtol}{\textsc{hdt{\small 0}l}}
\newcommand{\regt}{\textsc{regt}}
\newcommand{\proj}{\textsc{proj}}
\newcommand{\lsi}{\textsc{lsi}}
\newcommand{\tage}{\textsc{tage}}
\newcommand{\dexptime}{\textsc{Exptime}}
\newcommand{\conexptime}{\textsc{co-NExptime}}
\newcommand{\expspace}{\textsc{Expspace}}
\newcommand{\nlogspace}{\textsc{NLogspace}}
\newcommand{\exptime}{\textsc{Exptime}}
\newcommand{\pspace}{\textsc{Pspace}}
\newcommand{\nptime}{\textsc{NPtime}}
\newcommand{\ptime}{\textsc{Ptime}}
\newcommand{\Ptime}{\textsc{Ptime}}
\newcommand{\nat}{I\!\!N}
\newcommand{\neboner}{\text{n-}\textsc{eb}_{\text{1r}}}
\newcommand{\eboner}{\textsc{eb}_{\text{1r}}}
\newcommand{\dtr}{\textsc{dt}^{\text{R}}}
\newcommand{\dt}{\textsc{dt}}
\newcommand{\mttr}{\textsc{mtt}^{\text{R}}}
\newcommand{\dmsott}{\textsc{dmsott}}
\newcommand{\dtrfc}{\textsc{dt}_{\text{fc}}^{\text{R}}}
\def\rank{{\sf rank}}
\def\expand{{\sf expand}}
\def\strip{{\sf strip}}
\def\rhs{{\sf rhs}}
\def\rank{{\sf rank}}
\def\val{{\sf val}}
\title{Equivalence Problems for Tree Transducers:\\A Brief Survey}
\author{Sebastian Maneth
\institute{School of Informatics\\University of Edinburgh}
\email{smaneth@inf.ed.ac.uk}
}
\def\titlerunning{Equivalence Problems for Tree Transducers: A Brief Survey}
\def\authorrunning{S. Maneth}
\begin{document}
\maketitle
\begin{abstract}
The decidability of equivalence for three important classes of tree transducers is discussed. 
Each class can be obtained as a natural restriction of deterministic macro tree transducers (s):
(1)~no context parameters, i.e., top-down tree transducers,
(2)~linear size increase, i.e.,  definable tree transducers, and
(3)~monadic input and output ranked alphabets.
For the full class of s, decidability of equivalence remains a long-standing open problem.
\end{abstract}
\section{Introduction}

The macro tree transducer () was invented independently by
Engelfriet~\cite{eng80,DBLP:journals/jcss/EngelfrietV85}
and Courcelle~\cite{DBLP:journals/tcs/CourcelleF82,DBLP:journals/tcs/CourcelleF82a} 
(see also~\cite{DBLP:series/eatcs/FulopV98}).
As a model of syntax-directed translations, s generalize the
attribute grammars of Knuth~\cite{DBLP:journals/mst/Knuth68}. 
Note that one (annoying) issue of attribute
grammars is that they can be circular; s always terminate.
Macro tree transducers are a combination of context-free
tree grammars, invented by Rounds~\cite{DBLP:conf/stoc/Rounds69} and 
also known as ``macro tree grammars''~\cite{fis68}, and 
the top-down tree transducer of Rounds and 
Thatcher~\cite{DBLP:journals/mst/Rounds70,DBLP:journals/jcss/Thatcher70}:
the derivation of the grammar is (top-down) controlled by a given
input tree. In terms of a top-down transducer, the combination is
obtained by allowing nesting of state calls in the rules
(similar to the nesting of nonterminals in the productions of context-free tree grammars).
Top-down tree transducers generalize to trees the
finite state (string) transducers (also known as ``generalized sequential machines'', 
or s, 
see~\cite{gin66,Berstel79transductionsand}).
In terms of formal languages, compositions of s
give rise to
 a large hierarchy of string languages containing, e.g., the  and  hierarchies
(at level one they include the indexed languages of Aho~\cite{DBLP:journals/jacm/Aho68}),
see~\cite{DBLP:journals/jcss/EngelfrietM02}.
s can be applied in many scenarios, e.g., to type check
 transformations 
(they can simulate the -pebble transducers of Milo, Suciu, and Vianu~\cite{DBLP:journals/jcss/MiloSV03}),
see~\cite{DBLP:journals/acta/EngelfrietM03,DBLP:conf/pods/ManethBPS05,DBLP:conf/icdt/ManethPS07}, or
to efficiently implement streaming XQuery transformations~\cite{icde2014,DBLP:conf/aplas/NakanoM06}.
In terms of functional programs, s are particularly simple
programs that do primitive recursion over an input tree and 
only produce trees as output. Applications in programming 
languages exist 
include~\cite{DBLP:journals/scp/Vogler91,DBLP:phd/de/Voigtlander2005,DBLP:conf/aplas/NakanoM06}.

Equivalence of nondeterministic transducers is undecidable,
already for restricted string transducers~\cite{DBLP:journals/jacm/Griffiths68}. 
We therefore only consider deterministic transducers.
What is known about the equivalence problem for 
deterministic macro tree transducers?
Unfortunately not much in the general case.
Only a few subcases are known to be decidable.
Here we describe three of them:
\begin{enumerate}
\item[(1)] top-down tree transducers
\item[(2)] linear size increase transducers
\item[(3)] monadic tree transducers.
\end{enumerate}
The first one was solved long ago by Esik~\cite{DBLP:journals/actaC/Esik81}, but
was revived through the ``earliest canonical normal form''
by Engelfriet, Maneth, and Seidl~\cite{DBLP:journals/jcss/EngelfrietMS09}. 
The latter implies 
equivalence check for total top-down tree transducers. 
The second one is solved by the decidability of equivalence 
for deterministic  tree transducers of Engelfriet and Maneth~\cite{DBLP:journals/ipl/EngelfrietM06}.
The same authors have shown that every  of linear size increase
is effectively equal to an  transducer~\cite{DBLP:journals/siamcomp/EngelfrietM03}. Hence,
decidability of equivalence follows for s of linear size increase.
Here we give a direct proof using s.
The third result is about s over monadic trees.
These are string transducers with copying. 
The decidability of their equivalence problem follows
through a relationship with L-systems~\cite{DBLP:journals/jcss/EngelfrietRS80};
in particular, with the sequence equivalence problem of  systems.
The latter was first proved decidable by Culik~II and Karumh{\"a}ki~\cite{culkar86}.

\section{Preliminaries}

We deal with finite, ordered, ranked trees.
In such a tree, each node is labeled by a symbol from
a ranked alphabet such that the rank of the symbol is equal
to the number of children of the node.
Formally, a \emph{ranked alphabet} consists of a finite set 
together with a mapping  associating
to each symbol its rank. 
We write  to denote that the rank of  is equal to .
By  we denote the subset of symbols of  that have
rank .
Let  be a ranked alphabet.
The \emph{set of all trees over },
denoted , is the smallest set of strings  such that if
, , and , then also
.
For a tree of the form  we simply write .
For a set , we denote by  the set of all trees over
 such that the rank of each  is zero.
Let  be distinct symbols in  and 
let  such that none of the leaves in  are
labeled by  for .
Then by 
we denote the \emph{tree substitution} that replaces each leaf labeled
 by the tree .
Thus  denotes the tree
.

Let  for some ranked alphabet .
We denote the \emph{nodes of } by their Dewey dotted decimal path and
define the set  of nodes of  as

if  with , , and
. Thus,  denotes the root node,
and  denotes the th child of the node .
For a node  we denote by  its label, and,
for a tree  we denote by  
the tree obtained from  by replacing its subtree rooted at 
by the tree .
The \emph{size} of , denoted , is its number  of nodes.
The \emph{height} of , denoted , is defined as
 for
, 
, 
and .

We fix the set of \emph{input variables}  and the set
of \emph{formal context parameters} . 
For  we define  and .

A \emph{deterministic finite-state bottom-up tree automaton} is a tuple
 where  is a finite set of states, 
 is the set of final states, and for every 
and ,  is a function from  to .
The transition function  is extended to a mapping from  to  in 
the obvious way, and the set of trees accepted by  is
.

\section{Macro Tree Transducers}

\begin{definition}\rm
A \emph{(deterministic) macro tree transducer}  is a tuple 
such that  is a ranked alphabet of states with , 
 and  are ranked alphabets of input and output
symbols, respectively,  is the initial state, and
for every , , and , , 
the set of rules  contains at most one rule of the form

where  such that
if a node in  has its label in , then its first child has 
its label in . A rule as above is called a -rule and 
its right-hand side is denoted by .
\end{definition}

The translation  (often just denoted )
realized by an   is a partial function recursively defined as follows.
For each state  of rank , 
is the translation of  starting in state , i.e.,
``starting'' with a tree  where .
For instance, for ,  simply
equals .
In general, for an input tree ,
 is obtained from  by repeatedly replacing
a subtree of the form  
with 
by
the tree .
The latter tree will also be written 
.
We define .

By the definition above, all our s are deterministic 
and we refer to them as ``macro tree transducers'' and
denote their class of translations by .
An  is \emph{total} if there is exactly one rule of the above form. 
If each state of an  is of rank one, 
then it is a \emph{top-down tree transducer}.
The class of translations realized by (deterministic) 
top-down tree transducers is denoted by .
In Lemmas~\ref{lm:nParikh} and~\ref{lm:ab} we 
make use of nondeterministic top-down tree transducers
and mark the corresponding class there by the letter ``N''.
A transducer is nondeterministic if there are several 
-rules in . An  is \emph{monadic} if its
input and output ranked alphabets are monadic, i.e., only contain
symbols of rank  and .
An   is of \emph{linear size increase} if there is a constant  
such that  for every . 

As an example, consider the transducer  consisting of the rules 
in Figure~\ref{fig:mtt}.
\begin{figure}[htb]

\caption{The macro tree transducer  adds reverse Dewey paths to leaves}\label{fig:mtt}
\end{figure}
The alphabets of this transducer are
,
, and 
.
The   translates a binary tree into the same tree, but additionally 
adds under each leaf the reverse (Dewey) path of the node.
For instance,  is translated into 

as can be verified in this computation of 

Note that  is \emph{not} of linear size increase. 
Hence, it is not MSO definable.
Since the translation is neither top-down nor monadic, it falls into
a class of s for which we do not know a procedure to decide
equivalence.
As an exercise, the reader may wonder whether there is an  that
is similar to , but outputs Dewey paths below leaves (instead of their \emph{reverses}).
In contrast,
consider input trees with only exactly one -leaf (and all other leaves labeled differently).
Then an  of linear size increase can output under the unique -leaf its 
reverse Dewey path, i.e., this translation is  definable. Is it now possible
with an  to output the non-reversed Dewey path?

One of the most useful properties of s is the effective preservation
of regular tree languages by their inverses.
This is used inside the proofs of several results presented here.
For instance, to prove that for every  there is an equivalent one
which is nondeleting in the parameters (= strict), one can use the above
property as follows: given a state  of rank  and a subset
, the language  is regular, and hence
 is the regular set of inputs for which  outputs
only parameters in the set . Thus, by using regular look-ahead
we can determine which parameters are used, and can change the rules 
to call an appropriate state  which is only provided the parameters
in  which it uses. Regular look-ahead is explained in Section~\ref{sect:td}.

The following result is stated in Theorem~7.4 of~\cite{DBLP:journals/jcss/EngelfrietV85}.
A similar proof as below is given at the end of~\cite{DBLP:journals/acta/EngelfrietM03}.
A slightly simpler proof, for a slightly larger class, is presented
by Perst and Seidl for macro forest transducers~\cite{DBLP:journals/ipl/PerstS04}.

\begin{lemma}\rm\label{lm:inv}
Let  be an  with output alphabet  and let
 be a regular tree language (given by a
bottom-up tree automaton ). Then  is effectively regular.
In particular, the domain  is effectively regular.
\end{lemma}
\begin{proof}
Let  and .
We construct the new automaton .
The states of  are mappings  that associate with 
each state  a mapping . 
The set  consists of all  such that .
For  we define
 such that 
for every  and , 
.
Here,  is the extension of  to trees in 
by the rule  for all .
Now let  with .
For  we define
 where
for every  and , 
.
Now,  is the extension of  of above 
to symbols  by 
 for
every .
\end{proof}

\subsection{Bounded Balance}

Many algorithms for deciding equivalence of transducers are based
on the notion of bounded balance. 
Intuitively, two transducers have bounded balance if 
the difference of their outputs on any ``partial''
input is bounded by a constant.
For instance, for two  systems  with ,
Culik~II defines~\cite{DBLP:journals/tcs/Culik76} the balance of a string  as the
difference of the lengths of  and . He shows that
equivalence is decidable for  systems that have bounded difference.
In a subsequent article~\cite{DBLP:journals/iandc/CulikF77}, 
Culik~II and Fris show that any two equivalent  systems
in normal form have bounded difference, thus giving the first solution to the
famous  equivalence problem.

For a tree transducer  a partial input is an input tree which contains
exactly on distinguished leaf labeled , and  is a fresh symbol
not in . More precisely, a partial input is a tree
 where  is in the domain of  and  is a node of .
Since the transducer has no rules for , the computation on the
input  ``blocks'' at the -labeled node. Thus, the tree 
may contain subtrees of the form  where  is a
state of rank . 
For instance, consider the transducer  shown in the left of
Figure~\ref{fig:MN} and consider the partial input tree
. Then

It should be clear that if we replace each  by  so
that  is a tree with ,
then we obtain the output  of  on input tree .
For instance, we may pick  above; then we obtain
 which indeed equals .

Consider the trees  and  for two s  and .
What is the balance of these two trees?
There are two natural notions of balance: either we compare the 
sizes of , or we compare their heights.
Given two trees  we define their \emph{size-balance} 
(for short, \emph{s-balance}) as  and 
their \emph{height-balance} (for short, \emph{h-balance}) as
.
Two transducers  have \emph{bounded s-balance} (resp. h-balance)
if there exists a  such that for any partial input 
the s-balance (resp. h-balance) of  and  is 
at most .
Obviously, bounded h-balance implies bounded s-balance, but not vice versa.

Let  and  be equivalent s.
Do  and  have bounded size-balance?
To see that this is in general \emph{not} the case, it suffices to consider
two simple top-down tree transducers:
 translates a monadic partial input of the form 
into the full binary tree of height  containing 
occurrences of the subtree . The transducer  translates 
 into the full binary tree of height  with
 occurrences of the subtree .
\begin{figure}[htb]

\caption{Equivalent top-down tree transducers  and 
with unbounded size-balance}\label{fig:MN}
\end{figure}
The rules of  and  are shown in Figure~\ref{fig:MN}.
Clearly,  and  are of bounded height-balance (with constant ).
But, their size-balance is not bounded.

In a similar way it can be seen that equivalent s need
not have bounded height-balance.
This even holds for monadic s:
Consider the transducers  and  with rules shown
\begin{figure}[htb]

\caption{Equivalent monadic macro tree transducers  and 
with unbounded height-balance}\label{fig:mon}
\end{figure}
in Figure~\ref{fig:mon}.
Their height-balance on input  is equal to ; for instance,
for the tree  we obtain

which are trees of height  and , respectively.
We may verify that replacing  by  results in equal trees:

And for  we obtain that


Let us now show that equivalent top-down tree transducers 
have bounded height-balance.

\begin{lemma}\rm\label{lm:bd}
Equivalent top-down tree transducers
effectively have bounded height-balance.
\end{lemma}
\begin{proof}
Let  be equivalent top-down tree transducers 
with sets of states , respectively.
Note that they have the same domain .
Let  and .
Consider the two trees 
 for .
Let  be a smallest input tree such that 
.
It should be clear that the height of  is bounded
by some constant . 
In fact, let  be the height of a smallest tree in the
set ,
for any subsets  and . 
Since  is in such a set, its height is at most .
This bound  can be computed because by Lemma~\ref{lm:inv}
the sets  and 
are effectively regular, and regular
tree languages are effectively closed under intersection~\cite{tata07}.
In fact, it is not difficult to see that we can choose
.
Hence, there is a constant  such that 
 for any 
appearing in . 
Clearly we can take , where  is the maximal height
of the right-hand side of any rule of  and .
This means that  because
 and the substitutions

increase the height of  by at most .
\end{proof}

If the transducers  of Lemma~\ref{lm:bd} are
total, then  and  is the maximal size of the right-hand
side of any rule for an input leaf symbol, i.e., 
.

Let us consider an example of two equivalent 
top-down tree transducers  and  with
\begin{figure}[htb]

\caption{Equivalent top-down tree transducers  and }\label{fig:diff}
\end{figure}
output paths of different height.
The rules of  and  are given in Figure~\ref{fig:diff}.
Let us consider the input tree . We omit some
parentheses in monadic input trees.
We obtain

Similarly, for the transducer  we obtain

As the reader may verify, if  is replaced by the leaf , then
indeed the output trees  and  are the same, i.e.,
the transducers are equivalent.
Let us compare the trees  and .
On the one hand, the transducer  is ``ahead'' of the transducer
 in the output branch . It has already produced a -node at
that position, while  has not (and is in state  at that position).
On the other hand,  is ahead of  at two other positions in the
output: at the node  the transducer  has produced an -node
already, while  at that node is in state , and,
at node  the transducer  has output an -node, while
also here  is in state .

\section{Decidable Equivalence Problems}\label{sect:equiv}

\subsection{Top-Down Tree Transducers}\label{sect:td}

It was shown by {\'E}sik~\cite{DBLP:journals/actaC/Esik81} 
that the bounded height-difference 
of top-down tree transducers can be used to decide equivalence. 

\begin{theorem}\rm(\cite{DBLP:journals/actaC/Esik81})\label{theo:Esik}
Equivalence of top-down tree transducers is decidable.
\end{theorem}
\begin{proof}
We follow the version of the proof given by Engelfriet~\cite{eng80}.
\begin{figure}[htb]
\centerline{\input joost.pdf_t}
\caption{Two equivalent top-down tree transducers}\label{fig:joost}
\end{figure}
Consider two equivalent top-down tree transducers  and .
By Lemma~\ref{lm:bd} they have bounded height-balance by some constant .
Consider the trees  and
. An ``overlay'' of these two trees is
shown in Figure~\ref{fig:joost} (this is a copy of Figure~10 of~\cite{eng80}).
At the node where  is in state , the transducer  has
already produced the tree , i.e., at this node 
 is ``ahead'' of  by the amount .
Similarly, at the -labeled node,  is ahead of 
by the amount .
Clearly, the height of  and  is bounded by . 
Hence, there are only finitely many such trees  and .
We can construct a top-down tree automaton  which in its states
keeps track of all such ``difference trees''  and , while
simulating the runs of  and . It checks if the outputs are
consistent, and rejects if either the outputs are different or if 
the height of a difference tree is too large.
Finally, we check if  accepts the language ; this is
decidable because  is regular by Lemma~\ref{lm:inv}, and equivalence
of regular tree languages is decidable (see~\cite{tata07}).
\end{proof}

Note that {\'E}sik~\cite{DBLP:journals/actaC/Esik81} shows that
even for single-valued (i.e., functional) nondeterministic top-down
tree transducers, equivalence is decidable. 
It is open whether or not equivalence is decidable for
-valued nondeterministic top-down tree transducers
(but believed to be decidable along the same lines as for bottom-up
tree transducers~\cite{sei2014}, cf. the text below Theorem~\ref{theo:bu}).
A top-down tree transducer is letter-to-letter if the right-hand
side of each rule contains exactly one output symbol in .
It was shown by Andre and Bossut~\cite{DBLP:journals/tcs/AndreB98}
that equivalence is decidable for nondeleting \emph{nondeterministic} letter-to-letter
top-down tree transducers.
An interesting generalization of Theorem~\ref{theo:Esik} is given
by Courcelle and Franchi-Zannettacci~\cite{DBLP:journals/iandc/CourcelleF82}.
They show that equivalence is decidable for ``separated'' attribute grammars which are 
evaluated in two
independent phases: first a phase that computes all inherited
attributes, followed by a phase that computes all synthesized attributes
(top-down tree transducers are the special case of synthesized
attributes only).

\bigskip

{\bf Top-down Tree Transducers with Regular Look-Ahead.}\quad
Regular look-ahead means that the transducer ( or ) comes with
a (complete) deterministic bottom-up automaton (without final
states), called the ``look-ahead
automaton of ''. A rule of the look-ahead transducer is of the form

and is applicable to an input tree  only if 
the look-ahead automaton recognizes  in state  for all .   
Given two top-down tree
transducers with regular look-ahead , we can transform them into
ordinary transducers (without look-ahead)  such that the resulting
transducers are equivalent if and only if the original ones are.
This is done by changing the input alphabet so that 
for every original input symbol  of rank , 
it now contains the symbols 

for all possible look-ahead states  of  and  of .
Thus, for every , the 
new input alphabet has -many symbols.
It is easy to see that Lemma~\ref{lm:bd} also holds 
for transducers with look-ahead, by additionally requiring
that the domain  of the  is intersected with all input trees that represent
correct runs of the look-ahead automata. 
Thus, Theorem~\ref{theo:Esik} also holds for top-down tree transducers with look-ahead. 



\begin{theorem}\rm\label{theo:dtr}
Equivalence of top-down tree transducers with
regular look-ahead is decidable.
\end{theorem}


\medskip

{\bf Canonical Normal Form.}\quad
Consider two equivalent top-down tree transducers 
and let  be their domain. 
As the example transducers  and  with rules
in Figure~\ref{fig:diff} show, 
for a partial input tree , 
there may be positions in the output trees
where  is ahead of , and other positions
where  is ahead of . 
Such a scenario is also depicted in Figure~\ref{fig:joost}.

We say that  is \emph{earlier} than ,
if for every  and , 
the tree  is a prefix of the tree .
A tree  is a prefix of a tree  if for every 
with  it holds that . 
The question arises whether for every top-down translation there is an
equivalent unique earliest transducer  such that
 for ; we call such a transducer
a \emph{canonical transducer}.
The question was answered affirmative by
Engelfriet, Maneth, and Seidl~\cite{DBLP:journals/jcss/EngelfrietMS09}.
We only state this result for total transducers.

\begin{theorem}\rm\label{theo:EMS}
Let  be a total top-down tree transducer. An equivalent canonical 
transducer can be constructed in polynomial time. 
\end{theorem}
\begin{proof}
The canonical transducers are top-down tree transducers 
without an initial state, but with 
an \emph{axiom tree} . 
This means that the translation on input tree 
starts with the tree  
(instead of  for ordinary transducers).

Starting with , we define its axiom .
In a first step, an earliest transducer is constructed:
if there is a state  and an output symbol  (of rank )
such that  for every input symbol ,
then  is \emph{not} earliest. 
Intuitively, the symbol  should be produced earlier, at
each call of the state . Thus, the construction 
replaces  in all right-hand sides (and in the axiom ) by 

where the  are new states.
For every ,  
is defined as the -th subtree of the root of 
 -- beware, this right-hand side may have
changed due to the replacement above. 
Finally we remove  and its rules.
This step is repeated until it cannot be applied anymore. 
In this case  has become earliest and 
no  and  exists such that
 for all input trees  of . 
It should be clear that the earliest step can be carried
out in polynomial time.
In the second step, equivalent states are merged to obtain the canonical transducer;
the corresponding 
equivalence relation on states is computed using
fixed point iteration in cubic time (with respect to the size of ).
It is computed in such a way that if , then .
\end{proof}

Note that the availability of a canonical (``minimal'') transducer
has many advantages. For instance, it makes possible to formulate a
Myhill-Nerode like theorem which, in turn, makes possible Gold-style
learning of top-down tree transducers (in polynomial time),
as shown by Lemay, Maneth, and Niehren~\cite{DBLP:conf/pods/LemayMN10}.

Consider the two transducers  and  with the rules given
in Figure~\ref{fig:diff}. To construct a canonical equivalent transducer
for  according to Theorem~\ref{theo:EMS}, we observe that state
 of  is \emph{not} earliest: the root equals  for 
the right-hand sides of all -rules. We replace 
 by  in the -rule, and
introduce the two rules  and
. We remove  and have obtained 
an earliest transducer. The canonical transducer is constructed
by realizing that the states  and  are equivalent
and hence can be merged. The rules of the canonical transducer can
thus be given as

To construct the canonical transducer for , we observe that state 
is not earliest: the root equals  in both rules. 
We thus replace  everywhere by  where
 are new states. After this replacement, the 
current -rule is:

Thus, new -rules are
 and
.
The -rules are  and .
The resulting transducer is earliest.
We now compute that  and that .
We merge these pairs of states and obtain the same transducer
(up to renaming of states) as the canonical one of  above.
Hence,  and  are equivalent.

As a consequence of Theorem~\ref{theo:EMS}
we obtain that equivalence of total top-down
tree transducers can be decided in polynomial time. 

\begin{theorem}\rm\label{theo:complex}
Equivalence of total top-down tree transducers can be
decided in polynomial time.
\end{theorem}

The earliest normal form has also certain ``disadvantages''.
For instance, it does not preserve linearity (or nondeletingness) of the
transducer. Consider for  the rules

and .
When making earliest, these rules are removed and a rule such as 
 is replaced by

which is non-linear. It is also deleting:
.


\subsection{Bottom-Up Tree Transducers}

As a corollary of Theorem~\ref{theo:dtr} we obtain that
also for deterministic bottom-up tree transducers,
equivalence is decidable.
This follows from the fact that every deterministic bottom-up tree transducer
can be transformed into an equivalent deterministic top-down tree transducer
with regular look-ahead~\cite{DBLP:journals/mst/Engelfriet77}.

\begin{theorem}\rm\label{theo:bu}
Equivalence of deterministic bottom-up tree transducers is decidable.
\end{theorem}
\begin{proof}
A deterministic bottom-up tree transducer is a 
tuple 
where  is the set of final states and  contains
for every , , and 
at most one rule of the form 
 where
 is a tree in .
We construct in linear time a deterministic bottom-up 
tree automaton which for every rule
as above has the transition .
This automaton serves as the look-ahead automaton of a top-down
tree transducer with the unique state .
For a rule as above, the transducer has the rule

It should be clear that the resulting top-down tree transducer 
with look-ahead  (which has only the single state )
is equivalent to the given bottom-up tree transducer .
\end{proof}

The equivalence problem for bottom-up tree transducers was first solved
by Zachar~\cite{DBLP:journals/actaC/Zachar80}.
It was shown by Seidl~\cite{DBLP:journals/tcs/Seidl92} 
that equivalence can be decided in polynomial time for
single-valued (i.e., functional) nondeterministic bottom-up tree transducers. 
Note that this also follows from Theorem~\ref{theo:j} and the (polynomial time)
construction in the proof of Theorem~\ref{theo:bu}.
This result was extended to finite-valued nondeterministic bottom-up
tree transducers by Seidl~\cite{DBLP:journals/mst/Seidl94}.
For nondeterministic letter-to-letter bottom-up tree transducers, equivalence
was shown decidable by Andre and Bossut~\cite{DBLP:conf/tapsoft/AndreB95};
such transducers contain exactly one output symbol in the right-hand side
of each rule.
They reduce the problem to the equivalence of bottom-up relabelings which
was solved by Bozapalidis~\cite{DBLP:journals/tcs/Bozapalidis92}.
For deterministic bottom-up tree transducers the effective existence of a
canonical normal form, similar in spirit to the earliest normal
form of top-down tree transducers, was shown by Friese, Seidl, 
and Maneth~\cite{DBLP:journals/ijfcs/FrieseSM11}. They show that this normal
form can be constructed in polynomial time, if each state of the given
transducer produces either none or infinitely many outputs; hence, 
equivalence is decidable in polynomial time for such transducers.
Friese presents in her PhD thesis~\cite{Friese10Thesis} a Myhill-Nerode theorem
for bottom-up tree transducers.

\subsection{Linear Size Increase mtts}

It was shown by Engelfriet and Maneth~\cite{DBLP:journals/siamcomp/EngelfrietM03}
that total deterministic mtts of linear size increase 
characterize the total deterministic
 definable tree translations. In fact, even any composition of total deterministic
mtts, when restricted to linear size increase, is equal to an  definable translation,
as shown by Maneth~\cite{DBLP:conf/fsttcs/Maneth03}.
The  definable tree translations are a special instance of the
 definable graph translations, introduced by Courcelle and
Engelfriet, see~\cite{DBLP:books/daglib/0030804}.
Decidability of equivalence for deterministic  graph-to-string translations 
on a context-free graph language was proved by Engelfriet and Maneth~\cite{DBLP:journals/ipl/EngelfrietM06}.
It implies decidable equivalence also for  tree translations.
We present a proof of the latter here that only uses s and avoids going through . 

The idea of the proof stems from Gurari's proof~\cite{DBLP:journals/siamcomp/Gurari82} of
the decidability of equivalence for .
In a nutshell: the ranges of all the above translations are Parikh.
A language is \emph{Parikh} if its set of Parikh vectors is equal to the
set of Parikh vectors of a regular language. 
Let  be an alphabet.
The \emph{Parikh vector} of a string  
is the -tuple  of natural numbers  such that
for ,  equals the number of occurrences of  in .
For a language that is Parikh, it is decidable whether or not it contains
a string with Parikh vector  for some natural number .
This property is used to prove equivalence as follows.
Given two tree-to-string transducers  we first change  to 
produce a new end marker DM_1M_2a,bw/mmwna^nb^n\in L^{a,b}L^{a,b}nM_1, M_2abnM_1, M_2\ntrfcL^{a,b}abmnMcs\in T_\Sigmau\in V(s)q(x)qMM(s[u\leftarrow x])\leq c\ntrfctytXyXq(x_i)\regty\ntrfc(\regt)My\ntrfcR\in\regtx_iM'\dom{M'}=\dom{M}M'(s)M(s)s\in\dom{M}MMc\langle q_1,\dots, q_n\ranglen\leq cq_iM'(R)y\regtMkns'=s[u\leftarrow x]s\in T_\Sigmau\in V(s)M(s')\leq kqm+11\leq j\leq ms\in T_\Sigmay_jM_q(s)\leq n\mttMc|\tau_M(s)|\leq c\cdot |s|s\in T_\Sigma\mtt\mttlsi(y\mttlsi)\subseteq y\trfcMy_jq(q,\sigma)y\mttry\try\mttM_q(s)m+1w_jm+1qw_0,y_1w_1,\dots,y_mw_my_jM_q(s)M_1, M_2y\trfc\Sigma\Deltaa,b\in\Deltaa\not=bD\subseteq T_\SigmaL^{a,b}=\{a^m\# b^n\mid \exists s\in D: M_1(s)/m=a, M_2(s)/n=b\}Q_1, Q_2M_1,M_2M_1, M_2q_0p_0y\ntrfcM_1'Q_0=Q_1\cup \{q_a\mid q\in Q_1\}q_{0,a}M_1q(\sigma(x_1,\dots,x_k))\to w\quad\langle\cdots\rangleM_1w=uavq_a(\sigma(x_1,\dots,x_k))\to ua\quad\langle\cdots\ranglew=uq'(x_i)vq_a(\sigma(x_1,\dots,x_k))\to uq_a'(x_i)\quad\langle\cdots\rangleM_1'y\ntrfcM_1''\DeltaaM_1'M_2''M_2''(s)=\{b^n\mid M_2(s)/n=b\}y\ntrfcMM(s)=\{ a^m\# b^n\mid M_1(s)/m=a, M_2(s)/n=b\}\{ r_0\}\cup Q_1' \cup Q_2'r_0MM_1M_2MM_1''M_2''\sigma\in\Sigma^{(k)}k\geq 0q_{0,a}(\sigma(x_1,\dots,x_k))\to u\quad\langle q_1',\dots,q_k'\ranglep_{0,b}(\sigma(x_1,\dots,x_k))\to w\quad\langle p_1',\dots,p_k'\rangleMM(s)M_1''(s)\# M_2''(s)M(D)=L^{a,b}L^{a,b}M_1, M_2\mttM_i\dom{M_i}DM_iM_1M_2y\trfc\DeltaM_i\ symbol.
This can easily be done by first splitting the initial state  so that it appears
in the right-hand side of no rule, and then adding q_0M_1M_2a,b\in\Deltaa\not=bs\in DnM_1(s)/n=aM_2(s)/n=bL^{a,b}E=\{a^n\# b^n\mid n\in\nat\}L^{a,b}E\cap LELL^{a,b}\mtt\hdtol\mttqm+11\leq j\leq m\sigmay_j\rhs(q,\sigma)\mttM=(Q,\Sigma,\Delta,q_0,R)Q=Q^{(2)}\cup Q^{(1)}\Sigma^{(0)}=\Delta^{(0)}=\{\bot\}(q,\bot)\hdtol\hdtol\Sigma\Deltaw_1,w_2\in\Sigma^*h_j,g_j:\Sigma^*\to\Sigma^*1\leq j\leq nh,g:\Sigma^*\to\Delta^*k\geq 01\leq i_1,\dots,i_k\leq n\mtt\hdtols=a_1(\cdots a_n(e)\cdots)\strip(s)a_1\cdots a_n\mttM_1=(Q_1,\Gamma,\Pi,q_0,R_1)M_2=(Q_2,\Gamma,\Pi,p_0,R_2)Q_1Q_2Q=Q_1\cup Q_2\hdtol\Sigma, \Delta\Sigma=\Pi^{(1)}\cup Q\Delta=\Pi^{(1)}h_a, g_aa\in\Gamma^{(1)}\pi\in\Pi^{(1)}h_a(\pi)=g_a(\pi)=\piq\in Qq\in Q_1h_a(q)=\strip(\rhs_{M_1}(q,a))h_a(q)=qq\in Q_2g_a(q)=\strip(\rhs_{M_2}(q,a))g_a(q)=qt\in T_{\Pi\cup Q}(X_k\cup Y_m)\strip\strip(\pi(t))=\pi\cdot\strip(t)\pi\in\Gamma^{(1)}\strip(q(x_1,t))=q\cdot\strip(t)q\in Q^{(2)}\strip(q(x_1))=qq\in Q^{(1)}\strip(\bot)=\strip(y_1)=\epsilon\cdoth, gh(q)=\strip(\rhs_{M_1}(q,\bot))q\in Q_1h(q)=qg(q)=\strip(\rhs_{M_2}(q,\bot))q\in Q_2g(q)=qw_1=q_0w_2=p_0\hdtols=a_1(\cdots a_n(\bot)\cdots)\in T_\Gammah(h_{a_n}(\cdots h_{a_1}(w_1)\cdots))=\strip(M_1(s))g(g_{a_n}(\cdots g_{a_1}(w_2)\cdots))=\strip(M_2(s))\hdtolM_1M_2D\subseteq T_\GammaM_1(s)=M_2(s)s\in DDA\Gamma^{(1)}A=(R,\Gamma^{(1)},r_0,\delta,R_f)r\in Ra\in\Gamma^{(1)}\delta(r,a)Rr_0R_f\subseteq R\Sigma=\Pi^{(1)}\cup Q\Sigma'=\{\langle r,b\rangle\mid r\in R, b\in\Sigma\}\Delta=\Pi^{(1)}\hdtol\Sigma'\Deltaa\in\Gamma^{(1)}r\in Rr'=\delta(r,a)\pi\in\Pi^{(1)}h_a(\langle r,\pi\rangle)=g_a(\langle r,\pi\rangle)=\langle r',\pi\rangleq\in Q_1p\in Q_2h_a(\langle r,p\rangle)=\langle r,p\rangleg_a(\langle r,q\rangle)=\langle r,q\rangleg, hr\in R_fh(\langle r,q\rangle)=\strip(\rhs_{M_1}(q,\bot))g(\langle r,p\rangle)=\strip(\rhs_{M_2}(p,\bot))h(\langle r,b\rangle)=bg(\langle r,b\rangle)=br\not\in R_fh(\langle r,b\rangle) = g(\langle r,b\rangle) = \epsilonb\in\Sigmaw_1=\langle r_0,q_0\ranglew_2=\langle r_0,p_0\rangles=a_1(\cdots a_n(\bot)\cdots)\in T_\Gamma1\leq j\leq n\delta^*(r_0,a_1\cdots a_j)=rAra_1\cdots a_jgM_2As\not\in Du_1=h_{a_n}(\cdots h_{a_1}(w_1)\cdots)u_2=g_{a_n}(\cdots g_{a_1}(w_2)\cdots)r\not\in R_fh(u_1) = g(u_2) =\epsilons\not\in Ds\in Du_ih(u_i)=\strip(M_i(s))t=a_1(\cdots a_n(e)\cdots)\expand(t)a_1(\cdots a_n(e(\bot))\cdots)\mttrM\mttrN\tau_N=\{(\expand(s),\expand(t))\mid (s,t)\in\tau_M\}M\mttrN\Sigma'=\Sigma^{(1)}\cup\{a'^{(1)}\mid a\in\Sigma^{(0)}\}\Delta'=\Delta^{(1)}\cup\{a'^{(1)}\mid a\in\Delta^{(0)}\}\Sigma^{(1)}NMq\in Qa\in\Sigma^{(0)}\rhs(q,a)N\expand(s)N\mtt\mttM_1,M_2\SigmaA_1, A_2M_1, M_2M_1M_2M_1M_2D\mttN_1, N_2P_1,P_2A_1,A_2N_i\Sigma'=\{\langle\sigma,p_1,p_2\rangle\mid\sigma\in\Sigma^{(1)},p_1\in P_1, p_2\in P_2\}\langle\sigma, p_1,p_2\ranglep_1p_2(q,\langle\sigma,p_1,p_2\rangle)N_1(q,\sigma)\langle p_1\rangleM_1(q,\langle\sigma,p_1,p_2\rangle)N_2(q,\sigma)\langle p_2\rangleM_2N_1N_2tT_{\Sigma'}\gamma(t)T_\Sigma\langle \sigma,p,p'\rangle\sigmaE\subseteq T_{\Sigma'}ts=\gamma(t)DtA_1stA_2sN_iN_1N_2EM_1M_2\ttt\ttt\ttt\expspace\ttt\nlogspaceMs,tt=M(s)vtusvM(s[u\leftarrow x])\Deltavv\DeltaM[s\leftarrow a(x,\dots,x)]a=s[u]\expspace\nlogspaceM_1M_2svt_1=M_1(s)t_2=M_2(s)t_1[v]\neq t_2[v]vM_1M_2u_1u_2ssu_1u_2t_1t_2vvt_1t_2su_1u_2t_1t_2M_1M_2M_1M_2s't_1t_2M_1M_2t_1t_2\dexptimenA_1,\dots,A_n\dexptime\SigmaA_iM_1=(\{q_0,\dots, q_n\},\Sigma,\Sigma\cup\{\delta^{(n)}\},q_0,R)A_iq_iRM_{1,q_i}=\{(s,s)\mid s\in L(A_i)\}\sigma\in\Sigma\geq 1e\Sigma^{(0)}RM_2=(\{p\},\Sigma,\Sigma,p,\{p(e)\to e\})\tau_{M_2} = \{(e,e)\}L(A_i)s\in T_\SigmaM_{q_i}(s)i\in\{1,\dots, n\}\tau_{M_1}=\{(e,e)\}M_1,M_2sM(s)=\delta(s,s,\dots, s)M_1M_2L(A_i)\msoL^{a,b}a^nb^nL^{a,b}Aa^nb^nL(A)\nptime\conexptime\pspace\pspace\mso\dexptime\dexptime\Ptime\Ptime\Ptime\mtt\mtt\mtt\mtt\mtta^mb^nmanba^nb^na\not= b\mtt\hdtol\hdtol\mtt\tr\ttt\tr$s~\cite{DBLP:journals/corr/EngelfrietMS13}.

\bibliographystyle{eptcs}
\bibliography{bib} 
\end{document}
