

In this section, we explore a parameterized complexity approach
\cite{DowneyFellows1999,FlumGrohe2006,Niedermeier2006} for the
\textsc{Mod/Resc Parsimony Inference} problem. Due to the equivalence shown in the previous section, we focus
for convenience reasons on \textsc{Bipartite Biclique Edge
Cover}. In parameterized
complexity, problem instances are appended with an additional
parameter, usually denoted by , and the goal is to find an
algorithm for the given problem which runs
in time , where  is an arbitrary
computable function. In our context, our goal is to determine
whether a given input bipartite graph  with  vertices
has a biclique edge cover of size  in time .

\subsection{The kernelization}\label{s:sskern}

Fleischner \emph{et
al.}~\cite{FleischnerMujuniPaulusmaSzeider2009} studied the
\textsc{Bipartite Biclique Edge Cover} problem in the context
of parameterized complexity. The main result in their paper is to
provide a kernel for the problem based on the techniques given
by Gramm \emph{et al.}~\cite{GrammGuoHuffnerNiedermeier2006}
for the similar \textsc{Clique Edge Cover} problem.
Kernelization is a central technique in parameterized
complexity which is best described as a polynomial-time
transformation that converts instances of arbitrary size to
instances of a size bounded by the problem parameter (usually of
the same problem), while mapping ``yes''-instances to
``yes''-instances, and ``no''-instances to ``no''-instances. More
precisely, a \emph{kernelization algorithm}  for a
parameterized problem (language)  is a polynomial-time
algorithm such that there exists some computable function
, such that, given an instance  of , 
produces an instance  of  with:
\begin{itemize}
\item , and
\item .
\end{itemize}

We refer the reader to
\emph{e.g.}~\cite{GuoNiedermeier2007,Niedermeier2006} for more
information on kernelization.

A typical kernelization algorithm works with reduction rules,
which transform a given instance to a slightly smaller
equivalent instance in polynomial time. The typical argument
used when working with reduction rules is that once none of
these can be applied, the resultant instance has size bounded
by a function of the parameter. For the \textsc{Bipartite
Biclique Edge Cover}, two kernelization rules have been applied
by Fleischner \emph{et
al.}~\cite{FleischnerMujuniPaulusmaSzeider2009}:\\

\noindent \underline{\textsf{RULE 1}}: If  has a vertex with
no neighbors, remove this vertex without changing the
parameter.

\noindent \underline{\textsf{RULE 2}}: If  has two vertices
with identical neighbors, remove one of these vertices without
changing the parameter.

\begin{lemma}[\cite{FleischnerMujuniPaulusmaSzeider2009}]
\label{Lemma: Kernel}Applying rules 1 and 2 of above exhaustively gives a
kernelization algorithm for \textsc{Bipartite Biclique Edge
Cover} that runs in  time, and transforms an instance
 to an equivalent instance  with  and .
\end{lemma}

We add two additional rules, which will be necessary
for further interesting properties.

\noindent \underline{\textsf{RULE 3}}: If there is a vertex 
with exactly one neighbor  in , then remove both  and
, and decrease the parameter by one.

\begin{lemma}\label{lemma: lrul3}
Rule 3 is correct.
\end{lemma}

\begin{proof}
Assume a biclique cover of size  of the graph, and assume
that vertex  is a member of some of the bicliques in this
cover. By definition, at least one of the bicliques covers the
edge 
. Since this is the only edge 
adjacent to , the bicliques that cover
 include only vertex  among the vertices in its
bipartite vertex class.
If the bicliques do not cover all the
edges of , add them to each of the bicliques. \qed
\end{proof}


\noindent \underline{\textsf{RULE 4}}: If there is a vertex 
in  which is adjacent to all vertices in the opposite
bipartition class of , then remove  without decreasing
the parameter.

\begin{lemma}\label{Lemma: lrul4}
Rule 4 is correct.
\end{lemma}

\begin{proof}
After applying rule 3 above, each remaining vertex in the graph
has at least two neighbors. Assume a biclique cover of size 
of all the edges except those adjacent to vertex .
Assume w.l.o.g. that . Since each vertex
 has degree at least 2, it is adjacent to an edge
which is covered by the biclique cover. It therefore belongs to
some biclique in this cover. For each biclique in the cover,
add now vertex  to its set of vertices. Since  is
adjacent to all the vertices of 
, each changed component is a correct biclique and the
new solution covers all the edges, including those of vertex
, and is of same size. \qed
\end{proof}

Regarding the time complexity of the new rules we introduced,
it is clear that once a vertex has been found in which a rule
should be applied, applying each rule takes  time. Thus,
including the time necessary to find such a vertex, the time
required for each rule is . 
Since one can apply the
reduction rules at most  time, the total time required
for our extended kernelization remains . We remark that
although 
the new rules do not change the kernelization size, which
remains  vertices in a solution of size , they will be
useful in the following section.


\subsection{\textsc{Bipartite Biclique Edge Cover} and \textsc{Clique Edge Cover}}\label{s:ssrbeccec}

In this section, we show the connection between the \textsc{Bipartite Biclique
Edge Cover} and the \textsc{Clique Edge Cover} problems. We show that in the context of
fixed-parameter tractability, we can easily translate our problem to
the classical clique covering problem and then use it for a solution to our problem. For instance, it gives another way for the kernelization of the problem and can provide interesting heuristics, mentioned in~\cite{GrammGuoHuffnerNiedermeier2006}. 

Given a kernelized bipartite graph  as an instance to the
\textsc{Bipartite Biclique Edge Cover} problem, we transform
 into a (non-bipartite) graph  defined by 
and .

\begin{theorem}\label{t:qmqh}
The edges of  can be covered with  cliques iff the edges
of  can be covered with  cliques.
\end{theorem}

\begin{proof}
Suppose  is a biclique edge cover of . Then
each , , induces a clique in
. Furthermore, the only remaining edges which are not
covered in  are the ones between vertices in  and
, which can be covered by the two cliques induced by
these vertex sets in . Altogether this gives us 
cliques that cover all edges in . Conversely, take a clique
edge cover  of . Due to the fourth
kernelization rule, we know that there is no vertex in 
which is connected to all vertices in , and vice-versa,
in both  and . It follows that there must be at least
two cliques in , say  and , with
 and . Thus,
there is a subset of the cliques in  which
have vertices in both partition classes of , and which cover
all the edges in . Taking the corresponding bicliques in
, and adding duplicated bicliques if necessary, gives us  bicliques
that cover all edges in . \qed
\end{proof}




\subsection{Algorithms}\label{s:sssa}

After the kernelization algorithm is applied, the next step
is usually to solve the problem using brute-force. This is what is done
in~\cite{FleischnerMujuniPaulusmaSzeider2009}. However, the
time complexity given there is inaccurate, and the
parametric-dependent time bound of their algorithm is
 instead of the  bound 
stated in their paper. Furthermore, the algorithm they describe is
initially given for the related \textsc{Bipartite Biclique Edge
  Partition} problem (where each edge is allowed to appear exactly once in a
biclique), and the adaptation of such algorithm to the \textsc{Bipartite Biclique
Edge Cover} problem is left vague and imprecise.
Here, we suggest two possible brute-force procedures for the \textsc{Bipartite Biclique
Edge Cover} problem, each of which outperforms the algorithm
of~\cite{FleischnerMujuniPaulusmaSzeider2009} in the worst-case. We assume
throughout that we are working
with a kernelized instance obtained by applying the algorithm described in Section~\ref{s:sskern}, \emph{i.e.} a pair  where  is a bipartite graph with
at most  vertices (and consequently at most  edges).

\paragraph{The first brute-force algorithm:}
For each , try all possible partitions of the
edge-set  of  into  subsets. For each such
partition , check whether each of the
subgraphs  is a biclique, where
 is the subgraph of  induced by . If yes, report
 as a solution. If some  is
not a biclique, check whether edges in 
can be added to  in order to make the graph a biclique.
Continue with the next partition if some graph in
 cannot be appended in this way in
order to get a biclique, and otherwise report the solution
found. Finally, if the above procedure fails for all partitions
of  into  subsets, report that  does not have a
biclique edge cover of size .

\begin{lemma}
\label{Lemma: Correctness}
The above algorithm correctly determines whether  has a
bipartite biclique edge cover of size~ in time
.
\end{lemma}

\begin{proof}
Correctness of the above algorithm is immediate in case a
solution is found. To see that the algorithm is also correct
when it reports that no solution can be found, observe that for
any biclique edge cover  of , the set
 with  defines a partition of  (with some of the 's
possibly empty), and given this partition, the algorithm above
would find the biclique edge cover of . Correctness of the
algorithm thus follows.

Regarding the time complexity, the time needed for appending
edges to each subgraph is at most , and thus
a total of  time is required for the entire partition. The 
number of possible partitions of  into  disjoint set
is the \emph{Stirling number of the second kind} ,
which has been shown in~\cite{Korshunov1983} to be
asymptotically equal to
. Thus, the 
total complexity of the algorithm is . 
\qed
\end{proof}


\paragraph{The second brute-force algorithm:}
We generate
the set  of all possible inclusion-wise maximal
bicliques in , and try all possible -subsets of
 to see whether one covers all edges in .
Correctness of the algorithm is immediate
since one can always restrict oneself to using only
inclusion-wise maximal bicliques in a biclique
edge cover. To generate all maximal
bicliques, we first transform  into the graph  given in
Theorem~\ref{t:qmqh}. Thus, every
inclusion-wise maximal biclique in  is an inclusion-wise
maximal clique in . We then use the algorithm
of~\cite{TsukiyamaIdeAriyoshiShirakawa1977} on the complement
graph  of , \emph{i.e.} the graph defined
by  and .

\begin{theorem}
The \textsc{Bipartite Biclique Edge Cover} problem can be
solved in  time, where .
\end{theorem}

\begin{proof}
Given a bipartite graph  as an instance to \textsc{Bipartite
Biclique Edge Cover}, we first apply the kernelization algorithm
to obtain an equivalent graph  with  vertices, and then apply the brute-force algorithm
described above to determine whether  has a biclique edge
cover of size . Correctness of this algorithm follows
directly from Section~\ref{s:sskern} and the correctness of
the brute-force procedure. To analyze the time complexity of
this algorithm, we first note that Prisner showed that any
bipartite graph on  vertices has at most 
inclusion-wise maximal
bicliques~\cite{TsukiyamaIdeAriyoshiShirakawa1977}. This
implies that . The
algorithm of~\cite{Prisner2000} runs in
 time, which is
. Finally, the total
number of -subsets of  is
, and checking whether each of these subsets
covers the edges of  requires 
time. Thus, the total time complexity of the entire algorithm
is  \qed
\end{proof}

It is worthwhile mentioning that some particular bipartite
graphs have a number of inclusion-wise maximal bicliques, which
is polynomial in the number of their vertices. For these types
of bipartite graphs, we could improve on the worst-case analysis
given in the theorem above. For instance, a bipartite chordal
graph  has at most  inclusion-wise maximal
bicliques~\cite{TsukiyamaIdeAriyoshiShirakawa1977}. A bipartite
graph with  vertices and no induced cocktail-party graph of
order  has at
most  inclusion-wise maximal
bicliques~\cite{Prisner2000}. The cocktail party graph of order 
is the graph with nodes consisting of two rows of paired nodes in which all nodes but the paired ones are connected with a graph edge
(for a full definition, see~\cite{Prisner2000}). Observing that the algorithm in
Section~\ref{s:sskern} preserves cordiality and does not
introduce any new cocktail-party induced subgraphs, we obtain
the following corollary:

\begin{corollary}
The \textsc{Bipartite Biclique Edge Cover} problem can be
solved in  time when restricted to
chordal bipartite graphs, and in 
time when restricted to bipartite graphs with no induced
cocktail-party graphs of order .
\end{corollary}
