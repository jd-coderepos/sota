\documentclass[a4paper,11pt]{article}

\usepackage{fullpage}
\usepackage{latexsym}
\usepackage{amsmath}
\usepackage{amsthm}
\usepackage{graphicx}

\newtheorem{theorem}{Theorem}[section]
\newtheorem{lemma}[theorem]{Lemma}

\title{Maximum Flow in Directed Planar Graphs with\\Vertex Capacities}
\author{Haim Kaplan~\thanks{The Blavatnik School of Computer Science, 
Tel Aviv University, 69978 Tel Aviv, Israel, 
{\small \texttt{\{haimk,yahav.nussbaum\}@cs.tau.ac.il}}}\and Yahav Nussbaum~}

\date{}

\begin{document}

\maketitle

\begin{abstract}
    In this paper we present an   algorithm for  finding
    a maximum flow in a directed planar graph, where the vertices are subject to capacity constraints, in addition to the arcs.
    If the source and the sink are on the same face,
    then our algorithm can be implemented in   time.

    For general (not planar) graphs, vertex capacities do not make the problem more difficult,
    as there is a simple reduction that eliminates vertex capacities.
    However, this reduction does not preserve the planarity of the graph.
    The essence of our algorithm is a different reduction that does preserve the planarity,
      and can be implemented in linear time.
     For the special case of undirected planar graph,
      an algorithm with the same time complexity was recently claimed, but we show that it has a flaw.
\end{abstract}

\section{Introduction}

The problem of finding maximum flow in a graph  is a well-studied
problem with applications in many fields, see the book of Ahuja,
Magnanti and Orlin~\cite{AMO93} for a survey. The maximum flow
problem is also interesting if we restrict it to \emph{planar}
graphs, which are graphs that have an embedding in the plane without
crossing edges. The case of planar graphs appears in many
applications of the problem, for example road traffic or VLSI
design. The special structure of planar graphs allows us to get simpler
and more efficient algorithms for the maximum flow and related problems.

In the maximum flow problem, usually the arcs  of the graph have
capacities which limit the amount of flow that may go through each
arc. We study a version of the problem in which the vertices of the
graph also have capacities, which limit the amount of flow that may
enter each vertex. This version appears for example when
computing vertex disjoint paths in graphs, and  in other problems
where the vertices model objects which have a capacity.

Ford and Fulkerson~\cite[Chapter I.11]{FF62} studied this version of
the problem. They suggested the following simple reduction to
eliminate vertex capacities. We replace every vertex  with a
finite capacity  by two vertices  and . The arcs that were
directed into  now enter , and the arcs that were directed out of
 now leave . We also add a new arc with capacity  from
 to . Unfortunately, this reduction does not preserve the
planarity of the graph \cite{KN94}. Consider for example the
complete graph of four vertices, where one of the vertices has
finite capacity. This graph is planar. If we apply the construction
of Ford and Fulkerson we get a graph whose underlying undirected graph
is the compete graph with 5 vertices. This graph
is not planar by Kuratowski's Theorem.

The most efficient algorithm for  maximum flow in directed planar
graphs without vertex capacities, to date, was given by Borradaile
and Klein~\cite{BK} (Weihe~\cite{W97} gave an  algorithm with the
same time bound but assuming  certain connectivity condition on the
graph). Their paper also contains a survey of the history of the
maximum flow problem on planar graphs. The time bound of the
algorithm of \cite{BK} is  where  is the number of
vertices in the input graph. Borradaile and Klein ask whether their
algorithm can be generalized to the case where the flow is subject
to vertex capacities.

A planar  graph is a \emph{-planar} graph if the source and the
sink are on the same face. Hassin~\cite{H81} gave an algorithm for
the maximum flow problem in directed -planar graphs without
vertex capacities. The bottleneck of the algorithm is the
computation of single-source shortest-path distances, which takes
 time in a planar graph, using the algorithm of Henzinger et
al.~\cite{HKRS97}.

 Khuller and Naor~\cite{KN94} were the first to study the  problem of maximum flow with vertex
capacities in planar graphs. They gave various results, including an
 time algorithm for finding the value of the
maximum flow in -planar graphs (which can be improved to 
time using the algorithm of \cite{HKRS97}), an  time
algorithm for finding the maximum flow in -planar graphs, an
 time algorithm for finding the value of the maximum
flow in undirected planar graph, and an  time
algorithm for the same problem on directed planar graphs. If all
vertices have unit capacities, then we get the \emph{vertex-disjoint
paths problem}. Ripphausen-Lipa et al.~\cite{RLW97} solved this
problem in  time for undirected planar graph.

Recently, Zhang, Liang and Chen~\cite{ZLC08}, used a construction
similar to the one of \cite{KN94} to obtain a maximum flow for
undirected planar graph with vertex capacities that runs in  time. Their algorithm constructs a planar graph without vertex
capacities, and then uses the algorithm of \cite{BK} to find a
maximum flow on it, which is modified in  time to a
flow in the original graph with vertex capacities.
They also gave an  time algorithm for undirected
-planar graphs. Zhang et al.\ also ask in their paper if there
is an algorithm that solves the problem for directed planar graph.

In this paper we answer \cite{BK} and \cite{ZLC08},  and show a
linear time reduction of the problem of finding maximum flow in
directed planar graphs with arc and vertex capacities, to the
problem of finding maximum flow in directed graphs with only arc
capacities. This problem is more general than the one for undirected
planar graphs, since an undirected planar graph can be viewed as a
special case of a directed planar graph, in which there are two
opposite arcs between any pair of adjacent vertices.

We show how to apply the constructions of \cite{KN94} and \cite{ZLC08}
to directed planar graphs.
Given directed planar graph, we construct another directed planar graph
without vertex capacities, such that we can transform a maximum flow
in the new graph  back to a maximum flow in the original graph.
Since the new graph does not have vertex capacities we can find a
maximum flow in it using  the algorithm of \cite{BK} (or of
\cite{H81} if it is an -planar graph). The time bound of our
reduction is linear in the size of the graph, therefore we show that
vertex capacities do not increase the time complexity of the maximum
flow problem also for planar graphs.



In addition, we show that the algorithm of \cite{ZLC08}
unfortunately has a flaw. We give an undirected graph in  which this
algorithm does not find a correct maximum flow. Therefore, in fact our
algorithm is also the first to solve the problem for undirected
graphs.

The outline of the paper is as follows: In the next  section we give
some background and terminology. In Section \ref{sec:cut} we
describe the construction of \cite{KN94} that we use, and in Section
\ref{sec:ext} we describe the one of \cite{ZLC08}. In Section
\ref{sec:acy} we characterize when a maximum flow in the constructed graph
induces a maximum flow in the original graph. In Section
\ref{sec:canc} we show how to efficiently find such a flow in the
constructed graph.  Finally, in the last section we stick all the
pieces together to get our algorithm.

\section{Preliminaries}

We consider a simple directed planar graph  , where 
is the set of vertices and  is the set of arcs, with a given
planar embedding. The planar embedding of the graph  is
represented combinatorially, see \cite{NC88}
for survey on planar graphs. An arc  is directed
from  to . We denote the number of vertices by
, since  the graph is planar we have .

A \emph{path}  is a sequence of arcs  such that for  we have . If in addition  then  is a \emph{cycle}. We say that a path  \emph{contains} a vertex , if either  or  is in , for some vertex . The path  \emph{starts} at  and \emph{ends} at .
For ,  is the set of incoming arcs and  is the set of outgoing arcs.

The graph  has two distinguished vertices,  is the
\emph{source} and  is the \emph{sink}. The source  has no
incoming arcs, and the sink  has no outgoing arcs. Every arc ,
has a capacity , and in addition every vertex  has a capacity . A capacity might be . We assume that the
source and the sink have no capacities, if we wish to allow them to
have capacities, we can add a vertex  that will be the source
instead of , and an arc  with the desired capacity, and
similarly add a new sink , and an arc  with the desired
capacity. Note that this transformation keeps the graph planar, and
even -planar if it was so. It
is easy to extend the given embedding to accommodate , ,
and the arcs  and . A graph without vertex
capacities can be viewed as a special case in which 
for every vertex.

A function  is a \emph{flow function} if and only if
it satisfies the following three constraints:

Constraints \eqref{con:ec} are the \emph{arc capacity
  constraints}, Constraints \eqref{con:vc} are the
\emph{vertex capacity constraints} and Constraints \eqref{con:0}
are the \emph{flow conservation constraints}.

We say that  \emph{carries flow into}  if , and that  \emph{carries flow out of}  if .

The \emph{value} of a flow  is , the amount of flow which enters the sink. If the value of  is  then  is a \emph{circulation}. Our goal, in the \emph{maximum flow problem}, is to find a  flow function of maximum value.

For a flow function  we define a
 cycle   to be
a \emph{flow-cycle} if  for every arc in
.
We extend this definition to every function , even if it is not a flow.
If a function  has no flow-cycles we say that  is {\em acyclic}.
 An \emph{acyclic flow} is a  flow function
which is acyclic.

Let  we denote .
For a flow function , we may assume that  does not contain an arc  such that
both  and , because otherwise the flows in both directions can cancel each other.
For a path  we let .

The planar embedding of  partitions the plane into connected
regions called .
For a simpler description of our algorithm, we fix an embedding of 
such that  is on the boundary of the infinite face.
It is easy to convert any given embedding to such an embedding \cite{NC88}.

The
\emph{dual graph}  of  has a vertex  for every  face 
of , and an arc
 for every arc  of . The arc  connects the two vertices
corresponding to the faces incident to . The arc  is directed from
the vertex that corresponds to the face on the
left side of  to the one of the face on the right side of .
Intuitively,  is obtained from  by turning the arcs clockwise.
The dual graph  is planar, but it is may have loops or parallel arcs.
Every face  of  corresponds to a vertex  in , such that the
arcs that bound  are dual to the arcs that are incident to .
See Figure \ref{fig:Gd}.
The capacity of , , is interpreted in  as the \emph{length}
of .

\begin{figure}
    \centering
    \includegraphics{cut}
    \caption{A planar graph and its dual graph. The vertices of  are dots, and its arcs are solid. The vertices of  are circles, and its arcs are dashed. The bold arcs are an arc-cut in  and a cut-cycle in . Capacities are not shown in this figure.}
    \label{fig:Gd}
\end{figure}

In the construction we present below we add undirected edges to
directed graphs.
Each such undirected edge  can be represented by
two antiparallel
directed arcs  and , with the same capacity.
If  is
an undirected edge, then  is also undirected.

\subsection{Residual cycles} \label{sec:rcyc}

In this section we present the algorithm of Khuller, Naor and Klein~\cite{KNK93}
that finds a circulation without clockwise residual cycles, in a directed
planar graph. The complete details of the algorithm can be found in \cite{KNK93}
and also in \cite{BK}.
We also present an extension of this algorithm that changes a given flow into
another flow, with the same value, without clockwise residual cycles.
We use this algorithm later to get the linear time bound for our reduction.
This algorithm of \cite{KNK93} was given for directed planar graphs
without vertex capacities, so for this section assume that the graph 
does not have vertex capacities.

In this section we also assume that if  then also .
This assumption can be
satisfied, without changing the problem, by adding the arc  with
capacity  for every arc  such that  is not in .

Let  be a flow in .
The \emph{residual capacity} of an arc  with respect to  is defined as
. In other words, the residual capacity of 
is the amount of flow that we can add to , or reduce from .
The \emph{residual graph} of  with respect to  has the same vertex set
and arc set as , and the capacity for each edge  is .
A \emph{residual arc} with respect to  is an arc  with a positive
residual capacity.
A \emph{residual path} is a path made of residual arcs. A \emph{residual cycle} is a
cycle made of residual arcs.

Given embedded planar graph  with arc capacities,
 Khuller et al.~\cite{KNK93} showed how to
find a circulation  in , such that there are no
clockwise residual cycles with respect to .
Their procedure is as follows.

Let  be
the the infinite face of . Find
the shortest path distances from  to every vertex of .
Define a \emph{potential function}  that assigns to every face
 of , the shortest distance from  to
 in . Let  be an arc of  with face  to its left
and face  to its right. If  then
 (otherwise it is implied that
 gets the value  and ).

The function  satisfies the two constraints of a flow \cite{KNK93} (without vertex
capacities). The edge capacity constraints are satisfied, because for each ,
the arc  connects  to , so  is
at most the length of , which is the capacity of .
The flow conversion constraints are satisfied because for every cycle in ,
the sum of  for the arcs of the cycle is  \cite{H81,J87}, and the
vertices which incident to a specific vertex in  form a cycle in .
Therefore,  is a circulation.

The correctness of this algorithm follows from the fact that any clockwise
cycle  in  encloses some face . The shortest path from 
to  in  must contain an arc 
dual to an arc  of . Because  is in the shortest paths tree, the
algorithm assigns to  the value of the length of , which is
the same as the capacity of . Therefore the residual
capacity of  with respect to  is ,
and  is not a residual arc with respect to . Therefore,  is not
a residual cycle with respect to .

The bottleneck of the algorithm of \cite{KNK93} is the
computation of single-source shortest-path distances from the vertex
 in the dual graph. Henzinger
et al.~\cite{HKRS97} showed how to find these distances in a planar
graph in  time, so the algorithm of \cite{KNK93}
can be implemented in the same time bound.

Given a flow  in , we wish to find a flow  with the same value, such
that  does not have clockwise residual cycles. We can use the algorithm
of \cite{KNK93} which we described above for this problem.

Let  be the residual graph of  with respect to . We find a
circulation  in , such that  does not have clockwise
residual cycles with respect to , using the algorithm of \cite{KNK93}. Define
 to be the sum of  and , that is .
In other words, we add to  the flow of , and let the flows
on  and  to cancel each other.

The function  satisfies the two constraints of a flow without vertex
capacities. The capacity of an arc
 in  is  and therefore  is
smaller than this capacity, therefore , so  and the arc capacity constraints are satisfied in .
The conservation constraints are satisfied, because these constrains
are satisfied for  and for , and  is the sum of these two flows.

The value of the flow  is the sum of the values of  and . Since
 is a circulation, its value is , and so the value of  is the
same as the value of .

The flow  has the desired property that it has no clockwise residual
cycles. To show that, we show that if  is a clockwise residual cycle in
 with respect to , then  is also a residual cycle in 
with respect to , contrary to the way we find .
Let  be an arc of , and assume for contradiction
that  is not residual in  with respect to .  From our
assumption  and .
Therefore,  and  is not residual in  with respect to , in
contradiction to the fact that it is a member of .

\begin{lemma}
	Let  be a directed planar graph without vertex capacities, and let  be a flow in . We can find a flow  in , with the same value as , such that  does not have clockwise residual cycles, in  time.
\end{lemma}


\section{Minimum cut} \label{sec:cut}

In a graph without vertex capacities, a \emph{cut}  is a minimal subset of  such that every path from  to  contains an arc in . To avoid ambiguity later, when we introduce cuts that may contain vertices, we call such a cut an \emph{arc-cut}. See Figure \ref{fig:Gd}. The \emph{value} of an arc-cut  is . The \emph{minimum cut problem} asks to find an arc-cut of minimum value. The fundamental connection between maximum flow and the minimum cut problems was given by Ford and Fulkerson~\cite{FF62} in the Max-Flow Min-Cut Theorem:
\begin{theorem}\cite{FF62} \label{thm:mfmc}
    The value of the maximum flow (in a graph without vertex capacities) is equal to the value of the minimum arc-cut in the same graph.
\end{theorem}

Let  be a cycle in . We say that  is a \emph{cut-cycle} if
it separates the faces corresponding to  and , and goes
counterclockwise around  (or equivalently, clockwise around ).
See Figure \ref{fig:Gd}. The length of  is the sum of the lengths of its
 arcs. Johnson~\cite{J87} showed the following relation between
the value of minimum arc-cut and the value of shortest cut-cycle:
\begin{lemma} \cite{J87} \label{lem:mcc}
    Let  be a directed planar graph without vertex capacities. Then the value of the minimum arc-cut of , is the same as the length of the shortest cut-cycle in .
\end{lemma}

Ford and Fulkerson~\cite[Chapter I.11]{FF62} extended the definition of cuts to graphs
with vertex capacities. In such a graph, a cut  is a minimal subset
of  such that every path from  to  contains an arc or
a vertex in . The value of a cut  is similarly defined as
. Ford and Fulkerson also presented a version of the Max-Flow Min-Cut Theorem
for graphs with vertex capacities, in this case the value of maximal flow (subject to both arc and
vertex capacities) is equal to the value of the minimum cut (which contains both arcs and vertices).

Khuller and Naor~\cite{KN94} extended Lemma \ref{lem:mcc}
using a supergraph  of 
which they construct as follows.
Let  be a face of  that corresponds to a vertex  of  with finite
capacity. We add
 a new vertex  inside 
 and connect it by an (undirected) edge of length 
 to every vertex on the boundary of . See Figure \ref{fig:Gc}.

\begin{figure}
    \centering
    \includegraphics{Gc}
    \caption{Construction of  and  for the graph in Figure \ref{fig:Gd}.
    The graph  is presented as the dual graph of . The newly added (undirected) edges are without arrowheads. Capacities are not shown in this figure.}
    \label{fig:Gc}
\end{figure}

\begin{lemma} \cite{KN94} \label{lem:Gc}
    The values of the maximum flow and minimum cut in  are
equal to the length of the shortest cut-cycle in .
\end {lemma}

\section{The extended graph} \label{sec:ext}

 Zhang, Liang and Jiang~\cite{ZLJ06} and Zhang, Liang and Chen~\cite{ZLC08} construct the \emph{extended graph} for an undirected graph
with vertex capacities. We use the same construction for directed
planar graphs with vertex capacities.
The extended graph is defined as follows.
We replace every vertex
 which has a
 finite capacity with  vertices , where 
is the degree of . We connect every  to  with an (undirected)
edge of capacity of .
We make every arc that was adjacent
to , adjacent to some vertex  instead, such that each arc is
connected to a different vertex , and the clockwise order of the arcs
is preserved. We identify the new arc  or  with
the original arc  or . The resulting graph is denoted
by , and the cycle that replaces  in  by .
 The graph  is a simple directed planar graph
without vertex  capacities.
The arc set of  contains the arc set of .
See Figure \ref{fig:Gc}.

From the construction of  and  follows that  is the
dual of . Let  be a vertex with finite capacity and let 
be the corresponding face in . Then, in  we replaced 
with , and in  we placed  inside . The edges which
connects  to the boundary of  are dual to the edges of .

Combining Theorem \ref{thm:mfmc}, Lemma \ref{lem:mcc}
and Lemma \ref{lem:Gc} we get that the value of the maximum flow in
 is the same as the value of minimum arc-cut in , which is
the same as the value of the shortest cut-cycle in , which equals
to the value of the maximum flow in . And the next lemma follows.
\begin{lemma} \label{lem:Ge}
    The value of the maximum flow of  is equal to the value of the maximum flow of .
\end{lemma}

\section{Reduction from the extended graph to the original graph}
\label{sec:acy}
Let  be a flow in ,
we define 
to be the restriction of  to the arcs of .
 The next
lemma generalizes and corrects the result of Zhang et
al.~\cite[Theorem 3]{ZLC08}.

\begin{lemma} \label{lem:acyc}
    Let  be a flow function in . If  is acyclic then 
is a flow function in .
\end{lemma}
\begin{proof}
We show that  satisfies all three
conditions that a flow function in  should satisfy.

Since the capacities of common arcs of  and   are  the same,
 clearly satisfies arc capacity constraints.

Let . If  then  is also a vertex of , with the same incident arcs,
and therefore  satisfies the flow conservation constraint at
. Otherwise,  the amount of flow that enters  in ,
is the same amount that enters  in . Also, the amount
of flow that leaves  in  is the same amount that leaves
 in . Therefore we obtain that  satisfies the flow
conservation constraint at  by summing up the flow conservation
constraints that  satisfies for the vertices in .

It  is left to show that  satisfies the vertex capacities
constraints. (Note that these constraints are irrelevant for 
since in  we do not have vertex capacities.) Let  be a
vertex with finite capacity.  First, we show that the acyclicity of
 implies that in the cyclic order around  of arcs with positive
flow that are incident to  in , the arcs carrying flow into
 are consecutive, and the arcs carrying flow out of  are
consecutive. Note that this claimed consecutiveness is restricted to
arcs with positive flow, so arcs  with  can appear
anywhere in the cyclic order of the arcs incident to .


If there is only one arc carrying
 flow into  then the
claim is trivial. Otherwise, consider two arcs  and  carrying
flow  into . We  show that if we cyclically traverse the arcs
incident to  clockwise starting from , we either traverse all
arcs carrying flow out of  before traversing , or we traverse
them all following  but before we get back to .
 Since this hold for every pair of arcs  and 
carrying flow into , the desired consecutiveness follows.

Since  carries flow,
there is a path  from  to  of arcs with positive flow that
ends with . Similarly, there is a path  of arcs with positive
flow from  to  that ends with .

Let  be the last vertex of , before , that also
appears on . The vertex  must also be the last vertex on ,
before ,
that also appears on , as otherwise we get that  contains a
flow-cycle.
Let  be the suffix of  that starts at the arc of  that
goes out of , and let  be the suffix of  that starts at
the arc of  that goes out of . (These arcs are uniquely
defined since  does not contain a flow-cycle.) Since both
 and  goes from  to  the arcs of  and  partition
the plane into two regions, denote them by  and .

The vertex  is obviously not  since  has a finite capacity.
Furthermore, since there is an arc with positive flow outgoing of
each vertex along  and , none of these vertices can be . Therefore  is
either inside  or inside . We assume without loss of generality that
 is in .

 Let  be an arc that carries flow out of , such that . There must be a path  that starts with  and carries flow
from  to . See Figure \ref{fig:H}. Since  is in  and
 is in , we get that there exists a flow-cycle that starts
with a prefix of  and ends with a suffix of  or of ,
contradicting  the assumption that  is acyclic. Therefore,
all arcs that carry flow out of , carry it to a vertex in .
Our claim that the arcs carrying flow into
 are consecutive and the arcs carrying flow out of  are
consecutive then follows.

\begin{figure}
    \centering
    \includegraphics{H}
    \caption{The path  from  to  must cross one of the paths  or  from  to , and thus creates a flow-cycle. The shaded area is .}
    \label{fig:H}
\end{figure}

An arcs  with  incident to  in  corresponds to the same
arc  with  incident to a vertex in the cycle  in
. Therefore among the arcs incident to  with  those
that carry flow into vertices in  are consecutive.
 Therefore,
we can find two edges  and  of  such that if we remove
them  splits into two parts, such that all arcs that carry flow
into vertices in  are incident to one of these parts, and all
arcs that carry flow out of vertices in  are incident to the
other part.
 See
Figure \ref{fig:bb}. All the flow which enters  must go through
 and , in order to leave , because of the flow conservation
constraints on the vertices of .
The total capacity of  and  is ,
and therefore the total flow that enters  in  is at most
. Thus it follows that the total flow that enters  in
 is also at most , and the vertex capacity constraint
at  holds.
\end{proof}

\begin{figure}
    \centering
    \includegraphics{bb}
    \caption{The edges  and  separate between the incoming flow into  and the outgoing flow from this cycle. Dotted arcs do not carry flow. The arrowheads of edges of  indicate direction of flow.}
    \label{fig:bb}
\end{figure}

\section{Canceling flow-cycles} \label{sec:canc}

In order to use Lemma \ref{lem:acyc} we must find a maximum flow  in 
such that  is acyclic. In this section we show how to do
that.

The algorithm of Zhang et al.~\cite[Section 3]{ZLC08} for undirected planar
graphs finds a flow  in  and than cancels flow-cycles  in
 in an arbitrary order. They call the resulting flow 
and claim that this flow satisfies vertex capacities constraints.
This approach is flawed.  Figure \ref{fig:counter} shows an example
on which the algorithm of \cite{ZLC08} fails. After we cancel
flow-cycles in  in an arbitrary order it is possible that
there is no flow  in  whose restriction to  is ,
and  therefore Lemma \ref{lem:acyc} does not apply.

\begin{figure}
    \centering
    \includegraphics{counter}
    \caption{A counterexample to the algorithm of \cite{ZLC08}.
    The edges of the original undirected graph  are solid. The vertex  has capacity , the edges of  are dotted.
     The flow in every solid edge is , the flow in every dotted edges is , in the specified direction (the edges are undirected). The bold edges form a flow-cycle in , after we cancel it we remain with an acyclic flow in , but the amount of flow that enters  is . The correct solution is to cancel the flow in the two internal flow-cycles.}
    \label{fig:counter}
\end{figure}

As the example in Figure \ref{fig:counter} shows, it is not enough
to cancel arbitrary flow-cycles in . We can cancel a flow-cycle
in  only if there is a cycle  in  that contains it, such that
we can reduce flow along the cycle . In this case the cycle  in 
is a residual cycle with respect to . Therefore, in order to cancel a
flow-cycle in  with respect to  we must cancel a residual cycle in
 with respect to . Canceling a arbitrary residual cycle is not enough,
since we always want to reduce the flow that  assigns to arcs, and never
to increase it.

Let  be a flow in . We define a new capacity function  on the
arcs of  which guarantees that the flow in arcs of  never increases
beyond .
For  we let . The arcs of  which are not in 
are arcs of  for some vertex , for these arc we do not have to limit
the flow to the amount in , so we set . The flow
function  is also a flow function in  with the new capacity , by
the way we defined . Since  for every arc , every
flow in  with capacity  is also a flow in  with the original
capacity .

Instead of canceling the residual cycles one by one,
we apply to  and  the algorithm in Section \ref{sec:rcyc} and find a
new flow  with the same value as , such that there are no clockwise
residual cycles in  with respect to  and .
The following lemma shows the crucial property of .

\begin{lemma}
The restriction
 of  to  does not contain counterclockwise flow-cycles.
\end{lemma}
\begin{proof}
  Assume, for a contradiction, that there is a counterclockwise flow-cycle 
  with respect to  in . We choose  such that  does
  not contain any other counterclockwise flow-cycle inside its embedding in
  the plane. We show that we can extend  to a clockwise residual cycle
  with respect to  in the graph  with capacity ,
in contradiction to the way we constructed .

For every arc  of , , and therefore  is a residual arc
with respect to  and . If  does not contain a vertex  with
 then  is a clockwise residual cycle with respect to
 and , and we obtain a contradiction.

Let  be a vertex in  with . Let  and  be the arcs of  which are incident to .
These arcs correspond to arcs  and  in ,
where  and  are in .
Let  be the path from  to  which goes counterclockwise around
 (recall that  is undirected in ).
To show a complete residual cycle in , we argue that  is a residual path
in  with respect to  and , so we can use it to fill the gap between
 and  in .
An arc  of  is not residual if
and only if .
Without loss of generality we
assume that the vertex  in the path  is followed by .

Let  be the last arc in the path  from  to
.
Assume for contradiction that  is not residual with respect to 
and . Then  and so the total flow which enters into
 in  is . The only remaining arc that can carry flow
out of  is . Because  satisfies flow
conservation constraints .
But this is impossible since the capacity of the arc  is
. Therefore,  and  is residual
with respect to  and .

We now proceed by induction. Assume by induction that we already
know that the arc  on the path 
from  to  is residual with respect to  and .
If  then we are done. Otherwise, we
prove that  is also
residual with respect to  and . Since   is
residual it follows that
  .
Let  be the single arc of  incident to . If  is
directed out of  and  then since  is a cycle and
 is inside  in the embedding of , there must be a path
carrying flow that starts with  and continues to another vertex
on . (Recall that  is incident to the outer face.) This
implies that there is a counterclockwise flow-cycle with respect to
 and  inside the embedding of , in contradiction to
the choice of . Therefore  does not carry flow of  out of
 in . This implies, by the conservation constraint on ,
that ,
 so  is indeed residual
with respect to  and .

We showed that there is a residual path from  to .
Since  was an arbitrary vertex with  on  it follows that
we can extend  to a residual cycle in  with respect to  and
. Since  is a counterclockwise cycle, the residual cycle we got from
 is a clockwise cycle.
This contradicts the definition of , and therefore a counterclockwise
flow-cycle  with respect to  and  does not exist.
\end{proof}

We repeat the previous procedure symmetrically, by defining a new capacity
 which restricts the flow in  to the flow in , and applying a
symmetric version of the algorithm of Section \ref{sec:rcyc}. This way we
get from  a flow  of the same value, such that 
does not contain clockwise flow-cycles in . For
every  we changed the flow such that , so we did not create any new flow-cycles.
Therefore we have the following lemma.
\begin{lemma} \label{lem:fpp}
The flow function  has the same value as the flow function . The restriction
 of  to  is acyclic.
\end{lemma}

\section{The algorithm}

Combining together the results of the previous sections we get an
algorithm for finding maximum flow in a directed planar graph with
vertex capacities.

First, we construct  from  by replacing each vertex that has a
finite capacity with  as defined in Section \ref{sec:ext}. Next,
we find a maximum flow  in , which is a directed planar graph
without vertex capacities.  Last, we change
  to another flow  as in Section \ref{sec:canc}.

According to
Lemma \ref{lem:fpp}, the flow  is maximum flow in , and
 its restriction  is acyclic.
 By Lemma \ref{lem:acyc}, the function  is a flow in
 . And by Lemma \ref{lem:Ge},  is a maximum flow, since the amount of
 flow that  carries into  is the same as the amount of
 flow that  carries into .

 The construction of  from  takes  time. The computation
 of  from  also takes  time using the algorithm
we described in Section \ref{sec:rcyc}. Therefore,
 the only bottleneck of our algorithm is finding , a maximum flow
 in a directed planar graph without vertex capacities.

\begin{theorem}
  The maximum flow in a directed planar graph with both arc capacities
  and vertex capacities can be computed within the same time bound as the
  maximum flow in a directed planar graph with  arc capacities only.
\end{theorem}

The algorithm of Borradaile and Klein~\cite{BK} finds a maximum flow
in directed planar graph with arcs capacities in  time. If
 is a -planar graph, then  preserves this property. In
this case the algorithm of Hassin~\cite{H81}, using the algorithm of
\cite{HKRS97} for single-source shortest-path distances, finds a
maximum flow in  time.

\begin{thebibliography}{10}
\bibitem{AMO93}
R.~K.~Ahuja, T.~L.~Magnanti, and J.~B.~Orlin.
\newblock {\em Network Flows: Theory, Algorithms and Applications}.
\newblock Prentice-Hall, New Jersey, 1993.

\bibitem{BK}
G.~Borradaile and P.~Klein.
\newblock An  algorithm for maximum -flow in a directed planar graph
\newblock {\em J.\ ACM}, to appear.

\bibitem{FF62}
L.~R.~Ford and D.~R.~Fulkerson.
\newblock {\em Flows in Networks}.
\newblock Princeton University Press, New Jersey, 1962.

\bibitem{H81}
R.~Hassin.
\newblock Maximum flow in  planar networks.
\newblock {\em Information Processing Letters} 13: 107, 1981.

\bibitem{HKRS97}
M.~R.~Henzinger, P.~Klein, S.~Rao, and S.~Subramania.
\newblock Faster shortest-path algorithms for planar graphs.
\newblock {\em J.\ Comput.\ Syst.\ Sci.} 55: 3--23, 1997.

\bibitem{J87}
D.~B.~Johnson.
\newblock Parallel algorithms for minimum cuts and maximum flows in planar networks.
\newblock {\em J.\ ACM} 34: 950--967, 1987.

\bibitem{KN94}
S.~Khuller and J.~Naor.
\newblock Flow in planar graphs with vertex capacities.
\newblock {\em Algoirthmica} 11: 200--225, 1994.

\bibitem{KNK93}
S.~Khuller, J.~Naor, and P.~Klein.
\newblock The lattice structure of flow in planar graphs.
\newblock {\em SIAM J.\ Disc.\ Math.} 63: 477--490, 1993.

\bibitem{NC88}
T.~Nishizwki and N.~Chiba.
\newblock {\em Planar Graphs: Theory and Algorithms}, Ann.\ Discrete Math., Vol. 32.
\newblock North-Holland, 1988.

\bibitem{RLW97}
H.~Ripphausen-Lipa, D.~Wagner, and K.~Weihe.
\newblock The vertex-disjoint Menger problem in planar graphs.
\newblock {\em SIAM J. Comput.} 26: 331--349, 1997.

\bibitem{W97}
K.~Weihe.
\newblock Maximum -flows in planar networks in -time.
\newblock {\em J.\ Comput.\ Syst.\ Sci.} 55: 454--476, 1997.

\bibitem{ZLJ06}
X.~Zhang, W. Liang, and H.~Jiang.
\newblock Flow equivalent trees in node-edge-capacitied undirected planar graphs.
\newblock {\em Information Processing Letters} 100: 100--115 ,2006.

\bibitem{ZLC08}
X.~Zhang, W.~Liang, and G.~Chen.
\newblock Computing maximum flows in undirected planar networks with both edge and vertex capacities.
\newblock In {\em 14th Annual International Conference on Computing and Combinatorics (COCOON)},
Lecture Notes in Computer Science 5092: 577--586, 2008.

\end{thebibliography}

\end{document}
