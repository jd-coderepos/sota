
\section{Results and discussion}
\label{sec:results}

\subsection{Photoreceptor layer segmentation}

\begin{figure}[t]
 \centering
 \subfigure[Photoreceptors]{\includegraphics[width=0.45\columnwidth]{figures/t_vs_auc_photoreceptors.pdf}}
 \subfigure[Disruptions]{\includegraphics[width=0.45\columnwidth]{figures/t_vs_auc_interruptions.pdf}}
 \caption{AUCs in the validation set for different $T$ values.}
 \label{fig:val-t-num}
\end{figure}

\begin{table}[t!]  
\centering
\resizebox{\columnwidth}{!}{
\begin{tabular}{C{1.6cm}||C{1.2cm}|C{1.2cm}|C{1.2cm}||C{1cm}|C{1cm}|C{1.2cm}}
  \hline
  \multirow{4}{*}{\textbf{Model}} & \multicolumn{3}{c||}{\textbf{Test set A}}& \multicolumn{3}{c}{\textbf{Test set B}} \\
  & \multicolumn{3}{c||}{AMD (early, CNV), DME, RVO} & \multicolumn{3}{c}{Late AMD (GA)} \\
  \cline{2-7}
  & \multicolumn{2}{c|}{\textbf{Photoreceptors}} & \textbf{Disruptions} & \multicolumn{2}{c|}{\textbf{Photoreceptors}} & \textbf{Disruptions} \\
  \cline{2-7}
   & \textbf{AUC} & \textbf{Dice} & \textbf{AUC} & \textbf{AUC} & \textbf{Dice} & \textbf{AUC}  \\
  \hline
  \hline
  \textbf{U-Net}~\cite{ronneberger2015u} & 0.9566 & 0.8815 $\pm$0.06 & 0.5077 & 0.9390 & 0.8375 $\pm$0.07 & 0.8795 \\
  \hline
  \textbf{BRU-Net}~\cite{apostolopoulos2017pathological} & 0.9593 & 0.8767 $\pm$0.08 & 0.2621 & 0.9295 & 0.7890 $\pm$0.13 & 0.8333 \\
  \hline
  \textbf{BU-Net} $T=1$                  & 0.9466 & 0.8647 $\pm$0.08 & 0.2222 & 0.8969 & 0.7311 $\pm$0.14 & 0.8065\\
  \hline
  \textbf{BU-Net} $T=10$                 & 0.9505 &  0.8678 $\pm$0.08 & 0.2405 & 0.8998 & 0.7428 $\pm$0.14 & 0.8129\\
  \hline
  \hline
  \textbf{U2-Net} $T=1$                  & 0.9653  & 0.8932 $\pm$0.04 & \textbf{0.6712} & \textbf{0.9500} & \textbf{0.8546 $\pm$0.06} & 0.9085 \\
  \hline
  \textbf{U2-Net} $T=10$                 & \textbf{0.9669} & \textbf{0.8943 $\pm$0.04} & 0.6417 & 0.9472 & 0.8457 $\pm$0.08 & \textbf{0.9101} \\
  \hline
\end{tabular}
}
\caption{Quantitative evaluation on the test sets A and B.}
\label{table:comparison-table} 
\end{table}

We studied the changes in the AUC values when varying the number of MC samples $T$ in the validation set (Fig.~\ref{fig:val-t-num}). It can be seen that the segmentation performance was not significantly improved after $T=20$, while a small drop in the AUC of the disruptions is observed after $T=10$. 

Table~\ref{table:comparison-table} compares the results obtained on the test sets A and B using different segmentation models. The U2-Net achieved the highest performance in the two sets, both for photoreceptor segmentation and disruption detection. In the test set A, an improvement in the segmentation results of the U2-Net was observed when averaging through multiple MC samples, while the performance for disruption detection was slightly decreased. The opposite case was observed in the test set B, where MC sampling improved the performance for disruption detection but affecting the results for photoreceptor segmentation. In all the cases, the BU-Net performed poorly compared both to our U2-Net and the standard network.

Qualitative results of the U2-Net with $T=10$ samples are presented in Fig.~\ref{fig:qualitative-results}, jointly with their associated pixel-wise uncertainty estimates. The uncertainty maps were normalized using their maximum value for visualization purposes.


\begin{figure}[t]
 \centering
\subfigure[Dice$=0.9196$, $\overline{u}=6.7\times10^{-4}$]{\includegraphics[width=0.48\columnwidth]{figures/whole_fig_2.png}\label{fig:qualitative-results-a}}
 \subfigure[Dice$=0.5888$, $\overline{u}=13\times10^{-4}$]{\includegraphics[width=0.48\columnwidth]{figures/whole_fig_3.png}\label{fig:qualitative-results-b}}
 \caption{Qualitative results of the U2-Net ($T=10$) on the test set $A$. From top to bottom: B-scan, manual annotation, automated segmentation, and epistemic uncertainty map. B-scan level Dice and mean uncertainty $\overline{u}$ are also included.}
 \label{fig:qualitative-results}
\end{figure}


\subsection{Uncertainty estimation}

Fig.~\ref{fig:uncertainty-volume} shows the correlation between the photoreceptor layer segmentation performance (as measured using the Dice index) and the mean uncertainty for each of the volumes on the test set A. The linear regression line is included in the plot to illustrate the general trend of the results. The mean uncertainty at a volume level was observed to be inversely correlated with the segmentation performance ($R^2=0.7644$). A similar behavior was observed at a B-scan level (Fig.~\ref{fig:qualitative-results}).




\subsection{Discussion}

The proposed U2-Net allowed to improve both the segmentation and disruption detection results of the baseline U-Net, outperforming the more complex BRU-Net architecture (Table~\ref{table:comparison-table}). This is a result of a better generalization ability thanks to the incorporation of more dropout than in the baseline networks, and to the ability of leaky ReLUs to prevent vanishing gradients~\cite{xu2015empirical}. It was also observed that learning to predict the aleatoric uncertainty of the model (BU-Net) reduced the performance of the architecture in our two test sets (Table~\ref{table:comparison-table}). This might be due to the fact that this additional task acts as a strong regularizer that enforces the network to ignore infrequently occuring ambiguities. A general drop in segmentation performance was observed when applying the models on the test set B, although our U2-Net still achieved the best results. The improvement in the AUC values for disruption detection could be explained by the presence of more evident interruptions than those in the test set A. When averaging multiple MC samples, a trade-off between segmentation and disruption detection performance was observed. This might be caused by the methods producing more (less) true positive responses of the photoreceptors but at the same time more (less) false positive responses in the disrupted areas. Such a behavior is consistent with e.g. being more accurate in terms of the layer thickness or to better segment the photoreceptor layer under vessel shadows, but at the cost of ignoring small disruptions (Fig.~\ref{fig:qualitative-results-a}). Nevertheless, the MC sampling procedure has the added value of providing pixel-wise uncertainties that can be used to correct these errors. It is worth mentioning also that the dropout parameters of the U2-Net were fixed without fine tuning on the validation set. Further improvements in the results could be achieved by finding an optimal configuration.

In healthy photoreceptor layers, most of the epistemic uncertainty occurs in the upper and lower interfaces of the segmentation (Fig.\ref{fig:qualitative-results-a}, left arrow). This is in line with the high inter-observer variability observed during the manual annotation process. Under challenging scenarios with small disruptions (Fig.\ref{fig:qualitative-results-a}, right arrow) the model sometimes missinterpreted the area as a layer thinning, although with high uncertainty.In extremelly pathological scenarios (Fig.\ref{fig:qualitative-results-b}), the U2-Net was able to identify most of the largest areas of cell death in the layer, which are more evident due to the concomitant appeareance of cysts and retina thickenning. In Fig.\ref{fig:qualitative-results-b}, most of the wrong predictions were associated to areas of subtle disruptions, differences in the layer thickness (right arrows), or to inconsistencies in the edges of the disruptions (left arrow). In any case, high uncertainty values were observed in those areas. This indicates that the errors in the segmentation could be pointed out by the uncertainty estimates, and subsequently be manually corrected by human readers. This claim is also supported by the correlation analysis in Fig.~\ref{fig:uncertainty-volume}.

\begin{figure}[t]
 \centering
 \includegraphics[width=0.7\columnwidth]{figures/unet-cross-entropy_10_best_Dice.pdf}
 \caption{Correlation between segmentation performance and mean uncertainty for each OCT volume in the test set $A$.}
 \label{fig:uncertainty-volume}
\end{figure}
