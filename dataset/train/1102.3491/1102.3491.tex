\documentclass[letterpaper,11pt]{article}
\usepackage{fullpage}
\usepackage{amsmath,amsthm, amssymb}
\usepackage{color}		\usepackage{epsfig}
\usepackage{graphicx}
\usepackage{hyperref}
\usepackage[all]{hypcap}

\newcommand{\Nat}{{\mathbb N}}
\newcommand{\Integ}{{\mathbb Z}}

\newcommand{\Real}{{\mathbb R}}
\newcommand{\NP}{{\textrm NP}}
\newcommand{\M}{\mathcal{M}}
\newcommand{\I}{\mathcal{I}}

\newtheorem{theorem}{Theorem}[section]
\newtheorem{corollary}[theorem]{Corollary}
\newtheorem{lemma}[theorem]{Lemma}
\theoremstyle{definition}
\newtheorem{definition}[theorem]{Definition}
\newtheorem{alg}{Algorithm}

\hypersetup{
pdfauthor = {Jos\'e\ A.\ Soto},
pdftitle = {A simple PTAS for Weighted Matroid Matching on Strongly Base Orderable Matroids},
}


\begin{document}

\title{A simple PTAS for Weighted Matroid Matching on Strongly Base Orderable Matroids\footnote{This work was partially supported by NSF
  contract CCF-0829878 and by ONR grant N00014-11-1-0053.}\, \footnote{Extended abstract of this article to appear in the proceedings of the Sixth Latin-American Algorithms, Graphs and Optimization Symposium (LAGOS) 2011.}}
\author{Jos\'e A. Soto\thanks{MIT,
Dept.~of Math., Cambridge, MA 02139. \texttt{jsoto@math.mit.edu}.}}
\date{}
\maketitle

\begin{abstract}
We give a simple polynomial time approximation scheme for the weighted matroid matching problem on strongly base orderable matroids. We also show that even the unweighted version of this problem is NP-complete and not in oracle-coNP.
\end{abstract}



\section{Preliminaries}
We assume familiarity with basic graph and matroid theory. For reference, see Schrijver's book~\cite{Schrijver-book}. Given a matroid  and a graph , we say that a matching  is  \emph{feasible} for  if the set  of vertices covered by  is independent in . Given a weight function, , the \emph{weighted matroid matching problem} consists on finding a feasible matching of maximum weight. This problem is a common generalization to both the weighted graphic matching and the weighted matroid intersection problems.

Even for the unweighted version (where every edge has unit weight), the previous problem is not polynomially solvable when the matroid is given by an independence oracle~\cite{JensenK82,Lovasz81}: there are instances where checking that no feasible matching of a given size exists requires an exponential number of oracle queries. These instances can be modified to show \NP-completeness~\cite{Schrijver-book}.

For the unweighted version, Lov\'asz~\cite{Lovasz81} has given a polynomial time algorithm for linear matroids. Recently, Lee et al.~\cite{LeeSV10} have developed a PTAS for general matroids using local search techniques. For arbitrary weights, the situation is less developed. We only have polynomial time algorithms for the case where the matroid is a gammoid~\cite{tong1982solving}. An open problem of Lee et al.~\cite{LeeSV10} is to get a PTAS for the weighted version on general matroids.

In this note we focus on a special class of matroids containing gammoids and not comparable with linear matroids. A matroid  is \emph{strongly base orderable} if for every pair of bases ~and~, there is a bijection  such that for all , the set  is also a base of the matroid. On the positive side, we give a local search based PTAS for the weighted matroid matching problem on strongly base orderable matroids. On the negative side, we show that even the unweighted version of this problem is NP-complete and not in oracle co-NP. The negative results follow since the matroids used to prove the analogue hardness results for matroid matching are strongly base orderable.

An important special case of the weighted matroid matching problem is the \emph{weighted matroid parity problem}. In this version, the set  of edges (also denoted as \emph{pairs} or \emph{mates}) is itself a matching. The following reduction~\cite{LeeSV10} shows that both problems are equivalent. Given an instance  of the weighted matroid matching problem, define a new set  of elements containing as many copies of every vertex  as its degree in . Construct a new graph  by creating for every edge , one edge of the same weight between a copy of  and a copy of  in , in such a way that the new collection of edges  is a matching. Let  be the matroid in which a set  is independent if it has at most one copy of each element of  and the respective set  of original elements is independent in~. It is easy to check that the previous construction gives an approximation preserving reduction and that if  is strongly base orderable then so is the matroid . The last statement follows since strong base orderability of matroids is closed under taking parallel extensions \cite[Section 42.6c]{Schrijver-book}.

\section{A local search PTAS for Weighted Matroid Parity}
Consider a matroid , a partition  of  into  pairs and a weight function . A folklore result (see, e.g.~\cite{LeeSV10}) is that the \emph{greedy solution} obtained by processing the pairs in decreasing order of weight, and selecting only those pairs that keep the current solution feasible, is a 2-approximation for the weighted matroid parity problem.

\begin{definition} An \emph{-move} for a feasible set of pairs  is a choice of pairs  and , with , such that   is feasible. The value  is the \emph{gain of the move}. The move \emph{improves } if its gain is positive (equiv., if ). For a given , the move \emph{-improves } if its gain is at least  (equiv., if ).
\end{definition}

A feasible set is a \emph{local optimum for -moves} if no -move improves it. A feasible set is a \emph{-local optimum for -moves} if no -move -improves it. Consider the following local search algorithms for the unweighted and weighted matroid parity problems respectively, parameterized by an integer .

\begin{alg} \label{alg1} While there is an -move improving the current solution, initially empty, perform the -move of maximum gain. Return the solution found.
\end{alg}

\begin{alg} \label{alg2} Start from a greedy solution  for the problem. While there is an -move -improving the current solution, perform the -move with maximum gain. Return the solution found.
\end{alg}
\begin{lemma}\label{lem:localoptimum}
Given , Algorithm \ref{alg1} (resp.\ Algorithm \ref{alg2}) returns a local optimum (resp.\ a -local optimum) for -moves for the unweighted (resp.\ weighted) matroid parity problem in  time (resp.\  time).
\end{lemma}
\begin{proof}
Clearly, both algorithms are finite and correct and the greedy solution~ can be found in  time, which is . Given a feasible solution , the number of possible -moves for~ is . Then each iteration of both algorithms can be done in  time and using the same number of oracle calls. To conclude, we show that the number of iterations is bounded.

Since every iteration of the first algorithm improves the current solution by one unit and the optimum solution has cardinality at most , this algorithm performs at most  iterations. In the second algorithm, every iteration improves the weight of the current solution by a factor . Since the initial solution  and the solution returned are at most a factor two apart, the number of iterations is at most . \qedhere
\end{proof}

The following lemma, which we prove at the end of this section, is useful to obtain a PTAS when  is strongly base orderable.

\begin{lemma}\label{lem:main}
Let  be a positive integer and  be an optimal feasible solution of the weighted matroid parity problem on an strongly base orderable matroid.
If  is a local optimum for -moves then .
If  is a -local optimum for -moves then .
\end{lemma}

Given , set  and . By Lemma~\ref{lem:main}, any local optimum for -moves is a -approximation, and so is any -local optimum for -moves. By choosing  appropriately in Lemma \ref{lem:localoptimum}, we have the following corollary.

\begin{corollary} Algorithm \ref{alg1} (resp.\ Algorithm \ref{alg2}) returns a -approximation for the unweighted (resp.\ weighted) matroid parity problem for strongly base orderable matroids in time  (resp.\ ).
\end{corollary}

Using the reduction from matroid matching to matroid parity we obtain our main positive result.
\begin{theorem} There is a PTAS for both the weighted and the unweighted matroid matching problems on strongly base orderable matroids.
\end{theorem}

\begin{proof}[Proof of Lemma \ref{lem:main}]
For two feasible solutions  and , consider the collections  and  obtained by adding dummy pairs of zero weight to the set of smaller cardinality, so that . Let  be the set of elements added and  be the matroid obtained from  by first adding  as coloops and then truncating  to rank . By properties of strongly base orderable matroids (see, e.g.~\cite[Section 42.6c]{Schrijver-book}),  is also strongly base orderable and both  and  are bases of .

Consider a bijection  such that for all ,  is a base of . We can further impose  to be the identity map (see, e.g.~\cite[Section 42.6c]{Schrijver-book}). Let  be the multigraph on the set  having one edge between  and  if only one element in  is mapped to  and two parallel edges between them if . Since every vertex has degree two and the graph is bipartite,  is a collection of vertex-disjoint even cycles. Orient the edges so that every cycle is properly directed to obtain a digraph~. See Figure \ref{fig:figure1}.

\begin{figure}[h!!]
\centering
\begin{picture}(0,0)\includegraphics{FinalGraph.pstex}\end{picture}\setlength{\unitlength}{4144sp}\begingroup\makeatletter\ifx\SetFigFont\undefined \gdef\SetFigFont#1#2#3#4#5{\reset@font\fontsize{#1}{#2pt}\fontfamily{#3}\fontseries{#4}\fontshape{#5}\selectfont}\fi\endgroup \begin{picture}(5250,1332)(301,-481)
\put(3376,569){\makebox(0,0)[lb]{\smash{{\SetFigFont{9}{10.8}{\rmdefault}{\mddefault}{\updefault}{\color[rgb]{0,0,0}}}}}}
\put(2386,569){\makebox(0,0)[rb]{\smash{{\SetFigFont{9}{10.8}{\rmdefault}{\mddefault}{\updefault}{\color[rgb]{0,0,0}}}}}}
\put(5536,569){\makebox(0,0)[lb]{\smash{{\SetFigFont{9}{10.8}{\rmdefault}{\mddefault}{\updefault}{\color[rgb]{0,0,0}}}}}}
\put(4546,569){\makebox(0,0)[rb]{\smash{{\SetFigFont{9}{10.8}{\rmdefault}{\mddefault}{\updefault}{\color[rgb]{0,0,0}}}}}}
\put(4546,-421){\makebox(0,0)[rb]{\smash{{\SetFigFont{9}{10.8}{\rmdefault}{\mddefault}{\updefault}{\color[rgb]{0,0,0}}}}}}
\put(316,164){\makebox(0,0)[rb]{\smash{{\SetFigFont{9}{10.8}{\rmdefault}{\mddefault}{\updefault}{\color[rgb]{0,0,0}}}}}}
\put(316,-421){\makebox(0,0)[rb]{\smash{{\SetFigFont{9}{10.8}{\rmdefault}{\mddefault}{\updefault}{\color[rgb]{0,0,0}}}}}}
\put(1306,569){\makebox(0,0)[lb]{\smash{{\SetFigFont{9}{10.8}{\rmdefault}{\mddefault}{\updefault}{\color[rgb]{0,0,0}}}}}}
\put(2386,-421){\makebox(0,0)[rb]{\smash{{\SetFigFont{9}{10.8}{\rmdefault}{\mddefault}{\updefault}{\color[rgb]{0,0,0}}}}}}
\end{picture} \caption{In the left, feasible sets  and  sharing one pair. In the middle, sets  and  and the bijection . In the right, a possible orientation of .}
\label{fig:figure1}
\end{figure}

For a positive integer , let  be the collection of \emph{long cycles} in , having length at least  and  be the collection of \emph{short cycles} in , having length at most  (for convenience, include in  all the isolated nodes in  as \emph{zero-length cycles}). Let  and  be the sets of nodes in long and short cycles respectively. Note that  and  form a partition of .

For every cycle  of , . Hence  is a feasible matching for~. Therefore, for every short cycle , the collection of pairs  defines a -move for  in the original matroid .

For every , define the set  of nodes reachable from  by using at most  directed arcs in . This is,  is the node set of the directed path starting in , having  nodes from~ and having  nodes from . By construction, for every , the image under  of  contains . This observation, together with the definition of , implies that  is feasible for the matroid . Therefore,  defines a -move for  in .

The total gain of moves coming from short cycles, , is

and the total gain of moves from long cycles, , is


Suppose that  is optimal and  is a local optimum for -moves. Then each of the above moves has nonpositive gain. In particular, for each , we have . Therefore,

From here, we have that .

If  is a -local optimum for -moves, then each move has gain of at most . Since ,  and , we have

From here, we have that . \qedhere\end{proof}

\section{Matroid parity is hard on strongly base orderable matroids.}
Here we review the proofs of hardness of matroid parity on general matroids (see~\cite[Section 43.9]{Schrijver-book} for more details).
Let ,  be a set of even size and~ be a partition of  into pairs. Let ~be the matroid on , where  is independent if , or if  and  is not the union of  pairs in~. For every set  of size , let  be the matroid extending  by an independent set . Also, given a graph , construct  by extending  by all independent sets , where  is the vertex set of a clique of size  in . It can be shown that ,  and  are matroids, that  has no matching of size  feasible for , that  is the unique matching of size  feasible for , and that  contains a matching of size  feasible for  if and only if  has a clique of size .

Given , the above set  and an oracle for a matroid  that is known to be either  or any of the possible  with , it is known that the number of oracle calls needed to check if there is a matching of size  feasible for  is at least , showing that the matroid parity problem is not in oracle co-NP. Similarly, as the maximum-size clique problem is NP-complete, the matroid parity problem on matroids  constructed as above is NP-complete.

Note that  and  are special cases of , obtained from a graph having zero or one clique of cardinality . The following lemma implies that the hardness guarantees for the matroid parity problem are maintained if we restrict to strongly base orderable matroids.

\begin{lemma}\label{lem:lemsbo}
  For every  and every graph , the matroid  is strongly base orderable.
\end{lemma}
\begin{proof}
Let  and  be a graph. Create a set  having two copies of each element in , and interpret  as a partition of  into pairs or mates. The bases of  are then


Let  be two bases in . In what follows we construct a bijection , such that for all ,  is also a base. We divide the problem in cases. In each of them we impose conditions on  that can easily be satisfied by some bijection. In particular, for every case, we impose  to be the identity map. By assuming that  is the union of ~pairs, we either show that  or we get a contradiction, proving in both situations that  is a base.

Under the assumption that  is the union of  pairs, the following two properties hold.
\begin{description}
\item[Property 1:] An element is in  if and only if its mate is in .
\item[Property 2:] An element in  is in  if and only if its image or preimage under  is not in .
\end{description}

There are nine cases to consider for  and . For visual aid on the first five cases, see Figure \ref{fig:figure2}.

\begin{figure}[h!!t]
\centering
\scalebox{0.98}{\begin{picture}(0,0)\includegraphics{ArxivCases1.pstex}\end{picture}\setlength{\unitlength}{4144sp}\begingroup\makeatletter\ifx\SetFigFont\undefined \gdef\SetFigFont#1#2#3#4#5{\reset@font\fontsize{#1}{#2pt}\fontfamily{#3}\fontseries{#4}\fontshape{#5}\selectfont}\fi\endgroup \begin{picture}(5919,4747)(-11,-3896)
\put(676,-1411){\makebox(0,0)[b]{\smash{{\SetFigFont{10}{12.0}{\rmdefault}{\mddefault}{\updefault}{\color[rgb]{0,0,0}(i)}}}}}
\put(991,-691){\makebox(0,0)[b]{\smash{{\SetFigFont{9}{10.8}{\rmdefault}{\mddefault}{\updefault}{\color[rgb]{0,0,0}}}}}}
\put(3106,-1411){\makebox(0,0)[b]{\smash{{\SetFigFont{10}{12.0}{\rmdefault}{\mddefault}{\updefault}{\color[rgb]{0,0,0}(ii)}}}}}
\put(2701,-331){\makebox(0,0)[b]{\smash{{\SetFigFont{9}{10.8}{\rmdefault}{\mddefault}{\updefault}{\color[rgb]{0,0,0}}}}}}
\put(3511,-331){\makebox(0,0)[b]{\smash{{\SetFigFont{9}{10.8}{\rmdefault}{\mddefault}{\updefault}{\color[rgb]{0,0,0}}}}}}
\put(5626,-331){\makebox(0,0)[b]{\smash{{\SetFigFont{9}{10.8}{\rmdefault}{\mddefault}{\updefault}{\color[rgb]{0,0,0}}}}}}
\put(5221,-1411){\makebox(0,0)[b]{\smash{{\SetFigFont{10}{12.0}{\rmdefault}{\mddefault}{\updefault}{\color[rgb]{0,0,0}(iii)}}}}}
\put(4816,-331){\makebox(0,0)[b]{\smash{{\SetFigFont{9}{10.8}{\rmdefault}{\mddefault}{\updefault}{\color[rgb]{0,0,0}}}}}}
\put(1486,-3121){\makebox(0,0)[b]{\smash{{\SetFigFont{9}{10.8}{\rmdefault}{\mddefault}{\updefault}{\color[rgb]{0,0,0}}}}}}
\put(2296,-3121){\makebox(0,0)[b]{\smash{{\SetFigFont{9}{10.8}{\rmdefault}{\mddefault}{\updefault}{\color[rgb]{0,0,0}}}}}}
\put(1486,-2761){\makebox(0,0)[b]{\smash{{\SetFigFont{9}{10.8}{\rmdefault}{\mddefault}{\updefault}{\color[rgb]{0,0,0}}}}}}
\put(2296,-2761){\makebox(0,0)[b]{\smash{{\SetFigFont{9}{10.8}{\rmdefault}{\mddefault}{\updefault}{\color[rgb]{0,0,0}}}}}}
\put(1891,-3841){\makebox(0,0)[b]{\smash{{\SetFigFont{10}{12.0}{\rmdefault}{\mddefault}{\updefault}{\color[rgb]{0,0,0}(iv)}}}}}
\put(4276,-3841){\makebox(0,0)[b]{\smash{{\SetFigFont{10}{12.0}{\rmdefault}{\mddefault}{\updefault}{\color[rgb]{0,0,0}(v)}}}}}
\put(3241,-1816){\makebox(0,0)[b]{\smash{{\SetFigFont{9}{10.8}{\rmdefault}{\mddefault}{\updefault}{\color[rgb]{0,0,0}}}}}}
\put(3871,-1816){\makebox(0,0)[b]{\smash{{\SetFigFont{9}{10.8}{\rmdefault}{\mddefault}{\updefault}{\color[rgb]{0,0,0}}}}}}
\put(4681,-1816){\makebox(0,0)[b]{\smash{{\SetFigFont{9}{10.8}{\rmdefault}{\mddefault}{\updefault}{\color[rgb]{0,0,0}}}}}}
\put(4681,-2131){\makebox(0,0)[b]{\smash{{\SetFigFont{9}{10.8}{\rmdefault}{\mddefault}{\updefault}{\color[rgb]{0,0,0}}}}}}
\put(3871,-2131){\makebox(0,0)[b]{\smash{{\SetFigFont{9}{10.8}{\rmdefault}{\mddefault}{\updefault}{\color[rgb]{0,0,0}}}}}}
\put(3871,-3436){\makebox(0,0)[b]{\smash{{\SetFigFont{9}{10.8}{\rmdefault}{\mddefault}{\updefault}{\color[rgb]{0,0,0}}}}}}
\end{picture} }
\caption{Cases (i)--(v). In each case, , the top-left region is , the top-right region is  and the bottom region is . Thick lines represent .  Thin arrows represent a possible map .}
\label{fig:figure2}
\end{figure}

\begin{enumerate}
\item[(i)] If there is  whose mate is in , then any bijection  works.
\end{enumerate}
Since , any set  as above contains  but not its mate. Thus,  is not the union of  pairs.
\begin{enumerate}
  \item[(ii)] Else, if there are  and  with mates in , then select any such  and , and set .
  \item[(iii)] Else, if there are  and  with , then select any such pair and set .
\end{enumerate}
Cases (ii) and (iii) are similar: any set  contains either  or , but not its corresponding mate. Thus,  is not the union of  pairs.
\begin{enumerate}
\item[(iv)]\label{iv} Else, if there are pairs  and  in  with ,  and , then set .
\end{enumerate}
For case (iv), note that any set  as above contains both  and  but only one of their mates. Thus,   is not the union of  pairs.
\begin{enumerate}
\item[(v)] Else, if there are pairs  in  with , , for ,  , and , then set , for . (Also consider here the same case with  and  interchanged.)
\end{enumerate}
For case (v), consider a set  as above and assume  is the union of  pairs. Since  is in , we have , and so  must be in . In what follows we apply Properties 1 and 2 iteratively. Since  is in , its mate  is also in . Then,  is outside set , and so is its mate . This implies that  is in . But its mate  is outside , contradicting the assumption.

If none of the previous cases hold, then only one of  or  can contain an element with mate in , and  only one  can contain an element with mate in . Furthermore, none of the two sets can contain both types of elements. The rest of the pairs intersecting  are completely inside ,  or . Therefore, by possibly interchanging the roles of  and~, we can assume that (1) the set  contains  elements with mates in , and  pairs completely inside; (2) the set  contains  elements with mates in  and  pairs completely inside; (3) the set  contains  pairs completely inside; and (4) no pair other than the ones described before intersects .

Under these assumptions, let  and  be the pairs completely inside  and  respectively. Note that . In particular, if , then   and viceversa. We proceed with the rest of the cases. For visual aid, see Figure~\ref{fig:figure3}.

\begin{figure}[h!!t]
\centering
\scalebox{0.98}{\begin{picture}(0,0)\includegraphics{ArxivCases2.pstex}\end{picture}\setlength{\unitlength}{4144sp}\begingroup\makeatletter\ifx\SetFigFont\undefined \gdef\SetFigFont#1#2#3#4#5{\reset@font\fontsize{#1}{#2pt}\fontfamily{#3}\fontseries{#4}\fontshape{#5}\selectfont}\fi\endgroup \begin{picture}(7494,3892)(-11,-3041)
\put(4456,614){\makebox(0,0)[b]{\smash{{\SetFigFont{9}{10.8}{\rmdefault}{\mddefault}{\updefault}{\color[rgb]{0,0,0}}}}}}
\put(4456,299){\makebox(0,0)[b]{\smash{{\SetFigFont{9}{10.8}{\rmdefault}{\mddefault}{\updefault}{\color[rgb]{0,0,0}}}}}}
\put(5266,614){\makebox(0,0)[b]{\smash{{\SetFigFont{9}{10.8}{\rmdefault}{\mddefault}{\updefault}{\color[rgb]{0,0,0}}}}}}
\put(5266,299){\makebox(0,0)[b]{\smash{{\SetFigFont{9}{10.8}{\rmdefault}{\mddefault}{\updefault}{\color[rgb]{0,0,0}}}}}}
\put(4456,-646){\makebox(0,0)[b]{\smash{{\SetFigFont{9}{10.8}{\rmdefault}{\mddefault}{\updefault}{\color[rgb]{0,0,0}}}}}}
\put(4456,-961){\makebox(0,0)[b]{\smash{{\SetFigFont{9}{10.8}{\rmdefault}{\mddefault}{\updefault}{\color[rgb]{0,0,0}}}}}}
\put(5266,-646){\makebox(0,0)[b]{\smash{{\SetFigFont{9}{10.8}{\rmdefault}{\mddefault}{\updefault}{\color[rgb]{0,0,0}}}}}}
\put(5266,-961){\makebox(0,0)[b]{\smash{{\SetFigFont{9}{10.8}{\rmdefault}{\mddefault}{\updefault}{\color[rgb]{0,0,0}}}}}}
\put(4456,-1276){\makebox(0,0)[b]{\smash{{\SetFigFont{9}{10.8}{\rmdefault}{\mddefault}{\updefault}{\color[rgb]{0,0,0}}}}}}
\put(5266,-1276){\makebox(0,0)[b]{\smash{{\SetFigFont{9}{10.8}{\rmdefault}{\mddefault}{\updefault}{\color[rgb]{0,0,0}}}}}}
\put(271,614){\makebox(0,0)[b]{\smash{{\SetFigFont{9}{10.8}{\rmdefault}{\mddefault}{\updefault}{\color[rgb]{0,0,0}}}}}}
\put(271,299){\makebox(0,0)[b]{\smash{{\SetFigFont{9}{10.8}{\rmdefault}{\mddefault}{\updefault}{\color[rgb]{0,0,0}}}}}}
\put(1081,614){\makebox(0,0)[b]{\smash{{\SetFigFont{9}{10.8}{\rmdefault}{\mddefault}{\updefault}{\color[rgb]{0,0,0}}}}}}
\put(1081,299){\makebox(0,0)[b]{\smash{{\SetFigFont{9}{10.8}{\rmdefault}{\mddefault}{\updefault}{\color[rgb]{0,0,0}}}}}}
\put(1081,-16){\makebox(0,0)[b]{\smash{{\SetFigFont{9}{10.8}{\rmdefault}{\mddefault}{\updefault}{\color[rgb]{0,0,0}}}}}}
\put(1081,-331){\makebox(0,0)[b]{\smash{{\SetFigFont{9}{10.8}{\rmdefault}{\mddefault}{\updefault}{\color[rgb]{0,0,0}}}}}}
\put(271,-1591){\makebox(0,0)[b]{\smash{{\SetFigFont{9}{10.8}{\rmdefault}{\mddefault}{\updefault}{\color[rgb]{0,0,0}}}}}}
\put(271,-1276){\makebox(0,0)[b]{\smash{{\SetFigFont{9}{10.8}{\rmdefault}{\mddefault}{\updefault}{\color[rgb]{0,0,0}}}}}}
\put(2476,614){\makebox(0,0)[b]{\smash{{\SetFigFont{9}{10.8}{\rmdefault}{\mddefault}{\updefault}{\color[rgb]{0,0,0}}}}}}
\put(2476,299){\makebox(0,0)[b]{\smash{{\SetFigFont{9}{10.8}{\rmdefault}{\mddefault}{\updefault}{\color[rgb]{0,0,0}}}}}}
\put(3286,614){\makebox(0,0)[b]{\smash{{\SetFigFont{9}{10.8}{\rmdefault}{\mddefault}{\updefault}{\color[rgb]{0,0,0}}}}}}
\put(3286,299){\makebox(0,0)[b]{\smash{{\SetFigFont{9}{10.8}{\rmdefault}{\mddefault}{\updefault}{\color[rgb]{0,0,0}}}}}}
\put(3286,-1276){\makebox(0,0)[b]{\smash{{\SetFigFont{9}{10.8}{\rmdefault}{\mddefault}{\updefault}{\color[rgb]{0,0,0}}}}}}
\put(3286,-1591){\makebox(0,0)[b]{\smash{{\SetFigFont{9}{10.8}{\rmdefault}{\mddefault}{\updefault}{\color[rgb]{0,0,0}}}}}}
\put(2476,-331){\makebox(0,0)[b]{\smash{{\SetFigFont{9}{10.8}{\rmdefault}{\mddefault}{\updefault}{\color[rgb]{0,0,0}}}}}}
\put(2476,-16){\makebox(0,0)[b]{\smash{{\SetFigFont{9}{10.8}{\rmdefault}{\mddefault}{\updefault}{\color[rgb]{0,0,0}}}}}}
\put(6391,614){\makebox(0,0)[b]{\smash{{\SetFigFont{9}{10.8}{\rmdefault}{\mddefault}{\updefault}{\color[rgb]{0,0,0}}}}}}
\put(6391,299){\makebox(0,0)[b]{\smash{{\SetFigFont{9}{10.8}{\rmdefault}{\mddefault}{\updefault}{\color[rgb]{0,0,0}}}}}}
\put(7201,614){\makebox(0,0)[b]{\smash{{\SetFigFont{9}{10.8}{\rmdefault}{\mddefault}{\updefault}{\color[rgb]{0,0,0}}}}}}
\put(7201,299){\makebox(0,0)[b]{\smash{{\SetFigFont{9}{10.8}{\rmdefault}{\mddefault}{\updefault}{\color[rgb]{0,0,0}}}}}}
\put(6391,-1591){\makebox(0,0)[b]{\smash{{\SetFigFont{9}{10.8}{\rmdefault}{\mddefault}{\updefault}{\color[rgb]{0,0,0}}}}}}
\put(6391,-1276){\makebox(0,0)[b]{\smash{{\SetFigFont{9}{10.8}{\rmdefault}{\mddefault}{\updefault}{\color[rgb]{0,0,0}}}}}}
\put(7201,-1591){\makebox(0,0)[b]{\smash{{\SetFigFont{9}{10.8}{\rmdefault}{\mddefault}{\updefault}{\color[rgb]{0,0,0}}}}}}
\put(7201,-1276){\makebox(0,0)[b]{\smash{{\SetFigFont{9}{10.8}{\rmdefault}{\mddefault}{\updefault}{\color[rgb]{0,0,0}}}}}}
\put(676,-2986){\makebox(0,0)[b]{\smash{{\SetFigFont{10}{12.0}{\rmdefault}{\mddefault}{\updefault}{\color[rgb]{0,0,0}(vi)}}}}}
\put(2881,-2986){\makebox(0,0)[b]{\smash{{\SetFigFont{10}{12.0}{\rmdefault}{\mddefault}{\updefault}{\color[rgb]{0,0,0}(vii)}}}}}
\put(4861,-2986){\makebox(0,0)[b]{\smash{{\SetFigFont{10}{12.0}{\rmdefault}{\mddefault}{\updefault}{\color[rgb]{0,0,0}(viii)}}}}}
\put(6796,-2986){\makebox(0,0)[b]{\smash{{\SetFigFont{10}{12.0}{\rmdefault}{\mddefault}{\updefault}{\color[rgb]{0,0,0}(ix)}}}}}
\end{picture} }
\caption{Cases (vi)--(ix), corresponding to . In each case, the top-left region is , the top-right region is  and the bottom region is .
Thick lines represent .  Thin arrows represent a possible map . The elements of  and  are numbered from up to down as  and , respectively.}
\label{fig:figure3}
\end{figure}

\begin{enumerate}
\item[(vi)] If , let  be a set of  different elements of  with mates in . Set  and  for every .
\end{enumerate}
Assume that  is the union of  pairs. Then no element with mate in  is in . Using Property 2, the preimages under  of these elements are in~. In particular,  are in~. By repeatedly using Properties~1~and~2, starting on element , we conclude that all  are in , and that their images,  are not in . Furthermore, since the ~elements of  with mates in  are also in , we conclude that .

\begin{enumerate}
\item[(vii)] If , let  be a set of  different elements of  with mates in . Set  and  for every .
\end{enumerate}
Assume that  is the union of  pairs. Then every element with mate in  is in . Using Property~2, the images of these elements are not in~. This is, the elements  are not in . By repeatedly using Properties 1 and 2, starting on element , we conclude that .
\begin{enumerate}
\item[(viii)] If , let  be an element in  with mate in , and  be an element in  with mate in . Set ,  for every  and .
\end{enumerate}
If  is the union of  pairs, then every element with mate in  is in . In particular,  and, by Property 2, its image  is outside . By repeatedly using Properties 1 and 2, starting on element  we get . This proof also holds in the degenerated case where .

\begin{enumerate}
  \item[(ix)] If . Set ,  for every .
\end{enumerate}
For this final case, let  be a union of  pairs. If  then it is easy to see, using Properties 1 and 2, that . Similarly, if  then .

We have proved that for every case, and for every , if the set  is a union of pairs then . Thus,  is always a base of , and therefore  is an strongly base orderable matroid. \qedhere
\end{proof}

\paragraph{Acknowledgment} The author would like to thank Jan Vondr\'ak for helpful discussions.

\bibliographystyle{abbrv}
\bibliography{biblioMatroidMatching}

\end{document} 