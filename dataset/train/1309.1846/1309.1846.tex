\begin{filecontents*}{example.eps}
gsave
newpath
  20 20 moveto
  20 22 lineto
  22 22 lineto
  22 20 lineto
closepath
setlinewidth
gsave
  .4 setgray fill
 grestore
stroke
grestore
\end{filecontents*}
\RequirePackage{fix-cm}

\documentclass[smallextended]{svjour3}       \smartqed  \raggedbottom
\usepackage{amssymb}
\usepackage{graphicx}
\usepackage[inner=2.5cm,outer=2cm]{geometry}
\usepackage{algorithm}
\usepackage{algpseudocode}
\usepackage{algpascal}
\begin{document}

\title{Approximation Algorithms for the Load Balanced Capacitated Vehicle Routing Problem}





\author{Haniyeh Fallah  \and
        Farzad Didehvar  \and
        Farhad Rahmati  }



\institute{Corresponding author: Farzad Didehvar \at
                   \email{didehvar@aut.ac.ir}       
           \and
        Haniyeh Fallah  \at
              \email{hfallah@aut.ac.ir} 
             \and
            Farhad Rahmati \at
          \email{frahmati@aut.ac.ir} 
          \and
          1.  Department of Mathematics and Computer Science, Amirkabir University of Technology, Tehran, P. O. Box: 15875-4413, Iran} 


\date{Received: date / Accepted: date}



\maketitle
\begin{abstract}
\noindent We study the load balanced capacitated vehicle routing problem (LBCVRP): the problem is to design a collection of tours for a fixed fleet of vehicles with capacity  to distribute a supply from a single depot between a number of predefined clients, in a way that the total traveling cost is a minimum, and the vehicle loads are balanced. The unbalanced loads cause the decrease of distribution quality especially in business environments and flexibility in the logistics activities. The problem being NP-hard, we propose two approximation algorithms.\\
\noindent When the demands are equal, we present a approximation algorithm that finds balanced loads. Here,  is the approximation ratio for the known metric traveling salesman problem (TSP). This result leads to a  approximation ratio for the tree metrics since an optimal solution can be found for the TSP on a tree. We present an improved approximation algorithm. \\
 \noindent When the demands are unequal, we focus on obtaining approximate solutions since finding balanced loads is NP-complete. We propose an algorithm that provides a approximation for the balance of the loads. \\
\noindent We assume a second approach to get around the difficulties of the feasibility. In this approach, we redefine and convert the problem into a multi-objective problem. The algorithm we propose has a 4 factor of approximation. 

\keywords{Capacitated Vehicle Routing Problem\and Load Balanced Capacitated Vehicle Routing Problem \and Approximation Algorithms\and Fairness}
\end{abstract}

\section{Introduction}
\label{intro}
\subsection{Problem definition}
\indent The capacitated vehicle routing problem (CVRP) is a well-known and fundamental problem in the domain of operations research. It has many applications that are mostly related to transportation, logistics industry, communications, manufacturing, military and relief systems, etc. The CVRP is a graph problem that can be defined as follows: the quantity  should be delivered to the client  from a single depot using a vehicles fleet with capacity . A CVRP solution is a set of tours: each tour starts and ends at the depot after visiting a set of clients in a way that the sum of the clients' demands on a tour (the tour load) does not violate the capacity constraint. Each client must be visited exactly once. The objective is to minimize the total traveling cost. A special version is when each client demand is unit (i.e.  for a client ). In this case, the capacity restriction is given as a limit for the maximal number of clients in a vehicle tour. A unit demand CVRP is also known as the  (ECVRP). When the clients are allowed to be visited by more than one vehicle, we have the split delivery vehicle routing problem \cite{Dror}. This problem can be reduced to a unit demand CVRP by replacing a client  with demand  by  clients with unit demands and zero inter-distances. For an overview of the rich literature on the CVRP, we refer the reader to \cite{Altinkemer1990,Altinkemer1987,Baldacci,Bompadre,Chandran,Frederickson,Haimovich1988,Haimovich1985,Laporte,Li,Marinakis}. \\
\indent  A variety of the CVRP considers additional optimality objectives such as the balance of the routes (see \cite{Zhou}), the balance of the vehicle loads (see \cite{Bowerman}) or the minimum number of the vehicles (see \cite{Bowerman}). Chen et al. \cite{Chen2008} studied the CVRP with load balancing and time windows. Tsouros et al. \cite{Tsouros} considered the CVRP with load and route balance. Lee and Ueng \cite{Lee} studied the CVRP with load balancing where an attempt is made to balance the workload of the drivers. Bowerman et al. \cite{Bowerman} studied the CVRP with load and route balancing. In this paper, we consider the CVRP with load balancing. We denote the unequal demand version by LBCVRP and identify the equal/unit demand version by LBECVRP.\\
\indent Load balancing in a CVRP has practical applications in situations where it is required to distribute services fairly e.g. the collection of school children on a school bus \cite{Bowerman}. In their paper, Bowerman et al. \cite{Bowerman} investigated the design of school bus routes in an urban setting in Canada. Alongside the goal of minimizing the route numbers, they also followed further goal of adjusting the students number served in each route. The authors proposed a multi-objective optimization approach that minimizes the number of routes and at the same time balances the number of students. According to their assertions, the load balancing reduces the likelihood that the routes will be overfilled with an additional load, when fresh students enter into a school attendance zone, or if school attendance zones are re-expressed and redefined when adjusting pre-existing vehicle routes. \\
\indent In this article, we provide approximation algorithms for the LBECVRP and the LBCVRP since they are NP-hard problems. As per our knowledge, these problems have not yet been addressed by the approach of approximation algorithms. We use the approach of Vazirani \cite{Vazirani} to define an approximation algorithm and approximation guarantee. An approximation algorithm obtains a feasible solution in polynomial time and guarantees the solution quality.   \\
\indent Here, we restrict ourselves to tree metrics and we refer to it as the tree load balanced capacitated vehicle routing problem (TLBCVRP). In this case, each vehicle tour could be characterized by a tree (rooted at the depot) and a vehicle may pass through a node without serving it. Naturally, tree networks emerge in situations where access construction is more expensive than routing costs e.g. in pit mine railways and river networks (see \cite{Basnet,Labbe}). Further, CVRP on trees has applications in flexible manufacturing environments \cite{Berger}. In addition to the practical applications, the TLBCVRP is compelling theoretically. We are interested to know whether this problem can be solved more efficiently than the general LBCVRP. Here, we provide an improved approximation algorithm when the demands are equal. Through this article, we denote the equal demand version by TLBECVRP.
\subsection{Our results}
\indent  We give a load balancing algorithm that finds the balanced loads, in the equal demand case (LBECVRP). Given an instance of the LBECVRP and its balanced loads, we present the first approximation algorithm, where  is the approximation ratio for the known metric traveling salesman problem (TSP). Our algorithm is based on a method called route-first cluster-second method  (\cite{Beasley}) for the vehicle routing problems. This result leads to a approximation algorithm for the TLBECVRP, since TSP can be solved optimally on trees. We present an improved approximation algorithm. This algorithm proceeds in a sequence of rounds and partitions the tree into subtrees whose loads are balanced. Since the balanced loads differ in at most once, we reform the tree and prepare some strategies to partition it (see section 3). \\
\indent In the unequal demand case, finding balanced loads is NP-complete (see section 2.3), so we provide an approximate solution. We show there is a  bi-criteria approximation algorithm that finds a solution whose cost is  times the optimal and the balance of the loads is 4-approximation of the optimal balance. When the metric space is a tree, we show there is a  bi-criteria approximation algorithm. In a second approach, we redefine and convert the problem into a multi-objective problem and present a approximation algorithm.
\subsection{Related works}
\indent  The LBCVRP belongs to the class of combinatorial problems, known as balanced optimization problems. Martello et al. \cite{Martello} characterized a general class of these problems, for the first time. Since then, many variations and generalizations of them were studied (see e.g. \cite{Ahuja,Aydin,Cappanera}). Larusic and Punnen \cite{Larusic} studied the balanced traveling salesman problem which has applications in modeling optimization problems where it is important to equitably distribute resources. Some other assignment is allocated to Bassetto and Mason \cite{Bassetto} where a 2-period balanced traveling salesman problem is regarded in Euclidean graphs. A variety of the balanced vehicle routing problems have been studied by authors (see e.g. \cite{Borgulya,Chen2009,Chen2008,Lee,Matl,Tsouros,Yousefikhoshbakht}). \\ 
\indent The first approximation analysis of the CVRP is done by Haimovich and Rinnooy Kan \cite{Haimovich1985}. They presented a lower bound on the optimal cost of the ECVRP and derived approximation results based on this bound. Also, they provided the first PTAS for the ECVRP on Euclidean metrics. Further research articles (e.g., \cite{Altinkemer1990,Altinkemer1987,Li}) depend on this bound to improve or generalize approximation results for the CVRP (equal or unequal demands). Altinkemer and Gavish \cite{Altinkemer1987} have proposed a approximation algorithm for the CVRP which improves the result of Haimovich and Rinnooy Kan \cite{Haimovich1985}. Here,  is an approximation factor for the known metric Traveling Salesman Problem (TSP). The best result for  is  in general metrics \cite{Christofides}, and  for Euclidean metrics \cite{Mitchell}. In 1990, Altinkemer and Gavish \cite{Altinkemer1990} presented a approximation algorithm for the ECVRP. Recently, Bompadre et al. \cite{Bompadre} prepared lower bounds for the CVRP (equal and unequal demands) which improves the bound of Haimovich and Rinnooy Kan \cite{Haimovich1985}. For the CVRP with multiple depots, Li and Simchi-Levi \cite{Li} have presented a approximation algorithm. A version of the CVRP called delivery is considered by Charikar et al. \cite{Charikar}. They derive a approximation, for this problem. In the latter, a single vehicle with capacity  transports  items located at arbitrary locations to given demand points. The CVRP on trees (TCVRP) is introduced by Labbe et al. \cite{Labbe}. They proved NP-hardness of this problem and presented a approximation algorithm. Another approximation algorithm is given by Chandran and Raghavan \cite{Chandran}. Hamaguchi et al. \cite{Hamaguchi} studied the split delivery vehicle routing problem on trees and provided a approximation algorithm. Also, they showed that it is an NP-hard problem. Asano et al. \cite{Asano2001} improved this ratio to . Recently, Becker \cite{Becker} presented a tight approximation for this problem. \\

\indent  This paper is organized as follows. In section 2, we give a precise formulation of the problem. In section 3, we address the LBECVRP in general and on trees. We give a load balancing algorithm that finds the balanced loads. Then, we present two approximation algorithms. We study the general case for the unequal demands (LBCVRP) in general metrics and on trees, in section 4. We also discuss the multi-objective version of the LBCVRP, in this section. Finally, we summarize the results in the conclusion section.
\section{Problem Formulation}
\indent The LBCVRP is defined as follows. Let  be a graph with the set of  vertices , and the set of edges , and let  be the single depot. Each edge  has a non-negative weight  that represents its length/cost. The edge weights are symmetric (i.e., for each , ), and satisfy the triangle inequality. Clients are located at the vertices of the graph, and a client at  has a positive demand . Here,  denotes the set of natural numbers. We assume  is a complete graph unless otherwise stated. In this paper, the terms nodes, vertices, and clients have the same meaning.\\
\indent A fixed fleet size of  identical vehicles, located at the depot, has to be routed to supply the clients' demands. Each vehicle tour starts and ends at the depot after visiting a subset of the clients such that each client is visited exactly once (i.e. its demand is satisfied entirely by a single vehicle). For a tour , the load  is the sum of the demands of the nodes contained in , and the cost/length  is the sum of the edge weights incident on it. A load of each vehicle tour does not exceed the capacity constraint . Note that, a lower bound for the number of required vehicles is .\\
\indent A solution , of the LBCVRP corresponds to a partition  which verifies the following relations:

\noindent We use the term "capacitated partition" to refer to such a partition. Indeed, for each , we have . We say,  allocates the clients  to the  vehicle.\\
\indent Let  be the vehicles' loads in ,  may be described as the "load allocation vector" of . Henceforth, when we refer to a load allocation vector , implicitly we assume there is a solution  whose load allocation vector , is equal to : . \\
\indent The objective of the LBCVRP minimizes the total traveling cost and at the same time balances the vehicles' loads. We need to define a "balanced solution" and a "balanced load allocation". The following section gives some preliminaries regarding the balance criteria. We propose a mathematical model for the LBCVRP in section 2.2.
\subsection{Balance Criteria}
\noindent We use the term equity measure/function to refer to an index value that is calculated for a given load allocation. Range fairness is an accepted formulation of the fairness notion, in the load balancing area (see e.g. \cite{Chen2008,Tsouros}). The equity measure, in this case, is the difference between the largest and the smallest load. Ratio fairness (in the sense of the ratio between the largest and the smallest load) is another equity measure. Yousefikhoshbakht et al. \cite{Yousefikhoshbakht} used the ratio measure in a route balanced capacitated vehicle routing problem. \\
\indent One of the widely accepted properties of the equity measures is the Pigou-Dalton (PD) principle \cite{Cowell}, that is applied when the number and sum of outcomes of the allocations are identical: Let   be an allocated vector (e.g.
 the vehicle loads) and  be the equity function. Let   be formed as follows:  , ,   for all . The weak PD principle expresses that if , then . In the strong version, the inequality is strict () i.e. the new allocation should be more equitable. \\
\indent A simple calculation shows that either the range or the ratio criteria satisfy the weak PD principle. In this paper, we use the ratio measure to evaluate the fairness/balance of the allocated loads. The range measure could be used to obtain similar results. \\
\indent Given the number of vehicles , and a real parameter , a partition of vertices of  into  parts  is said to be -balanced (in the ratio sense) if and only if for each :

\noindent where . This condition simply implies that: 

\indent We use the "balanced condition" to refer to the inequality (1). It is evident that if a partition is balanced, it is also balanced for each . The minimum  for which this condition holds is called the "balanced ratio" of the set . So, 
 
\noindent is the balanced ratio of the set . 
\begin{definition}Let  be a solution for the LBCVRP with the corresponding partition . We say,  is the balanced solution, iff the balanced ratio of  is the smallest among the balanced ratios of all the capacitated partitions of  into  parts. Also,  is balanced, iff  is . The balanced ratio of  is defined to be the balanced ratio of . The load allocation vector of a balanced solution is a balanced load allocation. 
\end{definition}

 \indent For a given load deviation of ,  , a simple argument shows that if   is chosen and the set   is -balanced (in the ratio sense), then it is also balanced (in the range sense) with deviation , that is, . \\
 \\
 \indent In this paper, the notation  is used to describe a set of solution for the problem under consideration,  denotes a load allocation vector,  denotes a vehicle tour,  is the balanced ratio of a solution,  is used to denote the traveling cost,  is the number of available vehicles, and  is a tree.
\subsection {Mathematical Model}
\indent We assume the depot corresponds to the vertex  and  represents the set of clients. Moreover, we assign the demand  to the depot. Let  be the binary variables on the edges  that decide whether the edge  is presented in the route of the  vehicle. An edge  is presented in the route of the  vehicle if  and is not presented otherwise. The following is an integer programming formulation of the LBCVRP:\\
\\
\indent  Equality equations (3) ensure that all the vehicles depart from the depot. The constraints (4) indicate that a vehicle will depart from a client after visiting it. The constraints (5) specify the fact that each client is visited exactly once. Constraints (6) indicate that a vehicle cannot carry more than its capacity Q. The balance restriction for loads of the vehicles is guaranteed by the constraints (7). Restrictions (8) indicate that each edge in the graph has the value 1, if it is used, and 0 otherwise. \\
\indent  By the above formulation, the LBCVRP is programmed as the basic CVRP with the extra constraints of the balanced condition (constraints 7). To avoid ambiguity, we call this problem , when  is fixed. \\
\indent Using the equations (2)-(8), the inequalities (7) impose feasibility on the solutions. We will show finding a feasible solution for the , that satisfies the related balanced condition, is NP-complete (see sub-section 2.3). Let us consider an instance  of the LBCVRP where the distances between the locations are identical, and the clients' demands are as indicated in Fig. 1. There are a fleet of 2 identical vehicles in the depot, each having a capacity of 5. All possible solutions are listed below: , . The balanced ratio of the set  is , and the balanced ratio of  is . The minimum ratio is related to , and there is no solution with a balanced ratio smaller than . So,  is a balanced solution. \\
\indent One of the major problems is to find the lowest value of  for which the  has a feasible solution. We show a decision version  of this problem that asks to find whether it has a feasible solution, is NP-complete (see section 2.3), so determining the minimum value of  is NP-hard. \\
\begin{figure}[ht]
\begin{center}
\includegraphics[scale=0.28] {Fig1.jpg} 
\end{center}
\caption{An instance of the LBCVRP that has no solution with a balanced ratio smaller than 1/3.}
\label{fig1}
\end{figure}
\indent  The LBCVRP could be analyzed from another viewpoint as a multi-objective problem (see \cite{Bowerman}). This approach could be a good way to round around the difficulties of the feasibility. There are two objectives:\\
 \noindent : Minimization of the total traveling cost.\\
 \noindent : Minimization of the function .\\
 \indent The objective function  has been proposed by Bowerman et al. \cite{Bowerman}. We define the objective function , for . The parameter weightings effects can be examined by decision-makers to find the desired set of vehicle tours. For example, if load balancing introduces costly tours, then one can reduce the emphasis on load balancing. Similarly, if a desirable solution cannot be reached by the cost measure, then one can increase the weighting of the cost measure to force the tours to become inexpensive. We present a 4-approximation algorithm for the LBCVRP with the aforementioned objective function. 
\subsection{Complexity Analysis}
\noindent Here, we show the  is NP-hard. We need to prove that its decision version is NP-complete (see \cite{Garey}). Let   for  be a decision version of this problem that asks to find whether it has a feasible solution. The following theorem indicates   is NP-complete for each , . 
\begin{theorem}
For each , ,   is an NP-complete problem.
\end{theorem}
\begin{proof}
\noindent  First, we show  is NP-complete by a reduction to the partition problem. The partition problem is the task of deciding whether a given set  of positive integers has a sub-set   such that . This problem is a known NP-complete problem. We define  to be the set of clients and the depot. For each , we assign the demand . Let the distances between the clients and the depot be the same and the depot includes two vehicles each of capacity , . This problem is an instance of  and has a feasible solution, iff the partition problem has a solution. Furthermore, this reduction can be done in polynomial time. Hence, the  is an NP-complete problem. \\
\indent Now, we show  is NP-complete for each , . We reduce  to the  in polynomial time. Let  be a given instance of the  that has a set of clients  with demands . The number of vehicles in the depot is , and the capacity constraint is . We construct an instance  of  similar to , and just restrict the vehicle capacity constraint to . Clearly,  is a feasible solution of , iff it is a feasible solution of . This proves that for each ,  is NP-complete.\\
 
\end{proof}

\indent We conclude (from theorem 1) that finding the smallest value of  for the LBCVRP is NP-hard. In Section 3, we give an algorithm called "load balancing algorithm" that finds (in polynomial time) the smallest value of  for the LBECVRP. However, the LBECVRP is still NP-hard, since it is a generalization of the ECVRP. The complexity of solving the ECVRP relies on the capacity of the vehicles. When the vehicle capacity is 2, the ECVRP can be solved in polynomial time \cite{Asano1996}. However, it is NP-hard for any  \cite{Asano1996}. Albeit, the authors in \cite{Asano1996} showed that the ECVRP is indeed APX-complete for any , i.e., there exists  such that no  approximation algorithm exists unless .
 \section{LBECVRP}
\noindent In this section, we consider the LBECVRP and show that the fairest solution can be found. We present two approximation algorithms for the LBECVRP in general metrics and on trees that find fairest solutions. First, we define what we mean by the fairest solution, and then explain how we can find such a solution.\\
\indent We seek a set of vehicle tours that start from the depot and terminate there after visiting a set of clients. So, a routing has two components: the traveling route of a tour and the allocation of the clients that the tour will visit. The allocation of the clients should be as fair as possible (in the ratio sense). \\
\indent Let  be a load allocation vector, that is sorted in non-decreasing order, we define . We say  is the fairest (in the ratio sense) if the tuple  is the fairest in the min-max sense. Intuitively, the min-max fairness can be derived from the following approach: first, make sure that the maximum ratio is as small as possible and then ignoring this maximum ratio, make sure that the ratio between the load of the vehicles that can still get additional load and the minimum load is the smallest, and so on.\\
\indent Moreprecisly, let  and  be two tuples, each sorted in non-increasing order. We say  is lexicographically smaller than , if  or there is some index  for which  and  for all . We define a partial order (denoted by ) on the set including all the load allocation vectors. Given two load allocation vectors  and , we say  is fair than  (written ), iff  is lexicographically smaller than . We will say  and  are equivalent, iff both  and . So,  and  are equivalent, iff . Thus, the relation  is a total order on the equivalence classes of the load vectors. The fairest allocations are the vectors in the unique minimal equivalence class under . 
\begin{definition} Let  be a solution for the LBECVRP, and  be the allocated loads. We say  is the fairest solution, iff  is a fairest load allocation vector, that is,  belongs to the unique minimal equivalence class under .
\end{definition} 

\indent Definition 2, brings the relationship between the balanced solution and the fairest solution. The fairest solution has the smallest balanced ratio among the balanced ratios of all the capacitated partitions of the vertices of  into  parts. So, the fairest solution is indeed the balanced solution. However, the balanced solution may not be the fairest. In the following section, we present an algorithm called "load balancing algorithm" that explains how we can find such a fairest solution.  
\subsection{Load Balancing Algorithm}
\noindent In this section, we consider the following load balancing problem. Let  be a set of items each with volume of , and  be a set of bins. We assume each item ,  can be put in any bin in the set . All bins have the same capacity, denoted by . We wish to put each item in a bin, and our optimum solution   specifies the number of items (total volume) each bin  receives. A packing is a function  that puts each item  in a bin in . Note that, the items correspond to the clients, and the bins relate to the vehicles. So, the optimum solution  is indeed a load allocation to the  vehicles. Algorithm 1 finds the optimum packing for this problem. Lemma 1 shows this packing is indeed the best possible.\\
\indent More generally, each item  can have a different volume : each item should be put in a unique bin. This problem can be encoded in the un-splittable assignment problem (see \cite{Lenstra} for a detailed definition of the un-splittable assignment problem). \\


\begin{algorithm}

\caption{Load balancing algorithm}

 A set of  identical items that should be packed into a set of  bins.\\
 Optimum packing .
\begin{itemize}
\item[]  
\item[] Designate  items to each bin.

\item[] Distribute the rest items to  bins randomly. 

\item[] Sort the allocated loads in non-decreasing order to find the optimum packing .
\item[] 
\end{itemize}

\end{algorithm}

\indent For example, assume there are  items and  bins each with a capacity of . The best packing found using the algorithm is ; another packing  may be e.g. . In a packing , let  be the number of packed items into the bin , we will also refer to  as the degree of . Let  be the maximum degree of the optimum packing . In the above example, we have , also the value of the maximum degree of  is . Lemma 1 shows that the packing  is indeed the best possible. 
\begin{lemma}The load balancing algorithm finds an optimum packing of the items to the bins. 
\end{lemma}
\begin{proof} Let   be the optimum packing found by the algorithm, and  be some other packing, each sorted in non-decreasing order. The packings  and  are indeed two different load allocation vectors. Let  and , each are sorted in non-increasing order. We see the entries in  are equal to  or . Let  be the number of bins with degree . If , obviously the packing  is the best possible. So, we assume that .\\

\noindent  .
\begin{proof}Otherwise, let , so we have . This is in contradiction with the fact that:  and .
\end{proof}

\indent When the largest entry  of  is equal to , the smallest entry  is at most , so we have . The equality   occurs when . If , there exists an index  for which  because of . As it is stated before, the entries in  are equal to  or , so we have either , or . It is easy to see that both the cases imply: . This proves that the tuple  is lexicographically smaller than . Thus,  is a fairest load allocation vector.\\
 
\end{proof}

\indent  We assume   is the sequence of the vehicles' loads found using the load balancing algorithm. There are two possible cases:  or . In the following, we assume , and endeavor to find a set of tours whose loads are equal to . 
\subsection{LBECVRP in general metrics}
\noindent Our algorithm employs a tour partitioning heuristic (see \cite{Beasley}) to find the fairest solution (see algorithm 2). It has a similar flavor as the heuristics given in \cite{Altinkemer1990,Altinkemer1987,Haimovich1985}.  \\

\begin{algorithm}
\caption {} 
 An instance of the LBECVRP together optimum loads .\\
 A set of feasible vehicle tours whose loads are balanced.
\begin{itemize}
\item[] 
\item[] Use the Christofides algorithm \cite{Christofides} to find a salesman tour   spanning the vertices in .  Let  be the vertices in the order of their representation in . 
\item[]   to  
\begin{itemize}
\item[] Begin at  and find the following tours:\\
  . 
\item[] Find the total traveling cost of .
\end{itemize}
\item[] 
\item[]  Return the solution ,   whose total traveling cost is the smallest. 
\item[] 
\end{itemize}
\end{algorithm}
\indent In the following, we prove that the traveling cost of  is within  times the optimal value  of the LBECVRP. First, we prove a lemma to give a lower bound for the . Here,  is the edge cost connecting  to  for a node .
\begin{lemma}

\end{lemma}
\begin{proof}
\indent Let  be an optimal solution for the LBECVRP and . Let  and  be the traveling cost of . Since , we have:

\noindent Summing over all the tours in , we obtain:


\end{proof}
\begin{theorem}
The approximation factor of the algorithm 2 for the LBECVRP is .
\end{theorem}
\begin{proof}
Let  be the cost of the traveling salesman tour  covering the vertices in . Since the Christofides algorithm is used to obtain , we have . Indeed, the cost of an optimal traveling salesman tour is a lower bound for the . We see that each node  appears at most once as the first node of a tour and at most once as the end node of a tour during  iterations of the partitioning procedure. Hence, each edge  appears at most 2 times during the iterations. Besides, the edge  is not included in the solution where  appears as the first node of a tour. Thus,

\noindent Note that, if  does not appear as the first node of any tour during the  iterations of the partitioning procedure, we then have:  due to the triangle inequality. Hence, the right-hand side of (9) is an upper bound for the total traveling cost. Since  has a minimum traveling cost among the others, we have:

\noindent Besides, according to the load balancing algorithm, there are two cases: (a) , (b) .\\
\indent  First, assume . Hence,

\noindent The second inequality follows from the fact that . The third inequality follows from lemma 2. 

\indent Now, let . Since , we have . Hence,

\noindent Here,  is the factor of approximation for the TSP.\\
 
\end{proof}
\noindent  When the metric space is a tree, TSP can be solved optimally. Thus, the approximation factor of the above algorithm is  for tree networks, when . In the following, we present an improved -approximation algorithm for the TLBECVRP.
\subsection{LBECVRP on Trees}
\noindent In this section, we present a approximation algorithm for the TLBECVRP. In this case, a vehicle tour can be characterized by a rooted subtree, and so more than one vehicle may pass through the nodes of the tree. \\
\indent  Let  be a tree with the set of vertices  and the set of edges . We assume  is binary (as otherwise, we can split high degree vertices, and add zero-length edges), and rooted at the depot . We denote the unique path between the vertices  by , and identify its cost by . For any , the cost  is the sum of the edge weights in . Here,  is the subtree rooted at . The load  is defined similarly: . A solution to the TLBECVRP consists of a set of tours. We specify a tour by a set of clients that the vehicle visits. In fact, for a given set , an optimal tour for  can be trivially obtained: first compute a minimal subtree that spans , and then perform a depth-first search starting at vertex . Thus, we describe the  vehicle tour by , considering the fact that all the demands are equal to . \\
\indent Assume  is the sequence of the fairest loads and . We define a lower bound for the edge  in an optimal solution as follows:

\indent  is a lower bound for the edge , since at least  number of vehicles is required (for any solution) to serve the clients in , and each vehicle traverses  at least twice. In the following lemma, we give a lower bound for the optimal cost of the TLBECVRP.
\begin{lemma}
, is a lower bound for the optimal cost of the TLBECVRP.
\end{lemma}

\indent Our algorithm (see algorithm 3) is similar to the algorithms given in \cite{Asano2001,Hamaguchi,Naoki}. It proceeds in a sequence of rounds. Let  be the number of loads that are equal to , in the fairest load allocation. In the first  rounds, the algorithm finds some tours whose loads are equal to . In each round, it focuses on a particular -feasible node  which is -minimal and proposes a strategy. A vertex  is known as feasible if , and is known as minimal, if it is -feasible but none of its children is. In the next rounds, it finds some tours whose loads are equal to . In these rounds, we reform the tree to reason better about the structure of the resulting instance. We use the reforming operations given in \cite{Asano2001} and modify them ensuring that they preserve the lower bound and do not decrease the optimum cost. These operations can be made safely at any point in the algorithm until one of a few cases arise. Each case has a corresponding strategy that finds a set of tours. A feasible solution in the modified tree has a corresponding feasible solution in the original tree. In the following, we briefly restate the reforming operations given by Asano et al. \cite{Asano2001} and refer the reader to that paper for more details.\\
 \indent  We assume that the vertices with positive demands are exactly the leaves: if some internal vertex  has positive demand , we add a vertex  with demand  and edge  of length zero and set  to zero. Besides, we assume no non-root vertex has a degree exactly two, as no branching would occur there, and the two incident edges can be spliced into one. In the following, we list the reforming operations. Here, we use the names given in \cite{Becker}. 
\begin{itemize} 
\item[]  The condense operation merges a subtree whose demand is at most  into an edge. Precisely, for an internal node  with parent  , and , the condense operation replaces  by an edge  whose weight is  and . Then, the vertex  has a degree exactly two, and the two incident edges can be spliced into one that has the weight . If , then we delete the edge  and add the edge  with the cost . The demand of  will be considered in the final round of the algorithm. 
\item[]  The unit operation merges leaves of nodes. Let  be a subset of leaves of a node . Here,  denotes the weight of the edge . The unit operation examines any pair of leaves  and merges them, if . Exactly speaking, it replaces the demand of  with  and the weight  with  and then removes the leaf  together with the edge . If , then we delete the edge  and add the edge  with the cost . The demands of  will be considered in the final round of the algorithm. 
\item[]  If a node  with parent   has two children  and , then the unzip operation deletes  and adds edges  with weights  for .

\indent Asano et al. \cite{Asano2001} introduced the notion of a p-node and q-node. A p-node is an internal node whose children are leaves, and the sum of their demands is between  and . A q-node  is an internal node for which the load of the subtree  (rooted at ) is at least , but none of the children of  has this property. Note that, the demands at the leaves of a q-node are smaller than .
\item[]  If a node  has a child of p-node , and some other leaves directly connected to , and for a leaf node , , then  is placed as the child of  with the edge of weight equal to . 
\end{itemize}

\indent  Performing the unit, unzip and merge operations ensures that a p-node has exactly three leaves. If it has four leaves, then their demands exceed  which is in contradiction with the definition of a p-node. If it has two children, then the unzip operation connects its children directly by its parent. If it has one child, then its degree is exactly two, so the two incident edges can be spliced into one. It is obvious that the lower bound is not increased by the above reformations. However, the upper bound increases, since the possible tours are restricted. There are three possible cases for a q-node, after the reforming operations (see Fig. 2). 
\begin{itemize}
\item[]  There exists more than one child of p-node in a q-node.
\item[]  There exists no child of p-node in a q-node.
\item[]  There exists only one child of p-node in a q-node.\\
\end{itemize}
 
\begin{figure}[ht]
\begin{center} 
\includegraphics [scale=0.38]{Fig2.jpg} 
\end{center}
\caption{Three possible cases for a q-node.}
\label{fig2}
\end{figure}
\indent Our algorithm performs in a sequence of rounds similar to that of given by Asano et al. \cite{Asano2001} and Hamaguchi and Katoh \cite{Hamaguchi}. The main difference between our algorithm and their algorithms is that we need loads of the vehicles to be exactly equal to  or  (our given bounds), and the number of loads that are equal to  is limited, while in their algorithms the number of vehicles is not limited and the vehicles' loads are at the peak of the given bound  (not exactly ).\\
\indent  In each round, we apply an appropriate strategy to partition the tree. We need to compute the cost of the tours needed by the strategy and the reduced quantity from the lower bound. Let  be the problem instance for which we will apply the strategy, and  be the lower bound of its optimal cost given in lemma 3. Let  be the problem instance obtained after applying the strategy, and  be the lower bound of its optimal cost. The reduced quantity from the lower bound is defined as . The final round is when we do not have any q-node. In this round, there are some leaves whose loads are equal to , and there are either at most two leaves whose loads are smaller than  or at most one p-node. Since the vehicles' loads are equal to , the load of the remaining tree is a multiple of . 

\begin{algorithm}
\caption {} 
 An instance of the TLBECVRP together optimum loads .\\
 A set of feasible vehicle tours whose loads are balanced.
\begin{itemize}
\item[] 
\item[]   Number of optimum loads that are equal to .
\item[]   to  
\begin{itemize}
\item[] Apply strategy 1 to find some tours whose loads are equal to .
\item[] Calculate the cost of the tours needed by the strategy and the reduced quantity from the lower bound. 
\item[] Compute the ratio between the cost of the tours found and the reduced quantity from the lower bound.
\end{itemize}
\item[] 
\item[]   
\begin{itemize}
\item[] Apply the reforming operations to the remaining tree until any of them cannot be applied.
\item[] Focus on a particular q-node  and apply an appropriate strategy among the strategies  to find some tours whose loads are equal to .
\item[] Calculate the cost of the tours needed by each strategy and the reduced quantity from the lower bound. 
\item[] Compute the ratio between the cost of the tours found and the reduced quantity from the lower bound.
\item[]  i is the final round (there is no q-node) 
\begin{itemize}
\item[] Apply strategy 6 to find a set of tours whose loads are equal to .
\item[] Calculate the cost of the tours needed by the strategy and the reduced quantity from the lower bound. 
\item[] Compute the ratio between the cost of the tours found and the reduced quantity from the lower bound.
\end{itemize}
\item[]  
\end{itemize}
\item[] 
\item[] Return the set including the feasible tours found.
\item[] 
\end{itemize} 
\end{algorithm} 
\begin{theorem} The approximation factor of the algorithm 3 for the TLBECVRP is .
\end{theorem}
\noindent  The proof is by induction on the number of rounds similar to that of given in \cite{Asano2001,Naoki}. We assume the theorem holds for the problem instances that require at most  rounds, and consider the problem instance  for which our algorithm requires  rounds. Let  be an instance of the TLBECVRP obtained from  after the first round whose lower bound is , and let  be the reduced quantity from the lower bound at this round. Let ,  and  denote the total cost that is required by our algorithm to performs on the original problem , the cost of our algorithm in the first round and the total cost that is required by our algorithm to performs on the remaining problem , respectively, (i.e., ). Then, we have

\noindent From the induction hypothesis, we have , so it suffices to prove

\noindent Lemmas 4 through 8 prove that this inequality holds in each case. Thus, we have the theorem. \\
\\

\indent Let  be the number of rounds of the algorithm. The proposed algorithm performs the following strategy iteratively for each . The algorithm finds a deepest vertex  s.t. the load of the subtree  below  is at least . So, the load of the subtrees hanging off the 's children are smaller than . Note that, we have assumed  is binary and rooted at . Let  be the children of  that are connected to  by the edges  and  with weights/costs  and . The algorithm considers subtrees  and  satisfying 

\noindent Without loss of generality, we assume . We split the vertices  of  into  and  so that  where .\\
\indent  We allocate a vehicle to serve the clients in  and . The computation of such  is straightforward: perform a depth-first search on  starting at  in a way that:\\
\noindent (1) Initially set , , .\\
\noindent (2) Every time a new vertex  is visited, if  holds, set  and , otherwise (if ), set  and  and stop. If  holds, set , .\\
\indent The cost to serve the clients in  is at most

\noindent The reduced quantity from the lower bound is given by

\noindent since for each edge  the reduced quantity from the lower bound  is , since the decrease of  is exactly . Hence, the ratio between the cost of the tours and the reduced quantity from the lower bound is given by

\noindent The clients in  are served in the  round and their demands will not be considered in the later rounds of the algorithm. Hence, we get the following lemma.
 \begin{lemma}
If , the approximation ratio is at most .
\end{lemma}

 \indent Now, assume . The algorithm first reforms the remaining tree according to the operations given above until no more operation can be applied. Then, it performs some strategies for each of the cases explained above. We will use the following simple fact:\\
 \\
. , for each  and .\\
 \\
\indent  In this case, the algorithm focuses on arbitrary one or two p-nodes. Let  be the q-node and  denote the two p-nodes. For the p-node , we denote by  the children of the subtree , by  their demands, and by  weights of the edges between  and its children. Also, we denote by  the weight of the (unique) path between  and . The cost of the edge between  and  is denoted by . We have: , ; , ; and so . We assume . Similarly, we denote by  the children of the subtree , by  their demands, and by  weights of the edges between  and its leaves. Here, we denote by  the cost of the edge between  and . In a similar way, by assumption, , ; , ; and  hold, and  is assumed. \\

\indent There are two cases depending on whether  holds or not. If this inequality holds, the case is called subcase 1A. If it does not hold, the case is called subcase 1B.\\

\indent  We prepare the following strategy. \\
\indent  The strategy allocates two vehicles to serve the full demands of  and a partial demand of . The first vehicle utilizes the full demand of  and a partial demand of  to be filled to the maximum load of . The second vehicle employs the full demand of  and the rest demand at  to be loaded to the utmost bar of . Note that, if  or , the vehicles do not use the demand at . Thus, the ratio between the cost of these vehicles and the reduced quantity from the lower bound is given by

 \noindent The first inequality follows from fact 1 and the assumption that . We observe that the leaves , , are indeed the subtrees of the primitive tree , and a partial demand at  could be computed straightforwardly as we stated in strategy 1. \\

 \indent  Let  and . Without loss of generality, we assume . We have either , or : since , , and . First, assume . We prepare the following strategy. \\
 
\indent  This strategy also allocates two vehicles to serve the demands of  and a partial demand of . The first vehicle utilizes the full demand of  and a partial demand of  to be filled to the maximum load of . The second vehicle employs the full demand of  and the rest demand at  and a partial demand at  to be loaded to the utmost bar of . The remaining demand of  will be considered in the next rounds. The ratio is as follows:


\noindent The first inequality comes from the assumptions ,  and fact 1. \\
\indent In the case that , the strategy is the same as strategy 3, where the role of  and  is exchanged. Since , a similar way could be used to show that the ratio between the cost of the tours and the decrease in the lower bound is at most . Hence, we get the following lemma.
 \begin{lemma}
For  and in case 1, the approximation ratio is at most .
\end{lemma}

\indent  In this case, a q-node has no p-node. It follows from the definition of a q-node that it has at least three and at most four leaves. So, there are two cases: In subcase 2A, there exist three leaves in a q-node  whose total demand is at least . In subcase 2B, there exist four leaves in a q-node whose total demand is at least . \\
\indent  In this case, a q-node  has three children whose total demand is at least . This case can be considered as a special case of subcase 1A where the cost of the edge  is equal to . Thus, we use strategy 2.\\
\indent  In this case, a q-node  has four children. This case can be considered as a special case of case 3 where the cost of the edge between  and  is equal to . Thus, we will investigate in case 3. We conclude the following lemma.
\begin{lemma}
For  and in case 2A, the approximation ratio is at most .
\end{lemma}

\indent  In this case, there exist (exactly) one child of p-node  in a q-node , and at least one child  other than . Let  be the children of the subtree  with demands , respectively. If , strategy 2 can be used. So, we assume . We notice here that  exceeds , since otherwise the merge operation can be applied to shift  down to the position of the child of . We assume . We denote by  the path length between  to , and by  the cost of the edge between  and .\\

\indent  We assume . The case where , can be considered as a special case of the subcase 1B in which the cost of the edge between  and  is equal to . Thus, we use the strategy 3, except that we exchange the role of  and . Since , a similar way could be used to show that the ratio between the cost of the tours and the reduced quantity from the lower bound is at most . \\
\indent If , we prepare the following two strategies depending on whether 

\noindent  holds or not. If the above inequality holds, the case is called subcase . If it does not hold, the case is called the subcase . \\

\indent  We prepare the following strategy.\\
\indent  The strategy employs two vehicles to serve the full demands at , ,  and a partial demand of . The first vehicle utilizes the demand of  and a partial demand of  to be filled to the maximum load of . The second vehicle employs the demand of  and the remaining demand at  and possibly a partial demand at  to be loaded to the utmost bar of . The ratio is as follows: 

\noindent The first inequality comes from fact 1 and the assumptions , .\\

\indent  If , the algorithm applies the following strategy. \\
\indent  The strategy employs two vehicles to serve the full demands at ,  and possibly a partial demand of . A vehicle uses the demand of  and possibly a partial demand of  to be filled to the maximum load of . The other vehicle employs the full demand at  and possibly a partial demand at  to be loaded to the utmost bar of . There are two cases: (a) , and (b) . In the case (a), the ratio is as follows:

\noindent The first inequality is due to the facts , and . In the case (b), the ratio is as follows:

\noindent The first inequality follows from fact 1 and the assumptions , . Thus, we conclude the following lemma.
\begin{lemma}
For  and in case 3, the approximation ratio is at most .
\end{lemma}

\indent  In this round, there are some leaves whose loads are equal to . Also, there are three possible cases: (a) there is exactly one leaf whose load is smaller than , (b) there are exactly two leaves whose loads are smaller than , (c) there is a p-node  for which .

\indent In the cases (a) or (b), if there is a leaf  for which , we allocate a vehicle to optimally serve the full demand at . In the case (b), let  be the leaves for which , , . Let , we allocate a vehicle to serve the full demand at  and a partial demand at . Then, the ratio is  


\indent Thus, we can assume there is at most one leaf whose load is smaller than . In the case (c), if , we use strategy , so we assume . 

\indent First, assume there is one p-node  for which , and some other leaves  with demands , . Let , and  be three leaves of . We denote their demands by , and . Let  be the edges connecting ,  to their parent . We denote their costs by  and denote by  the cost of the edge between  and . Without loss of generality, we assume . The algorithm applies the following strategy.\\

\indent  The strategy employs  vehicles to serve the remaining demands. Each vehicle load must be equal to . The  vehicle, , serves a partial demand of . The  vehicle serves the full demand of  and a partial demand of , and the  vehicle serves the full demand of  and the remaining demand  at  and the remaining demands at ,  so that its load is equal to . The  vehicle, , serves the remaining demands at . Note that, , since . We see the edges weights , , and , appear at most 4 times while the lower bound of each edge is exactly . Thus, the ratio is as follows:

\indent Now, assume there is a leaf with demand smaller than  except  leaves whose demands are equal to . The strategy is the same as strategy 6, except that the  vehicle serves the full demand of  and the remaining demands at , so that its load is equal to . The  vehicle  serves the remaining demands in the same way as we said in the strategy 6. We see each weight , appears at most 4 times while the lower bound on each edge is . Thus, the ratio is at most 2. So, we get the following lemma.
\begin{lemma}
For  and in the final round, the approximation ratio is at most . 
\end{lemma}
\indent In the following, we study an extension to unequal demand problems.
\section{LBCVRP}
 \noindent As we stated before, the LBCVRP can be encoded in the un-splittable assignment problem. In this case, finding the fairest allocation is NP-complete \cite{Lenstra}. So, we focus on obtaining approximate solutions. We use the following natural definition of the notion of approximation. Let  be a load allocation vector, that is sorted in non-decreasing order, and . Let  be the allocated loads in an optimal solution, sorted in non-decreasing order and . We say  is a -approximation to  (written ), iff for each  the value of the th largest entry in  is at most  times the value of the th largest entry in . We say the balance/fairness of  is a approximation of the balance/fairness of , iff  is a -approximation to . 
 \begin{definition} Let  be a solution for the LBCVRP with the allocated loads , and let  be an optimal solution with the allocated loads . We say the balance/fairness of  is a approximation to the balance/fairness of , iff  is a -approximation to . 
\end{definition} 

 \indent In the following, we show there is an algorithm that finds a solution for the LBCVRP whose fairness is a approximation to the optimal.
  \subsection{LBCVRP in general metrics}
  \indent Let  be a -approximation algorithm for the CVRP that assigns a new vehicle tour to the clients whose demands are at least . We show there is a polynomial-time  bi-criteria approximation algorithm for the LBCVRP that produces a solution. The total traveling cost of the produced solution is guaranteed to be about  times the optimal, and its balance is a 4-approximation to the balance of the optimal solution. The best-known result for  is  by the algorithm of Altinkemer and Gavish \cite{Altinkemer1987} (see lemma 9). 
 \begin{lemma}  \cite{Altinkemer1987}
There is a tour partitioning heuristic for the CVRP that provides a solution whose cost is within  times the optimal.
\end{lemma} 

\indent The Altinkemer and Gavish algorithm in \cite{Altinkemer1987} assigns a new vehicle tour for the clients whose demands are at least . Let  be the solution found using this algorithm, and  be its balanced ratio. We have  where . In the following, we attempt to modify the tours in , to obtain a better-balanced ratio. Define a heavy (light) tour to be a tour whose load is at least (smaller than) . We modify the tours by removing some nodes from a heavy tour, and appending them to the tour with the smallest load, in the same order they are appeared (see Fig. 3).  \\
\begin{figure}[ht]
\begin{center} 
\includegraphics [scale=0.28]{Fig3.jpg} 
\end{center}
\caption{Modification of the tours.}
\label{fig3}
\end{figure}\\
\indent Let  be the light tour with the smallest load and  be a heavy tour that serves at least two clients. Modifying the tours , , produces two new tours  , . The edges  are added to ,  and the edge  is removed (see Fig. 3). We see  where  is the optimal value for the LBCVRP. Thus, the load balance would be improved at the loss of a constant factor of one in the traveling cost approximation ratio. \\
\indent Assume  is the allocated loads in the modified solution (sorted in non-decreasing order), and . We compare  with a vector of ones , since in the best case, all values of the allocated loads are equal. The smallest load is at least  in the modified solution (see theorem 4), so  is a approximation of . Thus, the balance of the modified solution is a approximation of the balance of the optimal solution. Moreover, we see the balanced ratio, , of the modified solution is at most : .
\begin{theorem}
The smallest load is guaranteed to be at least , after modifying the solutions. 
\end{theorem}
\begin{proof}
\noindent We start by showing that for each tour , . Let   be the light tour and  be a heavy tour that contains at least two nodes. If , the proof is complete. So, let . It is easy to see that  since otherwise they will be merged. Hence,

\noindent Let  be the first index for which , and let   be the vertices that are removed from  joined to . Let  and  be the respective new tours. We have  and it remains to show that . \\
\noindent Otherwise, let . According to the above, the index  is the first index for which  (indeed,  ). Furthermore,  since the Altinkemer and Gavish algorithm assigns a new vehicle tour for the nodes with demand at least . We remind that  . Thus,

which is a contradiction.\\

\end{proof}
\subsection{LBCVRP on Trees}
\noindent  Chandran and Raghavan in \cite{Chandran} presented a approximation algorithm for the CVRP on trees. Let  be the set of subtrees found using this algorithm, and  be its balanced ratio. We have . In this section, we improve this algorithm to find a better-balanced solution for the TLBCVRP (see algorithm 4). Let  be a binary tree rooted at . We define a non-grandparent node to be a node that has only leaf nodes as children. For a node , the vertices  are its children, and  is the subtree hanging off .

\begin{algorithm}
\caption {} 
 An instance of the TLBCVRP.\\
 A set of feasible vehicle tours. 
\begin{itemize}
\item[] 
\item[] Remove the nodes with demand at least  and replace a dummy vertex with zero demand.
\item[] Assign a new vehicle tour for each of the removed vertices. 
\item[] Apply the algorithm of Chandran and Raghavan \cite{Chandran}:
\begin{itemize}
\item[]  there is a non-grandparent node 
\begin{itemize}
\item[] Choose a non-grandparent node  and apply a bin-packing heuristic to pack the demands in  into a minimal set of bins. The sum of the demands in each bin does not violate the capacity . 
\item[] Remove the nodes in the subtree  from the graph and replace some nodes that represent the packed bins. For each bin, add one node to the tree as the child of  (parent of node ).  
\item[] Continue this procedure until all items get packed. 
\end{itemize}
\item[] 
\end{itemize}
\item[] Remove some of the clients from the subtree with the highest load and append them to the tree with the smallest load:
\begin{itemize}
\item[]  a heavy tree that contains at least two nodes.
\item[]  (vertice of ).
\item[]  the tree with the smallest load. 
\item[] Split the vertices  into  and  so that 
\noindent where , . The existence of such   is guaranteed by Theorem 4. 
\end{itemize}
\item[] Return the set of tours found.
\item[] 
\end{itemize}
\end{algorithm}

 \begin{theorem} 
Algorithm 4 is a  bi-criteria approximation algorithm for the TLBCVRP.
\end{theorem} 
\begin{proof} Assume that  is the set of obtained tours before load balancing. We see that loads of the found tours are at least  except at most once. Thus, a proof similar to that of Chandran and Raghavan can be used to show that the total distance traveled by all the vehicles is at most twice that of the optimal. Indeed, for an edge , at least  number of vehicles is required for any solution to travel  since each vehicle load does not exceed . Becides, loads of the obtained tours are at least , so the number of vehicles that travel  is at most . Thus, the total traveling cost is at most twice the lower bound, so it is at most twice the optimal.\\
\indent Modifying the loads increases the total traveling cost at most , where  is the optimal value of the TLBCVRP. Hence, the load balance improves at the loss of a constant factor of one in the traveling cost approximation ratio. We conclude that there is  bi-criteria approximation algorithm for the TLBCVRP that provides a solution, whose total traveling cost is at most  times the optimal, and its load balance is at most  times the best one.\\

\end{proof}
\subsection{LBCVRP as a multi-objective problem}
\noindent One of the main drawbacks using the first approach (equations (2)-(8)), is the difficulty of the existence of the feasible solutions (see section 2). To round around this difficulty, we assume a second approach. We redefine and convert the problem into a multi-objective problem with the objectives of  and . We combine the objectives into a single objective using a convex combination of them and consider the objective function of  for . \\
\indent Our purpose in this section is to show there is an algorithm that provides a solution whose cost is within  times the optimal (see theorem 6).
\begin{theorem}
Using the multi-objective approach to address the LBCVRP, there is an algorithm that provides a solution whose cost is within  times the optimal.
\end{theorem}
\begin{proof}
\indent We use the results given in lemma 9. Let  be the solution found using the Altinkemer and Gavish algorithm. When  is the objective function, we use  to denote the optimal value. Also,  is used to denote the optimal value of the function . It could be observed that . According to the lemma 9, we have . Here,  is the traveling cost of .\\ 
\indent  For an allocation  of loads,  is defined as . Because  has a constant value,  is a minimum, if  for each , so we have  . In addition, , since  and . Note that,  is an upper bound for the required number of vehicles. Hence, . Moreover, we have  , so . Thus, using the function of , the approximation factor of the solution found is : 


\end{proof}
 \section{Conclusion}
\noindent We studied the load balanced capacitated vehicle routing problem (LBCVRP)  to design two approximation algorithms. We presented a mathematical formulation for this problem and provided an algorithm to find the balanced loads in the equal demand version (LBECVRP). Given the balanced loads, we presented a  approximation algorithm to find a solution whose loads are balanced. Here  is the factor of approximation for the TSP. When the graph is a tree, the total traveling cost is at most  times the optimal, since an optimal solution can be found (in polynomial time) for the TSP on a tree. We showed there is an improved approximation algorithm in this case. \\
\indent The direction of future research is to study the LBECVRP on Euclidean metrics. Theoretically, we are interested to know whether the LBECVRP can be solved more efficiently on Euclidean spaces. \\
\indent In general problem of the LBCVRP, we focused on obtaining approximate solutions since deciding whether there is a feasible solution is NP-complete. We presented a  bi-criteria approximation algorithm. The approximation factor for the traveling cost of the obtained solution is , and the balance of the obtained loads is 4-approximation of the optimal balance. When the metric space is a tree, we showed there is a  bi-criteria approximation algorithm for this problem.\\
\indent Furthermore, we investigated the LBCVRP from another viewpoint as a multi-objective problem. We showed there is an algorithm that provides a approximation.
\begin{acknowledgements}
The authors thank Amirkabir University of Technology for facilities.
\end{acknowledgements}
 The authors declare that they have no conflict of interest.

\begin{thebibliography}{}
 
\bibitem{Ahuja} Ahuja, R.K.: The balanced linear programming problem. European Journal of Operational Research 101, 29-38 (1997)

\bibitem{Altinkemer1990} Altinkemer, K., Gavish, B.: Heuristics for equal weight delivery problems with constant error guarantees. Transportation Science 24, 294-297 (1990) 

\bibitem{Altinkemer1987} Altinkemer, K., Gavish, B.: Heuristics for unequal weight delivery problems with a fixed error guarantee. Operations Research Letters 6, 149-158 (1987)

 \bibitem{Asano1996} Asano, T., Katoh, N., Tamaki, H., Tokushama, T.: Covering points in the plane by k-tours: a polynomial approximation scheme for fixed k. Research Report RT0162, IBM, Tokyo Research Laboratory (1996)

\bibitem{Asano2001} Asano, T., Katoh, N., Kawashima, K.: A New Approximation Algorithm for the Capacitated Vehicle Routing Problem on a Tree. Journal of Combinatorial Optimization  5, 213-231 (2001)

\bibitem{Aydin} Aydin, K., Bateni, M.H., Mirrokni, V.: Distributed Balanced Partitioning via Linear Embedding. 16 Proceedings of the Ninth ACM International Conference on Web Search and Data Mining 387-396 (2016)

\bibitem{Becker}  Becker, A.: A Tight 4/3 Approximation for Capacitated Vehicle Routing in Trees. arXiv:1804.08791v1 (2018)

\bibitem{Berger} Berger, I., Bourjolly, J.M., Laporte, G.: Branch and bound algorithms for the multi-product assembly line balancing problem. European Journal of Operational Research 58, 215-222 (1992)

\bibitem{Baldacci} Baldacci, R., Toth, P., Vigo, D.: Exact algorithms for routing problems under vehicle capacity constraints. Annals of Operations Research 175, 213-245 (2010)
  
\bibitem{Basnet} Basnet, C., Foulds, L.R., Wilson, J.M.: Heuristics for vehicle routing on tree like networks. Journal of the Operational Research Society 50, 627-635 (1999)

\bibitem{Borgulya} Borgulya, I.: An algorithm for the capacitated vehicle routing problem with route balancing. Central European Journal of Operations Research 16, 331-343 (2008) 

\bibitem{Bassetto} Bassetto, T., Mason, F.: Heuristic algorithms for the 2-period balanced Travelling Salesman Problem in Euclidean graphs. European Journal of Operational Research 208, 253-262 (2011)

\bibitem{Beasley}  Beasley, J.: Route first-cluster second methods for vehicle routing. Omega 11, 403-408 (1983)
 
\bibitem{Bompadre} Bompadre, A., Dror, M., Orlin, J.B.: Improved bounds for vehicle routing solutions. Discrete Optimization 3, 299-316 (2006)

\bibitem{Bowerman} Bowerman, R., Hall, B., Calamai, P.: A Multiobjective Optimization Approach to Urban SchoolBus Routing: Formulation and Solution Method. Transportation research Part A: Policy and Practice 29, 107-123 (1995)
 
\bibitem{Cappanera} Cappanera, P., Scutella, M.G.: Balanced paths in a cyclic networks: tractable cases and related approaches. Networks 45, 104-111 (2005) 
  
\bibitem{Chandran} Chandran, B., Raghavan, S.: Modeling and Solving the Capacitated Vehicle Routing Problem on Trees. Operations Research/Computer Science Interfaces 43, 239-261 (2008)
 
\bibitem{Charikar} Charikar, M., Khuller, S., Raghavachari, B.: Algorithms for capacitated vehicle routing. SIAM Journal on Computing 31, 665-682 (2001)  

\bibitem{Chen2009}  Chen, J., Zhang, L., Chen, S.: Modeling of Vehicle Routing Problem with Load Balancing and Time Windows. Logistics 2773-2778 (2009)
  
\bibitem{Chen2008} Chen, J., Chen, S.: Optimization of Vehicle Routing Problem with Load Balancing and Time windows in Distribution Wireless Communications. Networking and Mobile Computing (2008). DOI: 10.1109/WiCom.2008.1527

\bibitem{Christofides} Christofides, N.: Worst-case analysis of a new heuristic for the travelling salesman problem. Technical Report, 388. Graduate School of Industrial Administration, Carnegie Mellon University (1976) 

\bibitem{Cowell}  Cowell, F.A.: Measurement of Inequality. In: Atkinson, A.B., Bourguignon, F. (eds.) Handbook of Income Distribution, North Holland, Amsterdam (2000) 
\bibitem{Dror}  Dror, M., Trudeau, P.: Savings by split delivery routing. Transportation Science 23, 141-145 (1989) 

\bibitem{Frederickson} Frederickson, G.N., Hecht, M.S., Kim, C.E.: Approximation algorithms for some routing problems. SIAM Journal on Computing 7, 178-193 (1978) 
 
\bibitem{Garey} Garey, M.R., Johnson, D.S.: Computers and Intractability: A Guide to the Theory of NP-Completeness. (1979) 

\bibitem{Haimovich1988}  Haimovich, M., Rinnooy Kan, A.H.G., Stougie, L.: Analysis of heuristics for vehicle routing problems. In: Golden, B.L., Assad, A.A. (eds.) Vehicle Routing: Methods and Studies, pp. 47-61. North-Holland, Amsterdam (1988)

\bibitem{Haimovich1985} Haimovich, M., Rinnooy Kan, A.H.G.: Bounds and heuristics for capacitated routing problems. Mathematics of Operations Research 10, 527-524 (1985)  

\bibitem{Hamaguchi}  Hamaguchi, S., Katoh, N.A.: Capacitated Vehicle Routing Problem on a Tree. In International Symposium on Algorithms and Computation 397-406 (1998)

\bibitem{Labbe} Labbe, M., Laporte, G., Mercure, H.: Capacitated Vehicle Routing on Trees. Operations Research 39, 616-622 (1991)

\bibitem{Larusic} Larusic, J., Punnen, A.P.: The balanced traveling salesman problem. Computers and Operations Research 868-875 (2011) 

\bibitem{Laporte}  Laporte, G., Semet, F.: Classical heuristics for the capacitated VRP. In: Toth, P., Vigo, D.(eds.) The Vehicle Routing Problem. Society for Industrial and Applied Mathematics, pp. 109-128. Philadelphia (2001) 

\bibitem{Lenstra}  Lenstra, J.K., Shmoys, D., Tardos, E.: Approximation algorithms for scheduling unrelated parallel machines. In Proc. 28th IEEE Symposium on Foundations of Computer Science 217-224 (1987) 

\bibitem{Lee} Lee, T.R., Ueng, J.H.: A study of vehicle routing problems with load-balancing. International Journal of Physical Distribution and Logistics Management 29, 646-657 (1999) 

\bibitem{Li}  Li, C.L., Simchi-Levi, D.: Worst-case analysis of heuristics for multi depot capacitated vehicle routing problems. INFORMS Journal On Computing 2, 64-73 (1990)  

\bibitem{Marinakis}  Marinakis, Y., Migdalas, A., Pardalos, P.M.: A new bilevel formulation for the vehicle routing problem and a solution method using a genetic algorithm. Journal of Global Optimization 38, 555-580 (2007)

\bibitem{Martello} Martello, S., Pulleyblank, W.R., Toth, P., Werra, D. de: Balanced optimization problems. Operations Research Letters 3, 275-278 (1984)

\bibitem{Matl} Matl, P., Hartl, R.F., Vidal, T.: Workload Equity in Vehicle Routing Problems: A Survey and Analysis. Transportation Science 52, 229-496 (2017) 

\bibitem{Mitchell} Mitchell, J.S.B.: Guillotine Subdivisions Approximate Polygonal Subdivisions: A Simple Polynomial-Time Approximation Scheme for Geometric TSP, k-MST, and Related Problems. SIAM Journal on Computing 28, 1298-1309 (1999) 

\bibitem{Naoki} Naoki, K., Taihei, Y.: An approximation algorithm for the pickup and delivery vehicle routing problem on trees. Discrete Applied Mathematics 154, 16, 2335-2349 (2006)

\bibitem{Tsouros} Tsouros, M., Pitsiava, M., Grammenidou, K.: Routing-Loading Balance Heuristic Algorithms for a Capacitated Vehicle Routing Problem. Information and Communication Technologies  ICTTA '06. 2nd (2006). DOI: 10.1109/ICTTA.2006.1684915 

\bibitem{Vazirani} Vazirani, V. V.: Approximation Algorithms. Springer (2001) 

\bibitem{Yousefikhoshbakht}   Yousefikhoshbakht, M., Didehvar, F., Rahmati, F.: An Effective Rank Based Ant System Algorithm for Solving the Balanced Vehicle Routing Problem. International Journal of Industrial Engineering 23, 13-25 (2016) 

\bibitem{Zhou}  Zhou, W., Song, T., He, F., Liu, X.: Multiobjective Vehicle Routing Problem with Route Balance Based on Genetic Algorithm. Discrete Dynamics in Nature and Society 2013, 25686, 9 pages (2013) 

\end{thebibliography}









\end{document}
