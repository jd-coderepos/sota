In this section we present a number of measurements and benchmarks
performed on the iDataCool system.  We first describe our sensing and
monitoring facilities.  The liquid-cooling installation is constantly
monitored, and relevant system parameters are logged electronically.
On the node level, we read out the individual processor core
temperatures from chip-internal sensors, we estimate the water in- and
outlet temperature of each node using the original air-flow
temperature sensors (which we attached to the copper pipe), and we
monitor the DC power consumption of each node.  On the cluster level,
we measure the in- and outlet temperature and the AC power consumption
of the 3-rack installation.  Our instrumentation also allows us to
determine the combined AC power consumption of the iDataCool cluster,
the GPU cluster, water pumps, the adsorption chiller, and the dry
recooler.  To determine the flow rates in the different water circuits
we use various kinds of flow meters.  As for the accuracy of our
equipment, we estimate the node-level temperature sensors to be
accurate to about 1$^\circ$C/2$^\circ$F, while the cluster-level
temperature sensors, which are in direct contact with the water, are
specified to have an accuracy of 0.2$^\circ$C/0.4$^\circ$F.  The
ultrasonic flow meter for the rack cooling circuit is specified to
have an accuracy of 1\%, while the flow meters for the other circuits
are much simpler and only about 10\% accurate. 

When plotting quantities as a function of the cooling-water
temperature we have the choice of using the rack inlet or outlet
temperature.  We chose the outlet temperature $T_\text{out}$ since
this is the quantity of interest for energy reuse purposes.  The
difference between inlet and outlet temperature can be controlled by
adjusting the water flow rate and is about 5$^\circ$C/9$^\circ$F in
our system.\footnote{At constant water flow rate this temperature
  difference decreases somewhat with the outlet temperature since the
  system is not perfectly insulated from the environment.  At higher
  temperatures more heat is lost to the air.}  For constant rack inlet
temperature, $T_\text{out}$ fluctuates slightly depending on the load
of the cluster and on the control parameters.  In the figures, the
horizontal error bars in the $T_\text{out}$ direction reflect these
fluctuations in time.

Some of our measurements [those presented in Figs.~\ref{fig:core},
\ref{fig:power}, and \ref{fig:increase}] were taken on a subset of 13
randomly selected nodes (six-core E5645 processors at 2.4 GHz with
Turbo Boost disabled) running a well-defined load (the standard
\texttt{stress} tool \cite{stress}).  The other measurements were
taken on the whole iDataCool system running in production mode, i.e.,
various jobs of different sizes and with different computing and
communication requirements are scheduled and executed by the batch
queueing system.

\begin{figure}[t]
  \centering
  \subfigure[][\label{fig:core}]{\includegraphics{plots/coretemp-vs-trck-crop}}\hfill
  \subfigure[][\label{fig:coredist}]{\includegraphics{plots/histogram-coretemps-crop}}\\
  \vspace*{-3mm}
  \caption{\subref{fig:core} Core temperatures of 13 compute nodes
    under \texttt{stress} and \subref{fig:coredist} core temperature
    distribution of the cluster in production mode.  In both plots
    only six-core E5645 processors are included.  The vertical error
    bars in \subref{fig:core} are the standard deviations after
    averaging over time and nodes.}
\end{figure}

In Fig.~\ref{fig:core} we show the average compute core temperature as
a function of the outlet temperature.  The average difference between
core and water temperature increases slightly, from
15$^\circ$C/59$^\circ$F to 17.5$^\circ$C/63.5$^\circ$F, over the range
of temperatures considered.  The error bars are rather large,
indicating a large variation between nodes.  A histogram of core
temperatures for $T_{\text{out}}=67^\circ$C/153$^\circ$F is shown in
Fig.~\ref{fig:coredist}.  The solid line is a Gaussian fit centered at
$84^\circ$C/183$^\circ$F with $\sigma=2.8^\circ$C/5.0$^\circ$F.  The
small bump at the low end of the histogram is due to idle nodes that
have a much lower core temperature.  Our interpretation of the large
spread visible in Fig.~\ref{fig:coredist} is that it is mainly due to
the first segment of the heat transfer path described in
Sect.~\ref{sec:idc-arch}, over which we have no control, while we can
control the second part very carefully.  Nevertheless, this spread is
a real problem if we aim, with energy reuse in mind, for a high outlet
temperature.  The cores throttle at about
100$^\circ$C/212$^\circ$F,\footnote{Note that there are other
  processors that throttle already at much lower temperatures.  Such
  processors are obviously not suitable for cooling with hot water.}
so the outlet temperature is limited by the core with the largest
difference between core and outlet temperature.  In our system this
largest difference is below 30$^\circ$C/54$^\circ$F so that we can
safely run at $T_{\text{out}}\le70^\circ$C/158$^\circ$F.  If we
desired higher temperatures we could sort out the ``bad'' chips and
run them at lower temperature in a separate system.  The high end of
the histogram in Fig.~\ref{fig:coredist} indicates that we could
perhaps gain another 5$^\circ$C/9$^\circ$F in this way.

\begin{figure}[t]
  \centering
  \subfigure[][\label{fig:power}]{\includegraphics{plots/coretempXnodepower-crop}}\hfill
  \subfigure[][\label{fig:powerdist}]{\includegraphics{plots/histogram-powers-crop}}
  \vspace*{-3mm}
  \caption{\subref{fig:power} Node power consumption of 13 compute
    nodes under \texttt{stress} and \subref{fig:powerdist} node power
    distribution of the cluster in production mode.  In both plots
    only six-core E5645 processors are included. }
\end{figure}

The power consumption per node also shows large fluctuations.  In
Fig.~\ref{fig:power} we present the DC power consumption of 13 nodes
vs.{} their average core temperature.  To quantify the spread we
measure the DC power on most six-core nodes for various temperatures,
interpolate to $80^\circ$C/176$^\circ$F, and then construct a
histogram of the interpolated node power, see
Fig.~\ref{fig:powerdist}.  The solid line is a Gaussian fit centered
at 206W with $\sigma=5.4$W.  We see that the individual CPUs vary
greatly in their power consumption even for the same coolant
temperature.  We again attribute most of these variations to the
manufacturing process of the chips, not to our liquid-cooling
solution.

\begin{figure}
  \centering
  \subfigure[][\label{fig:increase}]{\includegraphics{plots/powerincrease-crop}}\hfill
  \subfigure[][\label{fig:cop}]{\includegraphics{plots/chiller-cop-crop}}\hfill
  \vspace*{-3mm}
  \caption{\subref{fig:increase} Relative node power increase for 13
    nodes with six-core E5645 processors and \subref{fig:cop} COP of
    adsorption chiller.  The vertical error bars in
    \subref{fig:increase} are the standard deviations after averaging
    over nodes, while the vertical error bars in \subref{fig:cop}
    reflect the 10\% accuracy of the flow meters.}
\end{figure}

With higher cooling-water temperatures the power consumption of the
nodes increases, which has a negative effect on the total energy reuse
efficiency of the system.  To quantify this effect we plot in
Fig.~\ref{fig:increase} the relative average increase of the node
power consumption, which is about 7\% when going from
49$^\circ$C/120$^\circ$F to 70$^\circ$C/158$^\circ$F.  This should be
compared to the efficiency gain of the adsorption chiller, which is
quantified by the COP defined in Sect.~\ref{sec:installation} and
shown in Fig.~\ref{fig:cop}.  The temperature in the last plot starts
at 57$^\circ$C/135$^\circ$F since the adsorption chiller is in standby
mode for lower temperatures.  When going from 57$^\circ$C/135$^\circ$F
to 70$^\circ$C/158$^\circ$F the COP increases by 90\%, while the node
power consumption increases by only 5\%.  Thus the energy reuse
efficiency dramatically improves when running at higher temperatures.

In Fig.~\ref{fig:heat} we show the fraction of the electric power
delivered to the cluster that is transferred to the water in the rack
circuit.  We observe that this fraction drastically decreases with
temperature.  The reason for this decrease is the imperfect thermal
insulation of the iDataCool racks from the environment.\footnote{We
  did make serious insulation efforts, but since we retrofitted an
  existing system we were limited in what we could do.}  A higher
temperature difference between rack and air implies that more energy
is lost to the air.  The lesson from this figure is that in future
hot-water cooling designs serious attention should be paid to the
thermal insulation of the rack already in the early planning stages.
In Fig.~\ref{fig:etrans} we show the fraction of electric power that
is transferred to the driving circuit of the adsorption chiller, i.e.,
$P_d/P_\text{electric}$, as a function of the coolant temperature in
the rack circuit.\footnote{The energy balance in the rack circuit is
  $P_r=P_d+P_\text{add}+P_\text{loss}$, where 
  $P_r =$ heat-in-water $\times\; P_{\text{electric}}$,
  $P_\text{add}$ is the additional cooling power
  from the primary cooling circuit, and $P_\text{loss}$ is the heat
  per unit time that is lost to the environment due to imperfect
  thermal insulation of the plumbing.  $P_\text{add}$ is small at high
  temperatures, see Sect.~\ref{sec:installation}.  The fact that the
  numbers in Fig.~\ref{fig:etrans} are significantly lower than those
  in Fig.~\ref{fig:heat} thus implies that for our system
  $P_\text{loss}$ is rather large.} The increase shows that higher
coolant temperatures in the rack circuit lead to a better utilization
of the chiller, i.e., for our system the increase of the chiller
effectiveness with $T_{\text{out}}$ outweighs the reduced heat in
water.

\begin{figure}[t]
  \centering
  \subfigure[][\label{fig:heat}]{\includegraphics{plots/heat-in-water-crop}}\hfill
  \subfigure[][\label{fig:etrans}]{\includegraphics{plots/etrans-crop}}
  \vspace*{-3mm}
  \caption{\subref{fig:heat} Heat-in-water fraction and
    \subref{fig:etrans} transferred power.  The vertical error bars in
    \subref{fig:heat} combine temporal fluctuations of the inlet- and
    outlet coolant temperatures and the flow, while the vertical error
    bars in \subref{fig:cop} reflect the 10\% accuracy of the flow
    meters.}
\end{figure}


We do not show plots of the fraction of energy reused (or,
equivalently, of the energy reuse efficiency) since the cooling
capacity of our chiller is not high enough to convert all heat
from the iDataCool system to chilled water.  The fraction of energy
that could be reused (e.g., by adding another chiller) can be computed
by multiplying the numbers in Figs.~\ref{fig:cop} and \ref{fig:heat}
and is on the order of 25\% for
$T=60\ldots70^\circ$C/140\ldots158$^\circ$F.  With better thermal
insulation this fraction could increase by almost a factor of two at
$T=70^\circ$C/158$^\circ$F, see Fig.~\ref{fig:heat}.

