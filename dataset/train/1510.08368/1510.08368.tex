

\documentclass[letterpaper, 10 pt, conference]{ieeeconf}  



\IEEEoverridecommandlockouts                              

\overrideIEEEmargins                                      



\usepackage{amsmath} \usepackage{amssymb}  \usepackage{cite}

\usepackage{graphicx}


\newtheorem{theorem}{Theorem}
\newtheorem{lemma}[theorem]{Lemma}
\newtheorem{proposition}[theorem]{Proposition}
\newtheorem{corollary}[theorem]{Corollary}
\newtheorem{definition}{Definition}
\newtheorem{remark}{Remark}


\newcommand {\DF}[1]{\mbox{ }\medskip \\ {{\bf NOTE (Davide):} {\em #1}} \mbox{ }\medskip \\}


\title{\LARGE \bf
Switching control for incremental stabilization of nonlinear systems via contraction theory
}


\author{Mario di Bernardo and Davide Fiore\thanks{M. di Bernardo and D. Fiore are with the Department of Electrical Engineering and Information Technology, University of Naples Federico II, Via Claudio 21, 80125 Naples, Italy
        {\tt\small davide.fiore@unina.it}}\thanks{M. di Bernardo is also with the Department of Engineering Mathematics, University of Bristol, University Walk, BS8 1TR Bristol, U.K.
        {\tt\small mario.dibernardo@unina.it}}}


\begin{document}



\maketitle
\thispagestyle{empty}
\pagestyle{empty}


\begin{abstract}
In this paper we present a switching control strategy to incrementally stabilize a class of nonlinear dynamical systems. Exploiting recent results on contraction analysis of switched Filippov systems derived using regularization, sufficient conditions are presented to prove incremental stability of the closed-loop system. Furthermore, based on these sufficient conditions, a design procedure is proposed to design a switched control action that is active only where the open-loop system is not sufficiently incrementally stable in order to reduce the required control effort. The design procedure to either locally or globally incrementally stabilize a dynamical system is then illustrated by means of a representative example.
\end{abstract}


\section{Introduction}
Incremental stability has been established as a powerful tool to prove convergence in nonlinear dynamical systems \cite{angeli2002lyapunov}. An effective approach to obtain sufficient conditions for incremental stability comes from contraction theory \cite{lohmiller1998contraction,russo2010global,jouffroy2005some, forni2014differential,aminzare2014contraction}. More specifically, incremental exponential stability over a given forward invariant set is guaranteed if some matrix measure  of the system Jacobian matrix is uniformly negative in that set for all time. Moreover, contraction theory has been used as a synthesis tool to design incrementally stabilizing controllers and observers \cite{lohmiller1998contraction,manchester2014control,manchester2014output,van2008tracking}.

Piecewise smooth (PWS) systems are important in applications, ranging from problems in mechanics (friction, impact) and biology (genetic regulatory networks) to variable structure systems in control engineering \cite{filippov1988differential,cortes2008discontinuous,bernardo2008piecewise,utkin2013sliding}. Several results have been presented in literature to extend contraction analysis to these classes of nondifferentiable vector field \cite{lohmiller2000nonlinear,pavlov2005convergentp1,pavlov2005convergentp2,di2014contraction,lu2015contraction,di2013incremental,di2014incremental,fiore2015contraction}.

In this paper we discuss the problem of designing a switched feedback control to incrementally stabilize a nonlinear dynamical systems over some set of interest. Our approach is based on some of our previous analytical results on contraction and incremental stability of bimodal Filippov systems which were recently presented in \cite{fiore2015contraction}. In particular the switching control action resulting from our design procedure is active only where the open-loop system is not sufficiently incrementally stable. Such behavior can be usefully exploited to reduce the required control effort.

The rest of the paper is organized as follows. Section \ref{sec:background} summarizes the necessary mathematical preliminaries on contraction analysis and incremental stability of continuously differentiable systems and recalls a basic result on bimodal Filippov systems presented in \cite{fiore2015contraction}. Section \ref{sec:control_design} contains our main result on switched controlled systems and a design procedure to derive an incrementally stabilizing switching control input. The design procedure is illustrated with an example in Section \ref{sec:examples}. Conclusions are drawn in Section \ref{sec:conclusions}.



\section{Contraction analysis of PWS systems}
\label{sec:background}
\subsection{Incremental Stability and Contraction Theory}
Let  be an open set. Consider the system of ordinary differential equations

where  is a continuously differentiable vector field defined for  and , that is .

We denote by  the value of the solution  at time  of the differential equation \eqref{eq:dynamical_sys} with initial value .
We say that a set  is \emph{forward invariant} for system \eqref{eq:dynamical_sys}, if  implies  for all .

\begin{definition}
\label{def:incr_stability}
Let  be a forward invariant set and  some norm on . The system \eqref{eq:dynamical_sys} is said to be \emph{incrementally exponentially stable} () in  if there exist constants  and  such that 

, , where  and  are its two solutions.
\end{definition}

Results in contraction theory can be applied to a quite general class of subsets , known as K-reachable subsets \cite{russo2010global}. See Appendix for a definition.

\begin{definition}
\label{def:contraction}
The continuously differentiable vector field \eqref{eq:dynamical_sys} is said to be \emph{contracting} on a K-reachable set  if there exists some norm in , with associated matrix measure  (see Appendix), such that, for some constant  (the \emph{contraction rate})

\end{definition}
\vspace{0.2cm}
The basic result of nonlinear contraction analysis states that, if a system is contracting, then all of its trajectories are incrementally exponentially stable, as follows.
\begin{theorem}
\label{thm:contraction}
Suppose that  is a K-reachable forward-invariant subset of  and that the vector field \eqref{eq:dynamical_sys} is infinitesimally contracting with contraction rate  therein. Then, for every two solutions  and  with  we have that \eqref{eq:ies} holds with .
\end{theorem}
As a result, if a system is contracting in a forward-invariant subset then it converges towards an equilibrium point therein \cite{russo2010global,lohmiller1998contraction}.

In this paper we analyse contraction properties of dynamical systems based on norms and matrix measures \cite{russo2010global}. Other more general definitions exist in the literature, for example results based on Riemannian metrics \cite{lohmiller1998contraction} and Finsler-Lyapunov functions \cite{forni2014differential}. The relations between these three definitions and the definition of convergence \cite{pavlov2004convergent} have been investigated in \cite{forni2014differential}.

\subsection{Filippov systems}
The control input  we are going to design in this paper is a discontinuous function, this implies that even if the open-loop vector field is continuously differentiable the resulting closed-loop vector field is obviously not. In particular it belongs to a class of systems that has been investigated by Filippov \cite{filippov1988differential} and Utkin \cite{utkin2013sliding}. Switched (or bimodal) Filippov systems are dynamical systems  where  is a piecewise continuous vector field having a codimension-one submanifold  as its discontinuity set.
 
The submanifold  is called the \emph{switching manifold} and is defined as the zero set of a smooth function , that is

where  is a regular value of , i.e. . It divides  in two disjoint regions,  and  (see Figure \ref{fig:regions}).

Hence, a bimodal Filippov system can be defined as

where .  When the normal components of the vector fields either side of  point in the \textit{same} direction, the gradient of a trajectory is discontinuous, leading to Carath\'eodory solutions \cite{filippov1988differential}. In this case, the dynamics is described as \textit{crossing} or \textit{sewing}. But when the vector fields on either side of  both point toward it, the solutions are constrained to evolve along  and some additional dynamics needs to be given when such \emph{sliding} behavior occurs. To define this sliding vector field it is widely adopted the Filippov convention \cite{filippov1988differential}. 
\begin{remark}
In the following we assume that solutions of system \eqref{eq:filippov_bimodal} are defined in the sense of Filippov \cite{filippov1988differential} and they have the property of \emph{right-uniqueness} \cite[pag. 106]{filippov1988differential} holds in , i.e. for each point  there exists  such that any two solutions satisfying  coincide on the interval . Therefore, the escaping region is excluded from our analysis. 
\end{remark}

Definition \ref{def:contraction} was previously presented as a sufficient condition for a dynamical system to be incrementally exponentially stable, but  condition \eqref{eq:contraction_cond} can not be directly applied to system \eqref{eq:filippov_bimodal} because its vector field is not continuously differentiable. Therefore an extension of contraction analysis to PWS systems is not straightforward. In a recent work \cite{fiore2015contraction} sufficient conditions were derived for convergence of any two trajectories of a Filippov system between each other. Instead of directly analyzing the Filippov system, a regularized version  was considered given as

where  is the so-called transition function. See the original paper \cite{sotomayor1996regularization} from Sotomayor and Teixeira for further details on the regularization method adopted in \cite{fiore2015contraction}. 
In this new system the switching manifold  has been replaced by a boundary layer  (Figure \ref{fig:regions}) of width 

and more important  is continuously differentiable in , therefore condition \eqref{eq:contraction_cond} can be applied to it. Finally, results that are valid for Filippov systems \eqref{eq:filippov_bimodal} were recovered taking the limit for .
\begin{figure}[t]
\centering 
{\includegraphics[width=2.5in]{regions}}
\caption{Regions of state space: the switching manifold,  from \eqref{eq:switching_manifold}, 
,  (hatched zone) and  (grey zone) from \eqref{eq:regularizationregion}.} 
\label{fig:regions}
\end{figure}

The sufficient conditions for a bimodal Filippov system to be incrementally exponentially stable in a certain set are stated in the following theorem from \cite{fiore2015contraction}. 

\begin{theorem}
\label{thm:contracting_pws}
The bimodal Filippov system \eqref{eq:filippov_bimodal} is incrementally exponentially stable in a K-reachable set  with convergence rate  if there exists some norm in , with associated matrix measure , such that for some positive constants 

\end{theorem}
\vspace{0.3cm}
In the above relations  and  represent the closures of the sets  and , respectively.
The interested reader can refer to \cite{fiore2015contraction} for a complete proof and further details.

\section{Switching control design}
\label{sec:control_design}
\subsection{Problem formulation}
In this paper we consider the class of dynamical systems defined by

where ,  are state and feedback control input, and  ,  are continuously differentiable vector fields.

We want to find a discontinuous feedback control input  for system \eqref{eq:controlled_sys} such that the resulting closed-loop system is incrementally stabilized, either locally or globally. The control input  we are looking for has the following form

where  are continuously differentiable vector fields, and  is a scalar function as in \eqref{eq:switching_manifold}.

In particular, to minimize the control effort we want to exploit possible contracting properties of the open-loop vector field  to design a control input that is not active in the regions where  is already sufficiently incrementally stable.

\subsection{Main theorem}
The main result of this paper follows directly from Theorem \ref{thm:contracting_pws}.
\begin{theorem}
\label{thm:main_control}
The dynamical systems \eqref{eq:controlled_sys} with the switching control input \eqref{eq:control_input} is incrementally exponentially stable in a K-reachable set  with convergence rate  if there exist some norm in , with associated matrix measure  such that for some positive constants 

\end{theorem}
\vspace{0.3cm}
\begin{proof}
The closed-loop system with switching control \eqref{eq:control_input} is a Filippov system as \eqref{eq:filippov_bimodal} of the form

therefore Theorem \ref{thm:contracting_pws} can be directly applied giving the previous three conditions. And thus if these conditions hold then the switching control \eqref{eq:control_input} incrementally stabilizes system \eqref{eq:controlled_sys} with convergence rate .
\end{proof}

Note that

where we denoted with  and  the -th column of  and the -th component of , respectively.

\subsection{Design procedure}

In the following we present a possible approach to design a switching controller \eqref{eq:control_input} that incrementally stabilize system \eqref{eq:controlled_sys} in a desired set using conditions of Theorem \ref{thm:main_control}. Indeed if the designed  is such that conditions \eqref{eq:thm:control_condition1}-\eqref{eq:thm:control_condition3} are satisfied for a desired  then the discontinuous closed-loop system \eqref{eq:closed_loop_sys} is incrementally exponentially stable as required.

Specifically, suppose that it is required for the closed-loop system \eqref{eq:closed_loop_sys} to be incrementally stable with convergence rate  in a certain set  (where the open-loop system \eqref{eq:dynamical_sys} is not sufficiently contracting).

Suppose that in  there can be identified two disjoint subregions, one where condition \eqref{eq:contraction_cond} with  is not satisfied and the other one where it is satisfied (without the equality sign). Specifically, the two subregions are 


The key design idea is to choose the scalar function  in \eqref{eq:control_input} as

in this way the switching manifold  is defined as 




The final step is to find  and  such that conditions \eqref{eq:thm:control_condition1}-\eqref{eq:thm:control_condition3} are satisfied. Obviously with the selection of  made in \eqref{eq:H_control} the open-loop vector field  already satisfies the design requirements in , therefore in this case the simplest choice is 

and the control problem is reduced to find a  that satisfies \eqref{eq:thm:control_condition1} and \eqref{eq:thm:control_condition3}. In other terms, by selecting \eqref{eq:Sigma_control} as switching manifold the resulting switching control input can be active only in the region where the controlled system is not sufficiently contracting. 

This property can be exploited to reduce the average control energy compared to the one required by a continuous control input defined in the whole set  (eventually globally), as we will show in the next section through a simple example.

\section{Representative examples}
\label{sec:examples}

Here we present examples to illustrate the design procedure described in the previous section. The unweighted 1-norm will be used to highlight that non-Euclidean norms can be used in some cases as an alternative to Euclidean norms and that not only the analysis but the control synthesis too can be easier if they are used. See Appendix for the definition of the matrix measure induced by unweighted 1-norm .

The nonlinear system \eqref{eq:controlled_sys} that we want to incrementally stabilize in a certain set is 


The desired convergence rate  in this examples is set to 2, i.e. .

It can be easily seen that

Therefore the set  where system \eqref{eq:example_open_loop} is contracting with contraction rate , that is where it satisfies condition \eqref{eq:contraction_cond}, is  


In the following two design examples will be presented and discussed. In the first one we want to extend the region  where the system is incrementally stable to the set , and in the second one we want to make the system globally incrementally stable, that is .

In both cases, following the design procedure of Section \ref{sec:control_design}, the scalar function  of the switching controller is set as

and the switching manifold  as its zero set, that is as


Furthermore, as expected the control requirements are already satisfied in , and thus . The problem is now reduced to find a function  such that conditions \eqref{eq:thm:control_condition1} and \eqref{eq:thm:control_condition3} hold. Specifically, condition \eqref{eq:thm:control_condition1} is satisfied if the following quantity is made less than 

with .\\
In this simple example the first term in \eqref{eq:example_mu1} does not depend on  so it can be made less than  by simply setting . Therefore, in conclusion we need to find  such that

and then check if the resulting  satisfies \eqref{eq:thm:control_condition3} where .

\subsection{Example 1}
\label{sec:example_1}
As previously said, we want to extend the region where system \eqref{eq:example_open_loop} is contracting to a new set , in particular we choose . Therefore , and it can be easily proved that \eqref{eq:example_design_cond} is satisfied for , and thus, by integration, we have 


Condition \eqref{eq:thm:control_condition3} is also satisfied, since we have that for all 




In conclusion a switching control input that incrementally stabilize \eqref{eq:example_open_loop} in  is

In Figure \ref{fig:example1_sim}, we report numerical simulations of the evolution of the difference between two trajectories. The dashed line is the estimated decay from \eqref{eq:ies} with  and . It can be seen that as expected


\begin{figure}[t]
\centering 
{\includegraphics[width=3.2in]{example_1}}
\caption{System \eqref{eq:example_open_loop} in open-loop (dotted line) and with control \eqref{eq:example1_control} (solid line). Initial conditions in  and . The dashed line is the estimated exponential decay from \eqref{eq:ies} with  and . } 
\label{fig:example1_sim}
\end{figure}

\subsection{Example 2}
\label{sec:example_2}
If we want system \eqref{eq:example_open_loop} to be globally incrementally stable (that is ) condition \eqref{eq:example_design_cond} has to be verified with . It can be proved that such condition is satisfied choosing for example , and therefore by integration the control input defined in  is



Again, condition \eqref{eq:thm:control_condition3} is satisfied since

for all .

To conclude, system \eqref{eq:example_open_loop} is globally incrementally stabilized by the switching controller

In Figure \ref{fig:example2_sim}, we show numerical simulations of the evolution of the difference between two trajectories that confirm the theoretical results. Open-loop simulations are not reported in this case since the system is unstable for chosen initial conditions.

All simulations presented in this section were computed using the numerical solver in \cite{piiroinen2008event}.
\begin{figure}[t]
\centering 
{\includegraphics[width=3.2in]{example_2}}
\caption{Closed-loop system with control \eqref{eq:example2_control}. Initial conditions in  and . The dashed line is the estimated exponential decay from \eqref{eq:ies} with  and .} 
\label{fig:example2_sim}
\end{figure}



\subsection{Discussion}
As highlighted in the previous section, the control inputs designed here are active only in the region  of the state space where the open-loop system is not sufficiently incrementally stable, otherwise they are turned off. On the other hand, to satisfy the same stability requirements a continuous control feedback  has to be design such that

and thus it has to take non-zero values on the whole .
Therefore, the switching control law presented in this paper has the additional property that it can be turned off in  to reduce the required control energy.

For example, a continuous feedback control that satisfies control requirements as in Example 1 is

that is  in \eqref{eq:example1_control} extended to . Hence in this case it is clear that control input \eqref{eq:example1_control} uses less energy than \eqref{eq:example1_smooth_control}.

Instead, for what concerns Example 2, a continuous function  such that \eqref{eq:example_design_cond} holds on all  has to be at least cubic (while \eqref{eq:example2_control} is quadratic). Since their derivatives have to satisfy the same linear constraint \eqref{eq:example_design_cond} in , it easily follows that the -norm of the continuous control input will be always greater than the one of the discontinuous input. 





\section{Conclusions}
\label{sec:conclusions}

In this paper we formulated the problem of designing a switched control action to stabilize a nonlinear system using tools from contraction theory. Based on sufficient conditions for incremental exponential stability of switched bimodal Filippov system derived in \cite{fiore2015contraction}, we presented a control design strategy to incrementally stabilize a class of nonlinear systems. The effectiveness of the design methodology to derive both global and local results was illustrated through a simple but representative example. Moreover, we showed that different metrics rather than the Euclidean norm can be effectively used in the design of the controller.

Future work will be aimed at extending the class of systems stabilizable through such switched controllers and to construct state observers for these systems using methodologies similar to those presented here. Furthermore, it is of interest to reformulate the design procedure as a convex optimization problem to compute numerically both metrics and control gains.






\addtolength{\textheight}{-9.5cm}   









\section*{Appendix}
\subsection*{K-reachable sets}
Let  be any positive real number. A subset  is \emph{K-reachable} if for any two points  and  in  there is some continuously differentiable curve  such that ,  and .

For convex sets , we may pick , so  and we can take . Thus, convex sets are 1-reachable, and it is easy to show that the converse holds.


\subsection*{Matrix measure}
The \emph{matrix measure} \cite{vidyasagar2002nonlinear} associated to a matrix  is the function  defined as

The measure of a matrix  can be thought of as the one-sided directional derivative of the induced matrix norm function , evaluated at the point , in the direction of . See \cite{vidyasagar1978matrix} for a more general definition of matrix measure induced by a positive convex function and \cite{vidyasagar2002nonlinear,desoer1972measure} for a list of properties of this measure.

In this paper we often use the measure induced by unweighted 1-norm:

Other matrix measures often used in literature are the one induced by Euclidean norm
 
and the one induced by -norm

\vspace{0.1cm}








\bibliographystyle{IEEEtranBST/IEEEtran} 
\bibliography{IEEEtranBST/IEEEabrv,refs}   





\end{document}
