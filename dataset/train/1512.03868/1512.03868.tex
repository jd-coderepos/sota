\chapter{Negative Information and Tolerances} \label{chap:Neginfo}

In this chapter we present our later joint results with
Svetlana Shorina~\cite{BukatinShorina3,BukatinShorina4}.
In the previous chapter 
we showed how to obtain such a -structure
from a CC-valuation for any continuous Scott domain.
Certain pathologies in the behavior of

precluded us from extending this
method beyond bounded complete domains.

In this chapter, we obtain meaningful Scott continuous relaxed metrics
for continuous dcpo's by replacing  with
,
where .
This result can be understood
in terms of interplay between {\em negation duality} and
{\em Stone duality}.
Since we know how to construct a CC-valuation for any
continuous dcpo with countable basis, this method of
constructing partial and relaxed metrics via
CC-valuations and the resulting -structures
works for all continuous dcpo's with countable bases.

Escardo~\cite{Escardo, Smyth2} defined a topological space  to be
{\em weakly Hausdorff}, if its consistency relation is closed.
The consistency relation is given by formula
, where  is the
specialization order of .

A continuous dcpo with Scott topology is
weakly Hausdorff, if and only if
negative information  is observable, e.g. Scott open,
due to the fact that the relation
 is exactly the complement of
the consistency relation.

The technique of considering  works especially well,
when a continuous dcpo  satisfies the {\em Lawson condition} ---
the relative Scott and Lawson topologies on  are equal.
We obtain that the Lawson condition is equivalent to the
property . This will imply
that when the Lawson condition holds,
an induced classical metric on 
results.

Since the Lawson condition is
equivalent to the formula ,
which is a weakening of , which is, in turn, equivalent to
 being weakly Hausdorff, we call the spaces satisfying the Lawson
condition {\em very weakly Hausdorff}.

\section{Tolerances and a Smyth Conjecture}

Recently Mike Smyth~\cite{Smyth2} and Julian Webster~\cite{Webster}
advanced the approach in which tolerance is considered not as
an alternative to standard topology, but as a structure complementary 
to topology. In particular, it seems to be fruitful to equip
Scott domains with tolerances.

Sections~\ref{sec:smyth} 
and~\ref{sec:lowerb} make a small contribution to this emerging
theory.

Smyth~\cite{Smyth2} defines a tolerance as a reflexive symmetric
relation following Poincare and Zeeman. He defines a topological
tolerance space as a topological space equipped with a
tolerance relation closed in the product topology.

For a weakly Hausdorff space,  is
the least closed tolerance.

The set
 is an observable continuous representation of negative
information about 
for a weakly Hausdorff continuous dcpo  with the Scott
topology. We will see in Section~\ref{sec:bad} that, 
when  is not weakly Hausdorff the largest continuous
approximation of  is represented by
,
and the largest observable continuous representation of 
is represented by .

Smyth conjectured, that  or  is closely related to
the least symmetric closed tolerance on . In this chapter we
establish that, indeed, 
is the complement of this tolerance.

We also establish a relationship between this tolerance and
lower bounds of relaxed metrics on .

\section{Overview of the Chapter}

\subsection{Negation Duality and Problems with }

Certain pathologies in the behavior of  do not allow
us to use the formula 
 is
unbounded
when  is not bounded complete.
It is convenient to use {\em negation duality},
, when analyzing the behavior
of . Specifically, one can consider directed sets ,
and by considering  and , obtain the
following Lemma:

\begin{Lemma} For any dcpo , 

\end{Lemma}

\Proof
. Consider . 
.
If  is Scott open, ,
and .

. Consider directed , such that .
.
Condition  implies
that , and
.
\eproof

One can informally restate this, by saying that all  are
(Scott) observable~\cite{Smyth}, if and only if  is a Scott
continuous function .

We will see that for some continuous dcpo's certain 's are not
Scott open. This would be possible to tolerate, since, as we will see
later, co-continuity of valuations allows to extend those valuations to
Alexandrov open sets, and any  is still Alexandrov open.
However, the corresponding breakdown of equality
 for directed sets 
cannot be tolerated, because it tends to lead to the loss of
Scott continuity for the resulting relaxed metrics.

We could have replaced  with  (here and everywhere
in this chapter, interiors are being taken in Scott topology),
to rectify the problem of  being not Scott open, but this
does nothing to fix the broken continuity property.
Somewhat surprisingly, the {\em negation duality} helps us to
resolve this problem. 

\subsection{Negation Duality and Stone Duality}

We consider equality , which is just another way to state the
negation duality. Then, instead of taking ,
we consider .
Then the continuity property for the
resulting relaxed metric will be restored.
It is going to be technically convenient to replace 
not with , but with , but we do not think that this
feature is principal.

We will see that the reasons for  to work in this situation
can be best understood in terms of Stone duality.
In particular,
the map  gives rise to a map of the generators

of the free frame of Scott open sets of the powerset of  (ordered by
inclusion, the original ordering on  is ignored) to the frame
of open sets of the domain . 
Function  turns out to be a Scott continuous function
, which is dual to the function
 obtained by extending
map  onto the frame
.
We should emphasize here, that
what is going on in this chapter is a rather subtle interplay of
two {\em different} dualities --- negation duality and Stone duality ---
none of which seems to be reducible to another.

We will also see, that the function  is the
largest ``negative'' Scott continuous function ,
where the powerset of , , is ordered by the set-theoretic
inclusion.
This means that if  is
another Scott continuous function, 
such that for all , ,
then for all , .

\subsection{Lawson Condition}

In general we only get deficient -structures via the use of
 or .
Namely, the totality property does not hold in general,
and thus we still do not get an induced metric on .
However, in this situation it helps to impose the {\em Lawson
condition}, that relative Scott and Lawson topologies
on  are equal. 

The Lawson condition was introduced in~\cite{Lawson}, and is
widely used lately. As Lawson writes in the
Introduction to~\cite{Lawson}, ``this turns out to be a very fruitful
notion that permit great generality, but at the same time permits
the derivation of many important structure results''. 
This condition is now a standard part of the notion of
a {\em computational model} for a topological space~\cite{Flagg}.

In this chapter, we present two equivalent formulations of the
Lawson condition:  and .

The second of these equivalent formulations implies
that if the Lawson condition holds
for the domain  and, hence, for any 
maximal element , ,
then the totality property,
, holds, and the induced metric on 
results. Of course, the resulting metric topology is the same as
relative Scott and Lawson topologies on .

\subsection{Polish Spaces}

Lawson has shown in~\cite{Lawson}, that for every continuous
dcpo  with countable basis, Lawson condition implies that
 is a Polish space, i.e. that it is homeomorphic to
a complete, separable metric space.

However, since completeness of metric spaces is not a topological
invariant, this does not mean that metrics obtained by our
present methods have to be complete. Indeed, our present methods,
based on assignment of converging systems of finite weights for
the case of algebraic domains, yield a non-complete metric on
 for the domain  of Section~\ref{sec:lawson}.

This raises a lot of open questions, ranging from the question
of when our construction yields a complete metric space to
the question of whether methods of Edalat and Heckmann, used to
approximate complete metric spaces (see~\cite{Heckmann} for
the variant of their approach using partial metrics and, thus,
closest to our methods), can be extended to describe certain
non-complete metric spaces, like an open interval of the real line
or set .

\subsection{Historical Remarks}

Bob Flagg noted, that our results from Section~\ref{sec:good}
can be generalized to continuous dcpo's with compact Lawson topology.
Mike Smyth observed that this is a corollary of the following two facts.
The first fact is
that the conditions that all  are Scott open, that ,
and that the continuous dcpo is weakly Hausdorff are equivalent.
The second fact is that continuous dcpo's with compact
Lawson topology are weakly Hausdorff~\cite{Smyth2}.

Since  is equivalent to the space in question being weakly
Hausdorff, and since the Lawson condition can be reformulated as
 for maximal , we can offer an alternative name for
the spaces, satisfying the Lawson condition --- {\em very weakly
Hausdorff spaces}.

Using this terminology, we can say that one of the discoveries
of this chapter is that continuous dcpo's do not have to be
weakly Hausdorff to be satisfactorily 
relaxed metrizable by our methods,
but it is enough to require that they be very weakly Hausdorff.

\subsection{Structure of the Chapter}

In Section~\ref{sec:good} we show that the old formulas, based on ,
work for the class of {\em coherently continuous} dcpo's with countable
basis,
because 's are Scott open for this class of domains.

In Section~\ref{sec:bad} we analyze the pathologies of behavior
of  on a specific example. We then study the properties of
, which serves as a replacement for , and explain those
properties from the viewpoint of Stone duality.

In Section~\ref{sec:lawson} we find equivalent formulations of
the Lawson condition and use these formulations to establish the
totality property of the resulting -structures
with its ramifications for the induced metrics on .

In Section~\ref{sec:smyth} we talk about tolerances and prove
the Smyth Conjecture.

In Section~\ref{sec:lowerb} we establish that for the relaxed
metrics defined above,  if and only if ,
and build a continuous family of tolerances.

\section{When  Behaves Well}\label{sec:good}

In this section we study cases, when for all ,  is Scott open,
or, equivalently, when for all directed ,
. We already know, that this
situation takes place for continuous Scott domains.

Another class of domains, for which these properties can be established,
is the class of coherently continuous dcpo's with countable basis.
The term ``coherence'' here is understood in the weak sense
of~\cite{Abramsky} (weaker, than bounded completeness), and not in the
strong sense of~\cite{Gunter} (stronger, than bounded completeness).

\Def We say that a continuous dcpo  with the basis  is {\em
coherently
continuous}, if for any two basic elements ,
the set of their minimal upper bounds, , is finite,
and for any , if  and ,
then there is , such that .

\begin{theorem} If  is a coherently continuous dcpo with countable basis
K,
then for any ,  is open.
\end{theorem}
\Proof
Consider . Since the space has a countable basis,
we only need to show, that if there is a sequence of basic
elements, ,
such that , then some  belongs to .

By contradiction, assume that this is not the case. Then
for all ,  and  have an upper bound. Using the
presence of a countable basis again, approximate  with
a sequence of basic elements as well: , . Then for all ,
 and  have an upper bound.

Now we are going to build the sequence of (not necessarily basic)
elements, ,
such that for all ,  and .
Then  would be an upper bound of  and , yielding
the desired contradiction.

Consider an element . Define the height
of  as maximal , such that there is ,
such that . If there is no such maximal natural number,
we say that  is of infinite height. Using coherence condition,
it is easy to see, that there is an element 
of infinite height. Now consider only elements ,
such that . Using coherence condition, it is easy to
see once again, that there is an element  of
infinite height, such that . Continuing this
process, we obtain the desired sequence.
\eproof

Therefore, one can use  in order to obtain all results
of the previous section not only for continuous Scott domains,
but also for coherently continuous dcpo's with countable bases.

\section{When  Behaves Badly}\label{sec:bad}

\subsection{Example}

Let start with the example. We define an algebraic countable dcpo ,
as a following subset of the powerset of , ordered by subset inclusion.

, , , .
For convenience, we introduce a unique letter denotation for each of the
elements of : , , ,
,
, , ,
 We will use this notation throughout
the chapter. Observe, that all elements, except , are
compact, that elements  are total,
and that  iff .

\begin{figure}[h]
\begin{picture}(320,160)
\put(170,20){\circle*{3}}
\put(150,18){}
\put(170,20){\line(0,1){20}}
\put(170,40){\circle*{3}}
\put(150,38){}
\put(170,40){\line(0,1){20}}
\put(170,60){\circle*{3}}
\put(150,58){}
\put(170,60){\line(0,1){20}}
\put(170,80){\circle*{3}}
\put(150,78){}
\put(170,100){\circle{3}}
\put(170,110){\circle{3}}
\put(170,120){\circle{3}}
\put(170,140){\circle*{3}}
\put(150,138){}
\put(230,40){\circle*{3}}
\qbezier(170,20)(200,30)(230,40)
\put(210,40){}
\qbezier(230,40)(220,90)(210,140)
\put(210,140){\circle*{3}}
\put(208,150){}
\qbezier(230,40)(230,90)(230,140)
\put(230,140){\circle*{3}}
\put(228,150){}
\qbezier(230,40)(240,90)(250,140)
\put(250,140){\circle*{3}}
\put(248,150){}
\put(270,140){\circle{3}}
\put(280,140){\circle{3}}
\put(290,140){\circle{3}}
\qbezier(170,40)(220,50)(210,140)
\qbezier(170,40)(220,50)(230,140)
\qbezier(170,40)(220,50)(250,140)
\qbezier(170,60)(220,80)(230,140)
\qbezier(170,60)(220,80)(250,140)
\qbezier(170,80)(210,110)(250,140)
\end{picture}
\caption{Domain }
\end{figure}

Now we will see how {\em negation duality} works in this example.
In our previous terminology, , .
The role of a directed set  is played by an increasing sequence,
.
Notice that .

Note also that  and . 
You see, that  is not Scott open (although 
 is Scott open),
and dually, we obtain that 
(due to observation, that ), thus breaking  .

The breaking of this equality prevents the resulting relaxed metrics
from being Scott continuous, as  is compact and should naturally
carry a finite weight. Since all  and  are Scott open,
taking  as  instead of  would not fix this problem. 

\subsection{Solution}

Let us rewrite the negation duality as .
What works, somewhat surprisingly, is taking a subset of  via
taking the interior inside the right-hand side of this expression:
.

\begin{Lemma}\label{sec:jgood}
If  is a directed set, 
.
\end{Lemma}
\Proof
A potentially non-trivial part is to prove
.
Consider . By definition of ,
. Since  is Scott open,
there is , such that , that is
.
\eproof

In the next subsection, we explain this result in terms
of Stone duality. In our example domain , 
does not include , unlike , yielding
.

In general,  is Alexandrov open, but does not have to
be Scott open. E.g., in our example domain ,
we have that
, and thus  is not Scott open.
Due to co-continuity we can extend  to Alexandrov open
sets , by defining .

However, in order to use a setup of Section~\ref{new_part_metric},
it is much more convenient to define  and to use
the following Lemma.

\begin{Lemma} If for arbitrary Alexandrov open sets ,
the equality  holds,
then  holds as well.
\end{Lemma}
\Proof
The potentially non-trivial direction is to prove
.
Consider . By the Border Lemma (Lemma~\ref{border_lemma})
there is , such that . Then, because of the
condition of the Lemma we are currently proving, there is ,
such that . Then applying the Border Lemma again,
we obtain .
\eproof

Hence,  enables us to satisfy all the requirements
of the setup of Section~\ref{new_part_metric}, except for the requirement
that for all , which does not hold
in general. E.g. consider our example domain , and observe that
, but 
. (Observe, also that changing
 to  does not fix this.)

Thus the Theorem~\ref{MainTheorem_a} holds, but the 
Theorem~\ref{OnTotals_a}
about the equality of  and 
does not have to hold, and the resulting
induced metric on  cannot in general be obtained. E.g.,
in our example domain ,
we have that
, but 
, which
is, in general, not zero, since  is compact.

\subsection{Stone Duality}

The first Lemma in the previous subsection holds for the reasons,
which are not related to such specific
features of  as the use of  (any open set
can be used instead) and the fact, that
 and  belong to the same set .

We analyze this situation in the spirit of 
{\em Stone duality}~\cite{Johnstone,Vickers}, which is a contravariant
equivalence between categories of spatial frames (of open sets)
and sober topological spaces.

For the purpose of this subsection only, assume that there is
a continuous dcpo  (Scott topologies of continuous dcpo's are 
sober~\cite{Johnstone})
and a set , and that we are given a map ,
where  is the frame of Scott open sets of the
domain .

Now generalize the construction of  by considering the map
, where  is a powerset of 
ordered by set-theoretic inclusion and equipped with the Scott topology.
Define  by the formula: . 
Then observe that the proof of Lemma~\ref{sec:jgood}
still goes through, implying that  is a Scott continuous
function from  to .

Now observe, if one applies  to a subbasic
open set , one obtains \linebreak
.

Thus the map
 can be thought of as defined on the generators
 of the
free frame of all Scott open sets on 
and giving raise to the frame homomorphism
 (of course, ,
e.g. for basic open sets, , and the similar thing goes
for unions of basic sets).

Now, since we are dealing with sober spaces, Stone duality means,
that not only  can be obtained from the continuous
function , but also the continuous function  can be restored
from the frame morphism . And this is the essence of our
definition of , when we think about  as defined on the
generators of the frame .

\subsection{ Is the Largest Continuous Approximation of }

Both  and  can be considered as functions from  to the
powerset of , . However, in general, only 
is Scott continuous.
The following theorem shows that, in some sense,  is the best
we can do.

\begin{theorem}
If  is a Scott continuous function
and , then
. 
\end{theorem}
\Proof
Assume that such Scott continuous function  is given,
and for some , there is , such that ,
i.e. .
However,  means . Now consider a directed
set . We have that  and that all 
are way below . Then, taking into account 
and applying the Border Lemma, we obtain that .

However, the assumption of continuity of  means, that
. Hence, since ,
there is some , such that , hence ,
hence , contradicting the last formula of the previous
paragraph.
\eproof


A similar statement holds for  and ,
understood as functions from  to the dual domain of open sets
of .

\section{The Use of the Lawson Condition}\label{sec:lawson}

We start with the equivalent formulation of the Lawson condition.

\begin{Lemma} Given a continuous dcpo , the relative Scott and Lawson
topologies on  coincide if and only if for all ,
.
\end{Lemma}

For example, in the domain  from the previous section
Lawson condition does not hold. Indeed,
 and .  
However, . 
The set  is open in the
relative Lawson topology, but not in the relative Scott topology.

Now we are going to modify the domain , in order to obtain a
different example, which would satisfy the Lawson condition.
We add a new element, , to , so that  will be
a subset of the powerset of , ordered by
the set-theoretic inclusion. Let ,
where .

\begin{figure}[h]
\begin{picture}(320,180)
\put(170,20){\circle*{3}}
\put(150,18){}
\put(170,20){\line(0,1){20}}
\put(170,40){\circle*{3}}
\put(150,38){}
\put(170,40){\line(0,1){20}}
\put(170,60){\circle*{3}}
\put(150,58){}
\put(170,60){\line(0,1){20}}
\put(170,80){\circle*{3}}
\put(150,78){}
\put(170,100){\circle{3}}
\put(170,110){\circle{3}}
\put(170,120){\circle{3}}
\put(170,140){\circle*{3}}
\put(150,138){}
\put(170,140){\line(0,1){20}}
\put(170,160){\circle*{3}}
\put(150,158){}
\put(230,40){\circle*{3}}
\qbezier(170,20)(200,30)(230,40)
\put(210,40){}
\qbezier(230,40)(220,90)(210,140)
\put(210,140){\circle*{3}}
\put(208,150){}
\qbezier(230,40)(230,90)(230,140)
\put(230,140){\circle*{3}}
\put(228,150){}
\qbezier(230,40)(240,90)(250,140)
\put(250,140){\circle*{3}}
\put(248,150){}
\put(270,140){\circle{3}}
\put(280,140){\circle{3}}
\put(290,140){\circle{3}}
\qbezier(170,40)(220,50)(210,140)
\qbezier(170,40)(220,50)(230,140)
\qbezier(170,40)(220,50)(250,140)
\qbezier(170,60)(220,80)(230,140)
\qbezier(170,60)(220,80)(250,140)
\qbezier(170,80)(210,110)(250,140)
\end{picture}
\caption{Domain }
\end{figure}

Now , so 
this is still not a Scott open set,
however, . Notice, that  here, 
so  is still only
Alexandrov open.  
is the same as in , but now  is not
a total element. However, , because
, so , and
. The set 
is open in both Lawson and Scott relative topologies on
.

What is going on here is described by the following Theorem.

\begin{theorem} A continuous dcpo  satisfies the Lawson condition
if and only if
for all , .
Hence, if the Lawson condition holds, then
.
\end{theorem}
\Proof
Assume that the Lawson condition holds and .
Assume, that , i.e.  and
, using the totality of .
Thus, by duality, . Because  and
because due to the 
Lawson condition ,
we obtain , hence . 

Conversely, assume .
Let us prove ,
thus proving the Lawson condition.
Take . By negation duality, 
, then, by totality 
of  and our assumptions, , which, by definition of ,
means that . 

The rest follows from the observation,
that if ,  is Scott open.
\eproof

Hence if the Lawson condition holds, the resulting -structure
is not deficient, and the Theorem~\ref{OnTotals_a} holds.



\section{Tolerances and Negative Information}\label{sec:smyth}

\subsection{Tolerances, Distinguishability, and Observability}

Smyth~\cite{Smyth2} requires that a tolerance relation is closed
in the product topology. Here are informal reasons for this.

The typical meaning of two points being in the relation of tolerance,
, is that  cannot be distinguished from , i.e.
there is no way to establish, that  and  differ.

The natural way to interpret the statement, that  and  can
be distinguished, is to give some ``effective'' procedure for
making such a distinction. Thus, the property of being distinguishable
is observable~\cite{Smyth}. Correspondingly, the property 
is refutable, hence  should be closed.

The fact that the least closed tolerance for a weakly Hausdorff
continuous dcpo is  also is quite natural in this framework.
Indeed, domain elements are thought of as being only partially known
and dynamically increasing in the course of their lives. Hence the
fact that , that is , precisely means that  and  may approximate the
same ``genuine'' element , hence we cannot distinguish between them.
Since  is closed in the weakly Hausdorff case, its complement
is open, hence observable. That means that when  does
not hold, there is some ``finite'' way to distinguish between  and
.

\subsection{ and the Least Closed Tolerance (a Smyth Conjecture)}

Consider a continuous dcpo . In this subsection .
Recall that we defined
.

\begin{Lemma}
 is unbounded.
\end{Lemma}

\Proof
Using the Border Lemma we get  iff .  By the definition of ,  iff  i.e. .  
Finally recall that  iff  is unbounded .
\eproof

It is an immediate corollary
that  is symmetric. 

Let us show that  is open in
the product topology. If we fix a pair  
the set  is open, and our
 is the union of all such sets for all unbounded pairs 
.

\begin{theorem}
The relation  is the complement of the least closed tolerance.
\end{theorem}

\Proof 
Consider an open set , such that
. Consider a pair 
. Since  is
open, we can choose two open sets  and , such that
. 
Consider a pair . 
The pair  is bounded, otherwise
, 
so we can take  such that , , therefore , , so 

and . 
So the complement of  is not a tolerance, because it
is not
reflexive.
\eproof

\subsection{Examples}

In our example domain , the pair  is
unbounded, but belongs to the least closed tolerance, since
these elements cannot be distinguished by looking at the approximation
pairs,  .

Moreover, by adding to domain  elements
  and ,
we obtain a domain, where two different maximal elements,  and
, cannot be distinguished 
via finite observations, because all their
approximations are bounded.
Such a situation, when ,
where  is the least closed tolerance, cannot
occur in the presence of the Lawson condition, because the Lawson condition
is equivalent to .

\section{Tolerances and Lower Bounds of Relaxed Metrics}\label{sec:lowerb}

We are going to prove the following statement.

\begin{theorem}
.
\end{theorem}

\Proof
. Recall that
.
Notice that
 implies , which implies
that . Hence , hence  due to the strong
non-degeneracy of . Hence .

.  means  or
. It is enough to consider
. Since  is the largest element
of , we obtain , hence .
\eproof

Only upper bounds of relaxed metrics participate in the definition
of the relaxed metric topology. Hence lower bounds are usually considered
as only playing an auxiliary role in the computation of upper bounds.
Here we see an example of a quite different 
application of lower bounds.

\subsection{A Continuous Family of Tolerances}

A set ,
also forms a tolerance. Indeed, this is a symmetric, reflexive
relation. To see that it is closed, consider a Scott continuous
function , where  is
a segment  with the usual ordering and the induced
Scott topology, and observe that the set in question is
the inverse image of a Scott closed set 
under .

The resulting family of tolerances parametrized by  is
Scott continuous in the following sense. The dual domain for
 is domain . Here  is the same segment of numbers,
but with inverse ordering (),
an element  corresponds to the open set ,
the element  corresponds to the open set ,
and domains  and  are equipped with
the induced Scott topology.

The function 
is Scott continuous, and so is its restriction on .
Then  is the complement of the tolerance in
question, and we can think about this complement as representing
this tolerance in the dual domain .
 
\section{Open Problems}

It might be useful to extend the Stone duality analysis to
 and to be able to speak about the intuition behind
the Lawson condition in the spirit of~\cite{Smyth}.

Another open question is as follows.
If Lawson condition does not hold, can we obtain some negative
results about the existence of -structures
with totality property, or, more generally,
about the existence of relaxed metrics
with the property ?
Obviously, this question allows a number of variations,
e.g. we know now, that when this
 question is restricted to the case of ,
such negative results can indeed be obtained.

It seems that tolerances will play an increasingly important role
in domain theory. One particularly promising direction of development
is to use tolerances and especially their asymmetric generalizations
instead of transitivity of logical inference to formally
express and study the ideas of A.S.Esenin-Vol'pin and P.Vopenka,
that large numbers should be considered infinite,
and long proofs and computations should be considered 
meaningless~\cite{Vopenka}.


