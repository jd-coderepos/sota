\documentclass{stacs_proc}



\usepackage{graphicx}
\usepackage{epsfig}
\usepackage{psfrag}
\usepackage{wrapfig}





\newcommand{\bigO}{O}
\newcommand{\setR}{\mathbb{R}}
\newcommand{\D}{\mathcal{D}}
\newcommand{\R}{\mathcal{R}}
\newcommand{\C}{\mathcal{C}}
\newcommand{\T}{\mathcal{T}}




\DeclareMathOperator{\st}{s.t.}
\DeclareMathOperator{\intr}{int}
\DeclareMathOperator{\DT}{DT}
\DeclareMathOperator{\opt}{OPT}
\DeclareMathOperator{\con}{conv}



\begin{document}

\title[short title]{title}
\title[Geometric Set Cover and Hitting Sets for Polytopes in ]{Geometric Set
  Cover and Hitting Sets \\ for Polytopes in }


\author{S\"{o}ren Laue}{S\"{o}ren Laue}
\address{Max-Planck-Institut f\"{u}r Informatik, Campus E1 4, 66123
Saarbr\"{u}cken, Germany}  \email{soeren@mpi-inf.mpg.de}  \thanks{This work was supported by the Max Planck Center for Visual
  Computing and Communication (MPC-VCC) funded by the German Federal
  Ministry of Education and Research (FKZ 01IMC01).} 




\keywords{Computational Geometry, Epsilon-Nets, Set Cover, Hitting Sets}
\subjclass{F.2.2, G.2.1}




\begin{abstract}
  \noindent 
  Suppose we are given a finite set of points  in  and a
  collection of polytopes  that are all translates of the same
  polytope . We consider two problems in this paper. The first is
  the set cover problem where we want to select a minimal number of
  polytopes from the collection  such that their union covers all
  input points . The second problem that we consider is finding a
  hitting set for the set of polytopes , that is, we want to
  select a minimal number of points from the input points  such
  that every given polytope is hit by at least one point.
  
  We give the first constant-factor approximation algorithms for both
  problems. 
  We achieve this by providing an epsilon-net for translates of a
  polytope in  of size . 
\end{abstract}

\maketitle


\stacsheading{2008}{479-490}{Bordeaux}
\firstpageno{479}



\vspace{-0.5cm}
\section*{Introduction}\label{S:one}
Suppose we are given a set of  points  in  and a collection
of polytopes  that are all translates of the same polytope . We
consider two problems in this paper. The first is the set cover
problem where we want to select a minimal number of polytopes from the
collection  such that their union covers all input points . The
second problem that we consider is finding a hitting set for the set
of polytopes , that is, we want to select a minimal number of
points from the input points  such that every given polytope is hit
by at least one point.   

Both problems, the set cover problem and the hitting set problem which
are in fact dual to each other are very fundamental problems and have
been studied intensively. In a more general setting, where the sets
could be arbitrary subsets, both problems are known to be NP-hard, in
fact they are even hard to approximate within ~\cite{LY94}. Even when the sets are induced by geometric objects
it is widely believed that the corresponding set cover problem as well
as the hitting set problem are NP-hard. Several geometric versions of
these problems were even proven to be hard to approximate. Hence, we
are looking for algorithms that approximate both problems. We give the
first constant-factor approximation algorithms for the set cover
problem and the hitting set problem for translates of a polytope in
. The central idea to our approximation algorithms are small
\emph{epsilon-nets}. 

A set of elements  (also called points) along with a collection
 of subsets of  (also called ranges) is in general called a
\emph{set system}  and for geometric settings also known as
\emph{range spaces}. One essential characteristic of these set systems
is the \emph{Vapnik-Chervonenkis dimension}, or
\emph{VC-dimension}~\cite{VC71}. The VC-dimension is the cardinality
of the largest subset  for which  is the powerset
of . If the set  is finite, we say that the set system 
has bounded 
VC-dimension, otherwise we say the VC-dimension of  is 
unbounded. For instance, the set system induced by translates of a
polytope has VC-dimension three as well as the set system induced by
halfspaces in .  A set  is called an \emph{epsilon-net} for
a given set system  if  for every subset  for
which . In other words, an epsilon-net is a hitting set
for all subsets  whose cardinality is an -fraction of the
cardinality of the input point set .  

It is known that there exist epsilon-nets of size
 for any set system of
VC-dimension ~\cite{BEHW89, KPW92}. This bound is in fact tight for
arbitrary set systems as there exist set systems that do not admit
epsilon-nets of size less than this bound~\cite{PW90}. Such an
epsilon-net can be simply found by random sampling~\cite{M}. 

However, for special set systems that are induced by geometric objects
there do exist epsilon-nets of smaller size, namely of size
. It has been shown by Pach and
Woeginger~\cite{PW90} that halfspaces in  and translates of
polytopes in  admit epsilon-net of size
. Matou\v{s}ek et al.~\cite{MSW90} gave an
algorithm for computing  small epsilon-nets for pseudo-disks in 
and halfspaces in . The result for halfspaces in  also
follows from a more general statement by Matou\v{s}ek~\cite{M92}.  

Among other reasons for finding epsilon-nets of small size is the fact
that an epsilon-net of size  immediately implies an
approximation algorithm for the corresponding hitting set with
approximation guarantee of , where  denotes the
optimal solution to the hitting set~\cite{PA95}. This means, that for
arbitrary set systems of fixed VC-dimension we have an algorithm for
the hitting set problem with approximation .  
And for set systems that admit epsilon-nets of size  we
get an approximation algorithm to the hitting set problem with
constant approximation guarantee.    




Clarkson and Varadarajan~\cite{CV05} developed a technique that
connects the complexity of a union of geometric objects to the size of
the epsilon-net for the dual set system. Using this result, they are
able to develop, among other approximation algorithms for geometric
objects in , a constant-factor approximation algorithm for the
set cover problem induced by translates of unit cubes in .  

We extend their result to not only the set cover problem but also the
hitting set problem for arbitrary translates of a polytope in
. We do not require the polytope to be convex or fat. This is the
first constant-factor approximation algorithm for these two
problems. We achieve this by giving an epsilon-net for translates of a
polytope in  of size . We reduce the problem
of finding epsilon-nets for translates of a polytope to a family of
non-piercing objects in  and then generalize the epsilon-net
finder for pseudo-disks of Matou\v{s}ek et al.~\cite{MSW90} to our
setting.     

The set cover problem
which is studied by Hochbaum and Maass~\cite{HM85} where one is
allowed to move the objects is fundamentally different. They give a
PTAS for their problem.  






\section{Small Epsilon-Nets for Polytopes in }



Let  be a set of  points in  and let  be a family of
polytopes  that are all translates of the same bounded polytope
.
We want to find a set of polytopes of minimal cardinality among the
collection  that covers all input points . 
First, we find a small epsilon-net for this set system and use this
later for the constant-factor approximation of the hitting set
problem. Finally, we show how this then can be translated into a
solution for the set cover problem.  

 Throughout this paper we denote by  the polytope as well as
the subset of points from  that  covers and by  the family
of polytopes as well as the corresponding family of subsets of
. This will make the paper easier to read and it 
will be clear from the context whether we talk about the geometric
object or the corresponding set of points.


\subsection{From Polytopes in  to Non-Piercing Objects in }



So given such a set system  we want to find an epsilon-net
for it, i.e. we are looking for a set  such that every subset of
points  with  is stabbed by at least one point
from . 

We can cut the polytope  into, lets say  polytopes . If the polytope  contains  input points then one
of the polytopes  must contain at least 
input points. Hence, in order to find an -net for the set system
 
induced by translates of , it suffices to find -net
for the set systems induced by the translates of .

Following this reasoning we can reduce our problem for finding an
epsilon-net for the set system induced by translates of arbitrary
polytopes to translates of \emph{convex} polytopes by cutting the
possibly non-convex polytope into a set of convex polytopes.
Note that the number of these convex polytopes only depends on the
polytope  and hence is constant for fixed . 


Wlog. let  be from now on a convex polytope.
We can place a cubical grid onto the space  such that for
any translate of  every cubical grid cell contains at most
vertex of . This can be achieved by making the grid fine
enough. Clearly, the maximal number  of grid cells that can be
intersected by  is bounded and only depends on . Again, if
 contains  input points then at least one of the cells
must contain at least  of the input points. Hence, we
can restrict ourselves to finding epsilon-nets for translates of 
triangular cones
where all input points lie in a cube in . This just adds a
multiplicative constant to the size of the final epsilon-net.

The case when the cubical cell only contains a halfspace or the
intersection of two halfspaces can be either seen as a special case of
a cone or, in fact, be even treated separately in a much simpler
way. The case of a translate of a halfspace reduces to a
one-dimensional problem an admits an epsilon-net of size 1 and the
case of two intersecting halfspaces reduces to a problem on intervals
which admits an epsilon-net of size . 

In the following we will construct an epsilon-net for the set system
 that is induced by translates of a 
triangular cone . 

Given a cone , we call a set of points  in non--degenerate
position if every translate of  has at most three points of  on
its boundary. We can always perturb the input points  in such a way
that they are in non--degenerate position and the collection of
subsets of the form  where  is a translate of  does
not decrease~\cite{EW85}. Hence, we can restrict ourselves on
non--degenerate set of points .  

\begin{wrapfigure}[11]{R}{5cm}
  \begin{center}
    \psfrag{z}{}
    \psfrag{C}{}
    \psfrag{r}{}
    \epsfig{file=cone1.eps,width=5cm}
    \caption{The cone  and its internal ray .}
    \label{fig:cone1}
  \end{center}
\end{wrapfigure}
We place a
coordinate system  such that the input points all have -coordinate
greater than  and a ray  emitting from the apex of the cone  and
lying entirely in the cone should intersect the plane . We refer
to such a cone as a cone that \emph{opens to the bottom} and the ray  as
its \emph{internal ray}.
Figure~\ref{fig:cone1} illustrates this setup for the two-dimensional
case.


The following two definitions are helpful generalizations the lower
envelope. 



\begin{defi}
  Given a finite point set  and a triangular cone  that opens to
  the bottom consider the arrangement 
  of all translates of  that have a point of  on its boundary
  but no point of  in its interior. The upper set of plane segments
  that can be seen from above is called the \emph{lower envelope of
     with respect to cone }.
\end{defi}

Figure~\ref{fig:cone2} illustrates the definition of the lower envelope
in the two-dimensional case.
This definition is similar to the definition of alpha-shapes where
the cone is replaced by a ball.
We call all points that lie on the lower envelope with respect to cone
 \emph{lower envelope points} and denote this set by . 
\begin{figure}[h]
\begin{minipage}{0.49\textwidth}
  \begin{center}
    \psfrag{C}{}
    \epsfig{file=cone2.eps,width=6cm}
    \vspace{-3ex}
    \caption{The lower envelope with respect to cone , the
      corresponding cones are drawn dotted.} 
    \label{fig:cone2}
  \end{center}
\end{minipage}\hfill
\begin{minipage}{0.49\textwidth}
  \begin{center}
    \psfrag{C}{}
    \epsfig{file=cone3.eps,width=6cm}
     \vspace{-3ex}
    \caption{The flattened lower envelope with respect to cone ,
      lower envelope is drawn dotted.} 
    \label{fig:cone3}
  \end{center}
\end{minipage}
\end{figure}

\begin{defi}
  Let  be a triangular cone that opens to the bottom and let  be a finite set of points in non--degenerate position. Let
   be a cone that is flatter that  by small  and such that
  it contains  and the combinatorial structure of  and  is
  the same as for  and . See figure~\ref{fig:cone3} for an
  illustration. Then, the lower envelope of  with respect to 
  is called the \emph{flattened lower envelope of  with respect to cone }.
\end{defi}
Such a cone  always exists for a finite point set that is in
non--degenerate position.
From now on we will abbreviate the term lower envelope with respect to
cone  by lower envelope since we will throughout this paper only
talk about the same cone . The flattened lower envelope can  be
basically seen as a slightly flattened version of the lower envelope.



The next lemma shows that we can reduce the problem of finding an
epsilon-net with respect to cones of arbitrary point sets to lower
envelope points.

\begin{lemma}
  \label{lem:1}
  If for every finite point set  of lower envelope points in
  non--degenerate position there exists 
  an epsilon-net with respect to translates of a cone  of size
   then there exists 
  an epsilon-net with respect to translates of a cone  of size
   for every finite point 
  set   in non--degenerate position.
\end{lemma}
\begin{wrapfigure}[12]{r}{5.2cm}
  \begin{center}
    \epsfig{file=cone4.eps,width=5cm}
    \caption{The projection of points onto flattened lower envelope.}
    \label{fig:cone4}
  \end{center}
\end{wrapfigure}
.\vspace{-4ex}
\begin{proof}
  Let  be such a finite point set in non--degenerate
  position and let  denote the cone. Let  denote the set of
  lower envelope points. Let  be the set of all
  non-lower envelope points.   
We project all non-lower envelope points  along the
  internal ray  of cone  onto the flattened lower envelope
  (cf. figure~\ref{fig:cone4}). We denote the projection of a point
   by . 
  Let  be union of the projected points and . Clearly,  is
  a set of lower envelope points in non--degenerate position.  

  Suppose we have an epsilon-net  for this point
  set . From this epsilon-net  we will construct an
  epsilon-net  for the original point set . If a point from the set
   is in the epsilon-net , we also add it to the epsilon-net
   for . If however, a projected point  is in  then we
  add to  the three points  and  from the lower envelope  that determine the cone  on whose boundary also  lies. Note that whenever an arbitrary cone contains the point  then it has to contain one of the three points  or . 



  We have the following two properties:
  \begin{enumerate}
  \item If a cone contains at least  points from the set  then
    it contains at least  points from the set . 
  \item If a cone contains a point from the epsilon-net  for 
    then the cone contains a point from the epsilon-net  for .  
  \end{enumerate}    
  Both properties prove that the 
set  is indeed an epsilon-net for .
\end{proof}



The preceding lemma assures that we can restrict ourselves on a finite
set of lower envelope points in non--degenerate position. For such
a set system we will now construct a corresponding set system of
points in the plane and a collection of regions in the plane. 


\begin{defi}
  Let  be a cone and let  be a finite set of lower envelope
  points in non--degenerate position and let  be a collection
  of translates of . We define a projection  from the flattened
  lower envelope onto the plane  by projecting each point along
  the internal ray . Let the projection of all points 
  which all lie on the be denoted as the set . For each cone of the
  collection the image of the intersection of the cone with the
  flattened lower envelope is an object  and the family 
  of cones induces a family of objects which we will denote by . 
\end{defi}
Using the flattened lower envelope instead of the lower
envelope avoids degeneracy. The intersection of an arbitrary cone with
the flattened lower envelope is always a collection of line
segments. Furthermore, it makes everything continuous in the
sense that if a cone is moved continuously in  then the
intersection of the cone with the flattened lower envelope moves
continuously as well as its image of the projection . Note, that
 is injective.  

Analogously, we call a set of points  in non--degenerate
position if every  has at most three points on its boundary. 
We have the following lemma:
\begin{lemma}
  \label{lem:2}
  If for every finite point set  in non--degenerate
  position there exists 
  an epsilon-net with respect to the family of objects 
  produced by the projection  of size  then there exists 
  an epsilon-net with respect to cones of size  for every point
  set  of lower envelope points  in non--degenerate position.
\end{lemma}
\begin{proof}
  The proof follows easily from the fact that the image of a cone 
  under the projection  contains exactly those points that are the
  image of the points that are contained in . 
\end{proof}

We refer to a cone  as the corresponding cone of the object
. We will prove a few useful properties of the so constructed
set system . 

Notice, that the intersection of two triangular cones is again a
cone. Furthermore, the intersection of a possibly infinite family of
triangular cones is either empty or again a triangular cone since all
cones are closed. The intersection of the boundary of a cone with the
flattened lower envelope is either empty or a set of line segments
that form one simple closed cycle. Hence, the image of a cone under
the projection  is a closed and connected region whose boundary is
a closed and connected cycle. 
  
\vspace{-3ex}
\begin{wrapfigure}[11]{r}{4.3cm}
  \begin{center}
    \epsfig{file=piercing1.eps,width=4cm}
    \caption{A set of non-piercing objects}
    \label{fig:piercing1}
  \end{center}
\end{wrapfigure}
.
\begin{definition}
Two geometric objects(sets)  and  that are bounded
  by Jordan curves are said to be 
  \emph{non-piercing} 
  if 
the boundary of  and  cross at most twice.
  A family of geometric objects is called non-piercing if every
  two objects from this family are non-piercing. See
  figure~\ref{fig:piercing1} for an illustration. 
\end{definition}




 
\begin{lemma}
  \label{lem:piercing}
  The projection  produces a family  of non-piercing objects.
\end{lemma}

\begin{proof}
  Consider two cones  and  that intersect each other. If one
  is contained in the other, i.e.  then we are done, as
   and hence their boundaries cannot cross. So if
   and  intersect and none is subset of the other then the
  intersection of their boundaries are two rays emitting from the same
  point. Each of these rays intersects the flattened lower envelope
  exactly once. Hence, as the projection  is injective the boundary
  of the two images of the cones  and  under the projection
   intersect exactly twice. Thus, the objects are non-piercing.   
\end{proof}


\subsection{Small Epsilon-Nets for Non-Piercing Objects in }

In this subsection we will derive a few properties of the projection
that are necessary to apply the algorithm of Matou\v{s}ek et
al.~\cite{MSW90} for finding a small epsilon-net for
pseudo-disks. These properties also hold in general for any family of
non-piercing objects with the additional property that for any three
points there always exists an object that has these three points on
its boundary. 
However the proofs are a bit more involved. 
Since this does not lie in the scope of
this paper, we omit this here and focus only on the special family of
non-piercing objects that is produced by the projection described
above. 


Consider the family of all cones that have  and  on its
boundary. The intersection of all these cones is a cone  that
has  and  on its boundary. Connect  and  by a Jordan curve
 such that it lies entirely in the cone  and on the
flattened lower envelope, for instance part of the boundary of
 that intersects the flattened lower envelope. The image of
 under the projection  is a curve  embedded in
the plane. 


\begin{defi}
  Let  be a family of non-piercing objects and let  be a
  finite set of points.  
  We call two points  \emph{-Delaunay
    neighbors} if there exists an object  that has  and  on
  its boundary and no other point of  in its interior. 
  The
  -Delaunay graph of , in short -, is the graph that
  is embedded in the plane, has  as its vertex set and the edges
    
  between all -Delaunay neighbors  and . 
\end{defi}
Due to the definition of the -Delaunay edge between two
-Delaunay neighbors  and  it is guaranteed that whenever a
object  contains  as well as  then it also must contain
the -Delaunay edge .  
In the following we will prove
that this -Delaunay graph is in fact a triangulation of the vertex
set . 

\begin{lemma}
  \label{lem:tria}
  The -Delaunay graph of the given finite point set  in
  non--degenerate position is a triangulation.  
\end{lemma}
\begin{wrapfigure}[13]{r}{5.2cm}
    \vspace{-4ex}
  \begin{center}
    \psfrag{p}{}
    \psfrag{q}{}
    \psfrag{r}{}
    \psfrag{s}{}
    \epsfig{file=piercing2.eps,width=3cm}
    \caption{Two intersecting -Delaunay edges and their defining
      objects} 
    \label{fig:piercing2}
  \end{center}
\end{wrapfigure}
.\vspace{-4ex}
\begin{proof}
  First, we will prove that - is planar. Suppose
  otherwise, i.e. two edges  and  intersect each
  other in the plane. Since the cone  does not have any point
  in its interior and  also does not have any point in its
  interior and since each of these cones has at most  points on its
  boundary the objects  and  would have to
  pierce each other, see figure~\ref{fig:piercing2} for an
  illustration. Here, it is actually essential, that the set  is in
  non--degenerate position. Thus, the graph is planar. 

  The graph - itself consists of an outer face which is
  defined by cones of the lower envelope that have at most 2 points on
  their boundary and all other faces are triangles defined by the
  cones of the lower envelope that have exactly three points on its
  boundary.  
  Suppose an inner face  is not bounded by a triangle. Then, one
  can place the apex of a cone in such a way onto the flattened lower
  envelope such that its image under the projection  is a point
  which lies inside this face . By moving the cone upward one can
  ensure that the cone will finally have three points on its boundary
  whose image under the projection  are three vertices of the face
   but no point in its interior. Hence, the face  must be
  bounded by a triangle.   
  Hence, - is a triangulation of the set . 
\end{proof}

We call the points of  that lie define the outer face the
\emph{convex hull of  with respect to cone } and we denote it by
. 
It is a generalization of the standard convex hull and we will make
use of it later. 
For a standard triangulation one requires that the outer face is
determined by the convex hull. Here, we replaced the standard convex
hull by the convex hull with respect to cone . This is the
appropriate generalization that we need.  


\begin{lemma}
  \label{lem:connect}
  Let  be an object produced by the projection . The subgraph
   of - induced by the points of  that lie in  is
  connected. 
\end{lemma}

\begin{proof}
  We prove the connectivity using induction over the number of points
  that lie in . 
  If  contains at most 2 points that it must be connected by
  definition and the fact that we can slide down the corresponding
  cone until both points lie on the boundary. So lets assume that
  every object  that contains at most  points from the set 
  induces a connected subgraph . Now consider an object  that
  contains  points of . Consider the cone that is the
  intersection of all cones that contain  exactly those 
  points. This cone has exactly three points on its boundary. We can
  move the cone by a small  in such a way that each of the three
  points can be excluded separately. As all of these induced graphs
  are connected by induction hypothesis, the whole subgraph induced by
   must be connected.    
\end{proof}

We need two more lemmas. Both lemmas basically rely on the fact that
projection  is continuous. 
\begin{lemma}
  \label{lem:cont}
  Let  be a finite point set.
  \begin{enumerate}
  \item For any object , there exists an object  with
    .
  \item For any object , there exists an object  with
    .   
  \end{enumerate}
\end{lemma}

\begin{proof}
  Let  be the corresponding cone of . If we move  upward
  along the internal ray  by a small  then the corresponding
  object  of this cone will satisfy (1). On the other hand, if we
  move the cone  downward along the ray  by a small  then the
  corresponding object  will satisfy (2).  
\end{proof}


\begin{lemma}
  \label{lem:dedge}
  Let  be a finite point set in non--degenerate position, let
   be a -Delaunay edge in -. Then, there exists
  an object  with  and  on its boundary and with .  
\end{lemma}

\begin{proof}
  Let  be the object that assures that  is a -Delaunay
  edge, i.e.  has  and  on its boundary. Since the point set
   is in non--degenerate position  has at most three points
  on its boundary. If  has exactly two points on its boundary we
  are done. So lets assume that  has exactly three points on its
  boundary. Let  be the corresponding cone of  and let the
  corresponding points of  and  be  and .  
  Neither  nor  can lie on the intersection of two of the
  defining planes of cone  because otherwise the cone could still
  be moved in an upward direction such that all three points still lie
  on the boundary until the cone hits a fourth point. But this would
  mean that the point set was in -degenerate position. Hence, 
  and  lie in the interior of two of the plane segments of cone
  .  
  If we now move the cone  downward by a small  such that it
  still touches  and  then the corresponding object of this
  cone will only have  and  on its boundary.  
\end{proof}


Having these properties, we can basically directly apply the algorithm
for finding an epsilon-net for pseudo-disks  
from~\cite{MSW90}. We will describe the algorithm here and prove its
correctness for our setting. 

We are given a finite point set  in non--degenerate position
and we want to find a subset  of size  that stabs
any object  that contains at least  points of . 

Let . First, let  be pairwise disjoint subsets of
 with the following properties: Each  contains  points,
their union contains the convex hull of  with respect to cone ,
i.e.  and each  is representable by
 for an appropriate  cone . Such sets can be easily
constructed by repeatedly biting off points from  with a
suitable cone . Notice, that all these objects 
belong to the collection .  

Next, find a maximal pairwise disjoint collection 
of subsets of the remaining points  satisfying
 for some object  and each subset containing 
points. Obviously, there are at most  many subsets  in
total. 
For an illustration we refer to figure~\ref{fig:delauny1}. 
We assign all points in  the color  and call all other points
\emph{colorless}. Let  be the set of all colored points. 
Note, that if an object contains only colorless points then it
contains less that  points, since the collection of subsets 
was maximal. 
\begin{figure}
\begin{minipage}{0.49\textwidth}
  \begin{center}
    \epsfig{file=delauny1.eps,width=4cm}
    \caption{The sets  and the convex hull  with
      respect to cone . The -Delaunay triangulation is drawn
      dotted.} 
    \label{fig:delauny1}
  \end{center}
\end{minipage}\hfill
\begin{minipage}{0.49\textwidth}
  \begin{center}
    \psfrag{R}{}
    \epsfig{file=delauny2.eps,width=5cm}
    \caption{The corridor  which is split into two sub-corridors
      and two tri-colored triangles. The corners of the sub-corridors
      are marked by crosses.} 
    \label{fig:delauny2}
  \end{center}
\end{minipage}
\end{figure}

Let  be the -Delaunay graph of the set of colored points
, i.e. .  is indeed a triangulation
(cf. lemma~\ref{lem:tria}). In this graph we call a triangle
\emph{uni-colored}, \emph{bi-colored} or \emph{tri-colored} depending
upon the number of colors its vertices have. In a similar way we call
edges uni-colored or bi-colored. We call a maximal connected chain of
bi-colored triangles in  sharing bi-colored edges a \emph{corridor}
(cf. figure~\ref{fig:delauny2}).  
Since the graph  is planar and each of the induced subgraphs  is connected according to lemma~\ref{lem:connect} the number of
such corridors is at most  (\cite{MSW90}). All colorless points
are contained in the corridors and the tri-colored triangles because
any uni-colored triangle is contained it its color-defining object. We
break each corridor  into a minimum number of \emph{sub-corridors},
i.e. sub-chains of the chain that forms , so that each sub-corridor
contains at most  colorless points. Since there are less than 
colorless points and since the total number of corridors is  the
total number of sub-corridors is .  

Each sub-corridor is bounded by two chains of uni-colored edges which
we call \emph{sides} and by two bi-colored edges which we call
\emph{ends} of the sub-corridor. The endpoints of the sides are called
\emph{corners}. Let  be the set of all corners of all
sub-corridors. Since each sub-corridor has at most 4 corners the size
of  is . 
The set  is an epsilon-net for the set of non-piercing objects
.

The proof that  is indeed an epsilon-net relies in principle on the
fact that the collection  are non-piercing objects and follows
along the lines of~\cite{MSW90}. 
\begin{proof}
  Let  be an object that has no points of  on its boundary
  (cf. lemma~\ref{lem:cont}) and assume that  does not contain any
  points from . The theorem is proven when we can show that 
  then contains less than  points of . If  contains no
  colored point then we are done, because the sets  were a
  maximal. Hence,  must contain at least one colored point. If it
  contains two colored points, lets say  of color 1 and  of
  color 2, we can draw the following picture: Let  be the color
  defining object of color 1 and  the color defining object of
  color . Then  intersects  and  but cannot pierce
  them. The area between  and  is a sub-corridor whose ends
  we denote by  and . Lemma~\ref{lem:dedge}
  assures that there is an object  that has  and  on
  its boundary and there is an object  that has  and 
  on its boundary. Since  also does not contain any point from 
  which are the corners of the sub-corridors, i.e. it does not contain
   or  and since  and  as well as  and
   are non-piercing it must lie between two ends of one
  sub-corridor. See figure~\ref{fig:delauny3} for an
  illustration. Now, as all objects , ,  and 
  contain at most  points and the sub-corridor also contains at
  most  points  can contain at most 
  points of . 
\begin{figure}
\vspace{-2ex}
\begin{minipage}{0.45\textwidth}
  \begin{center}
    \psfrag{D}{}
    \psfrag{D1}{}
    \psfrag{D2}{}
    \psfrag{Da}{}
    \psfrag{Db}{}
    \psfrag{a1}{}
    \psfrag{a2}{}
    \psfrag{b1}{}
    \psfrag{b2}{}
    \epsfig{file=delauny3.eps,height=3cm}
    \vspace{-1ex}
    \caption{The case where  contains colored points of at least
      two colors.} 
    \label{fig:delauny3}
  \end{center}
\end{minipage}\hfill
\begin{minipage}{0.45\textwidth}
  \begin{center}
    \psfrag{D}{}
    \psfrag{D1}{}
    \psfrag{D2}{}
    \psfrag{Da}{}
    \psfrag{Db}{}
    \psfrag{a1}{}
    \psfrag{a2}{}
    \psfrag{b1}{}
    \psfrag{b2}{}
    \epsfig{file=delauny4.eps,height=3cm}
     \vspace{-1ex}
    \caption{The case where  contains colored points of exactly one
      color.} 
    \label{fig:delauny4}
  \end{center}
\end{minipage}
\end{figure}

The case where  only contains points of one color and colorless
points is very similar. There is basically only one setup and it is
depicted in figure~\ref{fig:delauny4}. Arguing as above it easy to see
in this case that  cannot contain more than  points
from .  
\end{proof}
Hence, we have the following theorem
\begin{theorem}
  Let  be the set of non-piercing objects in ,
  that is produced by the projection . For every finite point set
  in non--degenerate position there exists an epsilon-net of size 
  . 
\end{theorem}
Together with lemma~\ref{lem:1} and lemma~\ref{lem:2} this
immediately implies our main theorem  
\begin{theorem}
  Given a finite point set  and a polytope . The set
  system  induced by a set of translates of polytope  
  admits an epsilon-net of size . 
\end{theorem}







\vskip-0.3cm
\section{From Epsilon-Nets to Hitting Sets}
In this section we will describe a constant factor approximation
algorithm to the hitting set problem using the epsilon-net of size
 from the previous section. 
Recall that in the hitting set problem we are given a set of points
 and a set polytopes  that are all translates of the same
polytope and we would like to select a subset  of the input
points of minimal cardinality such that every polytope is stabbed by a
point in . We denote the corresponding set system by . 
The fractional hitting set problem is a relaxation of the original
hitting set problem and is defined by the following linear program: 
 

Let  denote the optimal size of the hitting set and  the
optimal value of the fractional hitting set problem. It is known that
the integrality gap is constant for set systems that admit an
epsilon-net of size ~\cite{PA95}.  

Let  be a weight function for the set . We define
the weight  of a subset  to be the sum of the weights of
the elements of . The weighted version of an epsilon-net is as
follows: 
\begin{defi}
  Consider a set system  and a weight function . A set  is called an \emph{epsilon-net with respect to 
    } if  for every subset  for which . 
\end{defi}
There are algorithms that compute a hitting set provided one has an
epsilon-net finder.  
The core idea to all these algorithms is to find a weight function  that assigns weights to the elements of  and finds an
appropriate  such that every set in  has weight at least . Once such weights are found it is then obvious that an
epsilon-net to this set system is automatically a hitting set. 

The algorithm given by Br\"{o}nnimann and Godrich~\cite{BG94} computes
these weights iteratively. Initially, all elements have weight
. Then, in each iteration an epsilon-net is computed and then
checked whether it is also a proper hitting set. If not, i.e. there is
a set which is not hit, then the weights of its elements are
doubled. This is done until a hitting set is found. This algorithm can
be seen as a deterministic analogue of the randomized natural
selection technique used for instance by Clarkson~\cite{C95}. 

Another algorithm is by Even et al.~\cite{ERS05}. Here, the weights of
the elements and  are directly found by the following linear
program: 
 

It suffices to approximate the solution to this linear problem. There
are numerous algorithms that find an approximate solution to such a
covering linear program efficiently~\cite{Y95, GK98}.

One can reduce the problem of finding a weighted epsilon-net to the
unweighted case. One just makes multiple copies of a point according
to its assigned weight and it can be shown that the cardinality of
this multiset can be bounded by ~\cite{CV05}. 
Hence, an -net for this set
system gives a hitting set for the original hitting set problem.
Hence, we have
\begin{theorem}
  There exists a polynomial time algorithm that computes a
  constant-factor approximation to the hitting set problem for
  translates of polytopes in . 
\end{theorem} 


\vskip-0.3cm
\section{From Hitting Set to Set Cover}

\begin{defi}
  The \emph{dual set system} of a set system  is the set
  system  where  and  consists of
  all subsets of  that contain . 
\end{defi}






Obviously, a set cover for the primal set system is a hitting set for
the dual set system. Hence, in order to solve the set cover problem
for a set system it suffices to solve the hitting set problem for the
dual set system. For arbitrary set systems, the dual set system can be
of quite different structure. In general it is only known that the
VC-dimension of the dual set system is less than , where 
is the VC-dimension of the primal set system~\cite{A83}.  

However, we observe that if the set system is induced by translates of
a polytope, then the dual is again induced by  translates of a
polytope. 
To see this, let  be the primal set system. One just reduces
each polytope  to a point, for instance each to its lowest
vertex. Let this be the set . Then, replace each point of  by a
translate of the polytope  which is the inversion of  in a
point. One easily verifies that the so constructed set system  of points  and collection of translates of polytope  is
indeed equivalent to the dual .      
This holds in fact for all .    
Hence, we can find a constant-factor approximation to the set cover
problem for translates of a polytope in  in polynomial time. 
This brings us to our final theorem
\begin{theorem}
  There exists a polynomial time algorithm that computes a
  constant-factor approximation to the set cover problem for
  translates of polytopes in . 
\end{theorem} 


\vskip-0.3cm
\section{Conclusions and Open Problems}
In this paper we have given the first constant-factor approximation
algorithm for finding a set cover for a set of points in  by a
given collection of  translates of a polytope as well as the first
constant-factor approximation algorithm for the corresponding hitting
set problem.  We achieved this result by providing an epsilon-net of
size  for the corresponding set system which is
optimal up to a multiplicative constant. 
Eventhough we can approximate a unit ball in  up to any given
precision by a polytope, the corresponding question, whether there
exists a constant-factor approximation algorithm for unit balls in
 still remains open.    

\vskip-0.3cm
\section*{Acknowledgements}
The author would like to thank Nabil H. Mustafa and Saurabh Ray for
useful discussion on the topic and an anonymous referee for pointing
out an error in a preliminary version of this paper. 












\bibliographystyle{plain}

\vskip-0.3cm
\begin{thebibliography}{10}

\bibitem{A83}
P.~Assouad.
\newblock {Densit\'e et dimension.}
\newblock {\em Ann. Inst. Fourier}, 33(3):233--282, 1983.

\bibitem{BEHW89}
A.~Blumer, A.~Ehrenfeucht, D.~Haussler, and M.~K. Warmuth.
\newblock Learnability and the vapnik-chervonenkis dimension.
\newblock {\em J. ACM}, 36(4):929--965, 1989.

\bibitem{BG94}
H.~Br\"{o}nnimann and M.~T. Goodrich.
\newblock Almost optimal set covers in finite vc-dimension: (preliminary
  version).
\newblock In {\em SoCG '94}, pages 293--302, New York, NY, USA, 1994. ACM
  Press.

\bibitem{C95}
K.~Clarkson.
\newblock Las vegas algorithms for linear and integer programming when the
  dimension is small.
\newblock {\em J. ACM}, 42(2):488--499, 1995.

\bibitem{CV05}
K.~L. Clarkson and K.~Varadarajan.
\newblock Improved approximation algorithms for geometric set cover.
\newblock In {\em SoCG '05}, pages 135--141, New York, NY, USA, 2005. ACM
  Press.

\bibitem{EW85}
H.~Edelsbrunner and E.~Welzl.
\newblock On the number of line separations of a finite set in the plane.
\newblock {\em J. Comb. Theory, Ser. A}, 38(1):15--29, 1985.

\bibitem{ERS05}
G.~Even, D.~Rawitz, and S.~Shahar.
\newblock Hitting sets when the vc-dimension is small.
\newblock {\em Inf. Process. Lett.}, 95(2):358--362, 2005.

\bibitem{GK98}
N.~Garg and J.~K\"{o}nemann.
\newblock Faster and simpler algorithms for multicommodity flow and other
  fractional packing problems.
\newblock In {\em FOCS '98}, page 300, Washington, DC, USA, 1998. IEEE Computer
  Society.

\bibitem{HM85}
D.~S. Hochbaum and W.~Maass.
\newblock Approximation schemes for covering and packing problems in image
  processing and vlsi.
\newblock {\em J. ACM}, 32(1):130--136, 1985.

\bibitem{KPW92}
J.~Koml\'{o}s, J.~Pach, and G.~J. Woeginger.
\newblock Almost tight bounds for epsilon-nets.
\newblock {\em Discrete and Computational Geometry}, 7:163--173, 1992.

\bibitem{LY94}
C.~Lund and M.~Yannakakis.
\newblock On the hardness of approximating minimization problems.
\newblock {\em J. ACM}, 41(5):960--981, 1994.

\bibitem{M}
J.~Matousek.
\newblock {\em Lectures on Discrete Geometry}.
\newblock Springer-Verlag New York, Inc., Secaucus, NJ, USA, 2002.

\bibitem{M92}
J.~Matou\v{s}ek.
\newblock Reporting points in halfspaces.
\newblock {\em Comput. Geom. Theory Appl.}, 2(3):169--186, 1992.

\bibitem{MSW90}
J.~Matou\v{s}ek, R.~Seidel, and E.~Welzl.
\newblock How to net a lot with little: Small epsilon-nets for disks and
  halfspaces.
\newblock In {\em SoCG '90}, pages 16--22, 1990.

\bibitem{PA95}
J.~Pach and P.~K. Agarwal.
\newblock {\em Combinatorial Geometry}.
\newblock Wiley, New York, 1995.

\bibitem{PW90}
J.~Pach and G.~Woeginger.
\newblock Some new bounds for epsilon-nets.
\newblock In {\em SoCG '90}, pages 10--15, New York, USA, 1990. ACM Press.

\bibitem{VC71}
V.~N. Vapnik and A.~Ya. Chervonenkis.
\newblock On the uniform convergence of relative frequencies of events to their
  probability.
\newblock {\em Theory Probab. Appl.}, 16:264--280, 1971.

\bibitem{Y95}
N.~E. Young.
\newblock Randomized rounding without solving the linear program.
\newblock In {\em SODA '95}, pages 170--178, Philadelphia, PA, USA, 1995.
  Society for Industrial and Applied Mathematics.

\end{thebibliography}

\end{document}