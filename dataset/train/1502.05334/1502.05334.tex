\documentclass{article}
\usepackage[margin=1in]{geometry}
\usepackage{amsmath,amssymb,amsthm}
\usepackage{graphicx}
\usepackage{cite}
\usepackage{hyperref}
\usepackage{microtype}

\newtheorem{theorem}{Theorem}
\renewcommand{\theoremautorefname}{Theorem}
\newtheorem{lemma}{Lemma}
\newcommand{\lemmaautorefname}{Lemma}
\newtheorem{definition}{Definition}
\newcommand{\definitionautorefname}{Definition}
\newtheorem{corollary}{Corollary}
\newcommand{\corollaryautorefname}{Corollary}
\newtheorem{observation}{Observation}
\newcommand{\observationautorefname}{Observation}

\title{Simple Recognition of Halin Graphs and Their Generalizations}
\author{David Eppstein\thanks{Computer Science Department, University of California, Irvine; Irvine, CA, USA. This material is based upon work supported by the National Science Foundation under Grant CCF-1228639 and by the Office of Naval Research under Grant No. N00014-08-1-1015.}}

\date{ }

\begin{document}
\maketitle

\begin{abstract}
We describe and implement two local reduction rules that can be used to recognize Halin graphs in linear time, avoiding the complicated planarity testing step of previous linear time Halin graph recognition algorithms. The same two rules can be used as the basis for linear-time algorithms for other algorithmic problems on Halin graphs, including decomposing these graphs into a tree and a cycle, finding a Hamiltonian cycle, or constructing a planar embedding.
These reduction rules can also be used to recognize a broader class of polyhedral graphs. These graphs, which we call the D3-reducible graphs, are the dual graphs of the polyhedra formed by gluing pyramids together on their triangular faces; their treewidth is bounded, and they necessarily have Lombardi drawings. 
\end{abstract}

\section{Introduction}

\emph{Halin graphs} are the graphs that can be formed from a tree with no degree-two vertices, embedded in the plane, by adding a cycle of edges connecting the leaf vertices of the tree in the cyclic order given by the embedding~\cite{Hal-CMA-71}. They are necessarily 3-connected  and planar, with a unique planar embedding up to the choice of the outer face of the embedding; we will adopt the convention that this outer face is the leaf cycle.
Unlike planar graphs more generally, Halin graphs have bounded treewidth (at most three), allowing problems such as the maximum independent set problem which are NP-hard on planar graphs to be solved in polynomial time on Halin graphs~\cite{Bod-ICALP-88}.

Two algorithms for recognizing Halin graphs in linear time are known. Sys{\l}o and Proskurowski~\cite{SysPro-GT-83} showed that a graph with  vertices and  edges is Halin if and only if it is planar and 3-connected, and has a face with exactly  vertices and edges. All of these conditions can be checked in linear time. Fomin and Thilikos~\cite{FomThi-JDA-06} instead observed that in a Halin graph, the outer face has at least  vertices, and that any planar graph can have at most four such faces. They proposed a recognition algorithm that constructs an (arbitrary) planar embedding, and tests for each large face whether its vertices all have degree three and whether removing the edges of the face from the graph leaves a tree with no degree-two vertices. Because there are only a constant number of faces to test, all steps of this algorithm can be performed in linear time. However, both of these algorithms use planarity testing, a problem whose many known linear-time algorithms~\cite{HopTar-JACM-74,BooLue-JCSS-76,ChiNisAbe-JCSS-85,ShiHsu-TCS-99,BoyMyr-JGAA-04,FraOssRos-IJFCS-06,Sch-MFCS-13} are complex and hard to implement. Linear-time 3-connectivity testing, also, has complex algorithms that have proven treacherous to implementors~\cite{HopTar-SJC-73,GutMut-GD-00}.

It would also be possible to base a linear time recognition algorithm for Halin graphs on Courcelle's theorem, which states that the monadic second-order logic of graphs has efficient decision algorithms for graphs of bounded treewidth~\cite{Cou-IC-90}. The existence of a decomposition of the edges of a given graph into a tree and a cycle through the leaves of the tree is straightforward to express in second-order logic. Expressing the correct ordering of the cycle with respect to the planar embedding of the tree is not as straightforward, but can be expressed logically as the statement that every subtree of the tree contacts a contiguous subpath of the cycle. Thus, to test whether a graph is Halin, one can construct a width-three tree-decomposition~\cite{Bod-SJC-96} and then check whether these logical expressions are valid for the decomposition. Such methods are again unlikely to lead to simple, practical, and implementable algorithms, because of the high constant factors resulting from the use of Courcelle's theorem. However, they could be used to recognize Halin graphs in logarithmic space~\cite{ElbJakTan-FOCS-10}.

An alternative approach that has proven successful for many other computational problems on graphs of bounded treewidth involves the notion of a \emph{reduction algorithm}, an algorithm that gradually shrinks the size of the input graph by applying \emph{reduction rules} based on local structures within a given graph~\cite{Flu-PhD-97}. If the reduction rules are chosen to be safe (preserving the property to be tested), complete (applicable to any large enough graph with the property), and terminating (always reducing some appropriate size function of the graphs they operate on), then the graphs with the property can be recognized by repeatedly applying reductions until no more can be found to apply, and then testing whether the remaining smaller graph belongs to a finite set of base cases. Such algorithms have been found for many specific graph classes~\cite{Duf-JMAA-65,ValTarLaw-SJC-82,ArnPro-SJADM-86,BodThi-Algs-99} and more generally are known to exist for all graph classes of bounded treewidth that can be recognized using Courcelle's theorem~\cite{ArnCouPro-JACM-93,Flu-PhD-97}. Thus, in particular, a reduction algorithm exists for recognizing Halin graphs. However, although (unsafe) reduction rules for Halin graphs have been used to solve the Steiner tree and edge-constrained Hamiltonian cycle problems in these graphs~\cite{SkoSys-ZM-87,Win-DAM-87}, to our knowledge, no explicit reduction algorithm for recognizing Halin graphs has been described.

Motivated by these considerations, we describe in this paper two simple reduction rules for Halin graphs that are safe, complete, and terminating, and that can be used to recognize Halin graphs in linear time. Our rules involve augmenting the vertices of the graph with a single additional bit of information, which we can interpret as a color of a vertex, black or white. We use these vertex colors to allow or disallow certain reductions, and then recolor certain vertices after each reduction.
The same two rules can be used as the basis for linear-time algorithms for other algorithmic problems on Halin graphs, including decomposing these graphs into a tree and a cycle, finding a Hamiltonian cycle, or constructing a planar embedding.

It is natural to consider the graphs obtained by simplifying these rules even further by leaving the graph vertices uncolored and allowing all reductions. We call the class of graphs that are recognized by the uncolored version of our two reduction rules the \emph{D3-reducible graphs} because the preconditions for both reduction rules involve triples of degree three vertices. As we show, the D3-reducible graphs are a generalization of the Halin graphs that, like the Halin graphs themselves, are automatically planar and 3-vertex-connected. Thus, by Steinitz's theorem~\cite{Ste-EMW-22}, they are the graphs of polyhedra, and we characterize the D3-reducible graphs geometrically as the dual graphs of the polyhedra that can be constructed by gluing together pyramids on their triangular faces. Additionally, we show that the D3-reducible graphs have treewidth at most four, and that they necessarily have planar \emph{Lombardi drawings}, drawings in which the edges are represented by circular arcs that meet at equal angles at each vertex. Planar Lombardi drawings were previously known to exist for Halin graphs and for planar graphs of maximum degree three~\cite{DunEppGoo-JGAA-12,Epp-DCG-14}, but beyond these classes their existence is somewhat mysterious; we do not even know whether they exist for all outerplanar graphs~\cite{LofNol-GD-12}.

\section{D3 reductions}

If  is a tree with four or more vertices, none of degree two, it can be reduced to  by reduction steps that either remove the two leaf children from a vertex of degree three or remove a leaf from a vertex of degree greater than three. Taking into account the cycle edges added to such a tree to form a Halin graph gives us the following two reduction rules:

\begin{description}
\item[D3a.] Let , , and  be three degree-three vertices that induce a triangle in the given graph , and whose neighbors outside the set  are all distinct. Replace these three vertices by a single vertex with the same three outside neighbors. (\autoref{fig:reductions}, left.)
\item[D3b.] Let , , and  be three degree-three vertices that induce a path, with  as the middle vertex, and suppose additionally that there is a single vertex  adjacent to all three of , , and .
Delete  from the graph and replace it by a new edge from  to . (\autoref{fig:reductions}, right.) We refer to  as the \emph{apex} of the reduction and  as the \emph{middle vertex} of the reduction.
\end{description}

\begin{figure}[t]
\centering
\includegraphics[scale=0.35]{reductions}
\caption{The two D3 reductions. Left: three degree-three vertices , , and  form a triangle with three distinct neighbors, and are collapsed to a single vertex . Right: three degree-three vertices , , and  form a path with one shared neighbor , and are contracted to a two-vertex path.}
\label{fig:reductions}
\end{figure}

We collectively refer to rules D3a and D3b as the D3 reductions.

\begin{definition}
We define a \emph{D3-reducible graph} to be a graph that can be reduced to the four-vertex complete graph  by a sequence of D3 reductions. We define a graph to be \emph{irreducible} if no additional D3 reductions can be applied to it.
\end{definition}

For instance,  is irreducible, because all triples of its degree-three vertices induce triangles but do not have three distinct neighbors outside of each of these triangles.
As we now show, it will not be necessary to search for a special reduction sequence that leads to  in order to recognize these graphs: all reduction sequences lead to isomorphic graphs.

\begin{lemma}
\label{lem:interchangeable}
Let  be any graph, and  and  be two D3 reductions that are both applicable in .
Then either  and  may both be applied independently (in either order) or the result of performing  is isomorphic to the result of performing .
\end{lemma}

\begin{figure}[b]
\centering
\includegraphics[scale=0.35]{confluent}
\caption{Non-independent pairs of D3 reductions. Left: D3a reductions collapsing either of the two yellow triangles give isomorphic results. Right: D3b reductions shortening either of the two overlapping paths of three yellow vertices give isomorphic results.}
\label{fig:confluent}
\end{figure}

\begin{proof}
If  is a triangle of degree-three vertices, and a D3 reduction that is not a D3a reduction of  is performed, then  remains a triangle of degree-three vertices.
And if  is an induced path with a shared neighbor , and a D3 reduction that is not a D3b reduction of  is performed, then  remains the middle vertex of an induced path with a shared neighbor~. So the only way one reduction  could prevent the future performance of another reduction  is by changing the neighbors of a middle vertex  of a D3b reduction or causing the outside neighbors of a triangle to become non-distinct.
For this to happen, we have the following cases:
\begin{itemize}
\item If  and  are both D3a reductions, then their two triangles of degree-three vertices are connected to each other by two or three edges. If they are connected by three edges, then the result of performing either reduction is . If they are connected by only two edges, then after either of the two reductions the result is a graph in which the two non-adjacent vertices of the two triangles are linked by a pair of triangles that share an edge (\autoref{fig:confluent}, left).
\item If  and  are both D3b reductions, then they can only affect each other if their two paths share an edge, so that they are both part of a path of four degree-three vertices that all are adjacent to the same apex. In this case the results of performing either reduction are a graph in which this four-vertex path has been replaced by a three-vertex path (\autoref{fig:confluent}, right).
\item If one of  and  is a D3a reduction and the other is a D3b reduction, then neither reduction can affect the other one: a D3a reduction can't change the neighbors of the middle vertex of a path, and a D3b reduction can't make previously-distinct vertices become the same as each other.\qedhere
\end{itemize}
\end{proof}

\begin{lemma}
\label{lem:confluence}
In a 3-vertex-connected graph, all maximal sequences of D3 reductions lead to isomorphic irreducible graphs.
\end{lemma}

\begin{proof}
Let  and  be two maximal sequences of reductions starting from the same graph , and let  be the first reduction in . We will prove that there exists a reduction sequence  such that  begins with , and such that  and  transform  into isomorphic graphs. The claimed result will then follow by induction on the length of the two sequences,
by applying the induction hypothesis to the graph obtained from  by reduction . As a base case, when either  or  is empty, then  is already irreducible, and there can only be one maximal sequence of reductions (the empty sequence).

To show that every sequence  has an equivalent sequence  beginning with , we again use induction on the length of .  cannot be empty, for operation  can be applied to  and the result of applying  should be an irreducible graph; therefore, we can define  to be the first operation in . We have three cases:
\begin{itemize}
\item If  then  already begins with  and the result follows.
\item If  and  both give isomorphic graphs then we can create the desired sequence  by replacing  by  and again the result follows.
\item In the remaining case, by \autoref{lem:interchangeable},  and  may be applied independently. Let  be the graph obtained from  by reduction , and let  be the reduction sequence on~ obtained from  by removing the first reduction~.
Then  may be applied to , and by induction there exists a reduction sequence  on~ that begins with  and has the same effect as . The desired reduction sequence  may be obtained by applying reductions  and  followed by the remaining reductions (after~) in~.\qedhere
\end{itemize}
\end{proof}

In the language of rewriting systems, \autoref{lem:confluence} means that D3 reductions are \emph{confluent}, or have the \emph{Church--Rosser property}. This allows us to apply them greedily without worrying about the ordering of the reductions.

\begin{theorem}
We can recognize D3-reducible graphs in linear time.
\end{theorem}

\begin{proof}
We maintain an adjacency list representation of the graph  after a sequence of reductions, allowing edge insertions and removals, degree tests, finding the endpoints of an edge, and finding the neighbors of a bounded-degree vertex in constant time per operation. We also maintain a collection  of vertices or former vertices of the graph that is guaranteed to contain at least one of the three degree-three vertices of each possible D3a reduction and the middle vertex of each possible D3b reduction. Our algorithm performs the following steps:
\begin{enumerate}
\item Initialize  to the set of all degree-three vertices in .
\item While  is non-empty:
\begin{enumerate}
\item Select and remove an arbitrary vertex  from .
\item If  and two of its neighbors form the triangle of a D3a reduction, perform that reduction; add the new vertex formed by the reduction and all its degree-three neighbors to .
\item Otherwise, if  and two of its neighbors form the path of a D3b reduction, perform that reduction; if this causes the apex of the reduction to have degree three, add it to .
\end{enumerate}
\item Test whether the resulting graph is .
\end{enumerate}
 initially contains  vertices. Each successful reduction adds  vertices to it, so the total number of vertices ever added to  is linear. The time for the algorithm is  for the initialization and final testing stages, and  per vertex in  for the inner loop, giving  in total. The correctness of the algorithm follows from \autoref{lem:confluence}.
\end{proof}

We remark that the graphs that can be obtained using only D3a reductions are the dual graphs of planar 3-trees, and the graphs that can be obtained using only D3b reductions are the wheel graphs.

\section{Shared properties with Halin graphs}

As we now show, D3-reducible graphs have many of the same properties that Halin graphs are known to have. We will use this fact to simplify some algorithmic computations on Halin graphs, such as the search for Hamiltonian cycles, by generalizing these computations to D3-reducible graphs.

\begin{theorem}
Every D3-reducible graph is 3-vertex-connected.
\end{theorem}

\begin{proof}
We use induction on the number of D3 reductions in an (arbitrarily chosen) sequence of reductions that takes the given graph  to . As a base case,  itself is connected. Otherwise, let  be the graph formed from  by the first reduction in the sequence. By induction, every two pairs of vertices in  have three vertex-disjoint paths connecting them, and we must show that the same is true in . We divide into cases:
\begin{itemize}
\item For pairs of vertices  outside the set of three vertices , , and  defining the reduction,
the three paths in  connecting  to  may be straightforwardly modified to give three paths in , replacing a path through the collapsed vertex of a D3a reduction by a path through two vertices, and a path through edge  of a D3b reduction by a path through edges  and .
\item For pairs of vertices one of which is , , or , the required three vertex-disjoint paths connecting the pair of vertices can be found by replacing , , or  by one of the corresponding vertices in , finding three vertex-disjoint paths in , and again making straightforward modifications to find three vertex-disjoint paths in the original graph.
\item In the remaining case, we are given two vertices of , , and , and must find three vertex-disjoint paths connecting them in . First, suppose that  is obtained by a D3a reduction; by symmetry, we may assume that we are finding paths connecting  and . In this case, two such paths exist within the triangle defining the D3a reduction; the third path can be found as one of the three paths connecting the outside neighbors of  and . Second, suppose that  is obtained from a D3b reduction and that we are connecting vertices  and . Again, two of the required paths from  to  exist: the induced path  and the path  through the apex of the reduction. The third path can be found as one of the three paths connecting the outside neighbors of  and . Third and finally, suppose that we are connecting vertices  and  of a D3b reduction. In this case, we have paths  and  within the reduced part of the graph, and a third path through  together with one of the three paths connecting the outside neighbors of  and .
\end{itemize}
Thus, for all pairs of vertices in , there exist three vertex-disjoint paths, and the result follows.
\end{proof}

An alternative proof of this theorem can be given by applying a known reduction-based characterization of the 3-vertex-connected graphs: a graph is 3-connected if and only if it can be reduced to  by the Barnette--Gr\"unbaum reduction rules~\cite[Thm.~1]{BarGru-MFGT-69}. These rules allow the removal of any edge from a graph and the suppression of any vertex of degree two created by this removal. Arbitrary edge removals can fail to reach , but a sequence of reduction steps leading to  can be found in polynomial time when it exists~\cite{Sch-Algo-12}. The same reduction rules have also been used as part of a linear-time planarity test~\cite{Sch-MFCS-13}. In our case, a D3a reduction can be seen as a Barnette--Gr\"unbaum reduction that removes a triangle edge, and a D3b reduction can be seen as a Barnette--Gr\"unbaum reduction that removes the edge between the middle vertex of a path and the apex of the reduction. Therefore, the reduction sequence for a D3-reducible graph also gives a Barnette--Gr\"unbaum reduction sequence taking the graph to~.

For the next property of D3-reducible graphs, recall that, when a 3-vertex-connected graph is planar, its planar embedding is unique up to the choice of the outer face, and its faces are exactly the induced cycles for which the graph induced by the complementary set of vertices is connected~\cite{Bru-JCTB-04}.

\begin{theorem}
Every D3-reducible graph is planar, and every triangle in the graph is a face of its unique planar embedding.
\end{theorem}

\begin{proof}
We use induction on the number of D3 reductions; as a base case,  clearly has the stated properties. For any other D3-reducible graph , suppose that the graph has a D3 reduction  leading to a smaller graph ; by the induction hypothesis,  is planar with all triangles as faces. We have two cases:
\begin{itemize}
\item If  is a D3a reduction, then  may be obtained from  by replacing a degree-three vertex  by a triangle. A planar embedding of  may be obtained from the embedding of  by adding one new edge to each of the three faces that meet at , and forming a new face triangle from the three new edges. The only new triangle created by this replacement is necessarily a face.
\item If  is a D3b reduction, then  may be obtained from  by subdividing an edge  that belongs to a triangle  and connecting the new subdivision vertex to the opposite apex  of the triangle. An embedding of  may be obtained in the same way, by splitting the face  of the embedding of  into two new triangular faces. The two new triangles formed from the subdivision are again faces of the subdivided embedding.\qedhere
\end{itemize}
\end{proof}

The proof of this result may be used to derive a linear-time algorithm to construct a planar embedding of a D3-reducible graph, more simply than using a general-purpose linear-time planar embedding algorithm, by reversing the sequence of reductions and maintaining a planar embedding for each step of the reversed reduction sequence.

\begin{theorem}
\label{thm:ham}
Every D3-reducible graph has a Hamiltonian cycle that can be found in linear time.
\end{theorem}

\begin{figure}[t]
\centering
\includegraphics[scale=0.35]{ham}
\caption{Cases for \autoref{thm:ham}. In each case the rightward pointing arrow describes a D3 reduction from the given graph  to a smaller graph , and the leftward pointing arrow shows how to modify the Hamiltonian cycle  in  (thick red edges) to give a Hamiltonian cycle in~.}
\label{fig:ham}
\end{figure}

\begin{proof}
We use induction on the number of D3 reductions.
As a base case,  is Hamiltonian. For any other D3-reducible graph , suppose that the graph has a D3 reduction  leading to a smaller graph ; by the induction hypothesis,  has a Hamiltonian cycle . We have three cases:
\begin{itemize}
\item If  is a D3a reduction that replaces triangle  by a new vertex , then let  and  be the two edges of  that pass through , and relabel the vertices if necessary so that  and  are edges in . Then to form a Hamiltonian cycle in , replace  and  in  by the four edges , , , and . (\autoref{fig:ham}, left.)
\item If  is a D3b reduction of path  with apex , and  passes through edge ,
then a Hamiltonian cycle in  may be obtained by replacing  by  and  in . (\autoref{fig:ham}, upper right.)
\item In the remaining case,  is a D3b reduction of path  with apex , and  does not pass through edge . Then (because  and  both have degree three in , and have two incident edges in ) the two edges  and  must both belong to .
In this case, a Hamiltonian cycle  in  may be obtained by replacing  by
 and . (\autoref{fig:ham}, lower right.)
\end{itemize}
This inductive proof translates directly to an algorithm that reverses the reduction sequence of the  graph and maintains a Hamiltonian cycle for the graph at each step of the reversed reduction sequence. Updating the cycle after each reversed reduction takes constant time so the total time for the algorithm is linear.
\end{proof}

Halin's original motivation for studying Halin graphs was that they provided a natural class of minimally three-connected graphs. This is also true more generally for D3-reducible graphs.

\begin{theorem}
For every D3-reducible graph , and every edge  of , the graph  formed by deleting  from  is not 3-vertex-connected.
\end{theorem}

\begin{proof}
Fix a planar embedding of , and let  and  be the faces of  on the two sides of  in the embedding. We prove more strongly that there exists a Jordan curve that passes through an interior point of , a vertex  in  (disjoint from ), an a vertex  of  (also disjoint from ) without passing through any other vertices or edges of~. Equivalently, there exists a face , distinct from  and , that includes both  and  as its vertices, so that the desired Jordan curve can be partitioned into three arcs: one in  from  to , one in  from  to , and one in  from  back to . Because edge  crosses this Jordan curve once, it necessarily separates  from , so  and  form a 2-separation of .

If either  or  has degree three, we may take  and  to be the two of its neighbors that are disjoint from , and  to be the third face (with  and ) that is incident to the degree-three vertex. Otherwise,  cannot be , so we may assume that it has a D3 reduction  taking it to a smaller D3-reducible graph . Since neither  nor  has degree three, the same edge  is also present in . By induction,  has faces , , and  and vertices  and  with the desired incidence relations to each other.
Whenever , , or  is not the triangular face resulting from a D3b reduction, it has a corresponding face , , or  in . We have the following cases.
\begin{itemize}
\item If  is a D3a reduction whose new supervertex is disjoint from  and , then the same vertices  and  and (possibly modified) faces , , and  have the same incidence relations in .
\item If  is a D3a reduction that is not disjoint from  and , we may assume by symmetry that  is the supervertex formed by contracting a triangle . Then in , faces  and  still meet at one of , , or ; relabel the triangle if necessary so that they meet at .
Then vertices  and  and faces , , and  have the desired incidence relations.
\item If  is a D3b reduction,   and  are adjacent in , and edge  is not created by reduction , then they remain adjacent in , and either of the two faces incident to them may be chosen as .
\item If  is a D3b reduction,   and  are adjacent in , and edge  is created by reduction  that removed the middle vertex  of a path , then in  the path  has the two triangles of the D3b reduction on one side of the path, and a single face  incident to both  and  on the other side of the path (since  necessarily has degree three). Again, , , , , and  have the desired incidence relation.
\item If  is a D3b-reduction, and  and  are not adjacent in , then  is a face of  with four or more vertices. Then  corresponds to a face  of  with either the same set of vertices, or with one more vertex (the middle vertex  of the path that was shortened by the D3b reduction). Vertices , , and faces , , and  have the desired incidence relation.
\end{itemize}
Thus, in all cases we have shown the existence of three faces and two vertices that, together with edge , support a Jordan curve separating  from .
\end{proof}

\section{Decomposition, duality, and graph drawing}

\begin{figure}[b]
\centering\includegraphics[height=2in]{trunctet}
\caption{A Lombardi drawing of the graph of the truncated tetrahedron (left) and the circle packing used to construct the drawing (right). This graph is D3-reducible and has treewidth~4.}
\label{fig:trunctet}
\end{figure}

Halin graphs all have treewidth three, but this is not true of D3-reducible graphs. In particular, the graph of the truncated tetrahedron (\autoref{fig:trunctet}) is D3-reducible, but has treewidth~four: contracting the six edges that do not belong to triangles produces the octahedral graph , which is one of the minor-minimal graphs of treewidth four. However, this example has the largest treewidth possible for these graphs. To prove this, we provide a structural description of the dual graphs of D3-reducible graphs, in terms of clique-sums, operations in which complete subgraphs of pairs of graphs are identified.

Suppose that  and  are two polyhedral graphs in which we have identified an explicit isomorphism between two triangular faces  of  and  of . Then we may glue these two graphs together by forming the disjoint union of  and  and then collapsing each identified pair of vertices --, --, and -- to a single supervertex. A general clique-sum operation would also allow the removal of some or all of the triangle edges but we do not do this. The result of this gluing operation is a larger polyhedral graph in which the two faces have become a single non-facial triangle. We may perform repeated gluing operations on a collection of polyhedral graphs in the same way, but each triangular face of a graph in the collection may take part only in one of these gluing operations (after which it is no longer a face). We do not allow a graph to be glued to itself, whether it is one of the given graphs or the result of previous gluing steps, because this would not necessarily preserve planarity. Gluing together  polyhedral graphs involves  gluing steps (each of which reduces the number of graphs by one), and we can represent these steps abstractly as the edges of a tree whose nodes correspond to the given graphs. The order in which the gluing steps are performed does not affect the result.

\begin{theorem}
\label{thm:glue}
A graph  is the planar dual of a D3-reducible graph if and only if  can be constructed by gluing together a collection of polyhedral graphs, as described above, such that each graph in the collection is a wheel graph (the graph of a tetrahedron or pyramid).
\end{theorem}

\begin{proof}
In one direction, suppose that  is formed by gluing together wheel graphs. We may order the gluing steps so that each step glues a single wheel onto another graph, rather than gluing together two graphs that are themselves the result of other gluing steps. Gluing a four-vertex wheel (the complete graph ) can be equivalently described as subdividing a triangular face of  into three smaller triangles; the time-reversed operation in the dual graph is a D3a reduction. Gluing a larger wheel may be described as a multiple-step process in which we first glue a four-vertex wheel and then increase the number of vertices in the glued wheel; each of these vertex-increasing operations is the dual to a time-reversed D3b reduction. Thus, reversing and dualizing the sequence of gluing and wheel-increase steps gives us a D3 reduction of the dual graph, showing that it is D3-reducible.

In the other direction, suppose that  is the dual graph of a D3-reducible graph . As a base case, if  is ,  is also  and is the graph of a four-vertex wheel. Otherwise, let  be a D3 reduction in  taking it to a smaller graph , and let  be the dual of . By induction, we may assume that  has a representation as a gluing of wheel graphs. If  is a D3a reduction, the dual operation to  un-subdivides a triangle of , and is equivalent to the time-reversal of gluing a four-vertex wheel onto . If  is a D3b reduction of a path  with apex  then the vertex  of  dual to triangle  has degree three; because each gluing step increases the degree of the glued vertex, this implies that  belongs only to a single wheel of the gluing for . The two vertices  and  are dual to adjacent triangles in , and the D3b reduction is the time-reversed dual of an operation that expands the edge between them into another triangle, increasing the number of vertices of this wheel. Thus, as before, reversing and dualizing the sequence of D3 reductions for  gives us a sequence of gluing steps for constructing .
\end{proof}

\begin{corollary}
The dual graph of a D3-reducible graph has treewidth three.
\end{corollary}

\begin{proof}
Every wheel graph is a Halin graph, so it has treewidth three, and it is known that clique-sums do not increase the treewidth of the graphs they combine~\cite{Lov-BAMS-06}.
\end{proof}

\begin{corollary}
\label{cor:tw4}
Every D3-reducible graph has treewidth at most four.
\end{corollary}

\begin{proof}
This follows from the fact that the treewidth of a graph is at most one more than the treewidth of its dual graph~\cite{BouMazTod-DM-03}.
\end{proof}

It would be of interest to find a direct proof of \autoref{cor:tw4} that leads to a simple linear-time construction of a width-four tree-decomposition.

The gluing construction for the duals of D3-reducible graphs can also be applied in the construction of graph drawings for the D3-reducible graphs themselves. It is a famous theorem that the vertices of every planar graph can be represented by interior-disjoint disks in such a way that two disks are tangent if and only if the corresponding two vertices are adjacent~\cite{Ste-ICP-05}. For duals of D3-reducible graphs this representation can be chosen with an additional property:

\begin{lemma}
\label{lem:cpack}
Let  be the dual of a D3-reducible graph . Then its vertices can be represented by interior-disjoint disks as above such that, for every face of , the disks for the vertices of the face are equivalent under a M\"obius transformation to a ring of  congruent disks with cocircular centers.
\end{lemma}

\begin{proof}
Every wheel graph has a representation of this form. If two polyhedral graphs  and  are glued together on triangular faces, their disk representations may also be obtained by gluing together the representations for  and , applying a M\"obius transformation to the representation of  make the three disks for the gluing face have the same size and position as they do in the representation of .  The result follows from \autoref{thm:glue}.
\end{proof}

\begin{corollary}
Every D3-reducible graph has a planar \emph{Lombardi drawing}, a drawing in which the vertices are represented by points and the edges are represented by circular arcs that meet at equal angles at each vertex.
\end{corollary}

\begin{proof}
A construction of the author for Lombardi drawings of cubic graphs~\cite{Epp-DCG-14} forms a disk representation for the dual graph.
It defines a distance function from points to these disks, where the distance from point  to disk  is the radius of two congruent disks that are tangent to each other at  and also both tangent to , and constructs the minimization diagram of this distance, a partition of the plane into cells within which one of the disks is closer than all the others~\cite[Sec.~3]{Epp-DCG-14}. This minimization diagram has circular arcs for boundaries~\cite[Lem.~2]{Epp-DCG-14}, which in the case of cubic graphs necessarily meet at angles of , forming a Lombardi drawing of the original graph. It is invariant under M\"obius transformations of the plane: transforming a circle packing and then constructing the minimization diagram, or constructing the diagram first and then transforming it, produces the same result~\cite[Lem.~1]{Epp-DCG-14}.

For a D3-reducible graph we use the same minimization diagram for the circle packing of \autoref{lem:cpack}.
The resulting minimization diagram again has piecewise-circular boundaries between cells and is invariant under M\"obius transformations of the plane. By symmetry and M\"obius invariance, these boundaries must meet at equal angles at a point within each face of the dual graph, forming a Lombardi drawing of the primal D3-reducible graph.
\end{proof}

An example of a Lombardi drawing constructed in this way for the graph of the truncated tetrahedron is shown in \autoref{fig:trunctet}. This graph has all vertices of degree three, a property already known to guarantee the existence of a Lombardi drawing~\cite{Epp-DCG-14}, but the same method works as well for D3-reducible graphs with vertices of higher degree. However, it is not true that the duals of D3-reducible graphs always have planar Lombardi drawings; indeed, it is known that some planar 3-trees (a special case of the duals of D3-reducible graphs) do not have such drawings~\cite{DunEppGoo-GD-11}.

\section{Halin graph recognition}

We return to our motivating problem of developing a simple algorithm for Halin graph recognition.
We have already seen that the D3 reductions will apply to any Halin graph; however, they also apply to some graphs that are not Halin graphs, so we need to modify the reduction process to avoid
these more general reductions. The key observation is the following:

\begin{observation}
\label{obs:reduction-is-at-leaves}
Let  be a Halin graph, constructed from a tree  with outer cycle . Then every D3a reduction in  must form a simpler Halin graph by removing the two children from a node of  that has only two children, both leaves, and every D3b reduction must form a simpler graph by removing a middle leaf child from a tree node that has three consecutive leaves among its children.
\end{observation}

A Halin graph may have more than one decomposition into a tree and a cycle, but if so this observation applies simultaneously to all of these decompositions. The reason the observation holds is that both the D3a and D3b reductions require the presence of a triangle in , and the only triangles in a Halin graph can be the ones formed by two leaf edges of  and an edge of .

Intuitively, whenever we perform a reduction in a Halin graph, \autoref{obs:reduction-is-at-leaves} gives us more information about the set of vertices that belong to the outer cycle. We will use this information to check whether an arbitrary D3-reducible graph is Halin.
Our algorithm for testing whether a graph  is Halin follows the same outline as the algorithm for testing whether  is D3-reducible, with the following modifications:
\begin{itemize}
\item We maintain a set  of vertices that are known to belong to the outer cycle of a Halin graph representation of ; initially,  is empty.
\item Whenever we perform a D3a reduction of a triangle , replacing it by a vertex , we first check whether all three of , , and  belong to ; if they do, we forbid this reduction.
Otherwise we perform the reduction and then add  to . Additionally, if any one or two of , , or  were already members of , we add the neighbor or neighbors of these known-outer vertices to  as well. Examples of this reduction are shown by the two rightmost arrows in \autoref{fig:nonhalin}.
\item Whenever we perform a D3b reduction of a path  with apex , removing the middle vertex , we check whether  is in ; if so, we forbid the reduction. Otherwise, we add  and  to . Additionally, if the reduction reduces the graph to , we add the fourth vertex (the one that is not ) to . Examples of this reduction are shown by the left and upper middle arrows in \autoref{fig:nonhalin}. The lower right graph of the figure gives an example in which there are four potential D3b reductions, but all are forbidden because their shared apex is in~.
\item When an irreducible graph is reached, as well as checking that it is isomorphic to , we check that it has at least one vertex that does not belong to . If so, we recognize it as a Halin graph; otherwise we do not. The upper right graph of \autoref{fig:nonhalin} gives an example in which the recognition algorithm can reach a  graph but fails to recognize the graph as Halin because all vertices belong to~.
\end{itemize}

\begin{figure}[t]
\centering
\includegraphics[scale=0.5]{nonhalin}
\caption{A D3-reducible graph that is not Halin and its possible reduction sequences (up to isomorphism) under the Halin graph recognition algorithm. The set of known-outer vertices is shown in red.}
\label{fig:nonhalin}
\end{figure}


\begin{lemma}
\label{lem:Halin-subset}
If  is a Halin graph then the
modified algorithm described above will never forbid a reduction. The graph remaining after each step will also be Halin, and the intersection of that graph with  will consist only of vertices that belong to the outer cycle of every decomposition of this graph into a tree and a cycle.
\end{lemma}

\begin{proof}
We prove the lemma by induction on the number of steps of the algorithm; initially  is empty and the result holds vacuously.

At each D3a reduction, in which a triangle  is contracted, one of the triangle vertices (say ) must be the parent of the other two vertices  and , which must be leaves in . By the induction hypothesis,  is not in  prior to the D3a reduction so the reduction will not be forbidden. After the reduction, the removal of two leaves causes the contracted supervertex to become a leaf in the reduced version of , so adding it to  is valid. And the only edges in  incident to  and  are the ones connecting them to their parent , so if this step causes the neighbors of  and  to become added to  the result is again valid.

At each D3b reduction, in which a path  with apex  is shortened, the three vertices , , and  must all be children of  in . By the induction hypotheses  will not belong to  and the reduction will not be forbidden. Vertices  and  remain leaf children of  after the reduction, so adding  and  to  is valid.
\end{proof}

\begin{corollary}
\label{cor:no-false-neg}
If  is a Halin graph then the algorithm described above will correctly recognize  as being a Halin graph.
\end{corollary}

\begin{proof}
By Lemma~\ref{lem:Halin-subset},  will be reduced to an irreducible graph, which must be , and this graph must have a face that forms a superset of .
Therefore, there will be at least one vertex of the irreducible graph that is not in , so the termination condition of the algorithm is met and the algorithm will necessarily recognize  as Halin.
\end{proof}

\begin{lemma}
\label{lem:no-false-pos}
If  is recognized by the algorithm described above, then it is indeed a Halin graph, and has a decomposition into a tree  and a cycle  in which the vertices of  all belong to the cycle. \end{lemma}

\begin{proof}
We prove the result by induction on the size of . If  is irreducible, it can only be recognized if it is , which is indeed a Halin graph. Otherwise, suppose that  is the first reduction found by the algorithm, and let  be the smaller graph formed from  by reduction . By induction,  is Halin, with a decomposition into a tree  and cycle  with . We have the following cases:
\begin{itemize}
\item If  is a D3a reduction of triangle , replacing these three vertices by a single vertex , then (because the algorithm adds  to )  must belong to , and must form a leaf of the tree .  Two edges  and  must belong to , and the third edge  cannot (because a cycle has degree two at each vertex). To form a Halin graph decomposition of , we replace  in  by , and add  and  as children of  to form the tree . We form the cycle  by replacing the edges  and  in  by the three edges , , and . The resulting tree and cycle decompose  in the manner required of a Halin graph, so  is Halin.

Vertices  and  belong to the cycle  but vertex  does not. Vertex  cannot have been part of  prior to performing reduction ,
because if it were then in  vertices  and  would both belong to , forcing edge  to belong to  (because  is Halin and in a Halin graph every edge between leaf vertices belongs to the outer cycle) contradictory to our assumption. Therefore, in  it remains true that the vertices of  all belong to cycle .
\item If  is a D3b reduction of path  with apex , removing  and shortening the path,
then after the reduction  and  belong to , so they must both be leaf vertices of . The edge  connecting them must belong to . The other two edges  and  of the triangle  in  must be leaf edges of , for the only other possibility (that together with  they form the outer cycle of a  graph) is prevented by the special handling of a  in the D3b reduction. We form the cycle  by replacing edge  by the path , and we form the cycle  by adding  as a leaf child of~. The resulting tree and cycle decompose  in the manner required of a Halin graph, so  is Halin.

In this case, all vertices other than  either belong to both  and , or belong to neither.
Therefore, the condition that  is a subset of  follows from the induction hypothesis that  is a subset of , together with the fact that the construction places ~in~.\qedhere
\end{itemize}
\end{proof}

\begin{theorem}
The algorithm described above correctly recognizes Halin graphs in linear time,
and can be modified to construct a decomposition of a Halin graph into a tree and a cycle in linear time.
\end{theorem}

\begin{proof}
The correctness of the algorithm follows from \autoref{cor:no-false-neg} and \autoref{lem:no-false-pos}. The modifications to the D3-reducibility algorithm add constant time per reduction so the time analysis is the same as for testing D3-reducibility. To construct a decomposition, we reverse the steps of the reduction and use \autoref{lem:no-false-pos} to maintain at each step of the reversed sequence a decomposition of the Halin graph from that step of the sequence; again, this adds constant time per step of the reduction.
\end{proof}

\section{Implementation}

To support our claim that the reduction-based method described here leads to simple and implementable algorithms, we developed a proof-of-concept implementation of our algorithms in the Python programming language, including the algorithms for testing D3-reducibility, finding Hamiltonian cycles in D3-reducible graphs, testing whether a graph is Halin, and finding the set of leaf nodes of an (arbitrarily chosen) decomposition of a Halin graph into a tree and a cycle.

\subsection{Graph representation}

To support constant-time graph reduction operations, adjacency tests, neighbor listing operations, and neighbor counting operations, we use a modified version of a graph representation scheme suggested by van Rossum~\cite{Ros-PP-98}. In van Rossum's representation, a graph is a Python dictionary object (a hash table) with vertices as its keys and with Python lists (dynamic arrays) of neighboring vertices as the associated values. There is no need for special vertex objects: vertices in this representation are allowed to be any type of object that can be used as keys in a dictionary, such as integers or strings.

However, a Python list does not allow constant-time membership testing, nor the constant-time removal of a vertex from a list of neighbors without knowing its position in the list. Therefore, we modify van Rossum's representation by
representing a graph as a dictionary with the vertices as keys and with Python sets of neighboring vertices as the associated values. The set data type was introduced to Python subsequently to van Rossum's original proposal for this representation. Using sets in place of lists allows more flexible and fast addition, removal, and membership testing in each vertex neighborhood.

\subsection{Software architecture}

In order to maximize the code re-use of our implementation, we designed it to have a central core that finds and performs D3 reductions on a graph, and that takes as arguments callback routines that either modify the sequence of reductions that can be performed or record information about the reductions as they are performed.

More specifically, the main subroutine of our implementation takes three arguments: a list of \emph{triangle hooks}, another list of \emph{path hooks}, and a \emph{finalizer}. These arguments have the following meanings:

\begin{itemize}
\item The triangle hooks are a list of subroutines that are called, in the order given by the list, before performing any D3a reduction. These subroutines take the graph and seven vertices as arguments: the six vertices forming the configuration to be reduced, and a seventh vertex that will replace the central triangle in this configuration. They return a Boolean value, true if the reduction should be allowed to happen and false otherwise. If any triangle hook returns false, the remaining ones on the list are not called; otherwise, all are called. As well as being used to constrain which D3a reductions occur, these hooks may also be used to record information about the sequence of reductions performed by the algorithm.
\item The path hooks are another list of subroutines, used in the same way for D3b reductions as the triangle hooks are used for D3a reductions. They each take five arguments: the graph, three path vertices and apex of a D3b reduction.
\item The finalizer takes as input the irreducible graph after all reductions are complete, and produces as output the value that should be returned as the result of the overall computation.
\end{itemize}

\noindent
We  implemented several additional sets of subroutines to be used as these arguments:
\begin{itemize}
\item To recognize Halin graphs, we use triangle and path hooks that maintain a set of vertices required to be part of the outer cycle, and that prevent reductions inconsistent with this requirement.
We use a finalizer that returns true when the graph can be reduced to  without requiring all four vertices to be part of the outer cycle, and false otherwise.
\item We also implemented alternative recognition algorithms based on D3 reductions for testing whether a given graph is the dual graph of a planar 3-tree, or whether it is a wheel. These algorithms use trivial triangle or path hooks that prevent any D3b reduction in the case of dual 3-trees, or that prevent any D3a reduction in the case of wheels, together with a finalizer that merely checks whether the reduced graph is .
\item We implemented a pair of triangle and path hooks that record a sequence of D3 reductions. We use these hooks as part of a subroutine that recursively calls any D3-reduction based recognition algorithm (such as our Halin graph recognition algorithm), reverses the recorded sequence of reductions made during the algorithm, and then calls a given pair of triangle and path functions (with the same arguments as the triangle and path hooks) in the order given by the reversed sequence. This can be used to inductively construct structures associated with Halin or D3-reducible graphs.
\item We implemented a method for finding the set of leaf vertices in a decomposition of a Halin graph into a cycle and a tree, using our subroutine for inductive construction together with additional triangle and path subroutines that update this set of leaf vertices through the reversal of any D3 reduction. For a Halin graph that has more than one valid decomposition, one is chosen arbitrarily.
\item We also implemented another method for constructing a Hamiltonian cycle in a D3-reducible graph, again using our subroutine for inductive construction together with additional triangle and path subroutines that update the Hamiltonian cycle through the reversal of any D3 reduction.
\end{itemize}

\subsection{Code size and testing}

Open-source Python code for our implementation is available online at
\url{http://www.ics.uci.edu/~eppstein/PADS/Halin.py}.

In our implementation, not counting comments, whitespace, and sanity checks, the basic D3 reducibility test takes 65 lines of code, and the subroutines to record and reverse a sequence of reductions take 12 lines of code. The additional subroutines for Halin graph recognition take 26 lines of code, the subroutines for finding the leaf vertices of a Halin graph take 15 lines of code, and the subroutines for constructing a Hamiltonian cycle take 28 lines of code. We believe that this code size is substantially smaller than what would be required for a Halin graph recognition algorithm based on general linear-time planarity testing methods.

We checked the correctness of our implementations by unit tests that run them and compare their output with the known correct output for several graphs. Our test cases include examples of Halin graphs, D3-reducible but non-Halin graphs, and non-D3-reducible graphs, on up to 40 vertices. The size of these test graphs was limited by the need to have independent human verification of the correctness of the results rather than by the performance of the algorithms.

Because the implementation of Python that we used is a slow interpreted language, we did not attempt to measure the runtime of our algorithms, as we feel that this measurement would not provide useful information about the efficiency of the same algorithms when implemented in a higher-performance environment.

\section{Conclusions and open problems}

We have developed simple and implementable algorithms for recognizing Halin graphs and for several related problems. These algorithms led us to the definition of a class of graphs, the D3-reducible graphs, that generalize the Halin graphs and share many of their important properties.

It would be of interest to determine more precisely where the D3-reducible graphs fit within the complicated hierarchy of known graph classes. For instance, as well as being a subclass of the polyhedral graphs (which also include the D3-reducible graphs) and the planar partial 3-trees (which don't), the Halin graphs are a subclass of the intersection graphs of rectangles~\cite{ChaFraSur-DM-09}. Is this also true of the  D3-reducible graphs?

\bibliographystyle{abuser}
\bibliography{halin}

\end{document}
