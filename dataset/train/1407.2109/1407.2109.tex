\documentclass[11pt]{article}

\title{\textbf{Planar Graphs: Random Walks and Bipartiteness Testing}\thanks{To appear in \emph{Random Structures and Algorithms}, {\tt http://doi.org/10.1002/rsa.20826}\,, license: CC-BY. A preliminary version appeared in \emph{Proceedings of the 52th IEEE Symposium on Foundations of Computer Science (FOCS)}, pages 423--432, Palm Springs, CA, October 22--25, 2011. IEEE Computer Society Press, Los Alamitos, CA, 2011.}}
\author{Artur Czumaj\thanks{Department of Computer Science, Centre for Discrete Mathematics and its Applications, University of Warwick. Email: A.Czumaj@warwick.ac.uk. Research supported in part by the \emph{Centre for Discrete Mathematics and its Applications (DIMAP)}, EPSRC award EP/D063191/1, by EPSRC award EP/G064679/1, and by a Weizmann-UK Making Connections Grant ``The Interplay between Algorithms and Randomness.''}
    \and
		Morteza Monemizadeh\thanks{Amazon, Palo Alto, CA, USA and Computer Science
Institute of Charles University, Faculty of Mathematics and Physics,
Prague, Czech Republic. The work was done when the author was at the Department of
Computer Science, Goethe-Universit{\"a}t Frankfurt, Germany, University of
Maryland, College Park, MD, USA and Computer Science Institute of Charles
University, Faculty of Mathematics and Physics, Prague, Czech Republic.
Email: monemi@iuuk.mff.cuni.cz, mmorteza@amazon.com.
Partially supported by the projects MO 2200/1-1 and 14-10003S of GA \v{C}R.
}
    \and
Krzysztof Onak\thanks{IBM T.J.\ Watson Research Center. Part of the work was done when the author was at the Massachusetts Institute of Technology and Carnegie Mellon University. Email: konak@us.ibm.com. Supported in part by a Simons Postdoctoral Fellowship and NSF grants 0732334 and 0728645.}
    \and
Christian Sohler\thanks{Department of Computer Science, TU Dortmund. Email: christian.sohler@tu-dortmund.de. Supported in part by the German Research Foundation (DFG), grant So 514/3-2 and ERC grant No.\ 307696.}
}

\date{August 2018}



\usepackage[T1]{fontenc}\usepackage{yfonts}
\usepackage{amsmath,amssymb,amsfonts}
\usepackage{graphicx}
\usepackage{subfigure}
\usepackage{latexsym}
\usepackage{graphicx}
\usepackage{times}
\usepackage{fancybox}
\usepackage{fullpage}
\usepackage{color}
\usepackage{paralist}
\usepackage{environ}
\usepackage{xstring}

\usepackage[T1]{fontenc}
\usepackage{yfonts}

\usepackage{ifpdf}
\newcommand{\mydriver}{hypertex}
\ifpdf
 \renewcommand{\mydriver}{pdftex}
\fi
\usepackage[breaklinks,\mydriver]{hyperref}

\DeclareMathOperator{\poly}{poly}

\newenvironment{proof}{\noindent {\bf Proof}.\ }{\qed \par\vskip 4mm\par}
\newenvironment{proofof}[1]{\noindent {\bf Proof of #1}.\ }{\qed \par\vskip 4mm\par}

\newtheorem{theorem}{Theorem}
\newtheorem{corollary}[theorem]{Corollary}
\newtheorem{lemma}[theorem]{Lemma}
\newtheorem{proposition}[theorem]{Proposition}
\newtheorem{defn}[theorem]{Definition}
\newenvironment{definition}[1][]{\IfStrEq{#1}{}{\begin{defn}}{\begin{defn}[#1]}\rm}{\end{defn}}
\newtheorem{remark}[theorem]{Remark}
\newtheorem{claim}[theorem]{Claim}
\newtheorem{invariant}[theorem]{Invariant}
\newtheorem{observation}[theorem]{Observation}
\newtheorem{example}[theorem]{Example}
\newtheorem{fact}[theorem]{Fact}

\newcommand{\sq}{\hbox{\rlap{}}}
\newcommand{\qed}{\hspace*{\fill}\sq}
\renewcommand{\qed}{\hspace*{\fill}\ensuremath{\blacksquare}}
\newcommand{\COMMENTED}[1]{{}}
\newcommand{\junk}[1]{\COMMENTED{#1}}
\newcommand{\REAL}{\ensuremath{\mathbb{R}}}
\newcommand{\NAT}{\ensuremath{\mathbb{N}}}
\newcommand{\NATURAL}{\NAT}
\renewcommand{\Pr}[1]{\ensuremath{\mathbf{Pr}[#1]}}
\newcommand{\PPr}[1]{\ensuremath{\mathbf{Pr}\big[#1\big]}}
\newcommand{\Ex}[1]{\ensuremath{\mathbf{E}[#1]}}
\newcommand{\EEx}[1]{\ensuremath{\mathbf{E}\big[#1\big]}}
\newcommand{\EExover}[2]{\ensuremath{\mathbf{E_{#2}}\big[#1\big]}}
\newcommand{\eps}{\ensuremath{\epsilon}}
\def\epsilon{\ensuremath{\varepsilon}}



\def\CC{\mathcal{C}}
\def\cc{\mathfrak{c}}

\renewcommand{\O}{\textcolor{red}{AAAA}}



\newlength{\savedparindent}
\newcommand{\SaveIndent}{\setlength{\savedparindent}{\parindent}}
\newcommand{\RestoreIndent}{\setlength{\parindent}{\savedparindent}}

\setlength{\fboxrule}{.25mm}

\newcommand{\InGray}[1]{\SaveIndent{} \noindent{} \fcolorbox[rgb]{0,0,0}{0.95,0.95,0.95}{
\begin{minipage}{0.965\linewidth} \RestoreIndent{}#1
\end{minipage}
} }

\newcommand{\InGrayMiddle}[1]{\SaveIndent{} \centerline{ \fcolorbox[rgb]{0,0,0}{0.95,0.95,0.95}{
\begin{minipage}{0.8\linewidth} \RestoreIndent{}#1
\end{minipage}
} } }

\newcommand{\InGrayNarrow}[1]{\SaveIndent{} \centerline{ \fcolorbox[rgb]{0,0,0}{0.95,0.95,0.95}{
\begin{minipage}{0.65\linewidth} \RestoreIndent{}#1
\end{minipage}
} } }



\newcommand{\RBE}{{\bf Random-Bipartiteness-Exploration}}
\newcommand{\RW}{{\bf Random-Walk}}
\newcommand{\CLF}{{\bf Cycle-Level-Filtering}}
\newcommand{\AL}{\textbf{Assigning-Levels}}
\newcommand{\CS}{\textbf{Cycle-Shortening}}

\newcommand{\leven}{\mathbf{0}}
\newcommand{\lodd}{\mathbf{1}}

\newcommand{\mG}{\mathcal{G}}
\newcommand{\pG}{\mathbb{G}}
\newcommand{\pC}{\mathbb{C}}

\newcommand{\pCt}{\pC^\bigtriangleup}

\NewEnviron{algo}{\smallskip\InGray{\mbox{}\\\BODY\mbox{}\
    \sum_{j} |B(j)| \le (k-1) \cdot |C|.

    a =
    \sum_{j \in R} |A(j)| <
    \sum_{j \in R} \frac{1}{2k} |B(j)| \le
    \frac{k-1}{2k} \cdot |C| <
    \frac{1}{2} \cdot |C|.

    \frac{1}{2k} b =
    \frac{1}{2k} \sum_{j\in R'} |B(j)| \le
    \sum_{j\in R'} |A(j)| \le
    c,

which implies the claim.
\end{proof}
\end{quote}

This finishes the proof of Lemma \ref{lemma:small-vertices}, which follows from Claims \ref{claim:a} and \ref{claim:b} as argued above.
\end{proof}

Our next lemma shows that if there are no self-loops, then we can cover a large number of cycles by well-contractible vertices.

\begin{lemma}
\label{lemma:contractible-vertices}
Let  be a set of edge disjoint cycles on a vertex set  of length at most . Let  be a partition that is good for  such that  contains no self-loops. Let  be the set of vertices in  that have at most 6 distinct neighbors in . If every cycle in  contains at least one vertex from , then there is a subset , , such that every cycle  has a vertex  that is a well-contractible vertex in .\footnote{We slightly abuse notation here as  may be no longer a good partition due to vertex removal. In this case, we can still define  the same way as above.}
\end{lemma}

\begin{proof}
For each vertex , we select  and  independently uniformly at random with among its neighbors in . If , we delete from  all cycles  that contain  and for which  contains any vertex other than  or .
If , we also select independently uniformly at random parities . In this case, we remove from  every cycle  that contains  unless  contains both the edge  of parity  and the edge  of parity . Let  be the set of remaining cycles. We observe that every cycle  in  contains a vertex  that is well-contractible (in fact, every vertex from  contained in  is well-contractible). The probability that a fixed cycle  is not deleted is at least . Thus, the expected size of  is at least , and therefore, there exists a set  of that size that satisfies the lemma.
\end{proof}

We now proceed to the main technical lemma that prepares our reduction step. Given a set  of cycles and a partition , we show that there is a set  and a subset  of  of size  such that we can simultaneously contract edges at all vertices from  to shorten all cycles in . Furthermore, the degree of each non-isolated vertex in the multigraph  is comparable to its degree in .

\begin{lemma}
\label{lemma:MainStep}
Let . Let  be a set of edge-disjoint cycles on vertex set  of length at most  such that  is planar and let  be a partition that is good for . There exists a set  of size , a set of vertices ,
and a partition  that is good for , such that
\begin{itemize}
\item  is obtained from  by setting for every vertex ,
    \deg_{G(C')}(u) > 0
\item  is an independent set in ,
\item every vertex in  is well-contractible in ,
\item every cycle in  that is not a self-loop contains a vertex from , and
\item every non-isolated vertex  from  satisfies .
\end{itemize}
\end{lemma}

\begin{proof}
Let  be the subset of cycles from  that are self-loops in  and let . If , then we choose ,  and  as in the lemma to satisfy the properties of Lemma~\ref{lemma:MainStep}.

It remains to consider the case that , in which case we select . We first set  and then iteratively modify  such that it remains a good partition for the current set of cycles and can be obtained as the statement of the lemma specifies.

We modify  in the way described earlier in the case of deletions of cycles. First, we set  for every vertex  with degree  in . Then we apply Lemma \ref{lemma:small-vertices} with  and  to obtain a set of cycles  of size , such that every cycle from  contains at least one vertex with at most 6 distinct neighbors. In order to maintain the first property in the lemma statement, we modify  by setting  for new isolated vertices in . Note that after the modification every cycle from  still contains at least one vertex with at most 6 distinct neighbors.

Next, we apply Lemma \ref{lemma:contractible-vertices} to  and the current . We obtain sets  and .  is the set of vertices that have at most  distinct neighbors in .  and every cycle in  contains a vertex from  that is well-contractible. Then we apply Lemma \ref{lemma:transformation} to  to obtain the final set of cycles . We modify  as before, by setting  for new isolated vertices in . Clearly,  and  satisfy the first and the fifth property specified in the lemma.

We set  to be any maximal independent subset of vertices in  that are well-contractible in . This ensures both the second and third property. It remains to show that the fourth property is satisfied. By the construction of , every cycle  contains a vertex  that is well-contractible in . If , we are done. Otherwise, since  is maximal, we know that a neighbor of  in  is in . By the definition of well-contractible vertices, this neighbor is also in , which completes the proof of the lemma.
\end{proof}



\subsection{Finalizing the proof of Lemma \ref{lemma:second_reduction}}

We are ready to complete the proof of Lemma \ref{lemma:second_reduction}.
For a planar graph , let  be a set of  odd-length cycles in  and let  be a partition that is good for  such that all cycles in  have length at most . We first apply Lemma \ref{lemma:MainStep} to obtain the sets  and  and the partition . We set .

Now we consider all vertices in  and contract each of them into one of its neighbors. This construction is well-defined since  is an independent set by the second property of Lemma \ref{lemma:MainStep}. For each vertex , we select one of its neighbors in  and call it . We want to contract all edges . A refinement  of  is created as follows. For each vertex  and for each vertex , we set . For all other vertices , we set . One can easily verify that the resulting mapping  is a good partition for .

Our next step is to argue that the length of the cycles in  is at most . We first note that the fourth property in the statement of Lemma \ref{lemma:MainStep} ensures that every cycle  that is not a self-loop contains a vertex . This vertex is well-contractible in ---due to the third property---and our construction above contracts an edge in  incident to this vertex.

Therefore, to complete the proof, we only need to argue that if \RW finds an odd-parity cycle with probability , then the probability that \RW finds an odd-parity cycle is also .

Each edge  in  belongs to a cycle  in . By our construction, a given edge  was either already present in , or is a result of a contraction and there is a vertex  such that  and  are edges of . Note that in either case, the parity of the corresponding path from  to  is maintained. The probability to move from  to  via edge  in  is . If  was already present in , then the probability to move from  to  via edge  in  is . The case that  corresponds to  and  in  is more complicated since parallel edges become relevant. Since  is a well-contractible vertex in , we know that all cycles in  that pass through  also go through  and , and all contain copies of the edges  and . Let us fix a copy of  and call it . In this case, the probability to move in  from  to  through edge  in two steps is
,
since the probability to move along  is  and the probability to take a copy of  is .
If we combine these arguments with the fifth property specified in Lemma \ref{lemma:MainStep}, we conclude that the probability to move from  to  along edge  in  differs from the probability to do the corresponding move in  by at most a factor of . We further conclude inductively that if a random walk in  moves in  steps from vertex  to vertex  with probability , then the same movement happens in  with probability . Furthermore, by definition of well-contractible vertices, if the walk in  contains a cycle of odd parity, then so does the corresponding walk in .

It remains to address the probability of choosing  as the starting vertex. Since we may contract many vertices into  during our construction, the probability of choosing  as a starting vertex in  can be significantly greater than the probability of choosing  in . To this end, recall that the probability to choose  as a starting vertex is . We note that with probability  we sample either  or a vertex  that is contracted into , i.e., a vertex  with . By the definition of well-contractible vertices, the probability to move from  to  in the first step of the random walk is at least . Summing up over all starting vertices (including ), we obtain that we end up at  (either in step 1 or 2 of the random walk) with probability at least . Thus, for every walk that happens in  with probability  there is a (set of) walks in  such that one of them happens with probability . Furthermore, if the walk in  contains an odd-parity cycle then so does the walk in . For different walks in , the sets of corresponding walks in  are disjoint. Thus, we do not double count and our result follows by observing that the walk in  has length at most .
\qed

\junk{
\section{Proof of Lemma \ref{lemma:ExistenceOfH}}
Let  be the final set of cycles  obtained by from our reduction.
We claim that the lemma \ref{lemma:ExistenceOfH} is satisfied by graph . This can be seen by starting the reduction with  and observing
that Lemmas \ref{lemma:small-vertices}, \ref{lemma:contractible-vertices}, and \ref{lemma:MainStep} are satisfied without modifications of the set of cycles . Thus, during the reduction
the set  does not change and at the end establishes the properties of Lemma \ref{lemma:ExistenceOfH} via Lemmas \ref{lemma:second_reduction} and \ref{lemma:Selfloops}.
}


\section{Extending the analysis to minor-free graphs}
\label{sec:minor-free}

While throughout the paper we focused on testing bipartiteness of planar graphs, our techniques can easily be extended to any class of minor-free graphs. Recall that a graph  is called a \emph{minor} of a graph  if  can be obtained from  via  a sequence of vertex and edge deletions and edge contractions. For any graph , a graph  is called \emph{-minor-free} if  is not a minor of . (For example, by Kuratowski's Theorem, a graph is planar if and only if it is -minor-free and -minor-free.)

Let us fix a graph  and consider the input graph  to be an -minor-free graph. We now argue now that entire analysis presented in the previous sections easily extends to testing bipartiteness of . The key observation is that our analysis in Sections \ref{sec:preliminaries}--\ref{sec:proof-of-lemma:second_reduction} relies only on the following properties of planar graphs:
\begin{enumerate}[\it (i)]
\item the number of edges in a planar graph is , where  is the number of vertices (Fact \ref{fact:planar_limited_edges}),
\item every minor of a planar graph is planar (Fact \ref{fact:planar_minor_planar}),
\item a direct implication of the Klein-Plotkin-Rao theorem for planar graphs (Lemma \ref{lemma:partition-into-conn-comp-of-small-diam}).
\end{enumerate}
The first two properties hold for any class of -minor-free-graphs (that is, the second property would be that every minor of an -minor-free-graph is -minor-free). Since the Klein-Plotkin-Rao theorem holds for any minor-free graph as well (cf.~\cite{KPR93}), so does a version of Lemma \ref{lemma:partition-into-conn-comp-of-small-diam} with a slightly different constant hidden by the big  notation. Therefore, we can proceed with nearly identical  analysis for -minor-free graphs and arrive at the following version of Theorem \ref{thm:main-bipartiteness}.

\begin{theorem}
\label{thm:main-bipartiteness-minor-free}
Let  be a fixed graph. There are positive functions  and  such that for any -minor-free-graph :
\begin{itemize}
\item if  is bipartite, then \RBE accepts , and
\item if  is -far from bipartite, then \RBE rejects  with probability at least .
\end{itemize}
\end{theorem}



\section{Conclusions}

In this paper we proved that bipartiteness is testable in constant time for arbitrary planar graphs. Our result was proven via a new type of analysis of random walks in planar graphs. Our analysis easily carries over to classes of graphs defined by arbitrary fixed forbidden minors.

This is merely the first step that poses the following main question:
\begin{center}
What graph properties can be tested in constant time in minor-free graphs?
\end{center}

Going through the analysis of the paper we obtain a running time of . While we did not try to optimize it, it seems that our technique
requires at least an exponential running time. An interesting open question is whether one can get polynomial or pseudopolynomial running time in .
This seems to require significantly new techniques. In particular, to the best of our knowledge, it is not known how to obtain polynomial running time even for bounded-degree planar graphs.






\newcommand{\Proc}{Proceedings of the~}
\newcommand{\ALENEX}{Workshop on Algorithm Engineering and Experiments (ALENEX)}
\newcommand{\BEATCS}{Bulletin of the European Association for Theoretical Computer Science (BEATCS)}
\newcommand{\CCCG}{Canadian Conference on Computational Geometry (CCCG)}
\newcommand{\CIAC}{Italian Conference on Algorithms and Complexity (CIAC)}
\newcommand{\COCOON}{Annual International Computing Combinatorics Conference (COCOON)}
\newcommand{\COLT}{Annual Conference on Learning Theory (COLT)}
\newcommand{\COMPGEOM}{Annual ACM Symposium on Computational Geometry}
\newcommand{\DCGEOM}{Discrete \& Computational Geometry}
\newcommand{\DISC}{International Symposium on Distributed Computing (DISC)}
\newcommand{\ECCC}{Electronic Colloquium on Computational Complexity (ECCC)}
\newcommand{\ESA}{Annual European Symposium on Algorithms (ESA)}
\newcommand{\FOCS}{IEEE Symposium on Foundations of Computer Science (FOCS)}
\newcommand{\FSTTCS}{Foundations of Software Technology and Theoretical Computer Science (FSTTCS)}
\newcommand{\ICALP}{Annual International Colloquium on Automata, Languages and Programming (ICALP)}
\newcommand{\ICCCN}{IEEE International Conference on Computer Communications and Networks (ICCCN)}
\newcommand{\ICDCS}{International Conference on Distributed Computing Systems (ICDCS)}
\newcommand{\IJCGA}{International Journal of Computational Geometry and Applications}
\newcommand{\INFOCOM}{IEEE INFOCOM}
\newcommand{\IPCO}{International Integer Programming and Combinatorial Optimization Conference (IPCO)}
\newcommand{\ISAAC}{International Symposium on Algorithms and Computation (ISAAC)}
\newcommand{\ISTCS}{Israel Symposium on Theory of Computing and Systems (ISTCS)}
\newcommand{\JACM}{Journal of the ACM}
\newcommand{\LNCS}{Lecture Notes in Computer Science}
\newcommand{\PODS}{ACM SIGMOD Symposium on Principles of Database Systems (PODS)}
\newcommand{\RANDOM}{International Workshop on Randomization and Approximation Techniques in Computer Science (RANDOM)}
\newcommand{\RSA}{Random Structures and Algorithms}
\newcommand{\SICOMP}{SIAM Journal on Computing}
\newcommand{\SODA}{Annual ACM-SIAM Symposium on Discrete Algorithms (SODA)}
\newcommand{\SPAA}{Annual ACM Symposium on Parallel Algorithms and Architectures (SPAA)}
\newcommand{\STACS}{Annual Symposium on Theoretical Aspects of Computer Science (STACS)}
\newcommand{\STOC}{Annual ACM Symposium on Theory of Computing (STOC)}
\newcommand{\SWAT}{Scandinavian Workshop on Algorithm Theory (SWAT)}
\newcommand{\TALG}{ACM Transactions on Algorithms}
\newcommand{\UAI}{Conference on Uncertainty in Artificial Intelligence (UAI)}
\newcommand{\WADS}{Workshop on Algorithms and Data Structures (WADS)}
\newcommand{\TCS}{Theory of Computing Systems}

\renewcommand{\Proc}{{\rm In} Proceedings of the~}



\begin{thebibliography}{1}

\bibitem{AFNS06}
N.~Alon, E.~Fischer, I.~Newman, and A.~Shapira.
\newblock A combinatorial characterization of the testable
  graph properties: it's all about regularity.
\newblock \emph{\SICOMP}, 39:143--167, 2009.

\bibitem{AK02}
N.~Alon and M.~Krivelevich.
\newblock Testing -colorability.
\newblock \emph{SIAM Journal on Discrete Mathematics}, 15(2):211-227, 2002.

\bibitem{BKN16}
J.~Babu, A.~Khoury, and I.~Newman.
\newblock Every Property of Outerplanar Graphs is Testable.
\newblock {\Proc APPROX-RANDOM}, pp. 21:1-21:19, 2016.

\bibitem{BSS08}
I.~Benjamini, O.~Schramm, and A.~Shapira.
\newblock Every minor-closed property of sparse graphs is
  testable.
\newblock \emph{Advances in Mathematics}, 223:2200--2218, 2010.

\bibitem{BOT02}
A.~Bogdanov, K.~Obata, and L.~Trevisan.
\newblock A lower bound for testing 3-colorability in
  bounded-degree graphs.
\newblock \emph{\Proc 43rd \FOCS}, pp. 93--102, 2002.

\bibitem{CSS09}
A.~Czumaj, A.~Shapira, and C.~Sohler.
\newblock Testing hereditary properties of nonexpanding
  bounded-degree graphs.
\newblock \emph{\SICOMP}, 38(6): 2499--2510, April 2009.

\bibitem{Gol10}
O.~Goldreich, editor.
\newblock \emph{Property Testing: Current Research and Surveys}.
\newblock Lecture Notes in Computer Science 6390, Springer Verlag,
  Berlin, Heidelberg, December 2010.

\bibitem{GGR98}
O.~Goldreich, S.~Goldwasser, and D.~Ron.
\newblock Property testing and its connection to learning
  and approximation.
\newblock \emph{\JACM}, 45(4): 653--750, July 1998.

\bibitem{GR97}
O.~Goldreich and D.~Ron.
\newblock Property testing in bounded degree graphs.
\newblock \emph{Algorithmica}, 32(2): 302--343, 2002.

\bibitem{GR99}
O.~Goldreich and D.~Ron.
\newblock A sublinear bipartiteness tester for bounded degree  graphs.
\newblock \emph{Combinatorica}, 19(3):335--373, 1999.

\bibitem{GR00}
O.~Goldreich and D.~Ron.
\newblock On testing expansion in bounded-degree graphs.
\newblock \emph{\ECCC}, Report No. 7, 2000.

\bibitem{I16}
Hiro Ito.
\newblock Every Property Is Testable on a Natural Class of Scale-Free Multigraphs.
\newblock \emph{\Proc 24th \ESA}, pp. 51:1-51:12, 2016.

\bibitem{HKNO09}
A.~Hassidim, J.~A. Kelner, H.~N. Nguyen, and K.~Onak.
\newblock Local graph partitions for approximation and testing.
\newblock \emph{\Proc 50th \FOCS}, pp. 22--31, 2009.

\bibitem{KKR04}
T.~Kaufman, M.~Krivelevich, and D.~Ron.
\newblock Tight bounds for testing bipartiteness in general graphs.
\newblock \emph{\SICOMP}, 33(6): 1441--1483, September 2004.

\bibitem{KPR93}
P.~Klein, S.~Plotkin, and S.~Rao.
\newblock Excluded minors, network decomposition, and multicommodity flow.
\newblock \emph{\Proc 25th \STOC}, pp. 682--690, 1993.

\bibitem{KY14}
M.~Kusumoto and Y.~Yoshida.
\newblock Testing forest-isomorphism in the adjacency list model.
\newblock \emph{\Proc 41st \ICALP}, pp. 763--774, 2014.

\bibitem{LPS88}
A.~Lubotzky, R.~Phillips, and P.~Sarnak.
\newblock Ramanujan graphs.
\newblock \emph{Combinatorica}, 8(3): 261--277, 1988.

\bibitem{MR06}
S.~Marko and D.~Ron.
\newblock Approximating the distance to properties in bounded-degree
  and general sparse graphs.
\newblock \emph{\TALG}, 5(2), Article No. 22, March 2009.

\bibitem{NS11}
I.~Newman and C.~Sohler.
\newblock Every property of hyperfinite graphs is testable.
\newblock \emph{\SICOMP}, 42(3): 1095-1112, 2013.

\bibitem{NO08}
H.~N. Nguyen and K.~Onak.
\newblock Constant-time approximation algorithms via local improvements.
\newblock \emph{\Proc 49th \FOCS}, pp. 327--336, 2008.

\bibitem{PR02}
M.~Parnas, D.~Ron.
\newblock Testing the diameter of graphs.
\newblock \emph{\RSA}, 20(2): 165-183, 2002.

\end{thebibliography}

\end{document}
