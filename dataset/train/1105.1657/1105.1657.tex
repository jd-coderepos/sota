

\documentclass{fsttcs}

\usepackage{amsmath,amssymb,latexsym,stmaryrd}
\usepackage[algoruled,vlined,english]{algorithm2e}
\usepackage[normalem]{ulem}
\usepackage{setspace}
\usepackage[backgroundcolor=orange!50!white]{todonotes}



\def\@envspa{\hspace{0.3em}}
\def\@sa{\hspace{-0.2em}}
\def\@sb{\hspace{0.5em}}
\def\@sc{\hspace{-0.1em}}
\def\sk{\smallskip}		

\def\mynote#1{{\sf  #1}}

\newcommand{\todopg}[1]{\todo[color=green!40!white]{PG: #1}}
\newcommand{\todoinlinepg}[1]{\todo[inline,color=green!40!white]{PG: #1}}

\makeatletter

\begingroup \catcode `|=0 \catcode `[= 1
\catcode`]=2 \catcode `\{=12 \catcode `\}=12
\catcode`\\=12 |gdef|@xcomment#1\end{comment}[|end[comment]]
|endgroup
\def\@comment{\let\do\@makeother \dospecials\catcode`\^^M=10\def\par{}}
\def\begincomment{\@comment\@xcomment}
\makeatother
\newenvironment{comment}{\begincomment}{}

\newcommand{\told}[1]{\textcolor{red}{\sout{#1}}}
\newcommand{\tnew}[1]{\textcolor{blue}{#1}}

\def\ie{{\it i.e. }}
\def\eg{{\it e.g. }}
\def\set#1{{\left\{ #1 \right\}}}
\def\tuple#1{{\left\langle #1 \right\rangle}}
\def\multi#1{{\llbracket #1 \rrbracket}}
\def\nats{{\mathbb{N}}}
\def\integers{{\mathbb{Z}}}

\newcommand{\multiset}[1]{{\mathbb{M}[ #1 ]}}

\def\mmap{\mathbf{m}}
\def\cmap{\mathbf{c}}
\def\card#1{\lvert {#1} \rvert}
\def\czero{\mathbf{0}}


\newcommand{\denote}[1]{\llbracket {#1} \rrbracket}
\newcommand{\dcl}[1]{\left\downarrow {#1}\right.}
\newcommand{\ucl}[1]{\left\uparrow {#1}\right.}
\newcommand{\fire}[1]{\left[ {#1}\right\rangle}
\def\prod{\mathcal{P}}
\def\cfl{\mathsf{CFL}}
\def\cfg{\mathsf{CFG}}
\def\pn{\mathsf{PN}}
\def\pni{\mathsf{PNI}}
\def\pnl{\mathsf{PNL}}
\def\pnw{\mathsf{PNW}}
\def\pnwl{\mathsf{PNWL}}

\def\addto{\mathit{add\_\mathord{\ast}\_to}}
\def\subto{\mathit{sub\_\mathord{\ast}\_to}}

\def\unidx{\mathit{unidx}}
\def\idx{\mathit{Sidx}}
\def\yield{\ensuremath{\mathcal{Y}}}


\sloppy

\begin{document}
\title{Approximating Petri Net Reachability Along Context-free Traces}
\runningtitle{Approximating Petri Net Reachability Along Context-free Traces}

\author{Mohamed Faouzi Atig\inst{1}, Pierre Ganty\inst{2}}


\institute{1}{Uppsala University, Sweden}
\institute{2}{IMDEA Software Institute, Spain}

\runningauthors{M. F. Atig, P. Ganty}



\begin{abstract}
We investigate the problem asking whether the intersection of a
context-free language ( and a Petri net language ( is empty.
Our contribution to solve this long-standing problem which relates, for
instance, to the reachability analysis of recursive programs over unbounded
data domain, is to identify a class of s called the finite-index
s for which the problem is decidable.  The -index approximation of
a   can be obtained by discarding all the words that cannot be derived
within a budget  on the number of occurrences of non-terminals.  A
finite-index  is thus a  which coincides with its -index
approximation for some . We decide whether the intersection of a finite-index  and a  is empty by reducing it to the reachability
problem of Petri nets with {\em weak} inhibitor arcs, a class of
systems with infinitely many states for which reachability is known to be
decidable.  Conversely, we show that the reachability problem for a Petri net
with weak inhibitor arcs reduces to the emptiness problem of a finite-index
 intersected with a .
\end{abstract}

\section{Introduction}



Automated verification of infinite-state systems, for instance programs with
(recursive) procedures and integer variables, is an important and a highly
challenging problem.  Pushdown automata (or equivalently context-free grammars) have been
proposed as an adequate formalism to model procedural programs. However
pushdown automata require finiteness of the data domain which is typically
obtained by abstracting the program's data, for instance, using the predicate
abstraction techniques \cite{Cousot77-POPL,GS97}.  In many cases, reasoning
over finite abstract domains yields to a too coarse analysis and is therefore
not precise. To palliate this problem, it is natural to model a procedural
program with integer variables as a pushdown automaton manipulating counters.
In general, pushdown automata with counters are Turing powerful 
which implies that basic decision problems are undecidable (this is true even
for the case finite-state automata with counters).  


Therefore one has to look for restrictions on the model which retain sufficient
expressiveness while allowing basic properties like reachability to be
algorithmically verified. One such restriction is to forbid the test of a
counter and a constant for equality.  In fact, forbidding test for equality
implies the decidability of the reachability problem  for the case of finite-state
automata with counters (i.e. Petri nets \cite{LEROUX-POPL2011,Reinhardt}). 

The verification problem  for pushdown automata with (restricted) counters
boils down to check whether a context-free language ( and a Petri net
language ( are disjoint or not. We denote this last problem
. 
 





The decidability of  is open and lies at the very edge of our
comprehension of infinite-state systems.  We see two breakthroughs contributing
to this question.  First, determining the emptiness of a  was known to
be decidable as early as the eighties. Then, in 2006, Reinhardt
\cite{Reinhardt} lifted this result to an extension of  with inhibitor
arcs (that allow to test if a counter equals 0) {which must satisfy some
additional topological conditions. By imposing a topology on the tests for
zero,} Reinhardt prevents his model to acquire Turing powerful capabilities. We
call his model  and the languages thereof .



Our contribution to the decidability of  comes under the form
of a partial answer which is better understood in terms of underapproximation.
In fact, given a   and the language  of a context-free
grammar we replace  by a subset  which is obtained by discarding
from  all the words that cannot be derived within a given budget  on the
number of non-terminal symbols. (In fact, the subset   contains any word of  that can be generated  by a  derivation  that  contains at most  non-terminal symbols at each derivation step.)  We show how to compute  by annotating the
variables of the context-free grammar for  with an allowance. What is
particularly appealing is that the coverage of  increases with the allowance.
{Approximations induced by allowances are non-trivial: every regular or
linear language is captured exactly with an allowance of ,  coincides with 
when the allowance is unbounded, and under commutativity of concatenation
 coincides with  for some allowance
.}

We call finite-index , or  for short, a context-free
language where each of its words can be derived within a given budget.  In this
paper, we prove the decidability of
 by reducing it to the
emptiness problem of . We also prove the converse reduction; showing
those two problems are equivalent. Hence, we offer a whole new perspective on
the emptiness problem for  and .  

To conclude the introduction let us mention the recent result of \cite{RB-mfcs11} which builds on
\cite{LEROUX-POPL2011} to give an alternative proof of Reinhardt's result
( reachability is decidable) for the particular case where one counter
only can be tested for zero.















\section{Preliminaries}\label{sec:prelim}

\subsection{Context-Free Languages}

An \emph{alphabet}  is a finite non-empty set of \emph{symbols}.
A \emph{word}  over an alphabet  is a finite sequence of symbols of
 where the empty sequence is denoted . 
We write  for the set of words over .
Let ,  defines a \emph{language}.  


A \emph{context-free grammar} (  is a tuple
 where  is a finite non-empty set of \emph{variables}
(\emph{non-terminal letters}),  is an alphabet of
\emph{terminal letters}, and  a finite set of \emph{productions} (the
production  may also be denoted by ). For every production , we use  to denote the variable . Observe that the
form of the productions is restricted, but it has been shown in \cite{LL10} that
every  can be transformed, in polynomial time, into an equivalent
grammar of this form.

Given two strings  we define the relation
, if there exists a production  and some words
 such that  and .  We use
 for the reflexive transitive closure of .  
Given , we define the language , or simply 
when  is clear form the context, as . 
A language  is \emph{context-free} ( if there exists a  
 and  such that
.


\subsection{Finite-index Approximation of Context-Free Languages}

Let ,  be a  and
.  A derivation from  given by
 is 
-\emph{index bounded} if for every  at most  symbols of
 are variables.  We denote by  the subset of  such
that for every  there exists a  index bounded derivation
.  We call  the \emph{-index
approximation} of  or more generically we say that  is a
\emph{finite-index approximation} of .\footnote{Finite-index
approximations were first studied in the 60's.} 

Let us now give some known properties of finite-index approximations.
Clearly .  Moreover, let  be
a regular or linear language\footnote{See \cite{HMU06} for definitions.}, then
there exists a  , and a variable  of  such that
.  Also Luker showed in \cite{Luker78} that if  for some , then 
for some .  More recently, \cite{EKL08:icalp,gmm10} showed some
form of completeness for finite-index approximation when commutativity of
concatenation is assumed. It shows that there exists a  such that
 where  denotes the language
obtained by permuting symbols of  for every .  As an
incompleteness result, Salomaa showed in \cite{Salomaa1969} that for the Dyck
language  over 1-pair of parentheses there is no  ,
variable  of  and  such that
. 

Inspired by \cite{egkl11-ipl,EKL10:JACM,EKL08:icalp} let us define the
  which annotates the variables of 
with a positive integer bounding the index of the derivations starting with
that variable. 

\begin{definition}
Let
  be the  context-free grammar defined as follows:  , and   is the smallest  set such that:
\begin{itemize}
\item For every , 
has the productions  and  for every
.
\item For every  with  , 
 for all
.
\end{itemize}
\label{def:cfgbounded}
\end{definition}

What follows is a consequence of several results from different papers by
Esparza \textit{et al}. For the sake of clarity we give a direct proof in the appendix.
{\begin{lemma}
	\label{lem1} Let . We have .
\end{lemma}
}


\subsection{Petri nets with Inhibitor Arcs}

Let  be a finite non-empty set, a \emph{multiset}  over  maps each symbol of  to a natural
number.  Let  be the set of all multiset over .

We sometimes use the following notation for multisets
 for the multiset
 such that ,
, and . The empty multiset is denoted
.  

Given  we define  to be the multiset such that , we also define the natural partial order
 on  as follows:  if{}f there
exists  such that
.  We also define
 as the multiset such that
 provided .



A {\em Petri net} with inhibitor arcs ( for short)
 consists of a finite non-empty set 
of \emph{places}, a finite set  of \emph{transitions} disjoint from , a
tuple  of functions ,  and , and an \emph{initial
marking} .  
A marking  ( of  assigns to each place
  \emph{tokens}.

A transition  is \emph{enabled at} , written , if
 and   for all .  A transition  that is enabled at  can be 
\emph{fired}, yielding a marking  such that . We write this fact as follows: .
We extend enabledness and firing inductively to
finite sequences of transitions as follows. 
Let .
If  we define  if{}f 
; else if  we
have  if{}f .  

From the above definition we find that  is a reachable  marking from
 if and only if there exists  such that .
Given a language  over the transitions of , the \emph{set of
reachable states from  along }, written ,
coincides with .
Incidentally, if  is unspecified then it is assumed to be  and we
simply write  for the set of states reachable from
. For clarity, we shall sometimes write the  in subscript,
e.g. .

A Petri net with {\em weak} inhibitor arcs ( for short)  is a   such that there is an index function  with the property: 

A Petri net ( for short) can be seen as a subclass of Petri nets with {\em weak} inhibitor arcs where  for all transitions . In this case, we shorten  as the pair .


The {\em reachability} problem for a   is the problem of deciding, for a given marking  , whether  holds. It is well known that reachability  for Petri nets with inhibitor arcs is  undecidable \cite{hack76}. However, the following holds:

\begin{theorem}{\cite{Reinhardt}}
	The reachability problem for  is decidable.
	\label{thm:reinhardt}
\end{theorem}

\subsection{The reachability problem for Petri nets along finite-index CFL}

Let us formally define the problem we are interested in.  Given: (1) a Petri net  where ; 
	(2) a   and ; (3) a marking ;
	and (4) a value .

\noindent
\hspace{0pt}\hspace{\stretch{1}}Does 
hold ?\hfill\vspace{0pt}





In what follows, we prove the interreducibility of the reachability
problem for  along finite-index  and the reachability
problem for .
 \section{From  reachability along  to  reachability}\label{sec:reduc_pierre}

In this section, we show that the reachability problem for Petri nets along
finite-index   is decidable. To this aim, let us fix an instance of the problem: a Petri net  where
,  a  ,
, and a natural number   . Moreover, let
 be the 
given by def.~\ref{def:cfgbounded}. 
	
Lemma \ref{lem1} shows that 
if and only if . Then, our decision
procedure, which determines if
, proceeds by reduction to the
reachability problem for  and is divided in two steps.  First, we
reduce the question  to the
existence of a successful execution in the program of Alg.~\ref{alg:traverse}
which, in turn, is reduced to a reachability problem for .
Let us describe Alg.~\ref{alg:traverse}.

\noindent \begin{minipage}[t]{.33\linewidth}
	{\bf Part 1.} Alg.~\ref{alg:traverse} gives the procedure  in which
 	 and  are
global arrays of markings with index ranging from  to  (i.e., for every
, ).  We say that a call 
\emph{successfully returns} if there exists an execution which eventually
rea\-ches line~\ref{return} (i.e., no assert fails) and the 
postcondition  for every  holds. Moreover we say that a
call  is \emph{proper} if  for all . 
Let
, we shall now demonstrate that a proper call
 successfully returns if and only if there exists
 such that .
\end{minipage}\hspace{\stretch{1}}\begin{minipage}[t]{.65\linewidth}
\vspace{-3pt}
\begin{algorithm}[H]
\DontPrintSemicolon
	\LinesNumbered
	\KwIn{A variable  of }
		\Begin{
Let  such that \;\nllabel{ln:pick}
		\Switch{p}{
		\uCase{\tcc*[f]{}}{\nllabel{ln:case0} 
			\; \nllabel{ln:set0}
			\; \nllabel{ln:set1}
		}
		\uCase{}{
			\nllabel{ln:case1}\;\nllabel{transfer}
			\;\nllabel{add}
			\;\nllabel{traverse_c}
			assert  \ for all \;\nllabel{assert1}
			\;\nllabel{traverse_b}
		}
		\Case{}{
		 \nllabel{ln:case2}\;\nllabel{transfer_bis}							 
		 \;\nllabel{add_bis}
		 \;  \nllabel{call-l-1_bis}
		 assert  \  for all \; \nllabel{assert2}
		 \; \nllabel{traverse_c_bis}
		}
		}
		\Return\;\nllabel{return}
	}
 \caption{\label{alg:traverse}}
\end{algorithm}\end{minipage}

\noindent \hspace*{\stretch{1}}
\begin{minipage}[t]{.5\linewidth}
	\vspace{0pt}
\begin{algorithm}[H]
   {\small    \DontPrintSemicolon
\KwIn{}
       \Begin{
         Let  s.t. \;
\eIf{}{\;
				 }(\tcp*[h]{}){\;
				 }
			 }
			 \caption{\label{alg:add}}}
\end{algorithm}\end{minipage}\hspace{\stretch{1}}
\begin{minipage}[t]{.35\linewidth}
	\vspace{0pt}
\begin{algorithm}[H]
     {\small  \DontPrintSemicolon
\KwIn{}
       \Begin{
        Let  s.t. \;
\; 
				\;
			 }
			 \caption{\label{alg:transfer}}}
\end{algorithm}\end{minipage}\hfill\hspace{0pt}



The formal statement is given at Lem.~\ref{lem:alg_equiv}. We give some
intuitions about Alg.~\ref{alg:traverse} first.

\medskip

The control flow of  matches the traversal of a derivation
tree of  such that at each node  goes first to
the subtree which carries the least index. The tree traversal is implemented
through recursive calls in . To see that the traversal goes first in the subtree
of least index, it suffices to look at the ordering of the recursive calls to
 in the code of Alg.~\ref{alg:traverse}, e.g. in case the
of line~\ref{ln:case1},  is called before
.

Reasoning in terms of derivation trees, we have
that the proper call  returns if{}f there exists a derivation
tree  of  with root variable 
such that the sequence of transitions given by the yield of  is enabled
from the marking stored in  and its firing yields the marking
stored in .

Because of the least index first traversal, it turns out that the arrays
 and  provide enough space to manage
all the intermediary results.

Also, we observe that when the procedure 
calls itself with the parameter, say , the call is a \emph{tail
recursive call}. This means that when  returns
then  immediately returns.  It is known from
programming techniques how to implement tail recursive call without consuming
space on the call stack. In the case of Alg.~\ref{alg:traverse}, we can do so
by having a global variable to store the parameter of  and
by replacing tail recursive calls with  statements.  For the
remaining recursive calls (line \ref{traverse_c} and \ref{call-l-1_bis}),
because the index of the callee is one less than the index of the caller, we conclude that a bounded
space consisting of  frames suffices for the call stack.

Those two insights (two arrays with  entries and a
stack with  frames) will be the key to show, in Part 2, that
 can be implemented as a .

\begin{lemma}
	Let , , and
	. Then, the proper call
	 with context 
  and 
successfully returns  if and only if  there exists  such that .\label{lem:alg_equiv}
\end{lemma}
\begin{proof}
{\bf If.}
We prove that if there exists  such that
 then the proper call
 with  and
 successfully returns.


Our proof is done by induction on the length  of the derivation of .
For the case , we necessarily have  for some .
In this case, the proper call  with  and
 executes as follows: 
 is picked and the case of line~\ref{ln:case0} executes successfully
since  holds.
In fact, after the assignment of line \ref{ln:set0} we have . From there, the call to 
can return with  which shows that 
successfully returns.

\medskip For the case , we have  which
necessarily has the form 
or  by def.\ of .
Assume we are in the latter case. Thus there exists  and  such
that  with 
and .
Observe that  and  and so
by induction hypothesis we find that the proper call 
with ,  successfully returns.
And so does, by induction hypothesis, the proper call  with
, .
Therefore let us consider the proper call  with , . We show it successfully returns.

First observe that the call to the procedure  is proper.
Next, at line~\ref{ln:pick}, pick .
Then the call  of line~\ref{transfer_bis} executes
such that  is updated to  and  to .
Next the call to the procedure  of line~\ref{add_bis} executes such that
both  and  are updated to . Recall that .

Finally we showed above that the proper call  successfully returns, the assert that follows too and
finally the proper call . Moreover it is routine to check that 
upon completion of  (and therefore
 we have  for all .

The left case (i.e.  is treated similarly.

\noindent {\bf Only If.}
Here we prove that if the proper call  successfully returns then
there exists  such that .

Our proof is done by induction on the number  of times line~\ref{ln:pick}
is executed during the execution of . 
In every case, line~\ref{ln:pick} is executed at least once.  
For the case , the algorithm necessarily executes the case of line~\ref{ln:case0}.
The definition of  shows that along a successful execution of
, the non deterministic choice of
line~\ref{ln:pick} necessarily returns a production of the form . Therefore, a successful execution must
execute line~\ref{ln:set0} and \ref{ln:set1} and then~\ref{return} after
which the postcondition  for all  holds.  Because the postcondition
holds, we find that  holds
before executing line~\ref{ln:set1}, hence that  before
executing line~\ref{ln:set0}, and finally that  by semantics of transition  and we are done.

For the case , the first non deterministic choice of line~\ref{ln:pick}
necessarily picks  of the form  or . Let us assume 
, hence that the case of
line~\ref{ln:case1} is executed.  Let  and  be respectively
the values of  and 
when  is invoked. Now, let
 be such that 
and such that upon completion of the call to  at
line~\ref{transfer} we have that 
and .  Moreover, let  be the
marking such that  upon completion of the
call to  at line~\ref{add}. Therefore we find that
 is updated to .
Next consider the successful proper call  of line~\ref{traverse_c} with
, .
Observe that because the execution of  yields the calls
 and , we find that the
number of times line~\ref{ln:pick} is executed in  and 
is strictly less than .
Therefore, the induction hypothesis shows that there exists  such that
 and .  Then
comes the successful assert of line~\ref{assert1} followed by the successful
proper call  of line~\ref{traverse_b} with
 and
.
Again by induction hypothesis, there exists  such that
 and .

Next we conclude from the monotonicity property of  that since
 then
, hence
that

and finally that  because .  Finally since
 we
conclude that  and we are
done.

The left case (i.e.  is treated similarly.\qed
\end{proof}
\pagebreak
 

\noindent {\bf Part 2.} In this section, we  show that it is possible to construct a   such
that  the problem asking if the call to
 successfully returns  can be reduced, in
polynomial time, to a reachability problem for .
Incidentally, we show that  is a , hence that the reachability problem for  along
finite-index  is decidable.

To describe  we use a generalization of the net program formalism introduced
by Esparza in \cite{esparza-course} which enrich the instruction set with the test for 0 of a variable.

A \emph{net program} is a finite sequence of \emph{labelled commands} separated by
semicolons.  Basic commands have the following form, where
 are \emph{labels} taken from some arbitrary set,
and  is a variable over the natural numbers, also called a \emph{counter}.

\medskip

\begin{minipage}[t]{.25\textwidth}
	\begin{itemize}
		\item[]  
		\item[]  
		\item[] 
	\end{itemize}
\end{minipage}\hspace{\stretch{1}}\begin{minipage}[t]{.35\textwidth}
	\begin{itemize}
		\item[]  
		\item[] 
		\item[] 
	\end{itemize}
\end{minipage}\hfill \begin{minipage}[t]{.20\textwidth}
	\begin{itemize}
		\item[] 
		\item[] 
	\end{itemize}
\end{minipage}


\medskip

A net program is \emph{syntactically correct} if the labels of commands are
pairwise different, and if the destinations of jumps corresponds to existing
labels. Moreover we require the net program to be decomposable into a main
program that only calls first-level \emph{subroutines}, which in turn only call
second level \emph{subroutines}, etc and the jump commands in a subroutine can
only have commands of the same subroutine as destinations.\footnote{Here we
consider the main program as a zero-level subroutine, i.e. jump commands in the
main program can only have commands of the main program as destinations.} Each
subroutine has a unique \emph{entry command} labelled with a subroutine name,
and a unique \emph{exit command} of the form .
Entry and exit labelled commands are distinct. 


A net program can only be executed once its variables have received initial
values. In this paper we assume that the initial values are always .
The semantics of net programs is that suggested by the syntax. 

The compilation of a syntactically correct net program to a  is
straightforward and omitted due to space constraints. See \cite{esparza-course}
for the compilation.



At Alg.~\ref{alg:np} is the net program that implements
Alg.~\ref{alg:traverse}.  In what follows assume , the set of places of
the underlying Petri net, to be  for .  The
counter variables of the net program are given by  and   which arranges counters into two matrices of dimension .  For clarity, our net programs use some abbreviations whose
semantics is clear from the syntax, e.g.   stands for the sequence
 

Let us now make a few observations of Alg.~\ref{alg:np}:

\noindent \textbullet{} at the top level we have the subroutine  which first sets up
	 and ,
	then simulates the call  and finally checks
	that the postcondition holds (label  before halting (label . 

\noindent \textbullet{} the counter variables  defines
		the parameter of the calls to . For instance, a call
		to  is simulated in the net program by incrementing 
		and then calling subroutine .


\noindent \hspace{\stretch{1}}\begin{minipage}[t]{.44\linewidth}
	\vspace{0pt}
	{\footnotesize
\begin{algorithm}[H]
	\SetKw{goto}{goto}
	\SetKw{gosub}{gosub}
	\SetKw{Kor}{or}
	\SetKw{halt}{halt}
\nlset{main:}\;
      \;
			\;
			\gosub \;
\nlset{}\If{}{\goto \textbf{success}\;}
\nlset{:} \goto  \Kor  \Kor\ \goto \;
[\ldots]\;
\nlset{:}  \;
      \;
      \;
\gosub \;
			\goto \textbf{exit}\;
			[\ldots]\;
\nlset{:}  \;
			\gosub \;
			\gosub \;
      \;
			\gosub \;
			\nlset{}\If{}{\goto \textbf{l1}\;}
			\nlset{l1:}  \;
			\goto \;
			[\ldots]\;
\nlset{:} \;
			\gosub \;
			\gosub \;
			\;
			\gosub \;
			\nlset{}\If{}{\goto \textbf{l2}\;}
			\nlset{l2:}  \;
			\goto \;
			[\ldots]\;
			\nlset{exit:} \Return{}\;
			[\ldots]\;
			\nlset{success:} \halt\;
\caption{ \footnotesize \textbf{main} invoking  with ,
 and subroutines  where  implementing the calls .\label{alg:np}}
\end{algorithm}}
\end{minipage}\hspace{\stretch{1}}\begin{minipage}[t]{.50\linewidth}
	\noindent \textbullet{} the non-deterministic jump at label  simulates
		the selection of a production rule  which will be fired next
		(if enabled else the program fails). 

	\noindent \textbullet{} 
the missing code for the subroutines
		,
		,
		and  can be found in the appendix although it is pretty obvious to infer from
		Alg.~\ref{alg:add} and Alg.~\ref{alg:transfer}. The code for ,
		 and  is also routine to write.

		\noindent \textbullet{}
the program is syntactically correct. First, the levels are assigned to
		subroutines as follows: the level of  is , the
		level of ,
		,
		,
		 and 
		is .  Given that level assignment, it is routine to check that
		subroutines of level  only call subroutines of level .
		Moreover, thanks to the programming techniques that allow to implement the
		tail recursive call as a \textbf{goto} instead of \textbf{gosub} we find
		that the program is synctactically correct.  (If we had used \textbf{gosub}
		everywhere, then the net program would be synctactically incorrect).  Also
		observe that each jump commands does not leave the subroutine inside which it is
		invoked.  

\noindent \textbullet{} the tests for 0 (labels
		 have a particular
		structure matching the level of the subroutines (level  for
		 and  for  and
		. So, after compilation of the net program into a  ,
		if we set a mapping  from the places of
		 to  such that  is mapped to  if  and every other place is mapped to  then we
		find that  is a . Clearly, deciding whether
		Alg.~\ref{alg:np} halts reduces to  reachability.
		Therefore, by Thm.~\ref{thm:reinhardt}, it is decidable whether Alg.~\ref{alg:np} halts.
\end{minipage}\hfill

\begin{lemma}
\label{lem2-part2}
Let , , and
. Then the proper call  with 
, 
successfully returns if{}f Alg.~\ref{alg:np} halts.
\end{lemma}
Hence from Lem.~\ref{lem1}, \ref{lem:alg_equiv} and \ref{lem2-part2}, we conclude the following.
\begin{corollary}
The reachability problem for
 along finite-index   can be reduced to the reachability problem for . 
\end{corollary}
 \section{From  reachability to   reachability along }\label{sec:reduc_faouzi}

In this section, we show that the reachability problem for  can be
reduced to the reachability problem of  along finite-index . To
this aim, let  be a ,  a marking,  and  an
index function such that \eqref{eq:reinhardt} holds.



Let    and .  Because it
simplifies the presentation we will make a few assumptions that yield no loss
of generality.  ( For every , we have , ( ,   ,
( , and ( for every , if  then
 (see \cite{Reinhardt}, Lemma 2.1).  Notice that the Petri net 
can not test if the place  is empty or not.



In the following, we show that it is possible to construct a Petri net (without
inhibitor arcs)  , a marking
, and a finite-index   such that:

if{}f
.


\smallskip \noindent {\bf Constructing  the Petri net :} 
Let  be a  which consists
in  unconnected  widget:
the widget  given by  without tests for zero (i.e.  is set to  for every  and the widgets  where each
 where
 and
. 
{ is depicted as follows:
.}
Finally, define  to be 
 for  and  elsewhere;
and .

{
Since we have the ability to restrict the possible sequences of transitions
that fire in , we can enforce the invariant that the sum of tokens in
 and  stays constant.  To do so it suffices to force that
whenever a token produced in  then a token is consumed from  and
vice versa. Call  the language enforcing that invariant. Then, let
 be a marking such that , observe that by
firing from  a sequence of the form: (  repeated  times, (
any sequence  and (  repeated  times; the marking
 that is reached is such that . This
suggests that to simulate faithfully a transition  of  that does
test  for  we allow the occurrence of the counterpart of  in
 right before ( or right after ( only. In what follows, we build
upon the above idea the language  which, as we we will show, coincides
with the finite-index approximation of some .
}

We need the following notation. Given  a word  and , we define  to be the word obtained from  by erasing
all the symbols that are not in . We extend it to languages as follows:
Let . Then .

\smallskip \noindent {\bf Constructing the language :} For every ,
let  and    be two words over  the alphabet 
such that  and  for all .   Observe that  firing  keeps unchanged the total number of tokens in  for
each . Let
 define .\footnote{Note that if  then
.} Also given  and
, define  as the set .

Define the s  inductively as follows: 
and for  define .  It is routine to check that  (since ) and  (since
. 
{Also,  is a regular
language and therefore there exists a   and a variable 
of  such that .  Now, let us assume that for 
there exists a   and a variable  such that
.  From the definition of  it is routine to
check that there exists a   and a variable  such
that . Finally we find that  can be
captured by the -index approximation of a .
}

\begin{lemma}
\label{sec5.lem7}
Let . If  such
that , then
  for all .
\end{lemma}Let us make a few observations about the transitions of  which were
carrying out 0 test in . In  no transition  such that
 is allowed, that is no test of place  for
 is allowed along any word of . The language 
imposes that the place  can only be tested for  along
. The intuition is that  allows to test
 for  provided all places  and  for 
are empty.

Let us introduce the following notations. Let  and
, we write  for the multiset of 
such that  for all .  We define the following
subsets of places of :  (resp.   is given by
 (resp.  .  The proofs of lemmata that follow
are done by induction and given in the appendix.

\begin{lemma}
\label{sec5.lem8}
Let , , and
 such that \linebreak  and . Then   .
\end{lemma}




\begin{lemma}
\label{sec5.lem9}
Let ,  such that   and  . Then there are  such that , , , and   .
\end{lemma}


\begin{lemma}
\label{sec.th1}
 if and only if .
\end{lemma}
\begin{proof}
({\em }) Assume that . 
Since  and
, the result of Lem.~\ref{sec5.lem9}
shows that there are  such that
, , ,
and . This implies that  since  and
 by definition.

\medskip
\noindent
({\em }) Assume that . 
The definition of  and   shows that

and therefore,  by Lem.~\ref{sec5.lem8}, we find 
that , hence that
 by definition of , and finally that  since .
\qed
\end{proof}



As an immediate consequence of Lemma \ref{sec.th1}, we obtain the following
result:

\begin{corollary}
The reachability problem for  can be reduced to the reachability problem for
 along finite-index . 
\end{corollary}
 

\section{Conclusion}
In this paper, we have defined  the class finite-index context-free languages  (which is an interesting sub-class of  context-free languages). We have shown that  the problem of checking whether the intersection of a finite-index context-free language and a Petri net language is empty is decidable. This result is obtained through a non-trivial reduction to  the reachability problem for Petri nets with weak inhibitor arcs. On the other hand, we have proved that the reachability problem for Petri nets with weak inhibitor arcs  can be reduced to the the emptiness problem of the language obtained from the intersection of a finite-index context-free language and a Petri net language, which implies by \cite{Lipton} that the latter is EXPSPACE-hard.



\begin{spacing}{0.9}
\bibliographystyle{abbrv}
\bibliography{ref}
\end{spacing}



\newpage
\appendix


\section{Missing Net programs}
Alg.~\ref{alg:subto} gives the net program which implements the call .

\begin{algorithm}[H]
\SetKw{goto}{goto}
\SetKw{gosub}{gosub}
\SetKw{Kor}{or}
\SetKw{halt}{halt}
	\nlset{}\goto \textbf{exit} \Kor  \Kor \ldots \Kor  \;
	\nlset{:}\;
	\;
	\goto \textbf{}\;
	[\ldots]\;
	\nlset{:}\;
	\;
	\goto \;
	\nlset{\textbf{exit:}} \Return{}\;
	\caption{\label{alg:subto}}
\end{algorithm}
Alg.~\ref{alg:addto} implements the call .

\begin{algorithm}[H]
\SetKw{goto}{goto}
\SetKw{gosub}{gosub}
\SetKw{Kor}{or}
\SetKw{halt}{halt}
\nlset{}\goto \textbf{exit} \Kor  \Kor \ldots \Kor  \;
	\nlset{:}\;
	\;
	\goto \;
	[\ldots]\;
	\nlset{:}\;
	\;
	\goto \;
	\nlset{\textbf{exit:}} \Return{}\;
	\caption{\label{alg:addto}}
\end{algorithm}
Alg.~\ref{alg:trfromto} implements the call .

\begin{algorithm}[H]
\SetKw{goto}{goto}
\SetKw{gosub}{gosub}
\SetKw{Kor}{or}
\SetKw{halt}{halt}
\nlset{}\goto \textbf{exit} \Kor  \Kor \ldots \Kor  \;
	\nlset{:} \;
	\;
	\goto \;
	[\ldots]\;
	\nlset{:}\;
	\;
	\goto \;
	\nlset{\textbf{exit:}} \Return{}\;
	\caption{\label{alg:trfromto}}
\end{algorithm}


\pagebreak
\section{Missing Proofs}

\subsection{Proof of Lemma~\ref{lem1}}
\begin{proof}
	Let , we shall demonstrate that
	 if{}f there exists a derivation 
	that is  index bounded.
	
\noindent {\bf Only if.}
We have  for some
. The proof is done by induction on .
For the case , we have , hence that  and  by definition of
 and finally that  is  index bounded.
For the case , the definition of  shows that there exists
a derivation of the form (1)  where  or (2)

where  which is treated similarly.  
Assume case (1) holds. Because  where 
we find, by induction hypothesis, that there exists a derivation  that is  index bounded. Also, since  where , 
the induction hypothesis shows that there exists a derivation  that is  index bounded.  Finally,
we conclude from  , that , hence
that there exists a derivation  that is  index bounded and we are done.

\noindent {\bf If.}
Let  for some
 be a  index bounded derivation. 
The proof is done by induction on .
For the case , we conclude from  is  index bounded that
 by definition of , hence that  by definition of  and finally that
.

For the case , there is a  index bounded derivation of the
form  such that one of the following
derivation is  index bounded:  or  where .

Assume the former case holds (the other is handled similarly).  Since the
derivation is  index bounded we find that  is
 index bounded and  is  bounded.  Because
 and  we find, by induction hypothesis, that  and .
Finally,  shows that , 
hence we deduce that , and finally that  holds.\qed
\end{proof}






\subsection{Proof of Lemma~\ref{sec5.lem7}}

\begin{proof} 
The proof is done by induction on .

\noindent {\bf Basis.} . 
Let , that is  for some .
The proof is by induction on . The case  ( is trivially
solved. Let , then  can be decomposed in 
where each  for  and
 is necesarily of the form . 
Finally since the firing of  keeps unchanged the total number of
tokens in  for each  then
so does all  and we are done.

\noindent {\bf Step.} .   
The definition of  shows that  for some . The proof is done by
induction on . The case  ( is trivially solved. For
 we have that  where  or .  If
, then using the above reasoning we find that the the firing
of any  keeps unchanged the total number of tokens in
 for each .  If
 then  for some , .
Since the result holds for every  by induction hypothesis,
we find that it also holds for  by definition of  and
 and because they fire an equal number of times.  Finally we use
the induction hypothesis on  (we can because  and we are done.\qed
\end{proof}


\subsection{Proof of Lemma~\ref{sec5.lem8}}

\begin{proof} The proof is done by induction on . 
\medskip

\noindent
{\bf Basis.} .
 and every transition  occurring in  is such
that , hence the def.\ of  and  show that .

\medskip

\noindent {\bf Step.} .
The definition of  shows that  for some . The proof is done by
induction on . The case  ( is trivially solved. For
 we have that  where  or .  If
 then  for some , .  Let
 such that
.
We conclude from  and
 that .  Next Lem.~\ref{sec5.lem7} shows that .  Hence, the induction
hypothesis on  shows that . Finally the definition of  shows that ,
hence that , and finally that
 since  and
.  
Also from the assumption ,
 and Lem.~\ref{sec5.lem7} we conclude that .

Let us now turn to the case .  Let  such that
, we conclude from ,  and Lem.~\ref{sec5.lem7} that
, hence that
 since ,
 and . 

Finally we use the induction hypothesis on  (we can because
(1)  and (2) we have shown that  in both cases) and we are done.
\qed
\end{proof}



\subsection{Proof of Lemma~\ref{sec5.lem9}}

\begin{proof} The proof is done by induction on .

\medskip

\noindent
{\bf Basis.} . First, let us observe that, since , the predicates
 and 
are vacuously true. Let  where . Then, there is a word  such
that . Let  defined as follows:
, and  for all .
Then, we have  which yields  since there are enough
tokens in the places . Moreover, we have  since no
transition in  has an arc to a place in .

\medskip

\noindent {\bf Step.} .  Since there is  such that , then either case must hold: 

\begin{itemize}
\item {\bf Case 1:} . Then, we can use the induction
  hypothesis to show that there are  and
    such that , ,
  , and
  . 
  Next, Lem.~\ref{sec5.lem7} shows that
  ,
  hence that  since
   and  for
  . 
  Let   where ,
  and let  such that  and  for .
  From the above we find that (i)   for
  , (ii)  (since  and by def.\ of  and
  (iii)  (since 
  we can show that  and we are done.
\item {\bf Case 2:}  for
  some   and  
  (also ).
  To simplify the presentation, we assume  that . (The general case
  can be handled in the same way.)  Then, there are  such that . Since ,  and
  , we have
  . Hence, we can apply the
  first case to the runs   and  , to show there are  such that , 
  , 
  
  for , 
  , and . 
Moreover  shows that
  there exist  and 
   such that .
  Therefore we can pick  such that
  in addition to the above constraints we have
  
  which is possible since  and
   for . Finally the above reasoning
  shows that 
  ,
  ,
  , hence that
   by definition of  and we are done
  since ,  for .\qed
\end{itemize}
\end{proof}






\end{document}
