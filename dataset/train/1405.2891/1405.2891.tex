\documentclass[usletter]{article}
\usepackage[totalwidth=450pt,totalheight=640pt]{geometry}
\usepackage{scrextend}
\newcommand{\longversion}[1]{#1}
\newcommand{\shortversion}[1]{}
\newcommand{\longshort}[2]{\longversion{#1}\shortversion{#2}}



\usepackage[cmex10]{amsmath}
\usepackage{amssymb}
\usepackage{amsthm,todonotes}
\usepackage{graphicx}
\usepackage{microtype}
\usepackage{color}
\hyphenation{op-tical net-works semi-conduc-tor}
\usepackage{float}
\usepackage{enumitem}


\newcommand{\fo}{\mathcal{FO}}
\newcommand{\rela}{\mathbf{A}}
\newcommand{\relb}{\mathbf{B}}
\newcommand{\cc}{\mathbf{C}}
\newcommand{\dd}{\mathbf{D}}
\newcommand{\pp}{\mathbf{P}}
\newcommand{\qq}{\mathbf{Q}}
\newcommand{\suchthat}{\;\ifnum\currentgrouptype=16 \middle\fi|\;}

\newtheorem{proposition}{Proposition}
\newtheorem{theorem}{Theorem}
\newtheorem{lemma}{Lemma}
\newtheorem{claim}{Claim}
\newtheorem{example}{Example}
\newtheorem{observation}{Observation}
\newtheorem{corollary}{Corollary}
\newtheorem{definition}{Definition}


\begin{document}
\shortversion{
\special{papersize=8.5in,11in}
\setlength{\pdfpageheight}{\paperheight}
\setlength{\pdfpagewidth}{\paperwidth}

\conferenceinfo{CSL-LICS 2014}{July 14--18, 2014, Vienna, Austria}
\copyrightyear{2014}
\copyrightdata{978-1-4503-2886-9}
\doi{nnnnnnn.nnnnnnn}









\titlebanner{banner above paper title}        \preprintfooter{short description of paper}   }

\title{Model Checking Existential Logic on Partially Ordered Sets\footnote{This research was supported by ERC Starting Grant (Complex Reason, 239962) 
and FWF Austrian Science Fund (Parameterized Compilation, P26200).}}


\longshort{
\newcommand*\samethanks[1][\value{footnote}]{\footnotemark[#1]}
\author{Simone Bova, Robert Ganian, and Stefan Szeider\\
\small Vienna University of Technology\\bigwedge_{a,a' \in A, a \neq a'}a \neq a' 
\wedge 
\bigwedge_{R \in \sigma} 
\left(
\bigwedge_{\mathbf{a} \in R^{\mathbf{A}}} R\mathbf{a}
\wedge 
\bigwedge_{\mathbf{a} \not\in R^{\mathbf{A}}} \neg R\mathbf{a}
\right)\text{.}(a_{1,1},a_{1,2},\ldots,a_{1,r}),\ldots,(a_{m,1},a_{m,2},\ldots,a_{m,r}) \in R^\mathbf{A}\text{,}(
f(a_{1,1},a_{2,1},\ldots,a_{m,1}),
\ldots,
f(a_{1,r},a_{2,r},\ldots,a_{m,r})
) \in R^\mathbf{A}\text{.}\textup{compil}(\pp,\cc_{i_1},\ldots,\cc_{i_{w'}},\lambda_1,\ldots,\lambda_{w'})\text{,}\textup{compil}(P)=C_{i_1} \times C_{i_2} \times \cdots \times C_{i_{w'}}\text{.}\{ (\mathbf{c},\mathbf{c}') \in L^{\textup{compil}(\mathbf{P})} \suchthat 
\text{} \} 
\text{.}s \colon \textup{compil}(P)^2 \to \textup{compil}(P)\label{eq:semilattice}
s(\mathbf{c},\mathbf{c}')=(d_1,\ldots,d_{w'})\text{.} 
(s(\mathbf{c},\mathbf{c}'), 
s(\mathbf{d},\mathbf{d}')) \in L\text{,}(s(\mathbf{c},\mathbf{c}'), 
s(\mathbf{d},\mathbf{d}')) \in R\text{,}\mathbf{Q}^* \in \textsc{Hom}(\mathbf{P}^*)\text{,}
\mathbf{Q}^*&=\textup{compil}(\mathbf{Q},\dd_{i_1},\ldots,\dd_{i_{w'}},\mu_1,\ldots,\mu_{w'})\text{,}\\
\mathbf{P}^*&=\textup{compil}(\mathbf{P},\cc_{i_1},\ldots,\cc_{i_{w'}},\lambda_1,\ldots,\lambda_{w'})\text{.} 
(
(\textup{min}^{\qq_{i_1}}(d_1,d'_1),\ldots,\textup{min}^{\qq_{i_{w'}}}(d_{w'},d'_{w'})),
\mathbf{d}
) \in R_{j,k}^{\qq^*}\text{,}(
(\textup{min}^{\qq_{i_1}}(d_1,d'_1),\ldots,\textup{min}^{\qq_{i_{w'}}}(d_{w'},d'_{w'})),
\mathbf{d}'
) \in R_{j,k}^{\qq^*}\text{.}h((\textup{min}^{\qq_{i_1}}(d_1,d'_1),\ldots,\textup{min}^{\qq_{i_{w'}}}(d_{w'},d'_{w'})))=\mathbf{c}''\text{.}
( &(\textup{bot}(\dd_{i_1}),\ldots,q,\ldots,\textup{bot}(\dd_{i_{w'}})), \\
  &(\textup{bot}(\dd_{i_1}),\ldots,q',\ldots,\textup{bot}(\dd_{i_{w'}}))) 
\in L^{\qq^*}\text{,} 
(\mathbf{c},\mathbf{c}') \in L^{\pp^*}\text{,}(\ldots,q,\ldots,q',\ldots) \in O_{(j,j')}^{\qq^*}\text{,}\mathbf{c} \in O_{(j,j')}^{\pp^*}\text{,}(\ldots,q,\ldots,q',\ldots) \in I_{\{j,j'\}}^{\qq^*}\text{,}\mathbf{c} \in I_{\{j,j'\}}^{\pp^*}\text{,}h((d_1,\ldots,d_{w'}))=(e(d_1),\ldots,e(d_{w'}))
P_i = &\{ \bot_i,a_i,b_i,c_i,d_i,\top_i \} \cup \{ l_{(i,j)}, u_{(i,j)} \mid j \in [n], (i,j) \in E^{\mathbf{G}} \}\text{.}

P_i = &\{ \bot_i,a_i,b_i,c_i,d_i,\top_i \} \\
    & \cup \{ l_{(i,j)}, u_{(i,j)} \mid j \in [n], (i,j) \in E^{\mathbf{G}} \}\text{.}

Q_{\phi}^v &=\left\{ (x_i,(j,j')) \suchthat 
\begin{array}{c}
i \in [n], x_i \in \delta_j, x_i \in \delta_{j'},j<j'\textup{,}\\
\textup{and } x_i \not\in \delta_{j''} \textup{ for all } j<j''<j'
\end{array} \right\}\text{,}

P_{\phi}^a &= \bigcup_{(\delta_i,j) \in Q_{\phi}^a} \{ (f,(\delta_i,j)) \mid f \in \{0,1\}^{\textup{var}(\delta_i)} \textup{ satisfies } \delta_i \}\text{,}\\ 
P_{\phi}^c &= \bigcup_{(\delta'_i,j) \in Q_{\phi}^c} \{ (f,(\delta'_i,j)) \mid f \in \{0,1\}^{\textup{var}(\delta_i)} \textup{ satisfies } \delta_i \}\text{,}\\
P_{\phi}^v &=\bigcup_{(x_i,(j,j')) \in Q_{\phi}^v} \{ (x_i,(j,j')), (\neg x_i,(j,j'))\}\text{,}

Q_{0} = & \{ c_{(i,j)}, c_{(i,m)}, c_{(m,j)} \mid i, j \in [m-1], i\neq j \}\text{,} \\
Q_{1} = & \{ f_{(i,j)} \mid i,j \in [m], i\neq j \} \text{,}\\
Q_{2} = & \{ d_{(i,j)} \mid 1 \leq i<j \leq m \} \text{,}

f_{(i,i_1)} & \succ^{\qq_\phi} c_{(i,i_1)} \prec^{\qq_\phi} f_{(i,i_2)} \succ^{\qq_\phi} \cdots \succ^{\qq_\phi} c_{(i,i_{m-1})} \prec^{\qq_\phi} f_{(i,i_{m-1})}\text{.}

f_{(i,i_1)} & \succ^{\qq_\phi} c_{(i,i_1)} \prec^{\qq_\phi} f_{(i,i_2)} \succ^{\qq_\phi} \cdots \\
\cdots & \succ^{\qq_\phi} c_{(i,i_{m-1})} \prec^{\qq_\phi} f_{(i,i_{m-1})}\text{.}

P_0 = & \{ c_{(i,j),a},c_{(i,m),a},c_{(m,j),a} \mid i,j \in [m-1], i\neq j, a \in [7] \}\text{,}\\
P_1 = & \{ f_{(i,j),a} \mid i,j\in [m], i\neq j, a \in [7] \}\text{,}\\
P_2 = & \{ d_{(i,j),(a,a')} \mid 1 \leq i<j \leq m, (a,a') \in [7]^2 \}\text{,}

f_{(i,i_{1}),a} & \succ^{\pp_\phi} c_{(i,i_1),a} \prec^{\pp_\phi} f_{(i,i_2),a} \succ^{\pp_\phi} \cdots \succ^{\pp_\phi} c_{(i,i_{m-1})} \prec^{\pp_\phi} f_{(i,i_{m-1}),a}\text{.}

f_{(i,i_{1}),a} & \succ^{\pp_\phi} c_{(i,i_1),a} \prec^{\pp_\phi} f_{(i,i_2),a} \succ^{\pp_\phi} \cdots\\ 
\cdots & \succ^{\pp_\phi} c_{(i,i_{m-1})} \prec^{\pp_\phi} f_{(i,i_{m-1}),a}\text{.}
}
\end{itemize}
Since  and 
,  
has bounded degree by Proposition~\ref{pr:diagram}.

\longshort{\begin{theorem}}{\begin{theorem}[]}
\label{th:degreenphard} 
 is -hard. 
\end{theorem}

\newcommand{\pfdegreenphard}[0]{
\begin{proof}
We give a polynomial-time many-one reduction from the satisfiability problem over 
 to the problem , 
which suffices since the source problem is -hard.  

The reduction maps an instance  of the satisfiability problem, 
say ,  
to the instance  of .  
The reduction is clearly polynomial-time computable.  

For correctness, let  be an assignment satisfying .  
Recall that  is a fixed ordering of the assignments in  satisfying , 
for all .  Let  be such that 
 for all .  It is easy to check 
that the function  defined by setting:
\begin{itemize}
\item  for all ;
\item  for all ;
\item  for all ;
\end{itemize}
embeds  into . 

Conversely, let  embed  into .  
We show that  is satisfiable.  Note that  for all , 
because  maps all -element chains in  into -element chains in , 
all -element chains in  link three elements in , , and , in this order, 
and all -element chains in  link three elements in , , and , in this order.  

We first claim that for all , 
there exists exactly one  such that, 
for all , 
it holds that .  Assume for 
a contradiction that  and  
for some , , and ; 
without loss of generality, let .  By (E2), 
 reaches  through a fence of length 
, starting in  and alternating steps in  and ; but by (D2), 
 does not reach  through a fence of length , 
starting in  and alternating steps in  and , contradicting the assumption that  is an embedding.

Let  be uniquely determined by the previous 
claim.  We now claim that, for all  such that , 
and all , 
it holds that .  Assume without loss of generality that .  
By (E1), .  
By hypothesis,  and .  
Therefore, since  is an embedding, ; 
thus, by (D1),  that is, 
 for all .  

By the above,  is a 
function from  to .  Since  
satisfies  for all , it follows that  satisfies , concluding the proof.
\end{proof}}


\longversion{\pfdegreenphard}

\longshort{\subsection{Isomorphism in Polynomial Time on Bounded Width Posets}\label{sect:wdtract}}{\subsection{Isomorphism in Polytime on Bounded Width Posets}\label{sect:wdtract}}

The insight on bounded width used to prove tractability of the embedding problem 
essentially scales to the isomorphism problem.

\begin{theorem}\label{th:isoptime}
Let  be a class of posets of bounded width.  Then, 
 is polynomial-time tractable.
\end{theorem}
\begin{proof}
The proof utilizes three known facts from the literature.  

Let  be any poset.  
For all , let 
 be \emph{downset} generated by  in  , i.e., 
 r \leq^{\mathbf{R}} s.  
Let  be the order defined 
by equipping the universe of all antichains in  
by the relation  if and only if 
.  Note that, if  
is considered as a constant, the construction of  
is polynomial-time computable from . 

The three needed facts are the following.  First, for any (finite) poset , 
the structure  is a (finite) distributive lattice \cite[Proposition~5.5.5]{Schroder03}.  
Second, the substructure of  
generated by join irreducible elements is isomorphic to  \cite[Theorem~5.5.6]{Schroder03}; 
recall that, if  is a lattice, 
then  is \emph{join irreducible} if, for all , 
if  is the least upper bound of  and , then  or .   
Third, the isomorphism problem restricted to finite distributive lattices 
is polynomial-time tractable \cite{GorazdIdziak95}.  

Using the previous facts, we design the following algorithm.
Let  be the upper bound on the width of posets in .  
Let  be an instance of .  Let .  
If , or , 
or , then reject; 
the condition is checkable in time  by Theorem~\ref{th:felsner}.  
Otherwise, in polynomial time, 
compute  and  
and accept if and only if  and  are isomorphic.  

The algorithm clearly runs in polynomial time.  For correctness, 
notice that  and  are isomorphic if and only if 
 and  are isomorphic.  For the nontrivial direction (backwards), 
if  is an isomorphism from  to , 
then let  be the restriction of  to the join irreducible elements 
of .  It is easy to check that  is bijective into 
the join irreducible elements of , hence, 
using the second fact mentioned above, 
 is an isomorphism between  and .
\end{proof}





\section{Conclusion}\label{sect:concl}

We embarked on the study of the model checking problem 
on 
posets; compared to graphs, 
the problem is largely unexplored, and we made a first contribution 
by studying basic syntactic fragments (existential logic) 
and fundamental poset invariants (including width, depth, and degree).  
Our complexity classification for existential logic also carries over to the \emph{jump number} 
(between size and width in Figure~\ref{fig:overwparcompl}); a future direction is to extend our study to \emph{dimension} 
(above width \cite{CaspardLeclercMonjardet12} and degree \cite{FurediKahn86} in Figure~\ref{fig:overwparcompl}).


Our main algorithmic result, fixed-parameter tractability of existential logic on bounded width posets, 
raises the natural question of whether model checking the full first-order logic 
is fixed-parameter tractable on classes of posets of bounded width. 
We propose this as a topic for future research.  










\shortversion{\acks

This research was supported by ERC Starting Grant (Complex Reason, 239962) 
and FWF Austrian Science Fund (Parameterized Compilation, P26200).}



\bibliographystyle{abbrvnat}



\longversion{\newpage}

\begin{thebibliography}{10}
\shortversion{\softraggedright}

\bibitem{AlonYusterZwick95}
N. Alon, R. Yuster, and U. Zwick.
\newblock Color-coding.
\newblock {\em J. ACM}, 42(4):844--856, 1995.



\bibitem{CaspardLeclercMonjardet12}
N. Caspard, B. Leclerc, and B. Monjardet.
\newblock {\em Finite Ordered Sets}.\newblock Cambridge University Press, 2012.



\bibitem{CourcelleMakowskyRotics00}
B.~Courcelle, J.~A. Makowsky, and U.~Rotics.
\newblock Linear time solvable optimization problems on graphs of bounded
  clique-width.
\newblock {\em Theory Comput. Syst.}, 33(2):125--150, 2000.

\bibitem{CourcelleOlariu00}
B.~Courcelle and S.~Olariu.
\newblock Upper bounds to the clique-width of graphs.
\newblock {\em Discr. Appl. Math.}, 101(1-3):77--114, 2000.

\bibitem{CourcelleEngelfriet12}
B. Courcelle and J. Engelfriet.
\newblock {\em Graph Structure and Monadic Second-Order Logic}.
\newblock Cambridge University Press, 2012.

\bibitem{FederVardi98}
T. Feder and M.~Y. Vardi.
\newblock The computational structure of monotone monadic {SNP} and constraint
  satisfaction: a study through {D}atalog and group theory.
\newblock {\em SIAM J. Comput.}, 28(1):57--104, 1998.

\bibitem{FelsnerRaghavanSpinrad03}
S. Felsner, V. Raghavan, and J. Spinrad.
\newblock Recognition algorithms for orders of small width and graphs of small
  Dilworth number.
\newblock {\em Order}, 20(4):351--364, 2003.

\bibitem{FlumGrohe06}
J. Flum and M. Grohe.
\newblock {\em Parameterized Complexity Theory}.
\newblock Springer, 2006.

\bibitem{FurediKahn86}
Z.~Furedi and J. Kahn.
\newblock On the dimensions of ordered sets of bounded degree.
\newblock {\em Order}, 3:15--20, 1986.

\bibitem{GorazdIdziak95}
T.~A. Gorazd and P.~M. Idziak.
\newblock The isomorphism problem for varieties generated by a two-element
  algebra.
\newblock {\em Algebr.\ Univ.}, 34(3):430--439, 1995.

\bibitem{GrahamGrotschelLovasz95}
R.~L. Graham, M.~Gr\"{o}tschel, and L.~Lov\'{a}sz, editors.
\newblock {\em Handbook of Combinatorics (Vol. 1)}.
\newblock MIT Press, 1995.

\bibitem{Grohe07a}
M. Grohe.
\newblock Logic, graphs, and algorithms.
\newblock In {\em Logic and
  Automata: History and Perspectives}, pp.\ 357--422. Amsterdam University Press, 2007.

\bibitem{GroheKreutzerSiebertz14}
M.~Grohe, S.~Kreutzer, and S.~Siebertz. 
\newblock Deciding First-Order Properties of Nowhere Dense Graphs.  
In {\em STOC}, 2014.  Preprint in {\em CoRR}, abs/1311.3899, 2013.



\bibitem{JeavonsCohenGyssens97}
P. Jeavons, D. Cohen, and M. Gyssens.
\newblock Closure properties of constraints.
\newblock {\em J. of the ACM}, 44(4):527--548, 1997.

\bibitem{NielsonNielsonHankin05}
F. Nielson, H.~R. Nielson, and C. Hankin.
\newblock {\em Principles of program analysis}.
\newblock Springer, 2005.

\bibitem{PrattTiuryn96}
V.~R. Pratt and J. Tiuryn.
\newblock Satisfiability of inequalities in a poset.
\newblock {\em Fund. Inform.}, 28(1-2):165--182, 1996.

\bibitem{RauschReinert10}
T. Rausch and K. Reinert.
\newblock {\em Problem Solving Handbook in Computational Biology and
  Bioinformatics}.
\newblock Springer, 2010.

\bibitem{Schroder03}
B.~Schr{\"o}der.
\newblock {\em Ordered Sets: An Introduction}.
\newblock Birkh{\"a}user, 2003.

\bibitem{Seese96}
D. Seese.
\newblock Linear time computable problems and first-order descriptions.
\newblock {\em Math.\ Struct.\ in Comp.\ Science}, 6(6):505--526,
  1996.

\end{thebibliography}






\end{document}
