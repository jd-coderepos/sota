\documentclass[runningheads]{llncs}
\usepackage{graphicx}

\usepackage{tikz}
\usepackage{comment} 
\usepackage{amsmath,amssymb} \usepackage{color}

\usepackage{bm}
\usepackage{multirow} \usepackage{ctable}   \newcommand*\samethanks[1][\value{footnote}]{\footnotemark[#1]}
\newcolumntype{L}[1]{>{\raggedright\let\newline\\\arraybackslash\hspace{0pt}}m{#1}}
\newcolumntype{C}[1]{>{\centering\let\newline\\\arraybackslash\hspace{0pt}}m{#1}}
\newcolumntype{R}[1]{>{\raggedleft\let\newline\\\arraybackslash\hspace{0pt}}m{#1}}


\begin{document}

\pagestyle{headings}
\mainmatter

\title{SMAP: Single-Shot Multi-Person \\Absolute 3D Pose Estimation} 

\titlerunning{SMAP}

\author{Jianan Zhen\inst{1,2,\star} \and Qi Fang\inst{1,\star} \and Jiaming Sun\inst{1,2} \and Wentao Liu\inst{2} \and \\ Wei Jiang\inst{1} \and Hujun Bao\inst{1} \and Xiaowei Zhou\inst{1}}

\authorrunning{Zhen et al.}

\institute{Zhejiang University \and SenseTime}

\maketitle

\begin{abstract}
Recovering multi-person 3D poses with absolute scales from a single RGB image is a challenging problem due to the inherent depth and scale ambiguity from a single view. Addressing this ambiguity requires to aggregate various cues over the entire image, such as body sizes, scene layouts, and inter-person relationships. However, most previous methods adopt a top-down scheme that first performs 2D pose detection and then regresses the 3D pose and scale for each detected person individually, ignoring global contextual cues.  
In this paper, we propose a novel system that first regresses a set of 2.5D representations of body parts and then reconstructs the 3D absolute poses based on these 2.5D representations with a depth-aware part association algorithm. Such a single-shot bottom-up scheme allows the system to better learn and reason about the inter-person depth relationship, improving both 3D and 2D pose estimation. The experiments demonstrate that the proposed approach achieves the state-of-the-art performance on the CMU Panoptic and MuPoTS-3D datasets and is applicable to in-the-wild videos. {\renewcommand{\thefootnote}{\fnsymbol{footnote}} \footnotetext[1]{Equal contribution. }}

\keywords{Human pose estimation \and 3D from a single image}
\end{abstract}

\section{Introduction}
 accuracy of the 3D root-relative pose. It calculates the Euclidean distance between the predicted and the groundtruth joint locations averaged over all joints.

\paragraph{\bf RtError.} Root Error (RtError) is defined as the Euclidean distance between the predicted and the groundtruth root locations. 

\paragraph{\bf 3DPCK.} 3DPCK is the percentage of correct keypoints. A keypoint is declared correct if the Euclidean distance between predicted and groundtruth coordinates is smaller than a threshold (15cm in our experiments). PCK measures relative pose accuracy with root alignment; PCK measures absolute pose accuracy without root alignment; and PCK only measures the accuracy of root joints. AUC means the area under curve of 3DPCK over various thresholds. 

\paragraph{\bf PCOD.} We propose a new metric named the percentage of correct ordinal depth (PCOD) relations between people. The insight is that predicting absolute depth from a single view is inherently ill-posed, while consistent ordinal relations between people are more meaningful and suffice many applications. For a pair of people , we compare their root depths and divide the ordinal depth relation into three classes: closer, farther, and roughly the same (within 30cm). PCOD equals the classification accuracy of predicted ordinal depth relations. 


\begin{table}[t]
	\centering
	\setlength\tabcolsep{1.0pt}
	\def\arraystretch{1.0}
	\caption{Results on the Panoptic dataset. For \cite{Moon_2019_ICCV_3DMPPE}, we used the code provided by the authors and trained it on the Panoptic dataset. *The average of \cite{zanfir2018deep} is recalculated following the standard practice in \cite{zanfir2018monocular}, i.e., average over activities.}
	\label{table:panoptic3d}
	\begin{tabular}{L{1.4cm}|L{2.85cm}*{5}{C{1.46cm}}}
	\specialrule{.1em}{.05em}{.05em}
    & \multicolumn{1}{c}{Method} & Haggling & Mafia & Ultim. & Pizza & Average  \\\hline
	\multirow{6}{*}{MPJPE} & ~PoPa et al.~\cite{popa2017deep} & 217.9 & 187.3 & 193.6 & 221.3 & 203.4 \\ 
	& ~Zanfir et al.~\cite{zanfir2018monocular} & 140.0 & 165.9 & 150.7 & 156.0 & 153.4 \\
	& ~Moon et al.~\cite{Moon_2019_ICCV_3DMPPE} & 89.6 & 91.3 & 79.6 & 90.1 & 87.6 \\
    & ~Zanfir et al.~\cite{zanfir2018deep} & 72.4 & 78.8 & 66.8 & 94.3 & 78.1*\\
    & ~\textbf{Ours w/o Refine} & 71.8 & 72.5 & 65.9 & 82.1 & 73.1\\
	& ~\textbf{Ours} & \textbf{63.1} & \textbf{60.3} & \textbf{56.6} & \textbf{67.1} & \textbf{61.8} \\
	\hline
	\multirow{3}{*}{RtError} & ~Zanfir et al.~\cite{zanfir2018monocular} & 257.8 & 409.5 & 301.1 & 294.0 & 315.5 \\ 
	& ~Moon et al.~\cite{Moon_2019_ICCV_3DMPPE} & 160.2 & 151.9 & 177.5 & 127.7 & 154.3 \\
	& ~\textbf{Ours} & \textbf{84.7} & \textbf{87.7} & \textbf{91.2} & \textbf{78.5} & \textbf{85.5} \\
	\hline
    \multirow{2}{*}{PCOD} & ~Moon et al.~\cite{Moon_2019_ICCV_3DMPPE} & 92.3 & 93.7 & 95.2 & 94.2 & 93.9 \\
	& ~\textbf{Ours} & \textbf{97.8} & \textbf{98.5} & \textbf{97.6} & \textbf{99.6} & \textbf{98.4} \\ 
	\specialrule{.1em}{.05em}{.05em}
	\end{tabular}
\end{table}
 
\begin{table}[t]
	\centering
	\setlength\tabcolsep{1.0pt}
	\caption{Results on the MuPoTS-3D dataset. All numbers are average values over 20 activities.}
	\label{table:mupots_all}
	\begin{tabular}{C{1.15cm}|L{1.8cm}|*{5}{C{1.2cm}}|C{1.16cm}C{1.16cm}}
	\specialrule{.1em}{.05em}{.05em}
    & & \multicolumn{5}{c|}{Matched people} & \multicolumn{2}{c}{All people} \\\hline
    & \multicolumn{1}{l|}{~Method} & PCK & PCK & PCK & AUC & PCOD & PCK & PCK\\ \hline
    \multirow{4}{*}{\shortstack{top\\down}}
    & ~Rogez.~\cite{rogez2017lcr} & 62.4 & - & - & - & - & 53.8 & -\\
    & ~Rogez.~\cite{rogez2019lcr} & 74.0 & - & - & - & - & 70.6 & -\\
    & ~Dabral.~\cite{dabral2019multi} & 74.2 & - & - & - & - & 71.3 & -\\
    & ~Moon.~\cite{Moon_2019_ICCV_3DMPPE} & \textbf{82.5} & 31.8 & 31.0 & 40.9 & 92.6 & \textbf{81.8} & 31.5 \\
    \hline
    \multirow{3}{*}{\shortstack{bottom\\up}}
    & ~Mehta.~\cite{mehta2018single} & 69.8 & - & - & - & - & 65.0 & - \\
	& ~Mehta.~\cite{mehta2019xnect} & 75.8 & - & - & - & - & 70.4 & - \\
	& ~\textbf{Ours} & \textbf{80.5} & \textbf{38.7} & \textbf{45.5} & \textbf{42.7} & \textbf{97.0} & \textbf{73.5} & \textbf{35.4}\\
	\specialrule{.1em}{.05em}{.05em}
	\end{tabular}
\end{table}

 
\subsection{Comparison with state-of-the-art methods}
\paragraph{\bf CMU Panoptic.} 
Table~\ref{table:panoptic3d} demonstrates quantitative comparison between state-of-the-art methods and our model. It indicates that our model outperforms previous methods in all metrics by a large margin. In particular, the error on the Pizza sequence decreases significantly compared with the previous work. As the Pizza sequence shares no similarity with the training set, this improvement shows our generalization ability. 

\begin{figure}[t!]
	\centering
	\includegraphics[width=0.8\linewidth,trim={0.2cm 5.5cm 0.2cm 6cm},clip]{figures/cmp.pdf}
 	\caption{
 		\textbf{Comparisons of root localization and relative pose}. The curves are PCK and PCK over different thresholds on the MuPoTS-3D dataset. Blue: result of \cite{Moon_2019_ICCV_3DMPPE}. Green: estimating the translation by minimizing the reprojection error. Red: our result.
 	}
 	\label{fig:rootrelPCK}
 \end{figure}

\paragraph{\bf MuPoTS-3D.} 
We follow the protocol of \cite{Moon_2019_ICCV_3DMPPE}. Additionally, PCK and PCOD are used to evaluate the 3D localization of people. In terms of the absolute pose which we are more concerned with, it can be observed from Table~\ref{table:mupots_all} that our model is superior to \cite{Moon_2019_ICCV_3DMPPE} in relevant metrics including PCK, PCK and PCOD by a large margin. It also demonstrates that our model has higher PCK compared with all bottom-up methods and most top-down methods except \cite{Moon_2019_ICCV_3DMPPE}. Note that we achieve higher AUC compared to \cite{Moon_2019_ICCV_3DMPPE} for the relative 3D pose of matched people.

\begin{figure*}[t!]
	\centering
	\includegraphics[width=0.95\linewidth,trim={1.4cm 4cm 1.3cm 7cm},clip]{figures/cases.pdf}
 	\caption{
		\textbf{Qualitative comparison}. The results of three example images. For each example, the top row shows the input image, and the bottom row shows the results of \cite{Moon_2019_ICCV_3DMPPE} (left) and the proposed method (right), respectively. The red circles highlight the difference in localization of human bodies between two methods.
 	}
 	\label{fig:case}
 \end{figure*}

\paragraph{\bf Comparison with top-down methods.} 
We provide additional analysis to compare our single-shot bottom-up method to the state-of-the-art top-down method \cite{Moon_2019_ICCV_3DMPPE}. Fig.~\ref{fig:rootrelPCK} shows thorough comparisons in terms of PCK and PCK. For root localization, we compare to two methods: 1) regressing the scale from each cropped bounding box using a neural network as in \cite{Moon_2019_ICCV_3DMPPE} and 2) estimating the 3D translation by optimizing reprojection error with the groundtruth 2D pose and the estimated relative 3D pose from \cite{Moon_2019_ICCV_3DMPPE} (`FitT' in Fig.~\ref{fig:rootrelPCK}). We achieve better PCK over various thresholds than both of them. Notably, we achieve roughly 100\% accuracy with a threshold 1m. As for relative pose estimation, \cite{Moon_2019_ICCV_3DMPPE} achieves higher PCK (@15cm) as it adopts a separate off-the-shelf network \cite{sun2018integral} that is particularly optimized for relative 3D pose estimation. Despite that, we obtain better PCK when the threshold is smaller and higher AUC. 

Fig.~\ref{fig:case} shows several scenarios (various poses, occlusion, and truncation) in which the top-down method \cite{Moon_2019_ICCV_3DMPPE} may fail as it predicts the scale for each detected person separately and ignore global context. Instead, the proposed bottom-up design is able to leverage features over the entire image instead of only using cropped features in individual bounding boxes.

Furthermore, our running time and memory remain almost unchanged with the number of people in the image while those of \cite{Moon_2019_ICCV_3DMPPE} grow faster with the number of people due to its top-down design, as shown in the supplementary material.

\paragraph{\bf Depth estimation.} Apart from our method, there are two alternatives for depth estimation: 1) regressing the full depth map rather than the root depth map. 2) using the cropped image as the input to the network rather than the whole image. For the first alternative, since there is no depth map annotation in existing multi-person outdoor datasets, we use the released model of the state-of-the-art human depth estimator \cite{Li_2019_CVPR}, which is particularly optimized for human depth estimation trained on a massive amount of in-the-wild `frozen people' videos. For the second alternative, \cite{Moon_2019_ICCV_3DMPPE} is the state-of-the-art method that estimates root depth from the cropped image, so we compare with it.
Fig.~\ref{fig:consistency} demonstrates scatter plots of the groundtruth root depth versus the predicted root depth of three methods on the MuPoTS-3D dataset. Ideally, the estimated depths should be linearly correlated to the ground truth, resulting a straight line in the scatter plot. Our model shows better consistency than baselines. Note that, while the compared methods are trained on different datasets, the images in the test set MuPoTS-3D are very different from the training images for all methods. Though not rigorous, this comparison is still reasonable to indicate the performance of these methods when applied to unseen images. 

\begin{figure}[t]
\centering
\includegraphics[width=0.9\linewidth,trim={0cm 0cm 0cm 0cm},clip]{figures/consistency.pdf}
 \caption{
	 \textbf{Comparison with alternative depth estimation methods}. The scatter plots show the consistency of root depth estimation on the MuPoTS-3D dataset. The X and Y axes are the predicted, groundtruth root depth, respectively. The dashed line means the ideal result, i.e., estimation equals ground truth. (a) `Read-out' root depths from the full depth map estimated by \cite{Li_2019_CVPR}. 
	 (b) State-of-the-art top-down approach \cite{Moon_2019_ICCV_3DMPPE}. 
	 (c) Our approach.
 }
 \label{fig:consistency}
\end{figure}

\subsection{Ablation analysis}



\begin{table}[t]
	\centering
	\setlength\tabcolsep{1.0pt}
	\def\arraystretch{1.0}
	\caption{Ablation study of the structure design on the MuPoTS-3D dataset. The default backbone is Hourglass model with three stages, and `Smaller Backbone' means one-stage model. }
	\label{table:ablation}
	\begin{tabular}{L{4.6cm}|*{5}{C{1.2cm}}}
	\specialrule{.1em}{.05em}{.05em}
    Design & Recall & PCK & PCK & PCK & PCOD\\\hline
    Full Model & \textbf{92.3} & \textbf{45.5} & \textbf{38.7} & \textbf{80.5} & \textbf{97.0} \\
	No Normalization & 92.3 & 5.7 & 8.7 & 78.9 & 95.7 \\
	No Multi-scale Supervision & 92.1 & 45.2 & 36.2 & 75.4 & 93.1 \\
	No RefineNet & 92.3 & 45.5 & 34.7 & 70.9 & 97.0 \\
	Smaller Backbone & 91.1 & 43.8 & 35.1 & 75.7 & 96.4 \\
	\specialrule{.1em}{.05em}{.05em}
	\end{tabular}
\end{table} \noindent{\bf Architecture.} Table~\ref{table:ablation} shows how different designs of our framework affect the multi-person 3D pose estimation accuracy:
1) the performance of our model will degrade severely without depth normalization. As we discussed in Section \ref{sec:representation}, normalizing depth values by the size of FoV makes depth learning easier. 
2) Multi-scale supervision is beneficial.
3) To show that our performance gain in terms of the absolute 3D pose is mostly attributed to our single-shot bottom-up design rather than the network size, we test with a smaller backbone. The results show that, even with a one-stage hourglass network, our method still achieves higher PCK and PCK than the top-down method  \cite{Moon_2019_ICCV_3DMPPE}. 



\begin{table}[t]
	\centering
	\setlength\tabcolsep{1.0pt}
	\def\arraystretch{1.0}
	\caption{\textbf{Ablation study of the part association}. `2DPA' means the 2D part association proposed by \cite{cao2017realtime}. `DAPA' means the depth-aware part association we proposed. Both of them are based on the same heatmaps and PAFs results.}
	\label{table:partassociation}
	\begin{tabular}{L{1.0cm}|C{1.1cm}C{1.15cm}|*{5}{C{1.2cm}}}
	\specialrule{.1em}{.05em}{.05em}
     & \multicolumn{2}{c|}{Panoptic} & \multicolumn{5}{c}{MuPoTS-3D}   \\\hline
     & Recall & 2DPCK & Recall & PCK & PCK & PCK & PCOD \\
     \hline
     2DPA & 94.3 & 92.4 & 92.1 & 45.3 & 38.6 & 80.2 & 96.5\\
     DAPA & \textbf{96.4} & \textbf{93.1} & \textbf{92.3} & \textbf{45.5} & \textbf{38.7} & \textbf{80.5} & \textbf{97.0}\\
	\specialrule{.1em}{.05em}{.05em}
	\end{tabular}
\end{table}


 
\begin{figure*}[t]
\centering
\includegraphics[width=1.0\linewidth,trim={1.5cm 1cm 1.5cm 0cm},clip]{figures/internet_demo.pdf} \caption{
	\textbf{Qualitative results on in-the-wild images from the Internet}.
 }
 \label{fig:lastdemo}
\end{figure*}

\paragraph{\bf Part association.} To compare the proposed depth-aware part association with the 2D part association in \cite{cao2017realtime}, 
we evaluate relevant metrics on Panoptic and MuPoTS-3D datasets. Note that the threshold of 2DPCK is the half of the head size. Table~\ref{table:partassociation} lists the results (2DPA vs. DAPA) and reveals that our depth-aware part association outperforms the 2D part association in all these metrics. Besides, Fig.~\ref{fig:2DAmbiguity} shows some qualitative examples.

\paragraph{\bf RefineNet.} 
Table~\ref{table:panoptic3d} and ~\ref{table:ablation} show that
RefineNet is able to improve both relative and absolute pose estimation. It is able to complete invisible keypoints and refine visible keypoints with a learned 3D pose prior. 
The improvement is more significant on the Panoptic dataset since the training and test images are captured by cameras with similar views. 

~\\
Please refer to the supplementary material for more experimental details and results.

 
\section{Conclusion}

We proposed a novel single-shot bottom-up framework to estimate absolute multi-person 3D poses from a single RGB image. The proposed framework uses a fully convolutional network to regress a set of 2.5D representations for multiple people, from which the absolute 3D poses can be reconstructed. Additionally, benefited from the depth estimation of human bodies, a novel depth-aware part association algorithm was proposed and proven to benefit 2D pose estimation in crowd scenes. Experiments demonstrated state-of-the-art performance as well as generalization ability of the proposed approach.

\paragraph{\bf Acknowledgements:} The authors would like to acknowledge support from NSFC (No. 61806176), Fundamental Research Funds for the Central Universities (2019QNA5022) and ZJU-SenseTime Joint Lab of 3D Vision.
 

\clearpage
\begin{center}
    \textbf{\Large Supplementary Material: \\SMAP: Single-Shot Multi-Person \\Absolute 3D Pose Estimation} 
\end{center}
\setcounter{section}{0}

In this supplementary material, we provide more experimental details and results. Additionally, qualitative results on in-the-wild images from the Internet are shown in the supplementary video.

\section{More details}
\subsection{Loss function}
There are three output branches of the network, illustrated in  Fig.~\ref{fig:pipeline}. The first branch regresses keypoint heatmaps  and PAFs  simultaneously, while the second branch regresses part relative-depth maps .  loss is applied to these two branches. The third branch predicts root depth map . According to 2D location of detected root , we can get the predicted root depth  and compared it with the groundtruth normalized depth  using  loss. The total loss is computed by weighted summation of all losses.
Our loss functions are as follows. 

where ,  are the number of joints, the number of detected people (root joints) respectively,  means each pixel location and superscript  denotes the groundtruth. The default settings are: =0.1, =5, =10. 

\subsection{Running time and memory}


\begin{table}[t]
	\centering
	\setlength\tabcolsep{1.0pt}
	\def\arraystretch{1.0}
	\caption{Running time and memory comparison.}
	\label{table:time_memory}
    \begin{tabular}{C{1cm}C{1.5cm}||C{1.8cm}C{1.8cm}||C{1.8cm}C{1.8cm}}
	\specialrule{.1em}{.05em}{.05em}
	& & \multicolumn{2}{c||}{3-people} & \multicolumn{2}{c}{20-people} \\
	\hline
    & & Time(ms) & Memory(M) & Time(ms) & Memory(M) \\
    \hline
    \multirow{3}{*}{\cite{Moon_2019_ICCV_3DMPPE}} & DetectNet & 120.0 & 899 & 120.0 & 899 \\ 
    & PoseNet & 14.7 & 815 & 71.8 & 1491 \\
    & RootNet & 13.0 & 803 & 58.9 & 1051 \\
    \hline
    \multirow{3}{*}{Ours} & SMAP & 57.0 & 1379 & 57.0 & 1379 \\
    & DAPA & 4.5 & - & 8.8 & - \\
    & RefineNet & 0.80 & 0.5 & 0.83 & 0.5 \\
    \specialrule{.1em}{.05em}{.05em}
    \end{tabular}
\end{table}
 Table \ref{table:time_memory} provides detailed information about running time and memory of the state-of-the-art top-down method \cite{Moon_2019_ICCV_3DMPPE} and our method. Note that our method is almost not affected by the number of people in the image. 

\section{More results compared with SOTA}
Due to the limited space, only the average PCK is reported in the main manuscript. Here we provide more thorough experimental results. Table~\ref{table:mupots_abs20} presents sequence-wise PCK on the MuPoTS-3D dataset and demonstrates that our PCK is higher than the state-of-the-art top-down method \cite{Moon_2019_ICCV_3DMPPE}, especially for outdoor scenarios (TS6-TS20). Table~\ref{table:mupots_rel} shows that our model has higher PCK compared with all bottom-up methods and most top-down methods except \cite{Moon_2019_ICCV_3DMPPE}. Note that we have higher AUC compared with \cite{Moon_2019_ICCV_3DMPPE} as we state in the main manuscript. Table \ref{table:h36m} shows the results on the Human3.6M dataset.

\section{More ablation analysis}
\subsection{Effect of the multi-task structure}
SMAP simultaneously output 2D information (keypoint heatmaps and PAFs), root depth map, and part relative-depth map. To analyze the impact of our single-shot multi-task structure on root localization, we delete some of the output branches and evaluate the performance, as indicated in Table~\ref{table:ablation_root}. One variant is only to regress the root position and its depth alone (row 2 of Table~\ref{table:ablation_root}). This variant can obtain an acceptable result, which reflects the significance of our bottom-up design for root localization. Another variant which adds the keypoint heatmaps and PAFs branches (row 3 of Table~\ref{table:ablation_root}) significantly improves the performance, indicating that 2D cues (pose, body size) are also beneficial to root depth estimation. Nevertheless, this variant is still inferior to the full model. 

\subsection{Influence of camera intrinsics}
Here we make three comparisons:1) full model with known camera intrinsics. 2) full model without camera intrinsics. 3) without normalization. 

RtError of our full model reaches 23.3cm on the MuPoTS-3D dataset. If the intrinsic parameter is not provided (use default intrinsics), RtError increases to 67cm. Note that the ordinal depth relation remains unchanged. If the model lacks normalization, RtError is as high as 120cm. 


\begin{table}[t]
	\centering
	\setlength\tabcolsep{1.0pt}
	\def\arraystretch{1.0}
	\caption{Sequence-wise PCK on the MuPoTS-3D dataset for matched groundtruths.}
	\label{table:mupots_abs20}
	\begin{tabular}{L{2.5cm}*{11}{C{0.73cm}}}
	\specialrule{.1em}{.05em}{.05em}
     & S1  & S2 & S3 & S4 & S5 & S6 & S7 & S8 & S9 & S10 \\ \hline
	Moon et al.~\cite{Moon_2019_ICCV_3DMPPE} & \textbf{59.5} & \textbf{45.3} & \textbf{51.4} & \textbf{46.2} & 53.0 & \textbf{27.4} & 23.7 & \textbf{26.4} & \textbf{39.1} & 23.6 & \\
    \textbf{Ours} & 42.1 & 41.4 & 46.5 & 16.3 & \textbf{53.0} & 26.4 & \textbf{47.5} & 18.7 & 36.7 & \textbf{73.5} &\\
    \specialrule{.1em}{.05em}{.05em}
    & S11 & S12 & S13 & S14 & S15 & S16 & S17 & S18 & S19 & S20 & Avg. \\ \hline
    Moon et al.~\cite{Moon_2019_ICCV_3DMPPE} & 18.3 & 14.9 & \textbf{38.2} & 29.5 & 36.8 & 23.6 & 14.4 & 20.0 & 18.8 & 25.4 & 31.8 \\
	\textbf{Ours} & \textbf{46.0} & \textbf{22.7} & 24.3 & \textbf{38.9} & \textbf{47.5} & \textbf{34.2} & \textbf{35.0} & \textbf{20.0} & \textbf{38.7} & \textbf{64.8} & \textbf{38.7}\\
    
	\specialrule{.1em}{.05em}{.05em}
	\end{tabular}
	\vspace*{-2mm}
\end{table}

\begin{table}[t]
	\centering
	\setlength\tabcolsep{1.0pt}
	\def\arraystretch{1.0}
	\caption{PCK on the MuPoTS-3D dataset for matched groundtruths. 
	`T' denotes top-down methods while `B' denotes bottom-up methods.}
	\label{table:mupots_rel}
	\begin{tabular}{L{2.5cm}L{0.5cm} *{11}{C{0.73cm}}}
	\specialrule{.1em}{.05em}{.05em}
    & & S1 & S2 & S3 & S4 & S5 & S6 & S7 & S8 & S9 & S10 \\ \hline
    Rogez et al.~\cite{rogez2017lcr} & T & 69.1 & 67.3 & 54.6 & 61.7 & 74.5 & 25.2 & 48.4 & 63.3 & 69.0 & 78.1 & \\
    Rogez et al.~\cite{rogez2019lcr} & T & 88.0 & 73.3 & 67.9 & 74.6 & 81.8 & 50.1 & 60.6 & 60.8 & 78.2 & 89.5 &  \\
    Dabral et al.~\cite{dabral2019multi} & T & 85.8 & 73.6 & 61.1 & 55.7 & 77.9 & 53.3 & 75.1 & 65.5 & 54.2 & 81.3 & \\
	Moon et al.~\cite{Moon_2019_ICCV_3DMPPE} & T & \textbf{94.4} & 78.6 & \textbf{79.0} & \textbf{82.1} & 86.6 & \textbf{72.8} & 81.9 & 75.8 & \textbf{90.2} & \textbf{90.4} & \\
	Mehta et al.~\cite{mehta2018single} & B & 81.0 & 64.3 & 64.6 & 63.7 & 73.8 & 30.3 & 65.1 & 60.7 & 64.1 & 83.9 &  \\
	Mehta et al.~\cite{mehta2019xnect} & B & 88.4 & 70.4 & 68.3 & 73.6 & 82.4 & 46.4 & 66.1 & \textbf{83.4} & 75.1 & 82.4 &  \\
    \textbf{Ours} & B & 89.9 & \textbf{88.3} & 78.9 & 78.2 & \textbf{87.6} & 51.0 & \textbf{88.5} & 71.6 & 70.3 & 89.2 &\\
    \specialrule{.1em}{.05em}{.05em}
    & & S11 & S12 & S13 & S14 & S15 & S16 & S17 & S18 & S19 & S20 & Avg. \\ \hline
    Rogez et al.~\cite{rogez2017lcr} & T & 53.8 & 52.2 & 60.5 & 60.9 & 59.1 & 70.5 & 76.0 & 70.0 & 77.1 & 81.4 & 62.4 \\
    Rogez et al.~\cite{rogez2019lcr} & T & 70.8 & 74.4 & 72.8 & 64.5 & 74.2 & 84.9 & 85.2 & 78.4 & 75.8 & 74.4 & 74.0 \\
    Dabral et al.~\cite{dabral2019multi} & T & \textbf{82.2} & 71.0 & 70.1 & 67.7 & 69.9 & 90.5 & 85.7 & \textbf{86.3} & 85.0 & \textbf{91.4} & 74.2 \\
    Moon et al.~\cite{Moon_2019_ICCV_3DMPPE} & T & 79.4 & 79.9 & \textbf{75.3} & \textbf{81.0} & \textbf{81.0} & 90.7 & 89.6 & 83.1 & 81.7 & 77.3 & \textbf{82.5} \\
    Mehta et al.~\cite{mehta2018single} & B & 71.5 & 69.6 & 69.0 & 69.6 & 71.1 & 82.9 & 79.6 & 72.2 & 76.2 & 85.9 & 69.8 \\
	Mehta et al.~\cite{mehta2019xnect} & B & 76.5 & 73.0 & 72.4 & 73.8 & 74.0 & 83.6 & 84.3 & 73.9 & \textbf{85.7} & 90.6 & 75.8 \\
	\textbf{Ours} & B & 76.3 & \textbf{82.0} & 70.8 & 65.2 & 80.4 & \textbf{91.6} & \textbf{90.4} & 83.4 & 84.3 & 91.2 & 80.5\\
	\specialrule{.1em}{.05em}{.05em}
	\end{tabular}
	\vspace*{-2mm}
\end{table}


\begin{table}[t]
	\centering
	\setlength\tabcolsep{1.0pt}
	\def\arraystretch{1.0}
	\caption{Sequence-wise PCK on the MuPoTS-3D dataset.}
	\label{table:mupots_abs20}
	\begin{tabular}{L{2.5cm}*{11}{C{0.73cm}}}
	\multicolumn{12}{l}{\textit{Accuracy for all groundtruths}} \\
	\specialrule{.1em}{.05em}{.05em}
     & S1 & S2 & S3 & S4 & S5 & S6 & S7 & S8 & S9 & S10 \\ \hline
    Moon et al.~\cite{Moon_2019_ICCV_3DMPPE} & \textbf{59.5} & \textbf{44.7} & \textbf{51.4} & \textbf{46.0} & \textbf{52.2} & \textbf{27.4} & 23.7 & \textbf{26.4} & \textbf{39.1} & 23.6 & \\
    \textbf{Ours} & 41.6 & 33.4 & 45.6 & 16.2 & 48.8 & 25.8 & \textbf{46.5} & 13.4 & 36.7 & \textbf{73.5} &\\
    \specialrule{.1em}{.05em}{.05em}
    & S11 & S12 & S13 & S14 & S15 & S16 & S17 & S18 & S19 & S20 & Avg. \\ \hline
    Moon et al.~\cite{Moon_2019_ICCV_3DMPPE} & 18.3 & 14.9 & \textbf{38.2} & 26.5 & 36.8 & 23.4 & 14.4 & \textbf{19.7} & 18.8 & 25.1 & 31.5 \\
	\textbf{Ours} & \textbf{43.6} & \textbf{22.7} & 21.9 & \textbf{26.7} & \textbf{47.1} & \textbf{32.5} & \textbf{31.4} & 18.0 & \textbf{33.8} & \textbf{47.8} & \textbf{35.4}\\
    \specialrule{.1em}{.05em}{.05em}
    \\
    \multicolumn{12}{l}{\textit{Accuracy for matched groundtruths}} \\
    \specialrule{.1em}{.05em}{.05em}
     & S1  & S2 & S3 & S4 & S5 & S6 & S7 & S8 & S9 & S10 \\ \hline
	Moon et al.~\cite{Moon_2019_ICCV_3DMPPE} & \textbf{59.5} & \textbf{45.3} & \textbf{51.4} & \textbf{46.2} & 53.0 & \textbf{27.4} & 23.7 & \textbf{26.4} & \textbf{39.1} & 23.6 & \\
    \textbf{Ours} & 42.1 & 41.4 & 46.5 & 16.3 & \textbf{53.1} & 26.4 & \textbf{47.5} & 18.7 & 36.7 & \textbf{73.5} &\\
    \specialrule{.1em}{.05em}{.05em}
    & S11 & S12 & S13 & S14 & S15 & S16 & S17 & S18 & S19 & S20 & Avg. \\ \hline
    Moon et al.~\cite{Moon_2019_ICCV_3DMPPE} & 18.3 & 14.9 & \textbf{38.2} & 29.5 & 36.8 & 23.6 & 14.4 & 20.0 & 18.8 & 25.4 & 31.8 \\
	\textbf{Ours} & \textbf{46.0} & \textbf{22.7} & 24.3 & \textbf{38.9} & \textbf{47.5} & \textbf{34.2} & \textbf{35.0} & \textbf{20.1} & \textbf{38.7} & \textbf{64.8} & \textbf{38.7}\\
	\specialrule{.1em}{.05em}{.05em}
	\end{tabular}
\end{table}

\begin{table}[t]
	\centering
	\setlength\tabcolsep{1.0pt}
	\def\arraystretch{1.0}
	\caption{PCK on the MuPoTS-3D dataset for all groundtruths. 
	`T' denotes top-down methods while `B' denotes bottom-up methods.}
	\label{table:mupots_rel_all}
	\begin{tabular}{L{2.5cm}L{0.5cm} *{11}{C{0.73cm}}}
	\specialrule{.1em}{.05em}{.05em}
    & & S1 & S2 & S3 & S4 & S5 & S6 & S7 & S8 & S9 & S10 \\ \hline
    Rogez et al.~\cite{rogez2017lcr} & T & 67.7 & 49.8 & 53.4 & 59.1 & 67.5 & 22.8 & 43.7 & 49.9 & 31.1 & 78.1 & \\
    Rogez et al.~\cite{rogez2019lcr} & T & 87.3 & 61.9 & 67.9 & 74.6 & 78.8 & 48.9 & 58.3 & 59.7 & 78.1 & 89.5 &  \\
    Dabral et al.~\cite{dabral2019multi} & T & 85.1 & 67.9 & 73.5 & 76.2 & 74.9 & 52.5 & 65.7 & 63.6 & 56.3 & 77.8 & \\
	Moon et al.~\cite{Moon_2019_ICCV_3DMPPE} & T & \textbf{94.4} & \textbf{77.5} & \textbf{79.0} & \textbf{81.9} & \textbf{85.3} & \textbf{72.8} & 81.9 & \textbf{75.7} & \textbf{90.2} & \textbf{90.4} & \\
	Mehta et al.~\cite{mehta2018single} & B & 81.0 & 59.9 & 64.4 & 62.8 & 68.0 & 30.3 & 65.0 & 59.2 & 64.1 & 83.9 &  \\
	Mehta et al.~\cite{mehta2019xnect} & B & 88.4 & 65.1 & 68.2 & 72.5 & 76.2 & 46.2 & 65.8 & 64.1 & 75.1 & 82.4 &  \\
    \textbf{Ours} & B & 88.8 & 71.2 & 77.4 & 77.7 & 80.6 & 49.9 & \textbf{86.6} & 51.3 & 70.3 & 89.2  &\\
    \specialrule{.1em}{.05em}{.05em}
    & & S11 & S12 & S13 & S14 & S15 & S16 & S17 & S18 & S19 & S20 & Avg. \\ \hline
    Rogez et al.~\cite{rogez2017lcr} & T & 50.2 & 51.0 & 51.6 & 49.3 & 56.2 & 66.5 & 65.2 & 62.9 & 66.1 & 59.1 & 53.8 \\
    Rogez et al.~\cite{rogez2019lcr} & T & 69.2 & 73.8 & 66.2 & 56.0 & 74.1 & 82.1 & 78.1 & 72.6 & 73.1 & 61.0 & 70.6 \\
    Dabral et al.~\cite{dabral2019multi} & T & 76.4 & 70.1 & 65.3 & 51.7 & 69.5 & 87.0 & 82.1 & 80.3 & 78.5 & 70.7 & 71.3 \\
    Moon et al.~\cite{Moon_2019_ICCV_3DMPPE} & T & \textbf{79.2} & 79.9 & \textbf{75.1} & \textbf{72.7} & \textbf{81.1} & \textbf{89.9} & \textbf{89.6} & \textbf{81.8} & \textbf{81.7} & \textbf{76.2} & \textbf{81.8} \\
    Mehta et al.~\cite{mehta2018single} & B & 67.2 & 68.3 & 60.6 & 56.5 & 69.9 & 79.4 & 79.6 & 66.1 & 66.3 & 63.5 & 65.0 \\
	Mehta et al.~\cite{mehta2019xnect} & B & 74.1 & 72.4 & 64.4 & 58.8 & 73.7 & 80.4 & 84.3 & 67.2 & 74.3 & 67.8 & 70.4 \\
	\textbf{Ours} & B & 72.3 & \textbf{81.7} & 63.6 & 44.8 & 79.7 & 86.9 & 81.0 & 75.2 & 73.6 & 67.2 & 73.5\\
	\specialrule{.1em}{.05em}{.05em}
	\end{tabular}
\end{table}

\begin{table}[t]
	\centering
	\setlength\tabcolsep{1.0pt}
	\def\arraystretch{1.0}
	\caption{MPJPE Results on Human3.6M dataset. Note that there is no groundtruth bounding box information in inference time.}
	\label{table:h36m}
	\begin{tabular}{L{3cm}C{1.7cm}}
	\specialrule{.1em}{.05em}{.05em}
	Method & MPJPE \\ \hline
    Rogez et al. \cite{rogez2017lcr} & 87.7 \\
    Mehta et al. \cite{mehta2018single} & 69.9 \\
    Dabral et al. \cite{dabral2019multi} & 65.2 \\
    Mehta et al. \cite{mehta2019xnect} & 63.6 \\
    Rogez et al. \cite{rogez2019lcr} & 63.5 \\
    Moon et al. \cite{Moon_2019_ICCV_3DMPPE} & 54.4 \\
    \textbf{Ours} & \textbf{54.1} \\
	
	\specialrule{.1em}{.05em}{.05em}
	\end{tabular}
\end{table}

\begin{table}[t]
	\centering
	\setlength\tabcolsep{1.0pt}
	\def\arraystretch{1.0}
	\caption{Ablation study of the structure design on the MuPoTS-3D dataset. Note that our full model consists of root depth, relative depth and 2D branches.}
	\label{table:ablation_root}
	\begin{tabular}{L{5.5cm}|*{5}{C{1.2cm}}}
	\specialrule{.1em}{.05em}{.05em}
    Design & Recall & PCK & PCK & PCK & PCOD\\\hline
    Full Model & \textbf{92.3} & \textbf{45.5} & \textbf{38.7} & \textbf{80.5} & \textbf{97.0} \\
    Root Depth Only & 85.0 & 29.9 & - & - & 88.3 \\
	Root Depth + 2D Branches & 92.1 & 43.6 & - & - & 96.7 \\
	
	\specialrule{.1em}{.05em}{.05em}
	\end{tabular}
\end{table} 
 
\clearpage
\bibliographystyle{splncs04}
\bibliography{egbib}
\end{document}
