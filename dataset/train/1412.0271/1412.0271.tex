\documentclass[preprint,12pt]{elsarticle}
\usepackage{lineno}

\usepackage[T1]{fontenc}
\usepackage[latin2]{inputenc}
\usepackage[english]{babel}
\usepackage{dsfont, complexity} \usepackage{amsmath,amsthm}
\usepackage{amssymb}
\usepackage{float, lmodern}
\usepackage{graphicx}
\usepackage{tikz}
\usepackage{xcolor}
\usepackage{paralist}
\usepackage{refcount}
\newcommand{\myqed}{}
\newcommand{\myproof}{\noindent\textit{Proof. }}

\usepackage{mathtools}
\DeclarePairedDelimiter\ceil{\lceil}{\rceil}
\DeclarePairedDelimiter\floor{\lfloor}{\rfloor}
\usetikzlibrary{arrows,decorations.markings,patterns,calc, positioning, backgrounds}

\tikzstyle{vertex} = [circle, draw=black, scale=0.7]
\tikzstyle{edgelabel} = [circle, fill=white, scale=0.8]

\newcommand{\ktwotwo}[2]{
\coordinate (p) at #1;
\begin{scope}[rotate around={#2:(p)}]
	\node[vertex, ultra thick, gray] (p) at (p) {};
	\node[vertex] (q) at () {};
	
	\draw [ultra thick, gray, dotted] (p) -- (p');
	\draw [] (p) -- (q');
	\draw [] (q) -- (q');
	\draw [] (q) -- (p');
\end{scope}
}
\newtheorem{theorem}{Theorem}[section]
\newtheorem{claim}[theorem]{Claim}
\newtheorem{pr}[theorem]{Problem}
\newtheorem{ex}[theorem]{Example}
\newtheorem{lemma}[theorem]{Lemma}
\newtheorem{cor}[theorem]{Corollary}
\newtheorem{defi}[theorem]{Definition}
\newtheorem{conj}[theorem]{Conjecture}
\newtheorem{ques}[theorem]{Question}
\newtheorem{remark}[theorem]{Remark}
\newcommand{\true}{\mathsf{true}}
\newcommand{\false}{\mathsf{false}}

\journal{Discrete Optimization}
\begin{document}
\begin{frontmatter}

\title{Stable Marriage and Roommates problems with restricted edges: Complexity and approximability\tnoteref{title}}

\tnotetext[title]{A preliminary version of this paper appeared in the Proceedings of SAGT 2015: the 8th International Symposium on Algorithmic Game Theory.}

\author[cseh]{\'Agnes Cseh}
\ead{cseh@math.tu-berlin.de}
\fntext[cseh]{Supported by COST Action IC1205 on Computational Social Choice and by the Deutsche Telekom Stiftung. Part of this work was carried out whilst visiting the University of Glasgow. Present address:  School of Computer Science, Reykjavik University, Menntavegur 1, 101 Reykjavik, Iceland.}

\author[manlove]{David F. Manlove}
\fntext[manlove]{Supported by Engineering and Physical Sciences Research Council grant EP/K010042/1.}
\ead{David.Manlove@glasgow.ac.uk}

\address[cseh]{Institute for Mathematics, Technische Universit\"at Berlin, Sekr. MA 5-2, Stra{\ss}e des 17. Juni 136, 10623 Berlin, Germany}

\address[manlove]{School of Computing Science, Sir Alwyn Williams Building, University of Glasgow, Glasgow G12 8QQ, UK}

\begin{abstract}
In the Stable Marriage and Roommates problems, a set of agents is given, each of them having a strictly ordered preference list over some or all of the other agents. A matching is a set of disjoint~pairs of mutually acceptable agents. If any two agents mutually prefer each other to their partner, then they block the matching, otherwise, the matching is said to be stable. We investigate the complexity of finding a solution satisfying additional constraints on restricted pairs of agents. Restricted pairs can be either \emph{forced} or \emph{forbidden}. A stable solution must contain all of the forced pairs, while it must contain none of the forbidden pairs.

Dias et al.~\cite{DFFS03} gave a polynomial-time algorithm  to decide whether such a solution exists in the presence of restricted edges. If the answer is no, one might look for a solution close to optimal. Since optimality in this context means that the matching is stable and satisfies all constraints on restricted pairs, there are two ways of relaxing the constraints by permitting a solution to: (1)~be blocked by as few as possible pairs, or (2)~violate as few as possible constraints on restricted pairs.

Our main theorems prove that for the (bipartite) Stable Marriage problem, case~(1) leads to -hardness and inapproximability results, whilst case~(2) can be solved in polynomial time.  For non-bipartite Stable Roommates instances, case~(2) yields an -hard problem.  In the case of -hard problems, we also discuss polynomially solvable special cases, arising from restrictions on the lengths of the preference lists, or upper bounds on the numbers of restricted pairs.
\end{abstract}

\begin{keyword}
stable matching \sep restricted edge \sep approximation algorithm 
 \MSC 05C70 \sep 68W40 \sep 05C85
\end{keyword}
\end{frontmatter}

\section{Introduction}
\label{se:intro}

In the classical \emph{Stable Marriage problem} ({\sc sm})~\cite{GS62}, a bipartite graph is given, where one colour class symbolises a set of men  and the other colour class stands for a set of women~. Man  and woman  are connected by edge  if they find one another mutually acceptable. Each participant provides a strictly ordered preference list of the acceptable agents of the opposite gender. An edge  \emph{blocks} matching  if it is not in , but each of  and  is either unmatched or prefers the other to their partner. A \emph{stable matching} is a matching not blocked by any edge. From the seminal paper of Gale and Shapley~\cite{GS62}, we know that the existence of such a stable solution is guaranteed and one can be found in linear time. Moreover, the solutions form a distributive lattice~\cite{Knu76}. The two extreme points of this lattice are called the \emph{man-} and \emph{woman-optimal stable matchings}~\cite{GS62}. These assign each man/woman their best partner reachable in any stable matching. Another interesting and useful property of stable solutions is the so-called Rural Hospitals Theorem. Part of this theorem states that if an agent is unmatched in one stable matching, then all stable solutions leave him unmatched~\cite{GS85}.

One of the most widely studied  extensions of {\sc sm} is the \emph{Stable Roommates problem} ({\sc sr})~\cite{GS62,Irv85}, defined on general graphs instead of bipartite graphs. The notion of a blocking edge is as defined above (except that it can now involve any two agents in general), but several results do not carry over to this setting. For instance, the existence of a stable solution is not guaranteed any more. On the other hand, there is a linear-time algorithm to find a stable matching or report that none exists~\cite{Irv85}. Moreover, the corresponding variant of the Rural Hospitals Theorem holds in the roommates case as well: the set of matched agents is the same for all stable solutions~\cite{GI89}. We summarise this observation as follows:
\begin{theorem}[Gusfield and Irving~\cite{GI89}]
Given an instance of {\sc sr}, the same set of agents is matched in all stable matchings.
\label{th:rural}
\end{theorem}

Both {\sc sm} and {\sc sr} are widely used in various applications. In markets where the goal is to maximise social welfare instead of profit, the notion of stability is especially suitable as an optimality criterion~\cite{Rot84}. For {\sc sm}, the oldest and most common area of applications is employer allocation markets~\cite{RS90}.  On one side, job applicants are represented, while the job openings form the other side. Each application corresponds to an edge in the bipartite graph. The employers rank all applicants to a specific job offer and similarly, each applicant sets up a preference list of jobs. Given a proposed matching  of applicants to jobs, if an employer--applicant pair exists such that the position is not filled or a worse applicant is assigned to it, and the applicant received no contract or a worse contract, then this pair blocks~. In this case the employer and applicant find it mutually beneficial to enter into a contract outside of , undermining its integrity. If no such blocking pair exists, then  is stable. Stability as an underlying concept is also used to allocate graduating medical students to hospitals in many countries~\cite{Rot08}. {\sc sr} on the other hand has applications in the area of P2P networks~\cite{GLMMRV07}.

Forced and forbidden edges in {\sc sm} and {\sc sr} open the way to formulate various special requirements on the sought solution. Such edges now form part of the extended problem instance: if an edge is \emph{forced}, it must belong to a constructed stable matching, whilst if an edge is \emph{forbidden}, it must not.
In certain market situations, a contract is for some reason particularly important, or to the contrary, not wished by the majority of the community or by the central authority in control. In such cases, forcing or forbidding the edge and then seeking a stable solution ensures that the wishes on these specific contracts are fulfilled while stability is guaranteed. Henceforth, the term \emph{restricted edge} will be used to refer either to a forbidden edge or a forced edge. The remaining edges of the graph are referred as \emph{unrestricted edges}.

Note that simply deleting forbidden edges or fixing forced edges and searching for a stable matching on the remaining instance does not solve the problem of finding a stable matching with restricted edges. Deleted edges (corresponding to forbidden edges, or those adjacent to forced edges) can block that matching. Therefore, to meet both requirements on restricted edges and stability, more sophisticated methods are needed.

The attention of the community was drawn very early to the characterization of stable matchings that must contain a prescribed set of edges. In the seminal book of Knuth~\cite{Knu76}, forced edges first appeared under the term \emph{arranged marriages}. Knuth presented an algorithm that finds a stable matching with a given set of forced edges or reports that none exists, given an instance of {\sc sm}. This method runs in  time, where  denotes the number of vertices in the graph. Gusfield and Irving~\cite{GI89} provided an algorithm for {\sc sm} based on rotations that terminates in  time, following  pre-processing time, where  is the set of forced edges. This latter method is favoured over Knuth's if multiple forced sets of small cardinality are proposed.

Forbidden edges appeared only in~2003 in the literature, and were first studied by Dias et al.~\cite{DFFS03}. In their paper, complete bipartite graphs were considered, but the methods can easily be extended to incomplete preference lists. Their main result was the following (in the following theorem, and henceforth,  is the total number of edges in the graph).

\begin{theorem}[Dias et al.~\cite{DFFS03}] 
\label{th:bip_decision} 
The problem of finding a stable matching in a  {\sc sm} instance with forced and forbidden edges or reporting that none exists is solvable in  time.
\end{theorem}

While Knuth's method relies on basic combinatorial properties of stable matchings, the other two algorithms make use of \emph{rotations}. We refer the reader to~\cite{GI89} for background on these. The problem of finding a stable matching with forced and forbidden edges in an {\sc sm} instance can easily be formulated as a weighted stable matching problem (that is, we seek a stable matching with minimum weight, where the weight of a matching  is the sum of the weights of the edges in~). Let us assign all forced edges weight~, all forbidden edges weight~, and all remaining edges weight~0. A stable matching satisfying all constraints on restricted edges exists if and only if there is a stable matching of weight~ in the weighted instance, where  is the set of forced edges. With the help of rotations, minimum weight stable matchings can be found in polynomial time~\cite{ILG87,Fed92,Rot92,Fed94} (see the final paragraph of Section~\ref{se:preliminaries} for more detail on the role played by each of these references).

Since finding a weight-minimal stable matching in {\sc sr} instances is an -hard task~\cite{Fed92}, it follows that solving the problem with forced and forbidden edges requires different methods from the aforementioned weighted transformation. Fleiner et al.~\cite{FIM07} showed that any {\sc sr} instance with forbidden edges can be converted into another stable matching problem involving ties that can be solved in  time~\cite{IM02} and the transformation has the same time complexity as well. Forced edges can easily be eliminated by forbidding all edges adjacent to them, therefore we can state the following result.

\begin{theorem}[Fleiner et al.~\cite{FIM07}]  
\label{th:sr_decision}
The problem of finding a stable matching in an {\sc sr}  instance with forced and forbidden edges or reporting that none exists is solvable in  time.
\end{theorem}

As we have seen so far, answering the question as to whether a stable solution containing all forced and avoiding all forbidden edges exists can be solved efficiently in the case of both {\sc sm} and {\sc sr}. We thus concentrate on cases where the answer to this question is no. What kind of approximate solutions exist then and how can we find them?

\paragraph{Our contribution} Since optimality is defined by two criteria, it is straightforward to define approximation from those two points of view. In case~BP, all constraints on restricted edges must be satisfied, and we seek a matching with the minimum number of blocking edges. In case~CV, we seek a stable matching that violates the fewest constraints on restricted edges.  The optimisation problems that arise from each of these cases are defined formally in Section~\ref{se:preliminaries}.

In Section~\ref{se:almost_stable}, we consider case~BP: that is, all constraints on restricted edges must be fulfilled, while the number of blocking edges is minimised. We show that in the {\sc sm} case, this problem is computationally hard and not approximable within  for any , unless . We also discuss special cases for which this problem becomes tractable. This occurs if the maximum degree of the graph is at most~2 or if the number of blocking edges in the optimal solution is a constant. We point out a striking difference in the complexity of the two cases with only forbidden and only forced edges: the problem is polynomially solvable if the number of forbidden edges is a constant, but by contrast it is -hard even if the instance contains a single forced edge. We also prove that when the restricted edges are either all forced or all forbidden, the optimisation problem remains -hard even on very sparse instances, where the maximum degree of a vertex is~3.

Case~CV, where the number of violated constraints on restricted edges is minimised while stability is preserved, is studied in Section~\ref{se:viol_const}. It is a rather straightforward observation that in {\sc sm}, the setting can be modelled and efficiently solved with the help of edge weights. Here we show that on non-bipartite graphs, the problem becomes -hard. As in case~BP, we also discuss the complexity of degree-constrained restrictions and establish that the -hardness results remain intact even for graphs with degree at most~3, while the case with degree at most~2 is polynomially solvable.

A structured overview of our results for general {\sc sm} and {\sc sr} instances is contained in Table~\ref{ta:results}. 
\begin{table}[ht]
	\centering
    \resizebox{\columnwidth}{!}{
		\begin{tabular}{|c|c|c|}
		\hline
			\phantom{n}& Stable Marriage & Stable Roommates \\ \hline
			\begin{tabular}{c}case BP: \\ min \# blocking edges \end{tabular} & \begin{tabular}{c}-hard to approximate \\ within  \end{tabular}&  \begin{tabular}{c}-hard to approximate \\ within  \end{tabular}\\  \hline
			 \begin{tabular}{c}case CV: min \# violated \\ restricted edge constraints \end{tabular} &\begin{tabular}{c}solvable \\ in polynomial time \end{tabular} & \begin{tabular}{c}-hard\end{tabular}\\
		\hline
		\end{tabular}}\newline
\caption{Summary of results}
\label{ta:results}
\end{table}

\section{Preliminaries and techniques}
\label{se:preliminaries}

In this section, we introduce the notation used in the remainder of the paper and also define the key problems that we investigate later. An instance  of the \emph{Stable Marriage problem} ({\sc sm}) consists of a bipartite graph  with  vertices and  edges, and a set : the set of strictly ordered, but not necessarily complete preference lists. These lists are provided on the set of adjacent vertices at each vertex. The \emph{Stable Roommates problem} ({\sc sr}) differs from {\sc sm} in one sense: the underlying graph  need not be bipartite. In both {\sc sm} and {\sc sr}, a matching  in  is sought, assigning each agent to at most one partner. If a vertex  is matched in , we denote by  the partner of  in .  An edge  \emph{blocks} , or forms a \emph{blocking pair} of  if either  is unmatched or prefers  to , and either  is unmatched or prefers  to . A matching that is not blocked by any edge is called \emph{stable}.

As already mentioned in the Introduction, an {\sc sr} instance need not admit a stable solution. The number of blocking edges is a characteristic property of every matching. The set of edges blocking  is denoted by~. A natural goal is to find a matching minimising~;
following the consensus in the literature, such a matching is called \emph{almost stable}. This approach has a broad literature: almost stable matchings have been investigated in {\sc sm}~\cite{KMV94,HIM09,BMM10} and {\sc sr}~\cite{ABM06,BMM12} instances.

All problems investigated in this paper deal with at least one set of restricted edges. The set of forbidden edges is denoted by~, while  stands for the set of forced edges. We assume throughout the paper that . A matching  satisfies all constraints on restricted edges if  and .

In Figure~\ref{fi:base}, a sample {\sc sm} instance on four men and four women can be seen. The preference ordering is shown on the edges. For instance, vertex  ranks  best, then , and  last. The set of forbidden edges  is marked by dotted grey edges. The unique stable matching  contains both forbidden edges. Later on, we will return to this sample instance to demonstrate approximation concepts on it.

\begin{figure}[h]
\centering
\begin{tikzpicture}[scale=0.9, transform shape]
\node[vertex, label=above:] (u_1) at (0, 3) {};
\node[vertex, label=above:] (u_2) at (4, 3) {};
\node[vertex, label=above:] (u_4) at (12, 3) {
\label{eq:2}    n &= 2(3 n_1 + 2mC + 4n_2 -K + C) \\
\label{eq:2a} & =6 n_1 + 8n_1C + 8n_2 -2K + 2C \\ 
\label{eq:3}   &\leq 6 n_1 + 8n_1 ((n_1 + n_2)^{B+1} + 1) + 8n_2 - 2K + 2 (n_1 + n_2)^{B+1} + 2 \\
\label{eq:4}  &\leq 14 n_1 + (n_1 + n_2)^{B+1} (8 n_1 + 2) + 8n_2 + 2 \\
\label{eq:5}  &\leq 14 n_1 + 14 n_2 + (n_1 + n_2)^{B+1} (14 n_1 + 14n_2) \\
\label{eq:6} 	&\leq  (n_1 + n_2)^{B+2}  + 14(n_1 + n_2)^{B+2} \\
\label{eq:7} 	&=  15 (n_1 + n_2)^{B+2}

\label{eq:9}    n &=  6 n_1 + 8n_1C + 9n_2 -2K + 2C \\
\label{eq:11}     &>(n_1 + n_2)^{B+1}  \\
\label{eq:12b}     &\geq 15^B

\label{eq:14}C &>(n_1 + n_2)^B \\
\label{eq:15} &\geq 15^{- \frac{B}{B+2}} n^{\frac{B}{B+2}}\\
\label{eq:16} &\geq n^{1-\frac{3}{B+2}}\\
\label{eq:17}  &\geq n^{1-\varepsilon}
M=\{uw\in E: u_iw_j\in M'\mbox{ for some  and }\}.   
w(e) = 
     \begin{cases}
       -1 &\quad\text{if }e\text{ is forced},\\
       0 &\quad\text{if }e\text{ is unrestricted},\\ 
       1 &\quad\text{if }e\text{ is forbidden}.
     \end{cases}
};
\node[vertex] (v2) at (2, 0.5) {};
\node[vertex] (v4) at (-1, 1.5) {};
\node[vertex] (v6) at (1, 3) {};
\node[vertex] (v2') at () {};
\node[vertex] (v4') at () {};
\node[vertex] (v6') at () {   
w(e) = 
     \begin{cases}
       \frac{|Q|}{|M|} -1 &\quad\text{if }e\text{ is forced},\\
       \frac{|Q|}{|M|} &\quad\text{if }e\text{ is unrestricted},\\ 
       \frac{|Q|}{|M|} +1 &\quad\text{if }e\text{ is forbidden}.
     \end{cases}\vspace{-5mm}
M=\{p_i\bar{q_i},\bar{p_i}q_i : v_i\in S\}\cup
    \{p_i\bar{p_i},q_i\bar{q_i} : v_i\notin S\}
\begin{array}{llllllll}
	V' & = \{v_i^r : 1\leq i\leq n\wedge r \in \{1,2,3\}\}\\
	W  & = \{w_i^r : 1\leq i\leq n\wedge r \in \{1,2,3\}\}\\
	E' & = \{e_j^s : 1\leq j\leq m\wedge s \in \{1,2\}\}\\
	F & = \{f_j^s : 1\leq j\leq m\wedge s \in \{1,2\}\}	
\end{array}

\begin{array}{rll}
v_i^1: & w_i^1 ~~ e(v_i^1) ~~ w_i^2 ~~~~ & (1\leq i\leq n) \vspace{1mm}\\
v_i^2: & w_i^2 ~~ e(v_i^2) ~~ w_i^3 & (1\leq i\leq n) \vspace{1mm}\\
v_i^3: & w_i^3 ~~ e(v_i^3) ~~ w_i^1 & (1\leq i\leq n) \vspace{1mm}\\
e_j^1: & e_j^2 ~~ v(e_j^1) ~~ f_j^2 & (1\leq j\leq m) \vspace{1mm}\\
e_j^2: & f_j^1 ~~ v(e_j^2) ~~ e_j^1 & (1\leq j\leq m) \vspace{1mm}
\end{array}

		\begin{array}{rll}
w_i^1: & v_i^3 ~~ v_i^1             & (1\leq i\leq n) \vspace{1mm}\\
w_i^2: & v_i^1 ~~ v_i^2             & (1\leq i\leq n) \vspace{1mm}\\
w_i^3: & v_i^2 ~~ v_i^3             & (1\leq i\leq n) \vspace{1mm}\\
f_j^1: & f_j^2 ~~ e_j^2             & (1\leq j\leq m) \vspace{1mm}\\
f_j^2: & e_j^1 ~~ f_j^1             & (1\leq j\leq m) \vspace{1mm}
\end{array}

\end{minipage}
\caption{Preference lists in the constructed instance of {\sc sr min forbidden}.}
\label{preflists}
\end{figure}

Finally we define some further notation in~. For each , , let  and let
, where addition is taken modulo~3. Note that each  contains exactly one forbidden edge, while  has no forbidden edge. Similarly for each , , let  and let
.

\begin{claim}
	\label{cl:bp1}
	 admits a stable matching in which every vertex is matched. 
\end{claim}

	\myproof Let . Starting with the argument that each , each  and each  receive its first-choice partner in , it is straightforward to verify that  is stable: the remaining vertices prefer, to their partners, only vertices that already have their first-choice partners.  Theorem~\ref{th:rural} implies then that every stable matching in  matches every vertex in~. \myqed 

In Claims~\ref{cl:bp2} and~\ref{cl:bp3} we show that  has a vertex cover  where  if and only if  has a stable matching  where~.

\begin{claim}
\label{cl:bp2}
	If  has a vertex cover  such that  in~, then there is a stable matching  in  with~.
\end{claim}

	\myproof We construct a matching  in  as follows. For each  (), if , add  to , otherwise add  to~. For each  (), if , add  to , otherwise add  to~. Then .

	Now we verify that  is stable in~. Suppose firstly that some  has its third-choice partner in , and prefers .  Then .  Let .  Then by definition,  and .  Since ,  is covered by  at its other endpoint (i.e., ).  By construction of , .  Hence  has its first-choice partner in .
    
    Now suppose that some  has its third-choice partner in , and prefers .  Then .  It follows that , since  is a vertex cover.  Let .  Then by definition,  and .  Thus .  By construction of , .  Hence  has its first-choice partner in .
\myqed


\begin{claim}
	\label{cl:bp3}
	If there is a stable matching  with  in , then  has a vertex cover  in  such that~.
\end{claim}

	\myproof We construct a set of vertices  in  as follows. Claim~\ref{cl:bp1} states that  matches every vertex in .  Hence for each  (), either  or~. In the former case add  to~. As , it follows that~. Also, for each  (), as  matches every vertex in , either  or~. 

	Assume that  is not a vertex cover in~, i.e., there is an edge  such that  and .  Suppose that  and .  Then  and .
    
    Now let  be such that  and let  be such that .  Then  and  prefers  to its partner in .  Similarly  and  prefers  to its partner in .
    
    If  then  blocks .  Otherwise  and  blocks .  This contradiction to the stability of  implies that  is a vertex cover in~.  \myqed


	For {\sc sr max forced}, an analogous proof can be derived if we define the set of forced edges to be .
\end{proof}

\begin{theorem}
\label{srminrest2}
	{\sc sr min restricted violations} is solvable in  time if every preference list is of length at most~2.
\end{theorem}

\begin{proof}
    Since the the set of matched vertices is the same in all stable matchings by Theorem~\ref{th:rural}, finding a stable matching in  time in these very strongly restricted instances marks all vertices that need to be matched. In each component , since  is a path or a cycle, there are at most two possible stable matchings satisfying these constraints. We choose the stable matching in  that violates fewer constraints.\end{proof}

Short preference lists are not the only case when {\sc sr min restricted violations} becomes tractable, as our last theorem shows.
    
\begin{theorem}
\label{srminrestconst}
	{\sc sr min restricted violations} is solvable in polynomial time if the number of restricted edges or the minimal number of violated constraints is constant.
\end{theorem}

\begin{proof}
Suppose firstly that  is a constant.  We will show how to solve {\sc sr min restricted violations dec} in polynomial time.  We assume that, for the purposes of this proof, the problem definition is modified so that, given an instance , we are required to find a stable matching  in  such that , or report that no such matching exists.

Our first observation is that this problem is trivially solvable if the target value  satisfies .  In this case, any stable matching will suffice.

Now assume that~.  Suppose firstly that there is a stable matching  in  violating  restrictions.  Then  and , where .  If we let  and  then  is a stable matching in  containing no edge in  and containing all edges in .

Hence to solve {\sc sr min restricted violations dec} we generate all subsets  of  of size , for each .  Then we run the algorithm of~\cite{FIM07} to determine in  time whether there is a stable matching containing no edge in  and containing all edges in .  

Thus  subsets are generated to determine whether the desired matching exists.  The number of rounds is thus , while each round takes  time to complete.  The overall running time is .

We now show how to use the above approach in order to solve {\sc sr min restricted violations}.  If we find a solution during course of this process then  admits a stable matching  such that .  In order to minimise  it suffices to use the above technique in combination with a binary search procedure on values of .  This requires  invocations of the algorithm for the decision problem, which is a constant, and hence the overall time complexity remains .
\medskip

Now suppose that  is the minimal number of violated constraints, and that  is constant. For each value of , where  starts from 0 and increases by 1 after each iteration, we execute the algorithm described above to solve {\sc sr min restricted violations dec}, noting that it is sufficient to generate all subsets of size exactly  at each iteration.  We terminate as soon as we find a stable matching  such that 
 .  This process is bound to halt, since by definition,  admits a stable matching  such that , so . Thus the overall time complexity of this approach is , which is polynomial if  is a constant.
\end{proof}

\section{Conclusion and open questions}
\label{sec:conc}
    In this paper, we investigated the stable marriage and the stable roommates problems on graphs with forced and forbidden edges. Since a solution satisfying all constraints need not exist, two relaxed problems were defined. In {\sc min bp sm restricted}, constraints on restricted edges are strict, while a matching with the  minimum number of blocking edges is searched for. On the other hand, in {\sc sr min restricted violations}, we seek stable solutions that violate as few constraints on restricted edges as possible. For both problems, we determined the complexity and studied several special cases.
    
One of the most striking open questions is the approximability of {\sc sr min restricted violations}. Even though the problem can easily be formulated as a weighted {\sc sr} problem, the 2-approximation for this latter problem~\cite{TS97,TS98} only holds for instances with specific  and  sets. This is due to the non-negativity and monotonicity constraints on the 2-approximation result. Another open question is formulated as Conjecture~\ref{co:2_infty}: the complexity of {\sc min bp sm restricted} is not known if each woman's preference list consists of at most 2 elements.

A more general direction of further research involves the {\sc sm min restricted violations} problem. We have shown that it can be solved in polynomial time, using algorithms for minimum weight stable marriage. The following question arises naturally: is there a faster method for {\sc sm min restricted violations} that avoids reliance on Feder's algorithm or linear programming methods?

Another natural generalisation is to consider preference lists involving ties. Our hardness results carry over to this case, but the positive results need to be revisited.

Besides the two main problems discussed in this paper, other approximation concepts can also be investigated in the framework of restricted edges.  For example one alternative would be to combine the two objectives that we considered.  Can we efficiently find matchings that minimise the total number of violated constraints, that is, ?

Counting the number of blocking pairs is the most prevalent, but not the only relaxation of stability that has been studied in the literature. Other relaxations can also be combined with the presence of restricted edges in the instance, such as the following concepts.
\begin{itemize}
 \item \emph{Maximum internally stable matchings}~\cite{Tan90}, where the goal is to maximise the set of pairs that are stable within themselves. 
 \item \emph{Maximum irreversible stable matchings}~\cite{BIM16}, where the goal is to maximise the number of irreversible pairs. A pair is \emph{irreversible} if, once it is contained in a matching, no agent from the pair will ever be in a blocking edge, irrespective of how the outside agents are matched.
 \item \emph{Socially stable matchings}~\cite{AIKMP13}, where only a fixed subset of pairs (described by a given social network graph) can be blocking.
\end{itemize}

\section*{Acknowledgements}
We would like to thank the anonymous reviewers of this paper and an earlier version of it for their valuable comments, which helped to improve the presentation, and for suggesting several of the open problems that feature in Section \ref{sec:conc}.

\bibliographystyle{elsarticle-harv} \bibliography{mybib}
\end{document}