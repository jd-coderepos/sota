\documentclass{llncs}

\usepackage{latexsym}
\usepackage[dvipdfmx]{graphicx}
\usepackage{fancybox}
\usepackage{algorithm}
\usepackage[noend]{algpseudocode}
\renewcommand{\algorithmiccomment}[1]{ // #1}
\usepackage{amsmath}
\usepackage{mediabb}
\usepackage{amssymb}




\def\unnumbered#1{\begin{trivlist}
\item[] {{#1}\kern0.5em}}
\def\endunnumbered{\end{trivlist}}

\def\myappenv#1#2{\begin{unnumbered}{{\bf #1~\ref{#2}.}}}
\def\endmyappenv{\end{unnumbered}}



 \usepackage{url}
\usepackage{cite}


\def\newblock{\hskip .11em plus.33em minus.07em}



\newcommand{\name}[1]{\textit{#1}}
\newcommand{\set}[1]{\{#1\}}
\newcommand{\seq}[1]{\lange#1\rangle}
\newcommand{\inset}[2]{\{#1\;|\;#2\}}

\newcommand{\Can}{{\it CAND}}
\newcommand{\Avo}{{\it AVO}}
\newcommand{\adj}{{\rm adj}}
\newcommand{\len}{{\rm len}}
\newcommand{\MyProc}[1]{{\rm \textsc{#1}}}


\begin{document}

\title{Efficient Enumeration of Induced Subtrees in a K-Degenerate Graph}


\author{Kunihiro Wasa\inst{1} \and Hiroki Arimura\inst{1} \and Takeaki Uno\inst{2}}
\institute{Hokkaido University, Graduate School of Information Science and Technology, Japan, 
\email{\{wasa, arim\}@ist.hokudai.ac.jp}
\and National Institute of Informatics, Japan, 
\email{uno@nii.jp}}
\maketitle

\begin{abstract}
    In this paper, 
    we address the problem of  
    enumerating all induced subtrees 
    in an input \name{-degenerate} graph, 
    where an \name{induced subtree} is 
    an acyclic and connected induced subgraph. 
    A graph  is a -degenerate graph 
    if for any its induced subgraph has a vertex 
    whose degree is less than or equal to , 
    and many real-world graphs have small degeneracies, 
    or very close to small degeneracies. 
    Although, the studies are on subgraphs enumeration, 
    such as trees, paths, and matchings, 
    but the problem addresses the subgraph
    enumeration, such as enumeration of subgraphs that are trees.
    Their induced subgraph versions have not been studied well.
    One of few example is for chordless paths and cycles.
    Our motivation is to reduce 
    the time complexity close to  for each solution. 
    This type of optimal algorithms are proposed 
    many subgraph classes such as trees, and spanning trees. 
    Induced subtrees are fundamental object 
    thus it should be studied deeply 
    and there possibly exist some efficient algorithms. 
    Our algorithm utilizes nice properties of -degeneracy 
    to state an effective amortized analysis. 
    As a result, the time complexity is reduced to 
     time per induced subtree. 
    The problem is solved in constant time for each in
    planar graphs, as a corollary. 
\end{abstract}




\section{Introduction}


\name{Subgraph enumeration problems} are enumeration problems 
that given a graph  and a graph class , 
output all subgraphs  of  satisfying  without duplicates. 
Subgraph enumeration problems are widely studied~\cite{Avis:Fukuda:DAM:1996,Ferreira:Grossi:Rizzi:ESA:2011,Shioura:Tamura:Uno:SIAM:1997,Tarjan:Read:1975,Wasa:COCOON:2012,Birmele:etal:SODA:2013,Uno:SIGAL:2003,Tarjan:1973}. 
Enumeration involves a huge number of solutions, 
thus enumeration algorithms are supposed to run in short time, 
with respect to the number of solutions . 
For example, if an algorithm runs in  time for small , 
other than preprocessing, 
we can consider the algorithm is efficient. 
In this case, 
we say that the algorithm runs in  time per solution, 
or  time for each solution. 
Further, 
the maximum computation time between two consecutive outputs called \name{delay} is 
also considered as a more efficiency of enumeration algorithms. 
Note that delay will not be  
even if an algorithm runs in  time per solution. 

Enumeration algorithms are widely studied in these days. 
Especially, 
the data mining area has a large amount of studies on pattern mining problem. 
The algorithms have to deal with huge databases and a huge number of solutions, 
thus there are great needs of the algorithm theory on efficient enumeration. 
As we show below, 
many recent studies focus on the development of small complexity algorithms. 
Compared to other algorithms, 
enumeration algorithms have some unique aspects. 
For example, 
by operating only on the differences between the solutions, 
one can develop algorithms that 
run in time shorter than the amount of exact output. 
Other than this, 
since the recursion is much more structured compared to optimization, 
we can develop a non-trivial amortized analysis. 
As a consequent, 
researches on the numeration algorithms have great interests. 

In what follows, 
we fix the input graph , 
and let , . 
In the 1970s, 
Tarjan and Read~\cite{Tarjan:Read:1975} studied 
a problem of enumerating spanning trees in the input graph. 
Their algorithm runs in  time. 
Shioura, Tamura, and Uno~\cite{Shioura:Tamura:Uno:SIAM:1997} 
is improved the complexity to  time. 
Tarjan~\cite{Tarjan:1973} proposed an algorithm for enumerating all cycle 
in  time, 
where  is all cycle in . 
Birmel\'{e} \textit{et al.}~\cite{Birmele:etal:SODA:2013} 
improved the complexity to 
in  total time. 
They also presented an enumeration algorithm for
all st-paths in the input graph  
in  total time, 
where  is all st-paths in . 
Ferreira \textit{et al.}~\cite{Ferreira:Grossi:Rizzi:ESA:2011} proposed 
an enumeration algorithm 
that enumerating all subtree having exactly  edges in  in  time. 
Wasa \textit{et al.}~\cite{Wasa:COCOON:2012} presented 
an improved version of Ferreira \textit{et al.}'s problem in constant time delay 
when the input is a tree. 
As we see, 
speed up of enumeration algorithms have been intensively studied in long history. 


Compared to these studies, 
induced subgraph enumerations have not been studied well.
Avis and Fukuda~\cite{Avis:Fukuda:DAM:1996} considered 
the connected induced subgraph enumeration problem. 
Their algorithm is based on reverse search, 
and runs in  time.
Uno~\cite{Uno:SIGAL:2003} proposed 
an enumeration algorithm for enumerating all chordless path 
connecting the given vertices  and  
and all chordless cycle in  time. 

In this paper, 
we address the problem of enumerating all induced subtrees
in the given graph, 
where an induced subtree is a connected induced subgraph that has no cycle. 
Assume that the set of vertices in an induced subtree is . 
Then,  is a feedback vertex set of . 
Feedback vertices are also 
fundamental graph objects and their enumeration problem is 
equivalent to that of induced subtrees. 
If the input graph  is a tree, 
the connected induced subgraph of  is a subtree. 
Thus, 
Wasa \textit{et al.}'s shows that 
the induced subtree enumeration problem can be solved in constant time delay 
when the input graph is a tree. 
Tree is a simple graph class, 
so we are motivated whether we can do better
in more general graph classes with non-trivial algorithms.


As a main result of this paper, 
we propose an algorithm for the -degenerate graph case. 
The algorithm runs in  time per solution, 
after  preprocessing time. 
The algorithm starts from the empty subgraph,
and adds a vertex recursively to enlarge the induced subtree.
The vertex to be added has to be adjacent to the current induced subtree, 
and has not to make a cycle. 
By using the degeneracy, 
we efficiently maintain the addible vertices, 
and the time complexity is bounded by a sophisticated amortized analysis.
Real world graphs usually have small degeneracies, 
or only few vertex removals result small degeneracies, 
the algorithm is expected to be efficient in practice. 
Compared to other graph classes, 
this is a strong point of -degenerate graphs. 
There have been not so many studies on the use of the degeneracy for enumeration algorithm, 
and thus our approach introduces 
one of new way of developing practically efficient 
and theoretically supported algorithms.


The rest of this paper is organized as follows: 
In Section~\ref{sec:prelim}, 
we gives definitions in this paper 
and the definition of our problem. 
In Section~\ref{sec:algorithm}, 
we propose a basic enumeration algorithm based on a binary partition method. 
In Section~\ref{sec:time:complexity}, 
we improve the algorithm by using a property of the degeneracy, 
and analyze its time complexity. 
Finally, we conclude this paper and give future works in Section~\ref{sec:conc}. 


\section{Preliminaries}
\label{sec:prelim}

\subsection{Graphs}

Let  be an \name{undirected graph}, 
where  is the set of \name{vertices} and  is the set of \name{edges}. 
In this paper, we assume that  is simple and finite. 
We denote by  the edge connecting  and . 
For any vertices  of , 
we say that  and  are \name{adjacent} to each other if .  
We denote by  the set of all vertices adjacent to  in . 
We define the \name{degree}  of  in  as the number of vertices adjacent to . 
In what follows,
if it is clear from context, 
we omit the subscript . 

A \name{path} in  is 
a sequence of distinct vertices , 
such that  and  are adjacent to each other for . 
If there is  in , 
we say that the path \name{connects}  and . 
The \name{length} of path  is the number of vertices in  minus one. 
For any path  of length larger than one, 
 is called a \name{cycle} if . 
We say that  is \name{connected} 
if there is a path connecting any pair of vertices in . 
 is a \name{tree} if  has no cycle and is connected. 


\subsection{Induced subtrees}

Let  be a subset of . 
We denote by  the graph \name{induced} by , 
where . 
We call  an \name{induced subgraph} of . 
If no confusion, we regard  as .
 is the size of . 
We say that  is an \name{induced subtree} (see Fig.~\ref{img:induced:subtree}),  
if  is a tree.  
In the following, we state the problem of this paper. 

\spnewtheorem*{wproblem}{Problem}{\itshape}{\itshape}
\begin{wproblem}[Induced subtree enumeration problem]
    \label{prob:main}
    Enumerate all induced subtrees in  . 
\end{wproblem}



\begin{figure}[t]
    \begin{center}
        \includegraphics[width=10em]{./induced_subtree.eps}
        \caption{
            An induced subtree  in . 
            In the figure, bolded vertices and edges represent vertices and edges in . 
             consists of . 
             is an induced subtree in  since  is connected and acyclic. 
        }
        \label{img:induced:subtree}
    \end{center}
\end{figure}






\subsection{-degenerate graphs}

A graph  is \name{-degenerate}~\cite{Lick:White:CJM:1970}
if any its induced subgraph of  has 
a vertex whose degree is less than or equal to . 
The \name{degeneracy} of  is defined 
as the smallest  satisfying the definition of -degenerate graphs. 
Examples of graph classes with constant degeneracy include 
trees, grid graphs, outerplanar graphs, and planer graphs, 
thus degenerate graph is a large class of sparse graphs. 
These degeneracy are 1, 2, 2, and 5, respectively. 

From the definition of -degeneracy, 
we obtain a vertex sequence  
satisfying the condition    

This condition  implies that 
there exists an \name{ordering} among vertices of  
such that for any vertex , 
the number of vertices adjacent to  larger than it is at most . 
Hereafter we assume that the vertices 
are indexed in this ordering. 
We say  (, respectively) 
if the index of  is smaller than  ( is larger than , respectively) with respect to this ordering. 
In Fig.~\ref{img:k:degeneracy}, 
we show an example of the ordering
satisfying . 
Matula and Beck~\cite{Matula:Beck:1983} proposed 
an algorithm for obtaining the degeneracy of  and the ordering satisfying . 
By iteratively choosing the smallest degree vertex and removing it from ,  
their algorithm finds such an ordering in  time. 









\section{Basic Binary Partition Algorithm}
\label{sec:algorithm}


\subsection{Candidate Sets and Forbidden Sets}
\label{subsec:can}

Let  be an induced subtree of . 
We define the \name{adjacency} of a vertex  to  
as , 
that is,  is the number of vertices of  adjacent to . 

\begin{lemma}
    \label{lem:neighbor}
    Let  be any induced subtree in  and  be any vertex . 
     is an induced subtree if and only if . 
\end{lemma}

\begin{proof}
    If , 
     is adjacent to two vertices  and  of . 
    Since  has a path  connecting  and , 
    the addition of  yields a cycle in . 
    If ,  is disconnected. 
    If ,  is connected. 
    Since the degree of  in  is one, 
     is not included in a cycle. 
    Thus,  does not contain a cycle. 
    \qed
\end{proof}

In each iteration, 
we maintain the \name{forbidden set}  
as the vertex set such that any vertex  in  
satisfies either  belongs to , 
 includes a cycle, 
or  is forbidden to include in the solution by some ancestor iterations of the iteration.  
We also maintain the \name{candidate set}  
as the set of vertices whose additions yield induced subtrees and are not included in . 
We maintain  and  for efficient computation. 
From Lemma~\ref{lem:neighbor}, 
they are disjoint, 
and for any vertex , 
if , 
 belongs to either  or . 

\begin{figure}[!t]
    \begin{center}
        \includegraphics[width=28em]{./k_degeneracy.eps}
        \caption{
            An example of an ordering of . 
In the right graph, 
            vertices are sorted by the ordering that satisfies . 
\label{img:k:degeneracy}}
    \end{center}
\end{figure}


\subsection{Basic Binary Partition}
\label{subsec:algorithm}


Our algorithm starts from the empty induced subtree . 
In each iteration given an induced subtree , 
we remove a vertex  from , 
and partition the problem into two; 
enumeration of all induced subtrees including , 
and those including  but not including . 
We recursively do this partition 
until there is no vertex in . 
The former can be solved 
by a recursive call with setting  to . 
The latter is solved 
by a recursive call with setting  to . 
In this way, 
we can enumerate all induced subtrees. 
We present 
the main routine \MyProc{ISE} of our algorithm 
in Algorithm~\ref{alg:ISE}.  
We show how to update candidate sets and forbidden sets in the next two lemmas. 

\begin{lemma}
    \label{lem:can:update}
    For an induced subtree  and a vertex , 
    when we add  to  and remove  from , 
     changes to  
    
\end{lemma}

\begin{proof}
    Any vertex in  other than  
    remains in  after the addition of  to  
    since the adjacencies of the vertices do not change. 
    If vertices in  are added to , 
    they are in , or they make cycles 
    since they are adjacent to  and other vertices in .  
    The adjacency of any vertex in  is zero for , 
    and one for . 
Any vertex  satisfying  
    is either in  or . 
    Thus, the statement holds. 
    \qed
\end{proof}


\begin{lemma}
    \label{lem:X:update}
    For an induced subtree  and a vertex , 
    when we add  to  and remove  from , 
     changes to 
    
\end{lemma}

\begin{proof}
    Any vertex  remains in  for , 
    since  always holds. 
    From the definition of the forbidden set, 
     is in  for . 
    Further, 
    any vertex  in  makes cycles when they are added to , 
    since  holds. 
By adding  to , 
    no other vertex is forbidden to be added, 
    thus the statement holds. 
    \qed
\end{proof}


\begin{theorem}
    \label{thm:correctness}
    Algorithm \MyProc{ISE} enumerates all induced subtrees 
    in the input graph  without duplicates. 
\end{theorem}



\begin{algorithm}[t]
    \caption{Main routine \MyProc{ISE}: Enumerating all induced subtrees in }
    \label{alg:ISE}
    \begin{algorithmic}[1]
        \Procedure{ISE}{}
        \State \textbf{if}  
               \textbf{then} output ; 
               \textbf{return};  
        \label{alg:ISE:output}
        \State choose the smallest vertex  from  and remove  from ; 
        \label{alg:ISE:choose}
        \State call ; 
        \label{alg:ISE:call1}
        \State call ; 
        \label{alg:ISE:call2}
        \EndProcedure
    \end{algorithmic}
\end{algorithm}





\section{Improved Binary Partition Algorithm}
\label{sec:time:complexity} 

From Lemma~\ref{lem:can:update} and Lemma~\ref{lem:X:update}, 
we can easily see that 
the computation time of updating 
the candidate set and the forbidden set 
is  by checking all vertices adjacent to . 
However, in this way, 
we must check some vertices again and again. 
Specifically, 
let us assume  are consecutively added to , 
and  is adjacent to ,  and another vertex in . 
When we add  to , 
we check whether we can add  to the candidate set of . 
After generating , we check  again 
when we add  to . 
In order to avoid this redundant checking, 
we improve the way of updating the candidate set and the forbidden set 
by using the following set. 


\begin{definition}
    Suppose that  is a vertex of  for an induced subtree of . 
    We define a set  
    as follows: 
    
\end{definition}

\begin{lemma}
    \label{lem:mod:update:Can}
    Let  be an induced subtree of , 
     be the smallest in the candidate set  of , 
    and  be the forbidden set of . 
    Then, the following formula holds: 
    
    where .  
\end{lemma}

\begin{proof}
    Let  be the set of vertices larger than . 
    Since  is the smallest vertex in , 
    . 
    From the definition of  
    and  is the smallest in 
    , 
    , 
    where . 
This concludes the lemma.
    \qed
\end{proof}


In what follows, 
we use an 
adjacency lists for the sets , , and , 
so that a removal and the recover of the removed element 
can be done in  time, 
and the merge of two sets can be done in linear time of their sizes.


\begin{lemma}
    \label{lem:mod:update:X:neighbor}
    When we add a vertex  to , 
    the update of   for all vertices  is done in  time.
\end{lemma}

\begin{proof}
    To update, 
    it is suffice to remove  from  from all . 
    Thus, it takes  time.
    \qed
\end{proof}

\begin{lemma}
    \label{lem:mod:update:Can:X}
    Let  be an induced subtree of , 
     be the smallest in the candidate set  of , 
    and  be the forbidden set of .  
    When we add  to  and remove  from , 
    the computation time of updating  and  are
     and  time, respectively.  
\end{lemma}

\begin{proof}
    Since  is the smallest vertex in , 
    , 
    where . 
    Since vertices in  are sorted by the ordering, 
    the computation time of  is . 
    Thus, 
    adding vertices in  and  to  
    and removing  from  are done in  time. 
From Lemma~\ref{lem:mod:update:Can}, 
    since , 
    the computation time of adding these vertex to  is . 
Hence, the lemma holds. 
    \qed 
\end{proof}



\begin{figure}[t]
    \begin{center}
        \includegraphics[width=35em]{./candidate.eps}
        \caption{
            This figure shows the changes between candidate set  
            by the addition of  to . 
             is an induced subtree and 
             is the candidate set of . 
            Let assume that  and . 
            Since  does not belongs to , 
             is skipped checking. 
        }
        \label{img:candidate}
    \end{center}
\end{figure}


In Fig.~\ref{img:candidate}, 
we show the changes of between 
the candidate set of  and that of  
after adding  to . 
We implement  and   by doubly linked lists. 
Thanks to the doubly linked list, 
the cost for a deletion and a recover of a vertex 
can be done in constant time. 


\begin{theorem}
    \label{thm:complexity}
    Let  be the input graph and  is the degeneracy of . 
    Our algorithm enumerates all induced subtrees in  
    in  time per solution 
    after  preprocessing time without duplicates 
    using  space. 
\end{theorem}

\begin{proof}
    Since the update of  and  is correct, 
    the correctness of the algorithm is obvious. 
(I) We discuss the time complexity of the preprocessing. 
    First, our algorithm computes an ordering of vertices 
    by Matula and Beck's algorithm~\cite{Matula:Beck:1983} 
    in  time. 
    Next, our algorithm sorts 
    vertices belonging to each adjacency list by using a bucket sort.
    Thus, the preprocessing time is . 


    (II) 
    We consider an iteration inputting , , and , and assume
    that  is the candidate set for . 
    Line~\ref{alg:ISE:output} and line~\ref{alg:ISE:choose} run in  time. 
    From Lemma~\ref{lem:mod:update:X:neighbor}, 
    line~\ref{alg:ISE:call1} needs  time. 
From Lemma~\ref{lem:mod:update:Can:X}, 
    since it is clear that ,   
    our algorithm needs  time for computing  and . 
The update of 's is done in  time, from Lemma~\ref{lem:mod:update:X:neighbor}. 
    We observe that for each vertex  
    such that  is removed from , 
     is in  of , 
    that will be generated by a descendant of this iteration.
    We charge the cost of constant time to remove  from  
    to the induced subtree . 
    Then, we can see that  is charged only from iterations inputting , 
    that divides the problem by  such that , 
    that is, the iteration generates .
    We consider the average amount of the charge over all induced subtrees of , 
    , and  is in  of . 
    Since the number of pairs  is at most , 
    we can see the average charge is  for each . 
    Thus, in summary, we can see the update time for  in an iteration is bounded by , on average.
    Thus, an iteration takes  time on average.
We observe that the sum of  over all iterations is no greater 
    than the sum of  over all induced subtrees, 
    since  is the candidate set of  and forbidden set , 
    and  is generated only from . 
    Further, we can see that  is generated only from  
    this iteration.
    Hence, 
    thus the sum of  over all induced subtrees is bounded 
    by the number of induced subtrees. 
    Therefore, 
    the computation time for each iteration is bounded by  on average. 

    In a binary partition algorithm, 
    each iteration at the leaf of the recursion 
    outputs a solution, 
    and each non-leaf iteration generates exactly two recursive calls. 
    Thus, the number of iterations (recursive calls) of a binary partition algorithm 
    is at most . 
    Hence, the computation time per induced subtree is . 
    All all sets the algorithm maintains are of size  in total. 


    We need a bit care to perform a recursive call. 
    When a recursive call is made, 
    we record the operations 
    to prepare the parameters given to the recursive call on the memory. 
    When the recursive call ends, 
    we apply the inverse operations of the recorded operations 
    to recover the variables such as  and . 
    In this way, 
    we can recover the variables from the updated ones 
    without increasing the time complexity. 
    Since no vertex is added or deleted from the same variable twice, 
    the accumulated space for the recorded operations is bounded by . 
From the above arguments, 
    our algorithm runs in  time per solution 
    after  preprocessing time using  space. 
    \qed
\end{proof}



\section{Conclusion}
\label{sec:conc}

In this paper, 
we have presented an algorithm for enumerating all induced subtrees in -degenerate graph. 
Our algorithm runs in  time per solution after linear preprocessing time 
using linear space. 
From this result, 
we obtain the following corollary; 
if the input graph has a constant degeneracy, 
our algorithm is optimal with respect to the computation time per solution. 
-degenerate graphs often appear in real-world data 
even when with much noise. 
Thus considering the applications, 
it is important to study on efficient computation on -degeneracy. 
This result is one of the first steps for such studies, 
and researches on enumeration algorithms on -degenerate graphs will be an important issue. 




\subsection*{Acknowledgement}
This work was partially supported by 
MEXT Grant-in-Aid for Scientific Research (A)  
and Grant-in-Aid for JSPS Fellows . 

{\small
\begin{thebibliography}{99}

\bibitem{Avis:Fukuda:DAM:1996}
Avis, D. and Fukuda, K.:
\newblock Reverse search for enumeration.
\newblock {\em DAM}, 65:21--46, 1996.

\bibitem{Birmele:etal:SODA:2013}
Birmel\'{e}, E., Ferreira R.~A., Grossi R., Marino A., Pisanti N., Rizzi R., and Sacomoto G.: 
\newblock Optimal Listing of Cycles and st-Paths in Undirected Graphs. 
\newblock In {\em Proc.~SODA 2013}, 1884--1896, 2013.

\bibitem{Ferreira:Grossi:Rizzi:ESA:2011}
Ferreira, R. , Grossi, R. , and Rizzi, R.:
\newblock Output-sensitive listing of bounded-size trees in undirected graphs.
\newblock In {\em Proc.~ESA 2011}, LNCS~6942, 275--286, 2011.



\bibitem{Lick:White:CJM:1970}
Lick, D.~R. and White, A.~T: 
\newblock -DEGENERATE GRAPHS. 
\newblock {\em Can. J. Math.}, XXII(5): 1082--1096, 1970.



\bibitem{Matula:Beck:1983}
Matula, D.~W. and Beck, L.~L.:
\newblock Smallest-last ordering and clustering and graph coloring algorithms.
\newblock {\em J. ACM}, 30(3): 417--427, 1983. 

\bibitem{Shioura:Tamura:Uno:SIAM:1997}
Shioura, A., Tamura, A.,  and Uno, T.:
\newblock An optimal algorithm for scanning all spanning trees of undirected graphs.
\newblock {\em SIAM J. Comput.}, 26(3):678--692, 1997.

\bibitem{Tarjan:1973}
Tarjan, R.~E.: 
\newblock 
Enumeration of the Elementary Circuits of a Directed Graph. 
\newblock {\em SIAM J. Comput.}, 2(3): 211--216, 1973. 

\bibitem{Tarjan:Read:1975}
Tarjan, R.~E. and Read, R.~C. .:
\newblock Bounds on backtrack algorithms for listing cycles, paths, and spanning trees.
\newblock {\em Networks}, 5(3):237--252, 1975.



\bibitem{Uno:SIGAL:2003}
Uno, T.: 
\newblock An output linear time algorithm for enumerating chordless cycles.
\newblock {\em Technical Notes, 92nd SIGAL of IPSJ}, pages 47--53, 2003.
\newblock In Japanese.

\bibitem{Wasa:COCOON:2012}
Wasa, K., Kaneta, Y., Uno, T., and Arimura, H.:
\newblock Constant time enumeration of bounded-size subtrees in trees and its application.
\newblock In {\em Proc.~COCOON 2012}, LNCS~7434, 347--359, 2012.


\end{thebibliography}
}

\end{document}
