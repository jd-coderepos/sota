\documentclass[10pt]{IEEEtran}
\pdfoutput=1 
\usepackage{amsmath}
\usepackage{amsthm}
\usepackage{color}
\usepackage{graphicx}
\usepackage{calc}
\usepackage{import}
\usepackage{url}
\usepackage{wrapfig}
\usepackage[all]{xy}
\usepackage{enumerate}

\newcommand{\mline}[1]{\begin{array}{c}#1\end{array}}
\newcommand{\outputs}{\mathsf{outputs}}
\newcommand{\net}{\mathsf{net}}
\newcommand{\word}{\mathsf{word}}
\newcommand{\sent}[1]{\ensuremath{\mathtt{#1}}} \newcommand{\GG}{{\cal G}}
\newcommand\tuple[1]{\langle #1 \rangle}
\newcommand{\sset}[2]{\left\{~#1  \left|
      \begin{array}{l}#2\end{array}
    \right.     \right\}}

\newtheorem{lemma}{Lemma}
\newtheorem{definition}{Definition}
\newtheorem{theorem}{Theorem}
\newtheorem{conjecture}{Conjecture}
\newtheorem{proposition}{Proposition}
\newtheorem{example}{Example}
\newtheorem{corollary}{Corollary}

\newcommand{\comment}[1]{\footnote{!!! #1}}


\pagestyle{plain}

\begin{document}

\title{The Quest for Optimal Sorting Networks: Efficient Generation of
       Two-Layer Prefixes\thanks{Supported by 
       the Israel Science Foundation, grant 182/13 and by
       the Danish Council for Independent Research, Natural Sciences.}}
\author{\IEEEauthorblockN{Michael Codish\\}
\IEEEauthorblockA{Department of Computer Science\\
    Ben-Gurion University of the Negev\\
    PoB 653\\
    Beer-Sheva, Israel 84105\\smallskipamount]
  
  \noindent\hfill~\raisebox{-\height/2}{\makebox{\includegraphics{saturated-2c.pdf}}}
  ~\raisebox{-\height/2}{\makebox{\includegraphics{4-13.pdf}}}
  ~\raisebox{-\height/2}{\makebox{\includegraphics{4-24.pdf}}}\hspace*\fill\\
\end{theorem}

\begin{proof}
  Although this formulation is more general, the proof of case~ is the
  same as the first case of the proof of Lemma~8 of~\cite{DBLP:conf/lata/BundalaZ14}, and
  the proof of cases~,  and~ is the same as the second case of the same
  proof.



  For case~, assume that  includes the given pattern and let the channels
  corresponding to those in the pattern be , ,  and~.  Add a
  comparator between channels  and  to obtain a network  that
  includes the following pattern.\medskip

  \hfill\includegraphics{4-1324.pdf}\hspace*\fill\medskip
  
  For a given input  of , let ,  and  denote the values on
  channel  respectively at input, after layer~, and at output.
  Exchanging the values of  with  and  with  has the effect
  of exchanging  with  and  with .
  Then the output  can be obtained as an output of :
  \begin{itemize}
  \item if , then  is the output of  corresponding to input
    ;
  \item if , then  is the output of  corresponding to input 
    obtained from  by permuting  with ,  with , and
    maintaining the value on all other channels.
  \end{itemize}
  Therefore  is not saturated.

  For case~ the construction is the same, and the thesis follows by
  comparing  with .
\end{proof}


As it turns out, these are actually \emph{all} of the patterns that make
a comparator network with first layer  non-saturated. We formalize
this observation in the following theorem.
\begin{theorem}
  \label{thm:sat-thm}
  If  is a non-redundant two-layer network on  channels with first
  layer  containing none of the
  patterns in Theorem~\ref{thm:sat-char}, then  is saturated.
\end{theorem}

\begin{proof}
  Let  be a non-redundant two-layer comparator network, and assume that the second layer of 
  has at least two unused channels (otherwise there is nothing to prove).  If one
  of these channels were unused at layer~, then the network would contain the
  pattern~,  or~. Thus, by Theorem~\ref{thm:sat-char}
  necessarily the two channels are connected at layer~. Again from the same theorem,
  we know that they must be both min-channels or both max-channels (otherwise case~ applies)
  and the channels they are connected to at layer~ cannot be connected at
  layer~, otherwise the network would be redundant.

  There are eight different cases to consider.  We
  detail the cases where the two unused channels are max channels. Assume that the
  four relevant channels are adjacent. This does not lose generality, since a first-layer preserving permutation
  can always be applied to  to make this hold. Label the channels , , 
  and  from top to bottom (so  and  are comparators at
  layer~ and channels~ and~ are unused at layer~).
  Let  be the number of channels above~ and  be the number of channels below~.
  Adding a comparator to  yields  where  is a comparator at
  layer~.
  The four possibilities depend on whether channels~ and~ are min- or
  max-channels,
and are represented in Figure~\ref{fig:sat-thm}.

  \begin{figure}[t]
    \hfill
    (i)~\raisebox{-\height/2}{\makebox{\includegraphics{saturated-minmin.pdf}}}
    \hfill
    (ii)~\raisebox{-\height/2}{\makebox{\includegraphics{saturated-minmax.pdf}}}
    \hfill
    (iii)~\raisebox{-\height/2}{\makebox{\includegraphics{saturated-maxmin.pdf}}}
    \hfill
    (iv)~\raisebox{-\height/2}{\makebox{\includegraphics{saturated-maxmax.pdf}}}
    \hspace*\fill

    \caption{Possible cases for channels~ and~ in the proof of
      Theorem~\ref{thm:sat-thm}.  To obtain , add a comparator between channels~ and .}
    \label{fig:sat-thm}
  \end{figure}

  \begin{itemize}
  \item Figure~\ref{fig:sat-thm}~(i):  and  are min-channels at layer~.

    Consider the input string .  This is transformed to
     by , so .  We now show that
    .  In order to obtain the  on channel~, the
    input string would necessarily have a~ on channel~ because of the
    comparator~ at layer~.  But then the output would also have
    a~ on channel~, hence it could not be .

  \item Figure~\ref{fig:sat-thm}~(ii):  is a min-channel at layer~, and  is a max-channel.

    The argument is similar, but using the
    input string .  This is transformed to  by , so
    , and the same reasoning as above shows that
    .

  \item Figure~\ref{fig:sat-thm}~(iii):  is a max-channel at layer~, and  is a min-channel.

    Consider again the input string .  As before, this is
    transformed to  by , so , and we
    show that .  As before, to obtain the  on
    channel~ the input string would necessarily have a~ on channel~ because
    of the comparator~ at layer~.  Now this  is propagated upwards
    by the second-layer comparator at~, which means that the output has
    a~ on one of the first  channels, hence it cannot be .

  \item Figure~\ref{fig:sat-thm}~(iv):  and~ are both max-channels at layer~.

    The reasoning is a bit more involved.
    Consider once more the input string .  Since channel~ is a
    second-layer max-channel connected w.l.o.g.~to a channel , the output produced by~ is
    . In
    order to obtain this output with network~, as before it is necessary to
    have inputs  on channels~ and~; but since there are only two s in
    the output, this means that channel~ must also be connected to channel 
    on layer~, which is impossible.
  \end{itemize}

\noindent  The cases where  and  are the unused (min) channels are similar.
\end{proof}


We believe the following generalization to hold.

\begin{conjecture}\label{conjecture}
  If the two-layer networks  and  on  channels are both saturated and
  non-equivalent, then .
\end{conjecture}

Particular cases of Conjecture~\ref{conjecture} are implied by
Theorem~\ref{thm:sat-thm}, but the general case remains open. The
conjecture has been verified experimentally for .


\section{Case studies:  and }
\label{sec:example}

This section provides a detailed analysis for the cases of
four-channel two-layer networks with first layer  and
five-channel two-layer networks with first layer .
Consider the following strategy to enumerate all possible
second layers: channel  may be connected to channels~, 
or~, or may be unused; if channel~ is connected to channel~,
then channel~ may be connected to channel~ or may be unused;
etc.  With this strategy, the ten networks in Figure~\ref{fig:4wire} are generated in
the order .

\begin{figure}
\smallskip\emph{Redundant nets:}\smallskip

\fbox{~\raisebox{-\height/2}{\includegraphics{4-1234.pdf}}
}
\fbox{~\raisebox{-\height/2}{\includegraphics{4-12.pdf}}
  ~\raisebox{-\height/2}{\includegraphics{4-34.pdf}}
}

\smallskip\emph{Non-saturated nets:}\smallskip

\fbox{~\raisebox{-\height/2}{\includegraphics{4-13.pdf}}
}
\fbox{~\raisebox{-\height/2}{\includegraphics{4-14.pdf}}
  ~\raisebox{-\height/2}{\includegraphics{4-23.pdf}}
}

\smallskip
\fbox{~\raisebox{-\height/2}{\includegraphics{4-24.pdf}}
}
\fbox{~\raisebox{-\height/2}{\includegraphics{4-0.pdf}}
}

\smallskip\emph{Saturated nets:}\smallskip

\fbox{~\raisebox{-\height/2}{\includegraphics{4-1324.pdf}}
}
\fbox{~\raisebox{-\height/2}{\includegraphics{4-1423.pdf}}
}\bigskip

\caption{The  two-layer standard networks on four channels with the Parberry first layer .}
\label{fig:4wire}
\end{figure}

The boxes around the networks represent classes of
equivalent networks.  There are only two non-trivial equivalence
classes.
The equivalence between nets  and  follows since
the permutation  transforms them into one another.  For
nets  and , applying the same permutation to  yields a net
that has a generalized comparator in layer~; untangling it results in
.

The nets in the first row are all redundant, as they repeat a
comparator from the first layer; since the redundant comparators can
be removed without altering the set of outputs, they can be simplified to
net~.
The nets in the second row are not saturated; by
Theorem~\ref{thm:sat-char}, nets~ and~ are not saturated, and
their extension~ produces a subset of their outputs;
a similar situation arises with net~ vs net~, and net~
produces a superset of the outputs of both  and .  We detail the sets of
binary outputs for nets~,  and~, which correspond to
Case~ of Theorem~\ref{thm:sat-char}, one of the two cases missing
from the corresponding result in~\cite{DBLP:conf/lata/BundalaZ14}.




For  the situation is similar to .  The generation algorithm from Section~\ref{sec:pathrep} produces the
26~two-layer networks in Figure~\ref{fig:5wire} in the order
.

\begin{figure}

\smallskip\emph{Redundant nets:}\smallskip

\fbox{~\raisebox{-\height/2}{\includegraphics{5-1234.pdf}}
}
\fbox{~\raisebox{-\height/2}{\includegraphics{5-1235.pdf}}
  ~\raisebox{-\height/2}{\includegraphics{5-1534.pdf}}
}
\fbox{~\raisebox{-\height/2}{\includegraphics{5-1245.pdf}}
  ~\raisebox{-\height/2}{\includegraphics{5-2534.pdf}}
}

\smallskip
\fbox{~\raisebox{-\height/2}{\includegraphics{5-12.pdf}}
  ~\raisebox{-\height/2}{\includegraphics{5-34.pdf}}
}

\smallskip\emph{Non-saturated nets:}\smallskip

\fbox{~\raisebox{-\height/2}{\includegraphics{5-13.pdf}}
}
\fbox{~\raisebox{-\height/2}{\includegraphics{5-14.pdf}}
  ~\raisebox{-\height/2}{\includegraphics{5-23.pdf}}
}
\fbox{~\raisebox{-\height/2}{\includegraphics{5-15.pdf}}
  ~\raisebox{-\height/2}{\includegraphics{5-35.pdf}}
}

\smallskip
\fbox{~\raisebox{-\height/2}{\includegraphics{5-24.pdf}}
}
\fbox{~\raisebox{-\height/2}{\includegraphics{5-25.pdf}}
  ~\raisebox{-\height/2}{\includegraphics{5-45.pdf}}
}
\fbox{~\raisebox{-\height/2}{\includegraphics{5-0.pdf}}
}

\smallskip\emph{Saturated nets:}\smallskip

\fbox{~\raisebox{-\height/2}{\includegraphics{5-1324.pdf}}
}
\fbox{~\raisebox{-\height/2}{\includegraphics{5-1325.pdf}}
  ~\raisebox{-\height/2}{\includegraphics{5-1345.pdf}}
}
\fbox{~\raisebox{-\height/2}{\includegraphics{5-1425.pdf}}
  ~\raisebox{-\height/2}{\includegraphics{5-2345.pdf}}
}

\smallskip
\fbox{~\raisebox{-\height/2}{\includegraphics{5-1423.pdf}}
}
\fbox{~\raisebox{-\height/2}{\includegraphics{5-1435.pdf}}
  ~\raisebox{-\height/2}{\includegraphics{5-1523.pdf}}
}
\fbox{~\raisebox{-\height/2}{\includegraphics{5-1524.pdf}}
  ~\raisebox{-\height/2}{\includegraphics{5-2435.pdf}}
}

  \caption{The  two-layer standard networks on five channels with the Parberry first layer .}
  \label{fig:5wire}
\end{figure}

As before, the boxes identify the equivalence classes, which again can all be
obtained by means of the permutation  and eventually reversing any
generalized comparators at the second-layer.  The first set of networks is
redundant, while the second set is not saturated by Theorem~\ref{thm:sat-char},
and once again it can easily be verified that each network in this group
contains a set of outputs that is a proper superset of a network in the third
group.  Furthermore, only one element from each box in the third group needs to be
considered.

Following the notation in~\cite{DBLP:conf/lata/BundalaZ14}, we denote
the total number of two-layer networks on  channels whose first
layer is  by ; 
the number of non-equivalent such networks (up to permutation of
channels) by ; 
and the corresponding values for saturated networks by  and
.  
From these analyses, we obtain , ,
; and , , , and
.  The values for , ,  and 
coincide with those in~\cite{DBLP:conf/lata/BundalaZ14}, whereas the
values we obtain for  and  coincide with those authors'
results after applying Lemma~\ref{lem:outputs} to eliminate representatives.
The difference in values in  and
 is probably due to an incomplete identification of the
equivalence classes (note that, for , case~ of
Theorem~\ref{thm:sat-char} is not necessary, so the notion of
saturated from \cite{DBLP:conf/lata/BundalaZ14} coincides with our
definition in the previous section).  The problem of computing the
equivalence classes \emph{efficiently} is the topic of the next
sections.


\section{Graph representation}

The results presented in \cite{DBLP:conf/lata/BundalaZ14} involve a
great deal of computational effort to identify permutations which
render various two-layer networks equivalent. Motivated by the
existence of sophisticated tools in the context of graph isomorphism,
we adopt a representation for comparator networks similar to the one
defined by Choi and Moon~\cite{DBLP:conf/gecco/ChoiM02}.
Let  be a comparator network on  channels.  The graph
representation of  is a directed and labeled graph, 
where each node in  corresponds to a comparator in  and
.  Let  denote the
comparator corresponding to a node . Then,  if
comparator  feeds into the comparator  in  and the
label  indicates if the channel from  to
 is the min or the max output of .  Note that the
number of channels cannot be inferred from the graph representation,
as unused channels are not represented.

Each node has at most two in-edges and at most two out-edges. Nodes
with less than two in-edges represent comparators that are connected
to the input channels of the network. Similarly, nodes with less than two
out-edges represent comparators which are connected to the output
channels. As such, if the graph contains  comparators, then the sum of
the in-degrees of the nodes and also the sum of the out-degrees of the
nodes is bounded by .

Clearly, graphs representing comparator networks are acyclic, and the
degrees of their vertices are bounded by~.  There is a strong
relationship between equivalence of comparator networks and
isomorphism of their corresponding graphs. Choi and
Moon~\cite{DBLP:conf/gecco/ChoiM02} state the following proposition,
which implies that the comparator network equivalence problem is
polynomially reduced to the bounded-valence graph isomorphism problem.

\begin{proposition}\label{proposition:isomorphism}
  Let  and  be -channel comparator networks. Then

\end{proposition}


\begin{example}
  The sorting networks~ and~ from Page~\pageref{ex:sn} are
  represented by the following graphs,
  which can be seen to be isomorphic by mapping the vertices as
, , , ,  and
.

{\small 


}
\end{example}

The graph isomorphism problem is one of a very small number of
problems belonging to NP, for which it is neither known that they are solvable in
polynomial time nor that they are NP-complete.  However, it is known that
the isomorphism of graphs of bounded valence (here: bounded degree) can be tested in
polynomial time~\cite{DBLP:journals/jcss/Luks82}, so the comparator
network equivalence problem can be efficiently solved.

An obvious approach for finding all two-layer prefixes modulo symmetry
is to generate all two-layer networks as demonstrated in
Section~\ref{sec:example}, and then apply graph isomorphism checking to find
canonical representatives of the equivalence classes. 
We evaluated this approach using the popular graph isomorphism tool
\verb!nauty!~\cite{DBLP:journals/jsc/McKayP14}, but found that the
exponential growth in the number of two-layer prefixes prevents this
approach from scaling.

Instead of a generate-and-test approach, in the next section we
present a scalable method for directly generating only one
representative two-layer prefix per equivalence class. Furthermore,
this approach also enables us to encode saturation as a syntactic
criterion in the generation process, i.e., to generate directly only
representatives of saturated two-layer prefixes.



\section{Path representation of two-layer networks}
\label{sec:pathrep}

In this section, we focus on two-layer networks where the first layer
is maximal (although not necessarily ).  These networks can be
uniquely represented in terms of the paths in their graph
representations. Furthermore, this representation can be read directly
from the network, and can be used to construct a canonical
representation of the network that completely characterizes the
equivalence classes in the generated graphs.  In the following, recall
that channels of a network are characterized as \emph{free},
\emph{min} or \emph{max} depending on the first layer.


\begin{definition}\label{def:path}
  A \emph{path} in a two-layer network  is a sequence
   of distinct channels such that each pair
  of consecutive channels is connected by a comparator in .
The \emph{word} corresponding to  is
  , where  is \sent{0}, \sent{1} or \sent{2}
  according to whether  is the free channel, a min channel or a
  max channel, respectively.
\end{definition}

A path is maximal if it is a simple path (with no repeated nodes) that
cannot be extended (in either direction).  A network is connected if
its graph representation is connected.

\begin{definition}
  Let  be a connected two-layer network on  channels.  Then
   is defined as follows.

  \begin{description}
  \item[Head]If  is odd, then  is the word
    corresponding to the maximal path in  starting with
    the (unique) free channel. 
\item[Stick]If  is even and  has two channels not used
    in layer~, then there are exactly two maximal paths in  starting
    with a free channel (which are reverse to one another), and
     is the lexicographically smallest of
    the words corresponding these two paths.
  \item[Cycle]If  is even and all channels are used by a
    comparator in layer , then  is obtained by removing
    the last letter from the lexicographically
    smallest word corresponding to a maximal path in  that
    begins with two channels connected in layer~.
  \end{description}
\end{definition}



\begin{example}
  \label{ex:words}
  Below are three connected networks, (), (), and (), with their
  maximal paths, pictured as (), (), and (), marked in bold.
  For instance, () corresponds to the path .

  \noindent
  \hspace*\fill
  ~\raisebox{-\height/2}{\includegraphics{words-C.pdf}}
  \hfill
  ~\raisebox{-\height/2}{\includegraphics{words-A.pdf}}
  \hfill
  ~\raisebox{-\height/2}{\includegraphics{words-E.pdf}}
  \hspace*\fill\
  \mathsf{Word} &::= \mathsf{Head} \mid \mathsf{Stick} \mid \mathsf{Cycle}\\
  \mathsf{Head} &::= \sent{0}(\sent{12}+\sent{21})^\ast\\
   \mathsf{Stick} &::= (\sent{12}+\sent{21})^+ \\
   \mathsf{Cycle} &::= \sent{12}(\sent{12}+\sent{21})^\ast(\sent{1}+\sent{2})
\begin{array}{c|r|r|r|r|r|r|r|r|r|r|r|r}
n & \multicolumn1{c|}{3}
 & \multicolumn1{c|}{4}
 & \multicolumn1{c|}{5}
 & \multicolumn1{c|}{6}
 & \multicolumn1{c|}{7}
 & \multicolumn1{c|}{8}
 & \multicolumn1{c|}{9}
 & \multicolumn1{c|}{10}
 & \multicolumn1{c|}{11}
 & \multicolumn1{c|}{12}
 & \multicolumn1{c|}{13}
 & \multicolumn1{c}{14}  \\ \hline
|G_n| & 4 & 10 & 26 & 76 & 232 & 764 & 2{,}620 & 9{,}496 & 35{,}696 & 140{,}152 & 568{,}504 & 2{,}390{,}480 \\
|S_n| & 2 & 4 & 10 & 28 & 70 & 230 & 676 & 2{,}456 & 7{,}916 & 31{,}374 & 109{,}856 & 467{,}716 \\
|R(G_n)| & 4 & 8 & 16 & 20 & 52 & 61 & 165 & 152 & 482 & 414 & 1{,}378 & 1{,}024 \\
|R(S_n)| & 2 & 2 & 6 & 6 & 14 & 15 & 37 & 27 & 88 & 70 & 212 & 136 \\ 
|R_n| & 1 & 2 & 4 & 5 & 8 & 12 & 22 & 21 & 48 & 50 & 117 & 94 \\ 
\hline
\end{array}\begin{array}{c|r|r|r|r|r}
n
 & \multicolumn1{c|}{15}
 & \multicolumn1{c|}{16}
 & \multicolumn1{c|}{17}
 & \multicolumn1{c|}{18}
 & \multicolumn1{c}{19} \\ \hline
|G_n| & 10{,}349{,}536 & 46{,}206{,}736 & 211{,}799{,}312 & 997{,}313{,}824 & 4{,}809{,}701{,}440 \\
|S_n| & 1{,}759{,}422 & 7{,}968{,}204 & 31{,}922{,}840 & 152{,}664{,}200 & 646{,}888{,}154\\
|R(G_n)| & 3{,}780 & 2{,}627 & 10{,}187 & 6{,}422 & 26{,}796 \\
|R(S_n)| & 494 & 323 & 1{,}149 & 651 & 2{,}632 \\ 
|R_n| & 262 & 211 & 609 & 411 & 1{,}367 \\ 
\hline
\end{array}\begin{array}{c|r|r|r|r|r|r|r|r|r|r|r|r}
n
 & \multicolumn1{c|}{20}
 & \multicolumn1{c|}{21}
 & \multicolumn1{c|}{22}
 & \multicolumn1{c|}{23}
 & \multicolumn1{c|}{24}
 & \multicolumn1{c|}{25}
 & \multicolumn1{c|}{26}
 & \multicolumn1{c|}{27}
 & \multicolumn1{c|}{28}
 & \multicolumn1{c|}{29}
 & \multicolumn1{c|}{30}
 & \multicolumn1{c}{31}
 \\ \hline
|R(S_n)| & 1{,}478 & 5{,}988 & 3{,}040 & 13{,}514 & 6{,}744 & 30{,}312 & 14{,}036 & 67{,}638 & 30{,}552 & 150{,}128 & 64{,}168 & 331{,}970 \\
|R_n| & 894 & 3{,}098 & 1{,}787 & 6{,}920 & 3{,}848 & 15{,}469 & 7{,}830 & 34{,}318 & 16{,}690 & 75{,}979 & 34{,}486 & 167{,}472 \\ \hline
\end{array}\begin{array}{c|r|r|r|r|r|r|r|r|r}
n
 & \multicolumn1{c|}{32}
 & \multicolumn1{c|}{33}
 & \multicolumn1{c|}{34}
 & \multicolumn1{c|}{35}
 & \multicolumn1{c|}{36}
 & \multicolumn1{c|}{37}
 & \multicolumn1{c|}{38}
 & \multicolumn1{c|}{39}
 & \multicolumn1{c}{40} \\ \hline
|R(S_n)| & 138{,}122 & 731{,}000 & 291{,}090 & 1{,}604{,}790 & 622{,}136 & 3{,}511{,}250 & 1{,}313{,}262 & 7{,}663{,}112 & 2{,}792{,}966 \\
|R_n| & 73{,}191 & 368{,}143 & 152{,}503 & 806{,}710 & 322{,}891 & 1{,}763{,}133 & 676{,}431 & 3{,}843{,}848 & 1{,}429{,}836 \\ \hline
\end{array}
  \mathsf{Word} &::= \mathsf{Head} \mid \mathsf{Stick} \mid \mathsf{Cycle}\\
  \mathsf{Stick} &::= \sent{12} \mid \mathsf{eStick} \mid \mathsf{oStick} \\
  \mathsf{Head} &::= \sent0 \mid \mathsf{eHead} \mid \mathsf{oHead}\\
  \mathsf{eStick} &::= \sent{12}(\sent{12}+\sent{21})^+\sent{21} \\
  \mathsf{eHead} &::= \sent0(\sent{12}+\sent{21})^\ast\sent{12}\\
  \mathsf{oStick} &::= \sent{21}(\sent{12}+\sent{21})^+\sent{12} \\
  \mathsf{oHead} &::= \sent0(\sent{12}+\sent{21})^\ast21\\
  \mathsf{Cycle} &::= \sent{12}(\sent{12}+\sent{21})^\ast(\sent{1}+\sent{2})


Furthermore, sentences are multi-sets  such that:
(i)~if  contains the words \sent0\ or \sent{12}, then all other elements of
 are cycles;
(ii)~if  contains an  or , then it contains
no  or .
With these restrictions, generating all saturated networks for 
can be done almost instantaneously.  The numbers  of
saturated two-layer networks and  of equivalence classes
modulo permutation are given in the first four lines in the table of Figure~\ref{fig:table}.

Bundala and Z{\'a}vodn{\'y} mention that the number of two-layer
networks could further be restricted by considering
reflections~\cite{DBLP:conf/lata/BundalaZ14} (with acknowledgement to
D.E.~Knuth).  The reflection of a comparator network on  channels is
the network obtained by replacing each comparator  by the comparator
; when the first layer is the set  of
comparators of the form , reflection leaves it unchanged.
Furthermore, they show that a two-layer network with first layer  can
be extended to a sorting network if, and only if, its reflection can be
extended to a sorting network, hence reflections can be removed from
 when searching for optimal depth sorting networks. 

Since the word representation is defined for any first layer, this symmetry
break can be encoded by a similar technique as the one applied for saturation.
By removing  and  from the above grammar,
we directly generate only the  representatives for -channel networks described
in~\cite{DBLP:conf/lata/BundalaZ14}.  Furthermore, it is
possible to have distinct cycles whose reflections are equivalent
(but not equal); this brings the number
of relevant two-layer networks on  channels to~.  The last line in the table of
Figure~\ref{fig:table} above details the number  of representatives modulo equivalence
and reflection for each value of .  We can compute the set
 in less than one minute and  in approximately two hours.




Having computed , we can directly verify the known value  for the optimal depth of a 16-channel sorting network, obtained only indirectly in \cite{DBLP:conf/lata/BundalaZ14}.
This direct proof involves
showing that none of the 
two-layer comparator networks in  extends to a sorting network
of depth~. For this, we use an encoding to Boolean
satisfiablity (SAT) as described in~\cite{DBLP:conf/lata/BundalaZ14},
where for each network 
in , we generate a
formula 
that
is satisfiable if and only if there exists a
sorting network of depth 
extending .
Showing the unsatisfiability of these  SAT instances can be performed in
parallel, with the hardest instance (a CNF with approx.\  clauses) requiring approx.\  seconds running on a single thread of a cluster of Intel Xeon E5-2620 nodes clocked at ~GHz.



However, this approach does not directly work for , where the best known
upper bound is .
Attempting to show that there is no sorting network of depth 
requires analyzing the networks in
. The resulting  formulas have more than five
million clauses each, and none could be solved within a couple of weeks.
It appears that finding the optimal
depth of sorting networks with more than  channels is a hard
challenge that will require prefixes with more than  layers.

\section{Conclusion}

We presented an efficient technique to generate, modulo symmetry, the
set  of all saturated two-layer comparator networks on ~channels,
as well as its restriction  to exclude networks
that are equivalent modulo reflection.

As noted by Parberry in~1991 and again by Bundala and Z{\'a}vodn{\'y} in~2014,
computing  and  is a crucial step in the search for
optimal depth sorting networks on ~channels.
Using our approach we can compute  in under a second vs
~minutes using the brute force approach applied in \cite{DBLP:conf/lata/BundalaZ14},
and
improve the number of relevant two-layer prefixes to be considered
from  to  by eliminating
reflections.

In personal communication, Bundala and Z{\'a}vodny state that their
brute-force approach does not scale beyond . This is not
suprising, as there is an exponential growth of the number of
candidate networks, a quadratic number of subsumption tests between
the candidate networks, and, for each subsumption test, a factorial number
of permutations and an exponential number of inputs to consider.

The smallest open instance of the optimal depth sorting network
problem is for . We can easily compute  (in ~seconds)
as well as its restriction to networks modulo
reflection. This later set consists of only  networks and is a key
ingredient to solving this
problem,
effectively reducing the search space more than
-fold.

\bibliographystyle{abbrv}
\bibliography{paper}











\end{document}
