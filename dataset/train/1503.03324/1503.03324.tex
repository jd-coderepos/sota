\documentclass[a4paper]{tlp2}
\pdfoutput=1


\usepackage{ifthen}
\newboolean{commentsaon}         \setboolean{commentsaon}{true}
 \setboolean{commentsaon}{false}  

\newboolean{commentson} \setboolean{commentson}{true}


\newcommand{\comment}[1]
{\ifthenelse{\boolean{commentson}\AND\boolean{commentsaon}}
   {{\par\noindent\mbox{}{\small\blue[ *** #1 ]\par}\noindent\par}}{}}

\newcommand{\commenta}[1]
{\ifthenelse{\boolean{commentsaon}}
   {{\par\noindent\mbox{}{\small\color[rgb]{0, .4, 0}[ *** #1 ]\par}\noindent\par}}{}}


\usepackage{xspace}
\usepackage{latexsym}
\usepackage{amssymb}   


\usepackage{color}
\newcommand\blue     {\color{blue}}
\newcommand\red     {\color{red}}

\usepackage{hyperref}
\usepackage[hyphenbreaks]{breakurl}



\newtheorem{theorem}{Theorem}
\newtheorem{corollary}[theorem]{Corollary}
\newtheorem{lemma}[theorem]{Lemma}
\newtheorem{proposition}[theorem]{Proposition}
\newtheorem{example}[theorem]{Example}
\newtheorem{definition}[theorem]{Definition}
\newtheorem{remark}[theorem]{Remark}



\newcommand*{\seq}[2][n]  {{#2_{1}, \allowbreak \ldots, \allowbreak #2_{#1}}}
\newcommand*{\SEQ}[3]
            {{\ensuremath{#1_{#2}, \allowbreak \ldots, \allowbreak #1_{#3}}}}

\newcommand*{\notmodels}{\mathrel{\,\not\!\models}}
\newcommand*{\partto}{\hookrightarrow}





\newcommand*{\mydash}{{\mbox{\tt-}}}
\newcommand*{\HU}{{\ensuremath{\cal H U}}\xspace}
\newcommand*{\HB}{{\ensuremath{\cal B}}\xspace}
\newcommand*{\M}{{\ensuremath{\cal M}}\xspace}
\usepackage[mathscr]{euscript} 
\newcommand*{\F}{{\ensuremath{\mathscr F}}\xspace}
\renewcommand*{\P}{{\ensuremath{\cal P}}\xspace}
\newcommand*{\X}{{\ensuremath{\cal X}}\xspace}

\newcommand*{\NN}{{\ensuremath{\mathbb{N}}}\xspace}

\newcommand*{\mylonger}[1]{\makebox[0pt][l]{#1}}

\usepackage[ddmmyyyy]{datetime}
\renewcommand{\dateseparator}{-}



\newcommand*{\mytoday}{{\rm\today}}



\begin{document}

\title 
[Program answers and least Herbrand models]
{On definite program answers \\ and least Herbrand models}



\author[W. Drabent]
{W{\l}odzimierz Drabent\\
         Institute of Computer Science,
         Polish Academy of Sciences,\\
         ul. Jana Kazimierza 5,
         01-248 Warszawa, Poland
         \\ and \\
Department of Computer and Information Science,
         Link\"oping University\\
         S -- 581\,83   Link\"oping, Sweden      \\
         \email{drabent\,{\it at}\/\,ipipan\,{\it dot}\/\,waw\,{\it dot}\/\,pl}
}



\submitted
{11-01-2016}









\maketitle

\begin{abstract}
A sufficient and necessary condition 
is given under which least Herbrand models exactly characterize the answers
of definite clause programs. 

To appear in Theory and Practice of Logic Programming (TPLP)
\end{abstract}

\begin{keywords}
logic programming, least Herbrand model, declarative semantics, function symbols
\end{keywords}



\section{Introduction}
The relation between answers of definite logic programs and their least Herbrand
models is not trivial.
In some cases the equivalence

does not hold
(where  is a definite program,  its least Herbrand model, and  a
query, i.e.\ a conjunction of atoms\footnote
{The semantics of non closed formulae is understood as usually
(see e.g.\  \cite{vanDalen,Apt-Prolog}),
so that  iff
, where  is an interpretation or a
theory,   a formula, and  its
universal closure.}).
So programs with the same least Herbrand model may have different sets of
answers. 
(By definition,  is an answer of  iff .)
For a simple counterexample \cite[Exercise 4.5]{Doets},
assume that the underlying language has only one function symbol, a constant .
Take a program .  
Now  but .
This counterexample can be in a natural way generalized for any  finite set
of function symbols,
see the comment following the proof of Prop.\,\ref{prop:counterexample}.



Equivalence (\ref{eq}) holds for ground queries
(\citeNP{Lloyd87}, Th.\,6.6;\,\,\citeNP{Apt-Prolog}, Th.\,4.30).
For a possibly nonground  (and a finite )
a sufficient condition for (\ref{eq})
is that there are infinitely many constants in the
underlying language
(\citeNP
{DBLP:books/mk/minker88/Maher88};  \citeNP[Corollary 4.39]{Apt-Prolog}).
\citeN{DBLP:books/mk/minker88/Maher88} states without proof
that instead of an infinite supply of constants it is sufficient that there is
a non constant function symbol not occurring in . 
The author is not aware of any proof of this property
(except for \cite[Appendix]{drabent.arxiv.coco14}).

This paper presents a more general sufficient condition,
and shows that the condition is also a necessary one.
To obtain the sufficient condition, we show a property of (possibly
nonground) atoms containing symbols not occurring in a program . 
Namely, when such atom is true in  then, under certain conditions, a
certain more general atom is a logical consequence of .
As an initial step, we obtain
a generalization of the theorem on constants \cite{shoenfield67},
for a restricted class of theories, namely definite clause programs.
We also give an alternative proof for the original theorem.









\paragraph{Related problem.}

This paper studies (in)equivalence of
two views at the declarative semantics of definite
clause programs.  One of them considers answers true in the least Herbrand
models of programs, the other -- answers that are logical consequences of
programs.  
 


The subject of this paper should be compared with a related issue
(which is outside of the scope of this paper).
There exists (in)equivalence between the declarative
semantics and the operational one, given by SLD-resolution.  
  As possibly first pointed in
(\citeNP{DM87};\,\,\citeyearNP{DM88}),
two logically equivalent programs 
(i.e.\ with the same models, and thus the same logical consequences)
may have different sets of SLD-computed answers for the same query.
For instance take
, and 

Then for a query  program  gives two distinct computed answers,
and  one.  This phenomenon gave rise to the {\em s-semantics}, 
see e.g.\ \cite{DBLP:journals/tcs/Bossi09} for overview and references.





\paragraph{Preliminaries.}

We consider definite clause logic programs.  A query is a conjunction of atoms.
A query   is an {\em answer} (or a {\em correct answer}) of a program   
iff .
\citeN{Apt-Prolog} calls it a correct instance (of some query).
We do not need to refer to SLD-computed answers, as
each computed answer is an
answer, and each answer is a computed answer for some query,
by soundness and completeness of SLD-resolution.
Similarly, we do not need to consider to which query  a given query is
an answer.



The Herbrand universe
(for the alphabet of function symbols of the underlying language) 
will be denoted by \HU, and the least Herbrand model of a program  by .
Remember that  depends on the underlying language.
We require .
Names of variables will begin with an upper-case letter.
Otherwise we use the standard definitions and notation of \cite{Apt-Prolog},
including the list notation of Prolog.
(However in discussing the semantics of first order formulae we use a
standard term ``variable assignment'' instead of ``state'' 
used in \cite{Apt-Prolog}.)



The paper is organized as follows.  The next section presents some 
necessary definitions.
Section \ref{sec:lemma} 
shows how existence of answers containing symbols not occurring in the program
implies existence of more general answers.  The main result of this section
is compared with theorem on constants \cite{shoenfield67}.
Section \ref{sec:main}
contains the central technical lemma of this paper.
Section \ref{sec:H}
studies when the least Herbrand models provide an exact characterization of
program answers.  A new sufficient condition for equivalence
(\ref{eq}) is presented, and it is shown in which sense the condition is a
necessary one.





\section{Definitions}



This section introduces three notions needed further on.
Let  be the set of function symbols of the underlying language;
let .  An {\em alien} w.r.t.\   
is a non-variable term with its main function symbol from .
An alien w.r.t.\ a theory  (for instance a program)
means an alien w.r.t.\ the set of function symbols occurring in .
An occurrence of an alien  (w.r.t.\ , in an atom or substitution)
will be called a {\em maximal alien} if the occurrence is not within
an alien .



By a generalization of a query we mean the result of systematic replacement
of maximal aliens in the query by new variables.  More formally, 
let \P be a theory or a set of function symbols.
Let the maximal aliens of a query  w.r.t.\ 
be the occurrences in  of distinct terms .
Let  be distinct variables not occurring in .
Let a query  be
obtained from  by replacing (each occurrence of) 
 by , for .  
(So .)
Such  will be called
{\em  generalized} for .
We will also call it a/the {\em generalization} of  (for \P).
Note that it is unique up to variable renaming. 

\begin{example}
The standard append program APPEND \cite[p.\,127]{Apt-Prolog}
contains two function symbols   and .
Terms  are aliens w.r.t.\ APPEND, term  is not.
Maximal aliens in 
are the first and the last occurrences of  and the (single) occurrences of
  and .
Atom 
is  generalized for APPEND.
\end{example}

Let  be a query not containing aliens w.r.t.\ \P, and 
 be a substitution such that .
Then  is a generalization of   for \P (and for )
    iff  
    where  are distinct aliens w.r.t.\ \P.



 The correspondence between a ground atom and its generalization
 is described, in other terms, in
 \cite[Def.\,4]{naish.tplp.floundering14}.
It is used in that paper to represent nonground atoms by
 ground ones, in analysis of floundering in the context of delays.


\section{On program answers and aliens}
\label{sec:lemma}



Given a query containing aliens which is an answer of a program ,
this section shows which more general queries are answers of .
The main result (Lemma \ref{lemma:alien:substitution})
is compared with theorem on constants,
used by \cite{DBLP:books/mk/minker88/Maher88} to prove 
equivalence (\ref{eq}) for a case with an infinite alphabet of constants.

It is rather obvious that answers containing aliens can be generalized.
Assume that a query
  is an answer of , and that  contains aliens w.r.t.\ . 
Then  is
a proper instance of some computed answer .
It is however not obvious which replacements of aliens in  by variables 
result in answers.


\begin{example}
By replacing aliens w.r.t.\  by variables in an answer , 
we obtain some queries which are answers of ,
and some which are not.
Let  and .
So .
Now  and  are not answers of , 
but ,   and   are.
\end{example}



\begin{lemma}
\label{lemma:alien:substitution}  
Let  be a program,  a query, and
  be a substitution where
  are distinct aliens w.r.t.\ .  Then

\end{lemma}



Note that terms  may be nonground 
(and may contain variables from ), some
 may be unifiable, or contain common variables,
 may contain variables other than 
and may contain aliens w.r.t. .
So  is not necessarily a generalization of  for ,
but it is one for .


\begin{example}
In the previous example, the cases in which the more general atom is an
answer of  satisfy conditions of Lemma \ref{lemma:alien:substitution},
and the remaining ones do not.

\end{example}

\begin{proof}[Proof (Lemma \ref{lemma:alien:substitution})]


Without loss of generality assume that variables  occur in .
Let   be the remaining variables of .
The ``only if'' case is obvious.

Assume .  
By completeness of SLD-resolution,
 is an instance of some computed answer  for  and :
.
Each function symbol occurring in  occurs in  or .
Moreover (for ) 
and the main symbol of  does not occur in ;
hence  is a variable.
As  are distinct, variables  are
distinct. 
  Similarly,   for , thus
 
  are distinct variables.
Thus  is a variant of  and, by soundness of
  SLD-resolution, .
\end{proof}





\begin{corollary}
\label{th:generalized}
\label{cor:generalized}
Let  be a program,  a query, and  be  generalized for .
Then  iff  .
\end{corollary}

\begin{proof}
 for a certain  .
The premises of Lemma \ref{lemma:alien:substitution}  
are satisfied by , , and 
(as  are aliens w.r.t.\ , but also w.r.t.\ ).
\end{proof}

\begin{example}
Consider again program APPEND.  Assume that the underlying language 
has more function symbols than those occurring in the program, 
i.e.\  ,  .
Assume that we know that the least Herbrand model 
contains an atom
  ,
where  are distinct aliens w.r.t.\ APPEND.
Note that , as equivalence (\ref{eq}) holds for ground queries.
{\sloppy\par}



By Corollary \ref{cor:generalized},
  ,
where  are distinct variables.
Hence, for any terms ,
\mbox{}.
{\sloppy\par}
\end{example}




\begin{example}
 Consider the map colouring program  \cite[Program 14.4]{Sterling-Shapiro}.
We skip any details, let us only mention that the names of colours and
 countries do not occur in the program.
 (The function symbols occurring in the program are 
 .)
By Corollary  \ref{cor:generalized},
 for any answer  of the program,
 the generalization  of  w.r.t.\  is an
 answer of the program.  So is each instance of .  
 Thus systematic replacing (some) names of colours or countries in  by
 other terms results in a query  which is an answer of the program.\footnote{Thus it is possible that neighbouring countries get the same colour.
This does not mean that the program is incorrect.
Its main predicate {\it color\_map}
describes a correct map colouring provided that its second argument
is a list of distinct colours.
} 
\end{example}











The proof of equivalence (\ref{eq})
for an infinite set of constants of 
\cite[proof of Prop.\,6]{DBLP:books/mk/minker88/Maher88} employs a so called
theorem on constants  \cite{shoenfield67},
see also free constant theorem in \cite[p.\,56]{HandbookLAILP:FOL}.
The theorem states that (\ref{th:alien:substitution1}) holds for an arbitrary
theory  and formula , when the distinct aliens  are constants.
Its proofs in \cite{shoenfield67,HandbookLAILP:FOL} are syntactical, but a
rather simple semantic proof is possible:



Let  be the set of function and predicate symbols from , let 
\X be the set of the free variables of .
Notice that
for any interpretation  (for ) and any variable assignment 
(for \X)
there exists a variable assignment 
(for )
and an interpretation  
(for ) such that
 for each ,
  for each , and all the symbols
of  have the same interpretation in  and .  
Thus  iff , and 
 iff .
Conversely, for each interpretation  for 
and variable assignment  for 
there exist  as above. (In particular, the two equivalences hold.)
Now the theorem follows:
\smallskip

\begin{tabular}{l}
 iff  \\
for every  (as above)  implies 
iff \\
for every  (as above)  implies
 iff \\
.
\end{tabular}
\smallskip




\citeN[p.\,634]{DBLP:books/mk/minker88/Maher88}
states that ``The same effect 
[as adding new constants] could be obtained with one new function symbol (of
arity ) to obtain new ground terms with new outermost function symbol.''
This idea does not apply to the proof of the previous paragraph;
when  are such terms then the proof fails.\footnote{Informally, this is because such new terms cannot be interpreted
  independently, in contrast to  new constants.
Sometimes no interpretation for the new symbol  is possible,
  such that   are interpreted as a given  values.
  For instance take  for .
  Then for any interpretation for ,
  if  have the same value then all  also have the same value.
} So do the proofs of  \cite{shoenfield67,HandbookLAILP:FOL}.
In the context of \cite{shoenfield67} -- first order logic with equality -- the
generalization of the theorem on constants to terms with a
new outermost symbol does not hold.
For a counterexample, note that 
 \ but \ 
.
The generalization in Lemma \ref{lemma:alien:substitution} is
sound and has a simple proof, due to restriction to 
definite programs and queries.


From Lemma \ref{lemma:alien:substitution} it follows that equivalence (\ref{eq})
holds whenever the underlying language has a non constant function symbol 
(or a sufficient number of constants) not occurring in .\footnote{Assume that   are the variables of , and that there exist
    distinct ground terms  with their main symbols
not occurring in .
    Let  .
Assume , so , 
    and  as  is ground.
By Lemma \ref{lemma:alien:substitution}, . 
} (See also \cite[Appendix]{drabent.arxiv.coco14} for a direct proof.)
We however aim for a more general sufficient condition for (\ref{eq}),
allowing  to occur in ;
in this case Lemma \ref{lemma:alien:substitution} is not applicable.



\section{Least Herbrand models and program answers}
\label{sec:main}



This section shows conditions under which
truth in  of a query with aliens implies 
that a certain more general query is an answer of .
This is a central technical result of this paper (Lemma \ref{lemma:MP}).  
From it,
the sufficient conditions for equivalence (\ref{eq}) follow rather
straightforwardly, as shown in the next section.
We begin with proving an auxiliary property, by means the two following lemmas.







\begin{lemma}
\label{lemma:unifier}
Two distinct terms have at most one unifier of the form  where 
is not a variable.  
\end{lemma}

\begin{proof}


Let ,   be distinct substitutions,
where neither of  is a variable.
We show that if  then  , 
for any distinct terms .
The proof is
by induction on the sum  of the sizes of .
(Any notion of term size would do, providing that  whenever  is
a proper subterm of .)
Assume that the
property holds for each   such that .

Let  and .
Notice that at most one of   is a variable.
(Otherwise  are  -- two distinct variables,
 or exactly one of   is a variable, contradiction.)
Assume that exactly one of , say , is a variable.
Then  (as ), so
~does not occur in  (as  are unifiable),
hence .
Now if  then  which is distinct from any instance of
.  Otherwise  , hence .

If both  are not variables then
, for .  For some~, 
and .
By the inductive assumption, 
;
thus .
\end{proof}



\begin{lemma}
\label{lemma:distinct}
let \P be a theory or a set of function symbols.
Let  be a sequence of distinct terms, where  
() are variables, and   are aliens w.r.t.~.
Assume that if   are ground then there exist ground
aliens  w.r.t.\ , pairwise distinct from  .
Then the sequence has
a ground instance  consisting of  distinct aliens
w.r.t.\ . 
\end{lemma}
\nopagebreak

\begin{proof}
Consider first the case of  ground.
Then 
is a substitution providing the required instance. 
{\sloppy\par}


Let some  () be nonground. 
Its main symbol, say , is a non-constant function symbol not
occurring in . Thus the set  of ground aliens w.r.t.\  is infinite.
\pagebreak[3]

Let  be the variables occurring in  .
For some  substitution  is not a unifier of any
pair  (),
as by  Lemma \ref{lemma:unifier} 
each such pair has at most one unifier of the form , .
Thus  is a sequence of  distinct terms.
Applying this step repetitively we obtain the required sequence
 of distinct ground terms.
\end{proof}





\begin{lemma}
\label{lemma:MP}
Let  be a program,
 an atom, and   be  generalized for .
If
\begin{enumerate}[(a)]
\item 
\label{lemma:MP:condition1}
the underlying language has a non-constant function symbol not occurring in~P,
or 

\item 
\label{lemma:MP:condition2}
 contains exactly  (distinct) variables, and
the underlying language has (at least)  constants not occurring in ,
\nopagebreak
\end{enumerate}
\nopagebreak
then  iff .
\end{lemma}



\begin{proof}
Note that  where 
,   are the variables 
of  not occurring in , and  are the maximal aliens in  
(precisely: the distinct terms whose occurrences in  are the maximal aliens
w.r.t.\ ).
Let  be the variables occurring in .


We construct a ground instance  of , such that 
 is  generalized for .
To apply Lemma \ref{lemma:distinct} to terms ,
note that if  are ground then there exist  ground aliens
w.r.t.\  pairwise distinct from .
(They are either the constants from condition \ref{lemma:MP:condition2}, 
or can be taken from the infinite set of ground aliens w.r.t.\  with the main
 symbol from condition  \ref{lemma:MP:condition1}.)
By Lemma \ref{lemma:distinct}, 
there exists a ground instance  , 
consisting of  distinct aliens w.r.t.\ ,
where the domain of  is .

Note that
.
The substitution
maps variables  to distinct aliens 
 w.r.t.\ .
So  is  generalized for .
Thus  and  satisfy the conditions of 
Corollary~\ref{cor:generalized}.


Now  implies  and then
 
(as equivalence (\ref{eq}) from Introduction holds for ground queries).
As , by
Corollary \ref{cor:generalized} .
The ``if'' case is obvious, as  is an instance of .
\end{proof}


















\begin{remark}
  The premises of Lemma \ref{lemma:MP} can be weakened by
stating that  are atoms such that  for a substitution
,
where  are distinct aliens w.r.t.\ , and variables
 do not occur in .
\end{remark}

\begin{proof}
Obtained by minor modifications of the proof above.
The first sentence, describing , is to be dropped.
Each ``w.r.t.\ '' is to be changed to ``w.r.t.\ ''.
In the third paragraph, substitution  together with  and
 satisfy the condition of Lemma \ref{lemma:alien:substitution}.
At the end of the proof, 
Lemma \ref{lemma:alien:substitution} should be applied instead of 
Corollary \ref{cor:generalized}.
\end{proof}





It remains to generalize Lemma \ref{lemma:MP} to arbitrary queries.

\begin{corollary}
\label{cor:MP}
Lemma \ref{lemma:MP} also holds for non-atomic queries.  Moreover, condition
\ref{lemma:MP:condition2} of the lemma can be replaced by:
\begin{enumerate}[(a)]
\setcounter{enumi}2
\item
\label{cor:MP:condition3}
for each atom  of  with   (distinct) variables,
the underlying language has (at least)  constants not occurring in .
\end{enumerate}
\end{corollary}

\begin{proof}
Note that condition \ref{lemma:MP:condition2} implies condition
\ref{cor:MP:condition3}.
So assume that the latter holds.
Let   generalized for  be .
Then each  is   generalized for .
So Lemma \ref{lemma:MP} applies to each .  Thus  implies
, for each .
Hence .
\end{proof}








\section{Characterization of program answers by the least Herbrand model}
\label{sec:H}

This section studies when the least Herbrand models exactly characterize the
program answers. 
First a sufficient condition is presented for equivalence (\ref{eq}) from
Introduction.
Then we show that the sufficient condition is also necessary.
Conditions \ref{th:MP:condition1}, \ref{th:MP:condition2} below are the same
as conditions \ref{lemma:MP:condition1}, \ref{cor:MP:condition3}
 of Lemma \ref{lemma:MP} and  Corollary \ref{cor:MP}.



\begin{theorem}
[Characterizing answers by ]
\label{th:MP}
Let  be a program, and  a query such that
\begin{enumerate}[(a)]
\item 
\label{th:MP:condition1}
the underlying language has a non-constant function symbol not occurring in~P,
or 
\item 
\label{th:MP:condition2}
for each atom  of  with   (distinct) variables,
the underlying language has (at least)  constants not occurring in .
\nopagebreak
\end{enumerate}
\nopagebreak
Then  iff .
\end{theorem}

Note that condition \ref{th:MP:condition1}
 implies that the equivalence holds for every query ,
including queries containing the new symbol.
Also, it holds for every query  and every finite program  when the
alphabet contains infinitely many function symbols, as then 
condition  \ref{th:MP:condition1} or \ref{th:MP:condition2} is satisfied.
From the theorem the known sufficient conditions follow: 
the alphabet containing infinitely many constants (and  finite),
or  ground.



Condition \ref{th:MP:condition2}
is implied by its simpler version:
the language has  constants not occurring in , 
 and each atom of  contains no more than  variables.


\begin{proof}[Proof of Th.\,\ref{th:MP}]
Let  be  generalized for .
By Corollary \ref{cor:MP},
  implies , hence , as  is an
instance of .
The reverse implication is obvious.
\end{proof}









We conclude with showing in which sense
the sufficient condition of Th.\,\ref{th:MP}
is also necessary.  As expected, it is strictly speaking not a necessary
condition for (\ref{eq}), 
as it is violated for some  for which (\ref{eq}) holds. 


\begin{example}


\noindent
Consider program APPEND and assume that the only function symbols of the
underlying language are , 
.
 Let .
Then  and , 
but the condition of Th.\,\ref{th:MP} is violated.

On the other hand, consider a program  of three clauses 
.\,;
{\small
.\,;
\hspace{0pt plus .5ex}.
}
Programs \mbox{APPEND} and  have the same least Herbrand model but
different sets of answers,
as e.g.\ .
The condition of Th.\,\ref{th:MP} is violated by , and the equivalence
does not hold.
Note that  cannot be used to append lists when new function symbols are
added to the language;  is then not an answer of~.
\sloppy
\end{example}






Roughly speaking, 
the sufficient conditions of Th.\,\ref{th:MP} and Lemma \ref{lemma:MP}
are also necessary, 
when all what is known about a program is the set of function
symbols employed in it:


\begin{proposition}
\label{prop:counterexample}
Let  be the set of function symbols of the underlying language, and
 be its finite subset.
Let  be a query, 
such that the predicate symbols of the atoms of  are distinct.
Assume that   iff , for each finite program 
such that 
 is the set of function symbols occurring in .  Then the 
sufficient condition of Th.\,\ref{th:MP} holds.
\end{proposition}
The proposition also holds when  and the considered program  are
infinite. 



\begin{proof}
Let  be a query whose atoms have distinct predicate symbols.
Assume that the sufficient condition of Th.\,\ref{th:MP} does not hold.
We show that for a certain program  
(such that  is the set of the function symbols occurring in ), 
\linebreak[3]
\mbox{} but .

As condition \ref{th:MP:condition1} of Th.\,\ref{th:MP} does not hold, 
all the non-constant function symbols of  are in .
As condition \ref{th:MP:condition2} does not hold, 
there is an atom  in  with  distinct variables ,
for which the number of constants from  not occurring in
 is \/; let    be the constants.
The atom can be represented as ,
where
 are those (distinct) constants of  which are not in .\footnote{Formally,  can be defined as the instance
  of an atom , whose
 (distinct) variables are , and 
 whose function symbols are from .
}
So .


Let  be the set of atoms of  except for .
Let  be  distinct variables.
Let  consist of the unary clauses of  and the
unary clauses of the form  where 
(i)~
(so a variable occurs twice),
or (ii)~
where , its arity is ,
and  is a
tuple of  distinct variables pairwise distinct from those in .
Note that  is finite iff  is.



Each ground atom  (where )
is an instance of some clause of ,
as if  are distinct terms then the main symbol of some of
them is in , and  is an instance of a clause of the form (ii),
otherwise  is an instance of a clause of the form (i).
Thus  , hence 
(as  for each atom  of  distinct from ).
To show that  , 
add new constants  to the alphabet.
Then  is not an instance of any clause of , 
so  is false in the least Herbrand model of 
with the extended alphabet.
\end{proof}






The proof provides a family of counterexamples for a claim that 
 and  are equivalent.
In particular, setting  (, ) results in 
,
a generalization 
of the counterexample from Introduction to any underlying
finite set  of function symbols.




From the proposition it follows that a more general sufficient condition
(than that of Th.\,\ref{th:MP}) for
the equivalence of  and  is impossible,
unless it uses more information about  than just the set of involved symbols.


\section{Conclusion}

In some cases the least Herbrand model does not characterize the set of
answers of a definite program.  
This paper generalizes the sufficient condition for
 iff , to
``a non-constant function symbol not in , or  constants not in 
for each atom  of ''.
It also shows
that the sufficient condition cannot be improved unless more is
known about the program than just which function symbols occur in it.
  As a side effect, it is shown
  which more general queries are implied to be answers of  by  being an
  answer. 





\paragraph{Acknowledgement.}
Comments of anonymous referees helped improving the presentation.

\bibliographystyle{acmtrans}
\bibliography{bibshorter,bibmagic,bibpearl}

\end{document}
