
\documentclass{llncs}
\usepackage{amssymb, amsmath}

\usepackage{verbatim}
\usepackage{graphicx}
\usepackage{subfigure}



\usepackage{color}
\usepackage[normalem]{ulem}

\newcommand{\oldversion}[1]{{}}
\newcommand{\rednote}[1]{{\color{red} #1}}
\newcommand{\bluenote}[1]{{\color{blue} #1}}
\newcommand{\greennote}[1]{{\color{green} #1}}
\newcommand{\remove}[1]{}
\newcommand{\rephrase}[3]{\noindent\textbf{#1 #2}.~\emph{#3}}
\newcommand{\del}[1]{\textcolor{red}{\sout{#1}}}
\newcommand{\rep}[2]{\textcolor{blue}{\sout{#1}} \textcolor{blue}{#2}}
\newcommand{\NP}{\mbox{\bfseries NP}}
\newcommand{\Pt}{\mbox{\bfseries P}}
\newcommand{\red}[1]{\color{red} #1 \color{black}\xspace}

\newenvironment{proofof}[1]{\par\addvspace\topsep\noindent
{\bf Proof #1:} \ignorespaces }{\qed}



\usepackage[pagebackref=true,breaklinks=true]{hyperref}





\begin{document}

\title{Drawing Partially Embedded and\\ Simultaneously Planar Graphs\thanks{A preliminary version of this paper appeared in~\cite{cfglms-dpespg-14}.}}


\author{ Timothy~M.~Chan\inst{1}
\and Fabrizio Frati\inst{2}
    \and Carsten~Gutwenger\inst{3}
    \and Anna~Lubiw\inst{1}
    \and Petra~Mutzel\inst{3}
    \and Marcus~Schaefer\inst{4}
}

\institute{Cheriton School of Computer Science, University of Waterloo, Canada.\\
\email{\{tmchan, alubiw\}@uwaterloo.ca}
\and
School of Information Technologies, The University of Sydney, Australia.\\
\email{fabrizio.frati@sydney.adu.au}
\and
Technische Universit\"{a}t Dortmund, Dortmund, Germany.\\
\email{\{carsten.gutwenger, petra.mutzel\}@tu-dortmund.de}
\and DePaul University, Chicago, Illinois, USA.\\
\email{mschaefer@cdm.depaul.edu}
}



\maketitle


\begin{abstract}
We investigate the problem of constructing planar drawings with few bends for two related problems, the \emph{partially embedded graph} problem---to extend a straight-line planar drawing of a subgraph to a planar drawing of the whole graph---and the \emph{simultaneous planarity} problem---to find planar drawings of two graphs that coincide on shared vertices and edges. In both cases we show that if the required planar drawings exist, then there are planar drawings with a linear number of bends per edge and, in the case of simultaneous planarity, a constant number of crossings between every pair of edges.  Our proofs provide efficient algorithms if the combinatorial embedding of the drawing is given. Our result on partially embedded graph drawing  generalizes a classic result by Pach and Wenger which shows that any planar graph can be drawn with a linear number of bends per edge if the location of each vertex is fixed.
 \end{abstract}

\section{Introduction}
\label{sec:intro}

In many practical applications we wish to draw a planar graph while satisfying some geometric or topological constraints. One natural situation is that we have a drawing of part of the graph and wish to extend it to a planar drawing of the whole graph. Pach and Wenger~\cite{PW01} considered a special case of this problem.  They showed that any planar graph can be drawn with its vertices lying at pre-assigned points in the plane and with a linear number of bends per edge. In this case the pre-drawn subgraph has no edges.

If the pre-drawn subgraph  has edges, a planar drawing of the whole graph  extending the given drawing  of  may not exist. Angelini et al.~\cite{ABFJKPR10} gave a linear-time algorithm for the corresponding decision problem; the algorithm returns, for a positive answer, a planar embedding of  that {\em extends} that of  (i.e., if we restrict the embedding of  to the edges and vertices of , we obtain the embedding corresponding to ). If one does not care about maintaining the actual planar drawing of  this is the end of the story, since standard methods can be used to find a straight-line planar drawing of  in which the drawing of  is topologically equivalent to the one of . In this paper we show how to draw  while preserving the actual drawing  of , so that each edge has a linear number of bends. This bound is worst-case optimal, as proved by Pach and Wenger~\cite{PW01} in the special case in which  has no edges.

A result analogous to ours was claimed by Fowler et al.~\cite{fjks-crp-11} for the special case in which  has the same vertex set as . Their algorithm draws the edges of  one by one, in any order so that edges connecting distinct connected components of  precede edges within the same connected component of ; each edge is drawn as a curve with the minimum number of bends. Fowler et al. claim that their algorithm constructs drawings with a linear number of bends per edge. However, we prove that there exists a tree, a planar drawing of its vertex set, and an order of the edges of the tree, such that drawing the edges in the given order as curves with the minimum number of bends results in some edges having an exponential number of bends.

The second graph drawing problem we consider is the \emph{simultaneous planarity} problem~\cite{BKR?}, also known as ``simultaneous embedding with fixed edges (SEFE)''.  The SEFE problem is strongly related to the partially embedded graph problem and---in a sense we will make precise later---generalizes it. We are given two planar graphs  and  that share a \emph{common subgraph}  (i.e.,  is composed of those vertices and edges that belong to both  and ). We wish to find a \emph{simultaneously planar drawing}, i.e., a planar drawing of  and a planar drawing of  that coincide on . Graphs  and  are \emph{simultaneously planar} if they admit such a drawing. Both  and  may have {\em private} edges that are not part of . In a simultaneous planar drawing the private edges of  may cross the private edges of . The simultaneous planarity problem arises in information visualization when we wish to display two relationships on two overlapping element sets.

The decision version of the simultaneous planarity problem is not known to be \NP-complete, or to be solvable in polynomial time, though it is known to be \NP-complete if more than two graphs are given~\cite{GJPSS06}. However, there is a combinatorial characterization of simultaneous planarity, based on the concept of a ``compatible embedding'', due to J{\"u}nger and Schulz~\cite{JS} (see below for details). Erten and Kobourov~\cite{EK}, who first introduced the problem, gave an efficient drawing algorithm for the special case where the two graphs share vertices but no edges.  In this case, a simultaneous planar drawing on a polynomial-size grid always exists in which each edge has at most two bends and therefore any two edges cross at most nine times, see~\cite{dl-seogpc-07,EK,k-setbepa-06}. In this paper we show that if two graphs have a simultaneous planar drawing, then there is a drawing on a polynomial-size grid in which every edge has a linear number of bends and in which any two edges cross at most 24 times. Our result is algorithmic, assuming a compatible embedding is given.

\subsection{Realizability Results}

Our paper addresses the following two drawing problems:

\begin{description}

\item[{\bf Planarity of a partially embedded graph (PEG).}]
Given a planar graph  and a straight-line planar drawing  of a subgraph  of ,
find a planar drawing of  that extends  (see~\cite{ABFJKPR10,JKR13}).


\item[{\bf Simultaneous planarity (SEFE).}]
Given two planar graphs  and  that share a subgraph ,
find a simultaneous planar drawing of  and  (see~\cite{BKR?}).
\end{description}

We prove the following results:

\begin{theorem}[Realizing a Partially Embedded Graph]\label{thm:PWG}
Let  be an -vertex planar graph, let  be a subgraph of , and let  be a straight-line planar drawing of . Suppose that  has a planar embedding  that extends the one of .
Then we can construct a planar drawing of  in -time which realizes , extends , and has at most  bends per edge.
 \end{theorem}

Theorem~\ref{thm:PWG} generalizes Pach and Wenger's classic result, which corresponds to the special case in which the pre-drawn subgraph has no edges.

\begin{theorem}[Realizing a Simultaneous Planar Embedding]
\label{thm:sim-draw}
Let  and  be simultaneously planar graphs on a total of  vertices with a shared subgraph .
If we are given a compatible embedding of the two graphs, then we can construct in  time a drawing that realizes the compatible embedding, and in which
any private edge of  and any private edge of 
intersect at most  times. In addition, we can ensure either one of the following two properties:

\begin{enumerate}
\item[] each edge of  is straight, and each private edge of  and of  has at most  bends; also, vertices, bends, and crossings lie on an  grid; or
\item[] each edge of  is straight and each private edge of  has at most  bends per edge.
\end{enumerate}
\end{theorem}

 Theorem~\ref{thm:PWG} provides a weak form of Theorem~\ref{thm:sim-draw}: If  and  are simultaneously planar, they admit a compatible embedding. Take any straight-line planar drawing of  realizing that embedding and extend the induced drawing of  to a drawing of . By Theorem~\ref{thm:PWG}, we obtain a simultaneous planar drawing where each edge of  is straight and each private edge of  has at most  bends per edge. Our stronger result of 24 crossings between any two edges is  obtained by modifying the proof  of Theorem~\ref{thm:PWG}, rather than applying that result directly.


Grilli et al.~\cite{Grilli2014} very recently and independently proved a result in some respect stronger than Theorem~\ref{thm:sim-draw}. They showed that two simultaneously planar graphs have a simultaneous planar drawing with at most  bends per edge, vastly better than our  bound. On the other hand, our bound of  crossings per pair of edges is better than the bound of  that can be derived from their result. Also, our algorithm allows us to construct simultaneous planar drawings in which each edge of one graph is straight or in which vertices, bends, and crossings lie on a polynomial-size grid. The former feature is not achievable by means of Grilli et al.'s algorithm; the latter one could be obtained from Grilli et al.'s result, at the expense of increasing the number of bends per edge to  (which corresponds to the number of crossings on a single private edge).

\subsection{Related Work}

The decision version of simultaneous planarity generalizes partially embedded planarity: given an instance  of the latter problem, we can augment  to a drawing of a -connected graph  and let .  Then  and  are simultaneously planar if and only if  has a planar embedding extending . In the other direction, the algorithm~\cite{ABFJKPR10} for testing planarity of partially embedded graphs solves the special case of the simultaneous planarity problem in which the embedding of the common graph  is fixed (which happens, e.g., if  or one of the two graphs is -connected).

Several optimization versions of partially embedded planarity and simultaneous planarity are \NP-hard. Patrignani showed that testing whether there is a straight-line drawing of a planar graph  extending a given drawing of a subgraph of  is \NP-complete~\cite{P06}, so bend minimization in partial embedding extensions is \NP-complete; Patrignani's result holds  even if a combinatorial embedding of  is given.\footnote{Patrignani does not explicitly claim \NP-completeness in the case in which the embedding of  is fixed, but that can be concluded by checking his construction; only the variable gadget, pictured in his Figure~3, needs minor adjustments.}
Bend minimization in simultaneous planar drawings is \NP-hard, since it is \NP-hard to decide whether there is a straight-line simultaneous drawing~\cite{EGJPSS08}.
Crossing minimization in simultaneous planar drawings is also \NP-hard, as follows from an \NP-hardness result on \emph{anchored planar drawings} by Cabello and Mohar~\cite{Cabello-Mohar}; see Theorem~\ref{thm:SPC} in Section~\ref{sec:S2} for a slightly stronger result.

Di Giacomo et al.~\cite{ddlmw-psetgpd-09} studied the special case of PEG in which the -vertex graph  to be drawn is a tree. They showed that, given a drawing  of a subtree  of , a drawing of  extending  can be computed in  time so that each edge of  has at most  bends.

Further, as mentioned above, the special cases of PEG and SEFE in which there are no edges in the pre-drawn subgraph and in the common subgraph have been already studied.

Concerning PEG, Pach and Wenger~\cite{PW01} proved the following result: given an -vertex planar graph  with fixed vertex locations, a planar drawing of  in which each edge has at most  bends can be constructed in  time. They also proved that such a bound is asymptotically tight in the worst case.
Regarding the constant, Badent et al.~\cite{bgl-dcgcps-08} improved the bound to  bends per edge.
Biedl and Floderus~\cite{Biedl-Floderus} considered the more general problem of drawing an -vertex planar graph on fixed vertex locations where the drawing is constrained to lie inside a -vertex polygon.   They show that there is a drawing with  bends per edge.

Concerning SEFE, Di Giacomo and Liotta~\cite{dl-seogpc-07} and independently Kammer~\cite{k-setbepa-06} proved the following result: given two planar graphs  and  sharing some vertices and no edge with a total number of  vertices, there exists an -time algorithm to construct a simultaneous planar drawing of  and  on a grid of size , where each edge has at most  bends, hence there are at most  crossings between any edge of  and any edge of . This improves upon a previous result of Erten and Kobourov~\cite{EK}. The algorithms in~\cite{dl-seogpc-07,EK,k-setbepa-06} make use of a drawing technique introduced by Kaufmann and Wiese~\cite{KW}.

Haeupler et al.~\cite{HJL} showed that if two simultaneously planar graphs  and  share a subgraph  that is connected, then there is a simultaneous planar drawing in which any edge of  and any edge of  intersect at most once. Introducing vertices at crossing points yields a planar graph, and a straight-line drawing of that graph provides a simultaneous planar drawing with  bends per edge,  crossings per edge, and with vertices, bends, and crossings on an  grid. Our result generalizes this to the case where the common graph  is not necessarily connected.

\subsection{Graph Drawing Terminology}\label{sec:definitions}

A \emph{rotation system} for a graph is a cyclic ordering of the edges incident to each vertex.
A rotation system of a connected graph determines its {\em facial walks}---the closed walks in which each edge  is followed by the next edge  in the cyclic order at .
The {\em size}  of a facial walk  is the number of vertices on , where we count vertex repetitions.  (Note that a graph that consists of a single vertex has a single facial walk of size 1; for any other connected graph the size of a facial walk is equal to the number of edges in the facial walk, counting repetitions.)
A rotation system is \emph{planar} if Euler's formula holds, i.e.,  where  is the number of vertices,  is the number of edges, and  is the number of facial walks.
A \emph{planar embedding} of a graph consists of a planar rotation system together with a specified outer face.  A fundamental result about connected planar graphs is that every planar drawing corresponds to a planar embedding, and conversely, every planar embedding can be realized as a planar drawing (and, in fact, as a straight-line planar drawing by F\'ary's theorem).
Furthermore, facial walks correspond to faces in the drawing.



These definitions do not handle the combinatorics of a planar drawing of a disconnected graph---namely the definition of planar embedding as stated above does not tell us how connected components nest into each other.



Following  J{\"u}nger and Schulz~\cite{JS}, we
define a \emph{topological embedding} of a (possibly non-connected) graph as follows: We specify a planar embedding for each connected component.  This determines a set of inner faces. For each connected component we specify a ``containing'' face, which may be an inner face of some other component or the unique outer face.   Furthermore, we forbid cycles of containment---in other words, if a connected component is contained in an inner face, which is contained in a component, etc., then this chain of containments must lead eventually to the unique outer face.

A {\em facial boundary} in a topological embedding of a graph is the collection of facial walks along the (not necessarily connected) boundary of a face. Each face (unless it is the outer face) has a distinguished facial walk we call the {\em outer} facial walk separating the remaining {\em inner} facial walks from the outer face of the embedding. The {\em size} of a facial boundary is the sum of the sizes of its facial walks.

A \emph{compatible embedding} of two planar graphs  and  consists of topological embeddings of  and  such that the common subgraph  inherits the same topological embedding from  as from  (where a subgraph inherits a topological embedding in a straightforward way; in particular, if we remove an edge that disconnects the graph, the face containment is determined by the edge that was removed). J{\"u}nger and Schulz~\cite{JS} proved that  and  are simultaneously planar if and only if they have a compatible embedding. For that proof, they construct a simultaneous planar drawing of  and  by extending a drawing of  (thus proving a form of our Theorem~\ref{thm:PWG}). However, their method does not yield any bounds on the number of bends or crossings.


\section{Partially Embedded Graphs}\label{sec:PEG}

In this section we prove Theorem~\ref{thm:PWG}; that is, we show how to construct a planar drawing of  that extends the planar straight-line drawing  and has a linear number of bends per edge assuming that we are given a planar embedding of  extending . It is sufficient to prove the result for a single face  of , since the embedding of  is given, and we know for each vertex and edge of  which face of  it lies in, so the drawings in different faces of  do not interfere with each other.

Pach and Wenger~\cite{PW01} proved their upper bound on the number of bends needed to draw a graph with fixed vertex locations by drawing a tree with its leaves at the fixed vertex locations, and ``routing'' all the edges close to the tree, sometimes crossing the tree but never crossing each other. We want to use their approach, but we have to deal with a more general problem. Instead of fixed vertex locations we have fixed facial boundaries. The solution is natural: We contract each facial walk  of  to a single vertex , fix a position for vertex  inside  near , and then apply the Pach-Wenger method to draw the contracted graph on the fixed vertex locations . We ensure that the contracted graph is drawn inside , indeed we
stay a small distance away
from the boundary of , inside a polygonal region  that is an ``inner approximation'' of . Inside  we draw a tree  with its leaves  at the fixed vertex locations, while suitably bounding the size of  so as to get our bound on the number of bends. We then route the edges of the contracted graph close to  as  Pach and Wenger do. Finally, to retrieve the original, uncontracted graph,  we route the edges incident to  to their true endpoint on the facial boundary ---these routes use the empty buffer zone between  and .

We fill in the details of this argument in Section~\ref{sec:PPWG}, but before doing so we introduce ``inner approximations'' in Section~\ref{sec:AF},
and formalize the tree argument in Section~\ref{sec:EPP}.

To simplify notation, we use  and  for the number of vertices and edges in a graph (or subgraph) .

\subsection{Approximating Faces}\label{sec:AF}

In the drawing , the face  is a region of the plane
homeomorphic to a disc with holes.  Each facial walk of  appears in the drawing as a  {\em closed polygonal arc}, i.e.~a sequence of straight-line segments joined in a path that returns to its starting point (repeated segments/vertices may occur).
We will refer to a facial walk and its drawing interchangeably.

We will approximate  by offsetting each of its facial walks into the interior of .
See Figure~\ref{fig:approx}.
Let  be the outer facial walk of , and let  be the inner facial walks.
An {\em inner -approximation of } is a simple polygon  (a closed polygonal arc with no self-intersections) such that:
\begin{enumerate}
\item   is {\em-close} to  , meaning that every point of  is within distance  of a point of ,
\item the inner facial walk  lies in the interior of , for each , and
\item the outer facial walk  lies in the exterior of .
\end{enumerate}
If in addition the 's form a {\em polygonal region} (a simple polygon with holes) with  as the outer polygon, then we say that the polygonal region is an  {\em inner -approximation of }.
\remove{The {\em Hausdorff distance}  of two sets (in a space with metric ) is defined as\footnote{The underlying metric  can be Euclidean or some other appropriate metric.}\\  {\begin{center} .\end{center}} Intuitively, the Hausdorff distance measures how far a point in one set can be from the other set. Sets  and  are {\em -close} if . Then  is an {\em inner -approximation of } if they are -close and there is a  so that all the points -close to  are a subset of .
}
The next lemma shows that we can build inner -approximations of . 

\begin{lemma}\label{cor:fw}
For any  we can efficiently construct an inner -approximation  of .
\end{lemma}

\begin{figure}[tb]
\centering
\includegraphics[width=0.8\textwidth]{approx-face-sparse.pdf}
\caption{(\emph{a}) A face  with outer facial walk  and inner facial walks . (\emph{b}) An inner approximation  (heavy blue lines).}
\label{fig:approx}
\end{figure}

See Figure~\ref{fig:approx} for an illustration of Lemma~\ref{cor:fw}. To prove the lemma, we construct---for every sufficiently small  and for every facial walk of ---an
inner -approximating polygon
 which
does not have too many bends, and so that the  are
{\em nested}
in the following sense: if , then
 lies in the interior of
 if  is an inner face, and vice versa otherwise.
There are various ways to achieve this. Pach and Wenger~\cite{PW01} use the Minkowski sum of the facial walk (in their case the facial walk of a tree) and a square diamond centered at .  We use a slightly different construction, because it seems easier (both computationally and conceptually) and it gives a slightly better bound on the number of bends (which is what we are most interested in): for the facial walk of an -vertex tree, Pach and Wenger construct a
polygon with  vertices, while ours  have  vertices. Our construction does have one disadvantage: the resulting drawings are tight, placing elements close together, for sharp (acute or obtuse) angles (the Minkowski-sum construction has the same problem for highly obtuse angles only).

\begin{lemma}\label{lem:fw}
 Let  be a facial walk in a face  of a drawing of a graph  in the plane. We can efficiently construct a
 nested family of inner -approximating polygons
 so that
each  has at most  vertices.
\end{lemma}

\begin{proof}
 Let  be a {\em corner} of , that is, two consecutive edges ,  and their shared vertex . At  erect the angle bisector of  and  of length  (inside ), and let  be the endpoint of the bisector different from . For computational reasons, it may be better to use the -norm at this point (the Euclidean norm will lead to square root expressions in the coordinates). If  is the sequence of vertices along , with , then  defines a
 closed polygonal chain.  If  is sufficiently small, namely less than half the distance between any vertex of  and a non-adjacent edge on , the
 polygonal chain
is free of self-crossings, and therefore bounds a simple polygon with  vertices. There are two special cases in which this argument does not work: if the facial walk is a facial walk on an isolated vertex or an isolated edge. In both of these cases, we can approximate  using a
 triangle.
\end{proof}

To prove Lemma~\ref{cor:fw} we can use
Lemma~\ref{lem:fw} to efficiently construct an inner -approximating polygon for
each facial walk of .
The resulting polygons are disjoint and form a polygonal region
as long as  is less than half the distance between any two non-adjacent vertices or edges of .


\remove{
Lemma~\ref{lem:fw} allows us to replace a facial boundary with a
{\em polygonal region},
that is, a collection of
simple polygons that bound a face which is very close to the original boundary, has bounded complexity, and can be constructed efficiently. This leads to a proof of Lemma~\ref{cor:fw}. Namely, approximate each facial walk of the facial boundary with an -close
polygon lying in . The
result is a polygonal region
as long as  is less than half the distance between any two non-adjacent vertices or edges. The upper bound of  will generally be a large overestimate, but allows for the possibility that all the inner walks are walks on isolated vertices. If there are no isolated vertices, then a walk of size  gets replaced by a polygon of size at most  (a tight bound for ), proving the slightly sharper upper bound.
}

\subsection{Extending Partial Embeddings}\label{sec:EPP}

Our main technical tool in the proof of Theorem~\ref{thm:PWG} is the following lemma. We suggest skipping the proof of this lemma in a first reading. Multigraphs, in this paper, may have multiple edges and loops.

\begin{lemma}\label{cor:Te}
Let  be a multigraph with a given planar embedding and fixed locations for a subset  of its vertices. Suppose we are given a straight-line drawing of a tree  whose leaves include all the vertices in  at their fixed locations. Then for every  there is a planar poly-line drawing of  that is -close to , that realizes the given embedding, where the vertices in  are at their fixed locations, where each edge has at most  bends, and where each edge comes close to each vertex  in  at most six times (where coming close to  means entering and leaving an -neighborhood of  or terminating at ).
\end{lemma}

Our proof of Lemma~\ref{cor:Te} will follow closely the structure of Pach and Wenger's algorithm~\cite{PW01} to draw a planar graph with fixed vertex locations. That algorithm has three ingredients:  making  Hamiltonian,  drawing the Hamiltonian cycle of , and  drawing the remaining edges of . We use their result  directly:

\begin{lemma}[{Pach, Wenger~\cite{PW01}}]\label{lem:PWHam}
  Given a planar graph  we can in linear time construct a Hamiltonian graph  with  by adding and subdividing edges of  (each edge is subdivided by at most two new vertices). \end{lemma}

We will use a slightly stronger version of Lemma~\ref{lem:PWHam} in which  is allowed to be a multigraph. Pach and Wenger's proof of Lemma~\ref{lem:PWHam} works in the presence of multiple edges and loops.

For part  Pach and Wenger show that a Hamiltonian cycle can be drawn at fixed vertex locations -close to a star connecting all the vertices. For our application, we replace their star with a straight-line drawing of a tree  whose leaves are the vertices . Lemma~\ref{lem:HCemb} shows how to draw the Hamiltonian cycle. Later we will see how to draw the remaining edges.

Independently of our result, the generalization of part  to trees has essentially been shown by Chan et al.~\cite{CHKL13}. Since their goal was to minimize edge lengths, they did not give an estimate on the number of bends.

\begin{lemma}
\label{lem:HCemb}
 Let  be a cycle with fixed vertex locations, and suppose we are given a straight-line planar drawing of a tree , in which the vertices of  are leaves of  at their fixed locations. Then for every  there is a planar poly-line drawing of  with at most  bends per edge and -close to .
\end{lemma}

\begin{proof}
 Let  be the vertices of  in their order along the cycle. We build a planar poly-line drawing of  as follows. Let  be an -approximation of  for  (which we construct using Lemma~\ref{lem:fw}). We start at . Suppose we have already built the poly-line drawing of  and we want to add . Let  be the unique path in  connecting   to . Create  from  by keeping only the vertices of  close to (approximating)
 vertices in .
 This removes parts of the walk along  which we patch up as follows: suppose  is an interior vertex of , and
  is incident to  which does not lie on . Then  is approximated by two vertices  and  which lie on bisectors formed by  with neighboring edges. Now  and  belong to , but the path along  between them got removed (since  does not belong to ). We
 add  to  to connect them. Note that  does not pass through  since  is incident to at least three edges ( and two edges of ), and it does not cross any edges of any  with , since  is monotone: if , then  for . See Figure~\ref{fig:treecycle} for an illustration.
\begin{figure}[tb]
\centering
\includegraphics[width=2.4in]{TreeCycle.pdf}
\caption{The underlying tree  is in black (thick edges), angle bisectors in gray; the  are drawn as thin black edges; to reduce clutter, we are not showing the remaining edges of ; the drawing of  is indicated by the green line.}
\label{fig:treecycle}
\end{figure}
Now both  and  correspond to unique vertices on  (since they are leaves), so we can pick the facial walk  on  which connects  to  and which avoids passing by . We now add line segments
 , , , ,  to the
 poly-line drawing of . We treat the final edge  similarly, except that we move along  back to  in the last step,
 which we can do, since none of the intermediate paths passed by . Each edge of  is replaced by a polygonal arc with at most  bends.
\end{proof}

The following lemma shows how to draw the remaining edges of , assuming that  is Hamiltonian. As mentioned earlier, this lemma is close to a result by Chan et al.~\cite{CHKL13}, except for the claim about the number of bends, and the rotation system (which we need for our main result).

\begin{lemma}\label{lem:Te}
Let  be a Hamiltonian multigraph with a given planar embedding and fixed vertex locations. Suppose we are given a straight-line drawing of a tree  whose leaves include all the vertices of  at their fixed locations. Then for every  there is a planar poly-line drawing of  that is -close to , that realizes the given embedding, and so that the vertices of  are at their fixed locations, every edge has at most  bends, and every edge comes close to any leaf of  at most twice.
\end{lemma}

The obvious idea---routing edges along the Hamiltonian cycle ---only gives a quadratic bound on the number of bends, since each edge would follow the path of a linear number of edges of , and each edge of  has a linear number of bends. Pach and Wenger came up with an ingenious way to construct auxiliary curves with few bends based on the level curves  which carry the cycle  in the proof of Lemma~\ref{lem:HCemb}.

\begin{proof}
 Let  be the Hamiltonian cycle of  and let  and  be the two outerplanar graphs composed of  and, respectively, of the edges of  outside and inside .  Using Lemma~\ref{lem:HCemb} we find a planar poly-line drawing of  on . We need to show how to draw  and  respecting the planar embeddings induced by the given embedding of . Let  and .
 We only describe how to draw , since  can be handled analogously. Let ,  be a -approximation of  constructed using Lemma~\ref{lem:fw}. For a fixed , each  crosses  twice: when  moves from  to , and when it finally moves back from  to . As in Pach and Wenger, we can then split  at the crossings and connect their free ends to  and , resulting (for each ) in two curves  and  connecting  to , where  lies outside  (these are the curves we use for ) and  inside  (these are the curves we use for ). Each such curve has at most  bends. As in the proof of Pach and Wenger, we can create edges  by concatenating  with . Since we chose  such approximations, we can do this for each edge in . There are two problems remaining: edges  now all pass through  and they could potentially cross (rather than just touch) there. Pach and Wenger show that any two edges touch, so the drawing can be modified close to  so as to separate all edges  from each other. This introduces at most one more bend per edge, so that the resulting edges have  bends. Finally, note that each edge  comes close to each leaf of  (including ) at most twice, once for  and once for  .
\end{proof}

We are finally ready to complete the proof of Lemma \ref{cor:Te}. We show how to apply Lemma~\ref{lem:Te} in case  is not Hamiltonian, and not all its vertices are assigned fixed locations.

\begin{proofof}{of Lemma~\ref{cor:Te}}
By Lemma~\ref{lem:PWHam}, we can construct a graph  with a Hamiltonian cycle  by subdividing each edge of  at most twice, and by adding some edges, where  has a planar embedding extending the embedding of .

Next we deal with the issue that not all vertices lie in , the set of vertices with fixed locations.  Traverse : whenever we encounter an edge of  with at least one endpoint not in , contract that edge. This yields a new Hamiltonian graph  with  and a planar embedding induced by the planar embedding of . Use Lemma~\ref{lem:Te} to construct a planar poly-line drawing of   at the fixed vertex locations, and -close to , so that each edge of  has at most  bends. Each vertex  of  corresponds to a set of vertices  which was contracted to , so the subgraph  of  induced by  is connected. Since we embedded  with the induced planar embedding of , we can now do some surgery to turn  back into .

The idea is to remove a small disc around vertex  in the drawing of , and to draw  inside this disc, connected to the appropriate edges leaving the disc.
This will involve introducing new vertices where edges cross into the disc.
The same idea was used in~\cite[Theorem 2]{HJL}.

To this end, we define a graph , which consists of , a cycle  containing  in its interior, and some further edges. Each vertex of  corresponds to an edge of  ``incident to'' , i.e., with an end-vertex in  and an end-vertex not in .
Vertices appear in  in the same order as the corresponding edges incident to  leave  (this order also corresponds to the cyclic order of the edges incident to  in ); each vertex of  corresponding to an edge  of  is connected to the end-vertex of  in . Finally,  contains further edges that triangulate its internal faces.

Consider a small disk  around . We erase the part of the drawing of  inside . We construct a straight-line convex drawing of  in which each vertex of  is mapped to the point in which the corresponding edge crosses the boundary of . This drawing always exists (and can be constructed efficiently), since  is -connected and internally-triangulated. Removing the edges that triangulate the internal faces of  completes the reintroduction of .

Overall, we added one bend to an edge with exactly one endpoint in . Since an edge can have endpoints in at most two , this process adds at most two bends per edge, so every edge has at most  bends. Since each edge of  was subdivided at most twice to obtain , each edge of  has at most  bends. Each edge of  comes close to each leaf of  at most twice, so each edge of  comes close to each vertex of  at most six times. This concludes the proof of Lemma~\ref{cor:Te}.
\end{proofof}

\subsection{Proof of Theorem~\ref{thm:PWG}}\label{sec:PPWG}

As we mentioned earlier, it is sufficient to prove the result for each face of , so fix such a face .
Let , with , be the facial walks of . We distinguish between facial walks consisting of isolated vertices, indexed by , and facial walks consisting of more than one vertex, with indices in .
Construct an inner -approximation  of ---that is,  without the isolated vertices---using Lemma~\ref{cor:fw}, and let  be the face bounded by  and isolated vertices , .
For  let  be the polygon in  that approximates .
Then
 by Lemma~\ref{lem:fw} and the fact that  has size at least 2.
Thus we have that .




We can triangulate  using at most  triangles, applying the following lemma with , , and .

\begin{lemma}[Based on O'Rourke~\protect{\cite[Lemma 5.2]{OR87}}]\label{le:number-of-triangles}
Given an -vertex polygonal region with  point-holes and  non-point-holes, this region can be triangulated by adding chords in time . The resulting triangulation has  triangles.
\end{lemma}

\begin{proof}
The time bound can be derived from the algorithm of O'Rourke~\cite[Lemma 5.1]{OR87}. Consider the total sum of all angles in triangles of the triangulation. Suppose there are  vertices on the outer face,  isolated vertices, and  vertices on non-point-holes (of which there are ). Then the total angle sum is  which equals , where  is the number of triangles. We conclude that .
\end{proof}





We use a result of Bern and Gilbert~\cite{BG92} to construct a straight-line drawing of the dual of the triangulation.  Bern and Gilbert place a vertex at the {\em incenter} of each
triangle (where the angle bisectors of the triangle meet) and prove that the straight-line edge joining two vertices in adjacent triangles lies within the union of the two triangles. Now take a spanning tree  of the dual. By Lemma~\ref{le:number-of-triangles},  has  vertices. For each facial walk , , we augment  with a new leaf   close to  and inside ; for each facial walk , , we add the isolated vertex of  to  as a new leaf . This adds  vertices to , so the number of vertices of  is now 


Let  be the embedded multigraph obtained by restricting  to vertices and edges lying inside or on the boundary of  and by contracting each facial walk  of  to a single vertex . We can now use Lemma~\ref{cor:Te} to embed  along  so that vertices  are drawn at their fixed locations. Each edge of  has at most  bends.


\begin{figure}[tb]
\centering
\includegraphics[width=0.8\textwidth]{approx-face.pdf}
\caption{A face  with outer facial walk  and inner facial walk .  (\emph{a}) The 5 edges of .  (\emph{b}) The inner approximation  (heavy blue lines), a triangulation of it (fine lines), and the dual spanning tree (dashed red) with extra vertices  and  close to   and , respectively.}

\label{fig:route1}
\end{figure}

We now want to connect edges in  to the boundary components they belong to. For facial walks , , there is nothing to do, since we chose  to be the isolated vertex which is the boundary component . So we may assume that we are dealing with boundary components consisting of more than one vertex. We will use the buffer between  and  to do this. In fact, we need to split the buffer zone into two,  so we apply
Lemma~\ref{cor:fw} a second time to obtain an inner -approximation  of , so that . See Figure~\ref{fig:route23}. Let  be the polygon that approximates  in . Note that .
Now for each walk  we extend the edges ending at  to their endpoint on . Since we maintained the cyclic order of -edges at , we can simply route these edges around  using approximations to  via Lemma~\ref{cor:fw}, and we can do so in .
This adds two bends to the edge near , plus at most one bend for each vertex of  except the one corresponding to the final destination vertex on .  In total we add at most  bends.
There is one difficulty: there are edges of  that pass by , separating it from the segment of  close to  (which is our gate to ). To remedy this difficulty, we first route all of these edges around the whole obstacle  in the  part of the buffer, which adds  bends to an edge every time it passes  (see Figure~\ref{fig:route23}, note that the edge starts with one bend close to the vertex).



Now we are free to route the -edges incident to  to their endpoints along . Since an edge can pass by and/or terminate at a vertex at most six times, the number of additional bends in each edge caused by going around
 is at most ; totalling this number over all boundary components of  yields a bound of at most
 bends along the whole edge (we can ignore  with , since we do not reroute around those components). Since each -edge started with  bends, each -edge now has at most
 bends.
 



In order to derive a bound in terms of , we use:

   (as discussed in the first part of this subsection),

  (as discussed in the first part of this subsection),

 


  (which can be easily proved by induction on , primarily, and on the number of -connected components of , if ), and




   (since each facial walk  with  consists of more than one vertex).
 

\remove{In order to derive a bound in terms of , we use   (as discussed in the first part of this subsection),  ,  ,   (since each facial walk  with  consists of more than one vertex), and   (as discussed in the first part of this subsection).
}

From (1) and (2) we get that .
Thus the number of bends in each -edge is at most

From (3) and (4), we conclude that each -edge has at most  bends.







\begin{figure}[tb]\centering
\includegraphics[width=\textwidth]{detour.pdf}
\caption{A close-up of the situation near inner facial walk .
(\emph{a}) After drawing  around the tree  (heavy dashed line), edges
 are incident to  in the correct cyclic order, but two other edges  and   pass by between  and .
(\emph{b})  We add a second approximation  and route the edges  and  (in dashed red) around   in the buffer zone between  and .
(\emph{c})  We route the edges incident to  in the buffer zone between  and .}
\label{fig:route23}
\end{figure}

Most of the steps in the construction can be performed in linear time. Building the triangulation takes time . The overall running time is thus bounded by the size of the resulting drawing which contains a linear number of edges each with a linear number of bends, yielding the quadratic running time.

{\bf Remark 1.} The algorithm we presented in this section provides a bound better than  bends per edge if the subgraph  of  for which a straight-line drawing  is given as part of the input is {\em induced}. If that is the case, then the embedded multigraph  defined in this section contains no self-loops; consequently, a Hamiltonian graph  can be constructed in linear time by adding vertices and edges and by subdividing edges of  so that each edge is subdivided by at most one new vertex (while in the general case we use two subdivision vertices per edge, see Lemma~\ref{lem:PWHam}). This immediately allows us to improve the bounds in Lemma~\ref{cor:Te} on the number of bends per edge to  and on the number of times each edge comes close to each vertex  to at most four. The same analysis as above and the improved bounds of Lemma~\ref{cor:Te} allow us to upper bound the number of bends per edge in Theorem~\ref{thm:PWG} by .

{\bf Remark 2.} An improvement upon the  bound of Theorem~\ref{thm:PWG} can be obtained by modifying the placement of , for each , and the route of the edges that go around . This modification makes the algorithm slightly more involved, so we preferred to omit it from the proof and to sketch it here. The main idea is that vertex  can be inserted not just at any point inside , but rather at a convex corner of  that approximates an occurrence  of a vertex of . Then each edge that goes around  and has to be ``wrapped around''  can save three bends (each time it passes by ) with respect to the route described in Figure~\ref{fig:route23}. To achieve this, we bend the edge at its intersection points with  and then connect it directly to the suitable approximations of the vertices next to  along . This route introduces  new bends each time an edge passes by . A similar argument can be used for the edges that terminate at some vertex of . This results in each -edge having at most  bends. Then the same calculations described above lead to a bound of  bends per edge.

\section{Extending Partial Drawings Greedily} \label{se:greedy}

Let  be a plane graph with a spanning subgraph  for which we have fixed a straight-line planar drawing .
For a given ordering  of the edges in  we say that a drawing  of  {\em greedily extends  with respect to } if it is obtained by drawing edges  in this order, so that  is drawn as a polygonal curve that respects the embedding of  and with the minimum number of bends, for .

Suppose  orders the edges of  so that the edges between distinct connected components of  precede edges between vertices in the same connected component of . For such orderings Fowler {\em et al.}\ claimed in~\cite{fjks-crp-11} that there exists a drawing  of  greedily extending  with respect to  in which each edge has  bends. However, in the following we confirm a claim of Schaefer~\cite{S13a} stating that greedy extensions do not, in general, lead to drawings with a polynomial number of bends.

\begin{theorem}\label{th:greedy}
For every  there exists an -vertex plane graph , a planar drawing  of , the empty spanning subgraph of , and an order  of the edges in  so that any drawing of  that greedily extends  with respect to  has edges with  bends.
\end{theorem}

\begin{proof}
We adapt an example by Kratochv\' il and Matou\v sek~\cite{km-sgrer-91}. Refer to Figure~\ref{fig:exponential}. Let , for any integer . Graph  consists of  isolated vertices, name them . The first  edges in  are  for ,  for ,  for , , , , , , and . All these edges are straight-line segments in any drawing  of  that greedily extends  with respect to . The last  edges in  are  in this order.

\begin{figure}[tb]
\centering
\includegraphics[scale=0.9]{Exponential_Drawing.pdf}
\caption{A drawing  of  that greedily extends  with respect to . Drawing  consists of the black circles. The first edges  edges in  are (black) straight-line segments. The last  edges  are (colored) polygonal lines whose bends have been made smooth to improve the readability. Only four of the latter edges are shown.}
\label{fig:exponential}
\end{figure}

Consider any drawing  of  that greedily extends  with respect to . We claim that edge  has  bends in . In fact, it suffices to prove that  has  intersections with the straight-line segment  in . Indeed,  has exactly one intersection with  in . Inductively assume that  has  intersections with  in ; we prove that  has  intersections with  in . This proof is accomplished by following Kratochv\' il and Matou\v sek~\cite{km-sgrer-91} almost {\em verbatim}. Since  does not cross , it has a bend  around , i.e., inside the square defined by , , , and . Thus the polygonal curve representing  in  consists of two parts---one from  to , the other from  to . Both of these parts may be used as an edge joining  and , after contracting  and  into , and  into . Hence, by induction, each of these two parts has  intersections with , and the whole edge  has  intersections with~.

Hence, in any drawing  of  that greedily extends  with respect to , one edge has  bends, which concludes the proof.
\end{proof}

We remark that the graph  in the proof of Theorem~\ref{th:greedy} is a tree, so every edge of  connects vertices in distinct connected components of .



\section{Simultaneous Planarity}
\label{sec:S2}



Before turning to our algorithm to draw simultaneously planar graphs, we justify our claim that minimizing the number of crossings in a simultaneous planar drawing is \NP-hard. This result follows from Cabello and Mohar's proof of  \NP-hardness for the \emph{anchored planarity} problem~\cite[Theorem 2.1]{Cabello-Mohar}, but a more direct proof of a slightly stronger result is possible by reduction from the \NP-complete crossing number problem.

\begin{theorem}\label{thm:SPC}
 Minimizing the number of crossings in a simultaneous planar drawing of two graphs is \NP-complete, even if one graph is the disjoint union of paths of length at most two and the other graph is a matching.
\end{theorem}

The result is sharp in the sense that if both  and  are matchings, the problem is easy, since the union of two matchings is always planar.

\begin{proof}
We use the fact that the (standard) crossing number problem is \NP-hard for cubic graphs~\cite{H06}. Let  be a cubic graph with  edges.
Subdivide each edge  or  times (we will shortly see which). At each of the original vertices of  choose two of the incident edges, and make them part of ; the third edge at each vertex is added to . Now add the remaining edges to  and  so that along each path between original vertices  and  edges alternate. If such a path ends with two -edges or two -edges, we need to subdivide it  times to make this possible; if it ends with one -edge and one -edge, we subdivide it  times. By this construction,  is a disjoint union of paths of length at most two, and  is a matching. Finally, the number of crossings in a simultaneous planar drawing of  and  is an upper bound on the crossing number of , and, since we subdivided each edge of  sufficiently often, the two numbers are equal: starting with a crossing-minimal drawing of , we can realize each crossing by aligning a -edge with a -edge.
\end{proof}



We now turn to the proof the Theorem~\ref{thm:sim-draw}.


\begin{proofof}{of Theorem~\ref{thm:sim-draw}}
We first note that it is easy to go from  to : Suppose
we have constructed, in time  a simultaneous planar drawing  so that private edges of  and  intersect at most  times, all edges of  are straight, and all private edges of  have at most  bends. We add dummy
vertices at the locations of the  crossings points in , and then construct a straight-line drawing of the resulting planar graph on a small grid. The number of bends per edge in the new drawing is at most , since each edge in  intersects fewer than  edges, and each one of them at most  times.

We are left with the proof of . That is, we have to construct in time  a simultaneous planar drawing of  in which private edges of  and  intersect at most  times, all edges of  are straight, and every private edge of  has at most  bends.

Start with an arbitrary straight-line planar drawing  of . We now construct a drawing  of  using an approach similar to the proof of Theorem~\ref{thm:PWG}. Drawing  induces a straight-line planar drawing  of . Thus, in order to determine , it remains to describe how to draw the private edges of . We will accomplish this independently for each face  of .

We construct a triangulation  of  by using all the vertices and edges of  that lie inside , as well as some extra edges we will specify shortly. Next, we execute the same algorithm we used in the proof of Theorem~\ref{thm:sim-draw}.  Namely, we construct a straight-line drawing of the dual  of  and we take a spanning tree  of . For each facial walk  of , we augment  with a leaf   close to  and inside , if , and coinciding with , if ; here,  is an inner -approximation of  constructed as earlier. Let  be the embedded multigraph obtained by restricting  to the vertices and edges inside or on the boundary of , and by contracting each facial walk  of  to a single vertex . We use Lemma~\ref{cor:Te} to construct a planar poly-line drawing of  that realizes the given embedding, that is -close to , and in which vertices  maintain their fixed locations. Finally, for boundary components with , we reconnect edges in  to the boundary components they belong to. In order to do this, we first ``wrap'' the edges of  passing by a vertex  around , and we then extend the edges of  incident to  to their endpoint on , by routing them around .

By construction every edge of  is straight. By Theorem~\ref{thm:PWG} every private edge of  has at most  bends. Also, the algorithmic steps are the same as for the proof of Theorem~\ref{thm:PWG}, hence the algorithm runs in  time. It remains to prove that any private edge of  and any private edge of  intersect at most ~times.

Consider any private edge  of  and any private edge  of . Recall that  is an edge of . Denote by  and  the facial walks that the end-vertices of  belong to.
Edge  can only intersect edge  in the following two situations:
when passing by  or  and when passing by the point  in which the edge of  dual to  crosses . We prove that each of these two types of intersections happens at most  times.

For the first type of intersections, Lemma~\ref{cor:Te} implies that edge  passes by each of  or  at most  times, hence at most  times in total.

For the second type of intersections, Lemma~\ref{lem:PWHam} implies that edge  is subdivided into at most three edges , , and  in order to turn  into a Hamiltonian graph. For each ,  either belongs to the Hamiltonian cycle of the subdivided  or not. In the former case,  is drawn as part of an -approximation  of , as in the proof of Lemma~\ref{lem:HCemb}, hence it crosses  at most twice. In the latter case,  is composed of two parts, denoted by  and , or by  and  in the proof of Lemma~\ref{lem:Te}. Each of , ,  and  is part of a -approximation of , which is part of . Hence, each of , ,  and  crosses  at most twice; thus  crosses  at most four times, and  crosses  close to  at most  times.
\end{proofof}


\section{Open Questions}

We conclude with three open questions.
We proved that if a graph has a planar drawing extending a straight-line planar drawing of a subgraph then there is such a drawing with at most  bends per edge.  This is asymptotically tight, but can the constant  be reduced? As sketched at the end of Section~\ref{sec:PEG}, a variation of our algorithm decreases this constant to , however new ideas seem to be needed in order to push the bound further down.

Our second result was that any two simultaneously planar graphs have a simultaneous planar drawing with at most  crossings per pair of edges and a linear number of bends per edge with a drawing on a polynomial-sized grid.  The only lower bound on the number of crossings between two edges in a simultaneous planar drawing is 2 (see~\cite{CJS08} or the figure in the margin for the entry ``simultaneous crossing number'' in~\cite{S13b}).  There is a large gap between  and .  Can two edges be forced to cross more than twice in a simultaneous planar drawing?
For the third open question, we note that
Grilli et al.~\cite{Grilli2014} showed that two simultaneously planar graphs have a drawing with at most  bends per edge, though with a larger constant for the number of crossings and not on a grid.  Is it possible to achieve the best of both results:   bends per edge,  crossings per pair of edges, and a nice grid?



\bibliographystyle{abbrvurl}
\bibliography{./SEFE2}

\end{document}
